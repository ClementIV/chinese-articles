吊古战场文

浩浩乎,平沙无垠,夐不见人。河水萦带,群山纠纷。黯兮惨悴,风悲日曛。蓬断草枯,凛若霜晨。鸟飞不下,兽铤亡群。亭长告余曰:“此古战场也,常覆三军。往往鬼哭,天阴则闻。”伤心哉!秦欤汉欤?将近代欤?

吾闻夫齐魏徭戍,荆韩召募。万里奔走,连年暴露。沙草晨牧,河冰夜渡。地阔天长,不知归路。寄身锋刃,腷臆谁愬?秦汉而还,多事四夷,中州耗斁,无世无之。古称戎夏,不抗王师。文教失宣,武臣用奇。奇兵有异于仁义,王道迂阔而莫为。呜呼噫嘻!

吾想夫北风振漠,胡兵伺便。主将骄敌,期门受战。野竖旌旗,川回组练。法重心骇,威尊命贱。利镞穿骨,惊沙入面,主客相搏,山川震眩。声析江河,势崩雷电。至若穷阴凝闭,凛冽海隅,积雪没胫,坚冰在须。鸷鸟休巢,征马踟蹰。缯纩无温,堕指裂肤。当此苦寒,天假强胡,凭陵杀气,以相剪屠。径截辎重,横攻士卒。都尉新降,将军复没。尸踣巨港之岸,血满长城之窟。无贵无贱,同为枯骨。可胜言哉!鼓衰兮力竭,矢尽兮弦绝,白刃交兮宝刀折,两军蹙兮生死决。降矣哉,终身夷狄;战矣哉,暴骨沙砾。鸟无声兮山寂寂,夜正长兮风淅淅。魂魄结兮天沉沉,鬼神聚兮云幂幂。日光寒兮草短,月色苦兮霜白。伤心惨目,有如是耶!

吾闻之:牧用赵卒,大破林胡,开地千里,遁逃匈奴。汉倾天下,财殚力痡。任人而已,岂在多乎!周逐猃狁,北至太原。既城朔方,全师而还。饮至策勋,和乐且闲。穆穆棣棣,君臣之间。秦起长城,竟海为关。荼毒生民,万里朱殷。汉击匈奴,虽得阴山,枕骸徧野,功不补患。

苍苍蒸民,谁无父母?提携捧负,畏其不寿。谁无兄弟?如足如手。谁无夫妇?如宾如友。生也何恩,杀之何咎?其存其没,家莫闻知。人或有言,将信将疑。悁悁心目,寤寐见之。布奠倾觞,哭望天涯。天地为愁,草木凄悲。吊祭不至,精魂无依。必有凶年,人其流离。呜呼噫嘻!时耶命耶?从古如斯!为之奈何?守在四夷。