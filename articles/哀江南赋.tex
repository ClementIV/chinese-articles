\article{哀江南赋}

粤以戊辰之年,建亥之月,大盗移国,金陵瓦解。余乃窜身荒谷,公私涂炭。华阳奔命,有去无归,中兴道销,穷于甲戌,三日哭于都亭,三年囚于别馆。

天道周星,物极不反。傅燮之但悲身世,无处求生;袁安之每念王室,自然流涕。昔桓君山之志事,杜元凯之平生,并有著书,咸能自序。潘岳之文采,始述家风;陆机之辞赋,先陈世德。信年始二毛,即逢丧乱,藐是流离,至于暮齿。《燕歌》远别,悲不自胜;楚老相逢,泣将何及!畏南山之雨,忽践秦庭;让东海之滨,遂餐周粟。下亭漂泊,高桥羁旅;楚歌非取乐之方,鲁酒无忘忧之用。追为此赋,聊以记言;不无危苦之辞,惟以悲哀为主。

日暮途远,人间何世?将军一去,大树飘零;壮士不还,寒风萧瑟。荆璧睨柱,受连城而见欺;载书横阶,捧珠盘而不定。钟仪君子,入就南冠之囚;季孙行人,留守西河之馆。申包胥之顿地,碎之以首;蔡威公之泪尽,加之以血。钓台移柳,非玉关之可望;华亭鹤唳,岂河桥之可闻?

孙策以天下为三分,众才一旅;项籍用江东之子弟,人惟八千;遂乃分裂山河,宰割天下。岂有百万义师,一朝卷甲;芟夷斩伐,如草木焉!江淮无涯岸之阻,亭壁无籓篱之固。头会箕敛者合从缔交;锄耰棘矜者因利乘便。将非江表王气,终于三百年乎?

是知并吞六合,不免轵道之灾;混一车书,无救平阳之祸。呜呼!山岳崩颓,既履危亡之运;春秋迭代,必有去故之悲。天意人事,可以凄怆伤心者矣。况复舟楫路穷,星汉非乘槎可上;风飙道阻,蓬莱无可到之期。穷者欲达其言,劳者须歌其事。陆士衡闻而抚掌,是所甘心;张平子见而陋之,固其宜矣。

赋文

我之掌庾承周,以世功而为族;经邦佐汉,用论道而当官。禀嵩华之玉石,润河洛之波澜。居负洛而重世,邑临河而宴安。逮永嘉之艰虞,始中原之乏主。民枕倚于墙壁,路交横于豺虎。值五马之南奔 ,逢三星之东聚 。彼凌江而建国,始播迁于吾祖。分南阳而赐田,裂东岳而胙土。诛茅宋玉之宅,穿径临江之府。水木交运,山川崩竭。家有直道,人多全节。训子见于纯深,事君彰于义烈。新野有生祠之庙,河南有胡书之碣。况乃少微真人,天山逸民,阶庭空谷,门巷蒲轮。移谈讲树,就简书筠。降生世德,载诞贞臣。文词高于甲观,楷模盛于漳滨。嗟有道而无凤,叹非时而有麟。既奸回之奰逆,终不悦于仁人。

王子滨洛之岁,兰成射策之年。始含香于建礼,仍矫翼于崇贤;游洊雷之讲肆,齿明离之胄筵。既倾蠡而酌海,遂测管而窥天。方塘水白,钓渚池圆。侍戎韬于武帐,听雅曲于文弦。乃解悬而通籍,遂崇文而会武。居笠毂而掌兵,出兰池而典午。论兵于江汉之君,拭玉于西河之主。

于是朝野欢娱,池台钟鼓。里为冠盖,门成邹鲁。连茂苑于海陵,跨横塘于江浦。东门则鞭石成桥,南极则铸铜为柱。橘则园植万株,竹则家封千户。西赆浮玉,南琛没羽。吴歈越吟,荆艳楚舞。草木之遇阳春,鱼龙之逢风雨。五十年中,江表无事。班超为定远之侯,王歙为和亲之使。马武无预于甲兵,冯唐不论于将帅。岂知山岳闇然,江湖潜沸,渔阳有闾左戍卒,离石有将兵都尉。

天子方删诗书,定礼乐;设重云之讲,开士林之学;谈劫烬之灰飞,辨常星之夜落。地平鱼齿,城危兽角;卧刁 斗于荥阳,绊龙媒于平乐。宰衡以干戈为儿戏,缙绅以清谈为庙略。乘渍水以胶船,驭奔驹以朽索。小人则将及水火,君子则方成猿鹤。敝箄不能救盐池之咸,阿胶不能止黄河之浊。既而鲂鱼赪尾,四郊多垒。殿狎江鸥,宫鸣野雉。湛庐去国,艅艎失水。见被发于伊川,知百年而为戎矣。

彼奸逆之炽盛,久游魂而放命。大则有鲸有鲵,小则为枭为獍。负其牛羊之力,肆其水草之性;非玉烛之能调,岂璇玑之可正。值天下之无为,尚有欲于羁縻。饮其琉璃之酒,赏其虎豹之皮;见胡柯于大夏,识鸟卵于条枝。豺牙密厉,虺毒潜吹。轻九鼎而欲问,闻三川而遂窥。

始则王子召戎,奸臣介胄。既官政而离逷,遂师言而泄漏。望廷尉之逋囚,反淮南之穷寇。出狄泉之苍鸟,起横江之困兽。地则石鼓鸣山,天则金精动宿。北阙龙吟,东陵麟斗。

尔乃桀黠构扇,冯陵畿甸。拥狼望于黄图,填卢山于赤县。青袍如草,白马如练。天子履端废朝,单于长围高宴。两观当戟,千门受箭;白虹贯日,苍鹰击殿;竟遭夏台之祸,终视尧城之变。官守无奔问之人,干戚非平戎之战。陶侃空争米船,顾荣虚摇羽扇。

将军死绥,路绝重围。烽随星落,书逐鸢飞。乃韩分赵裂,鼓卧旗折。失群班马,迷轮乱辙。猛士婴城,谋臣卷舌。昆阳之战象走林,常山之阵蛇奔穴。五郡则兄弟相悲,三州则父子离别。

护军慷慨,忠能死节,三世为将,终于此灭。济阳忠壮,身参末将,兄弟三人,义声俱唱。主辱臣死,名存身丧。敌人归元,三军凄怆。尚书多算,守备是长。云梯可拒,地道能防。有齐将之闭壁,无燕师之卧墙。大事去矣,人之云亡!

申子奋发,勇气咆勃。实总元戎,身先士卒。胄落鱼门,兵填马窟。屡犯通中,频遭刮骨。功业夭枉,身名埋没。或以隼翼鷃披,虎威狐假。沾渍锋镝,脂膏原野。兵弱虏强,城孤气寡。闻鹤唳而心惊,听胡笳而泪下。拒神亭而亡戟,临横江而弃马。崩于钜鹿之沙,碎于长平之瓦。

于是桂林颠覆,长洲麋鹿。溃溃沸腾,茫茫墋黩。天地离阻,神人惨酷。晋郑靡依,鲁卫不睦。竞动天关,争回地轴。探雀鷇而未饱,待熊蹯而讵熟?乃有车侧郭门,筋悬庙屋。鬼同曹社之谋,人有秦庭之哭。

尔乃假刻玺于关塞,称使者之酬对。逢鄂坂之讥嫌,值耏门之征税。乘白马而不前,策青骡而转碍。吹落叶之扁舟,飘长风于上游。彼锯牙而钩爪,又循江而习流。排青龙之战舰,斗飞燕之船楼。张辽临于赤壁,王濬下于巴丘。乍风惊而射火,或箭重而沉舟。未辨声于黄盖,已先沉于杜侯。落帆黄鹤之浦,藏船鹦鹉之洲。路已分于湘汉,星犹看于斗牛。

若乃阴陵失路,钓台斜趣。望赤壁而沾衣,舣乌江而不渡。雷池栅浦,鹊陵焚戍。旅舍无烟,巢禽无树。谓荆、衡之杞梓,庶江、汉之可恃。淮海维扬,三千馀里。过漂渚而寄食,托芦中而渡水。届于七泽,滨于十死。嗟天保之未定,见殷忧之方始。本不达于危行,又无情于禄仕。谬掌卫于中军,滥尸丞于御史。

信生世等于龙门,辞亲同于河洛。奉立身之遗训,受成书之顾托。昔三世而无惭,今七叶而始落。泣风雨于《梁山》,惟枯鱼之衔索。入欹斜之小径,掩蓬藋之荒扉。就汀洲之杜若,待芦苇之单衣。

于是西楚霸王,剑及繁阳。鏖兵金匮,校战玉堂。苍鹰赤雀,铁舳牙樯。沉白马而誓众,负黄龙而渡江,海潮迎舰,江萍送王。戎军屯于石城,戈船掩于淮泗。诸侯则郑伯前驱,盟主则荀罃暮至。剖巢燻穴,奔魑走魅。埋长狄于驹门,斩蚩尤于中冀。燃腹为灯,饮头为器。直虹贯垒,长星属地。昔之虎踞龙盘,加以黄旗紫气,莫不随狐兔而窟穴,与风尘而殄瘁。

西瞻博望,北临玄圃,月榭风台,池平树古。倚弓于玉女窗扉,系马于凤皇楼柱。仁寿之镜徒悬,茂陵之书空聚。若夫立德立言,谟明寅亮;声超于系表,道高于河上。更不遇于浮丘,遂无言于师旷。以爱子而托人,知西陵而谁望?非无北阙之兵,犹有云台之仗。 司徒之表里经纶,狐偃之惟王实勤。横琱戈而对霸主,执金鼓而问贼臣。平吴之功,壮于杜元凯;王室是赖,深于温太真。始则地名全节,终则山称枉人。南阳校书,去之已远;上蔡逐猎,知之何晚?镇北之负誉矜前,风飙凛然。水神遭箭,山灵见鞭。是以蛰熊伤马,浮蛟没船。才子并命,俱非百年

中宗之夷凶靖乱,大雪冤耻,去代邸而承基,迁唐郊而纂祀。反旧章于司隶,归馀风于正始。沉猜则方逞其欲,藏疾则自矜于己。天下之事没焉,诸侯之心摇矣。既而齐交北绝,秦患西起。况背关而怀楚,异端委而开吴。驱绿林之散卒,拒骊山之叛徒。营军梁溠,蒐乘巴渝。问诸淫昏之鬼,求诸厌劾之符。荆门遭廪延之戮,夏口滥逵泉之诛。蔑因亲以致爱,忍和乐于弯弧。既无谋于肉食,非所望于《论都》。未深思于五难,先自擅于三端。登阳城而避险,卧砥柱而求安。既言多于忌刻,实志勇而形残。但坐观于时变,本无情于急难。地惟黑子,城犹弹丸。其怨则黩,其盟则寒。岂冤禽之能塞海?非愚叟之可移山。况以沴气朝浮,妖精夜陨。赤鸟则三朝夹日,苍云则七重围轸。亡吴之岁既穷,入郢之年斯尽。

周含郑怒,楚结秦冤。有南风之不竞,值西邻之责言。俄而梯冲乱舞,冀马云屯。俴秦车于畅毂,沓汉鼓于雷门。下陈仓而连弩,渡临晋而横船。虽复楚有七泽,人称三户;箭不丽于六麋,雷无惊于九虎。辞洞庭兮落木,去涔阳兮极浦。炽火兮焚旗,贞风兮害蛊。乃使玉轴扬灰,龙文折柱。下江余城,长林故营。徒思拑马之秣,未见烧牛之兵。章曼支以毂走,宫之奇以族行。河无冰而马渡,关未晓而鸡鸣。忠臣解骨,君子吞声。章华望祭之所,云梦伪游之地。荒谷缢于莫敖,冶父囚于群帅。硎穽折拉,鹰鹯批㩌 。冤霜夏零,愤泉秋沸。城崩杞妇之哭,竹染湘妃之泪。

水毒秦泾,山高赵陉。十里五里,长亭短亭。饥随蛰燕,暗逐流萤。秦中水黑,关上泥青。于时瓦解冰泮,风飞雹散,浑然千里,淄渑一乱。雪暗如沙,冰横似岸。逢赴洛之陆机,见离家之王粲,莫不闻陇水而掩泣,向关山而长叹。况复君在交河,妾在青波。石望夫而逾远,山望子而逾多。才人之忆代郡,公主之去清河。栩阳亭有离别之赋,临江王有愁思之歌。别有飘飖武威,羁旅金微。班超生而望返,温序死而思归。李陵之双凫永去,苏武之一雁空飞。

若江陵之中否,乃金陵之祸始。虽借人之外力,实萧墙之内起。拨乱之主忽焉,中兴之宗不祀。伯兮叔兮,同见戮于犹子。荆山鹊飞而玉碎,隋岸蛇生而珠死。鬼火乱于平林,殇魂游于新市。梁故丰徙,楚实秦亡。不有所废,其何以昌?有妫之后,将育于姜。输我神器,居为让王。天地之大德曰生,圣人之大宝曰位。用无赖之子弟,举江东而全弃。惜天下之一家,遭东南之反气。以鹑首而赐秦,天何为而此醉?

且夫天道回旋,生民赖焉。余烈祖于西晋,始流播于东川。洎余身而七叶,又遭时而北迁。提挈老幼,关河累年。死生契阔,不可问天。况复零落将尽,灵光岿然!日穷于纪,岁将复始。逼切危虑,端忧暮齿。践长乐之神皋,望宣平之贵里。渭水贯于天门,骊山回于地市。幕府大将军之爱客,丞相平津侯之待士。见钟鼎于金张,闻弦歌于许史。岂知灞陵夜猎,犹是故时将军;咸阳布衣,非独思归王子!
