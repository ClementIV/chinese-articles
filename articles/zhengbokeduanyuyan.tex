
初,郑武公娶于申,曰武姜,生庄公及共叔段。庄公寤生,惊姜氏,故名曰寤生,遂恶之。爱共叔段,欲立之,亟请于武公,公弗许。及庄公即位,为之请制。公曰:“制,岩邑也,虢叔死焉;他邑唯命。”请京,使居之,谓之京城大叔。

祭仲曰:“都城过百雉,国之害也。先王之制,大都不过参国之一;中,五之一;小,九之一。今京不度,非制也,君将不堪。”公曰:“姜氏欲之,焉辟害?”对曰:“姜氏何厌之有!不如早为之所,无使滋蔓!蔓,难图也。蔓草犹不可除,况君之宠弟乎!”公曰:“多行不义,必自毙。子姑待之。”

既而大叔命西鄙北鄙贰于己。公子吕曰:“国不堪贰,君将若之何?欲与大叔,臣请事之。若弗与,则请除之,无生民心。”公曰:“无庸,将自及。”大叔又收贰以为己邑,至于廪延。子封曰:“可矣!厚将得众。”公曰:“不义不暱,厚将崩。”

大叔完聚,缮甲兵,具卒乘,将袭郑。夫人将启之。公闻其期,曰:“可矣!”命子封帅车二百乘以伐京。京叛大叔段,段入于鄢。公伐诸鄢。五月辛丑,大叔出奔共。

书曰:“郑伯克段于鄢。”段不弟,故不言弟;如二君,故曰克;称郑伯,讥失教也;谓之郑志,不言出奔,难之也。

遂置姜氏于城颖,而誓之曰:“不及黄泉,无相见也!”既而悔之。颖考叔为颖谷封人,闻之,有献于公。公赐之食,食舍肉。公问之。对曰:“小人有母,皆尝小人之食矣,未尝君之羹,请以遗之。”公曰:“尔有母遗,繄我独无!”颖考叔曰:“敢问何谓也?”公语之故,且告之悔。对曰:“君何患焉!若阙地及泉,隧而相见,其谁曰不然?”公从之。公入而赋:“大隧之中,其乐也融融!”姜出而赋:“大隧之外,其乐也洩洩!”遂为母子如初。

君子曰:“颖考叔,纯孝也,爱其母,施及庄公。《诗》曰:‘孝子不匮,永锡尔类。’其是之谓乎?”