\book{春秋繁露}

\article{楚莊王}

楚莊王殺陳夏徵舒,春秋貶其文,不予專討也。靈王殺齊慶封,而直稱楚子,何也?曰:莊王之行頤賢,而舒之罪重。以賢君討重罪,其於人心善。若不貶,孰知其非正經。《春秋》常於其嫌得者,見其不得也。是故齊桓不予專地而對,晉文不予致王而朝,楚莊弗予專殺而討。三者不得,則諸侯之得,殆此矣。此楚靈之所以稱子而討也。《春秋》之辭,多所況,是文約而法明也。問者曰:不予諸侯之專封,複見於陳蔡之滅。不予諸侯之專討,獨不複見於慶封之殺,何也?曰:《春秋》之用辭,已明者去之,未明者著之。今諸侯之不得專討,固已明矣。而慶封之罪未有所見也,故稱楚子以伯討之,著其罪之宜死,以為天下大禁。曰:人臣之行,貶主之位,亂國之臣,雖不篡殺,其罪皆宜死,比於此其云爾也。《春秋》曰:「晉伐鮮虞。」奚惡乎晉而同夷狄也?曰:《春秋》尊禮而重信。信重於地,禮尊於身。何以知其然也?宋伯姬疑禮而死於火,齊桓公疑信而虧其地,《春秋》賢而舉之,以為天下法,曰禮而信。不答,施無不報,天之數也。


今我君臣同姓適女,女無良心,不符號。有恐畏我,何其不夷狄也。公子慶父之亂,魯危殆亡,而齊侯安之。於彼無親,尚來擾我,如何與同姓而殘賊遇我。《詩》云:「宛彼鳴鳩,翰飛戾天。我心憂傷,念彼先人。明發不味,有懷二人。」人皆有此心也。今晉不以同姓憂我,而強大厭我,我心望焉。故言之不好。謂之晉而已,婉辭也。晉惡而不可親,公往而不敢到,乃人情耳。君子何恥而稱公有疾也?曰:惡無故自來。君子不恥,內省不疚,何憂於誌,是已矣。今《春秋》恥之者,昭公有以取之也。臣陵其君,始於文而甚於昭。公受亂陵夷,而無懼惕之心,囂囂然輕計妄討,犯大禮而取同姓,接不義而重自輕也。人之言曰:「國家治,則四鄰賀;國家亂,則四鄰散。」是故季孫專其位,而大國莫之正。出走八年,死乃得歸。身亡子危,困之到也。君子不恥其困,而恥其所以窮。昭公難逢此時,敬不取同姓,諜到於是。雖取同姓,能用孔子自輔,亦不到如是。時難而治簡,行枉而無救,是其所以窮也。


春秋分十二世以為三等:有見,有聞,有傳聞。有見三世,有聞四世,有傳聞五世。故哀、定、昭,君子之所見也。襄、成、文、宣,君子之所聞也。僖、閔、莊、桓、隱,君子之所傳聞也。所見六十一年,所聞八十五年,所傳聞九十六年。於所見微其辭,於所聞痛其禍,於傳聞殺其恩,與情俱也。是故逐季氏而言又雩,微其辭也。

子赤殺,弗忍書日,痛其禍也。子般殺而書乙未,殺其恩也。屈伸之志,詳略之文,皆應之。吾以其近近而遠遠,親親而疏疏也,亦知其貴貴而賤賤,重重而輕輕也。有知其厚厚而薄薄,善善而惡惡也,有知其陽陽而陰陰,白白而黑黑也。百物皆有合偶,偶之合之,仇之匹之,善矣。《詩》云:「成儀抑抑,德音秩秩。無怨無惡,率由仇匹。」此之謂也。《春秋》,義之大者也。視其溫辭,可以知其塞怨。是故於外,道而不顯,於內,諱而不隱。於尊亦然,於賢亦然。此其別內外、差賢不肖而等尊卑也。義不訕上,智不危身。故遠者以義諱,近者以智畏。畏與義兼,則世逾近而言逾謹矣。此定哀之所以微其辭。以故用則天下平,不用則安其身,《春秋》之道也。


《春秋》之道,奉天而法古。是故雖有巧手,弗循規矩,不能正方員。雖有察耳,不吹六律,不能定五音。雖有知心,不覽先王,不能平天下。亦天下之規矩六律已。故聖者法天,賢者法聖,此其大數也。得大數而治,失大數而亂,此治亂之分也。所聞天下無二道,故聖人異治同理也。


古今通達,故先賢傳其法於後世也。《春秋》之於世事也,善複古,譏易常,欲其法先王也。然而介以一言曰:「王者必改制。」自僻者得此以為辭,曰:古苟可循先王之道,何莫相因?世迷是聞,以疑正道而信邪言,甚可患也。答之曰:人有聞諸侯之君射《狸首》之樂者,於是自斷狸首,懸而射之,曰:安在於樂也!此聞其名而不知其實者也。今所謂新王必改制者,非改其道,非變其理,受命於天,易姓更王,非繼前王而王也。若一因前制,修故業,而無有所改,是與繼前王而王者無以別。受命之君,天之所大顯也。事父者承意,事君者儀誌。事天亦然。今天大顯已,物襲所代而率與同,則不顯不明,非天誌。故必徙居處、更稱號、改正朔、易服色者,無他焉,不敢不順天誌而明白顯也。若夫大綱、人倫、道理、政治、教化、習俗、文義盡如故,亦何改哉?故王者有改制之名,無易道之實。孔子曰:「無為而治者,其舜乎!」言其主堯之道而已。此非不易之效與?問者曰:物改而天授顯矣,其必更作樂,何也?曰:樂異乎是。制為應天改之,樂為應人作之。彼之所受命者,必民之所同樂也。是故大改制於初,所以明天命也。更作樂於終,所以見天功也。緣天下之所新樂而為之文曲,且以和政,且以同德。天下未遍合和,王者不虛作樂。樂者,盈於內而動發於外者也。應其治時,制禮作樂以成之。成者,本末質文皆以具矣。是故作樂者必反天下之所始樂於己以為本。舜時,民樂其昭堯之業也,故《韶》。「韶」者,昭也。禹之時,民樂其三聖相繼,故《夏》。「夏」者,大也。湯之時,民樂其救之於患害也,故《濩》。「濩」者,救也。文王之時,民樂其同師徵伐也,故《武》。「武」者,伐也。四者,天下同樂之,一也,其所同樂之端不可一也。作樂之法,必反本之所樂。所樂不同事,樂安得不世異?是故舜作《韶》而禹作《夏》,湯作《濩》而文王作《武》。四樂殊名,則各順其民始樂於己也。見其效矣。《詩》云:「文王受命,有此武功。既伐於崇,作邑於豐。」樂之風也。又曰:「王赫斯怒,爰整其旅。」當是時,紂為無道,諸侯大亂,民樂文王之怒而詠歌之也。周人德已洽天下,反本以為樂,謂之《大武》,言民所始樂者武也云爾。故凡樂者,作之於終,而名之以始,重本之義也。此觀之,正朔、服色之改,受命應天制禮作樂之異,人心之動也。二者離而複合,所為一也。

\article{玉杯}

\begin{pinyinscope}
《春秋》譏文公以喪取。難者曰:「喪之法,不過三年。三年之喪,二十五月。今按經,文公乃四十一月方取。取時無喪,出其法也久矣。何以謂之喪取。」曰:春秋之論事,莫重於誌。今取必納幣,納幣之月在喪分,故謂之喪取也。且文公以秋祭,以冬納幣,皆失於太蚤。《春秋》不譏其前,而顧譏其後,必以三年之喪,肌膚之情也。雖從俗而不能終,猶宜未平於心。今全無悼遠之志,反思念取事,是《春秋》之所甚疾也。故譏不出三年於首而已,譏以喪取也。不別先後,賤其無人心也。緣此以論禮,禮之所重者在其誌。誌敬而節具,則君子予之知禮。誌和而音雅,則君子予之知樂。誌哀而居約,則君子予之知喪。故曰:非虛加之,重誌之謂也。誌為質,物為文。文著於質,質不居文,文安施質?質文兩備,然後其禮成。文質偏行,不得有我爾之名。俱不能備而偏行之,寧有質而無文。雖弗予能禮,尚少善之,介葛廬來是也。有文無質,非直不子,乃少惡之,謂州公實來是也。然則《春秋》之序道也,先質而後文,右誌而左物。「禮云禮云,玉帛云乎哉?」推而前之,亦宜曰:朝云朝云,辭令云乎哉?「樂云樂云,鐘鼓云乎哉?」引而後之,亦宜曰:喪雲喪雲,衣服雲乎哉?是故孔子立新王之道,明其貴誌以反和,見其好誠以滅偽。其有繼周之弊,故若此也。
\end{pinyinscope}


《春秋》之法,以人隨君,以君隨天。一日不可無君,而猶三年稱子者,為君心之未當立也。此非以人隨君耶?孝子之心,三年不當。三年不當而逾年即位者,與天數俱終始也。此非以君隨天邪?故屈民而伸君,屈君而伸天,《春秋》之大義也。《春秋》論十二世之事,人道浹而王道備。法布二百四十二年之中,相耿左右,以成文采。其居參錯,非襲古也。是故論《春秋》者,合而通之,緣而求之,五其比,偶其類,覽其緒,屠其贅,是以人道浹而王法立。以為不然?今夫天子逾年即位,諸侯於封內三年稱子,皆不在經也,而操之與在經無以異。非無其辨也,有所見而經安受其贅也。故能以比貫類、以辨付贅者,大得之矣。


人受命於天,有善善惡惡之性,可養而不可改,可豫而不可去,若形體之可肥,而不可得革也。是故雖有至賢,能為君親含容其惡,不能為君親令無惡。事親亦然,皆忠孝之極也。非至賢安能如是?父不父則子不子,君不君則臣不臣耳。


文公不能服喪,不時奉祭,不以三年,又以喪取,取於大夫,以卑宗廊,亂其群祖以逆先公。小善無一,而大惡四五,故諸侯弗予盟,是惡惡之徵、不臣之效也。出侮於外,人奪於內,無位之君也。孔子曰:「政逮於大夫四世矣。」蓋自文公以來之謂也。


君子知在位者之不能以惡服人也,是故簡六藝以贍養之。《詩》《書》具其志,《禮》《樂》純其養,《易》《春秋》明其知。六學皆大,而各有所長。《詩》道誌,故長於質。《禮》制節,故長於文。《樂》詠德,故長於風。《書》著功,故長於事。《易》本天地,故長於數。《春秋》正是非,故長於治人。能兼得其所長,而不能遍舉其詳也。礦人主大節則知暗,大博則業厭。二者異失同貶,其傷必到,不可不察也。是故善為師者,既美其道,有慎其行,齊時蚤晚,任多少,適疾徐,造而勿趨,稽而勿苦,省其所為,而成其所湛,故力不勞而身大成。


《春秋》之好微與?其貴誌也。《春秋》修本末之義,達變故之應,通生死之志,遂人道之極者也。是故君殺賊討,則善而書其誅。若莫之討,則君不書葬,而賊不複見矣。不書葬,以為無臣子也;賊不複見,以其宜滅絕也。今趙質弒君,四年之後,別牘複見,非《春秋》之常辭也。古今之學者異而問之,曰:是弒君何以複見?猶曰:賊未討,何以書葬?何以書葬者,不宜書葬也而書葬。何以複見者,亦不宜複見也而複見。二者同貫,不得不相若也。質之複見,直以赴問,而辨不親弒,非不當誅也。則亦不得不謂悼公之書葬,直以赴問而辨不成弒,非不當罪也。若是則《春秋》之說亂矣,豈可法哉。無比而處之,誣辭也。今視其比,皆不當死,何以誅之?《春秋》赴問數百,應問數千,同留經中。翻援比類,以發其端。卒無妄言而得應於傳者。今使外賊不可誅,故皆複見,而問曰此複見何也,言莫妄於是,何以得應乎?故吾以其得應,知其問之不妄。以其問之不妄,知質之獄不可不察也。夫名為弒父而實免罪者,已有之矣;亦有名為弒君,而罪不誅者。逆而距之,不若徐而味之。且吾語質有本,《詩》云:「他人有心,予忖度之。」此言物莫無鄰,察視其外,可以見其內也。今案盾事而觀其心,願而不刑,合而信之,非篡弒之鄰也。按盾辭號乎天,苟內不誠,安能如是?是故訓其終始無弒之志。掛惡謀者,過在不遂去,罪在不討賊而已。臣之宜為君討賊也,猶子之宜為父嘗藥也。子不嘗藥,故加之弒父;臣不討賊,故加之弒君。所以示天下廢臣子之節,其惡之大若此也。故盾之不討賊,為弒君也,與止之不嘗藥為弒父無以異。盾不宜誅,以此參之。問者曰:夫謂之弒而有不誅,其論難知,非蒙之所能見也。故赦止之罪,以傳明之。盾不誅,無傳,何也?曰:世亂義廢,背上不臣,篡弒覆君者多,而有明大惡之誅,誰言其誅。故晉趙質、楚公子比皆不誅之文,而弗為傳,弗欲明之心也。問者曰:人弒其君,重卿在而弗能討者,非一國也。靈公弒,趙盾不在。不在之與在,惡有厚薄。《春秋》責在而不討賊者,弗擊臣子爾也。責不在而不討賊者,乃加弒焉,何其責厚惡之薄、薄惡之厚也?曰:《春秋》之道,視人所惑,為立說以大明之。今趙盾賢而不遂於理,皆見其善,莫見其罪,故因其所賢而加之大惡,擊之重責,使人湛思而自省悟以反道。曰:吁!君臣之大義,父子之道,乃到乎此,此所由惡薄而責之厚也。他國不討賊者,諸斗筲之民,何足數哉?弗擊人數而已。此所由惡厚而責薄也。傳曰:輕為重,重為輕,非是之謂乎?故公子比嫌可以立,趙盾嫌無臣責,許止嫌無子罪。《春秋》為人不知惡而恬行不備也,是故重累責之,以矯枉世而直之。矯者不過其正,弗能直。知此而義異矣。

\article{竹林}

《春秋》之常辭也,不予夷狄而予中國為禮,到之戰,偏然反之,何也?曰:《春秋》無通辭,從變而移。今晉變而為夷狄,楚變而為君子,故移其辭以從其事。夫莊王之舍鄭,有可貴之美,晉人不知其善,而欲擊之。所救已解,如挑與之戰,此無善善之心,而輕救民之意也,是以賤之。而不使得與賢者為禮。秦穆侮蹇叔而大敗。鄭文輕眾而喪師。《春秋》之敬賢重民如是。是故戰攻侵伐,雖數百起,必一二書,傷其害所重也。問者曰:其書戰伐甚謹。其惡戰伐無辭,何也?曰:會同之事,大者主小;戰伐之事,後者主先。苟不惡,何為使起之者居下。是其惡戰伐之且《春秋》之法,凶年不修舊,意在無苦民爾。苦民尚惡之。況傷民乎?傷民尚痛之,況殺民乎?故曰:凶年舊則譏。造邑則諱。是害民之小者,惡之小也;害民之大者,惡之大也。今戰伐之於民,其為害幾何?考意而觀指,則《春秋》之所惡者,不任德而任力,驅民而殘賊之。其所好者,設而勿用,仁義以服之也。詩云:「弛其文德,洽此四國。」《春秋》之所善也。夫德不足以親近,而文不足以來遠,而斷斷以戰伐為之者,此固《春秋》之所甚疾已,皆非義也。難者曰:《春秋》之書戰伐也,有惡有善也。惡詐擊而善偏戰,奈何以《春秋》為無義戰而盡惡之也?曰:凡《春秋》之記災異也,雖有數莖,猶謂之無麥苗也。今天下之大,三百年之久,戰攻侵攻不可勝數,而複者有二焉。是何以異於無麥苗之有數莖哉?不足以難之,故謂之無義戰也。以無義戰為不可,則無麥苗亦不可也;以無麥苗為可,則無義戰亦可矣。若《春秋》之於偏戰也,善其偏,不善其戰,有以效其然也。《春秋》愛人,而戰者殺人,君子奚說善殺其所愛哉?故《春秋》之於偏戰也,猶其於諸夏也。引之魯,則謂之外;引之夷狄,則謂之內。比之詐戰,則謂之義;比之不戰,則謂之不義。故盟不如不盟。然而有所謂善盟。戰不如不戰,然而有所謂善戰。不義之中有義,義之中有不義。辭不能及,皆在於指,非精心達思者,其孰能知之。《詩》云:「棠棣之華,偏其反而。豈不爾思?室是遠而。」孔子曰:「未之思,夫何遠之有!」由是觀之。見其指者,不任其辭。不任其辭,然後可與適道矣。



司馬子反為其君使。廢君命,與敵情,從其所請,與宋平。是內專政而外擅名也。專政則輕君,擅名則不臣,而《春秋》大之,奚由哉?曰:為其有慘怛之恩,不忍餓一國之民,使之相食。推恩者遠之而大,為仁者自然而美。今子反出己之心,矜宋之民,無計其閒,故大之也。難者曰:《春秋》之法,卿不憂諸侯,政不在大夫。子反為楚臣而恤宋民,是憂諸侯也;不複其君而與敵平,是政在大夫也。溴梁之盟,信在大夫,而諸侯刺之,為其奪君尊也。平在大夫,亦奪君尊,而《春秋》大之,此所間也。且《春秋》之義,臣有惡,擅名美。故忠臣不諫,欲其由君出也。《書》曰:「爾有嘉謀嘉猷,入告爾君於內,爾乃順之於外,曰:此謀此猷,惟我君之德。」此為人臣之法也。古之良大夫,其事君皆若是。今子反去君近而不複,莊王可見而不告,皆以其解二國之難為不得已也。奈其奪君名美何?此所惑也。曰:《春秋》之道,固有常有變,變用於變,常用於常,各止其科,非相妨也。今諸子所稱,皆天下之常,雷同之義也。子反之行,一曲之變。獨修之意也。夫目驚而體失其容,心驚而事有所忘,人之情也。通於驚之情者,取其一美,不盡其失。《詩》云:「采葑采菲,無以下體。」此之謂也。今子反往視宋,間人相食,大驚而哀之,不意之到於此也,是以心駭目動而違常禮。禮者,庶於仁、文,質而成體者也。今使人相食,大失其仁,安著其禮?方救其質,奚恤其文?《春秋》之辭,有所謂賤者,有賤乎賤者。夫有賤乎賤者,則亦有貴乎貴者矣。今讓者《春秋》之所貴。雖然見人相食,驚人相爨,救之忘其讓,君子之道有貴於讓者也。故說《春秋》者,無以平定之常義,疑變故之大則,義幾可諭矣。


《春秋》記天下之得失,而見所以然之故。甚幽而明,無傳而著,不可不察也。夫泰山之為大,弗察弗見,而況微渺者乎?故案《春秋》而適往事,窮其端而視其故,得誌之君子,有喜之人,不可不慎也。齊頃公親齊桓公之孫,國固廣大而地勢便利矣,又得霸主之余尊,而誌加於諸侯。以此之故,難使會同,而易使驕奢。即位九年,未嘗肯一與會同之事。有怒魯衛之志,而不從諸侯於清丘、斷道。春往伐魯,入其北郊,顧返伐衛,敗之新築。當是時也,方乘勝而誌廣,大國往聘,慢而弗敬其使者。晉魯懼怒,內悉其眾,外得黨與曹衛,四國相輔,大困之奸,獲齊頃公,逄丑父。深本頃公之所以大辱身,幾亡國,為天下笑,其端乃從懾魯勝衛起。伐魯,魯不敢出,擊衛,大敗之,因得氣而無敵國以興患也。故曰,得誌有喜,不可不戒。此其效也。自是之後,頃公恐懼,不聽聲樂,不飲酒食肉,內愛百姓,問疾吊霄,外敬諸侯。從會與盟,卒終其身,國家安寧。是福之本生於憂,而祝起於喜也。嗚呼!物之所由然,其於人切近,可不省邪?


逄丑父殺其身以生其君,何以不得謂知權?丑父欺晉,祭仲許宋,俱枉正以存其君。然而丑父之所為,難於祭仲,祭仲見賢而丑父猶見非,何也?曰:是非難別者在此。此其嫌疑相似而不同理者,不可不察。夫去位而避兄弟者,君子之所甚貴;獲虜逃遁者,君子之所甚賤。祭仲措其君於人所甚貴以生其君,故《春秋》以為知權而賢之。丑父措其君於人所甚賤以生其君,《春秋》以為不知權而簡之。其俱枉正以存君,相似也;其使君榮之與使君辱,不同理。故凡人之有為也,前枉而後義者,謂之中權,雖不能成,《春秋》善之,魯隱公、鄭祭仲是也。前正而後有枉者,謂之邪道,雖能成之,《春秋》不愛,齊頃公、逄丑父是也。夫冒大辱以生,其情無樂,故賢人不為也,而眾人疑焉。《春秋》以為人之不知義而疑也,故示之以義,曰國滅君死之,正也。正也者,正於天之為人性命也。天之為人性命,使行仁義而羞可恥,非若鳥獸然,苟為生,苟為利而已。是故《春秋》推天施而順人理,以到尊為不可以加於到辱大羞,故獲者絕之。以到辱為亦不可以加於到尊大位,故難失位弗君也。已反國複在位矣,而《春秋》猶有不君之辭,況其固然方獲而虜邪。其於義也,非君定矣。若非君,則丑父何權矣。故欺三軍為大罪於晉,其免頃公為辱宗廟於齊,是以雖難而《春秋》不愛。丑父大義,宜言於頃公曰:「君慢侮而怒諸侯,是失禮大矣。今被大辱而弗能死,是無恥也而複重罪。請俱死,無辱宗廟,無羞社稷。」如此,雖陷其身,尚有廉名。嘗此之時,死賢於生。故君子生以辱,不如死以榮,正是之謂也。由法論之,則丑父欺而不中權,忠而不中義,以為不然?複察《春秋》。《春秋》之序辭也,置王於春正之間,非日上奉天施而下正人,然後可以為王也云爾。


今善善惡惡,好榮憎辱,非人能自生,此天施之在人者也。君子以天施之在人者聽之,則丑父弗忠也。天施之在人者,使人有廉恥。有廉恥者,不生於大辱。大辱莫甚於去南面之位而束獲為虜也。曾子曰:「辱若可避,避之而已。及其不可避,君子視死如歸。」謂如頃公者也。


《春秋》曰:「鄭伐許。」奚惡於鄭而夷狄之也?曰:衛侯卒,鄭師侵之,是伐喪也。鄭與諸侯盟於蜀,以盟而歸,諸侯於是伐許,是叛盟也。伐喪無義,叛盟無信,無信無義,故大惡之。問者曰:「是君死,其子未逾年,有稱伯不子,法辭其罪何?曰:先王之制,有大喪者,三年不呼其門,順其誌之不在事也。《書》云:「高宗諒暗,三年不言。」居喪之義也。今縱不能如是,奈何其父卒未逾年即以喪舉兵也。《春秋》以薄恩,且施失其子心,故不複得稱子,謂之鄭伯,以辱之也。且其先君襄公伐喪叛盟,得罪諸侯,諸侯怒之未解,惡之未已。繼其業者,宜務善以覆之,父伐人喪,子以喪伐人,父加不義於人,子施失恩於親,以犯中國,是父負故惡於前,己起大惡於後。諸侯果怒而憎之,率而俱到,謀共擊之。鄭乃恐懼,去楚而成蟲牢之盟是也。楚與中國俠而擊之,鄭罷疲危亡,終身愁辜。無義而敗,由輕心然。孔子曰:「道千乘之國,敬事而信。」知其為得失之大也,故敬而慎之。今鄭伯既無子恩,又不熟計,舉兵不當,被患不窮,自取之也。是以生不得稱子,去其義也;死不得書葬,見其窮也。有國者視此。行身不放義,同事不審時,興事不審時,其何如此爾。


\article{玉英}

謂一元者,大始也。知元年誌者,大人之所重,小人之所輕。是故治國之端在正名。名之正,興五世,五傳之外,美惡乃形,可謂得其真矣,非子路之所能見。


非其位而即之,雖受之先君,《春秋》危之,宋繆公是也。非其位,不受之先君,而自即之,《春秋》危之,吳王僚是也。雖然,苟能行善得眾。《春秋》弗危,衛侯晉以立書葬是也。俱不宜立,而宋繆受之先君而危。衛宣弗受先君而不危,以此見得眾心之為大安也。故齊桓非直弗受之先君也。乃率弗宜為君者而立,罪亦重矣。然而知鞏懼,敬眾賢人,而以自覆蓋,知不背要盟以自湔浣也,遂為賢君,而霸諸侯。使齊桓被惡而無此美,得免殺戮乃幸已,何霸之有!魯桓忘其憂而禍逮其身。齊桓憂其憂而立功名。推而散之。凡人有憂而不知憂者凶,有憂而深憂之者吉。《易》曰:「複自道,何其咎。」此之謂也。匹夫之反道以除咎尚難,人主之反道以除咎甚易。《詩》云:「德如毛。」言其易也。


公觀魚於棠,何?惡也。凡人之性,莫不善義,然而不能義者,利敗之也。故君子終日言不及利,欲以勿言愧之而已,愧之以塞其源也。夫處位動風化者,徒言利之名爾,猶惡之,況求利乎?故天王使人求賻求金,皆為大惡而書。今非直使人也,親自求之,是為甚惡。譏何故言觀魚?猶言觀社也,皆諱大惡之辭也。


《春秋》有經禮,有變禮。為如安性平心者,經禮也。至有於性,雖不安,於心,雖不平,於道,無以易之,此變禮也。是故昏禮不稱主人,經禮也。辭窮無稱,稱主人,變禮也。天子三年然後稱王,經禮也。有故則未三年而稱王,變禮也。婦人無出境之事,經禮也。母為子娶婦,奔喪父母,變禮也。明乎經變之事,然後知輕重之分,可與適權矣。難者曰:《春秋》事同者辭同。此四者俱為變禮,而或達於經,或不達於經,何也?曰:《春秋》理百物,辨品類,別嫌微。修本未者也。是故星墜謂之隕,螽附謂之雨,其所發之處不同,或降於天,或發於地,其辭不可同也。今四者俱為變禮也同,而其所發亦不同。或發於男,或發於女,其辭不可同也。是或達於常,或達於變也。


桓之志無王,故不書王。其誌欲立,故書即位。書即位者,言其弒君兄也。不書王者,以言其背天子。是故隱不言立,桓不言王者,從其誌以見其事也。從賢之志以達其義,從不肖之志以著其惡。由此觀之,《春秋》之所善,善也,所不善,亦不善也,不可不兩省也。

經曰:「宋督弒其君與夷。」《傳》言:「莊公馮殺之。」不可及於經,何也?曰:非不可及於經,其及之端眇,不足以類鉤之,故難知也。《傳》曰:「臧孫許與晉卻克同時而聘乎齊。」按經無有,豈不微哉。不書其往而有避也。今此《傳》言莊公馮,而於經不書,亦以有避也。是以不書聘乎齊,避所羞也。不書莊公馮殺,避所善也。是故讓者《春秋》之所善。宣公不與其子而與其弟,其弟亦不與子而反之兄子,雖不中法,皆有讓高,不可棄也。故君子為之諱不居正之謂避,其後也亂。移之宋督以存善誌。若直書其篡,則宣繆之高滅,而善之無所見矣。難者曰:為賢者諱,皆言之,為宣繆諱,獨弗言,何也?曰:不成於賢也。其為善不法,不可取,亦不可棄。棄之則棄善誌也,取之則害王法。故不棄亦不載,以竟見之而已。苟誌於仁無惡,此之謂也。


器從名、地從主人之謂制。權之端焉,不可不察也。夫權雖反經,亦必在可以然之域。不在可以然之域,故雖死亡,終弗為也,公子目夷是也。故諸侯父子兄弟不宜立而立者,《春秋》視其國與宜立之君無以異也。此皆在可以然之域也。到於取乎莒,以之為同居,目曰「莒人滅,此在不可以然之域也。故諸侯在不可以然之域者,謂之大德,大德無逾閒者,謂正經。諸侯在可以然之域者,謂之小德,小德出入可也。權譎也,尚歸之以奉鉅經耳。故《春秋》之道,博而要,詳而反一也。公子目夷複其君,終不與國,祭仲已與,後改之,晉荀息死而不聽,衛曼姑拒而弗內,此四臣事異而同心,其義一也。目夷之弗與,重宗廟。祭仲與之,亦重宗廟。荀息死之,貴先君之命。曼姑拒之,亦貴先君之命民。事雖相反,所為同,俱為重宗廟、貴先帝之命耳。難者曰:公子目夷、祭仲之所以為者,皆存之事君,善之可矣。荀息、曼姑非有此事也,而所欲恃者皆不宜立者,何以得載乎義?曰:《春秋》之法,君立不宜立,不書,大夫立則書。書之者,弗予大夫之得立不宜立者也。不書,予君之得立之也。君之立不宜立者,非也。既立之,大夫奉之是也,荀息曼姑之所得為義也。


難紀季曰:《春秋》之法,大夫不得用地。又曰:公子無去國之義。又曰:君子不避外難。紀季犯此三者,何以為賢?賢臣故盜地以下敵,棄君以避難乎?曰:賢者不為是。是故托賢於紀季,以見季之弗為也。紀季弗為而紀侯使之可知矣。《春秋》之書事時,詭其實以有避也。其書人時,易其名以有諱也。故詭晉文得誌之實,以代諱避致王也。詭莒子號謂之人,避隱公也。易慶父之名謂之仲孫,變盛謂之成,諱大惡也。然則說《春秋》者,入則詭辭,隨其委曲而後得之。今紀季受命乎君而經書專,無善一名而文見賢,此皆詭辭,不可不察。《春秋》之於所賢也,固順其誌而一其辭,章其義而褒其美。今紀侯《春秋》之所貴也,是以聽其入齊之志,而詭其服罪之辭也,移之紀季。故告於齊者,實莊公為之,而《春秋》詭其辭,以予臧孫辰。以入於齊者,實紀侯為之,而《春秋》詭其辭,以與紀季。所以詭之不同,其實一也。難者曰:有國家者,人欲立之,固盡不聽,國滅君死之,正也,何賢乎紀侯?曰:齊將複讎,紀侯自知力不加而誌距之,故謂其弟曰:「我宗廟之主,不可以不死也。汝以往,服罪於齊,請以立五廟,使我先君歲時有所依歸。」率一國之眾,以衛九世之主。襄公逐之不去,求之弗予,上下同心而俱死之。故謂之大去。《春秋》賢死義,且得眾心也,故為諱滅。以為之諱,見其賢之也。以其賢之也,見其中仁義也。




\article{精華}

《春秋》慎辭,謹於名倫等物者也。是故小夷言伐而不得言戰,大夷言戰而不得言獲,中國言獲而不得言執,各有辭也。有小夷避大夷而不得言戰,大夷避中國而不得言獲,中國避天子而不得言執,名倫弗予,嫌於相臣之辭也。是故大小不逾等,貴賤如其倫,義之正也。

大雩者何?旱祭也。難者曰:大旱雩祭而請雨,大水鳴鼓而攻社,天地之所為,陰陽之所起也。或請焉,或攻焉,何也?曰:大旱,陽滅陰也。陽滅陰者,尊厭卑也,固其義也,雖大甚,拜請之而已,敢有加也?大水者,陰滅陽也。陰滅陽者,卑勝尊也,日食亦然。皆下犯上,以賤傷貴者,逆節也,故鳴鼓而攻之,朱絲而脅之,為其不義也。此亦《春秋》之不畏強御也。故變天地之位,正陰陽之序,直行其道而不忘其難,義之至也。是故脅嚴社而不為不敬靈,出天王而不為不尊上,辭父之命而不為不承親,絕母之屬而不為不孝慈,義矣夫。


難者曰:《春秋》之法,大夫無遂事。又曰:出境有可以安社稷、利國家者,則專之可也。又曰:大夫以君命出,進退在大夫也。又曰:聞喪徐行而不反也。夫既曰無遂事矣,又曰專之可也。既曰進退在大夫矣,又曰徐行而不反也。若相悖然,是何謂也?曰:四者各有所處。得其處則皆是也,失其處,則皆非也。《春秋》固有常義,又有應變。無遂事者,謂平生安寧也。專之可也者,謂救危除患也。進退在大夫者,謂將率用兵也。徐行不反者,謂不以親害尊,不以私妨公也。此之謂將得其私,知其指。故公子結受命往媵陳人之婦,於鄄。道生事,從齊桓盟,《春秋》弗非,以為救莊公之危。公子遂受命使京師,道生事之晉,《春秋》非之,以為是時僖公安寧無危。故有危而不專救,謂之不忠;無危而擅生事,是卑君也。故此二臣俱生事,《春秋》有是有非,其義然也。


齊桓挾賢相之能,用大國之資,即位五年,不能致一諸侯。於柯之盟,見其大信,一年而近國之君畢到,鄄幽之會是也。其後二十年之間亦久矣,尚未能大合諸侯也。至於救邢衛之事,見存亡繼絕之義,而明年遠國之君畢到,貫澤、陽谷之會是也。故曰親近者不以言,召遠者不以使,此其效也。其後矜功,振而自足,而不修德,故楚人滅弦而誌弗憂,江黃伐陳而不往救,損人之國而執其大夫,不救陳之患而責陳不納,不複安鄭,而必欲迫之以兵,功未良成而誌已滿矣。故曰:「管仲之器小哉!」此之謂也。


《春秋》之聽獄也,必本其事而原其誌。誌邪者不待成,首惡者罪特重,本直者其論輕。是故逄丑父當,而轅濤涂不宜執,魯季子追慶父,而吳季子釋闔廬。此四者罪同異論,其本殊也。俱欺仨三軍,或死或不死;俱弒君,或誅或不誅。聽訟折獄,可無審耶!故折獄而是也,理益明,教益行。折獄而非也,暗理迷眾,與教相妨。教,政之本也。獄,政之末也。其事異域,其用一也,不可不以相順,故君子重之也。


難晉事者曰:《春秋》之法,未逾年之君稱子,蓋人心之正也。到裡克殺奚齊,避此正辭而稱君之子,何也?曰:所聞《詩》無達詁,《易》無達佔,《春秋》無達辭,從變從義,而一以奉人。仁人錄其同姓之禍,固宜異操。晉,《春秋》之同姓也。驪姬一謀而三君死之,天下之所共痛也。本其所為為之者,蔽於所欲得位而不見其難也。《春秋》疾其所蔽,故去其正辭,徒言君之子而已。若謂奚齊曰:嘻嘻!為大國君之子,富貴足矣,何必以兄之位為欲居之,以到此乎云爾。錄所痛之辭也。故痛之中有痛,無罪而受其死者,申生、奚齊、卓子是也。惡之中有惡者,己立之,己殺之,不得如他臣之弒君者,齊公子商人是也。故晉禍痛而齊禍重。《春秋》傷痛而敦重,是以奪晉子繼位之辭與齊子成君之號,詳見之也。


古之人有言曰:不知來,視諸往。今《春秋》之為學也,道往而明來者也。然而其辭體天之微,故難知也。弗能察,寂若無;能察之,無物不在。是故為《春秋》者,得一端而多連之,見一空而博貫之,則天下盡矣。魯僖公以亂即位,而知親任季子。季子無恙之時,內無臣下之亂,外無諸侯之患,行之二十年,國家安寧。季子卒之後,魯不支鄰國之患,直乞師楚耳。僖公之情非輒不肖而國衰益危者,何也?以無季子也。以魯人之若是也,亦知他國之皆若是也。以他國之皆若是,亦知天下之皆若是也。此之謂連而貫之。故天下雖大,古今雖久,以是定矣。以所任賢,謂之主尊國安。所任非其人,謂之主卑國危。萬世必然,無所疑也。其在《易》曰:「鼎折足,覆公餗。」夫鼎折足者,任非其人也。覆公者,國家傾也。是故任非其人而國家不傾者,自古到今未嘗聞也。故吾按《春秋》而觀成敗,乃切於前世之興亡也。任賢臣者,國家之興也。夫知不足以知賢,無可奈何矣。知之不能任,大者以死亡,小者以亂危,其若是何邪?以莊公不知季子賢邪?安知病將死,召而授以國政。以殤公為不知孔父賢邪?安知孔父死,己必死,趨而救之。二主知皆足以知賢,而不決,不能任。故魯莊以危,宋殤以弒。使莊公早用季子,而宋殤素任孔父,尚將興鄰國,豈直免弒哉。此吾所而悲者也。

\article{王道}

《春秋》何貴乎元而言之?元者,始也,言本正也。道,王道也。王者,人之始也。王正則元氣和順、風雨時、景星見、黃龍下。王不正則上變天,賊氣並見。五帝三王之治天下,不敢有君民之心。什一而稅。教以愛,使以忠,敬長老,親親而尊尊,不奪民時,使民不過歲三日。民家給人足,無怨望忿怒之患,強弱之難,無讒賊妒疾之人。民修德而美好,被發銜哺而游,不慕富貴,恥惡不犯。父不哭子;兄不哭弟。毒蟲不螫,猛獸不搏,抵蟲不觸。故天為之下甘露,朱草生,醴泉出,風寸時,嘉禾興,鳳凰麒麟游於郊。囹圄空虛,書衣裳而民不犯。民情到樸而不文。郊天祀地,秩山川,以時到,封於泰山,禪於梁父。立明堂,宗祀先帝。以祖配天,天下諸侯各以其職來祭。貢土地所有,先以入宗廟,端冕盛服而後見先。德恩之報,奉先之應也。


桀紂皆聖王之後,驕溢妄行。侈宮室,廣苑囿,窮五采之變,極飭材之工,困野獸之足,竭山澤之列,食類惡之獸。奪民財食,高雕文刻鏤之觀,盡金玉骨象之工,窮白黑之變。深刑妄殺以陵下,聽鄭衛之音,充傾宮之志,靈虎文采之獸。以希見之意,嘗佞賜讒。以糟為丘,以酒為池。孤貧不養,殺聖賢而剖其心,生燔人聞其臭,剔孕婦見其化,朝涉之足察其拇,殺梅伯以為醢,刑鬼侯之女取其環。君臣畏恐,莫敢盡忠,紂愈自賢。周發兵,不期會於孟津者八百諸侯,共誅紂,大亡天下。《春秋》以為戒,曰:「蒲社災。」周衰,天子微弱,諸侯力政,大夫專國,士專邑,不能行度制法文之禮。諸侯背叛,莫修貢聘,奉獻天子。臣弒其君,子弒其父,孽殺其宗,不能統理,更相伐銼以廣地。以強相脅,不能制屬。強奄弱,眾暴寡,富使貧,並兼無已。臣下上僭,不能禁止。日為之食,星如雨,雨螽,沙鹿崩。夏大雨水,冬大雨雪,石於宋五,六退飛。霜不殺草,李梅實。正月不雨,到於秋七月。地震,梁山崩,壅河,三日不流。書晦。彗星見於東方,孛於大辰。鸛鵒來巢,《春秋》異之。以此見悖亂之徵。孔子明得失,差貴賤,反王道之本。譏天王以致太平。刺惡譏微,不遣小大,善無細而不舉,惡無細而不去,進善誅惡,絕諸本而已矣。


天王使宰喧來歸惠公仲子之賜,刺不及事也。天王伐鄭,譏親也,會王世子,譏微也。祭公來逆王後,譏失也。刺家父求車,武氏毛伯求賻金。王人救衛。王師敗於貿戎。天王不養,出居於鄭,殺母弟,王室亂,不能及外,無以先天下,召衛侯不能致,遣子突徵衛不能絕,無駭滅極不能誅。諸侯得以大亂,篡弒無已。臣下上逼,僭擬天子。諸侯強者行威,小國破滅。晉到三侵周,與天王戰於貿戎而大敗之。戎執凡伯於楚丘以歸。諸侯本怨隨惡,發兵相破,夷人宗廟社稷,不能統理。臣子強,到弒其君父。法度廢而不複用,威武絕而不複行。故鄭魯易地,晉文再致天子。齊桓會王世子,擅封邢、衛、杞,橫行中國,意欲王天下。魯舞八俏,北祭泰山,郊天祀地,如天子之為。以此之故,弒君三十二,細惡不絕之所致也。


《春秋》立義:祭天地,諸侯祭社稷,諸山川不在封內不祭。有天子在,諸侯不得專地,不得專封,不得專執天子之大夫,不得舞天子之樂,不得致天子之賦,不得適天子之貴。君親無將,將而誅。大夫不得廢置君命。立適,以長不以賢,立子以貴不以長。立夫人以適不以妾。親近以來遠,未有不先近而致遠者也。故內其國而外諸夏,內諸夏而外夷狄,言自近者始也。


諸侯來朝者得褒,邾妻儀父稱字,滕薛稱侯,荊得人,介葛盧得名。內出言如,諸侯來日朝,大夫來日聘,王道之意也。誅惡而不得遣細大,諸侯不得為匹夫興師,不得執天子之大夫,執天子之大夫與伐國同罪,執凡伯言伐。獻八俏,諱八言六。鄭魯易地,諱易言假。晉文再致天子,諱致言狩。桓公存邢、衛、杞,不見《春秋》,內心予之,行法絕而不予,止亂之道也,非諸侯所當為也。《春秋》之義,臣不討賊,非臣也。子不複仇,非子也。故誅趙盾賊不討者,不書葬,臣子之誅也。許世子止不嘗藥,而誅為弒父,楚公子比脅而立,而不免於死。齊桓晉文擅封,致天子,誅亂、繼絕、存亡,侵伐會同,常為本主。曰:桓公救中國,攘夷狄,卒服楚,晉文再致天子,皆止不誅,善其牧諸侯,奉獻天子而服周室,《春秋》予之為伯,誅意不誅辭之謂也。


魯隱之代桓立,祭仲之出忽立突,仇牧、孔父、荀息之死節,公子目夷不與楚國,此皆執權存國,行正世之義,守拳拳之心,《春秋》嘉氣義焉,故皆見之,複正之謂也。夷狄邾妻人、牟人、葛人,為其天王崩而相朝聘也,此其誅也。殺世子母弟直稱君,明失親親也。魯季子之免罪,吳季子之讓國,明親親之恩也。閽殺吳子餘祭,見刑人之不可近。鄭伯原卒於會,諱弒,痛強臣專君,君不得為善也。衛人殺州吁,齊人殺無知,明君臣之義,守國之正也。衛人立晉,美得眾也。君將不言率師,重君之義也。正月,公在楚,臣子思君,無一日無君之意也。誅受令,恩衛葆,以正囹圉之平也。言圍成,甲竿祠兵,以別迫脅之罪,誅意之法也。作南門。刻桷,丹楹,作雉門及兩觀。築三台,新延廄,譏驕溢不恤下也。故臧孫辰請於齊,孔子曰:「君子為國,必有三年之積。一年不熟乃請,失君之職也。誅犯始者,省刑,絕惡疾始也。大夫盟於澶淵,刺大夫之專政也。諸侯會同,賢為主,賢賢也。《春秋》紀纖芥之失,反之王道。追古貴信,結言而已,不到用牲盟而後成約。故曰:齊侯衛侯胥命於蒲。《傳》曰:「古者不盟,結言而退。」宋伯姬曰:「婦人夜出,傅母不在,不下堂。曰:古者周公東徵,則西國怨。桓公曰:「無貯粟,無鄣谷,無易樹子,無以妾為妻。」宋襄公曰:「不鼓不成列,不厄人。」莊王曰:「古者桿不穿,皮不蠹,則不出。」君子篤於禮,薄於利,要其人不要其土,告從不赦,不祥。強不陵弱。齊頃公吊死視疾,孔父正色而立於朝,人莫過而致難乎其君,齊國佐不辱君命而尊齊侯,此《春秋》之救文以質也。救文以質,見天下諸侯所以失其國者亦有焉。潞子欲合中國之禮義,離乎夷狄,未合乎中國,所以亡也。吳王夫差行強於越,臣人之主,妾人之妻,卒以自亡,宗廟夷,社稷滅。其可痛也。長王投死,於戲,豈不哀哉!晉靈行無禮,處台上彈君臣,枝解宰人而棄之,漏陽處父之謀,使陽處父死。及患趙盾之諫,欲殺之,卒為趙盾所弒。晉獻公行逆理,殺世子申生以驪姬立奚齊、卓子,皆殺死,國大亂,四世乃定,幾為秦所滅,從驪姬起也。楚平王行無度,殺伍子胥父兄。蔡昭公朝之,因請其裘,昭公不與。吳王非之。舉兵加楚,大敗之。君舍乎君室,大夫舍乎大夫室,妻楚王之母,貪暴之所致也。晉厲公行暴道,殺無罪人,一朝而殺大臣三人。明年,臣下畏恐,晉國殺之。陳侯佗淫乎蔡,蔡人殺之。古者諸侯出疆必具左右,備一師,以備不虞。今陳侯恣以身出入民間,到死閭裡之庸,甚非人君之行也。宋閔公矜婦人而心妒,與大夫萬博。萬與魯莊公曰:「天下諸侯宜為君者,唯魯侯爾。」閔公妒其言,曰:「此虜也,爾虜焉故。魯侯之美惡乎到?」萬怒,搏閔公絕。此以與臣博之過也。古者人君立於陰,大夫立於陽,所以別位,明貴賤。今與臣相對而博,置婦人在側,此君臣無別也。故使萬稱他國卑閔公之意,閔公藉萬而身與之博,下君自置。有辱之婦人之房,俱而矜婦人,獨得殺死之道也。《春秋傳》曰:「大夫不適君。」遠此逼也。梁內役民無已。其民不能堪,使民比地為伍,一家亡,五家殺刑。其民曰:先亡者封,後亡者刑。君者將使民以孝於父母,順於長老,守丘墓,承宗廟,世世祀其先。今求財不足,行罰如將不勝,殺戮如屠,仇仇其民,魚爛而亡,國中盡空。《春秋》曰:「梁亡。」亡者自亡也,非人亡之也。虞公貪財,不顧其難,愉耳悅目,受晉之璧、屈產之乘,假晉師道,還以自滅。宗廟破毀,社稷不祀,身死不葬,貪財之所致也。故《春秋》以此見物不空來,寶不虛出,自內出者,無匹不行,自外到者,無主不止,此其應也。楚靈王行強乎陳蔡,意廣以武,不顧其行,虜所美,內罷其眾。乾溪有物女,水盡則女見,水滿則不見。靈王舉發其國而役,三年不罷,楚國大怨。殺無罪臣成然,公子棄疾卒令靈王父子自殺而取其國。虞不離津澤,農不去疇土,而民相愛也。此非盈意之過耶?魯莊公好宮室,一年三起台。夫人內淫兩弟,國絕莫繼,為齊所存,夫人淫之過也。妃匹貴妾,可不慎邪?此皆內自強從心之敗己,見自強之敗,尚有正諫而不用,卒皆取亡。曹羈諫其君曰:「戎眾以無義,君無自適。」君不聽,果死戎寇。伍子胥諫吳王,以為越不可不敢。吳王不聽,到死伍子胥。還九年,越果大滅吳國。秦穆公將襲鄭,百裡、蹇叔諫曰:「千裡而襲人者,未有不亡者也。」穆公不聽。師果大敗中,匹馬只輪無反者。晉假道虞,虞公許之。宮之奇諫曰:「唇亡齒寒,虞虢之相救,非相賜也。君請勿許。」虞公不聽,後虞果亡於晉。《春秋》明此,存亡道可觀也。觀乎蒲社,知驕溢之罰。觀乎許田,知諸侯不得專封。觀乎齊桓、晉文、宋襄、楚莊,知任賢奉上之功。觀乎魯隱、祭仲、叔武、孔父、荀息、仇牧、吳季子、公子目夷,知忠臣之效。觀乎楚公子比,知臣子之道,效死之義。觀乎潞子,知無輔自詛之敗。觀乎公在楚,知臣子之恩。觀乎漏言,知忠道之絕。觀乎獻六羽,知上下之差。觀乎宋伯姬,知貞婦之信。觀乎吳王夫差,知強陵弱。觀乎晉獻公,知逆理近色之過。觀乎楚昭王之伐蔡,知無義之反。觀乎晉厲之妄殺無罪,知行暴之報。觀乎陳佗宋閔,知妒淫之禍。觀乎虞公、梁亡,知貪財枉法之窮。觀乎楚靈,知苦民之壤。觀乎魯莊之起台,知驕奢淫溢之失。觀乎衛侯朔,知不即召之罪。觀乎執凡伯,知犯上之法。觀乎晉缺之伐邾妻,知臣下作福之誅。觀乎公子,知臣窺君之意。觀乎世卿,知移權之敗。故明王視於冥冥,聽於無聲,天覆地載,天下萬國,莫敢不悉靖春職受命者,不示臣下以知之到也。故道同則不能相先,情同則不能相使,此其教也。由此觀之,未有去人君之權,能制其勢者也;未有貴賤無差,能全其位者也。故君子慎之。

\article{滅國上}

王者,民之所往。君者,不失其群者也。故能使萬民往之,而得天下之群者,無敵於天下。弒君三十六,亡國五十二。小國德薄,不朝聘大國,不與諸侯會聚,孤特不相守,獨居不同群,遭難莫之救,所以亡也。非獨公侯大人如此,生天地之間,根本微者,不可遭大風疾雨,立鑠消耗。衛侯朔固事齊襄,而天下患之,虞虢並力,晉獻難之。晉趙盾,一夫之士也,無尺寸之土,一介之眾也。而靈公據霸主之余尊,而欲誅之,窮變極詐,詐盡力竭,祝大及身。推盾之心,載小國之位,孰能亡之哉?故伍子胥,一夫之士也,去楚干闔廬,遂得意於吳。所托者誠是,何可御邪?楚王髡托其國於子玉得臣,而天下畏之。虞公托其國於宮之奇,晉獻患之。及髡殺得臣,天下輕之,虞公不用宮之奇,晉獻亡之。存亡之端,不可不知也。諸侯見加以兵,逃遁奔走,到於滅亡而莫之救,平生之素行可見也。隱代桓立,所謂僅存耳,使無駭帥師滅極,內無諫臣,外無諸侯之救;載亦由是也,宋、蔡、衛國伐之,鄭因勘和而取之。此無以異於遣重寶於道而莫之守,見者掇之也。鄧、失地而朝魯桓,鄧、彀失地,不亦宜乎?

\article{滅國下}

紀侯之所以滅者,乃九世之仇也。一旦之言,危百世之嗣,故曰大去。衛人侵成,鄭入成,及齊師圍成,三被大兵,終滅,莫之救,所恃者安在?齊桓公欲行霸道,譚遂違命,故滅而奔莒。不事大而事小,曹伯之所以戰死於位。諸侯莫助憂者。幽之會,齊桓數合諸侯,曹小,未嘗來也。魯大國,幽之會,莊公不往。戎人乃窺兵於濟西,由見魯孤獨而莫之救也。此時大夫廢君命,專救危者。魯莊公二十七年,齊桓為幽之會,衛人不來。其明年,桓公怒而大敗之。及伐山戎,張旗陳獲以驕諸侯。於是魯一年三築台,亂臣比三起於內,夷狄之兵仍滅於外,衛滅之端,以失幽之會。亂之本,存親內蔽。邢未嘗會齊桓也,附晉又微,晉侯獲於韓而背之,淮之會是也。齊桓卒,豎刁易牙之亂作。邢與狄伐其同姓,取之。其行如此,雖爾親,庸能親爾乎?是君也,其滅於同姓,衛侯毀滅邢是也。齊桓為幽之會,衛不到,桓怒而伐之。狄滅之,桓憂而立之。魯莊為柯之盟,劫汶陽,魯絕,桓立之。邢杞未嘗朝聘,齊桓見其滅,率諸侯而立之,用心如此,豈不霸哉?故以憂天下與之。

\article{隨本消息}

顏淵死,子曰:「天喪予。」子路死,子曰:「天祝予。」西狩獲麟,曰:「吾道窮,吾道窮。」三年,身隨而卒。天命成敗,聖人知之,有所不能救,命矣夫。

先晉獻之卒,齊桓為葵丘之會,再致其集。先齊孝未卒一年,魯僖乞師取。晉文之威,天子再致。先卒一年,魯僖公之心,分而事齊。文公不事晉。先齊侯潘卒一年,文公如晉,衛侯鄭伯皆不期來。齊侯已卒,諸侯果會晉大夫於新城。魯昭公以事楚之故,晉人不入。楚國強而得意,一年再會諸侯,伐強吳,為齊誅亂臣,遂滅厲。魯得其威以滅其明年,如晉,無河上之難。先晉昭之卒一年,無難。楚國內亂,臣弒君。諸侯會於平丘,謀誅楚亂臣,昭公不得與盟,大夫見執。吳大敗楚之黨六國於雞父。公如晉而大辱,《春秋》為之諱而言有疾。由此觀之,所行從不足恃,所事者不可不慎。此亦存亡榮辱之要也。先楚莊王卒之三年,晉滅赤狄潞氏及甲氏留吁。先楚子審卒之三年,鄭服蕭魚。晉侯周卒一年,先楚子昭卒之二年,與陳蔡伐鄭而大克。其明年,楚屈建會諸侯而張中國。卒之三年,諸夏之君朝於楚。楚子卷繼之,四年而卒。其國不為侵奪,而顧隆盛強大,中國不出年余,何也?楚子昭蓋諸侯可者也,天下之疾其君者,皆赴而乘之。兵四五出,常以眾擊少,以專擊散,義之盡也。先卒四五年,中國內乖,齊、晉、魯、衛之兵分守,大國襲小。諸夏再會陳儀,齊不肯往。吳在其南,而二君殺,中國在其北,而齊衛殺其君,慶封劫君亂國,石惡之徒聚而成群,衛據陳儀而為諼。林父據戚而以畔,宋公殺其世子,魯大饑。中國之行,亡國之跡也。譬如於文宣之際,中國之君,五年之中五君殺。以晉靈之行,使一大夫立於斐林,拱揖指揮,諸侯莫敢不出,此猶隰之有泮也。

\article{盟會要}

至意雖難喻,蓋聖人者貴除天下之患。貴除天下之患,故《春秋》重,而書天下之患遍矣。以為本於見天下之所以致患,其意欲以除天下之患,何謂哉?天下者無患,然後性可善;性可善,然後清廉之化流;清廉之化流,然後王道舉。禮樂興,其心在此矣。《傳》曰:諸侯相聚而盟。君子修國曰:此將率為也哉。是以君子以天下為憂也,患乃至於弒君三十六,亡國五十二,細惡不絕之所致也。辭已喻矣,故曰:立義以明尊卑之分,強干弱枝以明大小這職;別嫌疑之行,以明正世之義;采摭托意,以矯失禮。善無小而不舉,無惡小而不去,以純其美。別賢不肖以明其尊。親近以來遠,因其國而容天下,名倫等物不失其理。公心以是非,賞善誅惡而王澤洽,始於除患,正一而萬物備。故曰大矣哉其號,兩言而管天下。此之謂也。

\article{正貫}

《春秋》,大義之所本耶?六者之科,六者之旨之謂也。然後援天端,布流物,而貫通其理,則事變散其辭矣。故志得失之所從生,而後差貴賤之所始矣。論罪源深淺,定法誅,然後絕屬之分別矣。立義定尊卑之序,而後君臣之職明矣。載天下之賢方,表謙義之所在,則見復正焉耳。幽隱不相踰,而近之則密矣。而後萬變之應無窮者,故可施其用於人,而不悖其倫矣。是以必明其統於施之宜,故知其氣矣,然後能食其志也;知其聲矣,而後能扶其精也。知其行矣,而後能遂其形也;知其物矣,然後能別其情也。故唱而民和之,動而民隨之,是知引其天性所好,而厭其情之所憎者也。如是則言雖約,說必布矣;事雖小,功必大矣。聲響盛化運於物,散入於理,德在天地,神明休集,並行而不竭,盈於四海而訟聲詠。《書》曰:「八音克諧,無相奪倫,神人以和。」乃是謂也。故明於情性乃可與論為政,不然,雖勞無功。夙夜是寤,思慮惓心,猶不能睹,故天下有非者。三示當中孔子之所謂非,尚安知通哉!

\article{十指}

《春秋》二百四十二年之文,天下之大,事變之博,無不有也。雖然,大略之要有十指。十指者,事之所擊也,王化之所由得流也。舉事變見有重焉,一指也。見事變之所至者,一指也。因其所以至者而治之,一指也。強干弱枝,大本小末,一指也。別嫌疑,異同類,一指也。論賢才之義,別所長之能,一指也。親近來遠,同民所欲,一指也。承周文而反之質,一指也。木生火,火為夏,天之端,一指也。切刺譏之所罰,考變異之所加,天之端,一指也。舉事變見有重焉,則百姓安矣。見事變之所至者,則得失審矣。因其所以至而治之,則事之本正矣。強干弱枝,大本小末,則君臣之分明矣。別嫌疑,異同類,則是非著矣。論賢才之義,別所長之能,則百官序矣。承周文而反之質,則化所務立矣。親近來遠,同民所欲,則仁恩達矣。木生火,火為夏,則陰陽四時之理相受而次矣。切刺譏之所罰,考變異之所加,則天所欲為行矣。統此而舉之,德澤廣大,衍溢於四海,陰陽和調,萬物靡不得其理矣。說《春秋》者凡用是矣,此其法也。

\article{重政}

惟聖人能屬萬物於一而擊之元也,終不及本所從來而承之,不能遂其功。是以《春秋》變一謂之元,元猶原也,其義以隨天地終始也。故人惟有終始也而生,不必應四時之變,故元者為萬物之本,而人之元在焉。安在乎?乃在乎天地之前。故人雖生天氣及奉天氣者,不得與天元本、天元命而共達其所為也。故春正月者,承天地之所為也,繼天之所為而終之也,其道相與共功持業,安容言乃天地之元。天地之元奚為於此,惡施於人,大其貫承意之理矣。

能說鳥獸之類者,非聖人所欲說也。聖人所欲說,在於說仁義而理之,知其分科條別,貫所附,是乃聖人之所貴而已矣。不然,傳於眾辭,觀於眾物,說不急之言而以惑後進者,君子之所甚惡也。奚以為哉?聖人思慮不厭,書日繼之以夜,然後萬物察者,仁義矣。由此言之,尚自為得之哉。故曰:於乎!為人師者,可無慎邪!夫義出於經,經傳,大本也。棄營勞心也,苦誌盡情,頭白齒落,尚不合自錄也哉?

人始生有大命,是其體也。有變命存其間者,其政也。政不齊則人有忿怒之志,若將施危難之中,而時有隨、遭者,神明之所接,絕續之符也。亦有變其間,使之不齊如此,不可不省之,省之則重政之本矣。進義誅惡絕之本,而以其施,此舉湯武同而有異。湯武用之治往故。《春秋》明得失,差貴賤,本之天。王之所失天下者,使諸侯得以大亂之,說而後引而反之。故曰博而明,深而切矣。

\article{服制像}

天地之生萬物也以養人,故其可適者以養身體,其可威者以為容服,禮之所為同也。劍之在左,青龍之象也。刀之在右,白虎之象也。韍之在前,朱鳥之象也。冠之在首,玄武之象也。四者,人之盛飾也。夫能通古今,別然不然,乃能服此也。蓋玄武者,貌之最嚴有威者也,其像在後,其服反居首,武之至而不用矣。聖人之所以超然,雖欲從之,末由也已。夫執介冑而後能拒敵者,故非聖人之所貴也。君子顯之於服,而勇武者消其誌於貌也矣。故文德為貴,而威武為下,此天下之所以永全也。於《春秋》何以言之?孔父義形於色,而奸臣不敢容邪;虞有宮之奇,而獻公為之不寐;晉厲之強,中國以寢尸流血不已。故武王克殷,裨冕而笏。虎賁之王說劍,安在勇猛必任武殺然後威。是以君子所服為上矣,故望之儼然者,亦已至矣,豈可不察乎!

\article{二端}

《春秋》至意有二端,不本二端之所從起,亦未可與論異也,小大微著之分也。夫覽求微細於無端之處,誠知小之將為大也,微之將為著也。吉凶未形,聖人所獨立也,雖欲從之,末由也已,此之謂也。故王者受命,改正朔,不順數而往,必迎來而受之者,授受之義也。故聖人能擊心於微而致之著也。是故《春秋》之道,以元之深正天之端,以天之端正王之政,以王之政正諸侯之即位,以諸侯之即位正竟內之治,五者俱正而化大行。故書日蝕、星隕、有蜮、地震、夏大雨水、冬大雨雹、隕霜不殺草、自正月不雨至於秋七月、有鵒來巢,《春秋》異之,以此見悖亂之徵。是小者不得大,微者不得著,雖甚末,亦一端。孔子以此效之,吾所以貴微重始是也。因惡夫推災異之象於前,然後圖安危禍亂於後者,非《春秋》之所甚貴也。然而《春秋》舉之以為一端者,亦欲其省天譴而畏天威,內動於心誌,外見於事情,修身審己,明善心以反道者也,豈非貴微重始、慎終推效者哉!

\article{符瑞}

有非力之所能致而自至者,西狩獲麟,受命之符是也。然後托乎《春秋》正不正之間,而明改制之義。一統乎天子,而加憂於天下之憂也,天下所患。而欲以上通五帝,下極三王,以通百王之道,而隨天之終始,博得失之效,而考命象之為,極理以盡情性之宜,則天容遂矣。百官同望異路,一之者在主,率之者在相。

\article{俞序}

仲尼之作春秋也,上探正天端,王公之位,萬物民之所欲,下明得失,起賢才,以待後聖。故引史記,理往事,正是非,見王公。史記十二公之間,皆衰世之事,故門人惑。孔子曰:「吾因其行事而加乎王心焉。」以為見之空言,不如行事博深切明。故子貢、閔子、公肩子,言其切而為國家資也。其為切而至於殺君亡國,奔走不得保社稷,其所以然,是皆不明於道,不覽於《春秋》也。故衛子夏言,有國家者不可不學《春秋》,不學《春秋》,則無以見前後旁側之危,則不知國之大柄,君之重任也。故或脅窮失國,搶殺於位,一朝至爾。苟能述《春秋》之法,致行其道,豈徒除禍哉,乃堯舜之德也。故世子曰:「功及子孫,光輝百世,聖人之德,莫美於恕。」故予先言《春秋》詳己而異人,《春秋》之道,大得之則以王,小得之則以霸。故曾子、子石盛美齊侯,安諸侯,尊天子。霸王之道,皆本於仁。仁,天心,故次以天心。愛人之大者,莫大於思患而豫防之,故蔡得意於吳,魯得意於齊,而《春秋》皆不告,故次以言怨人不可邇,敵國不可狎,攘竊之國不可使久親,皆防患為民除患之意也。不愛民之漸乃至於死亡,故言楚靈王晉厲公生弒於位,不仁之所致也。故善宋襄公不厄人,不由其道而勝,不如由其道而敗,《春秋》貴之,將以變習俗而成王化也。故子夏言《春秋》重人,諸譏皆本此。或奢侈使人憤怨,或暴虐賊害人,終皆禍及身。故子池言魯莊築台,丹楹刻桷,晉厲之刑刻意者,皆不得以壽終。上奢侈,刑又急,皆不內恕,求備於人,故次以《春秋》緣人情,赦小過,而《傳》明之曰:「君子辭也。」孔子明得失,見成敗,疾時世之不仁,失王道之體,故緣人情,赦小過,《傳》又明之曰:「君子辭也。」孔子曰:「吾因行事,加吾王心焉。」假其位號以正人倫,因其成敗以明順逆,故其所善,則桓文行之而遂,其所惡,則亂國行之終以敗,故始言大惡殺君亡國,終言赦小過,是亦始於麤粗,終於精微,教化流行,德澤大洽,天下之人,人有士君子之行而少過矣,亦譏二名之意也。

\article{離合根}

天高其位而下其施,故為人主者,法天之行,是故內深藏,所以為神;外博觀,所以為明也;任群賢,所以為受成;乃不自勞於事,所以為尊也;凡愛群生,不以喜怒賞罰,所以為仁也。故為人主者,以無為為道,以不私為寶。立無為之位而乘備具之官,足不自動而相者導進,口不自言而擯者贊辭,心不自慮而群臣效當,故莫見其為之而功成矣。此人主所以法天之行也。為人臣者法地之道,暴其形,出其情以示人,高下、險易、堅耍、剛柔、肥、美惡,累可就財也。故其形宜不宜,可得而財也。為人臣者比地貴信而悉見其情於主,主亦得而財之,故王道威而不失。為人臣常竭情悉力而見其短長,使主上得而器使之,而猶地之竭竟其情也,故其形宜可得而財也。

\article{立元神}

君人者,國之元,發言動作,萬物之樞機。樞機之發,榮辱之端也。失之豪厘,駟不及追。故為人君者,謹本詳始,敬小慎微,誌如死灰,安精養神,寂莫無為。休形無見影,搶聲無出音,虛心下士,觀來察往。謀於眾賢,考求眾人,得其心遍見其情,察其好惡,以參忠佞,考其往行,驗之於今,計其蓄積,受於先賢。釋其讎怨,視其所爭,差其黨族,所依為臬,據位治人,用何為名,累日積久,何功不成。可以內參外,可以小佔大,必知其實,是謂開闔。君人者,國之本也。夫為國,其化莫大於崇本,崇本則君化若神,不崇本則君無以兼人。無以兼人,雖峻刑重誅,而民不從,是所謂驅國而棄之者也,患孰甚焉?何謂本?曰:天地人,萬物之本也。天生之,地養之,人成之。天生之以孝悌,地養之以衣食,人成之以禮樂,三者相為手足,合以成禮,不可一無也。無孝悌則亡其所以生,無衣食則亡其所以養,無禮樂,則亡其所以成也。三者皆亡,則民如麋鹿,各從其欲,家自為俗。父不能使子,君不能使臣,雖有城郭,名曰虛邑。如此,其君枕塊而僵,莫之危而自危,莫之喪而自亡,是謂自然之罰。自然之罰至,裹襲石室,分障險阻,猶不能逃之也。明主賢君必於其信,是故肅慎三本。郊祀致敬,共事祖禰,舉顯孝悌,表異孝行,所以奉天本也。秉耒躬耕,采桑親蠶,墾草殖彀,開闢以足衣食,所以奉地本也。立闢雍庠序,修孝悌敬讓,明以教化,感以禮樂,所以奉人本也。三者皆奉,則民如子弟,不敢自專,邦如父母,不待恩而愛,不須嚴而使,雖野居露宿,厚於宮室。如是者,其君安枕而臥,莫之助而自強,莫之綏而自安,是謂自然之賞。自然之賞至,雖退讓委國而去,百姓襁負其子隨而君之,君亦不得離也。故以德為國者,甘於飴蜜,固於膠漆,是以聖賢勉而崇本而不敢失也。君人者,國之證也,不可先倡,感而後應。故居倡之位而不行倡之勢,不居和之職而以和為德,常盡春下,故能為之上也。

體國之道,在於尊神。尊者所以奉其政也,神者所以就其化也,故不尊不畏,不神不化。夫欲為尊者在於任賢,欲為神者在於同心。賢者備股肱則君尊嚴而國安,同心相承則變化若神,莫見其所為而功德成,是謂尊神也。

天積眾精以自剛,天序日月星辰以自光,聖人序爵祿以自明。天所以剛者,非一精之力;聖人所以強者,非一賢之德也。故天道務盛其精,聖人務眾其賢。盛其精而壹其陽,眾其賢而同其心。壹其陽然後可以致其神,同其心然後可以致其功。是以建治之術,貴得賢而同心。為人君者,其要貴神。神者,不可得而視也,不可得而聽也,是故親而不見其形,聽而不聞其聲。聲之不聞,故莫得其響,不見其形,故莫得其影。莫得其影則無以曲直也,莫得其響則無以清濁也。無以曲直則其功不可得而敗,無以清濁則其名不可得而度也。所謂不見其形者,非不見其進止之形也,言其所以進止不可得而見也。所謂不聞其聲者,非不聞其號令之聲也,言其所以號令不可得而聞也。不見不聞,是謂冥昏。能冥則明,能昏則彰。能冥能昏,是謂神人。君貴居冥而明其位,處陰而向陽。惡人見其情而欲知人之心,是故為人君者執無源之慮,行無端之事,以不求奪,以不問問。吾以不求奪則我利矣,彼以不出出則彼費矣。吾以不問問則我神矣,彼以不對對則彼情矣。故終日問之,彼不知其所對,終日奪之,彼不知其所出。吾則以明而彼不知其所亡。故人臣居陽而為陰,人君居陰而為陽。陰道尚形而露情,陽道無端而貴神。

\article{保位權}

民無所好,君無以權也。民無所惡,君無以畏也。無以權,無以畏,則君無以禁制也。無以禁制,則比肩齊勢而無以為貴矣。故聖人之治國也,因天地之性情,孔窮之所利,以立尊卑之制,以等貴賤之差。設官府爵祿,利五味,盛五色,調五聲,以誘其耳目,自令清濁昭然殊體,榮辱踔然相駁,以感動其心,務致民令有所好。有所好然後可得而勸也。既有所勸,又有所畏,然後可得而制。制之者,制其所好,是以勸賞而不得多也。制其所惡,是以畏罰而不可過也。所好多則作福,所惡多則作威。作威則君亡權,天下相怨;作福則君亡德,天下相賤。故聖人之制民,使之有欲,不得過節;使之敦樸,不得無欲。無欲有欲,各得以足,而君道得矣。國之所以為國者德也,君之所以為君者威也,故德不可共,威不可分。德共則失恩,威分則失權。失權則君賤,失恩則民散。民散則國亂,君賤則臣叛。是故為人君者,固守其德,以附其民;固執其權,以正其臣。聲有順逆,必有清濁,形有善惡,必有曲直。故聖人聞其聲則別其清濁,見其形則異其曲直。於曲之中,必見其直;於直之中,必見其曲。於聲無小而不取,於形無小而不舉。不以著蔽微,不以眾掩寡,各應其事以致其報。黑白分明,然後民知所去就,民知所去就,然後可以致治,是為象則。為人君者居無為之位,行不言之教,寂而無聲,靜而無形,執一無端,為國源泉。因國以為身,因臣以為心。以臣言為聲,以臣事為形。有聲必有響,有形必有影。聲出於內,響報於外;形立於上,影應於下。響有清濁,影有曲直,響所報非一聲也,影所應非一形也。故為君虛心靜處,聰聽其響,明視其影,以行賞罰之象。其行賞罰也,響清則生清者榮,響濁則生濁者辱,影正則生正者進,影枉則生枉者絀。擊名考質,以參其實。賞不空施,罰不虛出。是以君臣分職而治,各敬而事,爭進其功,顯廣其名,而人君得載其中,此自然致力之術也。聖人由之,故功出於臣,名歸於君也。


\article{考功名}

考之法。考其所積也。天道積聚眾精以為光,聖人積聚眾善以為功。故日月之明,非一精之濼也;聖人致太平,非一善之功也。明所從生,不可為源,善所從出,不可為端,量勢立權,因事制義。故聖人之為天下同利也,其猶春氣之生草也,各因其生小大而量其多少,各順其勢,傾側而制於南北。故異孔而同歸,殊施而鈞德,其趣於同利除害一也。是以同利之要在於致之,不在於多少;除害之要在於去之,不在於南北。考絀陡,計事除廢,有益者謂之公,名責實,不得虛言,有功者賞,有罪者罰,功盛者賞顯,罪多者罰重。不能致功,雖有賢名不予之賞;官職不廢,雖有愚名,不加之罰。賞罰用於實,不用於名,賢愚在於質,不在於文。故是非不能混,喜怒不能傾,奸軌不能弄,萬物各得其冥,則百官勸職,爭進其功。

考試之法,大者緩,小者急,貴者舒而賤者促。諸侯月試其國,州伯時試其部,四試而一考。天子歲試天下,三試而一考,前後三考而絀陟,命之曰計。

考試之法,合其爵祿,並其秩,積其日,陳其實,計功量罪,以多除少,以名定實,先內弟之。其先比二三分以為上中下,以考進退,然後外集。通名曰進退,增減多少,有率為弟。九分三三列之,亦有上中下,以一為最,五為中,九為殿。有余歸之於中,中而上者有得,中而下者有負。得少者以一益之,至於四,負多者以四減之,至於一,皆逆行。三四十二而成於計,得滿計者絀陟之。次次每計,各逐其弟,以通來數。初次再計,次次四計,各不失故弟,而亦滿計絀陟之。

初次再計,謂上弟二也。次次四計,謂上弟三也。九年為一弟,二得九,並去其六,為置三弟,六六得等,為置二,並中者得三盡去之,並三三計得六,並得一計得六,此為四計也。絀者亦然。


\article{通國身}

氣之清者為精,人之清者為賢。治身者以積精為寶,身以心為本,國以君為主。精積於其本,則血氣相承受;賢積於其主,則上下相制使。血氣相承受,則形體無所苦;上下相制使,則百官各得其所。形體無所苦,然後身可得而安也;百官各得其所,然後國可得而守也。夫欲致精者,必虛靜其形;欲致賢者,必卑謙其身。形靜誌虛者,精氣之所趣也;謙尊自卑者,仁賢之所事也。故治身者務執虛靜以致精,治國者務盡卑謙以致賢。能致精則合明而壽,能致賢則德澤洽而國太平。

\article{三代改制質文}

《春秋》曰「王正月」,《傳》曰:「王者孰謂?謂文王也。曷為先言王而後言正月?王正月也。何以謂之王正月?曰:王者必受命而後王。王翥和改正朔,易服色,制禮樂,一統於天下,所以明易姓,非繼人,通以己受之於天也。王者受命而王,制此月以應變,故作科以奉天地,故謂之王正月也。王者改制作科奈何?曰:當十二色,歷各法而正色,絀三之前曰五帝,帝迭首一色,順數五而相複,咸作國號,遷宮邑,易官名,制禮作樂。故湯受命而王,應天變夏作殷號,時正白統。親夏故虞,絀唐謂之帝堯,以神農為赤帝。作宮邑於下洛之陽,名相官曰尹。作漢樂,制質禮以奉天。文王受命而王,應天變殷作周號,時正赤統。親殷故夏,絀虞謂之帝舜,以軒轅為黃帝,推神農以為九皇。作宮邑於豐。名相官曰宰。作武樂,制文禮以奉天。武王受命,作宮邑於,制爵五等,作象樂,繼文以奉天。周公輔成王受命,作宮邑於洛陽,成文武之制,作汋樂以奉天。殷湯之後稱邑,示天之變反命。故天子命無常。唯命是德慶。故《春秋》應天作新王之事,時正黑統。王魯,尚黑,絀夏,親周,故宋。樂宜親招武,故以虞錄親,樂制宜商,合伯子男為一等。然則其略說奈何?曰:三正以黑統初。正日月朔於營室,斗建寅。天統氣始通化物,物見萌達,其色黑。故朝正服黑,首服藻黑,正路輿質黑,馬黑,大節綬幟尚黑,郊牲黑,冠於阼,昏禮逆於庭,喪禮殯於東階之上。祭牲黑牡,樂器黑質。法不刑有懷任新產,是月不殺。聽朔廢刑發德,具存二王之後也。親赤統,故日分平明,平明朝正。正白統奈何?曰:正白統者,歷正日月朔於虛,斗建丑。天統氣始蛻化物,物始芽,其色白,故朝正服白,首服藻白,正路輿質白,大節綬幟尚白,旗白,大寶玉白,郊牲白,犧牲角繭。冠於堂,昏禮逆於堂,喪事殯於楹柱之間。祭牲白牡,薦尚肺。樂器白質。法不刑有身懷任,是月不殺。聽朔廢刑發德,具存二王之後也。親黑統,故日分鳴晨,鳴晨朝正。正赤統奈何?曰:正赤統者,歷正日月朔於牽牛,斗建子。天統氣始施化物,物始動,其色赤,故朝正服赤,首服藻赤,正路輿質赤,馬赤,大節綬,幟尚赤,旗赤,大寶玉赤,郊牲,犧牲角栗。冠於房,喪禮殯於西階之上。祭牲牡,薦尚心。樂器赤質。法不刑有身,重懷藏以養微,是月不殺。聽朔廢刑發德,具存二王之後也。親白統,故日分夜半,夜半朝正。改正之義,奉元而起。古之王者受命而王,改制稱號正月,服色定,然後郊告天地及群神,遠追祖安道爾,然後布天下。諸侯廟受,以告社稷宗廟山川,然後感應一其司。三統之變,近夷遐方無有,生煞者獨中國。而三代改正,必以三統天下。曰:三統五端,化四方之本也。天始廢始施,地必待中,是故三代必居中國。法天奉本,執端要以統天下,朝諸侯也。是以朝正之義,天子純統色衣,諸侯統衣纏緣紐,大夫士以冠,參近夷以綏,遐方各衣其服而朝,所以明乎天統之義也。其謂統三正者,曰:正者,正也,統致其氣,萬物皆應,而正統正,其余皆正,凡歲之要,在正月也。法正之道,正本而末應,正內而外應,動作舉錯,靡不變化隨從,可謂法正也。故君子曰:「武王其似正月矣。」《春秋》曰:「杞柏來朝。」王者之後稱公,杞何以稱伯?《春秋》上絀夏,下存周,以《春秋》當新王。《春秋》當新王者奈何?曰:王者之法,必正號,絀王謂之帝,封其後以小國,使奉祀之。下存二王之後以大國,使服其服,行其禮樂,稱客而朝。故同時稱帝者五,稱王者三,所以昭五端,通三統也。是故周人之王,尚推神農為九皇,而改號軒轅謂之黃帝,因存帝顓頊、帝嚳、帝堯之帝號,絀虞而號舜曰帝舜,錄五帝以小國。下存禹之後於杞,存湯之後於宋,以方百裡,爵號公。使服其服,行其禮樂,稱先王客而朝。《春秋》作新王之事,變周之制,當正黑統。而殷周為王者之後,絀夏改號禹謂之帝,錄其後以小國,故曰絀夏存周,以《春秋》當新王。不以杞侯,弗同王者之後也。稱子又稱伯何?見殊之小國也。黃帝之先謚,四帝之後謚,何也?曰:帝號必存五,帝代首天之色,號至五而反。周人之王,軒轅直首天黃號,故曰黃帝雲。帝號尊而謚卑,故四帝後謚也。帝,尊號也,錄以小何?曰:遠者號尊而地小,近者號卑而地大,親疏之義也。故王者有不易者,有三而複者,有四而複者,一朋而複者,有九而複者,明此通天地、陰陽、四時、日月、星辰、山川、人倫,德侔天地者稱皇帝,天佑而子之,號稱天子。故聖王生則稱天子,崩遷則存為三王,絀滅則為五帝,下至附庸,絀為九皇,下極其為民。有一謂之三代,故雖絕地,宗於代宗。故曰:聲名魂魄施於虛,極壽無疆。何謂再而複,四而複?《春秋》鄭忽何以名?《春秋》曰:伯子男一也,辭無所貶。何以為一?曰:周壽五等,《春秋》三等。《春秋》何三等?曰:王者以制,一商一夏,一質一文。商質者主天,夏文者主地,《春秋》者主人,故三等也。主天法商而王,其道佚陽,親親而多仁樸。故立嗣予子,篤母第,妾以子貴。昏冠之禮,字子以父。夫婦,對坐而食,喪禮別葬,祭禮先臊,夫妻昭穆別位。制爵三等,祿士二品。制郊宮明堂員,其屋高嚴侈員,惟祭器員。玉厚九恰好,白藻五絲,衣制大上,首服嚴員。驚輿尊蓋,法天列象,垂四驚。用錫舞,舞溢員。先毛血而後用聲。正刑多隱,親戚多諱。封禪於尚位。主地法夏而王,其道進陰,尊尊而多義節。故立嗣與孫,篤世子,妾不以子稱貴號。昏冠之禮,字子以母。別眇夫婦,同坐而食,喪禮合葬,祭禮先亨,婦從夫為昭穆。制爵五等,祿十三品。制郊宮明堂方,其屋卑污方,祭器方。玉厚八分,白藻四絲,衣制大下,首服卑退。驚輿卑,法地周象載,垂二驚。樂設鼓,用織施舞,舞溢方。先亨而後用聲。正刑天法,封壇於下位。主天法質而王,其道佚陽,故立嗣予子,篤母弟,妾以子貴。昏冠之禮,字子以父。別眇夫婦,對坐而食,喪禮別葬,祭禮先嘉疏,夫婦昭穆別位。制爵三等,祿士二品。制郊宮明堂內員外橢,其屋如倚靡員橢,祭器橢。玉厚七分,白藻三絲,衣長前衽,首服員轉。驚輿尊蓋,備天列象,垂四驚。樂鼓,用羽龠舞,舞溢橢。先用玉聲而後烹,正刑多隱,親戚多赦。封壇於左位。主地法文而王,其道進陰,尊尊而多禮文。故立嗣予孫,篤世子,妾不以子稱貴號。昏冠之禮,字子以母。別眇夫妻,同坐而食,喪禮合葬,祭禮先,婦從夫為昭穆。制爵五等,祿士三品。制郊宮明堂內方外衡,其屋習而衡,祭器衡同,作秩機。玉厚六分,白藻三絲,衣長後衽,首服習而垂流。驚輿卑,備地用象載,垂二驚。樂縣鼓,用萬舞,舞溢衡。先烹而後用樂,正刑天法,封壇於左位。

四法修於所故,祖於先帝,故四法如四時然,終而複始,窮則反本。四法之天施符授聖人,王法則性命形乎先祖,大昭乎王君。故天將授舜,主天法商而王,祖錫姓為姚氏。至舜形體大上而員首,而明有二童子,性長於天文,純於孝慈。天將授禹,主地法夏而王,祖錫姓為姒氏,至禹生發於背,形體長,長足,疾行先左,隨以右,勞左佚右也。性長於行,習地明水。天將授湯,主天法質而王,祖錫姓為子氏。謂契母吞玄鳥卵生契,契先發於胸。性長於人倫。至湯,體長專小,足左扁而右便,勞右佚左也。性長於天光,質易純仁。天將授文王,主地法文而王,祖錫姓姬氏。謂後稷母姜原履天之跡而生後稷。後稷長於邰土,播田五。至文王,形體博長,有四乳大足,性長於地文勢。故帝使禹、皋論姓,知殷之德陽德也,故以子為姓;知周之德陰德也,故以姬為姓。故殷王改文,以男書子,周王以女書姬。故天道合以其類動,非聖人孰能明之?

\article{官制象天}

王者制官,三公、九卿、二十七大夫、八士元士,凡百二十人,而列臣備矣。吾聞聖王所取儀,金天之大經,官制亦角者,此其儀與?三人而為一選,儀於三月而為一時也。四選而止,儀於四時而終也。三公者,王之所以自持也。天以三成之,王以三自持。立成數以為植而四重之,其可以無失矣。備天數以參事,治謹於道之意也。此百二十臣者,皆先王之所與直道而行也。是故天子自參以三公,三公自參以九卿,九卿自參以三大夫,三大夫自參以三士。三人為選者四重,自三之道以治天下,若天之四重,自三之時以終始歲也。一陽而三春,非自三之時與?而天四重之,其數同矣。天有四時,時三月;王有四選,選三臣。是故有孟、有仲、有季,一時之情也;有上、有下、有中,一選之情也。三臣而為一選,四選而止,人情盡矣。人之材固有四選,如天之時固有四變也。聖人為一選,君子為一選,善人為一選,正人為一選,由此而下者,不足選也。四選之中,各有節也。是故天選四堤十二而人變盡矣。盡人之變合之天,唯聖人者能之,所以立王事也。何謂天之大經?三起而成日,三日而成規,三旬而成月,三月而成時,三時而成功。寒暑與和,三而成物;日月與星,三而成光;天地與人,三而成德。由此觀之,三而一成,天之大經也,以此為天制。是故禮三讓而成一節,官三人而成一選。三公為一選,三卿為一選,三大夫為一選,三士為一選,凡四選。三臣應天之制,凡四時之三月也。是故其以三為選,取諸天之經;其以四為制,取諸天之時;其以十二臣為一條,取諸歲之度;其至十條而止,取之天端。何謂天之端?曰:天有十端,十端而止已。天為一端,地為一端,陰為一端,陽為一端,火為一端,金為一端,木為一端,水為一端,土為一端,人為一端,凡十端而畢,天之數也。天數畢於十,王者受十端於天,而一條之率。每條一端以十二時,如天之每終一歲以十二月也。十者天之數也,十二者歲之度也。用歲之度,條天之數,十二而天數畢。是故終十歲而用百二十月,條十端亦用百二十臣,以率被之,皆合於天。其率三臣而成一慎,故八十一元士為二十七慎,以持二十七大夫;二十七大夫為九慎,以持九卿;九卿為三慎,以持三公;三公為一慎,以持天子。天子積四十慎以為四選,選一慎三臣,皆天數也。是故以四選率之,則選三十人,三四十二,百二十人,亦天數也。十端積四十慎,慎三臣,三四十二,百二十人,亦天數也。以三公之勞率之,則公四十人,三四十二,百二十人,亦天數也。故散而名之為百二十臣,選而寶之為十二長,所以名之雖多,莫若謂之四選十二長,然而分別率之,皆有所合,無不中天數者也。求天數之微,莫若於人。從之身有四肢,每肢有三節,三四十二,十二節相持而形體立矣。天有四時,每一時有三月,三四十二,十二月相受而歲數終矣。官有四選,每一選有三人,三四十二,十二臣相參而事治行矣。以此見天之數,人之形,官之制,相參相得也。人之與天,多此類者,而皆微忽,不可不察也。天地之理,分一歲之變為以四時,四時亦天之四選已。是故春者少陽之選也,夏者太陽之選也,秋者少陰之選也,冬者太陰之選也。四選之中各有孟、仲、季,是選之中有選,故一歲之中有四時,一時之中有三長,天之節也。人生於天而體天之節,故亦有大小厚薄之變,人之氣也。先王因人之氣,而分其變以為四選,是故三公之位,聖人之選也。三卿之位,君子之選也;三大夫之位,善人之選也;三士之位,正直之選也。分人之變以為四選,選立三臣,如天之分歲之變以為四時,時有三節也。天以四時之選十二節相和而成歲,王以四位之選與十二臣相砥礪而致極,道必極於其所至,然後能得天地之美也。

\article{堯舜不擅移、湯武不專殺}

堯舜何緣而得擅移天下哉?《孝經》之語曰:「事父孝,故事天明。」事天與父,同禮也。今父有以重予子,子不敢擅予他人,人心皆然。則王者亦天之子也,天以天下予堯舜,堯舜受命於天而王天下,猶子安敢擅以所重受於天者予他人也。天有不以予堯舜漸奪之,故明為子道,則堯舜之不私傳天下而擅移位也,無所疑也。儒者以湯武為至聖大賢也,以為全道究義盡美者,故列之堯舜,謂之聖王,如法則之。今足下以湯武為不義,然則足下之所謂義者,何世之王也?曰;弗知。弗知者,以天下王為無義者耶?其有義者而足下不知耶?則答之以神農。應之曰:神農之為天子,與天地俱起乎?將有所伐乎?神農氏有所伐可,湯武有所伐獨不可,何也?且天之生民,非為王也,而天立王以為民也。故其德足以安樂民者,天予之;其惡足以賊害民者,天奪之。《詩》云:「殷士膚敏,裸將於京,侯服於周,天命靡常。」言天之無常予,無常奪也。故封泰山之上。禪梁父之下,易姓而王,德如堯舜者七十二人。王者,天之所予也,其所伐皆天之所奪也。今唯以湯武之伐桀紂為不義,則七十二王亦有伐也。推足下之說,將以七十二王為皆不義也!故夏無道而殷伐之,殷無道而周伐之,周無道而秦伐之,秦無道而漢伐之。有道伐無道,此天理也,所從來久矣,寧能至湯武而然耶?夫非湯武之伐桀紂者,亦將非秦之伐周,漢之伐秦,非徒不知天理,又不明人禮。禮,子為父隱惡。今使伐人者而信不義,當為國諱之,豈宜如誹謗者,此所謂一言而再過者也。君也者,掌令者也,令行而禁止也。今桀紂令天下而不行,禁天下而不止,安在其能臣天下也?果不能臣天下,何謂湯武弒?

\article{服制}

率得十六萬國三分之,則各度爵而制服,量祿而用財。飲食有量,衣服有制,宮室有度,畜產人徒有數,舟車甲器有禁。生有軒冕、之服位、貴祿、田宅之分,雖有賢才美體,無其爵不敢服其服;雖有富家多貲,天子服有文章,不得以燕公以朝;將軍大夫不得以燕;將軍大夫以朝官吏;命士止於帶緣。散民不敢服雜采,百工商賈不敢服狐貉,刑余戮民不敢服絲玄乘馬,謂之服制。

\article{度制}

孔子曰:「不患貧而患不均。」故有所積重,則有所空虛矣。大富則驕,大貧則憂。憂則為盜,驕則為暴,此眾人之情也。聖者則於眾人之情,見亂之所從生。故其制人道而差上下也,使富者足以示貴而不至於驕,貧者足以養生而不至於憂。以此為度而調均之,是以財不匱而上下相安,故易治也。今世棄其度制,而各從其欲。欲無所窮,而欲得自恣,其勢無極。大人病不足於上,而小民贏瘠於下,則富者愈貪利而不肯為義,貧者日犯禁而不可得止,是世之所以難治也。


孔子曰:「君子不盡利以遺民。」《詩》云「彼有遺秉,此有不斂,伊寡婦之利。」故君子仕則不稼,田則不漁,食時不力珍,大夫不坐羊,士不坐犬。《詩》曰:「采葑采菲,無以下體。德音莫違,及爾同死。」以此防民,民猶忘義而爭利,以亡其身。天不重與,有角不得有上齒。故已有大者,不得有小者,天數也。夫已有大者又兼小者,天不能足之,況人乎?故明聖者象天所為,為制度,使諸有大奉祿亦皆不得兼小利,與民爭利業,乃天理也。


凡百亂之源,皆出嫌疑纖微,以漸寢稍長至於大。聖人章其疑者,別其微者,絕其纖者,不得嫌以蚤防之。聖人之道,眾堤防之類也。謂之度制,謂之禮節。故貴賤有等,衣服有制,朝廷有位,鄉黨有序,則民有所讓而不敢爭,所以一之也。《書》曰:「舉服有庸,誰敢弗讓,敢不敬應。」此之謂也。


凡衣裳之生也,為蓋形暖身也。然而染五采,飾文章者,非以為益肌膚血氣之情也,將以貴貴尊賢,而明別上下之倫,使教亟行,使化易成,為治為之也。若去其度制,使人人從其欲,快其意,以逐無窮,是大亂人倫,而靡斯財用也,失文采所遂生之意矣。上下之倫不別,其勢不能相治,故苦亂也。嗜欲之物無限,其勢不能相足,故苦貧也。今欲以亂為治,以貧為富,非反之制度不可。古者天子衣文,諸侯不以燕,大夫衣祿,士不以燕,庶人衣縵,此其大略也。



\article{爵國}

《春秋》曰:「會宰周公。」又曰:「公會齊侯、宋公、鄭伯、許男、滕子。」又曰:「初獻六羽。」《傳》曰:「天子三公稱公,王者之後稱公,其余大國稱侯,小國稱伯、子、男。」凡五等。故周爵五等,士三品,文多而實少。《春秋》三等,合伯、子、男為一爵,《春秋》曰:「荊。」《傳》曰:「氏不若人,人不若名,名不若字。」凡四等,命曰附庸,三代共之。然則其地列奈何?曰:天子邦圻千裡,公侯百裡,伯七十裡,子男五十裡,附庸字者方三十裡,名者方二十裡,人氏者方十五裡。《春秋》曰:「宰周公。」《傳》曰:「天子三公。」「祭伯來」,《傳》曰:「天子大夫。」「宰渠伯糾」,《傳》曰:「下大夫。」「石尚」,《傳》曰:「天子之士也。」「王人」,《傳》曰:「微者,謂下士也。」凡五等。《春秋》曰:「作三軍。」《傳》曰:「何以書?譏。何譏爾?古者上卿、下卿、上士、下士。」凡四等。小國之大夫與次國下卿同,次國大夫與大國下卿同,大國下大夫與天子下士同。二十四等,祿八差。有大功德者受大爵士,功德小者受小爵士,大材者執大官位,小材者受小官位,如其能,宣治之至也。故萬人者曰英,千人者曰俊,百人者曰杰,十人者曰豪。豪杰俊英不相陵,故治天下如視諸掌上。其數何法以然?曰:天子分左右五等,三百六十三人,法天一歲之數。五時色之象也。通佐十上卿與下卿而二百二十人,天庭之象也。倍諸侯之數也。諸侯之外佐四等,百二十人,法四時六甲之數也。通佐五,而六十人,法日辰之數也。佐之必三三而相複,何?曰:時三月而成大,辰三而成象。諸侯之爵或五何?法天地之數也。五官亦然。然則立置有司,分指數柰何?曰:「諸侯大國四軍,古之制也。其一軍以奉公家也。凡口軍三者何?曰:大國十六萬口而立口軍三。何以言之?曰:以井田準數之。方裡而一井,一井而九百畝而立口。方裡八家,一家百畝,以食五口。上農夫耕百畝,食九口,次八人,次七人,次六人,次五人。多寡相補,率百畝而三口,方裡而二十四口。方裡者十,得二百四十口。方十裡為方裡者百。得二千四百口。方百裡為方裡者萬,得二十四萬口。法三分而除其一。城池、郭邑、屋室、閭巷、街路市、官府、園囿、萎、台沼、椽采,得良田方十裡者六十六,與方裡六十六,定率得十六萬口。三分之,則各五萬三千三百三十三口,為大口軍三。此公侯也。天子地方千裡,為方百裡者百。亦三分除其一,定得田方百裡者六十六,與方十裡者六十六,定率得千六百萬口。九分之,各得百七十七萬七千七百七十七口,為京口軍九。三京口軍以奉王家。故天子立一後,一世夫人,中左右夫人,四姬,三良人。立一世子,三公,九卿,二十七大夫,八十一元士,二百四十三下士。有七上卿,二十一下卿,六十三元士,百二十九下士。王後置一太傅、太母,三伯,三丞。世夫人,四姬,三良人,各有師傅。世子一人,太傅,三傅,三率,三少。士入仕宿衛天子者比下士,下士者如上士之下數。王後御衛者,上下御各五人。世夫人、中左右夫人、四姬,上下御各五人。三良人,各五人。世子幻妃姬及士衛者,如公侯之制。王後傅,上下史五人;三伯,上下史各五人;少伯,史各五人。世子太傅,上下史各五人;少傅,亦各五人;三率、三下率,亦各五人。三公,上下史各五人;卿,上下史各五人;大夫,上下史各五人;元士,上下史各五人;上下卿、上下士之史,上下亦各五人。卿大夫、元士,臣各三人。故公侯方百裡,三分除其一,定得田方十裡者六十六,與方裡六十六,定率得十六萬口。三分之,為大國口軍三,而立大國。一夫人,一世婦,左右婦,三姬,二良人。立一世子,三卿,九大夫,二十七上士,八十一下士,亦有五通大夫,立上下士。上卿位比天子之元士,今八百石。下卿六百石,上士四百石,下士三百石。夫人一傅母,三伯,三丞。世婦,左右婦,三姬,二良人,各有師保。世子一上傅、丞。士宿衛公者,比公者,比上卿者有三人,下卿六人,比上下士者如上下之數。夫人衛御者,上下御各五人;世婦、左右婦,上下御各五人;二卿,御各五人;世子上傅,上下史各五人;丞,史各五人;三卿、九大夫,上士史各五人,下士史各五人;通大夫、士,上下史各五人;卿,臣二人。此公侯之制也。公侯賢者為州方伯,錫斧鉞,置虎賁百人。故伯七十裡,七七四十九,三分除其一,定得田方十裡者二十八,與方十裡者六十六,定率得十萬九千二百一十二口,為次國口軍三,而立次國。一夫人,世婦,左右婦,三良人,二孺子。妝世子,三卿,九大夫,二十七上士,八士下士,與五通大夫,五上士,十五下士。其上卿,位比大國之下卿,今六百石;下卿四百石,上士三百石,下士二百石。夫人一傅母,三伯,三丞。世婦,左右婦,三良人,二御人,各有帥保。世子一上下傅。士宿衛公者,比上卿者三人,下卿六人,比上下士如上下之數。夫人御衛者,上下御各五人。世婦、左右婦,上下御各五人;二御,各五人;世子上傅,上下史各五人,丞、史各五人;三卿、九大夫上下史各五人,下士史各五人;通大夫,上下史各五人;卿,臣二人。故子男方五十裡,五五二十五,為方十裡者六十六,定率得發口,為小國口軍三,而立小國。夫人,世婦,左右婦,三良人,二孺子。立一世子,三卿,九大夫,二十七上士,八十一下士,與五通大夫,五上士,十五下士。其上卿比次國之下卿,今四百石。下卿三百石,上士二百石,下士百石。夫人一傅母,三伯,三丞。世婦,左右婦,三良人,一御人,各有師保。世子一上下傅。士宿衛公者,比上卿者三人,下卿六人。夫人御衛者,上下御各五人;世婦,左右婦,上下御卿六人。夫人御衛者,上下御各五人;世婦,左右婦,上下御各五人;二御人,各五人;世子上傅,上下史各五人;三卿、九大夫,上下史各五人;士,各五人;通大夫,上下史亦各五人;卿,臣二人。此周制也。《春秋》合伯子男為一等,故附庸字者地方三十裡,三三而九,三分而除其一,定得田方十裡者六,定率得一萬四千四百口,為口師三,而立一宗婦、二妾、一世子,宰歪、士一,秩士五人。宰視子男下卿,今三百石。宗婦有師保,御者三人,妾各二人,世子一傅。士宿衛君者,比上卿,下卿一人,上下各如其數。世子傅,上下史各五人。稱名善者,地方半字君之地。三分除其一,定得田方十裡者三,定率得七千二百口。一世子宰,今二百石。下四半三半二十五。三分除其一,定得田方十裡者一,與方裡者五,定率得三千六百口。一世子宰,今百石,史五人,宗婦仕衛世子臣。

\article{仁義法}

《春秋》之所治,人與我也。所以治人與我者,仁與義也。以仁安人,以義正我,故仁之為言人也,義之為言我也,言名以別矣。仁之於人,義之與我者,河不察也。眾人不察,乃反以仁自裕,而以義設人。詭其處而逆其理,鮮不亂矣。是故人莫欲亂,而大抵常亂。凡以暗於人我之分,而不省仁義之所在也。是故《春秋》為仁義法。仁之法在愛人,不在愛我。義之法在正我,不在正人。我不自正,雖能正人,弗予為義。人不被其愛,雖厚自愛,不予為仁。昔者晉靈公殺膳宰以淑飲食,彈大夫以娛其意,非不厚自愛也,然而不得為淑人者,不愛人也。質於愛民,以下至於鳥獸昆蟲莫不愛。不愛,奚足謂仁?仁者,愛人之名也。酅《傳》無大之之辭。自為追,則善其所恤遠也。兵已加焉,乃往救之,則弗美。未至豫備之,則美之,善其救害之先也。夫救蚤而先之,則害無由起,而天下無害矣。然則觀物之動,而先覺其萌,絕亂塞害於將然而未形之時,《春秋》之志也,非堯舜之智,知禮之本,孰能當此?故救害而先知之,明也。公之所恤遠,而《春秋》美之。詳其美恤遠之意,則天地之間然後快其仁矣。非三王之德,選賢之精,孰能如此?是以知明先,以仁厚遠。遠而愈賢、近而愈不肖者,愛也。故王者愛及四夷,霸者愛及諸侯,安者愛及封內,危者愛及旁側,亡者愛及獨身。獨身者,雖立天子諸侯之位,一夫之人耳,無臣民之用矣。如此者,莫之亡而自亡也。《春秋》不言伐梁者,而言梁亡,蓋愛獨及其身者也。故曰仁者愛人,不在愛我,此其法也。義雲者,非謂正人,謂正我。雖有亂世枉上,莫不欲正人。奚謂義?昔者楚靈王討陳蔡之賊,齊桓公執袁濤涂之罪,非不能正人也,然而《春秋》弗予,不得為義者,我不正也。闔廬能正楚蔡之難矣,而《春秋》奪之義辭,以其身不正也。潞子之於諸侯,無所能正,《春秋》予之有義,其身正也,趨而利也。故曰義在正我,不在正人,此其法也。夫我無之求諸人,我有之而誹諸人,人之所不能受也。其理逆矣,何可謂義?義者,謂宜在我者。宜在我者,而後可以稱義。故言義者,合我與宜,以為一言。以此操之,義之為言我也。故曰有為而得義者,謂之自得;有為而失義者,謂之自失。人好義者,謂之自好;人不好義者,謂之不自好。以此參之,義,我也,明矣。是義與仁殊。仁謂往,義謂來,仁大遠,義大近。愛在人謂之仁,義在我謂之義。仁主人,義主我也。故曰仁者人也,義者我也,此之謂也。君子求仁義之別,以紀人我之間,然後辨乎內外之分,而著於順逆之處也。是故內治反理以正身,據禮以勸福。外治推恩以廣施,寬制以容眾。孔子謂冉子曰:「治民者,先富之而後加教。」語樊遲曰:「治身者,先難後獲。」以此之謂治身之與治民,所先後者不同焉矣。《詩》曰:「飲之食之,教之誨之。」先飲食而後教誨,謂治人也。又曰:「坎坎伐輻,彼君子兮,不素餐兮。」先其事,後其食,謂治身也。《春秋》刺上之過,而矜下之苦,小惡在外弗舉,在我書而誹之。凡此六者,以仁治人。義治我,躬自厚而薄責於外,此之謂也。且《論》已見之,而人不察,不攻人之惡,非仁之寬與?自攻其惡,非義之全與?此謂之仁造人,義造我,何以異乎?故自稱其惡謂之情,稱人之惡謂之賊;求諸己謂之厚,求諸人謂之薄;自責以備謂之明。責人以備謂之惑。是故以自治之節治人,是居上不寬也;以治人之度自治,是為禮不敬也。為禮不敬,則傷行而民弗尊;居上不寬,則傷厚而民弗親。弗親則弗信,弗尊則弗敬。二端之政詭於上,而僻行之則誹於下,仁義之處可無論乎?夫目不視弗見,心弗論不得。雖有天下之至味,弗嚼弗知其旨也;雖有聖人之至道,弗論不知其義也。

\article{必仁且知}

莫近於仁,莫急於智。不仁而有勇力材能,則狂而操利兵也;不智而辯慧狷給,則迷而乘良馬也。故不仁不智而有材能,將以其材能以輔其邪狂之心,而贊其僻違之行,適足以大其非而甚其惡耳。其強足以覆過,其御足以犯詐,其慧足以其辨足以飾非,其堅足以斷闢,其嚴足以拒諫。此非無材能也,其施之不當而處之不義也。有否心者,不可藉便執,其質愚者不與利器。《論》之所謂不知人也者,恐不知別此等也。仁而不智,則愛而不別也;智而不仁,則知而不為也。故仁者所以愛人類也,智者所以除其害也。


何謂仁?仁者怛愛人,謹翕不爭,好惡敦倫,無傷惡之心,無隱忌之志,無嫉妒之氣,無感愁之欲,無險之事,無闢違之行。故其心舒,其誌平,其氣和,其欲節,其事易,其行道,故能平易和理而無爭也。如此者謂之仁。

何謂之智?先言而後當。凡人欲舍行為,皆以其智先規而後為之。其規是者,其所為得,其所事當,其行遂,其名榮,其身故利而無患,福及子孫,德加萬民,湯武是也。其規非者,其所為不得,其所事不當,其行不遂,其名辱,害及其身,絕世無複,殘類滅宗亡國是也。故曰莫急於智。智者見禍福遠,其知利害蚤,物動而知其化,事同而知其歸,見始而知其母,言之而無敢嘩,立之而不可廢,取之而不可舍,前後不相悖,終始有類,思之而有複,及之而不可厭。其言寡而足,約而喻,簡而達,省而具,少而不可益,多而不可損。其動中倫,其言當務。如是者謂之智。


其大略之類,天地之物有不常之變者,謂之異,小者謂之災。災常先至而異乃隨之。災者,天之譴也;異者,天之威也。譴之而不知,乃畏之以威。《詩》云「畏天之威。」殆此謂也。凡災異之本,盡生於國家之失。國家之失乃始萌芽,而天出災害以譴告之;譴告之而不知變,乃見怪異以驚駭之,驚駭之尚不知畏恐,其殃咎乃至。以此見天意之仁而不欲陷人也。謹案災異以見天意。天意有欲也,有不欲也。所欲所不欲者,人內以自省,宜有懲於心;外以觀其事,宜有驗於國。故見天意者之於災異也,畏之而不惡也,以為天欲振吾過,救吾失,故以此報我也。《春秋》之法,上變古易常,應是而有天災者,謂幸國。孔子曰:「天之所幸,有為不善而屢極。」楚莊王以天不見災,地不見孽,則禱之於山川,曰:「天其將亡予邪?不說吾過,極吾罪也。」以此觀之,天災之應過而至也,異之顯明可畏也。此乃天之所欲救也,《春秋》之所獨幸也,莊王所以禱而請也。聖主賢君尚樂受忠臣之諫,而況受天譴也?

\article{身之養重於義}

天之生人也,使人生義與利。利以養其體,義以養其心。心不得義不能樂,體不得利不能安。義者心之養也,利者體之養也。體莫貴於心,故養莫重於義,義之養生人大於利。奚以知之?今人大有義而甚無利,雖貧與賤,尚榮其行,以自好而樂生,原憲、曾、閔之屬是也。人甚有利而大無義,雖甚富,則羞辱大惡。惡深,非立死其罪者,即旋傷殃憂爾,莫通能以樂生而終其身,刑戮夭折之民是也。夫人有義者,雖貧能自樂也。而大無義者,雖富莫能自存。吾以此實義之養生人,大於利而厚於財也。民不能知而常反之,皆忘義而殉利,去理而走邪,以賊其身而禍其家。此非其自為計不忠也,則其知之所不能明也。今握棗與錯金,以示嬰兒,嬰兒必取棗而不敢金也。握一斤金與千萬之珠,以示野人,野人必取金而不敢珠也。故物之於人,小者易知也,其於大者難見也。今利之於人小而義之於人大者,無怪民之皆趨利而不趨義也,固其所暗,故民不陷。《詩》云:「示我顯德行。」此之謂也。先王顯德以示民,民樂而歌之以為詩,說而化之以為欲。故不令而自行,不禁而自止,從上之意,不待使之,若自然矣。故曰:聖人天地動、四時化者,非有他也,其見義大故能動,動故能化,化故能大行,化大行故法不犯,法不犯故刑不用,刑不用則堯舜之功德。此大治之道也,先聖傅授而複也。故孔子曰:「誰能出不由戶,何莫由斯道也。」今不示顯德行,民暗於義,不能;迷於道不能解,因欲大嚴以必正之,直殘賊天民而薄主德耳,其勢不行。仲尼曰:「國有道,雖加刑,無刑也。國無道,雖殺之,不可勝也。」其所謂有道無道者,示之以顯德行與不示爾。

\article{對膠西王越大夫不得為仁}

命令相曰:「大夫蠡、大夫種、大夫庸、大夫睪、大夫車成,越王與此五大夫謀伐吳,遂滅之,雪會稽之恥,卒為霸主。範蠡去之,種死之。寡人以此二大夫者為皆賢。孔子曰:『殷有三仁。』今以越王之賢,與蠡種之能,此三人者,寡人亦以為越有三仁。其於君何如?桓公決疑於管仲,寡人決疑於君。」仲舒伏地再拜對曰:「仲舒智褊而學淺,不足以決之。雖然,王有問於臣,臣不敢不悉以對,禮也。臣仲舒聞,昔者魯君問於柳下惠曰:『我欲攻齊,何如?』柳下惠對曰:『不可。』退而有憂色,曰:『吾聞之也,謀伐國者,不問於仁人也。此何為至於我?』但見問而尚羞之,而況乃與為詐以伐吳乎?其不宜明矣。以此觀之,越本無一仁,而安得三仁?仁人者正其道不謀其利,修其理不急其功,致無為而習俗大化,可謂仁聖矣。三王是也。《春秋》之義,貴信而賤詐。詐人而勝之,雖有功,君子弗為也。是以仲尼之門,五尺童子,言羞稱五伯。為其詐以成功,苟為而已也,故不足稱於大君子之門。五伯者,比於他諸侯為賢者,比於仁賢,何賢之有?譬猶武夫比於美玉也。臣仲舒伏地再拜以聞。」

\article{觀德}

天地者,萬物之本,先祖之所出也。廣大無極,其德昭明,歷年眾多,永永無疆。天出至明,眾知類也,其伏無不也。地出至晦,星日為明,不敢暗。君臣、父子、夫婦之道取之此。大禮之終也,臣子三年不敢當。雖當之,必稱先君,必稱先人,不敢貪至尊也。百禮之貴,皆編於月。月編於時,時編於君,君編於天。天之所棄,天下弗桀紂是也。天子之所誅絕,臣子弗得立,蔡世子逢丑父是也。天父父所絕,子孫不得屬,魯莊公之不得念母,術輒之辭父命是也。故受命而海內順之,猶眾星之共北辰,流水之宗滄海也。況生天地之間,法太祖先人之容貌,則其至德取象,眾名尊貴,是以聖人為貴也。泰伯至德之侔天地也,上帝為之廢適易姓而子之。讓其至德,海內懷歸之。泰伯三讓而不敢就位。伯邑考知群心貳,自引而激,順神明也。至德以受命,豪英高明之人輻歸之。高者列為公侯,下至卿大夫,濟濟乎哉,皆以德序。是故吳魯同姓也,鐘離之會不得序而稱君,殊魯而會之,為其夷狄之行也。雞父之戰,吳不得與中國為禮。至於伯莒黃池之行,變而反道,乃爵而不殊。召陵之會,魯君在是而不得為主,避齊桓也。魯桓即位十三年,齊、宋、衛、燕舉師而東,紀、鄭與魯戮力而報之。後其日,以魯不得遍,避紀侯與鄭厲公也。《春秋》常辭,夷狄不得與中國為禮。至之戰,夷狄反道,中國不得與夷狄為禮,避楚莊也。邢衛,魯之同姓也,狄人滅之,《春秋》為諱,避齊桓也。當其如此也,惟德是親,是故周之子孫,其親等也,四時等也,而春最先。十二月等也,而正月最先。德等也,則先親親。魯十二公等也,而定哀最尊。衛俱諸夏也,善稻之會,獨先內之,為其與我同姓也。吳俱夷狄也,相之會,獨先外之,為其與我同姓也。滅國十五有余,獨先諸夏,魯晉俱諸夏也,譏二名,獨先及之。盛伯郜子俱當絕,而獨不名,為其與我同姓兄弟也。外出者眾,以母弟出,獨大惡之,為其亡母背骨肉也。滅人者莫絕,衛侯毀滅同姓獨絕,賤基其本祖而忘先也。親等從近者始,立適以長,母以子貴先。甲戌、己丑,陳侯鮑卒,書所見也,而不言其暗者。隕石於宋五,六退飛,耳聞而記,目見而書,或徐或察,皆以其先接於我者序之。其於會朝聘之禮亦猶是。諸侯與盟者眾矣,而儀父獨漸進。鄭僖公方來會我而道殺,《春秋》致其意,謂之如會。潞子離狄而歸,黨以得亡,《春秋》謂之子,以領其意。包來、首戴、洮、踐土與操之會,陳鄭去我,謂之逃歸;鄭處而不來,謂之乞盟;陳侯後至,謂之如會,莒人疑我,貶而稱人。諸侯朝魯者眾矣,而滕薛獨稱侯。州公化我,奪爵而無號。吳楚國先聘我者見賢,曲棘與之戰,先憂我者見尊。

\article{奉本}

禮者,繼天地,體陰陽,而慎主客,序尊卑、貴賤、大小之位,而差外內、遠近、新故之級者也,以德多為象。萬物以廣博眾多,歷年久者為象。其在天而象天者,莫大日月,繼天地之光明,莫不照也。星莫大於大辰,北斗常星。部星三百,衛星三千。大火二十六星,伐十三星,半斗星,二十八宿。多者宿二十八九。其猶著百莖而共一本,龜千歲而人寶。是以三代傳決疑焉。其得地體者,莫如山阜。人之得天得眾者,莫如受命之天子。下至公、侯、伯、子、男,海內之心懸於天子,疆內之民統於諸侯。日月食,告凶,不以其行。有星於東方,常星不見,地震,梁山沙鹿崩,宋、衛、陳、鄭災,王公大夫篡弒者,《春秋》皆書以為大異;不言眾星之入、雨,原隰之襲崩,一國之小民死亡,不決疑於眾草木也。唯田邑之稱,多著主名。君將不言臣,臣不言師,王夷、君獲,不言師敗。孔子曰:「唯天為大,唯堯則之。」則之者,大也。「巍巍乎其有成功也」,言其尊大以成功也。齊桓晉文不尊周室,不能霸;三代聖人不則天地,不能至王。階此而觀之,可以知天地之貴矣。夫流深者其水不測,尊至者其敬無窮。是故天之所加,雖為災害,猶承而大之,其欽無窮,震夷伯之廟是也。天無錯舛之災,地有震動之異。天子所誅絕,所敗師,雖不中道,而《春秋》者不敢闕,謹之也。故師出者眾矣,莫言還。至師及齊師圍成,成降於齊師,其君劫外,不得已,故可直言也。至於他師,皆其君之過也,而曰非師之罪。是臣子之不為君父受罪,罪不臣子莫大焉。夫至明者其照無疆,至晦者其暗無疆。今《春秋》緣魯以言王義,殺隱桓以為遠祖,宗定哀以為考妣,至尊且高,至顯且明。其基壞之所加,潤澤之所被,條條無疆,前是常數,十年鄰之,幽人近其墓而高明。微國之君,卒葬之禮,錄而辭繁。遠夷之君,內而不外。當此之時,魯無鄙疆,諸侯之伐哀者皆言我。邾妻庶其、鼻我,邾妻大夫。其於我無以親,以近之故,乃得顯明。隱桓,親《春秋》之先人也,益師卒而不日。於稷之會,言其成宋亂,以遠外也。黃池之會,以兩伯之辭,言不以為外,以近內也。

\article{深察名號}

治天下之端,在審辨大。辨大之端,在深察名號。錄其首章之意,以窺其中之事,則是非可知,逆順自著,其幾通於天地矣。是非之正,取之逆順,逆順之正,取之名號,名號之正,取之天地,天地為名號之大義也。古之聖人,而效天地謂之號,鳴而施命謂之名。名之為言,鳴與命也,號之為言,而效也。而效天地者為號,鳴而命者為名。名號異聲而同本,皆鳴號而達天意者也。天不言,使人發其意;弗為,使人行其中。名則聖人所發天意,不可不深觀也。受命之君,天意之所予也。故號為天子者,宜視天如父,事天以孝道也。號為諸侯者,宜謹視所候奉之天子也。號為大夫者,宜厚其忠信,敦其禮義,使善大於匹夫之義,足以化也。士者,事也;民者,瞑也。士不及化,可使守事從上而已。各有分。分中委曲,名眾於號,號春大全。曲有名。名也者,名其別離分散也。號凡而略,名詳而目。目者,遍辨其事也;凡者,獨舉其大也。一日祭。祭之散名,春曰祠,夏曰祗,秋曰嘗,冬曰蒸。獵禽獸者號,一日田。田之散名,春苗,秋搜,冬狩,夏。無有不皆中天意者。物莫不有凡號,號莫不有散名,如是。是故事各順於名,名各順於天。天人之際,合而為一。同而通理,動而相益,順而相受,謂之德道。《詩》曰:「維號斯言,有倫有跡。」此之謂也。


深察王號之大意,其中有五科:皇科、方科、匡科、黃科、往科。合此五科,以一言謂之王。王者皇也,王者方也,王者匡也,王者黃也,王者往也。是故王意不普大而皇,則道不能正直而方;道不能正直而方,則德不能匡運周遍;德不能匡運周遍,則美不能黃;美不能黃,則四方不能往;四方不能往,則不全於王。故曰:天覆無外,地載兼愛,風行令而一其威,雨布施而均其德。王術之謂也。


深察君號之大意,春中亦有五科:元科、原科、權科、溫科、群科。合此五科,以一言謂之君。君者元也,君者原也。君者權也,君者溫也,君者群也。是故君意不比於元,則動而失本;動而失本,則所為不立;所為不立,則不效於原,不效於原,則自委舍;自委舍,則化不行。用權於變,則失中適之宜;失中適之宜,則道不平,德不溫;道不平,德不溫,則眾不親安;眾不親安,則離散不群;離散不群,則不全於君。


名生於真,非其真,弗以為名。名者,聖人之所以真物也。名之為言真也。故凡百譏有者,各反其真,則者還昭昭耳。欲審曲直,莫如引繩;欲審是非,莫如引名。名之審於是非也,猶繩之審於曲直也。詰其名實,觀其離合,則是非之情不可以相讕已。今世暗於性,言之者不同,胡不試反性之名。性之名非生與?如其生之自然之資謂之性。性者質也。詰性之質於善之名,能中之與?既不能中矣,而尚謂之質善,何哉?性之名不得離質。離質如毛,則非性已,不可不察也。《春秋》辨物之理,以正其名。名物如其真,不失秋毫之末。故名隕石,則後其五,言退,則先其六。聖人之謹於正名如此。君子於其言,無所苟而已,五石、六之辭是也。眾惡於內,弗使得發於外者,心也。也。人之受氣苟無惡者,心何哉?吾以心之名,得人之誠。人之誠,有貪有仁。仁貪之氣,兩在於身。身之名,取諸天。天兩有陰陽之施,身亦兩有貪仁之性。天有陰陽禁,身有情欲,與天道一也。是以陰之行不得干春夏,而月之魄常厭於日光。乍全乍傷,天之禁陰如此,安得不損其欲而輟其情以應天。天所禁而身禁之,故曰身猶天也。禁天所禁,非禁天也。必知天性不乘於教,終不能。察實以為名,無教之時,性何遽若是。故性比於禾,善比於米。米出禾中,而禾未可全為米也。善出性中,而性未可全為善也。善與米,人之所繼天而成於外,非在天所為之內也。天之所為,有所至而止。止之內謂之天性,止之外謂之人事。事在性外,而性不得不成德。民之號,取之瞑也。使性而已善,則何故以瞑為號?以者言,弗扶將,則顛陷猖狂,安能善?性有似目,目臥幽而瞑,待覺而後見。當其未覺,可謂有見質,而不可謂見。今萬民之性,有其質而未能覺,譬如瞑者待覺,教之然後善。當其未覺,可謂有善質,而不可說服力善,與目之瞑而覺,一概之比也。靜心徐察之,其言可見矣。性而瞑之未覺;天所為也。效天所為,為之起號,故謂之民。民之為言,固猶瞑也,隨其名號以入其理,則得之矣。是正名號者於天地,天地之所生,謂之性情。性情相與為一瞑。情亦性也。謂性已善,奈其情何?故聖人莫謂性善,累其名也。身之有性情也,若天之有陰陽也。言人之質而無其情,猶言天之陽而無其陰也。窮論者,無時受也。名性,不以上,不以下,以其中名之。性如繭如卵。卵待覆而成誰,繭待繅而為絲,性待教而為善。此之謂真天。天生民性有善質,而未能善,於是為之立王以善之,此天意也。民受未能善之性於天,而退受成性之教於王。王承天意,以成民之性為任者也。今案其真質,而謂民性已善者,是失天意而去王任也。萬民之性苟已善,則王者受命尚何任也?其設名不正,故棄重任而違大命,非法言也。《春秋》之辭,內事之待外者,從外言之。貪今萬民之性,待外教然後能善,善當與教,不當與性。與性,則多累而不精,自成功而無賢聖,此世長者之所誤出也,非《春秋》為辭之術也。不法之言、無驗之說,君子之所外,何以為哉?或曰:性有善端,心有善質,尚安非善?應之曰:非也。繭有絲而繭非絲也,獷有而卵非也。比類率然,有何疑焉。天生民有《六經》,言性者不當異。然其或曰性也善,或曰性未善,則所謂善者,各異意也。性有善端,動之愛父母,善於禽獸,則謂之善。此孟子之善。循三綱五紀,通八端之理,忠信而博愛,敦厚而好禮,乃可謂善。此聖人之善也。是故孔子曰:「善人吾不得而見之,得見有常者斯可矣。」由是觀之,聖人之所謂善,未易當也,非善於禽獸則謂之善也。使動其端善於禽獸則可謂之善,善奚為弗見也?夫善於禽獸之未得為善也,猶知於草木而不得名知。萬民之性善於禽獸而不得名善,知之名乃取之聖。聖人之所命,天下以為正。正朝夕者視北辰,正嫌疑者視聖人。聖人以為無王之世,不教之民,莫能當善。善之難當如此,而謂萬民之性皆能當之,過矣。質於禽獸之性,則萬民之性善矣;質於人道之善,則民性弗及也。萬民之性善於禽痹嗜許之,聖人之所謂善者弗許。吾質之命性者異孟子。孟子下質於禽獸之所為,故曰性已善;吾上質於聖人之所為,故謂性未善。善過性,聖人過善。《春秋》大元,故謹於正名。名非所始,如之何謂未善已善也。


\article{實性}

孔子曰:「名不正則言不順。」今謂性已善,不幾於無教而如其自然!又不順於為政為道矣。且名者性之實,實者性之質。質無教之時,何遽能善?善如米,性如禾。禾雖出米,而禾未可謂米也。性雖出善,而性未可謂善也。米與善,人之繼天而成於外也,非在天所為之內也。天所為,有所至而止。止之內謂之天,止之外謂之王教。王教在性外,而性不得不遂。故曰性有善質,而未能為善也。豈敢美辭,其實然也。天之所為,止於繭麻與禾。以麻為布,以繭為絲,以米為飯,以性為善,此皆聖人所繼天而進也,非情性質樸之能至也,故不可謂性。正朝夕者視北辰,正嫌疑者視聖人。聖人之所名,天下以為正。今按聖人言中,本無性善名,而有善人吾不得見之矣。使萬民之性皆已能善,善人者何為不見也?觀孔子言此之意,以為善甚難當。而孟子以為萬民性皆能當之,過矣。聖人之性不可以名性,斗筲之性又不可以名性,中民之性如繭如卵。卵待覆二十日而後能為雛,繭待繰以涫湯而後能為絲,性待漸於教訓而後能為善。善,教訓之所然也,非質樸之所能至也,故不謂性。性者宜知名矣,無所待而起,生而所自有也。善所自有,則教訓已非性也。是以米出於粟,而粟不可謂米;玉出於璞,而璞不可謂玉;善出於性,而性不可謂善。其比多在物者為然,在性者以為不然,何不通於類也?卵之性未能作雛也,繭之性未能作絲也,麻之性未能為縷也,粟之性未能為米恥。《春秋》別物之理以正其名,名物必各因基其真。真其義也,真真情也,乃以為名。名石則後其五,退飛則先其六,此皆其真也。聖人於言無所苟而已矣。性者,天質之樸也;善者,王教之化也。無其質,則王教不能化;無其王教,則質樸不能善。質而不以善性,其名不正,故不受也。

\article{諸侯}

生育養長,成而更生,終而複始,其事所以利活民者無已。天雖不言,其欲贍足之意可見也。古之聖人,見天意之厚於人也,故南面而君天下,和以兼利之。為其遠者目不能見,其隱者耳不能聞,於是千裡之外,割地分民,而建國立君,使為天子視所不見,聽所不聞,朝者召而問之也。諸侯之為言,猶諸候也。

\article{五行對}

河閒獻王問溫城董君曰:「《孝經》曰『夫孝,天之經,地之義。』何謂也?」對曰:「天有五行,木火土金水是也。木生火,火生土,土生金,金生水。水為冬,金為秋,土為季夏,木為春。春主生,夏主長,季夏主養,秋主收,冬主藏。藏,冬之所成也。是故父之所生,其子長之;父之所長,其子養之;父之所養,其子成之。諸父所為,其子皆奉承而續行之,不敢不致如父之意,盡為人之道也。故五行者,五行也。由此觀之,父授之,子受之,乃天之道也。故曰:夫孝者,天之經也。此之謂也。」王曰:「善哉。天經既得聞之矣,願聞地之義。」對曰:「地出雲為雨,起氣為風。風雨者,地之所為。地不敢有其功名,必上之於天。命若從天氣者,故曰天風天雨也,莫曰地風地雨也。勤勞在地,名一歸於天,非至有義,其孰能行此?故下事上,如地事天也,可謂大忠矣。土者,火之子也。五行莫貴於土。土之於四時無所命者,不與火分功名。木名春,火名夏,金名秋,水名冬。忠臣之義,孝子之行,取之土。土者,五行最貴者也,其義不可以加矣。五聲莫貴於宮,五味莫美於甘,五色莫盛於黃,此謂孝者地之義也。」王曰:「善載!」

\article{為人者天}
為生不能為人,為人者天也。人之人本於天,天亦人之曾祖父也。人之形體,化天數而成;人之血氣,化天誌而仁;人之德行,化天理而義。人之好惡,化天之暖清;人之喜怒,化天之寒暑;人之受命,化天之四時。人生有喜怒哀樂之答,春秋冬夏之類也。春之答也,怒,秋之答也;樂,夏之答也;哀,冬之答也。天之副在乎人。人之情性有由天者矣。故曰受,由天之號也。為人主也,道莫明省身之天,如天出之也。使其出也,答天之出四時而必忠其受也,是可生可殺,而不可使為亂。故曰:「非道不行,非法不言。」此之謂也。


傅曰:唯天子受命於天,天下受命於天子,一國則受命於君。君命順,則民有順命;君命逆,則民有逆命。故曰:「一人有慶,兆民賴之。」此之謂也。


傅曰:政有三端:父子不親,則致其愛慈;大臣不和,則敬順其禮;百姓不安,則力其孝弟。孝弟者,所以安百姓也。力者,勉行之身以化之。天地之數,不能獨以寒暑成歲,必有春夏秋冬。聖人之道,不能獨以威勢成政,必有教化。故曰:先之以博愛,教以仁也;難得者,君子不貴,教以養也。雖天子必有尊也,教以孝也;必有先也,教以弟也。教化之功不大乎?


傅曰:天生之,地載之,聖人教之。君者,民之心也;民者,君之體也。心之所好,體必安之;君之所好,民必從之。故君民者,貴孝弟而好禮義,重仁廉而輕財利,躬親職此於上,而萬民聽,故曰:「先王見教之可以化民也。」此之謂也。衣服容貌者,所以說目也;聲音應對者,所以說耳也;好惡去就者,所以說心也。故君子衣服中而容貌恭,則目說矣;言理應對遜,則耳說矣;好仁厚而惡淺薄,就善人而遠僻鄙,則心說矣。故曰:「行思可樂,容止可觀。」此之謂也。


\article{五行之義}

天有五行:一曰木,二曰火,三曰土,四曰金,五曰水。木,五行之始也;水,五行之終也;土,五行之中也。此其天次之序也。木生火,火生土,土生金,金生水,水生木,此其父子也。木居左,金居右,火居前,水居後,土居中央,此其父子之序,相受而布。是故木受水,而火受木,土受火,金受土,水受金也。諸授之者,皆其父也;受之者,皆其子也。常因其父以使其子,天之道也。是故木已生而火養之,金已死而水藏之,火樂木而養以陽,水克金而喪以陰,土之事火竭其忠。故五行者,乃孝子忠臣之行也。五行之為言也,猶五行與?是故以得辭也,聖人知之,故多其愛而少嚴,厚養生而謹送終,就天之制也。以子而迎成養,如火之樂木也。喪父,如水之克金也。事君,若土之敬天也。可謂有行人矣。五行之隨,各如其序,五行之官,各致其能。是故木居東方而主春氣,火居南方而主夏氣,金居西方而主秋氣,水居北方而主冬氣。是故木主生而金主殺,火主暑而水主寒,使人必以其序,官人必以其能,天之數也。土居中央,為之天潤。土者,天之股肱也。其德茂美,不可名以一時之事,故五行而四時者。土兼之也。金木水火雖各職,不因土,方不立,若酸咸辛苦之不因甘肥不能成味也。甘者,五味之本也;土者,五行之主也。五行之主土氣也,猶五味之有甘肥也,不得不成。是故聖人之行,莫貴於忠,土德之謂也。人官之大者,不名所職,相其是矣。天官之大者,不名所生,土是矣。

\article{陽尊陰卑}

天之大數,畢於十旬。旬天地之間,十而畢舉;旬生長之功,十而畢成。十者,天數之所止也。古之聖人,因天數之所止,以為數紀。十如更始,民世世傳之,而不知省其所起。知省其所起,則見天數之所始;見天數之所始,則知貴賤逆順所在;知貴賤逆順所在,則天地之情著,聖人之寶出矣。是故陽氣以正月始出於地,生育長養於上。至其功必成也,而積十月。人亦十月而生,合於天數也。是故天道十月而成,人亦十月而成,合於天道也。故陽氣出於東北,入於西北,發於孟春,畢於孟冬,而物莫不應是。陽始出,物亦始出;陽方盛,物亦方盛;陽初衰,物亦初衰。物隨陽而出入,數隨陽而終始,三王之正隨陽而更起。以此見之,貴陽而賤陰也。故數日者,據書而不據夜;數歲者,據陽而不據陰。陰不得達之義。是故《春秋》之於禮也,達宋公而不達紀侯之母。紀侯之母宜稱而不達,宋公不宜稱而達,達陽而不達陰,以天道制之也。丈夫雖賤皆為陽,婦人雖貴皆為陰。陰之中亦相為陰,陽之中亦相為陽。諸在上者皆為其下陽,諸在下者皆為其上陰。何名何有,皆並一於陽,昌力而辭功。故出去起雨,必令從之下,不敢有其所出,上善而下惡。惡者受之,善者不受。土若地,義之至也。是故《春秋》君不名惡,臣不名善,善皆歸於君,惡皆歸於臣。臣之義比於地,故為人臣者,視地之事天也。為人子者,視土之事火也。雖居中央,亦歲七十二日之王,傅於火以調和養長,然而弗名者,皆並功於火,火得以盛,不敢與父分功美,孝之至也。是故孝子之行,忠臣之義,皆法於地也。地事天也,猶下之事上也。地,天之合也,物無合會之義。是故推天地之精,運陰陽之類,以別順逆之理。安所加以不在?在上下,在大小,在強弱,在賢不肖,在善惡。惡之屬盡為陰,善之屬盡為陽。陽為德,陰為刑。刑反德而順於德,亦權之類也。雖曰權,皆在權成。是故陽行於順,陰行於逆。順行而逆者,陰也。是故天以陰為權,以陽為經。陽出而南,陰出而北。經用於盛,權用於末。以此見天之顯經隱權,前德而後刑也。故曰:陽天之德,陰天之刑也。陽氣暖而陰氣寒,陽氣予而陰氣奪,陽氣仁而陰氣戾,陽氣寬而陰氣急,陽氣愛而陰氣惡,陽氣生而陰氣殺。是故陽常居實位而行於盛,陰常居空位而行於末。天之好仁而近,惡戾之變而遠,大德而小刑之意也。先經而後權,貴陽而賤陰也。故陰,夏入居下,不得任歲事,冬出居上,置之空處也。養長之時伏於下,遠去之,弗使得為陽也。無事之時起之空處,使之備次陳,守閉塞也。此皆天之近陽而遠陰,大德而小刑也。是故人主近天之所近,遠天之所遠;大天之所大,小天之所小。是故天數右陽而不右陰,務德而不務刑。刑之不可任以成世也,猶陰之不可任以成歲也。為政而任刑,謂之逆天,非王道也。

\article{王道通三}

古之造文者,三書而連其中,謂之王。三書者,天地與人也,而連其中者,通其道也。取天地與人之中以為貫而參通之,非王者孰能當是?是故王者唯天之施,施其時而成之,法其命而循之諸人,法其烽而以起事,治其道而以出法,治其誌而歸之於仁。仁之美者在於天。天,仁也。天覆育萬物,既化而生之,有養而成之,事功無已,終而複始,凡舉歸之以奉人。察於天之意,無窮極之仁也。人之受命於天也,取仁於天而仁也。是故人之受命天之尊,父兄子弟之親,有忠信慈惠之心,有禮義廉讓之行,有是非逆順之治,文理燦然而厚,積知廣大有而博,唯人道為可以參天。天常以愛利為意,以養長為事,春秋冬夏皆其用也。王者亦常以愛利天下為意,以安樂一世為事,好惡喜怒而備用也。然而主之好惡喜怒,乃天之春夏秋冬也,其俱暖清寒暑而以變化成功也。天出此物者,時則歲美,不時則歲惡。人主出此四者,義則世治,不義則世亂。是故治世與美歲同數,亂世與惡歲同數,以此見人理之副天道也。天有寒有暑。夫喜怒哀樂之發,與清暖寒暑,其實一貫也。喜氣為暖而當春,怒氣為清而當秋,樂氣為太陽而當夏,哀氣為太陰而當冬。四氣者,天與人所同有也,非人所能蓄也,故可節而不可止也。節之而順,止之而亂。人生於天,而取化於天。喜氣取諸春,樂氣取諸夏,怒氣取諸秋,哀氣取諸冬,四氣之心也。四肢之答各有處,如四時;寒暑不可移,若肢體。肢體移易其處,謂之壬人;寒暑移易其處,謂之敗歲;喜怒移易其處,謂之亂世。明王正喜以當春,正怒以當秋,正樂以當夏,正哀以當冬。上下法此,以取天之道。春氣愛,秋氣嚴,夏氣樂,冬氣哀。愛氣以生物,嚴氣以成功,樂氣以養生,哀氣以喪終,天之志也。是故春氣暖者,天之所以愛而生之;秋氣清者,天之所以嚴而成之;夏氣溫者,天之所以樂而養之;冬氣寒者,天之所以哀而藏之。春主生,夏主養,秋主收,冬主藏。生溉其樂以養,死溉其哀以藏,為人子者也。故四時之行,父子之道也;天地之志,君臣之義也;陰陽之理,聖人之法也。陰,刑氣也;陽,德氣也。陰始於秋,陽始於春。春之為言,猶偆偆也;秋之為言,猶湫湫也。偆偆者,喜樂之貌也,湫湫者,憂悲之狀也。是故春喜夏樂,秋憂冬悲,悲死而樂生。以夏養春,以冬藏秋,大人之志也。是故先愛而後嚴,樂生而哀終,天之當也。而人資諸天。天固有此,然而無所之如其身而已矣。人主立於生殺之位,與天共持變化之勢,物莫不應天化。天地之化如四時。所好之風出,則為暖氣而有生於俗;所惡之風出,則為清氣而有殺於俗。喜則為暑氣而有養長也,怒則為寒氣而有閉塞也。人主以好惡喜怒變習俗,而天以暖清寒暑化草木。喜怒時而當則歲美,不時而妄則歲惡。天地人主一也。然則人主之好惡喜怒,乃天之暖清寒暑也,不可不審其處而出也。當暑而寒,當寒而暑,必為惡歲矣。人主當喜而怒,當怒而喜,必為亂世矣。是故人主之大守,在於謹藏而禁內,使好惡喜怒必當義乃出,若暖清寒暑之必當其時乃發也。人主掌此而無失,使乃好惡喜怒未嘗差也,如春秋冬夏之未嘗過也,可謂參天矣。深藏此四者而勿使妄發,可謂天矣。

\article{天容}

天之道,有序而時,有度而節,變而有常,反而有相奉,微而至遠,踔而致精,一而少積蓄,廣而實,虛而盈。聖人視天而行。是故其禁而審好惡喜怒之處也,欲合諸天之非其時,不出暖清寒暑也;其告之以政令而化風之清微也,欲合諸天之顛倒其一而以成歲也;其羞淺末華虛而貴敦厚忠信也,欲合諸天之默然不言而功德積成也;其不阿黨偏私而美測愛兼利也,欲合諸天之所以成物者少霜而多露也。其內自省以是而外顯,不可以不時,人主有喜怒,不可以不時。可亦為時,時亦為義,喜怒以類合,其理一也。故義不義者,時之合類也,而喜怒乃寒暑之別氣也。

\article{天辨在人}

難者曰:陰陽之會,一歲再遇。遇於南方者以中夏,遇於北方者以中冬。冬喪物之氣也,則其會於是何?如金木水火,各奉其所主以從陰陽,相與一力而並功。其實非獨陰陽也,然而陰陽因之以起,助其所主。故少陽因木而起,助春之生也;太陽因火而起,助夏之養也;少陰因金而起,助秋之成也;太陽因水而起,助冬之藏也。陰雖與水並氣而合冬,其實不同,故水獨有喪而陰不與焉。是以陰陽會於中冬者,非其喪也。春愛誌也,夏樂誌也,秋嚴誌也,冬哀誌也。故愛而有嚴,樂而有哀,四時之則也。喜怒之禍,哀樂之義,不獨在人,亦在於天,而春夏之陽,秋冬之陰,不獨在天,亦在於人。人無春氣,何以博愛而容眾?人無秋氣,何以立嚴而成功?人無夏氣,何以盛養而樂生?人無冬氣,何以哀死而恤喪?天無喜氣,亦何以暖而春生育?天無怒氣,亦何以清而秋殺就?天無樂氣,亦何以疏陽而夏養長?天無哀氣,亦何以激陰而冬閉藏?故曰:天乃有喜怒哀樂之行,人亦有春秋冬夏之氣者,合類之謂也。匹夫雖賤,而可以見德刑之用矣。是故陰陽之行,終各六月,遠近同度,而所在異處。陰之行,春居東方,秋居西方,夏居空右,冬居空左,夏居空下,冬居空上,此陰之常處也。陽之行,春居上,冬居下,此陽之常處也。陰終歲四移,而陽常居實,非親陽而疏陰,任德而遠刑與?天之志,常置陰空處,稍取之以為助。故刑者德之輔,陰者陽之助也,陽者歲之主也。天下之草木隨陽而生落,天下之三王隨陽而改正,天下之尊卑隨陽而序位。幼者居陽之所少,老者居陽之所老,貴者居陽之所盛,賤者居陽之所衰。藏者,言其不得當陽。不當陽者臣子是也,當陽者勻是也。故人主南面,以陽為位也。陽貴而陰賤,天之制也。禮之尚右,非尚陰也,敬老陽而尊成功也。

\article{陰陽位}

陽氣始出東北而南行,就其位也;西轉而北入,藏其休也。陰氣始出東南而北行,亦就其位也;西轉而南入,屏其伏也。是故陽以南方為位,以北方為休;陰以北方為位,以南方為伏。陽至其位而大暑熱。陰至其位而大寒凍。陽至其休而入化於地,陰至其伏而避德於下。是故夏出長於上、冬入化於下者,陽也;夏入守虛地於下,冬出守虛位於上者,陰也。陽出實入實,陰出空入空,天之任陽不任陰,好德不好刑,如是也。故陰陽終歲各一出。

\article{陰陽終始}

天之道,終而複始。故北方者,天之所終始也,陰陽之所合別也。冬至之後,陰而西入,陽仰而東出,出入之處常相反也。多少調和之適,常相順也。有多而無溢,有少而無絕。春夏陽多而陰少,秋冬陽少而陰多,多少無常,未嘗不分而相散也。以出入相損益,以多少相溉濟也。天所起一,動而再倍,常乘反衛再登之勢,以就同類,與之相報,故其氣相俠,而以變化相輸也。春秋之中,陰陽之氣俱相並也。中春以生,中秋以殺。由此見之,天之所起其氣積,天之所廢其氣隨。故至春少陽東出就木,與之俱生,至夏太陽南出就火,與之俱暖。此非各就其類而與之相起與?少陽就木,太陽就火,火木相稱,各就其正。此非正其倫與?至於秋時,少陰同而不得以秋從金,從金而傷火功,雖不得以從金,亦以秋出於東方,其其處而適其事,以成歲功。此非權與?陰之行,固常居虛而不得居實。至於冬而止空虛,太陽乃得北就其類,而與水起寒。是故天之道有倫有經、有權。

\article{陰陽義}

天地之常,一陰一陽。陽者天之德也,陰者天之刑也。跡陰陽終歲之行,以觀天之所親而任。成天之功,猶謂之空,空者之實也。故清凓之於歲也,若酸醎之於味也,僅有而已矣。聖人之治,亦從而然。天之少陰用於功,太陰用於空。人之少陰用於嚴,而太陰用於喪。喪亦空,空亦喪也。是故天之道以三時成生,以一時喪死。死之者,謂百物枯落也;喪之者,謂陰氣悲哀也。天亦有喜怒之氣、哀樂之心,與人相副。以類合之,天人一也。春,喜氣也,故生;秋,怒氣也,故殺;夏,樂氣也,故養;冬,哀氣也,故藏。四者天人同有之。有其理而一用之。與天同者大治,與天異者大亂。故為人主之道,莫明於在身之與天同者而用之,使喜怒必當義而出,如寒暑之必當其時乃發也。使德之厚於刑也,如陽之多於陰也。是故天之行陰氣也,少取以成秋,其余以歸之冬。聖人之行陰氣也,少取以立嚴,其余以歸之喪。喪亦人之冬氣,故人之太陰,不用於刑而用於喪,天之太陰,不用於物而用於空。空亦為喪,喪亦為空,其實一也,皆喪死亡之心也。

\article{陰陽出入上下}
天道大數,相反之物也,不得俱出,陰陽是也。春出陽而入陰,秋出陰而入陽,夏右陽而左陰,冬右陰而左陽。陰出則陽入,陽出則陰入;陰右則陽左,陰左則陽右。是故春俱南,秋俱北,而不同道;夏交於前,冬交於後,而不同理。立行而不相亂,澆滑而各持分,此之謂天之意。而何以從事?天之道,初薄大冬,陰陽各從一方來,而移於後。陰由東方來西,陽由西方來東,至於中冬之月,相遇北方,合而為一,謂之曰至。別而相去,陰適右,陽適左。適左者其道順,適右者其道逆。逆氣左上,順氣右下,故下暖而上寒。以此見天之冬右陰而左陽也,上所右而下所左也。冬月盡,而陰陽俱南還,陽南還出於寅,陰南還入於戌,此陰陽所始出地入地之見處也。至於仲春之月,陽在正東,陰在正西,謂之春分。春分者,陰陽相半也,故晝夜均而寒暑平。陰日損而隨陽,陽日益而鴻,故為暖熱初得。大夏之月,相遇南方,合而為一,謂之日至。別而相去,陽適右,陰適左。適左由下,適右由上,上暑而下寒,以此見天之夏右陽而左陰也。上其所右,下其所左。夏月畫,而陰陽俱北還。陽北還而入於申,陰北還而出於辰,此陰陽之所始出地入地之見處也。至於中秋之月,陽在正西,陰在正東,謂之秋分。秋分者,陰陽相半也,故晝夜均,而寒暑平。陽日損而隨陰,陰日益而鴻,故至於季秋而始霜,至於孟冬而始寒,小雪而物咸成,大寒而物畢藏,天地之功終矣。

\article{天道無二}

天之常道,相反之物也,不得兩起,故謂之一。一而不二者,天之行也。陰與陽,相反之物也,故或出或入,或右或左,春俱南,秋俱北,夏交於前,冬交於後,行而不同路,交會而各代理,此其文與?天之道,有一出一入,一休一伏,其度一也,然而不同意。陽之出,常懸於前而任歲事;陰之出,常懸於後而守空虛。陽之休也,功已成於上而伏於下;陰之伏也,不得近義而遠其處也。天之任陽不任陰,好德不好刑如是。故陽出而前,陰出而後,尊德而卑刑之心見矣。陽出而積於夏,任德以歲事也;陰出而積於冬,錯刑於空處也。必以此察之。天無常於物,而一於時。時之所宜,而一為之。故開一塞一,起一廢一,至畢時而止,終有複始於一。一者,一也。是於天凡在陰位者皆惡亂善,不得主名,天之道也。故常一而不滅,天之道。事無大小,物無難易。反天之道,無成者。是以目不能二視,耳不能二聽,手不能二事。一手畫方,一手畫圓,莫能成。人為小易之物,而終不能成,反天之不可行如是。是故古之人物而書文,心止於一中者,謂之忠;持二中者,謂之患。患,人之中不一者也。不一者,故患之所由生也。是故君子賤二而貴一。人孰無善?善不一,故不足以立身。常不一,故不足以臻功。《詩》云:「上帝臨汝,無二爾心。」知天道者之言也。

\article{煖燠常多}

天之道,出陽為煖以生之,出陰為清以成之。是故非薰也不能有育,非溧也不能有熟,歲之精也。知心而不省薰與溧孰多者,用之必與天戾。與天戾,雖勞不成。是自正月至於十月,而天之功畢。計其間,陰與陽各居幾何,薰與溧其日孰多。距物之初生,至其畢成,露與霜其下孰倍。故從中春至於秋,氣溫柔和調。及季秋九月,陰乃始多於陽,天於是時出溧下霜。出溧下霜,而天降物固已皆成矣。故九月者,天之功大究於是月也,十月而悉畢。故案其跡,數其實,清溧之日少少耳。功已畢成之後,陰乃大出。天之成功也,少陰與而太陰不與,少陰在內而太陰在外。故霜加於物,而雪加於空,空者地而已,不逮物也。功已畢成之後,物未複生之前,太陰之所當出也。雖曰陰,亦以太陽資化其位,而不知所受之。故聖主在上位,天覆地載,風令雨施。雨施者,布德均也;風令者,言令直也。《詩》云:「識不知,順帝之則。」言弗能知識,而效天之所為云爾。禹水湯旱,非常經也,適遭世氣之變,而陰陽失平。堯視民如子,民視堯如父母。《尚書》曰:「二下有八載,放動乃殂落,百姓如喪考妣。四海之內,於密八音三年。」三年陽氣於陰,陰氣大同,此禹所以有水名也。桀,天下之殘賊也;湯,天下之盛德也。天睛除殘賊而得盛德大善者再,是重陽也,故湯有旱之名。皆適遭之變,非禹湯之過。毋以適遭之變疑平生之常,則所守不失,則正道益明。

\article{基義}

凡物必有合。合,必有上,必有下,必有左,必有右,必有前,必有後,必有表,必有裹。有美必有惡,有順必有逆,有喜必有怒,有寒必有暑,有書必有夜,此皆其合也。陰者陽之合,妻者夫之合,子者父之合,臣者君之合。物莫無合,而合各有陰陽。陽兼於陰,陰兼於陽,夫兼於妻,妻兼於夫,父兼於子,子兼於父,君兼於臣,臣兼於君。君臣、父子、夫婦之義,皆取諸陰陽之道。君為陽,臣為陰;父為陽,子為陰;夫為陽,妻為陰。陰道無所獨行。其始也不得專起,其終也不得分功,有所兼之義。是故臣兼功於君,子兼功於父,妻兼功於夫,陰兼功於陽,地兼功於天。舉而上者,抑而下也;有屏而左也,有引而右也;有親而任也,有疏而遠也;有欲日益也,有欲日損也。益其用而損其妨。有時損少而益多,有時損多而益少。少而不至絕,多而不至溢。陰陽二物,終歲各壹出。壹其出,遠近同度而不同意。陽之出也,常懸於前而任事;陰之出也,常懸於後而守空處。此見天之親陽而疏陰,任德而不任刑也。是故仁義制度之數,盡取之天。天為君而覆露之,地為臣而持載之;陽為夫而生之,陰為婦而助之;春為父而生之,夏為子而養之;秋為死而棺之,冬為痛而喪之。王道之三綱,可求於天。天出陽,為暖以生之;地出陰,為清以成之。不暖不生,不清不成。然而計其多少之分,則暖暑居百而清寒居一。德教之與刑罰猶此也。故聖人多其愛而少其嚴,厚其德而簡其刑,以此配天。天之大數必有十旬。旬,天地之數,十而畢舉,旬,生長之功,十而畢成。天之氣徐,乍寒乍暑。故寒不凍,暑不,以其有余徐來,不暴卒也。《易》曰「履霜堅冰」,蓋言遜也。然則上堅不逾等,果是天之所為,弗作而成也。人之所為,亦當弗作而極也。凡有興者,稍稍上之以遜順往,使人心說而安之,無使人心恐。故曰:君子以人治人,能願。此之謂也。聖人之道,同諸天地,蕩諸四海,變易習俗。

\article{四時之副}

天之道,春暖以生,夏暑以養,秋清以殺,冬寒以藏。暖暑清寒,異氣而同功,皆天之所以成歲也。聖人副天之所行以為政,故以慶副暖而當春,以賞副暑而當夏,以罰副清而當秋,以刑副寒而當冬。慶賞罰刑,畢事而同功,皆王者之所以成德也。慶賞罰刑與春夏秋冬,以類相應也,如合符。故曰王者配天,謂其道。天有四時,王有四政,四政若四時,通類也,天人所同有也。慶為春,賞為夏,罰為秋,刑為冬。慶賞罰刑之不可不具也,如春夏秋冬不可不備也。慶賞罰刑,當其處不可不發,若暖暑清寒,當其時不可不出也。慶賞罰刑各有正處,如春夏秋冬各有時也。四政者,不可以相干也,猶四時不可相干也。四政者,不可以易處也,猶四時不可易處也。故慶賞罰刑有不行於其正處者,《春秋》譏也。

\article{人副天數}

天德施,地德化,人德義。天氣上,地氣下,人氣在其間。春生夏長,百物以同;秋殺冬收,百物以藏。故莫精於氣,莫富於地,莫神於天。天地之精所以生物者,莫貴於人。人受命乎天也,故超然有以倚。物疾莫能為仁義,唯人獨能為仁義;物疾莫能偶天地,唯人獨能偶天地。人有三百六十節,偶天之數也;形體骨肉,偶地之厚也。上有耳目聰明,日月之象也;體有空穹進脈,川谷之象也;心有哀樂喜怒,神氣之類也。觀人之禮一,何高物之甚,而類於天也。物旁折取天之陰陽以生活耳,而人乃爛然有文理。是故凡物之形,莫不伏從旁折天地而行,人獨題直立端尚,正正當之。是故所取天地少者,旁折之;所取天地多者,正當之。此見人之絕於物而參天地。是故人之身,首而員,象天容也;發,象星辰也;耳目戾戾,象日月也;鼻口呼吸,象風氣也;胸中達知,象神明也,腹胞實虛,象百物也。百物者最近地,故要以下,地也。天地之象,以要為帶。頸以上者,精神尊嚴,明天類之狀也;頸而下者,豐厚卑辱,土壤之比也。足布而方,地形之象也。是故禮,帶置紳必直其頸,以別心也。帶而上者盡為陽,帶而下者盡為陰,陽,天氣也;陰,地氣也。故陰陽之動,使人足病,喉起,則地氣痹起,則地氣上為雲雨,而象亦應之也。天地之符,陰陽之副,常設於身,身猶天也,數與之相參,故命與之相連也。天以終歲之數,成人之身,故小節三百六十六,副日數也;大節十二分,副月數也;內有五藏,副五行數也;外有四肢,副四時數也;乍視乍瞑,副晝夜也;乍剛乍柔,副冬夏也;乍哀乍樂,副陰陽也;心有計慮,副度數也;行有倫理,副天地也。此皆暗膚著身,與人俱生,比而偶之合。於其可數也,副數;不可數者,副類。皆當同而副天,一也。是故陳其有形以著其無形者,拘其可數以著其不可數者。以此言道之,亦宜以類相應,猶其形也,以數相中也。

\article{同類相動}

今平地注水,去燥就濕,均薪施火,去濕就燥。百物去其所與異,而從春所與同,故氣同則會,聲比則應,其驗然也。試調琴瑟而錯之,鼓其宮則他宮應之,鼓其商而他商應之,五音比而自鳴,非有神,其數然也。美事召美類,惡事召惡類,類之相應而起也。如馬鳴則馬應之,牛鳴則牛應之。帝王之將同也,其美祥亦先見;其將亡也,妖孽亦先見。物故以類相召也,故以龍致雨,以扇逐暑,軍之所處以棘楚。美惡皆有從來,以為命,莫知其處所。天將陰雨,人之病故為之先動,是陰相應而起也。天將欲陰雨,又使人欲睡臥者,陰氣也。有憂亦使人臥者,是陰相求也;有喜者,使人不欲臥者,是陰相索也。水得夜益長數分,東風而酒湛溢,病者至夜而疾益甚,雞至幾明,皆鳴而相薄。陽陰之氣,因可以類相益損也。天有陰陽,人亦有陰陽。天地之陰氣起,而人之陰氣應之而起,人之陰氣起,而天地之陰氣亦宜應之而起,其道一也。明於此者,欲致雨則動陰以起陰,欲止雨則動陽以起陽,故致雨非神也。而疑於神者,其理微妙也。非獨陰陽之氣可以類進退也,雖不祥禍福所從生,亦由是也。無非己先起之,而物以類應之而動者也。故聰明聖神,內視反聽,言為明聖,內視反聽,故獨明聖者知其本心皆在此耳。故琴瑟報彈其宮,他宮自鳴而應之,此物之以類動者也。其動以聲而無形,人不見其動之形,則謂之自鳴也。又相動無形,則謂之自然,其實非自然也,有使之然者矣。物固有實使之,其使之無形。《尚書大傅》言:「周將同之時,有大赤鳥銜之種,而集王屋之上者,武王喜,諸大夫皆喜。周公曰:『茂哉!茂哉!天之見此以勸之也。』」

\article{五行相勝}

木者,司農也。司農為奸,朋黨比周,以蔽主明,退匿賢士,絕滅公卿,教民奢侈,賓客交通,不勸田事,博戲雞,走狗弄馬,長幼無禮,大小相瞄,為寇賊,橫恣絕理。司徒誅之,齊桓是也。行霸任兵,侵蔡,蔡潰,遂伐楚,楚人降伏,以安中國。木者,君之官也。夫木者農也,農者民也,不順如叛,則命司徒誅其率正矣。故曰金勝木。


火者,司馬也。司馬為讒,反言易辭以譖人,內離骨肉之親,外疏忠臣,賢聖旋亡,讒邪日昌,魯上大夫季孫是也。專權擅政,薄國威德,反以怠惡,譖其賢臣,劫惑其君。孔子為魯司寇,據義行法,季孫自消,墮費城,兵甲有差。夫火者,大朝,有邪讒熒惑其君,執法誅之。執法者水也,故曰水勝火。


土者,君之官也。其相司營。司營為神,主所為皆曰可,主所言皆曰善,順主指,聽從為比。進主所善,以快主意,導主以邪,陷主不義。大為宮室,多為台榭,雕文刻鏤,五色成光。賦斂無度,以奪民財;多發繇役,以奪民時,百姓罷弊而叛,及其身弒。夫土者,君之官也,君大奢侈,過度失禮,民叛矣。其民叛,其君窮矣。故曰木勝土。


金者,司徒也。司徒為賊,內得於君,外驕軍士,專權擅勢,誅殺無罪,侵伐暴虐,攻戰妄取,令不行,禁不止,將率不親,士卒不使,兵弱地削,令君有恥,則司馬誅之,楚殺其司徒得臣是也。得臣數戰破敵,內得於君,驕蹇不其下,卒不為使,當敵而弱,以危楚國,司馬誅之。金者,司徒,司徒弱,不能使士眾,則司馬誅之,故曰火勝金。


水者,司寇也。司寇為亂,足恭不謹,巧言令色,阿黨不平,慢令爭誅,誅殺無罪,則司營誅之,營蕩是也。為齊司寇。太公封於齊,問焉以治國之要,營蕩對曰:「任仁義而已。」太公曰:「任仁義奈何?」營蕩對曰:「仁者愛人,義者尊老。」太公曰:愛人尊老奈何?」營蕩對曰:「愛人者,有子不食其力;尊老者,妻長而夫拜之。」太公曰:「寡人欲以仁義治齊,今子以仁義亂齊,寡人立而誅之,以定齊國。」夫水者,執法司寇也。執法附黨不平,則司營誅之,故曰土勝水。


\article{五刑相生}

天地之氣,合而為一,分為陰陽,判為四時,列為五行。行者行也,其行不同,故謂之五行。五行者,五官也,比相生而間相勝也。故為治,逆之則亂,順之則治。


東方者木,農之本。司農尚仁,進經術之士,道之以帝王之路,將順其美,匡其惡。執規而生,至溫潤下,知地形肥磽美惡,立事生則,因地之宜,召公是也。親入南畝之中,觀民墾草發淄,耕種五穀,積蓄有余,家給人足,倉庫充實。司馬,本朝也。本朝者火也,故曰木生火。


南方者火也,本朝。司馬尚智,進賢聖之士,上知天文,其形兆未見,其萌芽未生,昭然獨見存亡之機,得失之要,治亂之源,豫禁未然之前,至忠厚仁,輔翼其君,周公是也。成王幼弱,周公相,誅管叔蔡叔,以定天下。天下既寧以安君。官者,司營也。司營者土也,故曰火生土。


中央者土,君官也。司營尚信,卑身賤體,夙同夜寐,稱述往古,以厲主意。明見成敗,微諫納善,防滅其惡,絕源塞執繩而制四方,至忠厚信,以事其君,據義割恩,太公是也。應天因時之化,威武強御以成。大理者,司徒也。司徒者金也,故曰土生金。


西方者金,大理司徒也。司徒尚義,臣死君而眾人死父。親有尊卑,位有上下,各死其事,事不逾矩,執權而伐。兵不苟克,取不苟得,義而後行,至廉而威,質直剛毅,子是也。伐有罪,討不義,是以百姓附親,邊境安寧,寇賊不發,邑無獄訟,則親安。執法者,司寇也。司寇附親,邊境安寧,寇賊不發,邑無獄訟,則親安。執法者,司寇也。司寇者,水也。故曰金生水。


北方者水,執法司寇也。司寇尚禮,君臣有位,長幼有序,朝廷有爵,鄉黨以齒,升降揖讓,般伏拜竭,折旋中矩,立而折,拱則抱鼓,執衡而藏,至清廉平,賂遣不受,請謁不聽,據法聽訟,無有所阿,孔子是也。為魯司寇,斷獄屯屯,與眾共之,不敢自專。是死者不恨,生者不怨,百工維時,以成器械。器械既成,以給司農。司農者,田官也。田官者木,故曰水生木。



\article{五行逆順}

木者春,生之性,農之本也。勸農事,無奪民時,使民,歲不過三日,行什一之稅,進經術之士。挺群禁,出輕擊,去稽留,除桎梏,開門闔,通障塞。恩及草木,則樹木華美,而朱草生;恩及鱗蟲,則魚大為,鯨不見,如人君出入不時,走狗試馬,馳騁不反宮室,好淫樂,飲酒沈湎,從恣,不顧政治,事多發役,以奪民時,作謀增稅,以奪民財,民病疥搔,溫體,足痛。咎及於木,則茂木枯槁,工匠之輪多傷敗。毒水群,漉陂如漁,咎及鱗蟲,則魚不為,群龍深藏,鯨出見。


火者夏,成長,本朝也。舉賢良,進茂才,官得其能,任得其力,賞有功,封有德,出貨財,振困乏,使四方。恩及於火,則火順人而甘露降;恩及羽蟲,則飛鳥大為,黃鵠出見,鳳凰翔。如人君惑於讒邪,內離骨肉,外疏忠臣,至殺世子,誅殺不辜,逐忠臣,以妾為妻,棄法令,婦妾為政,賜予不當,則民病備壅腫,目不明。咎及於火,則大旱,必有火;咎及羽蟲,則飛鳥不為,冬應不來,梟鴟群嗚,成熟百種,君之官。循宮室之制,謹夫婦之別,加親戚之恩。恩及於土,則五穀成,而嘉禾同。恩及蟲,則百姓親附,城郭充實,賢聖皆遷,仙人降。如人君好淫佚,妻妾過度,犯親戚,侮父兄,欺罔百姓,大為台榭,五色成光,雕文刻鏤,則民病心腹宛黃,舌爛痛。咎及於土,則五穀不成;暴虐妄誅,咎及蟲,蟲不為,百姓叛去,賢聖放亡。


金者秋,殺氣之始也。建立旗鼓,杖把旄鉞,以誅賊殘,禁暴虐,安集,故動眾同師,必應義理,出則祠兵,入則振旅,以閒習之。因於搜狩,存不忘亡,安不忘危。飭兵甲,警百官,誅不法。恩及於金石,則涼風出;恩及於毛蟲,則走獸大為,麒麟至。如人君好戰,侵陵諸侯,貪城邑之賂,輕百姓之命,則民病喉咳嗽,筋攣,鼻鼽塞。咎及於金,則鑄化凝滯,凍堅不成;焚林而獵,咎及毛蟲,則走獸不為,白虎妄搏,麒麟遠去。


水者冬,藏至陰也。宗廟祭祀之始,敬四時之祭,昭穆之序。天子祭天,諸侯祭土。閉門閭,大搜索,斷刑罰,執當罪,飭關梁,禁外徙。恩及於水,則豐醴泉出;恩及介蟲,則黿鼉大為,如人君簡宗廟,不禱祀,廢祭祀,執法不順,逆天時,則民病流腫,痿,孔窺不通。咎及於水,霧氣冥冥,必有大水,水為民害;咎及介蟲,則龜深藏,黿鼉。


\article{治水五行}

日冬至,七十二日木用事,其氣燥濁而青。七十二日火用事,其氣慘陽而赤。七十二日土用事,其氣濕濁而黃。七十二日金用事,其氣慘淡而白。七十二日水用事,其氣清寒而黑。七十二日複得木。木用事,則行柔惠,挺群禁。至於立春,出輕擊,去稽留,除桎梏,開門闔,存幼孤,矜寡獨,無伐木。火用事,則正封疆,循田疇。至於立夏,舉賢良,封有德,賞有功,出使四方,無縱火。土用事,則養長老,存幼孤,矜寡獨,賜孝弟,施恩澤,無同土功。金用事,則修城郭,繕牆垣,審群禁,飭甲兵,警百官,誅不法,存長老,無焚金石。水用事,則閉門閭,大搜索,斷刑罰,執當罪,飭關梁,禁外徙,無決堤。

\article{治亂五行}

火干木,蟄蟲蚤出,雷蚤行。

土干木,胎夭卵鳥蟲多傷。金干木,有兵。水干木,春下霜。


土干火,則多雷。金干火,草木夷。木干火,則地動。

金干土,則五穀傷,有殃。水干土,夏寒雨霜。木干土,蟲不為。


水干金,則魚不為。木干金,則草木再生。火干金,則草木秋榮。土干金,五穀不成。

木干水,冬蟄不藏。土干水,則蟄蟲冬出。火干水,則星墜。金干水,則冬大寒。

\article{五行變救}
五行變至,當救之以德,施之天下,則咎除。不救以德,不出三年,天當雨石。木有變,春凋秋榮。秋木冰,春多雨。此繇役眾,賦斂重,百姓貧窮叛去,道多饑人。救之者,省繇役,薄賦斂,出倉谷,振困窮矣。火有變,冬溫夏寒。此王者不明,善者不賞,惡者不絀,不肖在位,賢者伏匿,則寒暑失序,而民疾疫。救之者,舉賢良,賞有功,封有德。土有變,大風至,五穀傷。此不信仁賢,不敬父兄,淫無度,宮室榮。救之者,省宮室,去雕文,舉孝悌,恤黎元。金有變,畢昴為回,三覆有武,多兵,多盜寇。此棄義貪財,輕民命,重貨賂,百姓趣利,多奸軌。救之者,舉廉潔,立正直,隱武行文,束甲械。水有變,冬濕多霧,春夏雨雹。此法令緩,刑罰不行。救之者,憂囹圄,案奸宄,誅有罪,舊五日。

\article{五行五事}

王者與臣無禮,貌不肅敬,則木不曲直,而夏多暴風。風者,木之氣也,其音角也,故應之以暴風。王者言不從,則金不從革,而秋多霹靂。霹靂者,金氣也,其音商也,故應之以霹靂。王者視不明,則火不炎上,而秋多電。電者,火氣也,其陰徵也,故應之以電。王者聽不聰,則水不潤下,而春夏多暴雨。雨者,水氣也,其音羽也,故應之以暴雨。王者心不能容,則稼穡不成,而秋多雷。雷者,土氣也,其音宮也。故應之以雷。


五事,一曰貌,二曰言,三曰視,四曰聽,五曰思。何謂也?夫五事者,人之所受命於天也,而王者所修而治民也。故王者為民,治則不可以不明,準繩不可以不正。王者貌曰恭,恭者敬也。言曰從,視曰明,明者知賢不肖,分明黑白也。聽曰聰,聰者能聞事而審其意也。思曰容,容者言無不容。恭作肅,從作,明作哲,聰作謀,容作聖。何謂也?恭作肅,言王者誠能內有恭敬之姿,而天下莫不肅矣。從作,言王者言可從,明正從行而天下治矣。明作哲,哲者知也,王者明則賢者進,不肖者退,天下知善而勸之,知惡而恥之矣。聰作謀,謀者謀事也,王者聰則聞事與臣下謀之,故事無失謀矣。王者心寬大無不容,則聖能施設,事各得其宜也。


王者能敬,則肅,肅則春氣得,故肅者主春。春陽氣微,萬物柔易,移弱可化,於時陰氣為賊,故王者欽。欽不以議陰事,然後萬物遂生,而木可曲直也。春行秋政,則草木凋;行冬政,則雪;行夏政,則殺。春失政則。


王者能治,則義立,義立則秋氣得,故者主秋。秋氣始殺,王者行小刑罰,民不犯則禮義成。於時陽氣為賊,故王者輔以官牧之事,然後萬物成熟。秋草木不榮華,秋行春政,則華;行夏政,則喬;行冬政,則落。秋失政,則春大風不解,雷不發聲。


王者能知,則知善惡,知善惡則夏氣得,故哲者主夏。夏陽氣始盛,萬物兆長,王者不搶明,則道不退塞。而夏至之後,大暑隆,萬物茂育懷任,王者恐明不知賢不肖,分明白黑。於時寒為賊,故王者輔以賞賜之事,然後夏草木不霜,火炎上也。夏行春政,則風;行秋政,則水;行冬政,則落。夏失政,則冬不凍冰,五穀不藏,大寒不解。


王者無失謀,然後冬氣得,故謀者主冬。冬陰氣始盛,草木必死,王者能聞事,審謀慮之,則不侵伐。不侵伐且殺,則死者不恨,生者不怨。冬日至之後,大寒降,萬物藏於下。於時暑為賊,故王者輔之以急斷之事,以水潤下也。冬行春政,則蒸;行夏政,則雷;行秋政,則旱。冬失政,則夏草木不實。五穀疾枯。




\article{郊語}

人之言:醞去煙,鴟羽去眯,慈石取鐵,頸金取火,蠶珥絲於室,而絕於堂,蕪荑生於燕,橘枳死於荊,此十物者,皆奇而可怪,非人所意也。夫非人所意而然,既已有之矣,或者吉凶禍福、利不利之所從生,無有奇怪,非人所意,如是者乎?此等可畏也。孔子曰:「君子有三畏:畏天命,畏大人,畏聖人之言。」彼豈無傷害於人,如孔子徒畏之哉!以此見天之不可不畏敬,猶主上之不可不謹事。不謹事主,其禍來至顯,不畏敬天,其殃來至暗。暗者不見其端,若自然也。故曰:堂堂如天,殃言不必立校,默而無聲,潛而無形也。由是觀之,天殃與主罰所以別者,暗與顯耳。不然,其來逮人,殆無以異。孔子同之,俱言可畏也。天地神明之心,與人事成敗之真,固莫之能見也,唯聖人能見之。聖人者,見人之所不見者也,故聖人之言亦可畏也。奈何如廢郊禮?郊禮者,人所最甚重也。廢聖人所最甚重,而吉凶利害在於冥冥不可得見之中,雖已多受其病,何從知之?故曰:問聖人者,問其所為而無問其所以為也。問其所以為,終弗能見,不如勿問。問為而為之,不為而勿為,是與聖人同實也,何過之有?《詩》云:「不騫不忘,率由舊章。」舊章者,先聖人之故文章也。率由,各有修從之也。此言先聖人之故文章者,雖不能深見而詳知其則,猶不知其美譽之功矣。故古之聖王,文章之最重者也,前世王莫不從重,栗精奉之,以事上天。至於秦而獨闕然廢之,一何不率由舊章之大甚也!天者,百神之大君也。事天不備,雖百神猶無益也。何以言其然也?祭而地神者,《春秋》譏之。孔子曰:「獲罪於天,無所禱也。」是其法也。故未見秦國致天福如周國也。《詩》云:「唯此文王,小心翼翼,昭事上帝,允懷多福。」多福者,非謂人也,事功也,謂天之所福也。傅曰:「周國子多賢,蕃殖至於駢孕男者四,四乳而得八男,皆君子俊雄也。」此天之所以興周國也,非周國之所能為也。今秦與周俱得為天子,而所以事天者異於周。以郊為百神始,始入歲首,必以正月上辛日先享天,乃敢於地,先貴之義也。夫歲先之與歲弗行也,相去遠矣。天下福若無可怪者,然所以久弗行者,非灼灼見其當而故弗行也,典禮之官常嫌疑,莫能昭昭明其當也。今切以為其當與不當,可內反於心而定也。堯謂舜曰「天之歷數在爾躬。」言察身以知天也。今身有子,孰不欲其有子禮也。聖人正名,名不虛生。天子者,則天之子也。以身度天,獨何為不欲其子之有子禮也。今為其天子,而闕然無祭於天,天何必善之?所聞曰:天下和平,則災害不生。今災害生,見天下未和平也。天下所未和平者,天子之教化不行也。《詩》曰:「有覺德行,四國順之。」覺者著也,王者有明著之德行於世,則四方莫不響應,風化善於彼矣。故曰:悅於慶賞,嚴於刑罰,疾於法令。

\article{郊義}

郊義,《春秋》之法,王者歲一祭天於郊,四祭於宗廟。宗廟因於四時之易,郊因於新歲之初,聖人有以起之,其以祭不可不親也。天者,百神之君也,王者之所最尊也。以最尊天之故,故易始歲更紀,即以其初郊。郊必以正月上辛者,言以所最尊,首一歲之事。每更紀者以郊,郊祭首之,先貴之義,尊天之道也。

\article{郊祭}

《春秋》之義,國有大喪者,止宗廟之祭,而不止郊祭,不敢以父母之喪,廢事天地之禮也。父母之喪,至哀痛悲苦也,尚不敢廢郊也,孰足以廢郊者?故其在禮,亦曰:「喪者不祭,唯祭天為越喪而行事。」夫古之畏敬天而重天郊,如此甚也。今群臣學士不探察,曰:「萬民多貧,或頗饑寒,足郊乎?」是何言之誤!天子父母事天,而子孫畜萬民。民未遍飽,無用祭天者,是猶子孫未得食,無用食父母也。言莫逆於是,是其去禮遠也。先貴而後賤,孰貴於天子?天子號天之子也。奈何受為天子之號,而無天子之禮?天子不可不祭天也,無異人之不可以不食父。為人子而不事父者,天下莫能以為可。今為天之子而不事天,何以異是?是故天子每至歲首,必先郊祭以離開天,乃敢為地,行子禮也;每將同師,必先郊祭以告天,乃敢徵伐,行子道也。文王受天命而王天下,先郊乃敢行事,而興師伐崇。其《詩》曰:「芃芃棫樸,薪之槱之。濟濟辟王,左右趨之,濟濟闢王,左右奉璋。奉璋峨峨,髦土攸宜。」此郊辭也。其下曰:「淠彼涇舟,丞徒楫之。周王於邁,六師及之。」此伐辭也。其下曰:「文王受命,有此武功,既伐於崇,作邑於豐。」以此辭者,見文王受命則郊,郊乃伐崇,伐崇之時,民何處央乎?

\article{四祭}

古者歲四祭。四祭者,因四時之所生孰,而祭其先祖父母也。故春曰祠,夏曰祗,秋曰嘗,冬曰蒸。此言不失其時,以奉祭先祖也。過時不祭,則失為人子之道也。祠者,以正月始食韭也;祗者,以四月食麥也;嘗者,以七月嘗黍稷也;蒸者,以十月進初稻也。此天之經也,地之義也。孝子孝婦,緣天之時,因地之利。藝之稻麥黍稷,菜生谷熟,永思吉日,供具祭物,齊戒沐浴,潔清致敬,祀其先祖父母。孝子孝婦不使時過,己處之以愛敬,行之以恭讓,亦殆免於罪矣。

已受命而王,必先祭天,乃行王事,文王之伐崇是也。《詩》曰:「濟濟闢王,左右奉璋。奉璋峨峨,髦士攸宜。」此文王之郊也。其下之辭曰:「淠彼涇舟,丞徒楫之。周王於邁,六師及之。」此文王之伐崇也。上言奉璋,下言伐崇,以是見文王之先郊而後伐也。文王受命則郊,郊乃伐崇,崇國之民,方困於暴亂之君,未得被聖人德澤,而文王已郊矣。

\article{郊祀}

周宣王時,天下旱,歲惡甚,王憂之。其《詩》曰:「倬彼雲漢,昭回於天。王曰鳴呼!何辜今之人?天降喪亂,饑饉薦臻。靡神不舉,靡愛斯牲,圭璧既卒,寧莫我聽。旱既太甚,蘊隆蟲蟲。不殄祀,自郊徂宮。上下奠瘞,靡神不宗。後稷不克,上帝不臨。耗射下土,寧丁我躬。」宣王自以為不能乎後稷,不中乎上帝,故有此災。有此災,愈恐懼而謹事天。天若不予是家,是家者安得立為天子?立為天子者,天予是家。天予是家者,天使是家。天使是家者,是家天之所予也,天之所使也。天已予之,天已使之,其間不可以接天何哉?故《春秋》凡譏郊,未嘗譏君德不成於郊也。乃不郊而祭山川,失祭之敘,逆於禮,故必譏之。以此觀之,不祭天者,乃不可祭小神也。郊因先卜,不吉不敢郊。百神之祭不卜,而郊獨卜,郊祭最大也。《春秋》譏喪祭,不譏喪郊,郊不闢喪,喪尚不闢,況他物。郊祝曰:「皇皇上天,照臨下土。集地之靈,降甘風雨。庶物群生,各得其所。靡今靡古,維予一人某敬拜皇天之祜。」夫不自為言,而為庶物群生言,以人心庶天無尤焉。天無尤焉,而辭恭順,家珂喜也。右郊祀九句。九句者,陽數也。

\article{順命}

父者,子之天也;天者,父之天也。無天而生,未之有也。天者萬物之祖,萬物非天不生。獨陰不生,獨陽不生,陰陽與天地參然後生。故曰:父之子也可尊,母之子也可卑,尊者取尊號,卑者取卑號。故德侔天地者,皇天右而子之,號稱天子。其次有五等之爵以尊之,皆以國邑為號。其無德於天地之間者,州國人民,甚者不得系國邑。皆絕骨肉之屬,離人倫,謂之暗盜而已。無名姓號氏於天地之間,至賤乎賤者也。其尊至德,巍巍乎不可以加矣;其卑至賤,冥冥其無下矣。《春秋》列序位尊卑之陳,累累乎可得而觀也。雖暗且愚,莫不昭然。公子慶父,罪亦不當系於國,以親之故為之諱,而謂之齊仲孫,去其公子之親也。故有大罪,不奉其天命者,皆棄其天倫。人於天也,以道受命;其於人,以言受命。不若於道者,天絕之;不若於言者,人絕之。臣子大受命於君,辭而出疆,唯有社稷國家之危,猶得發辭而專安之,盟是也。天子受命於天,諸侯受命於天子,子受命於父,臣妾受命於君,妻受命於夫。諸所受命者,其尊皆天也,雖謂受命於天亦可。天子不能奉天之命,則廢而稱公,王者之後是也。公侯不能奉天子之命,則名絕而不得就位,衛侯朔是也。子不奉父命,則有伯討之罪,衛世子蒯聵是也。臣不奉君命,雖善以叛,言晉趙鞅入於晉陽以叛是也。妾不奉君之命,則媵女先至者是也。妻不奉夫之命,則絕,夫不言及是也。曰:不奉順於天者,其罪如此。

孔子曰:「畏天命,畏大人,畏聖人之言。」其祭社稷、宗廟、山川、鬼神,不以其道,無災無害。至於祭天不享,其卜不從,使其牛口傷,鼷鼠食其角。或言食牛,或言食而死,或食而生,或不食而自死,或改卜而牛死,或卜而食其角。過有深淺薄厚,而災有簡甚,不可不察也。猶郊之變,因其災而之變,應而無為也。見百事之變之所不知而自然者,勝言與?以此見其可畏。專誅絕者其唯天乎?臣殺君,子殺父,三十有余,諸其賤者則損。以此觀之,可畏者其唯天命、大人乎?亡國五十有余,皆不事畏者也。況不畏大人,大人專誅之。君之滅者,何日之有哉?魯宣達聖人之言,變古易常,而災立至。聖人之言可不慎?此三畏者,異指而同致,故聖人同之,俱言其可畏也。

\article{郊事對}

廷尉臣湯昧死言:臣湯承制,以郊事問故膠西相仲舒。臣仲舒對曰:「所聞古者天子之禮,莫重於郊。郊常以正月上辛者,所以先百神而最居前。禮,三年喪,不祭其先,而不敢廢郊。郊重於宗廟,天尊於人也。《王制》曰:『祭天地之牛繭栗,宗廟之牛握,賓客之牛尺。』此言德滋美而牲滋微也。《春秋》曰:『魯祭周公,用白牡。』色白貴純也。帝牲在滌三月,牲貴肥潔,而不貪其大也。凡養牲之道,務在肥潔而已。駒犢未能勝爭芻豢之食,莫如令食其母便。」臣湯謹問仲舒:「魯祀周公用白牲,非禮也?」「周公子用,群公不毛。周公,諸公也,何以得用純牲?」臣仲舒對曰:「武王崩,成王立而在襁褓之中,周公繼文武之業,成二聖之功,德漸天地,澤被四海,故成王賢而貴之。《詩》云:『無德不報。』故成王使祭周公以白牡,上不得與天子同色,下有異於諸侯。臣仲舒愚以為報德之禮。」臣湯問仲舒:「天子祭天,諸侯祭土,魯何緣以祭郊?」臣仲舒對曰:「周公傅成王,成王遂及聖,功莫大於此。周公,聖人也,有祭於天道。故成王令魯郊也。」臣湯問仲舒:「魯祭周公用白牡,其郊何用?」臣仲舒對曰:魯郊用純。周色上赤,魯以天子命郊,故以。」臣湯問仲舒:「祠宗廟或以鶩當鳧,鶩非鳧,可用否?」仲舒對曰:「鶩非鳧,鳧非鶩也。臣聞孔子入太廟,每事問,慎之至也。陛下祭躬親,齊戒沐浴,以承宗廟,甚敬謹,奈何以鳧當鶩,鶩當鳧?名實不相應,以承太廟,不亦不稱乎?臣仲舒愚以為不可。臣犬馬齒衰,賜骸骨,伏陋巷。陛下乃幸使九卿問臣以朝廷之事,臣愚陋。曾不足以承明詔,奉大對。臣仲舒昧死以聞。」

\article{執贄}
凡執贄,天子用暢,公侯用玉,卿用羔,大夫用雁。雁乃有類於長者,長者在民上,必施然有先後之隨,必然有行列之治,故大夫以為贄。羔有角而不任,設備而不用,類好仁者;執之不鳴,殺之不諦,類死義者;羔食於其母,必跪而受之,類知禮者;故羊之為言猶祥與!故卿以為贄。玉有似君子。子曰:「人而不曰如之何、如之何者,吾末如之何也矣。」故匿病者不得良醫,羞問者聖人去之,以為遠功而近有災,是則不有。玉至清而不蔽其惡,內有瑕積,必見之於外,故君子不隱其短。不知則問,不能則學,取之玉也。君子比之玉,玉潤而不污,是仁而至清潔也;廉而不殺,是義而不害也;堅而不堅,過而不濡。視之如庸,展之如石,狀如石,搔而不可從繞,潔白如素,而不受污,玉類備者,故公侯以為贄。暢有似於聖人者,純仁淳粹,而有知之貴也,擇於身者盡為德音,發於事者盡為潤澤。積美陽芬香,以通之天。合之為一,而達其臭,氣暢於天。其淳粹無擇,與聖人一也,故天子以為贄。而各以事上也。觀贄之意,可以見其事。

\article{山川頌}

山則,嵬巍,久不崩,似夫仁人誌士。孔子曰:「山川神只立,寶藏殖,器用資,曲直合,大者可以為宮室台榭,小者可以為舟輿浮灄。大者無不中,小者無不入,持斧則斫,折鐮則艾。生人立,禽獸伏,死人入,多其功而不言,是以君子取譬也。」且積土成山,無損也,成其高,無害也,成其大,無虧也。小其上,久長安,後世無有去就,儼然獨處,惟山之意。《詩》云:「節彼南山,惟石岩岩。赫赫師尹,民具爾瞻。」此之謂也。

水則源泉混混,晝夜不竭,既似力者;盈科後行,既似持平者,循微赴下,不遺小間,既似察者,循谷不迷,或奏萬裡而必至,既似知者;障防山而能清淨,既似知命者;不清而入,潔清而出,既似善化者;赴千仞之壑,入而不旋,既似勇者;物皆困於火,而水獨勝之,既似武者;孔子在川上曰:「逝者如斯夫,不舍晝夜。」此之謂也。

\article{求雨}

春旱求雨。今懸邑以水日禱社稷山川,家人祀戶。無伐名木,無斬山林。八日。於邑東門之外為四通之壇,方八尺,植蒼繒八。其神共工,祭之以生魚八,玄酒,具清酒、膊脯。擇巫之潔清辯利者以為祝。祝齊三日,服蒼衣,先再拜,乃跪陳,陳已,複再拜,乃起。祝曰:「昊天生五穀以養人,今五穀病旱,恐不成實,敬進清酒、膊脯,再拜請雨,寸幸大澍。」以甲乙日為大蒼龍一,長八丈,居中央。為小龍七,各長四丈。於東方。皆東鄉,其間相去八尺。小童八人,皆齊三日,服青衣而舞之。田嗇夫亦齊三日,服青衣而立之。鑿社通之於閭外之溝,取五暇蟆,錯置社之中。池方八尺,深一尺,置水暇蟆焉。具清酒、膊脯,祝齊三日,服蒼衣,拜跪,陳祝如初。取三歲雄雞與三歲蝦豬,令民邑裡南門,開邑裡北門,具老蝦豬一,置之於裡北門之外。市中亦置蝦豬一,聞鼓聲,皆燒蝦豬尾。取死人骨埋之,開山淵,積薪而燔之。通道橋之壅塞不行者,決瀆之。幸而得雨,報以豚一,酒、鹽、黍財足,以茅為席,毋斷。夏求雨。令懸邑以水日,家人祀灶。無舉土功,更火浚井。暴釜於壇,臼杵術,為四通之壇於邑南門之外,方七尺,植赤繒七。其神尤,祭之以赤雄雞七,玄酒,具清酒、膊脯。祝齊三日,服赤衣,拜跪陳祝如春辭。以丙丁日為大赤龍一,長七丈,居中央。又為小龍六,各長三丈五尺,於南方。皆南鄉,其間相去七尺。壯者七人,皆齊三日,服赤衣而舞之。司空嗇夫亦齊三日,服赤衣而立之。鑿社而通之閭外之溝。取五蝦,錯置裡社之中,池方七尺,深一尺。祝齊,衣赤衣,拜跪陳祝如初。取三歲雄雞、蝦豬,燔之四通神宇。開陰閉陽如春也。


季夏禱山陵以助之。令縣邑十日壹徙市,於邑南門之外。五日禁男子無得行入市。家人祠中溜。無舉土功。聚巫市傍,為之結蓋。為四通之壇於中央,植黃繒五。其神後稷,祭之以母五,玄酒,具清酒、膊脯。令各為祝齊三日,衣黃衣。皆如春祠。以戊己日為大黃龍一,長五丈,居中央。又為小龍四,各長二丈五尺。丈夫五人,皆齊三日,服黃衣而舞之。老者五人,亦齊三日,亦通社中於閭外之溝,蝦池方五尺,深一尺。他皆如前。秋暴巫至九日,無舉火事,家人祠門。為四通之壇於邑西門之外,方九尺,植白繒九。祭之以桐木魚九,玄酒,具清酒、膊脯。衣白衣。以庚辛日為大白龍一,長九丈,居中央。為小龍八,各長四丈五尺,於西方。皆西鄉,其間相去九尺。鰥者九人,皆齊三日,服白衣而舞之。司馬亦齊三日,衣白衣而立之蝦池方九尺,深一尺。他皆如前。


冬舞龍六日,禱於名山以助之。家人祠井。無壅水。為四通之壇於邑北門之外,方六尺,植黑繒六。其神玄冥,祭之以黑狗子六,玄酒,具清酒、膊脯。祝齊三日,衣黑衣,祝禮如春。以壬癸日為大黑龍一,長六丈,居中同在。又為小龍五,各長三丈,於北方。皆北鄉,其間相去六尺。老者六人,皆齊三日,衣黑衣而舞之。尉亦齊三日,服黑衣而立之。蝦池皆如春。


四時皆以水日,為龍,必取潔土為之,結蓋,龍成而發之。四時皆以庚子之日,令吏民夫婦皆偶處。凡求雨之大體,丈夫欲藏匿,女子欲和而樂。



\article{止雨}

雨太多,令縣邑以土日,塞水瀆,絕道,蓋井,禁婦人不得行入市。令縣鄉裡皆掃社下。縣邑若丞合史、嗇夫三人以上,祝一人;鄉嗇夫若吏三人以上,祝一人;裡正父老三人以上,祝一人,皆齊三日,各衣時衣。具豚一,黍鹽美酒財足,祭社。擊鼓三日,而祝。先再拜,乃跪陳,陳已,複再拜,乃起。祝曰:「嗟!天生五穀以養人,今淫雨太多,五穀不和。敬進肥牲清酒,以請社靈,幸為止雨,除民所苦,無使陰滅陽。陰滅陽,不順於天。天之常意,在於利人,人願止雨,敢告於社。」鼓而無歌,至罷乃止。凡止雨之大體,女子欲其藏而匿也,丈夫欲其和而樂也。開陽而閉陰,闔水而開火。以朱絲縈社十周。衣赤衣赤。三日罷。

二十一年八月甲申,朔。丙午,江都相仲舒告內史中尉:陰雨太久,恐傷五穀,趣止雨。止雨之禮,廢陰起陽。書十七縣,八十離鄉,乃都官吏千石以下,夫婦在官者,咸遣婦歸。女子不得至市,市無詣井,蓋之,勿令泄。鼓用牲於社。祝之曰:「雨以太多,五穀不和,敬進肥牲,以請社靈,社靈幸為止雨,除民所苦,無使陰滅陽。陰滅陽,不順於天。天意常在於利民,願止雨。敢告。」鼓用牲於社,皆壹以辛亥之日,書到即起,縣社令長,若丞尉官長,各城邑社嗇夫,裡吏正裡人皆出,至於社下,鋪而罷。三日而止。未至三日,天亦止。

\article{祭義}

五穀,食物之牲也,天之所以為人賜也。宗廟上四時之所成,受賜而薦之宗廟,敬之性也,於祭之而宜矣。宗廟之祭,物之厚無上也。春上豆實,夏上尊實,秋上實,豆實,韭也,春之所始生也。尊實,也,夏之所受初也。實,黍也,秋之所先成也。敦實,稻也,冬之所畢熟也。始生故曰祠,善其司也;夏約故曰祗,貴所受初也;先成故曰嘗,嘗言甘也;畢熟故曰蒸,蒸言眾也。奉四時所受於天者而上之,為上祭,貴天賜,且尊宗廟也。孔子受君賜則以祭,況受天賜乎。一年之中,天賜四至,至則上之,此宗廟所以歲四祭也。故君子未嘗不食新,新天賜至,必先薦之,乃敢食之,尊天、敬宗廟之心也。尊天,美義也;敬宗廟,大禮也。聖人之所謹也。不多而欲潔清,不貪數而欲恭敬。君子之祭也,躬親之,致其中心之誠,盡敬潔之道,以接至尊,故鬼享之。享之如此,乃可謂之能祭。者,察也,以善逮鬼神之謂也。善乃逮不可聞見者,故謂之察。吾以名之所享,故祭之不虛,安所可察哉!祭之為言際也與?祭然後能見不見。見不見之見者,然後知天命鬼神。知天命鬼神,然後明祭之意。明祭之意,乃知重祭事。孔子曰:「吾不與祭,如不祭。祭神如神在。」重祭事,如事生。故聖人於鬼神也,畏之而不敢欺也,信之而不獨任,事之而不專恃。恃其公,報有德也;幸其不私,與人福也。其見於《詩》曰:「嗟爾君子,毋恆安息。靜共爾位,好是正直。神之聽之,介爾景福。」正直者得福也,不正者不得福,此其法也。以《詩》為天下法矣,何謂不法哉?其辭直而重,有再歡之,欲人省其意也。而人尚不省,何其忘哉!孔子曰:「書之重,辭之複。嗚呼!不可不察也。其中必有美者焉。」

\article{循天之道}

循天之道,以養其身,謂之道也。天有兩和以成二中,歲立其中,用之無窮。是北方之中用合陰,而物始動於下;南方之中用合陽,而養始美於上。其動於下者,不得東方之和不能生,中春是也。其養於上者,不得西方之和不能成,中秋是也。然則天地之美惡,在兩和之處,二中之所來歸而遂其為也。是故東方生而西方成,東方和生北方之所起,西方和成南方之所養長。起之不至於和之所不能生,養長之不至於和之所不能成。成於和,生必和也;始於中,止必中也。中者,天地之所終始也;而和者,天地之所生成也。夫德莫大於和,而道莫正於中。中者,天地之美達理也,聖人之所保守也。《詩》云:「不剛不柔,布政優優。」此非中和之謂與?是故能以中和理天下者,其德大盛;能以中和養其身者,其壽極命。男女之法,法陰與陽。陽氣起於北方,至南方而盛,盛極而合乎陰。陰氣起乎中夏,至中冬而盛,盛極而合乎陽。不盛不合,是故十月而壹俱盛,終歲而乃再合。天地久節,以此為常,是故先法之內矣,養身以全,使男子不堅牡不家室,陰不極盛不相接。是故身精明,難衰而堅固,壽考無忒,此天地之道也。天氣先盛牡而後施精,故其精固;地氣盛牝而後化,故其化良。是故陰陽之會,冬合北方而物動於下,夏合南方而物動於上。上下之大動,皆在日至之後。為寒則凝冰襲地,為熱則焦沙爛石。氣之精至於是,故天地之化,春氣生而百物皆出,夏氣養而百物皆長,秋氣殺而百物皆死,冬氣收而百物皆藏。是故惟天地之氣而精,出入無形,而物莫不應,實之至也。君子法乎其所貴。天地之陰陽當男女,人之男女當陰陽。陰陽亦可以謂男女,男女亦可以謂陰陽。天地之經,至東方之中而所生大養,至西方之中而所養大成,一歲四起業,而必於中。中之所為,而必就於和,故曰和其要也。和者,天之正也,陰陽之平也,其氣最良,物之所生也。誠擇其和者,以為大得天地之奉也。天地之道,雖有不和者,必歸之於和,而所為有功;雖有不中者,必止之於中,而所為不失。是故陽之行,始於北方之中,而止於南方之中;陰之行,始於南方之中,而止於北方之中。陰陽之道不同,至於盛而皆止於中,其所始起皆必於中。中者,天地之太極也,日月之所至而卻也,長短之隆,不得過中,天地之制也。兼和與不和,中與不中,而時用之,盡以為功。是故時無不時者,天地之道也。順天之道,節者天之制也,陽者天之寬也,陰者天之急也,中者天之用也,和者天之功也。舉天地之道,而美於和,是故物生,皆貴氣而迎養之。孟子曰:「我善養吾浩然之氣者也。」謂行必終禮,而心自喜,常以陽得生其意也。公孫之養氣曰:「裹藏泰實則氣不通,泰虛則氣不足,熱勝則氣,寒勝則氣,泰勞則氣不入,泰佚則氣宛至,怒則氣高,喜則氣散,憂則氣狂,懼則氣懾。凡此十者,缺之害也,而皆生於不中和。故君子怒則反中而自說以和,喜則反中而收之以正,憂則反中而舒之以意,懼則反中而實之以精。」夫中和之不可不反如此。故君子道至,氣則華而上。凡氣從心。心,氣之君也,何為而氣不隨也。是以天下之道者,皆言內心其本也。故仁人之所以多壽者,外無貪而內清淨,心和平而不失中正,取天地之美以養其身,是其且多且治。鶴之所以壽者,無宛氣於中,猿之所以壽者,好引其末,是故氣四越。天氣常下施於地,是故道者亦引氣於足;天之氣常動而不滯,是故道者亦不宛氣。苟不治,雖滿不虛。是故君子養而和之,節而法之,去其群泰,取其眾和。高台多陽,廣室多陰,遠天地之和也,故聖人弗為,適中而已矣。法人八尺,四尺其中也。宮者,中央之音也;甘者,中央之味也;四尺者,中央之制也。是故三王之禮,味皆尚甘,聲皆尚和。處其身所以常自漸於天地之道,其道同類,一氣之辨也。法天者乃法人之辨。天之道,向春夏而陰去。是故佔之人霜降而迎女,冰泮而殺內,與陰俱近,與陽俱遠也。天地之氣,不致盛滿,不交陰陽。是故君子甚愛氣而游於房,以體天也。氣不傷於以盛通,而傷於不時、天並。不與陰陽俱往來,謂之不時;恣其欲而不顧天數,謂之天並。君子治身,不敢違天。是故新牡十日而一游於房,中年者倍新牡,始衰者倍中年,中衰者倍始衰,大衰者以月當新牡之日,而上與天地同節矣。此其大略也,然而其要皆期於不極盛不相遇。疏春而曠夏,謂不遠天地之數。民皆知愛其衣食,而不愛其天氣。天氣之於人,重於衣食。衣食盡,尚猶有閒,氣盡而立終。故養生之大者,乃在愛氣。氣從神而成,神從意而出。心之所之謂意,意勞者神擾,神擾者氣少,氣少者難久矣。故君子閒欲止惡以平意,平意以靜神,靜神以養氣。氣多而治,則養身之大者得矣。古之道士有言曰:將欲無陵,固守一德。此言神無離形,則氣多內充,而忍饑寒也。和樂者,生之外泰也;精神者,生之內充也。外泰不若內充,而況外傷乎?忿恤憂恨者,生之傷也;和說勸善者,生之養也。君子慎小物而無大敗也。行中正,聲向榮,氣意和平,居處虞樂,可謂養生矣。凡養生者,莫精於氣。是故春襲葛,夏居密陰,秋避殺風,冬避秤潔,就其和也。衣欲常漂,食欲常饑。體欲常勞,而無長佚,居多也。凡衛地之物,乘於其泰而生,厭於其勝而死,四時之變是也。故冬之水氣,東加於春而木生,乘其泰也。春之生,西至金而死,厭於勝也。生於木者,至金而死;生於金者,至火而死。春之所生而不得過秋,秋之所生不得過夏,天之數也。飲食臭味,每至一時,亦有所勝,有所不勝,之理不可不察也。四時不同氣,氣各有所宜,宜之所在,其物代美。視代美而代養之,同時美者雜食之,是皆其所宜也。故以冬美,而荼以夏成,此可以見冬夏之所宜服矣。冬,水氣也,甘味也,乘於水氣而美者,甘勝寒也。之為言濟與?濟,大水也。夏,火氣也,荼,苦味也,乘於火氣而成者,苦勝暑也。天無所言,而意以物。物不與群物同時而生死者,必深察之,是天之所以告人也。故成告之甘,荼成告之苦也。君子察物而成告謹,是以至不可食之時,而盡遠甘物,至荼成就也。天所獨代之成者,君子獨代之,是冬夏之所宜也。春秋雜物其和,而冬夏代服其宜,則當得天地之美,四時和矣。凡擇味之大體,各因其時之所美,而違天不遠矣。是故當百物大生之時,群物皆生,而此物獨死。可食者,告其味之便於人也;其不食者,告殺穢除害之不待秋也。當物之大枯之時,群物皆死,如此物獨生。其可食者,益食之,天為之利人,獨代生之;其不可食,益畜之。天愍州華之間,故生宿麥,中歲而熟之。君子察物之異,以求天意,大可見矣。是故男女體其盛,臭味取其勝,居處就其和,勞佚居其中,寒暖無失適,饑飽無過平,欲惡度理,動靜順性,喜怒止於中,憂懼反之正,此中和常在乎其身,謂之得天地泰。得天地泰者,其壽引而長;不得天地泰者,其壽傷而短。短長之質,人之所由受於天也。是故壽有短長,養有得失,及至其末之,大卒而必讎,於此莫之得離,故壽之為言,猶讎也。天下之人雖眾,不得不各讎其所生,而壽夭於其所自行。自行可久之道者,其壽讎於久;自行不可久之道者,其壽亦讎於不久。久與不久之情,各讎其生平之所行,今如後至,不可得勝,故曰:壽者讎也。然則人之所自行,乃與其壽夭相益損也。其自行佚而壽長者,命益之也;其自行端而壽短者,命損之也。以天命之所損益,疑人之所得失,此大惑也。是故天長之而人傷之者,其長損;天短之而人養之者,其短益。夫損益者皆人,人其天之繼?出其質而人弗繼,豈獨立哉!

\article{天地之行}

天地之行美也。是以天高其位而下其施,藏其形而見其光,序列星而近至精,考陰陽而降霜露。高其位所以為尊也,下其施所以為仁也,藏其形所以為神也,見其光所以為明也,序列星所以相承也,近至精所以為剛也,考陰陽所以成歲也,降霜露所以生殺也。為人君者,其法取象於天。故貴爵而臣國,所以為仁也;深居隱處,不見其體,所以為神也;任賢使能,觀聽四方,所以為明也;量能授官,賢愚有差,所以相承也;引賢自近,以備股肱,所以為剛也;考實事功,次序殿最,所以成世也;有功者進,無功者退,所以賞罰也。是故天執其道為萬物主,君執其常為一國主。天不可以不剛,主不可以不堅。天不剛則列星亂其行,主不堅則邪臣亂其官。星亂則亡其天,臣亂則亡其君。故為天者務剛其氣,為君者務堅其政,剛堅然後陽道制命。地卑其位而上其氣,暴其形而著其情,受其死而獻其生,成其事而歸其功。卑其位所以事天也,上其氣所以養陽也,暴其形所以為忠也,著其情所以為信也,受其死所以藏終也,獻其生所以助明也,成其事所以助化也,歸其功所以致義也。為人臣者,其法取象於地。故朝夕進退。奉職應對,所以事貴也;供設飲食,候視疾,所以致養也;委身致命,事無專制,所以為忠也;竭愚寫情,不飾春過,所以為信也;伏節死難,不惜其命,所以救窮也;推進光榮,褒揚其善,所以助明也;受命宣恩,輔成君子,所以助化也;功成事就,歸德於上,所以致義也。是故地明其理為萬物母,臣明其職為一國宰。母不可以不信,宰不可以不忠。母不信則草木傷其根,宰不忠則奸臣危其君。根傷則亡其枝葉,君危則亡其國。故為地者務暴其形,為臣者務著其情。

一國之君,其猶一體之心也。隱居深宮,若心之藏於胸;至貴無與敵,若心之神無與雙也。高清明而下重濁,若身之貴目而賤足也;任群臣無所親,若四肢之各有職也;內有四輔,若心之有肝肺脾腎也;外有百官,若心之有形體孔竅也;親聖近賢,若神明皆聚於心也;上下相承順,若肢體相為使也;布恩施惠,若元氣之流皮毛腠理也;百姓皆得其所,若血氣和平,形體無所苦也;無為致太平,若神氣自通於淵也;致黃龍鳳皇,若神明之致玉女芝英也。君明,臣蒙其功,若心之神,體得以全;臣賢,君蒙其恩,若形體之靜而心得以安。上亂下被其患,若耳目不聰明而手足為傷也;臣不忠而君滅亡,若形體妄動而心為之喪。是故君臣之禮,若心之與體,心不可以不堅,君不可以不賢;體不可以不順,臣不可以不忠。心所以全者,體之力也;君所以安者,臣之功也。

\article{威德所生}

	
天有和有德,有平有威,有相受之意,有為政之理,不可不審也。春者,天之和也;夏者,天之德也;秋者,天之平也;冬者,天之威也。天之序,必先和然後發德,必先平然後發威。此可以見不和不可以發慶賞之德,不平不可以發刑罰之威。又可以見德生於和,威生於平也。不和無德,不平無威,天之道也,達者以此見之矣。我雖有所愉而喜,必先和心以求其當,然後發慶賞以立其德。雖有所忿而怒,必先平心以求其政,然後發刑罰以立其威。能常若是者謂之天德,行天德者謂之聖人。為人主者,居至德之位,操殺生之勢,以變化民。民之從主也,如草木之應四時也。喜怒當寒暑,威德當冬夏。冬夏者,威德之合也;寒暑者,喜怒之偶也。喜怒之有時而當發,寒暑亦有時而當出,其理一也。當喜而不喜,猶當暑而不暑;當怒而不怒,猶當寒而不寒也;當德而不德,猶當夏而不夏也;當威而不威,猶當冬而不冬也。喜怒威德之不可以不直處而發也,如寒暑冬夏之不可不當其時而出也。故謹善惡之端。何以效其然也?《春秋》采善不遺小,掇惡不遺大,諱而不隱,罪而不忽,以是非,正理以褒貶。喜怒之發,威德之處,無不皆中其應,可以參寒暑冬夏之不失其時已。故曰聖人配天。

\article{如天之為}

陰陽之氣,在上天,亦在人。在人者為好惡喜怒,在天者為暖清寒暑。出入上下、左右、前後,平行而不止,未嘗有所稽留滯鬱也。其在人者,亦宜行而無留,若四時之條條然也。夫喜怒哀樂之止動也,此天之所為人性命者。臨其時而欲發其應,亦天應也,與暖清寒暑之至其時而欲發無異。若留德而待春夏,留刑而待秋冬也,此有順四時之名,實逆於天地之經。在人者亦天也,奈何其久留天氣,使之鬱滯,不得以其正周行也。是故天行谷朽寅,而秋生麥,告除穢而繼乏也。所以成功繼乏,以贍人也。天之生有大經也,而所周行者,又有害功也,除而殺殛者,行急皆不待時也,天之志也,而聖人承之以治。是故春修仁而求善,秋修義而求惡,冬修刑而致清,夏修德而致寬。此所以順天地,體陰陽。然而方求善之時,見惡而不釋;方求惡之時,見善亦立行;方致清之時,見大善亦立舉之;方致寬之時,見大惡亦立去之。以效天地之方生之時有殺也,方殺之時有生也。是故誌意隨天地,緩急仿陰陽。然而人事之宜行者,無所鬱滯,且恕於人,順於天,天人之道兼舉,此謂執其中。天非以春生人,以秋殺人也。當生者曰生,當死者曰死,非殺物之義待四時也。而人之所治也,安取久留當行之理,而必待四時也。此之謂壅,非其中也。人有喜怒哀樂,猶天之有春夏秋冬也。喜怒哀樂之至其時而欲發也,若春夏秋冬之至其時而欲出也,皆天氣之然也。其宜直行而無鬱滯,一也。天終歲乃一遍此四者,而人主終日不知過此四之數,其理故不可以相待。且天之欲利人,非直其欲利谷也。除穢不待時,況穢人乎!

\article{天地陰陽}

天、地、陰、陽、木、火、土、金、水,九,與人而十者,天之數畢也。故數者至十而止,書者以十為終,皆取之此。聖人何其貴者?起於天,至於人而畢。畢之外謂之物,物者投所貴之端,而不在其中。以此見人之超然萬物之上,而最為天下貴也。人,下長萬物,上參天地。故其治亂之故,動靜順逆之氣,乃損益陰陽之化,而搖蕩四海之內。物之難知者若神,不可謂不然也。今投地死傷而不勝相助,投淖相動而近,投水相動而愈遠。由此觀之,夫物愈淖而愈易變動搖蕩也。今氣化之淖,非直水也。而人主以眾動之無已時,是故常以治亂之氣,與天地之化相而不治也。世治而民和,誌平而氣正,則天地之化精,而萬物之美起。世亂而民乖,誌僻而氣逆,則天地之化傷,氣生災害起。是故治世之德,潤草木,澤流四海,功過神明。亂世之所起亦博。若是,皆因天地之化,以成敗物,乘陰陽之資,以任其所為,故為惡愆人力而功傷,名自過也。天地之間,有陰陽之氣,常漸人者,若水常漸魚也。所以異於水者,可見與不可見耳,其澹澹也。然則人之居天地之間,其猶魚之離水,一也。其無間若氣而淖於水。水之比於氣也,若泥之比於水也。是天地之間,若虛而實,人常漸是澹澹之中,而以治亂之氣,與之流通相也。故人氣調和,而天地之化美,於惡而味敗,此易之物也。推物之類,以易見難者,其情可得。治亂之氣。邪正之風,是天地之化者也。生於化而反化,與運連也。《春秋》舉世事之道,夫有書天,之盡與不盡,王者之任也。《詩》云:「天難諶斯,不易維王。」此之謂也。夫王者不可以不知天。知天,詩人之所難也。天意難見也,其道難理。是故明陽陰、入出、實虛之處,所以觀天之志。辨五行之本末順逆、小大廣狹,所以觀天道也。天誌仁,其道也義。為人主者,予奪生殺,各當其義,若四時;列官置吏,必以其能,若五行;好仁惡戾,任德遠刑,若陰陽。此之謂能配天。天者其道長萬物,而王者長人。人主之大,天地之參也;好惡之分,陰陽之理也;喜怒之發,寒暑之比也;官職之事,五行之義也。以此長天地之間,蕩四海之內,陰陽之氣,與天地相雜。是故人言:既曰王者參天地矣,苟參天地,則是化矣,豈獨天地之精哉。王者亦參而之,治則以正氣天地之化,亂則以邪氣天地之化,同者相益,異者相損之數也,無可疑者矣。

\article{天道施}

天道施,地道化,人道義。聖人見端而知本,精之至也;得一而應萬,類之治也。動其本者不知靜其末,受其始者不能辭其終。利者盜之本也,妄者亂之始也。夫受亂之始,動盜之本,而欲民之靜,不可得也。故君子非禮而不言,非禮而不動。好色而無禮則流,飲食而無禮則爭,流爭則亂。夫禮,體情而防亂者也。民之情,不能制其欲,使之度禮。目視正色,耳聽正聲,口食正味,身行正道,非奪之情也,所以安其情也。變謂之情,雖持異物性亦然者,故曰內也。變變之變,謂之外。故雖以情,然不為性說。故曰:外物之動性,若神之不守也。積習漸靡,物之微者也。其入人不知,習忘乃為,常然若性,不可不察也。純知輕思則慮達,節欲順行則倫得,以諫爭靜為宅,以禮義為道則文德。是故至誠遺物而不與變,躬寬無爭而不以與欲推,眾強弗能入。蜩蛻濁穢之中,含得命施之理,與萬物遷徙而不自失者,聖人之心也。

名者,所以別物也。親者重,疏者輕,尊者文,卑者質,近者詳,遠者略,文辭不隱情,明情不遺文,人心從之而不逆,古今通貫而不亂,名之義也。男女猶道也。人生別言禮義,名號之由人事起也。不順天道,謂之不義,察天人之分,觀道命之異,可以知禮之說矣。見善者不能無好,見不善者不能無惡,好惡去就,不能堅守,人道者,人之所由樂而不亂,複而不厭者,萬物載名而生,聖人因其象而命之。然而可易也,皆有義從也,故正名以名義也。物也者,洪名也,皆名也,而物有私名,此物也,非夫物。故曰:萬物動而不形者,意也;形而不易者,德也;樂而不亂,複而不厭者,道也。





