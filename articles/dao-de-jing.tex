\article{道德經}

\begin{pinyinscope}
道可道,非常道。名可名,非常名。無名天地之始;有名萬物之母。故常無欲,以觀其妙;常有欲,以觀其徼。此兩者,同出而異名,同謂之玄。玄之又玄,衆妙之門。

天下皆知美之為美,斯惡已。皆知善之為善,斯不善已。故有無相生,難易相成,長短相較,高下相傾,音聲相和,前後相隨。是以聖人處無為之事,行不言之教;萬物作焉而不辭,生而不有。為而不恃,功成而弗居。夫唯弗居,是以不去。

不尚賢,使民不爭;不貴難得之貨,使民不為盜;不見可欲,使心不亂。是以聖人之治,虛其心,實其腹,弱其志,強其骨。常使民無知無欲。使夫知者不敢為也。為無為,則無不治。

道沖而用之或不盈。淵兮似萬物之宗。挫其銳,解其紛,和其光,同其塵。湛兮似或存。吾不知誰之子,象帝之先。

天地不仁,以萬物為芻狗;聖人不仁,以百姓為芻狗。天地之間,其猶橐籥乎?虛而不屈,動而愈出。多言數窮,不如守中。

谷神不死,是謂玄牝。玄牝之門,是謂天地根。綿綿若存,用之不勤。

天長地久。天地所以能長且久者,以其不自生,故能長生。是以聖人後其身而身先;外其身而身存。非以其無私耶?故能成其私。

上善若水。水善利萬物而不爭,處衆人之所惡,故幾於道。居善地,心善淵,與善仁,言善信,正善治,事善能,動善時。夫唯不爭,故無尤。

持而盈之,不如其已;揣而銳之,不可長保。金玉滿堂,莫之能守;富貴而驕,自遺其咎。功遂身退天之道。

載營魄抱一,能無離乎?專氣致柔,能嬰兒乎?滌除玄覽,能無疵乎?愛民治國,能無知乎?天門開闔,能為雌乎?明白四達,能無知乎?生之、畜之,生而不有,為而不恃,長而不宰,是謂玄德。

三十輻,共一轂,當其無,有車之用。埏埴以為器,當其無,有器之用。鑿戶牖以為室,當其無,有室之用。故有之以為利,無之以為用。

五色令人目盲;五音令人耳聾;五味令人口爽;馳騁田獵,令人心發狂;難得之貨,令人行妨。是以聖人為腹不為目,故去彼取此。

寵辱若驚,貴大患若身。何謂寵辱若驚?寵為下,得之若驚,失之若驚,是謂寵辱若驚。何謂貴大患若身?吾所以有大患者,為吾有身,及吾無身,吾有何患?故貴以身為天下,若可寄天下;愛以身為天下,若可託天下。

視之不見,名曰夷;聽之不聞,名曰希;搏之不得,名曰微。此三者不可致詰,故混而為一。其上不皦,其下不昧。繩繩不可名,復歸於無物。是謂無狀之狀,無物之象,是謂惚恍。迎之不見其首,隨之不見其後。執古之道,以御今之有。能知古始,是謂道紀。

古之善為士者,微妙玄通,深不可識。夫唯不可識,故強為之容。豫兮若冬涉川;猶兮若畏四鄰;儼兮其若容;渙兮若冰之將釋;敦兮其若樸;曠兮其若谷;混兮其若濁;孰能濁以靜之徐清?孰能安以久動之徐生?保此道者,不欲盈。夫唯不盈,故能蔽不新成。

致虛極,守靜篤。萬物並作,吾以觀復。夫物芸芸,各復歸其根。歸根曰靜,是謂復命。復命曰常,知常曰明。不知常,妄作凶。知常容,容乃公,公乃王,王乃天,天乃道,道乃久,沒身不殆。

太上,下知有之;其次,親而譽之;其次,畏之;其次,侮之。信不足,焉有不信焉。悠兮,其貴言。功成事遂,百姓皆謂我自然。

大道廢,有仁義;智慧出,有大偽;六親不和,有孝慈;國家昏亂,有忠臣。

絕聖棄智,民利百倍;絕仁棄義,民復孝慈;絕巧棄利,盜賊無有。此三者以為文不足。故令有所屬:見素抱樸,少私寡欲。

絕學無憂,唯之與阿,相去幾何?善之與惡,相去若何?人之所畏,不可不畏。荒兮其未央哉!衆人熙熙,如享太牢,如春登臺。我獨怕兮其未兆;如嬰兒之未孩;儽儽兮若無所歸。衆人皆有餘,而我獨若遺。我愚人之心也哉!沌沌兮,俗人昭昭,我獨若昏。俗人察察,我獨悶悶。澹兮其若海,飂兮若無止,衆人皆有以,而我獨頑似鄙。我獨異於人,而貴食母。

孔德之容,唯道是從。道之為物,唯恍唯惚。忽兮恍兮,其中有象;恍兮忽兮,其中有物。窈兮冥兮,其中有精;其精甚真,其中有信。自古及今,其名不去,以閱衆甫。吾何以知衆甫之狀哉?以此。

曲則全,枉則直,窪則盈,弊則新,少則得,多則惑。是以聖人抱一為天下式。不自見,故明;不自是,故彰;不自伐,故有功;不自矜,故長。夫唯不爭,故天下莫能與之爭。古之所謂曲則全者,豈虛言哉!誠全而歸之。

希言自然,故飄風不終朝,驟雨不終日。孰為此者?天地。天地尚不能久,而況於人乎?故從事於道者,道者,同於道;德者,同於德;失者,同於失。同於道者,道亦樂得之;同於德者,德亦樂得之;同於失者,失亦樂得之。信不足,焉有不信焉。

企者不立;跨者不行;自見者不明;自是者不彰;自伐者無功;自矜者不長。其在道也,曰:餘食贅行。物或惡之,故有道者不處。

有物混成,先天地生。寂兮寥兮,獨立不改,周行而不殆,可以為天下母。吾不知其名,字之曰道,強為之名曰大。大曰逝,逝曰遠,遠曰反。故道大,天大,地大,王亦大。域中有四大,而王居其一焉。人法地,地法天,天法道,道法自然。

重為輕根,靜為躁君。是以聖人終日行不離輜重。雖有榮觀,燕處超然。奈何萬乘之主,而以身輕天下?輕則失本,躁則失君。

善行無轍迹,善言無瑕讁;善數不用籌策;善閉無關楗而不可開,善結無繩約而不可解。是以聖人常善救人,故無棄人;常善救物,故無棄物。是謂襲明。故善人者,不善人之師;不善人者,善人之資。不貴其師,不愛其資,雖智大迷,是謂要妙。

知其雄,守其雌,為天下谿。為天下谿,常德不離,復歸於嬰兒。知其白,守其黑,為天下式。為天下式,常德不忒,復歸於無極。知其榮,守其辱,為天下谷。為天下谷,常德乃足,復歸於樸。樸散則為器,聖人用之,則為官長,故大制不割。

將欲取天下而為之,吾見其不得已。天下神器,不可為也,為者敗之,執者失之。故物或行或隨;或歔或吹;或強或羸;或挫或隳。是以聖人去甚,去奢,去泰。

以道佐人主者,不以兵強天下。其事好還。師之所處,荊棘生焉。大軍之後,必有凶年。善有果而已,不敢以取強。果而勿矜,果而勿伐,果而勿驕。果而不得已,果而勿強。物壯則老,是謂不道,不道早已。

夫佳兵者,不祥之器,物或惡之,故有道者不處。君子居則貴左,用兵則貴右。兵者不祥之器,非君子之器,不得已而用之,恬淡為上。勝而不美,而美之者,是樂殺人。夫樂殺人者,則不可以得志於天下矣。吉事尚左,凶事尚右。偏將軍居左,上將軍居右,言以喪禮處之。殺人之衆,以哀悲泣之,戰勝以喪禮處之。

道常無名。樸雖小,天下莫能臣也。侯王若能守之,萬物將自賓。天地相合,以降甘露,民莫之令而自均。始制有名,名亦既有,夫亦將知止,知止所以不殆。譬道之在天下,猶川谷之與江海。

知人者智,自知者明。勝人者有力,自勝者強。知足者富。強行者有志。不失其所者久。死而不亡者壽。

大道汎兮,其可左右。萬物恃之而生而不辭,功成不名有。衣養萬物而不為主,常無欲,可名於小;萬物歸焉,而不為主,可名為大。以其終不自為大,故能成其大。

執大象,天下往。往而不害,安平大。樂與餌,過客止。道之出口,淡乎其無味,視之不足見,聽之不足聞,用之不足既。

將欲歙之,必固張之;將欲弱之,必固強之;將欲廢之,必固興之;將欲奪之,必固與之。是謂微明。柔弱勝剛強。魚不可脫於淵,國之利器不可以示人。

道常無為而無不為。侯王若能守之,萬物將自化。化而欲作,吾將鎮之以無名之樸。無名之樸,夫亦將無欲。不欲以靜,天下將自定。

上德不德,是以有德;下德不失德,是以無德。上德無為而無以為;下德為之而有以為。上仁為之而無以為;上義為之而有以為。上禮為之而莫之應,則攘臂而扔之。故失道而後德,失德而後仁,失仁而後義,失義而後禮。夫禮者,忠信之薄,而亂之首。前識者,道之華,而愚之始。是以大丈夫處其厚,不居其薄;處其實,不居其華。故去彼取此。

昔之得一者:天得一以清;地得一以寧;神得一以靈;谷得一以盈;萬物得一以生;侯王得一以為天下貞。其致之,天無以清,將恐裂;地無以寧,將恐發;神無以靈,將恐歇;谷無以盈,將恐竭;萬物無以生,將恐滅;侯王無以貴高將恐蹶。故貴以賤為本,高以下為基。是以侯王自稱孤、寡、不穀。此非以賤為本耶?非乎?故致數譽無譽。不欲琭琭如玉,珞珞如石。

反者道之動;弱者道之用。天下萬物生於有,有生於無。

上士聞道,勤而行之;中士聞道,若存若亡;下士聞道,大笑之。不笑不足以為道。故建言有之:明道若昧;進道若退;夷道若纇;上德若谷;太白若辱;廣德若不足;建德若偷;質真若渝;大方無隅;大器晚成;大音希聲;大象無形;道隱無名。夫唯道,善貸且成。

道生一,一生二,二生三,三生萬物。萬物負陰而抱陽,沖氣以為和。人之所惡,唯孤、寡、不穀,而王公以為稱。故物或損之而益,或益之而損。人之所教,我亦教之。強梁者不得其死,吾將以為教父。

天下之至柔,馳騁天下之至堅。無有入無間,吾是以知無為之有益。不言之教,無為之益,天下希及之。

名與身孰親?身與貨孰多?得與亡孰病?是故甚愛必大費;多藏必厚亡。知足不辱,知止不殆,可以長久。

大成若缺,其用不弊。大盈若沖,其用不窮。大直若屈,大巧若拙,大辯若訥。躁勝寒靜勝熱。清靜為天下正。

天下有道,卻走馬以糞。天下無道,戎馬生於郊。禍莫大於不知足;咎莫大於欲得。故知足之足,常足矣。

不出戶知天下;不闚牖見天道。其出彌遠,其知彌少。是以聖人不行而知,不見而名,不為而成。

為學日益,為道日損。損之又損,以至於無為。無為而無不為。取天下常以無事,及其有事,不足以取天下。

聖人無常心,以百姓心為心。善者,吾善之;不善者,吾亦善之;德善。信者,吾信之;不信者,吾亦信之;德信。聖人在天下,歙歙為天下渾其心,百姓皆注其耳目,聖人皆孩之。

出生入死。生之徒,十有三;死之徒,十有三;人之生,動之死地,十有三。夫何故?以其生,生之厚。蓋聞善攝生者,陸行不遇兕虎,入軍不被甲兵;兕無所投其角,虎無所措其爪,兵無所容其刃。夫何故?以其無死地。

道生之,德畜之,物形之,勢成之。是以萬物莫不尊道而貴德。道之尊,德之貴,夫莫之命常自然。故道生之,德畜之;長之育之;亭之毒之;養之覆之。生而不有,為而不恃,長而不宰,是謂玄德。

天下有始,以為天下母。既得其母,以知其子,既知其子,復守其母,沒身不殆。塞其兌,閉其門,終身不勤。開其兌,濟其事,終身不救。見小曰明,守柔曰強。用其光,復歸其明,無遺身殃;是為習常。

使我介然有知,行於大道,唯施是畏。大道甚夷,而民好徑。朝甚除,田甚蕪,倉甚虛;服文綵,帶利劍,厭飲食,財貨有餘;是謂盜夸。非道也哉!

善建不拔,善抱者不脫,子孫以祭祀不輟。修之於身,其德乃真;修之於家,其德乃餘;修之於鄉,其德乃長;修之於國,其德乃豐;修之於天下,其德乃普。故以身觀身,以家觀家,以鄉觀鄉,以國觀國,以天下觀天下。吾何以知天下然哉?以此。

含德之厚,比於赤子。蜂蠆虺蛇不螫,猛獸不據,攫鳥不搏。骨弱筋柔而握固。未知牝牡之合而全作,精之至也。終日號而不嗄,和之至也。知和曰常,知常曰明,益生曰祥。心使氣曰強。物壯則老,謂之不道,不道早已。

知者不言,言者不知。塞其兑,閉其門,挫其銳,解其分,和其光,同其塵,是謂玄同。故不可得而親,不可得而踈;不可得而利,不可得而害;不可得而貴,不可得而賤。故為天下貴。

以正治國,以奇用兵,以無事取天下。吾何以知其然哉?以此:天下多忌諱,而民彌貧;民多利器,國家滋昏;人多伎巧,奇物滋起;法令滋彰,盜賊多有。故聖人云:我無為,而民自化;我好靜,而民自正;我無事,而民自富;我無欲,而民自樸。

其政悶悶,其民淳淳;其政察察,其民缺缺。禍兮福之所倚,福兮禍之所伏。孰知其極?其無正。正復為奇,善復為妖。人之迷,其日固久。是以聖人方而不割,廉而不劌,直而不肆,光而不燿。

治人事天莫若嗇。夫唯嗇,是謂早服;早服謂之重積德;重積德則無不克;無不克則莫知其極;莫知其極,可以有國;有國之母,可以長久;是謂深根固柢,長生久視之道。

治大國若烹小鮮。以道蒞天下,其鬼不神;非其鬼不神,其神不傷人;非其神不傷人,聖人亦不傷人。夫兩不相傷,故德交歸焉。

大國者下流,天下之交,天下之牝。牝常以靜勝牡,以靜為下。故大國以下小國,則取小國;小國以下大國,則取大國。故或下以取,或下而取。大國不過欲兼畜人,小國不過欲入事人。夫兩者各得其所欲,大者宜為下。

道者萬物之奧。善人之寶,不善人之所保。美言可以市,尊行可以加人。人之不善,何棄之有?故立天子,置三公,雖有拱璧以先駟馬,不如坐進此道。古之所以貴此道者何?不曰:以求得,有罪以免耶?故為天下貴。

為無為,事無事,味無味。大小多少,報怨以德。圖難於其易,為大於其細;天下難事,必作於易,天下大事,必作於細。是以聖人終不為大,故能成其大。夫輕諾必寡信,多易必多難。是以聖人猶難之,故終無難矣。

其安易持,其未兆易謀。其脆易泮,其微易散。為之於未有,治之於未亂。合抱之木,生於毫末;九層之臺,起於累土;千里之行,始於足下。為者敗之,執者失之。是以聖人無為故無敗;無執故無失。民之從事,常於幾成而敗之。慎終如始,則無敗事,是以聖人欲不欲,不貴難得之貨;學不學,復衆人之所過,以輔萬物之自然,而不敢為。

古之善為道者,非以明民,將以愚之。民之難治,以其智多。故以智治國,國之賊;不以智治國,國之福。知此兩者亦𥡴式。常知𥡴式,是謂玄德。玄德深矣,遠矣,與物反矣,然後乃至大順。

江海所以能為百谷王者,以其善下之,故能為百谷王。是以聖人欲上民,必以言下之;欲先民,必以身後之。是以聖人處上而民不重,處前而民不害。是以天下樂推而不厭。以其不爭,故天下莫能與之爭。

天下皆謂我道大,似不肖。夫唯大,故似不肖。若肖久矣。其細也夫!我有三寶,持而保之。一曰慈,二曰儉,三曰不敢為天下先。慈故能勇;儉故能廣;不敢為天下先,故能成器長。今舍慈且勇;舍儉且廣;舍後且先;死矣!夫慈以戰則勝,以守則固。天將救之,以慈衛之。

善為士者,不武;善戰者,不怒;善勝敵者,不與;善用人者,為之下。是謂不爭之德,是謂用人之力,是謂配天古之極。

用兵有言:吾不敢為主,而為客;不敢進寸,而退尺。是謂行無行;攘無臂;扔無敵;執無兵。禍莫大於輕敵,輕敵幾喪吾寶。故抗兵相加,哀者勝矣。

吾言甚易知,甚易行。天下莫能知,莫能行。言有宗,事有君。夫唯無知,是以不我知。知我者希,則我者貴。是以聖人被褐懷玉。

知不知上;不知知病。夫唯病病,是以不病。聖人不病,以其病病,是以不病。

民不畏威,則大威至。無狎其所居,無厭其所生。夫唯不厭,是以不厭。是以聖人自知不自見;自愛不自貴。故去彼取此。

勇於敢則殺,勇於不敢則活。此兩者,或利或害。天之所惡,孰知其故?是以聖人猶難之。天之道,不爭而善勝,不言而善應,不召而自來,繟然而善謀。天網恢恢,踈而不失。

民不畏死,奈何以死懼之?若使民常畏死,而為奇者,吾得執而殺之,孰敢?常有司殺者殺。夫司殺者,是大匠斲;夫代大匠斲者,希有不傷其手矣。

民之飢,以其上食稅之多,是以飢。民之難治,以其上之有為,是以難治。民之輕死,以其求生之厚,是以輕死。夫唯無以生為者,是賢於貴生。

人之生也柔弱,其死也堅強。萬物草木之生也柔脆,其死也枯槁。故堅強者死之徒,柔弱者生之徒。是以兵強則不勝,木強則共。強大處下,柔弱處上。

天之道,其猶張弓與?高者抑之,下者舉之;有餘者損之,不足者補之。天之道,損有餘而補不足。人之道,則不然,損不足以奉有餘。孰能有餘以奉天下,唯有道者。是以聖人為而不恃,功成而不處,其不欲見賢。

天下莫柔弱於水,而攻堅強者莫之能勝,其無以易之。弱之勝強,柔之勝剛,天下莫不知,莫能行。是以聖人云:受國之垢,是謂社稷主;受國不祥,是謂天下王。正言若反。

和大怨,必有餘怨;安可以為善?是以聖人執左契,而不責於人。有德司契,無德司徹。天道無親,常與善人。

小國寡民。使有什伯之器而不用;使民重死而不遠徙。雖有舟輿,無所乘之,雖有甲兵,無所陳之。使民復結繩而用之,甘其食,美其服,安其居,樂其俗。鄰國相望,雞犬之聲相聞,民至老死,不相往來。

信言不美,美言不信。善者不辯,辯者不善。知者不博,博者不知。聖人不積,既以為人己愈有,既以與人己愈多。天之道,利而不害;聖人之道,為而不爭。
\end{pinyinscope}
