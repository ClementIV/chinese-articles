\article{迎敵祠}

\begin{pinyinscope}
敵以東方來,迎之東壇,壇高八尺,堂密八。年八十者八人,主祭青旗。青神長八尺者八,弩八,八發而止。將服必青,其牲以雞。敵以南方來,迎之南壇,壇高七尺,堂密七,年七十者七人,主祭赤旗,赤神長七尺者七。弩七,七發而止。將服必赤,其牲以狗。敵以西方來,迎之西壇,壇高九尺,堂密九。年九十者九人,主祭白旗。素神長九尺者九,弩九,九發而止。將服必白,其牲以羊。敵以北方來,迎之北壇,壇高六尺,堂密六。年六十者六人主祭黑旗。黑神長六尺者六,弩六,六發而止。將服必黑,其牲以彘。從外宅諸名大祠,靈巫或禱焉,給禱牲。

凡望氣,有大將氣,有小將氣,有往氣,有來氣,有敗氣,能得明此者可知成敗、吉凶。舉巫、醫、卜有所,長具藥,宮之,善為舍。巫必近公社,必敬神之。巫卜以請守,守獨智巫卜望氣之請而已。其出入為流言,驚駭恐吏民,謹微察之,斷,罪不赦。望氣舍近守官。牧賢大夫及有方技者若工,弟之。舉屠、酤者置廚給事,弟之。

凡守城之法,縣師受事,出葆,循溝防,築薦通塗,脩城。百官共財,百工即事,司馬視城脩卒伍。設守門,二人掌右閹,二人掌左閹,四人掌閉,百甲坐之。城上步一甲、一戟,其贊三人。五步有五長,十步有什長,百步有百長,旁有大率,中有大將,皆有司吏卒長。城上當階,有司守之,移中中處澤急而奏之。士皆有職。城之外,矢之所遝,壞其牆,無以為客菌。三十里之內,薪、蒸、水皆入內。狗、彘、豚、雞食其肉,斂其骸以為醢腹,病者以起。城之內薪蒸廬室,矢之所遝皆為之涂菌。令命昏緯狗纂馬,掔緯。靜夜聞鼓聲而譟,所以閹客之氣也,所以固民之意也,故時譟則民不疾矣。

祝、史乃告於四望、山川、社稷,先於戎,乃退。公素服誓于太廟,曰:「其人為不道,不脩義詳,唯乃是王,曰:予必懷亡爾社稷,滅爾百姓。二三子夙夜自厲,以勤寡人,和心比力兼左右,各死而守。既誓,公乃退食。舍於中太廟之右,祝、史舍于社。百官具御,乃斗鼓于門,右置旂,左置旌于隅練名。射參發,告勝,五兵咸備,乃下,出挨,升望我郊。乃命鼓,俄升,役司馬射自門右,蓬矢射之,茅參發,弓弩繼之,校自門左,先以揮,木石繼之。祝、史、宗人告社,覆之以甑。


\end{pinyinscope}