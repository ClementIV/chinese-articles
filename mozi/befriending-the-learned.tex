\article{親士}

\begin{pinyinscope}
入國而不存其士,則亡國矣。見賢而不急,則緩其君矣。非賢無急,非士無與慮國,緩賢忘士而能以其國存者,未曾有也。

昔者文公出走而正天下,桓公去國而霸諸侯,越王句踐遇吳王之醜,而尚攝中國之賢君。三子之能達名成功於天下也,皆於其國抑而大醜也。太上無敗,其次敗而有以成,此之謂用民。

吾聞之曰:「非無安居也,我無安心也。非無足財也,我無足心也。」是故君子自難而易彼,眾人自易而難彼,君子進不敗其志,內究其情,雖雜庸民,終無怨心,彼有自信者也。是故為其所難者,必得其所欲焉,未聞為其所欲,而免其所惡者也。是故偪臣傷君,諂下傷上。君必有弗弗之臣,上必有詻詻之下。分議者延延,而支苟者詻詻,焉可以長生保國。

臣下重其爵位而不言,近臣則喑,遠臣則唫,怨結於民心,諂諛在側,善議障塞,則國危矣。桀紂不以其無天下之士邪?殺其身而喪天下。故曰:「歸國寶,不若獻賢而進士。」

今有五錐,此其銛,銛者必先挫。有五刀,此其錯,錯者必先靡,是以甘井近竭,招木近伐,靈龜近灼,神蛇近暴。是故比干1之殪,其抗也;孟賁之殺,其勇也;西施之沈,其美也;吳起之裂,其事也。故彼人者,寡不死其所長,故曰:「太盛難守也。」

故雖有賢君,不愛無功之臣;雖有慈父,不愛無益之子。是故不勝其任而處其位,非此位之人也;不勝其爵而處其祿,非此祿之主也。良弓難張,然可以及高入深;良馬難乘,然可以任重致遠;良才難令,然可以致君見尊。是故江河不惡小谷之滿己也,故能大。聖人者,事無辭也,物無違也,故能為天下器。是故江河之水,非一源之水1也。千鎰之裘,非一狐之白也。夫惡有同方取不取同而已者乎?蓋非兼王之道也。是故天地不昭昭,大水不潦潦,大火不燎燎,王德不堯堯者,乃千人之長也。

其直如矢,其平如砥,不足以覆萬物,是故谿陝者速涸,逝淺者速竭,墝埆者其地不育。王者淳澤不出宮中,則不能流國矣。


\end{pinyinscope}