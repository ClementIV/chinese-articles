\article{非攻下}

\begin{pinyinscope}
子墨子言曰:「今天下之所譽善者,其說將何哉1?為其上中天之利,而中中鬼之利,而下中人之利,故譽之與?意亡非為其上中天之利,而中中鬼之利,而下中人之利,故譽之與?雖使下愚之人,必曰:『將為其上中天之利,而中中鬼之利,而下中人之利,故譽之』。今天下之所同義者,聖王之法也。今天下之諸侯將猶多皆免攻伐并兼,則是有譽義之名,而不察其實也。此譬猶盲者之與人,同命白黑之名,而不能分其物也,則豈謂有別哉?是故古之知者之為天下度也,必順慮其義,而後為之行,是以動則不疑,速通成得其所欲,而順天鬼百姓之利,則知者之道也。是故古之仁人有天下者,必反大國之說,一天下之和,總四海之內,焉率天下之百姓,以農臣事上帝山川鬼神。利人多,功故又大,是以天賞之,鬼富之,人譽之,使貴為天子,富有天下,名參乎天地,至今不廢。此則知者之道也,先王之所以有天下者也。

今王公大人天下之諸侯則不然,將必皆差論其爪牙之士,皆列其舟車之卒伍,於此為堅甲利兵,以往攻伐無罪之國。入其國家邊境,芟刈其禾稼,斬其樹木,墮其城郭,以湮其溝池,攘殺其牲牷,燔潰其祖廟,勁殺其萬民,覆其老弱,遷其重器,卒進而柱乎鬥,曰『死命為上,多殺次之,身傷者為下,又況失列北橈乎哉,罪死無赦』,以譂其眾。夫無兼國覆軍,賊虐萬民,以亂聖人之緒。意將以為利天乎?夫取天之人,以攻天之邑,此刺殺天民,剝振神之位,傾覆社稷,攘殺其犧牲,則此上不中天之利矣。意將以為利鬼乎?夫殺之人,滅鬼神之主,廢滅先王,賊虐萬民,百姓離散,則此中不中鬼之利矣。意將以為利人乎?夫殺之人,為利人也博矣。又計其費此,為周生之本,竭天下百姓之財用,不可勝數也,則此下不中人之利矣。

今夫師者之相為不利者也,曰:將不勇,士不分,兵不利,教不習,師不眾,率不利和,威不圉,害之不久,爭之不疾,孫之不強。植心不堅,與國諸侯疑,與國諸侯疑,則敵生慮,而意羸矣。偏具此物,而致從事焉,則是國家失卒,而百姓易務也。今不嘗觀其說好攻伐之國?若使中興師,君子庶人也,必且數千,徒倍十萬,然後足以師而動矣。久者數歲,速者數月,是上不暇聽治,士不暇治其官府,農夫不暇稼穡,婦人不暇紡績織紝,則是國家失卒,而百姓易務也,然而又與其車馬之罷弊也,幔幕帷蓋,三軍之用,甲兵之備,五分而得其一,則猶為序疏矣。然而又與其散亡道路,道路遼遠,糧食下繼傺,食飲之時,廁役以此飢寒凍餒疾病,而轉死溝壑中者,不可勝計也。此其為不利於人也,天下之害厚矣。而王公大人,樂而行之。則此樂賊滅天下之萬民也,豈不悖哉!今天下好戰之國,齊、晉、楚、越,若使此四國者得意於天下,此皆十倍其國之眾,而未能食其地也。是人不足而地有餘也。今又以爭地之故,而反相賊也,然則是虧不足,而重有餘也」。

今遝夫好攻伐之君,又飾其說以非子墨子曰:「以攻伐之為不義,非利物與?昔者禹征有苗,湯伐桀,武王伐紂,此皆立為聖王,是何故也?」子墨子曰:「子未察吾言之類,未明其故者也。彼非所謂攻,謂誅也。昔者三苗大亂,天命殛之,日妖宵出,雨血三朝,龍生於廟,犬哭乎巿,夏冰,地坼及泉,五穀變化,民乃大振。高陽乃命玄宮,禹親把天之瑞令以征有苗,四電誘袛,有神人面鳥身,若瑾以侍,搤矢有苗之祥,苗師大亂,後乃遂幾。禹既已克有三苗,焉磨為山川,別物上下,卿制大極,而神民不違,天下乃靜。則此禹之所以征有苗也。遝至乎夏王桀,天有酷命,日月不時,寒暑雜至,五穀焦死,鬼呼國,鶴鳴十夕餘。天1乃命湯於鑣宮,用受夏之大命,夏德大亂,予既卒其命於天矣,往而誅之,必使汝堪之。湯焉敢奉率其眾,是以鄉有夏之境,帝乃使陰暴毀有夏之城。少少有神來告曰:『夏德大亂,往攻之,予必使汝大堪之。予既受命於天,天命融隆火,于夏之城閒西北之隅。湯奉桀眾以克有,屬諸侯於薄,薦章天命,通于四方,而天下諸侯莫敢不賓服。則此湯之所以誅桀也。遝至乎商王紂天不序其德,祀用失時。兼夜中,十日雨土于薄,九鼎遷止,婦妖宵出,有鬼宵吟,有女為男,天雨肉,棘生乎國道,王兄自縱也。赤鳥銜珪,降周之岐社,曰:『天命周文王伐殷有國。』泰顛來賓,河出綠圖,地出乘黃。武王踐功,夢見三神曰2:予既沈漬殷紂于酒德矣,往攻之,予必使汝大堪之』。武王乃攻狂夫,反商之周,天賜武王黃鳥之旗。王既已克殷,成帝之來,分主諸神,祀紂先王,通維四夷,而天下莫不賓,焉襲湯之緒,此即武王之所以誅紂也。若以此三聖王者觀之,則非所謂攻也,所謂誅也」。

則夫好攻伐之君,又飾其說以非子墨子曰:「子以攻伐為不義,非利物與?昔者楚熊麗始討此睢山之閒,越王繄虧」,出自有遽,始邦於越,唐叔與呂尚邦齊晉。此皆地方數百里,今以并國之故,四分天下而有之。是故何也?」子墨子曰:「子未察吾言之類,未明其故者也。古者天子之始封諸侯也,萬有餘,今以并國之故,萬國有餘皆滅,而四國獨立。此譬猶醫之藥萬有餘人,而四人愈也,則不可謂良醫矣。」

則夫好攻伐之君又飾其說曰:「我非以金玉、子女、壤地為不足也,我欲以義名立於天下,以德求諸侯也。」子墨子曰:「今若有能以義名立於天下,以德求諸侯者,天下之服可立而待也。夫天下處攻伐久矣,譬若傅子之為馬然。今若有能信效先利天下諸侯者,大國之不義也,則同憂之;大國之攻小國也,則同救之;小國城郭之不全也,必使修之;布粟之絕,則委之;幣帛不足,則共之。以此效大國,則小國之君說,人勞我逸,則我甲兵強。寬以惠,緩易急,民必移。易攻伐以治我國,攻必倍。量我師舉之費,以爭諸侯之斃,則必可得而序利焉。督以正,義其名,必務寬吾眾,信吾師,以此授諸侯之師,則天下無敵矣。其為下不可勝數也。此天下之利,而王公大人不知而用,則此可謂不知利天下之巨務矣。」是故子墨子曰:「今且天下之王公大人士居子,中情將欲求興天下之利,除天下之害,當若繁為攻伐,此實天下之巨害也。今欲為仁義,求為上士,尚欲中聖王之道,下欲中國家百姓之利,故當若非攻之為說,而將不可不察者此也。」


\end{pinyinscope}