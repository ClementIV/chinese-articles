\article{小取}

\begin{pinyinscope}
夫辯者,將以明是非之分,審治亂之紀,明同異之處,察名實之理,處利害,決嫌疑。焉摹略萬物之然,論求群言之比。以名舉實,以辭抒意,以說出故,以類取,以類予。有諸己不非諸人,無諸己不求諸人。屬於:[邏輯]

或也者,不盡也。假者,今不然也。效者,為之法也,所效者所以為之法也。故中效,則是也;不中效,則非也。此效也。辟也者,舉他1物而以明之也。侔也者,比辭而俱行也。援也者,曰「子然,我奚獨不可以然也?」推也者,以其所不取之同於其所取者,予之也。是猶謂也者,同也。吾豈謂也者,異也。屬於:[邏輯]

夫物有以同而不率遂同。辭之侔也,有所至而正。其然也,有所以然也;其然也1同,其所以然不必同。其取之也,有所2以取之。其取之也同,其所以取之不必同。是故辟、侔、援、推之辭,行而異,轉而危,遠而失,流而離本,則不可不審也,不可常用也。故言多方,殊類,異故,則不可偏觀也。夫物或乃是而然,或是而不然,或一周3而一不周4,或一是而一不是也。不可常用也。故言多方,殊類,異故,則不可偏觀也,5非也。屬於:[邏輯]

白馬,馬也;乘白馬,乘馬也。驪馬,馬也;乘驪馬,乘馬也。獲,人也;愛獲,愛人也。臧,人也;愛臧,愛人也。此乃是而然者也。屬於:[邏輯]

獲之親1,人也;獲事其親,非事人也。其弟,美人也;愛弟,非愛美人也。車,木也;乘車,非乘木也。船,木也;入2船,非入3木也。盜人,人也,多盜,非多人也,無盜非無人也。奚以明之?惡多盜,非惡多人也;欲無盜,非欲無人也。世相與共是之。若若是,則雖盜人人也,愛盜非愛人也;不愛盜非不愛人也;殺盜人非殺人也,無難盜無難4矣。此與彼同類,世有彼而不自非也,墨者有此而非之,無他故5焉,所謂內膠外閉與心毋空乎?內膠而不解也,此乃是而不然6者也。屬於:[邏輯]

且夫讀書,非書也;好讀書,1好書也。且鬥雞,非雞也;好鬥雞,好雞也。且入井,非入井也;止且入井,止入井也。且出門,非出門也;止且出門,止出門也。若若是,且夭,非夭也;壽夭也。有命,非命也;非執有命,非命也,無難矣。此與彼同類2,世有彼而不自非也,墨者有此而罪非之,無也故焉,所謂內膠外閉與心毋空乎?內膠而不解也。此乃不3是而然者也。屬於:[邏輯]

愛人,待周愛人而後為愛人。不愛人,不待周不愛人;不周愛,因為不愛人矣。乘馬,不待周乘馬然後為乘馬也;有乘於馬,因為乘馬矣。逮至不乘馬,待周不乘馬而後不乘馬而後不乘馬1。此一周而一不周者也。屬於:[邏輯]

居於國,則為居國;有一宅於國,而不為有國。桃之實,桃也;棘之實,非棘也。問人之病,問人也;惡人之病,非惡人也。人之鬼,非人也;兄之鬼,兄也。祭人1之鬼,非祭人也;祭兄之鬼,乃祭兄也。之馬之目盼則為之馬盼;之馬之目大,而不謂之馬大。之牛之毛黃,則謂之牛黃;之牛之毛眾,而不謂之牛眾。一馬,馬也;二馬,馬也。馬四足者,一馬而四足也,非兩馬而四足也。一馬,馬也。馬或白者,二馬而或白也,非一馬而或白。此乃一是而一非者也。屬於:[邏輯]


\end{pinyinscope}