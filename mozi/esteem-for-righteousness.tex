\article{貴義}

\begin{pinyinscope}
子墨子曰:「萬事莫貴於義。今謂人曰:『予子冠履,而斷子之手足,子為之乎?』必不為,何故?則冠履不若手足之貴也。又曰:『予子天下而殺子之身,子為之乎?』必不為,何故?則天下不若身之貴也。爭一言以相殺,是貴義於其身也。故曰,萬事莫貴於義也。」

子墨子自魯即齊,過故人,謂子墨子曰:「今天下莫為義,子獨自苦而為義,子不若已。」子墨子曰:「今有人於此,有子十人,一人耕而九人處,則耕者不可以不益急矣。何故?則食者眾,而耕者寡也。今天下莫為義,則子如勸我者也,何故止我?」子墨子南游於楚,見楚獻惠王,獻惠王以老辭,使穆賀見子墨子。子墨子說穆賀,穆賀大說,謂子墨子曰:「子之言則成善矣!而君王,天下之大王也,毋乃曰『賤人之所為』,而不用乎?」子墨子曰:「唯其可行。譬若藥然,天子食之以順其疾,豈曰『一草之本』而不食哉?今農夫入其稅於大人,大人為酒醴粢盛以祭上帝鬼神,豈曰『賤人之所為』而不享哉?故雖賤人也,上比之農,下比之藥,曾不若一草之本乎?且主君亦嘗聞湯之說乎?昔者,湯將往見伊尹,令彭氏之子御。彭氏之子半道而問曰:『君將何之?』湯曰:『將往見伊尹。』彭氏之子曰:『伊尹,天下之賤人也。若君欲見之,亦令召問焉,彼受賜矣。』湯曰:『非女所知也。今有藥此,食之則耳加聰,目加明,則吾必說而強食之。今夫伊尹之於我國也,譬之良醫善藥也。而子不欲我見伊尹,是子不欲吾善也。』因下彭氏之子,不使御。彼苟然,然後可也」。

子墨子曰:「凡言凡動,利於天鬼百姓者為之;凡言凡動,害於天鬼百姓者舍之;凡言凡動,合於三代聖王堯舜禹湯文武者為之;凡言凡動,合於三代暴王桀紂幽厲者舍之。」

子墨子曰:「言足以遷行者,常之;不足以遷行者,勿常。不足以遷行而常之,是蕩口也。」

子墨子曰:「必去六辟。嘿則思,言則誨,動則事,使三者代御,必為聖人。必去喜,去怒,去樂,去悲,去愛,而用仁義。手足口鼻耳,從事於義,必為聖人。」

子墨子謂二三子曰:「為義而不能,必無排其道。譬若匠人之斲而不能,無排其繩。」

子墨子曰:「世之君子,使之為一犬一彘之宰,不能則辭之;使為一國之相,不能而為之。豈不悖哉!」

子墨子曰:「今瞽曰:『鉅者白也,黔者黑也。』雖明目者無以易之。兼白黑,使瞽取焉,不能知也。故我曰瞽不知白黑者,非以其名也,以其取也。今天下之君子之名仁也,雖禹湯無以易之。兼仁與不仁,而使天下之君子取焉,不能知也。故我曰天下之君子不知仁者,非以其名也,亦以其取也。」

子墨子曰:「今士之用身,不若商人之用一布之慎也。商人用一布布,不敢繼苟而讎焉,必擇良者。今士之用身則不然,意之所欲則為之,厚者入刑罰,薄者被毀醜,則士之用身不若商人之用一布之慎也。」

子墨子曰:「世之君子欲其義之成,而助之修其身則慍,是猶欲其牆之成,而人助之築則慍也,豈不悖哉!」

子墨子曰:「古之聖王,欲傳其道於後世,是故書之竹帛,鏤之金石,傳遺後世子孫,欲後世子孫法之也。今聞先王之遺而不為,是廢先王之傳也。」

子墨子南遊使衛,關中載書甚多,弦唐子見而怪之,曰:「吾夫子教公尚過曰:『揣曲直而已。』今夫子載書甚多,何有也?」子墨子曰:「昔者周公旦朝讀書百篇,夕見漆十士。故周公旦佐相天子,其脩至於今。翟上無君上之事,下無耕農之難,吾安敢廢此?翟聞之:『同歸之物,信有誤者。』然而民聽不鈞,是以書多也。今若過之心者,數逆於精微,同歸之物,既已知其要矣,是以不教以書也。而子何怪焉?」

子墨子謂公良桓子曰:「衛,小國也,處於齊、晉之閒,猶貧家之處於富家之閒也。貧家而學富家之衣食多用,則速亡必矣。今簡子之家,飾車數百乘,馬食菽粟者數百匹,婦人衣文繡者數百人,吾取飾車、食馬之費,與繡衣之財以畜士,必千人有餘。若有患難,則使百人處於前,數百於後,與婦人數百人處前後,孰安?吾以為不若畜士之安也。」

子墨子仕人於衛,所仕者至而反。子墨子曰:「何故反?」對曰:「與我言而不當。曰『待女以千盆。』授我五百盆,故去之也。」子墨子曰:「授子過千盆,則子去之乎?」對曰:「不去。」子墨子曰:「然則,非為其不審也,為其寡也。」

子墨子曰:「世俗之君子,視義士不若負粟者。今有人於此,負粟息於路側,欲起而不能,君子見之,無長少貴賤,必起之。何故也?曰義也。今為義之君子,奉承先王之道以語之,縱不說而行,又從而非毀之。則是世俗之君子之視義士也,不若視負粟者也。」

子墨子曰:「商人之四方,市賈信徙,雖有關梁之難,盜賊之危,必為之。今士坐而言義,無關梁之難,盜賊之危,此為信徙,不可勝計,然而不為。則士之計利不若商人之察也。」

子墨子北之齊,遇日者。日者曰:「帝以今日殺黑龍於北方,而先生之色黑,不可以北。」子墨子不聽,遂北,至淄水,不遂而反焉。日者曰:「我謂先生不可以北。」子墨子曰:「南之人不得北,北之人不得南,其色有黑者有白者,何故皆不遂也?且帝以甲乙殺青龍於東方,以丙丁殺赤龍於南方,以庚辛殺白龍於西方,以壬癸殺黑龍於北方,若用子之言,則是禁天下之行者也。是圍心而虛天下也,子之言不可用也。」

子墨子曰:「吾言足用矣,舍言革思者,是猶舍穫而拾粟也。以其言非吾言者,是猶以卵投石也,盡天下之卵,其石猶是也,不可毀也。」


\end{pinyinscope}