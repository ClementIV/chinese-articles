\article{尚賢上}

\begin{pinyinscope}
子墨子言曰:「今1者王公大人為政於國家者,皆欲國家之富,人民之眾,刑政之治,然而不得富而得貧,不得眾而得寡,不得治而得亂,則是本失其所欲,得其所惡,是其故何也?」

子墨子言曰:「是在王公大人為政於國家者,不能以尚賢事能為政也。是故國有賢良之士眾,則國家之治厚,賢良之士寡,則國家之治薄。故大人之務,將在於眾賢而己。」

曰:「然則眾賢之術將柰何哉?」

子墨子言曰:「譬若欲眾其國之善射御之士者,必將富之,貴之,敬之,譽之,然后國之善射御之士,將可得而眾也。況又有賢良之士厚乎德行,辯乎言談,博乎道術者乎,此固國家之珍,而社稷之佐也,亦必且富之,貴之,敬之,譽之,然后國之良士,亦將可得而眾也。

是故古者聖王之為政也1,言曰:「不義不富,不義不貴,不義不親,不義不近。」是以國之富貴人聞之,皆退而謀曰:『始我所恃者,富貴也,今上舉義不辟貧賤,然則我不可不為義。』親者聞之,亦退而謀曰:『始我所恃者親也,今上舉義不辟親2疏,然則我不可不為義。』近者聞之,亦退而謀曰:『始我所恃者近也,今上舉義不避遠,然則我不可不為義。』遠者聞之,亦退而謀曰:『我始以遠為無恃,今上舉義不辟遠,然則我不可不為義。』逮至遠鄙郊外之臣,門庭庶子,國中之眾、四鄙之萌人聞之,皆競為義。是其故何也?曰:上之所以使下者,一物也,下之所以事上者,一術也。譬之富者有高牆深宮,牆立既,謹上為鑿一門,有盜人入,闔其自入而求之,盜其無自出。是其故何也?則上得要也。

故古者聖王之為政,列德而尚賢,雖在農與工肆之人,有能則舉之,高予之爵,重予之祿,任之以事,斷予之令,曰:「爵位不高則民弗敬,蓄祿不厚則民不信,政令不斷則民不畏」,舉三者授之賢者,非為賢賜也,欲其事之成。故當是時,以德就列,以官服事,以勞殿賞,量功而分祿。故官無常貴,而民無終賤,有能則舉之,無能則下之,舉公義,辟私怨,此若言之謂也。故古者堯舉舜於服澤之陽,授之政,天下平;禹舉益於陰方之中,授之政,九州成;湯舉伊尹於庖廚之中,授之政,其謀得;文王舉閎夭泰顛於罝罔之中,授之政,西土服。故當是時,雖在於厚祿尊位之臣,莫不敬懼而施,雖在農與工肆之人,莫不競勸而尚意。故士者所以為輔相承嗣也。故得士則謀不困,體不勞,名立而功成,美章1而惡不生,則由得士也。」

是故子墨子言曰:「得意賢士不可不舉,不得意賢士不可不舉,尚欲祖述堯舜禹湯之道,將不可以不尚賢。夫尚賢者,政之本也。」


\end{pinyinscope}