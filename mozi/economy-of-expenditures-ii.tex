\article{節用中}

\begin{pinyinscope}
子墨子言曰:「古者明王聖人,所以王天下,正諸侯者,彼其愛民謹忠,利民謹厚,忠信相連,又示之以利,是以終身不饜,歿世1而不卷。古者明王聖人,其所以王天下正諸侯者,此也。

是故古者聖王,制為節用之法曰:『凡天下群百工,輪車、韗鞄、陶、冶、梓匠,使各從事其所能』,曰:『凡足以奉給民用,則止。』諸加費不加于民利者,聖王弗為1。

古者聖王制為飲食之法曰:『足以充虛繼氣,強股肱,耳目聰明,則止。不極五味之調,芬香之和,不致遠國珍怪異物。』何以知其然?古者堯治天下,南撫交阯北降幽都,東西至日所出入,莫不賓服。逮至其厚愛,黍稷不二,羹胾不重,飯於土塯,啜於土形,斗以酌。俛仰周旋威儀之禮,聖王弗為。

古者聖王制為衣服之法曰:『冬服紺緅之衣,輕且暖,夏服絺綌之衣,輕且凊,則止。』諸加費不加於民利者,聖王弗為。古者聖人為猛禽狡獸,暴人害民,於是教民以兵行,日帶劍,為刺則入,擊則斷,旁擊而不折,此劍之利也。甲為衣則輕且利,動則兵且從,此甲之利也。車為服重致遠,乘之則安,引之則利,安以不傷人,利以速至,此車之利也。古者聖王為大川廣谷之不可濟,於是利為舟楫,足以將之則止。雖上者三公諸侯至,舟楫不易,津人不飾,此舟之利也。

古者聖王制為節葬之法曰:『衣三領,足以朽肉,棺三寸,足以朽骸,堀穴深不通於泉,流不發洩則止。死者既葬,生者毋久喪用哀。』

古者人之始生,未有宮室之時,因陵丘堀穴而處焉。聖王慮之,以為堀穴曰:『冬可以辟風寒』,逮夏,下潤溼,上熏烝,恐傷民之氣,于是作為宮室而利。」然則為宮室之法將柰何哉?子墨子言曰:「其旁可以圉風寒,上可以圉雪霜雨露,其中蠲潔,可以祭祀,宮牆足以為男女之別則止,諸加費不加民利者,聖王弗為。」


\end{pinyinscope}