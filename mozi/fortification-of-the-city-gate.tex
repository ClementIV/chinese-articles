\article{備城門}

\begin{pinyinscope}
禽滑釐問於子墨子曰:「由聖人之言,鳳鳥之不出,諸侯畔殷周之國,甲兵方起於天下,大攻小,強執弱,吾欲守小國,為之柰何?」子墨子曰:「何攻之守?」禽滑釐對曰:「今之世常所以攻者:臨、鉤、衝、梯、堙、水、穴、突、空洞、蟻傅、轒轀軒車,敢問守此十二者柰何?」子墨子曰:「我城池修,守器具,樵粟足,上下相親,又得四鄰諸侯之救,此所以持也。且守者雖善,而君不用之,則猶若不可以守也。若君用之守者,又必能乎守者,不能而君用之,則猶若不可以守也。然則守者必善而君尊用之,然後可以守也。

凡守圉城之法,厚以高,壕池深以廣,樓撕脩,守備繕利,薪食足以支三月以上,人眾以選,吏民和,大臣有功勞於上者多,主信以義,萬民樂之無窮。不然,父母墳墓在焉;不然,山林草澤之饒足利;不然,地形之難攻而易守也;不然,則有深怨於適而有大功於上;不然則賞明可信而罰嚴足畏也。此十四者具,則民亦不宜上矣。然後城可守。十四者無一,則雖善者不能守矣。

故凡守城之法,備城門為縣門沈機,長二丈,廣八尺,為之兩相如;門扇數令相接三寸,施土扇上,無過二寸。塹中深丈五,廣比扇,塹長以力為度,塹之末為之縣,可容一人所。客至,諸門戶皆令鑿而慕孔。孔之。各為二幕二,一鑿而繫繩,長四尺。城四面四隅皆為高樓磿撕,使重室子居亓上,磿適,視亓態狀,與亓進左右所移處,失磿斬。

適人為穴而來,我亟使穴師選士,迎而穴之,為之且內弩以應之。

民室材木瓦石,可以益城之備者,盡上之。不從令者斬。

皆築,七尺一居屬,五步一壘。五築有銕。長斧,柄長八尺。十步一長鎌,柄長八尺。十步一斲,長椎,柄長六尺,頭長尺,兌亓兩端。三步一大鋌,前長尺,蚤長五寸。兩鋌交之置如平,不如平不利,兌亓兩末。穴隊若衝隊,必審如攻隊之廣狹,而令邪穿亓穴,令亓廣必夷客隊。

疏束樹木,令足以為柴摶,貫前面樹,長丈七尺一以為外面,以柴摶從橫施之,外面以強塗,毋令土漏。令亓廣厚,能任三丈五尺之城以上。以柴木土稍杜之,以急為故。前面之長短,豫蚤接之,令能任塗,足以為堞,善塗亓外,令毋可燒拔也。

大城丈五為閨門,廣四尺。

為郭門,郭門在外,為衡,以兩木當門,鑿亓木維敷上堞。

為斬縣梁,令穿,斷城以板橋,邪穿外,以板次之,倚殺如城報。城內有傅堞,因以內堞為外。鑿亓閒,深丈五尺,室以樵,可燒之以待適。

令耳屬城,為再重樓。下鑿城外堞內深丈五,廣丈二。樓若令耳,皆令有力者主敵,善射者主發,佐以厲矢。

治裾諸,延堞,高六尺,部廣四尺,皆為兵弩簡格。

轉射機,機長六尺,貍一尺。兩材合而為之轀,轀長二尺,中鑿夫之為通臂,臂長至垣。二十步一,令善射者佐之,令一人下上之勿離。

城上百步一樓,樓四植,植皆為通舄,下高丈,上九尺,廣、袤各丈六尺,皆為文。三十步一突,九尺,廣十尺,高八尺,鑿廣三尺,袤二尺,為文。

城上為攢火,矢長以城高下為度,置火亓末。

城上九尺一弩、一戟、一椎、一斧、一艾,皆積絫石、蒺藜。

渠長丈六尺,夫長丈二尺,臂長六尺,亓貍者三尺,樹渠毋傅堞五寸。

藉莫長八尺,廣七尺,亓木也廣五尺,中藉苴為之橋,索亓端;適攻,令一人下上之,勿離。

城上二十步一藉車,當隊者不用此數。

城上三十步一礱灶。

持水者必以布麻斗、革盆,十步一。柄長八尺,斗大容二斗以上到三斗。敝綌、新布長六尺,中拙柄,長丈,十步一,必以大繩為箭。

城上十步一鈂。

水缶,容三石以上,小大相雜。盆、蠡各二財。

為卒乾飯,人二斗,以備陰雨,面使積燥處。令使守為城內堞外行餐。

置器備,殺沙礫鐵,皆為坏斗。令陶者為薄缶,大容一斗以上至二斗,即用取,三祕合束。

堅為斗城上隔。棧高丈二,剡亓一末。

為閨門,閨門兩扇,令可以各自閉也。

救闉池者,以火與爭,鼓橐,馮埴外內,以柴為燔。

靈丁,三丈一,犬牙施之。十步一人,居柴內帑,柴半,為狗犀者環之。牆七步而一。

救車火,為熛矢射火城門上,鑿扇上為棧,塗之,持水麻斗、革盆救之。門扇薄植,皆鑿半尺,一寸一涿弋,弋長二寸,見一寸,相去七寸,厚塗之以備火。城門上所鑿以救門火者,各一垂水,容三石以上,小大相雜。門植關必環錮,以錮金若鐵鍱之。門關再重,鍱之以鐵,必堅。梳關,關二尺,梳關一莧,封以守印,時令人行貌封,及視關入桓淺深。門者皆無得挾斧、斤、鑿、鋸、椎。

城上二步一渠,渠立程,丈三尺,冠長十丈,辟長六尺。二步一荅,廣九尺,袤十二尺。

二步置連梃、長斧、長椎各一物;槍二十枚,周置二步中。

二步一木弩,必射五十步以上。及多為矢,即毋竹箭,以楛、桃、柘、榆,可。蓋求齊鐵夫,播以射衝及櫳樅。

二步積石,石重千鈞以上者,五百枚。毋百,以亢疾犁、壁,皆可善方。

二步積苙,大一圍,長丈,二十枚。

五步一罌,盛水有奚,奚蠡大容一斗。

五步積狗屍五百枚,狗屍長三尺,喪以弟,瓮亓端,堅約弋。

十步積摶,大二圍以上,長八尺者二十枚。

二十五步一灶,灶有鐵鐕容石以上者一,戒以為湯。及持沙,毋下千石。

三十步置坐侯樓,樓出於堞四尺,廣三尺,廣四尺,板周三面,密傅之,夏蓋亓上。

五十步一藉車,藉車必為鐵纂。

五十步一井屏,周垣之,高八尺。

五十步一方,方尚必為關籥守之。

五十步積薪,毋下三百石,善蒙塗,毋令外火能傷也。

百步一櫳樅,起地高五丈,三層,下廣前面八尺,後十三尺,亓上稱議衰殺之。

百步一木樓,樓廣前面九尺,高七尺,樓囪居坫,出城十二尺。

百步一井,井十罋,以木為繫連。水器容四斗到六斗者百。

百步一積雜稈,大二圍以上者五十枚。

百步為櫓,櫓廣四尺,高八尺。為衝術,

百步為幽竇,廣三尺高四尺者千。

二百步一立樓,城中廣二丈五尺二,長二丈,出樞五尺。

城上廣三步到四步,乃可以為使鬥。俾倪廣三尺,高二尺五寸。陛高二尺五,廣長各三尺,遠廣各六尺。城上四隅童異高五尺四尉舍焉。

城上七尺一渠,長丈五尺,貍三尺,去堞五寸,夫長丈二尺,臂長六尺。半植一鑿,內後長五寸。夫兩鑿,渠夫前端下堞四寸而適。鑿渠、鑿坎,覆以瓦,冬日以馬夫寒,皆待命,若以瓦為坎。

城上千步一表,長丈,棄水者操表搖之。五十步一廁,與下同圂。之廁者,不得操。

城上三十步一藉車,當隊者不用。

城上五十步一道陛,高二尺五寸,長十步。城上五十步一樓,樓撕必再重。

土樓百步一,外門發樓,左右渠之。為樓加藉幕,棧上出之以救外。

城上皆毋得有室,若也可依匿者,盡除去之。

城下州道內百步一積薪,毋下三千石以上,善塗之。

城上十人一什長,屬一吏士、一帛尉。

百步一亭,高垣丈四尺,厚四尺,為閨門兩扇,令各可以自閉。亭一尉,尉必取有重厚忠信可任事者。

二舍共一井爨,灰、康、秕、秠馬矢,皆謹收藏之。

城上之備:渠譫、藉車、行棧、行樓、到,頡皋、連梃、長斧、長椎、長茲、距、飛衝、縣口、批屈。樓五十步一,堞下為爵穴,三尺而一為薪皋,二圍長四尺半必有潔。

瓦石:重二升以上,上。城上沙,五十步一積。灶置鐵鐕焉,與沙同處。

木大二圍,長丈二尺以上,善耿亓本,名曰長從,五十步三十。木橋長三丈,毋下五十。復使卒急為壘壁,以蓋瓦復之。

用瓦木罌,容十升以上者,五十步而十,盛水,且用之。五十二者十步而二。

城下里中家人,各葆亓左右前後,如城上。城小人眾,葆離鄉老弱國中及也大城。

寇至,度必攻,主人先削城編,唯勿燒寇在城下,時換吏卒署,而毋換亓養,養毋得上城。寇在城下,收諸盆罋,耕積之城下,百步一積,積五百。

城門內不得有室,為周官桓吏,四尺為倪。行棧內閈,二關一堞。

除城場外,去池百步,牆垣樹木小大俱壞伐,除去之。寇所從來若昵道、傒近,若城場,皆為扈樓。立竹箭天中。

守堂下為大樓,高臨城,堂下周散,道中應客,客待見,時召三老在葆宮中者,與計事得先。行德計謀合,乃入葆。葆入守,無行城,無離舍。諸守者,審知卑城淺池,而錯守焉。晨暮卒歌以為度,用人少易守。

守法:五十步丈夫十人、丁女二十人、老小十人,計之五十步四十人。城下樓卒,率一步一人,二十步二十人。城小大以此率之,乃足以守圉。

客馮面而蛾傅之,主人則先之知,主人利,客適。客攻以遂,十萬物之眾,攻無過四隊者,上術廣五百步,中術三百步,下術五十步。諸不盡百五步者,主人利而客病。廣五百步之隊,丈夫千人,丁女子二千人,老小千人,凡四千人,而足以應之,此守術之數也。使老小不事者,守於城上不當術者。

城持出必為明填,令吏民皆智知之。從一人百人以上,持出不操填章,從人非亓故人,乃亓稹章也,千人之將以上止之,勿令得行。行及吏卒從之,皆斬,具以聞於上。此守城之重禁之,夫姦之所生也,不可不審也。

城上為爵穴,下堞三尺,廣亓外,五步一。爵穴大容苴,高者六尺,下者三尺,疏數自適為之。塞外塹,去格七尺,為縣梁。城笮陜不可塹者,勿塹。城上三十步一聾灶,人擅苣長五節。寇在城下,聞鼓音,燔苣,復鼓,內苣爵穴中,照外。

諸藉車皆鐵什,藉車之柱長丈七尺,亓貍者四尺;夫長三丈以上,至三丈五尺,馬頰長二尺八寸,試藉車之力而為之困,失四分之三在上。藉車,夫長三尺,四二三在上,馬頰在三分中。馬頰長二尺八寸,夫長二十四尺,以下不用。治困以大車輪。藉車桓長丈二尺半,諸藉車皆鐵什,復車者在之。

寇闉池來,為作水甬,深四尺,堅慕貍之。十尺一,覆以瓦而待令。以木大圍長二尺四分而早鑿之,置炭火亓中而合慕之,而以藉車投之。為疾犁投,長二尺五寸,大二圍以上。涿弋,弋長七寸,弋閒六寸,剡亓末。狗走,廣七寸,長尺八寸,蚤長四寸,犬耳施之。」

子墨子曰:「守城之法,必數城中之木,十人之所舉為十挈,五人之所舉為五挈,凡輕重以挈為人數。為薪樵挈,壯者有挈,弱者有挈,皆稱亓任。凡挈輕重所為,吏人各得亓任。城中無食則為大殺。去城門五步大塹之,高地三丈下地至,施賊亓中,上為發梁,而機巧之,比傳薪土,使可道行,旁有溝壘,毋可踰越,而出佻且比,適人遂入,引機發梁,適人可禽。適人恐懼而有疑心,因而離。」


\end{pinyinscope}