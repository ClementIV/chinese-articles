\article{旗幟}

\begin{pinyinscope}
守城之法,木為蒼旗,火為赤旗,薪樵為黃旗,石為白旗,水為黑旗,食為菌旗,死士為倉英之旗,竟士為雩旗,多卒為雙兔之旗,五尺童子為童旗,女子為梯末之旗,弩為狗旗,戟為旌旗,劍盾為羽旗,車為龍旗,騎為鳥旗。凡所求索旗名不在書者,皆以其形名為旗。城上舉旗,備具之官致財物,之足而下旗。

凡守城之法:石有積,樵薪有積,菅茅有積,雚葦有積,木有積,炭有積,沙有積,松柏有積,蓬艾有積,麻脂有積,金鐵有積,粟米有積;井灶有處,重質有居,五兵各有旗,節各有辨;法令各有貞;輕重分數各有請:主慎道路者有經。

亭尉各為幟,竿長二丈五,帛長丈五,廣半幅者大。寇傅攻前池外廉,城上當隊鼓三,舉一幟;到水中周,鼓四,舉二幟;到藩,鼓五,舉三幟;到馮垣,鼓六,舉四幟;到女垣,鼓七,舉五幟;到大城,鼓八,舉六幟;乘大城半以上,鼓無休。夜以火,如此數。寇卻解,輒部幟如進數,而無鼓。

城為隆,長五十尺,四面四門將長四十尺,其次三十尺,其次二十五尺,其次二十尺,其次十五尺,高無下四十五尺。

城上吏卒置之背,卒於頭上,城中吏卒民男女,皆辨異衣章微職,令男女可知。1城下吏卒置之肩。左軍於左肩,中軍置之胸。各一鼓,中軍一三。每鼓三、十擊之,諸有鼓之吏,謹以次應之,當應鼓而不應,不當應而應鼓,主者斬。

道廣三十步,於城下夾階者,各二,其井置鐵甕。於道之外為屏,三十步而為之圜,高丈。為民圂,垣高十二尺以上。巷術周道者,必為之門,門二人守之,非有信符,勿行,不從令者斬。

城中吏卒民男女,皆辨異衣章微職,令男女可知。1諸守牲格者,三出卻適,守以令召賜食前,予大旗,署百戶邑若他人財物,建旗其署,令皆明白知之,曰某子旗。牲格內廣二十五步,外廣十步,表以地形為度。

勒卒,中教解前後左右,卒勞者更休之。


\end{pinyinscope}