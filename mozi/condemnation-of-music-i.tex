\article{非樂上}

\begin{pinyinscope}
子墨子言曰:「仁之事者,必務求興天下之利,除天下之害,將以為法乎天下。利人乎,即為;不利人乎,即止。且夫仁者之為天下度也,非為其目之所美,耳之所樂,口之所甘,身體之所安,以此虧奪民衣食之財,仁者弗為也。」是故子墨子之所以非樂者,非以大鍾、鳴鼓、琴瑟、竽笙之聲,以為不樂也;非以刻鏤華文章之色,以為不美也;非以犓豢煎炙之味,以為不甘也;非以高臺厚榭邃野之居,以為不安也。雖身知其安也,口知其甘也,目知其美也,耳知其樂也,然上考之不中聖王之事,下度之不中萬民之利。是故子墨子曰:「為樂,非也。」

今王公大人,雖無造為樂器,以為事乎國家,非直掊潦水折壤坦而為之也,將必厚措斂乎萬民,以為大鍾、鳴鼓、琴瑟、竽笙之聲。古者聖王亦嘗厚措斂乎萬民,以為舟車,既以成矣,曰:『吾將惡許用之?曰:舟用之水,車用之陸,君子息其足焉,小人休其肩背焉。』故萬民出財齎而予之,不敢以為慼恨者,何也?以其反中民之利也。然則樂器反中民之利亦若此,即我弗敢非也。然則當用樂器譬之若聖王之為舟車也,即我弗敢非也。民有三患:飢者不得食,寒者不得衣,勞者不得息,三者民之巨患也。然即當為之撞巨鍾、擊鳴鼓、彈琴瑟、吹竽笙而揚干戚,民衣食之財將安可得乎?即我以為未必然也。意舍此。今有大國即攻小國,有大家即伐小家,強劫弱,眾暴寡,詐欺愚,貴傲賤,寇亂盜賊並興,不可禁止也。然即當為之撞巨鍾、擊鳴鼓、彈琴瑟、吹竽笙而揚干戚,天下之亂也,將安可得而治與?即我未必然也。」是故子墨子曰:「姑嘗厚措斂乎萬民,以為大鍾、鳴鼓、琴瑟、竽笙之聲,以求興天下之利,除天下之害而無補也。」是故子墨子曰:「為樂,非也。」

今王公大人,唯毋處高臺厚榭之上而視之,鍾猶是延鼎也,弗撞擊將何樂得焉哉?其說將必撞擊之,惟勿撞擊,將必不使老與遲者,老與遲者耳目不聰明,股肱不畢強,聲不和調,明不轉朴。將必使當年,因其耳目之聰明,股肱之畢強,聲之和調,眉之轉朴。使丈夫為之,廢丈夫耕稼樹藝之時,使婦人為之,廢婦人紡績織紝之事。今王公大人唯毋為樂,虧奪民衣食之財,以拊樂如此多也。」是故子墨子曰:「為樂,非也。」

今大鍾、鳴鼓、琴瑟、竽笙之聲既已具矣,大人鏽然奏而獨聽之,將何樂得焉哉?其說將必與賤人不與君子。與君子1聽之,廢君子聽治;與賤人聽之,廢賤人之從事。今王公大人惟毋為樂,虧奪民之衣食之財,以拊樂如此多也。」是故子墨子曰:「為樂,非也。」

昔者齊康公興樂萬,萬人不可衣短褐,不可食糠糟,曰食飲不美,面目顏色不足視也;衣服不美,身體從容醜羸,不足觀也。是以食必粱肉,衣必文繡,此掌不從事乎衣食之財,而掌食乎人者也。」是故子墨子曰:「今王公大人惟毋為樂1,虧奪民衣食之財,以拊樂如此多也。」是故子墨子曰:「為樂,非也。」

今人固與禽獸麋鹿、蜚鳥、貞蟲異者也,今之禽獸麋鹿、蜚鳥、貞蟲,因其羽毛以為衣裘,因其蹄蚤以為褲屨,因其水草以為飲食。故唯使雄不耕稼樹藝,雌亦不紡績織紝,衣食之財固已具矣。今人與此異者也,賴其力者生,不賴其力者不生。君子不強聽治,即刑政亂;賤人不強從事,即財用不足。今天下之士君子,以吾言不然,然即姑嘗數天下分事,而觀樂之害。王公大人蚤朝晏退,聽獄治政,此其分事也;士君子竭股肱之力,亶其思慮之智,內治官府,外收斂關市、山林、澤梁之利,以實倉廩府庫,此其分事也;農夫蚤出暮入,耕稼樹藝,多聚叔粟,此其分事也;婦人夙興夜寐,紡績織紝,多治麻絲葛緒綑布縿,此其分事也。今惟毋在乎王公大人說樂而聽之,即必不能蚤朝晏退,聽獄治政,是故國家亂而社稷危矣。今惟毋在乎士君子說樂而聽之,即必不能竭股肱之力,亶其思慮之智,內治官府,外收斂關市、山林、澤梁之利,以實倉廩府庫,是故倉廩府庫不實。今惟毋在乎農夫說樂而聽之,即必不能蚤出暮入,耕稼樹藝,多聚叔1粟,是故叔粟不足2。今惟毋在乎婦人說樂而聽之,即不必能3夙興夜寐,紡績織紝,多治麻絲葛緒綑布縿,是故布縿不興。曰:孰為大人之聽治而廢國家之從事?曰:樂也。」是故子墨子曰:「為樂,非也。」

何以知其然也?曰先王之書,湯之官刑有之曰:「其恆舞于宮,是謂巫風。其刑君子出絲二衛,小人否,似二伯黃徑。」乃言曰:『嗚乎!舞佯佯,黃言孔章,上帝弗常,九有以亡,上帝不順,降之百1𦍙,其家必懷喪。』察九有之所以亡者,徒從飾樂也。於武觀曰:『啟乃淫溢康樂,野于飲食,將將銘莧磬以力,湛濁于酒,渝食于野,萬舞翼翼,章聞于大,天用弗式。』故上者天鬼弗戒,下者萬民弗利。」

是故子墨子曰:「今天下士君子,請將欲求興天下之利,除天下之害,當在樂之為物,將不可不禁而止也。」


\end{pinyinscope}