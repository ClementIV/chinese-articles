\article{魯問}

\begin{pinyinscope}
魯君謂子墨子曰:「吾恐齊之攻我也,可救乎?」子墨子曰:「可。昔者,三代之聖王禹湯文武,百里之諸侯也,說忠行義,取天下。三代之暴王桀紂幽厲,讎怨行暴,失天下。吾願主君,之上者尊天事鬼,下者愛利百姓,厚為皮幣,卑辭令,亟遍禮四鄰諸侯,敺國而以事齊,患可救也,非此,顧無可為者。」

齊將伐魯,子墨子謂項子牛曰:「伐魯,齊之大過也。昔者,吳王東伐越,棲諸會稽,西伐楚,葆昭王於隨。北伐齊,取國子以歸於吳。諸侯報其讎,百姓苦其勞,而弗為用,是以國為虛戾,身為刑戮也。昔者,智伯伐范氏與中行氏,兼三晉之地,諸侯報其讎,百姓苦其勞,而弗為用,是以國為虛戾,身為刑戮用是也。故大國之攻小國也,是交相賊也,過必反於國。」

子墨子見齊大王曰:「今有刀於此,試之人頭,倅然斷之,可謂利乎?」大王曰:「利。」子墨子曰:「多試之人頭,倅然斷之,可謂利乎?」大王曰:「利。」子墨子曰:「刀則利矣,孰將受其不祥?」大王曰:「刀受其利,試者受其不祥。」子墨子曰:「并國覆軍,賊殺百姓,孰將受其不祥?」大王俯仰而思之曰:「我受其不祥。」

魯陽文君將攻鄭,子墨子聞而止之,謂陽文君曰:「今使魯四境之內,大都攻其小都,大家伐其小家,殺其人民,取其牛馬狗豕布帛米粟貨財,則何若?」魯陽文君曰:「魯四境之內,皆寡人之臣也。今大都攻其小都,大家伐其小家,奪之貨財,則寡人必將厚罰之。」子墨子曰:「夫天之兼有天下也,亦猶君之有四境之內也。今舉兵將以攻鄭,天誅亓不至乎?」魯陽文君曰:「先生何止我攻鄭也?我攻鄭,順於天之志。鄭人三世殺其父,天加誅焉,使三年不全。我將助天誅也。」子墨子曰:「鄭人三世殺其父而天加誅焉,使三年不全。天誅足矣,今又舉兵將以攻鄭,曰『吾攻鄭也,順於天之志』。譬有人於此,其子強梁不材,故其父笞之,其鄰家之父舉木而擊之,曰:『吾擊之也,順於其父之志』,則豈不悖哉?」

子墨子謂魯陽文君曰:「攻其鄰國,殺其民人,取其牛馬、粟米、貨財,則書之於竹帛,鏤之於金石,以為銘於鍾鼎,傳遺後世子孫曰:『莫若我多。』今賤人也,亦攻其鄰家,殺其人民,取其狗豕食糧衣裘,亦書之竹帛,以為銘於席豆,以遺後世子孫曰:『莫若我多。』亓可乎?」魯陽文君曰:「然吾以子之言觀之,則天下之所謂可者,未必然也。」

子墨子為魯陽文君曰:「世俗之君子,皆知小物而不知大物。今有人於此,竊一犬一彘則謂之不仁,竊一國一都則以為義。譬猶小視白謂之白,大視白則謂之黑。是故世俗之君子,知小物而不知大物者,此若言之謂也。」

魯陽文君語子墨子曰:「楚之南有啖人之國者橋,其國之長子生,則鮮而食之,謂之宜弟。美,則以遺其君,君喜則賞其父。豈不惡俗哉?」子墨子曰:「雖中國之俗,亦猶是也。殺其父而賞其子,何以異食其子而賞其父者哉?苟不用仁義,何以非夷人食其子也?」

魯君之嬖人死,魯君為之誄,魯人因說而用之。子墨子聞之曰:「誄者,道死人之志也,今因說而用之,是猶以來首從服也。」

魯陽文君謂子墨子曰:「有語我以忠臣者,令之俯則俯,令之仰則仰,處則靜,呼則應,可謂忠臣乎?」子墨子曰:「令之俯則俯,令之仰則仰,是似景也。處則靜,呼則應,是似響也。君將何得於景與響哉?若以翟之所謂忠臣者,上有過則微之以諫,己有善,則訪之上,而無敢以告。外匡其邪,而入其善,尚同而無下比,是1以美善在上,而怨讎在下,安樂在上,而憂慼在臣。此翟之所2謂忠臣者也。」

魯君謂子墨子曰:「我有二子,一人者好學,一人者好分人財,孰以為太子而可?」子墨子曰:「未可知也,或所為賞與為是也。魡者之恭,非為魚賜也;餌鼠以蟲,非愛之也。吾願主君之合其志功而觀焉。」

魯人有因子墨子而學其子者,其子戰而死,其父讓子墨子。子墨子曰:子欲學子之子,今學成矣,戰而死,而子慍,而猶欲糶,糶讎,則慍也。豈不費哉?」

魯之南鄙人,有吳慮者,冬陶夏耕,自比於舜。子墨子聞而見之。吳慮謂子墨子「義耳義耳,焉用言之哉?」子墨子曰:「子之所謂義者,亦有力以勞人,有財以分人乎?」吳慮曰:「有。」子墨子曰:「翟嘗計之矣。翟慮耕而食天下之人矣,盛,然後當一農之耕,分諸天下,不能人得一升粟。籍而以為得一升粟,其不能飽天下之飢者,既可睹矣。翟慮織而衣天下之人矣,盛,然後當一婦人之織,分諸天下,不能人得尺布。籍而以為得尺布,其不能煖天下之寒者,既可睹矣。翟慮被堅執銳救諸侯之患,盛,然後當一夫之戰,一夫之戰其不御三軍,既可睹矣。翟以為不若誦先王之道,而求其說,通聖人之言,而察其辭,上說王公大人,次匹夫徒步之士。王公大人用吾言,國必治;匹夫徒步之士用吾言,行必脩。故翟以為雖不耕而食飢,不織而衣寒,功賢於耕而食之、織而衣之者也。故翟以為雖不耕織乎,而功賢於耕織也。」吳慮謂子墨子曰:「義耳義耳,焉用言之哉?」子墨子曰:「籍設而天下不知耕,教人耕,與不教人耕而獨耕者,其功孰多?」吳慮曰:「教人耕者其功多。」子墨子曰:「籍設而攻不義之國,鼓而使眾進戰,與不鼓而使眾進戰,而獨進戰者,其功孰多?」吳慮曰:「鼓而進眾者其功多。」子墨子曰:「天下匹夫徒步之士,少知義而教天下以義者,功亦多,何故弗言也?若得鼓而進於義,則吾義豈不益進哉?」

子墨子游公尚過於越。公尚過說越王,越王大說,謂公尚過曰:「先生苟能使子墨子於越而教寡人,請裂故吳之地,方五百里,以封子墨子。」公尚過許諾。遂為公尚過束車五十乘,以迎子墨子於魯,曰:「吾以夫子之道說越王,越王大說,謂過曰,苟能使子墨子至於越,而教寡人,請裂故吳之地,方五百里,以封子。」子墨子謂公尚過曰:「子觀越王之志何若?意越王將聽吾言,用我道,則翟將往,量腹而食,度身而衣,自比於群臣,奚能以封為哉?抑越不聽吾言,不用吾道,而吾往焉,則是我以義糶也。鈞之糶,亦於中國耳,何必於越哉?」

子墨子游,魏越曰:「既得見四方之君子,則將先語?」子墨子曰:「凡入國,必擇務而從事焉。國家昏亂,則語之尚賢、尚同;國家貧,則語之節用、節葬;國家說音湛湎,則語之非樂、非命;國家遙僻無禮,則語之尊天、事鬼;國家務奪侵凌,即語之兼愛、非攻,故曰擇務而從事焉。」

子墨子出曹公子而於宋三年而反,睹子墨子曰:「始吾游於子之門,短褐之衣,藜藿之羹,朝得之,則夕弗得,祭祀鬼神。今而以夫子之教,家厚於始也。有家厚,謹祭祀鬼神。然而人徒多死,六畜不蕃,身湛於病,吾未知夫子之道之可用也。」子墨子曰:「不然!夫鬼神之所欲於人者多,欲人之處高爵祿則以讓賢也,多財則以分貧也。夫鬼神豈唯攫黍拑肺之為欲哉?今子處高爵祿而不以讓賢,一不祥也;多財而不以分貧,二不祥也。今子事鬼神唯祭而已矣,而曰:『病何自至哉?』是猶百門而閉一門焉,曰『盜何從入?』若是而求福於有怪之鬼,豈可哉?」

魯祝以一豚祭,而求百福於鬼神。子墨子聞之曰:「是不可,今施人薄而望人厚,則人唯恐其有賜於己也。今以一豚祭,而求百福於鬼神,唯恐其以牛羊祀也。古者聖王事鬼神,祭而已矣。今以豚祭而求百福,則其富不如其貧也。」

彭輕生子曰:「往者可知,來者不可知。」子墨子曰:「籍設而親在百里之外,則遇難焉,期以一日也,及之則生,不及則死。今有固車良馬於此,又有奴馬四隅之輪於此,使子擇焉,子將何乘?對曰:「乘良馬固車,可以速至。」子墨子曰:「焉在矣來!」

孟山譽王子閭曰:「昔白公之禍,執王子閭斧鉞鉤要,直兵當心,謂之曰:『為王則生,不為王則死。』王子閭曰:『何其侮我也!殺我親而喜我以楚國,我得天下而不義,不為也,又況於楚國乎?』遂而不為。王子閭豈不仁哉?」子墨子曰:「難則難矣,然而未仁也。若以王為無道,則何故不受而治也?若以白公為不義,何故不受王,誅白公然而反王?故曰難則難矣,然而未仁也。」

子墨子使勝綽事項子牛。項子牛三侵魯地,而勝綽三從。子墨子聞之,使高孫子請而退之曰:「我使綽也,將以濟驕而正嬖也。今綽也祿厚而譎夫子,夫子三侵魯,而綽三從,是鼓鞭於馬靳也。翟聞之:『言義而弗行,是犯明也。』綽非弗之知也,祿勝義也。」

昔者楚人與越人舟戰於江,楚人順流而進,迎流而退,見利而進,見不利則其退難。越人迎流而進,順流而退,見利而進,見不利則其退速,越人因此若埶,亟敗楚人。公輸子自魯南游楚,焉始為舟戰之器,作為鉤強之備,退者鉤之,進者強之,量其鉤強之長,而制為之兵,楚之兵節,越之兵不節,楚人因此若埶,亟敗越人。公輸子善其巧,以語子墨子曰:「我舟戰有鉤強,不知子之義亦有鉤強乎?」子墨子曰:「我義之鉤強,賢於子舟戰之鉤強。我鉤強,我鉤之以愛,揣之以恭。弗鉤以愛,則不親;弗揣以恭,則速狎;狎而不親則速離。故交相愛,交相恭,猶若相利也。今子鉤而止人,人亦鉤而止子,子強而距人,人亦強而距子,交相鉤,交相強,猶若相害也。故我義之鉤強,賢子舟戰之鉤強。」

公輸子削竹木以為鵲,成而飛之,三日不下,公輸子自以為至巧。子墨子謂公輸子曰:「子之為鵲也,不如匠之為車轄。須臾劉三寸之木,而任五十石之重。故所為功,利於人謂之巧,不利於人謂之拙。」

公輸子謂子墨子曰:「吾未得見之時,我欲得宋,自我得見之後,予我宋而不義,我不為。」子墨子曰:「翟之未得見之時也,子欲得宋,自翟得見子之後,予子宋而不義,子弗為,是我予子宋也。子務為義,翟又將予子天下。」


\end{pinyinscope}