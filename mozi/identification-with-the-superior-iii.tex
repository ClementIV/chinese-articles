\article{尚同下}

\begin{pinyinscope}
子墨子言曰:「知者之事,必計國家百姓所以治者而為之,必計國家百姓之所以亂者而辟之。然計國家百姓之所以治者何也?上之為政,得下之情則治,不得下之情則亂。何以知其然也?上之為政,得下之情,則是明於民之1善非也。若苟2明於民之善非也,則得善人而賞之,得暴人而罰之也。善人賞而暴人罰,則國必治。上之為政也,不得下之情,則是不明於民之善非也。若苟不明於民之善非,則是不得善人而賞之,不得暴人而罰之。善人不賞而暴人不罰,為政若此,國眾必亂。故賞不得下之情,而不可不察者也。」

然計得下之情將柰何可?故子墨子曰:「唯能以尚同一義為政,然後可矣。何以知尚同一義之可而為政於天下也?然胡不審稽古之治為政之說乎。古者,天之始生民,未有正長也,百姓為人。若苟百姓為人,是一人一義,十人十義,百人百義,千人千義,逮至人之眾不可勝計也,則其所謂義者,亦不可勝計。此皆是其義,而非人之義,是以厚者有鬥,而薄者有爭。是故天下之欲同一天下之義也,是故選擇賢者,立為天子。天子以其知力為未足獨治天下,是以選擇其次立為三公。三公又以其知力為未足獨左右天子也,是以分國建諸侯。諸侯又以其知力為未足獨治其四境之內也,是以選擇其次立為卿之宰。卿之宰又以其知力為未足獨左右其君也,是以選擇其次立而為鄉長家君。是故古者天子之立三公、諸侯、卿之宰、鄉長家君,非特富貴游佚而擇之也,將使助治亂刑政也。故古者建國設都,乃立后王君公,奉以卿士師長,此非欲用說也,唯辯而使助治天助1明也。

今此何為人上而不能治其下,為人下而不能事其上,則是上下相賊也,何故以然?則義不同也。若苟義不同者有黨,上以若人為善,將賞之,若人唯使得上之賞,而辟百姓之毀,是以為善者,必未可使勸,見有賞也。上以若人為暴,將罰之,若人唯使得上之罰,而懷百姓之譽,是以為暴者,必未可使沮,見有罰也。故計上之賞譽,不足以勸善,計其毀罰,不足以沮暴。此何故以然?則義不同也。1」

然1則欲同一天下之義,將柰何可?故子墨子言曰:「然胡不賞使家君試用家君,發憲布令其家,曰:『若見愛利家者,必以告,若見惡賊家者,亦必以告。若見愛利家以告,亦猶愛利家者也,上得且賞之,眾聞則譽之,若見惡賊家不以告,亦猶惡賊家者也,上得且罰之,眾聞則非之。』是以遍若家之人,皆欲得其長上之賞譽,辟其毀罰。是以善言之,不善言之,2家君得善人而賞之,得暴人而罰之。善人之賞,而暴人之罰,則家必治矣。然計若家之所以治者何也?唯以尚同一義為政故也。

家既已治,國之道盡此已邪?則未也。國之為家數也甚多,此皆是其家,而非人之家,是以厚者有亂,而薄者有爭,故又使家君總其家之義1,以尚同於國君。國君亦為發憲布令於國之眾,曰:『若見愛利國者,必以告,若見惡賊國者,亦必以告。若見愛利國以告者,亦猶愛利國者也,上得且賞之,眾聞則譽之,若見惡賊國不以告者,亦猶惡賊國者也,上得且罰之,眾聞則非之。』是以遍若國之人,皆欲得其長上之賞譽,避其毀罰。是以民見善者言之,見不善者言之,國君得善人而賞之,得暴人而罰之。善人賞而暴人罰,則國必治矣。然計若國之所以治者何也?唯能以尚同一義為政故也。

國既已治矣,天下之道盡此已邪?則未也。天下之為國數也甚多,此皆是其國,而非人之國,是以厚者有戰,而薄者有爭。故又使國君選其國之義,以尚同於天子。天子亦為發憲布令於天下之眾,曰:『若見愛利天下者,必以告,若見惡賊天下者,亦以告。若見愛利天下以告者,亦猶愛利天下者也,上得則賞之,眾聞則譽之。若見惡賊天下不以告者,亦猶惡賊天下者也,上得且罰之,眾聞則非之。』是以遍天下之人,皆欲得其長上之賞譽,避其毀罰,是以見善不善者告之。天子得善人而賞之,得暴人而罰之,善人賞而暴人罰之1,天下必治矣。然計天下之所以治者何也?唯而以尚同一義為政故也。

天下既已治,天子又總天下之義,以尚同於天。故當尚同之為說也,尚用之天子,可以治天下矣;中用之諸侯,可而治其國矣;小用之家君,可用而治其家矣。是故大用之,治天下不窕,小用之,治一國一家而不橫者,若道之謂也。」

故曰治天下之國若治一家,使天下之民若使一夫。意獨子墨子有此,而先王無此其有邪?則亦然也。聖王皆以尚同為政,故天下治。何以知其然也?於先王之書也大誓之言然,曰:「小人見姦巧乃聞,不言也,發罪鈞。」此言見淫辟不以告者,其罪亦猶淫辟者也。

故古之聖王治天下也,其所差論,以自左右羽翼者皆良,外為之人,助之視聽者眾。故與人謀事,先人得之;與人舉事,先人成之;光1譽令聞,先人發之。唯信身而從事,故利若此。古者有語焉,曰:「一目之2視也,不若二目之視也。一耳之聽也,不若二耳之聽也。一手之操也,不若二手之3彊也。」夫唯能信身而從事,故利若此。是故古之聖王之治天下也,千里之外有賢人焉,其鄉里之人皆未之均聞見也,聖王得而賞之。千里之內有暴人焉,其鄉里未之均聞4見也,聖王得而罰之。故唯毋以聖王為聰耳明目與?豈能一視而通見千里之外哉!一聽而通聞千里之外哉!聖王不往而視也,不就而聽也。然而使天下之為寇亂盜賊者,周流天下無所重足者,何也?其以尚同為政善也。

是故子墨子曰:「凡使民尚同者,愛民不疾,民無可使,曰必疾愛而使之,致信而持之,富貴以道其前,明罰以率其後。為政若此,唯欲毋與我同,將不可得也。」

是以子墨子曰:「今天下王公大人士君子,中情將欲為仁義,求為上1士,上欲中聖王2之道,下欲中國家百姓之利,故當尚同之說,而不可不3察尚同為政之本,而治要也。」


\end{pinyinscope}