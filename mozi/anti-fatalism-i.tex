\article{非命上}

\begin{pinyinscope}
子墨子言曰:「古者王公大人,為政國家者,皆欲國家之富,人民之眾,刑政之治。然而不得富而得貧,不得眾而得寡,不得治而得亂,則是本失其所欲,得其所惡,是故何也?」子墨子言曰:「執有命者以集於民閒者眾。執有命者之言曰:『命富則富,命貧則貧;命眾則眾,命寡則寡;命治則治,命亂則亂;命壽則壽,命夭則夭;命雖強勁,何益哉?』以上說王公大人,下以駔百姓之從事,故執有命者不仁。故當執有命者之言,不可不明辨。」

然則明辨此之說將柰何哉?子墨子言曰:「必立儀,言而毋儀,譬猶運鈞之上而立朝夕者也,是非利害之辨,不可得而明知也。故言必有三表。」何謂三表?子墨子言曰:「有本之者,有原之者,有用之者。於何本之?上本之於古者聖王之事。於何原之?下原察百姓耳目之實。於何用之?廢以為刑政,觀其中國家百姓人民之利。此所謂言有三表也。

然而今天下之士君子,或以命為有。蓋嘗尚觀於聖王之事,古者桀之所亂,湯受而治之;紂之所亂,武王受而治之。此世未易民未渝,在於桀紂,則天下亂;在於湯武,則天下治,豈可謂有命哉!

然而今天下之士君子,或以命為有。蓋嘗尚觀於先王之書,先王之書,所以1出國家,布施百姓者2,憲也。先王之憲,亦嘗有曰『福不可請,而禍不可諱,敬無益,暴無傷』者乎?所以聽獄制罪者,刑也。先王之刑亦嘗有曰『福不可請,禍不可諱,敬無益,暴無傷』者乎?所以整設師旅,進退師徒者,誓也。先王之誓亦嘗有曰:『福不可請,禍不可諱,敬無益,暴無傷』者乎?」是故子墨子言曰:「吾當未鹽數,天下之良書不可盡計數,大方論數,而五者是也。今雖毋求執有命者之言,不必得,不亦可錯乎?今用執有命者之言,是覆天下之義,覆天下之義者,是立命者也,百姓之誶也。說百姓之誶者,是滅天下之人也」。然則所為欲義在上者,何也?曰:「義人在上,天下必治,上帝山川鬼神,必有幹主,萬民被其大利。」何以知之?子墨子曰:「古者湯封於亳,絕長繼短,方地百里,與其百姓兼相愛,交相利,移則分。率其百姓,以上尊天事鬼,是以天鬼富之,諸侯與之,百姓親之,賢士歸之,未歿其世,而王天下,政諸侯。昔者文王封於岐周,「絕長繼短,方地百里,與其百姓兼相愛、交相利,則,是以近者安其政,遠者歸其德。聞文王者,皆起而趨之。罷不肖股肱不利者,處而願之曰:『柰何乎使文王之地及我,吾則吾利,豈不亦猶文王之民也哉。』是以天鬼富之,諸侯與之,百姓親之,賢士歸之,未歿其世,而王天下,政諸侯。鄉者言曰:義人在上,天下必治,上帝山川鬼神,必有幹主,萬民被其大利。吾用此知之。

是故古之聖王發憲出令,設以為賞罰以勸賢,是以入則孝慈於親戚,出則弟長於鄉里,坐處有度,出入有節,男女有辨。是故使治官府,則不盜竊,守城則不崩叛,君有難則死,出亡則送。此上之所賞,而百姓之所譽也。執有命者之言曰:『上之所賞,命固且賞,非賢故賞也。上之所罰,命固且罰,不暴故罰也。』是故入則不慈孝於親戚,出則不弟長於鄉里,坐處不度,出入無節,男女無辨。是故治官府則盜竊,守城則崩叛,君有難則不死,出亡則不送。此上之所罰,百姓之所非毀也。執有命者言曰:『上之所罰,命固且罰,不暴故罰也。上之所賞,命固且賞,非賢故賞也。』以此為君則不義,為臣則不忠,為父則不慈,為子則不孝,為兄則不良,為弟則不弟,而強執此者,此特凶言之所自生,而暴人之道也1。

然則何以知命之為暴人之道?昔上世之窮民,貪於飲食,惰於從事,是以衣食1之財不足,而飢寒凍餒之憂至,不知曰『我罷不肖,從事不疾』,必曰『我命固且貧』。昔上世暴王不忍其耳目之淫,心涂之辟,不順其親戚,遂以亡失國家,傾覆社稷,不知曰『我罷不肖,為政不善』,必曰『吾命固失之。』於仲虺之告曰:『我聞于夏人,矯天命布命于下,帝伐之惡,龔喪厥師。』此言湯之所以非桀之執有命也。於太誓曰:『紂夷處,不用事上帝鬼神,禍厥先神禔不祀,乃曰吾民有命,無廖排漏,天亦縱棄之而弗葆。』此言武王所以非紂執有命也。今用執有命者之言,則上不聽治,下不從事。上不聽治,則刑政亂;下不從事,則財用不足,上無以供粢盛酒醴,祭祀上帝鬼神,下無以2降綏天下賢可之士,外無以應待諸侯之賓客,內無以食飢衣寒,將養老弱。故命上不利於天,中不利於鬼,下不利於人,而強執此者,此特凶言之所自生,而暴人之道也。」

是故子墨子言曰:「今天下之士君子,忠實欲天下之富而惡其貧,欲天下之治而惡其亂,執有命者之言,不可不非,此天下之大害也。」


\end{pinyinscope}