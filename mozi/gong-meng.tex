\article{公孟}

\begin{pinyinscope}
公孟子謂子墨子曰:「君子共己以待,問焉則言,不問焉則止。譬若鍾然,扣則鳴,不扣則不鳴。」子墨子曰:「是言有三物焉,子乃今知其一身也,又未知其所謂也。若大人行淫暴於國家,進而諫,則謂之不遜,因左右而獻諫,則謂之言議。此君子之所疑惑也。若大人為政,將因於國家之難,譬若機之將發也然,君子之必以諫,然而大人之利,若此者,雖不扣必鳴者也。若大人舉不義之異行,雖得大巧之經,可行於軍旅之事,欲攻伐無罪之國,有之也,君得之,則必用之矣。以廣辟土地,著稅偽材,出必見辱,所攻者不利,而攻者亦不利,是兩不利也。若此者,雖不扣必鳴者也。且子曰:『君子共己待,問焉則言,不問焉則止,譬若鍾然,扣則鳴,不扣則不鳴。』今未有扣,子而言,是子之謂不扣而鳴邪?是子之所謂非君子邪?」

公孟子謂子墨子曰:「實為善,人孰不知?譬若良玉,處而不出有餘糈。譬若美女,處而不出,人爭求之。行而自衒,人莫之取也。今子遍從人而說之,何其勞也?」子墨子曰:「今夫世亂,求美女者眾,美女雖不出,人多求之;今求善者寡,不強說人,人莫之知也。且有二生,於此善筮。一行為人筮者,一處而不出者。行為人筮者與處而不出者,其糈孰多?」公孟子曰:「行為人筮者其糈多。」子墨子曰:「仁義鈞。行說人者,其功善亦多,何故不行說人也!」

公孟子戴章甫,搢忽,儒服,而以見子墨子曰:「君子服然後行乎?其行然後服乎?」子墨子曰:「行不在服。」公孟子曰:「何以知其然也?」子墨子曰:「昔者,齊桓公高冠博帶,金劍木盾,以治其國,其國治。昔者,晉文公大布之衣,牂羊之裘,韋以帶劍,以治其國,其國治。昔者,楚莊王鮮冠組纓,縫衣博袍,以治其國,其國治。昔者,越王句踐剪髮文身,以治其國,其國治。此四君者,其服不同,其行猶一也。翟以是知行之不在服也。」公孟子曰:「善!吾聞之曰『宿善者不祥』,請舍忽,易章甫,復見夫子可乎?」子墨子曰:「請因以相見也。若必將舍忽、易章甫,而後相見,然則行果在服也。」

公孟子曰:「君子必古言服,然後仁。」子墨子曰:「昔者,商王紂,卿士費仲,為天下之暴人,箕子、微子為天下之聖人,此同言而或仁不仁也。周公旦為天下之聖人,關叔為天下之暴人,此同服或仁或不仁。然則不在古服與古言矣。且子法周而未法夏也,子之古非古也。」

公孟子謂子墨子曰:「昔者聖王之列也,上聖立為天子,其次立為卿、大夫,今孔子博於詩、書,察於禮樂,詳於萬物,若使孔子當聖王,則豈不以孔子為天子哉?」子墨子曰:「夫知者,必尊天事鬼,愛人節用,合焉為知矣。今子曰:『孔子博於詩書,察於禮樂,詳於萬物』,而曰可以為天子,是數人之齒,而以為富。」

公孟子曰:「貧富壽夭,齰然在天,不可損益。」又曰:「君子必學。」子墨子曰:「教人學而執有命,是猶命人葆而去亓冠也。」

公孟子謂子墨子曰:「有義不義,無祥不祥。」子墨子曰:「古聖王皆以鬼神為神明,而為禍福,執有祥不祥,是以政治而國安也。自桀紂以下,皆以鬼神為不神明,不能為禍福,執無祥不祥,是以政亂而國危也。故先王之書,子亦有之曰:『亓傲也,出於子,不祥。』此言為不善之有罰,為善之有賞。」

子墨子謂公孟子曰:「喪禮,君與父母、妻、後子死,三年喪服,伯父、叔父、兄弟期,族人五月,姑、姊、舅、甥皆有數月之喪。或以不喪之閒,誦詩三百,弦詩三百,歌詩三百,舞詩三百。若用子之言,則君子何日以聽治?庶人何日以從事?」公孟子曰:「國亂則治之,國治則為禮樂。國治則從事,國富則為禮樂。子墨子曰:「國之治。治之廢,則國之治亦廢。國之富也,從事,故富也。從事廢,則國之富亦廢。故雖治國,勸之無饜,然後可也。今子曰:『國治,則為禮樂,亂則治之』,是譬猶噎而穿井也,死而求醫也。古者三代暴王桀紂幽厲,薾為聲樂,不顧其民,是以身為刑僇,國為戾虛者,皆從此道也。」

公孟子曰:「無鬼神。」又曰:「君子必學祭祀。」子墨子曰:「執無鬼而學祭禮,是猶無客而學客禮也,是猶無魚而為魚𦊟也。」

公孟子謂子墨子曰:「子以三年之喪為非,子之三日之喪亦非也。」子墨子曰:「子以三年之喪非三日之喪,是猶裸謂撅者不恭也。」

公孟子謂子墨子曰:「知有賢於人,則可謂知乎?」子墨子曰:「愚之知有以賢於人,而愚豈可謂知矣哉?」

公孟子曰:「三年之喪,學吾之慕父母。」子墨子曰:「夫嬰兒子之知,獨慕父母而已。父母不可得也,然號而不止,此亓故何也?即愚之至也。然則儒者之知,豈有以賢於嬰兒子哉?」

子墨子曰問於儒者:「何故為樂?」曰:「樂以為樂也。」子墨子曰:「子未我應也。今我問曰:『何故為室?』曰:『冬避寒焉,夏避暑焉,室以為男女之別也。』則子告我為室之故矣。今我問曰:『何故為樂?』曰:『樂以為樂也。』是猶曰『何故為室』?曰『室以為室也』。」

子墨子謂程子曰:「儒之道足以喪天下者,四政焉。儒以天為不明,以鬼為不神,天鬼不說,此足以喪天下。又厚葬久喪,重為棺槨,多為衣衾,送死若徙,三年哭泣,扶後起,杖後行,耳無聞,目無見,此足以喪天下。又弦歌鼓舞,習為聲樂,此足以喪天下。又以命為有,貧富壽夭,治亂安危有極矣,不可損益也,為上者行之,必不聽治矣;為下者行之,必不從事矣,此足以喪天下。」程子曰:「甚矣!先生之毀儒也。」子墨子曰:「儒固無此若四政者,而我言之,則是毀也。今儒固有此四政者,而我言之,則非毀也,告聞也。」程子無辭而出。子墨子曰:「迷之!」反,後坐,進復曰:「鄉者先生之言有可聞者焉,若先生之言,則是不譽禹,不毀桀紂也。」子墨子曰:「不然,夫應孰辭,稱議而為之,敏也。厚攻則厚吾,薄攻則薄吾。應孰辭而稱議,是猶荷轅而擊蛾也。」

子墨子與程子辯,稱於孔子。程子曰:「非儒,何故稱於孔子也?」子墨子曰:「是亦當而不可易者也。今鳥聞熱旱之憂則高,魚聞熱旱之憂則下,當此雖禹湯為之謀,必不能易矣。鳥魚可謂愚矣,禹湯猶云因焉。今翟曾無稱於孔子乎?」

有游於子墨子之門者,身體強良,思慮徇通,欲使隨而學。子墨子曰:「姑學乎,吾將仕子。」勸於善言而學。其年,而責仕於子墨子。子墨子曰:「不仕子,子亦聞夫魯語乎?魯有昆弟五人者,亓父死,亓長子嗜酒而不葬,亓四弟曰:『子與我葬,當為子沽酒。』勸於善言而葬。已葬,而責酒於其四弟。四弟曰:『吾末予子酒矣,子葬子父,我葬吾父,豈獨吾父哉?子不葬,則人將笑子,故勸子葬也。』今子為義,我亦為義,豈獨我義也哉?子不學,則人將笑子,故勸子於學。」

有游於子墨子之門者,子墨子曰:「盍學乎?」對曰:「吾族人無學者。」子墨子曰:「不然,夫好美者,豈曰吾族人莫之好,故不好哉?夫欲富貴者,豈曰我族人莫之欲,故不欲哉?好美、欲富貴者,不視人猶強為之。夫義,天下之大器也,何以視人必強為之?」

有游於子墨子之門者,謂子墨子曰:「先生以鬼神為明知,能為禍人哉福?為善者富之,為暴者禍之。今吾事先生久矣,而福不至,意者先生之言有不善乎?鬼神不明乎?我何故不得福也?」子墨子曰:「雖子不得福,吾言何遽不善?而鬼神何遽不明?子亦聞乎匿徒之刑之有刑乎?」對曰:「未之得聞也。」子墨子曰:「今有人於此,什子,子能什譽之,而一自譽乎?」對曰:「不能。」「有人於此,百子,子能終身譽亓善,而子無一乎?」對曰:「不能。」子墨子曰:「匿一人者猶有罪,今子所匿者若此亓多,將有厚罪者也,何福之求?」

子墨子有疾,跌鼻進而問曰:先生以鬼神為明,能為禍福,為善者賞之,為不善者罰之。今先生聖人也,何故有疾?意者先生之言有不善乎?鬼神不明知乎?」子墨子曰:「雖使我有病,何遽不明?人之所得於病者多方,有得之寒暑,有得之勞苦,百門而閉一門焉,則盜何遽無從入?」

二三子有復於子墨子學射者,子墨子曰:「不可,夫知者必量亓力所能至而從事焉,國士戰且扶人,猶不可及也。今子非國士也,豈能成學又成射哉?」

二三子復於子墨子曰:「告子曰:『言義而行甚惡。』請棄之。」子墨子曰:「不可,稱我言以毀我行,愈於亡。有人於此,翟甚不仁,尊天、事鬼、愛人,甚不仁,猶愈於亡也。今告子言談甚辯,言仁義而不吾毀,告子毀,猶愈亡也。」

二三子復於子墨子曰:「告子勝為仁。」子墨子曰:「未必然也!告子為仁,譬猶跂以為長,隱以為廣,不可久也。」

告子謂子墨子曰:「我治國為政。」子墨子曰:「政者,口言之,身必行之。今子口言之,而身不行,是子之身亂也。子不能治子之身,惡能治國政?子姑亡,子之身亂之矣!」


\end{pinyinscope}