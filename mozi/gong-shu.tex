\article{公輸}

\begin{pinyinscope}
公輸盤為楚造雲梯之械,成,將以攻宋。子墨子聞之,起於齊,行十日十夜而至於郢,見公輸盤。公輸盤曰:「夫子何命焉為?」子墨子曰:「北方有侮臣,願藉子殺之。」公輸盤不說。子墨子曰:「請獻十金。」公輸盤曰:「吾義固不殺人。」子墨子起,再拜曰:「請說之。吾從北方,聞子為梯,將以攻宋。宋何罪之有?荊國有餘於地,而不足於民,殺所不足,而爭所有餘,不可謂智。宋無罪而攻之,不可謂仁。知而不爭,不可謂忠。爭而不得,不可謂強。義不殺少而殺眾,不可謂知類。」公輸盤服。子墨子曰:「然,乎不已乎?」公輸盤曰:「不可。吾既已言之王矣。」子墨子曰:「胡不見我於王?」公輸盤曰:「諾」。

子墨子見王,曰:「今有人於此,舍其文軒,鄰有敝轝,而欲竊之;舍其錦繡,鄰有短褐,而欲竊之;舍其粱肉,鄰有糠糟,而欲竊之。此為何若人?」王曰:「必為竊疾矣。」子墨子曰:「荊之地,方五千里,宋之地,方五百里,此猶文軒之與敝轝也;荊有雲夢,犀兕麋鹿滿之,江漢之魚鱉黿鼉為天下富,宋所為無雉兔狐貍者也,此猶粱肉之與糠糟也;荊有長松、文梓、楩柟、豫章,宋無長木,此猶錦繡之與短褐也。臣以三事之攻宋也,為與此同類,臣見大王之必傷義而不得。」王曰:「善哉!雖然,公輸盤為我為雲梯,必取宋。」

於是見公輸盤,子墨子解帶為城,以牒為械,公輸盤九設攻城之機變,子墨子九距之,公輸盤之攻械盡,子墨子之守圉有餘。公輸盤詘,而曰:「吾知所以距子矣,吾不言。」子墨子亦曰:「吾知子之所以距我,吾不言。」楚王問其故,子墨子曰:「公輸子之意,不過欲殺臣。殺臣,宋莫能守,可攻也。然臣之弟子禽滑釐等三百人,已持臣守圉之器,在宋城上而待楚寇矣。雖殺臣,不能絕也。」楚王曰:「善哉!吾請無攻宋矣。」

子墨子歸,過宋,天雨,庇其閭中,守閭者不內也。故曰:「治於神者,眾人不知其功,爭於明者,眾人知之。」


\end{pinyinscope}