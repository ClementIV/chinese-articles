\article{節用上}

\begin{pinyinscope}
聖人為政一國,一國可倍也;大之為政天下,天下可倍也。其倍之非外取地也,因其國家,去其無用之費1,足以倍之。聖王為政,其發令興事,使民用財也,無不加用而為者,是故用財不費,民德不勞,其興利多矣。其為衣裘何?以為冬以圉寒,夏以圉暑。凡為衣裳之道,冬加溫,夏加凊者,芊䱉不加者去之。其為宮室何?以為冬以圉風寒,夏以圉暑雨,有盜賊加固者,芊䱉不加者去之。其為甲盾五兵何?以為以圉寇亂盜賊,若有寇亂盜賊,有甲盾五兵者勝,無者不勝。是故聖人作為甲盾五兵。凡為甲盾五兵加輕以利,堅而難折者,芊䱉不加者去之。其為舟車何?以為車以行陵陸,舟以行川谷,以通四方之利。凡為舟車之道,加輕以利者,芊䱉不加者去之。凡其為此物也,無不2加用而為者,是故用財不費,民德不勞,其興利多矣3。

有去大人之好聚珠玉、鳥獸、犬馬,以益衣裳、宮室、甲盾、五兵、舟車之數於數倍乎!若則不難,故孰為難倍?唯人為難倍。然人有可倍也。昔者1聖王為法曰:「丈夫年二十,毋敢不處家。女子年十五,毋敢不事人。」此聖王之法也。聖王即沒,于民次也,其欲蚤處家者,有所二十年處家;其欲晚處家者,有所四十年處家。以其蚤與其晚相踐,後聖王之法十年。若純三年而字,子生可以二三年矣。此不惟使民蚤處家而可以倍與?且不然已。

今天下為政者,其所以寡人之道多,其使民勞,其籍歛厚,民財不足,凍餓死者不可勝數也。且大人惟毋興師以攻伐鄰國,久者終年,速者數月,男女久不相見,此所以寡人之道也。與居處不安,飲食不時,作疾病死者,有與侵就伏橐,攻城野戰死者,不可勝數。此不令為政者,所以寡人之道數術而起與?聖人為政特無此,不聖人為政,其所以眾人之道亦數術而起與?」故子墨子曰:「去無用之費,1聖王之道,天下之大利也。」


\end{pinyinscope}