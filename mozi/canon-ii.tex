\article{經下}

\begin{pinyinscope}
止,類以行之1。說在同。屬於:[邏輯]

止:彼以此其然也,說是其然也;我以此其不然也,疑是其然也。屬於:[邏輯] 
1. 之 : 原作「人」。

推類之難,說在之大小。屬於:[邏輯]

謂四足獸,與牛馬1與,物盡異2,大小也。此然是必然,則俱。屬於:[邏輯] 
1. 牛馬 : 原作「生鳥」。2. 異 : 原作「與」。

物盡同名:二與鬥,愛,食與招,白與視,麗與,夫與履。

馬1麋同名俱鬥、不俱二,二與鬥也。包肝肺,子愛也。橘、茅,食與抬也。白馬多白,視馬不多視,白與視也。為麗不必麗,不必麗與暴也,為非以人,是不為非。若為夫勇,不為夫。為屨以買衣為屨,夫與屨也。1. 馬 : 原作「為」。

一偏棄之,謂而固是也,說在因。

二與一亡,不與一在,偏去未。有之1實也,而後謂之;無之2實也,則無謂也。不若敷與美,謂是則是固美也,謂也則是非美。無謂則報也。1. 之 : 原作「文」。2. 之 : 原作「文」。

不可偏去而二,說在見與俱、一與二、廣與修1。

見不見離,一二不相盈,廣修堅白。1. 修 : 原作「循」。自孫詒讓《墨子閒詁》改。

不能而不害。說在害。

舉不重不與箴,非力之任也;為握者之觭倍,非智之任也。若耳目。

異類不吡,說在量。屬於:[邏輯]

異:木與夜孰長?智與粟孰多?爵、親、行、賈四者孰貴?麋與霍孰高?麋與霍孰霍?𧈳與瑟孰瑟?屬於:[邏輯]

偏去莫加少,說在故。

偏:俱一無變。屬於:[邏輯]

假必誖,說在不然。屬於:[邏輯]

假:假必非也而後假。狗,假霍也,猶氏霍也。屬於:[邏輯]

物之所以然,與所以知之,與所以使人知之,不必同。說在病。屬於:[知識論]

物:或傷之,然也;見之,智也。告1之,使智也。屬於:[知識論] 
1. 告 : 原作「吉」。

疑,說在逢、循、遇、過。

疑:逢為務則士,為牛廬者夏寒,逢也。舉之則輕,廢之則重,非有力也。沛從削,非巧也,若石羽,循也。鬥者之敝也,以飲酒,若以日中,是不可智也,愚也。智與?以已為然也與?愚也。

合,與一,或復否,說在拒。



歐物一體也,說在俱一、惟是。

俱:俱一,若牛馬四足;惟是,當牛馬。數牛數馬則牛馬二;數牛馬則牛馬一。若數指,指五而五一。

宇:或徙,說在長宇久。

長宇:徙而有處,宇。宇南北,在且有在莫,宇徙久。

不堅白,說在無久與宇。1

1. 無久與宇。 : 從第116條移到此處。  高亨《墨經校詮》

無久與宇。1堅白,說在因。

無堅得白,必相盈也。1. 無久與宇。 : 移到第115條。  高亨《墨經校詮》

在諸其所然未者然,說在於是推之。

在:堯善治,自今在諸古也。自古在之今,則堯不能治也。

景不徙,說在改為。

景:光至景亡,若在盡可1息。1. 可 : 原作「古」。

景二,說在重。

景:二光夾一光,一光者景也。

景到,在午有端與景長,說在端。

景:光之人煦若射。下者之人也高,高者之人也下。足敝下光,故成景於上。首敝上光,故成景於下。在遠近有端,與於光,故景㢓內也。

景迎日,說在慱。

景日之光反燭人,則景在日與人之間。

景之小大,說在地正1遠近。

景:木柂,景短大。木正,景長小。火2小於木,則景大於木。非獨小也,遠近。1. 正 : 原作「缶」。2. 火 : 原作「大」。

臨鑒而立,景到。多而若少,說在寡區。

臨:正鑒景寡。貌能、白黑、遠近、柂正、異於光。鑒景當俱,就,去亦1當俱,俱用北。鑒者之臭於鑒,無所不鑒。景之臭無數而必過,正故同處,其體俱然,鑒分。1. 亦 : 原作「尒」。自孫詒讓《墨子閒詁》改。

鑒位,景一小而易,一大而正1,說在中之外內。

鑒:中之內,鑒者近中,則所鑒大,景亦大;遠中,則所鑒小,景亦小,而必正。起於中緣正而長其直也。中之外,鑒者近中,則所鑒大,景亦大;遠中,則所鑒小,景亦小,而必易。合於而長其直也。1. 正 : 原作「缶」。

鑑團,景一天,而必正1,說在得。

鑒:鑒者近則所鑒大,景亦大;亣2遠,所鑒小,景亦小,而必正。景過正故招。1. 正 : 原作「缶」。2. 亣 : 原作「亦」。

負1而不撓,說在勝。

負:衡木加2重焉而不撓,極勝重也。右校交繩,無加焉而撓,極不勝重也。衡加重於其一旁必捶,權重相若也。相衡則本短標長,兩加焉重相若,則標必下,標得權也。1. 負 : 原作「貞」。自孫詒讓《墨子閒詁》改。2. 加 : 原作「如」。孫詒讓《墨子閒詁》

契與枝板,說在薄。

挈:有力也,引無力也。不正所挈之止於施也,繩制挈之也,若以錐刺之。挈,長重者下,短輕者上,上者愈得,下下者愈亡。繩直權重相若,則正矣。收,上者愈喪,下者愈得,上者權中盡,則遂。

挈:兩輪高,兩輪為輲,車梯也。重其前,弦其前,載弦其前,載弦其軲,而縣重於其前。是梯挈且挈則行。凡重,上弗挈,下弗收,旁弗劫,則下直杝,或害之也流。梯者不得流直也。今也廢尺於平地,重不下,無旁也。若夫繩之引軲也,是猶自舟中引橫也。

倚者不可正,說在剃。

倚:倍、拒、堅、䠳,倚焉則不正。

推之必往,說在廢材。

誰:𥩵石、壘石耳。夾𡨦者法也。方石去地尺,關石於其下,縣絲於其上,使適至方石。不下,柱也。膠絲去石,挈也;絲絕,引也。木變而名易,收也。

買無貴,說在仮其賈。

買:刀、糴相為賈。刀輕則糴不貴,刀重則糴不易。王刀無變,糴有變。歲變糴,則歲變刀。若鬻子。

賈宜則讐,說在盡。

賈:盡也者,盡去其以不讎也。其所以不讎去,則讎正1。賈也宜不宜正2欲不欲,若敗邦鬻室嫁子。1. 正 : 原作「缶」。2. 正 : 原作「缶」。

無說而懼,說在弗心。

無:子在軍,不必其死生;聞戰,亦不必其生。前也不懼,今也懼。

或,過名也,說在實。

或:知是之非此也,有知是之不在此也,然而謂此南北,過而以已為然。始也謂此南方,故今也謂此南方。

知知之否之足用也誖,說在無以也。

智:論之非智無以也。

謂辯無勝,必不當。說在辯。屬於:[邏輯]

謂:「所謂非同也,則異也。同則或謂之狗,其或謂之犬也;異則或謂之牛,牛或謂之馬也。俱無勝。」是不辯也。辯也者,或謂之是,或謂之非,當者勝也。屬於:[邏輯]

無不讓也,不可。說在始。

無:讓者酒,未讓始也。不可讓也。

於一,有知焉,有不知焉,說在存。

於石一也,堅白二也,而在石。故有智焉,有不智焉,可。

有指於二,而不可逃,說在以二絫。

有指:子智是,有智是吾所先舉,重則。子智是,而不智吾所先舉也,是一。謂「有智焉,有不智焉」也。若智之,則當指之智告我,則我智之,兼指之以二也。衡指之,參直之也。若曰,「必獨指吾所舉,毋舉吾所不舉」,則者固不能獨指。所欲相不傳,意若未校。且其所智是也,所不智是也,則是智是之不智也,惡得為一?謂而「有智焉,有不智焉」。

所知而弗能指,說在春也、逃臣、狗犬、貴者。

所:春也,其執固不可指也。逃臣,不智其處。狗犬,不智其名也。遺者,巧弗能兩也。

知狗而自謂不知犬,過也,說在重。屬於:[邏輯]

智:智狗,重智犬,則過;不重,則不過。屬於:[邏輯]

通意後對,說在不知其誰謂也。

通:問者曰,「子智𩥡乎?」應之曰,「𩥡何謂也?」彼曰,「𩥡施。」則智之。若不問𩥡何謂,徑應以弗智,則過。且應必應問之時。若應長,應有深淺大常中在兵人長。

所存與者,於存與孰存,駟異說。

所:室堂,所存也。其子,存者也。據在者而問室堂,惡可存也?主室堂而問存者,孰存也?是一主存者以問所存,一主所存以問存者。

五行毋常勝,說在宜。

五:合水土火火。離。然火鑠金,火多也。金靡炭,金多也。合之府水,木離木若識麋輿魚之數,惟所利。

無欲惡之為益損也,說在宜。

無:欲惡傷生損壽,說以少連。是誰愛也,嘗多粟。或者欲不有能傷也,若酒之於人也。且𢜔人利人,愛也。則唯𢜔弗治也。

損而不害,說在餘。

損:飽者去餘,適足不害。能害飽,若傷麋之無脾也。且有損而后益智者,若虐病之之於虐也。

知而不以五路,說在久。屬於:[知識論]

智:以目見。而目以火見,而火不見。惟以五路智,久不當,以目見若以火見。屬於:[知識論]

必熱,說在頓。

火:謂火熱也,非以火之熱我有,若視日。

知其所以、不知,說在以名、取。

智:雜所智與所不智而問之,則必曰:「是所智也,是所不智也。」取、去俱能之,是兩智之也。

無不必待有,說在所謂。

無:若無焉,則有之而后無;無天陷,則無之而無。

擢慮不疑,說在有無。

擢:疑,無謂也。臧,也今死,而春也得文,文死也可。且猶是也。

且然不可正,而不害用工,說在宜。

且:且必然,且已、必已。且用工而後已者,必用工後已。

均之絕不,說在所均。

均:髮均,縣輕;而髮絕,不均也。均,其絕也莫絕。

堯之義也,生於今而處於古。而異時。說在所義二。

堯:霍,或以名視人,或以實視人。舉友富商也,是以名視人也。指是臛也,是以實視人也。堯之義也,是聲也於今,所義之實處於古。若殆於城門與於臧也。

狗,犬也,而殺狗非殺犬也,可。說在重。屬於:[邏輯]

狗:狗,犬也。謂之殺犬,可。若兩脾。屬於:[邏輯]

使:殷、美,說在使。

使:令使也。我使我,我不使,亦使我。殿戈亦使殿,不美,亦使殿。

荊之大,其沈淺也,說在具。

荊:沈,荊之見也。則沈淺非荊淺也,若易五之一。

以檻為摶,於以為,無知也。說在意。

以:楹之摶也,見之,其於意也不易,先智,意相也。若楹輕於秋,其於意也洋然。

意未可知,說在可用,過仵。

段、椎、錐俱事於履,可用也。成繪屢過椎,與成椎過繪屢,同,過仵也。

一少於二而多於五,說在建住。

一:五有一焉,一有五焉。十,二焉。

非半,弗𣃈,則不動。說在端。

非:𣃈半,進前取也,前則中無為半,猶端也。前後取則「端中」也。𣃈必半,「無」與「非半」,不可斫也。

可無也,有之而不可去。說在嘗然。

可無也:已給則當給,不可無也。久有窮無窮。

正1而不可擔,說在摶。

正:九,無所處而不中縣,摶也。1. 正 : 原作「缶」。

宇進無近,說在敷。

傴宇不可偏舉,字也。進行者先敷近,後敷遠。

行循以久,說在先後。

行:者行者必先近而後遠。遠修近修也,先後久也。民行修必以久也。

一法者之相與也盡,若方之相召也。說在方。

一:方貌盡。俱有法而異,或木或石,不害其方之相合也,盡貌,猶方也。物俱然。

狂舉不可以知異,說在有不可。屬於:[邏輯]

牛狂與馬惟異,以牛有齒、馬有尾,說牛之非馬也,不可。是俱有,不偏有、偏無有。曰之與馬不類,用牛角、馬無角,是類不同也。若舉牛有角、馬無角,以是為類之不同也,是狂舉也,猶牛有齒,馬有尾。屬於:[邏輯]

牛馬之非牛,與可之同,說在兼。屬於:[邏輯]

「或不非牛而『非牛也』可,則或非牛或牛而『牛也』可。故曰:『牛馬非牛也』未可,『牛馬牛也』未可。」則或可或不可,而曰「牛馬牛也,未可」亦不可。且牛不二,馬不二,而牛馬二。則牛不非牛,馬不非馬,而牛馬非牛非馬,無難。屬於:[邏輯]

循此循此與彼此同。說在異。屬於:[邏輯]

彼:正名者彼此彼此可。彼彼止於彼,此此止於此,彼此不可。彼且此也,彼此亦可。彼此止於彼此,若是而彼此也,則彼亦且此此也。屬於:[邏輯]

唱和同患,說在功。

唱無過,無所周,若粺。和無過,使也,不得已。唱而不和,是不學也。智少而不學,必寡。和而不唱,是不教也。智而不教,功適息。使人奪人衣,罪或輕或重;使人予人酒,或厚或薄。

聞所不知若所知,則兩知之,說在告。屬於:[知識論]

聞:在外者所不知也。或曰,「在室者之色若是其色」,是所不智若所智也。猶白若黑也,誰勝?是若其色也,若白者必白。今也智其色之若白也,故智其白也。夫名以所明正所不智,不以所不智疑所明。若以尺度所不智長。外,親智也;室中,說智也。屬於:[知識論]

以言為盡誖,誖。說在其言。屬於:[邏輯]

以:誖,不可也。出入之言可,是不誖,則是有可也。之人之言不可,以當必不審。屬於:[邏輯]

惟吾謂非名也,則不可。說在仮。

惟:謂是霍,可。而猶之非夫霍也,謂彼是是也,不可。謂者毋惟乎其謂。彼猶惟乎其謂,則吾謂不行。彼若不惟其謂,則不行也。

無窮不害兼,說在盈否。

無:「南者有窮則可盡,無窮則不可盡。有窮無窮未可智,則可盡不可盡不可盡未可智。人之盈之否未可智,而必人之可盡不可盡亦未可智,而必人之可盡愛也,誖。」人若不盈先窮,則人有窮也,盡有窮無難。盈無窮,則無窮盡也,盡有窮無難。

不知其數而知其盡也,說在明者。

不:「二智其數,惡智愛民之盡文也?或者遺乎?」其問也盡問人,則盡愛其所問。若不智其數而智愛之盡文也,無難。

不知其所處,不害愛之。說在喪子者。



仁義之為內外也,內,說在仵顏。屬於:[倫理]

仁:仁,愛也;義,利也。愛利,此也,所愛所利,彼也。愛利不相為內外,所愛利亦不相為內外。其為仁,內也,義,外也,舉愛與所利也,是狂舉也。若左目出,右目入。屬於:[倫理]

學之益也,說在誹者。

學:也以為不知學之無益也,故告之也。是使智學之無益也,是教也。以學為無益也教,誖。

誹之可否,不以眾寡。說在可非。屬於:[邏輯]

論誹誹之可不可以理,之可誹,雖多誹,其誹是也。其理不可誹,雖少誹,非也。今也謂多誹者不可,是猶以長論短。屬於:[邏輯]

非誹者諄,說在弗非。屬於:[邏輯]

不:誹非,己之誹也。不非誹,非可非也,不可非也。是不非誹也。屬於:[邏輯]

物甚不甚,說在若是。

物:甚長、甚短,莫長於是,莫短於是。是之是也,非是也者,莫甚於是。

取下以求上也,說在澤。

取:高下以善不善為度,不若山澤。處下善於處上,下所請上也。

是是與是同,說在不州。屬於:[邏輯]

不是:是則是且是焉。今是不文於是而不於是,故是不之是。不文則是而不文焉。今是文於是而文與是,故文與是不文同說也。屬於:[邏輯]


\end{pinyinscope}