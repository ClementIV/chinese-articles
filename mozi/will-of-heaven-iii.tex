\article{天志下}

\begin{pinyinscope}
子墨子言曰:「天下之所以亂者,其說將何哉?則是天下士君子,皆明於小而不明於大。何以知其明於小不明於大也?以其不明於天之意也。何以知其不明於天之意也?以處人之家者知之。今人處若家得罪,將猶有異家所,以避逃之者,然且父以戒子,兄以戒弟,曰:『戒之慎之,處人之家,不戒不慎之,而有處人之國者乎?』今人處若國得罪,將猶有異國所,以避逃之者矣,然且父以戒子,兄以戒弟,曰:『戒之慎之,處人之國者,不可不戒慎也!』今人皆處天下而事天,得罪於天,將無所以避逃之者矣。然而莫知以相極戒也,吾以此知大物則不知者也。」

是故子墨子言曰:「戒之慎之,必為天之所欲,而去天之所惡。曰天之所欲者何也?所惡者何也?天欲義而惡其不義者也。何以知其然也?曰義者正也。何以知義之為正也?天下有義則治,無義則亂,我以此知義之為正也。然而正者,無自下正上者,必自上正下。是故庶人不得次己而為正,有士正之;士不得次己而為正,有大夫正之;大夫不得次己而為正,有諸侯正之;諸侯不得次己而為正,有三公正之;三公不得次己而為正,有天子正之;天子不得次己而為政,有天正之。今天下之士君子,皆明於天子之正天下也,而不明於天之1正天子2也。是故古者聖人,明以此說人曰:『天子有善,天能賞之;天子有過,天能罰之。』天子賞罰不當,聽獄不中,天下疾病禍福,霜露不時,天子必且犓豢其牛羊犬彘,絜為粢盛酒醴,以禱祠祈福於天,我未嘗聞天之禱祈福於天子也,吾以此知天之重且貴於天子也。是故義者不自愚且賤者出,必自貴且知者出。曰誰為知?天為知。然則義果自天出也。今天下之士君子之欲為義者,則不可不順天之意矣。

曰順天之意何若?曰兼愛天下之人。何以知兼愛天下之人也?以兼而食之也。何以知其兼而食之也?自古及今無有遠靈孤夷之國,皆犓豢其牛羊犬彘,絜為粢盛酒醴,以敬祭祀上帝山川鬼神,以此知兼而食之也。苟兼而食焉,必兼而愛之。譬之若楚、越之君,今是楚王食於楚之四境之內,故愛楚之人;越王食於越,1故愛越之人。今天兼天下而食焉,我以此知其兼愛天下之人也。

且天之愛百姓也,不盡物而止矣。今天下之國,粒食之民,國1殺一不辜者,必有一2不祥。曰誰殺不辜?曰人也。孰予之不辜?曰天也。若天之中實不愛此民也,何故而人有殺不辜,而天予之不祥哉?且天之愛百姓厚矣,天之愛百姓別矣,既可得而知也。何以知天之愛百姓也?吾以賢者之必賞善罰暴也。何以知賢者之必賞善罰暴也?吾以昔者三代之聖王知之。故昔也三代之聖王堯舜禹湯文武之兼愛天下也,從而利之,移其百姓之意焉,率以敬上帝山川鬼神,天以為從其所愛而愛之,從其所利而利之,於是加其賞焉,使之處上位,立為天子以法也,名之曰『聖人』,以此知3其賞善之證。是故昔也三代之暴王桀紂幽厲之兼惡天下也,從而賊之,移其百姓之意焉,率以詬侮上帝山川鬼神,天以為不從其所愛而惡之,不從其所利而賊之,於是加其罰焉,使之父子離散,國家滅亡,抎失社稷,憂以及其身。是以天下之庶民屬而毀之,業萬世子孫繼嗣,毀之賁不之廢也,名之曰『失王』,以此知其罰暴之證。今天下之士君子,欲為義者,則不可不順天之意矣。

曰順天之意者,兼也;反天之意者,別也。兼之為道也,義正;別之為道也,力正。曰義正者何若?曰大不攻小也,強不侮弱也,眾不賊寡也,詐不欺愚也,貴不傲賤也,富不驕貧也,壯不奪老也。是以天下之庶國,莫以水火毒藥兵刃以相害也。若事上利天,中利鬼,下利人,三利而無所不利,是謂天德。故凡從事此者,聖知也,仁義也,忠惠也,慈孝也,是故聚斂天下之善名而加之。是其故何也?則順天之意也。曰力正者何若?曰大則攻小也,強則侮弱也,眾則賊寡也,詐則欺愚也,貴則傲賤也,富則驕貧也,壯則奪老也。是以天下之庶國,方以水火毒藥兵刃以相賊害也。若事上不利天,中不利鬼,下不利人,三不利而無所利,是謂之賊。故凡從事此者,寇亂也,盜賊也,不仁不義,不忠不惠,不慈不孝,是故聚斂天下之惡名而加之。是其故何也?則反天之意也。」

故子墨子置立天之,以為儀法,若輪人之有規,匠人之有矩也。今輪人以規,匠人以矩,以此知方圜之別矣。是故子墨子置立天之,以為儀法。吾以此知天下之士君子之去義遠也。何以知天下之士君子之去義遠也?今知氏大國之君寬者然曰:「吾處大國而不攻小國,吾何以為大哉!」是以差論蚤牙之士,比列其舟車之卒,以攻罰無罪之國,入其溝境,刈其禾稼,斬其樹木,殘其城郭,以御其溝池,焚燒其祖廟,攘殺其犧牷,民之格者,則剄殺之,不格者,則係操而歸,丈夫以為僕圉胥靡,婦人以為舂酋。則夫好攻伐之君,不知此為不仁義,以告四鄰諸侯曰:「吾攻國覆軍,殺將若干人矣。」其鄰國之君亦不知此為不仁義也,有具其皮幣,發其總處,使人饗賀焉。則夫好攻伐之君,有重不知此為不仁不義也,有書之竹帛,藏之府庫。為人後子者,必且欲順其先君之行,曰:「何不當發吾府1庫,視吾先君之法美。」必不曰文、武之為正為正2者若此矣,曰吾攻國覆軍殺將若干人矣。則夫好攻伐之君,不知此為不仁不義也,其鄰國之君不知此為不仁不義也,是以攻伐世世而不已者,此吾所謂大物則不知也。

所謂小物則知之者何若?今有人於此,入人之場園,取人之桃李瓜薑者,上得且罰之,眾聞則非之,是何也?曰不與其勞,獲其實,已非其有所取之故,而況有踰於人之牆垣,抯格人之子女者乎?與角人之府庫,竊人之金玉蚤絫者乎?與踰人之欄牢,竊人之牛馬者乎?而況有殺一不辜人乎?今王公大人之為政也,自殺一不辜人者;踰人之牆垣,抯格人之子女者;與角人之府庫,竊人之金玉蚤絫者乎1;與踰人之欄牢,竊人之2牛馬者;與入人之場園,竊人之3桃李瓜薑者,今王公大人之加罰此也,雖古之堯舜禹湯文武之為政,亦無以異此矣。今天下之諸侯,將猶皆侵凌攻伐兼并,此為殺一不辜人者,數千萬矣;此為踰人之牆垣,格人之子女者,與角人府庫,竊人金玉蚤絫者,數千萬矣;踰人之欄牢,竊人之牛馬者,與入人之場園,竊人之桃李瓜薑者,數千萬矣,而自曰義也。故子墨子言曰:「是蕡我者,則豈有以異是蕡黑白甘苦之辯者哉!今有人於此,少而示之黑謂之黑,多示之黑謂白,必曰吾目亂,不知黑白之別。今有人於此,能少嘗之甘謂甘,多嘗謂苦,必曰吾口亂,不知其甘苦之味。今王公大人之政也,或殺人,其國家禁之,此蚤越有能多殺其鄰國之人,因以為文義,此豈有異蕡白黑、甘苦之別者哉?」

故子墨子置天之,以為儀法。非獨子墨子以天之志為法也,於先王之書大夏之道之然:「帝謂文王,予懷而明德,毋大聲以色,毋長夏以革,不識不知,順帝之則。」此誥文王之以天志為法也,而順帝之則也。且今天下之士君子,中實將欲為仁義,求為上士,上欲中聖王之道,下欲中國家百姓之利者,當天之志,而不可不察也。天之志者,義之經也。


\end{pinyinscope}