\article{經上}

\begin{pinyinscope}
故,所得而後成也。屬於:[邏輯]

故:小故,有之不必然,無之必不然。體也,若有端。大故,有之必然,1無之必不2然,若見之成見也。屬於:[邏輯] 
1. 然, : 舊脫。  孫詒讓《墨子閒詁》2. 之必不 : 舊脫。  孫詒讓《墨子閒詁》

體,分於兼也。屬於:[幾何學]

體:若二之一,尺之端也。屬於:[幾何學]

知,材也。屬於:[知識論]

知:材,知也者。所以知也而必知。若明。屬於:[知識論]

慮,求也。屬於:[知識論]

慮:慮也者,以其知有求也,而不必得之。若睨。屬於:[知識論]

知,接也。屬於:[知識論]

知:知也者,以其知過物而能貌之。若見。屬於:[知識論]

𢜔,明也。屬於:[知識論]

𢜔1:𢜔2也者,以其知論物而其知之也著。若明。屬於:[知識論] 
1. 𢜔 : 原作「恕」。自孫詒讓《墨子閒詁》改。2. 𢜔 : 原作「恕」。自孫詒讓《墨子閒詁》改。

仁,體愛也。屬於:[倫理]

仁:愛己者,非為用己也。不若愛馬。著若明。1屬於:[倫理] 
1. 著若明。 : 刪除。 衍文。 孫詒讓《墨子閒詁》

義,利也。屬於:[倫理]

義:志以天下為芬,而能能利之,不必用。屬於:[倫理]

禮,敬也。屬於:[倫理]

禮:貴者公,賤者名,而俱有敬僈。焉等,異論也。屬於:[倫理]

行,為也。

行:所為不差1名,行也;所為差2名,巧也。若為盜。1. 差 : 原作「善」。2. 差 : 原作「善」。

實,榮也。

實:其志氣之見也,使人知1己。不若金聲玉服。1. 知 : 原作「如」。

忠,以為利而強君1也。屬於:[倫理]

忠:不利,弱子亥足將入止容。屬於:[倫理] 
1. 君 : 原作「低」。自孫詒讓《墨子閒詁》改。

孝,利親也。屬於:[倫理]

孝:以親為芬,而能能利親。不必得。屬於:[倫理]

信,言合於意也。

信:不以其言之當也,使人視城得金。

佴,自作也。

佴:與人遇,人衆,𢝺。

𧨜,作嗛也。

𧨜:為是為是之台彼也,弗為也。

廉,作非也。

廉:己惟為之,知其也𦖷也。

令,不為所作也。

所令非身弗行。

任,士損己而益所為也。

任:為身之所惡,以成人之所急。

勇,志之所以敢也。

勇:以其敢於是也,命之;不以其不敢於彼也,害之。

力,刑之所以奮也。

力:重之謂,下與重,奮1也。1. 奮 : 原作「舊」。

生,刑與知處也。

生:楹之生。商不可必也。

臥,知無知也。

臥:

夢,臥而以為然也。

夢:

平,知無欲惡也。

平:惔然。

利,所得而喜也。

利:得是而喜,則是利也。其害也,非是也。

害,所得而惡也。

害:得是而惡,則是害也。其利也,非是也。

治,求得也。

治:吾事治矣,人有治南北。

譽,明美也。

譽之必其行也。其言之忻,使人督之。

誹,明惡也。

誹:必其行也,其言之忻。

舉,擬實也。

舉1:告以之2名,舉彼實故也3。1. 舉 : 原作「譽」。2. 之 : 原作「文」。3. 故也 : 原作「也故」。

言,出舉也。

言也者,諸口能之,出民者也。民若畫俿也。言也謂,言猶名1致也。1. 名 : 原作「石」。

且且1,言然也。

且:自前曰且,自後曰已。方然亦且。若石者也。21. 且 : 刪除。 衍文。 2. 若石者也。 : 刪除。 衍文。

君,臣萌通約也。

君:以若名者也。

功,利民也。

功:不待時,若衣裘。功不待時,若衣裘。11. 功不待時,若衣裘。 : 刪除。 衍文。

賞,上報下之功也。

賞:上報下之功也。11. 上報下之功也。 : 從第37條移到此處。

罪,犯禁也。

罪:不在禁,惟害無罪,殆姑。上報下之功也。11. 上報下之功也。 : 移到第36條。

罰,上報下之罪也。

罰:上報下之罪也。

同,異而俱於之一也。

侗:二人而俱見是楹也,若事君。

久,彌異時也。

今久:古今且莫。

宇,彌異所也。

宇:東西家南北。

窮,或有前不容尺也。

窮:或不容尺,有窮;莫不容尺,無窮也。

盡,莫不然也。屬於:[邏輯]

盡:俱1止動。屬於:[邏輯] 
1. 俱 : 原作「但」。

始,當時也。

始:時或有久,或無久,始當無久。

化,徵易也。

化:若鼃為鶉。

損,偏去也。

損:偏也者兼之體1也。其體或去存,謂其存者損。1. 體 : 原作「禮」。

大益。



環1,俱柢2。

環3:俱柢4也。1. 環 : 原作「儇」。自孫詒讓《墨子閒詁》改。2. 俱柢 : 原作「稹秖」。自孫詒讓《墨子閒詁》改。3. 環 : 原作「儇」。自孫詒讓《墨子閒詁》改。4. 俱柢 : 原作「昫民」。自孫詒讓《墨子閒詁》改。

庫,易也。

庫:區穴若斯貌常。

動,或徒1也。

動:偏祭從者,戶樞免瑟。

讀此書旁行。1. 徒 : 原作「從」。自孫詒讓《墨子閒詁》改。

止,以久也。

止:無久之不止,當牛非馬,若矢1過楹。有久之不止,當馬非馬,若人過梁。1. 矢 : 原作「夫」。

必,不已也。

必:謂臺執者也。若弟兄一然者一不然者,必「不必」也,是非必也。

平,同高也。屬於:[幾何學]



同長,以正1相盡也。

同:楗2與狂之同長也。1. 正 : 原作「缶」。2. 楗 : 原作「捷」。

中,同長也。屬於:[幾何學]

心中:自是往相若也。屬於:[幾何學]

厚,有所大也。屬於:[幾何學]

厚:惟無所大。屬於:[幾何學]

日中,正1南也。

1. 正 : 原作「缶」。

直,參也。



圜,一中同長也。屬於:[幾何學]

圜:規寫攴也。屬於:[幾何學]

方,柱隅四讙也。

方:矩見攴也。屬於:[幾何學]

倍,為二也。屬於:[幾何學]

倍:二尺與尺但去一。屬於:[幾何學]

端,體之無厚1而最前者也。屬於:[幾何學]

端:是無間2也。屬於:[幾何學] 
1. 厚 : 原作「序」。2. 間 : 原作「同」。

有間,中也。屬於:[幾何學]

有間1:謂夾之者也。屬於:[幾何學] 
1. 間 : 原作「聞」。

間,不及旁也。屬於:[幾何學]

間1:謂夾者也。尺前於區穴而後於端,不夾於端與區內。及及非齊之,及也。屬於:[幾何學] 
1. 間 : 原作「聞」。

纑,間虛也。

纑:間1虛也者,兩木之間,謂其無木者也。1. 間 : 舊脫。

盈,莫不有也。

盈:無盈無厚。

堅白,不相外也。

於石1無所往而不得,得二,堅。異處不相盈,相非,是相外也。1. 石 : 原作「尺」。

攖,相得也。

攖:尺與尺俱不盡,端與端俱盡。尺與或盡或不盡。堅白之攖相盡,體攖不相盡。端。

仳1,有以相攖,有不相攖也。

仳:兩有端而后可。1. 仳 : 原作「似」。

次,無間而不攖攖也。

次:無厚而后可。

法,所若而然也。

法:意、規、員三也,俱可以為法。

佴,所然也。

佴:然也者,民若法也。

說,所以明也。屬於:[邏輯]



彼1,不可兩不可也。

彼:凡牛、樞非牛,兩也。無以非也。1. 彼 : 原作「攸」。

辯,爭彼也。辯勝,當也。屬於:[邏輯]

辯:或謂之牛,或謂之非牛,是爭彼也。是不俱當。不俱當,必或不當,不若當犬。屬於:[邏輯]

為,窮知而𠐴於欲也。

為:欲𩁥其指,智不知其害,是智之罪也。若智之慎文也無遺,於其害也,而猶欲𩁥之,則離之是猶食脯也。騷之利害,未知也,欲而騷,是不以所疑止所欲也。廧外之利害,未可知也,趨之而得力,則弗趨也,是以所疑止所欲也。觀「為,窮知而𠐴於欲」之理,𩁥脯而非𢜔也,𩁥指而非愚也,所為與不所與為相疑也,非謀也。

已,成、亡。

已:為衣,成也;治病,亡也。

使,謂、故。屬於:[邏輯]

使:令,謂「謂」也,不必成。濕,「故」也,必待所為之成也。屬於:[邏輯]

名,達、類、私。屬於:[邏輯]

名:物,達也。有實必待之名1也。命之馬,類也。若實也者,必以是名也。命之臧,私也。是名也止於是實也。聲出口,俱有名,若姓字2灑。屬於:[邏輯] 
1. 之名 : 原作「文多」。2. 字 : 原作「宇」。自孫詒讓《墨子閒詁》改。

謂,移、舉、加。

謂:狗、犬,命也。狗犬,舉也。叱狗,加也。

知,聞、說、親,名、實、合、為。屬於:[知識論]

知:傳受之,聞也;方不㢓,說也;身觀焉,親也。所以謂,名也;所謂,實也。名實耦,合也。志行,為也。屬於:[知識論]

聞,傳、親。屬於:[知識論]

聞:或告之,傳也;身觀焉,親也。屬於:[知識論]

見,體、盡。

見:時者,體也;二者,盡也。

合,正1、宜、必。

合2:兵立反中、志工,正也;臧之為,宜也:非彼必不有,必也。聖者用而勿必,必也者可勿疑。1. 正 : 原作「缶」。2. 合 : 原作「古」。

欲正1權利,且惡正2權害。

仗者,兩而勿偏。1. 正 : 原作「缶」。2. 正 : 原作「缶」。

為,存、亡、易、蕩、治、化。

為:早臺,存也。病,亡也。買鬻,易也。霄盡,蕩也。順長,治也。鼃買,化也。

同,重、體、合、類。屬於:[邏輯]

同:二名一實,重同也。不外於兼,體同也。俱處於室,合同也。有以同,類同也。屬於:[邏輯]

異,二、不1體、不合、不類。屬於:[邏輯]

異:二必異,二也。不連屬,不體也。不同所,不合也。不有同,不類也。屬於:[邏輯] 
1. 不 : 舊脫。  孫詒讓《墨子閒詁》

同異交得放有無。

同異交得:於福家良恕,有無也。比度,多少也。免軔還園,去就也。鳥折用桐,堅柔也。劍尤甲1,死生也。處室子子2母,長少也。兩絕勝,白黑也。中央旁也。論、行行行3、學、實,是非也。雞4宿,成未也。兄弟,俱適也。身處志往,存亡也。霍,為姓故也。賈宜,貴賤也。1. 甲 : 原作「早」。2. 子 : 刪除。 衍文。 3. 行行 : 刪除。  4. 雞 : 原作「難」。

聞,耳之聰也。屬於:[知識論]



循所聞而得其意,心之察也。



言,口之利也。



執所言而意得見,心之辯也。



諾,不一、利用。

諾:超、誠1、負2、正3也。相從、相去、先知、是、可,五色。長短、前後、輕重援正五諾,皆人於知有說;過五諾,若員,無直無說;用五諾,若自然矣。4。1. 誠 : 原作「城」。2. 負 : 原作「員」。3. 正 : 原作「止」。4. 正五諾,皆人於知有說;過五諾,若員,無直無說;用五諾,若自然矣。 : 從第100條移到此處。

服,執誽。音利。1

執服難。成言務成之,九則求執之。1. 音利。 : 刪除。 畢沅注:「『音利』二字舊注,未詳其義」。

巧轉則求其故。屬於:[邏輯]



法同,則觀其同。屬於:[邏輯]

法:法取同,觀巧傳。屬於:[邏輯]

法異,則觀其宜。屬於:[邏輯]

法:取此擇彼,問故觀宜。以人之有黑者有不黑者也,止黑人,與以有愛於人有不愛於人,心愛人,是孰宜?屬於:[邏輯]

止,因以別道。屬於:[邏輯]

止1:彼舉然者,以為此其然也,則舉不然者而問之。屬於:[邏輯] 
1. 止 : 原作「心」。

正1,無非。

若聖人有非而不非。正五諾,皆人於知有說;過五諾,若員,無直無說;用五諾,若自然矣。21. 正 : 原作「缶」。2. 正五諾,皆人於知有說;過五諾,若員,無直無說;用五諾,若自然矣。 : 移到第94條。


\end{pinyinscope}