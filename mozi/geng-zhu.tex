\article{耕柱}

\begin{pinyinscope}
子墨子怒耕柱子,耕柱子曰:「我毋俞於人乎?」子墨子曰:「我將上大行,駕驥與羊,子將誰敺?」耕柱子曰:「將敺驥也。」子墨子曰:「何故敺驥也?」耕柱子曰:「驥足以責。」子墨子1曰:「我亦以子為足以責。」

巫馬子謂子墨子曰:「鬼神孰與聖人明智?」子墨子曰:「鬼神之明智於聖人,猶聰耳明目之與聾瞽也。昔者夏后開使蜚廉折金於山川,而陶鑄之於昆吾;是使翁難雉乙1卜於白若之龜,曰:『鼎成三足而方』,不炊而自烹,不舉而自臧,不遷而自行,以祭於昆吾之虛2,上鄉」!乙3又4言兆之由曰:『饗矣!逢逢白雲,一南一北,一西一東,九鼎既成,遷於三國。』夏后氏失之,殷人受之;殷人失之,周人受之。夏后、殷、周之相受也。數百歲矣。使聖人聚其良臣與其桀相而謀,豈能智數百歲之後哉!而鬼神智之。是故曰,鬼神之明智於聖人也,猶聰耳明目之與聾瞽也。」

治徒娛、縣子碩問於子墨子曰:「為義孰為大務?」子墨子曰:「譬若築牆然,能築者築,能實壤者實壤,能欣者欣,然後牆成也。為義猶是也。能談辯者談辯,能說書者說書,能從事者從事,然後義事成也。」

巫馬子謂子墨子曰:「子兼愛天下,未云利也;我不愛天下,未云賊也。功皆未至,子何獨自是而非我哉?」子墨子曰:「今有燎者於此,一人奉水將灌之,一人摻火將益之,功皆未至,子何貴於二人?」巫馬子曰:「我是彼奉水者之意,而非夫摻火者之意。」子墨子1曰:「吾亦是吾意,而非子之意也。」

子墨子游荊耕柱子於楚,二三子過之,食之三升,客之不厚。二三子復於子墨子曰:「耕柱子處楚無益矣。二三子過之,食之三升,客之不厚。」子墨子曰:「未可智也。」毋幾何而遺十金於子墨子,曰:「後生不敢死,有十金於此,願夫子之用也。」子墨子曰:「果未可智也。」

巫馬子謂子墨子曰:「子1之為義也,人不見而助2,鬼不見而富,而子為之,有狂疾!」子墨子曰:「今使子有二臣於此,其一人者見子從事,不見子則不從事;其一人者見子亦從事,不見子亦從事,子誰貴於此二人?」巫馬子曰:「我貴其見我亦從事,不見我亦從事者。」子墨子曰:「然則,是子亦貴有狂疾也。」

子夏子徒問於子墨子曰:「君子有鬥乎?」子墨子曰:「君子無鬥。」子夏之徒曰:「狗豨猶有鬥,惡有士而無鬥矣?」子墨子曰:「傷矣哉!言則稱於湯文,行則譬於狗豨,傷矣哉!」

巫馬子謂子墨子曰:「舍今之人而譽先王,是譽槁骨也。譬若匠人然,智槁木也,而不智生木。」子墨子曰:「天下之所以生者,以先王之道教也。今譽先王,是譽天下之所以生也。可譽而不譽,非1仁也。」子墨子曰:「和氏之璧,隋侯之珠,三棘六異,此諸侯之所謂良寶也。可以富國家,眾人民,治刑政,安社稷乎?曰不可。所謂貴良寶者,為其可以利也。而和氏之璧、隋侯之珠、三棘六異不可以利人,是非天下之良寶也。今用義為政於國家,人民必眾,刑政必治,社稷必安。所為貴良寶者,可以利民也,而義可以利人,故曰,義天下之良寶也。」

葉公子高問政於仲尼曰:「善為政者若之何?」仲尼對曰:「善為政者,遠者近之,而舊者新之。」子墨子聞之曰:「葉公子高未得其問也,仲尼亦未得其所以對也。葉公子高豈不知善為政者之遠者近也,而舊者新是哉?問所以為之若之何也。不以人之所不智告人,以所智告之,故葉公子高未得其問也,仲尼亦未得其所以對也。」

子墨子謂魯陽文君曰:「大國之攻小國,譬猶童子之為馬也。童子之為馬,足用而勞。今大國之攻小國也,攻者農夫不得耕,婦人不得織,以守為事;攻人者,亦農夫不得耕,婦人不得織,以攻為事。故大國之攻小國也,譬猶童子之為馬也。」

子墨子曰:「言足以復行者,常之;不1足以舉行者,勿常。不足以舉行而常之,是蕩口也。」

子墨子使管黔敖游高石子於衛,衛君致祿甚厚,設之於卿。高石子三朝必盡言,而言無行者。去而之齊,見子墨子曰:「衛君以夫子之故,致祿甚厚,設我於卿。石三朝必盡言,而言無行,是以去之也。衛君無乃以石為狂乎?」子墨子曰:「去之苟道,受狂何傷!古者周公旦非關叔,辭三公東處於商蓋,人皆謂之狂。後世稱其德,揚其名,至今不息。且翟聞之為義非避毀就譽,去之苟道,受狂何傷!」高石子曰:「石去之,焉敢不道也。昔者夫子有言曰:『天下無道,仁士不處厚焉。』今衛君無道,而貪其祿爵,則是我為苟啗人食也。」子墨子說,而召子禽子曰:「姑聽此乎!夫倍義而鄉祿者,我常聞之矣。倍祿而鄉義者,於高石子焉見之也。」

子墨子曰:「世俗之君子,貧而謂之富,則怒,無義而謂之有義,則喜。豈不悖哉!」

公孟子曰:「先人有則三而已矣。」子墨子曰:「孰先人而曰有則三而已矣?子未智人之先有。」

後生有反子墨子而反者,「我豈有罪哉?吾反後」。子墨子曰:「是猶三軍北,失後之人求賞也。」

公孟子曰:「君子不作,術而已。」子墨子曰:「不然,人之其不君子者,古之善者不誅,今也善者不作。其次不君子者,古之善者不遂,己有善則作之,欲善之自己出也。今誅而不作,是無所異於不好遂而作者矣。吾以為古之善者則誅之,今之善者則作之,欲善之益多也。」

巫馬子謂子墨子曰:「我與子異,我不能兼愛。我愛鄒人於越人,愛魯人於鄒人,愛我鄉人於魯人,愛我家人於鄉人,愛我親於我家人,愛我身於吾親,以為近我也。擊我則疾,擊彼則不疾於我,我何故疾者之不拂,而不疾者之拂?故有我有殺彼以我,無殺我以利。」子墨子曰:「子之義將匿邪,意將以告人乎?」巫馬子曰:「我何故匿我義?吾將以告人。」子墨子曰:「然則,一人說子,一人欲殺子以利己;十人說子,十人欲殺子以利己;天下說子,天下欲殺子以利己。一人不說子,一人欲殺子,以子為施不祥言者也;十人不說子,十人欲殺子,以子為施不祥言者也;天下不說子,天下欲殺子,以子為施不祥言者也。說子亦欲殺子,不說子亦欲殺子,是所謂經者口也,殺常之身者也。」子墨子曰:「子之言惡利也?若無所利而不言,是蕩口也。」

子墨子謂魯陽文君曰:「今有一人於此,羊牛犓豢,維人但割而和之,食之不可1勝食也。見人之作餅,則還然竊之,曰:『舍余食。』不知日月安不足乎,其有竊疾乎?」魯陽文君曰:「有竊疾也。」子墨子曰:「楚四竟之田,曠蕪而不可勝辟,謼虛2數千,不可勝,見宋、鄭之閒邑,則還然竊之,此與彼異乎?」魯陽文君曰:「是猶彼也,實有竊疾也。」

子墨子曰:「季孫紹與孟伯常治魯國之政,不能相信,而祝於叢社,曰:『苟使我和。』是猶弇其目,而祝於叢社曰:『苟使我皆視』。豈不繆哉!」

子墨子謂駱滑氂曰:「吾聞子好勇。」駱滑氂曰:「然,我聞其鄉有勇士焉,吾必從而殺之。」子墨子曰:「天下莫不欲與其所好,度其所惡。今子聞其鄉有勇士焉,必從而殺之,是非好勇也,是惡勇也。」


\end{pinyinscope}