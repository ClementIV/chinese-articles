\article{大取}

\begin{pinyinscope}
天之愛人也,薄於聖人之愛人也;其利人也,厚於聖人之利人也。大人之愛小人也,薄於小人之愛大人也;其利小人也,厚於小人之利大人也。屬於:[倫理]

以臧為其親也,而愛之,非愛其親也;以臧為其親也,而利之,非利其親也。以樂為利其子,而為其子欲之,愛其子也;以樂為利其子,而為其子求之,非利其子也。屬於:[邏輯]

於所體之中,而權輕重之謂權。權,非為是也,非非為非也。權,正也。斷指以存腕,利之中取大,害之中取小也。害之中取小也,非取害也,取利也。其所取者,人之所執也。遇盜人,而斷指以免身,利也;其遇盜人害也。斷指與斷腕,利於天下相若,無擇也;死生利若,一無擇也。殺一人以存天下,非殺一人以利天下也。殺己以存天下,是殺己以利天下。於事為之中而權輕重之謂求,求為之,非也,害之中取小,求為義非為義也。屬於:[邏輯]

為暴人語天之為是也而性,為暴人歌天之為非也。諸陳執既有所為,而我為之陳執;執之所為因,吾所為也。若陳執未有所為,而我為之陳執,陳執因吾所為也。暴人為我為天之。以人非為是也,而性不可正而正之。

利之中取大非不得已也;害之中取小,不得已也。所未有而取焉是利之中取大也;於所既有而棄焉,是害之中取小也。屬於:[倫理]

義可厚,厚之;義可薄,薄之。謂倫列。德行、君上、老長、親戚,此皆所厚也。為長厚,不為幼薄。親厚,厚;親薄,薄。親至,薄不至。義厚親,不稱行而顧行。屬於:[倫理]

為天下厚禹,為禹也。為天下厚愛禹,乃為禹之人愛也。厚禹之加於天下,而厚禹不加於天下。若惡盜之為加於天下,而惡盜不加於天下。

愛人不外己,己在所愛之中。己在所愛,愛加於己。倫列之愛己,愛人也。屬於:[倫理]

聖人惡疾病,不惡危難。正體不動,欲人之利也,非惡人之害也。

聖人不為其室臧之故,在於臧。屬於:[倫理]

聖人不得為子之事。聖人之法死亡親,為天下也。厚親,分也;以死亡之,體渴興利。有厚薄而毋倫列之興利,為己。語經,語經也。非白馬焉。執駒焉說求之,舞說非也,漁大之舞大,非也。三物必具,然後足以生。1屬於:[倫理]

臧之愛己,非為愛己之人也。厚不外己,愛無厚薄。舉己,非賢也。義,利;不義,害。志功為辯。

有有於秦馬,有有於馬也,智來者之馬也。

愛眾眾世與愛寡世相若,兼愛之,有相若。愛尚世與愛後世,一若今之世人也。鬼,非人也;兄之鬼,兄也。

天下之利驩。聖人有愛而無利,俔日之言也,乃客之言也。天下無人,子墨子之言也猶在。

不得已而欲之,非欲之非欲之也。非殺臧也。專殺盜,非殺盜也。凡學愛人。

小圜之圜,與大圜之圜同。方至尺之不至也,與不至鐘之至,不異。其不至同者,遠近之謂也。屬於:[邏輯]

是璜也,是玉也。意楹,非意木也,意是楹之木也。意指之人也,非意人也。意獲也,乃意禽也。志功,不可以相從也。

利人也,為其人也;富人,非為其人也,有為也以富人。富人也,治人有為鬼焉。

為賞譽利一人,非為賞譽利人也,亦不至無貴於人。智親之一利,未為孝也,亦不至於智不為己之利於親也。智是之世之有盜也,盡愛是世。智是室之有盜也,不盡是室也。智其一人之盜也,不盡是二人。雖其一人之盜,茍不智其所在,盡惡其弱也。

諸聖人所先,為人欲名實。名實不必名。苟是石也白,敗是石也,盡與白同。是石也唯大,不與大同。是有便謂焉也。以形貌命者,必智是之某也,焉智某也,不可以形貌命者,唯不智是之某也,智某可也。諸以居運命者,苟人於其中者,皆是也,去之因非也。諸以居運命者,若鄉里齊荊者,皆是。諸以形貌命者,若山丘室廟者,皆是也。

智與意異。重同,具同,連同,同類之同,同名之同;丘同,鮒同,是之同,然之同,同根之同。有非之異,有不然之異。有其異也,為其同也,為其同也異。一曰乃是而然,二曰乃是而不然,三曰遷,四曰強。屬於:[邏輯]

子深其深,淺其淺,益其益,尊其尊。察次山比因至優指復;次察聲端名因請復。正夫辭惡者,人右以其請得焉。諸所遭執,而欲惡生者,人不必以其請得焉。聖人之附瀆也,仁而無利愛。利愛生於慮。昔者之慮也,非今日之慮也;昔者之愛人也,非今之愛人也。愛獲之愛人也,生於慮獲之利,非慮臧之利也;而愛臧之愛人也,乃愛獲之愛人也。去其愛而天下利,弗能去也。昔之知墻,非今日之知墻也。貴為天子,其利人不厚於正夫。二子事親,或遇孰,或遇凶,其親也相若,非彼其行益也,非加也。外執無能厚吾利者。藉藏也死而天下害,吾持養臧也萬倍,吾愛臧也不加厚。

長人之異,短人之同,其貌同者也,故同。指之人也與首之人也異,人之體非一貌者也,故異。將劍與挺劍異。劍,以形貌命者也,其形不一,故異。楊木之木與桃木之木也同。諸非以舉量數命者,敗之盡是也,故一人指,非一人也;是一人之指,乃是一人也。方之一面,非方也;方木之面,方木也。屬於:[邏輯]

三物必具,然後足以生。1夫辭2以故生,以理長,以類行也者。立辭而不明於其所生,妄3也。今人非道無所行,唯有強股肱而不明於道,其困也,可立而待也。夫辭以類行者也,立辭而不明於其類,則必困矣。屬於:[邏輯]

故浸淫之辭,其類在於鼓栗。聖人也,為天下也,其類在於追迷。或壽或卒,其利天下也指若,其類在譽石,一日而百萬生,愛不加厚,其類在惡害。愛二世有厚薄,而愛二世相若。其類在蛇文。愛之相若,擇而殺其一人,其類在阬下之鼠。小仁與大仁,行厚相若,其類在申。凡興利除害也,其類在漏雍。厚親,不稱行而類行,其類在江上井。「不為己」之可學也,其類在獵走。愛人非為譽也,其類在逆旅。愛人之親,若愛其親,其類在官茍。兼愛相若,一愛相若。一愛相若,其類在死也。屬於:[邏輯]


\end{pinyinscope}