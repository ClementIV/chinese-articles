\article{備穴}

\begin{pinyinscope}
禽子再拜再拜,曰:「敢問古人有善攻者,穴土而入,縛柱施火,以壞吾城,城壞,或中人為之柰何?」子墨子曰:「問穴土之守邪?備穴者城內為高樓,以謹候望適人。適人為變,築垣聚土非常者,若彭有水濁非常者,此穴土也,急塹城內穴亓土直之。穿井城內,五步一井,傅城足,高地,丈五尺,下地,得泉三尺而止。令陶者為罌,容四十斗以上,固幎之以薄皮革,置井中,使聰耳者伏罌而聽之,審知穴之所在,鑿穴迎之。

令陶者為月明,長二尺五寸六圍,中判之,合而施之穴中,偃一,覆一。柱之外善周塗,亓傅柱者勿燒。柱者勿燒柱善塗亓竇際,勿令泄。兩旁皆如此,與穴俱前。下迫地,置康若灰亓中,勿滿。灰康長五竇,左右俱雜相如也。穴內口為灶,令如窯,令容七八員艾,左右竇皆如此,灶用四橐。穴且遇,以頡皋衝之,疾鼓橐熏之,必令明習橐事者勿令離灶口。連版以穴高下、廣陜為度,令穴者與版俱前,鑿亓版令容矛,參分亓疏數,令可以救竇。穴則遇,以版當之,以矛救竇,勿令塞竇,竇則塞,引版而卻,過一竇而塞之,鑿亓竇,通亓煙,煙通,疾鼓橐以熏之。從穴內聽穴之左右,急絕亓前,勿令得行。若集客穴,塞之以柴塗,令無可燒版也。然則穴土之攻敗矣。

寇至吾城,急非常也,謹備穴。穴疑有應寇,急穴穴未得,慎毋追。

凡殺以穴攻者,二十步一置穴,穴高十尺,鑿十尺,鑿如前,步下三尺,十步擁穴,左右橫行,高廣各十尺殺。

俚兩罌,深平城置板亓上,連板以井聽。五步一密。用梓若松為穴戶,戶穴有兩蒺藜,皆長極亓戶,戶為環,壘石外埻,高七尺,加堞亓上。勿為陛與石,以縣陛上下出入。具鑪橐,橐以牛皮,鑪有兩缶,以橋鼓之百十,每亦熏四十什,然炭杜之,滿鑪而蓋之,毋令氣出。適人疾近五百穴穴高若下,不至吾穴,即以伯鑿而求通之。穴中與適人遇,則皆圉而毋逐,且戰北,以須鑪火之然也,即去而入壅穴殺。有鼠竄,為之戶及關籥獨順,得往來行亓中。穴壘之中各一狗,狗吠即有人也。

斬艾與柴長尺,乃置窯灶中,先壘窯壁迎穴為連版。

鑿井傳城足,三丈一,視外之廣陜而為鑿井,慎勿失。城卑穴高從穴難。鑿井城上,為三四井,內新甀井中,伏而聽之。審之知穴之所在,穴而迎之。穴且遇,為頡皋,必以堅材為夫,以利斧施之,命有力者三人用頡皋衝之,灌以不潔十餘石。

趣伏此井中,置艾亓上,七八員,盆蓋井口,毋令煙上泄,旁亓橐口,疾鼓之。

以車輪為轀。一束樵,染麻索塗中以束之。鐵鎖,縣正當寇穴口。鐵鎖長三丈,端環,一端鉤。

鼠穴高七尺,五寸廣,柱閒也尺,二尺一柱,柱下傅舄,二柱共一員十一。兩柱同質,橫員士,柱大二圍半,必固亓員士,無柱與柱交者。

穴二窯,皆為穴月屋,為置吏、舍人,各一人,必置水。塞穴門以車兩走,為轀,塗亓上,以穴高下廣陜為度,令入穴中四五尺,維置之。當穴者客爭伏門,轉而塞之為窯,容三員艾者,令亓突入伏尺。伏傅突一旁,以二橐守之,勿離。穴矛以鐵,長四尺半,大如鐵服說,即刃之二矛。內去竇尺,邪鑿之,上穴當心,亓矛長七尺。穴中為環利率,穴二。

鑿井城下,俟亓身井且通,居版上,而鑿亓一遍,已而移版,鑿一遍。頡皋為兩夫,而旁貍亓植,而數鉤亓兩端。諸作穴者五十人,男女相半。五十人。攻內為傳士之口,受六參,約枲繩以牛亓下,可提而與投,已則穴七人守退,壘之中為大廡一,藏穴具亓中。難穴,取城外池脣木月散之什,斬亓穴,深到泉。難近穴為鐵鈇。金與扶林長四尺,財自足。客即穴,亦穴而應之。

為鐵鉤鉅長四尺者,財自足,穴徹,以鉤客穴者。為短矛、短戟、短弩、虻矢,財自足,穴徹以鬥。以金劍為難,長五尺,為銎、木柄;柄有慮枚,以左客穴。

戒持罌,容三十斗以上,貍穴中,丈一,以聽穴者聲。

為穴,高八尺,廣,善為傅置。具鑪牛皮橐,皮及缶,衛穴二,蓋陳靃及艾,穴徹熏之以。

斧金為斫,柯長三尺,衛穴四。為壘,衛穴四十,屬四。為斤、斧、鋸、鑿、瞿、財自足。為鐵校,衛穴四。

為中櫓,高十丈半,廣四尺。為橫穴八櫓,蓋具稿枲,財自足,以燭穴中。

益持醯,客即熏,以救目,救目分方鼓穴,以盆盛醯置穴中,大盆毋少四斗。即熏,以目臨醯上及以洒目。」


\end{pinyinscope}