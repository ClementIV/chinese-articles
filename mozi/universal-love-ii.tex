\article{兼愛中}

\begin{pinyinscope}
子墨子言曰:「仁人之所以為事者,必興天下之利,除去天下之害,以此為事者也。」然則天下之利何也?天下之害何也?子墨子言曰:「今若國之與國之相攻,家之與家之相篡,人之與人之相賊,君臣不惠忠,父子不慈孝,兄弟不和調,此則天下之害也。」

然則察1此害亦何用生哉?以不相愛生邪?子墨子言:「以不相愛生。今諸侯獨知愛其國,不愛人之國,是以不憚舉其國以攻人之國。今家主獨知愛其家,而不愛人之家,是以不憚舉其家以篡人之家。今人獨知愛其身,不愛人之身,是以不憚舉其身以賊人之身。是故諸侯不相愛則必野戰。家主不相愛則必相篡,人與人不相愛則必相賊,君臣不相愛則不惠忠,父子不相愛則不慈孝,兄弟不相愛則不和調。天下之人皆不相愛,強必執弱,富必侮貧,貴必敖賤,詐必欺愚。凡天下禍篡怨恨,其所以起者,以不相愛生也,是以仁者非之。」

既以非之,何以易之?子墨子言曰:「以兼相愛交相利之法易之。」然則兼相愛交相利之法將柰何哉?子墨子言:「視人之國若視其國,視人之家若視其家,視人之身若視其身。是故諸侯相愛則不野戰,家主相愛則不相篡,人與人相愛則不相賊,君臣相愛則惠忠,父子相愛則慈孝,兄弟相愛則和調。天下之人皆相愛,強不執弱,眾不劫寡,富不侮貧,1貴不敖賤,詐不欺愚。凡天下禍篡怨恨可使毋起者,以相愛生也,是2以仁者譽之。」

然而今天下之士君臣相愛則惠忠,父子相愛則慈孝,兄弟相愛則和調。天下之人皆相愛,強不執弱,眾不劫寡,富不侮貧,1君子2曰:「然,乃若兼則善矣,雖然,天下之難物于故也。」子墨子言曰:「天下之士君子,特不識其利,辯其故也。今若夫攻城野戰,殺身為名,此天下百姓之所皆難也,苟君說之,則士眾能為之。況於兼相愛,交相利,則與此異。夫愛人者,人必從而愛之;利人者,人必從而利之;惡人者,人必從而惡之;害人者,人必從而害之。此何難之有!特上弗以為政,士不以為行故也。

昔者晉文公好士之惡衣,故文公之臣皆牂羊之裘,韋以帶劍,練帛之冠,入以見於君,出以踐於1朝。是其故何也?君說之,故臣為之也。昔者楚靈王好士細要,故靈王之臣皆以一飯為節,肱息然後帶,扶牆然後起。比期年,朝有黧黑之色。是其故何也?君說之,故臣能之也。昔越王句踐好士之勇,教馴其臣,和合之焚舟失火,試其士曰:『越國之寶盡在此!』越王親自鼓其士而進之。曰2士聞鼓音,破碎亂行,蹈火而死者左右百人有餘。越王擊金而退之。」

是故子墨子言曰:「乃若夫少食惡衣,殺身而為名,此天下百姓之所皆難也,若苟君說之,則眾能為之。況兼相愛,交相利,與此異矣。夫愛人者,人亦從而愛之;利人者,人亦從而利之;惡人者,人亦從而惡之;害人者,人亦從而害之。此何難之有焉,特上不以為政而士不以為行故也。」

然而今天下之士君子曰:「然,乃若兼則善矣。雖然,不可行之物也,譬若挈太山越河濟也。」子墨子言:「是非其譬也。夫挈太山而越河濟,可謂畢劫有力矣,自古及今未有能行之者也。況乎兼相愛,交相利,則與此異,古者聖王行之。何以知其然?古者禹治天下,西為西河漁竇,以泄渠孫皇之水;北為防原泒,注后之邸,呼池之竇,洒為底柱,鑿為龍門,以利燕、代、胡、貉與西河之民;東方漏之陸防孟諸之澤,灑為九澮,以楗東土之水,以利冀州之民;南為江、漢、淮、汝,東流之,注五湖之處,以利荊、楚、干、1越與南夷之民。此言禹之事,吾今行兼矣。昔者文王之治西土,若日若月,乍光于四方于西土,不為大國侮小國,不為眾庶侮鰥寡,不為暴勢奪穡人黍、稷、狗、彘。天屑臨文王慈,是以老而無子者,有所得終其壽;連獨無兄弟者,有所雜於生人之閒;少失其父母者,有所放依而長。此文王之事,則吾今行兼矣。昔者武王將事泰山隧,傳曰:『泰山,有道曾孫周王有事,大事既獲,仁人尚作,以祗商夏,蠻夷醜貉。雖有周親,不若仁人,萬方有罪,維予一人。』此言武王之事,吾今行兼矣。」

是故子墨子言曰:「今天下之君子,忠實欲天下之士1富,而惡其貧;欲天下之治,而惡其亂,當兼相愛,交相利,此聖王之法,天下之治道也,不可不務為也。」


\end{pinyinscope}