\article{非攻上}

\begin{pinyinscope}
今有一人,入人園圃,竊其桃李,眾聞則非之,上為政者得則罰之。此何也?以虧人自利也。至攘人犬豕雞豚者,其不義又甚入人園圃竊桃李。是何故也?以虧人愈多,其不仁茲甚,罪益厚。至入人欄廄,取人馬牛者,其不仁義又甚攘人犬豕雞豚。此何故也?以其虧人愈多。苟虧人愈多,其不仁茲甚,罪益厚。至殺不辜人也,扡其衣裘,取戈劍者,其不義又甚入人欄廄取人馬牛。此何故也?以其虧人愈多。苟虧人愈多,其不仁茲甚矣,罪益厚。當此,天下之君子1皆知而非之,謂之不義。今至大為攻國,則弗知非,從而譽之,謂之義。此可2謂知義與不義之別乎?

殺一人謂之不義,必有一死罪矣,若以此說往,殺十人十重不義,必有十死罪矣;殺百人百重不義,必有百死罪矣。當此,天下之君子皆知而非之,謂之不義。今至大為不義攻國,則弗知1非,從而譽之,謂之義,情不知其不義也,故書其言以遺後世。若知其不義也,夫奚說書其不義以遺後世哉?今有人於此,少見黑曰黑,多見黑曰白,則以此人不知白黑之辯矣;少嘗苦曰苦,多嘗苦曰甘,則必以此人為不知甘苦之辯矣。今小為非,則知而非之。大為非攻國,則不知而2非,從而譽之,謂之義。此可謂知義與不義之辯乎?是以知天下之君子也,辯義與不義之亂也。


\end{pinyinscope}