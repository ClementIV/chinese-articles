\article{法儀}

\begin{pinyinscope}
子墨子曰:「天下從事者,不可以無法儀,無法儀而其事能成者無有也1。雖至士之為將相者,皆有法,雖至百工從事者,亦皆有法。百工為方以矩,為圓以規,直衡以水,2以繩,正以縣。無巧工、不巧工,皆以此五者為法。巧者能中之,不巧者雖不能中,放依以從事,猶逾己。故百工從事,皆有法所度。」

今大者治天下,其次治大國,而無法所度,此不若百工辯也。然則奚以為治法而可?當皆法其父母,奚若?天下之1為父母者眾,而仁者寡,若皆法其父母,此法不仁也。法不仁不可以為法。當皆法其學,奚若?天下之為學者眾,而仁者寡,若皆法其學,此法不仁也。法不仁不可以為法。當皆法其君,奚若?天下之為君者眾,而仁者寡,若皆法其君,此法不仁也。法不仁不可以為法。故父母、學、君三者,莫可以為治法而可2。

然則奚以為治法而可?故曰莫若法天。天之行廣而無私,其施厚而不德,其明久而不衰,故聖王法之。既以天為法,動作有為,必度於天,天之所欲則為之,天所不欲則止。然而天何欲何惡者也?天必欲人之相愛相利,而不欲人之相惡相賊也。奚以知天之欲人之相愛相利,而不欲人之相惡相賊也?以其兼而愛之,兼而利之也。奚以知天兼而愛之,兼而利之也?以其兼而有之,兼而食之也。

今天下無大小國,皆天之邑也。人無幼長貴賤,皆天之臣也。此以莫不犓羊牛1、豢犬豬,絜為酒醴粢盛,以敬事天,此不為兼而有之,兼而食之邪?天苟兼而有食之,夫奚說以不欲人之相愛相利也?故曰:「愛人利人者,天必福之,惡人賊人者,天必禍之。」曰2:「殺不辜者,得不祥焉。夫奚說人為其相殺而天與禍乎?是以知3天欲人相愛相利,而不欲人相惡相賊也。」

昔之聖王禹、湯、文、武,兼愛1天下之百姓,率以尊天事鬼,其利人多,故天福之,使立為天子,天下諸侯皆賓事之。暴王桀、紂、幽、厲,兼惡天下之百姓,率以詬天侮鬼。其賊2人多,故天禍之,使遂失其國家,身死為僇於天下。後世子孫毀之,至今不息。故為不善以得禍者,桀、紂、幽、厲是也。愛人利人以得福者,禹、湯、文、武是也。愛人利人以得福者有矣,惡人賊人以得禍者亦有矣!


\end{pinyinscope}