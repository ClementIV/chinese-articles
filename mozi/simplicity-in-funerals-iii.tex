\article{節葬下}

\begin{pinyinscope}
子墨子言曰:「仁者之為天下度也,辟之無以異乎孝子之為親度也。今孝子之為親度也,將柰何哉?曰:『親貧則從事乎富之,人民寡則從事乎眾之,眾亂則從事乎治之。』當其於此也,亦有力不足,財不贍,智不智,然後己矣。無敢舍餘力,隱謀遺利,而不為親為之者矣。若三務者1,孝子之為親度也,既若此矣。

雖仁者之為1天下度,亦猶此也。曰:『天下貧則從事乎富之,人民寡則從事乎眾之,眾而亂則從事乎治之。』當其於此,亦有力不足,財不贍、智不智,然後已矣。無敢舍餘力,隱謀遺利,而不為天下為之者矣。若三務者,此仁者之為天下度也2,既若此矣。

今逮至昔者三代聖王既沒,天下失義,後世之君子,或以厚葬久喪以為仁也,義也,孝子之事也;或以厚葬久喪以為非仁義,非孝子之事也。曰二子者,言則相非,行即相反,皆曰:『吾上祖述堯舜禹湯文武之道者也。』而言即相非,行即相反,於此乎後世之君子,皆疑惑乎二子者言也。若苟疑惑乎之二子者言,然則姑嘗傳而為政乎國家萬民而觀之。計厚葬久喪,奚當此三利者?我意若使法其言,用其謀,厚葬久喪實可以富貧眾寡,定危治亂乎,此仁也,義也,孝子之事也1,為人謀者不可不勸也。仁者將興之天下,誰賈而使民譽之,終勿廢也。意亦使法其言,用其謀,厚葬久喪實不可以富貧眾寡,定危理亂乎,此非仁非義,非孝子之事也,為人謀者不可不沮也。仁者將興之天下,誰賈而使民譽之,終勿廢也。意亦使法其言,用其謀,厚葬久喪實不可以富貧眾寡,定危理亂乎,此非仁非義,非孝子之事也,為人謀者不可不沮也。2仁者將求除之天下,相廢而使人非之,終身勿為。

且故興天下之利,除天下之害,令國家百姓之不治也,自古及今,未嘗之有也。何以知其然也?今天下之士君子,將猶多皆疑惑厚葬久喪之為中是非利害也。」故子墨子言曰:「然則姑嘗稽之,今雖毋法執厚葬久喪者言,以為事乎國家。此存乎王公大人有喪者,曰棺槨必重,葬埋必厚,衣衾必多,文繡必繁,丘隴必巨;存乎匹夫賤人死者,殆竭家室;乎諸侯死者,虛車府,然後金玉珠璣比乎身,綸組節約,車馬藏乎壙,又必多為屋幕。鼎鼓几梴壺濫,戈劍羽旄齒革,挾而埋之,滿意。若送從,曰天子殺殉,眾者數百,寡者數十。將軍大夫殺殉,眾者數十,寡者數人。處喪之法將柰何哉?曰哭泣不秩聲翁,縗絰垂涕,處倚廬,寢苫枕塊,又相率強不食而為飢,薄衣而為寒,使面目陷陬,顏色黧黑耳目不聰明,手足不勁強,不可用也。又曰上士之操喪也,必扶而能起,杖而能行,以此共三年。若法若言,行若道使王公大人行此,則必不能蚤朝,五官六府,辟草木,實倉廩。使農夫行此。則必不能蚤出夜入,耕稼樹藝。使百工行此,則必不能修舟車為器皿矣。使婦人行此,則必不能夙興夜寐,紡績織紝。細計厚葬。為多埋賦之財者也。計久喪,為久禁從事者也。財以成者,扶而埋之;後得生者,而久禁之,以此求富,此譬猶禁耕而求穫也,富之說無可得焉。

是故求以富家而既已不可矣,欲以眾人民,意者可邪?其說又不可矣。今唯無以厚葬久喪者為政,君死,喪之三年;父母死,喪之三年;妻與後子死者,五皆喪之三年;然後伯父叔父兄弟孽子其;族人五月;姑姊甥舅皆有月數。則毀瘠必有制矣,使面目陷陬,顏色黧黑,耳目不聰明,手足不勁強,不可用也。又曰上士操喪也,必扶而能起,杖而能行,以此共三年。若法若言,行若道,苟其飢約,又若此矣,是故百姓冬不仞寒,夏不仞暑,作疾病死者,不可勝計也。此其為敗男女之交多矣。以此求眾,譬猶使人負劍,而求其壽也。眾之說無可得焉。

是故求以眾人民,而既以不可矣,欲以治刑政,意者可乎?其說又不可矣。今唯無以厚葬久喪者為政,國家必貧,人民必寡,刑政必亂。若法若言,行若道,使為上者行此,則不能聽治;使為下者行此,則不能從事。上不聽治,刑政必亂;下不行1從事,衣食之財必不足。若苟不足,為人弟者,求其兄而不得不弟弟必將怨其兄矣;為人子者,求其親而不得,不孝子必是怨其親矣;為人臣者,求之君而不得,不忠臣必且亂其上矣。是以僻淫邪行之民,出則無衣也,入則無食也,內續奚吾,並為淫暴,而不可勝禁也。是故盜賊眾而治者寡。夫眾盜賊而寡治者,以此求治,譬猶使人三還而毋負己也,治之說無可得焉。

是故求以治刑政,而既已不可矣,欲以禁止大國之攻小國也,意者可邪?其說又不可矣。是故昔者聖王既沒,天下失義,諸侯力征。南有楚、越之王,而北有齊、晉之君,此皆砥礪其卒伍,以攻伐并兼為政於天下。是故凡大國之所以不攻小國者,積委多,城郭修,上下調和,是故大國不耆攻之,無積委,城郭不修,上下不調和,是故大國耆攻之。今唯無以厚葬久喪者為政,國家必貧,人民必寡,刑政必亂。若苟貧,是無以為積委也;若苟寡,是城郭溝渠者寡也;若苟亂,是出戰不克,入守不固。

此求禁止大國之攻小國也,而既已不可矣。欲以干上帝鬼神之褔,意者可邪?其說又不可矣。今唯無以厚葬久喪者為政,國家必貧,人民必寡,刑政必亂。若苟貧,是粢盛酒醴不淨潔也;若苟寡,是事上帝鬼神者寡也;若苟亂,是祭祀不時度也。今又禁止事上帝鬼神,為政若此,上帝鬼神,始得從上撫之曰:『我有是人也,與無是人也,孰愈?』曰:『我有是人也,與無是人也,無擇也。』則惟上帝鬼神降之罪厲之禍罰而棄之,則豈不亦乃其所哉!

故古聖王制為葬埋之法,曰:『棺三寸,足以朽體;衣衾三領,足以覆惡。以及其葬也,下毋及泉,上毋通臭,壟若參耕之畝,則止矣。死則既以葬矣,生者必無久哭,而疾而從事,人為其所能,以交相利也。』此聖王之法也。」

今執厚葬久喪者之言曰:「厚葬久喪雖使不可以富貧眾寡,定危治亂,然此聖王之1道也。」子墨子曰:「不然。昔者堯北教乎八狄,道死,葬蛩山之陰,衣衾三領,榖木之棺,葛以緘之,既窆而後哭,滿埳無封。已葬,而牛馬乘之。舜西教乎七戎,道死,葬南己之市,衣衾三領,榖木之棺,葛以緘之,已葬,而市人乘之。禹東教乎九夷,道死,葬會稽之山,衣衾三領,桐棺三寸,葛以緘之,絞之不合,通之不埳,土地之深,下毋及泉,上毋通臭。既葬,收餘壤其上,壟若參耕之畝,則止矣。若以此若三聖王者觀之,則厚葬久喪果非聖王之道。故三王者,皆貴為天子,富有天下,豈憂財用之不足哉?以為如此葬埋之法。

今王公大人之為葬埋,則異於此。必大棺中棺,革闠三操,璧玉即具,戈劍鼎鼓壺濫,文繡素練,大鞅萬領,輿馬女樂皆具,曰必捶涂差通,壟雖凡山陵。此為輟民之事,靡民之財,不可勝計也,其為毋用若此矣。」是故子墨子曰:「鄉者,吾本言曰,意亦使法1其言,用其謀,計厚葬久喪,請可以富貧眾寡,定危治亂乎,則仁也,義也,孝子之事也,為人謀者,不可不勸也;意亦使法其言,用其謀,若人厚葬久喪,實不可以富貧眾寡,定危治亂乎,則非仁也,非義也,非孝子之事也,為人謀者,不可不沮也。是故求以富國家,甚得貧焉;欲以眾人民,甚得寡焉;欲以治刑政,甚得亂焉;求以禁止大國之攻小國也,而既已不可矣;欲以干上帝鬼神之福,又得禍焉。上稽之堯舜禹湯文武之道而政逆之,下稽之桀紂幽厲之事,猶合節也。若以此觀,則厚葬久喪其非聖王之道也。」

今執厚葬久喪者言曰:「厚葬久喪,果非聖王之道,夫胡說中國之君子,為而不已,操而不擇哉?」子墨子曰:「此所謂便其習而義其俗者也。昔者越之東有輆沐之國者,其長子生,則解而食之。謂之『宜弟』;其大父死,負其大母而棄之,曰鬼妻不可與居處。此上以為政,下以為俗,為而不已,操而不擇,則此豈實仁義之道哉?此所謂便其習而義其俗者也。楚之南有炎人國者,其親戚死朽其肉而棄之,然後埋其骨,乃成為孝子。秦之西有儀渠之國者,其親戚死,聚柴薪而焚之,燻上,謂之登遐,然後成為孝子。此上以為政,下以為俗,為而不已,操而不擇,則此豈實仁義之道哉?此所謂便其習而義其俗者也。若以此若三國者觀之,則亦猶薄矣。若以1中國之君子觀之,則亦猶厚矣。如彼則大厚,如此則大薄,然則葬埋之有節矣。故衣食者,人之生利也,然且猶尚有節;葬埋者,人之死利也,夫何獨無節於此乎。子墨子制為葬埋之法曰:「棺三寸,足以朽骨;衣三領,足以朽肉;掘地之深,下無菹漏,氣無發洩於上,壟足以期其所,則止矣。哭往哭來,反從事乎衣食之財,佴乎祭祀,以致孝於親。故曰子墨子之法,不失死生之利者,此也。

故子墨子言曰:「今天下之士君子,中請1將欲為仁義,求為上士,上欲中聖王之道,下欲中國家百姓之利,故當若節喪之為政,而不可不察此者也。」


\end{pinyinscope}