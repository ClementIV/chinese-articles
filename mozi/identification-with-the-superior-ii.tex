\article{尚同中}

\begin{pinyinscope}
子墨子曰:「方今之時,復古之民始生,未有正長之時,蓋其語曰『天下之人異義』。是以一人一義,十人十義,百人百義,其人數茲眾,其所謂義者亦茲眾。是以人是其義,而非人之義,故相交非也。內之父子兄弟作怨讎,皆有離散之心,不能相和合。至乎舍餘力不以相勞,隱匿良道不以相教,腐臭餘財不以相分,天下之亂也,至如禽獸然,無君臣上下長幼之節,父子兄弟之禮,是以天下亂焉。

明乎民之無正長以一同天下之義,而天下亂也。是故選擇天下賢良聖知辯慧之人,立以為天子,使從事乎一同天下之義。天子既以立矣,以為唯其耳目之請,不能獨一同天下之義,是故選擇天下贊閱賢良聖知辯慧之人,置以為三公,與從事乎一同天下之義。天子三公既已立矣,以為天下博大,山林遠土之民,不可得而一也,是故靡分天下,設以為萬諸侯國君,使從事乎一同其國之義。國君既已立矣,又以為唯其耳目之請,不能一同其國之義,是故擇其國之賢者,置以為左右將軍大夫,以遠至乎鄉里之長與從事乎一同其國之義。

天子諸侯之君,民之正長,既已定矣,天子為發政施教曰:『凡聞見善者,必以告其上,聞見不善者,亦必以告其上。上之所是,必亦是之,上之所非,必亦非之,已有善傍薦之,上有過規諫之。尚同義其上,而毋有下比之心,上得則賞之,萬民聞則譽之。意若聞見善,不以告其上,聞見不善,亦不以告其上,上之所是不能是,上之所非不能非,己有善不能傍薦之,上有過不能規諫之,下比而非其上者,上得則誅罰之,萬民聞則非毀之』。故古者聖王之為刑政賞譽也,甚明察以審信。

是以舉天下之人,皆欲得上之賞譽,而畏上之毀罰。是故里長順天子政,而一同其里之義。里長既同其里之義,率其里之萬民,以尚同乎鄉長,曰:『凡里之萬民,皆尚同乎鄉長,而不敢下比。鄉長之所是,必亦是之,鄉長之所非,必亦非之。去而不善言,學鄉長之善言;去而不善行,學鄉長之善行。鄉長固鄉之賢者也,舉鄉人以法鄉長,夫鄉何說而不治哉?』察鄉長之所以治鄉者何故之以也?曰唯以其能一同其鄉之義,是以鄉治。

鄉長治1其鄉,而鄉既已治矣,有率其鄉萬民,以尚同乎國君,曰:『凡鄉之萬民,皆上同乎國君,而不敢下比。國君之所是,必亦是之,國君之所非,必亦非之。去而不善言,學國君之善言;去而不善行,學國君之善行。國君固國之賢者也,舉國人以法國君,夫國何說而不治哉?』察國君之所以治國,而國治者,何故之以也?曰唯以其能一同其國之義,是以國治。

國君治其國,而國1既已治矣,有率其國之萬民,以尚同乎天子,曰:『凡國之萬民上同乎天子,而不敢下比。天子之所是,必亦是之,天子之所非,必亦非之。去而不善言,學天子之善言;去而不善行,學天子之善行。天子者,固天下之仁人也,舉天下之萬民以法天子,夫天下何說而不治哉?』察天子之所以治天下者,何故之以也?曰唯以其能一同天下之義,是以天下治。

夫既尚同乎天子,而未上同乎天者,則天菑將猶未止也。故當若天降寒熱不節,雪霜雨露不時,五穀不孰,六畜不遂,疾菑戾疫、飄風苦雨,荐臻而至者,此天之降罰也,將以罰下人之不尚同乎天者也。故古者聖王,明天鬼之所欲,而避天鬼之所憎,以求興天下之利,除天下之害。是以率天下之萬民,齊戒沐浴,潔為酒醴粢盛,以祭祀天鬼。其事鬼神也,酒醴粢盛不敢不蠲潔,犧牲不敢不腯肥,珪璧幣帛不敢不中度量,春秋祭祀不敢失時幾,聽獄不敢不中,分財不敢不均,居處不敢怠慢。曰其為正長若此,是故天鬼之福可得也。萬民之所便利而能彊從事焉,則萬民之親可得也。其為政若此,是以謀事,舉事成,入守固,1上者天鬼有厚乎其為政長也,下者萬民有便利乎其為政長也。天鬼之所深厚而彊從事焉,則2出誅勝者,何故之以也?曰唯以尚同為政者也。故古者聖王之為政若此。」

今天下之人曰:「方今之時,天鬼之福可得也。萬民之所便利而能彊從事焉,則萬民之親可得也。其為政若此,是以謀事,舉事成,入守固,1上者天鬼有厚乎其為政長也,下者萬民有便利乎其為政長也。天鬼之所深厚而彊從事焉,則2天下之正長猶未廢乎天下也,而天下之所以亂者,何故之以也?」子墨子曰:「方今之時之以正長,則本與古者異矣,譬之若有苗之以五刑然。昔者聖王制為五刑,以治天下,逮至有苗之制五刑,以亂天下。則此豈刑不善哉?用刑則不善也。是以先王之書呂刑之道曰:『苗民否用練折則刑,唯作五殺之刑,曰法。』則此言善用刑者以治民,不善用刑者以為五殺,則此豈刑不善哉?用刑則不善。故遂以為五殺。是以先王之書術令之道曰:『唯口出好興戎。』則此言善用口者出好,不善用口者以為讒賊寇戎。則此豈口不善哉?用口則不善也,故遂以為讒賊寇戎。

故古者之置正長也,將以治民也,譬之若絲縷之有紀,而罔罟之有綱也,將以運役天下淫暴,而一同其義也。是以先王之書,《相年》之道曰:「夫建國設都,乃作后王君公,否用泰也,輕大夫師長,否用佚也,維辯使治天均。」則此語古者上帝鬼神之建設國都,立正長也,非高其爵,厚其祿,富貴佚而錯之也,將以為萬民興利除害,富貴貧寡,安危治亂也。故古者聖王之為若此。

今王公大人之為刑政則反此。政以為便譬,宗於父兄故舊,以為左右,置以為正長。民知上置正長之非正以治民也,是以皆比周隱匿,而莫肯尚同其上。是故上下不同義。若苟上下不同義,賞譽不足以勸善,而刑罰不足以沮暴。何以知其然也?

曰:上唯毋立而為政乎國家,為民正長,曰:「人可賞吾,將賞之。」若苟上下不同義,上之所賞,則眾之所非,曰人眾與處,於眾得非。則是雖使得上之賞,未足以勸乎!上唯毋立而為政乎國家,為民正長,曰:「人可罰,吾將罰之。」若苟上下不同義,上之所罰,則眾之所譽。曰人眾與處,於眾得譽,則是雖使得上之罰,未足以沮乎!若立而為政乎國家,為民正長,賞譽不足以勸善,而刑罰不可以沮暴,則是不與鄉吾本言「民始生未有正長之時」同乎!若有正長與無正長之時同,則此非所以治民一眾之道。

故古者聖王唯而審1以尚同,以為正長,是故2上下情請為通。上有隱事遺利,下得而利之;下有蓄怨積害,上得而除之。是以數千萬里之外,有為善者,其室人未遍知,鄉里未遍聞,天子得而賞之。數千萬里之外,有為不善者,其室人未遍知,鄉里未遍聞,天子得而罰之。是以舉天下之人皆恐懼振動惕慄,不敢為淫暴,曰:「天子之視聽也神。」先王之言曰:「非神也,夫唯能使人之耳目助己視聽,使人之吻助己言談,使人之心助己思慮,使人之股肱助己動作」。助之視聽者眾,則其所聞見者遠矣;助之言談者眾,則其德音之所撫循者博矣;助之思慮者眾,則其談謀度速得矣;助之動作者眾,即其舉事速成矣。故古者聖人之所以濟事成功,垂名於後世者,無他故異物焉,曰唯能以尚同為政者也。

是以先王之書《周頌》之道之曰:「載來見彼王,聿求厥章。」則此語古者國君諸侯之以春秋來朝聘天子之廷,受天子之嚴教,退而治國,政之所加,莫敢不賓。當此之時,本無有敢紛天子之教者。《詩》曰:「我馬維駱,六轡沃若,載馳載驅,周爰咨度。」又曰:「我馬維騏,六轡若絲,載馳載驅,周爰咨謀。」即此語也。古者國君諸侯之聞見善與不善也,皆馳驅以告天子,是以賞當賢,罰當暴,不殺不辜,不失有罪,則此尚同之功也。」

是故子墨子曰:「今天下之王公大人士君子,請將欲富其國家,眾其人民,治其刑政,定其社稷,當若尚同之不可不察,此之本也。」


\end{pinyinscope}