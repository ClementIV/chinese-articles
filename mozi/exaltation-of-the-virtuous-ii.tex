\article{尚賢中}

\begin{pinyinscope}
子墨子言曰:「今王公大人之君人民,主社稷,治國家,欲脩保而勿失,故不察尚賢為政之本也。何以知尚賢之為政本也?曰自貴且智者,為政乎愚且賤者,則治;自愚且賤者,為政乎貴且智者,則亂。是以知尚賢之為政本也。故古者聖王甚尊尚賢而任使能,不黨父兄,不偏貴富,不嬖顏色,賢者舉而上之,富而貴之,以為官長;不肖者抑而廢之,貧而賤之以為徒役,是以民皆勸其賞,畏其罰,相率而為賢。者以賢者眾,而不肖者寡,此謂進賢。然後聖人聽其言,跡其行,察其所能,而慎予官,此謂事能。故可使治國者,使治國,可使長官者,使長官,可使治邑者,使治邑。凡所使治國家,官府,邑里,此皆國之賢者也。

賢者之治國者也,蚤朝晏退,聽獄治政,是以國家治而刑法正。賢者之長官也,夜寢夙興,收斂關市、山林、澤梁之利,以實官府,是以官府實而財不散。賢者之治邑也,蚤出莫入,耕稼、樹藝、聚菽粟,是以菽粟多而民足乎食。故國家治則刑法正,官府實則萬民富。上有以絜為酒醴栥盛,以祭祀天鬼;外有以為皮幣,與四鄰諸侯交接,內有以食飢息勞,將養其萬民。外有以懷天下之賢人。是故上者天鬼富之,外者諸侯與之,內者萬民親之,賢人歸之,以此謀事則得,舉事則成,入守則固,出誅則彊。故唯昔三代聖王堯、舜、禹、湯、文、武,之所以王天下正諸侯者,此亦其法已。

既曰若法,未知所以行之術,則事猶若未成,是以必為置三本。何謂三本?曰爵位不高則民不敬也,蓄祿不厚則民不信也,政令不斷則民不畏也。故古聖王高予之爵,重予之祿,任之以事,斷予之令,夫豈為其臣賜哉,欲其事之成也。《詩》曰:『告女憂卹,誨女予爵,孰能執熱,鮮不用濯。』則此語古者國君諸侯之不可以不執善,承嗣輔佐也。譬之猶執熱之有濯也。將休其手焉。古者聖王唯毋得賢人而使之,般爵以貴之,裂地以封之,終身不厭。賢人唯毋得明君而事之,竭四肢之力以任君之事,終身不倦。若有美善則歸之上,是以美善在上,而所怨謗在下,寧樂在君,憂慼在臣,故古者聖王之為政若此。

今王公大人亦欲效人以尚賢使能為政,高予之爵,而祿不從也。夫高爵而無祿,民不信也。曰:『此非中實愛我也,假藉而用我也。』夫假藉之民,將豈能親其上哉!故先王言曰:『貪於政者「不能分人以事,厚於貨者不能分人以祿。」事則不與,祲則不分,請問天下之賢人將何自至乎王公大人之側哉?若苟賢者不至乎王公大人之側,則此不肖者在左右也。不肖者在左右,則其所譽不當賢,而所罰不當暴,王公大人尊此以為政乎國家,則賞亦必不當賢,而罰亦必不當暴。若苟賞不當賢而罰不當暴,則是為賢者不勸而為暴者不沮矣。是以入則不慈孝父母,出則不長弟鄉里,居處無節,出入無度,男女無別。使治官府則盜竊,守城則倍畔,君有難則不死,出亡則不從,使斷獄則不中,分財則不均,與謀事不得,舉事不成,入守不固,出誅不彊。故雖昔者三代暴王桀紂幽厲之所以失措其國家,傾覆其社稷者,已此故也。何則?皆以明小物而不明大物也。

今王公大人,有一衣裳不能制也,必藉良工;有一牛羊不能殺也,必藉良宰。故當若之二物者,王公大人未知以尚賢使能為政也。逮至其國家之亂,社稷之危,則不知使能以治之,親戚則使之,無故富貴、面目佼好則使之。夫無故富貴、面目佼好則使之,豈必智且有慧哉!若使之治國家,則此使不智慧者治國家也,國家之亂既可得而知已。且夫王公大人有所愛其色而使,其心不察其知而與其愛。是故不能治百人者,使處乎千人之官,不能治千人者,使處乎萬人之官。此其故何也?曰處若官者爵高而祿厚,故愛其色而使之焉。夫不能治千人者,使處乎萬人之官,則此官什倍也。夫治之法將日至者也,日以治之,日不什脩,知以治之,知不什益,而予官什倍,則此治一而棄其九矣。雖日夜相接以治若官,官猶若不治,此其故何也?則王公大人不明乎以尚賢使能為政也。故以尚賢使能為政而治者,夫若言之謂也,以下賢為政而亂者,若吾言之謂也。

今王公大人中實將欲治其國家,欲脩保而勿失,胡不察尚賢為政之本也?且以尚賢為政之本者,亦豈獨子墨子之言哉!此聖王之道,先王之書距年之言也。傳曰:『求聖君哲人,以裨輔而身』,《湯誓》云:『聿求元聖,與之戮力同心,以治天下。』則此言聖之不失以尚賢使能為政也。故古者聖王唯能審以尚賢使能為政,無異物雜焉,天下皆得其利。古者舜耕歷山,陶河瀕,漁雷澤,堯得之服澤之陽,舉以為天子,與接天下之政,治天下之民。伊摯,有莘氏女之私臣,親為庖人,湯得之,舉以為己相,與接天下之政,治天下之民。傅說被褐帶索。庸築乎傅巖,武丁得之,舉以為三公,與接天下之政,治天下之民。此何故始賤卒而貴,始貧卒而富?則王公大人明乎以尚賢使能為政。是以民無飢而不得食,寒而不得衣,勞而不得息,亂而不得治者。

故古聖王以審以尚賢使能為政,而取法於天。雖天亦不辯貧富、貴賤、遠邇、親疏、賢者舉而尚之,不肖者抑而廢之。然則富貴為賢,以得其賞者誰也?曰若昔者三代聖王堯、舜、禹、湯、文、武者是也。所以得其賞何也?曰其為政乎天下也,兼而愛之,從而利之,又率天下之萬民以尚尊天、事鬼、愛利萬民,是故天鬼賞之,立為天子,以為民父母,萬民從而譽之曰『聖王』,至今不已。則此富貴為賢,以得其賞者也。然則富貴為暴,以得其罰者誰也?曰若昔者三代暴王桀、紂、幽、厲者是也。何以知其然也?曰其為政乎天下也,兼而憎之,從而賊之,又率天下之民以詬天侮鬼,賊傲萬民,是故天鬼罰之,使身死而為刑戮,子孫離散,室家喪滅,絕無後嗣,萬民從而非之曰「暴王」,至今不已。則此富貴為暴,而以得其罰者也。然則親而不善,以得其罰者誰也?曰若昔者伯鯀,帝之元子,廢帝之德庸,既乃刑之于羽之郊,乃熱照無有及也,帝亦不愛。則此親而不善以得其罰者也。然則天之所使能者誰也?曰若昔者禹、稷、皋陶是也。何以知其然也?先王之書呂刑道之曰:『皇帝清問下民,有辭有苗。曰群后之肆在下,明明不常,鰥寡不蓋,德威維威,德明維明。乃名三后,恤功於民,伯夷降典,哲民維刑。禹平水土,主名山川。稷隆播種,農殖嘉穀。三后成功,維假於民。』則此言三聖人者,謹其言,慎其行,精其思慮,索天下之隱事遺利,以上事天,則天鄉其德,下施之萬民,萬民被其利,終身無已。故先王之言曰:『此道也,大用之天下則不窕,小用之則不困,脩用之則萬民被其利,終身無已。』周頌道之曰:『聖人之德,若天之高,若地之普,其有昭於天下也。若地之固,若山之承,不坼不崩。若日之光,若月之明,與天地同常。』則此言聖人之德,章明博大,埴固,以脩久也。故聖人之德蓋總乎天地者也。

今王公大人欲王天下,正諸侯,夫無德義將何以哉?其說將必挾震威彊。今王公大人將焉取挾震威彊哉?傾者民之死也。民生為甚欲,死為甚憎,所欲不得而所僧屢至,自古及今未嘗能有以此王天下、正諸侯者也。今大人欲王天下,正諸侯,將欲使意得乎天下,名成乎後世,故不察尚賢為1政之本也。此聖人之厚行也。」


\end{pinyinscope}