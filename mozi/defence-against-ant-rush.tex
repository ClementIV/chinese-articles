\article{備蛾傅}

\begin{pinyinscope}
禽子再拜再拜曰:「敢問適人強弱,遂以傅城,後上先斷,以為法程,斬城為基,掘下為室,前上不止,後射既疾,為之柰何?」

子墨子曰:「子問蛾傅之守邪?蛾傅者,將之忿者也。守為行臨射之,校機藉之,擢之,太氾迫之,燒荅覆之,沙石雨之,然則蛾傅之攻敗矣。備蛾傅為縣脾,以木板厚二寸,前後三尺,旁廣五尺,高五尺,而折為下磨車,轉徑尺六寸。令一人操二丈四方,刃其兩端,居縣脾中,以鐵璅敷縣二脾上衡,為之機,令有力四人下上之,弗離。施縣脾,大數二十步一,攻隊所在六步一。

為纍荅廣從丈各二尺,以木為上衡,以麻索大編之,染其索塗中,為鐵璅,鉤其兩端之縣。客則蛾傅城,燒荅以覆之,連梃,抄大皆救之。以車兩走,軸閒廣大以圉,犯之。刺其兩端。以束輪,遍編塗其上。室中以榆若蒸,以棘為旁,命曰火捽,一曰傳湯,以當隊。客則乘隊,燒傳湯,斬維而下之,令勇士隨而擊之,以為勇士前行,城上輒塞壞城。

城下足為下說鑱杙,長五尺,大圉半以上,皆剡其末,為五行,行閒廣三尺,貍三尺,大耳樹之。為連殳,長五尺,大十尺。梃長二尺,大六寸,索長二尺。椎,柄長六尺,首長尺五寸。斧,柄長六尺,刃必利,皆築其一後。荅廣丈二尺,其長丈六尺,垂前衡四寸,兩端接尺相覆,勿令魚鱗槮,著其後行。中央木繩一,長二丈六尺,荅樓不會者以牒塞,數暴乾,荅為格,令風上下。堞惡疑壞者,先貍木十尺一枚一,節壞,鄧植以押慮盧薄於木,盧薄表八尺,廣七寸,經尺一,數施一擊而下之,為上下釫而斫之。

經一鈞、禾樓、羅石、縣荅,植內毋植外。

杜格,貍四尺,高者十丈,木長短相雜,兌其上,而外內厚塗之。

為前行行棧、縣荅。隅為樓,樓必曲裡。土五步一,毋其二十畾。雀穴十尺一,下堞三尺,廣其外。轉傅城上,樓及散與池革盆。若轉,攻卒擊其後,煖失治。車革火。

凡殺蛾傅而攻者之法,置薄城外,去城十尺,薄厚十尺。伐操之法,大小盡木斷之,以十尺為斷,離而深貍堅築之,毋使可拔。

二十步一殺,有鬲,厚十尺。殺有兩門,門廣五步,薄門板梯貍之,勿築,令易拔。城上希薄門而置搗。

縣火,四尺一椅,五步一灶,灶門有爐炭。傳令敵人盡入,車火燒門,縣火次之,出載而立,其廣終隊,兩載之間一火,皆立而待鼓音而然,即俱發之。敵人辟火而復攻,縣火復下,敵人甚病。

敵引哭而榆,則令吾死士左右出穴門擊遺師,令賁士、主將皆聽城鼓之音而出,又聽城鼓之音而入。因素出兵將施伏,夜半,而城上四面鼓噪,敵人必或,破軍殺將。以白衣為服,以號相得。


\end{pinyinscope}