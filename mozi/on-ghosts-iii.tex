\article{明鬼下}

\begin{pinyinscope}
子墨子言曰:「逮至昔三代聖王既沒,天下失義,諸侯力正,是以存夫為人君臣上下者之不惠忠也,父子弟兄之不慈孝弟長貞良也,正長之不強於聽治,賤人之不強於從事也,民之為淫暴寇亂1盜賊,以兵刃毒藥水火,退無罪人乎道路率徑,奪人車馬衣裘以自利者並作,由此始,是以天下亂。此其故何以然也?則皆以疑惑鬼神之有與無之別,不明乎鬼神之能賞賢而罰暴也。今若使天下之人,偕若信鬼神之能賞賢而罰暴也,則夫天下豈亂哉!」

今執無鬼者曰:「鬼神者,固無有。」旦暮以為教誨乎天下,之1疑天下之眾,使天下之眾皆疑惑乎鬼神有無之別,是以天下亂。是故子墨子曰:「今天下之王公大人士君子,實將欲求興天下之利,除天下之害,故當鬼神之有與無之別,以為將不可以不2明察此者也。既以鬼神有無之別,以為不可不察已。」

然則吾為明察此,其說將柰何而可?子墨子曰:「是與天下之所以察知有與無之道者,必以眾之耳目之實知有與亡為儀者也,請惑聞之見之,則必以為有,莫聞莫見,則必以為無1。若是,何不嘗入一鄉一里而問之,自古以及今,生民以來者,亦有嘗見鬼神之物,聞鬼神之聲,則鬼神何謂無乎?若莫聞莫見,則鬼神可謂有乎?」

今執無鬼者言曰:「夫天下之為聞見鬼神之物者,不可勝計也,亦孰為聞見鬼神有無之物哉?」子墨子言1曰:「若以眾之所同見,與眾之所同聞,則若昔者杜伯是也。周宣王殺其臣杜伯而不辜,杜伯曰:『吾君殺我而不辜,若以死者為無知則止矣;若死而有知,不出三年,必使吾君知之。』其三年,周宣王合諸侯而田於圃,田車數百乘,從數千,人滿野。日中,杜伯乘白馬素車,朱衣冠,執朱弓,挾朱矢,追周宣王,射之車上,中心折脊,殪車中,伏弢而死。當是之時,周人從者莫不見,遠者莫不聞,著在周之《春秋》。為君者以教其臣,為父者以䜘其子,曰:『戒之慎之!凡殺不辜者,其得不祥,鬼神之誅,若此之憯遫也2!以若書之說觀之,則鬼神之有,豈可疑哉?

非惟若書之說為然也1,昔者鄭穆公,當晝日中處乎廟,有神入門而左,鳥身,素服三絕,面狀正方。鄭穆公見之,乃恐懼奔,神曰:『無懼!2帝享女明德,使予錫女壽十年有九,使若國家蕃昌,子孫茂,毋失。鄭穆公再拜稽首曰:『敢問神名3?』曰:『予為句芒。』若以鄭穆公之所身見為儀,則鬼神之有,豈可疑哉?

非惟若書之說為然也,昔者,燕簡公殺其臣莊子儀而不辜,莊子儀曰:『吾君王殺我而不辜,死人毋知亦已,死人有知,不出三年,必使吾君知之』。期年,燕將馳祖,燕之有祖,當齊之社稷,宋之有桑林,楚之有雲夢也,此男女之所屬而觀也。日中,燕簡公方將馳於祖塗,莊子儀荷朱杖而擊之,殪之車上。當是時,燕人從者莫不見,遠者莫不聞,著在燕之春秋。諸侯傳而語之曰『凡殺不辜者,其得不祥,鬼神之誅,若此其憯遫也!』以若書之說觀之,則鬼神之有,豈可疑哉?

非惟若書之說為然也,昔者,宋文君鮑之時,有臣曰𥙐觀辜,固嘗從事於厲,祩子杖揖出與言曰:『觀辜是何珪璧之不滿度量?酒醴粢盛之不淨潔也?犧牲之不全肥?春秋冬夏「選」失時?豈女為之與?意鮑為之與?』觀辜曰:『鮑幼弱在荷繈之中,鮑何與識焉。官臣觀辜特為之』。祩子舉揖而槁之,殪之壇上。當是時1,宋人從者莫不見,遠者莫不聞,著在宋之春秋。諸侯傳而語之曰:『諸不敬慎祭祀者,鬼神之誅,至若此其憯遫也2!』以若書之說觀之,鬼神之有,豈可疑哉?

非惟若書之說為然也。昔者,齊莊君之臣1有所謂王里國、中里徼者,此二子者,訟三年而獄不斷。齊君由謙殺之恐不辜,猶謙釋之。恐失有罪,乃使之人共一羊,盟齊之神社,二子許諾。於是泏洫𢵣羊而漉其血,讀王里國之辭既已終矣,讀中里徼之辭未半也,羊起而觸之,折其腳,祧神之而槁之,殪之盟所。當是時,齊人從者莫不見,遠者莫不聞,著在齊之春秋。諸侯傳而語之曰:『請品先不以其請者,鬼神之誅,至若此其憯遫也。』以若書之說觀之,鬼神之有,豈可疑哉?」

是故子墨子言曰:「雖有深谿博林,幽澗毋人之所,施行不可以不董,見有鬼神視之」。

今執無鬼者曰:「夫眾人耳目之請,豈足以斷疑哉?柰何其欲為高君子於天下,而有復信眾之耳目之請哉?」子墨子1曰:若以眾之耳目之請,以為不足信也,不以斷疑。不識若昔者三代聖王堯舜禹湯文武者,足以為法乎?故於此乎,自中人以上皆曰:若昔者三代聖王,足以為法矣。若苟昔者三代聖王足以為法,然則姑嘗上觀聖王之事。昔者,武王之攻殷誅紂也,使諸侯分其祭曰:『使親者受內祀,疏者受外祀。』故武王必以鬼神為有,是故攻殷伐紂,使諸侯分其祭。若鬼神無有,則武王何祭分哉?

非惟武王之事為然也,故聖王其賞也必於祖,其僇也必於社。賞於祖者何也?告分之均也;僇於社者何也?告聽之中也。非惟若書之說為然也,且惟昔者虞夏、商、周三代之聖王,其始建國營都日,必擇國之正壇,置以為宗廟;必擇木之脩茂者,立以為菆位;必擇國之父兄慈孝貞良者,以為祝宗;必擇六畜之勝腯肥倅,毛以為犧牲;珪璧琮璜,稱財為度;必擇五穀之芳黃,以為酒醴粢盛,故酒醴粢盛,與歲上下也。故古聖王治天下也,故必先鬼神而後人者此也。故曰官府選效,必先祭器祭服,畢藏於府,祝宗有司,畢立於朝,犧牲不與昔聚群。故古者聖王之為政若此。

古者聖王必以鬼神為,其務鬼神厚矣,又恐後世子孫不能知也,故書之竹帛,傳遺後世子孫;咸恐其腐蠹絕滅,後世子孫不得而記,故琢之盤盂,鏤之金石,以重之;有恐後世子孫不能敬莙以取羊,故先王之書,聖人一尺之帛,一篇之書,語數鬼神之有也,重有重之。此其故何?則聖王務之。今執無鬼者曰:『鬼神者,固無有。』則此反聖王之務。反聖王之務,則非所以為君子之道也!」

今執無鬼者之言曰:「先王之書,慎無一尺之帛,一篇之書,語數鬼神之有,重有重之1,亦何書之亦何書2有之3哉?」子墨子曰:「《周書》、《大雅》有之,《大雅》曰:『文王在上,於昭于天,周雖舊邦,其命維新。有周不顯,帝命不時。文王陟降,在帝左右。穆穆文王,令問不已』。若鬼神無有,則文王既死,彼豈能在帝之左右哉?此吾所以知《周書》之鬼也。

且《周書》獨鬼,而《商書》不鬼,則未足以為法也。然則姑嘗上觀乎商書,曰:『嗚呼!古者有夏,方未有禍之時,百獸貞蟲,允及飛鳥,莫不比方。矧隹人面,胡敢異心?山川鬼神,亦莫敢不寧。若能共允,隹天下之合,下土之葆』。察山川鬼神之所以莫敢不寧者,以佐謀禹也。此吾所以知商書之鬼也。

且商書獨鬼,而夏書不鬼,則未足以為法也。然則姑嘗上觀乎夏書禹誓曰:『大戰于甘,王乃命左右六人,下聽誓于中軍,曰:「有扈氏威侮五行,怠棄三正,天用劋絕其命。」有曰:「日中。今予與有扈氏爭一日之命。且爾卿大夫庶人,予非爾田野葆士之欲也,予共行天之罰也。左不共于左,右不共于右,若不共命,御非爾馬之政,若不共命」』。是以賞于祖而僇于社。賞于祖者何也?言分命之均也。僇于社者何也?言聽獄之事也。故古聖王必以鬼神為賞賢而罰暴,是故賞必於祖而僇必於社。此吾所以知夏書之鬼也。故尚者夏書,其次商周之書,語數鬼神之有也,重有重之,此其故何也?則聖王務之。以若書之說觀之,則鬼神之有,豈可疑哉?於古曰:『吉日丁卯,周代祝社方,歲於社者1考,以延年壽』。若無鬼神,彼豈有所延年壽哉!」

是故子墨子曰:「嘗若鬼神之能賞賢如罰暴也。蓋本施之國家,施之萬民,實所以治國家利萬民之道也。若以為不然,是以吏治官府之不絜廉,男女之為無別者,鬼神見之;民之為淫暴寇亂盜賊,以兵刃毒藥水火,退無罪人乎道路,奪人車馬衣裘以自利者,有鬼神見之。是以吏治官府,不敢不絜廉,見善不敢不賞,見暴不敢不罪。民之為淫暴寇亂盜賊,以兵刃毒藥水火,退無罪人乎道路,奪車馬衣裘以自利者,由此止。是以莫放幽閒,擬乎鬼神之明顯,明有一人畏上誅罰,是以天下治。

故鬼神之明,不可為幽閒廣澤,山林深谷,鬼神之明必知之。鬼神之罰,不可為1富貴眾強,勇力強武,堅甲利兵,鬼神之罰必勝之。若以為不然,昔者夏王桀,貴為天子,富有天下,上詬天侮鬼,下殃傲天下之萬民,祥上帝伐元山帝行,故於此乎,天乃使湯至明罰焉。湯以車九兩,鳥陳鴈行,湯乘大贊,犯遂夏眾,入之郊逐,王乎禽推哆大戲。故昔夏王桀,貴為天子,富有天下,有勇力2之人3推哆大戲,生列兕虎,指畫殺人,人民之眾兆億,侯盈厥澤陵,然不能以此圉鬼神之誅。此吾所謂鬼神之罰,不可為富貴眾強、勇力強武、堅甲利兵者,此也。

且不惟此為然。昔者殷王紂,貴為天子,富有天下,上詬天侮鬼,下殃傲天下之萬民,播棄黎老,賊誅孩子,楚毒無罪,刲剔孕婦,庶舊鰥寡,號咷無告也。故於此乎,天乃使武王至明罰焉。武王以擇車百兩,虎賁之卒四百人,先庶國節窺戎,與殷人戰乎牧之野,王乎禽費中、惡來,眾畔百走。武王逐奔入宮,萬年梓株折紂而繫之赤環,載之白旗,以為天下諸侯僇。故昔者殷王紂,貴為天子,富有天下,有勇力之人費中、惡來、崇侯虎指寡殺人,人民之眾兆億,侯盈厥澤陵,然不能以此圉鬼神之誅。此吾所謂鬼神之罰,不可為富貴眾強、勇力強武、堅甲利兵者,此也。且禽艾之道之曰:『得璣無小,滅宗無大』。則此言鬼神之所賞,無小必賞之;鬼神之所罰,無大必罰之」。

今執無鬼者曰:「意不忠親之利,而害為孝子乎?」子墨子曰:「古之今之為鬼,非他也,有天鬼,亦有山水鬼神者,亦有人死而為鬼者。今有子先其父死,弟先其兄死者矣,意雖使然,然而天下之陳物曰『先生者先死』,若是,則先死者非父則母,非兄而姒也。今絜為酒醴粢盛,以敬慎祭祀,若使鬼神請有,是得其父母姒兄而飲食之也,豈非厚利哉?若使鬼神請亡,是乃費其所為酒醴粢盛之財耳。自夫費之,非1特注之汙壑而棄之也,內者宗族,外者鄉里,皆得如具飲食之。雖使鬼神請亡,此猶可以合驩聚眾,取親於鄉里。」今執無鬼者言曰:「鬼神者固請無有,是以不共其酒醴粢盛犧牲之財。吾非乃今愛其酒醴粢盛犧牲之財乎?其所得者臣將何哉?」此上逆聖王之書,內逆民人孝子之行,而為上士於天下,此非所以為上士之2道也3。是故子墨子曰:「今吾為祭祀也,非直注之汙壑而棄之也,上以交鬼之福,下以合驩聚眾,取親乎鄉里。若神有,則是得吾父母弟兄而食之也。則此豈非天下利事也哉!」

是故子墨子曰:「今天下之王公大人士君子,中實將欲求興天下之利,除天下之害,當若鬼神之有也,將不可不尊明也,聖王之道也」。


\end{pinyinscope}