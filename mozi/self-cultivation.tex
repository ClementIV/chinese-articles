\article{修身}

\begin{pinyinscope}
君子戰雖有陳,而勇為本焉。喪雖有禮,而哀為本焉。士雖有學,而行為本焉。是故置本不安者,無務豐末。近者不親,無務來遠。親戚不附,無務外交。事無終始,無務多業。舉物而闇,無務博1聞。

是故先王之治天下也,必察邇來遠,君子察邇而邇脩者也。見不脩行,見毀,而反之身者也,此以怨省而行脩矣。譖慝之言,無入之耳,批扞之聲,無出之口,殺傷人之孩,無存之心,雖有詆訐之民,無所依矣。

故君子力事日彊,願欲日逾,設壯日盛。君子之道也,貧則見廉,富則見義,生則見愛,死則見哀。四行者不可虛假,反之身者也。藏於心者,無以竭愛。動於身者,無以竭恭。出於口者,無以竭馴。暢之四支,接之肌膚,華髮隳顛1,而猶弗舍者,其唯聖人乎!

志不彊者智不達,言不信者行不果。據財不能以分人者,不足與友。守道不篤,偏物不博,辯是非不察者,不足與游。本不固者末必幾,雄而不脩者,其後必惰,原濁者流不清,行不信者名必秏1。名不徒生而譽不自長,功成名遂,名譽不可虛假,反之身者也。務言而緩行,雖辯必不聽。多力而伐功,雖勞必不圖。慧者心辯而不繁說,多力而不伐功,此以名譽揚天下。言無務為多而務為智,無務為文而務為察。故彼智無察,在身而情,反其路者也。善無主於心者不留,行莫辯於身者不立。名不可簡而成也,譽不可巧而立也,君子以身戴行者也。思利尋焉,忘名忽焉,可以為士於天下者,未嘗有也。


\end{pinyinscope}