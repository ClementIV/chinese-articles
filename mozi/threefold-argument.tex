\article{三辯}

\begin{pinyinscope}
程繁問於子墨子曰:「夫子曰:1『聖王不為樂』,昔諸侯倦於聽治,息於鐘鼓之樂;士大夫倦於聽治,息於竽瑟之樂;農夫春耕、夏耘、秋斂、冬藏,息於瓴2缶之樂。今夫子曰:『聖王不為樂』,此譬之猶馬駕而不稅,弓張而不弛,無乃非有血氣者之所不3能至邪?」

子墨子曰:「昔者堯舜有茅茨者,且以為禮,且以為樂。湯放桀於大水,環天下自立以為王,事成功立,無大後患,因先王之樂,又1自作樂,命曰《護》,又脩2《九招》。武王勝殷殺紂,環天下自立以為王,事成功立,無大後患,因先王之樂,又自作樂,命曰《象》。周成王因先王3之樂,又自作樂,4命曰《騶虞》。周成王之治天下也,不若武王。武王之治天下也,不若成湯。成湯之治天下也,不若堯舜。故其樂逾繁者,其治逾寡。自此觀之,樂非所以治天下也。」

程繁曰:「子曰:『聖王無樂』。此亦樂已,若之何其謂聖王無樂也?」子墨子曰:「聖王之命也,多寡之。食之利也,以知饑而食之者智也,因為無智1矣。今聖有樂而少,此亦無也。」


\end{pinyinscope}