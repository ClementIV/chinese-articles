\article{兼愛上}

\begin{pinyinscope}
聖人以治天下為事者也,必知亂之所自起,焉能治之,不知亂之所自起,則不能治。譬之如醫之攻人之疾者然,必知疾之所自起,焉能攻之;不知疾之所自起,則弗能攻。治亂者何獨不然,必知亂之所自起,焉能治之;不知亂之所自起,則弗能治。聖人以治天下為事者也,不可不察亂之所自起。

當察亂何自起?起不相愛。臣子之不孝君父,所謂亂也。子自愛不愛父,故虧父而自利;弟自愛不愛兄,故虧兄而自利;臣自愛不愛君,故虧君而自利,此所謂亂也。雖父之不慈子,兄之不慈弟,君之不慈臣,此亦天下之所謂亂也。父自愛也不愛子,故虧子而自利;兄自愛也不愛弟,故虧弟而自利;君自愛也不愛臣,故虧臣而自利。是何也?皆起不相愛。

雖至天下之為盜賊者亦然,盜愛其室不愛其異室,故竊異室以利其室;賊愛其身不愛人,故賊人以利其身。此何也?皆起不相愛。雖至大夫之相亂家,諸侯之相攻國者亦然。大夫各愛其1家,不愛異家,故亂異家以利其2家;諸侯各愛其國,不愛異國,故攻異國以利其國,天下之亂物具此而已矣。察此何自起?皆起不相愛。

若使天下兼相愛,愛1人若愛其身,猶有不孝者乎?視父兄與君若其身,2惡施不孝?猶有不慈者乎?視弟子3與臣若其身,惡施不慈?故4不孝不慈5亡有6,猶有盜賊乎?故視人之室若其室,誰竊?視人身若其身,誰賊?故盜賊亡有。猶有大夫之相亂家、諸侯之相攻國者乎?視人家若其家,誰亂?視人國若其國,誰攻?故大夫之相亂家、諸侯之相攻國者亡有。

若使天下兼相愛,國與國不相攻,家與家不相亂,盜賊無有,君臣父子皆能孝慈,若此則天下治。故聖人以治天下為事者,惡得不禁惡而勸愛?故天下兼相愛則治,交1相惡則亂。故子墨子曰:「不可以不勸愛人者,此也。」


\end{pinyinscope}