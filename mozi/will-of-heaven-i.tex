\article{天志上}

\begin{pinyinscope}
子墨子言曰:「今天下之士君子,知小而不知大。何以知之?以其處家者知之。若處家得罪於家長,猶有鄰家所避逃之。然且親戚兄弟所知識,共相儆戒,皆曰:『不可不戒矣!不可不慎矣!惡有處家而得罪於家長,而可為也!』非獨處家者為然,雖處國亦然。處國得罪於國君,猶有鄰國所避逃之,然且親戚兄弟所知識,共相儆戒皆曰:『不可不戒矣!不可不慎矣!誰亦有處國得罪於國君,而可為也』!此有所避逃之者也,相儆戒猶若此其厚,況無所避逃之者,相儆戒豈不愈厚,然後可哉?且語言有之曰:『焉而晏日焉而得罪,將惡避逃之?』曰無所避逃之。夫天不可為林谷幽門無人,明必見之。然而天下之士1君子之於2天也,忽然不知以相儆戒,此我所以知天下士君子知小而不知大也。

然則天亦何欲何惡?天欲義而惡不義。然則率天下之百姓以從事於義,則我乃為天之所欲也。我為天之所欲,天亦為我所欲。然則我1何欲何惡?我欲福祿而惡禍祟。若我不為天之所欲,而為天之所不欲,2然則我率天下之百姓,以從事於禍祟中也。然則何以3知天之欲義而惡不義?曰天下有義則生,無義則死;有義則富,無義則貧;有義則治,無義則亂。然則天欲其生而惡其死,欲其富而惡其貧,欲其治而惡其亂,此我所以知天欲義而惡不義也。

曰且夫義者政也,無從下之政上,必從上之政下。是故庶人竭力從事,未得次己而為政,有士政之;士竭力從事,未得次己而為政,有將軍大夫政之;將軍大夫竭力從事,未得次己而為政,有三公諸侯政之;三公諸侯竭力聽治,未得次己而為政,有天子政之;天子未得次己而為政,有天政之。天子為政於三公、諸侯、士、庶人,天下之士君子固明知,天之為政於天子,天下百姓未得之明知也。故昔三代聖王禹湯文武,欲以天之為政於天子,明說天下之百姓,故莫不犓牛羊,豢犬彘,潔為粢1盛酒醴,以祭祀上帝鬼神,而求祈福於天。我未嘗聞天下之所求祈福於天子者也,我所以知天之為政於天子者也。

故天子者,天下之窮貴也,天下之窮富也,故於富且貴者,當天意而不可不順,順天意者,兼相愛,交相利,必得賞。反天意者,別相惡,交相賊,必得罰。然則是誰順天意而得賞者?誰反天意而得罰者?」子墨子言曰:「昔三代聖王禹湯文武,此順天意而得賞也。昔三代之暴王桀紂幽厲,此反天意而得罰者也。然則禹湯文武其得賞何以也?」子墨子言曰:「其事上尊天,中事鬼神,下愛人,故天意曰:『此之我所愛,兼而愛之;我所利,兼而利之。愛人者此為博焉,利人者此為厚焉。』故使貴為天子,富有天下,業萬世子孫,傳稱其善,方施天下,至今稱之,謂之聖王。」然則桀紂幽厲得其罰何以也?」子墨子言曰:「其事上詬天,中詬鬼,下賊人,故天意曰:『此之我所愛,別而惡之,我所利,交而賊之。惡人者此為之博也,賊人者此為之厚也。』故使不得終其壽,不歿其世,至今毀之,謂之暴王。

然則何以知天之愛天下之百姓?以其兼而明之。何以知其兼而明之?以其兼而有之。何以知其兼而有之?以其兼而食焉。何以知其兼而食焉?四海之內,粒食之民,莫不犓牛羊,豢犬彘,潔為粢盛酒醴,以祭祀於上帝鬼神,天有邑人,何用弗愛也?且吾言殺一不辜者必有一不祥。殺不辜者誰也?則人也。予之不祥者誰也?則天也。若以天為不愛天下之百姓,則何故以人與人相殺,而天予之不祥?此我所以知天之愛天下之百姓也。

順天意者,義政也。反天意者,力政也。然義政1將柰何哉?」子墨子言曰:「處大國不攻小國,處大家不篡小家,強者不劫弱,貴者不傲賤,多詐者不欺愚。此必上利於天,中利於鬼,下利於人,三利無所不利,故舉天下美名加之,謂之聖王,力政者則與此異,言非此,行反此,猶倖馳也。處大國攻小國,處大家篡小家,強者劫弱,貴者傲賤,多詐欺愚。此上不利於天,中不利於鬼,下不利於人。三不利無所利,故舉天下惡名加之,謂之暴王。」

子墨子言曰:「我有天志,譬若輪人之有規,匠人之有矩,輪匠執其規矩,以度天下之方圜,曰:『中者是也,不中者非也。』今天下之士君子之書,不可勝載,言語不可盡計,上說諸侯,下說列士,其於仁義則大相遠也。何以知之?曰我得天下之明法以度之。」


\end{pinyinscope}