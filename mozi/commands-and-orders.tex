\article{號令}

\begin{pinyinscope}
安國之道,道任地始,地得其任則功成,地不得其任則勞而無功。人亦如此,備不先具者無以安主,吏卒民多心不一者,皆在其將長。諸行賞罰及有治者,必出於王公。數使人行勞賜守邊城關塞、備蠻夷之勞苦者,舉其守率之財用有餘、不足,地形之當守邊者,其器備常多者。邊縣邑視其樹木惡則少用,田不辟、少食,無大屋草蓋,少用桑。多財,民好食。為內堞,內行棧,置器備其上,城上吏、卒、養,皆為舍道內,各當其隔部。養什二人,為符者曰養吏一人,辨護諸門。門者及有守禁者皆無令無事者得稽留止其旁,不從令者戮。敵人且至,千丈之城,必郭迎之,主人利。不盡千丈者勿迎也,視敵之居曲,眾少而應之,此守城之大體也。其不在此中者,皆心術與人事參之。凡守城者以亟傷敵為上,其延日持久以待救之至,明於守者也,不能此,乃能守城。

守城之法,敵去邑百里以上,城將如今,盡召五官及百長,以富人重室之親,舍之官府,謹令信人守衛之,謹密為故。

及傅城,守將營無下三百人,四面四門之將,必選擇之有功勞之臣及死事之後重者,從卒各百人。門將并守他門,他門之上必夾為高樓,使善射者居焉。女郭、馮垣一人,一人守之,使重室子。

五十步一擊。因城中里為八部,部一吏,吏各從四人,以行衝術及里中。里中父老小不舉守之事及會計者,分里以為四部,部一長,以苛往來,不以時行、行而有他異者,以得其姦。吏從卒四人以上有分者,大將必與為信符,大將使人行,守操信符,信不合及號不相應者,伯長以上輒止之,以聞大將。當止不止及從吏卒縱之,皆斬。諸有罪自死罪以上,皆遝父母、妻子、同產。

諸男女有守於城上者,什、六弩、四兵。丁女子、老少,人一矛。

卒有驚事,中軍疾擊鼓者三,城上道路、里中巷街,皆無得行,行者斬。女子到大軍,令行者男子行左,女子行右,無並行,皆就其守,不從令者斬。離守者三日而一徇,此所以備姦也。里正與皆守宿里門,吏行其部,至里門,正與開門內吏。與行父老之守及窮巷幽閒無人之處。姦民之所謀為外心,罪車裂。正與父老及吏主部者,不得皆斬,得之,除,又賞之黃金,人二鎰。大將使使人行守,長夜五循行,短夜三循行。四面之吏亦皆自行其守,如大將之行,不從令者斬。

諸灶必為屏,火突高出屋四尺。慎無敢失火,失火者斬,其端失火以為事者,車裂。伍人不得,斬;得之,除。救火者無敢讙譁,及離守絕巷救火者斬。其正及父老有守此巷中部吏,皆得救之,部吏亟令人謁之大將,大將使信人將左右救之,部吏失不言者斬。諸女子有死罪及坐失火皆無有所失,逮其以火為亂事者如法。

圍城之重禁:敵人卒而至嚴令吏民無敢讙囂、三最、並行、相視、坐泣流涕、若視、舉手相探、相指、相呼、相麾、相踵、相投、相擊、相靡以身及衣、訟駮言語及非令也而視敵動移者,斬。伍人不得,斬;得之,除。伍人踰城歸敵,伍人不得,斬;與伯歸敵,隊吏斬;與吏歸敵,隊將斬。歸敵者父母、妻子、同產皆車裂。先覺之,除。當術需敵離地,斬。伍人不得,斬;得之,除。

其疾鬥卻敵於術,敵下終不能復上,疾鬥者隊二人,賜上奉。而勝圍,城周里以上,封城將三十里地為關內侯,輔將如令賜上卿,丞及吏比於丞者,賜爵五大夫,官吏、豪傑與計堅守者,十人及城上吏比五官者,皆賜公乘。男子有守者,爵人二級,女子賜錢五千,男女老小先分守者,人賜錢千,復之三歲,無有所與,不租稅。此所以勸吏民堅守勝圍也。

卒侍大門中者,曹無過二人。勇敢為前行,伍坐,令各知其左右前後。擅離署,戮。門尉晝三閱之,莫,鼓擊門閉一閱,守時令人參之,上逋者名。鋪食皆於署,不得外食。守必謹微察視謁者、執盾、中涓及婦人侍前者,志意、顏色、使令、言語之請。及上飲食,必令人嘗,皆非請也,擊而請故。守有所不說謁者、執盾、中涓及婦人侍前者,守曰斷之。衝之,若縛之,不如令,及後縛者,皆斷。必時素誡之。諸門下朝夕立若坐,各令以年少長相次,旦夕就位,先佑有功有能,其餘皆以次立。五日官各上喜戲、居處不莊、好侵侮人者一。

諸人士外使者來,必令有以執將。出而還若行縣,必使信人先戒舍室,乃出迎,門守乃入舍。為人下者常司上之,隨而行,松上不隨下。必須口口隨。

客卒守主人,及其為守衛,主人亦守客卒。城中戍卒,其邑或以下寇,謹備之,數錄其署,同邑者,弗令共所守。與階門吏為符,符合入,勞;符不合,牧,守言。若城上者,衣服,他不如令者。

宿鼓在守大門中,莫,令騎若使者操節閉城者,皆以執圭。昏鼓鼓十,諸門亭皆閉之。行者斷,必繫問行故,乃行其罪。晨見掌文,鼓縱行者,諸城門吏各入請籥,開門已,輒復上籥。有符節不用此令。寇至,樓鼓五,有周鼓,雜小鼓乃應之。小鼓五後從軍,斷。命必足畏,賞必足利,令必行,令出輒人隨,省其可行、不行。號,夕有號,失號,斷。為守備鬥而署之曰某程,置署街街衢階若門,令往來者皆視而放。諸吏卒民有謀殺傷其將長者,與謀反同罪,有能捕告,賜黃金二十斤,謹罪。非其分職而擅取之,若非其所當治而擅治為之,斷。諸吏卒民非其部界而擅入他部界,輒收,以屬都司空若候,候以聞守,不收而擅縱之,斷。能捕得謀反、賣城、踰城歸敵者一人,以令為除死罪二人,城旦四人。反城事父母去者,去者之父母妻子。

悉舉民室材木、瓦若藺石數,署長短小大,當舉不舉,吏有罪。諸卒民居城上者各葆其左右,左右有罪而不智也,其次伍有罪。若能身捕罪人若告之吏,皆構之。若非伍而先知他伍之罪,皆倍其構賞。

城外令任,城內守任,令、丞、尉亡得入當,滿十人以上,令、丞、尉奪爵各二級;百人以上,令、丞、尉免以卒戍。諸取當者,必取寇虜,乃聽之。

募民欲財物粟米以貿易凡器者,卒以賈予。邑人知識、昆弟有罪,雖不在縣中而欲為贖,若以粟米、錢金、布帛、他財物免出者,令許之。傳言者十步一人,稽留言及乏傳者,斷。諸可以便事者,亟以疏傳言守。吏卒民欲言事者,亟為傳言請之吏,稽留不言諸者,斷。

縣各上其縣中豪傑若謀士、居大夫、重厚口數多少。

官府城下吏卒民家,前後左右相傳保火。火發自燔,燔曼延燔人,斷。諸以眾彊凌弱少及彊姦人婦女,以讙譁者,皆斷。

諸城門若亭,謹候視往來行者符,符傳疑,若無符,皆詣縣廷言。請問其所使;其有符傳者,善舍官府。其有知識、兄弟欲見之,為召,勿令里巷中。三老、守閭令厲繕夫為答。若他以事者微者,不得入里中。三老不得入家人。傳令里中有以羽,羽在三所差,家人各令其官中,失令,若稽留令者,斷。家有守者治食。吏卒民無符節,而擅入里巷官府,吏、三老、守閭者失苛止,皆斷。

諸盜守器械、財物及相盜者,直一錢以上,皆斷。吏卒民各自大書於桀,著之其署隔。守案其署,擅入者,斷。城上日壹發席蓐,令相錯發,有匿不言人所挾藏在禁中者,斷。

吏卒民死者,輒召其人,與次司空葬之,勿令得坐泣。傷甚者令歸治病家善養,予醫給藥,賜酒日二升、肉二斤,令吏數行閭,視病有瘳,輒造事上。詐為自賊傷以辟事者,族之。事已,守使吏身行死傷家,臨戶而悲哀之。

寇去事已,塞禱。守以令益邑中豪傑力鬥諸有功者,必身行死傷者家以弔哀之,身見死事之後。城圍罷,主亟發使者往勞,舉有功及死傷者數使爵祿,守身尊寵,明白貴之,令其怨結於敵。

城上卒若吏各保其左右,若欲以城為外謀者,父母、妻子、同產皆斷。左右知不捕告,皆與同罪。城下里中家人皆相葆,若城上之數。有能捕告之者,封之以千家之邑;若非其左右及他伍捕告者,封之二千家之邑。

城禁:使、卒、民不欲寇微職和旌者,斷。不從令者,斷。非擅出令者,斷。失令者,斷。倚戟縣下城,上下不與眾等者,斷。無應而妄讙呼者,斷。縱失者,斷。譽客內毀者,斷。離署而聚語者,斷。聞城鼓聲而伍後上署者,斷。人自大書版,著之其署隔,守必自謀其先後,非其署而妄入之者,斷。離署左右,共入他署,左右不捕,挾私書,行請謁及為行書者,釋守事而治私家事,卒民相盜家室、嬰兒,皆斷無赦。人舉而藉之。無符節而橫行軍中者,斷。客在城下,因數易其署而無易其養,譽敵:少以為眾,亂以為治,敵攻拙以為巧者,斷。客、主人無得相與言及相藉,客射以書,無得譽,外示內以善,無得應,不從令者,皆斷。禁無得舉矢書,若以書射寇,犯令者父母、妻子皆斷,身梟城上。有能捕告之者,賞之黃金二十斤。非時而行者,唯守及摻太守之節而使者。

守入臨城,必謹問父老,吏大夫,諸有怨仇讎不相解者,召其人,明白為之解之。守必自異其人而藉之,孤之,有以私怨害城若吏事者,父母、妻子皆斷。其以城為外謀者,三族。有能得若捕告者,以其所守邑,小大封之,守還授其印,尊寵官之,令吏大夫及卒民皆明知之。豪傑之外多交諸侯者,常請之,令上通知之,善屬之,所居之吏上數選具之,令無得擅出入,連質之。術鄉長者、父老、豪傑之親戚父母、妻子,必尊寵之,若貧人食不能自給食者,上食之。及勇士父母親戚妻子皆時賜酒肉,必敬之,舍之必近太守。守樓臨質宮而善周,必密塗樓,令下無見上,上見下,下無知上有人無人。

守之所親,舉吏貞廉、忠信、無害、可任事者,其飲食酒肉勿禁,錢金、布帛、財物各自守之,慎勿相盜。葆宮之牆必三重,牆之垣,守者皆累瓦釜牆上。門有吏,主者門里,筦閉,必須太守之節。葆衛必取戍卒有重厚者。請擇吏之忠信者,無害可任事者。

令將衛,自築十尺之垣,周還牆門、閨者,非令衛司馬門。

望氣者舍必近太守,巫舍必近公社,必敬神之。巫祝史與望氣者必以善言告民,以請上報守,守獨知其請而已。無與望氣妄為不善言驚恐民,斷弗赦。

度食不足,食民各自占,家五種石升數,為期,其在蓴害,吏與雜訾,期盡匿不占,占不悉,令吏卒覹得,皆斷。有能捕告,賜什三。收粟米、布帛、錢金,出內畜產,皆為平直其賈,與主券人書之。事已,皆各以其賈倍償之。又用其賈貴賤、多少賜爵,欲為吏者許之,其不欲為吏,而欲以受賜賞爵祿,若贖出親戚、所知罪人者,以令許之。其受構賞者令葆宮見,以與其親。欲以復佐上者,皆倍其爵賞。某縣某里某子家食口二人,積粟六百石,某里某子家食口十人,積粟百石。出粟米有期日,過期不出者出王公有之,有能得若告之,賞之什三。慎無令民知吾粟米多少。

守入城,先以候為始,得輒宮養之,勿令知吾守衛之備。候者為異宮,父母妻子皆同其宮,賜衣食酒肉,信吏善待之。候來若復,就閒,守宮三難,外環隅為之樓,內環為樓,樓入葆宮丈五尺為復道。葆不得有室。三日一發席蓐,略視之,布茅宮中,厚三尺以上。發候,必使鄉邑忠信、善重士,有親戚、妻子,厚奉資之。必重發候,為養其親,若妻子,為異舍,無與員同所,給食之酒肉。遣他候,奉資之如前候,反,相參審信,厚賜之候三發三信,重賜之。不欲受賜而欲為吏者,許之二百石之吏。守珮授之印。其不欲為吏而欲受構賞祿,皆如前。有能入深至主國者,問之審信,賞之倍他候。其不欲受賞,而欲為吏者,許之三百石之吏。扞士受賞賜者,守必身自致之其親之其親之所,見其見守之任。其欲復以佐上者,其構賞、爵祿、罪人倍之。

出候無過十里,居高便所樹表,表三人守之,比至城者三表,與城上烽燧相望,晝則舉烽,夜則舉火。聞寇所從來,審知寇形必攻,論小城不自守通者,盡葆其老弱粟米畜產。遺卒候者無過五十人,客至堞去之。慎無厭建。候者曹無過三百人,日暮出之,為微職。空隊、要塞之人所往來者,令可以跡者,無下里三人,平明而跡。各立其表,城上應之。候出越陳表,遮坐郭門之外內,立其表,令卒之半居門內,令其少多無可知也。即有驚,見寇越陳去,城上以麾指之,遮坐擊鼓正期,以戰備從麾所指,望見寇,舉一垂;入竟,舉二垂;狎郭,舉三垂;入郭,舉四垂;狎城,舉五垂。夜以火,皆如此。

去郭百步,牆垣、樹木小大盡伐除之。外空井,盡窒之,無令可得汲也。外空窒盡發之,木盡伐之。諸可以攻城者盡內城中,令其人各有以記之。事以,各以其記取之。事為之券,書其枚數。當遂材木不能盡內,即燒之,無令客得而用之。

人自大書版,著之其署忠。有司出其所治,則從淫之法,其罪射。矜色謾正,淫囂不靜,當路尼眾,舍事後就,踰時不寧,其罪射。讙囂駴眾,其罪殺。非上不諫,次主凶言,其罪殺。無敢有樂器、獘騏軍中,有則其罪射。非有司之令,無敢有車馳、人趨,有則其罪射。無敢散牛馬軍中,有則其罪射。飲食不時,其罪射。無敢歌哭於軍中,有則其罪射。令各執罰盡殺,有司見有罪而不誅,同罰,若或逃之,亦殺。凡將率鬥其眾失法,殺。凡有司不使去卒、吏民聞誓令,代之服罪。凡戮人於市,死上目行。

謁者侍令門外,為二曹,夾門坐,鋪食更,無空。門下謁者一長,守數令入中,視其亡者,以督門尉與其官長,及亡者入中報。四人夾令門內坐,二人夾散門外坐。客見,持兵立前,鋪食更,上侍者名。

守室下高樓,候者望見乘車若騎卒道外來者,及城中非常者,輒言之守。守以須城上候城門及邑吏來告其事者以驗之,樓下人受候者言,以報守。

中涓二人,夾散門內坐,門常閉,鋪食更,中涓一長者。環守宮之術衢,置屯道,各垣其兩旁,高丈,為埤倪,立初雞足置,夾挾視葆食。而札書得必謹案視參食者,即不法,止詰之。屯道垣外術衢街皆為樓,高臨里中,樓一鼓壟灶。即有物故,鼓,吏至而止。夜以火指鼓所。城下五十步一廁,廁與上同圂。請有罪過而可無斷者,令杼廁利之。


\end{pinyinscope}