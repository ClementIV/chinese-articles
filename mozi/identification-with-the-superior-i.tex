\article{尚同上}

\begin{pinyinscope}
子墨子言曰:「古者民始生,未有刑政之時,蓋其語『人異義』。是以一人則一義,二人則二義,十人則十義,其人茲眾,其所謂義者亦茲眾。是以人是其義,以非人之義,故文相非也。是以內者父子兄弟作怨惡,離散不能相和合。天下之百姓,皆以水火毒藥相虧害,至有餘力不能以相勞,腐臭1餘財不以相分,隱匿良道不以相教,天下之亂,若禽獸然。

夫明虖天下之所以亂者,生於無政長。是故選天下之賢可者,立以為天子。天子立,以其力為未足,又選擇天下之賢可者,置立之以為三公。天子三公既以立,以天下為博大,遠國異土之民,是非利害之辯,不可一二而明知,故畫分萬國,立諸侯國君,諸侯國君既已立,以其力為未足,又選擇其國之賢可者,置立之以為正長。

正長既已具,天子發政於天下之百姓,言曰:『聞善而不善,皆以告其上。上之所是,必皆是之,所非必皆非之,上有過則規諫之,下有善則傍薦之。上同而不下比者,此上之所賞,而下之所譽也。意若聞善而不善,不以告其上,上之所是,弗能是,上之所非,弗能非,上有過弗規諫,下有善弗傍薦,下比不能上同者,此上之所罰,而百姓所毀也。』上以此為賞罰,甚明察以審信。

是故里長者,里之仁人也。里長發政里之百姓,言曰:『聞善而不善,必以告其鄉長。鄉長之所是,必皆是之,鄉長之所非,必皆非之。去若不善言,學鄉長之善言;去若不善行,學鄉長之善行,則鄉何說以亂哉?』察鄉之所治者何也?鄉長唯能壹同鄉之義,是以鄉治也。

鄉長者,鄉之仁人也。鄉長發政鄉之百姓,言曰:『聞善而不善者,必以告國君。國君之所是,必皆是之,國君之所非,必皆非之。去若不善言,學國君之善言,去若不善行,學國君之善行,則國何說以亂哉。』察國之所以治者何也?國君唯能壹同國之義,是以國治也。

國君者,國之仁人也。國君發政國之百姓,言曰:『聞善而不善。必以告天子。天子之所是,皆是之,天子之所非,皆非之。去若不善言,學天子之善言;去若不善行,學天子之善行,則天下何說以亂哉。』察天下之所以治者何也?天子唯能壹同天下之義,是以天下以1治也。

天下之百姓皆上同於天子,而不上同於天,則菑猶未去也。今若天飄風苦雨,溱溱而至者,此天之所以罰百姓之不上同於天者也。」

是故子墨子言曰:「古者聖王為五刑,請以治其民。譬若絲縷之有紀,罔罟之有綱,所連收天下之百姓不尚同其上者也。」


\end{pinyinscope}