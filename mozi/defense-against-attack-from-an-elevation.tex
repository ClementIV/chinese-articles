\article{備高臨}

\begin{pinyinscope}
禽子再拜再拜曰:「敢問適人積土為高,以臨吾城,薪土俱上,以為羊黔,蒙櫓俱前,遂屬之城,兵弩俱上,為之柰何?」

子墨子曰:「子問羊黔之守邪?羊黔者將之拙者也,足以勞卒,不足以害城。守為臺城,以臨羊黔,左右出巨,各二十尺,行城三十尺,強弩射之,技機藉之,奇器口口之,然則羊黔之攻敗矣。

備臨以連弩之車材大方一方一尺,長稱城之薄厚。兩軸三輪,輪居筐中,重下上筐。左右旁二植,左右有衡植,衡植左右皆圜內,內徑四寸。左右縳弩皆於植,以弦鉤弦,至於大弦。弩臂前後與筐齊,筐高八尺,弩軸去下筐三尺五寸。連弩機郭同銅,一石三十鈞。引弦鹿長奴。筐大三圍半,左右有鉤距,方三寸,輪厚尺二寸,鉤距臂博尺四寸,厚七寸,長六尺。橫臂齊筐外,蚤尺五寸,有距,博六寸,厚三寸,長如筐,有儀,有詘勝,可上下。為武重一石以材大圍五寸。矢長十尺,以繩系箭矢端,如如戈射,以磿鹿卷收。矢高弩臂三尺,用弩無數,出人六十枚,用小矢無留。十人主此車。遂具寇,為高樓以射道,城上以荅、羅,矢。


\end{pinyinscope}