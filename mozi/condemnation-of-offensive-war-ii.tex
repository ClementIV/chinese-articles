\article{非攻中}

\begin{pinyinscope}
子墨子言曰:「古者王公大人,為政於國家者,情欲譽之審,賞罰之當,刑政之不過失。」是故子墨子曰:「古者有語:『謀而不得,則以往知來,以見知隱』。謀若此,可得而知矣。」

今師徒唯毋興起,冬行恐寒,夏行恐暑,此不可以冬夏為者也。春則廢民耕稼樹藝,秋則廢民穫斂。今唯毋廢一時,則百姓飢寒凍餒而死者,不可勝數。今嘗計軍上,竹箭羽旄幄幕,甲盾撥劫,往而靡壞腑爛不反者,不可勝數;又與矛戟戈劍乘車,其往則1碎折靡壞而不反者,不可勝數;與其牛馬肥而往,瘠而反,往死亡而不反者,不可勝數;與其涂道之脩遠,糧食輟絕而不繼,百姓死者,不可勝數也;與其居處之不安,食飲之不時,飢飽之不節,百姓之道疾病而死者,不可勝數;喪師多不可勝數,喪師盡不可勝計,則是鬼神之喪其主後,亦不可勝數。

國家發政,奪民之用,廢民之利,若此甚眾,然而何為為之?曰:「我貪伐勝之名,及得之利,故為之。」子墨子言曰:「計其所自勝,無所可用也。計其所得,反不如所喪者之多。今攻三里之城,七里之郭,攻此不用銳,且無殺而徒得此然也。殺人多必數於萬,寡必數於千,然後三里之城、七里之郭,且可得也。今萬乘之國,虛數於千,不勝而入廣衍數於萬,不勝而辟。然則土地者,所有餘也,士民者,所不足也。今盡士民之死,嚴下上之患,以爭虛城,則是棄所不足,而重所有餘也。為政若此,非國之務者也。」

飾攻戰者言曰:「南則荊、吳之王,北則齊、晉之君,始封於天下之時,其土地之方,未至有數百里也;人徒之眾,未至有數十萬人也。以攻戰之故,土地之博至有數千里也;人徒之眾至有數百萬人。故當攻戰而不可為也。」子墨子言曰:「雖四五國則得利焉,猶謂之非行道也。譬若醫之藥人之有病者然。今有醫於此,和合其祝藥之于天下之有病者而藥之,萬人食此,若醫四五人得利焉,猶謂之非行藥也。故孝子不以食其親,忠臣不以食其君。古者封國於天下,尚者以耳之所聞,近者以目之所見,以攻戰亡者,不可勝數。何以知其然也?東方自莒之國者,其為國甚小,閒於大國之閒,不敬事於大,大國亦弗之從而愛利。是以東者越人夾削其壤地,西者齊人兼而有之。計莒之所以亡於齊越之間者,以是攻戰也。雖南者陳、蔡,其所以亡於吳越之閒者,亦以攻戰。雖北者且不一著何,其所以亡於燕、代、胡、貊之閒者,亦以攻戰也。」是故子墨子言曰:「古者王公大人,情欲得而惡失,欲安而惡危,故當攻戰而不可不非。」

飾攻戰者之言曰:「彼不能收用彼眾,是故亡。我能收用我眾,以此攻戰於天下,誰敢不賓服哉?」子墨子言曰:「子雖能收用子之眾,子豈若古者吳闔閭哉?古者吳闔閭教七年,奉甲執兵,奔三百里而舍焉,次注林,出於冥隘之徑,戰於柏舉,中楚國而朝宋與及魯。至夫差之身,北而攻齊,舍於汶上,戰於艾陵,大敗齊人而葆之大山;東而攻越,濟三江五湖,而葆之會稽。九夷之國莫不賓服。於是退不能賞孤,施舍群萌,自恃其力,伐其功,譽其智,怠於教,遂築姑蘇之臺,七年不成。及若此,則吳有離罷之心。越王句踐視吳上下不相得,收其眾以復其讎,入北郭,徙大內,圍王宮而吳國以亡。昔者晉有六將軍,而智伯莫為強焉。計其土地之博,人徒之眾,欲以抗諸侯,以為英名。攻戰之速,故差論其爪牙之士,皆列其1舟車之眾,以攻中行氏而有之。以其謀為既已足矣,又攻茲范氏而大敗之,并三家以為一家,而不止,又圍趙襄子於晉陽。及若此,則韓、魏亦相從而謀曰:『古者有語,脣亡則齒寒』。趙氏朝亡,我夕從之,趙氏夕;亡,我朝從之。《詩》曰『魚水不務,陸將何及乎!』」是以三主之君,一心戮力辟門除道,奉甲興士,韓、魏自外,趙氏自內,擊智伯大敗之。」是故子墨子言曰:「古者有語曰:『君子不鏡於水而鏡於人,鏡於水,見面之容,鏡於人,則知吉與凶。今以攻戰為利,則蓋嘗鑒之於智伯之事乎?此其為不吉而凶,既可得而知矣。』」


\end{pinyinscope}