\article{尚賢下}

\begin{pinyinscope}
子墨子言曰:「天下之王公大人皆欲其國家之富也,人民之眾也,刑法之治也,然而不識以尚賢為政其國家百姓,王公大人本失尚賢為政之本也。若苟王公大人本失尚賢為政之本也,則不能毋舉物示之乎?今若有一諸侯於此,為政其國家也,曰:『凡我國能射御之士,我將賞貴之,不能射御之士,我將罪賤之。』問於若國之士,孰喜孰懼?我以為必能射御之士喜,不能射御之士懼。我賞因而誘之矣,曰:『凡我國之忠信之士,我將賞貴之,不忠信之士,我將罪賤之。』問於若國之士,孰喜孰懼?我以為必忠信之士喜,不忠不信之士懼。今惟毋以尚賢為政其國家百姓,使國為善者勸,為暴者沮,大以為政於天下,使天下之為善者勸,為暴者沮。然昔吾所以貴堯舜禹湯文武之道者,何故以哉?以其唯毋臨眾發政而治民,使天下之為善者可而勸也,為暴者可而沮也。然則此尚賢者也,與堯舜禹湯文武之道同矣。

而今天下之士君子,居處言語皆尚賢,逮至其臨眾發政而治民,莫知尚賢而使能,我以此知天下之士君子,明於1小而不明於大也。何以知其然乎?今王公大人,有一牛羊之財不能殺,必索良宰;有一衣裳之財不能制,必索良工。當王公大人之於此也,雖有骨肉之親,無故富貴、面目美好者,實知其不能也,不使之也,是何故?恐其敗財也。當王公大人之於此也,則不失尚賢而使能。王公大人有一罷馬不能治,必索良醫;有一危弓不能張,必索良工。當王公大人之於此也,雖有骨肉之親,無故富貴、面目美好者,實知其不能也,必不使。是何故?恐其敗財也。當王公大人之於此也,則不失尚賢而使能。逮至其國家則不然,王公大人骨肉之親,無故富貴、面目美好者,則舉之,則王公大人之親其國家也,不若親其一危弓、罷馬、衣裳、牛羊之財與。我以此知天下之士君子皆明於小,而不明於大也。此譬猶瘖者而使為行人,聾者而使為樂師。

是故古之聖王之治天下也,其所富,其所貴,未必王公大人骨肉之親、無故富貴、面目美好者也。是故昔者舜耕於歷山,陶於河瀕,漁於雷澤,灰於常陽。堯得之服澤之陽,立為天子,使接天下之政,而治天下之民。昔伊尹為莘氏女師僕,使為庖人,湯得而舉之,立為三公,使接天下之政,治天下之民。昔者傅說居北海之洲,圜土之上,衣褐帶索,庸築於傅巖之城,武丁得而舉之,立為三公,使之接天下之政,而治天下之民。是故昔者堯之舉舜也,湯之舉伊尹也,武丁之舉傅說也,豈以為骨肉之親、無故富貴、面目美好者哉?惟法其言,用其謀,行其道,上可而利天,中可而利鬼,下可而利人,是故推而上之。

古者聖王既審尚賢欲以為政,故書之竹帛,琢之槃盂,傳以遺後世子孫。於先王之書呂刑之書然,王曰:『於!來!有國有土,告女訟刑,在今而安百姓,女何擇言人,何敬不刑,何度不及。』能擇人而敬為刑,堯、舜、禹、湯、文、武之道可及也。是何也?則以尚賢及之,於先王之書豎年之言然,曰:『晞夫聖、武、知人,以屏輔而身。』此言先王之治天下也,必選擇賢者以為其群屬輔佐。曰今也天下之士君子,皆欲富貴而惡貧賤。曰然。女何為而得富貴而辟貧賤?莫若為賢。為賢之道將柰何?曰有力者疾以助人,有財者勉以分人,有道者勸以教人。若此則飢者得食,寒者得衣,亂者得治。若飢則得食,寒則得衣,亂則得治,此安生生。

今王公大人其所富,其所貴,皆王公大人骨肉之親,無故富貴、面目美好者也。今王公大人骨肉之親,無故富貴、面目美好者,焉故必知哉!若不知,使治其國家,則其國家之亂可得而知也。今天下之士君子皆欲富貴而惡貧賤。然女何為而得富貴,而辟貧賤哉?曰莫若為王公大人骨肉之親,無故富貴、面目美好者1。王公大人骨肉之親,無故富貴、面目美好者,此非可學能者也。使不知辯,德行之厚若禹、湯、文、武不加得也,王公大人骨肉之親,躄、瘖、聾,暴為桀、紂,不加失也。是故以賞不當賢,罰不當暴,其所賞者已無故矣,其所罰者亦無罪。是以使百姓皆攸心解體,沮以為善,垂其股肱之力而不相勞來也;腐臭餘財,而不相分資也,隱慝良道,而不相教誨也。若此,則飢者不得食,寒者2不得衣,亂者不得治3推而上之以。

是故昔者堯有舜,舜有禹,禹有皋陶,湯有小臣,武王有閎夭、泰顛、南宮括、散宜生,而天下和,庶民阜,是以近者安之,遠者歸之。日月之所照,舟車之所及,雨露之所漸,粒食之所養,1得此不勸譽。且今天下之王公大人士君子,中實將欲為仁義,求為士,上欲中聖王之道,下欲中國家百姓之利,而天下和,庶民阜,是以近者安之,遠者歸之。日月之所照,舟車之所及,雨露之所漸,粒食之所養,2故尚賢之為說,而不可不察此者也。尚賢者,天鬼百姓之利,而政事之本也。」


\end{pinyinscope}