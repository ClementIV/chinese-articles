\article{七患}

\begin{pinyinscope}
子墨子曰:國有七患。七患者何?城郭溝池不可守而治宮室,一患也。邊國至境四鄰莫救,二患也。先盡民力無用之功,賞賜無能之人,民力盡於無用,財寶虛於待1客,三患也。仕者持2祿,游者愛佼3,君脩法討臣,臣懾而不敢拂,四患也。君自以為聖智而不問事,自以為安彊而無守備,四鄰謀之不知戒,五患也。所信4者5不忠,所忠者6不信,六患也。畜種菽粟不足以食之,大臣不足以7事之,賞賜不能喜,誅罰不能威,七患也。以七患居國,必無社8稷;以七患守城,敵至國傾。七患之所當,國必有殃。

凡五穀者,民之所仰也,君之所以為養也。故民無仰則君無養,民無食則不可事。故食不可不務也,地不可不力也,用不可不節也。五穀盡收,則五味盡御於主,不盡收則不盡御。一穀不收謂之饉,二穀不收謂之旱,三穀不收謂之凶,四穀不收謂之餽,五穀不收謂之饑。歲饉,則仕者大夫以下皆損祿五分之一。旱,則損五分之二。凶則損五分之三。餽,則損五分之四。饑,則盡無祿,稟食而已矣。故凶饑存乎國,人君徹鼎食五分之三1,大夫徹縣,士不入學,君朝之衣不革制,諸侯之客,四鄰之使,雍飧2而不盛,徹驂騑,塗不芸,馬不食粟,婢妾不衣帛,此告不足之至也。

今有負其子而汲者,隊其子於井中,其母必從而道之。今歲凶,民饑道餓,重其子此疚於隊,其可無察邪?故時年歲善,則民仁且良;時年歲凶,則民吝且惡。夫民何常此之有?為者疾,食者眾,則歲無豐。故曰:「財不足則反之時,食不足則反之用。」故先民以時生財,固本而用財,則財足。故雖上世之聖王,豈能使五穀常收而旱水不至哉?然而無凍餓之民者,何也?其力時急而自養儉也。故《夏書》曰:「禹七年水。」《殷書》曰:「湯五年旱。」此其離凶餓甚矣。然而民不凍餓者,何也?其生財密,其用之節也。

故倉1無備粟,不可以待凶饑;庫無備兵,雖有義不能征無義;城郭不備全,不可以自守;心無備慮,不可以應卒。是若慶忌無去之心,不能輕出。夫桀無待湯之備,故放;紂無待武王之備,故殺。桀、紂貴為天子,富有天下,然而皆滅亡於百里之君者,何也?有富貴而不為備也。故備者,國之重也;食者,國之寶也;兵者,國之爪也。城者所以自守也。此三者國之具也。

故曰:以其極賞,以賜無功,虛其府庫,以備車馬、衣裘、奇怪,苦其役徒,以治宮室觀樂;死又厚為棺槨,多為衣裘。生時治臺榭,死又脩墳墓。故民苦於外,府庫單於內,上不厭其樂,下不堪其苦。故國離寇敵則傷,民見凶饑則亡,此皆備不具之罪也。且夫食者,聖人之所寶也。故《周書》曰:「國無三年之食者,國非其國也;家無三年之食者,子非其子也。」此之謂國備。


\end{pinyinscope}