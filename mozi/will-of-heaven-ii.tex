\article{天志中}

\begin{pinyinscope}
子墨子言曰:「今天下之君子之欲為仁義者,則不可不察義之所從出。既曰不可以不察義之所從出,然則義何從出?」子墨子曰:「義不從愚且賤者出,必自貴且知者出。何以知義之不從愚且賤者出,而必自貴且知者出也?曰:義者,善政也。何以知義之為1善政也?曰:天下有義則治,無義則亂,是以知義之為2善政也。夫愚且賤者,不得為政乎貴且知者,然後得為政乎愚且賤者,此吾所以知義之不從愚且賤者出,而必自貴且知者出也。然則孰為貴?孰為知?曰:天為貴,天為知而已矣。然則義果自天出矣。」

今天下之人曰:「當若天子之貴於1諸侯,諸侯之貴於2大夫,傐明知之。然吾未知天之貴且知於天子也。」子墨子曰:「吾所以知天之貴且知於天子者有矣。曰:天子為善,天能賞之;天子為暴,天能罰之;天子有疾病禍祟,必齋戒沐浴,潔為酒醴粢盛,以祭祀天鬼,則天能除去之,然吾未知天之祈福於天子也。此吾所以知天之貴且知於天子者。不止此而已矣,又以先王之書馴天明不解之道也知之。曰:『明哲維天,臨君下土。』則此語天之貴且知於天子。不知亦有貴知夫天者乎?曰:天為貴,天為知而已矣。然則義果自天出矣。」

是故子墨子曰:「今天下之君子,中實將欲遵道利民,本察仁義之本,天之意不可不慎也。」既以天之意以為不可不慎已,然則天之將何欲何憎?子墨子曰:「天之意不欲大國之攻小國也,大家之亂小家也,強之暴寡,詐之謀愚,貴之傲賤,此天之所不欲也。不1止2此而已,欲人之有力相營,有道相教,有財相分也。又欲上之強聽治也,下之強從事也。上強聽治,則國家治矣,下強從事則財用足矣。若國家治財3用足,則內有以潔為酒醴粢盛,以祭祀天鬼;外有以為環璧珠玉,以聘撓四鄰。諸侯之冤不興矣,邊境兵甲不作矣。內有以食飢息勞,持養其萬民,則君臣上下惠忠,父子弟兄慈孝。故唯毋明乎順天之意,奉而光施之天下,則刑政治,萬民和,國家富,財用足,百姓皆得煖衣飽食,便寧無憂。」是故子墨子曰:「今天下之君子,中實將欲遵道利民,本察仁義之本,天之意不可不慎也!

且夫天子之有天下也,辟之無以異乎國君諸侯之有四境之內也。今國君諸侯之有四境之內也,夫豈欲其臣國萬民之相為不利哉?今若處大國則攻小國,處大家則亂小家,欲以此求賞譽,終不可得,誅罰必至矣。夫天之有天下也,將無已異此。今若處大國則1攻小國,處大都則伐小都,欲以此求福祿於天,福祿終不得,而禍祟必至矣。然有所不為天之所欲,而為天之所不欲,則夫天亦且不為人之所欲,而為人之所不欲矣。人之2所不欲者何也?曰病疾禍3祟也。若已不為天之所欲,而為天之所不欲,是率天下之萬民以從事乎禍祟之中也。故古者聖王明知天鬼之所福,而辟天鬼之所憎,以求興天下之利,而除天下之害。是以天之為寒熱也節,四時調,陰陽雨露也時,五穀孰,六畜遂,疾災戾疫凶饑則不至。」是故子墨子曰:「今天下之君子,中實將欲遵道4利民,本察仁義之本,天意不可不慎也!

且夫天下蓋有不仁不祥者,曰當若子之不事父,弟之不事兄,臣之不事君也。故天下之君子,與謂之不祥者。今夫天兼天下而愛之,撽遂萬物以利之,若豪之末,非天之所為也,而民得而利之,則可謂否矣。然獨無報夫天,而不知其為不仁不祥也。此吾所謂君子明細而不明大也。

且吾所以知天之愛民之厚者有矣,曰以磨為日月星辰,以昭道之;制為四時春秋冬夏,以紀綱之;雷降雪霜雨露,以長遂五穀麻絲,使民得而財利之;列為山川谿谷,播賦百事,以臨司民之善否;為王公侯伯,使之賞賢而罰暴;賊金木鳥獸,從事乎五穀麻絲,以為民衣食之財。自古及今,未嘗不有此也。今有人於此,驩若愛其子,竭力單務以利之,其子長,而無報子求父,故天下之君子與謂之不仁不祥。今夫天兼天下而愛之,撽遂萬物以利之,若豪之末,非天之所為,而民得而利之,則可謂否矣,然獨無報夫天,而不知其為不仁不祥也。此吾所謂君子明細而不明大也。

且吾所以知天愛民之厚者,不止此而足矣。曰殺不辜者,天予不祥。不辜者誰也?曰人也。予之不祥者誰也?曰天也。若天不愛民之厚,夫胡說人殺不辜,而天予之不祥哉?此吾之所1以知天之愛民之厚也。

且吾所以知天之愛民之厚者,不止此而已矣。曰愛人利人,順天之意,得天之賞者有之;憎人賊人1,反天之意,得天之罰者亦有矣。夫愛人利人,順天之意,得天之賞者誰也?曰若昔三代聖王,堯舜禹湯文武者是也。堯舜禹湯文武焉所從事?曰從事兼,不從事別。兼者,處大國不攻小國,處2大家不亂小家,強不劫弱,眾不暴寡,詐不謀愚,貴不傲賤。觀其事,上利乎天,中利乎鬼,下利乎人,三利無所不利,是謂天德。聚斂天下之美名而加之焉,曰:此仁也,義也,愛人利人,順天之意,得天之賞者也。不止此而已,書於竹帛,鏤之金石,琢之槃盂,傳遺後世子孫。曰將何以為?將以識夫愛人利人,順天之意,得天之賞者也。皇矣道之曰:『帝謂文王,予懷明德,不大聲以色,不長夏以革,不識不知,順帝之則。』帝善其順法則也,故舉殷以賞之,使貴為天子,富有天下,名譽至今不息。故夫愛人利人,順天之意,得天之賞者,既可得留而已。夫憎人賊人,反天之意,得天之罰者誰也?曰若昔者三代暴王桀紂幽厲者是也。桀紂幽厲焉所從事?曰從事別,不從事兼。別者,處大國則攻小國,處大家則亂小家,強劫弱,眾暴寡,詐謀愚,貴傲賤。觀其事,上不利乎天,中不利乎鬼,下不利乎人,三不利無所利,是謂天賊。聚斂天下之醜名而加之焉,曰此非仁也,非義也。憎人賊人,反天之意,得天之罰者也。不止此而已,又書其事於竹帛,鏤之金石,琢之槃盂,傳遺後世子孫。曰將何以為?將以識夫憎人賊人,反天之意,得天之罰者也。大誓之道之曰:『紂越厥夷居,不肯事上帝,棄厥先神祇不祀,乃曰吾有命,毋廖𠏿務天下3。天亦縱棄紂而不葆。』察天以縱棄紂而不葆者,反天之意也。故夫憎人賊人,反天之意,得天之罰者,既可得而知也。」

是故子墨子之有天之,辟人無以異乎輪人之有規,匠人之有矩也。今夫輪人操其規,將以量度天下之圜與不圜也,曰:中吾規者謂之圜,不中吾規者謂之不圜。是以圜與不圜,皆可得而知也。此其故何?則圜法明也。匠人亦操其矩,將以量度天下之方與不方也。曰:中吾矩者謂之方,不中吾矩者謂之不方。是以方與不方,皆可得而知之。此其故何?則方法明也。故子墨子之有天之意也,上將以度天下之王公大人之為刑政也,下將以量天下之萬民為文學出言談也。觀其行,順天之意,謂之善意行,反天之意,謂之不善意行;觀其言談,順天之1意,謂之善言談,反天之意,謂之不善言談;觀其刑政,順天之意,謂之善刑政,反天之意,謂之不善刑政。故置此以為法,立此以為儀,將以量度天下之王公大人卿大夫之仁與不仁,譬之猶分黑白也。是故子墨子曰:「今天下之王公大人士君子,中實將欲遵道利民,本察仁義之本,天之意不可不順也。順天之意者,義之法也。」


\end{pinyinscope}