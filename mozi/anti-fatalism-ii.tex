\article{非命中}

\begin{pinyinscope}
子墨子言曰:「凡出言談,由文學之為道也,則不可而不先立義法。若言而無義,譬猶立朝夕於員鈞之上也,則雖有巧工,必不能得正焉。然今天下之情偽,未可得而識也,故使言有三法。三法者何也?有本之者,有原之者,有用之者。於其本之也,考之天鬼之志,聖王之事;於其原之也,徵以先王之書;用之柰何,發而為刑。此言之三法也。

今天下之士君子或以命為亡,我所以知命之有與亡者,以眾人耳目之情,知有與亡。有聞之,有見之,謂之有;莫之聞,莫之見,謂之亡。然胡不1嘗考之百姓之情?自古以及今,生民以來者,亦嘗見命之物,聞命之聲者乎?則未嘗有也。若以百姓為愚不肖,耳目之情不足因而為法,然則胡不嘗考之諸侯之傳言流語乎?自古以及今,生民以來者,亦嘗有聞命之聲,見命之體者乎?則未嘗有也。然胡不嘗考之聖王之事?古之聖王,舉孝子而勸之事親,尊賢良而勸之為善,發憲布令以教誨,明2賞罰以勸沮。若此,則亂者可使治,而危者可使安矣。若以為不然,昔者,桀之所亂,湯治之;紂之所亂,武王治之。此世不渝而民不改,上變政而民易教,其在湯武則治,其在桀紂則亂,安危治亂,在上之發政也,則豈可謂有命哉!夫曰有命云者亦不然矣。

今夫有命者言曰:『我非作之後世也,自昔三代有若言以傳流矣。今故先生對之?』曰:夫有命者,不志昔也三代之聖善人與?意亡昔三代之暴不肖人也?何以知之?初之列士桀大夫,慎言知行,此上有以規諫其君長,下有以教順其百姓,故上有以規諫其君長,下有以教順其百姓,1故上得其君長之賞,下得其百姓之譽。列士桀大夫聲聞不廢,流傳至今,而天下皆曰其力也,必不能曰我見命焉2。

是故昔者三代之暴王,不繆其耳目之淫,不慎其心志之辟,外之敺騁田獵畢弋,內沈於酒樂,1不顧其國家百姓之政。繁為無用,暴逆百姓,使下不親其上,是故國為虛厲,身在刑僇之中,必不能曰我見命焉2。是故昔者三代之暴王,不繆其耳目之淫,不慎其心志之辟,外之敺騁田獵畢弋,內沈於酒樂,3不肯曰:4『我5罷不肖,我為刑政不善』,必曰:『我命故且亡。』雖昔也三代之窮民,亦由此也。內之不能善事其親戚,外不能善事其君長,惡恭儉而好簡易,貪飲食而惰從事,衣食之財不足,使身至有饑寒凍餒之憂,必不能曰:『我罷不肖,我從事不疾』,必曰:『我命固且窮。』雖昔也三代之偽民,亦猶此也。繁飾有命,以教眾愚樸人久矣。聖王之患此也,故書之竹帛,琢之金石,於先王之書仲虺之告曰:『我聞有夏,人矯天命,布命于下,帝式是惡,用闕師。』此語夏王桀之執有命也,湯與仲虺共非之。先王之書太誓之言然曰:『紂夷之居,而不用事上帝,棄闕其先神而不祀也,曰:「我民有命,毋僇其務。」天不亦棄縱而不葆。』此言紂之執有命也,武王以太誓非也。有於三代不國有之曰:『女毋崇天之有命也。』命三不國亦言命之無也。於召公之執令於然,且:「『敬哉!無天命,惟予二人,而無造言,不自降天之哉得之。』在於商、夏之詩書曰:『命者暴王作之。』且今天下之士君子,將欲辯是非利害之故,當天有命者,不可不疾非也。」執有命者,此天下之厚害也,是故子墨子非也。


\end{pinyinscope}