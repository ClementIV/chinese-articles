\article{非儒下}

\begin{pinyinscope}
儒者曰:「親親有術,尊賢有等。」言親疏尊卑之異也。其禮曰:「喪父母三年,妻,後子三年,伯父叔父弟兄庶子其,戚族人五月。」若以親疏為歲月之數,則親者多而疏者少矣,是妻後子與父同也。若以尊卑為歲月數,則是尊其妻子與父母同,而親伯父宗兄而卑子也,逆孰大焉。其親死,列尸弗斂1,登屋窺井,挑鼠穴,探滌器,而求其人矣。以為實在則贛愚甚矣;如其亡也必求焉,偽亦大矣!取妻,身迎,袨端為僕,秉轡授綏,如仰嚴親,昏禮威儀,如承祭祀。顛覆上下,悖逆父母,下則妻子,妻子上侵事親,若此可謂孝乎?儒者:「迎妻,妻之奉祭祀,子將守宗廟,故重之。」應之曰:「此誣言也,其宗兄守其先宗廟數十年,死喪之其,兄弟之妻奉其先之祭祀弗散,則喪妻子三年,必非以守奉祭祀也。夫憂妻子以大負絫,有曰『所以重親也』,為欲厚所至私,輕所至重,豈非大姦也哉!」

有強執有命以說議曰:「壽夭貧富,安危治亂,固有天命,不可損益。窮達賞罰幸否有極,人之知力,不能為焉。」群吏信之,則怠於分職;庶人信之,則怠於從事。吏1不治則亂,農事緩則貧,貧且亂政之本,而儒者以為道教,是賊天下之人者也。

且夫繁飾禮樂以淫人,久喪偽哀以謾親,立命緩貧而高浩居,倍本棄事而安怠傲,貪於飲食,惰於作務,陷於飢寒,危於凍餒,無以違之。是若人氣,鼸鼠藏,而羝羊視,賁彘起。君子笑之。怒曰:「散人!焉知良儒。」夫夏乞麥禾,五穀既收,大喪是隨,子姓皆從,得厭飲食,畢治數喪,足以至矣。因人之家翠,以為,恃人之野以為尊,富人有喪,乃大說,喜曰:「此衣食之端也。」

儒者曰:「君子必服古言然後仁。」應之曰:「所謂古之言服1者,皆嘗新矣,而古人言之,2服之,則非3君子也。然則必服非君子之服,言非君子之言,而後仁乎?」

又曰:「君子循而不作。」應之曰:「古者羿作弓,伃作甲,奚仲作車,巧垂作舟,然則今之鮑函車匠皆君子也,而羿、伃、奚仲、巧垂皆小人邪?且其所循人必或作之,然則其所循皆小人道也?」

又曰:「君子勝不逐奔,揜函弗射,施則助之胥車。」應之曰:「若皆仁人也,則無說而相與。仁人以其取舍是非之理相告,無故從有故也,弗知從有知也,無辭必服,見善必遷,何故相?若兩暴交爭,其勝者欲不逐奔,掩函弗射,施則助之胥車,雖盡能猶且不得為君子也。意暴殘之國也,聖將為世除害,興師誅罰,勝將因用儒術令士卒曰毋逐奔,揜函勿射,施則助之胥車。」暴亂之人也得活,天下害不除,是為群殘父母,而深賤世也,不義莫大焉!」

又曰:「君子若鍾,擊之則鳴,弗擊不鳴。應之曰:「夫仁人事上竭忠,事親得孝,務善則美,有過則諫,此為人臣之道也。今擊之則鳴,弗擊不鳴,隱知豫力,恬漠待問而後對,雖有君親之大利,弗問不言,若將有大寇亂,盜賊將作,若機辟將發也,他人不知,己獨知之,雖其君親皆在,不問不言。是夫大亂之賊也!以是為人臣不忠,為子不孝,事兄不弟,交,遇人不貞良。夫執後不言之朝物,見利使己雖恐後言,君若言而未有利焉,則高拱下視,會噎為深,曰:『唯其未之學也。』用誰急,遺行遠矣。夫一道術學業仁義者,皆大以治人,小以任官,遠施周偏,近以脩身,不義不處,非理不行,務興天下之利,曲直周旋,利則止,此君子之道也。以所聞孔丘之行,則本與此相反謬也。」

齊景公問晏子曰:「孔子為人何如?」晏子不對,公又復問,不對。景公曰:「以孔丘語寡人者眾矣,俱以賢人也。今寡人問之,而子不對,何也?」晏子對曰:「嬰不肖,不足以知賢人。雖然,嬰聞所謂賢人者,入人之國必務合其君臣之親,而弭其上下之怨。孔丘之荊,知白公之謀,而奉之以石乞,君身幾滅,而白公僇。嬰聞賢人得上不虛,得下不危,言聽於君必利人,教行下必於上,是以言明而易知也,行明而易1從也,行義可明乎民,謀慮可通乎君臣。今孔丘深慮同謀以奉賊,勞思盡知以行邪,勸下亂上,教臣殺君,非賢人之行也;入人之國而與人之賊,非義之類也;知人不忠,趣之為亂,非仁義之也。逃人而後謀,避人而後言,行義不可明於民,謀慮不可通於君臣,嬰不知孔丘之有異於白公也,是以不對。」景公曰:「嗚乎!貺寡人者眾矣,非夫子,則吾終身不知孔丘之與白公同也。」

孔丘之齊見景公,景公說,欲封之以尼谿,以告晏子。晏子曰:「不可夫儒浩居而自順者也,不可以教下;好樂而淫人,不可使親治;立命而怠事,不可使守職;宗喪循哀,不可使慈民;機服勉容,不可使導眾。孔丘盛容脩飾以蠱世,弦歌鼓舞以聚徒,繁登降之禮以示儀,務趨翔之節以觀眾,博學不可使議世,勞思不可以補民1,絫壽不能盡其學,當年不能行其禮,積財不能贍其樂,繁飾邪術以營世君,盛為聲樂以淫遇民,其道不可以期世,其學不可以導眾。今君封之,以利齊俗,非所以導國先眾。」公曰:2「善!」於是厚其3禮,留其封,敬見而不問其道。孔丘乃恚,怒於景公與晏子,乃樹鴟夷子皮於田常之門,告南郭惠子以所欲為,歸於魯。有頃,閒齊將伐魯,告子貢曰:「賜乎!舉大事於今之時矣!」乃遣子貢之齊,因南郭惠子以見田常,勸之伐吳,以教高、國、鮑、晏,使毋得害田常之亂,勸越伐吳。三年之內,齊、吳破國之難,伏尸以言術數。孔丘之誅也。

孔丘為魯司寇,舍公家而奉季孫。季孫相魯君而走,季孫與邑人爭門關,決植。

孔丘窮於蔡陳之閒,藜羹不糝,十日,子路為享豚,孔丘不問肉之所由來而食;號人衣以酤酒,孔丘不問酒之所由來而飲。哀公迎孔子,席不端弗坐,割不正弗食,子路進,請曰:「何其與陳、蔡反也?」孔丘曰:「來!吾語女,曩與女為苟生,今與女為苟1義。」夫飢約則不辭妄取,以活身,贏飽則2偽行以自飾,汙邪詐偽,孰大於此!

孔丘與其門弟子閒坐,曰:「夫舜見瞽叟孰然1,此時天下圾乎!周公旦非其人也邪?何為舍其家室而託寓也?」孔丘所行,心術所至也。其徒屬弟子皆效孔丘。子貢、季路輔孔悝亂2乎衛,陽貨亂乎齊,佛肸以中牟叛,桼雕刑殘,莫大焉。夫為弟子後生,其師,必脩其言,法其行,力不足,知弗及而後已。今孔丘之行如此,儒士則可以疑矣。


\end{pinyinscope}