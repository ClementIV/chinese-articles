\article{非命下}

\begin{pinyinscope}
子墨子言曰:「凡出言談,則必可而不先立儀而言。若不先立儀而言,譬之猶運鈞之上而立朝夕焉也。我以為雖有朝夕之辯,必將終未可得而從定也。是故言有三法。何謂三法?曰:有考之者,有1原之者,有用之者。惡乎考之?考先聖大王之事。惡乎原之?察眾之耳目之請?惡乎用之?發而為政乎國,察萬民而觀之。此謂三法也。

故昔者三代聖王禹湯文武方為政乎天下之時,曰:必務舉孝子而勸之事親,尊賢良之人而教之為善。是故出政施教,賞善罰暴。且以為若此,則天下之亂也,將屬可得而治也,社稷之危也,將屬可得而定也。若以為不然,昔桀之所亂,湯治之;紂之所亂,武王治之。當此之時,世不渝而民不易,上變政而民改俗。存乎桀紂而天下亂,存乎湯武而天下治。天下之治也,湯武之力也;天下之亂也,桀紂之罪也。若以此觀之,夫安危治亂存乎上之為政也,則夫豈可謂有命哉!故昔者禹湯文武方為政乎天下之時,曰『必使飢者得食,寒者得衣,勞者得息,亂者得治』,遂得光譽令問於天下。夫豈可以為命哉?故以為其力也!今賢良之人,尊賢而好功道術,故上得其王公大人之賞,下得其萬民之譽,遂得光譽令問於天下。亦豈以為其命哉?又以為力也!然今夫有命者,不識昔也三代之聖善人與,意亡昔三代之暴不肖人與?若以說觀之,則必非昔三代聖善人也,必暴不肖人也。然今以命為有者,昔三代暴王桀紂幽厲,貴為天子,富有天下,於此乎,不而矯其耳目之欲,而從其心意之辟,外之敺騁、田獵、畢弋,內湛於酒樂,而不顧其國家百姓之政,繁為無用,暴逆百姓,遂失其宗廟。其言不曰『吾罷不肖,吾聽治不強』,必曰『吾命固將失之』。雖昔也三代罷不肖之民,亦猶此也。不能善事親戚君長,甚惡恭儉而好簡易,貪飲食而惰從事,衣食之財不足,是以身有陷乎飢寒凍餒之憂。其言不曰『吾罷不肖,吾從事不強』,又曰『吾命固將窮。』昔三代偽民亦猶此也。

昔者暴王作之,窮人1術之,此皆疑眾遲樸,先聖王之患之也,固在前矣。是以書之竹帛,鏤之金石,琢之盤盂,傳遺後世子孫。曰何書焉存?禹之總德有之曰:『允不著,惟天民不而葆,既防凶心,天加之咎,不慎厥德,天命焉葆』?仲虺之告曰:『我聞有夏,人矯天命,于下,帝式是增,用爽厥師。』彼用無為有,故謂矯,若有而謂有,夫豈為矯哉!昔者,桀執有命而行,湯為仲虺之告以非之。太誓之言也,於去發曰:『惡乎君子!天有顯德,其行甚章,為鑑不遠,在彼殷王。謂人有命,謂敬不可行,謂祭無益,謂暴無傷,上帝不常,九有以亡,上帝不順,祝降其喪,惟我有周,受之大帝。』昔者紂執有命而行,武王為太誓、去發以非之。曰:子胡不尚考之乎商周虞夏之記,從十簡之篇以尚,皆無之,將何若者也?」

是故子墨子曰:「今天下之君子之為文學出言談也,非將勤勞其惟舌,而利其脣呡也,中實將欲為其國家邑里萬民刑政者也。今也王公大人之所以蚤朝晏退,聽獄治政,終朝均分,而不敢息1怠倦者,何也?曰:彼以為強必治,不強必亂;強必寧,不強必危,故不敢怠倦。今也卿大夫之所以竭股肱之力,殫其思慮之知,內治官府,外斂關市、山林、澤梁之利,以實官府,而不敢怠倦者,何也?曰:彼以為強必貴,不強必賤;強必榮,不強必辱,故不敢怠倦。今也農夫之所以蚤出暮入,強乎耕稼樹藝,多聚叔粟,而不敢怠倦者,何也?曰:彼以為強必富,不強必貧;強必飽,不強必飢,故不敢怠倦。今也婦人之所以2夙興夜寐,強乎紡績織紝,多治麻絲葛緒捆布縿,而不敢怠倦者,何也?曰:彼以為強必富,不強必貧,強必煖,不強必寒,故不敢怠倦。今雖毋在乎王公大人,蕢若信有命而致行之,則必怠乎聽獄治政矣,卿大夫必怠乎治官府矣,農夫必怠乎耕稼樹藝矣,婦人必怠乎紡績織紝矣。王公大人怠乎聽獄治政,卿大夫怠乎治官府,則我以為天下必亂矣。農夫怠乎耕稼樹藝,婦人怠乎紡織績紝,則我以為天下衣食之財將必不足矣。若以為政乎天下,上以事天鬼,天鬼不使;下以持養百姓,百姓不利,必離散不可得用也。是以入守則不固,出誅則不勝,故雖昔者三代暴王桀紂幽厲之所以共抎其國家,傾覆其社稷者,此也。」是故子墨子言曰:「今天下之士君子,中實將欲求興天下之利,除天下之害,當若有命者之3言,不可不強非4也。曰:命者,暴王所作,窮人所術,非仁者之言也。今之為仁義者,將不可不察而強非者,此也。」


\end{pinyinscope}