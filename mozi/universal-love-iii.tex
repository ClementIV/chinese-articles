\article{兼愛下}

\begin{pinyinscope}
子墨子言曰:「仁人之事者,必務求興天下之利,除天下之害。」然當今之時,天下之害孰為大?曰:「若大國之攻小國也,大家之亂小家也,強之劫弱,眾之暴寡,詐之謀愚,貴之敖賤,此天下之害也。又與為人君者之不惠也,臣者之不忠也,父者之不慈也,子者之不孝也,此又天下之害也。又與今人之賤人,執其兵刃、毒藥、水、火,以交相虧賊,此又天下之害也。」姑嘗本原若眾害之所自生1,此胡自生?此自愛人利人生與?即必曰非然也,必曰從惡人賊人生。分名乎天下惡人而賊人者,兼與?別與?即必曰2別也。然即之交別者,果生天下之大害者與?是故別非也。」

子墨子曰:「非人者必有以易之,若非人而無以易之,譬之猶以水救火也,其說將必無可焉。」是故子墨子曰:「兼以易別。然即兼之可以易別之故何也?曰:藉為人之國,若為其國,夫誰獨舉其國以攻人之國者哉?為彼者由為己也。為人之都,若為其都,夫誰獨舉其都以伐人之都者哉?為彼猶為己也。為人之家,若為其家,夫誰獨舉其家以亂人之家者哉?為彼猶為己也,然即國、都不相攻伐,人家不相亂賊,此天下之害與?天下之利與?即必曰天下之利也。姑嘗本原若眾利之所自生,此胡自生?此自惡人賊人生與?即必曰非然也,必曰從愛人利人生。分名乎天下愛人而利人者,別與?兼與?即必曰兼也。然即之交兼者,果生天下之大利者與。」是故子墨子曰:「兼是也。且鄉吾本言曰:『仁人之事者,必務求興天下之利,除天下之害。』今吾本原兼之所生,天下之大利者也;吾本原別之所生,天下之大害者也。」是故子墨子曰:「別非而兼是者,出乎若方也。

今吾將正求與天下之利而取之,以兼為正,是以聰耳明目相與視聽乎,是以股肱畢強相為動為1宰乎,而有道肆相教誨。是以老而無妻子者,有所侍養以終其壽;幼弱孤童之無父母者,有所放依以長其身。今唯毋以兼為正,即若其利也,不識天下之士,所以皆聞兼而非者,其故何也?」

然而天下之士非兼者之言,猶未止也。曰:「即善矣。雖然,豈可用哉?」子墨子曰:「用而不可,雖我亦將非之。且焉有善而不可用者?姑嘗兩而進之。誰以為二士,使其一士者執別,使其一士者執兼。是故別士之言曰:『吾豈能為吾友之身,若為吾身,為吾友之親,若為吾親。』是故退睹其友,飢即不食,寒即不衣,疾病不侍養,死喪不葬埋。別士之言若此,行若此。兼士之言不然,行亦不然,曰:『吾聞為高士於天下者,必為其友之身,若為其身,為其友之親,若為其親,然後可以為高士於1天下。』是故退睹其友,飢則食之,寒則衣之,疾病侍養之,死喪葬埋之。兼士之言若此,行若此。若之二士2者,言相非而行相反與?當使若二士者,言必信,行必果,使言行之合猶合符節也,無言而不行也。然即敢問,今有平原廣野於此,被甲嬰冑將往戰3,死生之權未可識也;又有君大夫之遠使於巴、越、齊、荊,往來及否未及否未4可識也,然即敢問,不識將惡也家室,奉承親戚,提挈妻子,而寄託之?不識於兼之有是乎?於別之有是乎?我以為當其於此也,天下無愚夫愚婦,雖非兼之人,必寄託之於兼之有是也。此言而非兼,擇即取兼,即此言行費也。不識天下之士,所以皆聞兼而非之者,其故何也?」

然而天下之士非兼者之言,猶未止也。曰:「意可以擇士,而不可以擇君乎?」「姑嘗兩而進之。誰以為二君,使其一君者執兼,使其一君者執別,是故別君之言曰『吾惡能為吾萬民之身,若為吾身,此泰非天下之情也。人之生乎地上之無幾何也,譬之猶駟馳而過隙也』。是故退睹其萬民,飢即不食,寒即不衣,疾病不侍養,死喪不葬埋。別君之言若此,行若此。兼君之言不然,行亦不然。曰:「吾聞為明君於天下者,必先萬民之身,後為其身,然後可以為明君於天下。」是故退睹其1萬民,飢即食之,寒即衣之,疾病侍養之,死喪葬埋之。兼君之言若此,行若此。然即交若之二君者,言相非而行相反與?常使若二君者,言必信,行必果,使言行之合猶合符節也,無言而不行也。然即敢問,今歲有癘疫,萬民多有勤苦凍餒,轉死溝壑中者,既已眾矣。不識將擇之二君者,將何從也?我以為當其於此也,天下無愚夫愚婦,雖非兼者,必從兼君是也。言而非兼,擇即取兼2,此言行拂也。不識天下所以皆聞兼而非之者,其故何也?」

然而天下之士非兼者之言也,猶1未止也。曰:「兼即仁矣義矣,雖然,豈可為哉?吾譬兼之不可為也,猶挈泰山以超江河也。故兼者直願之也,夫豈可為之物哉?」子墨子曰:「夫挈泰山以趙江河,自古之及今,生民而來,未嘗有也。今若夫兼相愛、交相利,此自先聖六王者親行之。」何知先聖六王之親行之也?子墨子曰:「吾非與之並世同時,親聞其聲,見其色也。以其所書於竹帛,鏤於金石,琢於槃盂,傳遺後世子孫者知之。」《泰誓》曰:「文王若日若月,乍照光於四方於西土。」即此言文王之兼愛天下之博大也,譬之日月,兼照天下之無有私也。即此文王兼也。雖子墨子之所謂兼者,於文王取法焉。

「且不唯《泰誓》為然,雖《禹誓》即亦猶是也。禹曰:『濟濟有群,咸聽朕言,非惟小子,敢行稱亂,蠢茲有苗,用天之罰,若予既率爾群對諸群,以征有苗。』禹之征有苗也,非以求以重富貴、干福祿、樂耳目也,以求興天下之利,除天下之害。」即此禹兼也。雖子墨子之所謂兼者,於禹求焉。

「且不唯《禹誓》為然雖《湯說》即亦猶是也。湯曰:『惟予小子履,敢用玄牡,告於上天后曰:「今天大旱,即當朕身履,未知得罪于上下,有善不敢蔽,有罪不敢赦,簡在帝心。萬方有罪,即當朕身,朕身有罪,無及萬方。」即此言湯貴為天子,富有天下,然且不憚以身為犧牲,以祠說于上帝鬼神。』即此湯兼也。雖子墨子之所謂兼者,於湯取法焉。

「且不惟《誓命》與《湯說》為然,《周詩》即亦猶是也。《周詩》曰:『王道蕩蕩,不偏不黨,王道平平,不黨不偏。其直若矢,其易若厎,君子之所履,小人之所視』,若吾言非語道之謂也,古者文武為正,均分賞賢罰暴,勿有親戚弟兄之所阿。」即此文武兼也。雖子墨子之所謂兼者,於文武取法焉。不識天下之人,所以皆聞兼而非之者,其故何也?

然而天下之非兼者之言,猶未止,曰:「意不忠親之利,而害為孝乎?」子墨子曰:「姑嘗本原之孝子之為親度者。吾不識孝子之為親度者,亦欲人愛利其親與?意欲人之惡賊其親與?以說觀之,即欲人之愛利其親也。然即吾惡先從事即得此?若我先從事乎愛利人之親,然後人報我愛利吾親乎?意我先從事乎惡人之親,然後人報我以愛利吾親乎?即必吾先從事乎愛利人之親,然後人報我以愛利吾親也。然即之交孝子者,果不得已乎,毋先從事愛利人之親者與?意以天下之孝子為遇而不足以為正乎?姑嘗本原之先王之所書,《大雅》之所道曰:『無言而不讎,無德而不報』『投我以桃,報之以李。』即此言愛人者必見愛也,而惡人者必見惡也。不識天下之士,所以皆聞兼而非之者,其故何也?

意以為難而不可為邪?嘗有難此而可為者。昔荊靈王好小要,當靈王之身,荊國之士飯不踰乎一,固據而後興,扶垣而後行。故約食為其難為也,然後為而靈王說之,未踰於世而民可移也,即求以鄉其上也。昔者越王句踐好勇,教其士臣三年,以其知為未足以知之也,焚舟失火,鼓而進之,其士偃前列,伏水火而死,有不可勝數也。當此之時,不鼓而退也,越國之士可謂顫矣。故焚身為其難為也,然後為之越王說之,未踰於世而民可移也,即求以鄉上也。昔者晉文公好苴服,當文公之時,晉國之士,大布之衣,牂羊之裘,練帛之冠,且苴之屨,入見文公,出以踐之朝。故苴服為其難為也,然後為而文公說之,未踰於世而民可移也,即求以鄉其上也。是故約食、焚舟、苴服,此天下之至難為也,然後為而上說之,未踰於世而民可移也。何故也?即求以鄉其上也。今若夫兼相愛,交1相利,此其有利且易為也,不可勝計也,我以為則無有上說之者而已矣。苟有上說之者,勸之以賞譽,威之以刑罰,我以為人之於就兼相愛交相利也,譬之猶火之就上,水之就下也,不可防止於天下。

故兼者聖王之道也,王公大人之所以安也,萬民衣食之所以足也。故君子莫若審兼而務行之,為人君必惠,為人臣必忠,為人父必慈,為人子必孝,為人兄必友,為人弟必悌。故君子莫若欲為惠君、忠臣、慈父、孝子、友兄、悌弟,當若兼之不可不行也,此聖王之道而萬民之大利也。


\end{pinyinscope}