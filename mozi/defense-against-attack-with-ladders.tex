\article{備梯}

\begin{pinyinscope}
禽滑釐子事子墨子三年,手足胼胝,面目黧黑,役身給使,不敢問欲。子墨子其哀之,乃管酒塊脯,寄于大山昧葇坐之,以樵禽子。禽子再拜而嘆。子墨子曰:「亦何欲乎?」禽子再拜再拜曰:「敢問守道?」

子墨子曰:「姑亡,姑亡。古有亓術者,內不親民,外不約治,以少閒眾,以弱輕強,身死國亡,為天下笑。子亓慎之,恐為身薑。」

禽子再拜頓首,願遂問守道。曰:「敢問客眾而勇,煙資吾池,軍卒並進,雲梯既施,攻備已具,武士又多,爭上吾城,為之柰何?」

子墨子曰: 「問雲梯之守邪?雲梯者重器也,亓動移甚難。守為行城,雜樓相見,以環亓中。以適廣陜為度,環中藉幕,毋廣亓處。行城之法,高城二十尺,上加堞,廣十尺,左右出巨各二十尺,高、廣如行城之法。

為爵穴煇鼠,施荅亓外,機、衝、錢、城,廣與隊等,雜亓閒以鐫、劍,持衝十人,執劍五人,皆以有力者。令案目者視適,以鼓發之,夾而射之,重而射之,披機藉之,城上繁下矢、石、沙、炭以雨之,薪火、水湯以濟之,審賞行罰,以靜為故,從之以急,毋使生慮。若此,則雲梯之攻敗矣。

守為行堞,堞高六尺而一等,施劍亓面,以機發之,衝至則去之,不至則施之。雀穴三尺而一,蒺藜投必遂而立,以車推引之。

裾城外,去城十尺,裾厚十尺。伐裾,小大盡本斷之,以十尺為傳,雜而深埋之,堅築,毋使可拔。二十步一殺,殺有一鬲,鬲厚十尺,殺有兩門,門廣五尺。裾門一,施淺埋,弗築,令易拔。城希裾門而直桀。

縣火,四尺一鉤樴,五步一灶,灶門有鑪炭。令適人盡入,煇火燒門,縣火次之。出載而立,亓廣終隊。兩載之閒一火,皆立而待鼓而然火,即具發之。適人除火而復攻,縣火復下,適人甚病,故引兵而去。則令我死士左右出穴門擊潰師,令賁士、主將皆聽城鼓之音而出,又聽城鼓之音而入。因素出兵施伏,夜半城上四面鼓噪,適人必或,有此必破軍殺將。以白衣為服,以號相得,若此,則雲梯之攻敗矣。」


\end{pinyinscope}