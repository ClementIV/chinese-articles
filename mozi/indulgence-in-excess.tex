\article{辭過}

\begin{pinyinscope}
子墨子曰:古之民,未知為宮室1時,就陵阜而居,穴而處,下潤濕傷民,故聖王作為宮室。為宮室之法,曰:室2高足以辟潤濕,邊足以圉風寒,上足以待雪霜雨露,宮牆之高,足以別男女之禮,謹此則止。費凡3財勞力,不加利者,不為也。役,脩其城郭,則民勞而不傷;以其常正,收其租稅,則民費而不病。民所苦者非此也,苦於厚作斂於百姓。4是故聖王作為宮室,便於生,不以為觀樂也。作為衣服帶履,便於身,不以為辟怪也,故節於身,誨於民,是以天下之民可得而治,財用可得而足。

當今之主,其為宮室則與此異矣。必厚作斂於百姓,暴奪民衣食之財,以為宮室,臺榭曲直之望,青黃刻鏤之飾。為宮室若此,故左右皆法象之,是以其財不足以待凶饑、振1孤寡,故國貧而民難治也。君實欲天下之治,而惡其亂也,當為宮室不可不節。

古之民,未知為衣服時,衣皮帶茭,冬則不輕而溫,夏則不輕而凊。聖王以為不中人之情,故作誨婦人治役,脩其城郭,則民勞而不傷;以其常正,收其租稅,則民費而不病。民所苦者非此也,苦於厚作斂於百姓。1絲麻,梱布絹,以為民衣。為衣服之法:冬則練帛之中,足以為輕且煖;夏則絺綌之中,足以為輕且2凊,謹此則止。故聖人之3為衣服,適身體和肌膚而足矣。非榮耳目而觀愚民也。當是之時,堅車良馬不知貴也,刻鏤文采,不知喜也。何則?其所道之然。故民衣食之財,家足以待旱水凶饑者,何也?得其所以自養之情,而不感於外也。是以其民儉而易治,其君用財節而易贍也。府庫實滿,足以待不然。兵革不頓,士民不勞,足以征不服。故霸王之業,可行於天下矣。

當今之主,其為衣服則與此異矣,冬則輕煗,夏則輕凊,皆已具矣。必厚作斂於百姓,暴奪民衣食之財,以為錦繡文采靡曼之衣,鑄金以為鉤,珠玉以為珮,女工作文采,男工作刻鏤,以為1身服,此非云益煗之情也。單財勞力,畢歸之於無用也2,以此觀之,其為衣服非為身體,皆為觀好,是以其民淫僻而難治,其君奢侈而難諫也。夫以奢侈之君,御妤淫僻之民,欲國無亂,不可得也。君實欲天下之治而惡其亂,當為衣服不可不節。

古之民未知為飲食時,素食而分處,故聖人作誨男耕稼樹藝,以為民食。其為食也,足以增氣充虛,彊體適腹而巳矣。故其用財節,其自養儉,民富國治。今則不然,厚作斂於百姓,以為美食芻豢,蒸炙魚鱉,大國累百器,小國累十器,前方丈,目不能遍視,手不能遍操,口不能遍味,冬則凍冰,夏則餲1饐,人君為飲食如此,故左右象之。是以富貴者奢侈,孤寡者凍餒,雖2欲無亂,不可得也。君實欲天下治而惡其亂,當為食飲,不可不節。

古之民未知為舟車時,重任不移,遠道不至,故聖王作為舟車,以便民之事。其為舟車也,完1固輕利,可以任重致遠,其為用財少,而為利多,是以民樂而利之。故法令不急而行,民不勞而上2足用,故民歸之。

當今之主,其為舟車與此異矣。完1固輕利皆已具,必厚作斂於百姓,以飾舟車。飾車以文采,飾舟以刻鏤,女子廢其紡織而脩文采,故民寒。男子離其耕稼而脩刻鏤,故民饑。人君為舟車若此,故左右象之,是以其民饑寒並至,故為姦袤。姦邪2多則刑罰深,刑罰深則國亂。君實欲天下之治而惡其亂,當為舟車,不可不節。

凡回於天地之間,包於四海之內,天壤之情,陰陽之和,莫不有也,雖至聖不能更也。何以知其然?聖人有傳:天地也,則曰上下;四時也,則曰陰陽;人情也,則曰男女;禽獸也,則曰牡牝雄雌也。真天壤之情,雖有先王不能更也。雖上世至聖,必蓄私,不以傷行,故民無怨。宮無拘女,故天下無寡夫。內無拘女,外無寡夫,故天下之民眾。當今之君,其蓄私也,大國拘女累千,小國累百,是以天下之男多寡無妻,女多拘無夫,男女1失時,故民少。君實欲民之眾而惡其寡,當蓄私不可不節。

凡此五者,聖人之所儉節也,小人之所淫佚也。儉節則昌,淫佚則亡,此五者不可不節。夫婦節而天地和,風雨節而五穀孰1,衣服節而肌膚和。


\end{pinyinscope}