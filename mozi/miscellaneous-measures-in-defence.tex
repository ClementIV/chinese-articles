\article{雜守}

\begin{pinyinscope}
禽子問曰:「客眾而勇,輕意見威,以駭主人。薪土俱上,以為羊坽,積土為高,以臨吾民,蒙櫓俱前,遂屬之城,兵弩俱上,為之柰何?」

子墨子曰:「子問羊坽之守邪?羊坽者攻之拙者也,足以勞卒,不足以害城。羊坽之攻,遠攻則遠禦,近攻則近禦,害不至城。矢石無休,左右趣射,蘭為柱後,望已固。厲吾銳卒,慎無使顧,守者重下,攻者輕去。養勇高奮,民心百倍,多執數賞,卒乃不怠。

作土不休,不能禁禦,遂屬之城,以禦雲梯之法應之。凡待堙、衝、雲梯、臨之法,必應城以禦之,石不足,則以木槨之。左百步,右百步,繁下矢、石、沙、灰以雨之,薪火、水湯以濟之。選厲銳卒,慎無使顧,審賞行罰,以靜為故,從之以急,無使生慮,養勇高憤,民心百倍,多執數賞,卒乃不怠。衝、臨、梯皆以衝衝之。

渠長丈五尺,其埋者三尺,夫長丈二尺。渠廣丈六尺,其梯丈二尺,荅之垂者四尺。樹渠無傅堞五寸,梯渠十丈一梯,渠荅大數,里二百五十八,渠荅百二十九。

諸外道可要塞以難寇,其甚害者為築三亭,亭三隅,織女之,令能相救。諸鉅阜、山林、溝瀆、丘陵、阡陌、郭門、若閻術,可要塞及為微識,可以跡知往來者少多及所伏藏之處。

葆民,先舉城中官府、民宅、室署,大小調處,葆者或欲從兄弟、知識者許之。外宅粟米、畜產、財物諸可以佐城者,送入城中,事即急,則使積門內。民獻粟米布帛金錢牛馬畜產,皆為置平價,與主券書之。

使人各得其所長,天下事當,均其分職,天下事得,皆其所喜,天下事備,強弱有數,天下事具矣。

築郵亭者圜之,高三丈以上,令倚殺。為臂梯,梯兩臂長三丈,連版三尺,報以繩連之。塹再匝,為縣梁。壟灶,亭一鼓。寇烽、警烽、亂烽,傳火以次應之,至主國止,其事急者引而上下之。烽火已舉,輒五鼓傳,又以火屬之,言寇所從來者少多,毋弇建,去來屬次烽勿罷。望見寇,舉一烽;入境,舉二烽;射要,舉三烽三鼓;郭會,舉四烽四鼓;城會,舉五烽五鼓;夜以火,如此數。守烽者事急。

候無過五十,寇至堞,隨去之,無弇逮。日暮出之,令皆為微識。距阜、山林,皆令可以跡,平明而跡,跡者無下里三人,各立其表,城上應之。候出置田表,斥坐郭內外立旗幟,卒半在內,令多少無可知。即有警,舉外表,見寇,舉次表。城上以麾指之,斥坐鼓整旗,以戰備從麾所指。田者男子以戰備從斥,女子亟走入。即見寇,鼓傳到城止。守表者三人,更立郵表而望,守數令騎若吏行旁視,有以知其所為。其曹一鼓。望見寇,鼓傳到城止。

斗食,終歲三十六石;參食,終歲二十四石;四食,終歲十八石;五食,終歲十四石四斗;六食,終歲十二石。斗食食五升,參食食參升小半,四食食二升半,五食食二升,六食食一升大半,日再食。救死之時,日二升者二十日,日三升者三十日,日四升者四十日,如是,而民免於九十日之約矣。

寇近,亟收諸離鄉金器,若銅鐵及他可以佐守事者。先舉縣官室居、官府不急者,材之大小長短及凡數,即急先發。寇薄,發屋,伐木,雖有請謁,勿聽。入柴,勿積魚鱗簪,當遂,令易取也。材木不能盡入者,燔之,無令寇得用之。積木,各以長短大小惡美形相從,城四面外各積其內,諸木大者皆以為關鼻,乃積聚之。

城守司馬以上,父母、昆弟、妻子,有質在主所,乃可以堅守。署都司空,大城四人,候二人,縣候面一,亭尉、次司空、亭一人。吏侍守所者才足,廉信,父母昆弟妻子又在葆宮中者,乃得為侍吏。諸吏必有質,乃得任事。守大門者二人,夾門而立,令行者趣其外。各四戟,夾門立,而其人坐其下。吏日五閱之,上逋者名。

池外廉有要有害,必為疑人,令往來行夜者射之,誅其疏者。牆外水中,為竹箭,箭尺廣二步,箭下於水五寸,雜長短,前外廉三行,外外向,內亦內向。三十步一弩廬,廬廣十尺,長丈二尺。

隊有急,亟發其近者往佐,其次襲其處。

守節出入,使主節必疏書,署其情,令若其事,而須其還報以檢驗之。節出,使所出門者,輒言節出時操者名。

百步一隊。

閤通守舍,相錯穿室。治復道,為築墉,墉善其上。

取蔬,令民家有三年蓄蔬食,以備湛旱、歲不為。常令邊縣豫種畜芫、芒、烏喙、椒葉,外宅溝井填可,塞不可,置此其中。安則示以危,危示以安。

寇至,諸門戶令皆鑿而幎竅之,各為二類,一鑿而屬繩,繩長四尺,大如指。寇至,先殺牛、羊、彘、雞、狗、鳧、鴈,收其皮革、筋、角、脂、腦、羽皆剝之。使檟桐栗,為鐵錍,後蘭為衡柱。事急,卒不可遠,令掘外宅林。課多少,若治城上為擊,三隅之。重五斤以上諸林木,渥水中,無過一筏。塗茅屋若積薪者,厚五寸以上。吏各舉其部界中財物,可以佐守備者上。

有讒人,有利人,有惡人,有善人,有長人,有謀士,有勇士,有巧士,有使士,有內人者,外人者,有善人者,有善鬥人者,守必察其所以然者,應名乃納之。民相惡,若議吏,吏所解,皆札書藏之,以須告者之至以參驗之。諸小睨五尺,不可卒者,為署吏,令給事官府若舍。

藺石、厲矢,諸材器用,皆謹部,各有積分數。為軺車以梓,載矢以軺車,輪轂,廣十尺,轅長丈,為四輪,廣六尺。為板箱,長與轅等,高四尺,善蓋上治中,令可載矢。」

子墨子曰:「凡不守者有五:城大人少,一不守也;城小人眾,二不守也;人眾食寡,三不守也;市去城遠,四不守也;蓄積在外,富人在墟,五不守也。率萬家而城方三里。」


\end{pinyinscope}