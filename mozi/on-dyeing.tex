\article{所染}

\begin{pinyinscope}
子墨子言見染絲者而歎曰:「染於蒼則蒼,染於黃則黃。所入者變,其色亦變。五入必而已,則為五色矣。故染不可不慎也。」

非獨染絲然也,國亦有染。舜染於許由1、伯陽,禹染於皋陶、伯益,湯染於伊尹、仲虺,武王染於太公、周公。此四王者所染當,故王天下,立為天子,功名蔽天地。舉天下之仁義顯人,必稱此四王者。

夏桀染於干辛、推哆,殷紂染於崇侯、惡來,厲王染於厲公長父、榮夷終,幽王染於傅公夷、蔡公穀。此四王者所染不當,故國殘身死,為天下僇。舉天下不義辱人,必稱此四王者。

齊桓染於管仲、鮑叔,晉文染於舅犯、高偃,楚莊染於孫叔、沈尹,吳闔閭染於伍員、文義,越句踐染於范蠡大夫種。此五君者1所染當,故霸諸侯,功名傅於後世。

范吉射染於長柳朔、王胜,中行寅染於籍秦、高彊,吳夫差染於王孫雒、太宰嚭,知伯搖染於智國、張武,中山尚染於魏義、偃長,宋康染於唐鞅、佃1不禮。此六君者所染不當,故國家殘亡,身為刑戮,宗廟破滅,絕無後類,君臣離散,民人流亡。舉天下之貪暴苛擾者,必稱此六君也。

凡君之所以安者,何也?以其行理也,行理性於染當。故善為君者,勞於論人,而佚於治官。不能為君者,傷形費神,愁心勞意,然國逾危,身逾辱。此六君者,非不重其國,愛其身也,以不知要故也。不知要者,所染不當也。

非獨國有染也,士亦有染。其友皆好仁義,淳謹畏令,則家日益,身日安,名日榮,處官得其理矣,則段干木、禽子、傅說之徒是也。其友皆好矜奮,創作比周,則家日損,身日危,名日辱,處官失其理矣,則子西、易牙、豎刀1之徒是也。《詩》曰:「必擇所堪。」必謹所堪者,此之謂也。


\end{pinyinscope}