\article{循吏列傳}

\begin{pinyinscope}
太史公曰:法令所以導民也,刑罰所以禁姦也。文武不備,良民懼然身修者,官未曾亂也。奉職循理,亦可以為治,何必威嚴哉?

孫叔敖者,楚之處士也。虞丘相進之於楚莊王,以自代也。三月為楚相,施教導民,上下和合,世俗盛美,政緩禁止,吏無姦邪,盜賊不起。秋冬則勸民山採,春夏以水,各得其所便,民皆樂其生。

莊王以為幣輕,更以小為大,百姓不便,皆去其業。市令言之相曰:「市亂,民莫安其處,次行不定。」相曰:「如此幾何頃乎?」市令曰:「三月頃。」相曰:「罷,吾今令之復矣。」後五日,朝,相言之王曰:「前日更幣,以為輕。今市令來言曰『市亂,民莫安其處,次行之不定』。臣請遂令復如故。」王許之,下令三日而市復如故。

楚民俗好庳車,王以為庳車不便馬,欲下令使高之。相曰:「令數下,民不知所從,不可。王必欲高車,臣請教閭里使高其梱。乘車者皆君子,君子不能數下車。」王許之。居半歲,民悉自高其車。

此不教而民從其化,近者視而效之,遠者四面望而法之。故三得相而不喜,知其材自得之也;三去相而不悔,知非己之罪也。

子產者,鄭之列大夫也。鄭昭君之時,以所愛徐摯為相,國亂,上下不親,父子不和。大宮子期言之君,以子產為相。為相一年,豎子不戲狎,斑白不提挈,僮子不犁畔。二年,市不豫賈。三年,門不夜關,道不拾遺。四年,田器不歸。五年,士無尺籍,喪期不令而治。治鄭二十六年而死,丁壯號哭,老人兒啼,曰:「子產去我死乎!民將安歸?」

公儀休者,魯博士也。以高弟為魯相。奉法循理,無所變更,百官自正。使食祿者不得與下民爭利,受大者不得取小。

客有遺相魚者,相不受。客曰:「聞君嗜魚,遺君魚,何故不受也?」相曰:「以嗜魚,故不受也。今為相,能自給魚;今受魚而免,誰復給我魚者?吾故不受也。」

食茹而美,拔其園葵而棄之。見其家織布好,而疾出其家婦,燔其機,云「欲令農士工女安所讎其貨乎」?

石奢者,楚昭王相也。堅直廉正,無所阿避。行縣,道有殺人者,相追之,乃其父也。縱其父而還自系焉。使人言之王曰:「殺人者,臣之父也。夫以父立政,不孝也;廢法縱罪,非忠也;臣罪當死。」王曰:「追而不及,不當伏罪,子其治事矣。」石奢曰:「不私其父,非孝子也;不奉主法,非忠臣也。王赦其罪,上惠也;伏誅而死,臣職也。」遂不受令,自刎而死。

李離者,晉文公之理也。過聽殺人,自拘當死。文公曰:「官有貴賤,罰有輕重。下吏有過,非子之罪也。」李離曰:「臣居官為長,不與吏讓位;受祿為多,不與下分利。今過聽殺人,傅其罪下吏,非所聞也。」辭不受令。文公曰:「子則自以為有罪,寡人亦有罪邪?」李離曰:「理有法,失刑則刑,失死則死。公以臣能聽微決疑,故使為理。今過聽殺人,罪當死。」遂不受令,伏劍而死。

太史公曰:孫叔敖出一言,郢市復。子產病死,鄭民號哭。公儀子見好布而家婦逐。石奢縱父而死,楚昭名立。李離過殺而伏劍,晉文以正國法。


\end{pinyinscope}