\article{扁鵲倉公列傳}

\begin{pinyinscope}
扁鵲者,勃海郡鄭人也,姓秦氏,名越人。少時為人舍長。舍客長桑君過,扁鵲獨奇之,常謹遇之。長桑君亦知扁鵲非常人也。出入十餘年,乃呼扁鵲私坐,閒與語曰:「我有禁方,年老,欲傳與公,公毋泄。」扁鵲曰:「敬諾。」乃出其懷中藥予扁鵲:「飲是以上池之水,三十日當知物矣。」乃悉取其禁方書盡與扁鵲。忽然不見,殆非人也。扁鵲以其言飲藥三十日,視見垣一方人。以此視病,盡見五藏癥結,特以診脈為名耳。為醫或在齊,或在趙。在趙者名扁鵲。

當晉昭公時,諸大夫彊而公族弱,趙簡子為大夫,專國事。簡子疾,五日不知人,大夫皆懼,於是召扁鵲。扁鵲入視病,出,董安于問扁鵲,扁鵲曰:「血脈治也,而何怪!昔秦穆公嘗如此,七日而寤。寤之日,告公孫支與子輿曰:『我之帝所甚樂。吾所以久者,適有所學也。帝告我:「晉國且大亂,五世不安。其後將霸,未老而死。霸者之子且令而國男女無別。」』公孫支書而藏之,秦策於是出。夫獻公之亂,文公之霸,而襄公敗秦師於殽而歸縱淫,此子之所聞。今主君之病與之同,不出三日必閒,閒必有言也。」

居二日半,簡子寤,語諸大夫曰:「我之帝所甚樂,與百神游於鈞天,廣樂九奏萬舞,不類三代之樂,其聲動心。有一熊欲援我,帝命我射之,中熊,熊死。有羆來,我又射之,中羆,羆死。帝甚喜,賜我二笥,皆有副。吾見兒在帝側,帝屬我一翟犬,曰:『及而子之壯也以賜之。』帝告我:『晉國且世衰,七世而亡。嬴姓將大敗周人於范魁之西,而亦不能有也。』」董安于受言,書而藏之。以扁鵲言告簡子,簡子賜扁鵲田四萬畝。

其後扁鵲過虢。虢太子死,扁鵲至虢宮門下,問中庶子喜方者曰:「太子何病,國中治穰過於眾事?」中庶子曰:「太子病血氣不時,交錯而不得泄,暴發於外,則為中害。精神不能止邪氣,邪氣畜積而不得泄,是以陽緩而陰急,故暴蹷而死。」扁鵲曰:「其死何如時?」曰:「雞鳴至今。」曰:「收乎?」曰:「未也,其死未能半日也。」「言臣齊勃海秦越人也,家在於鄭,未嘗得望精光侍謁於前也。聞太子不幸而死,臣能生之。」中庶子曰:「先生得無誕之乎?何以言太子可生也!臣聞上古之時,醫有俞跗,治病不以湯液醴灑,鑱石撟引,案扤毒熨,一撥見病之應,因五藏之輸,乃割皮解肌,訣脈結筋,搦髓腦,揲荒爪幕,湔浣腸胃,漱滌五藏,練精易形。先生之方能若是,則太子可生也;不能若是而欲生之,曾不可以告咳嬰之兒。」終日,扁鵲仰天嘆曰:「夫子之為方也,若以管窺天,以郄視文。越人之為方也,不待切脈望色聽聲寫形,言病之所在。聞病之陽,論得其陰;聞病之陰,論得其陽。病應見於大表,不出千里,決者至眾,不可曲止也。子以吾言為不誠,試入診太子,當聞其耳鳴而鼻張,循其兩股以至於陰,當尚溫也。」

中庶子聞扁鵲言,目眩然而不瞚,舌撟然而不下,乃以扁鵲言入報虢君。虢君聞之大驚,出見扁鵲於中闕,曰:「竊聞高義之日久矣,然未嘗得拜謁於前也。先生過小國,幸而舉之,偏國寡臣幸甚。有先生則活,無先生則棄捐填溝壑,長終而不得反。」言末卒,因噓唏服臆,魂精泄橫,流涕長潸,忽忽承睫,悲不能自止,容貌變更。扁鵲曰:「若太子病,所謂『尸蹷』者也。夫以陽入陰中,動胃繵緣,中經維絡,別下於三焦、膀胱,是以陽脈下遂,陰脈上爭,會氣閉而不通,陰上而陽內行,下內鼓而不起,上外絕而不為使,上有絕陽之絡,下有破陰之紐,破陰絕陽,(之)色[已]廢脈亂,故形靜如死狀。太子未死也。夫以陽入陰支蘭藏者生,以陰入陽支蘭藏者死。凡此數事,皆五藏蹙中之時暴作也。良工取之,拙者疑殆。」

扁鵲乃使弟子子陽厲鍼砥石,以取外三陽五會。有閒,太子蘇。乃使子豹為五分之熨,以八減之齊和煮之,以更熨兩脅下。太子起坐。更適陰陽,但服湯二旬而復故。故天下盡以扁鵲為能生死人。扁鵲曰:「越人非能生死人也,此自當生者,越人能使之起耳。」

扁鵲過齊,齊桓侯客之。入朝見,曰:「君有疾在腠理,不治將深。」桓侯曰:「寡人無疾。」扁鵲出,桓侯謂左右曰:「醫之好利也,欲以不疾者為功。」後五日,扁鵲復見,曰:「君有疾在血脈,不治恐深。」桓侯曰:「寡人無疾。」扁鵲出,桓侯不悅。後五日,扁鵲復見,曰;「君有疾在腸胃閒,不治將深。」桓侯不應。扁鵲出,桓侯不悅。後五日,扁鵲復見,望見桓侯而退走。桓侯使人問其故。扁鵲曰:「疾之居腠理也,湯熨之所及也;在血脈,鍼石之所及也;其在腸胃,酒醪之所及也;其在骨髓,雖司命無柰之何。今在骨髓,臣是以無請也。」後五日,桓侯體病,使人召扁鵲,扁鵲已逃去。桓侯遂死。

使聖人預知微,能使良醫得蚤從事,則疾可已,身可活也。人之所病,病疾多;而醫之所病,病道少。故病有六不治:驕恣不論於理,一不治也;輕身重財,二不治也;衣食不能適,三不治也;陰陽并,藏氣不定,四不治也;形羸不能服藥,五不治也;信巫不信醫,六不治也。有此一者,則重難治也。

扁鵲名聞天下。過邯鄲,聞貴婦人,即為帶下醫;過雒陽,聞周人愛老人,即為耳目痹醫;來入咸陽,聞秦人愛小兒,即為小兒醫:隨俗為變。秦太醫令李醯自知伎不如扁鵲也,使人刺殺之。至今天下言脈者,由扁鵲也。

太倉公者,齊太倉長,臨菑人也,姓淳于氏,名意。少而喜醫方術。高后八年,更受師同郡元里公乘陽慶。慶年七十餘,無子,使意盡去其故方,更悉以禁方予之,傳黃帝、扁鵲之脈書,五色診病,知人死生,決嫌疑,定可治,及藥論,甚精。受之三年,為人治病,決死生多驗。然左右行游諸侯,不以家為家,或不為人治病,病家多怨之者。

文帝四年中,人上書言意,以刑罪當傳西之長安。意有五女,隨而泣。意怒,罵曰:「生子不生男,緩急無可使者!」於是少女緹縈傷父之言,乃隨父西。上書曰:「妾父為吏,齊中稱其廉平,今坐法當刑。妾切痛死者不可復生而刑者不可復續,雖欲改過自新,其道莫由,終不可得。妾願入身為官婢,以贖父刑罪,使得改行自新也。」書聞,上悲其意,此歲中亦除肉刑法。

意家居,詔召問所為治病死生驗者幾何人也,主名為誰。

詔問故太倉長臣意:「方伎所長,及所能治病者?有其書無有?皆安受學?受學幾何歲?嘗有所驗,何縣里人也?何病?醫藥已,其病之狀皆何如?具悉而對。」臣意對曰:

自意少時,喜醫藥,醫藥方試之多不驗者。至高后八年,得見師臨菑元里公乘陽慶。慶年七十餘,意得見事之。謂意曰:「盡去而方書,非是也。慶有古先道遺傳黃帝、扁鵲之脈書,五色診病,知人生死,決嫌疑,定可治,及藥論書,甚精。我家給富,心愛公,欲盡以我禁方書悉教公。」臣意即曰:「幸甚,非意之所敢望也。」臣意即避席再拜謁,受其脈書上下經、五色診、奇咳術、揆度陰陽外變、藥論、石神、接陰陽禁書,受讀解驗之,可一年所。明歲即驗之,有驗,然尚未精也。要事之三年所,即嘗已為人治,診病決死生,有驗,精良。今慶已死十年所,臣意年盡三年,年三十九歲也。

齊侍御史成自言病頭痛,臣意診其脈,告曰:「君之病惡,不可言也。」即出,獨告成弟昌曰:「此病疽也,內發於腸胃之閒,後五日當擁腫,後八日嘔膿死。」成之病得之飲酒且內。成即如期死。所以知成之病者,臣意切其脈,得肝氣。肝氣濁而靜,此內關之病也。脈法曰「脈長而弦,不得代四時者,其病主在於肝。和即經主病也,代則絡脈有過」。經主病和者,其病得之筋髓里。其代絕而脈賁者,病得之酒且內。所以知其後五日而擁腫,八日嘔膿死者,切其脈時,少陽初代。代者經病,病去過人,人則去。絡脈主病,當其時,少陽初關一分,故中熱而膿未發也,及五分,則至少陽之界,及八日,則嘔膿死,故上二分而膿發,至界而擁腫,盡泄而死。熱上則熏陽明,爛流絡,流絡動則脈結發,脈結發則爛解,故絡交。熱氣已上行,至頭而動,故頭痛。

齊王中子諸嬰兒小子病,召臣意診切其脈,告曰:「氣鬲病。病使人煩懣,食不下,時嘔沫。病得之(少)[心]憂,數忔食飲。」臣意即為之作下氣湯以飲之,一日氣下,二日能食,三日即病愈。所以知小子之病者,診其脈,心氣也,濁躁而經也,此絡陽病也。脈法曰「脈來數疾去難而不一者,病主在心」。周身熱,脈盛者,為重陽。重陽者,逿心主。故煩懣食不下則絡脈有過,絡脈有過則血上出,血上出者死。此悲心所生也,病得之憂也。

齊郎中令循病,眾醫皆以為蹙入中,而刺之。臣意診之,曰:「湧疝也,令人不得前後溲。」循曰:「不得前後溲三日矣。」臣意飲以火齊湯,一飲得前[后]溲,再飲大溲,三飲而疾愈。病得之內。所以知循病者,切其脈時,右口氣急,脈無五藏氣,右口脈大而數。數者中下熱而湧,左為下,右為上,皆無五藏應,故曰湧疝。中熱,故溺赤也。

齊中御府長信病,臣意入診其脈,告曰:「熱病氣也。然暑汗,脈少衰,不死。」曰:「此病得之當浴流水而寒甚,已則熱。」信曰:「唯,然!往冬時,為王使於楚,至莒縣陽周水,而莒橋梁頗壞,信則擥車轅未欲渡也,馬驚,即墮,信身入水中,幾死,吏即來救信,出之水中,衣盡濡,有閒而身寒,已熱如火,至今不可以見寒。」臣意即為之液湯火齊逐熱,一飲汗盡,再飲熱去,三飲病已。即使服藥,出入二十日,身無病者。所以知信之病者,切其脈時,并陰。脈法曰「熱病陰陽交者死」。切之不交,并陰。并陰者,脈順清而愈,其熱雖未盡,猶活也。腎氣有時閒濁,在太陰脈口而希,是水氣也。腎固主水,故以此知之。失治一時,即轉為寒熱。

齊王太后病,召臣意入診脈,曰:「風癉客脬,難於大小溲,溺赤。」臣意飲以火齊湯,一飲即前後溲,再飲病已,溺如故。病得之流汗出循。循者,去衣而汗晞也。所以知齊王太后病者,臣意診其脈,切其太陰之口,溼然風氣也。脈法曰「沈之而大堅,浮之而大緊者,病主在腎」。腎切之而相反也,脈大而躁。大者,膀胱氣也;躁者,中有熱而溺赤。

齊章武里曹山跗病,臣意診其脈,曰:「肺消癉也,加以寒熱。」即告其人曰:「死,不治。適其共養,此不當醫治。」法曰「後三日而當狂,妄起行,欲走;後五日死」。即如期死。山跗病得之盛怒而以接內。所以知山跗之病者,臣意切其脈,肺氣熱也。脈法曰「不平不鼓,形獘」。此五藏高之遠數以經病也,故切之時不平而代。不平者,血不居其處;代者,時參擊并至,乍躁乍大也。此兩絡脈絕,故死不治。所以加寒熱者,言其人尸奪。尸奪者,形獘;形獘者,不當關灸鑱石及飲毒藥也。臣意未往診時,齊太醫先診山跗病,灸其足少陽脈口,而飲之半夏丸,病者即泄注,腹中虛;又灸其少陰脈,是壞肝剛絕深,如是重損病者氣,以故加寒熱。所以後三日而當狂者,肝一絡連屬結絕乳下陽明,故絡絕,開陽明脈,陽明脈傷,即當狂走。後五日死者,肝與心相去五分,故曰五日盡,盡即死矣。

齊中尉潘滿如病少腹痛,臣意診其脈,曰:「遺積瘕也。」臣意即謂齊太仆臣饒、內史臣繇曰:「中尉不復自止於內,則三十日死。」後二十餘日,溲血死。病得之酒且內。所以知潘滿如病者,臣意切其脈深小弱,其卒然合合也,是脾氣也。右脈口氣至緊小,見瘕氣也。以次相乘,故三十日死。三陰俱摶者,如法;不俱摶者,決在急期;一摶一代者,近也。故其三陰摶,溲血如前止。

陽虛侯相趙章病,召臣意。眾醫皆以為寒中,臣意診其脈曰:「迵風。」迵風者,飲食下嗌而輒出不留。法曰「五日死」,而後十日乃死。病得之酒。所以知趙章之病者,臣意切其脈,脈來滑,是內風氣也。飲食下嗌而輒出不留者,法五日死,皆為前分界法。後十日乃死,所以過期者,其人嗜粥,故中藏實,中藏實故過期。師言曰「安穀者過期,不安穀者不及期」。

濟北王病,召臣意診其脈,曰:「風蹶胸滿。」即為藥酒,盡三石,病已。得之汗出伏地。所以知濟北王病者,臣意切其脈時,風氣也,心脈濁。病法「過入其陽,陽氣盡而陰氣入」。陰氣入張,則寒氣上而熱氣下,故胸滿。汗出伏地者,切其脈,氣陰。陰氣者,病必入中,出及瀺水也。

齊北宮司空命婦出於病,眾醫皆以為風入中,病主在肺,刺其足少陽脈。臣意診其脈,曰:「病氣疝,客於膀胱,難於前後溲,而溺赤。病見寒氣則遺溺,使人腹腫。」出於病得之欲溺不得,因以接內。所以知出於病者,切其脈大而實,其來難,是蹶陰之動也。脈來難者,疝氣之客於膀胱也。腹之所以腫者,言蹶陰之絡結小肮也。蹶陰有過則脈結動,動則腹腫。臣意即灸其足蹶陰之脈,左右各一所,即不遺溺而溲清,小肮痛止。即更為火齊湯以飲之,三日而疝氣散,即愈。

故濟北王阿母自言足熱而懣,臣意告曰:「熱蹶也。」則刺其足心各三所,案之無出血,病旋已。病得之飲酒大醉。

濟北王召臣意診脈諸女子侍者,至女子豎,豎無病。臣意告永巷長曰:「豎傷脾,不可勞,法當春嘔血死。」臣意言王曰:「才人女子豎何能?」王曰:「是好為方,多伎能,為所是案法新,往年市之民所,四百七十萬,曹偶四人。」王曰:「得毋有病乎?」臣意對曰:「豎病重,在死法中。」王召視之,其顏色不變,以為不然,不賣諸侯所。至春,豎奉劍從王之廁,王去,豎後,王令人召之,即仆於廁,嘔血死。病得之流汗。流汗者,(同)法病內重,毛發而色澤,脈不衰,此亦(關)內[關]之病也。

齊中大夫病齲齒,臣意灸其左大陽明脈,即為苦參湯,日嗽三升,出入五六日,病已。得之風,及臥開口,食而不嗽。

菑川王美人懷子而不乳,來召臣意。臣意往,飲以莨碭藥一撮,以酒飲之,旋乳。臣意復診其脈,而脈躁。躁者有餘病,即飲以消石一齊,出血,血如豆比五六枚。

齊丞相舍人奴從朝入宮,臣意見之食閨門外,望其色有病氣。臣意即告宦者平。平好為脈,學臣意所,臣意即示之舍人奴病,告之曰:「此傷脾氣也,當至春鬲塞不通,不能食飲,法至夏泄血死。」宦者平即往告相曰:「君之舍人奴有病,病重,死期有日。」相君曰:「卿何以知之?」曰:「君朝時入宮,君之舍人奴盡食閨門外,平與倉公立,即示平曰,病如是者死。」相即召舍人(奴)而謂之曰:「公奴有病不?」舍人曰:「奴無病,身無痛者。」至春果病,至四月,泄血死。所以知奴病者,脾氣周乘五藏,傷部而交,故傷脾之色也,望之殺然黃,察之如死青之茲。眾醫不知,以為大蟲,不知傷脾。所以至春死病者,胃氣黃,黃者土氣也,土不勝木,故至春死。所以至夏死者,脈法曰「病重而脈順清者曰內關」,內關之病,人不知其所痛,心急然無苦。若加以一病,死中春;一愈順,及一時。其所以四月死者,診其人時愈順。愈順者,人尚肥也。奴之病得之流汗數出,(灸)[炙]於火而以出見大風也。

菑川王病,召臣意診脈,曰:「蹶上為重,頭痛身熱,使人煩懣。」臣意即以寒水拊其頭,刺足陽明脈,左右各三所,病旋已。病得之沐發未乾而臥。診如前,所以蹶,頭熱至肩。

齊王黃姬兄黃長卿家有酒召客,召臣意。諸客坐,未上食。臣意望見王后弟宋建,告曰:「君有病,往四五日,君要脅痛不可俛仰,又不得小溲。不亟治,病即入濡腎。及其未舍五藏,急治之。病方今客腎濡,此所謂『腎痹』也。」宋建曰:「然,建故有要脊痛。往四五日,天雨,黃氏諸倩見建家京下方石,即弄之,建亦欲效之,效之不能起,即復置之。暮,要脊痛,不得溺,至今不愈。」建病得之好持重。所以知建病者,臣意見其色,太陽色乾,腎部上及界要以下者枯四分所,故以往四五日知其發也。臣意即為柔湯使服之,十八日所而病愈。

濟北王侍者韓女病要背痛,寒熱,眾醫皆以為寒熱也。臣意診脈,曰:「內寒,月事不下也。」即竄以藥,旋下,病已。病得之欲男子而不可得也。所以知韓女之病者,診其脈時,切之,腎脈也,嗇而不屬。嗇而不屬者,其來難,堅,故曰月不下。肝脈弦,出左口,故曰欲男子不可得也。

臨菑氾里女子薄吾病甚,眾醫皆以為寒熱篤,當死,不治。臣意診其脈,曰:「蟯瘕。」蟯瘕為病,腹大,上膚黃麤,循之戚戚然。臣意飲以芫華一撮,即出蟯可數升,病已,三十日如故。病蟯得之於寒溼,寒溼氣宛篤不發,化為蟲。臣意所以知薄吾病者,切其脈,循其尺,其尺索刺麤,而毛美奉發,是蟲氣也。其色澤者,中藏無邪氣重病。

齊淳于司馬病,臣意切其脈,告曰:「當病迵風。迵風之狀,飲食下嗌輒後之。病得之飽食而疾走。」淳于司馬曰:「我之王家食馬肝,食飽甚,見酒來,即走去,驅疾至舍,即泄數十出。」臣意告曰:「為火齊米汁飲之,七八日而當愈。」時醫秦信在旁,臣意去,信謂左右閣都尉曰:「意以淳于司馬病為何?」曰:「以為迵風,可治。」信即笑曰:「是不知也。淳于司馬病,法當後九日死。」即後九日不死,其家復召臣意。臣意往問之,盡如意診。臣即為一火齊米汁,使服之,七八日病已。所以知之者,診其脈時,切之,盡如法。其病順,故不死。

齊中郎破石病,臣意診其脈,告曰:「肺傷,不治,當後十日丁亥溲血死。」即後十一日,溲血而死。破石之病,得之墮馬僵石上。所以知破石之病者,切其脈,得肺陰氣,其來散,數道至而不一也。色又乘之。所以知其墮馬者,切之得番陰脈。番陰脈入虛里,乘肺脈。肺脈散者,固色變也乘也。所以不中期死者,師言曰:「病者安穀即過期,不安穀則不及期」。其人嗜黍,黍主肺,故過期。所以溲血者,診脈法曰「病養喜陰處者順死,養喜陽處者逆死」。其人喜自靜,不躁,又久安坐,伏几而寐,故血下泄。

齊王侍醫遂病,自練五石服之。臣意往過之,遂謂意曰:「不肖有病,幸診遂也。」臣意即診之,告曰:「公病中熱。論曰『中熱不溲者,不可服五石』。石之為藥精悍,公服之不得數溲,亟勿服。色將發臃。」遂曰:「扁鵲曰『陰石以治陰病,陽石以治陽病』。夫藥石者有陰陽水火之齊,故中熱,即為陰石柔齊治之;中寒,即為陽石剛齊治之。」臣意曰:「公所論遠矣。扁鵲雖言若是,然必審診,起度量,立規矩,稱權衡,合色脈表裏有餘不足順逆之法,參其人動靜與息相應,乃可以論。論曰『陽疾處內,陰形應外者,不加悍藥及鑱石』。夫悍藥入中,則邪氣辟矣,而宛氣愈深。診法曰『二陰應外,一陽接內者,不可以剛藥』。剛藥入則動陽,陰病益衰,陽病益箸,邪氣流行,為重困於俞,忿發為疽。」意告之後百餘日,果為疽發乳上,入缺盆,死。此謂論之大體也,必有經紀。拙工有一不習,文理陰陽失矣。

齊王故為陽虛侯時,病甚,眾醫皆以為蹷。臣意診脈,以為痹,根在右脅下,大如覆杯,令人喘,逆氣不能食。臣意即以火齊粥且飲,六日氣下;即令更服丸藥,出入六日,病已。病得之內。診之時不能識其經解,大識其病所在。

臣意嘗診安陽武都裏成開方,開方自言以為不病,臣意謂之病苦沓風,三歲四支不能自用,使人瘖,瘖即死。今聞其四支不能用,瘖而未死也。病得之數飲酒以見大風氣。所以知成開方病者,診之,其脈法奇咳言曰「藏氣相反者死」。切之,得腎反肺,法曰「三歲死」也。

安陵阪裏公乘項處病,臣意診脈,曰:「牡疝。」牡疝在鬲下,上連肺。病得之內。臣意謂之:「慎毋為勞力事,為勞力事則必嘔血死。」處後蹴踘,要蹷寒,汗出多,即嘔血。臣意復診之,曰:「當旦日日夕死。」即死。病得之內。所以知項處病者,切其脈得番陽。番陽入虛裏,處旦日死。一番一絡者,牡疝也。

臣意曰:他所診期決死生及所治已病眾多,久頗忘之,不能盡識,不敢以對。

問臣意:「所診治病,病名多同而診異,或死或不死,何也?」對曰:「病名多相類,不可知,故古聖人為之脈法,以起度量,立規矩,縣權衡,案繩墨,調陰陽,別人之脈各名之,與天地相應,參合於人,故乃別百病以異之,有數者能異之,無數者同之。然脈法不可勝驗,診疾人以度異之,乃可別同名,命病主在所居。今臣意所診者,皆有診籍。所以別之者,臣意所受師方適成,師死,以故表籍所診,期決死生,觀所失所得者合脈法,以故至今知之。」

問臣意曰:「所期病決死生,或不應期,何故?」對曰:「此皆飲食喜怒不節,或不當飲藥,或不當鍼灸,以故不中期死也。」

問臣意:「意方能知病死生,論藥用所宜,諸侯王大臣有嘗問意者不?及文王病時,不求意診治,何故?」對曰:「趙王、膠西王、濟南王、吳王皆使人來召臣意,臣意不敢往。文王病時,臣意家貧,欲為人治病,誠恐吏以除拘臣意也,故移名數,左右不修家生,出行游國中,問善為方數者事之久矣,見事數師,悉受其要事,盡其方書意,及解論之。身居陽虛侯國,因事侯。侯入朝,臣意從之長安,以故得診安陵項處等病也。」

問臣意:「知文王所以得病不起之狀?」臣意對曰:「不見文王病,然竊聞文王病喘,頭痛,目不明。臣意心論之,以為非病也。以為肥而蓄精,身體不得搖,骨肉不相任,故喘,不當醫治。脈法曰『年二十脈氣當趨,年三十當疾步,年四十當安坐,年五十當安臥,年六十已上氣當大董』。文王年未滿二十,方脈氣之趨也而徐之,不應天道四時。後聞醫灸之即篤,此論病之過也。臣意論之,以為神氣爭而邪氣入,非年少所能復之也,以故死。所謂氣者,當調飲食,擇晏日,車步廣志,以適筋骨肉血脈,以瀉氣。故年二十,是謂『易眢』。法不當砭灸,砭灸至氣逐。」

問臣意:「師慶安受之?聞於齊諸侯不?」對曰:「不知慶所師受。慶家富,善為醫,不肯為人治病,當以此故不聞。慶又告臣意曰:『慎毋令我子孫知若學我方也。』」

問臣意:「師慶何見於意而愛意,欲悉教意方?」對曰:「臣意不聞師慶為方善也。意所以知慶者,意少時好諸方事,臣意試其方,皆多驗,精良。臣意聞菑川唐裏公孫光善為古傳方,臣意即往謁之。得見事之,受方化陰陽及傳語法,臣意悉受書之。臣意欲盡受他精方,公孫光曰:『吾方盡矣,不為愛公所。吾身已衰,無所復事之。是吾年少所受妙方也,悉與公,毋以教人。』臣意曰:『得見事侍公前,悉得禁方,幸甚。意死不敢妄傳人。』居有閒,公孫光閒處,臣意深論方,見言百世為之精也。師光喜曰:『公必為國工。吾有所善者皆疏,同產處臨菑,善為方,吾不若,其方甚奇,非世之所聞也。吾年中時,嘗欲受其方,楊中倩不肯,曰「若非其人也」。胥與公往見之,當知公喜方也。其人亦老矣,其家給富。』時者未往,會慶子男殷來獻馬,因師光奏馬王所,意以故得與殷善。光又屬意於殷曰:『意好數,公必謹遇之,其人聖儒。』即為書以意屬陽慶,以故知慶。臣意事慶謹,以故愛意也。」

問臣意曰:「吏民嘗有事學意方,及畢盡得意方不?何縣里人?」對曰:「臨菑人宋邑。邑學,臣意教以五診,歲餘。濟北王遣太醫高期、王禹學,臣意教以經脈高下及奇絡結,當論俞所居,及氣當上下出入邪[正]逆順,以宜鑱石,定砭灸處,歲餘。菑川王時遣太倉馬長馮信正方,臣意教以案法逆順,論藥法,定五味及和齊湯法。高永侯家丞杜信,喜脈,來學,臣意教以上下經脈五診,二歲餘。臨菑召裏唐安來學,臣意教以五診上下經脈,奇咳,四時應陰陽重,未成,除為齊王侍醫。」問臣意:「診病決死生,能全無失乎?」臣意對曰:「意治病人,必先切其脈,乃治之。敗逆者不可治,其順者乃治之。心不精脈,所期死生視可治,時時失之,臣意不能全也。」

太史公曰:女無美惡,居宮見妒;士無賢不肖,入朝見疑。故扁鵲以其伎見殃,倉公乃匿跡自隱而當刑。緹縈通尺牘,父得以後寧。故老子曰「美好者不祥之器」,豈謂扁鵲等邪?若倉公者,可謂近之矣。


\end{pinyinscope}