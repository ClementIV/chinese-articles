\article{白起王翦列傳}

\begin{pinyinscope}
白起者,郿人也。善用兵,事秦昭王。昭王十三年,而白起為左庶長,將而擊韓之新城。是歲,穰侯相秦,舉任鄙以為漢中守。其明年,白起為左更,攻韓、魏於伊闕,斬首二十四萬,又虜其將公孫喜,拔五城。起遷為國尉。涉河取韓安邑以東,到乾河。明年,白起為大良造。攻魏,拔之,取城小大六十一。明年,起與客卿錯攻垣城,拔之。後五年,白起攻趙,拔光狼城。後七年,白起攻楚,拔鄢、鄧五城。其明年,攻楚,拔郢,燒夷陵,遂東至竟陵。楚王亡去郢,東走徙陳。秦以郢為南郡。白起遷為武安君。武安君因取楚,定巫、黔中郡。昭王三十四年,白起攻魏,拔華陽,走芒卯,而虜三晉將,斬首十三萬。與趙將賈偃戰,沈其卒二萬人於河中。昭王四十三年,白起攻韓陘城,拔五城,斬首五萬。四十四年,白起攻南陽太行道,絕之。

四十五年,伐韓之野王。野王降秦,上黨道絕。其守馮亭與民謀曰:「鄭道已絕,韓必不可得為民。秦兵日進,韓不能應,不如以上黨歸趙。趙若受我,秦怒,必攻趙。趙被兵,必親韓。韓趙為一,則可以當秦。」因使人報趙。趙孝成王與平陽君、平原君計之。平陽君曰:「不如勿受。受之,禍大於所得。」平原君曰:「無故得一郡,受之便。」趙受之,因封馮亭為華陽君。

四十六年,秦攻韓緱氏、藺,拔之。

四十七年,秦使左庶長王龁攻韓,取上黨。上黨民走趙。趙軍長平,以按據上黨民。四月,龁因攻趙。趙使廉頗將。趙軍士卒犯秦斥兵,秦斥兵斬趙裨將茄。六月,陷趙軍,取二鄣四尉。七月,趙軍筑壘壁而守之。秦又攻其壘,取二尉,敗其陣,奪西壘壁。廉頗堅壁以待秦,秦數挑戰,趙兵不出。趙王數以為讓。而秦相應侯又使人行千金於趙為反閒,曰:「秦之所惡,獨畏馬服子趙括將耳,廉頗易與,且降矣。」趙王既怒廉頗軍多失亡,軍數敗,又反堅壁不敢戰,而又聞秦反閒之言,因使趙括代廉頗將以擊秦。秦聞馬服子將,乃陰使武安君白起為上將軍。而王龁為尉裨將,令軍中有敢泄武安君將者斬。趙括至,則出兵擊秦軍。秦軍詳敗而走,張二奇兵以劫之。趙軍逐勝,追造秦壁。壁堅拒不得入,而秦奇兵二萬五千人絕趙軍後,又一軍五千騎絕趙壁閒,趙軍分而為二,糧道絕。而秦出輕兵擊之。趙戰不利,因筑壁堅守,以待救至。秦王聞趙食道絕,王自之河內,賜民爵各一級,發年十五以上悉詣長平,遮絕趙救及糧食。

至九月,趙卒不得食四十六日,皆內陰相殺食。來攻秦壘,欲出。為四隊,四五復之,不能出。其將軍趙括出銳卒自搏戰,秦軍射殺趙括。括軍敗,卒四十萬人降武安君。武安君計曰:「前秦已拔上黨,上黨民不樂為秦而歸趙。趙卒反覆。非盡殺之,恐為亂。」乃挾詐而盡阬殺之,遺其小者二百四十人歸趙。前後斬首虜四十五萬人。趙人大震。

四十八年十月,秦復定上黨郡。秦分軍為二:王龁攻皮牢,拔之;司馬梗定太原。韓、趙恐,使蘇代厚幣說秦相應侯曰:「武安君禽馬服子乎?」曰:「然。」又曰:「即圍邯鄲乎?」曰:「然。」「趙亡則秦王王矣,武安君為三公。武安君所為秦戰勝攻取者七十餘城,南定鄢、郢、漢中,北禽趙括之軍,雖周、召、呂望之功不益於此矣。今趙亡,秦王王,則武安君必為三公,君能為之下乎?雖無欲為之下,固不得已矣。秦嘗攻韓,圍邢丘,困上黨,上黨之民皆反為趙,天下不樂為秦民之日久矣。今亡趙,北地入燕,東地入齊,南地入韓、魏,則君之所得民亡幾何人。故不如因而割之,無以為武安君功也。」於是應侯言於秦王曰:「秦兵勞,請許韓、趙之割地以和,且休士卒。」王聽之,割韓垣雍、趙六城以和。正月,皆罷兵。武安君聞之,由是與應侯有隙。

其九月,秦復發兵,使五大夫王陵攻趙邯鄲。是時武安君病,不任行。四十九年正月,陵攻邯鄲,少利,秦益發兵佐陵。陵兵亡五校。武安君病愈,秦王欲使武安君代陵將。武安君言曰:「邯鄲實未易攻也。且諸侯救日至,彼諸侯怨秦之日久矣。今秦雖破長平軍,而秦卒死者過半,國內空。遠絕河山而爭人國都,趙應其內,諸侯攻其外,破秦軍必矣。不可。」秦王自命,不行;乃使應侯請之,武安君終辭不肯行,遂稱病。

秦王使王龁代陵將,八九月圍邯鄲,不能拔。楚使春申君及魏公子將兵數十萬攻秦軍,秦軍多失亡。武安君言曰:「秦不聽臣計,今如何矣!」秦王聞之,怒,彊起武安君,武安君遂稱病甐。應侯請之,不起。於是免武安君為士伍,遷之陰密。武安君病,未能行。居三月,諸侯攻秦軍急,秦軍數卻,使者日至。秦王乃使人遣白起,不得留咸陽中。武安君既行,出咸陽西門十里,至杜郵。秦昭王與應侯群臣議曰:「白起之遷,其意尚怏怏不服,有餘言。」秦王乃使使者賜之劍,自裁。武安君引劍將自剄,曰:「我何罪于天而至此哉?」良久,曰:「我固當死。長平之戰,趙卒降者數十萬人,我詐而盡阬之,是足以死。」遂自殺。武安君之死也,以秦昭王五十年十一月。死而非其罪,秦人憐之,鄉邑皆祭祀焉。

王翦者,頻陽東鄉人也。少而好兵,事秦始皇。始皇十一年,翦將攻趙閼與,破之,拔九城,十八年,翦將攻趙。歲餘,遂拔趙,趙王降,盡定趙地為郡。明年,燕使荊軻為賊於秦,秦王使王翦攻燕。燕王喜走遼東,翦遂定燕薊而還。秦使翦子王賁擊荊,荊兵敗。還擊魏,魏王降,遂定魏地。

秦始皇既滅三晉,走燕王,而數破荊師。秦將李信者,年少壯勇,嘗以兵數千逐燕太子丹至於衍水中,卒破得丹,始皇以為賢勇。於是始皇問李信:「吾欲攻取荊,於將軍度用幾何人而足?」李信曰:「不過用二十萬人。」始皇問王翦,王翦曰:「非六十萬人不可。」始皇曰:「王將軍老矣,何怯也!李將軍果勢壯勇,其言是也。」遂使李信及蒙恬將二十萬南伐荊。王翦言不用,因謝病,歸老於頻陽。李信攻平與,蒙恬攻寢,大破荊軍。信又攻鄢郢,破之,於是引兵而西,與蒙恬會城父。荊人因隨之,三日三夜不頓舍,大破李信軍,入兩壁,殺七都尉,秦軍走。

始皇聞之,大怒,自馳如頻陽,見謝王翦曰:「寡人以不用將軍計,李信果辱秦軍。今聞荊兵日進而西,將軍雖病,獨忍棄寡人乎!」王翦謝曰:「老臣罷病悖亂,唯大王更擇賢將。」始皇謝曰:「已矣,將軍勿復言!」王翦曰:「大王必不得已用臣,非六十萬人不可。」始皇曰:「為聽將軍計耳。」於是王翦將兵六十萬人,始皇自送至灞上。王翦行,請美田宅園池甚眾。始皇曰:「將軍行矣,何憂貧乎?」王翦曰:「為大王將,有功終不得封侯,故及大王之向臣,臣亦及時以請園池為子孫業耳。」始皇大笑。王翦既至關,使使還請善田者五輩。或曰:「將軍之乞貸,亦已甚矣。」王翦曰:「不然。夫秦王怚而不信人。今空秦國甲士而專委於我,我不多請田宅為子孫業以自堅,顧令秦王坐而疑我邪?」

王翦果代李信擊荊。荊聞王翦益軍而來,乃悉國中兵以拒秦。王翦至,堅壁而守之,不肯戰。荊兵數出挑戰,終不出。王翦日休士洗沐,而善飲食撫循之,親與士卒同食。久之,王翦使人問軍中戲乎?對曰:「方投石超距。」於是王翦曰:「士卒可用矣。」荊數挑戰而秦不出,乃引而東。翦因舉兵追之,令壯士擊,大破荊軍。至蘄南,殺其將軍項燕,荊兵遂敗走。秦因乘勝略定荊地城邑。歲餘,虜荊王負芻,竟平荊地為郡縣。因南征百越之君。而王翦子王賁,與李信破定燕、齊地。

秦始皇二十六年,盡并天下,王氏、蒙氏功為多,名施於後世。

秦二世之時,王翦及其子賁皆已死,而又滅蒙氏。陳勝之反秦,秦使王翦之孫王離擊趙,圍趙王及張耳鉅鹿城。或曰:「王離,秦之名將也。今將彊秦之兵,攻新造之趙,舉之必矣。」客曰:「不然。夫為將三世者必敗。必敗者何也?必其所殺伐多矣,其後受其不祥。今王離已三世將矣。」居無何,項羽救趙,擊秦軍,果虜王離,王離軍遂降諸侯。

太史公曰:鄙語云「尺有所短,寸有所長」。白起料敵合變,出奇無窮,聲震天下,然不能救患於應侯。王翦為秦將,夷六國,當是時,翦為宿將,始皇師之,然不能輔秦建德,固其根本,偷合取容,以至圽身。及孫王離為項羽所虜,不亦宜乎!彼各有所短也。


\end{pinyinscope}