\article{歷書}

\begin{pinyinscope}
昔自在古,歷建正作於孟春。於時冰泮發蟄,百草奮興,秭鳺先滜。物乃歲具,生於東,次順四時,卒于冬分。時雞三號,卒明。撫十二[月]節,卒于丑。日月成,故明也。明者孟也,幽者幼也,幽明者雌雄也。雌雄代興,而順至正之統也。日歸于西,起明於東;月歸於東,起明于西。正不率天,又不由人,則凡事易壞而難成矣。

王者易姓受命,必慎始初,改正朔,易服色,推本天元,順承厥意。

太史公曰:神農以前尚矣。蓋黃帝考定星歷,建立五行,起消息,正閏餘,於是有天地神祇物類之官,是謂五官。各司其序,不相亂也。民是以能有信,神是以能有明德。民神異業,敬而不瀆,故神降之嘉生,民以物享,災禍不生,所求不匱。

少暤氏之衰也,九黎亂德,民神雜擾,不可放物,禍菑薦至,莫盡其氣。顓頊受之,乃命南正重司天以屬神,命火正黎司地以屬民,使復舊常,無相侵瀆。

其後三苗服九黎之德,故二官咸廢所職,而閏餘乖次,孟陬殄滅,攝提無紀,歷數失序。堯復遂重黎之後,不忘舊者,使復典之,而立羲和之官。明時正度,則陰陽調,風雨節,茂氣至,民無夭疫。年耆禪舜,申戒文祖,云「天之歷數在爾躬」。舜亦以命禹。由是觀之,王者所重也。

夏正以正月,殷正以十二月,周正以十一月。蓋三王之正若循環,窮則反本。天下有道,則不失紀序;無道,則正朔不行於諸侯。

幽、厲之後,周室微,陪臣執政,史不記時,君不告朔,故疇人子弟分散,或在諸夏,或在夷狄,是以其禨祥廢而不統。周襄王二十六年閏三月,而春秋非之。先王之正時也,履端於始,舉正於中,歸邪於終。履端於始,序則不愆;舉正於中,民則不惑;歸邪於終,事則不悖。

其後戰國并爭,在於彊國禽敵,救急解紛而已,豈遑念斯哉!是時獨有鄒衍,明於五德之傳,而散消息之分,以顯諸侯。而亦因秦滅六國,兵戎極煩,又升至尊之日淺,未暇遑也。而亦頗推五勝,而自以為獲水德之瑞,更名河曰「德水」,而正以十月,色上黑。然歷度閏餘,未能睹其真也。

漢興,高祖曰「北畤待我而起」,亦自以為獲水德之瑞。雖明習歷及張蒼等,咸以為然。是時天下初定,方綱紀大基,高后女主,皆未遑,故襲秦正朔服色。

至孝文時,魯人公孫臣以終始五德上書,言「漢得土德,宜更元,改正朔,易服色。當有瑞,瑞黃龍見」。事下丞相張蒼,張蒼亦學律歷,以為非是,罷之。其後黃龍見成紀,張蒼自黜,所欲論著不成。而新垣平以望氣見,頗言正歷服色事,貴幸,後作亂,故孝文帝廢不復問。

至今上即位,招致方士唐都,分其天部;而巴落下閎運算轉歷,然後日辰之度與夏正同。乃改元,更官號,封泰山。因詔御史曰:「乃者,有司言星度之未定也,廣延宣問,以理星度,未能詹也。蓋聞昔者黃帝合而不死,名察度驗,定清濁,起五部,建氣物分數。然蓋尚矣。書缺樂弛,朕甚閔焉。朕唯未能循明也,紬績日分,率應水德之勝。今日順夏至,黃鐘為宮,林鐘為徵,太蔟為商,南呂為羽,姑洗為角。自是以後,氣復正,羽聲復清,名復正變,以至子日當冬至,則陰陽離合之道行焉。十一月甲子朔旦冬至已詹,其更以七年為太初元年。年名『焉逢攝提格』,月名『畢聚』,日得甲子,夜半朔旦冬至。」

歷術甲子篇太初元年,歲名「焉逢攝提格」,月名「畢聚」,日得甲子,夜半朔旦冬至。

正北十二無大餘,無小餘;無大餘,無小餘;焉逢攝提格太初元年。

十二大餘五十四,小餘三百四十八;大餘五,小餘八;端蒙單閼二年。

閏十三大餘四十八,小餘六百九十六;大餘十,小餘十六;游兆執徐三年。

十二大餘十二,小餘六百三;大餘十五,小餘二十四;彊梧大荒落四年。

十二大餘七,小餘十一;大餘二十一,無小餘;徒維敦牂天漢元年。

閏十三大餘一,小餘三百五十九;大餘二十六,小餘八;祝犁協洽二年。

十二大餘二十五,小餘二百六十六;大餘三十一,小餘十六;商橫涒灘三年。

十二大餘十九,小餘六百一十四;大餘三十六,小餘二十四;昭陽作鄂四年。

閏十三大餘十四,小餘二十二;大餘四十二,無小餘;橫艾淹茂太始元年。

十二大餘三十七,小餘八百六十九;大餘四十七,小餘八;尚章大淵獻二年。

閏十三大餘三十二,小餘二百七十七;大餘五十二,小餘一十六;焉逢困敦三年。

十二大餘五十六,小餘一百八十四;大餘五十七,小餘二十四;端蒙赤奮若四年。

十二大餘五十,小餘五百三十二;大餘三,無小餘;游兆攝提格征和元年。

閏十三大餘四十四,小餘八百八十;大餘八,小餘八;彊梧單閼二年。

十二大餘八,小餘七百八十七;大餘十三,小餘十六;徒維執徐三年。

十二大餘三,小餘一百九十五;大餘十八,小餘二十四;祝犁大芒落四年。

閏十三大餘五十七,小餘五百四十三;大餘二十四,無小餘;商橫敦牂後元元年。

十二大餘二十一,小餘四百五十;大餘二十九,小餘八;昭陽汁洽二年。

閏十三大餘十五,小餘七百九十八;大餘三十四,小餘十六;橫艾涒灘始元元年。

正西十二大餘三十九,小餘七百五;大餘三十九,小餘二十四;尚章作噩二年。

十二大餘三十四,小餘一百一十三;大餘四十五,無小餘;焉逢淹茂三年。

閏十三大餘二十八,小餘四百六十一;大餘五十,小餘八;端蒙大淵獻四年。

十二大餘五十二,小餘三百六十八;大餘五十五,小餘十六;游兆困敦五年。

十二大餘四十六,小餘七百一十六;無大餘,小餘二十四;彊梧赤奮若六年。

閏十三大餘四十一,小餘一百二十四;大餘六,無小餘;徒維攝提格元鳳元年。

十二大餘五,小餘三十一;大餘十一,小餘八;祝犁單閼二年。

十二大餘五十九,小餘三百七十九;大餘十六,小餘十六;商橫執徐三年。

閏十三大餘五十三,小餘七百二十七;大餘二十一,小餘二十四;昭陽大荒落四年。

十二大餘十七,小餘六百三十四;大餘二十七,無小餘;橫艾敦牂五年。

閏十三大餘十二,小餘四十二;大餘三十二,小餘八;尚章汁洽六年。

十二大餘三十五,小餘八百八十九;大餘三十七,小餘十六;焉逢涒灘元平元年十二大餘三十,小餘二百九十七;大餘四十二,小餘二十四;端蒙作噩本始元年。

閏十三大餘二十四,小餘六百四十五;大餘四十八,無小餘;游兆閹茂二年。

十二大餘四十八,小餘五百五十二;大餘五十三,小餘八;彊梧大淵獻三年。

十二大餘四十二,小餘九百;大餘五十八,小餘十六;徒維困敦四年。

閏十三大餘三十七,小餘三百八;大餘三,小餘二十四;祝犁赤奮若地節元年。

十二大餘一,小餘二百一十五;大餘九,無小餘;商橫攝提格二年。

閏十三大餘五十五,小餘五百六十三;大餘十四,小餘八;昭陽單閼三年。

正南十二大餘十九,小餘四百七十;大餘十九,小餘十六;橫艾執徐四年。

十二大餘十三,小餘八百一十八;大餘二十四,小餘二十四;尚章大荒落元康元年。

閏十三大餘八,小餘二百二十六;大餘三十,無小餘;焉逢敦牂二年。

十二大餘三十二,小餘一百三十三;大餘三十五,小餘八;端蒙協洽三年。

十二大餘二十六,小餘四百八十一;大餘四十,小餘十六;游兆涒灘四年。

閏十三大餘二十,小餘八百二十九;大餘四十五,小餘二十四;彊梧作噩神雀元年。

十二大餘四十四,小餘七百三十六;大餘五十一,無小餘;徒維淹茂二年。

十二大餘三十九,小餘一百四十四;大餘五十六,小餘八;祝犁大淵獻三年。

閏十三大餘三十三,小餘四百九十二;大餘一,小餘十六;商橫困敦四年。

十二大餘五十七,小餘三百九十九;大餘六,小餘二十四;昭陽赤奮若五鳳元年。

閏十三大餘五十一,小餘七百四十七;大餘十二,無小餘;橫艾攝提格二年。

十二大餘十五,小餘六百五十四;大餘十七,小餘八;尚章單閼三年。

十二大餘十,小餘六十二;大餘二十二,小餘十六;焉逢執徐四年。

閏十三大餘四,小餘四百一十;大餘二十七,小餘二十四;端蒙大荒落甘露元年。

十二大餘二十八,小餘三百一十七;大餘三十三,無小餘;游兆敦執二年。

十二大餘二十二,小餘六百六十五;大餘三十八,小餘八;彊梧協洽三年。

閏十三大餘十七,小餘七十三;大餘四十三,小餘十六;徒維涒灘四年。

十二大餘四十,小餘九百二十;大餘四十八,小餘二十四;祝犁作噩黃龍元年。

閏十三大餘三十五,小餘三百二十八;大餘五十四,無小餘;商橫淹茂初元元年。正東十二大餘五十九,小餘二百三十五;大餘五十九,小餘八;昭陽大淵獻二年。

十二大餘五十三,小餘五百八十三;大餘四,小餘十六;橫艾困敦三年。

閏十三大餘四十七,小餘九百三十一;大餘九,小餘二十四;尚章赤奮若四年。

十二大餘十一,小餘八百三十八;大餘十五,無小餘;焉逢攝提格五年。

十二大餘六,小餘二百四十六;大餘二十,小餘八;端蒙單閼永光元年。

閏十三無大餘,小餘五百九十四;大餘二十五,小餘十六;游兆執徐二年。

十二大餘二十四,小餘五百一;大餘三十,小餘二十四;彊梧大荒落三年。

十二大餘十八,小餘八百四十九;大餘三十六,無小餘;徒維敦牂四年。

閏十三大餘十三,小餘二百五十七;大餘四十一,小餘八;祝犁協洽五年。

十二大餘三十七,小餘一百六十四;大餘四十六,小餘十六;商橫涒灘建昭元年。

閏十三大餘三十一,小餘五百一十二;大餘五十一,小餘二十四;昭陽作噩二年。

十二大餘五十五,小餘四百一十九;大餘五十七,無小餘;橫艾閹茂三年。

十二大餘四十九,小餘七百六十七;大餘二,小餘八;尚章大淵獻四年。

閏十三大餘四十四,小餘一百七十五;大餘七,小餘十六;焉逢困敦五年。

十二大餘八,小餘八十二;大餘十二,小餘二十四;端蒙赤奮若竟寧元年。

十二大餘二,小餘四百三十;大餘十八,無小餘;游兆攝提格建始元年。

閏十三大餘五十六,小餘七百七十八;大餘二十三,小餘八;彊梧單閼二年。

十二大餘二十,小餘六百八十五;大餘二十八,小餘十六;徒維執徐三年。

閏十三大餘十五,小餘九十三;大餘三十三,小餘二十四;祝犁大荒落四年。

右歷書:大餘者,日也。小餘者,月也。端(旃)蒙者,年名也。支:丑名赤奮若,寅名攝提格。干:丙名游兆。正北,冬至加子時;正西,加酉時;正南,加午時;正東,加卯時。


\end{pinyinscope}