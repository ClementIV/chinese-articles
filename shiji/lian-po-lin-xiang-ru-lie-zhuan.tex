\article{廉頗藺相如列傳}

\begin{pinyinscope}
廉頗者,趙之良將也。趙惠文王十六年,廉頗為趙將伐齊,大破之,取陽晉,拜為上卿,以勇氣聞於諸侯。藺相如者,趙人也,為趙宦者令繆賢舍人。

趙惠文王時,得楚和氏璧。秦昭王聞之,使人遺趙王書,願以十五城請易璧。趙王與大將軍廉頗諸大臣謀:欲予秦,秦城恐不可得,徒見欺;欲勿予,即患秦兵之來。計未定,求人可使報秦者,未得。宦者令繆賢曰:「臣舍人藺相如可使。」王問:「何以知之?」對曰:「臣嘗有罪,竊計欲亡走燕,臣舍人相如止臣,曰:『君何以知燕王?』臣語曰:『臣嘗從大王與燕王會境上,燕王私握臣手,曰「願結友」。以此知之,故欲往。』相如謂臣曰:『夫趙彊而燕弱,而君幸於趙王,故燕王欲結於君。今君乃亡趙走燕,燕畏趙,其勢必不敢留君,而束君歸趙矣。君不如肉袒伏斧質請罪,則幸得脫矣。』臣從其計,大王亦幸赦臣。臣竊以為其人勇士,有智謀,宜可使。」於是王召見,問藺相如曰:「秦王以十五城請易寡人之璧,可予不?」相如曰:「秦彊而趙弱,不可不許。」王曰:「取吾璧,不予我城,柰何?」相如曰:「秦以城求璧而趙不許,曲在趙。趙予璧而秦不予趙城,曲在秦。均之二策,寧許以負秦曲。」王曰:「誰可使者?」相如曰:「王必無人,臣願奉璧往使。城入趙而璧留秦;城不入,臣請完璧歸趙。」趙王於是遂遣相如奉璧西入秦。

秦王坐章臺見相如,相如奉璧奏秦王。秦王大喜,傳以示美人及左右,左右皆呼萬歲。相如視秦王無意償趙城,乃前曰:「璧有瑕,請指示王。」王授璧,相如因持璧卻立,倚柱,怒髪上沖冠,謂秦王曰:「大王欲得璧,使人發書至趙王,趙王悉召群臣議,皆曰『秦貪,負其彊,以空言求璧,償城恐不可得』。議不欲予秦璧。臣以為布衣之交尚不相欺,況大國乎!且以一璧之故逆彊秦之驩,不可。於是趙王乃齋戒五日,使臣奉璧,拜送書於庭。何者?嚴大國之威以修敬也。今臣至,大王見臣列觀,禮節甚倨;得璧,傳之美人,以戲弄臣。臣觀大王無意償趙王城邑,故臣復取璧。大王必欲急臣,臣頭今與璧俱碎於柱矣!」相如持其璧睨柱,欲以擊柱。秦王恐其破璧,乃辭謝固請,召有司案圖,指從此以往十五都予趙。相如度秦王特以詐詳為予趙城,實不可得,乃謂秦王曰:「和氏璧,天下所共傳寶也,趙王恐,不敢不獻。趙王送璧時,齋戒五日,今大王亦宜齋戒五日,設九賓於廷,臣乃敢上璧。」秦王度之,終不可彊奪,遂許齋五日,舍相如廣成傳。相如度秦王雖齋,決負約不償城,乃使其從者衣褐,懷其璧,從徑道亡,歸璧于趙。

秦王齋五日後,乃設九賓禮於廷,引趙使者藺相如。相如至,謂秦王曰:「秦自繆公以來二十餘君,未嘗有堅明約束者也。臣誠恐見欺於王而負趙,故令人持璧歸,閒至趙矣。且秦彊而趙弱,大王遣一介之使至趙,趙立奉璧來。今以秦之彊而先割十五都予趙,趙豈敢留璧而得罪於大王乎?臣知欺大王之罪當誅,臣請就湯鑊,唯大王與群臣孰計議之。」秦王與群臣相視而嘻。左右或欲引相如去,秦王因曰:「今殺相如,終不能得璧也,而絕秦趙之驩,不如因而厚遇之,使歸趙,趙王豈以一璧之故欺秦邪!」卒廷見相如,畢禮而歸之。

相如既歸,趙王以為賢大夫使不辱於諸侯,拜相如為上大夫。秦亦不以城予趙,趙亦終不予秦璧。

其後秦伐趙,拔石城。明年,復攻趙,殺二萬人。

秦王使使者告趙王,欲與王為好會於西河外澠池。趙王畏秦,欲毋行。廉頗、藺相如計曰:「王不行,示趙弱且怯也。」趙王遂行,相如從。廉頗送至境,與王訣曰:「王行,度道里會遇之禮畢,還,不過三十日。三十日不還,則請立太子為王。以絕秦望。」王許之,遂與秦王會澠池。秦王飲酒酣,曰:「寡人竊聞趙王好音,請奏瑟。」趙王鼓瑟。秦御史前書曰「某年月日,秦王與趙王會飲,令趙王鼓瑟」。藺相如前曰:「趙王竊聞秦王善為秦聲,請奏盆缻秦王,以相娛樂。」秦王怒,不許。於是相如前進缻,因跪請秦王。秦王不肯擊缻。相如曰:「五步之內,相如請得以頸血濺大王矣!」左右欲刃相如,相如張目叱之,左右皆靡。於是秦王不懌,為一擊缻。相如顧召趙御史書曰「某年月日,秦王為趙王擊缻」。秦之群臣曰:「請以趙十五城為秦王壽」。藺相如亦曰:「請以秦之咸陽為趙王壽。」秦王竟酒,終不能加勝於趙。趙亦盛設兵以待秦,秦不敢動。

既罷歸國,以相如功大,拜為上卿,位在廉頗之右。廉頗曰:「我為趙將,有攻城野戰之大功,而藺相如徒以口舌為勞,而位居我上,且相如素賤人,吾羞,不忍為之下。」宣言曰:「我見相如,必辱之。」相如聞,不肯與會。相如每朝時,常稱病,不欲與廉頗爭列。已而相如出,望見廉頗,相如引車避匿。於是舍人相與諫曰:「臣所以去親戚而事君者,徒慕君之高義也。今君與廉頗同列,廉君宣惡言而君畏匿之,恐懼殊甚,且庸人尚羞之,況於將相乎!臣等不肖,請辭去。」藺相如固止之,曰:「公之視廉將軍孰與秦王?」曰:「不若也。」相如曰:「夫以秦王之威,而相如廷叱之,辱其群臣,相如雖駑,獨畏廉將軍哉?顧吾念之,彊秦之所以不敢加兵於趙者,徒以吾兩人在也。今兩虎共鬬,其勢不俱生。吾所以為此者,以先國家之急而後私讎也。」廉頗聞之,肉袒負荊,因賓客至藺相如門謝罪。曰:「鄙賤之人,不知將軍寬之至此也。」卒相與驩,為刎頸之交。

是歲,廉頗東攻齊,破其一軍。居二年,廉頗復伐齊幾,拔之。後三年,廉頗攻魏之防陵、安陽,拔之。後四年,藺相如將而攻齊,至平邑而罷。其明年,趙奢破秦軍閼與下。

趙奢者,趙之田部吏也。收租稅而平原君家不肯出趙,奢以法治之,殺平原君用事者九人。平原君怒,將殺奢。奢因說曰:「君於趙為貴公子,今縱君家而不奉公則法削,法削則國弱,國弱則諸侯加兵,諸侯加兵是無趙也,君安得有此富乎?以君之貴,奉公如法則上下平,上下平則國彊,國彊則趙固,而君為貴戚,豈輕於天下邪?」平原君以為賢,言之於王。王用之治國賦,國賦大平,民富而府庫實。

秦伐韓,軍於閼與。王召廉頗而問曰:「可救不?」對曰:「道遠險狹,難救。」又召樂乘而問焉,樂乘對如廉頗言。又召問趙奢,奢對曰:「其道遠險狹,譬之猶兩鼠鬬於穴中,將勇者勝。」王乃令趙奢將,救之。

兵去邯鄲三十里,而令軍中曰:「有以軍事諫者死。」秦軍軍武安西,秦軍鼓譟勒兵,武安屋瓦盡振。軍中候有一人言急救武安,趙奢立斬之。堅壁,留二十八日不行,復益增壘。秦閒來入,趙奢善食而遣之。閒以報秦將,秦將大喜曰:「夫去國三十里而軍不行,乃增壘,閼與非趙地也。」趙奢既已遣秦閒,卷甲而趨之,二日一夜至,今善射者去閼與五十里而軍。軍壘成,秦人聞之,悉甲而至。軍士許歷請以軍事諫,趙奢曰:「內之。」許歷曰:「秦人不意趙師至此,其來氣盛,將軍必厚集其陣以待之。不然,必敗。」趙奢曰:「請受令。」許歷曰:「請就鈇質之誅。」趙奢曰:「胥後令邯鄲。」許歷復請諫,曰:「先據北山上者勝,後至者敗。」趙奢許諾,即發萬人趨之。秦兵後至,爭山不得上,趙奢縱兵擊之,大破秦軍。秦軍解而走,遂解閼與之圍而歸。

趙惠文王賜奢號為馬服君,以許歷為國尉。趙奢於是與廉頗、藺相如同位。

後四年,趙惠文王卒,子孝成王立。七年,秦與趙兵相距長平,時趙奢已死,而藺相如病甐,趙使廉頗將攻秦,秦數敗趙軍,趙軍固壁不戰。秦數挑戰,廉頗不肯。趙王信秦之閒。秦之閒言曰:「秦之所惡,獨畏馬服君趙奢之子趙括為將耳。」趙王因以括為將,代廉頗。藺相如曰:「王以名使括,若膠柱而鼓瑟耳。括徒能讀其父書傳,不知合變也。」趙王不聽,遂將之。

趙括自少時學兵法,言兵事,以天下莫能當。嘗與其父奢言兵事,奢不能難,然不謂善。括母問奢其故,奢曰:「兵,死地也,而括易言之。使趙不將括即已,若必將之,破趙軍者必括也。」及括將行,其母上書言於王曰:「括不可使將。」王曰:「何以?」對曰:「始妾事其父,時為將,身所奉飯飲而進食者以十數,所友者以百數,大王及宗室所賞賜者盡以予軍吏士大夫,受命之日,不問家事。今括一旦為將,東向而朝,軍吏無敢仰視之者,王所賜金帛,歸藏於家,而日視便利田宅可買者買之。王以為何如其父?父子異心,願王勿遣。」王曰:「母置之,吾已決矣。」括母因曰:「王終遣之,即有如不稱,妾得無隨坐乎?」王許諾。

趙括既代廉頗,悉更約束,易置軍吏。秦將白起聞之,縱奇兵,詳敗走,而絕其糧道,分斷其軍為二,士卒離心。四十餘日,軍餓,趙括出銳卒自博戰,秦軍射殺趙括。括軍敗,數十萬之眾遂降秦,秦悉阬之。趙前後所亡凡四十五萬。明年,秦兵遂圍邯鄲,歲餘,幾不得脫。賴楚、魏諸侯來救,乃得解邯鄲之圍。趙王亦以括母先言,竟不誅也。

自邯鄲圍解五年,而燕用栗腹之謀,曰「趙壯者盡於長平,其孤未壯」,舉兵擊趙。趙使廉頗將,擊,大破燕軍於鄗,殺栗腹,遂圍燕。燕割五城請和,乃聽之。趙以尉文封廉頗為信平君,為假相國。

廉頗之免長平歸也,失勢之時,故客盡去。及復用為將,客又復至。廉頗曰:「客退矣!」客曰:「吁!君何見之晚也?夫天下以市道交,君有勢,我則從君,君無勢則去,此固其理也,有何怨乎?」居六年,趙使廉頗伐魏之繁陽,拔之。

趙孝成王卒,子悼襄王立,使樂乘代廉頗。廉頗怒,攻樂乘,樂乘走。廉頗遂奔魏之大梁。其明年,趙乃以李牧為將而攻燕,拔武遂、方城。

廉頗居梁久之,魏不能信用。趙以數困於秦兵,趙王思復得廉頗,廉頗亦思復用於趙。趙王使使者視廉頗尚可用否。廉頗之仇郭開多與使者金,令毀之。趙使者既見廉頗,廉頗為之一飯斗米,肉十斤,被甲上馬,以示尚可用。趙使還報王曰:「廉將軍雖老,尚善飯,然與臣坐,頃之三遺矢矣。」趙王以為老,遂不召。

楚聞廉頗在魏,陰使人迎之。廉頗一為楚將,無功,曰:「我思用趙人。」廉頗卒死于壽春。

李牧者,趙之北邊良將也。常居代鴈門,備匈奴。以便宜置吏,市租皆輸入莫府,為士卒費。日擊數牛饗士,習射騎,謹烽火,多閒諜,厚遇戰士。為約曰:「匈奴即入盜,急入收保,有敢捕虜者斬。」匈奴每入,烽火謹,輒入收保,不敢戰。如是數歲,亦不亡失。然匈奴以李牧為怯,雖趙邊兵亦以為吾將怯。趙王讓李牧,李牧如故。趙王怒,召之,使他人代將。

歲餘,匈奴每來,出戰。出戰,數不利,失亡多,邊不得田畜。復請李牧。牧杜門不出,固稱疾。趙王乃復彊起使將兵。牧曰:「王必用臣,臣如前,乃敢奉令。」王許之。

李牧至,如故約。匈奴數歲無所得。終以為怯。邊士日得賞賜而不用,皆願一戰。於是乃具選車得千三百乘,選騎得萬三千匹,百金之士五萬人,彀者十萬人,悉勒習戰。大縱畜牧,人民滿野。匈奴小入,詳北不勝,以數千人委之。單于聞之,大率眾來入。李牧多為奇陳,張左右翼擊之,大破殺匈奴十餘萬騎。滅襜襤,破東胡,降林胡,單于奔走。其後十餘歲,匈奴不敢近趙邊城。

趙悼襄王元年,廉頗既亡入魏,趙使李牧攻燕,拔武遂、方城。居二年,龐煖破燕軍,殺劇辛。後七年,秦破殺趙將扈輒於武遂,斬首十萬。趙乃以李牧為大將軍,擊秦軍於宜安,大破秦軍,走秦將桓齮。封李牧為武安君。居三年,秦攻番吾,李牧擊破秦軍,南距韓、魏。

趙王遷七年,秦使王翦攻趙,趙使李牧、司馬尚御之。秦多與趙王寵臣郭開金,為反閒,言李牧、司馬尚欲反。趙王乃使趙蔥及齊將顏聚代李牧。李牧不受命,趙使人微捕得李牧,斬之。廢司馬尚。後三月,王翦因急擊趙,大破殺趙蔥,虜趙王遷及其將顏聚,遂滅趙。

太史公曰:知死必勇,非死者難也,處死者難。方藺相如引璧睨柱,及叱秦王左右,勢不過誅,然士或怯懦而不敢發。相如一奮其氣,威信敵國,退而讓頗,名重太山,其處智勇,可謂兼之矣!


\end{pinyinscope}