\article{刺客列傳}

\begin{pinyinscope}
曹沫者,魯人也,以勇力事魯莊公。莊公好力。曹沫為魯將,與齊戰,三敗北。魯莊公懼,乃獻遂邑之地以和。猶復以為將。

齊桓公許與魯會于柯而盟。桓公與莊公既盟於壇上,曹沫執匕首劫齊桓公,桓公左右莫敢動,而問曰:「子將何欲?」曹沫曰:「齊彊魯弱,而大國侵魯亦甚矣。今魯城壞即壓齊境,君其圖之。」桓公乃許盡歸魯之侵地。既已言,曹沫投其匕首,下壇,北面就群臣之位,顏色不變,辭令如故。桓公怒,欲倍其約。管仲曰:「不可。夫貪小利以自快,棄信於諸侯,失天下之援,不如與之。」於是桓公乃遂割魯侵地,曹沫三戰所亡地盡復予魯。

其後百六十有七年而吳有專諸之事。

專諸者,吳堂邑人也。伍子胥之亡楚而如吳也,知專諸之能。伍子胥既見吳王僚,說以伐楚之利。吳公子光曰:「彼伍員父兄皆死於楚而員言伐楚,欲自為報私讎也,非能為吳。」吳王乃止。伍子胥知公子光之欲殺吳王僚,乃曰:「彼光將有內志,未可說以外事。」乃進專諸於公子光。

光之父曰吳王諸樊。諸樊弟三人:次曰餘祭,次曰夷眛,次曰季子札。諸樊知季子札賢而不立太子,以次傳三弟,欲卒致國于季子札。諸樊既死,傳餘祭。餘祭死,傳夷眛。夷眛死,當傳季子札;季子札逃不肯立,吳人乃立夷眛之子僚為王。公子光曰:「使以兄弟次邪,季子當立;必以子乎,則光真適嗣,當立。」故嘗陰養謀臣以求立。

光既得專諸,善客待之。九年而楚平王死。春,吳王僚欲因楚喪,使其二弟公子蓋餘、屬庸將兵圍楚之灊;使延陵季子於晉,以觀諸侯之變。楚發兵絕吳將蓋餘、屬庸路,吳兵不得還。於是公子光謂專諸曰:「此時不可失,不求何獲!且光真王嗣,當立,季子雖來,不吾廢也。」專諸曰:「王僚可殺也。母老子弱,而兩弟將兵伐楚,楚絕其後。方今吳外困於楚,而內空無骨鯁之臣,是無如我何。」公子光頓首曰:「光之身,子之身也。」

四月丙子,光伏甲士於窟室中,而具酒請王僚。王僚使兵陳自宮至光之家,門戶階陛左右,皆王僚之親戚也。夾立侍,皆持長鈹。酒既酣,公子光詳為足疾,入窟室中,使專諸置匕首魚炙之腹中而進之。既至王前,專諸擘魚,因以匕首刺王僚,王僚立死。左右亦殺專諸,王人擾亂。公子光出其伏甲以攻王僚之徒,盡滅之,遂自立為王,是為闔閭。闔閭乃封專諸之子以為上卿。

其後七十餘年而晉有豫讓之事。

豫讓者,晉人也,故嘗事范氏及中行氏,而無所知名。去而事智伯,智伯甚尊寵之。及智伯伐趙襄子,趙襄子與韓、魏合謀滅智伯,滅智伯之後而三分其地。趙襄子最怨智伯,漆其頭以為飲器。豫讓遁逃山中,曰:「嗟乎!士為知己者死,女為說己者容。今智伯知我,我必為報讎而死,以報智伯,則吾魂魄不愧矣。」乃變名姓為刑人,入宮涂廁,中挾匕首,欲以刺襄子。襄子如廁,心動,執問涂廁之刑人,則豫讓,內持刀兵,曰:「欲為智伯報仇!」左右欲誅之。襄子曰:「彼義人也,吾謹避之耳。且智伯亡無後,而其臣欲為報仇,此天下之賢人也。」卒醳去之。

居頃之,豫讓又漆身為厲,吞炭為啞,使形狀不可知,行乞於市。其妻不識也。行見其友,其友識之,曰:「汝非豫讓邪?」曰:「我是也。」其友為泣曰:「以子之才,委質而臣事襄子,襄子必近幸子。近幸子,乃為所欲,顧不易邪?何乃殘身苦形,欲以求報襄子,不亦難乎!」豫讓曰:「既已委質臣事人,而求殺之,是懷二心以事其君也。且吾所為者極難耳!然所以為此者,將以愧天下後世之為人臣懷二心以事其君者也。」

既去,頃之,襄子當出,豫讓伏於所當過之橋下。襄子至橋,馬驚,襄子曰:「此必是豫讓也。」使人問之,果豫讓也。於是襄子乃數豫讓曰:「子不嘗事范、中行氏乎?智伯盡滅之,而子不為報讎,而反委質臣於智伯。智伯亦已死矣,而子獨何以為之報讎之深也?」豫讓曰:「臣事范、中行氏,范、中行氏皆眾人遇我,我故眾人報之。至於智伯,國士遇我,我故國士報之。」襄子喟然嘆息而泣曰:「嗟乎豫子!子之為智伯,名既成矣,而寡人赦子,亦已足矣。子其自為計,寡人不復釋子!」使兵圍之。豫讓曰:「臣聞明主不掩人之美,而忠臣有死名之義。前君已寬赦臣,天下莫不稱君之賢。今日之事,臣固伏誅,然願請君之衣而擊之,焉以致報讎之意,則雖死不恨。非所敢望也,敢布腹心!」於是襄子大義之,乃使使持衣與豫讓。豫讓拔劍三躍而擊之,曰:「吾可以下報智伯矣!」遂伏劍自殺。死之日,趙國志士聞之,皆為涕泣。

其後四十餘年而軹有聶政之事。

聶政者,軹深井里人也。殺人避仇,與母、姊如齊,以屠為事。

久之,濮陽嚴仲子事韓哀侯,與韓相俠累有卻。嚴仲子恐誅,亡去,游求人可以報俠累者。至齊,齊人或言聶政勇敢士也,避仇隱於屠者之閒。嚴仲子至門請,數反,然後具酒自暢聶政母前。酒酣,嚴仲子奉黃金百溢,前為聶政母壽。聶政驚怪其厚,固謝嚴仲子。嚴仲子固進,而聶政謝曰:「臣幸有老母,家貧,客游以為狗屠,可以旦夕得甘毳以養親。親供養備,不敢當仲子之賜。」嚴仲子辟人,因為聶政言曰:「臣有仇,而行游諸侯眾矣;然至齊,竊聞足下義甚高,故進百金者,將用為大人麤糲之費,得以交足下之驩,豈敢以有求望邪!」聶政曰:「臣所以降志辱身居市井屠者,徒幸以養老母;老母在,政身未敢以許人也。」嚴仲子固讓,聶政竟不肯受也。然嚴仲子卒備賓主之禮而去。

久之,聶政母死。既已葬,除服,聶政曰:「嗟乎!政乃市井之人,鼓刀以屠;而嚴仲子乃諸侯之卿相也,不遠千里,枉車騎而交臣。臣之所以待之,至淺鮮矣,未有大功可以稱者,而嚴仲子奉百金為親壽,我雖不受,然是者徒深知政也。夫賢者以感忿睚眦之意,而親信窮僻之人,而政獨安得嘿然而已乎!且前日要政,政徒以老母;老母今以天年終,政將為知己者用。」乃遂西至濮陽,見嚴仲子曰:「前日所以不許仲子者,徒以親在;今不幸而母以天年終。仲子所欲報仇者為誰?請得從事焉!」嚴仲子具告曰:「臣之仇韓相俠累,俠累又韓君之季父也,宗族盛多,居處兵衛甚設,臣欲使人刺之,(眾)終莫能就。今足下幸而不棄,請益其車騎壯士可為足下輔翼者。」聶政曰:「韓之與衛,相去中閒不甚遠,今殺人之相,相又國君之親,此其勢不可以多人,多人不能無生得失,生得失則語泄,語泄是韓舉國而與仲子為讎,豈不殆哉!」遂謝車騎人徒,聶政乃辭獨行。

杖劍至韓,韓相俠累方坐府上,持兵戟而衛侍者甚眾。聶政直入,上階刺殺俠累,左右大亂。聶政大呼,所擊殺者數十人,因自皮面決眼,自屠出腸,遂以死。

韓取聶政尸暴於市,購問莫知誰子。於是韓(購)縣[購]之,有能言殺相俠累者予千金。久之莫知也。

政姊榮聞人有刺殺韓相者,賊不得,國不知其名姓,暴其尸而縣之千金,乃於邑曰:「其是吾弟與?嗟乎,嚴仲子知吾弟!」立起,如韓,之市,而死者果政也,伏尸哭極哀,曰:「是軹深井里所謂聶政者也。」市行者諸眾人皆曰:「此人暴虐吾國相,王縣購其名姓千金,夫人不聞與?何敢來識之也?」榮應之曰:「聞之。然政所以蒙污辱自棄於市販之閒者,為老母幸無恙,妾未嫁也。親既以天年下世,妾已嫁夫,嚴仲子乃察舉吾弟困污之中而交之,澤厚矣,可柰何!士固為知己者死,今乃以妾尚在之故,重自刑以絕從,妾其柰何畏歿身之誅,終滅賢弟之名!」大驚韓市人。乃大呼天者三,卒於邑悲哀而死政之旁。

晉、楚、齊、衛聞之,皆曰:「非獨政能也,乃其姊亦烈女也。鄉使政誠知其姊無濡忍之志,不重暴骸之難,必絕險千里以列其名,姊弟俱僇於韓市者,亦未必敢以身許嚴仲子也。嚴仲子亦可謂知人能得士矣!」

其後二百二十餘年秦有荊軻之事。

荊軻者,衛人也。其先乃齊人,徙於衛,衛人謂之慶卿。而之燕,燕人謂之荊卿。

荊卿好讀書擊劍,以術說衛元君,衛元君不用。其後秦伐魏,置東郡,徙衛元君之支屬於野王。

荊軻嘗游過榆次,與蓋聶論劍,蓋聶怒而目之。荊軻出,人或言復召荊卿。蓋聶曰:「曩者吾與論劍有不稱者,吾目之;試往,是宜去,不敢留。」使使往之主人,荊卿則已駕而去榆次矣。使者還報,蓋聶曰:「固去也,吾曩者目攝之!」

荊軻游於邯鄲,魯句踐與荊軻博,爭道,魯句踐怒而叱之,荊軻嘿而逃去,遂不復會。

荊軻既至燕,愛燕之狗屠及善擊筑者高漸離。荊軻嗜酒,日與狗屠及高漸離飲於燕市,酒酣以往,高漸離擊筑,荊軻和而歌於市中,相樂也,已而相泣,旁若無人者。荊軻雖游於酒人乎,然其為人沈深好書;其所游諸侯,盡與其賢豪長者相結。其之燕,燕之處士田光先生亦善待之,知其非庸人也。

居頃之,會燕太子丹質秦亡歸燕。燕太子丹者,故嘗質於趙,而秦王政生於趙,其少時與丹驩。及政立為秦王,而丹質於秦。秦王之遇燕太子丹不善,故丹怨而亡歸。歸而求為報秦王者,國小,力不能。其後秦日出兵山東以伐齊、楚、三晉,稍蠶食諸侯,且至於燕,燕君臣皆恐禍之至。太子丹患之,問其傅鞠武。武對曰:「秦地遍天下,威脅韓、魏、趙氏,北有甘泉、谷口之固,南有涇、渭之沃,擅巴、漢之饒,右隴、蜀之山,左關、殽之險,民眾而士厲,兵革有餘。意有所出,則長城之南,易水以北,未有所定也。柰何以見陵之怨,欲批其逆鱗哉!」丹曰:「然則何由?」對曰:「請入圖之。」

居有閒,秦將樊於期得罪於秦王,亡之燕,太子受而捨之。鞠武諫曰:「不可。夫以秦王之暴而積怒於燕,足為寒心,又況聞樊將軍之所在乎?是謂『委肉當餓虎之蹊』也,禍必不振矣!雖有管、晏,不能為之謀也。願太子疾遣樊將軍入匈奴以滅口。請西約三晉,南連齊、楚,北購於單于,其後乃可圖也。」太子曰:「太傅之計,曠日彌久,心惛然,恐不能須臾。且非獨於此也,夫樊將軍窮困於天下,歸身於丹,丹終不以迫於彊秦而棄所哀憐之交,置之匈奴,是固丹命卒之時也。願太傅更慮之。」鞠武曰:「夫行危欲求安,造禍而求福,計淺而怨深,連結一人之後交,不顧國家之大害,此所謂『資怨而助禍』矣。夫以鴻毛燎於爐炭之上,必無事矣。且以鵰鷙之秦,行怨暴之怒,豈足道哉!燕有田光先生,其為人智深而勇沈,可與謀。」太子曰:「願因太傅而得交於田先生,可乎?」鞠武曰:「敬諾。」出見田先生,道「太子願圖國事於先生也」。田光曰:「敬奉教。」乃造焉。

太子逢迎,卻行為導,跪而蔽席。田光坐定,左右無人,太子避席而請曰:「燕秦不兩立,願先生留意也。」田光曰:「臣聞騏驥盛壯之時,一日而馳千里;至其衰老,駑馬先之。今太子聞光盛壯之時,不知臣精已消亡矣。雖然,光不敢以圖國事,所善荊卿可使也。」太子曰:「願因先生得結交於荊卿,可乎?」田光曰:「敬諾。」即起,趨出。太子送至門,戒曰:「丹所報,先生所言者,國之大事也,願先生勿泄也!」田光俛而笑曰:「諾。」僂行見荊卿,曰:「光與子相善,燕國莫不知。今太子聞光壯盛之時,不知吾形已不逮也,幸而教之曰『燕秦不兩立,願先生留意也』。光竊不自外,言足下於太子也,願足下過太子於宮。」荊軻曰:「謹奉教。」田光曰:「吾聞之,長者為行,不使人疑之。今太子告光曰:『所言者,國之大事也,願先生勿泄』,是太子疑光也。夫為行而使人疑之,非節俠也。」欲自殺以激荊卿,曰:「願足下急過太子,言光已死,明不言也。」因遂自刎而死。

荊軻遂見太子,言田光已死,致光之言。太子再拜而跪,膝行流涕,有頃而後言曰:「丹所以誡田先生毋言者,欲以成大事之謀也。今田先生以死明不言,豈丹之心哉!」荊軻坐定,太子避席頓首曰:「田先生不知丹之不肖,使得至前,敢有所道,此天之所以哀燕而不棄其孤也。今秦有貪利之心,而欲不可足也。非盡天下之地,臣海內之王者,其意不厭。今秦已虜韓王,盡納其地。又舉兵南伐楚,北臨趙;王翦將數十萬之眾距漳、鄴,而李信出太原、雲中。趙不能支秦,必入臣,入臣則禍至燕。燕小弱,數困於兵,今計舉國不足以當秦。諸侯服秦,莫敢合從。丹之私計愚,以為誠得天下之勇士使於秦,闚以重利;秦王貪,其勢必得所願矣。誠得劫秦王,使悉反諸侯侵地,若曹沫之與齊桓公,則大善矣;則不可,因而刺殺之。彼秦大將擅兵於外而內有亂,則君臣相疑,以其閒諸侯得合從,其破秦必矣。此丹之上願,而不知所委命,唯荊卿留意焉。」久之,荊軻曰:「此國之大事也,臣駑下,恐不足任使。」太子前頓首,固請毋讓,然後許諾。於是尊荊卿為上卿,舍上舍。太子日造門下,供太牢具,異物閒進,車騎美女恣荊軻所欲,以順適其意。

久之,荊軻未有行意。秦將王翦破趙,虜趙王,盡收入其地,進兵北略地至燕南界。太子丹恐懼,乃請荊軻曰:「秦兵旦暮渡易水,則雖欲長侍足下,豈可得哉!」荊軻曰:「微太子言,臣願謁之。今行而毋信,則秦未可親也。夫樊將軍,秦王購之金千斤,邑萬家。誠得樊將軍首與燕督亢之地圖,奉獻秦王,秦王必說見臣,臣乃得有以報。」太子曰:「樊將軍窮困來歸丹,丹不忍以己之私而傷長者之意,願足下更慮之!」

荊軻知太子不忍,乃遂私見樊於期曰:「秦之遇將軍可謂深矣,父母宗族皆為戮沒。今聞購將軍首金千斤,邑萬家,將柰何?」於期仰天太息流涕曰:「於期每念之,常痛於骨髓,顧計不知所出耳!」荊軻曰:「今有一言可以解燕國之患,報將軍之仇者,何如?」於期乃前曰:「為之柰何?」荊軻曰:「願得將軍之首以獻秦王,秦王必喜而見臣,臣左手把其袖,右手揕其匈,然則將軍之仇報而燕見陵之愧除矣。將軍豈有意乎?」樊於期偏袒搤捥而進曰:「此臣之日夜切齒腐心也,乃今得聞教!」遂自剄。太子聞之,馳往,伏尸而哭,極哀。既已不可柰何,乃遂盛樊於期首函封之。

於是太子豫求天下之利匕首,得趙人徐夫人匕首,取之百金,使工以藥焠之,以試人,血濡縷,人無不立死者。乃裝為遣荊卿。燕國有勇士秦舞陽,年十三,殺人,人不敢忤視。乃令秦舞陽為副。荊軻有所待,欲與俱;其人居遠未來,而為治行。頃之,未發,太子遲之,疑其改悔,乃復請曰:「日已盡矣,荊卿豈有意哉?丹請得先遣秦舞陽。」荊軻怒,叱太子曰:「何太子之遣?往而不返者,豎子也!且提一匕首入不測之彊秦,仆所以留者,待吾客與俱。今太子遲之,請辭決矣!」遂發。

太子及賓客知其事者,皆白衣冠以送之。至易水之上,既祖,取道,高漸離擊筑,荊軻和而歌,為變徵之聲,士皆垂淚涕泣。又前而為歌曰:「風蕭蕭兮易水寒,壯士一去兮不復還!」復為羽聲慨,士皆瞋目,發盡上指冠。於是荊軻就車而去,終已不顧。

遂至秦,持千金之資幣物,厚遺秦王寵臣中庶子蒙嘉。嘉為先言於秦王曰:「燕王誠振怖大王之威,不敢舉兵以逆軍吏,願舉國為內臣,比諸侯之列,給貢職如郡縣,而得奉守先王之宗廟。恐懼不敢自陳,謹斬樊於期之頭,及獻燕督亢之地圖,函封,燕王拜送于庭,使使以聞大王,唯大王命之。」秦王聞之,大喜,乃朝服,設九賓,見燕使者咸陽宮。荊軻奉樊於期頭函,而秦舞陽奉地圖柙,以次進。至陛,秦舞陽色變振恐,群臣怪之。荊軻顧笑舞陽,前謝曰:「北蕃蠻夷之鄙人,未嘗見天子,故振慴。願大王少假借之,使得畢使於前。」秦王謂軻曰:「取舞陽所持地圖。」軻既取圖奏之,秦王發圖,圖窮而匕首見。因左手把秦王之袖,而右手持匕首揕之。未至身,秦王驚,自引而起,袖絕。拔劍,劍長,操其室。時惶急,劍堅,故不可立拔。荊軻逐秦王,秦王環柱而走。群臣皆愕,卒起不意,盡失其度。而秦法,群臣侍殿上者不得持尺寸之兵;諸郎中執兵皆陳殿下,非有詔召不得上。方急時,不及召下兵,以故荊軻乃逐秦王。而卒惶急,無以擊軻,而以手共搏之。是時侍醫夏無且以其所奉藥囊提荊軻也。秦王方環柱走,卒惶急,不知所為,左右乃曰:「王負劍!」負劍,遂拔以擊荊軻,斷其左股。荊軻廢,乃引其匕首以擿秦王,不中,中桐柱。秦王復擊軻,軻被八創。軻自知事不就,倚柱而笑,箕踞以罵曰:「事所以不成者,以欲生劫之,必得約契以報太子也。」於是左右既前殺軻,秦王不怡者良久。已而論功,賞群臣及當坐者各有差,而賜夏無且黃金二百溢,曰:「無且愛我,乃以藥囊提荊軻也。」

於是秦王大怒,益發兵詣趙,詔王翦軍以伐燕。十月而拔薊城。燕王喜、太子丹等盡率其精兵東保於遼東。秦將李信追擊燕王急,代王嘉乃遺燕王喜書曰:「秦所以尤追燕急者,以太子丹故也。今王誠殺丹獻之秦王,秦王必解,而社稷幸得血食。」其後李信追丹,丹匿衍水中,燕王乃使使斬太子丹,欲獻之秦。秦復進兵攻之。後五年,秦卒滅燕,虜燕王喜。

其明年,秦并天下,立號為皇帝。於是秦逐太子丹、荊軻之客,皆亡。高漸離變名姓為人庸保,匿作於宋子。久之,作苦,聞其家堂上客擊筑,傍偟不能去。每出言曰:「彼有善有不善。」從者以告其主,曰:「彼庸乃知音,竊言是非。」家丈人召使前擊筑,一坐稱善,賜酒。而高漸離念久隱畏約無窮時,乃退,出其裝匣中筑與其善衣,更容貌而前。舉坐客皆驚,下與抗禮,以為上客。使擊筑而歌,客無不流涕而去者。宋子傳客之,聞於秦始皇。秦始皇召見,人有識者,乃曰:「高漸離也。」秦皇帝惜其善擊筑,重赦之,乃矐其目。使擊筑,未嘗不稱善。稍益近之,高漸離乃以鉛置筑中,復進得近,舉筑樸秦皇帝,不中。於是遂誅高漸離,終身不復近諸侯之人。

魯句踐已聞荊軻之刺秦王,私曰:「嗟乎,惜哉其不講於刺劍之術也!甚矣吾不知人也!曩者吾叱之,彼乃以我為非人也!」

太史公曰:世言荊軻,其稱太子丹之命,「天雨粟,馬生角」也,太過。又言荊軻傷秦王,皆非也。始公孫季功、董生與夏無且游,具知其事,為余道之如是。自曹沫至荊軻五人,此其義或成或不成,然其立意較然,不欺其志,名垂後世,豈妄也哉!


\end{pinyinscope}