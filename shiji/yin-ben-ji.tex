\article{殷本紀}

\begin{pinyinscope}
殷契,母曰簡狄,有娀氏之女,為帝嚳次妃。三人行浴,見玄鳥墮其卵,簡狄取吞之,因孕生契。契長而佐禹治水有功。帝舜乃命契曰:「百姓不親,五品不訓,汝為司徒而敬敷五教,五教在寬。」封于商,賜姓子氏。契興於唐、虞、大禹之際,功業著於百姓,百姓以平。

契卒,子昭明立。昭明卒,子相土立。相土卒,子昌若立。昌若卒,子曹圉立。曹圉卒,子冥立。冥卒,子振立。振卒,子微立。微卒,子報丁立。報丁卒,子報乙立。報乙卒,子報丙立。報丙卒,子主壬立。主壬卒,子主癸立。主癸卒,子天乙立,是為成湯。

成湯,自契至湯八遷。湯始居亳,從先王居,作帝誥。

湯征諸侯。葛伯不祀,湯始伐之。湯曰:「予有言:人視水見形,視民知治不。」伊尹曰:「明哉!言能聽,道乃進。君國子民,為善者皆在王官。勉哉,勉哉!」湯曰:「汝不能敬命,予大罰殛之,無有攸赦。」作湯征。

伊尹名阿衡。阿衡欲奸湯而無由,乃為有莘氏媵臣,負鼎俎,以滋味說湯,致于王道。或曰,伊尹處士,湯使人聘迎之,五反然後肯往從湯,言素王及九主之事。湯舉任以國政。伊尹去湯適夏。既醜有夏,復歸于亳。入自北門,遇女鳩、女房,作女鳩女房。

湯出,見野張網四面,祝曰:「自天下四方皆入吾網。」湯曰:「嘻,盡之矣!」乃去其三面,祝曰:「欲左,左。欲右,右。不用命,乃入吾網。」諸侯聞之,曰:「湯德至矣,及禽獸。」

當是時,夏桀為虐政淫荒,而諸侯昆吾氏為亂。湯乃興師率諸侯,伊尹從湯,湯自把鉞以伐昆吾,遂伐桀。湯曰:「格女眾庶,來,女悉聽朕言。匪台小子敢行舉亂,有夏多罪,予維聞女眾言,夏氏有罪。予畏上帝,不敢不正。今夏多罪,天命殛之。今女有眾,女曰:『我君不恤我眾,捨我嗇事而割政』。女其曰:『有罪,其柰何』?夏王率止眾力,率奪夏國。眾有率怠不和,曰:『是日何時喪?予與女皆亡』!夏德若茲,今朕必往。爾尚及予一人致天之罰,予其大理女。女毋不信,朕不食言。女不從誓言,予則帑僇女,無有攸赦。」以告令師,作湯誓。於是湯曰:「吾甚武」,號曰武王。

桀敗於有娀之虛,桀奔於鳴條,夏師敗績。湯遂伐三嵕,俘厥寶玉,義伯、仲伯作典寶。湯既勝夏,欲遷其社,不可,作夏社。伊尹報。於是諸侯畢服,湯乃踐天子位,平定海內。

湯歸至于泰卷陶,中壘作誥。既絀夏命,還亳,作湯誥:「維三月,王自至於東郊。告諸侯群后:『毋不有功於民,勤力乃事。予乃大罰殛女,毋予怨。』曰:『古禹、皋陶久勞于外,其有功乎民,民乃有安。東為江,北為濟,西為河,南為淮,四瀆已修,萬民乃有居。后稷降播,農殖百穀。三公咸有功于民,故後有立。昔蚩尤與其大夫作亂百姓,帝乃弗予,有狀。先王言不可不勉。』曰:『不道,毋之在國,女毋我怨。』」以令諸侯。伊尹作咸有一德,咎單作明居。

湯乃改正朔,易服色,上白,朝會以晝。

湯崩,太子太丁未立而卒,於是乃立太丁之弟外丙,是為帝外丙。帝外丙即位三年,崩,立外丙之弟中壬,是為帝中壬。帝中壬即位四年,崩,伊尹乃立太丁之子太甲。太甲,成湯適長孫也,是為帝太甲。帝太甲元年,伊尹作伊訓,作肆命,作徂后。

帝太甲既立三年,不明,暴虐,不遵湯法,亂德,於是伊尹放之於桐宮。三年,伊尹攝行政當國,以朝諸侯。

帝太甲居桐宮三年,悔過自責,反善,於是伊尹乃迎帝太甲而授之政。帝太甲修德,諸侯咸歸殷,百姓以寧。伊尹嘉之,乃作太甲訓三篇,褒帝太甲,稱太宗。

太宗崩,子沃丁立。帝沃丁之時,伊尹卒。既葬伊尹於亳,咎單遂訓伊尹事,作沃丁。

沃丁崩,弟太庚立,是為帝太庚。帝太庚崩,子帝小甲立。帝小甲崩,弟雍己立,是為帝雍己。殷道衰,諸侯或不至。

帝雍己崩,弟太戊立,是為帝太戊。帝太戊立伊陟為相。亳有祥桑谷共生於朝,一暮大拱。帝太戊懼,問伊陟。伊陟曰:「臣聞妖不勝德,帝之政其有闕與?帝其修德。」太戊從之,而祥桑枯死而去。伊陟贊言于巫咸。巫咸治王家有成,作咸艾,作太戊。帝太戊贊伊陟于廟,言弗臣,伊陟讓,作原命。殷復興,諸侯歸之,故稱中宗。

中宗崩,子帝中丁立。帝中丁遷于隞。河亶甲居相。祖乙遷于邢。帝中丁崩,弟外壬立,是為帝外壬。仲丁書闕不具。帝外壬崩,弟河亶甲立,是為帝河亶甲。河亶甲時,殷復衰。

河亶甲崩,子帝祖乙立。帝祖乙立,殷復興。巫賢任職。

祖乙崩,子帝祖辛立。帝祖辛崩,弟沃甲立,是為帝沃甲。帝沃甲崩,立沃甲兄祖辛之子祖丁,是為帝祖丁。帝祖丁崩,立弟沃甲之子南庚,是為帝南庚。帝南庚崩,立帝祖丁之子陽甲,是為帝陽甲。帝陽甲之時,殷衰。

自中丁以來,廢適而更立諸弟子,弟子或爭相代立,比九世亂,於是諸侯莫朝。

帝陽甲崩,弟盤庚立,是為帝盤庚。帝盤庚之時,殷已都河北,盤庚渡河南,復居成湯之故居,乃五遷,無定處。殷民咨胥皆怨,不欲徙。盤庚乃告諭諸侯大臣曰:「昔高后成湯與爾之先祖俱定天下,法則可修。捨而弗勉,何以成德!」乃遂涉河南,治亳,行湯之政,然後百姓由寧,殷道復興。諸侯來朝,以其遵成湯之德也。

帝盤庚崩,弟小辛立,是為帝小辛。帝小辛立,殷復衰。百姓思盤庚,乃作盤庚三篇。帝小辛崩,弟小乙立,是為帝小乙。

帝小乙崩,子帝武丁立。帝武丁即位,思復興殷,而未得其佐。三年不言,政事決定於冢宰,以觀國風。武丁夜夢得聖人,名曰說。以夢所見視群臣百吏,皆非也。於是乃使百工營求之野,得說於傅險中。是時說為胥靡,筑於傅險。見於武丁,武丁曰是也。得而與之語,果聖人,舉以為相,殷國大治。故遂以傅險姓之,號曰傅說。

帝武丁祭成湯,明日,有飛雉登鼎耳而呴,武丁懼。祖己曰:「王勿憂,先修政事。」祖己乃訓王曰:「唯天監下典厥義,降年有永有不永,非天夭民,中絕其命。民有不若德,不聽罪,天既附命正厥德,乃曰其奈何。鳴呼!王嗣敬民,罔非天繼,常祀毋禮于棄道。」武丁修政行德,天下咸驩,殷道復興。

帝武丁崩,子帝祖庚立。祖己嘉武丁之以祥雉為德,立其廟為高宗,遂作高宗肜日及訓。

帝祖庚崩,弟祖甲立,是為帝甲。帝甲淫亂,殷復衰。

帝甲崩,子帝廩辛立。帝廩辛崩,弟庚丁立,是為帝庚丁。帝庚丁崩,子帝武乙立。殷復去亳,徙河北。

帝武乙無道,為偶人,謂之天神。與之博,令人為行。天神不勝,乃僇辱之。為革囊,盛血,卬而射之,命曰「射天」。武乙獵於河渭之閒,暴雷,武乙震死。子帝太丁立。帝太丁崩,子帝乙立。帝乙立,殷益衰。

帝乙長子曰微子啟,啟母賤,不得嗣。少子辛,辛母正后,辛為嗣。帝乙崩,子辛立,是為帝辛,天下謂之紂。

帝紂資辨捷疾,聞見甚敏;材力過人,手格猛獸;知足以距諫,言足以飾非;矜人臣以能,高天下以聲,以為皆出己之下。好酒淫樂,嬖於婦人。愛妲己,妲己之言是從。於是使師涓作新淫聲,北里之舞,靡靡之樂。厚賦稅以實鹿臺之錢,而盈鉅橋之粟。益收狗馬奇物,充仞宮室。益廣沙丘苑臺,多取野獸蜚鳥置其中。慢於鬼神。大聚樂戲於沙丘,以酒為池,縣肉為林,使男女裸相逐其閒,為長夜之飲。

百姓怨望而諸侯有畔者,於是紂乃重刑辟,有炮格之法。以西伯昌、九侯、鄂侯為三公。九侯有好女,入之紂。九侯女不喜淫,紂怒,殺之,而醢九侯。鄂侯爭之彊,辨之疾,并脯鄂侯。西伯昌聞之,竊嘆。崇侯虎知之,以告紂,紂囚西伯羑里。西伯之臣閎夭之徒,求美女奇物善馬以獻紂,紂乃赦西伯。西伯出而獻洛西之地,以請除炮格之刑。紂乃許之,賜弓矢斧鉞,使得征伐,為西伯。而用費中為政。費中善諛,好利,殷人弗親。紂又用惡來。惡來善毀讒,諸侯以此益疏。

西伯歸,乃陰修德行善,諸侯多叛紂而往歸西伯。西伯滋大,紂由是稍失權重。王子比干諫,弗聽。商容賢者,百姓愛之,紂廢之。及西伯伐饑國,滅之,紂之臣祖伊聞之而咎周,恐,奔告紂曰:「天既訖我殷命,假人元龜,無敢知吉,非先王不相我後人,維王淫虐用自絕,故天棄我,不有安食,不虞知天性,不迪率典。今我民罔不欲喪,曰『天曷不降威,大命胡不至』?今王其柰何?」紂曰:「我生不有命在天乎!」祖伊反,曰:「紂不可諫矣。」西伯既卒,周武王之東伐,至盟津,諸侯叛殷會周者八百。諸侯皆曰:「紂可伐矣。」武王曰:「爾未知天命。」乃復歸。

紂愈淫亂不止。微子數諫不聽,乃與大師、少師謀,遂去。比干曰:「為人臣者,不得不以死爭。」乃彊諫紂。紂怒曰:「吾聞聖人心有七竅。」剖比干,觀其心。箕子懼,乃詳狂為奴,紂又囚之。殷之大師、少師乃持其祭樂器奔周。周武王於是遂率諸侯伐紂。紂亦發兵距之牧野。甲子日,紂兵敗。紂走入,登鹿臺,衣其寶玉衣,赴火而死。周武王遂斬紂頭,縣之[大]白旗。殺妲己。釋箕子之囚,封比干之墓,表商容之閭。封紂子武庚、祿父,以續殷祀,令修行盤庚之政。殷民大說。於是周武王為天子。其後世貶帝號,號為王。而封殷後為諸侯,屬周。

周武王崩,武庚與管叔、蔡叔作亂,成王命周公誅之,而立微子於宋,以續殷後焉。

太史公曰:余以頌次契之事,自成湯以來,采於書詩。契為子姓,其後分封,以國為姓,有殷氏、來氏、宋氏、空桐氏、稚氏、北殷氏、目夷氏。孔子曰,殷路車為善,而色尚白。


\end{pinyinscope}