\article{穰侯列傳}

\begin{pinyinscope}
穰侯魏冉者,秦昭王母宣太后弟也。其先楚人,姓羋氏。

秦武王卒,無子,立其弟為昭王。昭王母故號為羋八子,及昭王即位,羋八子號為宣太后。宣太后非武王母。武王母號曰惠文后,先武王死。宣太后二弟:其異父長弟曰穰侯,姓魏氏,名冉;同父弟曰羋戎,為華陽君。而昭王同母弟曰高陵君、涇陽君。而魏冉最賢,自惠王、武王時任職用事。武王卒,諸弟爭立,唯魏冉力為能立昭王。昭王即位,以冉為將軍,衛咸陽。誅季君之亂,而逐武王后出之魏,昭王諸兄弟不善者皆滅之,威振秦國。昭王少,宣太后自治,任魏冉為政。

昭王七年,樗里子死,而使涇陽君質於齊。趙人樓緩來相秦,趙不利,乃使仇液之秦,請以魏冉為秦相。仇液將行,其客宋公謂液曰:「秦不聽公,樓緩必怨公。公不若謂樓緩曰『請為公毋急秦』。秦王見趙請相魏冉之不急,且不聽公。公言而事不成,以德樓子;事成,魏冉故德公矣。」於是仇液從之。而秦果免樓緩而魏冉相秦。

欲誅呂禮,禮出奔齊。昭王十四年,魏冉舉白起,使代向壽將而攻韓、魏,敗之伊闕,斬首二十四萬,虜魏將公孫喜。明年,又取楚之宛、葉。魏冉謝病免相,以客卿壽燭為相。其明年,燭免,復相冉,乃封魏冉於穰,復益封陶,號曰穰侯。

穰侯封四歲,為秦將攻魏。魏獻河東方四百里。拔魏之河內,取城大小六十餘。昭王十九年,秦稱西帝,齊稱東帝。月餘,呂禮來,而齊、秦各復歸帝為王。魏冉復相秦,六歲而免。免二歲,復相秦。四歲,而使白起拔楚之郢,秦置南郡。乃封白起為武安君。白起者,穰侯之所任舉也,相善。於是穰侯之富,富於王室。

昭王三十二年,穰侯為相國,將兵攻魏,走芒卯,入北宅,遂圍大梁。梁大夫須賈說穰侯曰:「臣聞魏之長吏謂魏王曰:『昔梁惠王伐趙,戰勝三梁,拔邯鄲;趙氏不割,而邯鄲復歸。齊人攻衛,拔故國,殺子良;衛人不割,而故地復反。衛、趙之所以國全兵勁而地不并於諸侯者,以其能忍難而重出地也。宋、中山數伐割地,而國隨以亡。臣以為衛、趙可法,而宋、中山可為戒也。秦,貪戾之國也,而毋親。蠶食魏氏,又盡晉國,戰勝暴子,割八縣,地未畢入,兵復出矣。夫秦何厭之有哉!今又走芒卯,入北宅,此非敢攻梁也,且劫王以求多割地。王必勿聽也。今王背楚、趙而講秦,楚、趙怒而去王,與王爭事秦,秦必受之。秦挾楚、趙之兵以復攻梁,則國求無亡不可得也。願王之必無講也。王若欲講,少割而有質;不然,必見欺。』此臣之所聞於魏也,願君[王]之以是慮事也。《周書》曰『惟命不于常』,此言幸之不可數也。夫戰勝暴子,割八縣,此非兵力之精也,又非計之工也,天幸為多矣。今又走芒卯,入北宅,以攻大梁,是以天幸自為常也。智者不然。臣聞魏氏悉其百縣勝甲以上戍大梁,臣以為不下三十萬。以三十萬之眾守梁七仞之城,臣以為湯、武復生,不易攻也。夫輕背楚、趙之兵,陵七仞之城,戰三十萬之眾,而志必舉之,臣以為自天地始分以至于今,未嘗有者也。攻而不拔,秦兵必罷,陶邑必亡,則前功必棄矣。今魏氏方疑,可以少割收也。願君逮楚、趙之兵未至於梁,亟以少割收魏。魏方疑而得以少割為利,必欲之,則君得所欲矣。楚、趙怒於魏之先己也,必爭事秦,從以此散,而君後擇焉。且君之得地豈必以兵哉!邦晉國,秦兵不攻,而魏必效絳安邑。又為陶開兩道,幾盡故宋,衛必效單父。秦兵可全,而君制之,何索而不得,何為而不成!願君熟慮之而無行危。」穰侯曰:「善。」乃罷梁圍。

明年,魏背秦,與齊從親。秦使穰侯伐魏,斬首四萬,走魏將暴鳶,得魏三縣。穰侯益封。

明年,穰侯與白起客卿胡陽復攻趙、韓、魏,破芒卯於華陽下,斬首十萬,取魏之卷、蔡陽、長社,趙氏觀津。且與趙觀津,益趙以兵,伐齊。齊襄王懼,使蘇代為齊陰遺穰侯書曰:「臣聞往來者言曰『秦將益趙甲四萬以伐齊』,臣竊必之敝邑之王曰『秦王明而熟於計,穰侯智而習於事,必不益趙甲四萬以伐齊』。是何也?夫三晉之相與也,秦之深讎也。百相背也,百相欺也,不為不信,不為無行。今破齊以肥趙。趙,秦之深讎,不利於秦。此一也。秦之謀者,必曰『破齊,獘晉、楚,而後制晉、楚之勝』。夫齊,罷國也,以天下攻齊,如以千鈞之弩決潰癕也,必死,安能獘晉、楚?此二也。秦少出兵,則晉、楚不信也;多出兵,則晉、楚為制於秦。齊恐,不走秦,必走晉、楚。此三也。秦割齊以啖晉、楚,晉、楚案之以兵,秦反受敵。此四也。是晉、楚以秦謀齊,以齊謀秦也,何晉、楚之智而秦、齊之愚?此五也。故得安邑以善事之,亦必無患矣。秦有安邑,韓氏必無上黨矣。取天下之腸胃,與出兵而懼其不反也,孰利?臣故曰秦王明而熟於計,穰侯智而習於事,必不益趙甲四萬以代齊矣。」於是穰侯不行,引兵而歸。

昭王三十六年,相國穰侯言客卿灶,欲伐齊取剛、壽,以廣其陶邑。於是魏人范睢自謂張祿先生,譏穰侯之伐齊,乃越三晉以攻齊也,以此時奸說秦昭王。昭王於是用范睢。范睢言宣太后專制,穰侯擅權於諸侯,涇陽君、高陵君之屬太侈,富於王室。於是秦昭王悟,乃免相國,令涇陽之屬皆出關,就封邑。穰侯出關,輜車千乘有餘。穰侯卒於陶,而因葬焉。秦復收陶為郡。

太史公曰:穰侯,昭王親舅也。而秦所以東益地,弱諸侯,嘗稱帝於天下,天下皆西鄉稽首者,穰侯之功也。及其貴極富溢,一夫開說,身折勢奪而以憂死,況於羈旅之臣乎!


\end{pinyinscope}