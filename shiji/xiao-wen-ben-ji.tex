\article{孝文本紀}

\begin{pinyinscope}
孝文皇帝,高祖中子也。高祖十一年春,已破陳豨軍,定代地,立為代王,都中都。太后薄氏子。即位十七年,高后八年七月,高后崩。九月,諸呂呂產等欲為亂,以危劉氏,大臣共誅之,謀召立代王,事在呂后語中。

丞相陳平、太尉周勃等使人迎代王。代王問左右郎中令張武等。張武等議曰:「漢大臣皆故高帝時大將,習兵,多謀詐,此其屬意非止此也,特畏高帝、呂太后威耳。今已誅諸呂,新啑血京師,此以迎大王為名,實不可信。願大王稱疾毋往,以觀其變。」中尉宋昌進曰:「群臣之議皆非也。夫秦失其政,諸侯豪桀并起,人人自以為得之者以萬數,然卒踐天子之位者,劉氏也,天下絕望,一矣。高帝封王子弟,地犬牙相制,此所謂盤石之宗也,天下服其彊,二矣。漢興,除秦苛政,約法令,施德惠,人人自安,難動搖,三矣。夫以呂太后之嚴,立諸呂為三王,擅權專制,然而太尉以一節入北軍,一呼士皆左袒,為劉氏,叛諸呂,卒以滅之。此乃天授,非人力也。今大臣雖欲為變,百姓弗為使,其黨寧能專一邪?方今內有朱虛、東牟之親,外畏吳、楚、淮南、瑯邪、齊、代之彊。方今高帝子獨淮南王與大王,大王又長,賢聖仁孝,聞於天下,故大臣因天下之心而欲迎立大王,大王勿疑也。」代王報太后計之,猶與未定。卜之龜,卦兆得大橫。占曰:「大橫庚庚,余為天王,夏啟以光。」代王曰:「寡人固已為王矣,又何王?」卜人曰:「所謂天王者乃天子。」於是代王乃遣太后弟薄昭往見絳侯,絳侯等具為昭言所以迎立王意。薄昭還報曰:「信矣,毋可疑者。」代王乃笑謂宋昌曰:「果如公言。」乃命宋昌參乘,張武等六人乘傳詣長安。至高陵休止,而使宋昌先馳之長安觀變。

昌至渭橋,丞相以下皆迎。宋昌還報。代王馳至渭橋,群臣拜謁稱臣。代王下車拜。太尉勃進曰:「願請閒言。」宋昌曰:「所言公,公言之。所言私,王者不受私。」太尉乃跪上天子璽符。代王謝曰:「至代邸而議之。」遂馳入代邸。群臣從至。丞相陳平、太尉周勃、大將軍陳武、御史大夫張蒼、宗正劉郢、朱虛侯劉章、東牟侯劉興居、典客劉揭皆再拜言曰:「子弘等皆非孝惠帝子,不當奉宗廟。臣謹請(與)陰安侯列侯頃王后與瑯邪王、宗室、大臣、列侯、吏二千石議曰:『大王高帝長子,宜為高帝嗣。』願大王即天子位。」代王曰:「奉高帝宗廟,重事也。寡人不佞,不足以稱宗廟。願請楚王計宜者,寡人不敢當。」群臣皆伏固請。代王西鄉讓者三,南鄉讓者再。丞相平等皆曰:「臣伏計之,大王奉高帝宗廟最宜稱,雖天下諸侯萬民以為宜。臣等為宗廟社稷計,不敢忽。願大王幸聽臣等。臣謹奉天子璽符再拜上。」代王曰:「宗室將相王列侯以為莫宜寡人,寡人不敢辭。」遂即天子位。

群臣以禮次侍。乃使太仆嬰與東牟侯興居清宮,奉天子法駕,迎于代邸。皇帝即日夕入未央宮。乃夜拜宋昌為衛將軍,鎮撫南北軍。以張武為郎中令,行殿中。還坐前殿。於是夜下詔書曰:「閒者諸呂用事擅權,謀為大逆,欲以危劉氏宗廟,賴將相列侯宗室大臣誅之,皆伏其辜。朕初即位,其赦天下,賜民爵一級,女子百戶牛酒,酺五日。」

辛亥,皇帝即阼,謁高廟。右丞相平徙為左丞相,太尉勃為右丞相,大將軍灌嬰為太尉。諸呂所奪齊楚故地,皆復與之。

壬子,遣車騎將軍薄昭迎皇太后於代。皇帝曰:「呂產自置為相國,呂祿為上將軍,擅矯遣灌將軍嬰將兵擊齊,欲代劉氏,嬰留滎陽弗擊,與諸侯合謀以誅呂氏。呂產欲為不善,丞相陳平與太尉周勃謀奪呂產等軍。朱虛侯劉章首先捕呂產等。太尉身率襄平侯通持節承詔入北軍。典客劉揭身奪趙王呂祿印。益封太尉勃萬戶,賜金五千斤。丞相陳平、灌將軍嬰邑各三千戶,金二千斤。朱虛侯劉章、襄平侯通、東牟侯劉興居邑各二千戶,金千斤。封典客揭為陽信侯,賜金千斤。」

十二月,上曰:「法者,治之正也,所以禁暴而率善人也。今犯法已論,而使毋罪之父母妻子同產坐之,及為收帑,朕甚不取。其議之。」有司皆曰:「民不能自治,故為法以禁之。相坐坐收,所以累其心,使重犯法,所從來遠矣。如故便。」上曰:「朕聞法正則民愨,罪當則民從。且夫牧民而導之善者,吏也。其既不能導,又以不正之法罪之,是反害於民為暴者也。何以禁之?朕未見其便,其孰計之。」有司皆曰:「陛下加大惠,德甚盛,非臣等所及也。請奉詔書,除收帑諸相坐律令。」

正月,有司言曰:「蚤建太子,所以尊宗廟。請立太子。」上曰:「朕既不德,上帝神明未歆享,天下人民未有嗛志。今縱不能博求天下賢聖有德之人而禪天下焉,而曰豫建太子,是重吾不德也。謂天下何?其安之。」有司曰:「豫建太子,所以重宗廟社稷,不忘天下也。」上曰:「楚王,季父也,春秋高,閱天下之義理多矣,明於國家之大體。吳王於朕,兄也,惠仁以好德。淮南王,弟也,秉德以陪朕。豈為不豫哉!諸侯王宗室昆弟有功臣,多賢及有德義者,若舉有德以陪朕之不能終,是社稷之靈,天下之福也。今不選舉焉,而曰必子,人其以朕為忘賢有德者而專於子,非所以憂天下也。朕甚不取也。」有司皆固請曰:「古者殷周有國,治安皆千餘歲,古之有天下者莫長焉,用此道也。立嗣必子,所從來遠矣。高帝親率士大夫,始平天下,建諸侯,為帝者太祖。諸侯王及列侯始受國者皆亦為其國祖。子孫繼嗣,世世弗絕,天下之大義也,故高帝設之以撫海內。今釋宜建而更選於諸侯及宗室,非高帝之志也。更議不宜。子某最長,純厚慈仁,請建以為太子。」上乃許之。因賜天下民當代父後者爵各一級封將軍薄昭為軹侯。

三月,有司請立皇后。薄太后曰:「諸侯皆同姓,立太子母為皇后。」皇后姓竇氏。上為立后故,賜天下鰥寡孤獨窮困及年八十已上孤兒九歲已下布帛米肉各有數。上從代來,初即位,施德惠天下,填撫諸侯四夷皆洽驩,乃循從代來功臣。上曰:「方大臣之誅諸呂迎朕,朕狐疑,皆止朕,唯中尉宋昌勸朕,朕以得保奉宗廟。已尊昌為衛將軍,其封昌為壯武侯。諸從朕六人,官皆至九卿。」

上曰:「列侯從高帝入蜀、漢中者六十八人皆益封各三百戶,故吏二千石以上從高帝潁川守尊等十人食邑六百戶,淮陽守申徒嘉等十人五百戶,衛尉定等十人四百戶。封淮南王舅父趙兼為周陽侯,齊王舅父駟鈞為清郭侯。」秋,封故常山丞相蔡兼為樊侯。

人或說右丞相曰:「君本誅諸呂,迎代王,今又矜其功,受上賞,處尊位,禍且及身。」右丞相勃乃謝病免罷,左丞相平專為丞相。

二年十月,丞相平卒,復以絳侯勃為丞相。上曰:「朕聞古者諸侯建國千餘(歲),各守其地,以時入貢,民不勞苦,上下驩欣,靡有遺德。今列侯多居長安,邑遠,吏卒給輸費苦,而列侯亦無由教馴其民。其令列侯之國,為吏及詔所止者,遣太子。」

十一月晦,日有食之。十二月望,日又食。上曰:「朕聞之,天生蒸民,為之置君以養治之。人主不德,布政不均,則天示之以菑,以誡不治。乃十一月晦,日有食之,適見于天,菑孰大焉!朕獲保宗廟,以微眇之身託于兆民君王之上,天下治亂,在朕一人,唯二三執政猶吾股肱也。朕下不能理育群生,上以累三光之明,其不德大矣。令至,其悉思朕之過失,及知見思之所不及,匄以告朕。及舉賢良方正能直言極諫者,以匡朕之不逮。因各飭其任職,務省繇費以便民。朕既不能遠德,故憪然念外人之有非,是以設備未息。今縱不能罷邊屯戍,而又飭兵厚衛,其罷衛將軍軍。太仆見馬遺財足,餘皆以給傳置。」

正月,上曰:「農,天下之本,其開籍田,朕親率耕,以給宗廟粢盛。」

三月,有司請立皇子為諸侯王。上曰:「趙幽王幽死,朕甚憐之,已立其長子遂為趙王。遂弟辟彊及齊悼惠王子朱虛侯章、東牟侯興居有功,可王。」乃立趙幽王少子辟彊為河閒王,以齊劇郡立朱虛侯為城陽王,立東牟侯為濟北王,皇子武為代王,子參為太原王,子揖為梁王。

上曰:「古之治天下,朝有進善之旌,誹謗之木,所以通治道而來諫者。今法有誹謗妖言之罪,是使眾臣不敢盡情,而上無由聞過失也。將何以來遠方之賢良?其除之。民或祝詛上以相約結而後相謾,吏以為大逆,其有他言,而吏又以為誹謗。此細民之愚無知抵死,朕甚不取。自今以來,有犯此者勿聽治。」

九月,初與郡國守相為銅虎符、竹使符。

三年十月丁酉晦,日有食之。十一月,上曰:「前日(計)[詔]遣列侯之國,或辭未行。丞相朕之所重,其為朕率列侯之國。」絳侯勃免丞相就國,以太尉潁陰侯嬰為丞相。罷太尉官,屬丞相。四月,城陽王章薨。淮南王長與從者魏敬殺辟陽侯審食其。

五月,匈奴入北地,居河南為寇。帝初幸甘泉。六月,帝曰:「漢與匈奴約為昆弟,毋使害邊境,所以輸遺匈奴甚厚。今右賢王離其國,將眾居河南降地,非常故,往來近塞,捕殺吏卒,驅保塞蠻夷,令不得居其故,陵轢邊吏,入盜,甚敖無道,非約也。其發邊吏騎八萬五千詣高奴,遣丞相潁陰侯灌嬰擊匈奴。」匈奴去,發中尉材官屬衛將軍軍長安。

辛卯,帝自甘泉之高奴,因幸太原,見故群臣,皆賜之。舉功行賞,諸民里賜牛酒。復晉陽中都民三歲。留游太原十餘日。

濟北王興居聞帝之代,欲往擊胡,乃反,發兵欲襲滎陽。於是詔罷丞相兵,遣棘蒲侯陳武為大將軍,將十萬往擊之。祁侯賀為將軍,軍滎陽。七月辛亥,帝自太原至長安。乃詔有司曰:「濟北王背德反上,詿誤吏民,為大逆。濟北吏民兵未至先自定,及以軍地邑降者,皆赦之,復官爵。與王興居去來,亦赦之。」八月,破濟北軍,虜其王。赦濟北諸吏民與王反者。

六年,有司言淮南王長廢先帝法,不聽天子詔,居處毋度,出入擬於天子,擅為法令,與棘蒲侯太子奇謀反,遣人使閩越及匈奴,發其兵,欲以危宗廟社稷。群臣議,皆曰「長當棄市」帝不忍致法於王,赦其罪,廢勿王。群臣請處王蜀嚴道、邛都,帝許之。長未到處所,行病死,上憐之。後十六年,追尊淮南王長謚為厲王,立其子三人為淮南王、衡山王、廬江王。

十三年夏,上曰:「蓋聞天道禍自怨起而福繇德興。百官之非,宜由朕躬。今祕祝之官移過于下,以彰吾之不德,朕甚不取。其除之。」

五月,齊太倉令淳于公有罪當刑,詔獄逮徙系長安。太倉公無男,有女五人。太倉公將行會逮,罵其女曰:「生子不生男,有緩急非有益也!」其少女緹縈自傷泣,乃隨其父至長安,上書曰:「妾父為吏,齊中皆稱其廉平,今坐法當刑。妾傷夫死者不可復生,刑者不可復屬,雖復欲改過自新,其道無由也。妾願沒入為官婢,贖父刑罪,使得自新。」書奏天子,天子憐悲其意,乃下詔曰:「蓋聞有虞氏之時,畫衣冠異章服以為僇,而民不犯。何則?至治也。今法有肉刑三,而姦不止,其咎安在?非乃朕德薄而教不明歟?吾甚自愧。故夫馴道不純而愚民陷焉。《詩》曰『愷悌君子,民之父母』。今人有過,教未施而刑加焉?或欲改行為善而道毋由也。朕甚憐之。夫刑至斷支體,刻肌膚,終身不息,何其楚痛而不德也,豈稱為民父母之意哉!其除肉刑。」

上曰:「農,天下之本,務莫大焉。今勤身從事而有租稅之賦,是為本末者毋以異,其於勸農之道未備。其除田之租稅。」

十四年冬,匈奴謀入邊為寇,攻朝那塞,殺北地都尉卬。上乃遣三將軍軍隴西、北地、上郡,中尉周舍為衛將軍,郎中令張武為車騎將軍,軍渭北,車千乘,騎卒十萬。帝親自勞軍,勒兵申教令,賜軍吏卒。帝欲自將擊匈奴,群臣諫,皆不聽。皇太后固要帝,帝乃止。於是以東陽侯張相如為大將軍,成侯赤為內史,欒布為將軍,擊匈奴。匈奴遁走。

春,上曰:「朕獲執犧牲珪幣以事上帝宗廟,十四年于今,歷日(縣)[綿]長,以不敏不明而久撫臨天下,朕甚自愧。其廣增諸祀墠場珪幣。昔先王遠施不求其報,望祀不祈其福,右賢左戚,先民後己,至明之極也。今吾聞祠官祝釐,皆歸福朕躬,不為百姓,朕甚愧之。夫以朕不德,而躬享獨美其福,百姓不與焉,是重吾不德。其令祠官致敬,毋有所祈。」

是時北平侯張蒼為丞相,方明律歷。魯人公孫臣上書陳終始傳五德事,言方今土德時,土德應黃龍見,當改正朔服色制度。天子下其事與丞相議。丞相推以為今水德,始明正十月上黑事,以為其言非是,請罷之。

十五年,黃龍見成紀,天子乃復召魯公孫臣,以為博士,申明土德事。於是上乃下詔曰:「有異物之神見于成紀,無害於民,歲以有年。朕親郊祀上帝諸神。禮官議,毋諱以勞朕。」有司禮官皆曰:「古者天子夏躬親禮祀上帝於郊,故曰郊。」於是天子始幸雍,郊見五帝,以孟夏四月答禮焉。趙人新垣平以望氣見,因說上設立渭陽五廟。欲出周鼎,當有玉英見。

十六年,上親郊見渭陽五帝廟,亦以夏答禮而尚赤。

十七年,得玉杯,刻曰「人主延壽」。於是天子始更為元年,令天下大酺。其歲,新垣平事覺,夷三族。

後二年,上曰:「朕既不明,不能遠德,是以使方外之國或不寧息。夫四荒之外不安其生,封畿之內勤勞不處,二者之咎,皆自於朕之德薄而不能遠達也。閒者累年,匈奴并暴邊境,多殺吏民,邊臣兵吏又不能諭吾內志,以重吾不德也。夫久結難連兵,中外之國將何以自寧?今朕夙興夜寐,勤勞天下,憂苦萬民,為之怛惕不安,未嘗一日忘於心,故遣使者冠蓋相望,結軼於道,以諭朕意於單于。今單于反古之道,計社稷之安,便萬民之利,親與朕俱棄細過,偕之大道,結兄弟之義,以全天下元元之民。和親已定,始于今年。」

後六年冬,匈奴三萬人入上郡,三萬人入雲中。以中大夫令勉為車騎將軍,軍飛狐;故楚相蘇意為將軍,軍句注;將軍張武屯北地;河內守周亞夫為將軍,居細柳;宗正劉禮為將軍,居霸上;祝茲侯軍棘門:以備胡。數月,胡人去,亦罷。

天下旱,蝗。帝加惠:令諸侯毋入貢,弛山澤,減諸服御狗馬,損郎吏員,發倉庾以振貧民,民得賣爵。

孝文帝從代來,即位二十三年,宮室苑囿狗馬服御無所增益,有不便,輒弛以利民。嘗欲作露臺,召匠計之,直百金。上曰:「百金中民十家之產,吾奉先帝宮室,常恐羞之,何以臺為!」上常衣綈衣,所幸慎夫人,令衣不得曳地,幃帳不得文繡,以示敦樸,為天下先。治霸陵皆以瓦器,不得以金銀銅錫為飾,不治墳,欲為省,毋煩民。南越王尉佗自立為武帝,然上召貴尉佗兄弟,以德報之,佗遂去帝稱臣。與匈奴和親,匈奴背約入盜,然令邊備守,不發兵深入,惡煩苦百姓。吳王詐病不朝,就賜几杖。群臣如袁盎等稱說雖切,常假借用之。群臣如張武等受賂遺金錢,覺,上乃發御府金錢賜之,以愧其心,弗下吏。專務以德化民,是以海內殷富,興於禮義。

後七年六月己亥,帝崩於未央宮。遺詔曰:「朕聞蓋天下萬物之萌生,靡不有死。死者天地之理,物之自然者,奚可甚哀。當今之時,世咸嘉生而惡死,厚葬以破業,重服以傷生,吾甚不取。且朕既不德,無以佐百姓;今崩,又使重服久臨,以離寒暑之數,哀人之父子,傷長幼之志,損其飲食,絕鬼神之祭祀,以重吾不德也,謂天下何!朕獲保宗廟,以眇眇之身託于天下君王之上,二十有餘年矣。賴天地之靈,社稷之福,方內安寧,靡有兵革。朕既不敏,常畏過行,以羞先帝之遺德;維年之久長,懼于不終。今乃幸以天年,得復供養于高廟。朕之不明與嘉之,其奚哀悲之有!其令天下吏民,令到出臨三日,皆釋服。毋禁取婦嫁女祠祀飲酒食肉者。自當給喪事服臨者,皆無踐。绖帶無過三寸,毋布車及兵器,毋發民男女哭臨宮殿。宮殿中當臨者,皆以旦夕各十五舉聲,禮畢罷。非旦夕臨時,禁毋得擅哭。已下,服大紅十五日,小紅十四日,纖七日,釋服。佗不在令中者,皆以此令比率從事。布告天下,使明知朕意。霸陵山川因其故,毋有所改。歸夫人以下至少使。」令中尉亞夫為車騎將軍,屬國悍為將屯將軍,郎中令武為復土將軍,發近縣見卒萬六千人,發內史卒萬五千人,藏郭穿復土屬將軍武。

乙巳,群臣皆頓首上尊號曰孝文皇帝。

太子即位于高廟。丁未,襲號曰皇帝。

孝景皇帝元年十月,制詔御史:「蓋聞古者祖有功而宗有德,制禮樂各有由。聞歌者,所以發德也;舞者,所以明功也。高廟酎,奏武德、文始、五行之舞。孝惠廟酎,奏文始、五行之舞。孝文皇帝臨天下,通關梁,不異遠方。除誹謗,去肉刑,賞賜長老,收恤孤獨,以育群生。減嗜欲,不受獻,不私其利也。罪人不帑,不誅無罪。除(肉)[宮]刑,出美人,重絕人之世。朕既不敏,不能識。此皆上古之所不及,而孝文皇帝親行之。德厚侔天地,利澤施四海,靡不獲福焉。明象乎日月,而廟樂不稱。朕甚懼焉。其為孝文皇帝廟為昭德之舞,以明休德。然後祖宗之功德著於竹帛,施于萬世,永永無窮,朕甚嘉之。其與丞相、列侯、中二千石、禮官具為禮儀奏。」丞相臣嘉等言:「陛下永思孝道,立昭德之舞以明孝文皇帝之盛德。皆臣嘉等愚所不及。臣謹議:世功莫大於高皇帝,德莫盛於孝文皇帝,高皇廟宜為帝者太祖之廟,孝文皇帝廟宜為帝者太宗之廟。天子宜世世獻祖宗之廟。郡國諸侯宜各為孝文皇帝立太宗之廟。諸侯王列侯使者侍祠天子,歲獻祖宗之廟。請著之竹帛,宣布天下。」制曰:「可。」

太史公曰:孔子言「必世然後仁。善人之治國百年,亦可以勝殘去殺」。誠哉是言!漢興,至孝文四十有餘載,德至盛也。廩廩鄉改正服封禪矣,謙讓未成於今。嗚呼,豈不仁哉!


\end{pinyinscope}