\article{夏本紀}

\begin{pinyinscope}
夏禹,名曰文命。禹之父曰鯀,鯀之父曰帝顓頊,顓頊之父曰昌意,昌意之父曰黃帝。禹者,黃帝之玄孫而帝顓頊之孫也。禹之曾大父昌意及父鯀皆不得在帝位,為人臣。

當帝堯之時,鴻水滔天,浩浩懷山襄陵,下民其憂。堯求能治水者,群臣四嶽皆曰鯀可。堯曰:「鯀為人負命毀族,不可。」四嶽曰:「等之未有賢於鯀者,願帝試之。」於是堯聽四嶽,用鯀治水。九年而水不息,功用不成。於是帝堯乃求人,更得舜。舜登用,攝行天子之政,巡狩。行視鯀之治水無狀,乃殛鯀於羽山以死。天下皆以舜之誅為是。於是舜舉鯀子禹,而使續鯀之業。

堯崩,帝舜問四嶽曰:「有能成美堯之事者使居官?」皆曰:「伯禹為司空,可成美堯之功。」舜曰:「嗟,然!」命禹:「女平水土,維是勉之。」禹拜稽首,讓於契、后稷、皋陶。舜曰:「女其往視爾事矣。」

禹為人敏給克勤;其德不違,其仁可親,其言可信;聲為律,身為度,稱以出;亹亹穆穆,為綱為紀。

禹乃遂與益、后稷奉帝命,命諸侯百姓興人徒以傅土,行山表木,定高山大川。禹傷先人父鯀功之不成受誅,乃勞身焦思,居外十三年,過家門不敢入。薄衣食,致孝于鬼神。卑宮室,致費於溝淢。陸行乘車,水行乘船,泥行乘橇,山行乘檋。左準繩,右規矩,載四時,以開九州,通九道,陂九澤,度九山。令益予眾庶稻,可種卑溼。命后稷予眾庶難得之食。食少,調有餘相給,以均諸侯。禹乃行相地宜所有以貢,及山川之便利。

禹行自冀州始。冀州:既載壺口,治梁及岐。既修太原,至于嶽陽。覃懷致功,至於衡漳。其土白壤。賦上上錯,田中中,常、衛既從,大陸既為。鳥夷皮服。夾右碣石,入于海。

濟、河維沇州:九河既道,雷夏既澤,雍、沮會同,桑土既蠶,於是民得下丘居土。其土黑墳,草繇木條。田中下,賦貞,作十有三年乃同。其貢漆絲,其篚織文。浮於濟、漯,通於河。

海岱維青州:堣夷既略,濰、淄其道。其土白墳,海濱廣潟,厥田斥鹵。田上下,賦中上。厥貢鹽絺,海物維錯,岱畎絲、枲、鉛、松、怪石,萊夷為牧,其篚酓絲。浮於汶,通於濟。

海岱及淮維徐州:淮、沂其治,蒙、羽其藝。大野既都,東原底平。其土赤埴墳,草木漸包。其田上中,賦中中。貢維土五色,羽畎夏狄,嶧陽孤桐,泗濱浮磬,淮夷蠙珠臮魚,其篚玄纖縞。浮于淮、泗,通于河。

淮海維揚州:彭蠡既都,陽鳥所居。三江既入,震澤致定。竹箭既布。其草惟夭,其木惟喬,其土涂泥。田下下,賦下上上雜。貢金三品,瑤、琨、竹箭,齒、革、羽、旄,島夷卉服,其篚織貝,其包橘、柚錫貢。均江海,通淮、泗。

荊及衡陽維荊州:江、漢朝宗于海。九江甚中,沱、涔已道,云土、夢為治。其土涂泥。田下中,賦上下。貢羽、旄、齒、革,金三品,杶、榦、栝、柏,礪、砥、砮、丹,維箘簬、楛,三國致貢其名,包匭菁茅,其篚玄纁璣組,九江入賜大龜。浮于江、沱、涔、(于)漢,踰于雒,至于南河。

荊河惟豫州:伊、雒、瀍、澗既入于河,滎播既都,道荷澤,被明都。其土壤,下土墳壚。田中上,賦雜上中。貢漆、絲、絺、紵,其篚纖絮,錫貢磬錯。浮於雒,達於河。

華陽黑水惟梁州:汶、嶓既藝,沱、涔既道,蔡、蒙旅平,和夷厎績。其土青驪。田下上,賦下中三錯。貢璆、鐵、銀、鏤、砮、磬,熊、羆、狐、貍、織皮。西傾因桓是來,浮于潛,踰于沔,入于渭,亂于河。

黑水西河惟雍州:弱水既西,涇屬渭汭。漆、沮既從,灃水所同。荊、岐已旅,終南、敦物至于鳥鼠。原隰厎績,至于都野。三危既度,三苗大序。其土黃壤。田上上,賦中下。貢璆、琳、瑯玕。浮于積石,至于龍門西河,會于渭汭。織皮昆侖、析支、渠搜,西戎即序。

道九山:汧及岐至于荊山,踰于河;壺口、雷首至于太嶽;砥柱、析城至于王屋;太行、常山至于碣石,入于海;西傾、朱圉、鳥鼠至于太華;熊耳、外方、桐柏至于負尾;道嶓冢,至于荊山;內方至于大別;汶山之陽至衡山,過九江,至于敷淺原。

道九川:弱水至於合黎,餘波入于流沙。道黑水,至于三危,入于南海。道河積石,至于龍門,南至華陰,東至砥柱,又東至于盟津,東過雒汭,至于大邳,北過降水,至于大陸,北播為九河,同為逆河,入于海。嶓冢道瀁,東流為漢,又東為蒼浪之水,過三澨,入于大別,南入于江,東匯澤為彭蠡,東為北江,入于海。汶山道江,東別為沱,又東至于醴,過九江,至于東陵,東迆北會于匯,東為中江,入于梅。道沇水,東為濟,入于河,泆為滎,東出陶丘北,又東至于荷,又東北會于汶,又東北入于海。道淮自桐柏,東會于泗、沂,東入于海。道渭自鳥鼠同穴,東會于灃,又東北至于涇,東過漆、沮,入于河。道雒自熊耳,東北會于澗、瀍,又東會于伊,東北入于河。

於是九州攸同,四奧既居,九山刊旅,九川滌原,九澤既陂,四海會同。六府甚修,眾土交正,致慎財賦,咸則三壤成賦。中國賜土姓:「祗臺德先,不距朕行。」

令天子之國以外五百里甸服:百里賦納總,二百里納銍,三百里納秸服,四百里粟,五百里米。甸服外五百里侯服:百里采,二百里任國,三百里諸侯。侯服外五百里綏服:三百里揆文教,二百里奮武衛。綏服外五百里要服:三百里夷,二百里蔡。要服外五百里荒服:三百里蠻,二百里流。

東漸于海,西被于流沙,朔、南暨:聲教訖于四海。於是帝錫禹玄圭,以告成功于天下。天下於是太平治。

皋陶作士以理民。帝舜朝,禹、伯夷、皋陶相與語帝前。皋陶述其謀曰:「信其道德,謀明輔和。」禹曰:「然,如何?」皋陶曰:「於!慎其身修,思長,敦序九族,眾明高翼,近可遠在已。」禹拜美言,曰:「然。」皋陶曰:「於!在知人,在安民。」禹曰:「吁!皆若是,惟帝其難之。知人則智,能官人;能安民則惠,黎民懷之。能知能惠,何憂乎驩兜,何遷乎有苗,何畏乎巧言善色佞人?」皋陶曰:「然,於!亦行有九德,亦言其有德。」乃言曰:「始事事,寬而栗,柔而立,願而共,治而敬,擾而毅,直而溫,簡而廉,剛而實,彊而義,章其有常,吉哉。日宣三德,蚤夜翊明有家。日嚴振敬六德,亮采有國。翕受普施,九德咸事,俊乂在官,百吏肅謹。毋教邪淫奇謀。非其人居其官,是謂亂天事。天討有罪,五刑五用哉。吾言厎可行乎?」禹曰:「女言致可績行。」皋陶曰:「余未有知,思贊道哉。」

帝舜謂禹曰:「女亦昌言。」禹拜曰;「於,予何言!予思日孳孳。」皋陶難禹曰:「何謂孳孳?」禹曰:「鴻水滔天,浩浩懷山襄陵,下民皆服於水。予陸行乘車,水行乘舟,泥行乘橇,山行乘檋,行山刊木。與益予眾庶稻鮮食。以決九川致四海,浚畎澮致之川。與稷予眾庶難得之食。食少,調有餘補不足,徙居。眾民乃定,萬國為治。」皋陶曰:「然,此而美也。」

禹曰:「於,帝!慎乃在位,安爾止。輔德,天下大應。清意以昭待上帝命,天其重命用休。」帝曰:「吁,臣哉,臣哉!臣作朕股肱耳目。予欲左右有民,女輔之。余欲觀古人之象。日月星辰,作文繡服色,女明之。予欲聞六律五聲八音,來始滑,以出入五言,女聽。予即辟,女匡拂予。女無面諛。退而謗予。敬四輔臣。諸眾讒嬖臣,君德誠施皆清矣。」禹曰:「然。帝即不時,布同善惡則毋功。」

帝曰:「毋若丹朱傲,維慢游是好,毋水行舟,朋淫于家,用絕其世。予不能順是。」禹曰:「予(辛壬)娶涂山,[辛壬]癸甲,生啟予不子,以故能成水土功。輔成五服,至于五千里,州十二師,外薄四海,咸建五長,各道有功。苗頑不即功,帝其念哉。」帝曰:「道吾德,乃女功序之也。」

皋陶於是敬禹之德,令民皆則禹。不如言,刑從之。舜德大明。

於是夔行樂,祖考至,群后相讓,鳥獸翔舞,簫韶九成,鳳皇來儀,百獸率舞,百官信諧。帝用此作歌曰:「陟天之命,維時維幾。」乃歌曰:「股肱喜哉,元首起哉,百工熙哉!」皋陶拜手稽首揚言曰:「念哉,率為興事,慎乃憲,敬哉!」乃更為歌曰:「元首明哉,股肱良哉,庶事康哉!」(舜)又歌曰:「元首叢脞哉,股肱惰哉,萬事墮哉!」帝拜曰:「然,往欽哉!」於是天下皆宗禹之明度數聲樂,為山川神主。

帝舜薦禹於天,為嗣。十七年而帝舜崩。三年喪畢,禹辭辟舜之子商均於陽城。天下諸侯皆去商均而朝禹。禹於是遂即天子位,南面朝天下,國號曰夏后,姓姒氏。

帝禹立而舉皋陶薦之,且授政焉,而皋陶卒。封皋陶之後於英、六,或在許。而后舉益,任之政。

十年,帝禹東巡狩,至于會稽而崩。以天下授益。三年之喪畢,益讓帝禹之子啟,而辟居箕山之陽。禹子啟賢,天下屬意焉。及禹崩,雖授益,益之佐禹日淺,天下未洽。故諸侯皆去益而朝啟,曰:「吾君帝禹之子也」。於是啟遂即天子之位,是為夏后帝啟。

夏后帝啟,禹之子,其母涂山氏之女也。

有扈氏不服,啟伐之,大戰於甘。將戰,作甘誓,乃召六卿申之。啟曰:「嗟!六事之人,予誓告女:有扈氏威侮五行,怠棄三正,天用勦絕其命。今予維共行天之罰。左不攻于左,右不攻于右,女不共命。御非其馬之政,女不共命。用命,賞于祖;不用命,僇于社,予則帑僇女。」遂滅有扈氏。天下咸朝。

夏后帝啟崩,子帝太康立。帝太康失國,昆弟五人,須于洛汭,作五子之歌。

太康崩,弟中康立,是為帝中康。帝中康時,羲、和湎淫,廢時亂日。胤往征之,作胤征。

中康崩,子帝相立。帝相崩,子帝少康立。帝少康崩,子帝予立。帝予崩,子帝槐立。帝槐崩,子帝芒立。帝芒崩,子帝泄立。帝泄崩,子帝不降立。帝不降崩,弟帝扃立。帝扃崩,子帝廑立。帝廑崩,立帝不降之子孔甲,是為帝孔甲。帝孔甲立,好方鬼神,事淫亂。夏后氏德衰,諸侯畔之。天降龍二,有雌雄,孔甲不能食,未得豢龍氏。陶唐既衰,其后有劉累,學擾龍于豢龍氏,以事孔甲。孔甲賜之姓曰御龍氏,受豕韋之後。龍一雌死,以食夏后。夏后使求,懼而遷去。

孔甲崩,子帝皋立。帝皋崩,子帝發立。帝發崩,子帝履癸立,是為桀。帝桀之時,自孔甲以來而諸侯多畔夏,桀不務德而武傷百姓,百姓弗堪。乃召湯而囚之夏臺,已而釋之。湯修德,諸侯皆歸湯,湯遂率兵以伐夏桀。桀走鳴條,遂放而死。桀謂人曰:「吾悔不遂殺湯於夏臺,使至此。」湯乃踐天子位,代夏朝天下。湯封夏之後,至周封於杞也。

太史公曰:禹為姒姓,其後分封,用國為姓,故有夏后氏、有扈氏、有男氏、斟尋氏、彤城氏、褒氏、費氏、杞氏、繒氏、辛氏、冥氏、斟(氏)戈氏。孔子正夏時,學者多傳夏小正云。自虞、夏時,貢賦備矣。或言禹會諸侯江南,計功而崩,因葬焉,命曰會稽。會稽者,會計也。


\end{pinyinscope}