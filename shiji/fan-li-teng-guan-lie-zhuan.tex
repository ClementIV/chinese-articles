\article{樊酈滕灌列傳}

\begin{pinyinscope}
舞陽侯樊噲者,沛人也。以屠狗為事,與高祖俱隱。

初從高祖起豐,攻下沛。高祖為沛公,以噲為舍人。從攻胡陵、方與,還守豐,擊泗水監豐下,破之。復東定沛,破泗水守薛西。與司馬夷戰碭東,卻敵,斬首十五級,賜爵國大夫。常從,沛公擊章邯軍濮陽,攻城先登,斬首二十三級,賜爵列大夫。復常從,從攻城陽,先登。下戶牖,破李由軍,斬首十六級,賜上閒爵。從攻圍東郡守尉於成武,卻敵,斬首十四級,捕虜十一人,賜爵五大夫。從擊秦軍,出亳南。河閒守軍於杠里,破之。擊破趙賁軍開封北,以卻敵先登,斬候一人,首六十八級,捕虜二十七人,賜爵卿。從攻破楊熊軍於曲遇。攻宛陵,先登,斬首八級,捕虜四十四人,賜爵封號賢成君。從攻長社、轘轅,絕河津,東攻秦軍於尸,南攻秦軍於犨。破南陽守齮於陽城。東攻宛城,先登。西至酈,以卻敵,斬首二十四級,捕虜四十人,賜重封。攻武關,至霸上,斬都尉一人,首十級,捕虜百四十六人,降卒二千九百人。

項羽在戲下,欲攻沛公。沛公從百餘騎因項伯面見項羽,謝無有閉關事。項羽既饗軍士,中酒,亞父謀欲殺沛公,令項莊拔劍舞坐中,欲擊沛公,項伯常(肩)[屏]蔽之。時獨沛公與張良得入坐,樊噌在營外,聞事急,乃持鐵盾入到營。營衛止噲,噲直撞入,立帳下。項羽目之,問為誰。張良曰:「沛公參乘樊噲。」項羽曰:「壯士。」賜之卮酒彘肩。噲既飲酒,拔劍切肉食,盡之。項羽曰:「能復飲乎?」噲曰:「臣死且不辭,豈特卮酒乎!且沛公先入定咸陽,暴師霸上,以待大王。大王今日至,聽小人之言,與沛公有隙,臣恐天下解,心疑大王也。」項羽默然。沛公如廁,麾樊噲去。既出,沛公留車騎,獨騎一馬,與樊噲等四人步從,從閒道山下歸走霸上軍,而使張良謝項羽。項羽亦因遂已,無誅沛公之心矣。是日微樊噲奔入營譙讓項羽,沛公事幾殆。

明日,項羽入屠咸陽,立沛公為漢王。漢王賜噲爵為列侯,號臨武侯。遷為郎中,從入漢中。

還定三秦,別擊西丞白水北,雍輕車騎於雍南,破之。從攻雍、斄城,先登。擊章平軍好畤,攻城,先登陷陣,斬縣令丞各一人,首十一級,虜二十人,遷郎中騎將。從擊秦車騎壤東,卻敵,遷為將軍。攻趙賁,下郿、槐裏、柳中、咸陽;灌廢丘,最。至櫟陽,賜食邑杜之樊鄉。從攻項籍,屠煮棗。擊破王武、程處軍於外黃。攻鄒、魯、瑕丘、薛。項羽敗漢王於彭城,盡復取魯、梁地。噲還至滎陽,益食平陰二千戶,以將軍守廣武。一歲,項羽引而東。從高祖擊項籍,下陽夏,虜楚周將軍卒四千人。圍項籍於陳,大破之。屠胡陵。

項籍既死,漢王為帝,以噲堅守戰有功,益食八百戶。從高帝攻反燕王臧荼,虜荼,定燕地。楚王韓信反,噲從至陳,取信,定楚。更賜爵列侯,與諸侯剖符,世世勿絕,食舞陽,號為舞陽侯,除前所食。以將軍從高祖攻反韓王信於代。自霍人以往至雲中,與絳侯等共定之,益食千五百戶。因擊陳豨與曼丘臣軍,戰襄國,破柏人,先登,降定清河、常山凡二十七縣,殘東垣,遷為左丞相。破得綦毋卹、尹潘軍於無終、廣昌。破豨別將胡人王黃軍於代南,因擊韓信軍於參合。軍所將卒斬韓信,破豨胡騎橫谷,斬將軍趙既,虜代丞相馮梁、守孫奮、大將王黃、將軍、(太卜)太仆解福等十人。與諸將共定代鄉邑七十三。其後燕王盧綰反,噲以相國擊盧綰,破其丞相抵薊南,定燕地,凡縣十八,鄉邑五十一。益食邑千三百戶,定食舞陽五千四百戶。從,斬首百七十六級,虜二百八十八人。別,破軍七,下城五,定郡六,縣五十二,得丞相一人,將軍十二人,二千石已下至三百石十一人。

噲以呂后女弟呂須為婦,生子伉,故其比諸將最親。

先黥布反時,高祖嘗病甚,惡見人,臥禁中,詔戶者無得入群臣。群臣絳、灌等莫敢入。十餘日,噲乃排闥直入,大臣隨之。上獨枕一宦者臥。噲等見上流涕曰:「始陛下與臣等起豐沛,定天下,何其壯也!今天下已定,又何憊也!且陛下病甚,大臣震恐,不見臣等計事,顧獨與一宦者絕乎?且陛下獨不見趙高之事乎?」高帝笑而起。

其後盧綰反,高帝使噲以相國擊燕。是時高帝病甚,人有惡噲黨於呂氏,即上一日宮車晏駕,則噲欲以兵盡誅滅戚氏、趙王如意之屬。高帝聞之大怒,乃使陳平載絳侯代將,而即軍中斬噲。陳平畏呂后,執噲詣長安。至則高祖已崩,呂后釋噲,使復爵邑。

孝惠六年,樊噲卒,謚為武侯。子伉代侯。而伉母呂須亦為臨光侯,高后時用事專權,大臣盡畏之。伉代侯九歲,高后崩。大臣誅諸呂、呂須婘屬,因誅伉。舞陽侯中絕數月。孝文帝既立,乃復封噲他庶子市人為舞陽侯,復故爵邑。市人立二十九歲卒,謚為荒侯。子他廣代侯。六歲,侯家舍人得罪他廣,怨之,乃上書曰:「荒侯市人病不能為人,令其夫人與其弟亂而生他廣,他廣實非荒侯子,不當代後。」詔下吏。孝景中六年,他廣奪侯為庶人,國除。

曲周侯酈商者,高陽人。陳勝起時,商聚少年東西略人,得數千。沛公略地至陳留,六月餘,商以將卒四千人屬沛公於岐。從攻長社,先登,賜爵封信成君。從沛公攻緱氏,絕河津,破秦軍洛陽東。從攻下宛、穰,定十七縣。別將攻旬關,定漢中。

項羽滅秦,立沛公為漢王。漢王賜商爵信成君,以將軍為隴西都尉。別將定北地、上郡。破雍將軍焉氏,周類軍栒邑,蘇駔軍於泥陽。賜食邑武成六千戶。以隴西都尉從擊項籍軍五月,出鉅野,與鐘離眛戰,疾鬬,受梁相國印,益食邑四千戶。以梁相國將從擊項羽二歲三月,攻胡陵。

項羽既已死,漢王為帝。其秋,燕王臧荼反,商以將軍從擊荼,戰龍脫,先登陷陣,破荼軍易下,卻敵,遷為右丞相,賜爵列侯,與諸侯剖符,世世勿絕,食邑涿五千戶,號曰涿侯。以右丞相別定上谷,因攻代,受趙相國印。以右丞相趙相國別與絳侯等定代、鴈門,得代丞相程縱、守相郭同、將軍已下至六百石十九人。還,以將軍為太上皇衛一歲七月。以右丞相擊陳豨,殘東垣。又以右丞相從高帝擊黥布,攻其前拒,陷兩陳,得以破布軍,更食曲周五千一百戶,除前所食,凡別破軍三,降定郡六,縣七十三,得丞相、守相、大將各一人,小將二人,二千石已下至六百石十九人。

商事孝惠、高后時,商病,不治。其子寄,字況,與呂祿善。及高后崩,大臣欲誅諸呂,呂祿為將軍,軍於北軍,太尉勃不得入北軍,於是乃使人劫酈商,令其子況紿呂祿,呂祿信之,故與出游,而太尉勃乃得入據北軍,遂誅諸呂。是歲商卒,謚為景侯。子寄代侯。天下稱酈況賣交也。

孝景前三年,吳、楚、齊、趙反,上以寄為將軍,圍趙城,十月不能下。得俞侯欒布自平齊來,乃下趙城,滅趙,王自殺,除國。孝景中二年,寄欲取平原君為夫人,景帝怒,下寄吏,有罪,奪侯。景帝乃以商他子堅封為繆侯,續酈氏後。繆靖侯卒,子康侯遂成立。遂成卒,子懷侯世宗立。世宗卒,子侯終根立,為太常,坐法,國除。

汝陰侯夏侯嬰,沛人也。為沛廄司御。每送使客還,過沛泗上亭,與高祖語,未嘗不移日也。嬰已而試補縣吏,與高祖相愛。高祖戲而傷嬰,人有告高祖。高祖時為亭長,重坐傷人,告故不傷嬰,嬰證之。後獄覆,嬰坐高祖系歲餘,掠笞數百,終以是脫高祖。

高祖之初與徒屬欲攻沛也,嬰時以縣令史為高祖使。上降沛一日,高祖為沛公,賜嬰爵七大夫,以為太仆。從攻胡陵,嬰與蕭何降泗水監平,平以胡陵降,賜嬰爵五大夫。從擊秦軍碭東,攻濟陽,下戶牖,破李由軍雍丘下,以兵車趣攻戰疾,賜爵執帛。常以太仆奉車從擊章邯軍東阿、濮陽下,以兵車趣攻戰疾,破之,賜爵執珪。復常奉車從擊趙賁軍開封,楊熊軍曲遇。嬰從捕虜六十八人,降卒八百五十人,得印一匱。因復常奉車從擊秦軍雒陽東,以兵車趣攻戰疾,賜爵封轉為滕公。因復奉車從攻南陽,戰於藍田、芷陽,以兵車趣攻戰疾,至霸上。項羽至,滅秦,立沛公為漢王。漢王賜嬰爵列侯,號昭平侯,復為太仆,從入蜀、漢。

還定三秦,從擊項籍。至彭城,項羽大破漢軍。漢王敗,不利,馳去。見孝惠、魯元,載之。漢王急,馬罷,虜在後,常蹶兩兒欲棄之,嬰常收,竟載之,徐行面雍樹乃馳。漢王怒,行欲斬嬰者十餘,卒得脫,而致孝惠、魯元於豐。

漢王既至滎陽,收散兵,復振,賜嬰食祈陽。復常奉車從擊項籍,追至陳,卒定楚,至魯,益食茲氏。

漢王立為帝。其秋,燕王臧荼反,嬰以太仆從擊荼。明年,從至陳,取楚王信。更食汝陰,剖符世世勿絕。以太仆從擊代,至武泉、雲中,益食千戶。因從擊韓信軍胡騎晉陽旁,大破之。追北至平城,為胡所圍,七日不得通。高帝使使厚遺閼氏,冒頓開圍一角。高帝出欲馳,嬰固徐行,弩皆持滿外向,卒得脫。益食嬰細陽千戶。復以太仆從擊胡騎句注北,大破之。以太仆擊胡騎平城南,三陷陳,功為多,賜所奪邑五百戶。以太仆擊陳豨、黥布軍,陷陳卻敵,益食千戶,定食汝陰六千九百戶,除前所食。

嬰自上初起沛,常為太仆,竟高祖崩。以太仆事孝惠。孝惠帝及高后德嬰之脫孝惠、魯元於下邑之閒也,乃賜嬰縣北第第一,曰「近我」,以尊異之。孝惠帝崩,以太仆事高后。高后崩,代王之來,嬰以太仆與東牟侯入清宮,廢少帝,以天子法駕迎代王代邸,與大臣共立為孝文皇帝,復為太仆。八歲卒,謚為文侯。子夷侯灶立,七年卒。子共侯賜立,三十一年卒。子侯頗尚平陽公主。立十九歲,元鼎二年,坐與父御婢姦罪,自殺,國除。

潁陰侯灌嬰者,睢陽販繒者也。高祖之為沛公,略地至雍丘下,章邯敗殺項梁,而沛公還軍於碭,嬰初以中涓從擊破東郡尉於成武及秦軍於扛裏,疾鬬,賜爵七大夫。從攻秦軍亳南、開封、曲遇,戰疾力,賜爵執帛,號宣陵君。從攻陽武以西至雒陽,破秦軍尸北,北絕河津,南破南陽守齮陽城東,遂定南陽郡。西入武關,戰於藍田,疾力,至霸上,賜爵執珪,號昌文君。

沛公立為漢王,拜嬰為郎中,從入漢中,十月,拜為中謁者。從還定三秦,下櫟陽,降塞王。還圍章邯於廢丘,未拔。從東出臨晉關,擊降殷王,定其地。擊項羽將龍且、魏相項他軍定陶南,疾戰,破之。賜嬰爵列侯,號昌文侯,食杜平鄉。

復以中謁者從降下碭,以至彭城。項羽擊,大破漢王。漢王遁而西,嬰從還,軍於雍丘。王武、魏公申徒反,從擊破之。攻下黃,西收兵,軍於滎陽。楚騎來眾,漢王乃擇軍中可為(車)騎將者,皆推故秦騎士重泉人李必、駱甲習騎兵,今為校尉,可為騎將。漢王欲拜之,必、甲曰:「臣故秦民,恐軍不信臣,臣願得大王左右善騎者傅之。」灌嬰雖少,然數力戰,乃拜灌嬰為中大夫,令李必、駱甲為左右校尉,將郎中騎兵擊楚騎於滎陽東,大破之。受詔別擊楚軍後,絕其餉道,起陽武至襄邑。擊項羽之將項冠於魯下,破之,所將卒斬右司馬、騎將各一人。擊破柘公王武,軍於燕西,所將卒斬樓煩將五人,連尹一人。擊王武別將桓嬰白馬下,破之,所將卒斬都尉一人。以騎渡河南,送漢王到雒陽,使北迎相國韓信軍於邯鄲。還至敖倉,嬰遷為御史大夫。

三年,以列侯食邑杜平鄉。以御史大夫受詔將郎中騎兵東屬相國韓信,擊破齊軍於歷下,所將卒虜車騎將軍華毋傷及將吏四十六人。降下臨菑,得齊守相田光。追齊相田橫至嬴、博,破其騎,所將卒斬騎將一人,生得騎將四人。攻下嬴、博,破齊將軍田吸於千乘,所將卒斬吸。東從韓信攻龍且、留公旋於高密,卒斬龍且,生得右司馬、連尹各一人,樓煩將十人,身生得亞將周蘭。

齊地已定,韓信自立為齊王,使嬰別將擊楚將公杲於魯北,破之。轉南,破薛郡長,身虜騎將一人。攻(博)[傅]陽,前至下相以東南僮、取慮、徐。度淮,盡降其城邑,至廣陵。項羽使項聲、薛公、郯公復定淮北。嬰度淮北,擊破項聲、郯公下邳,斬薛公,下下邳,擊破楚騎於平陽,遂降彭城,虜柱國項佗,降留、薛、沛、酂、蕭、相。攻苦、譙,復得亞將周蘭。與漢王會頤鄉。從擊項籍軍於陳下,破之,所將卒斬樓煩將二人,虜騎將八人。賜益食邑二千五百戶。

項籍敗垓下去也,嬰以御史大夫受詔將車騎別追項籍至東城,破之。所將卒五人共斬項籍,皆賜爵列侯。降左右司馬各一人,卒萬二千人,盡得其軍將吏。下東城、歷陽。渡江,破吳郡長吳下,得吳守,遂定吳、豫章、會稽郡。還定淮北,凡五十二縣。

漢王立為皇帝,賜益嬰邑三千戶。其秋,以車騎將軍從擊破燕王臧荼。明年,從至陳,取楚王信。還,剖符,世世勿絕,食潁陰二千五百戶,號曰潁陰侯。

以車騎將軍從擊反韓王信於代,至馬邑,受詔別降樓煩以北六縣,斬代左相,破胡騎於武泉北。復從擊韓信胡騎晉陽下,所將卒斬胡白題將一人。受詔并將燕、趙、齊、梁、楚車騎,擊破胡騎於硰石。至平城,為胡所圍,從還軍東垣。

從擊陳豨,受詔別攻豨丞相侯敞軍曲逆下,破之,卒斬敞及特將五人。降曲逆、盧奴、上曲陽、安國、安平。攻下東垣。

黥布反,以車騎將軍先出,攻布別將於相,破之,斬亞將樓煩將三人。又進擊破布上柱國軍及大司馬軍。又進破布別將肥誅。嬰身生得左司馬一人,所將卒斬其小將十人,追北至淮上。益食二千五百戶。布已破,高帝歸,定令嬰食穎陰五千戶,除前所食邑。凡從得二千石二人,別破軍十六,降城四十六,定國一,郡二,縣五十二,得將軍二人,柱國、相國各一人,二千石十人。

嬰自破布歸,高帝崩,嬰以列侯事孝惠帝及呂太后。太后崩,呂祿等以趙王自置為將軍,軍長安,為亂。齊哀王聞之,舉兵西,且入誅不當為王者。上將軍呂祿等聞之,乃遣嬰為大將,將軍往擊之。嬰行至滎陽,乃與絳侯等謀,因屯兵滎陽,風齊王以誅呂氏事,齊兵止不前。絳侯等既誅諸呂,齊王罷兵歸,嬰亦罷兵自滎陽歸,與絳侯、陳平共立代王為孝文皇帝。孝文皇帝於是益封嬰三千戶,賜黃金千斤,拜為太尉。

三歲,絳侯勃免相就國,嬰為丞相,罷太尉官。是歲,匈奴大入北地、上郡,令丞相嬰將騎八萬五千往擊匈奴。匈奴去,濟北王反,詔乃罷嬰之兵。後歲餘,嬰以丞相卒,謚曰懿侯。子平侯阿代侯。二十八年卒,子彊代侯。十三年,彊有罪,絕二歲。元光三年,天子封灌嬰孫賢為臨汝侯,續灌氏後,八歲,坐行賕有罪,國除。

太史公曰:吾適豐沛,問其遺老,觀故蕭、曹、樊噲、滕公之家,及其素,異哉所聞!方其鼓刀屠狗賣繒之時,豈自知附驥之尾,垂名漢廷,德流子孫哉?余與他廣通,為言高祖功臣之興時若此云。


\end{pinyinscope}