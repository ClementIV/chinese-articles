\article{酷吏列傳}

\begin{pinyinscope}
孔子曰:「導之以政,齊之以刑,民免而無恥。導之以德,齊之以禮,有恥且格。」老氏稱:「上德不德,是以有德;下德不失德,是以無德。法令滋章,盜賊多有。」太史公曰:信哉是言也!法令者治之具,而非制治清濁之源也。昔天下之網嘗密矣,然姦偽萌起,其極也,上下相遁,至於不振。當是之時,吏治若救火揚沸,非武健嚴酷,惡能勝其任而愉快乎!言道德者,溺其職矣。故曰「聽訟,吾猶人也,必也使無訟乎」。「下士聞道大笑之」。非虛言也。漢興,破觚而為圜,斲雕而為樸,網漏於吞舟之魚,而吏治烝烝,不至於姦,黎民艾安。由是觀之,在彼不在此。

高后時,酷吏獨有侯封,刻轢宗室,侵辱功臣。呂氏已敗,遂(禽)[夷]侯封之家。孝景時,晁錯以刻深頗用術輔其資,而七國之亂,發怒於錯,錯卒以被戮。其后有郅都、寧成之屬。

郅都者,楊人也。以郎事孝文帝。孝景時,都為中郎將,敢直諫,面折大臣於朝。嘗從入上林,賈姬如廁,野彘卒入廁。上目都,都不行。上欲自持兵救賈姬,都伏上前曰:「亡一姬復一姬進,天下所少寧賈姬等乎?陛下縱自輕,柰宗廟太后何!」上還,彘亦去。太后聞之,賜都金百斤,由此重郅都。

濟南瞯氏宗人三百餘家,豪猾,二千石莫能制,於是景帝乃拜都為濟南太守。至則族瞯氏首惡,餘皆股栗。居歲餘,郡中不拾遺。旁十餘郡守畏都如大府。

都為人勇,有氣力,公廉,不發私書,問遺無所受,請寄無所聽。常自稱曰:「已倍親而仕,身固當奉職死節官下,終不顧妻子矣。」

郅都遷為中尉。丞相條侯至貴倨也,而都揖丞相。是時民樸,畏罪自重,而都獨先嚴酷,致行法不避貴戚,列侯宗室見都側目而視,號曰「蒼鷹」。

臨江王徵詣中尉府對簿,臨江王欲得刀筆為書謝上,而都禁吏不予。魏其侯使人以閒與臨江王。臨江王既為書謝上,因自殺。竇太后聞之,怒,以危法中都,都免歸家。孝景帝乃使使持節拜都為鴈門太守,而便道之官,得以便宜從事。匈奴素聞郅都節,居邊,為引兵去,竟郅都死不近鴈門。匈奴至為偶人象郅都,令騎馳射莫能中,見憚如此。匈奴患之。竇太后乃竟中都以漢法。景帝曰:「都忠臣。」欲釋之。竇太后曰:「臨江王獨非忠臣邪?」於是遂斬郅都。

寧成者,穰人也。以郎謁者事景帝。好氣,為人小吏,必陵其長吏;為人上,操下如束溼薪。滑賊任威。稍遷至濟南都尉,而郅都為守。始前數都尉皆步入府,因吏謁守如縣令,其畏郅都如此。及成往,直陵都出其上。都素聞其聲,於是善遇,與結驩。久之,郅都死,后長安左右宗室多暴犯法,於是上召寧成為中尉。其治效郅都,其廉弗如,然宗室豪桀皆人人惴恐。

武帝即位,徙為內史。外戚多毀成之短,抵罪髡鉗。是時九卿罪死即死,少被刑,而成極刑,自以為不復收,於是解脫,詐刻傳出關歸家。稱曰:「仕不至二千石,賈不至千萬,安可比人乎!」乃貰貸買陂田千餘頃,假貧民,役使數千家。數年,會赦。致產數千金,為任俠,持吏長短,出從數十騎。其使民威重於郡守。

周陽由者,其父趙兼以淮南王舅父侯周陽,故因姓周陽氏。由以宗家任為郎,事孝文及景帝。景帝時,由為郡守。武帝即位,吏治尚循謹甚,然由居二千石中,最為暴酷驕恣。所愛者,撓法活之;所憎者,曲法誅滅之。所居郡,必夷其豪。為守,視都尉如令。為都尉,必陵太守,奪之治。與汲黯俱為忮,司馬安之文惡,俱在二千石列,同車未嘗敢均茵伏。

由后為河東都尉,時與其守勝屠公爭權,相告言罪。勝屠公當抵罪,義不受刑,自殺,而由棄市。

自寧成、周陽由之后,事益多,民巧法,大抵吏之治類多成、由等矣。

趙禹者,斄人。以佐史補中都官,用廉為令史,事太尉亞夫。亞夫為丞相,禹為丞相史,府中皆稱其廉平。然亞夫弗任,曰:「極知禹無害,然文深,不可以居大府。」今上時,禹以刀筆吏積勞,稍遷為御史。上以為能,至太中大夫。與張湯論定諸律令,作見知,吏傳得相監司。用法益刻,蓋自此始。

張湯者,杜人也。其父為長安丞,出,湯為兒守舍。還而鼠盜肉,其父怒,笞湯。湯掘窟得盜鼠及餘肉,劾鼠掠治,傳爰書,訊鞫論報,并取鼠與肉,具獄磔堂下。其父見之,視其文辭如老獄吏,大驚,遂使書獄。父死后,湯為長安吏,久之。

周陽侯始為諸卿時,嘗系長安,湯傾身為之。及出為侯,大與湯交,遍見湯貴人。湯給事內史,為寧成掾,以湯為無害,言大府,調為茂陵尉,治方中。

武安侯為丞相,徵湯為史,時薦言之天子,補御史,使案事。治陳皇后蠱獄,深竟黨與。於是上以為能,稍遷至太中大夫。與趙禹共定諸律令,務在深文,拘守職之吏。已而趙禹遷為中尉,徙為少府,而張湯為廷尉,兩人交驩,而兄事禹。禹為人廉倨。為吏以來,舍毋食客。公卿相造請禹,禹終不報謝,務在絕知友賓客之請,孤立行一意而已。見文法輒取,亦不覆案,求官屬陰罪。湯為人多詐,舞智以御人。始為小吏,乾沒,與長安富賈田甲、魚翁叔之屬交私。及列九卿,收接天下名士大夫,己心內雖不合,然陽浮慕之。

是時上方鄉文學,湯決大獄,欲傅古義,乃請博士弟子治尚書、春秋補廷尉史,亭疑法。奏讞疑事,必豫先為上分別其原,上所是,受而著讞決法廷尉絜令,揚主之明。奏事即譴,湯應謝,鄉上意所便,必引正、監、掾史賢者,曰:「固為臣議,如上責臣,臣弗用,愚抵於此。」罪常釋。(聞)[閒]即奏事,上善之,曰:「臣非知為此奏,乃正、監、掾史某為之。」其欲薦吏,揚人之善蔽人之過如此。所治即上意所欲罪,予監史深禍者;即上意所欲釋,與監史輕平者。所治即豪,必舞文巧詆;即下戶羸弱,時口言,雖文致法,上財察。於是往往釋湯所言。湯至於大吏,內行修也。通賓客飲食。於故人子弟為吏及貧昆弟,調護之尤厚。其造請諸公,不避寒暑。是以湯雖文深意忌不專平,然得此聲譽。而刻深吏多為爪牙用者,依於文學之士。丞相弘數稱其美。及治淮南、衡山、江都反獄,皆窮根本。嚴助及伍被,上欲釋之。湯爭曰:「伍被本畫反謀,而助親幸出入禁闥爪牙臣,乃交私諸侯如此,弗誅,後不可治。」於是上可論之。其治獄所排大臣自為功,多此類。於是湯益尊任,遷為御史大夫。

會渾邪等降,漢大興兵伐匈奴,山東水旱,貧民流徙,皆仰給縣官,縣官空虛。於是丞上指,請造白金及五銖錢,籠天下鹽鐵,排富商大賈,出告緡令,鉏豪彊并兼之家,舞文巧詆以輔法。湯每朝奏事,語國家用,日晏,天子忘食。丞相取充位,天下事皆決於湯。百姓不安其生,騷動,縣官所興,未獲其利,姦吏并侵漁,於是痛繩以罪。則自公卿以下,至於庶人,咸指湯。湯嘗病,天子至自視病,其隆貴如此。

匈奴來請和親,群臣議上前。博士狄山曰:「和親便。」上問其便,山曰:「兵者凶器,未易數動。高帝欲伐匈奴,大困平城,乃遂結和親。孝惠、高后時,天下安樂。及孝文帝欲事匈奴,北邊蕭然苦兵矣。孝景時,吳楚七國反,景帝往來兩宮閒,寒心者數月。吳楚已破,竟景帝不言兵,天下富實。今自陛下舉兵擊匈奴,中國以空虛,邊民大困貧。由此觀之,不如和親。」上問湯,湯曰:「此愚儒,無知。」狄山曰:「臣固愚忠,若御史大夫湯乃詐忠。若湯之治淮南、江都,以深文痛詆諸侯,別疏骨肉,使蕃臣不自安。臣固知湯之為詐忠。」於是上作色曰:「吾使生居一郡,能無使虜入盜乎?」曰:「不能。」曰:「居一縣?」對曰:「不能。」復曰:「居一障閒?」山自度辯窮且下吏,曰:「能。」於是上遣山乘鄣。至月餘,匈奴斬山頭而去。自是以後,群臣震慴。

湯之客田甲,雖賈人,有賢操。始湯為小吏時,與錢通,及湯為大吏,甲所以責湯行義過失,亦有烈士風。

湯為御史大夫七歲,敗。

河東人李文嘗與湯有卻,已而為御史中丞,恚,數從中文書事有可以傷湯者,不能為地。湯有所愛史魯謁居,知湯不平,使人上蜚變告文姦事,事下湯,湯治論殺文,而湯心知謁居為之。上問曰:「言變事縱跡安起?」湯詳驚曰:「此殆文故人怨之。」謁居病臥閭里主人,湯自往視疾,為謁居摩足。趙國以冶鑄為業,王數訟鐵官事,湯常排趙王。趙王求湯陰事。謁居嘗案趙王,趙王怨之,并上書告:「湯,大臣也,史謁居有病,湯至為摩足,疑與為大姦。」事下廷尉。謁居病死,事連其弟,弟系導官。湯亦治他囚導官,見謁居弟,欲陰為之,而詳不省。謁居弟弗知,怨湯,使人上書告湯與謁居謀,共變告李文。事下減宣。宣嘗與湯有卻,及得此事,窮竟其事,未奏也。會人有盜發孝文園瘞錢,丞相青翟朝,與湯約俱謝,至前,湯念獨丞相以四時行園,當謝,湯無與也,不謝。丞相謝,上使御史案其事。湯欲致其文丞相見知,丞相患之。三長史皆害湯,欲陷之。

始長史朱買臣,會稽人也。讀春秋。莊助使人言買臣,買臣以楚辭與助俱幸,侍中,為太中大夫,用事;而湯乃為小吏,跪伏使買臣等前。已而湯為廷尉,治淮南獄,排擠莊助,買臣固心望。及湯為御史大夫,買臣以會稽守為主爵都尉,列於九卿。數年,坐法廢,守長史,見湯,湯坐床上,丞史遇買臣弗為禮。買臣楚士,深怨,常欲死之。王朝,齊人也。以術至右內史。邊通,學長短,剛暴彊人也,官再至濟南相。故皆居湯右,已而失官,守長史,詘體於湯。湯數行丞相事,知此三長史素貴,常淩折之。以故三長史合謀曰:「始湯約與君謝,已而賣君;今欲劾君以宗廟事,此欲代君耳。吾知湯陰事。」使吏捕案湯左田信等,曰湯且欲奏請,信輒先知之,居物致富,與湯分之,及他姦事。事辭頗聞。上問湯曰:「吾所為,賈人輒先知之,益居其物,是類有以吾謀告之者。」湯不謝。湯又詳驚曰:「固宜有。」減宣亦奏謁居等事。天子果以湯懷詐面欺,使使八輩簿責湯。湯具自道無此,不服。於是上使趙禹責湯。禹至,讓湯曰:「君何不知分也。君所治夷滅者幾何人矣?今人言君皆有狀,天子重致君獄,欲令君自為計,何多以對簿為?」湯乃為書謝曰:「湯無尺寸功,起刀筆吏,陛下幸致為三公,無以塞責。然謀陷湯罪者,三長史也。」遂自殺。

湯死,家產直不過五百金,皆所得奉賜,無他業。昆弟諸子欲厚葬湯,湯母曰:「湯為天子大臣,被汙惡言而死,何厚葬乎!」載以牛車,有棺無槨。天子聞之,曰:「非此母不能生此子。」乃盡案誅三長史。丞相青翟自殺。出田信。上惜湯。稍遷其子安世。

趙禹中廢,已而為廷尉。始條侯以為禹賊深,弗任。及禹為少府,比九卿。禹酷急,至晚節,事益多,吏務為嚴峻,而禹治加緩,而名為平。王溫舒等后起,治酷於禹。禹以老,徙為燕相。數歲,亂悖有罪,免歸。後湯十餘年,以壽卒于家。

義縱者,河東人也。為少年時,嘗與張次公俱攻剽為群盜。縱有姊姁,以醫幸王太后。王太后問:「有子兄弟為官者乎?」姊曰:「有弟無行,不可。」太后乃告上,拜義姁弟縱為中郎,補上黨郡中令。治敢行,少蘊藉,縣無逋事,舉為第一。遷為長陵及長安令,直法行治,不避貴戚。以捕案太后外孫修成君子仲,上以為能,遷為河內都尉。至則族滅其豪穰氏之屬,河內道不拾遺。而張次公亦為郎,以勇悍從軍,敢深入,有功,為岸頭侯。

寧成家居,上欲以為郡守。御史大夫弘曰:「臣居山東為小吏時,寧成為濟南都尉,其治如狼牧羊。成不可使治民。」上乃拜成為關都尉。歲餘,關東吏隸郡國出入關者,號曰「寧見乳虎,無值寧成之怒」。義縱自河內遷為南陽太守,聞寧成家居南陽,及縱至關,寧成側行送迎,然縱氣盛,弗為禮。至郡,遂案寧氏,盡破碎其家。成坐有罪,及孔、暴之屬皆奔亡,南陽吏民重足一跡。而平氏朱彊、杜衍、杜周為縱牙爪之吏,任用,遷為廷史。軍數出定襄,定襄吏民亂敗,於是徙縱為定襄太守。縱至,掩定襄獄中重罪輕系二百餘人,及賓客昆弟私入相視亦二百餘人。縱一捕鞠,曰「為死罪解脫」。是日皆報殺四百餘人。其后郡中不寒而栗,猾民佐吏為治。

是時趙禹、張湯以深刻為九卿矣,然其治尚寬,輔法而行,而縱以鷹擊毛摯為治。后會五銖錢白金起,民為姦,京師尤甚,乃以縱為右內史,王溫舒為中尉。溫舒至惡,其所為不先言縱,縱必以氣淩之,敗壞其功。其治,所誅殺甚多,然取為小治,姦益不勝,直指始出矣。吏之治以斬殺縛束為務,閻奉以惡用矣。縱廉,其治放郅都。上幸鼎湖,病久,已而卒起幸甘泉,道多不治。上怒曰:「縱以我為不復行此道乎?」嗛之。至冬,楊可方受告緡,縱以為此亂民,部吏捕其為可使者。天子聞,使杜式治,以為廢格沮事,棄縱市。後一歲,張湯亦死。

王溫舒者,陽陵人也。少時椎埋為姦。已而試補縣亭長,數廢。為吏,以治獄至廷史。事張湯,遷為御史。督盜賊,殺傷甚多,稍遷至廣平都尉。擇郡中豪敢任吏十餘人,以為爪牙,皆把其陰重罪,而縱使督盜賊,快其意所欲得。此人雖有百罪,弗法;即有避,因其事夷之,亦滅宗。以其故齊趙之郊盜賊不敢近廣平,廣平聲為道不拾遺。上聞,遷為河內太守。

素居廣平時,皆知河內豪姦之家,及往,九月而至。令郡具私馬五十匹,為驛自河內至長安,部吏如居廣平時方略,捕郡中豪猾,郡中豪猾相連坐千餘家。上書請,大者至族,小者乃死,家盡沒入償臧。奏行不過二三日,得可事。論報,至流血十餘里。河內皆怪其奏,以為神速。盡十二月,郡中毋聲,毋敢夜行,野無犬吠之盜。其頗不得,失之旁郡國,黎來,會春,溫舒頓足嘆曰:「嗟乎,令冬月益展一月,足吾事矣!」其好殺伐行威不愛人如此。天子聞之,以為能,遷為中尉。其治復放河內,徙諸名禍猾吏與從事,河內則楊皆、麻戊,關中楊贛、成信等。義縱為內史,憚未敢恣治。及縱死,張湯敗後,徙為廷尉,而尹齊為中尉。

尹齊者,東郡茌平人。以刀筆稍遷至御史。事張湯,張湯數稱以為廉武,使督盜賊,所斬伐不避貴戚。遷為關內都尉,聲甚於寧成。上以為能,遷為中尉,吏民益凋敝。尹齊木彊少文,豪惡吏伏匿而善吏不能為治,以故事多廢,抵罪。上復徙溫舒為中尉,而楊仆以嚴酷為主爵都尉。

楊仆者,宜陽人也。以千夫為吏。河南守案舉以為能,遷為御史,使督盜賊關東。治放尹齊,以為敢摯行。稍遷至主爵都尉,列九卿。天子以為能。南越反,拜為樓船將軍,有功,封將梁侯。為荀彘所縛。居久之,病死。

而溫舒復為中尉。為人少文,居廷惛惛不辯,至於中尉則心開。督盜賊,素習關中俗,知豪惡吏,豪惡吏盡復為用,為方略。吏苛察,盜賊惡少年投缿購告言姦,置伯格長以牧司姦盜賊。溫舒為人讇善事有埶者;即無埶者,視之如奴。有埶家,雖有姦如山,弗犯;無埶者,貴戚必侵辱。舞文巧詆下戶之猾,以熏大豪。其治中尉如此。姦猾窮治,大抵盡靡爛獄中,行論無出者。其爪牙吏虎而冠。於是中尉部中中猾以下皆伏,有勢者為游聲譽,稱治。治數歲,其吏多以權富。

溫舒擊東越還,議有不中意者,坐小法抵罪免。是時天子方欲作通天臺而未有人,溫舒請覆中尉脫卒,得數萬人作。上說,拜為少府。徙為右內史,治如其故,姦邪少禁。坐法失官。復為右輔,行中尉事。如故操。

歲餘,會宛軍發,詔徵豪吏,溫舒匿其吏華成,及人有變告溫舒受員騎錢,他姦利事,罪至族,自殺。其時兩弟及兩婚家亦各自坐他罪而族。光祿徐自為曰:「悲夫,夫古有三族,而王溫舒罪至同時而五族乎!」

溫舒死,家直累千金。後數歲,尹齊亦以淮陽都尉病死,家直不滿五十金。所誅滅淮陽甚多,及死,仇家欲燒其尸,尸亡去歸葬。

自溫舒等以惡為治,而郡守、都尉、諸侯二千石欲為治者,其治大抵盡放溫舒,而吏民益輕犯法,盜賊滋起。南陽有梅免、白政,楚有殷中、杜少,齊有徐勃,燕趙之閒有堅盧、范生之屬。大群至數千人,擅自號,攻城邑,取庫兵,釋死罪,縛辱郡太守、都尉,殺二千石,為檄告縣趣具食;小群以百數,掠鹵鄉里者,不可勝數也。於是天子始使御史中丞、丞相長史督之。猶弗能禁也,乃使光祿大夫范昆、諸輔都尉及故九卿張德等衣繡衣,持節,虎符發兵以興擊,斬首大部或至萬餘級,及以法誅通飲食,坐連諸郡,甚者數千人。數歲,乃頗得其渠率。散卒失亡,復聚黨阻山川者,往往而群居,無可柰何。於是作「沈命法」,曰群盜起不發覺,發覺而捕弗滿品者,二千石以下至小吏主者皆死。其後小吏畏誅,雖有盜不敢發,恐不能得,坐課累府,府亦使其不言。故盜賊寖多,上下相為匿,以文辭避法焉。

減宣者,楊人也。以佐史無害給事河東守府。衛將軍青使買馬河東,見宣無害,言上,徵為大廐丞。官事辨,稍遷至御史及中丞。使治主父偃及治淮南反獄,所以微文深詆,殺者甚眾,稱為敢決疑。數廢數起,為御史及中丞者幾二十歲。王溫舒免中尉,而宣為左內史。其治米鹽,事大小皆關其手,自部署縣名曹實物,官吏令丞不得擅搖,痛以重法繩之。居官數年,一切郡中為小治辨,然獨宣以小致大,能因力行之,難以為經。中廢。為右扶風,坐怨成信,信亡藏上林中,宣使郿令格殺信,吏卒格信時,射中上林苑門,宣下吏詆罪,以為大逆,當族,自殺。而杜周任用。

杜周者,南陽杜衍人。義縱為南陽守,以為爪牙,舉為廷尉史。事張湯,湯數言其無害,至御史。使案邊失亡,所論殺甚眾。奏事中上意,任用,與減宣相編,更為中丞十餘歲。

其治與宣相放,然重遲,外寬,內深次骨。宣為左內史,周為廷尉,其治大放張湯而善候伺。上所欲擠者,因而陷之;上所欲釋者,久系待問而微見其冤狀。客有讓周曰:「君為天子決平,不循三尺法,專以人主意指為獄。獄者固如是乎?」周曰:「三尺安出哉?前主所是著為律,後主所是疏為令,當時為是,何古之法乎!」

至周為廷尉,詔獄亦益多矣。二千石系者新故相因,不減百餘人。郡吏大府舉之廷尉,一歲至千餘章。章大者連逮證案數百,小者數十人;遠者數千,近者數百里。會獄,吏因責如章告劾,不服,以笞掠定之。於是聞有逮皆亡匿。獄久者至更數赦十有餘歲而相告言,大抵盡詆以不道以上。廷尉及中都官詔獄逮至六七萬人,吏所增加十萬餘人。

周中廢,後為執金時,逐盜,捕治桑弘羊、衛皇后昆弟子刻深,天子以為盡力無私,遷為御史大夫。家兩子,夾河為守。其治暴酷皆甚於王溫舒等矣。杜周初徵為廷史,有一馬,且不全;及身久任事,至三公列,子孫尊官,家訾累數巨萬矣。

太史公曰:自郅都、杜周十人者,此皆以酷烈為聲。然郅都伉直,引是非,爭天下大體。張湯以知陰陽,人主與俱上下,時數辯當否,國家賴其便。趙禹時據法守正。杜周從諛,以少言為重。自張湯死后,網密,多詆嚴,官事寖以秏廢。九卿碌碌奉其官,救過不贍,何暇論繩墨之外乎!然此十人中,其廉者足以為儀表,其污者足以為戒,方略教導,禁姦止邪,一切亦皆彬彬質有其文武焉。雖慘酷,斯稱其位矣。至若蜀守馮當暴挫,廣漢李貞擅磔人,東郡彌仆鋸項,天水駱璧推咸,河東褚廣妄殺,京兆無忌、馮翊殷周蝮鷙,水衡閻奉樸擊賣請,何足數哉!何足數哉!


\end{pinyinscope}