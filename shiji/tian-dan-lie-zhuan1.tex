\article{田儋列傳}

\begin{pinyinscope}
田儋者,狄人也,故齊王田氏族也。儋從弟田榮,榮弟田橫,皆豪,宗彊,能得人。

陳涉之初起王楚也,使周市略定魏地,北至狄,狄城守。田儋詳為縛其奴,從少年之廷,欲謁殺奴。見狄令,因擊殺令,而召豪吏子弟曰:「諸侯皆反秦自立,齊,古之建國,儋,田氏,當王。」遂自立為齊王,發兵以擊周市。周市軍還去,田儋因率兵東略定齊地。

秦將章邯圍魏王咎於臨濟,急。魏王請救於齊,齊王田儋將兵救魏。章邯夜銜枚擊,大破齊、魏軍,殺田儋於臨濟下。儋弟田榮收儋餘兵東走東阿。

齊人聞王田儋死,乃立故齊王建之弟田假為齊王,田角為相,田間為將,以距諸侯。

田榮之走東阿,章邯追圍之。項梁聞田榮之急,乃引兵擊破章邯軍東阿下。章邯走而西,項梁因追之。而田榮怒齊之立假,乃引兵歸,擊逐齊王假。假亡走楚。齊相角亡走趙;角弟田閒前求救趙,因留不敢歸。田榮乃立田儋子市為齊王。榮相之,田橫為將,平齊地。

項梁既追章邯,章邯兵益盛,項梁使使告趙、齊,發兵共擊章邯。田榮曰:「使楚殺田假,趙殺田角、田閒,閒肯出兵。」楚懷王曰:「田假與國之王,窮而歸我,殺之不義。」趙亦不殺田角、田閒以市於齊。齊曰:「蝮螫手則斬手,螫足則斬足。何者?為害於身也。今田假、田角、田閒於楚、趙,非直手足戚也,何故不殺?且秦復得志於天下,則齮龁用事者墳墓矣。」楚、趙不聽,齊亦怒,終不肯出兵。章邯果敗殺項梁,破楚兵,楚兵東走,而章邯渡河圍趙於鉅鹿。項羽往救趙,由此怨田榮。

項羽既存趙,降章邯等,西屠咸陽,滅秦而立侯王也,乃徙齊王田市更王膠東,治即墨。齊將田都從共救趙,因入關,故立都為齊王,治臨淄。故齊王建孫田安,項羽方渡河救趙,田安下濟北數城,引兵降項羽,項羽立田安為濟北王,治博陽。田榮以負項梁不肯出兵助楚、趙攻秦,故不得王;趙將陳餘亦失職,不得王:二人俱怨項王。

頊王既歸,諸侯各就國,田榮使人將兵助陳餘,令反趙地,而榮亦發兵以距擊田都,田都亡走楚。田榮留齊王市,無令之膠東。市之左右曰:「項王彊暴,而王當之膠東,不就國,必危。」市懼,乃亡就國。田榮怒,追擊殺齊王市於即墨,還攻殺濟北王安。於是田榮乃自立為齊王,盡并三齊之地。

項王聞之,大怒,乃北伐齊。齊王田榮兵敗,走平原,平原人殺榮。項王遂燒夷齊城郭,所過者盡屠之。齊人相聚畔之。榮弟橫,收齊散兵,得數萬人,反擊項羽於城陽。而漢王率諸侯敗楚,入彭城。項羽聞之,乃醳齊而歸,擊漢於彭城,因連與漢戰,相距滎陽。以故田橫復得收齊城邑,立田榮子廣為齊王,而橫相之,專國政,政無巨細皆斷於相。

橫定齊三年,漢王使酈生往說下齊王廣及其相國橫。橫以為然,解其歷下軍。漢將韓信引兵且東擊齊。齊初使華無傷、田解軍於歷下以距漢,漢使至,乃罷守戰備,縱酒,且遣使與漢平。漢將韓信已平趙、燕,用蒯通計,度平原,襲破齊歷下軍,因入臨淄。齊王廣、相橫怒,以酈生賣己,而亨酈生。齊王廣東走高密,相橫走博(陽),守相田光走城陽,將軍田既軍於膠東。楚使龍且救齊,齊王與合軍高密。漢將韓信與曹參破殺龍且,虜齊王廣。漢將灌嬰追得齊守相田光。至博(陽),而橫聞齊王死,自立為齊王,還擊嬰,嬰敗橫之軍於嬴下。田橫亡走梁,歸彭越。彭越是時居梁地,中立,且為漢,且為楚。韓信已殺龍且,因令曹參進兵破殺田既於膠東,使灌嬰破殺齊將田吸於千乘。韓信遂平齊,乞自立為齊假王,漢因而立之。

後歲餘,漢滅項籍,漢王立為皇帝,以彭越為梁王。田橫懼誅,而與其徒屬五百餘人入海,居島中。高帝聞之,以為田橫兄弟本定齊,齊人賢者多附焉,今在海中不收,後恐為亂,乃使使赦田橫罪而召之。田橫因謝曰:「臣亨陛下之使酈生,今聞其弟酈商為漢將而賢,臣恐懼,不敢奉詔,請為庶人,守海島中。」使還報,高皇帝乃詔衛尉酈商曰:「齊王田橫即至,人馬從者敢動搖者致族夷!」乃復使使持節具告以詔商狀,曰:「田橫來,大者王,小者乃侯耳;不來,且舉兵加誅焉。」田橫乃與其客二人乘傳詣雒陽。

未至三十里,至尸鄉廄置,橫謝使者曰:「人臣見天子當洗沐。」止留。謂其客曰:「橫始與漢王俱南面稱孤,今漢王為天子,而橫乃為亡虜而北面事之,其恥固已甚矣。且吾亨人之兄,與其弟并肩而事其主,縱彼畏天子之詔,不敢動我,我獨不愧於心乎?且陛下所以欲見我者,不過欲一見吾面貌耳。今陛下在洛陽,今斬吾頭,馳三十里閒,形容尚未能敗,猶可觀也。」遂自剄,令客奉其頭,從使者馳奏之高帝。高帝曰:「嗟乎,有以也夫!起自布衣,兄弟三人更王,豈不賢乎哉!」為之流涕,而拜其二客為都尉,發卒二千人,以王者禮葬田橫。

既葬,二客穿其冢旁孔,皆自剄,下從之。高帝聞之,乃大驚,大田橫之客皆賢。吾聞其餘尚五百人在海中,使使召之。至則聞田橫死,亦皆自殺。於是乃知田橫兄弟能得士也。

太史公曰:甚矣蒯通之謀,亂齊驕淮陰,其卒亡此兩人!蒯通者,善為長短說,論戰國之權變,為八十一首。通善齊人安期生,安期生嘗干項羽,項羽不能用其筴。已而項羽欲封此兩人,兩人終不肯受,亡去。田橫之高節,賓客慕義而從橫死,豈非至賢!余因而列焉。不無善畫者,莫能圖,何哉?


\end{pinyinscope}