\article{韓世家}

\begin{pinyinscope}
韓之先與周同姓,姓姬氏。其後苗裔事晉,得封於韓原,曰韓武子。武子後三世有韓厥,從封姓為韓氏。

韓厥,晉景公之三年,晉司寇屠岸賈將作亂,誅靈公之賊趙盾。趙盾已死矣,欲誅其子趙朔。韓厥止賈,賈不聽。厥告趙朔令亡。朔曰:「子必能不絕趙祀,死不恨矣。」韓厥許之。及賈誅趙氏,厥稱疾不出。程嬰、公孫杵臼之藏趙孤趙武也,厥知之。

景公十一年,厥與郤克將兵八百乘伐齊,敗齊頃公于鞍,獲逢丑父。於是晉作六卿,而韓厥在一卿之位,號為獻子。

晉景公十七年,病,卜大業之不遂者為祟。韓厥稱趙成季之功,今後無祀,以感景公。景公問曰:「尚有世乎?」厥於是言趙武,而復與故趙氏田邑,續趙氏祀。

晉悼公之(十)[七]年,韓獻子老。獻子卒,子宣子代。宣字徙居州。

晉平公十四年,吳季札使晉,曰:「晉國之政卒歸於韓、魏、趙矣。」晉頃公十二年,韓宣子與趙、魏共分祁氏、羊舌氏十縣。晉定公十五年,宣子與趙簡子侵伐范、中行氏。宣子卒,子貞子代立。貞子徙居平陽。

貞子卒,子簡子代。簡子卒,子莊子代。莊子卒,子康子代。康子與趙襄子、魏桓子共敗知伯,分其地,地益大,大於諸侯。

康子卒,子武子代。武子二年,伐鄭,殺其君幽公。十六年,武子卒,子景侯立。

景侯虔元年,伐鄭,取雍丘。二年,鄭敗我負黍。

六年,與趙、魏俱得列為諸侯。

九年,鄭圍我陽翟。景侯卒,子列侯取立。

列侯三年,聶政殺韓相俠累。九年,秦伐我宜陽,取六邑。十三年,列侯卒,子文侯立。是歲魏文侯卒。

文侯二年,伐鄭,取陽城。伐宋,到彭城,執宋君。七年,伐齊,至桑丘。鄭反晉。九年,伐齊,至靈丘。十年,文侯卒,子哀侯立。

哀侯元年,與趙、魏分晉國。二年,滅鄭,因徙都鄭。

六年,韓嚴弒其君哀侯。而子懿侯立。

懿侯二年,魏敗我馬陵。五年,與魏惠王會宅陽。九年,魏敗我澮。十二年,懿侯卒,子昭侯立。

昭侯元年,秦敗我西山。二年,宋取我黃池。魏取朱。六年,伐東周,取陵觀、邢丘。

八年,申不害相韓,修術行道,國內以治,諸侯不來侵伐。

十年,韓姬弒其君悼公。十一年,昭侯如秦。二十二年,申不害死。二十四年,秦來拔我宜陽。

二十五年,旱,作高門。屈宜臼曰:「昭侯不出此門。何也?不時。吾所謂時者,非時日也,人固有利不利時。昭侯嘗利矣,不作高門。往年秦拔宜陽,今年旱,昭侯不以此時卹民之急,而顧益奢,此謂『時絀舉贏』。」二十六年,高門成,昭侯卒,果不出此門。子宣惠王立。

宣惠王五年,張儀相秦。八年,魏敗我將韓舉。十一年,君號為王。與趙會區鼠。十四,秦伐敗我鄢。

十六年,秦敗我修魚,虜得韓將鯁、申差於濁澤。韓氏急,公仲謂韓王曰:「與國非可恃也。今秦之欲伐楚久矣,王不如因張儀為和於秦,賂以一名都,具甲,與之南伐楚,此以一易二之計也。」韓王曰:「善。」乃警公仲之行,將西購於秦。楚王聞之大恐,召陳軫告之。陳軫曰:「秦之欲伐楚久矣,今又得韓之名都一而具甲,秦韓并兵而伐楚,此秦所禱祀而求也。今已得之矣,楚國必伐矣。王聽臣為之警四境之內,起師言救韓,命戰車滿道路,發信臣,多其車,重其幣,使信王之救己也。縱韓不能聽我,韓必德王也,必不為鴈行以來,是秦韓不和也,兵雖至,楚不大病也。為能聽我絕和於秦,秦必大怒,以厚怨韓。韓之南交楚,必輕秦;輕秦,其應秦必不敬:是因秦、韓之兵而免楚國之患也。」楚王曰:「善。」乃警四境之內,興師言救韓。命戰車滿道路,發信臣,多其車,重其幣。謂韓王曰:「不穀國雖小,已悉發之矣。願大國遂肆志於秦,不穀將以楚殉韓。」韓王聞之大說,乃止公仲之行。公仲曰:「不可。夫以實伐我者秦也,以虛名救我者楚也。王恃楚之虛名,而輕絕彊秦之敵,王必為天下大笑。且楚韓非兄弟之國也,又非素約而謀伐秦也。已有伐形,因發兵言救韓,此必陳軫之謀也。且王已使人報於秦矣,今不行,是欺秦也。夫輕欺彊秦而信楚之謀臣,恐王必悔之。」韓王不聽,遂絕於秦。秦因大怒,益甲伐韓,大戰,楚救不至韓。十九年,大破我岸門。太子倉質於秦以和。

二十一年,與秦共攻楚,敗楚將屈丐,斬首八萬於丹陽。是歲,宣惠王卒,太子倉立,是為襄王。

襄王四年,與秦武王會臨晉。其秋,秦使甘茂攻我宜陽。五年,秦拔我宜陽,斬首六萬。秦武王卒。六年,秦復與我武遂。九年,秦復取我武遂。十年,太子嬰朝秦而歸。十一年,秦伐我,取穰。與秦伐楚,敗楚將唐眛。

十二年,太子嬰死。公子咎、公子蟣蝨爭為太子。時蟣蝨質於楚。蘇代謂韓咎曰:「蟣蝨亡在楚,楚王欲內之甚。今楚兵十餘萬在方城之外,公何不令楚王筑萬室之都雍氏之旁,韓必起兵以救之,公必將矣。公因以韓楚之兵奉蟣蝨而內之,其聽公必矣,必以楚韓封公也。」韓咎從其計。

楚圍雍氏,韓求救於秦。秦未為發,使公孫昧入韓。公仲曰:「子以秦為且救韓乎?」對曰:「秦王之言曰『請道南鄭、藍田,出兵於楚以待公』,殆不合矣。」公仲曰:「子以為果乎?」對曰:「秦王必祖張儀之故智。楚威王攻梁也,張儀謂秦王曰:『與楚攻魏,魏折而入於楚,韓固其與國也,是秦孤也。不如出兵以到之,魏楚大戰,秦取西河之外以歸。』今其狀陽言與韓,其實陰善楚。公待秦而到,必輕與楚戰。楚陰得秦之不用也,必易與公相支也。公戰而勝楚,遂與公乘楚,施三川而歸。公戰不勝楚,楚塞三川守之,公不能救也。竊為公患之。司馬庚三反於郢,甘茂與昭魚遇於商於,其言收璽,實類有約也。」公仲恐,曰:「然則柰何?」曰:「公必先韓而後秦,先身而後張儀。公不如亟以國合於齊楚,齊楚必委國於公。公之所惡者張儀也,其實猶不無秦也。」於是楚解雍氏圍。

蘇代又謂秦太后弟羋戎曰:「公叔伯嬰恐秦楚之內蟣蝨也,公何不為韓求質子於楚?楚王聽入質子於韓,則公叔伯嬰知秦楚之不以蟣蝨為事,必以韓合於秦楚。秦楚挾韓以窘魏,魏氏不敢合於齊,是齊孤也。公又為秦求質子於楚,楚不聽,怨結於韓。韓挾齊魏以圍楚,楚必重公。公挾秦楚之重以積德於韓,公叔伯嬰必以國待公。」於是蟣蝨竟不得歸韓。韓立咎為太子。齊、魏王來。

十四年,與齊、魏王共擊秦,至函谷而軍焉。十六年,秦與我河外及武遂。襄王卒,太子咎立,是為釐王。

釐王三年,使公孫喜率周、魏攻秦。秦敗我二十四萬,虜喜伊闕。五年,秦拔我宛。六年,與秦武遂地二百里。十年,秦敗我師于夏山。十二年,與秦昭王會西周而佐秦攻齊。齊敗,湣王出亡。十四年,與秦會兩周閒。二十一年,使暴捐救魏,為秦所敗,捐走開封。

二十三年,趙、魏攻我華陽。韓告急於秦,秦不救。韓相國謂陳筮曰:「事急,願公雖病,為一宿之行。」陳筮見穰侯。穰侯曰:「事急乎?故使公來。」陳筮曰:「未急也。」穰侯怒曰:「是可以為公之主使乎?夫冠蓋相望,告敝邑甚急,公來言未急,何也?」陳筮曰:「彼韓急則將變而佗從,以未急,故復來耳。」穰侯曰:「公無見王,請今發兵救韓。」八日而至,敗趙、魏於華陽之下。是歲,釐王卒,子桓惠王立。

桓惠王元年,伐燕。九年,秦拔我陘,城汾旁。十年,秦擊我於太行,我上黨郡守以上黨郡降趙。十四年,秦拔趙上黨,殺馬服子卒四十餘萬於長平。十七年,秦拔我陽城、負黍。二十二年,秦昭王卒。二十四年,秦拔我城皋、滎陽。二十六年,秦悉拔我上黨。二十九年,秦拔我十三城。

三十四年,桓惠王卒,子王安立。

王安五年,秦攻韓,韓急,使韓非使秦,秦留非,因殺之。

九年,秦虜王安,盡入其地,為潁州郡。韓遂亡。

太史公曰:韓厥之感晉景公,紹趙孤之子武,以成程嬰、公孫杵臼之義,此天下之陰德也。韓氏之功,於晉未睹其大者也。然與趙、魏終為諸侯十餘世,宜乎哉!


\end{pinyinscope}