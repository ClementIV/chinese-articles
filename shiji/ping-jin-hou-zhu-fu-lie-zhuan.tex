\article{平津侯主父列傳}

\begin{pinyinscope}
丞相公孫弘者,齊菑川國薛縣人也,字季。少時為薛獄吏,有罪,免。家貧,牧豕海上。年四十餘,乃學春秋雜說。養後母孝謹。

建元元年,天子初即位,招賢良文學之士。是時弘年六十,徵以賢良為博士。使匈奴,還報,不合上意,上怒,以為不能,弘乃病免歸。

元光五年,有詔徵文學,菑川國復推上公孫弘。弘讓謝國人曰:「臣已嘗西應命,以不能罷歸,願更推選。」國人固推弘,弘至太常。太常令所徵儒士各對策,百餘人,弘第居下。策奏,天子擢弘對為第一。召入見,狀貌甚麗,拜為博士。是時通西南夷道,置郡,巴蜀民苦之,詔使弘視之。還奏事,盛毀西南夷無所用,上不聽。

弘為人恢奇多聞,常稱以為人主病不廣大,人臣病不儉節。弘為布被,食不重肉。后母死,服喪三年。每朝會議,開陳其端,令人主自擇,不肯面折庭爭。於是天子察其行敦厚,辯論有餘,習文法吏事,而又緣飾以儒術,上大說之。二歲中,至左內史。弘奏事,有不可,不庭辯之。嘗與主爵都尉汲黯請閒,汲黯先發之,弘推其後,天子常說,所言皆聽,以此日益親貴。嘗與公卿約議,至上前,皆倍其約以順上旨。汲黯庭詰弘曰:「齊人多詐而無情實,始與臣等建此議,今皆倍之,不忠。」上問弘。弘謝曰:「夫知臣者以臣為忠,不知臣者以臣為不忠。」上然弘言。左右幸臣每毀弘,上益厚遇之。

元朔三年,張歐免,以弘為御史大夫。是時通西南夷,東置滄海,北筑朔方之郡。弘數諫,以為罷敝中國以奉無用之地,願罷之。於是天子乃使朱買臣等難弘置朔方之便。發十策,弘不得一。弘乃謝曰:「山東鄙人,不知其便若是,願罷西南夷、滄海而專奉朔方。」上乃許之。

汲黯曰:「弘位在三公,奉祿甚多。然為布被,此詐也。」上問弘。弘謝曰:「有之。夫九卿與臣善者無過黯,然今日庭詰弘,誠中弘之病。夫以三公為布被,誠飾詐欲以釣名。且臣聞管仲相齊,有三歸,侈擬於君,桓公以霸,亦上僭於君。晏嬰相景公,食不重肉,妾不衣絲,齊國亦治,此下比於民。今臣弘位為御史大夫,而為布被,自九卿以下至於小吏,無差,誠如汲黯言。且無汲黯忠,陛下安得聞此言。」天子以為謙讓,愈益厚之。卒以弘為丞相,封平津侯。

弘為人意忌,外寬內深。諸嘗與弘有卻者,雖詳與善,陰報其禍。殺主父偃,徙董仲舒於膠西,皆弘之力也。食一肉脫粟之飯。故人所善賓客,仰衣食,弘奉祿皆以給之,家無所餘。士亦以此賢之。

淮南、衡山謀反,治黨與方急。弘病甚,自以為無功而封,位至丞相,宜佐明主填撫國家,使人由臣子之道。今諸侯有畔逆之計,此皆宰相奉職不稱,恐竊病死,無以塞責。乃上書曰:「臣聞天下之通道五,所以行之者三。曰君臣,父子,兄弟,夫婦,長幼之序,此五者天下之通道也。智,仁,勇,此三者天下之通德,所以行之者也。故曰『力行近乎仁,好問近乎智,知恥近乎勇』。知此三者,則知所以自治;知所以自治,然後知所以治人。天下未有不能自治而能治人者也,此百世不易之道也。今陛下躬行大孝,鑒三王,建周道,兼文武,厲賢予祿,量能授官。今臣弘罷駑之質,無汗馬之勞,陛下過意擢臣弘卒伍之中,封為列侯,致位三公。臣弘行能不足以稱,素有負薪之病,恐先狗馬填溝壑,終無以報德塞責。願歸侯印,乞骸骨,避賢者路。」天子報曰:「古者賞有功,襃德,守成尚文,遭遇右武,未有易此者也。朕宿昔庶幾獲承尊位,懼不能寧,惟所與共為治者,君宜知之。蓋君子善善惡惡,(君宜知之)君若謹行,常在朕躬。君不幸罹霜露之病,何恙不已,乃上書歸侯,乞骸骨,是章朕之不德也。今事少閒,君其省思慮,一精神,輔以醫藥。」因賜告牛酒雜帛。居數月,病有瘳,視事。

元狩二年,弘病,竟以丞相終。子度嗣為平津侯。度為山陽太守十餘歲,坐法失侯。

主父偃者,齊臨菑人也。學長短縱橫之術,晚乃學易、春秋、百家言。游齊諸生閒,莫能厚遇也。齊諸儒生相與排擯,不容於齊。家貧,假貸無所得,乃北游燕、趙、中山,皆莫能厚遇,為客甚困。孝武元光元年中,以為諸侯莫足游者,乃西入關見衛將軍。衛將軍數言上,上不召。資用乏,留久,諸公賓客多厭之,乃上書闕下。朝奏,暮召入見。所言九事,其八事為律令,一事諫伐匈奴。其辭曰:

臣聞明主不惡切諫以博觀,忠臣不敢避重誅以直諫,是故事無遺策而功流萬世。今臣不敢隱忠避死以效愚計,願陛下幸赦而少察之。

《司馬法》曰:「國雖大,好戰必亡;天下雖平,忘戰必危。」天下既平,天子大凱,春蒐秋狝,諸侯春振旅,秋治兵,所以不忘戰也。且夫怒者逆德也,兵者凶器也,爭者末節也。古之人君一怒必伏尸流血,故聖王重行之。夫務戰勝窮武事者,未有不悔者也。昔秦皇帝任戰勝之威,蠶食天下,并吞戰國,海內為一,功齊三代。務勝不休,欲攻匈奴,李斯諫曰:「不可。夫匈奴無城郭之居,委積之守,遷徙鳥舉,難得而制也。輕兵深入,糧食必絕;踵糧以行,重不及事。得其地不足以為利也,遇其民不可役而守也。勝必殺之,非民父母也。靡獘中國,快心匈奴,非長策也。」秦皇帝不聽,遂使蒙恬將兵攻胡,辟地千里,以河為境。地固澤(咸)鹵,不生五穀。然後發天下丁男以守北河。暴兵露師十有餘年,死者不可勝數,終不能踰河而北。是豈人眾不足,兵革不備哉?其勢不可也。又使天下蜚芻芻粟,起於黃、腄、瑯邪負海之郡,轉輸北河,率三十鐘而致一石。男子疾耕不足於糧馕,女子紡績不足於帷幕。百姓靡敝,孤寡老弱不能相養,道路死者相望,蓋天下始畔秦也。

及至高皇帝定天下,略地於邊,聞匈奴聚於代谷之外而欲擊之。御史成進諫曰:「不可。夫匈奴之性,獸聚而鳥散,從之如搏影。今以陛下盛德攻匈奴,臣竊危之。」高帝不聽,遂北至於代谷,果有平城之圍。高皇帝蓋悔之甚,乃使劉敬往結和親之約,然後天下忘干戈之事。故兵法曰「興師十萬,日費千金」。夫秦常積眾暴兵數十萬人,雖有覆軍殺將系虜單于之功,亦適足以結怨深讎,不足以償天下之費。夫上虛府庫,下敝百姓,甘心於外國,非完事也。夫匈奴難得而制,非一世也。行盜侵驅,所以為業也,天性固然。上及虞夏殷周,固弗程督,禽獸畜之,不屬為人。夫上不觀虞夏殷周之統,而下(修)[循]近世之失,此臣之所大憂,百姓之所疾苦也。且夫兵久則變生,事苦則慮易。乃使邊境之民獘靡愁苦而有離心,將吏相疑而外市,故尉佗、章邯得以成其私也。夫秦政之所以不行者,權分乎二子,此得失之效也。故《周書》曰「安危在出令,存亡在所用」。願陛下詳察之,少加意而熟慮焉。

是時趙人徐樂、齊人嚴安俱上書言世務,各一事。徐樂曰:

臣聞天下之患在於土崩,不在於瓦解,古今一也。何謂土崩?秦之末世是也。陳涉無千乘之尊,尺土之地,身非王公大人名族之后,無鄉曲之譽,非有孔、墨、曾子之賢,陶朱、猗頓之富也,然起窮巷,奮棘矜,偏袒大呼而天下從風,此其故何也?由民困而主不恤,下怨而上不知(也),俗已亂而政不修,此三者陳涉之所以為資也。是之謂土崩。故曰天下之患在於土崩。何謂瓦解?吳、楚、齊、趙之兵是也。七國謀為大逆,號皆稱萬乘之君,帶甲數十萬,威足以嚴其境內,財足以勸其士民,然不能西攘尺寸之地而身為禽於中原者,此其故何也?非權輕於匹夫而兵弱於陳涉也,當是之時,先帝之德澤未衰而安土樂俗之民眾,故諸侯無境外之助。此之謂瓦解,故曰天下之患不在瓦解。由是觀之,天下誠有土崩之勢,雖布衣窮處之士或首惡而危海內,陳涉是也。況三晉之君或存乎!天下雖未有大治也,誠能無土崩之勢,雖有彊國勁兵不得旋踵而身為禽矣,吳、楚、齊、趙是也。況群臣百姓能為亂乎哉!此二體者,安危之明要也,賢主所留意而深察也。

閒者關東五穀不登,年歲未復,民多窮困,重之以邊境之事,推數循理而觀之,則民且有不安其處者矣。不安故易動。易動者,土崩之勢也。故賢主獨觀萬化之原,明於安危之機,修之廟堂之上,而銷未形之患。其要,期使天下無土崩之勢而已矣。故雖有彊國勁兵,陛下逐走獸,射蜚鳥,弘游燕之囿,淫縱恣之觀,極馳騁之樂,自若也。金石絲竹之聲不絕於耳,帷帳之私俳優侏儒之笑不乏於前,而天下無宿憂。名何必湯武,俗何必成康!雖然,臣竊以為陛下天然之聖,寬仁之資,而誠以天下為務,則湯武之名不難侔,而成康之俗可復興也。此二體者立,然後處尊安之實,揚名廣譽於當世,親天下而服四夷,餘恩遺德為數世隆,南面負扆攝袂而揖王公,此陛下之所服也。臣聞圖王不成,其敝足以安。安則陛下何求而不得,何為而不成,何征而不服乎哉!嚴安上書曰:

臣聞周有天下,其治三百餘歲,成康其隆也,刑錯四十餘年而不用。及其衰也,亦三百餘歲,故五伯更起。五伯者,常佐天子興利除害,誅暴禁邪,匡正海內,以尊天子。五伯既沒,賢聖莫續,天子孤弱,號令不行。諸侯恣行,彊陵弱,眾暴寡,田常篡齊,六卿分晉,并為戰國,此民之始苦也。於是彊國務攻,弱國備守,合從連橫,馳車擊轂,介胄生蟣蝨,民無所告愬。

及至秦王,蠶食天下,并吞戰國,稱號曰皇帝,主海內之政,壞諸侯之城,銷其兵,鑄以為鐘虡,示不復用。元元黎民得免於戰國,逢明天子,人人自以為更生。向使秦緩其刑罰,薄賦斂,省繇役,貴仁義,賤權利,上篤厚,下智巧,變風易俗,化於海內,則世世必安矣。秦不行是風而修其故俗,為智巧權利者進,篤厚忠信者退;法嚴政峻,諂諛者眾,日聞其美,意廣心軼。欲肆威海外,乃使蒙恬將兵以北攻胡,辟地進境,戍於北河,蜚芻芻粟以隨其后。又使尉[佗]屠睢將樓船之士南攻百越,使監祿鑿渠運糧,深入越,越人遁逃。曠日持久,糧食絕乏,越人擊之,秦兵大敗。秦乃使尉佗將卒以戍越。當是時,秦禍北構於胡,南掛於越,宿兵無用之地,進而不得退。行十餘年,丁男被甲,丁女轉輸,苦不聊生,自經於道樹,死者相望。及秦皇帝崩,天下大叛。陳勝、吳廣舉陳,武臣、張耳舉趙,項梁舉吳,田儋舉齊,景駒舉郢,周市舉魏,韓廣舉燕,窮山通谷豪士并起,不可勝載也。然皆非公侯之后,非長官之吏也。無尺寸之勢,起閭巷,杖棘矜,應時而皆動,不謀而俱起,不約而同會,壤長地進,至于霸王,時教使然也。秦貴為天子,富有天下,滅世絕祀者,窮兵之禍也。故周失之弱,秦失之彊,不變之患也。

今欲招南夷,朝夜郎,降羌僰,略濊州,建城邑,深入匈奴,燔其蘢城,議者美之。此人臣之利也,非天下之長策也。今中國無狗吠之驚,而外累於遠方之備,靡敝國家,非所以子民也。行無窮之欲,甘心快意,結怨於匈奴,非所以安邊也。禍結而不解,兵休而復起,近者愁苦,遠者驚駭,非所以持久也。今天下鍛甲砥劍,橋箭累弦,轉輸運糧,未見休時,此天下之所共憂也。夫兵久而變起,事煩而慮生。今外郡之地或幾千里,列城數十,形束壤制,旁脅諸侯,非公室之利也。上觀齊晉之所以亡者,公室卑削,六卿大盛也;下觀秦之所以滅者,嚴法刻深,欲大無窮也。今郡守之權,非特六卿之重也;地幾千里,非特閭巷之資也;甲兵器械,非特棘矜之用也:以遭萬世之變,則不可稱諱也。

書奏天子,天子召見三人,謂曰:「公等皆安在?何相見之晚也!」於是上乃拜主父偃、徐樂、嚴安為郎中。偃數見,上疏言事,詔拜偃為謁者,遷為中大夫。一歲中四遷偃。

偃說上曰:「古者諸侯不過百里,彊弱之形易制。今諸侯或連城數十,地方千里,緩則驕奢易為淫亂,急則阻其彊而合從以逆京師。今以法割削之,則逆節萌起,前日晁錯是也。今諸侯子弟或十數,而適嗣代立,餘雖骨肉,無尺寸地封,則仁孝之道不宣。願陛下令諸侯得推恩分子弟,以地侯之。彼人人喜得所願,上以德施,實分其國,不削而稍弱矣。」於是上從其計。又說上曰:「茂陵初立,天下豪桀并兼之家,亂眾之民,皆可徙茂陵,內實京師,外銷姦猾,此所謂不誅而害除。」上又從其計。

尊立衛皇后,及發燕王定國陰事,蓋偃有功焉。大臣皆畏其口,賂遺累千金。人或說偃曰:「太橫矣。」主父曰:「臣結發游學四十餘年,身不得遂,親不以為子,昆弟不收,賓客棄我,我阸日久矣。且丈夫生不五鼎食,死即五鼎烹耳。吾日暮途遠,故倒行暴施之。」

偃盛言朔方地肥饒,外阻河,蒙恬城之以逐匈奴,內省轉輸戍漕,廣中國,滅胡之本也。上覽其說,下公卿議,皆言不便。公孫弘曰:「秦時常發三十萬眾筑北河,終不可就,已而棄之。」主父偃盛言其便,上竟用主父計,立朔方郡。

元朔二年,主父言齊王內淫佚行僻,上拜主父為齊相。至齊,遍召昆弟賓客,散五百金予之,數之曰:「始吾貧時,昆弟不我衣食,賓客不我內門;今吾相齊,諸君迎我或千里。吾與諸君絕矣,毋復入偃之門!」乃使人以王與姊姦事動王,王以為終不得脫罪,恐效燕王論死,乃自殺。有司以聞。

主父始為布衣時,嘗游燕、趙,及其貴,發燕事。趙王恐其為國患,欲上書言其陰事,為偃居中,不敢發。及為齊相,出關,即使人上書,告言主父偃受諸侯金,以故諸侯子弟多以得封者。及齊王自殺,上聞大怒,以為主父劫其王令自殺,乃徵下吏治。主父服受諸侯金,實不劫王令自殺。上欲勿誅,是時公孫弘為御史大夫,乃言曰:「齊王自殺無後,國除為郡,入漢,主父偃本首惡,陛下不誅主父偃,無以謝天下。」乃遂族主父偃。

主父方貴幸時,賓客以千數,及其族死,無一人收者,唯獨洨孔車收葬之。天子后聞之,以為孔車長者也。

太史公曰:公孫弘行義雖修,然亦遇時。漢興八十餘年矣,上方鄉文學,招俊乂,以廣儒墨,弘為舉首。主父偃當路,諸公皆譽之,及名敗身誅,士爭言其惡。悲夫!

太皇太后詔大司徒大司空:「蓋聞治國之道,富民為始;富民之要,在於節儉。《孝經》曰『安上治民,莫善於禮』。『禮,與奢也寧儉』。昔者管仲相齊桓,霸諸侯,有九合一匡之功,而仲尼謂之不知禮,以其奢泰侈擬於君故也。夏禹卑宮室,惡衣服,后聖不循。由此言之,治之盛也,德優矣,莫高於儉。儉化俗民,則尊卑之序得,而骨肉之恩親,爭訟之原息。斯乃家給人足,刑錯之本也歟?可不務哉!夫三公者,百寮之率,萬民之表也。未有樹直表而得曲影者也。孔子不云乎,『子率而正,孰敢不正』。『舉善而教不能則勸』。維漢興以來,股肱宰臣身行儉約,輕財重義,較然著明,未有若故丞相平津侯公孫弘者也。位在丞相而為布被,脫粟之飯,不過一肉。故人所善賓客皆分奉祿以給之,無有所餘。誠內自克約而外從制。汲黯詰之,乃聞于朝,此可謂減於制度而可施行者也。德優則行,否則止,與內奢泰而外為詭服以釣虛譽者殊科。以病乞骸骨,孝武皇帝即制曰『賞有功,褒有德,善善惡惡,君宜知之。其省思慮,存精神,輔以醫藥』。賜告治病,牛酒雜帛。居數月,有瘳,視事。至元狩二年,竟以善終于相位。夫知臣莫若君,此其效也。弘子度嗣爵,后為山陽太守,坐法失侯。夫表德章義,所以率俗厲化,聖王之制,不易之道也。其賜弘後子孫之次當為後者爵關內侯,食邑三百戶,徵詣公車,上名尚書,朕親臨拜焉。」

班固稱曰:公孫弘、卜式、兒寬皆以鴻漸之翼困於燕雀,遠跡羊豕之閒,非遇其時,焉能致此位乎?是時漢興六十餘載,海內乂安,府庫充實,而四夷未賓,制度多闕,上方欲用文武,求之如弗及。始以蒲輪迎枚生,見主父而嘆息。群臣慕向,異人并出。卜式試於芻牧,弘羊擢於賈豎,衛青奮於奴仆,日磾出於降虜,斯亦曩時版筑飯牛之朋矣。漢之得人,於茲為盛。儒雅則公孫弘、董仲舒、兒寬,篤行則石建、石慶,質直則汲黯、卜式,推賢則韓安國、鄭當時,定令則趙禹、張湯,文章則司馬遷、相如,滑稽則東方朔、枚皋,應對則嚴助、朱買臣,歷數則唐都、落下閎,協律則李延年,運籌則桑弘羊,奉使則張騫、蘇武,將帥則衛青、霍去病,受遺則霍光、金日磾。其餘不可勝紀。是以興造功業,制度遺文,後世莫及。孝宣承統,纂修洪業,亦講論六藝,招選茂異,而蕭望之、梁丘賀、夏侯勝、韋玄成、嚴彭祖、尹更始以儒術進,劉向、王褒以文章顯。將相則張安世、趙充國、魏相、邴吉、于定國、杜延年,治民則黃霸、王成、龔遂、鄭弘、邵信臣、韓延壽、尹翁歸、趙廣漢之屬,皆有功跡見述於後。累其名臣,亦其次也。


\end{pinyinscope}