\article{袁盎鼂錯列傳}

\begin{pinyinscope}
袁盎者,楚人也,字絲。父故為群盜,徙處安陵。高后時,盎嘗為呂祿舍人。及孝文帝即位,盎兄噲任盎為中郎。

絳侯為丞相,朝罷趨出,意得甚。上禮之恭,常自送之。袁盎進曰:「陛下以丞相何如人?」上曰:「社稷臣。」盎曰:「絳侯所謂功臣,非社稷臣,社稷臣主在與在,主亡與亡。方呂后時,諸呂用事,擅相王,劉氏不絕如帶。是時絳侯為太尉,主兵柄,弗能正。呂后崩,大臣相與共畔諸呂,太尉主兵,適會其成功,所謂功臣,非社稷臣。丞相如有驕主色。陛下謙讓,臣主失禮,竊為陛下不取也。」後朝,上益莊,丞相益畏。已而絳侯望袁盎曰:「吾與而兄善,今兒廷毀我!」盎遂不謝。

及絳侯免相之國,國人上書告以為反,徵系清室,宗室諸公莫敢為言,唯袁盎明絳侯無罪。絳侯得釋,盎頗有力。絳侯乃大與盎結交。

淮南厲王朝,殺辟陽侯,居處驕甚。袁盎諫曰:「諸侯大驕必生患,可適削地。」上弗用。淮南王益橫。及棘蒲侯柴武太子謀反事覺,治,連淮南王,淮南王徵,上因遷之蜀,轞車傳送。袁盎時為中郎將,乃諫曰:「陛下素驕淮南王,弗稍禁,以至此,今又暴摧折之。淮南王為人剛,如有遇霧露行道死,陛下竟為以天下之大弗能容,有殺弟之名,柰何?」上弗聽,遂行之。

淮南王至雍,病死,聞,上輟食,哭甚哀。盎入,頓首請罪。上曰:「以不用公言至此。」盎曰:「上自寬,此往事,豈可悔哉!且陛下有高世之行者三,此不足以毀名。」上曰:「吾高世行三者何事?」盎曰:「陛下居代時,太后嘗病,三年,陛下不交睫,不解衣,湯藥非陛下口所嘗弗進。夫曾參以布衣猶難之,今陛下親以王者修之,過曾參孝遠矣。夫諸呂用事,大臣專制,然陛下從代乘六傳馳不測之淵,雖賁育之勇不及陛下。陛下至代邸,西向讓天子位者再,南面讓天子位者三。夫許由一讓,而陛下五以天下讓,過許由四矣。且陛下遷淮南王,欲以苦其志,使改過,有司衛不謹,故病死。」於是上乃解,曰:「將柰何?」盎曰:「淮南王有三子,唯在陛下耳。」於是文帝立其三子皆為王。盎由此名重朝廷。

袁盎常引大體慨。宦者趙同以數幸,常害袁盎,袁盎患之。盎兄子種為常侍騎,持節夾乘,說盎曰:「君與鬬,廷辱之,使其毀不用。」孝文帝出,趙同參乘,袁盎伏車前曰:「臣聞天子所與共六尺輿者,皆天下豪英。今漢雖乏人,陛下獨奈何與刀鋸餘人載!」於是上笑,下趙同。趙同泣下車。

文帝從霸陵上,欲西馳下峻阪。袁盎騎,并車擥轡。上曰:「將軍怯邪?」盎曰:「臣聞千金之子坐不垂堂,百金之子不騎衡,聖主不乘危而徼幸。今陛下騁六騑,馳下峻山,如有馬驚車敗,陛下縱自輕,柰高廟、太后何?」上乃止。

上幸上林,皇后、慎夫人從。其在禁中,常同席坐。及坐,郎署長布席,袁盎引卻慎夫人坐。慎夫人怒,不肯坐。上亦怒,起,入禁中。盎因前說曰:「臣聞尊卑有序則上下和。今陛下既已立后,慎夫人乃妾,妾主豈可與同坐哉!適所以失尊卑矣。且陛下幸之,即厚賜之。陛下所以為慎夫人,適所以禍之。陛下獨不見『人彘』乎?」於是上乃說,召語慎夫人。慎夫人賜盎金五十斤。

然袁盎亦以數直諫,不得久居中,調為隴西都尉。仁愛士卒,士卒皆爭為死。遷為齊相。徙為吳相,辭行,種謂盎曰:「吳王驕日久,國多姦。今茍欲劾治,彼不上書告君,即利劍刺君矣。南方卑溼,君能日飲,毋何,時說王曰毋反而已。如此幸得脫。」盎用種之計,吳王厚遇盎。

盎告歸,道逢丞相申屠嘉,下車拜謁,丞相從車上謝袁盎。袁盎還,愧其吏,乃之丞相舍上謁,求見丞相。丞相良久而見之。盎因跪曰:「願請閒。」丞相曰:「使君所言公事,之曹與長史掾議,吾且奏之;即私邪,吾不受私語。」袁盎即跪說曰:「君為丞相,自度孰與陳平、絳侯?」丞相曰:「吾不如。」袁盎曰:「善,君即自謂不如。夫陳平、絳侯輔翼高帝,定天下,為將相,而誅諸呂,存劉氏;君乃為材官蹶張,遷為隊率,積功至淮陽守,非有奇計攻城野戰之功。且陛下從代來,每朝,郎官上書疏,未嘗不止輦受其言,言不可用置之,言可受採之,未嘗不稱善。何也?則欲以致天下賢士大夫。上日聞所不聞,明所不知,日益聖智;君今自閉鉗天下之口而日益愚。夫以聖主責愚相,君受禍不久矣。」丞相乃再拜曰:「嘉鄙野人,乃不知,將軍幸教。」引入與坐,為上客。

盎素不好鼂錯,鼂錯所居坐,盎去;盎坐,錯亦去:兩人未嘗同堂語。及孝文帝崩,孝景帝即位,鼂錯為御史大夫,使吏案袁盎受吳王財物,抵罪,詔赦以為庶人。

吳楚反,聞,鼂錯謂丞史曰:「夫袁盎多受吳王金錢,專為蔽匿,言不反。今果反,欲請治盎宜知計謀。」丞史曰:「事未發,治之有絕。今兵西鄉,治之何益!且袁盎不宜有謀。」鼂錯猶與未決。人有告袁盎者,袁盎恐,夜見竇嬰,為言吳所以反者,願至上前口對狀。竇嬰入言上,上乃召袁盎入見。鼂錯在前,及盎請辟人賜閒,錯去,固恨甚。袁盎具言吳所以反狀,以錯故,獨急斬錯以謝吳,吳兵乃可罷。其語具在吳事中。使袁盎為太常,竇嬰為大將軍。兩人素相與善。逮吳反。諸陵長者長安中賢大夫爭附兩人,車隨者日數百乘。

及鼂錯已誅,袁盎以太常使吳。吳王欲使將,不肯。欲殺之,使一都尉以五百人圍守盎軍中。袁盎自其為吳相時,(嘗)有從史嘗盜愛盎侍兒,盎知之,弗泄,遇之如故。人有告從史,言「君知爾與侍者通」,乃亡歸。袁盎驅自追之,遂以侍者賜之,復為從史。及袁盎使吳見守,從史適為守盎校尉司馬,乃悉以其裝齎置二石醇醪,會天寒,士卒饑渴,飲酒醉,西南陬卒皆臥,司馬夜引袁盎起,曰:「君可以去矣,吳王期旦日斬君。」盎弗信,曰:「公何為者?」司馬曰:「臣故為從史盜君侍兒者。」盎乃驚謝曰;「公幸有親,吾不足以累公。」司馬曰:「君弟去,臣亦且亡,辟吾親,君何患!」乃以刀決張,道從醉卒(直)隧[直]出。司馬與分背,袁盎解節毛懷之,杖,步行七八里,明,見梁騎,騎馳去,遂歸報。

吳楚已破,上更以元王子平陸侯禮為楚王,袁盎為楚相。嘗上書有所言,不用。袁盎病免居家,與閭里浮沈,相隨行,鬬雞走狗。雒陽劇孟嘗過袁盎,盎善待之。安陵富人有謂盎曰:「吾聞劇孟博徒,將軍何自通之?」盎曰:「劇孟雖博徒,然母死,客送葬車千餘乘,此亦有過人者。且緩急人所有。夫一旦有急叩門,不以親為解,不以存亡為辭,天下所望者,獨季心、劇孟耳。今公常從數騎,一旦有緩急,寧足恃乎!」罵富人,弗與通。諸公聞之,皆多袁盎。

袁盎雖家居,景帝時時使人問籌策。梁王欲求為嗣,袁盎進說,其後語塞。梁王以此怨盎,曾使人刺盎。刺者至關中,問袁盎,諸君譽之皆不容口。乃見袁盎曰:「臣受梁王金來刺君,君長者,不忍刺君。然後刺君者十餘曹,備之!」袁盎心不樂,家又多怪,乃之棓生所問占。還,梁刺客後曹輩果遮刺殺盎安陵郭門外。

鼂錯者,潁川人也。學申商刑名於軹張恢先所,與雒陽宋孟及劉禮同師。以文學為太常掌故。

錯為人陗直刻深。孝文帝時,天下無治尚書者,獨聞濟南伏生故秦博士,治尚書,年九十餘,老不可徵,乃詔太常使人往受之。太常遣錯受尚書伏生所。還,因上便宜事,以書稱說。詔以為太子舍人、門大夫、家令。以其辯得幸太子,太子家號曰「智囊」。數上書孝文時,言削諸侯事,及法令可更定者。書數十上,孝文不聽,然奇其材,遷為中大夫。當是時,太子善錯計策,袁盎諸大功臣多不好錯。

景帝即位,以錯為內史。錯常數請閒言事,輒聽,寵幸傾九卿,法令多所更定。丞相申屠嘉心弗便,力未有以傷。內史府居太上廟壖中,門東出,不便,錯乃穿兩門南出,鑿廟壖垣。丞相嘉聞,大怒,欲因此過為奏請誅錯。錯聞之,即夜請閒,具為上言之。丞相奏事,因言錯擅鑿廟垣為門,請下廷尉誅。上曰:「此非廟垣,乃壖中垣,不致於法。」丞相謝。罷朝,怒謂長史曰:「吾當先斬以聞,乃先請,為兒所賣,固誤。」丞相遂發病死。錯以此愈貴。

遷為御史大夫,請諸侯之罪過,削其地,收其枝郡。奏上,上令公卿列侯宗室集議,莫敢難,獨竇嬰爭之,由此與錯有卻。錯所更令三十章,諸侯皆諠譁疾鼂錯。錯父聞之,從潁川來,謂錯曰:「上初即位,公為政用事,侵削諸侯,別疏人骨肉,人口議多怨公者,何也?」鼂錯曰:「固也。不如此,天子不尊,宗廟不安。」錯父曰:「劉氏安矣,而鼂氏危矣,吾去公歸矣!」遂飲藥死,曰:「吾不忍見禍及吾身。」死十餘日,吳楚七國果反,以誅錯為名。及竇嬰、袁盎進說,上令鼂錯衣朝衣斬東市。

鼂錯已死,謁者仆射鄧公為校尉,擊吳楚軍為將。還,上書言軍事,謁見上。上問曰:「道軍所來,聞鼂錯死,吳楚罷不?」鄧公曰:「吳王為反數十年矣,發怒削地,以誅錯為名,其意非在錯也。且臣恐天下之士噤口,不敢復言也!」上曰:「何哉?」鄧公曰:「夫鼂錯患諸侯彊大不可制,故請削地以尊京師,萬世之利也。計畫始行,卒受大戮,內杜忠臣之口,外為諸侯報仇,臣竊為陛下不取也。」於是景帝默然良久,曰:「公言善,吾亦恨之。」乃拜鄧公為城陽中尉。

鄧公,成固人也,多奇計。建元中,上招賢良,公卿言鄧公,時鄧公免,起家為九卿。一年,復謝病免歸。其子章以修黃老言顯於諸公閒。

太史公曰:袁盎雖不好學,亦善傅會,仁心為質,引義慨。遭孝文初立,資適逢世。時以變易,及吳楚一說,說雖行哉,然復不遂。好聲矜賢,竟以名敗。鼂錯為家令時,數言事不用;後擅權,多所變更。諸侯發難,不急匡救,欲報私讎,反以亡軀。語曰「變古亂常,不死則亡」,豈錯等謂邪!


\end{pinyinscope}