\article{張釋之馮唐列傳}

\begin{pinyinscope}
張廷尉釋之者,堵陽人也,字季。有兄仲同居。以訾為騎郎,事孝文帝,十歲不得調,無所知名。釋之曰:「久宦減仲之產,不遂。」欲自免歸。中郎將袁盎知其賢,惜其去,乃請徙釋之補謁者。釋之既朝畢,因前言便宜事。文帝曰:「卑之,毋甚高論,令今可施行也。」於是釋之言秦漢之閒事,秦所以失而漢所以興者久之。文帝稱善,乃拜釋之為謁者仆射。

釋之從行,登虎圈。上問上林尉諸禽獸簿,十餘問,尉左右視,盡不能對。虎圈嗇夫從旁代尉對上所問禽獸簿甚悉,欲以觀其能口對響應無窮者。文帝曰:「吏不當若是邪?尉無賴!」乃詔釋之拜嗇夫為上林令。釋之久之前曰:「陛下以絳侯周勃何如人也?」上曰:「長者也。」又復問:「東陽侯張相如何如人也?」上復曰:「長者。」釋之曰:「夫絳侯、東陽侯稱為長者,此兩人言事曾不能出口,豈斅此嗇夫諜諜利口捷給哉!且秦以任刀筆之吏,吏爭以亟疾苛察相高,然其敝徒文具耳,無惻隱之實。以故不聞其過,陵遲而至於二世,天下土崩。今陛下以嗇夫口辯而超遷之,臣恐天下隨風靡靡,爭為口辯而無其實。且下之化上疾於景響,舉錯不可不審也。」文帝曰:「善。」乃止不拜嗇夫。

上就車,召釋之參乘,徐行,問釋之秦之敝。具以質言。至宮,上拜釋之為公車令。

頃之,太子與梁王共車入朝,不下司馬門,於是釋之追止太子、梁王無得入殿門。遂劾不下公門不敬,奏之。薄太后聞之,文帝免冠謝曰:「教兒子不謹。」薄太后乃使使承詔赦太子、梁王,然後得入。文帝由是奇釋之,拜為中大夫。

頃之,至中郎將。從行至霸陵,居北臨廁。是時慎夫人從,上指示慎夫人新豐道,曰:「此走邯鄲道也。」使慎夫人鼓瑟,上自倚瑟而歌,意慘悽悲懷,顧謂群臣曰:「嗟乎!以北山石為槨,用紵絮斮陳,蕠漆其閒,豈可動哉!」左右皆曰:「善。」釋之前進曰:「使其中有可欲者,雖錮南山猶有郄;使其中無可欲者,雖無石槨,又何戚焉!」文帝稱善。其後拜釋之為廷尉。

頃之,上行出中渭橋,有一人從橋下走出,乘輿馬驚。於是使騎捕,屬之廷尉。釋之治問。曰:「縣人來,聞蹕,匿橋下。久之,以為行已過,即出,見乘輿車騎,即走耳。」廷尉秦當,一人犯蹕,當罰金。文帝怒曰:「此人親驚吾馬,吾馬賴柔和,令他馬,固不敗傷我乎?而廷尉乃當之罰金!」釋之曰:「法者天子所與天下公共也。今法如此而更重之,是法不信於民也。且方其時,上使立誅之則已。今既下廷尉,廷尉,天下之平也,一傾而天下用法皆為輕重,民安所措其手足?唯陛下察之。」良久,上曰:「廷尉當是也。」

其後有人盜高廟坐前玉環,捕得,文帝怒,下廷尉治。釋之案律盜宗廟服御物者為奏,奏當棄市。上大怒曰:「人之無道,乃盜先帝廟器,吾屬廷尉者,欲致之族,而君以法奏之,非吾所以共承宗廟意也。」釋之免冠頓首謝曰:「法如是足也。且罪等,然以逆順為差。今盜宗廟器而族之,有如萬分之一,假令愚民取長陵一抔土,陛下何以加其法乎?」久之,文帝與太后言之,乃許廷尉當。是時,中尉條侯周亞夫與梁相山都侯王恬開見釋之持議平,乃結為親友。張廷尉由此天下稱之。

後文帝崩,景帝立,釋之恐,稱病。欲免去,懼大誅至;欲見謝,則未知何如。用王生計,卒見謝,景帝不過也。

王生者,善為黃老言,處士也。嘗召居廷中,三公九卿盡會立,王生老人,曰「吾韤解」,顧謂張廷尉:「為我結韤!」釋之跪而結之。既已,人或謂王生曰:「獨柰何廷辱張廷尉,使跪結韤?」王生曰:「吾老且賤,自度終無益於張廷尉。張廷尉方今天下名臣,吾故聊辱廷尉,使跪結韤,欲以重之。」諸公聞之,賢王生而重張廷尉。

張廷尉事景帝歲餘,為淮南王相,猶尚以前過也。久之,釋之卒。其子曰張摯,字長公,官至大夫,免。以不能取容當世,故終身不仕。

馮唐者,其大父趙人。父徙代。漢興徙安陵。唐以孝著,為中郎署長,事文帝。文帝輦過,問唐曰:「父老何自為郎?家安在?」唐具以實對。文帝曰:「吾居代時,吾尚食監高袪數為我言趙將李齊之賢,戰於鉅鹿下。今吾每飯,意未嘗不在鉅鹿也。父知之乎?」唐對曰:「尚不如廉頗、李牧之為將也。」上曰:「何以?」唐曰:「臣大父在趙時,為官(卒)[率]將,善李牧。臣父故為代相,善趙將李齊,知其為人也。」上既聞廉頗、李牧為人,良說,而搏髀曰:「嗟乎!吾獨不得廉頗、李牧時為吾將,吾豈憂匈奴哉!」唐曰:「主臣!陛下雖得廉頗、李牧,弗能用也。」上怒,起入禁中。良久,召唐讓曰:「公柰何眾辱我,獨無閒處乎?」唐謝曰:「鄙人不知忌諱。」

當是之時,匈奴新大入朝那,殺北地都尉卬。上以胡寇為意,乃卒復問唐曰:「公何以知吾不能用廉頗、李牧也?」唐對曰:「臣聞上古王者之遣將也,跪而推轂,曰閫以內者,寡人制之;閫以外者,將軍制之。軍功爵賞皆決於外,歸而奏之。此非虛言也。臣大父言,李牧為趙將居邊,軍市之租皆自用饗士,賞賜決於外,不從中擾也。委任而責成功,故李牧乃得盡其智能,遣選車千三百乘,彀騎萬三千,百金之士十萬,是以北逐單于,破東胡,滅澹林,西抑彊秦,南支韓、魏。當是之時,趙幾霸。其後會趙王遷立,其母倡也。王遷立,乃用郭開讒,卒誅李牧,令顏聚代之。是以兵破士北,為秦所禽滅。今臣竊聞魏尚為雲中守,其軍市租盡以饗士卒,[出]私養錢,五日一椎牛,饗賓客軍吏舍人,是以匈奴遠避,不近雲中之塞。虜曾一入,尚率車騎擊之,所殺其眾。夫士卒盡家人子,起田中從軍,安知尺籍伍符。終日力戰,斬首捕虜,上功莫府,一言不相應,文吏以法繩之。其賞不行而吏奉法必用。臣愚,以為陛下法太明,賞太輕,罰太重。且雲中守魏尚坐上功首虜差六級,陛下下之吏,削其爵,罰作之。由此言之,陛下雖得廉頗、李牧,弗能用也。臣誠愚,觸忌諱,死罪死罪!」文帝說。是日令馮唐持節赦魏尚,復以為雲中守,而拜唐為車騎都尉,主中尉及郡國車士。

七年,景帝立,以唐為楚相,免。武帝立,求賢良,舉馮唐。唐時年九十餘,不能復為官,乃以唐子馮遂為郎。遂字王孫,亦奇士,與余善。

太史公曰:張季之言長者,守法不阿意;馮公之論將率,有味哉!有味哉!語曰「不知其人,視其友」。二君之所稱誦,可著廊廟。書曰「不偏不黨,王道蕩蕩;不黨不偏,王道便便」。張季、馮公近之矣。


\end{pinyinscope}