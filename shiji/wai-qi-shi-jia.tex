\article{外戚世家}

\begin{pinyinscope}
自古受命帝王及繼體守文之君,非獨內德茂也,蓋亦有外戚之助焉。夏之興也以涂山,而桀之放也以末喜。殷之興也以有娀,紂之殺也嬖妲己。周之興也以姜原及大任,而幽王之禽也淫於褒姒。故易基乾坤,詩始關雎,書美釐降,春秋譏不親迎。夫婦之際,人道之大倫也。禮之用,唯婚姻為兢兢。夫樂調而四時和,陰陽之變,萬物之統也。可不慎與?人能弘道,無如命何。甚哉,妃匹之愛,君不能得之於臣,父不能得之於子,況卑下乎!既驩合矣,或不能成子姓;能成子姓矣,或不能要其終:豈非命也哉?孔子罕稱命,蓋難言之也。非通幽明之變,惡能識乎性命哉?

太史公曰:秦以前尚略矣,其詳靡得而記焉。漢興,呂娥姁為高祖正后,男為太子。及晚節色衰愛弛,而戚夫人有寵,其子如意幾代太子者數矣。及高祖崩,呂后夷戚氏,誅趙王,而高祖後宮唯獨無寵疏遠者得無恙。

呂后長女為宣平侯張敖妻,敖女為孝惠皇后。呂太后以重親故,欲其生子萬方,終無子,詐取後宮人子為子。及孝惠帝崩,天下初定未久,繼嗣不明。於是貴外家,王諸呂以為輔,而以呂祿女為少帝后,欲連固根本牢甚,然無益也。

高后崩,合葬長陵。祿、產等懼誅,謀作亂。大臣征之,天誘其統,卒滅呂氏。唯獨置孝惠皇后居北宮。迎立代王,是為孝文帝,奉漢宗廟。此豈非天邪?非天命孰能當之?

薄太后,父吳人,姓薄氏,秦時與故魏王宗家女魏媼通,生薄姬,而薄案死山陰,因葬焉。

及諸侯畔秦,魏豹立為魏王,而魏媼內其女於魏宮。媼之許負所相,相薄姬,云當生天子。是時項羽方與漢王相距滎陽,天下未有所定。豹初與漢擊楚,及聞許負言,心獨喜,因背漢而畔,中立,更與楚連和。漢使曹參等擊虜魏王豹,以其國為郡,而薄姬輸織室。豹已死,漢王入織室,見薄姬有色,詔內後宮,歲餘不得幸。始姬少時,與管夫人、趙子兒相愛,約曰:「先貴無相忘。」已而管夫人、趙子兒先幸漢王。漢王坐河南宮成皋臺,此兩美人相與笑薄姬初時約。漢王聞之,問其故,兩人具以實告漢王。漢王心慘然,憐薄姬,是日召而幸之。薄姬曰:「昨暮夜妾夢蒼龍據吾腹。」高帝曰:「此貴徵也,吾為女遂成之。」一幸生男,是為代王。其後薄姬希見高祖。

高祖崩,諸御幸姬戚夫人之屬,呂太后怒,皆幽之,不得出宮。而薄姬以希見故,得出,從子之代,為代王太后。太后弟薄昭從如代。

代王立十七年,高后崩。大臣議立後,疾外家呂氏彊,皆稱薄氏仁善,故迎代王,立為孝文皇帝,而太后改號曰皇太后,弟薄昭封為軹侯。

薄太后母亦前死,葬櫟陽北。於是乃追尊薄案為靈文侯,會稽郡置園邑三百家,長丞已下吏奉守冢,寢廟上食祠如法。而櫟陽北亦置靈文侯夫人園,如靈文侯園儀。薄太后以為母家魏王後,早失父母,其奉薄太后諸魏有力者,於是召復魏氏,[及尊]賞賜各以親疏受之。薄氏侯者凡一人。

薄太后後文帝二年,以孝景帝前二年崩,葬南陵。以呂后會葬長陵,故特自起陵,近孝文皇帝霸陵。

竇太后,趙之清河觀津人也。呂太后時,竇姬以良家子入宮侍太后。太后出宮人以賜諸王,各五人,竇姬與在行中。竇姬家在清河,欲如趙近家,請其主遣宦者吏:「必置我籍趙之伍中。」宦者忘之,誤置其籍代伍中。籍奏,詔可,當行。竇姬涕泣,怨其宦者,不欲往,相彊,乃肯行。至代,代王獨幸竇姬,生女嫖,後生兩男。而代王王后生四男。先代王未入立為帝而王后卒。及代王立為帝,而王后所生四男更病死。孝文帝立數月,公卿請立太子,而竇姬長男最長,立為太子。立竇姬為皇后,女嫖為長公主。其明年,立少子武為代王,已而又徙梁,是為梁孝王。

竇皇后親蚤卒,葬觀津。於是薄太后乃詔有司,追尊竇后父為安成侯,母曰安成夫人。令清河置園邑二百家,長丞奉守,比靈文園法。

竇皇后兄竇長君,弟曰竇廣國,字少君。少君年四五歲時,家貧,為人所略賣,其家不知其處。傳十餘家,至宜陽,為其主入山作炭,(寒)[暮]臥岸下百餘人,岸崩,盡壓殺臥者,少君獨得脫,不死。自卜數日當為侯,從其家之長安。聞竇皇后新立,家在觀津,姓竇氏。廣國去時雖小,識其縣名及姓,又常與其姊採桑墮,用為符信,上書自陳。竇皇后言之於文帝,召見,問之,具言其故,果是。又復問他何以為驗?對曰:「姊去我西時,與我決於傳舍中,丐沐沐我,請食飯我,乃去。」於是竇后持之而泣,泣涕交橫下。侍御左右皆伏地泣,助皇后悲哀。乃厚賜田宅金錢,封公昆弟,家於長安。

絳侯、灌將軍等曰:「吾屬不死,命乃且縣此兩人。兩人所出微,不可不為擇師傅賓客,又復效呂氏大事也。」於是乃選長者士之有節行者與居。竇長君、少君由此為退讓君子,不敢以尊貴驕人。

竇皇后病,失明。文帝幸邯鄲慎夫人、尹姬,皆毋子。孝文帝崩,孝景帝立,乃封廣國為章武侯。長君前死,封其子彭祖為南皮侯。吳楚反時,竇太后從昆弟子竇嬰,任俠自喜,將兵,以軍功為魏其侯。竇氏凡三人為侯。

竇太后好黃帝、老子言,帝及太子諸竇不得不讀黃帝、老子,尊其術。

竇太后後孝景帝六歲(建元六年)崩,合葬霸陵。遺詔盡以東宮金錢財物賜長公主嫖。

王太后,槐裏人,母曰臧兒。臧兒者,故燕王臧荼孫也。臧兒嫁為槐裏王仲妻,生男曰信,與兩女。而仲死,臧兒更嫁長陵田氏,生男蚡、勝。臧兒長女嫁為金王孫婦,生一女矣,而臧兒卜筮之,曰兩女皆當貴。因欲奇兩女,乃奪金氏。金氏怒,不肯予決,乃內之太子宮。太子幸愛之,生三女一男。男方在身時,王美人夢日入其懷。以告太子,太子曰:「此貴徵也。」未生而孝文帝崩,孝景帝即位,王夫人生男。

先是臧兒又入其少女兒姁,兒姁生四男。

景帝為太子時,薄太后以薄氏女為妃。及景帝立,立妃曰薄皇后。皇后毋子,毋寵。薄太后崩,廢薄皇后。

景帝長男榮,其母栗姬。栗姬,齊人也。立榮為太子。長公主嫖有女,欲予為妃。栗姬妒,而景帝諸美人皆因長公主見景帝,得貴幸,皆過栗姬,栗姬日怨怒,謝長公主,不許。長公主欲予王夫人,王夫人許之。長公主怒,而日讒栗姬短於景帝曰:「栗姬與諸貴夫人幸姬會,常使侍者祝唾其背,挾邪媚道。」景帝以故望之。

景帝嘗體不安,心不樂,屬諸子為王者於栗姬,曰:「百歲後,善視之。」栗姬怒,不肯應,言不遜。景帝恚,心嗛之而未發也。

長公主日譽王夫人男之美,景帝亦賢之,又有曩者所夢日符,計未有所定。王夫人知帝望栗姬,因怒未解,陰使人趣大臣立栗姬為皇后。大行奏事畢,曰:「『子以母貴,母以子貴』,今太子母無號,宜立為皇后。」景帝怒曰:「是而所宜言邪!」遂案誅大行,而廢太子為臨江王。栗姬愈恚恨,不得見,以憂死。卒立王夫人為皇后,其男為太子,封皇后兄信為蓋侯。

景帝崩,太子襲號為皇帝。尊皇太后母臧兒為平原君。封田蚡為武安侯,勝為周陽侯。

景帝十三男,一男為帝,十二男皆為王。而兒姁早卒,其四子皆為王。王太后長女號日平陽公主,次為南宮公主,次為林慮公主。

蓋侯信好酒。田蚡、勝貪,巧於文辭。王仲蚤死,葬槐裏,追尊為共侯,置園邑二百家。及平原君卒,從田氏葬長陵,置園比共侯園。而王太后後孝景帝十六歲,以元朔四年崩,合葬陽陵。王太后家凡三人為侯。

衛皇后字子夫,生微矣。蓋其家號曰衛氏,出平陽侯邑。子夫為平陽主謳者。武帝初即位,數歲無子。平陽主求諸良家子女十餘人,飾置家。武帝祓霸上還,因過平陽主。主見所侍美人。上弗說。既飲,謳者進,上望見,獨說衛子夫。是日,武帝起更衣,子夫侍尚衣軒中,得幸。上還坐,驩甚。賜平陽主金千斤。主因奏子夫奉送入宮。子夫上車,平陽主拊其背曰:「行矣,彊飯,勉之!即貴,無相忘。」入宮歲餘,竟不復幸。武帝擇宮人不中用者,斥出歸之。衛子夫得見,涕泣請出。上憐之,復幸,遂有身,尊寵日隆。召其兄衛長君弟青為侍中。而子夫後大幸,有寵,凡生三女一男。男名據。

初,上為太子時,娶長公主女為妃。立為帝,妃立為皇后,姓陳氏,無子。上之得為嗣,大長公主有力焉,以故陳皇后驕貴。聞衛子夫大幸,恚,幾死者數矣。上愈怒。陳皇后挾婦人媚道,其事頗覺,於是廢陳皇后,而立衛子夫為皇后。

陳皇后母大長公主,景帝姊也,數讓武帝姊平陽公主曰:「帝非我不得立,已而棄捐吾女,壹何不自喜而倍本乎!」平陽公主曰:「用無子故廢耳。」陳皇后求子,與醫錢凡九千萬,然竟無子。

衛子夫已立為皇后,先是衛長君死,乃以衛青為將軍,擊胡有功,封為長平侯。青三子在襁褓中,皆封為列侯。及衛皇后所謂姊衛少兒,少兒生子霍去病,以軍功封冠軍侯,號驃騎將軍。青號大將軍。立衛皇后子據為太子。衛氏枝屬以軍功起家,五人為侯。

及衛后色衰,趙之王夫人幸,有子,為齊王。

王夫人蚤卒。而中山李夫人有寵,有男一人,為昌邑王。

李夫人蚤卒,其兄李延年以音幸,號協律。協律者,故倡也。兄弟皆坐姦,族。是時其長兄廣利為貳師將軍,伐大宛,不及誅,還,而上既夷李氏,後憐其家,乃封為海西侯。

他姬子二人為燕王、廣陵王。其母無寵,以憂死。

及李夫人卒,則有尹婕妤之屬,更有寵。然皆以倡見,非王侯有土之士女,不可以配人主也。

褚先生曰:臣為郎時,問習漢家故事者鐘離生。曰:王太后在民閒時所生(子)[一]女者,父為金王孫。王孫已死,景帝崩後,武帝已立,王太后獨在。而韓王孫名嫣素得幸武帝,承閒白言太后有女在長陵也。武帝曰:「何不蚤言!」乃使使往先視之,在其家。武帝乃自往迎取之。蹕道,先驅旄騎出橫城門,乘輿馳至長陵。當小市西入里,里門閉,暴開門,乘輿直入此里,通至金氏門外止,使武騎圍其宅,為其亡走,身自往取不得也。即使左右群臣入呼求之。家人驚恐,女亡匿內中床下。扶持出門,令拜謁。武帝下車泣曰:「嚄!大姊,何藏之深也!」詔副車載之,迴車馳還,而直入長樂宮。行詔門著引籍,通到謁太后。太后曰:「帝倦矣,何從來?」帝曰:「今者至長陵得臣姊,與俱來。」顧曰:「謁太后!」太后曰:「女某邪?」曰:「是也。」太后為下泣,女亦伏地泣。武帝奉酒前為壽,奉錢千萬,奴婢三百人,公田百頃,甲第,以賜姊。太后謝曰:「為帝費焉。」於是召平陽主、南宮主、林慮主三人俱來謁見姊,因號曰修成君。有子男一人,女一人。男號為修成子仲,女為諸侯王王后。此二子非劉氏,以故太后憐之。修成子仲驕恣,陵折吏民,皆患苦之。

衛子夫立為皇后,后弟衛青字仲卿,以大將軍封為長平侯。四子,長子伉為侯世子,侯世子常侍中,貴幸。其三弟皆封為侯,各千三百戶,一曰陰安侯,二曰發干侯,三曰宜春侯,貴震天下。天下歌之曰:「生男無喜,生女無怒,獨不見衛子夫霸天下!」

是時平陽主寡居,當用列侯尚主。主與左右議長安中列侯可為夫者,皆言大將軍可。主笑曰:「此出吾家,常使令騎從我出入耳,柰何用為夫乎?」左右侍御者曰:「今大將軍姊為皇后,三子為侯,富貴振動天下,主何以易之乎?」於是主乃許之。言之皇后,令白之武帝,乃詔衛將軍尚平陽公主焉。

褚先生曰:丈夫龍變。傳曰:「蛇化為龍,不變其文;家化為國,不變其姓。」丈夫當時富貴,百惡滅除,光耀榮華,貧賤之時何足累之哉!

武帝時,幸夫人尹婕妤。邢夫人號娙娥,眾人謂之「娙何」。娙何秩比中二千石,容華秩比二千石,婕妤秩比列侯。常從婕妤遷為皇后。

尹夫人與邢夫人同時并幸,有詔不得相見。尹夫人自請武帝,願望見邢夫人,帝許之。即令他夫人飾,從御者數十人,為邢夫人來前。尹夫人前見之,曰:「此非邢夫人身也。」帝曰:「何以言之?」對曰:「視其身貌形狀,不足以當人主矣。」於是帝乃詔使邢夫人衣故衣,獨身來前。尹夫人望見之,曰:「此真是也。」於是乃低頭俛而泣,自痛其不如也。諺曰:「美女入室,惡女之仇。」

褚先生曰:浴不必江海,要之去垢;馬不必騏驥,要之善走;士不必賢世,要之知道;女不必貴種,要之貞好。傳曰:「女無美惡,入室見妒;士無賢不肖,入朝見嫉。」美女者,惡女之仇。豈不然哉!

鉤弋夫人姓趙氏,河閒人也。得幸武帝,生子一人,昭帝是也。武帝年七十,乃生昭帝。昭帝立時,年五歲耳。

衛太子廢後,未復立太子。而燕王旦上書,願歸國入宿衛。武帝怒,立斬其使者於北闕。

上居甘泉宮,召畫工圖畫周公負成王也。於是左右群臣知武帝意欲立少子也。後數日,帝譴責鉤弋夫人。夫人脫簪珥叩頭。帝曰:「引持去,送掖庭獄!」夫人還顧,帝曰:「趣行,女不得活!」夫人死雲陽宮。時暴風揚塵,百姓感傷。使者夜持棺往葬之,封識其處。其後帝閒居,問左右曰:「人言云何?」左右對曰:「人言且立其子,何去其母乎?」帝曰:「然。是非兒曹愚人所知也。往古國家所以亂也,由主少母壯也。女主獨居驕蹇,淫亂自恣,莫能禁也。女不聞呂后邪?」故諸為武帝生子者,無男女,其母無不譴死,豈可謂非賢聖哉!昭然遠見,為後世計慮,固非淺聞愚儒之所及也。謚為「武」,豈虛哉!


\end{pinyinscope}