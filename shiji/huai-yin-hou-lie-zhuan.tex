\article{淮陰侯列傳}

\begin{pinyinscope}
淮陰侯韓信者,淮陰人也。始為布衣時,貧無行,不得推擇為吏,又不能治生商賈,常從人寄食飲,人多厭之者,常數從其下鄉南昌亭長寄食,數月,亭長妻患之,乃晨炊蓐食。食時信往,不為具食。信亦知其意,怒,竟絕去。

信釣於城下,諸母漂,有一母見信饑,飯信,竟漂數十日。信喜,謂漂母曰:「吾必有以重報母。」母怒曰:「大丈夫不能自食,吾哀王孫而進食,豈望報乎!」

淮陰屠中少年有侮信者,曰:「若雖長大,好帶刀劍,中情怯耳。」眾辱之曰:「信能死,刺我;不能死,出我袴下。」於是信孰視之,俛出袴下,蒲伏。一市人皆笑信,以為怯。

及項梁渡淮,信杖劍從之,居戲下,無所知名。項梁敗,又屬項羽,羽以為郎中。數以策干項羽,羽不用。漢王之入蜀,信亡楚歸漢,未得知名,為連敖。坐法當斬,其輩十三人皆已斬,次至信,信乃仰視,適見滕公,曰:「上不欲就天下乎?何為斬壯士!」滕公奇其言,壯其貌,釋而不斬。與語,大說之。言於上,上拜以為治粟都尉,上未之奇也。

信數與蕭何語,何奇之。至南鄭,諸將行道亡者數十人,信度何等已數言上,上不我用,即亡。何聞信亡,不及以聞,自追之。人有言上曰:「丞相何亡。」上大怒,如失左右手。居一二日,何來謁上,上且怒且喜,罵何曰:「若亡,何也?」何曰:「臣不敢亡也,臣追亡者。」上曰:「若所追者誰何?」曰:「韓信也。」上復罵曰:「諸將亡者以十數,公無所追;追信,詐也。」何曰:「諸將易得耳。至如信者,國士無雙。王必欲長王漢中,無所事信;必欲爭天下,非信無所與計事者。顧王策安所決耳。」王曰:「吾亦欲東耳,安能郁郁久居此乎?」何曰:「王計必欲東,能用信,信即留;不能用,信終亡耳。」王曰:「吾為公以為將。」何曰:「雖為將,信必不留。」王曰:「以為大將。」何曰:「幸甚。」於是王欲召信拜之。何曰:「王素慢無禮,今拜大將如呼小兒耳,此乃信所以去也。王必欲拜之,擇良日,齋戒,設壇場,具禮,乃可耳。」王許之。諸將皆喜,人人各自以為得大將。至拜大將,乃韓信也,一軍皆驚。

信拜禮畢,上坐。王曰:「丞相數言將軍,將軍何以教寡人計策?」信謝,因問王曰:「今東鄉爭權天下,豈非項王邪?」漢王曰:「然。」曰:「大王自料勇悍仁彊孰與項王?」漢王默然良久,曰:「不如也。」信再拜賀曰:「惟信亦為大王不如也。然臣嘗事之,請言項王之為人也。項王喑噁叱,千人皆廢,然不能任屬賢將,此特匹夫之勇耳。項王見人恭敬慈愛,言語嘔嘔,人有疾病,涕泣分食飲,至使人有功當封爵者,印刓敝,忍不能予,此所謂婦人之仁也。項王雖霸天下而臣諸侯,不居關中而都彭城。有背義帝之約,而以親愛王,諸侯不平。諸侯之見項王遷逐義帝置江南,亦皆歸逐其主而自王善地。項王所過無不殘滅者,天下多怨,百姓不親附,特劫於威彊耳。名雖為霸,實失天下心。故曰其彊易弱。今大王誠能反其道:任天下武勇,何所不誅!以天下城邑封功臣,何所不服!以義兵從思東歸之士,何所不散!且三秦王為秦將,將秦子弟數歲矣,所殺亡不可勝計,又欺其眾降諸侯,至新安,項王詐阬秦降卒二十餘萬,唯獨邯、欣、翳得脫,秦父兄怨此三人,痛入骨髓。今楚彊以威王此三人,秦民莫愛也。大王之入武關,秋豪無所害,除秦苛法,與秦民約,法三章耳,秦民無不欲得大王王秦者。於諸侯之約,大王當王關中,關中民咸知之。大王失職入漢中,秦民無不恨者。今大王舉而東,三秦可傳檄而定也。」於是漢王大喜,自以為得信晚。遂聽信計,部署諸將所擊。

八月,漢王舉兵東出陳倉,定三秦。漢二年,出關,收魏、河南,韓、殷王皆降。合齊、趙共擊楚。四月,至彭城,漢兵敗散而還。信復收兵與漢王會滎陽,復擊破楚京、索之閒,以故楚兵卒不能西。

漢之敗卻彭城,塞王欣、翟王翳亡漢降楚,齊、趙亦反漢與楚和。六月,魏王豹謁歸視親疾,至國,即絕河關反漢,與楚約和。漢王使酈生說豹,不下。其八月,以信為左丞相,擊魏。魏王盛兵蒲阪,塞臨晉,信乃益為疑兵,陳船欲度臨晉,而伏兵從夏陽以木罌缻渡軍,襲安邑。魏王豹驚,引兵迎信,信遂虜豹,定魏為河東郡。漢王遣張耳與信俱,引兵東,北擊趙、代。後九月,破代兵,禽夏說閼與。信之下魏破代,漢輒使人收其精兵,詣滎陽以距楚。

信與張耳以兵數萬,欲東下井陘擊趙。趙王、成安君陳餘聞漢且襲之也,聚兵井陘口,號稱二十萬。廣武君李左車說成安君曰:「聞漢將韓信涉西河,虜魏王,禽夏說,新喋血閼與,今乃輔以張耳,議欲下趙,此乘勝而去國遠鬬,其鋒不可當。臣聞千里餽糧,士有饑色,樵蘇後爨,師不宿飽。今井陘之道,車不得方軌,騎不得成列,行數百里,其勢糧食必在其後。願足下假臣奇兵三萬人,從閒道絕其輜重;足下深溝高壘,堅營勿與戰。彼前不得鬬,退不得還,吾奇兵絕其後,使野無所掠,不至十日,而兩將之頭可致於戲下。願君留意臣之計。否,必為二子所禽矣。」成安君,儒者也,常稱義兵不用詐謀奇計,曰:「吾聞兵法十則圍之,倍則戰。今韓信兵號數萬,其實不過數千。能千里而襲我,亦已罷極。今如此避而不擊,後有大者,何以加之!則諸侯謂吾怯,而輕來伐我。」不聽廣武君策,廣武君策不用。

韓信使人閒視,知其不用,還報,則大喜,乃敢引兵遂下。未至井陘口三十里,止舍。夜半傳發,選輕騎二千人,人持一赤幟,從閒道萆山而望趙軍,誡曰:「趙見我走,必空壁逐我,若疾入趙壁,拔趙幟,立漢赤幟。」令其裨將傳飱,曰:「今日破趙會食!」諸將皆莫信,詳應曰:「諾。」謂軍吏曰:「趙已先據便地為壁,且彼未見吾大將旗鼓,未肯擊前行,恐吾至阻險而還。」信乃使萬人先行,出,背水陳。趙軍望見而大笑。平旦,信建大將之旗鼓,鼓行出井陘口,趙開壁擊之,大戰良久。於是信、張耳詳棄鼓旗,走水上軍。水上軍開入之,復疾戰。趙果空壁爭漢鼓旗,逐韓信、張耳。韓信、張耳已入水上軍,軍皆殊死戰,不可敗。信所出奇兵二千騎,共候趙空壁逐利,則馳入趙壁,皆拔趙旗,立漢赤幟二千。趙軍已不勝,不能得信等,欲還歸壁,壁皆漢赤幟,而大驚,以為漢皆已得趙王將矣,兵遂亂,遁走,趙將雖斬之,不能禁也。於是漢兵夾擊,大破虜趙軍,斬成安君泜水上,禽趙王歇。

信乃令軍中毋殺廣武君,有能生得者購千金。於是有縛廣武君而致戲下者,信乃解其縛,東鄉坐,西鄉對,師事之。諸將效首虜,(休)畢賀,因問信曰:「兵法右倍山陵,前左水澤,今者將軍令臣等反背水陳,曰破趙會食,臣等不服。然竟以勝,此何術也?」信曰:「此在兵法,顧諸君不察耳。兵法不曰『陷之死地而後生,置之亡地而後存』?且信非得素拊循士大夫也,此所謂『驅市人而戰之』,其勢非置之死地,使人人自為戰;今予之生地,皆走,寧尚可得而用之乎!」諸將皆服曰:「善。非臣所及也。」

於是信問廣武君曰:「仆欲北攻燕,東伐齊,何若而有功?」廣武君辭謝曰:「臣聞敗軍之將,不可以言勇,亡國之大夫,不可以圖存。今臣敗亡之虜,何足以權大事乎!」信曰:「仆聞之,百里奚居虞而虞亡,在秦而秦霸,非愚於虞而智於秦也,用與不用,聽與不聽也。誠令成安君聽足下計,若信者亦已為禽矣。以不用足下,故信得侍耳。」因固問曰:「仆委心歸計,願足下勿辭。」廣武君曰:「臣聞智者千慮,必有一失;愚者千慮,必有一得。故曰『狂夫之言,聖人擇焉』。顧恐臣計未必足用,願效愚忠。夫成安君有百戰百勝之計,一旦而失之,軍敗鄗下,身死泜上。今將軍涉西河,虜魏王,禽夏說閼與,一舉而下井陘,不終朝破趙二十萬眾,誅成安君。名聞海內,威震天下,農夫莫不輟耕釋耒,褕衣甘食,傾耳以待命者。若此,將軍之所長也。然而眾勞卒罷,其實難用。今將軍欲舉倦獘之兵,頓之燕堅城之下,欲戰恐久力不能拔,情見勢屈,曠日糧竭,而弱燕不服,齊必距境以自彊也。燕齊相持而不下,則劉項之權未有所分也。若此者,將軍所短也。臣愚,竊以為亦過矣。故善用兵者不以短擊長,而以長擊短。」韓信曰:「然則何由?」廣武君對曰:「方今為將軍計,莫如案甲休兵,鎮趙撫其孤,百里之內,牛酒日至,以饗士大夫醳兵,北首燕路,而後遣辯士奉咫尺之書,暴其所長於燕,燕必不敢不聽從。燕已從,使諠言者東告齊,齊必從風而服,雖有智者,亦不知為齊計矣。如是,則天下事皆可圖也。兵固有先聲而後實者,此之謂也。」韓信曰:「善。」從其策,發使使燕,燕從風而靡。乃遣使報漢,因請立張耳為趙王,以鎮撫其國。漢王許之,乃立張耳為趙王。

楚數使奇兵渡河擊趙,趙王耳、韓信往來救趙,因行定趙城邑,發兵詣漢。楚方急圍漢王於滎陽,漢王南出,之宛、葉閒,得黥布,走入成皋,楚又復急圍之。六月,漢王出成皋,東渡河,獨與滕公俱,從張耳軍修武。至,宿傳舍。晨自稱漢使,馳入趙壁。張耳、韓信未起,即其臥內上奪其印符,以麾召諸將,易置之。信、耳起,乃知漢王來,大驚。漢王奪兩人軍,即令張耳備守趙地。拜韓信為相國,收趙兵未發者擊齊。

信引兵東,未渡平原,聞漢王使酈食其已說下齊,韓信欲止。范陽辯士蒯通說信曰:「將軍受詔擊齊,而漢獨發閒使下齊,寧有詔止將軍乎?何以得毋行也!且酈生一士,伏軾掉三寸之舌,下齊七十餘城,將軍將數萬眾,歲餘乃下趙五十餘,為將數歲,反不如一豎儒之功乎?」於是信然之,從其計,遂渡河。齊已聽酈生,即留縱酒,罷備漢守御。信因襲齊歷下軍,遂至臨菑。齊王田廣以酈生賣己,乃亨之,而走高密,使使之楚請救。韓信已定臨菑,遂東追廣至高密西。楚亦使龍且將,號稱二十萬,救齊。

齊王廣、龍且并軍與信戰,未合。人或說龍且曰:「漢兵遠鬬窮戰,其鋒不可當。齊、楚自居其地戰,兵易敗散。不如深壁,令齊王使其信臣招所亡城,亡城聞其王在,楚來救,必反漢。漢兵二千里客居,齊城皆反之,其勢無所得食,可無戰而降也。」龍且曰:「吾平生知韓信為人,易與耳。且夫救齊不戰而降之,吾何功?今戰而勝之,齊之半可得,何為止!」遂戰,與信夾濰水陳。韓信乃夜令人為萬餘囊,滿盛沙,壅水上流,引軍半渡,擊龍且,詳不勝,還走。龍且果喜曰:「固知信怯也。」遂追信渡水。信使人決壅囊,水大至。龍且軍大半不得渡,即急擊,殺龍且。龍且水東軍散走,齊王廣亡去。信遂追北至城陽,皆虜楚卒。

漢四年,遂皆降平齊。使人言漢王曰:「齊偽詐多變,反覆之國也,南邊楚,不為假王以鎮之,其勢不定。願為假王便。」當是時,楚方急圍漢王於滎陽,韓信使者至,發書,漢王大怒,罵曰:「吾困於此,旦暮望若來佐我,乃欲自立為王!」張良、陳平躡漢王足,因附耳語曰:「漢方不利,寧能禁信之王乎?不如因而立,善遇之,使自為守。不然,變生。」漢王亦悟,因復罵曰:「大丈夫定諸侯,即為真王耳,何以假為!」乃遣張良往立信為齊王,徵其兵擊楚。

楚已亡龍且,項王恐,使盱眙人武涉往說齊王信曰:「天下共苦秦久矣,相與力擊秦。秦已破,計功割地,分土而王之,以休士卒。今漢王復興兵而東,侵人之分,奪人之地,已破三秦,引兵出關,收諸侯之兵以東擊楚,其意非盡吞天下者不休,其不知厭足如是甚也。且漢王不可必,身居項王掌握中數矣,項王憐而活之,然得脫,輒倍約,復擊項王,其不可親信如此。今足下雖自以與漢王為厚交,為之盡力用兵,終為之所禽矣。足下所以得須臾至今者,以項王尚存也。當今二王之事,權在足下。足下右投則漢王勝,左投則項王勝。項王今日亡,則次取足下。足下與項王有故,何不反漢與楚連和,參分天下王之?今釋此時,而自必於漢以擊楚,且為智者固若此乎!」韓信謝曰:「臣事項王,官不過郎中,位不過執戟,言不聽,畫不用,故倍楚而歸漢。漢王授我上將軍印,予我數萬眾,解衣衣我,推食食我,言聽計用,故吾得以至於此。夫人深親信我,我倍之不祥,雖死不易。幸為信謝項王!」

武涉已去,齊人蒯通知天下權在韓信,欲為奇策而感動之,以相人說韓信曰:「仆嘗受相人之術。」韓信曰:「先生相人何如?」對曰:「貴賤在於骨法,憂喜在於容色,成敗在於決斷,以此參之,萬不失一。」韓信曰:「善。先生相寡人何如?」對曰:「願少閒。」信曰:「左右去矣。」通曰:「相君之面,不過封侯,又危不安。相君之背,貴乃不可言。」韓信曰:「何謂也?」蒯通曰:「天下初發難也,俊雄豪桀建號壹呼,天下之士雲合霧集,魚鱗襍遝,熛至風起。當此之時,憂在亡秦而已。今楚漢分爭,使天下無罪之人肝膽涂地,父子暴骸骨於中野,不可勝數。楚人起彭城,轉鬬逐北,至於滎陽,乘利席卷,威震天下。然兵困於京、索之閒,迫西山而不能進者,三年於此矣。漢王將數十萬之眾,距鞏、雒,阻山河之險,一日數戰,無尺寸之功,折北不救,敗滎陽,傷成皋,遂走宛、葉之閒,此所謂智勇俱困者也。夫銳氣挫於險塞,而糧食竭於內府,百姓罷極怨望,容容無所倚。以臣料之,其勢非天下之賢聖固不能息天下之禍。當今兩主之命縣於足下。足下為漢則漢勝,與楚則楚勝。臣願披腹心,輸肝膽,效愚計,恐足下不能用也。誠能聽臣之計,莫若兩利而俱存之,參分天下,鼎足而居,其勢莫敢先動。夫以足下之賢聖,有甲兵之眾,據彊齊,從燕、趙,出空虛之地而制其後,因民之欲,西鄉為百姓請命,則天下風走而響應矣,孰敢不聽!邦大弱彊,以立諸侯,諸侯已立,天下服聽而歸德於齊。案齊之故,有膠、泗之地,懷諸侯以德,深拱揖讓,則天下之君王相率而朝於齊矣。蓋聞天與弗取,反受其咎;時至不行,反受其殃。願足下孰慮之。」

韓信曰:「漢王遇我甚厚,載我以其車,衣我以其衣,食我以其食。吾聞之,乘人之車者載人之患,衣人之衣者懷人之憂,食人之食者死人之事,吾豈可以鄉利倍義乎!」蒯生曰:「足下自以為善漢王,欲建萬世之業,臣竊以為誤矣。始常山王、成安君為布衣時,相與為刎頸之交,後爭張黶、陳澤之事,二人相怨。常山王背項王,奉項嬰頭而竄,逃歸於漢王。漢王借兵而東下,殺成安君泜水之南,頭足異處,卒為天下笑。此二人相與,天下至驩也。然而卒相禽者,何也?患生於多欲而人心難測也。今足下欲行忠信以交於漢王,必不能固於二君之相與也,而事多大於張黶、陳澤。故臣以為足下必漢王之不危己,亦誤矣。大夫種、范蠡存亡越,霸句踐,立功成名而身死亡。野獸已盡而獵狗亨。夫以交友言之,則不如張耳之與成安君者也;以忠信言之,則不過大夫種、范蠡之於句踐也。此二人者,足以觀矣。願足下深慮之。且臣聞勇略震主者身危,而功蓋天下者不賞。臣請言大王功略:足下涉西河,虜魏王,禽夏說,引兵下井陘,誅成安君,徇趙,脅燕,定齊,南摧楚人之兵二十萬,東殺龍且,西鄉以報,此所謂功無二於天下,而略不世出者也。今足下戴震主之威,挾不賞之功,歸楚,楚人不信;歸漢,漢人震恐:足下欲持是安歸乎?夫勢在人臣之位而有震主之威,名高天下,竊為足下危之。」韓信謝曰:「先生且休矣,吾將念之。」

後數日,蒯通復說曰:「夫聽者事之候也,計者事之機也,聽過計失而能久安者,鮮矣。聽不失一二者,不可亂以言;計不失本末者,不可紛以辭。夫隨廝養之役者,失萬乘之權;守儋石之祿者,闕卿相之位。故知者決之斷也,疑者事之害也,審豪氂之小計,遺天下之大數,智誠知之,決弗敢行者,百事之禍也。故曰『猛虎之猶豫,不若蜂蠆之致螫;騏驥之跼躅,不如駑馬之安步;孟賁之狐疑,不如庸夫之必至也;雖有舜禹之智,吟而不言,不如瘖聾之指麾也』。此言貴能行之。夫功者難成而易敗,時者難得而易失也。時乎時,不再來。願足下詳察之。」韓信猶豫不忍倍漢,又自以為功多,漢終不奪我齊,遂謝蒯通。蒯通說不聽,已詳狂為巫。

漢王之困固陵,用張良計,召齊王信,遂將兵會垓下。項羽已破,高祖襲奪齊王軍。漢五年正月,徙齊王信為楚王,都下邳。

信至國,召所從食漂母,賜千金。及下鄉南昌亭長,賜百錢,曰:「公,小人也,為德不卒。」召辱己之少年令出胯下者以為楚中尉。告諸將相曰:「此壯士也。方辱我時,我寧不能殺之邪?殺之無名,故忍而就於此。」

項王亡將鐘離眛家在伊廬,素與信善。項王死後,亡歸信。漢王怨眛,聞其在楚,詔楚捕眛。信初之國,行縣邑,陳兵出入。漢六年,人有上書告楚王信反。高帝以陳平計,天子巡狩會諸侯,南方有雲夢,發使告諸侯會陳:「吾將游雲夢。」實欲襲信,信弗知。高祖且至楚,信欲發兵反,自度無罪,欲謁上,恐見禽。人或說信曰:「斬眛謁上,上必喜,無患。」信見眛計事。眛曰:「漢所以不擊取楚,以眛在公所。若欲捕我以自媚於漢,吾今日死,公亦隨手亡矣。」乃罵信曰:「公非長者!」卒自剄。信持其首,謁高祖於陳。上令武士縛信,載後車。信曰:「果若人言,『狡兔死,良狗亨;高鳥盡,良弓藏;敵國破,謀臣亡。』天下已定,我固當亨!」上曰:「人告公反。」遂械系信。至雒陽,赦信罪,以為淮陰侯。

信知漢王畏惡其能,常稱病不朝從。信由此日夜怨望,居常鞅鞅,羞與絳、灌等列。信嘗過樊將軍噲,噲跪拜送迎,言稱臣,曰:「大王乃肯臨臣!」信出門,笑曰:「生乃與噲等為伍!」上常從容與信言諸將能不,各有差。上問曰:「如我能將幾何?」信曰:「陛下不過能將十萬。」上曰:「於君何如?」曰:「臣多多而益善耳。」上笑曰:「多多益善,何為為我禽?」信曰:「陛下不能將兵,而善將將,此乃言之所以為陛下禽也。且陛下所謂天授,非人力也。」

陳豨拜為鉅鹿守,辭於淮陰侯。淮陰侯挈其手,辟左右與之步於庭,仰天嘆曰:「子可與言乎?欲與子有言也。」豨曰:「唯將軍令之。」淮陰侯曰:「公之所居,天下精兵處也;而公,陛下之信幸臣也。人言公之畔,陛下必不信;再至,陛下乃疑矣;三至,必怒而自將。吾為公從中起,天下可圖也。」陳豨素知其能也,信之,曰:「謹奉教!」漢十年,陳豨果反。上自將而往,信病不從。陰使人至豨所,曰:「弟舉兵,吾從此助公。」信乃謀與家臣夜詐詔赦諸官徒奴,欲發以襲呂后、太子。部署已定,待豨報。其舍人得罪於信,信囚,欲殺之。舍人弟上變,告信欲反狀於呂后。呂后欲召,恐其黨不就,乃與蕭相國謀,詐令人從上所來,言豨已得死,列侯群臣皆賀。相國紿信曰:「雖疾,彊入賀。」信入,呂后使武士縛信,斬之長樂鐘室。信方斬,曰:「吾悔不用蒯通之計,乃為兒女子所詐,豈非天哉!」遂夷信三族。

高祖已從豨軍來,至,見信死,且喜且憐之,問:「信死亦何言?」呂后曰:「信言恨不用蒯通計。」高祖曰:「是齊辯士也。」乃詔齊捕蒯通。蒯通至,上曰:「若教淮陰侯反乎?」對曰:「然,臣固教之。豎子不用臣之策,故令自夷於此。如彼豎子用臣之計,陛下安得而夷之乎!」上怒曰:「亨之。」通曰:「嗟乎,冤哉亨也!」上曰:「若教韓信反,何冤?」對曰:「秦之綱絕而維弛,山東大擾,異姓并起,英俊烏集。秦失其鹿,天下共逐之,於是高材疾足者先得焉。蹠之狗吠堯,堯非不仁,狗因吠非其主。當是時,臣唯獨知韓信,非知陛下也。且天下銳精持鋒欲為陛下所為者甚眾,顧力不能耳。又可盡亨之邪?」高帝曰:「置之。」乃釋通之罪。

太史公曰:吾如淮陰,淮陰人為余言,韓信雖為布衣時,其志與眾異。其母死,貧無以葬,然乃行營高敞地,令其旁可置萬家。余視其母冢,良然。假令韓信學道謙讓,不伐己功,不矜其能,則庶幾哉,於漢家勳可以比周、召、太公之徒,後世血食矣。不務出此,而天下已集,乃謀畔逆,夷滅宗族,不亦宜乎!


\end{pinyinscope}