\article{大宛列傳}

\begin{pinyinscope}
大宛之跡,見自張騫。張騫,漢中人。建元中為郎。是時天子問匈奴降者,皆言匈奴破月氏王,以其頭為飲器,月氏遁逃而常怨仇匈奴,無與共擊之。漢方欲事滅胡,聞此言,因欲通使。道必更匈奴中,乃募能使者。騫以郎應募,使月氏,與堂邑氏(故)胡奴甘父俱出隴西。經匈奴,匈奴得之,傳詣單于。單于留之,曰:「月氏在吾北,漢何以得往使?吾欲使越,漢肯聽我乎?」留騫十餘歲,與妻,有子,然騫持漢節不失。

居匈奴中,益寬,騫因與其屬亡鄉月氏,西走數十日至大宛。大宛聞漢之饒財,欲通不得,見騫,喜,問曰:「若欲何之?」騫曰:「為漢使月氏,而為匈奴所閉道。今亡,唯王使人導送我。誠得至,反漢,漢之賂遺王財物不可勝言。」大宛以為然,遣騫,為發導繹,抵康居,康居傳致大月氏。大月氏王已為胡所殺,立其太子為王。既臣大夏而居,地肥饒,少寇,志安樂,又自以遠漢,殊無報胡之心。騫從月氏至大夏,竟不能得月氏要領。

留歲餘,還,并南山,欲從羌中歸,復為匈奴所得。留歲餘,單于死,左谷蠡王攻其太子自立,國內亂,騫與胡妻及堂邑父俱亡歸漢。漢拜騫為太中大夫,堂邑父為奉使君。

騫為人彊力,寬大信人,蠻夷愛之。堂邑父故胡人,善射,窮急射禽獸給食。初,騫行時百餘人,去十三歲,唯二人得還。

騫身所至者大宛、大月氏、大夏、康居,而傳聞其旁大國五六,具為天子言之。曰:

大宛在匈奴西南,在漢正西,去漢可萬里。其俗土著,耕田,田稻麥。有蒲陶酒。多善馬,馬汗血,其先天馬子也。有城郭屋室。其屬邑大小七十餘城,眾可數十萬。其兵弓矛騎射。其北則康居,西則大月氏,西南則大夏,東北則烏孫,東則扜穼、于窴。于窴之西,則水皆西流,注西海;其東水東流,注鹽澤。鹽澤潛行地下,其南則河源出焉。多玉石,河注中國。而樓蘭、姑師邑有城郭,臨鹽澤。鹽澤去長安可五千里。匈奴右方居鹽澤以東,至隴西長城,南接羌,鬲漢道焉。

烏孫在大宛東北可二千里,行國,隨畜,與匈奴同俗。控弦者數萬,敢戰。故服匈奴,及盛,取其羈屬,不肯往朝會焉。

康居在大宛西北可二千里,行國,與月氏大同俗。控弦者八九萬人。與大宛鄰國。國小,南羈事月氏,東羈事匈奴。

奄蔡在康居西北可二千里,行國,與康居大同俗。控弦者十餘萬。臨大澤,無崖,蓋乃北海云。

大月氏在大宛西可二三千里,居媯水北。其南則大夏,西則安息,北則康居。行國也,隨畜移徙,與匈奴同俗。控弦者可一二十萬。故時彊,輕匈奴,及冒頓立,攻破月氏,至匈奴老上單于,殺月氏王,以其頭為飲器。始月氏居敦煌、祁連閒,及為匈奴所敗,乃遠去,過宛,西擊大夏而臣之,遂都媯水北,為王庭。其餘小眾不能去者,保南山羌,號小月氏。

安息在大月氏西可數千里。其俗土著,耕田,田稻麥,蒲陶酒。城邑如大宛。其屬小大數百城,地方數千里,最為大國。臨媯水,有市,民商賈用車及船,行旁國或數千里。以銀為錢,錢如其王面,王死輒更錢,效王面焉。畫革旁行以為書記。其西則條枝,北有奄蔡、黎軒。

條枝在安息西數千里,臨西海。暑溼。耕田,田稻。有大鳥,卵如甕。人眾甚多,往往有小君長,而安息役屬之,以為外國。國善眩。安息長老傳聞條枝有弱水、西王母,而未嘗見。

大夏在大宛西南二千餘里媯水南。其俗土著,有城屋,與大宛同俗。無大(王)[君]長,往往城邑置小長。其兵弱,畏戰。善賈市。及大月氏西徙,攻敗之,皆臣畜大夏。大夏民多,可百餘萬。其都曰藍市城,有市販賈諸物。其東南有身毒國。

騫曰:「臣在大夏時,見邛竹杖、蜀布。問曰:『安得此?』大夏國人曰:『吾賈人往市之身毒。身毒在大夏東南可數千里。其俗土著,大與大夏同,而卑溼暑熱云。其人民乘象以戰。其國臨大水焉。』以騫度之,大夏去漢萬二千里,居漢西南。今身毒國又居大夏東南數千里,有蜀物,此其去蜀不遠矣。今使大夏,從羌中,險,羌人惡之;少北,則為匈奴所得;從蜀宜徑,又無寇。」天子既聞大宛及大夏、安息之屬皆大國,多奇物,土著,頗與中國同業,而兵弱,貴漢財物;其北有大月氏、康居之屬,兵彊,可以賂遺設利朝也。且誠得而以義屬之,則廣地萬里,重九譯,致殊俗,威德遍於四海。天子欣然,以騫言為然,乃令騫因蜀犍為發閒使,四道并出:出駹,出冉,出徙,出邛、僰,皆各行一二千里。其北方閉氐、筰,南方閉巂、昆明。昆明之屬無君長,善寇盜,輒殺略漢使,終莫得通。然聞其西可千餘里有乘象國,名曰滇越,而蜀賈姦出物者或至焉,於是漢以求大夏道始通滇國。初,漢欲通西南夷,費多,道不通,罷之。及張騫言可以通大夏,乃復事西南夷。

騫以校尉從大將軍擊匈奴,知水草處,軍得以不乏,乃封騫為博望侯。是歲元朔六年也。其明年,騫為衛尉,與李將軍俱出右北平擊匈奴。匈奴圍李將軍,軍失亡多;而騫後期當斬,贖為庶人。是歲漢遣驃騎破匈奴西(城)[域]數萬人,至祁連山。其明年,渾邪王率其民降漢,而金城、河西西并南山至鹽澤空無匈奴。匈奴時有候者到,而希矣。其後二年,漢擊走單于於幕北。

是後天子數問騫大夏之屬。騫既失侯,因言曰:「臣居匈奴中,聞烏孫王號昆莫,昆莫之父,匈奴西邊小國也。匈奴攻殺其父,而昆莫生棄於野。烏嗛肉蜚其上,狼往乳之。單于怪以為神,而收長之。及壯,使將兵,數有功,單于復以其父之民予昆莫,令長守於西(城)[域]。昆莫收養其民,攻旁小邑,控弦數萬,習攻戰。單于死,昆莫乃率其眾遠徙,中立,不肯朝會匈奴。匈奴遣奇兵擊,不勝,以為神而遠之,因羈屬之,不大攻。今單于新困於漢,而故渾邪地空無人。蠻夷俗貪漢財物,今誠以此時而厚幣賂烏孫,招以益東,居故渾邪之地,與漢結昆弟,其勢宜聽,聽則是斷匈奴右臂也。既連烏孫,自其西大夏之屬皆可招來而為外臣。」天子以為然,拜騫為中郎將,將三百人,馬各二匹,牛羊以萬數,齎金幣帛直數千巨萬,多持節副使,道可使,使遺之他旁國。

騫既至烏孫,烏孫王昆莫見漢使如單于禮,騫大慚,知蠻夷貪,乃曰:「天子致賜,王不拜則還賜。」昆莫起拜賜,其他如故。騫諭使指曰:「烏孫能東居渾邪地,則漢遣翁主為昆莫夫人。」烏孫國分,王老,而遠漢,未知其大小,素服屬匈奴日久矣,且又近之,其大臣皆畏胡,不欲移徙,王不能專制。騫不得其要領。昆莫有十餘子,其中子曰大祿,彊,善將眾,將眾別居萬餘騎。大祿兄為太子,太子有子曰岑娶,而太子蚤死。臨死謂其父昆莫曰:「必以岑娶為太子,無令他人代之。」昆莫哀而許之,卒以岑娶為太子。大祿怒其不得代太子也,乃收其諸昆弟,將其眾畔,謀攻岑娶及昆莫。昆莫老,常恐大祿殺岑娶,予岑娶萬餘騎別居,而昆莫有萬餘騎自備,國眾分為三,而其大總取羈屬昆莫,昆莫亦以此不敢專約於騫。

騫因分遣副使使大宛、康居、大月氏、大夏、安息、身毒、于窴、扜穼及諸旁國。烏孫發導譯送騫還,騫與烏孫遣使數十人,馬數十匹報謝,因令窺漢,知其廣大。

騫還到,拜為大行,列於九卿。歲餘,卒。

烏孫使既見漢人眾富厚,歸報其國,其國乃益重漢。其後歲餘,騫所遣使通大夏之屬者皆頗與其人俱來,於是西北國始通於漢矣。然張騫鑿空,其後使往者皆稱博望侯,以為質於外國,外國由此信之。

自博望侯騫死後,匈奴聞漢通烏孫,怒,欲擊之。及漢使烏孫,若出其南,抵大宛、大月氏相屬,烏孫乃恐,使使獻馬,願得尚漢女翁主為昆弟。天子問群臣議計,皆曰「必先納聘,然後乃遣女」。初,天子發書易,云「神馬當從西北來」。得烏孫馬好,名曰「天馬」。及得大宛汗血馬,益壯,更名烏孫馬曰「西極」,名大宛馬曰「天馬」云。而漢始筑令居以西,初置酒泉郡以通西北國。因益發使抵安息、奄蔡、黎軒、條枝、身毒國。而天子好宛馬,使者相望於道。諸使外國一輩大者數百,少者百餘人,人所齎操大放博望侯時。其后益習而衰少焉。漢率一歲中使多者十餘,少者五六輩,遠者八九歲,近者數歲而反。

是時漢既滅越,而蜀、西南夷皆震,請吏入朝。於是置益州、越巂、牂柯、沈黎、汶山郡,欲地接以前通大夏。乃遣使柏始昌、呂越人等歲十餘輩,出此初郡抵大夏,皆復閉昆明,為所殺,奪幣財,終莫能通至大夏焉。於是漢發三輔罪人,因巴蜀士數萬人,遣兩將軍郭昌、衛廣等往擊昆明之遮漢使者,斬首虜數萬人而去。其後遣使,昆明復為寇,竟莫能得通。而北道酒泉抵大夏,使者既多,而外國益厭漢幣,不貴其物。

自博望侯開外國道以尊貴,其後從吏卒皆爭上書言外國奇怪利害,求使。天子為其絕遠,非人所樂往,聽其言,予節,募吏民毋問所從來,為具備人眾遣之,以廣其道。來還不能毋侵盜幣物,及使失指,天子為其習之,輒覆案致重罪,以激怒令贖,復求使。使端無窮,而輕犯法。其吏卒亦輒復盛推外國所有,言大者予節,言小者為副,故妄言無行之徒皆爭效之。其使皆貧人子,私縣官齎物,欲賤市以私其利外國。外國亦厭漢使人人有言輕重,度漢兵遠不能至,而禁其食物以苦漢使。漢使乏絕積怨,至相攻擊。而樓蘭、姑師小國耳,當空道,攻劫漢使王恢等尤甚。而匈奴奇兵時時遮擊使西國者。使者爭遍言外國災害,皆有城邑,兵弱易擊。於是天子以故遣從驃侯破奴將屬國騎及郡兵數萬,至匈河水,欲以擊胡,胡皆去。其明年,擊姑師,破奴與輕騎七百餘先至,虜樓蘭王,遂破姑師。因舉兵威以困烏孫、大宛之屬。還,封破奴為浞野侯。王恢數使,為樓蘭所苦,言天子,天子發兵令恢佐破奴擊破之,封恢為浩侯。於是酒泉列亭鄣至玉門矣。

烏孫以千匹馬聘漢女,漢遣宗室女江都翁主往妻烏孫,烏孫王昆莫以為右夫人。匈奴亦遣女妻昆莫,昆莫以為左夫人。昆莫曰「我老」,乃令其孫岑娶妻翁主。烏孫多馬,其富人至有四五千匹馬。

初,漢使至安息,安息王令將二萬騎迎於東界。東界去王都數千里。行比至,過數十城,人民相屬甚多。漢使還,而後發使隨漢使來觀漢廣大,以大鳥卵及黎軒善眩人獻于漢。及宛西小國驩潛、大益,宛東姑師、扜穼、蘇薤之屬,皆隨漢使獻見天子。天子大悅。

而漢使窮河源,河源出于窴,其山多玉石,采來,天子案古圖書,名河所出山曰崑崙云。

是時上方數巡狩海上,乃悉從外國客,大都多人則過之,散財帛以賞賜,厚具以饒給之,以覽示漢富厚焉。於是大觳抵,出奇戲諸怪物,多聚觀者,行賞賜,酒池肉林,令外國客遍觀(名)[各]倉庫府藏之積,見漢之廣大,傾駭之。及加其眩者之工,而觳抵奇戲歲增變,甚盛益興,自此始。

西北外國使,更來更去。宛以西,皆自以遠,尚驕恣晏然,未可詘以禮羈縻而使也。自烏孫以西至安息,以近匈奴,匈奴困月氏也,匈奴使持單于一信,則國國傳送食,不敢留苦;及至漢使,非出幣帛不得食,不市畜不得騎用。所以然者,遠漢,而漢多財物,故必市乃得所欲,然以畏匈奴於漢使焉。宛左右以蒲陶為酒,富人藏酒至萬餘石,久者數十歲不敗。俗嗜酒,馬嗜苜蓿。漢使取其實來,於是天子始種苜蓿、蒲陶肥饒地。及天馬多,外國使來眾,則離宮別觀旁盡種蒲萄、苜蓿極望。自大宛以西至安息,國雖頗異言,然大同俗,相知言。其人皆深眼,多須髯,善市賈,爭分銖。俗貴女子,女子所言而丈夫乃決正。其地皆無絲漆,不知鑄錢器。及漢使亡卒降,教鑄作他兵器。得漢黃白金,輒以為器,不用為幣。

而漢使者往既多,其少從率多進熟於天子,言曰:「宛有善馬在貳師城,匿不肯與漢使。」天子既好宛馬,聞之甘心,使壯士車令等持千金及金馬以請宛王貳師城善馬。宛國饒漢物,相與謀曰:「漢去我遠,而鹽水中數敗,出其北有胡寇,出其南乏水草。又且往往而絕邑,乏食者多。漢使數百人為輩來,而常乏食,死者過半,是安能致大軍乎?無柰我何。且貳師馬,宛寶馬也。」遂不肯予漢使。漢使怒,妄言,椎金馬而去。宛貴人怒曰:「漢使至輕我!」遣漢使去,令其東邊郁成遮攻殺漢使,取其財物。於是天子大怒。諸嘗使宛姚定漢等言宛兵弱,誠以漢兵不過三千人,彊弩射之,即盡虜破宛矣。天子已嘗使浞野侯攻樓蘭,以七百騎先至,虜其王,以定漢等言為然,而欲侯寵姬李氏,拜李廣利為貳師將軍,發屬國六千騎,及郡國惡少年數萬人,以往伐宛。期至貳師城取善馬,故號「貳師將軍」。趙始成為軍正,故浩侯王恢使導軍,而李哆為校尉,制軍事。是歲太初元年也。而關東蝗大起,蜚西至敦煌。

貳師將軍軍既西過鹽水,當道小國恐,各堅城守,不肯給食。攻之不能下。下者得食,不下者數日則去。比至郁成,士至者不過數千,皆饑罷。攻郁成,郁成大破之,所殺傷甚眾。貳師將軍與哆、始成等計:「至郁成尚不能舉,況至其王都乎?」引兵而還。往來二歲。還至敦煌,士不過什一二。使使上書言:「道遠多乏食;且士卒不患戰,患饑。人少,不足以拔宛。願且罷兵,益發而復往。」天子聞之,大怒,而使使遮玉門,曰軍有敢入者輒斬之!貳師恐,因留敦煌。

其夏,漢亡浞野之兵二萬餘於匈奴。公卿及議者皆願罷擊宛軍,專力攻胡。天子已業誅宛,宛小國而不能下,則大夏之屬輕漢,而宛善馬絕不來,烏孫、侖頭易苦漢使矣,為外國笑。乃案言伐宛尤不便者鄧光等,赦囚徒材官,益發惡少年及邊騎,歲餘而出敦煌者六萬人,負私從者不與。牛十萬,馬三萬餘匹,驢騾橐它以萬數。多齎糧,兵弩甚設,天下騷動,傳相奉伐宛,凡五十餘校尉。宛王城中無井,皆汲城外流水,於是乃遣水工徙其城下水空以空其城。益發戍甲卒十八萬,酒泉、張掖北,置居延、休屠以衛酒泉,而發天下七科適,及載糒給貳師。轉車人徒相連屬至敦煌。而拜習馬者二人為執驅校尉,備破宛擇取其善馬云。

於是貳師后復行,兵多,而所至小國莫不迎,出食給軍。至侖頭,侖頭不下,攻數日,屠之。自此而西,平行至宛城,漢兵到者三萬人。宛兵迎擊漢兵,漢兵射敗之,宛走入葆乘其城。貳師兵欲行攻郁成,恐留行而令宛益生詐,乃先至宛,決其水源,移之,則宛固已憂困。圍其城,攻之四十餘日,其外城壞,虜宛貴人勇將煎靡。宛大恐,走入中城。宛貴人相與謀曰:「漢所為攻宛,以王毋寡匿善馬而殺漢使。今殺王毋寡而出善馬,漢兵宜解;即不解,乃力戰而死,未晚也。」宛貴人皆以為然,共殺其王毋寡,持其頭遣貴人使貳師,約曰:「漢毋攻我。我盡出善馬,恣所取,而給漢軍食。即不聽,我盡殺善馬,而康居之救且至。至,我居內,康居居外,與漢軍戰。漢軍熟計之,何從?」是時康居候視漢兵,漢兵尚盛,不敢進。貳師與趙始成、李哆等計:「聞宛城中新得秦人,知穿井,而其內食尚多。所為來,誅首惡者毋寡。毋寡頭已至,如此而不許解兵,則堅守,而康居候漢罷而來救宛,破漢軍必矣。」軍吏皆以為然,許宛之約。宛乃出其善馬,令漢自擇之,而多出食食給漢軍。漢軍取其善馬數十匹。中馬以下牡牝三千餘匹,而立宛貴人之故待遇漢使善者名昧蔡以為宛王,與盟而罷兵。終不得入中城。乃罷而引歸。

初,貳師起敦煌西,以為人多,道上國不能食,乃分為數軍,從南北道。校尉王申生、故鴻臚壺充國等千餘人,別到郁成。郁成城守,不肯給食其軍。王申生去大軍二百里,(偵)[偩]而輕之,責郁成。郁成食不肯出,窺知申生軍日少,晨用三千人攻,戮殺申生等,軍破,數人脫亡,走貳師。貳師令搜粟都尉上官桀往攻破郁成。郁成王亡走康居,桀追至康居。康居聞漢已破宛,乃出郁成王予桀,桀令四騎士縛守詣大將軍。四人相謂曰:「郁成王漢國所毒,今生將去,卒失大事。」欲殺,莫敢先擊。上邽騎士趙弟最少,拔劍擊之,斬郁成王,齎頭。弟、桀等逐及大將軍。

初,貳師后行,天子使使告烏孫,大發兵并力擊宛。烏孫發二千騎往,持兩端,不肯前。貳師將軍之東,諸所過小國聞宛破,皆使其子弟從軍入獻,見天子,因以為質焉。貳師之伐宛也,而軍正趙始成力戰,功最多;及上官桀敢深入,李哆為謀計,軍入玉門者萬餘人,軍馬千餘匹。貳師后行,軍非乏食,戰死不能多,而將吏貪,多不愛士卒,侵牟之,以此物故眾。天子為萬里而伐宛,不錄過,封廣利為海西侯。又封身斬郁成王者騎士趙弟為新畤侯。軍正趙始成為光祿大夫,上官桀為少府,李哆為上黨太守。軍官吏為九卿者三人,諸侯相、郡守、二千石者百餘人,千石以下千餘人。奮行者官過其望,以適過行者皆絀其勞。士卒賜直四萬金。伐宛再反,凡四歲而得罷焉。

漢已伐宛,立昧蔡為宛王而去。歲餘,宛貴人以為昧蔡善諛,使我國遇屠,乃相與殺昧蔡,立毋寡昆弟曰蟬封為宛王,而遣其子入質於漢。漢因使使賂賜以鎮撫之。

而漢發使十餘輩至宛西諸外國,求奇物,因風覽以伐宛之威德。而敦煌置酒泉都尉;西至鹽水,往往有亭。而侖頭有田卒數百人,因置使者護田積粟,以給使外國者。

太史公曰:《禹本紀》言「河出崑崙。崑崙其高二千五百餘里,日月所相避隱為光明也。其上有醴泉、瑤池」。今自張騫使大夏之後也,窮河源,惡睹本紀所謂崑崙者乎?故言九州山川,《尚書》近之矣。至《禹本紀》、《山海經》所有怪物,余不敢言之也。


\end{pinyinscope}