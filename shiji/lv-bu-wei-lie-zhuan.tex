\article{呂不韋列傳}

\begin{pinyinscope}
呂不韋者,陽翟大賈人也。往來販賤賣貴,家累千金。

秦昭王四十年,太子死。其四十二年,以其次子安國君為太子。安國君有子二十餘人。安國君有所甚愛姬,立以為正夫人,號曰華陽夫人。華陽夫人無子。安國君中男名子楚,子楚母曰夏姬,毋愛。子楚為秦質子於趙。秦數攻趙,趙不甚禮子楚。

子楚,秦諸庶孽孫,質於諸侯,車乘進用不饒,居處困,不得意。呂不韋賈邯鄲,見而憐之,曰「此奇貨可居」。乃往見子楚,說曰:「吾能大子之門。」子楚笑曰:「且自大君之門,而乃大吾門!」呂不韋曰:「子不知也,吾門待子門而大。」子楚心知所謂,乃引與坐,深語。呂不韋曰:「秦王老矣,安國君得為太子。竊聞安國君愛幸華陽夫人,華陽夫人無子,能立適嗣者獨華陽夫人耳。今子兄弟二十餘人,子又居中,不甚見幸,久質諸侯。即大王薨,安國君立為王,則子毋幾得與長子及諸子旦暮在前者爭為太子矣。」子楚曰:「然。為之柰何?」呂不韋曰:「子貧,客於此,非有以奉獻於親及結賓客也。不韋雖貧,請以千金為子西游,事安國君及華陽夫人,立子為適嗣。」子楚乃頓首曰:「必如君策,請得分秦國與君共之。」

呂不韋乃以五百金與子楚,為進用,結賓客;而復以五百金買奇物玩好,自奉而西游秦,求見華陽夫人姊,而皆以其物獻華陽夫人。因言子楚賢智,結諸侯賓客遍天下,常曰「楚也以夫人為天,日夜泣思太子及夫人」。夫人大喜。不韋因使其姊說夫人曰:「吾聞之,以色事人者,色衰而愛弛。今夫人事太子,甚愛而無子,不以此時蚤自結於諸子中賢孝者,舉立以為適而子之,夫在則重尊,夫百歲之後,所子者為王,終不失勢,此所謂一言而萬世之利也。不以繁華時樹本,即色衰愛弛後,雖欲開一語,尚可得乎?今子楚賢,而自知中男也,次不得為適,其母又不得幸,自附夫人,夫人誠以此時拔以為適,夫人則竟世有寵於秦矣。」華陽夫人以為然,承太子閒,從容言子楚質於趙者絕賢,來往者皆稱譽之。乃因涕泣曰:「妾幸得充後宮,不幸無子,願得子楚立以為適嗣,以託妾身。」安國君許之,乃與夫人刻玉符,約以為適嗣。安國君及夫人因厚餽遺子楚,而請呂不韋傅之,子楚以此名譽益盛於諸侯。

呂不韋取邯鄲諸姬絕好善舞者與居,知有身。子楚從不韋飲,見而說之,因起為壽,請之。呂不韋怒,念業已破家為子楚,欲以釣奇,乃遂獻其姬。姬自匿有身,至大期時,生子政。子楚遂立姬為夫人。

秦昭王五十年,使王齮圍邯鄲,急,趙欲殺子楚。子楚與呂不韋謀,行金六百斤予守者吏,得脫,亡赴秦軍,遂以得歸。趙欲殺子楚妻子,子楚夫人趙豪家女也,得匿,以故母子竟得活。秦昭王五十六年,薨,太子安國君立為王,華陽夫人為王后,子楚為太子。趙亦奉子楚夫人及子政歸秦。

秦王立一年,薨,謚為孝文王。太子子楚代立,是為莊襄王。莊襄王所母華陽后為華陽太后,真母夏姬尊以為夏太后。莊襄王元年,以呂不韋為丞相,封為文信侯,食河南雒陽十萬戶。

莊襄王即位三年,薨,太子政立為王,尊呂不韋為相國,號稱「仲父」。秦王年少,太后時時竊私通呂不韋。不韋家僮萬人。

當是時,魏有信陵君,楚有春申君,趙有平原君,齊有孟嘗君,皆下士喜賓客以相傾。呂不韋以秦之彊,羞不如,亦招致士,厚遇之,至食客三千人。是時諸侯多辯士,如荀卿之徒,著書布天下。呂不韋乃使其客人人著所聞,集論以為八覽、六論、十二紀,二十餘萬言。以為備天地萬物古今之事,號曰呂氏春秋。布咸陽市門,懸千金其上,延諸侯游士賓客有能增損一字者予千金。

始皇帝益壯,太后淫不止。呂不韋恐覺禍及己,乃私求大陰人嫪毐以為舍人,時縱倡樂,使毐以其陰關桐輪而行,令太后聞之,以啗太后。太后聞,果欲私得之。呂不韋乃進嫪毐,詐令人以腐罪告之。不韋又陰謂太后曰:「可事詐腐,則得給事中。」太后乃陰厚賜主腐者吏,詐論之,拔其須眉為宦者,遂得侍太后。太后私與通,絕愛之。有身,太后恐人知之,詐卜當避時,徙宮居雍。嫪毐常從,賞賜甚厚,事皆決於嫪毐。嫪毐家僮數千人,諸客求宦為嫪毐舍人千餘人。

始皇七年,莊襄王母夏太后薨。孝文王后曰華陽太后,與孝文王會葬壽陵。夏太后子莊襄王葬芷陽,故夏太后獨別葬杜東,曰「東望吾子,西望吾夫。後百年,旁當有萬家邑」。

始皇九年,有告嫪毐實非宦者,常與太后私亂,生子二人,皆匿之。與太后謀曰「王即薨,以子為後」。於是秦王下吏治,具得情實,事連相國呂不韋。九月,夷嫪毐三族,殺太后所生兩子,而遂遷太后於雍。諸嫪毐舍人皆沒其家而遷之蜀。王欲誅相國,為其奉先王功大,及賓客辯士為游說者眾,王不忍致法。

秦王十年十月,免相國呂不韋。及齊人茅焦說秦王,秦王乃迎太后於雍,歸復咸陽,而出文信侯就國河南。

歲餘,諸侯賓客使者相望於道,請文信侯。秦王恐其為變,乃賜文信侯書曰:「君何功於秦?秦封君河南,食十萬戶。君何親於秦?號稱仲父。其與家屬徙處蜀!」呂不韋自度稍侵,恐誅,乃飲酖而死。秦王所加怒呂不韋、嫪毐皆已死,乃皆復歸嫪毐舍人遷蜀者。

始皇十九年,太后薨,謚為帝太后,與莊襄王會葬茝陽。

太史公曰:不韋及嫪毐貴,封號文信侯。人之告嫪毐,毐聞之。秦王驗左右,未發。上之雍郊,毐恐禍起,乃與黨謀,矯太后璽發卒以反蘄年宮。發吏攻毐,毐敗亡走,追斬之好畤,遂滅其宗。而呂不韋由此絀矣。孔子之所謂「聞」者,其呂子乎?


\end{pinyinscope}