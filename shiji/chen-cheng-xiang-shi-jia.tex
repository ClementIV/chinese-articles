\article{陳丞相世家}

\begin{pinyinscope}
陳丞相平者,陽武戶牖鄉人也。少時家貧,好讀書,有田三十畝,獨與兄伯居。伯常耕田,縱平使游學。平為人長[大]美色。人或謂陳平曰:「貧何食而肥若是?」其嫂嫉平之不視家生產,曰:「亦食糠覈耳。有叔如此,不如無有。」伯聞之,逐其婦而棄之。

及平長,可娶妻,富人莫肯與者,貧者平亦恥之。久之,戶牖富人有張負,張負女孫五嫁而夫輒死,人莫敢娶。平欲得之。邑中有喪,平貧,侍喪,以先往後罷為助。張負既見之喪所,獨視偉平,平亦以故後去。負隨平至其家,家乃負郭窮巷,以獘席為門,然門外多有長者車轍。張負歸,謂其子仲曰:「吾欲以女孫予陳平。」張仲曰:「平貧不事事,一縣中盡笑其所為,獨柰何予女乎?」負曰:「人固有好美如陳平而長貧賤者乎?」卒與女。為平貧,乃假貸幣以聘,予酒肉之資以內婦。負誡其孫曰:「毋以貧故,事人不謹。事兄伯如事父,事嫂如母。」平既娶張氏女,齎用益饒,游道日廣。

裏中社,平為宰,分肉食甚均。父老曰:「善,陳孺子之為宰!」平曰:「嗟乎,使平得宰天下,亦如是肉矣!」

陳涉起而王陳,使周市略定魏地,立魏咎為魏王,與秦軍相攻於臨濟。陳平固已前謝其兄伯,從少年往事魏王咎於臨濟。魏王以為太仆。說魏王不聽,人或讒之,陳平亡去。

久之,項羽略地至河上,陳平往歸之,從入破秦,賜平爵卿。項羽之東王彭城也,漢王還定三秦而東,殷王反楚。項羽乃以平為信武君,將魏王咎客在楚者以往,擊降殷王而還。項王使項悍拜平為都尉,賜金二十溢。居無何,漢王攻下殷(王)。項王怒,將誅定殷者將吏。陳平懼誅,乃封其金與印,使使歸項王,而平身閒行杖劍亡。渡河,船人見其美丈夫獨行,疑其亡將,要中當有金玉寶器,目之,欲殺平。平恐,乃解衣躶而佐刺船,船人知其無有,乃止。

平遂至修武降漢,因魏無知求見漢王,漢王召入。是時萬石君奮為漢王中涓,受平謁,入見平。平等七人俱進,賜食。王曰:「罷,就舍矣。」平曰:「臣為事來,所言不可以過今日。」於是漢王與語而說之,問曰:「子之居楚何官?」曰:「為都尉。」是日乃拜平為都尉,使為參乘,典護軍。諸將盡讙,曰:「大王一日得楚之亡卒,未知其高下,而即與同載,反使監護軍長者!」漢王聞之,愈益幸平。遂與東伐項王。至彭城,為楚所敗。引而還,收散兵至滎陽,以平為亞將,屬於韓王信,軍廣武。

絳侯、灌嬰等咸讒陳平曰:「平雖美丈夫,如冠玉耳,其中未必有也。臣聞平居家時,盜其嫂;事魏不容,亡歸楚;歸楚不中,又亡歸漢。今日大王尊官之,令護軍。臣聞平受諸將金,金多者得善處,金少者得惡處。平,反覆亂臣也,願王察之。」漢王疑之,召讓魏無知。無知曰:「臣所言者,能也;陛下所問者,行也。今有尾生、孝己之行而無益處於勝負之數,陛下何暇用之乎?楚漢相距,臣進奇謀之士,顧其計誠足以利國家不耳。且盜嫂受金又何足疑乎?」漢王召讓平曰:「先生事魏不中,遂事楚而去,今又從吾游,信者固多心乎?」平曰:「臣事魏王,魏王不能用臣說,故去事項王。項王不能信人,其所任愛,非諸項即妻之昆弟,雖有奇士不能用,平乃去楚。聞漢王之能用人,故歸大王。臣躶身來,不受金無以為資。誠臣計畫有可采者,(顧)[願]大王用之;使無可用者,金具在,請封輸官,得請骸骨。」漢王乃謝,厚賜,拜為護軍中尉,盡護諸將。諸將乃不敢復言。

其後,楚急攻,絕漢甬道,圍漢王於滎陽城。久之,漢王患之,請割滎陽以西以和。項王不聽。漢王謂陳平曰:「天下紛紛,何時定乎?」陳平曰:「項王為人,恭敬愛人,士之廉節好禮者多歸之。至於行功爵邑,重之,士亦以此不附。今大王慢而少禮,士廉節者不來;然大王能饒人以爵邑,士之頑鈍嗜利無恥者亦多歸漢。誠各去其兩短,襲其兩長,天下指麾則定矣。然大王恣侮人,不能得廉節之士。顧楚有可亂者,彼項王骨鯁之臣亞父、鐘離眛、龍且、周殷之屬,不過數人耳。大王誠能出捐數萬斤金,行反閒,閒其君臣,以疑其心,項王為人意忌信讒,必內相誅。漢因舉兵而攻之,破楚必矣。」漢王以為然,乃出黃金四萬斤,與陳平,恣所為,不問其出入。

陳平既多以金縱反閒於楚軍,宣言諸將鐘離眛等為項王將,功多矣,然而終不得裂地而王,欲與漢為一,以滅項氏而分王其地。項羽果意不信鐘離眛等。項王既疑之,使使至漢。漢王為太牢具,舉進。見楚使,即詳驚曰:「吾以為亞父使,乃項王使!」復持去,更以惡草具進楚使。楚使歸,具以報項王。項王果大疑亞父。亞父欲急攻下滎陽城,項王不信,不肯聽。亞父聞項王疑之,乃怒曰:「天下事大定矣,君王自為之!願請骸骨歸!」歸未至彭城,疽發背而死。陳平乃夜出女子二千人滎陽城東門,楚因擊之,陳平乃與漢王從城西門夜出去。遂入關,收散兵復東。

其明年,淮陰侯破齊,自立為齊王,使使言之漢王。漢王大怒而罵,陳平躡漢王。漢王亦悟,乃厚遇齊使,使張子房卒立信為齊王。封平以戶牖鄉。用其奇計策,卒滅楚。常以護軍中尉從定燕王臧荼。

漢六年,人有上書告楚王韓信反。高帝問諸將,諸將曰:「亟發兵阬豎子耳。」高帝默然。問陳平,平固辭謝,曰:「諸將云何?」上具告之。陳平曰:「人之上書言信反,有知之者乎?」曰:「未有。」曰:「信知之乎?」曰:「不知。」陳平曰:「陛下精兵孰與楚?」上曰:「不能過。」平曰:「陛下將用兵有能過韓信者乎?」上曰:「莫及也。」平曰:「今兵不如楚精,而將不能及,而舉兵攻之,是趣之戰也,竊為陛下危之。」上曰:「為之柰何?」平曰:「古者天子巡狩,會諸侯。南方有雲夢,陛下弟出偽游雲夢,會諸侯於陳。陳,楚之西界,信聞天子以好出游,其勢必無事而郊迎謁。謁,而陛下因禽之,此特一力士之事耳。」高帝以為然,乃發使告諸侯會陳,「吾將南游雲夢」。上因隨以行。行未至陳,楚王信果郊迎道中。高帝豫具武士,見信至,即執縛之,載後車。信呼曰:「天下已定,我固當烹!」高帝顧謂信曰:「若毋聲!而反,明矣!」武士反接之。遂會諸侯于陳,盡定楚地。還至雒陽,赦信以為淮陰侯,而與功臣剖符定封。

於是與平剖符,世世勿絕,為戶牖侯。平辭曰:「此非臣之功也。」上曰:「吾用先生謀計,戰勝剋敵,非功而何?」平曰:「非魏無知臣安得進?」上曰;「若子可謂不背本矣。」乃復賞魏無知。其明年,以護軍中尉從攻反者韓王信於代。卒至平城,為匈奴所圍,七日不得食。高帝用陳平奇計,使單于閼氏,圍以得開。高帝既出,其計祕,世莫得聞。

高帝南過曲逆,上其城,望見其屋室甚大,曰:「壯哉縣!吾行天下,獨見洛陽與是耳。」顧問御史曰:「曲逆戶口幾何?」對曰:「始秦時三萬餘戶,閒者兵數起,多亡匿,今見五千戶。」於是乃詔御史,更以陳平為曲逆侯,盡食之,除前所食戶牖。

其後常以護軍中尉從攻陳豨及黥布。凡六出奇計,輒益邑,凡六益封。奇計或頗祕,世莫能聞也。

高帝從破布軍還,病創,徐行至長安。燕王盧綰反,上使樊噲以相國將兵攻之。既行,人有短惡噲者。高帝怒曰:「噲見吾病,乃冀我死也。」用陳平謀而召絳侯周勃受詔床下,曰:「陳平亟馳傳載勃代噲將,平至軍中即斬噲頭!」二人既受詔,馳傳未至軍,行計之曰:「樊噲,帝之故人也,功多,且又乃呂后弟呂媭之夫,有親且貴,帝以忿怒故,欲斬之,則恐後悔。寧囚而致上,上自誅之。」未至軍,為壇,以節召樊噲。噲受詔,即反接載檻車,傳詣長安,而令絳侯勃代將,將兵定燕反縣。

平行聞高帝崩,平恐呂太后及呂媭讒怒,乃馳傳先去。逢使者詔平與灌嬰屯於滎陽。平受詔,立復馳至宮,哭甚哀,因奏事喪前。呂太后哀之,曰:「君勞,出休矣。」平畏讒之就,因固請得宿衛中。太后乃以為郎中令,曰:「傅教孝惠。」是後呂媭讒乃不得行。樊噲至,則赦復爵邑。

孝惠帝六年,相國曹參卒,以安國侯王陵為右丞相,陳平為左丞相。

王陵者,故沛人,始為縣豪,高祖微時,兄事陵。陵少文,任氣,好直言。及高祖起沛,入至咸陽,陵亦自聚黨數千人,居南陽,不肯從沛公。及漢王之還攻項籍,陵乃以兵屬漢。項羽取陵母置軍中,陵使至,則東鄉坐陵母,欲以招陵。陵母既私送使者,泣曰:「為老妾語陵,謹事漢王。漢王,長者也,無以老妾故,持二心。妾以死送使者。」遂伏劍而死。項王怒,烹陵母。陵卒從漢王定天下。以善雍齒,雍齒,高帝之仇,而陵本無意從高帝,以故晚封,為安國侯。

安國侯既為右丞相,二歲,孝惠帝崩。高后欲立諸呂為王,問王陵,王陵曰:「不可。」問陳平,陳平曰:「可。」呂太后怒,乃詳遷陵為帝太傅,實不用陵。陵怒,謝疾免,杜門竟不朝請,七年而卒。

陵之免丞相,呂太后乃徙平為右丞相,以辟陽侯審食其為左丞相。左丞相不治,常給事於中。

食其亦沛人。漢王之敗彭城西,楚取太上皇、呂后為質,食其以舍人侍呂后。其後從破項籍為侯,幸於呂太后。及為相,居中,百官皆因決事。

呂媭常以前陳平為高帝謀執樊噲,數讒曰:「陳平為相非治事,日飲醇酒,戲婦女。」陳平聞,日益甚。呂太后聞之,私獨喜。面質呂媭於陳平曰:「鄙語曰『兒婦人口不可用』,顧君與我何如耳。無畏呂媭之讒也。」

呂太后立諸呂為王,陳平偽聽之。及呂太后崩,平與太尉勃合謀,卒誅諸呂,立孝文皇帝,陳平本謀也。審食其免相。

孝文帝立,以為太尉勃親以兵誅呂氏,功多;陳平欲讓勃尊位,乃謝病。孝文帝初立,怪平病,問之。平曰:「高祖時,勃功不如臣平。及誅諸呂,臣功亦不如勃。願以右丞相讓勃。」於是孝文帝乃以絳侯勃為右丞相,位次第一;平徙為左丞相,位次第二。賜平金千斤,益封三千戶。

居頃之,孝文皇帝既益明習國家事,朝而問右丞相勃曰:「天下一歲決獄幾何?」勃謝曰:「不知。」問:「天下一歲錢穀出入幾何?」勃又謝不知,汗出沾背,愧不能對。於是上亦問左丞相平。平曰:「有主者。」上曰:「主者謂誰?」平曰:「陛下即問決獄,責廷尉;問錢穀,責治粟內史。」上曰:「茍各有主者,而君所主者何事也?」平謝曰:「主臣!陛下不知其駑下,使待罪宰相。宰相者,上佐天子理陰陽,順四時,下育萬物之宜,外鎮撫四夷諸侯,內親附百姓,使卿大夫各得任其職焉。」孝文帝乃稱善。右丞相大慚,出而讓陳平曰:「君獨不素教我對!」陳平笑曰:「君居其位,不知其任邪?且陛下即問長安中盜賊數,君欲彊對邪?」於是絳侯自知其能不如平遠矣。居頃之,絳侯謝病請免相,陳平專為一丞相。

孝文帝二年,丞相陳平卒,謚為獻侯。子共侯買代侯。二年卒,子簡侯恢代侯。二十三年卒,子何代侯。二十三年,何坐略人妻,棄市,國除。

始陳平曰:「我多陰謀,是道家之所禁。吾世即廢,亦已矣,終不能復起,以吾多陰禍也。」然其後曾孫陳掌以衛氏親貴戚,願得續封陳氏,然終不得。

太史公曰:陳丞相平少時,本好黃帝、老子之術。方其割肉俎上之時,其意固已遠矣。傾側擾攘楚魏之閒,卒歸高帝。常出奇計,救紛糾之難,振國家之患。及呂后時,事多故矣,然平竟自脫,定宗廟,以榮名終,稱賢相,豈不善始善終哉!非知謀孰能當此者乎?


\end{pinyinscope}