\article{伍子胥列傳}

\begin{pinyinscope}
伍子胥者,楚人也,名員。員父曰伍奢。員兄曰伍尚。其先曰伍舉,以直諫事楚莊王,有顯,故其後世有名於楚。

楚平王有太子名曰建,使伍奢為太傅,費無忌為少傅。無忌不忠於太子建。平王使無忌為太子取婦於秦,秦女好,無忌馳歸報平王曰:「秦女絕美,王可自取,而更為太子取婦。」平王遂自取秦女而絕愛幸之,生子軫。更為太子取婦。

無忌既以秦女自媚於平王,因去太子而事平王。恐一旦平王卒而太子立,殺己,乃因讒太子建。建母,蔡女也,無寵於平王。平王稍益疏建,使建守城父,備邊兵。

頃之,無忌又日夜言太子短於王曰:「太子以秦女之故,不能無怨望,願王少自備也。自太子居城父,將兵,外交諸侯,且欲入為亂矣。」平王乃召其太傅伍奢考問之。伍奢知無忌讒太子於平王,因曰:「王獨柰何以讒賊小臣疏骨肉之親乎?」無忌曰:「王今不制,其事成矣。王且見禽。」於是平王怒,囚伍奢,而使城父司馬奮揚往殺太子。行未至,奮揚使人先告太子:「太子急去,不然將誅。」太子建亡奔宋。

無忌言於平王曰:「伍奢有二子,皆賢,不誅且為楚憂。可以其父質而召之,不然且為楚患。」王使使謂伍奢曰:「能致汝二子則生,不能則死。」伍奢曰:「尚為人仁,呼必來。員為人剛戾忍訽,能成大事,彼見來之并禽,其勢必不來。」王不聽,使人召二子曰:「來,吾生汝父;不來,今殺奢也。」伍尚欲往,員曰:「楚之召我兄弟,非欲以生我父也,恐有脫者後生患,故以父為質,詐召二子。二子到,則父子俱死。何益父之死?往而令讎不得報耳。不如奔他國,借力以雪父之恥,俱滅,無為也。」伍尚曰:「我知往終不能全父命。然恨父召我以求生而不往,後不能雪恥,終為天下笑耳。」謂員:「可去矣!汝能報殺父之讎,我將歸死。」尚既就執,使者捕伍胥。伍胥貫弓執矢向使者,使者不敢進,伍胥遂亡。聞太子建之在宋,往從之。奢聞子胥之亡也,曰:「楚國君臣且苦兵矣。」伍尚至楚,楚并殺奢與尚也。

伍胥既至宋,宋有華氏之亂,乃與太子建俱奔於鄭。鄭人甚善之。太子建又適晉,晉頃公曰:「太子既善鄭,鄭信太子。太子能為我內應,而我攻其外,滅鄭必矣。滅鄭而封太子。」太子乃還鄭。事未會,會自私欲殺其從者,從者知其謀,乃告之於鄭。鄭定公與子產誅殺太子建。建有子名勝。伍胥懼,乃與勝俱奔吳。到昭關,昭關欲執之。伍胥遂與勝獨身步走,幾不得脫。追者在後。至江,江上有一漁父乘船,知伍胥之急,乃渡伍胥。伍胥既渡,解其劍曰:「此劍直百金,以與父。」父曰:「楚國之法,得伍胥者賜粟五萬石,爵執珪,豈徒百金劍邪!」不受。伍胥未至吳而疾,止中道,乞食。至於吳,吳王僚方用事,公子光為將。伍胥乃因公子光以求見吳王。

久之,楚平王以其邊邑鐘離與吳邊邑卑梁氏俱蠶,兩女子爭桑相攻,乃大怒,至於兩國舉兵相伐。吳使公子光伐楚,拔其鐘離、居巢而歸。伍子胥說吳王僚曰:「楚可破也。願復遣公子光。」公子光謂吳王曰:「彼伍胥父兄為戮於楚,而勸王伐楚者,欲以自報其讎耳。伐楚未可破也。」伍胥知公子光有內志,欲殺王而自立,未可說以外事,乃進專諸於公子光,退而與太子建之子勝耕於野。

五年而楚平王卒。初,平王所奪太子建秦女生子軫,及平王卒,軫竟立為後,是為昭王。吳王僚因楚喪,使二公子將兵往襲楚。楚發兵絕吳兵之後,不得歸。吳國內空,而公子光乃令專諸襲刺吳王僚而自立,是為吳王闔廬。闔廬既立,得志,乃召伍員以為行人,而與謀國事。

楚誅其大臣郤宛、伯州犁,伯州犁之孫伯嚭亡奔吳,吳亦以嚭為大夫。前王僚所遣二公子將兵伐楚者,道絕不得歸。後聞闔廬弒王僚自立,遂以其兵降楚,楚封之於舒。闔廬立三年,乃興師與伍胥、伯嚭伐楚,拔舒,遂禽故吳反二將軍。因欲至郢,將軍孫武曰:「民勞,未可,且待之。」乃歸。

四年,吳伐楚,取六與灊。五年,伐越,敗之。六年,楚昭王使公子囊瓦將兵伐吳。吳使伍員迎擊,大破楚軍於豫章,取楚之居巢。

九年,吳王闔廬謂子胥、孫武曰:「始子言郢未可入,今果何如?」二子對曰:「楚將囊瓦貪,而唐、蔡皆怨之。王必欲大伐之,必先得唐、蔡乃可。」闔廬聽之,悉興師與唐、蔡伐楚,與楚夾漢水而陳。吳王之弟夫概將兵請從,王不聽,遂以其屬五千人擊楚將子常。己卯,楚昭王出奔。庚辰,吳王入郢。子常敗走,奔鄭。於是吳乘勝而前,五戰,遂至郢。

昭王出亡,入雲夢;盜擊王,王走鄖。鄖公弟懷曰:「平王殺我父,我殺其子,不亦可乎!」鄖公恐其弟殺王,與王奔隨。吳兵圍隨,謂隨人曰:「周之子孫在漢川者,楚盡滅之。」隨人欲殺王,王子綦匿王,己自為王以當之。隨人卜與王於吳,不吉,乃謝吳不與王。

始伍員與申包胥為交,員之亡也,謂包胥曰:「我必覆楚。」包胥曰:「我必存之。」及吳兵入郢,伍子胥求昭王。既不得,乃掘楚平王墓,出其尸,鞭之三百,然後已。申包胥亡於山中,使人謂子胥曰:「子之報讎,其以甚乎!吾聞之,人眾者勝天,天定亦能破人。今子故平王之臣,親北面而事之,今至於僇死人,此豈其無天道之極乎!」伍子胥曰:「為我謝申包胥曰,吾日莫途遠,吾故倒行而逆施之。」於是申包胥走秦告急,求救於秦。秦不許。包胥立於秦廷,晝夜哭,七日七夜不絕其聲。秦哀公憐之,曰:「楚雖無道,有臣若是,可無存乎!」乃遣車五百乘救楚擊吳。六月,敗吳兵於稷。會吳王久留楚求昭王,而闔廬弟夫概乃亡歸,自立為王。闔廬聞之,乃釋楚而歸,擊其弟夫概。夫概敗走,遂奔楚。楚昭王見吳有內亂,乃復入郢。封夫概於堂谿,為堂谿氏。楚復與吳戰,敗吳,吳王乃歸。

後二歲,闔廬使太子夫差將兵伐楚,取番。楚懼吳復大來,乃去郢,徙於鄀。當是時,吳以伍子胥、孫武之謀,西破彊楚,北威齊晉,南服越人。

其後四年,孔子相魯。

後五年,伐越。越王句踐迎擊,敗吳於姑蘇,傷闔廬指,軍卻。闔廬病創將死,謂太子夫差曰:「爾忘句踐殺爾父乎?」夫差對曰:「不敢忘。」是夕,闔廬死。夫差既立為王,以伯嚭為太宰,習戰射。二年後伐越,敗越於夫湫。越王句踐乃以餘兵五千人棲於會稽之上,使大夫種厚幣遺吳太宰嚭以請和,求委國為臣妾。吳王將許之。伍子胥諫曰:「越王為人能辛苦。今王不滅,後必悔之。」吳王不聽,用太宰嚭計,與越平。

其後五年,而吳王聞齊景公死而大臣爭寵,新君弱,乃興師北伐齊。伍子胥諫曰:「句踐食不重味,弔死問疾,且欲有所用之也。此人不死,必為吳患。今吳之有越,猶人之有腹心疾也。而王不先越而乃務齊,不亦謬乎!」吳王不聽,伐齊,大敗齊師於艾陵,遂威鄒魯之君以歸。益疏子胥之謀。

其後四年,吳王將北伐齊,越王句踐用子貢之謀,乃率其眾以助吳,而重寶以獻遺太宰嚭。太宰嚭既數受越賂,其愛信越殊甚,日夜為言於吳王。吳王信用嚭之計。伍子胥諫曰:「夫越,腹心之病,今信其浮辭詐偽而貪齊。破齊,譬猶石田,無所用之。且盤庚之誥曰:『有顛越不恭,劓殄滅之,俾無遺育,無使易種于茲邑。』此商之所以興。願王釋齊而先越;若不然,後將悔之無及。」而吳王不聽,使子胥於齊。子胥臨行,謂其子曰:「吾數諫王,王不用,吾今見吳之亡矣。汝與吳俱亡,無益也。」乃屬其子於齊鮑牧,而還報吳。

吳太宰嚭既與子胥有隙,因讒曰:「子胥為人剛暴,少恩,猜賊,其怨望恐為深禍也。前日王欲伐齊,子胥以為不可,王卒伐之而有大功。子胥恥其計謀不用,乃反怨望。而今王又復伐齊,子胥專愎彊諫,沮毀用事,徒幸吳之敗以自勝其計謀耳。今王自行,悉國中武力以伐齊,而子胥諫不用,因輟謝,詳病不行。王不可不備,此起禍不難。且嚭使人微伺之,其使於齊也,乃屬其子於齊之鮑氏。夫為人臣,內不得意,外倚諸侯,自以為先王之謀臣,今不見用,常鞅鞅怨望。願王早圖之。」吳王曰:「微子之言,吾亦疑之。」乃使使賜伍子胥屬鏤之劍,曰:「子以此死。」伍子胥仰天嘆曰:「嗟乎!讒臣嚭為亂矣,王乃反誅我。我令若父霸。自若未立時,諸公子爭立,我以死爭之於先王,幾不得立。若既得立,欲分吳國予我,我顧不敢望也。然今若聽諛臣言以殺長者。」乃告其舍人曰:「必樹吾墓上以梓,令可以為器;而抉吾眼縣吳東門之上,以觀越寇之入滅吳也。」乃自剄死。吳王聞之大怒,乃取子胥尸盛以鴟夷革,浮之江中。吳人憐之,為立祠於江上,因命曰胥山。

吳王既誅伍子胥,遂伐齊。齊鮑氏殺其君悼公而立陽生。吳王欲討其賊,不勝而去。其後二年,吳王召魯衛之君會之橐皋。其明年,因北大會諸侯於黃池,以令周室。越王句踐襲殺吳太子,破吳兵。吳王聞之,乃歸,使使厚幣與越平。後九年,越王句踐遂滅吳,殺王夫差;而誅太宰嚭,以不忠於其君,而外受重賂,與己比周也。

伍子胥初所與俱亡故楚太子建之子勝者,在於吳。吳王夫差之時,楚惠王欲召勝歸楚。葉公諫曰:「勝好勇而陰求死士,殆有私乎!」惠王不聽。遂召勝,使居楚之邊邑鄢,號為白公。白公歸楚三年而吳誅子胥。

白公勝既歸楚,怨鄭之殺其父,乃陰養死士求報鄭。歸楚五年,請伐鄭,楚令尹子西許之。兵未發而晉伐鄭,鄭請救於楚。楚使子西往救,與盟而還。白公勝怒曰:「非鄭之仇,乃子西也。」勝自礪劍,人問曰:「何以為?」勝曰:「欲以殺子西。」子西聞之,笑曰:「勝如卵耳,何能為也。」

其後四歲,白公勝與石乞襲殺楚令尹子西、司馬子綦於朝。石乞曰:「不殺王,不可。」乃劫(之)王如高府。石乞從者屈固負楚惠王亡走昭夫人之宮。葉公聞白公為亂,率其國人攻白公。白公之徒敗,亡走山中,自殺。而虜石乞,而問白公尸處,不言將亨。石乞曰:「事成為卿,不成而亨,固其職也。」終不肯告其尸處。遂亨石乞,而求惠王復立之。

太史公曰:怨毒之於人甚矣哉!王者尚不能行之於臣下,況同列乎!向令伍子胥從奢俱死,何異螻蟻。棄小義,雪大恥,名垂於後世,悲夫!方子胥窘於江上,道乞食,志豈嘗須臾忘郢邪?故隱忍就功名,非烈丈夫孰能致此哉?白公如不自立為君者,其功謀亦不可勝道者哉!


\end{pinyinscope}