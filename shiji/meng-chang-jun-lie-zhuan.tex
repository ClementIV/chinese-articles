\article{孟嘗君列傳}

\begin{pinyinscope}
孟嘗君名文,姓田氏。文之父曰靖郭君田嬰。田嬰者,齊威王少子而齊宣王庶弟也。田嬰自威王時任職用事,與成侯鄒忌及田忌將而救韓伐魏。成侯與田忌爭寵,成侯賣田忌。田忌懼,襲齊之邊邑,不勝,亡走。會威王卒,宣王立,知成侯賣田忌,乃復召田忌以為將。宣王二年,田忌與孫臏、田嬰俱伐魏,敗之馬陵,虜魏太子申而殺魏將龐涓。宣王七年,田嬰使於韓、魏,韓、魏服於齊。嬰與韓昭侯、魏惠王會齊宣王東阿南,盟而去。明年,復與梁惠王會甄。是歲,梁惠王卒。宣王九年,田嬰相齊。齊宣王與魏襄王會徐州而相王也。楚威王聞之,怒田嬰。明年,楚伐敗齊師於徐州,而使人逐田嬰。田嬰使張丑說楚威王,威王乃止。田嬰相齊十一年,宣王卒,湣王即位。即位三年,而封田嬰於薛。

初,田嬰有子四十餘人。其賤妾有子名文,文以五月五日生。嬰告其母曰:「勿舉也。」其母竊舉生之。及長,其母因兄弟而見其子文於田嬰。田嬰怒其母曰:「吾令若去此子,而敢生之,何也?」文頓首,因曰:「君所以不舉五月子者,何故?」嬰曰:「五月子者,長與戶齊,將不利其父母。」文曰:「人生受命於天乎?將受命於戶邪?」嬰默然。文曰:「必受命於天,君何憂焉。必受命於戶,則可高其戶耳,誰能至者!」嬰曰:「子休矣。」

久之,文承閒問其父嬰曰:「子之子為何?」曰:「為孫。」「孫之孫為何?」曰:「為玄孫。」「玄孫之孫為何?」曰:「不能知也。」文曰:「君用事相齊,至今三王矣,齊不加廣而君私家富累萬金,門下不見一賢者。文聞將門必有將,相門必有相。今君後宮蹈綺縠而士不得(短)[裋]褐,仆妾餘粱肉而士不厭糟糠。今君又尚厚積餘藏,欲以遺所不知何人,而忘公家之事日損,文竊怪之。」於是嬰乃禮文,使主家待賓客。賓客日進,名聲聞於諸侯。諸侯皆使人請薛公田嬰以文為太子,嬰許之。嬰卒,謚為靖郭君。而文果代立於薛,是為孟嘗君。

孟嘗君在薛,招致諸侯賓客及亡人有罪者,皆歸孟嘗君。孟嘗君舍業厚遇之,以故傾天下之士。食客數千人,無貴賤一與文等。孟嘗君待客坐語,而屏風後常有侍史,主記君所與客語,問親戚居處。客去,孟嘗君已使使存問,獻遺其親戚。孟嘗君曾待客夜食,有一人蔽火光。客怒,以飯不等,輟食辭去。孟嘗君起,自持其飯比之。客慚,自剄。士以此多歸孟嘗君。孟嘗君客無所擇,皆善遇之。人人各自以為孟嘗君親己。

秦昭王聞其賢,乃先使涇陽君為質於齊,以求見孟嘗君。孟嘗君將入秦,賓客莫欲其行,諫,不聽。蘇代謂曰:「今旦代從外來,見木禺人與土禺人相與語。木禺人曰:『天雨,子將敗矣。』土禺人曰:『我生於土,敗則歸土。今天雨,流子而行,未知所止息也。』今秦,虎狼之國也,而君欲往,如有不得還,君得無為土禺人所笑乎?」孟嘗君乃止。

齊湣王二十五年,復卒使孟嘗君入秦,昭王即以孟嘗君為秦相。人或說秦昭王曰:「孟嘗君賢,而又齊族也,今相秦,必先齊而後秦,秦其危矣。」於是秦昭王乃止。囚孟嘗君,謀欲殺之。孟嘗君使人抵昭王幸姬求解。幸姬曰:「妾願得君狐白裘。」此時孟嘗君有一狐白裘,直千金,天下無雙,入秦獻之昭王,更無他裘。孟嘗君患之,遍問客,莫能對。最下坐有能為狗盜者,曰:「臣能得狐白裘。」乃夜為狗,以入秦宮臧中,取所獻狐白裘至,以獻秦王幸姬。幸姬為言昭王,昭王釋孟嘗君。孟嘗君得出,即馳去,更封傳,變名姓以出關。夜半至函谷關。秦昭王後悔出孟嘗君,求之已去,即使人馳傳逐之。孟嘗君至關,關法雞鳴而出客,孟嘗君恐追至,客之居下坐者有能為雞鳴,而雞齊鳴,遂發傳出。出如食頃,秦追果至關,已後孟嘗君出,乃還。始孟嘗君列此二人於賓客,賓客盡羞之,及孟嘗君有秦難,卒此二人拔之。自是之後,客皆服。

孟嘗君過趙,趙平原君客之。趙人聞孟嘗君賢,出觀之,皆笑曰:「始以薛公為魁然也,今視之,乃眇小丈夫耳。」孟嘗君聞之,怒。客與俱者下,斫擊殺數百人,遂滅一縣以去。

齊湣王不自得,以其遣孟嘗君。孟嘗君至,則以為齊相,任政。

孟嘗君怨秦,將以齊為韓、魏攻楚,因與韓、魏攻秦,而借兵食於西周。蘇代為西周謂曰:「君以齊為韓、魏攻楚九年,取宛、葉以北以彊韓、魏,今復攻秦以益之。韓、魏南無楚憂,西無秦患,則齊危矣。韓、魏必輕齊畏秦,臣為君危之。君不如令敝邑深合於秦,而君無攻,又無借兵食。君臨函谷而無攻,令敝邑以君之情謂秦昭王曰『薛公必不破秦以彊韓、魏。其攻秦也,欲王之令楚王割東國以與齊,而秦出楚懷王以為和』。君令敝邑以此惠秦,秦得無破而以東國自免也,秦必欲之。楚王得出,必德齊。齊得東國益彊,而薛世世無患矣。秦不大弱,而處三晉之西,三晉必重齊。」薛公曰:「善。」因令韓、魏賀秦,使三國無攻,而不借兵食於西周矣。是時,楚懷王入秦,秦留之,故欲必出之。秦不果出楚懷王。

孟嘗君相齊,其舍人魏子為孟嘗君收邑入,三反而不致一入。孟嘗君問之,對曰:「有賢者,竊假與之,以故不致入。」孟嘗君怒而退魏子。居數年,人或毀孟嘗君於齊湣王曰:「孟嘗君將為亂。」及田甲劫湣王,湣王意疑孟嘗君,孟嘗君乃奔。魏子所與粟賢者聞之,乃上書言孟嘗君不作亂,請以身為盟,遂自剄宮門以明孟嘗君。湣王乃驚,而蹤跡驗問,孟嘗君果無反謀,乃復召孟嘗君。孟嘗君因謝病,歸老於薛。湣王許之。

其後,秦亡將呂禮相齊,欲困蘇代。代乃謂孟嘗君曰:「周最於齊,至厚也,而齊王逐之,而聽親弗相呂禮者,欲取秦也。齊、秦合,則親弗與呂禮重矣。有用,齊、秦必輕君。君不如急北兵,趨趙以和秦、魏,收周最以厚行,且反齊王之信,又禁天下之變。齊無秦,則天下集齊,親弗必走,則齊王孰與為其國也!」於是孟嘗君從其計,而呂禮嫉害於孟嘗君。

孟嘗君懼,乃遺秦相穰侯魏冉書曰:「吾聞秦欲以呂禮收齊,齊,天下之彊國也,子必輕矣。齊秦相取以臨三晉,呂禮必并相矣,是子通齊以重呂禮也。若齊免於天下之兵,其讎子必深矣。子不如勸秦王伐齊。齊破,吾請以所得封子。齊破,秦畏晉之彊,秦必重子以取晉。晉國敝於齊而畏秦,晉必重子以取秦。是子破齊以為功,挾晉以為重;是子破齊定封,秦、晉交重子。若齊不破,呂禮復用,子必大窮。」於是穰侯言於秦昭王伐齊,而呂禮亡。

後齊湣王滅宋,益驕,欲去孟嘗君。孟嘗君恐,乃如魏。魏昭王以為相,西合於秦、趙,與燕共伐破齊。齊湣王亡在莒,遂死焉。齊襄王立,而孟嘗君中立於諸侯,無所屬。齊襄王新立,畏孟嘗君,與連和,復親薛公。文卒,謚為孟嘗君。諸子爭立,而齊魏共滅薛。孟嘗絕嗣無後也。

初,馮驩聞孟嘗君好客,躡蹻而見之。孟嘗君曰;「先生遠辱,何以教文也?」馮驩曰:「聞君好士,以貧身歸於君。」孟嘗君置傳舍十日,孟嘗君問傳舍長曰:「客何所為?」答曰:「馮先生甚貧,猶有一劍耳,又蒯緱。彈其劍而歌曰『長鋏歸來乎,食無魚』。」孟嘗君遷之幸舍,食有魚矣。五日,又問傳舍長。答曰:「客復彈劍而歌曰『長鋏歸來乎,出無輿』。」孟嘗君遷之代舍,出入乘輿車矣。五日,孟嘗君復問傳舍長。舍長答曰:「先生又嘗彈劍而歌曰『長鋏歸來乎,無以為家』。」孟嘗君不悅。

居朞年,馮驩無所言。孟嘗君時相齊,封萬戶於薛。其食客三千人。邑入不足以奉客,使人出錢於薛。歲餘不入,貸錢者多不能與其息,客奉將不給。孟嘗君憂之,問左右:「何人可使收債於薛者?」傳舍長曰:「代舍客馮公形容狀貌甚辯,長者,無他伎能,宜可令收債。」孟嘗君乃進馮驩而請之曰:「賓客不知文不肖,幸臨文者三千餘人,邑入不足以奉賓客,故出息錢於薛。薛歲不入,民頗不與其息。今客食恐不給,願先生責之。」馮驩曰;「諾。」辭行,至薛,召取孟嘗君錢者皆會,得息錢十萬。乃多釀酒,買肥牛,召諸取錢者,能與息者皆來,不能與息者亦來,皆持取錢之券書合之。齊為會,日殺牛置酒。酒酣,乃持券如前合之,能與息者,與為期;貧不能與息者,取其券而燒之。曰:「孟嘗君所以貸錢者,為民之無者以為本業也;所以求息者,為無以奉客也。今富給者以要期,貧窮者燔券書以捐之。諸君彊飲食。有君如此,豈可負哉!」坐者皆起,再拜。

孟嘗君聞馮驩燒券書,怒而使使召驩。驩至,孟嘗君曰:「文食客三千人,故貸錢於薛。文奉邑少,而民尚多不以時與其息,客食恐不足,故請先生收責之。聞先生得錢,即以多具牛酒而燒券書,何?」馮驩曰:「然。不多具牛酒即不能畢會,無以知其有餘不足。有餘者,為要期。不足者,雖守而責之十年,息愈多,急,即以逃亡自捐之。若急,終無以償,上則為君好利不愛士民,下則有離上抵負之名,非所以厲士民彰君聲也。焚無用虛債之券,捐不可得之虛計,令薛民親君而彰君之善聲也,君有何疑焉!」孟嘗君乃拊手而謝之。

齊王惑於秦、楚之毀,以為孟嘗君名高其主而擅齊國之權,遂廢孟嘗君。諸客見孟嘗君廢,皆去。馮驩曰:「借臣車一乘,可以入秦者,必令君重於國而奉邑益廣,可乎?」孟嘗君乃約車幣而遣之。馮驩乃西說秦王曰:「天下之游士馮軾結靷西入秦者,無不欲彊秦而弱齊;馮軾結靷東入齊者,無不欲彊齊而弱秦。此雄雌之國也,勢不兩立為雄,雄者得天下矣。」秦王跽而問之曰:「何以使秦無為雌而可?」馮驩曰:「王亦知齊之廢孟嘗君乎?」秦王曰:「聞之。」馮驩曰:「使齊重於天下者,孟嘗君也。今齊王以毀廢之,其心怨,必背齊;背齊入秦,則齊國之情,人事之誠,盡委之秦,齊地可得也,豈直為雄也!君急使使載幣陰迎孟嘗君,不可失時也。如有齊覺悟,復用孟嘗君,則雌雄之所在未可知也。」秦王大悅,乃遣車十乘黃金百鎰以迎孟嘗君。馮驩辭以先行,至齊,說齊王曰:「天下之游士馮軾結靷東入齊者,無不欲彊齊而弱秦者;馮軾結靷西入秦者,無不欲彊秦而弱齊者。夫秦齊雄雌之國,秦彊則齊弱矣,此勢不兩雄。今臣竊聞秦遣使車十乘載黃金百鎰以迎孟嘗君。孟嘗君不西則已,西入相秦則天下歸之,秦為雄而齊為雌,雌則臨淄、即墨危矣。王何不先秦使之未到,復孟嘗君,而益與之邑以謝之?孟嘗君必喜而受之。秦雖彊國,豈可以請人相而迎之哉!折秦之謀,而絕其霸彊之略。」齊王曰:「善。」乃使人至境候秦使。秦使車適入齊境,使還馳告之,王召孟嘗君而復其相位,而與其故邑之地,又益以千戶。秦之使者聞孟嘗君復相齊,還車而去矣。

自齊王毀廢孟嘗君,諸客皆去。後召而復之,馮驩迎之。未到,孟嘗君太息嘆曰:「文常好客,遇客無所敢失,食客三千有餘人,先生所知也。客見文一日廢,皆背文而去,莫顧文者。今賴先生得復其位,客亦有何面目復見文乎?如復見文者,必唾其面而大辱之。」馮驩結轡下拜。孟嘗君下車接之,曰:「先生為客謝乎?」馮驩曰:「非為客謝也,為君之言失。夫物有必至,事有固然,君知之乎?」孟嘗君曰:「愚不知所謂也。」曰:「生者必有死,物之必至也;富貴多士,貧賤寡友,事之固然也。君獨不見夫(朝)趣市[朝]者乎?明旦,側肩爭門而入;日暮之後,過市朝者掉臂而不顧。非好朝而惡暮,所期物忘其中。今君失位,賓客皆去,不足以怨士而徒絕賓客之路。願君遇客如故。」孟嘗君再拜曰:「敬從命矣。聞先生之言,敢不奉教焉。」

太史公曰:吾嘗過薛,其俗閭里率多暴桀子弟,與鄒、魯殊。問其故,曰:「孟嘗君招致天下任俠,姦人入薛中蓋六萬餘家矣。」世之傳孟嘗君好客自喜,名不虛矣。


\end{pinyinscope}