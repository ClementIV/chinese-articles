\article{游俠列傳}

\begin{pinyinscope}
韓子曰:「儒以文亂法,而俠以武犯禁。」二者皆譏,而學士多稱於世云。至如以術取宰相卿大夫,輔翼其世主,功名俱著於春秋,固無可言者。及若季次、原憲,閭巷人也,讀書懷獨行君子之德,義不茍合當世,當世亦笑之。故季次、原憲終身空室蓬戶,褐衣疏食不厭。死而已四百餘年,而弟子志之不倦。今游俠,其行雖不軌於正義,然其言必信,其行必果,已諾必誠,不愛其軀,赴士之阸困,既已存亡死生矣,而不矜其能,羞伐其德,蓋亦有足多者焉。

且緩急,人之所時有也。太史公曰:昔者虞舜窘於井廩,伊尹負於鼎俎,傅說匿於傅險,呂尚困於棘津,夷吾桎梏,百里飯牛,仲尼畏匡,菜色陳、蔡。此皆學士所謂有道仁人也,猶然遭此菑,況以中材而涉亂世之末流乎?其遇害何可勝道哉!

鄙人有言曰:「何知仁義,已饗其利者為有德。」故伯夷丑周,餓死首陽山,而文武不以其故貶王;跖、蹻暴戾,其徒誦義無窮。由此觀之,「竊鉤者誅,竊國者侯,侯之門仁義存」,非虛言也。

今拘學或抱咫尺之義,久孤於世,豈若卑論儕俗,與世沈浮而取榮名哉!而布衣之徒,設取予然諾,千里誦義,為死不顧世,此亦有所長,非茍而已也。故士窮窘而得委命,此豈非人之所謂賢豪閒者邪?誠使鄉曲之俠,予季次、原憲比權量力,效功於當世,不同日而論矣。要以功見言信,俠客之義又曷可少哉!

古布衣之俠,靡得而聞已。近世延陵、孟嘗、春申、平原、信陵之徒,皆因王者親屬,藉於有土卿相之富厚,招天下賢者,顯名諸侯,不可謂不賢者矣。比如順風而呼,聲非加疾,其埶激也。至如閭巷之俠,修行砥名,聲施於天下,莫不稱賢,是為難耳。然儒、墨皆排擯不載。自秦以前,匹夫之俠,湮滅不見,余甚恨之。以余所聞,漢興有朱家、田仲、王公、劇孟、郭解之徒,雖時捍當世之文罔,然其私義廉絜退讓,有足稱者。名不虛立,士不虛附。至如朋黨宗彊比周,設財役貧,豪暴侵淩孤弱,恣欲自快,游俠亦丑之。余悲世俗不察其意,而猥以朱家、郭解等令與暴豪之徒同類而共笑之也。

魯朱家者,與高祖同時。魯人皆以儒教,而朱家用俠聞。所藏活豪士以百數,其餘庸人不可勝言。然終不伐其能,歆其德,諸所嘗施,唯恐見之。振人不贍,先從貧賤始。家無餘財,衣不完采,食不重味,乘不過軥牛。專趨人之急,甚己之私。既陰脫季布將軍之阸,及布尊貴,終身不見也。自關以東,莫不延頸願交焉。

楚田仲以俠聞,喜劍,父事朱家,自以為行弗及。田仲已死,而雒陽有劇孟。周人以商賈為資,而劇孟以任俠顯諸侯。吳楚反時,條侯為太尉,乘傳車將至河南,得劇孟,喜曰:「吳楚舉大事而不求孟,吾知其無能為已矣。」天下騷動,宰相得之若得一敵國云。劇孟行大類朱家,而好博,多少年之戲。然劇孟母死,自遠方送喪蓋千乘。及劇孟死,家無餘十金之財。而符離人王孟亦以俠稱江淮之閒。

是時濟南瞷氏、陳周庸亦以豪聞,景帝聞之,使使盡誅此屬。其後代諸白、梁韓無辟、陽翟薛兄、陜韓孺紛紛復出焉。

郭解,軹人也,字翁伯,善相人者許負外孫也。解父以任俠,孝文時誅死。解為人短小精悍,不飲酒。少時陰賊,慨不快意,身所殺甚眾。以軀借交報仇,藏命作姦剽攻,(不)休(及)[乃]鑄錢掘冢,固不可勝數。適有天幸,窘急常得脫,若遇赦。及解年長,更折節為儉,以德報怨,厚施而薄望。然其自喜為俠益甚。既已振人之命,不矜其功,其陰賊著於心,卒發於睚金故云。而少年慕其行,亦輒為報仇,不使知也。解姊子負解之勢,與人飲,使之嚼。非其任,彊必灌之。人怒,拔刀刺殺解姊子,亡去。解姊怒曰:「以翁伯之義,人殺吾子,賊不得。」棄其尸於道,弗葬,欲以辱解。解使人微知賊處。賊窘自歸,具以實告解。解曰:「公殺之固當,吾兒不直。」遂去其賊,罪其姊子,乃收而葬之。諸公聞之,皆多解之義,益附焉。

解出入,人皆避之。有一人獨箕倨視之,解遣人問其名姓。客欲殺之。解曰:「居邑屋至不見敬,是吾德不修也,彼何罪!」乃陰屬尉史曰:「是人,吾所急也,至踐更時脫之。」每至踐更,數過,吏弗求。怪之,問其故,乃解使脫之。箕踞者乃肉袒謝罪。少年聞之,愈益慕解之行。

雒陽人有相仇者,邑中賢豪居閒者以十數,終不聽。客乃見郭解。解夜見仇家,仇家曲聽解。解乃謂仇家曰:「吾聞雒陽諸公在此閒,多不聽者。今子幸而聽解,解柰何乃從他縣奪人邑中賢大夫權乎!」乃夜去,不使人知,曰:「且無用,(待我)待我去,令雒陽豪居其閒,乃聽之。」

解執恭敬,不敢乘車入其縣廷。之旁郡國,為人請求事,事可出,出之;不可者,各厭其意,然後乃敢嘗酒食。諸公以故嚴重之,爭為用。邑中少年及旁近縣賢豪,夜半過門常十餘車,請得解客舍養之。

及徙豪富茂陵也,解家貧,不中訾,吏恐,不敢不徙。衛將軍為言:「郭解家貧不中徙。」上曰:「布衣權至使將軍為言,此其家不貧。」解家遂徙。諸公送者出千餘萬。軹人楊季主子為縣掾,舉徙解。解兄子斷楊掾頭。由此楊氏與郭氏為仇。

解入關,關中賢豪知與不知,聞其聲,爭交驩解。解為人短小,不飲酒,出未嘗有騎。已又殺楊季主。楊季主家上書,人又殺之闕下。上聞,乃下吏捕解。解亡,置其母家室夏陽,身至臨晉。臨晉籍少公素不知解,解冒,因求出關。籍少公已出解,解轉入太原,所過輒告主人家。吏逐之,跡至籍少公。少公自殺,口絕。久之,乃得解。窮治所犯,為解所殺,皆在赦前。軹有儒生侍使者坐,客譽郭解,生曰:「郭解專以姦犯公法,何謂賢!」解客聞,殺此生,斷其舌。吏以此責解,解實不知殺者。殺者亦竟絕,莫知為誰。吏奏解無罪。御史大夫公孫弘議曰:「解布衣為任俠行權,以睚眥殺人,解雖弗知,此罪甚於解殺之。當大逆無道。」遂族郭解翁伯。

自是之後,為俠者極眾,敖而無足數者。然關中長安樊仲子,槐裏趙王孫,長陵高公子,西河郭公仲,太原鹵公孺,臨淮兒長卿,東陽田君孺,雖為俠而逡逡有退讓君子之風。至若北道姚氏,西道諸杜,南道仇景,東道趙他、羽公子,南陽趙調之徒,此盜跖居民閒者耳,曷足道哉!此乃鄉者朱家之羞也。

太史公曰:吾視郭解,狀貌不及中人,言語不足採者。然天下無賢與不肖,知與不知,皆慕其聲,言俠者皆引以為名。諺曰:「人貌榮名,豈有既乎!」於戲,惜哉!


\end{pinyinscope}