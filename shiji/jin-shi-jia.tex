\article{晉世家}

\begin{pinyinscope}
晉唐叔虞者,周武王子而成王弟。初,武王與叔虞母會時,夢天謂武王曰:「余命女生子,名虞,余與之唐。」及生子,文在其手曰「虞」,故遂因命之曰虞。

武王崩,成王立,唐有亂,周公誅滅唐。成王與叔虞戲,削桐葉為珪以與叔虞,曰:「以此封若。」史佚因請擇日立叔虞。成王曰:「吾與之戲耳。」史佚曰:「天子無戲言。言則史書之,禮成之,樂歌之。」於是遂封叔虞於唐。唐在河、汾之東,方百里,故曰唐叔虞。姓姬氏,字子于。

唐叔子燮,是為晉侯。晉侯子寧族,是為武侯。武侯之子服人,是為成侯。成侯子福,是為厲侯。厲侯之子宜臼,是為靖侯。靖侯已來,年紀可推。自唐叔至靖侯五世,無其年數。

靖侯十七年,周厲王迷惑暴虐,國人作亂,厲王出奔于彘,大臣行政,故曰「共和」。

十八年,靖侯卒,子釐侯司徒立。釐侯十四年,周宣王初立。十八年,釐侯卒,子獻侯籍立。獻侯十一年卒,子穆侯費王立。

穆侯四年,取齊女姜氏為夫人。七年,伐條。生太子仇。十年,伐千畝,有功。生少子,名曰成師。晉人師服曰:「異哉,君之命子也!太子曰仇,仇者讎也。少子曰成師,成師大號,成之者也。名自命也;物自定也。今適庶名反逆,此後晉其能毋亂乎?」

二十七年,穆侯卒,弟殤叔自立,太子仇出奔。殤叔三年,周宣王崩。四年,穆侯太子仇率其徒襲殤叔而立,是為文侯。

文侯十年,周幽王無道,犬戎殺幽王,周東徙。而秦襄公始列為諸侯。

三十五年,文侯仇卒,子昭侯伯立。

昭侯元年,封文侯弟成師于曲沃。曲沃邑大於翼。翼,晉君都邑也。成師封曲沃,號為桓叔。靖侯庶孫欒賓相桓叔。桓叔是時年五十八矣,好德,晉國之眾皆附焉。君子曰:「晉之亂其在曲沃矣。末大於本而得民心,不亂何待!」

七年,晉大臣潘父弒其君昭侯而迎曲沃桓叔。桓叔欲入晉,晉人發兵攻桓叔。桓叔敗,還歸曲沃。晉人共立昭侯子平為君,是為孝侯。誅潘父。

孝侯八年,曲沃桓叔卒,子胛桓叔,是為曲沃莊伯。孝侯十五年,曲沃莊伯弒其君晉孝侯于翼。晉人攻曲沃莊伯,莊伯復入曲沃。晉人復立孝侯子郄為君,是為鄂侯。

鄂侯六年卒。曲沃莊伯聞晉鄂侯卒,乃興兵伐晉。周平王使虢公將兵伐曲沃莊伯,莊伯走保曲沃。晉人共立鄂侯子光,是為哀侯。

哀侯二年曲沃莊伯卒,子稱代莊伯立,是為曲沃武公。哀侯六年,魯弒其君隱公。哀侯八年,晉侵陘廷。陘廷與曲沃武公謀,九年,伐晉于汾旁,虜哀侯。晉人乃立哀侯子小子為君,是為小子侯。

小子元年,曲沃武公使韓萬殺所虜晉哀侯。曲沃益彊,晉無如之何。

晉小子之四年,曲沃武公誘召晉小子殺之。周桓王使虢仲伐曲沃武公,武公入于曲沃,乃立晉哀侯弟緡為晉侯。

晉侯緡四年,宋執鄭祭仲而立突為鄭君。晉侯十九年,齊人管至父弒其君襄公。

晉侯二十八年,齊桓公始霸。曲沃武公伐晉侯緡,滅之,盡以其寶器賂獻于周釐王。釐王命曲沃武公為晉君,列為諸侯,於是盡併晉地而有之。

曲沃武公已即位三十七年矣,更號曰晉武公。晉武公始都晉國,前即位曲沃,通年三十八年。

武公稱者,先晉穆侯曾孫也,曲沃桓叔孫也。桓叔者,始封曲沃。武公,莊伯子也。自桓叔初封曲沃以至武公滅晉也,凡六十七歲,而卒代晉為諸侯。武公代晉二歲,卒。與曲沃通年,即位凡三十九年而卒。子獻公詭諸立。

獻公元年,周惠王弟穨攻惠王,惠王出奔,居鄭之櫟邑。

五年,伐驪戎,得驪姬、驪姬弟,俱愛幸之。

八年,士蒍說公曰:「故晉之群公子多,不誅,亂且起。」乃使盡殺諸公子,而城聚都之,命曰絳,始都絳。九年,晉群公子既亡奔虢,虢以其故再伐晉,弗克。十年,晉欲伐虢,士蒍曰:「且待其亂。」

十二年,驪姬生奚齊。獻公有意廢太子,乃曰:「曲沃吾先祖宗廟所在,而蒲邊秦,屈邊翟,不使諸子居之,我懼焉。」於是使太子申生居曲沃,公子重耳居蒲,公子夷吾居屈。獻公與驪姬子奚齊居絳。晉國以此知太子不立也。太子申生,其母齊桓公女也,曰齊姜,早死。申生同母女弟為秦穆公夫人。重耳母,翟之狐氏女也。夷吾母,重耳母女弟也。獻公子八人,而太子申生、重耳、夷吾皆有賢行。及得驪姬,乃遠此三子。

十六年,晉獻公作二軍。公將上軍,太子申生將下軍,趙夙御戎,畢萬為右,伐滅霍,滅魏,滅耿。還,為太子城曲沃,賜趙夙耿,賜畢萬魏,以為大夫。士蒍曰:「太子不得立矣。分之都城,而位以卿,先為之極,又安得立!不如逃之,無使罪至。為吳太伯,不亦可乎,猶有令名。」太子不從。卜偃曰:「畢萬之後必大。萬,盈數也;魏,大名也。以是始賞,天開之矣。天子曰兆民,諸侯曰萬民,今命之大,以從盈數,其必有眾。」初,畢萬卜仕於晉國,遇屯之比。辛廖占之曰:「吉。屯固比入,吉孰大焉。其後必蕃昌。」

十七年,晉侯使太子申生伐東山。裏克諫獻公曰:「太子奉冢祀社稷之粢盛,以朝夕視君膳者也,故曰冢子。君行則守,有守則從,從曰撫軍,守曰監國,古之制也。夫率師,專行謀也;誓軍旅,君與國政之所圖也:非太子之事也。師在制命而已,稟命則不威,專命則不孝,故君之嗣適不可以帥師。君失其官,率師不威,將安用之?」公曰:「寡人有子,未知其太子誰立。」裏克不對而退,見太子。太子曰:「吾其廢乎?」裏克曰:「太子勉之!教以軍旅,不共是懼,何故廢乎?且子懼不孝,毋懼不得立。修己而不責人,則免於難。」太子帥師,公衣之偏衣,佩之金玦。裏克謝病,不從太子。太子遂伐東山。

十九年,獻公曰:「始吾先君莊伯、武公之誅晉亂,而虢常助晉伐我,又匿晉亡公子,果為亂。弗誅,後遺子孫憂。」乃使荀息以屈產之乘假道於虞。虞假道,遂伐虢,取其下陽以歸。

獻公私謂驪姬曰:「吾欲廢太子,以奚齊代之。」驪姬泣曰:「太子之立,諸侯皆已知之,而數將兵,百姓附之,柰何以賤妾之故廢適立庶?君必行之,妾自殺也。」驪姬詳譽太子,而陰令人譖惡太子,而欲立其子。

二十一年,驪姬謂太子曰:「君夢見齊姜,太子速祭曲沃,歸釐於君。」太子於是祭其母齊姜於曲沃,上其薦胙於獻公。獻公時出獵,置胙於宮中。驪姬使人置毒藥胙中。居二日,獻公從獵來還,宰人上胙獻公,獻公欲饗之。驪姬從旁止之,曰:「胙所從來遠,宜試之。」祭地,地墳;與犬,犬死;與小臣,小臣死。驪姬泣曰:「太子何忍也!其父而欲弒代之,況他人乎?且君老矣,旦暮之人,曾不能待而欲弒之!」謂獻公曰:「太子所以然者,不過以妾及奚齊之故。妾願子母辟之他國,若早自殺,毋徒使母子為太子所魚肉也。始君欲廢之,妾猶恨之;至於今,妾殊自失於此。」太子聞之,奔新城。獻公怒,乃誅其傅杜原款。或謂太子曰:「為此藥者乃驪姬也,太子何不自辭明之?」太子曰:「吾君老矣,非驪姬,寢不安,食不甘。即辭之,君且怒之。不可。」或謂太子曰:「可奔他國。」太子曰:「被此惡名以出,人誰內我?我自殺耳。」十二月戊申,申生自殺於新城。

此時重耳、夷吾來朝。人或告驪姬曰:「二公子怨驪姬譖殺太子。」驪姬恐,因譖二公子:「申生之藥胙,二公子知之。」二子聞之,恐,重耳走蒲,夷吾走屈,保其城,自備守。初,獻公使士蒍為二公子筑蒲、屈城,弗就。夷吾以告公,公怒士蒍。士蒍謝曰:「邊城少寇,安用之?」退而歌曰:「狐裘蒙茸,一國三公,吾誰適從!」卒就城。及申生死,二子亦歸保其城。

二十二年,獻公怒二子不辭而去,果有謀矣,乃使兵伐蒲。蒲人之宦者勃鞮命重耳促自殺。重耳踰垣,宦者追斬其衣袪。重耳遂奔翟。使人伐屈,屈城守,不可下。

是歲也,晉復假道於虞以伐虢。虞之大夫宮之奇諫虞君曰:「晉不可假道也,是且滅虞。」虞君曰:「晉我同姓,不宜伐我。」宮之奇曰:「太伯、虞仲,太王之子也,太伯亡去,是以不嗣。虢仲、虢叔,王季之子也,為文王卿士,其記勳在王室,藏於盟府。將虢是滅,何愛于虞?且虞之親能親於桓、莊之族乎?桓、莊之族何罪,盡滅之。虞之與虢,脣之與齒,脣亡則齒寒。」虞公不聽,遂許晉。宮之奇以其族去虞。其冬,晉滅虢,虢公丑奔周。還,襲滅虞,虜虞公及其大夫井伯百里奚以媵秦穆姬,而修虞祀。荀息牽曩所遺虞屈產之乘馬奉之獻公,獻公笑曰:「馬則吾馬,齒亦老矣!」

二十三年,獻公遂發賈華等伐屈,屈潰。夷吾將奔翟。冀芮曰:「不可,重耳已在矣,今往,晉必移兵伐翟,翟畏晉,禍且及。不如走梁,梁近於秦,秦彊,吾君百歲後可以求入焉。」遂奔梁。二十五年,晉伐翟,翟以重耳故,亦擊晉於齧桑,晉兵解而去。

當此時,晉彊,西有河西,與秦接境,北邊翟,東至河內。

驪姬弟生悼子。

二十六年夏,齊桓公大會諸侯於葵丘。晉獻公病,行後,未至,逢周之宰孔。宰孔曰:「齊桓公益驕,不務德而務遠略,諸侯弗平。君弟毋會,毋如晉何。」獻公亦病,復還歸。病甚,乃謂荀息曰:「吾以奚齊為後,年少,諸大臣不服,恐亂起,子能立之乎?」荀息曰:「能。」獻公曰:「何以為驗?」對曰:「使死者復生,生者不慚,為之驗。」於是遂屬奚齊於荀息。荀息為相,主國政。秋九月,獻公卒。裏克、邳鄭欲內重耳,以三公子之徒作亂,謂荀息曰:「三怨將起,秦、晉輔之,子將何如?」荀息曰:「吾不可負先君言。」十月,裏克殺奚齊于喪次,獻公未葬也。荀息將死之,或曰不如立奚齊弟悼子而傅之,荀息立悼子而葬獻公。十一月,裏克弒悼子于朝,荀息死之。君子曰:「詩所謂『白珪之玷,猶可磨也,斯言之玷,不可為也』,其荀息之謂乎!不負其言。」初,獻公將伐驪戎,卜曰「齒牙為禍」。及破驪戎,獲驪姬,愛之,竟以亂晉。

裏克等已殺奚齊、悼子,使人迎公子重耳於翟,欲立之。重耳謝曰:「負父之命出奔,父死不得修人子之禮侍喪,重耳何敢入!大夫其更立他子。」還報裏克,裏克使迎夷吾於梁。夷吾欲往,呂省、郤芮曰:「內猶有公子可立者而外求,難信。計非之秦,輔彊國之威以入,恐危。」乃使郤芮厚賂秦,約曰:「即得入,請以晉河西之地與秦。」及遺裏克書曰:「誠得立,請遂封子於汾陽之邑。」秦繆公乃發兵送夷吾於晉。齊桓公聞晉內亂,亦率諸侯如晉。秦兵與夷吾亦至晉,齊乃使隰朋會秦俱入夷吾,立為晉君,是為惠公。齊桓公至晉之高梁而還歸。

惠公夷吾元年,使邳鄭謝秦曰:「始夷吾以河西地許君,今幸得入立。大臣曰:『地者先君之地,君亡在外,何以得擅許秦者?』寡人爭之弗能得,故謝秦。」亦不與裏克汾陽邑,而奪之權。四月,周襄王使周公忌父會齊、秦大夫共禮晉惠公。惠公以重耳在外,畏裏克為變,賜裏克死。謂曰:「微裏子寡人不得立。雖然,子亦殺二君一大夫,為子君者不亦難乎?」裏克對曰:「不有所廢,君何以興?欲誅之,其無辭乎?乃言為此!臣聞命矣。」遂伏劍而死。於是邳鄭使謝秦未還,故不及難。

晉君改葬恭太子申生。秋,狐突之下國,遇申生,申生與載而告之曰:「夷吾無禮,余得請於帝,將以晉與秦,秦將祀余。」狐突對曰:「臣聞神不食非其宗,君其祀毋乃絕乎?君其圖之。」申生曰:「諾,吾將復請帝。後十日,新城西偏將有巫者見我焉。」許之,遂不見。及期而往,復見,申生告之曰:「帝許罰有罪矣,弊於韓。」兒乃謠曰:「恭太子更葬矣,後十四年,晉亦不昌,昌乃在兄。」

邳鄭使秦,聞裏克誅,乃說秦繆公曰:「呂省、郤稱、冀芮實為不從。若重賂與謀,出晉君,入重耳,事必就。」秦繆公許之,使人與歸報晉,厚賂三子。三子曰:「幣厚言甘,此必邳鄭賣我於秦。」遂殺邳鄭及裏克、邳鄭之黨七輿大夫。邳鄭子豹奔秦,言伐晉,繆公弗聽。

惠公之立,倍秦地及裏克,誅七輿大夫,國人不附。二年,周使召公過禮晉惠公,惠公禮倨,召公譏之。

四年,晉饑,乞糴於秦。繆公問百里奚,百里奚曰:「天菑流行,國家代有,救菑恤鄰,國之道也。與之。」邳鄭子豹曰:「伐之。」繆公曰:「其君是惡,其民何罪!」卒與粟,自雍屬絳。

五年,秦饑,請糴於晉。晉君謀之,慶鄭曰:「以秦得立,已而倍其地約。晉饑而秦貸我,今秦饑請糴,與之何疑?而謀之!」虢射曰:「往年天以晉賜秦,秦弗知取而貸我。今天以秦賜晉,晉其可以逆天乎?遂伐之。」惠公用虢射謀,不與秦粟,而發兵且伐秦。秦大怒,亦發兵伐晉。

六年春,秦繆公將兵伐晉。晉惠公謂慶鄭曰:「秦師深矣,柰何?」鄭曰:「秦內君,君倍其賂;晉饑秦輸粟,秦饑而晉倍之,乃欲因其饑伐之:其深不亦宜乎!」晉卜御右,慶鄭皆吉。公曰:「鄭不孫。」乃更令步陽御戎,家仆徒為右,進兵。九月壬戌,秦繆公、晉惠公合戰韓原。惠公馬騺不行,秦兵至,公窘,召慶鄭為御。鄭曰:「不用卜,敗不亦當乎!」遂去。更令梁繇靡御,虢射為右,輅秦繆公。繆公壯士冒敗晉軍,晉軍敗,遂失秦繆公,反獲晉公以歸。秦將以祀上帝。晉君姊為繆公夫人,衰绖涕泣。公曰:「得晉侯將以為樂,今乃如此。且吾聞箕子見唐叔之初封,曰『其後必當大矣』,晉庸可滅乎!」乃與晉侯盟王城而許之歸。晉侯亦使呂省等報國人曰:「孤雖得歸,毋面目見社稷,卜日立子圉。」晉人聞之,皆哭。秦繆公問呂省:「晉國和乎?」對曰:「不和。小人懼失君亡親,不憚立子圉,曰『必報讎,寧事戎、狄』。其君子則愛君而知罪,以待秦命,曰『必報德』。有此二故,不和。」於是秦繆公更舍晉惠公,餽之七牢。十一月,歸晉侯。晉侯至國,誅慶鄭,修政教。謀曰:「重耳在外,諸侯多利內之。」欲使人殺重耳於狄。重耳聞之,如齊。

八年,使太子圉質秦。初,惠公亡在梁,梁伯以其女妻之,生一男一女。梁伯卜之,男為人臣,女為人妾,故名男為圉,女為妾。

十年,秦滅梁。梁伯好土功,治城溝,民力罷怨,其眾數相驚,曰「秦寇至」,民恐惑,秦竟滅之。

十三年,晉惠公病,內有數子。太子圉曰:「吾母家在梁,梁今秦滅之,我外輕於秦而內無援於國。君即不起,病大夫輕,更立他公子。」乃謀與其妻俱亡歸。秦女曰:「子一國太子,辱在此。秦使婢子侍,以固子之心。子亡矣,我不從子,亦不敢言。」子圉遂亡歸晉。十四年九月,惠公卒,太子圉立,是為懷公。

子圉之亡,秦怨之,乃求公子重耳,欲內之。子圉之立,畏秦之伐也。乃令國中諸從重耳亡者與期,期盡不到者盡滅其家。狐突之子毛及偃從重耳在秦,弗肯召。懷公怒,囚狐突。突曰:「臣子事重耳有年數矣,今召之,是教之反君也。何以教之?」懷公卒殺狐突。秦繆公乃發兵送內重耳,使人告欒、郤之黨為內應,殺懷公於高梁,入重耳。重耳立,是為文公。

晉文公重耳,晉獻公之子也。自少好士,年十七,有賢士五人:曰趙衰;狐偃咎犯,文公舅也;賈佗;先軫;魏武子。自獻公為太子時,重耳固已成人矣。獻公即位,重耳年二十一。獻公十三年,以驪姬故,重耳備蒲城守秦。獻公二十一年,獻公殺太子申生,驪姬讒之,恐,不辭獻公而守蒲城。獻公二十二年,獻公使宦者履鞮趣殺重耳。重耳踰垣,宦者逐斬其衣袪。重耳遂奔狄。狄,其母國也。是時重耳年四十三。從此五士,其餘不名者數十人,至狄。

狄伐咎如,得二女:以長女妻重耳,生伯鯈、叔劉;以少女妻趙衰,生盾。居狄五歲而晉獻公卒,裏克已殺奚齊、悼子,乃使人迎,欲立重耳。重耳畏殺,因固謝,不敢入。已而晉更迎其弟夷吾立之,是為惠公。惠公七年,畏重耳,乃使宦者履鞮與壯士欲殺重耳。重耳聞之,乃謀趙衰等曰:「始吾奔狄,非以為可用與,以近易通,故且休足。休足久矣,固願徙之大國。夫齊桓公好善,志在霸王,收恤諸侯。今聞管仲、隰朋死,此亦欲得賢佐,盍往乎?」於是遂行。重耳謂其妻曰:「待我二十五年不來,乃嫁。」其妻笑曰:「犁二十五年,吾冢上柏大矣。雖然,妾待子。」重耳居狄凡十二年而去。

過衛,衛文公不禮。去,過五鹿,饑而從野人乞食,野人盛土器中進之。重耳怒。趙衰曰:「土者,有土也,君其拜受之。」

至齊,齊桓公厚禮,而以宗女妻之,有馬二十乘,重耳安之。重耳至齊二歲而桓公卒,會豎刀等為內亂,齊孝公之立,諸侯兵數至。留齊凡五歲。重耳愛齊女,毋去心。趙衰、咎犯乃於桑下謀行。齊女侍者在桑上聞之,以告其主。其主乃殺侍者,勸重耳趣行。重耳曰:「人生安樂,孰知其他!必死於此,不能去。」齊女曰:「子一國公子,窮而來此,數士者以子為命。子不疾反國,報勞臣,而懷女德,竊為子羞之。且不求,何時得功?」乃與趙衰等謀,醉重耳,載以行。行遠而覺,重耳大怒,引戈欲殺咎犯。咎犯曰:「殺臣成子,偃之願也。」重耳曰:「事不成,我食舅氏之肉。」咎犯曰:「事不成,犯肉腥臊,何足食!」乃止,遂行。

過曹,曹共公不禮,欲觀重耳駢脅。曹大夫釐負羈曰:「晉公子賢,又同姓,窮來過我,柰何不禮!」共公不從其謀。負羈乃私遺重耳食,置璧其下。重耳受其食,還其璧。

去,過宋。宋襄公新困兵於楚,傷於泓,聞重耳賢,乃以國禮禮於重耳。宋司馬公孫固善於咎犯,曰:「宋小國新困,不足以求入,更之大國。」乃去。

過鄭,鄭文公弗禮。鄭叔瞻諫其君曰:「晉公子賢,而其從者皆國相,且又同姓。鄭之出自厲王,而晉之出自武王。」鄭君曰:「諸侯亡公子過此者眾,安可盡禮!」叔瞻曰:「君不禮,不如殺之,且後為國患。」鄭君不聽。

重耳去之楚,楚成王以適諸侯禮待之,重耳謝不敢當。趙衰曰:「子亡在外十餘年,小國輕子,況大國乎?今楚大國而固遇子,子其毋讓,此天開子也。」遂以客禮見之。成王厚遇重耳,重耳甚卑。成王曰:「子即反國,何以報寡人?」重耳曰:「羽毛齒角玉帛,君王所餘,未知所以報。」王曰:「雖然,何以報不穀?」重耳曰:「即不得已,與君王以兵車會平原廣澤,請辟王三舍。」楚將子玉怒曰:「王遇晉公子至厚,今重耳言不孫,請殺之。」成王曰:「晉公子賢而困於外久,從者皆國器,此天所置,庸可殺乎?且言何以易之!」居楚數月,而晉太子圉亡秦,秦怨之;聞重耳在楚,乃召之。成王曰:「楚遠,更數國乃至晉。秦晉接境,秦君賢,子其勉行!」厚送重耳。

重耳至秦,繆公以宗女五人妻重耳,故子圉妻與往。重耳不欲受,司空季子曰:「其國且伐,況其故妻乎!且受以結秦親而求入,子乃拘小禮,忘大丑乎!」遂受。繆公大歡,與重耳飲。趙衰歌黍苗詩。繆公曰:「知子欲急反國矣。」趙衰與重耳下,再拜曰:「孤臣之仰君,如百穀之望時雨。」是時晉惠公十四年秋。惠公以九月卒,子圉立。十一月,葬惠公。十二月,晉國大夫欒、郤等聞重耳在秦,皆陰來勸重耳、趙衰等反國,為內應甚眾。於是秦繆公乃發兵與重耳歸晉。晉聞秦兵來,亦發兵拒之。然皆陰知公子重耳入也。唯惠公之故貴臣呂、郤之屬不欲立重耳。重耳出亡凡十九歲而得入,時年六十二矣,晉人多附焉。

文公元年春,秦送重耳至河。咎犯曰:「臣從君周旋天下,過亦多矣。臣猶知之,況於君乎?請從此去矣。」重耳曰:「若反國,所不與子犯共者,河伯視之!」乃投璧河中,以與子犯盟。是時介子推從,在船中,乃笑曰:「天實開公子,而子犯以為己功而要市於君,固足羞也。吾不忍與同位。」乃自隱渡河。秦兵圍令狐,晉軍于廬柳。二月辛丑,咎犯與秦晉大夫盟于郇。壬寅,重耳入于晉師。丙午,入于曲沃。丁未,朝于武宮,即位為晉君,是為文公。群臣皆往。懷公圉奔高梁。戊申,使人殺懷公。

懷公故大臣呂省、郤芮本不附文公,文公立,恐誅,乃欲與其徒謀燒公宮,殺文公。文公不知。始嘗欲殺文公宦者履鞮知其謀,欲以告文公,解前罪,求見文公。文公不見,使人讓曰:「蒲城之事,女斬予袪。其後我從狄君獵,女為惠公來求殺我。惠公與女期三日至,而女一日至,何速也?女其念之。」宦者曰:「臣刀鋸之餘,不敢以二心事君倍主,故得罪於君。君已反國,其毋蒲、翟乎?且管仲射鉤,桓公以霸。今刑餘之人以事告而君不見,禍又且及矣。」於是見之,遂以呂、郤等告文公。文公欲召呂、郤,呂、郤等黨多,文公恐初入國,國人賣己,乃為微行,會秦繆公於王城,國人莫知。三月己丑,呂、郤等果反,焚公宮,不得文公。文公之衛徒與戰,呂、郤等引兵欲奔,秦繆公誘呂、郤等,殺之河上,晉國復而文公得歸。夏,迎夫人於秦,秦所與文公妻者卒為夫人。秦送三千人為衛,以備晉亂。

文公修政,施惠百姓。賞從亡者及功臣,大者封邑,小者尊爵。未盡行賞,周襄王以弟帶難出居鄭地,來告急晉。晉初定,欲發兵,恐他亂起,是以賞從亡未至隱者介子推。推亦不言祿,祿亦不及。推曰:「獻公子九人,唯君在矣。惠、懷無親,外內棄之;天未絕晉,必將有主,主晉祀者,非君而誰?天實開之,二三子以為己力,不亦誣乎?竊人之財,猶曰是盜,況貪天之功以為己力乎?下冒其罪,上賞其姦,上下相蒙,難與處矣!」其母曰:「盍亦求之,以死誰懟?」推曰:「尤而效之,罪有甚焉。且出怨言,不食其祿。」母曰:「亦使知之,若何?」對曰:「言,身之文也;身欲隱,安用文之?文之,是求顯也。」其母曰:「能如此乎?與女偕隱。」至死不復見。

介子推從者憐之,乃懸書宮門曰:「龍欲上天,五蛇為輔。龍已升雲,四蛇各入其宇,一蛇獨怨,終不見處所。」文公出,見其書,曰:「此介子推也。吾方憂王室,未圖其功。」使人召之,則亡。遂求所在,聞其入綿上山中,於是文公環綿上山中而封之,以為介推田,號曰介山,「以記吾過,且旌善人」。

從亡賤臣壺叔曰;「君三行賞,賞不及臣,敢請罪。」文公報曰:「夫導我以仁義,防我以德惠,此受上賞。輔我以行,卒以成立,此受次賞。矢石之難,汗馬之勞,此復受次賞。若以力事我而無補吾缺者,此[復]受次賞。三賞之後,故且及子。」晉人聞之,皆說。

二年春,秦軍河上,將入王。趙衰曰;「求霸莫如入王尊周。周晉同姓,晉不先入王,後秦入之,毋以令于天下。方今尊王,晉之資也。」三月甲辰,晉乃發兵至陽樊,圍溫,入襄王于周。四月,殺王弟帶。周襄王賜晉河內陽樊之地。

四年,楚成王及諸侯圍宋,宋公孫固如晉告急。先軫曰:「報施定霸,於今在矣。」狐偃曰:「楚新得曹而初婚於衛,若伐曹、衛,楚必救之,則宋免矣。」於是晉作三軍。趙衰舉郤縠將中軍,郤臻佐之;使狐偃將上軍,狐毛佐之,命趙衰為卿;欒枝將下軍,先軫佐之;荀林父御戎,魏犫為右:往伐。冬十二月,晉兵先下山東,而以原封趙衰。

五年春,晉文公欲伐曹,假道於衛,衛人弗許。還自河南度,侵曹,伐衛。正月,取五鹿。二月,晉侯、齊侯盟于斂盂。衛侯請盟晉,晉人不許。衛侯欲與楚,國人不欲,故出其君以說晉。衛侯居襄牛,公子買守衛。楚救衛,不卒。晉侯圍曹。三月丙午,晉師入曹,數之以其不用釐負羈言,而用美女乘軒者三百人也。令軍毋入僖負羈宗家以報德。楚圍宋,宋復告急晉。文公欲救則攻楚,為楚嘗有德,不欲伐也;欲釋宋,宋又嘗有德於晉:患之。先軫曰:「執曹伯,分曹、衛地以與宋,楚急曹、衛,其勢宜釋宋。」於是文公從之,而楚成王乃引兵歸。

楚將子玉曰:「王遇晉至厚,今知楚急曹、衛而故伐之,是輕王。」王曰:「晉侯亡在外十九年,困日久矣,果得反國,險阸盡知之,能用其民,天之所開,不可當。」子玉請曰:「非敢必有功,願以閒執讒慝之口也。」楚王怒,少與之兵。於是子玉使宛春告晉:「請復衛侯而封曹,臣亦釋宋。」咎犯曰:「子玉無禮矣,君取一,臣取二,勿許。」先軫曰:「定人之謂禮。楚一言定三國,子一言而亡之,我則毋禮。不許楚,是棄宋也。不如私許曹、衛以誘之,執宛春以怒楚,既戰而後圖之。」晉侯乃囚宛春於衛,且私許復曹、衛。曹、衛告絕於楚。楚得臣怒,擊晉師,晉師退。軍吏曰:「為何退?」文公曰:「昔在楚,約退三舍,可倍乎!」楚師欲去,得臣不肯。四月戊辰,宋公、齊將、秦將與晉侯次城濮。己巳,與楚兵合戰,楚兵敗,得臣收餘兵去。甲午,晉師還至衡雍,作王宮于踐土。

初,鄭助楚,楚敗,懼,使人請盟晉侯。晉侯與鄭伯盟。

五月丁未,獻楚俘於周,駟介百乘,徒兵千。天子使王子虎命晉侯為伯,賜大輅,彤弓矢百,玈弓矢千,秬鬯一卣,珪瓚,虎賁三百人。晉侯三辭,然后稽首受之。周作晉文侯命:「王若曰:父義和,丕顯文、武,能慎明德,昭登於上,布聞在下,維時上帝集厥命于文、武。恤朕身、繼予一人永其在位。」於是晉文公稱伯。癸亥,王子虎盟諸侯於王庭。

晉焚楚軍,火數日不息,文公嘆。左右曰:「勝楚而君猶憂,何?」文公曰:「吾聞能戰勝安者唯聖人,是以懼。且子玉猶在,庸可喜乎!」子玉之敗而歸,楚成王怒其不用其言,貪與晉戰,讓責子玉,子玉自殺。晉文公曰:「我擊其外,楚誅其內,內外相應。」於是乃喜。

六月,晉人復入衛侯。壬午,晉侯度河北歸國。行賞,狐偃為首。或曰:「城濮之事,先軫之謀。」文公曰:「城濮之事,偃說我毋失信。先軫曰『軍事勝為右』,吾用之以勝。然此一時之說,偃言萬世之功,柰何以一時之利而加萬世功乎?是以先之。」

冬,晉侯會諸侯於溫,欲率之朝周。力未能,恐其有畔者,乃使人言周襄王狩于河陽。壬申,遂率諸侯朝王於踐土。孔子讀史記至文公,曰「諸侯無召王」、「王狩河陽」者,春秋諱之也。

丁丑,諸侯圍許。曹伯臣或說晉侯曰:「齊桓公合諸侯而國異姓,今君為會而滅同姓。曹,叔振鐸之後;晉,唐叔之後。合諸侯而滅兄弟,非禮。」晉侯說,復曹伯。

於是晉始作三行。荀林父將中行,先縠將右行,先蔑將左行。

七年,晉文公、秦繆公共圍鄭,以其無禮於文公亡過時,及城濮時鄭助楚也。圍鄭,欲得叔瞻。叔瞻聞之,自殺。鄭持叔瞻告晉。晉曰:「必得鄭君而甘心焉。」鄭恐,乃閒令使謂秦繆公曰:「亡鄭厚晉,於晉得矣,而秦未為利。君何不解鄭,得為東道交?」秦伯說,罷兵。晉亦罷兵。

九年冬,晉文公卒,子襄公歡立。是歲鄭伯亦卒。

鄭人或賣其國於秦,秦繆公發兵往襲鄭。十二月,秦兵過我郊。襄公元年春,秦師過周,無禮,王孫滿譏之。兵至滑,鄭賈人弦高將市于周,遇之,以十二牛勞秦師。秦師驚而還,滅滑而去。

晉先軫曰:「秦伯不用蹇叔,反其眾心,此可擊。」欒枝曰:「未報先君施於秦,擊之,不可。」先軫曰:「秦侮吾孤,伐吾同姓,何德之報?」遂擊之。襄公墨衰绖。四月,敗秦師于殽,虜秦三將孟明視、西乞秫、白乙丙以歸。遂墨以葬文公。文公夫人秦女,謂襄公曰:「秦欲得其三將戮之。」公許,遣之。先軫聞之,謂襄公曰:「患生矣。」軫乃追秦將。秦將渡河,已在船中,頓首謝,卒不反。

後三年,秦果使孟明伐晉,報殽之敗,取晉汪以歸。四年,秦繆公大興兵伐我,度河,取王官,封殽尸而去。晉恐,不敢出,遂城守。五年,晉伐秦,取新城,報王官役也。

六年,趙衰成子、欒貞子、咎季子犯、霍伯皆卒。趙盾代趙衰執政。

七年八月,襄公卒。太子夷皋少。晉人以難故,欲立長君。趙盾曰:「立襄公弟雍。好善而長,先君愛之;且近於秦,秦故好也。立善則固,事長則順,奉愛則孝,結舊好則安。」賈季曰:「不如其弟樂。辰嬴嬖於二君,立其子,民必安之。」趙盾曰:「辰嬴賤,班在九人下,其子何震之有!且為二君嬖,淫也。為先君子,不能求大而出在小國,僻也。母淫子僻,無威;陳小而遠,無援:將何可乎!」使士會如秦迎公子雍。賈季亦使人召公子樂於陳。趙盾廢賈季,以其殺陽處父。十月,葬襄公。十一月,賈季奔翟。是歲,秦繆公亦卒。

靈公元年四月,秦康公曰:「昔文公之入也無衛,故有呂、郤之患。」乃多與公子雍衛。太子母繆嬴日夜抱太子以號泣於朝,曰:「先君何罪?其嗣亦何罪?舍適而外求君,將安置此?」出朝,則抱以適趙盾所,頓首曰:「先君奉此子而屬之子,曰『此子材,吾受其賜;不材,吾怨子』。今君卒,言猶在耳,而棄之,若何?」趙盾與諸大夫皆患繆嬴,且畏誅,乃背所迎而立太子夷皋,是為靈公。發兵以距秦送公子雍者。趙盾為將,往擊秦,敗之令狐。先蔑、隨會亡奔秦。秋,齊、宋、衛、鄭、曹、許君皆會趙盾,盟於扈,以靈公初立故也。

四年,伐秦,取少梁。秦亦取晉之郩。六年,秦康公伐晉,取羈馬。晉侯怒,使趙盾、趙穿、郤缺擊秦,大戰河曲,趙穿最有功。七年,晉六卿患隨會之在秦,常為晉亂,乃詳令魏壽餘反晉降秦。秦使隨會之魏,因執會以歸晉。

八年,周頃王崩,公卿爭權,故不赴。晉使趙盾以車八百乘平周亂而立匡王。是年,楚莊王初即位。十二年,齊人弒其君懿公。

十四年,靈公壯,侈,厚斂以彫墻。從臺上彈人,觀其避丸也。宰夫胹熊蹯不熟,靈公怒,殺宰夫,使婦人持其尸出棄之,過朝。趙盾、隨會前數諫,不聽;已又見死人手,二人前諫。隨會先諫,不聽。靈公患之,使鉏麑刺趙盾。盾閨門開,居處節,鉏麑退,嘆曰:「殺忠臣,棄君命,罪一也。」遂觸樹而死。

初,盾常田首山,見桑下有餓人。餓人,示瞇明也。盾與之食,食其半。問其故,曰:「宦三年,未知母之存不,願遺母。」盾義之,益與之飯肉。已而為晉宰夫,趙盾弗復知也。九月,晉靈公飲趙盾酒,伏甲將攻盾。公宰示瞇明知之,恐盾醉不能起,而進曰:「君賜臣,觴三行可以罷。」欲以去趙盾,令先,毋及難。盾既去,靈公伏士未會,先縱齧狗名敖。明為盾搏殺狗。盾曰:「棄人用狗,雖猛何為。」然不知明之為陰德也。已而靈公縱伏士出逐趙盾,示瞇明反擊靈公之伏士,伏士不能進,而竟脫盾。盾問其故,曰:「我桑下餓人。」問其名,弗告。明亦因亡去。

盾遂奔,未出晉境。乙丑,盾昆弟將軍趙穿襲殺靈公於桃園而迎趙盾。趙盾素貴,得民和;靈公少,侈,民不附,故為弒易。盾復位。晉太史董狐書曰「趙盾弒其君」,以視於朝。盾曰:「弒者趙穿,我無罪。」太史曰:「子為正卿,而亡不出境,反不誅國亂,非子而誰?」孔子聞之,曰:「董狐,古之良史也,書法不隱。宣子,良大夫也,為法受惡。惜也,出疆乃免。」

趙盾使趙穿迎襄公弟黑臀于周而立之,是為成公。

成公者,文公少子,其母周女也。壬申,朝于武宮。

成公元年,賜趙氏為公族。伐鄭,鄭倍晉故也。三年,鄭伯初立,附晉而棄楚。楚怒,伐鄭,晉往救之。

六年,伐秦,虜秦將赤。

七年,成公與楚莊王爭彊,會諸侯于扈。陳畏楚,不會。晉使中行桓子伐陳,因救鄭,與楚戰,敗楚師。是年,成公卒,子景公據立。

景公元年春,陳大夫夏徵舒弒其君靈公。二年,楚莊王伐陳,誅徵舒。

三年,楚莊王圍鄭,鄭告急晉。晉使荀林父將中軍,隨會將上軍,趙朔將下軍,郤克、欒書、先縠、韓厥、鞏朔佐之。六月,至河。聞楚已服鄭,鄭伯肉袒與盟而去,荀林父欲還。先縠曰:「凡來救鄭,不至不可,將率離心。」卒度河。楚已服鄭,欲飲馬于河為名而去。楚與晉軍大戰。鄭新附楚,畏之,反助楚攻晉。晉軍敗,走河,爭度,船中人指甚眾。楚虜我將智罃。歸而林父曰:「臣為督將,軍敗當誅,請死。」景公欲許之。隨會曰:「昔文公之與楚戰城濮,成王歸殺子玉,而文公乃喜。今楚已敗我師,又誅其將,是助楚殺仇也。」乃止。

四年,先縠以首計而敗晉軍河上,恐誅,乃奔翟,與翟謀伐晉。晉覺,乃族縠。縠,先軫子也。

五年,伐鄭,為助楚故也。是時楚莊王彊,以挫晉兵河上也。

六年,楚伐宋,宋來告急晉,晉欲救之,伯宗謀曰:「楚,天方開之,不可當。」乃使解揚紿為救宋。鄭人執與楚,楚厚賜,使反其言,令宋急下。解揚紿許之,卒致晉君言。楚欲殺之,或諫,乃歸解揚。七年,晉使隨會滅赤狄。

八年,使郤克於齊。齊頃公母從樓上觀而笑之。所以然者,郤克僂,而魯使蹇,衛使眇,故齊亦令人如之以導客。郤克怒,歸至河上,曰:「不報齊者,河伯視之!」至國,請君,欲伐齊。景公問知其故,曰:「子之怨,安足以煩國!」弗聽。魏文子請老休,辟郤克,克執政。

九年,楚莊王卒。晉伐齊,齊使太子彊為質於晉,晉兵罷。

十一年春,齊伐魯,取隆。魯告急衛,衛與魯皆因郤克告急於晉。晉乃使郤克、欒書、韓厥以兵車八百乘與魯、衛共伐齊。夏,與頃公戰於砹傷困頃公。頃公乃與其右易位,下取飲,以得脫去。齊師敗走,晉追北至齊。頃公獻寶器以求平,不聽。郤克曰:「必得蕭桐姪子為質。」齊使曰:「蕭桐姪子,頃公母;頃公母猶晉君母,柰何必得之?不義,請復戰。」晉乃許與平而去。

楚申公巫臣盜夏姬以奔晉,晉以巫臣為邢大夫。

十二年冬,齊頃公如晉,欲上尊晉景公為王,景公讓不敢。晉始作六(卿)[軍],韓厥、鞏朔、趙穿、荀騅、趙括、趙旃皆為卿。智罃自楚歸。

十三年,魯成公朝晉,晉弗敬,魯怒去,倍晉。晉伐鄭,取氾。

十四年,梁山崩。問伯宗,伯宗以為不足怪也。

十六年,楚將子反怨巫臣,滅其族。巫臣怒,遺子反書曰:「必令子罷於奔命!」乃請使吳,令其子為吳行人,教吳乘車用兵。吳晉始通,約伐楚。

十七年,誅趙同、趙括,族滅之。韓厥曰:「趙衰、趙盾之功豈可忘乎?柰何絕祀!」乃復令趙庶子武為趙後,復與之邑。

十九年夏,景公病,立其太子壽曼為君,是為厲公。後月餘,景公卒。

厲公元年,初立,欲和諸侯,與秦桓公夾河而盟。歸而秦倍盟,與翟謀伐晉。三年,使呂相讓秦,因與諸侯伐秦。至涇,敗秦於麻隧,虜其將成差。

五年,三郤讒伯宗,殺之。伯宗以好直諫得此禍,國人以是不附厲公。

六年春,鄭倍晉與楚盟,晉怒。欒書曰:「不可以當吾世而失諸侯。」乃發兵。厲公自將,五月度河。聞楚兵來救,范文子請公欲還。郤至曰:「發兵誅逆,見彊辟之,無以令諸侯。」遂與戰。癸巳,射中楚共王目,楚兵敗於鄢陵。子反收餘兵,拊循欲復戰,晉患之。共王召子反,其侍者豎陽穀進酒,子反醉,不能見。王怒,讓子反,子反死。王遂引兵歸。晉由此威諸侯,欲以令天下求霸。

厲公多外嬖姬,歸,欲盡去群大夫而立諸姬兄弟。寵姬兄曰胥童,嘗與郤至有怨,及欒書又怨郤至不用其計而遂敗楚,乃使人閒謝楚。楚來詐厲公曰:「鄢陵之戰,實至召楚,欲作亂,內子周立之。會與國不具,是以事不成。」厲公告欒書。欒書曰:「其殆有矣!願公試使人之周微考之。」果使郤至於周。欒書又使公子周見郤至,郤至不知見賣也。厲公驗之,信然,遂怨郤至,欲殺之。八年,厲公獵,與姬飲,郤至殺豕奉進,宦者奪之。郤至射殺宦者。公怒,曰:「季子欺予!」將誅三郤,未發也。郤鉤欲攻公,曰:「我雖死,公亦病矣。」郤至曰:「信不反君,智不害民,勇不作亂。失此三者,誰與我?我死耳!」十二月壬午,公令胥童以兵八百人襲攻殺三郤。胥童因以劫欒書、中行偃于朝,曰:「不殺二子,患必及公。」公曰:「一旦殺三卿,寡人不忍益也。」對曰:「人將忍君。」公弗聽,謝欒書等以誅郤氏罪:「大夫復位。」二子頓首曰:「幸甚幸甚!」公使胥童為卿。閏月乙卯,厲公游匠驪氏,欒書、中行偃以其黨襲捕厲公,囚之,殺胥童,而使人迎公子周于周而立之,是為悼公。

悼公元年正月庚申,欒書、中行偃弒厲公,葬之以一乘車。厲公囚六日死,死十日庚午,智罃迎公子周來,至絳,刑雞與大夫盟而立之,是為悼公。辛巳,朝武宮。二月乙酉,即位。

悼公周者,其大父捷,晉襄公少子也,不得立,號為桓叔,桓叔最愛。桓叔生惠伯談,談生悼公周。周之立,年十四矣。悼公曰:「大父、父皆不得立而辟難於周,客死焉。寡人自以疏遠,毋幾為君。今大夫不忘文、襄之意而惠立桓叔之後,賴宗廟大夫之靈,得奉晉祀,豈敢不戰戰乎?大夫其亦佐寡人!」於是逐不臣者七人,修舊功,施德惠,收文公入時功臣後。秋,伐鄭。鄭師敗,遂至陳。

三年,晉會諸侯。悼公問群臣可用者,祁傒舉解狐。解狐,傒之仇。復問,舉其子祁午。君子曰:「祁傒可謂不黨矣!外舉不隱仇,內舉不隱子。」方會諸侯,悼公弟楊干亂行,魏絳戮其仆。悼公怒,或諫公,公卒賢絳,任之政,使和戎,戎大親附。十一年,悼公曰:「自吾用魏絳,九合諸侯,和戎、翟,魏子之力也。」賜之樂,三讓乃受之。冬,秦取我櫟。

十四年,晉使六卿率諸侯伐秦,度涇,大敗秦軍,至棫林而去。

十五年,悼公問治國於師曠。師曠曰:「惟仁義為本。」冬,悼公卒,子平公彪立。

平公元年,伐齊,齊靈公與戰靡下,齊師敗走。晏嬰曰:「君亦毋勇,何不止戰?」遂去。晉追,遂圍臨菑,盡燒屠其郭中。東至膠,南至沂,齊皆城守,晉乃引兵歸。

六年,魯襄公朝晉。晉欒逞有罪,奔齊。八年,齊莊公微遣欒逞於曲沃,以兵隨之。齊兵上太行,欒逞從曲沃中反,襲入絳。絳不戒,平公欲自殺,范獻子止公,以其徒擊逞,逞敗走曲沃。曲沃攻逞,逞死,遂滅欒氏宗。逞者,欒書孫也。其入絳,與魏氏謀。齊莊公聞逞敗,乃還,取晉之朝歌去,以報臨菑之役也。

十年,齊崔杼弒其君莊公。晉因齊亂,伐敗齊於高唐去,報太行之役也。

十四年,吳延陵季子來使,與趙文子、韓宣子、魏獻子語,曰:「晉國之政,卒歸此三家矣。」

十九年,齊使晏嬰如晉,與叔向語。叔向曰:「晉,季世也。公厚賦為臺池而不恤政,政在私門,其可久乎!」晏子然之。

二十二年,伐燕。二十六年,平公卒,子昭公夷立。

昭公六年卒。六卿彊,公室卑。子頃公去疾立。

頃公六年,周景王崩,王子爭立。晉六卿平王室亂,立敬王。

九年,魯季氏逐其君昭公,昭公居乾侯。十一年,衛、宋使使請晉納魯君。季平子私賂范獻子,獻子受之,乃謂晉君曰:「季氏無罪。」不果入魯君。

十二年,晉之宗家祁傒孫,叔向子,相惡於君。六卿欲弱公室,乃遂以法盡滅其族。而分其邑為十縣,各令其子為大夫。晉益弱,六卿皆大。

十四年,頃公卒,子定公午立。定公十一年,魯陽虎奔晉,趙鞅簡子捨之。十二年,孔子相魯。

十五年,趙鞅使邯鄲大夫午,不信,欲殺午,午與中行寅、范吉射親攻趙鞅,鞅走保晉陽。定公圍晉陽。荀櫟、韓不信、魏侈與范、中行為仇,乃移兵伐范、中行。范、中行反,晉君擊之,敗范、中行。范、中行走朝歌,保之。韓、魏為趙鞅謝晉君,乃赦趙鞅,復位。二十二年,晉敗范、中行氏,二子奔齊。

三十年,定公與吳王夫差會黃池,爭長,趙鞅時從,卒長吳。

三十一年,齊田常弒其君簡公,而立簡公弟驁為平公。三十三年,孔子卒。

三十七年,定公卒,子出公鑿立。

出公十七年,知伯與趙、韓、魏共分范、中行地以為邑。出公怒,告齊、魯,欲以伐四卿。四卿恐,遂反攻出公。出公奔齊,道死。故知伯乃立昭公曾孫驕為晉君,是為哀公。

哀公大父雍,晉昭公少子也,號為戴子。戴子生忌。忌善知伯,蚤死,故知伯欲盡并晉,未敢,乃立忌子驕為君。當是時,晉國政皆決知伯,晉哀公不得有所制。知伯遂有范、中行地,最彊。

哀公四年,趙襄子、韓康子、魏桓子共殺知伯,盡并其地。

十八年,哀公卒,子幽公柳立。

幽公之時,晉畏,反朝韓、趙、魏之君。獨有絳、曲沃,餘皆入三晉。

十五年,魏文侯初立。十八年,幽公淫婦人,夜竊出邑中,盜殺幽公。魏文侯以兵誅晉亂,立幽公子止,是為烈公。

烈公十九年,周威烈王賜趙、韓、魏皆命為諸侯。

二十七年,烈公卒,子孝公頎立。孝公九年,魏武侯初立,襲邯鄲,不勝而去。十七年,孝公卒,子靜公俱酒立。是歲,齊威王元年也。

靜公二年,魏武侯、韓哀侯、趙敬侯滅晉後而三分其地。靜公遷為家人,晉絕不祀。

太史公曰:晉文公,古所謂明君也,亡居外十九年,至困約,及即位而行賞,尚忘介子推,況驕主乎?靈公既弒,其後成、景致嚴,至厲大刻,大夫懼誅,禍作。悼公以後日衰,六卿專權。故君道之御其臣下。固不易哉!


\end{pinyinscope}