\article{楚元王世家}

\begin{pinyinscope}
楚元王劉交者,高祖之同母少弟也,字游。

高祖兄弟四人,長兄伯,伯蚤卒。始高祖微時,嘗辟事,時時與賓客過巨嫂食。嫂厭叔,叔與客來,嫂詳為羹盡,櫟釜,賓客以故去。已而視釜中尚有羹,高祖由此怨其嫂。及高祖為帝,封昆弟,而伯子獨不得封。太上皇以為言,高祖曰:「某非忘封之也,為其母不長者耳。」於是乃封其子信為羹頡侯。而王次兄仲於代。

高祖六年,已禽楚王韓信於陳,乃以弟交為楚王,都彭城。即位二十三年卒,子夷王郢立。夷王四年卒,子王戊立。

王戊立二十年,冬,坐為薄太后服私姦,削東海郡。春,戊與吳王合謀反,其相張尚、太傅趙夷吾諫,不聽。戊則殺尚、夷吾,起兵與吳西攻梁,破棘壁。至昌邑南,與漢將周亞夫戰。漢絕吳楚糧道,士卒饑,吳王走,楚王戊自殺,軍遂降漢。

漢已平吳楚,孝景帝欲以德侯子續吳,以元王子禮續楚。竇太后曰:「吳王,老人也,宜為宗室順善。今乃首率七國,紛亂天下,柰何續其後!」不許吳,許立楚後。是時禮為漢宗正。乃拜禮為楚王,奉元王宗廟,是為楚文王。

文王立三年卒,子安王道立。安王二十二年卒,子襄王注立。襄王立十四年卒,子王純代立。王純立,地節二年,中人上書告楚王謀反,王自殺,國除,入漢為彭城郡。

趙王劉遂者,其父高祖中子,名友,謚曰「幽」。幽王以憂死,故為「幽」。高后王呂祿於趙,一歲而高后崩。大臣誅諸呂呂祿等,乃立幽王子遂為趙王。

孝文帝即位二年,立遂弟辟彊,取趙之河閒郡為河閒王,(以)[是]為文王。立十三年卒,子哀王福立。一年卒,無子,絕後,國除,入于漢。

遂既王趙二十六年,孝景帝時坐晁錯以適削趙王常山之郡。吳楚反,趙王遂與合謀起兵。其相建德、內史王悍諫,不聽。遂燒殺建德、王悍,發兵屯其西界,欲待吳與俱西。北使匈奴,與連和攻漢。漢使曲周侯酈寄擊之。趙王遂還,城守邯鄲,相距七月。吳楚敗於梁,不能西。匈奴聞之,亦止,不肯入漢邊。欒布自破齊還,乃并兵引水灌趙城。趙城壞,趙王自殺,邯鄲遂降。趙幽王絕後。

太史公曰:國之將興,必有禎祥,君子用而小人退。國之將亡,賢人隱,亂臣貴。使楚王戊毋刑申公,遵其言,趙任防與先生,豈有篡殺之謀,為天下僇哉?賢人乎,賢人乎!非質有其內,惡能用之哉?甚矣,「安危在出令,存亡在所任」,誠哉是言也!


\end{pinyinscope}