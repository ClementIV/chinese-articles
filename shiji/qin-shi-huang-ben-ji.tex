\article{秦始皇本紀}

\begin{pinyinscope}
秦始皇帝者,秦莊襄王子也。莊襄王為秦質子於趙,見呂不韋姬,悅而取之,生始皇。以秦昭王四十八年正月生於邯鄲。及生,名為政,姓趙氏。年十三歲,莊襄王死,政代立為秦王。當是之時,秦地已并巴、蜀、漢中,越宛有郢,置南郡矣;北收上郡以東,有河東、太原、上黨郡;東至滎陽,滅二周,置三川郡。呂不韋為相,封十萬戶,號曰文信侯。招致賓客游士,欲以并天下。李斯為舍人。蒙驁、王齮、麃公等為將軍。王年少,初即位,委國事大臣。

晉陽反,元年,將軍蒙驁擊定之。二年,麃公將卒攻卷,斬首三萬。三年,蒙驁攻韓,取十三城。王齮死。十月,將軍蒙驁攻魏氏暢、有詭。歲大饑。四年,拔暢、有詭。三月,軍罷。秦質子歸自趙,趙太子出歸國。十月庚寅,蝗蟲從東方來,蔽天。天下疫。百姓內粟千石,拜爵一級。五年,將軍驁攻魏,定酸棗、燕、虛、長平、雍丘、山陽城,皆拔之,取二十城。初置東郡。冬雷。六年,韓、魏、趙、衛、楚共擊秦,取壽陵。秦出兵,五國兵罷。拔衛,迫東郡,其君角率其支屬徙居野王,阻其山以保魏之河內。七年,彗星先出東方,見北方,五月見西方。將軍驁死。以攻龍、孤、慶都,還兵攻汲。彗星復見西方十六日。夏太后死。八年,王弟長安君成蟜將軍擊趙,反,死屯留,軍吏皆斬死,遷其民於臨洮。將軍壁死,卒屯留、蒲鶴反,戮其尸。河魚大上,輕車重馬東就食。

嫪毐封為長信侯。予之山陽地,令毐居之。宮室車馬衣服苑囿馳獵恣毐。事無小大皆決於毐。又以河西太原郡更為毐國。九年,彗星見,或竟天。攻魏垣、蒲陽。四月,上宿雍。己酉,王冠,帶劍。長信侯毐作亂而覺,矯王御璽及太后璽以發縣卒及衛卒、官騎、戎翟君公、舍人,將欲攻蘄年宮為亂。王知之,令相國昌平君、昌文君發卒攻毐。戰咸陽,斬首數百,皆拜爵,及宦者皆在戰中,亦拜爵一級。毐等敗走。即令國中:有生得毐,賜錢百萬;殺之,五十萬。盡得毐等。衛尉竭、內史肆、佐弋竭、中大夫令齊等二十人皆梟首。車裂以徇,滅其宗。及其舍人,輕者為鬼薪。及奪爵遷蜀四千餘家,家房陵。(四)[是]月寒凍,有死者。楊端和攻衍氏。彗星見西方,又見北方,從斗以南八十日。十年,相國呂不韋坐嫪毐免。桓齮為將軍。齊、趙來置酒。齊人茅焦說秦王曰:「秦方以天下為事,而大王有遷母太后之名,恐諸侯聞之,由此倍秦也。」秦王乃迎太后於雍而入咸陽,復居甘泉宮。

大索,逐客,李斯上書說,乃止逐客令。李斯因說秦王,請先取韓以恐他國,於是使斯下韓。韓王患之。與韓非謀弱秦。大梁人尉繚來,說秦王曰:「以秦之彊,諸侯譬如郡縣之君,臣但恐諸侯合從,翕而出不意,此乃智伯、夫差、湣王之所以亡也。願大王毋愛財物,賂其豪臣,以亂其謀,不過亡三十萬金,則諸侯可盡。」秦王從其計,見尉繚亢禮,衣服食飲與繚同。繚曰:「秦王為人,蜂準,長目,摯鳥膺,豺聲,少恩而虎狼心,居約易出人下,得志亦輕食人。我布衣,然見我常身自下我。誠使秦王得志於天下,天下皆為虜矣。不可與久游。」乃亡去。秦王覺,固止,以為秦國尉,卒用其計策。而李斯用事。

十一年,王翦、桓齮、楊端和攻鄴,取九城。王翦攻閼與、橑楊,皆并為一軍。翦將十八日,軍歸斗食以下,什推二人從軍取鄴安陽,桓齮將。十二年,文信侯不韋死,竊葬。其舍人臨者,晉人也逐出之;秦人六百石以上奪爵,遷;五百石以下不臨,遷,勿奪爵。自今以來,操國事不道如嫪毐、不韋者籍其門,視此。秋,復嫪毐舍人遷蜀者。當是之時,天下大旱,六月至八月乃雨。

十三年,桓齮攻趙平陽,殺趙將扈輒,斬首十萬。王之河南。正月,彗星見東方。十月,桓齮攻趙。十四年,攻趙軍於平陽,取宜安,破之,殺其將軍。桓齮定平陽、武城。韓非使秦,秦用李斯謀,留非,非死雲陽。韓王請為臣。

十五年,大興兵,一軍至鄴,一軍至太原,取狼孟。地動。十六年九月,發卒受地韓南陽假守騰。初令男子書年。魏獻地於秦。秦置麗邑。十七年,內史騰攻韓,得韓王安,盡納其地,以其地為郡,命曰潁川。地動。華陽太后卒。民大饑。

十八年,大興兵攻趙,王翦將上地,下井陘,端和將河內,羌瘣伐趙,端和圍邯鄲城。十九年,王翦、羌瘣盡定取趙地東陽,得趙王。引兵欲攻燕,屯中山。秦王之邯鄲,諸嘗與王生趙時母家有仇怨,皆阬之。秦王還,從太原、上郡歸。始皇帝母太后崩。趙公子嘉率其宗數百人之代,自立為代王,東與燕合兵,軍上谷。大饑。

二十年,燕太子丹患秦兵至國,恐,使荊軻刺秦王。秦王覺之,體解軻以徇,而使王翦、辛勝攻燕。燕、代發兵擊秦軍,秦軍破燕易水之西。二十一年,王賁攻(薊)[荊]。乃益發卒詣王翦軍,遂破燕太子軍,取燕薊城,得太子丹之首。燕王東收遼東而王之。王翦謝病老歸。新鄭反。昌平君徙於郢。大雨雪,深二尺五寸。

二十二年,王賁攻魏,引河溝灌大梁,大梁城壞,其王請降,盡取其地。

二十三年,秦王復召王翦,彊起之,使將擊荊。取陳以南至平輿,虜荊王。秦王游至郢陳。荊將項燕立昌平君為荊王,反秦於淮南。二十四年,王翦、蒙武攻荊,破荊軍,昌平君死,項燕遂自殺。

二十五年,大興兵,使王賁將,攻燕遼東,得燕王喜。還攻代,虜代王嘉。王翦遂定荊江南地;降越君,置會稽郡。五月,天下大酺。

二十六年,齊王建與其相後勝發兵守其西界,不通秦。秦使將軍王賁從燕南攻齊,得齊王建。

秦王初并天下,令丞相、御史曰:「異日韓王納地效璽,請為藩臣,已而倍約,與趙、魏合從畔秦,故興兵誅之,虜其王。寡人以為善,庶幾息兵革。趙王使其相李牧來約盟,故歸其質子。已而倍盟,反我太原,故興兵誅之,得其王。趙公子嘉乃自立為代王,故舉兵擊滅之。魏王始約服入秦,已而與韓、趙謀襲秦,秦兵吏誅,遂破之。荊王獻青陽以西,已而畔約,擊我南郡,故發兵誅,得其王,遂定其荊地。燕王昏亂,其太子丹乃陰令荊軻為賊,兵吏誅,滅其國。齊王用后勝計,絕秦使,欲為亂,兵吏誅,虜其王,平齊地。寡人以眇眇之身,興兵誅暴亂,賴宗廟之靈,六王咸伏其辜,天下大定。今名號不更,無以稱成功,傳後世。其議帝號。」丞相綰、御史大夫劫、廷尉斯等皆曰:「昔者五帝地方千里,其外侯服夷服諸侯或朝或否,天子不能制。今陛下興義兵,誅殘賊,平定天下,海內為郡縣,法令由一統,自上古以來未嘗有,五帝所不及。臣等謹與博士議曰:『古有天皇,有地皇,有泰皇,泰皇最貴。』臣等昧死上尊號,王為『泰皇』。命為『制』,令為『詔』,天子自稱曰『朕』。」王曰:「去『泰』,著『皇』,采上古『帝』位號,號曰『皇帝』。他如議。」制曰:「可。」追尊莊襄王為太上皇。制曰:「朕聞太古有號毋謚,中古有號,死而以行為謐。如此,則子議父,臣議君也,甚無謂,朕弗取焉。自今已來,除謚法。朕為始皇帝。後世以計數,二世三世至于萬世,傳之無窮。」

始皇推終始五德之傳,以為周得火德,秦代周德,從所不勝。方今水德之始,改年始,朝賀皆自十月朔。衣服旄旌節旗皆上黑。數以六為紀,符、法冠皆六寸,而輿六尺,六尺為步,乘六馬。更名河曰德水,以為水德之始。剛毅戾深,事皆決於法,刻削毋仁恩和義,然後合五德之數。於是急法,久者不赦。

丞相綰等言:「諸侯初破,燕、齊、荊地遠,不為置王,毋以填之。請立諸子,唯上幸許。」始皇下其議於群臣,群臣皆以為便。廷尉李斯議曰:「周文武所封子弟同姓甚眾,然後屬疏遠,相攻擊如仇讎,諸侯更相誅伐,周天子弗能禁止。今海內賴陛下神靈一統,皆為郡縣,諸子功臣以公賦稅重賞賜之,甚足易制。天下無異意,則安寧之術也。置諸侯不便。」始皇曰:「天下共苦戰鬬不休,以有侯王。賴宗廟,天下初定,又復立國,是樹兵也,而求其寧息,豈不難哉!廷尉議是。」

分天下以為三十六郡,郡置守、尉、監。更名民曰「黔首」。大酺。收天下兵,聚之咸陽,銷以為鐘鐻,金人十二,重各千石,置廷宮中。一法度衡石丈尺。車同軌。書同文字。地東至海暨朝鮮,西至臨洮、羌中,南至北向戶,北據河為塞,并陰山至遼東。徙天下豪富於咸陽十二萬戶。諸廟及章臺、上林皆在渭南。秦每破諸侯,寫放其宮室,作之咸陽北阪上,南臨渭,自雍門以東至涇、渭,殿屋複道周閣相屬。所得諸侯美人鐘鼓,以充入之。

二十七年,始皇巡隴西、北地,出雞頭山,過回中。焉作信宮渭南,已更命信宮為極廟,象天極。自極廟道通酈山,作甘泉前殿。筑甬道,自咸陽屬之。是歲,賜爵一級。治馳道。

二十八年,始皇東行郡縣,上鄒嶧山。立石,與魯諸儒生議,刻石頌秦德,議封禪望祭山川之事。乃遂上泰山,立石,封,祠祀。下,風雨暴至,休於樹下,因封其樹為五大夫。禪梁父。刻所立石,其辭曰:

皇帝臨位,作制明法,臣下修飭。二十有六年,初并天下,罔不賓服。親巡遠方黎民,登茲泰山,周覽東極。從臣思跡,本原事業,祗誦功德。治道運行,諸產得宜,皆有法式。大義休明,垂于後世,順承勿革。皇帝躬聖,既平天下,不懈於治。夙興夜寐,建設長利,專隆教誨。訓經宣達,遠近畢理,咸承聖志。貴賤分明,男女禮順,慎遵職事。昭隔內外,靡不清凈,施于後嗣。化及無窮,遵奉遺詔,永承重戒。

於是乃并勃海以東,過黃、腄,窮成山,登之罘,立石頌秦德焉而去。

南登瑯邪,大樂之,留三月。乃徙黔首三萬戶瑯邪臺下,復十二歲。作瑯邪臺,立石刻,頌秦德,明得意。曰:

維二十八年,皇帝作始。端平法度,萬物之紀。以明人事,合同父子。聖智仁義,顯白道理。東撫東土,以省卒士。事已大畢,乃臨于海。皇帝之功,勸勞本事。上農除末,黔首是富。普天之下,摶心揖志。器械一量,同書文字。日月所照,舟輿所載。皆終其命,莫不得意。應時動事,是維皇帝。匡飭異俗,陵水經地。憂恤黔首,朝夕不懈。除疑定法,咸知所辟。方伯分職,諸治經易。舉錯必當,莫不如畫。皇帝之明,臨察四方。尊卑貴賤,不踰次行。瑯邪不容,皆務貞良。細大盡力,莫敢怠荒。遠邇辟隱,專務肅莊。端直敦忠,事業有常。皇帝之德,存定四極。誅亂除害,興利致福。節事以時,諸產繁殖。黔首安寧,不用兵革。六親相保,終無寇賊。驩欣奉教,盡知法式。六合之內,皇帝之土。西涉流沙,南盡北戶。東有東海,北過大夏。人跡所至,無不臣者。功蓋五帝,澤及牛馬。莫不受德,各安其宇。

維秦王兼有天下,立名為皇帝,乃撫東土,至于瑯邪。列侯武城侯王離、列侯通武侯王賁、倫侯建成侯趙亥、倫侯昌武侯成、倫侯武信侯馮毋擇、丞相隗林、丞相王綰、卿李斯、卿王戊、五大夫趙嬰、五大夫楊樛從,與議於海上。曰:「古之帝者,地不過千里,諸侯各守其封域,或朝或否,相侵暴亂,殘伐不止,猶刻金石,以自為紀。古之五帝三王,知教不同,法度不明,假威鬼神,以欺遠方,實不稱名,故不久長。其身未歿,諸侯倍叛,法令不行。今皇帝并一海內,以為郡縣,天下和平。昭明宗廟,體道行德,尊號大成。群臣相與誦皇帝功德,刻于金石,以為表經。」

既已,齊人徐市等上書,言海中有三神山,名曰蓬萊、方丈、瀛洲,僊人居之。請得齋戒,與童男女求之。於是遣徐市發童男女數千人,入海求僊人。

始皇還,過彭城,齋戒禱祠,欲出周鼎泗水。使千人沒水求之,弗得。乃西南渡淮水,之衡山、南郡。浮江,至湘山祠。逢大風,幾不得渡。上問博士曰:「湘君神?」博士對曰:「聞之,堯女,舜之妻,而葬此。」於是始皇大怒,使刑徒三千人皆伐湘山樹,赭其山。上自南郡由武關歸。

二十九年,始皇東游。至陽武博狼沙中,為盜所驚。求弗得,乃令天下大索十日。

登之罘,刻石。其辭曰:

維二十九年,時在中春,陽和方起。皇帝東游,巡登之罘,臨照于海。從臣嘉觀,原念休烈,追誦本始。大聖作治,建定法度,顯箸綱紀。外教諸侯,光施文惠,明以義理。六國回辟,貪戾無厭,虐殺不已。皇帝哀眾,遂發討師,奮揚武德。義誅信行,威燀旁達,莫不賓服。烹滅彊暴,振救黔首,周定四極。普施明法,經緯天下,永為儀則。大矣哉!宇縣之中,承順聖意。群臣誦功,請刻于石,表垂于常式。

其東觀曰:二十九年,皇帝春游,覽省遠方。逮于海隅,遂登之罘,昭臨朝陽。觀望廣麗,從臣咸念,原道至明。聖法初興,清理疆內,外誅暴彊。武威旁暢,振動四極,禽滅六王。闡并天下,甾害絕息,永偃戎兵。皇帝明德,經理宇內,視聽不怠。作立大義,昭設備器,咸有章旗。職臣遵分,各知所行,事無嫌疑。黔首改化,遠邇同度,臨古絕尤。常職既定,後嗣循業,長承聖治。群臣嘉德,祗誦聖烈,請刻之罘。

旋,遂之瑯邪,道上黨入。

三十年,無事。

三十一年十二月,更名臘曰「嘉平」。賜黔首裏六石米,二羊。始皇為微行咸陽,與武士四人俱,夜出逢盜蘭池,見窘,武士擊殺盜,關中大索二十日。米石千六百。

三十二年,始皇之碣石,使燕人盧生求羨門、高誓。刻碣石門。壞城郭,決通隄防。其辭曰:

遂興師旅,誅戮無道,為逆滅息。武殄暴逆,文復無罪,庶心咸服。惠論功勞,賞及牛馬,恩肥土域。皇帝奮威,德并諸侯,初一泰平。墮壞城郭,決通川防,夷去險阻。地勢既定,黎庶無繇,天下咸撫。男樂其疇,女修其業,事各有序。惠被諸產,久并來田,莫不安所。群臣誦烈,請刻此石,垂著儀矩。

因使韓終、侯公、石生求僊人不死之藥。始皇巡北邊,從上郡入。燕人盧生使入海還,以鬼神事,因奏錄圖書,曰「亡秦者胡也」。始皇乃使將軍蒙恬發兵三十萬人北擊胡,略取河南地。

三十三年,發諸嘗逋亡人、贅婿、賈人略取陸梁地,為桂林、象郡、南海,以適遣戍。西北斥逐匈奴。自榆中并河以東,屬之陰山,以為(三)[四]十四縣,城河上為塞。又使蒙恬渡河取斑闕、(陶)[陽]山、北假中,筑亭障以逐戎人。徙謫,實之初縣。禁不得祠。明星出西方。三十四年,適治獄吏不直者,筑長城及南越地。

始皇置酒咸陽宮,博士七十人前為壽。仆射周青臣進頌曰:「他時秦地不過千里,賴陛下神靈明聖,平定海內,放逐蠻夷,日月所照,莫不賓服。以諸侯為郡縣,人人自安樂,無戰爭之患,傳之萬世。自上古不及陛下威德。」始皇悅。博士齊人淳于越進曰:「臣聞殷周之王千餘歲,封子弟功臣,自為枝輔。今陛下有海內,而子弟為匹夫,卒有田常、六卿之臣,無輔拂,何以相救哉?事不師古而能長久者,非所聞也。今青臣又面諛以重陛下之過,非忠臣。」始皇下其議。丞相李斯曰:「五帝不相復,三代不相襲,各以治,非其相反,時變異也。今陛下創大業,建萬世之功,固非愚儒所知。且越言乃三代之事,何足法也?異時諸侯并爭,厚招游學。今天下已定,法令出一,百姓當家則力農工,士則學習法令辟禁。今諸生不師今而學古,以非當世,惑亂黔首。丞相臣斯昧死言:古者天下散亂,莫之能一,是以諸侯并作,語皆道古以害今,飾虛言以亂實,人善其所私學,以非上之所建立。今皇帝并有天下,別黑白而定一尊。私學而相與非法教,人聞令下,則各以其學議之,入則心非,出則巷議,夸主以為名,異取以為高,率群下以造謗。如此弗禁,則主勢降乎上,黨與成乎下。禁之便。臣請史官非秦記皆燒之。非博士官所職,天下敢有藏詩、書、百家語者,悉詣守、尉雜燒之。有敢偶語詩書者棄市。以古非今者族。吏見知不舉者與同罪。令下三十日不燒,黥為城旦。所不去者,醫藥卜筮種樹之書。若欲有學法令,以吏為師。」制曰:「可。」

三十五年,除道,道九原抵雲陽,塹山堙谷,直通之。於是始皇以為咸陽人多,先王之宮廷小,吾聞周文王都豐,武王都鎬,豐鎬之閒,帝王之都也。乃營作朝宮渭南上林苑中。先作前殿阿房,東西五百步,南北五十丈,上可以坐萬人,下可以建五丈旗。周馳為閣道,自殿下直抵南山。表南山之顛以為闕。為復道,自阿房渡渭,屬之咸陽,以象天極閣道絕漢抵營室也。阿房宮未成;成,欲更擇令名名之。作宮阿房,故天下謂之阿房宮。隱宮徒刑者七十餘萬人,乃分作阿房宮,或作麗山。發北山石槨,乃寫蜀、荊地材皆至。關中計宮三百,關外四百餘。於是立石東海上朐界中,以為秦東門。因徙三萬家麗邑,五萬家雲陽,皆復不事十歲。

盧生說始皇曰:「臣等求芝奇藥僊者常弗遇,類物有害之者。方中,人主時為微行以辟惡鬼,惡鬼辟,真人至。人主所居而人臣知之,則害於神。真人者,入水不濡,入火不爇,陵雲氣,與天地久長。今上治天下,未能恬倓。願上所居宮毋令人知,然後不死之藥殆可得也。」於是始皇曰:「吾慕真人,自謂『真人』,不稱『朕』。」乃令咸陽之旁二百里內宮觀二百七十復道甬道相連,帷帳鐘鼓美人充之,各案署不移徙。行所幸,有言其處者,罪死。始皇帝幸梁山宮,從山上見丞相車騎眾,弗善也。中人或告丞相,丞相後損車騎。始皇怒曰:「此中人泄吾語。」案問莫服。當是時,詔捕諸時在旁者,皆殺之。自是後莫知行之所在。聽事,群臣受決事,悉於咸陽宮。

侯生盧生相與謀曰:「始皇為人,天性剛戾自用,起諸侯,并天下,意得欲從,以為自古莫及己。專任獄吏,獄吏得親幸。博士雖七十人,特備員弗用。丞相諸大臣皆受成事,倚辨於上。上樂以刑殺為威,天下畏罪持祿,莫敢盡忠。上不聞過而日驕,下懾伏謾欺以取容。秦法,不得兼方不驗,輒死。然候星氣者至三百人,皆良士,畏忌諱諛,不敢端言其過。天下之事無小大皆決於上,上至以衡石量書,日夜有呈,不中呈不得休息。貪於權勢至如此,未可為求僊藥。」於是乃亡去。始皇聞亡,乃大怒曰:「吾前收天下書不中用者盡去之。悉召文學方術士甚眾,欲以興太平,方士欲練以求奇藥。今聞韓眾去不報,徐市等費以巨萬計,終不得藥,徒姦利相告日聞。盧生等吾尊賜之甚厚,今乃誹謗我,以重吾不德也。諸生在咸陽者,吾使人廉問,或為訞言以亂黔首。」於是使御史悉案問諸生,諸生傳相告引,乃自除犯禁者四百六十餘人,皆阬之咸陽,使天下知之,以懲後。益發謫徙邊。始皇長子扶蘇諫曰:「天下初定,遠方黔首未集,諸生皆誦法孔子,今上皆重法繩之,臣恐天下不安。唯上察之。」始皇怒,使扶蘇北監蒙恬於上郡。

三十六年,熒惑守心。有墜星下東郡,至地為石,黔首或刻其石曰「始皇帝死而地分」。始皇聞之,遣御史逐問,莫服,盡取石旁居人誅之,因燔銷其石。始皇不樂,使博士為僊真人詩,及行所游天下,傳令樂人歌弦之。秋,使者從關東夜過華陰平舒道,有人持璧遮使者曰:「為吾遺滈池君。」因言曰:「今年祖龍死。」使者問其故,因忽不見,置其璧去。使者奉璧具以聞。始皇默然良久,曰:「山鬼固不過知一歲事也。」退言曰:「祖龍者,人之先也。」使御府視璧,乃二十八年行渡江所沈璧也。於是始皇卜之,卦得游徙吉。遷北河榆中三萬家。拜爵一級。

三十七年十月癸丑,始皇出游。左丞相斯從,右丞相去疾守。少子胡亥愛慕請從,上許之。十一月,行至雲夢,望祀虞舜於九疑山。浮江下,觀籍柯,渡海渚。過丹陽,至錢唐。臨浙江,水波惡,乃西百二十里從狹中渡。上會稽,祭大禹,望于南海,而立石刻頌秦德。其文曰:

皇帝休烈,平一宇內,德惠修長。三十有七年,親巡天下,周覽遠方。遂登會稽,宣省習俗,黔首齋莊。群臣誦功,本原事跡,追首高明。秦聖臨國,始定刑名,顯陳舊章。初平法式,審別職任,以立恒常。六王專倍,貪戾傲猛,率眾自彊。暴虐恣行,負力而驕,數動甲兵。陰通閒使,以事合從,行為辟方。內飾詐謀,外來侵邊,遂起禍殃。義威誅之,殄熄暴悖,亂賊滅亡。聖德廣密,六合之中,被澤無疆。皇帝并宇,兼聽萬事,遠近畢清。運理群物,考驗事實,各載其名。貴賤并通,善否陳前,靡有隱情。飾省宣義,有子而嫁,倍死不貞。防隔內外,禁止淫泆,男女絜誠。夫為寄豭,殺之無罪,男秉義程。妻為逃嫁,子不得母,咸化廉清。大治濯俗,天下承風,蒙被休經。皆遵度軌,和安敦勉,莫不順令。黔首修絜,人樂同則,嘉保太平。後敬奉法,常治無極,輿舟不傾。從臣誦烈,請刻此石,光垂休銘。

還過吳,從江乘渡。并海上,北至瑯邪。方士徐市等入海求神藥,數歲不得,費多,恐譴,乃詐曰:「蓬萊藥可得,然常為大鮫魚所苦,故不得至,願請善射與俱,見則以連弩射之。」始皇夢與海神戰,如人狀。問占夢,博士曰:「水神不可見,以大魚蛟龍為候。今上禱祠備謹,而有此惡神,當除去,而善神可致。」乃令入海者齎捕巨魚具,而自以連弩候大魚出射之。自瑯邪北至榮成山,弗見。至之罘,見巨魚,射殺一魚。遂并海西。

至平原津而病。始皇惡言死,群臣莫敢言死事。上病益甚,乃為璽書賜公子扶蘇曰:「與喪會咸陽而葬。」書已封,在中車府令趙高行符璽事所,未授使者。七月丙寅,始皇崩於沙丘平臺。丞相斯為上崩在外,恐諸公子及天下有變,乃祕之,不發喪。棺載轀涼車中,故幸宦者參乘,所至上食。百官奏事如故,宦者輒從轀涼車中可其奏事。獨子胡亥、趙高及所幸宦者五六人知上死。趙高故嘗教胡亥書及獄律令法事,胡亥私幸之。高乃與公子胡亥、丞相斯陰謀破去始皇所封書賜公子扶蘇者,而更詐為丞相斯受始皇遺詔沙丘,立子胡亥為太子。更為書賜公子扶蘇、蒙恬,數以罪,(其)賜死。語具在李斯傳中。行,遂從井陘抵九原。會暑,上轀車臭,乃詔從官令車載一石鮑魚,以亂其臭。

行從直道至咸陽,發喪。太子胡亥襲位,為二世皇帝。九月,葬始皇酈山。始皇初即位,穿治酈山,及并天下,天下徒送詣七十餘萬人,穿三泉,下銅而致槨,宮觀百官奇器珍怪徙臧滿之。令匠作機弩矢,有所穿近者輒射之。以水銀為百川江河大海,機相灌輸,上具天文,下具地理。以人魚膏為燭,度不滅者久之。二世曰:「先帝後宮非有子者,出焉不宜。」皆令從死,死者甚眾。葬既已下,或言工匠為機,臧皆知之,臧重即泄。大事畢,已臧,閉中羨,下外羨門,盡閉工匠臧者,無復出者。樹草木以象山。

二世皇帝元年,年二十一。趙高為郎中令,任用事。二世下詔,增始皇寢廟犧牲及山川百祀之禮。令群臣議尊始皇廟。群臣皆頓首言曰:「古者天子七廟,諸侯五,大夫三,雖萬世世不軼毀。今始皇為極廟,四海之內皆獻貢職,增犧牲,禮咸備,毋以加。先王廟或在西雍,或在咸陽。天子儀當獨奉酌祠始皇廟。自襄公已下軼毀。所置凡七廟。群臣以禮進祠,以尊始皇廟為帝者祖廟。皇帝復自稱『朕』。」

二世與趙高謀曰:「朕年少,初即位,黔首未集附。先帝巡行郡縣,以示彊,威服海內。今晏然不巡行,即見弱,毋以臣畜天下。」春,二世東行郡縣,李斯從。到碣石,并海,南至會稽,而盡刻始皇所立刻石,石旁著大臣從者名,以章先帝成功盛德焉:

皇帝曰:「金石刻盡始皇帝所為也。今襲號而金石刻辭不稱始皇帝,其於久遠也如後嗣為之者,不稱成功盛德。」丞相臣斯、臣去疾、御史大夫臣德昧死言:「臣請具刻詔書刻石,因明白矣。臣昧死請。」制曰:「可。」

遂至遼東而還。

於是二世乃遵用趙高,申法令。乃陰與趙高謀曰:「大臣不服,官吏尚彊,及諸公子必與我爭,為之柰何?」高曰:「臣固願言而未敢也。先帝之大臣,皆天下累世名貴人也,積功勞世以相傳久矣。今高素小賤,陛下幸稱舉,令在上位,管中事。大臣鞅鞅,特以貌從臣,其心實不服。今上出,不因此時案郡縣守尉有罪者誅之,上以振威天下,下以除去上生平所不可者。今時不師文而決於武力,願陛下遂從時毋疑,即群臣不及謀。明主收舉餘民,賤者貴之,貧者富之,遠者近之,則上下集而國安矣。」二世曰:「善。」乃行誅大臣及諸公子,以罪過連逮少近官三郎,無得立者,而六公子戮死於杜。公子將閭昆弟三人囚於內宮,議其罪獨後。二世使使令將閭曰:「公子不臣,罪當死,吏致法焉。」將閭曰:「闕廷之禮,吾未嘗敢不從賓贊也;廊廟之位,吾未嘗敢失節也;受命應對,吾未嘗敢失辭也。何謂不臣?願聞罪而死。」使者曰:「臣不得與謀,奉書從事。」將閭乃仰天大呼天者三,曰:「天乎!吾無罪!」昆弟三人皆流涕拔劍自殺。宗室振恐。群臣諫者以為誹謗,大吏持祿取容,黔首振恐。

四月,二世還至咸陽,曰:「先帝為咸陽朝廷小,故營阿房宮為室堂。未就,會上崩,罷其作者,復土酈山。酈山事大畢,今釋阿房宮弗就,則是章先帝舉事過也。」復作阿房宮。外撫四夷,如始皇計。盡徵其材士五萬人為屯衛咸陽,令教射狗馬禽獸。當食者多,度不足,下調郡縣轉輸菽粟芻稿,皆令自齎糧食,咸陽三百里內不得食其穀。用法益刻深。

七月,戍卒陳勝等反故荊地,為「張楚」。勝自立為楚王,居陳,遣諸將徇地。山東郡縣少年苦秦吏,皆殺其守尉令丞反,以應陳涉,相立為侯王,合從西鄉,名為伐秦,不可勝數也。謁者使東方來,以反者聞二世。二世怒,下吏。後使者至,上問,對曰:「群盜,郡守尉方逐捕,今盡得,不足憂。」上悅。武臣自立為趙王,魏咎為魏王,田儋為齊王。沛公起沛。項梁舉兵會稽郡。

二年冬,陳涉所遣周章等將西至戲,兵數十萬。二世大驚,與群臣謀曰:「柰何?」少府章邯曰:「盜已至,眾彊,今發近縣不及矣。酈山徒多,請赦之,授兵以擊之。」二世乃大赦天下,使章邯將,擊破周章軍而走,遂殺章曹陽。二世益遣長史司馬欣、董翳佐章邯擊盜,殺陳勝城父,破項梁定陶,滅魏咎臨濟。楚地盜名將已死,章邯乃北渡河,擊趙王歇等於鉅鹿。

趙高說二世曰:「先帝臨制天下久,故群臣不敢為非,進邪說。今陛下富於春秋,初即位,柰何與公卿廷決事?事即有誤,示群臣短也。天子稱朕,固不聞聲。」於是二世常居禁中,與高決諸事。其後公卿希得朝見,盜賊益多,而關中卒發東擊盜者毋已。右丞相去疾、左丞相斯、將軍馮劫進諫曰:「關東群盜并起,秦發兵誅擊,所殺亡甚眾,然猶不止。盜多,皆以戌漕轉作事苦,賦稅大也。請且止阿房宮作者,減省四邊戍轉。」二世曰:「吾聞之韓子曰:『堯舜采椽不刮,茅茨不翦,飯土塯,啜土形,雖監門之養,不觳於此。禹鑿龍門,通大夏,決河亭水,放之海,身自持筑臿,脛毋毛,臣虜之勞不烈於此矣。』凡所為貴有天下者,得肆意極欲,主重明法,下不敢為非,以制御海內矣。夫虞、夏之主,貴為天子,親處窮苦之實,以徇百姓,尚何於法?朕尊萬乘,毋其實,吾欲造千乘之駕,萬乘之屬,充吾號名。且先帝起諸侯,兼天下,天下已定,外攘四夷以安邊竟,作宮室以章得意,而君觀先帝功業有緒。今朕即位二年之閒,群盜并起,君不能禁,又欲罷先帝之所為,是上毋以報先帝,次不為朕盡忠力,何以在位?」下去疾、斯、劫吏,案責他罪。去疾、劫曰:「將相不辱。」自殺。斯卒囚,就五刑。

三年,章邯等將其卒圍鉅鹿,楚上將軍項羽將楚卒往救鉅鹿。冬,趙高為丞相,竟案李斯殺之。夏,章邯等戰數卻,二世使人讓邯,邯恐,使長史欣請事。趙高弗見,又弗信。欣恐,亡去,高使人捕追不及。欣見邯曰:「趙高用事於中,將軍有功亦誅,無功亦誅。」項羽急擊秦軍,虜王離,邯等遂以兵降諸侯。八月己亥,趙高欲為亂,恐群臣不聽,乃先設驗,持鹿獻於二世,曰:「馬也。」二世笑曰:「丞相誤邪?謂鹿為馬。」問左右,左右或默,或言馬以阿順趙高。或言鹿(者),高因陰中諸言鹿者以法。後群臣皆畏高。

高前數言「關東盜毋能為也」,及項羽虜秦將王離等鉅鹿下而前,章邯等軍數卻,上書請益助,燕、趙、齊、楚、韓、魏皆立為王,自關以東,大氐盡畔秦吏應諸侯,諸侯咸率其眾西鄉。沛公將數萬人已屠武關,使人私於高,高恐二世怒,誅及其身,乃謝病不朝見。二世夢白虎齧其左驂馬,殺之,心不樂,怪問占夢。卜曰:「涇水為祟。」二世乃齋於望夷宮,欲祠涇,沈四白馬。使使責讓高以盜賊事。高懼,乃陰與其婿咸陽令閻樂、其弟趙成謀曰:「上不聽諫,今事急,欲歸禍於吾宗。吾欲易置上,更立公子嬰。子嬰仁儉,百姓皆載其言。」使郎中令為內應,詐為有大賊,令樂召吏發卒,追劫樂母置高舍。遣樂將吏卒千餘人至望夷宮殿門,縛衛令仆射,曰:「賊入此,何不止?」衛令曰:「周廬設卒甚謹,安得賊敢入宮?」樂遂斬衛令,直將吏入,行射,郎宦者大驚,或走或格,格者輒死,死者數十人。郎中令與樂俱入,射上幄坐幃。二世怒,召左右,左右皆惶擾不鬬。旁有宦者一人,侍不敢去。二世入內,謂曰:「公何不蚤告我?乃至於此!」宦者曰:「臣不敢言,故得全。使臣蚤言,皆已誅,安得至今?」閻樂前即二世數曰:「足下驕恣,誅殺無道,天下共畔足下,足下其自為計。」二世曰:「丞相可得見否?」樂曰:「不可。」二世曰:「吾願得一郡為王。」弗許。又曰:「願為萬戶侯。」弗許。曰:「願與妻子為黔首,比諸公子。」閻樂曰:「臣受命於丞相,為天下誅足下,足下雖多言,臣不敢報。」麾其兵進。二世自殺。

閻樂歸報趙高,趙高乃悉召諸大臣公子,告以誅二世之狀。曰:「秦故王國,始皇君天下,故稱帝。今六國復自立,秦地益小,乃以空名為帝,不可。宜為王如故,便。」立二世之兄子公子嬰為秦王。以黔首葬二世杜南宜春苑中。令子嬰齋,當廟見,受王璽。齋五日,子嬰與其子二人謀曰:「丞相高殺二世望夷宮,恐群臣誅之,乃詳以義立我。我聞趙高乃與楚約,滅秦宗室而王關中。今使我齋見廟,此欲因廟中殺我。我稱病不行,丞相必自來,來則殺之。」高使人請子嬰數輩,子嬰不行,高果自往,曰:「宗廟重事,王柰何不行?」子嬰遂刺殺高於齋宮,三族高家以徇咸陽。子嬰為秦王四十六日,楚將沛公破秦軍入武關,遂至霸上,使人約降子嬰。子嬰即系頸以組,白馬素車,奉天子璽符,降軹道旁。沛公遂入咸陽,封宮室府庫,還軍霸上。居月餘,諸侯兵至,項籍為從長,殺子嬰及秦諸公子宗族。遂屠咸陽,燒其宮室,虜其子女,收其珍寶貨財,諸侯共分之。滅秦之後,各分其地為三,名曰雍王、塞王、翟王,號曰三秦。項羽為西楚霸王,主命分天下王諸侯,秦竟滅矣。後五年,天下定於漢。

太史公曰:秦之先伯翳,嘗有勳於唐虞之際,受土賜姓。及殷夏之閒微散。至周之衰,秦興,邑于西垂。自繆公以來,稍蠶食諸侯,竟成始皇。始皇自以為功過五帝,地廣三王,而羞與之侔。善哉乎賈生推言之也!曰:

秦并兼諸侯山東三十餘郡,繕津關,據險塞,修甲兵而守之。然陳涉以戍卒散亂之眾數百,奮臂大呼,不用弓戟之兵,鉏櫌白梃,望屋而食,橫行天下。秦人阻險不守,關梁不闔,長戟不刺,彊弩不射。楚師深入,戰於鴻門,曾無藩籬之艱。於是山東大擾,諸侯并起,豪俊相立。秦使章邯將而東征,章邯因以三軍之眾要市於外,以謀其上。群臣之不信,可見於此矣。子嬰立,遂不寤。藉使子嬰有庸主之材,僅得中佐,山東雖亂,秦之地可全而有,宗廟之祀未當絕也。

秦地被山帶河以為固,四塞之國也。自繆公以來,至於秦王,二十餘君,常為諸侯雄。豈世世賢哉?其勢居然也。且天下嘗同心并力而攻秦矣。當此之世,賢智并列,良將行其師,賢相通其謀,然困於阻險而不能進,秦乃延入戰而為之開關,百萬之徒逃北而遂壞。豈勇力智慧不足哉?形不利,勢不便也。秦小邑并大城,守險塞而軍,高壘毋戰,閉關據阨,荷戟而守之。諸侯起於匹夫,以利合,非有素王之行也。其交未親,其下未附,名為亡秦,其實利之也。彼見秦阻之難犯也,必退師。安土息民,以待其敝,收弱扶罷,以令大國之君,不患不得意於海內。貴為天子,富有天下,而身為禽者,其救敗非也。

秦王足己不問,遂過而不變。二世受之,因而不改,暴虐以重禍。子嬰孤立無親,危弱無輔。三主惑而終身不悟,亡,不亦宜乎?當此時也,世非無深慮知化之士也,然所以不敢盡忠拂過者,秦俗多忌諱之禁,忠言未卒於口而身為戮沒矣。故使天下之士,傾耳而聽,重足而立,拑口而不言。是以三主失道,忠臣不敢諫,智士不敢謀,天下已亂,姦不上聞,豈不哀哉!先王知雍蔽之傷國也,故置公卿大夫士,以飾法設刑,而天下治。其彊也,禁暴誅亂而天下服。其弱也,五伯征而諸侯從。其削也,內守外附而社稷存。故秦之盛也,繁法嚴刑而天下振;及其衰也,百姓怨望而海內畔矣。故周五序得其道,而千餘歲不絕。秦本末并失,故不長久。由此觀之,安危之統相去遠矣。野諺曰「前事之不忘,後事之師也」。是以君子為國,觀之上古,驗之當世,參以人事,察盛衰之理,審權勢之宜,去就有序,變化有時,故曠日長久而社稷安矣。

秦孝公據殽函之固,擁雍州之地,君臣固守而窺周室,有席卷天下,包舉宇內,囊括四海之意,并吞八荒之心。當是時,商君佐之,內立法度,務耕織,修守戰之備,外連衡而鬬諸侯,於是秦人拱手而取西河之外。

孝公既沒,惠王、武王蒙故業,因遺冊,南兼漢中,西舉巴、蜀,東割膏腴之地,收要害之郡。諸侯恐懼,會盟而謀弱秦,不愛珍器重寶肥美之地,以致天下之士,合從締交,相與為一。當是時,齊有孟嘗,趙有平原,楚有春申,魏有信陵。此四君者,皆明知而忠信,寬厚而愛人,尊賢重士,約從離衡,并韓、魏、燕、楚、齊、趙、宋、衛、中山之眾。於是六國之士有寧越、徐尚、蘇秦、杜赫之屬為之謀,齊明、周最、陳軫、昭滑、樓緩、翟景、蘇厲、樂毅之徒通其意,吳起、孫臏、帶佗、兒良、王廖、田忌、廉頗、趙奢之朋制其兵。常以十倍之地,百萬之眾,叩關而攻秦。秦人開關延敵,九國之師逡巡遁逃而不敢進。秦無亡矢遺鏃之費,而天下諸侯已困矣。於是從散約解,爭割地而奉秦。秦有餘力而制其敝,追亡逐北,伏尸百萬,流血漂鹵。因利乘便,宰割天下,分裂河山,彊國請服,弱國入朝。延及孝文王、莊襄王,享國日淺,國家無事。

及至秦王,續六世之餘烈,振長策而御宇內,吞二周而亡諸侯,履至尊而制六合,執棰拊以鞭笞天下,威振四海。南取百越之地,以為桂林、象郡,百越之君俛首系頸,委命下吏。乃使蒙恬北筑長城而守藩籬,卻匈奴七百餘里,胡人不敢南下而牧馬,士不敢彎弓而報怨。於是廢先王之道,焚百家之言,以愚黔首。墮名城,殺豪俊,收天下之兵聚之咸陽,銷鋒鑄鐻,以為金人十二,以弱黔首之民。然後斬華為城,因河為津,據億丈之城,臨不測之谿以為固。良將勁弩守要害之處,信臣精卒陳利兵而誰何,天下以定。秦王之心,自以為關中之固,金城千里,子孫帝王萬世之業也。

秦王既沒,餘威振於殊俗。陳涉,罋牖繩樞之子,甿隸之人,而遷徙之徒,才能不及中人,非有仲尼、墨翟之賢,陶朱、猗頓之富,躡足行伍之閒,而倔起什伯之中,率罷散之卒,將數百之眾,而轉攻秦。斬木為兵,揭竿為旗,天下雲集響應,贏糧而景從,山東豪俊遂并起而亡秦族矣。

且夫天下非小弱也,雍州之地,殽函之固自若也。陳涉之位,非尊於齊、楚、燕、趙、韓、魏、宋、衛、中山之君;鉏櫌棘矜,非錟於句戟長鎩也;適戍之眾,非抗於九國之師;深謀遠慮,行軍用兵之道,非及鄉時之士也。然而成敗異變,功業相反也。試使山東之國與陳涉度長絜大,比權量力,則不可同年而語矣。然秦以區區之地,千乘之權,招八州而朝同列,百有餘年矣。然後以六合為家,殽函為宮,一夫作難而七廟墮,身死人手,為天下笑者,何也?仁義不施而攻守之勢異也。

秦并海內,兼諸侯,南面稱帝,以養四海,天下之士斐然鄉風,若是者何也?曰:近古之無王者久矣。周室卑微,五霸既歿,令不行於天下,是以諸侯力政,彊侵弱,眾暴寡,兵革不休,士民罷敝。今秦南面而王天下,是上有天子也。既元元之民冀得安其性命,莫不虛心而仰上,當此之時,守威定功,安危之本在於此矣。

秦王懷貪鄙之心,行自奮之智,不信功臣,不親士民,廢王道,立私權,禁文書而酷刑法,先詐力而後仁義,以暴虐為天下始。夫并兼者高詐力,安定者貴順權,此言取與守不同術也。秦離戰國而王天下,其道不易,其政不改,是其所以取之守之者[無]異也。孤獨而有之,故其亡可立而待。借使秦王計上世之事,并殷周之跡,以制御其政,後雖有淫驕之主而未有傾危之患也。故三王之建天下,名號顯美,功業長久。

今秦二世立,天下莫不引領而觀其政。夫寒者利裋褐而饑者甘糟糠,天下之嗷嗷,新主之資也。此言勞民之易為仁也。鄉使二世有庸主之行,而任忠賢,臣主一心而憂海內之患,縞素而正先帝之過,裂地分民以封功臣之後,建國立君以禮天下,虛囹圉而免刑戮,除去收帑汙穢之罪,使各反其鄉里,發倉廩,散財幣,以振孤獨窮困之士,輕賦少事,以佐百姓之急,約法省刑以持其後,使天下之人皆得自新,更節修行,各慎其身,塞萬民之望,而以威德與天下,天下集矣。即四海之內,皆讙各自安樂其處,唯恐有變,雖有狡猾之民,無離上之心,則不軌之臣無以飾其智,而暴亂之姦止矣。二世不行此術,而重之以無道,壞宗廟與民,更始作阿房宮,繁刑嚴誅,吏治刻深,賞罰不當,賦斂無度,天下多事,吏弗能紀,百姓困窮而主弗收恤。然後姦偽并起,而上下相遁,蒙罪者眾,刑戮相望於道,而天下苦之。自君卿以下至于眾庶,人懷自危之心,親處窮苦之實,咸不安其位,故易動也。是以陳涉不用湯武之賢,不藉公侯之尊,奮臂於大澤而天下響應者,其民危也。故先王見始終之變,知存亡之機,是以牧民之道,務在安之而已。天下雖有逆行之臣,必無響應之助矣。故曰「安民可與行義,而危民易與為非」,此之謂也。貴為天子,富有天下,身不免於戮殺者,正傾非也。是二世之過也。

襄公立,享國十二年。初為西畤。葬西垂。生文公。

文公立,居西垂宮。五十年死,葬西垂。生靜公。

靜公不享國而死。生憲公。

憲公享國十二年,居西新邑。死,葬衙。生武公、德公、出子。

出子享國六年,居西陵。庶長弗忌、威累、參父三人,率賊賊出子鄙衍,葬衙。武公立。

武公享國二十年。居平陽封宮。葬宣陽聚東南。三庶長伏其罪。德公立。

德公享國二年。居雍大鄭宮。生宣公、成公、繆公。葬陽。初伏,以御蠱。

宣公享國十二年。居陽宮。葬陽。初志閏月。

成公享國四年,居雍之宮。葬陽。齊伐山戎、孤竹。

繆公享國三十九年。天子致霸。葬雍。繆公學著人。生康公。

康公享國十二年。居雍高寢。葬竘社。生共公。

共公享國五年,居雍高寢。葬康公南。生桓公。

桓公享國二十七年。居雍太寢。葬義裏丘北。生景公。

景公享國四十年。居雍高寢,葬丘裏南。生畢公。

畢公享國三十六年。葬車裏北。生夷公。

夷公不享國。死,葬左宮。生惠公。

惠公享國十年。葬車里(康景)。生悼公。

悼公享國十五年。葬僖公西。城雍。生剌龔公。

剌龔公享國三十四年。葬入里。生躁公、懷公。其十年,彗星見。

躁公享國十四年。居受寢。葬悼公南。其元年,彗星見。

懷公從晉來。享國四年。葬櫟圉氏。生靈公。諸臣圍懷公,懷公自殺。

肅靈公,昭子子也。居涇陽。享國十年。葬悼公西。生簡公。

簡公從晉來。享國十五年。葬僖公西。生惠公。其七年。百姓初帶劍。

惠公享國十三年。葬陵圉。生出公。

出公享國二年。出公自殺,葬雍。

獻公享國二十三年。葬囂圉。生孝公。

孝公享國二十四年。葬弟圉。生惠文王。其十三年,始都咸陽。

惠文王享國二十七年。葬公陵。生悼武王。

悼武王享國四年,葬永陵。

昭襄王享國五十六年。葬茝陽。生孝文王。

孝文王享國一年。葬壽陵。生莊襄王。

莊襄王享國三年。葬茝陽。生始皇帝。呂不韋相。

獻公立七年,初行為市。十年,為戶籍相伍。

孝公立十六年。時桃李冬華。

惠文王生十九年而立。立二年,初行錢。有新生嬰兒曰「秦且王」。

悼武王生十九年而立。立三年,渭水赤三日。

昭襄王生十九年而立。立四年,初為田開阡陌。

孝文王生五十三年而立。

莊襄王生三十二年而立。立二年,取太原地。莊襄王元年,大赦,修先王功臣,施德厚骨肉,布惠於民。東周與諸侯謀秦,秦使相國不韋誅之,盡入其國。秦不絕其祀,以陽人地賜周君,奉其祭祀。

始皇享國三十七年。葬酈邑。生二世皇帝。始皇生十三年而立。

二世皇帝享國三年。葬宜春。趙高為丞相安武侯。二世生十二年而立。

右秦襄公至二世,六百一十歲。

孝明皇帝十七年十月十五日乙丑,曰:

周歷已移,仁不代母。秦直其位,呂政殘虐。然以諸侯十三,并兼天下,極情縱欲,養育宗親。三十七年,兵無所不加,制作政令,施於後王。蓋得聖人之威,河神授圖,據狼、狐,蹈參、伐,佐政驅除,距之稱始皇。

始皇既歿,胡亥極愚,酈山未畢,復作阿房,以遂前策。云「凡所為貴有天下者,肆意極欲,大臣至欲罷先君所為」。誅斯、去疾,任用趙高。痛哉言乎!人頭畜鳴。不威不伐惡,不篤不虛亡,距之不得留,殘虐以促期,雖居形便之國,猶不得存。

子嬰度次得嗣,冠玉冠,佩華紱,車黃屋,從百司,謁七廟。小人乘非位,莫不怳忽失守,偷安日日,獨能長念卻慮,父子作權,近取於戶牖之閒,竟誅猾臣,為君討賊。高死之後,賓婚未得盡相勞,餐未及下咽,酒未及濡脣,楚兵已屠關中,真人翔霸上,素車嬰組,奉其符璽,以歸帝者。鄭伯茅旌鸞刀,嚴王退舍。河決不可復壅,魚爛不可復全。賈誼、司馬遷曰:「向使嬰有庸主之才,僅得中佐,山東雖亂,秦之地可全而有,宗廟之祀未當絕也。」秦之積衰,天下土崩瓦解,雖有周旦之材,無所復陳其巧,而以責一日之孤,誤哉!俗傳秦始皇起罪惡,胡亥極,得其理矣。復責小子,云秦地可全,所謂不通時變者也。紀季以酅,春秋不名。吾讀秦紀,至於子嬰車裂趙高,未嘗不健其決,憐其志。嬰死生之義備矣。


\end{pinyinscope}