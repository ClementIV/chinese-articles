\article{河渠書}

\begin{pinyinscope}
《夏書》曰:禹抑洪水十三年,過家不入門。陸行載車,水行載舟,泥行蹈毳,山行即橋。以別九州,隨山浚川,任土作貢。通九道,陂九澤,度九山。然河菑衍溢,害中國也尤甚。唯是為務。故道河自積石歷龍門,南到華陰,東下砥柱,及孟津、雒汭,至于大邳。於是禹以為河所從來者高,水湍悍,難以行平地,數為敗,乃二渠以引其河。北載之高地,過降水,至于大陸,播為九河,同為逆河,入于勃海九川既疏,九澤既灑,諸夏艾安,功施于三代。

自是之後,滎陽下引河東南為鴻溝,以通宋、鄭、陳、蔡、曹、衛,與濟、汝、淮、泗會。于楚,西方則通渠漢水、雲夢之野,東方則通[鴻]溝江淮之閒。於吳,則通渠三江、五湖。於齊,則通菑濟之閒。於蜀,蜀守冰鑿離碓,辟沫水之害,穿二江成都之中。此渠皆可行舟,有餘則用溉浸,百姓饗其利。至于所過,往往引其水益用溉田疇之渠,以萬億計,然莫足數也。

西門豹引漳水溉鄴,以富魏之河內。

而韓聞秦之好興事,欲罷之,毋令東伐,乃使水工鄭國閒說秦,令鑿涇水自中山西邸瓠口為渠,并北山東注洛三百餘里,欲以溉田。中作而覺,秦欲殺鄭國。鄭國曰:「始臣為閒,然渠成亦秦之利也。」秦以為然,卒使就渠。渠就,用注填閼之水,溉澤鹵之地四萬餘頃,收皆畝一鐘。於是關中為沃野,無凶年,秦以富彊,卒并諸侯,因命曰鄭國渠。

漢興三十九年,孝文時河決酸棗,東潰金隄,於是東郡大興卒塞之。

其後四十有餘年,今天子元光之中,而河決於瓠子,東南注鉅野,通於淮、泗。於是天子使汲黯、鄭當時興人徒塞之,輒復壞。是時武安侯田蚡為丞相,其奉邑食鄃。鄃居河北,河決而南則鄃無水菑,邑收多。蚡言於上曰:「江河之決皆天事,未易以人力為彊塞,塞之未必應天。」而望氣用數者亦以為然。於是天子久之不事復塞也。

是時鄭當時為大農,言曰:「異時關東漕粟從渭中上,度六月而罷,而漕水道九百餘里,時有難處。引渭穿渠起長安,并南山下,至河三百餘里,徑,易漕,度可令三月罷;而渠下民田萬餘頃,又可得以溉田:此損漕省卒,而益肥關中之地,得穀。」天子以為然,令齊人水工徐伯表,悉發卒數萬人穿漕渠,三歲而通。通,以漕,大便利。其後漕稍多,而渠下之民頗得以溉田矣。

其後河東守番系言:「漕從山東西,歲百餘萬石,更砥柱之限,敗亡甚多,而亦煩費。穿渠引汾溉皮氏、汾陰下,引河溉汾陰、蒲阪下,度可得五千頃。五千頃故盡河壖棄地,民茭牧其中耳,今溉田之,度可得穀二百萬石以上。穀從渭上,與關中無異,而砥柱之東可無復漕。」天子以為然,發卒數萬人作渠田。數歲,河移徙,渠不利,則田者不能償種。久之,河東渠田廢,予越人,令少府以為稍入。

其後人有上書欲通褒斜道及漕事,下御史大夫張湯。湯問其事,因言:「抵蜀從故道,故道多阪,回遠。今穿褒斜道,少阪,近四百里;而褒水通沔,斜水通渭,皆可以行船漕。漕從南陽上沔入褒,褒之絕水至斜,閒百餘里,以車轉,從斜下下渭。如此,漢中之穀可致,山東從沔無限,便於砥柱之漕。且褒斜材木竹箭之饒,擬於巴蜀。」天子以為然,拜湯子卬為漢中守,發數萬人作褒斜道五百餘里。道果便近,而水湍石,不可漕。

其後莊熊羆言:「臨晉民願穿洛以溉重泉以東萬餘頃故鹵地。誠得水,可令畝十石。」於是為發卒萬餘人穿渠,自徵引洛水至商顏山下。岸善崩,乃鑿井,深者四十餘丈。往往為井,井下相通行水。水穨以絕商顏,東至山嶺十餘里閒。井渠之生自此始。穿渠得龍骨,故名曰龍首渠。作之十餘歲,渠頗通,猶未得其饒。

自河決瓠子後二十餘歲,歲因以數不登,而梁楚之地尤甚。天子既封禪巡祭山川,其明年,旱,乾封少雨。天子乃使汲仁、郭昌發卒數萬人塞瓠子決。於是天子已用事萬里沙,則還自臨決河,沈白馬玉璧于河,令群臣從官自將軍已下皆負薪窴決河。是時東郡燒草,以故薪柴少,而下淇園之竹以為楗。

天子既臨河決,悼功之不成,乃作歌曰:「瓠子決兮將柰何?皓皓旰旰兮閭殫為河!殫為河兮地不得寧,功無已時兮吾山平。吾山平兮鉅野溢,魚沸郁兮柏冬日。延道弛兮離常流,蛟龍騁兮方遠遊。歸舊川兮神哉沛,不封禪兮安知外!為我謂河伯兮何不仁,泛濫不止兮愁吾人?齧桑浮兮淮、泗滿,久不反兮水維緩。」一曰:「河湯湯兮激潺湲,北渡污兮浚流難。搴長茭兮沈美玉,河伯許兮薪不屬。薪不屬兮衛人罪,燒蕭條兮噫乎何以御水!穨林竹兮楗石菑,宣房塞兮萬福來。」於是卒塞瓠子,筑宮其上,名曰宣房宮。而道河北行二渠,復禹舊跡,而梁、楚之地復寧,無水災。

自是之後,用事者爭言水利。朔方、西河、河西、酒泉皆引河及川谷以溉田;而關中輔渠、靈軹引堵水;汝南、九江引淮;東海引鉅定;泰山下引汶水:皆穿渠為溉田,各萬餘頃。佗小渠披山通道者,不可勝言。然其著者在宣房。

太史公曰:余南登廬山,觀禹疏九江,遂至于會稽太湟,上姑蘇,望五湖;東闚洛汭、大邳,迎河,行淮、泗、濟、漯洛渠;西瞻蜀之岷山及離碓;北自龍門至于朔方。曰:甚哉,水之為利害也!余從負薪塞宣房,悲瓠子之詩而作河渠書。


\end{pinyinscope}