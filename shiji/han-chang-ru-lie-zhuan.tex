\article{韓長孺列傳}

\begin{pinyinscope}
御史大夫韓安國者,梁成安人也,後徙睢陽。嘗受韓子、雜家說於騶田生所。事梁孝王為中大夫。吳楚反時,孝王使安國及張羽為將,捍吳兵於東界。張羽力戰,安國持重,以故吳不能過梁。吳楚已破,安國、張羽名由此顯。

梁孝王,景帝母弟,竇太后愛之,令得自請置相、二千石,出入游戲,僭於天子。天子聞之,心弗善也。太后知帝不善,乃怒梁使者,弗見,案責王所為。韓安國為梁使,見大長公主而泣曰:「何梁王為人子之孝,為人臣之忠,太后曾弗省也?夫前日吳、楚、齊、趙七國反時,自關以東皆合從西鄉,惟梁最親為艱難。梁王念太后、帝在中,而諸侯擾亂,一言泣數行下,跪送臣等六人,將兵擊卻吳楚,吳楚以故兵不敢西,而卒破亡,梁王之力也。今太后以小睗苛禮責望梁王。梁王父兄皆帝王,所見者大,故出稱蹕,入言警,車旗皆帝所賜也,即欲以侘鄙縣,驅馳國中,以夸諸侯,令天下盡知太后、帝愛之也。今梁使來,輒案責之。梁王恐,日夜涕泣思慕,不知所為。何梁王之為子孝,為臣忠,而太后弗恤也?」大長公主具以告太后,太后喜曰:「為言之帝。」言之,帝心乃解,而免冠謝太后曰:「兄弟不能相教,乃為太后遺憂。」悉見梁使,厚賜之。其後梁王益親驩。太后、長公主更賜安國可直千餘金。名由此顯,結於漢。

其後安國坐法抵罪,蒙獄吏田甲辱安國。安國曰:「死灰獨不復然乎?」田甲曰:「然即溺之。」居無何,梁內史缺,漢使使者拜安國為梁內史,起徒中為二千石。田甲亡走。安國曰:「甲不就官,我滅而宗。」甲因肉袒謝。安國笑曰:「可溺矣!公等足與治乎?」卒善遇之。

梁內史之缺也,孝王新得齊人公孫詭,說之,欲請以為內史。竇太后聞,乃詔王以安國為內史。

公孫詭、羊勝說孝王求為帝太子及益地事,恐漢大臣不聽,乃陰使人刺漢用事謀臣。及殺故吳相袁盎,景帝遂聞詭、勝等計畫,乃遣使捕詭、勝,必得。漢使十輩至梁,相以下舉國大索,月餘不得。內史安國聞詭、勝匿孝王所,安國入見王而泣曰:「主辱臣死。大王無良臣,故事紛紛至此。今詭、勝不得,請辭賜死。」王曰:「何至此?」安國泣數行下,曰:「大王自度於皇帝,孰與太上皇之與高皇帝及皇帝之與臨江王親?」孝王曰:「弗如也。」安國曰:「夫太上、臨江親父子之閒,然而高帝曰『提三尺劍取天下者朕也』,故太上皇終不得制事,居于櫟陽。臨江王,適長太子也,以一言過,廢王臨江;用宮垣事,卒自殺中尉府。何者?治天下終不以私亂公。語曰:『雖有親父,安知其不為虎?雖有親兄,安知其不為狼?』今大王列在諸侯,悅一邪臣浮說,犯上禁,橈明法。天子以太后故,不忍致法於王。太后日夜涕泣,幸大王自改,而大王終不覺寤。有如太后宮車即晏駕,大王尚誰攀乎?」語未卒,孝王泣數行下,謝安國曰:「吾今出詭、勝。」詭、勝自殺。漢使還報,梁事皆得釋,安國之力也。於是景帝、太后益重安國。孝王卒,共王即位,安國坐法失官,居家。

建元中,武安侯田蚡為漢太尉,親貴用事,安國以五百金物遺蚡。蚡言安國太后,天子亦素聞其賢,即召以為北地都尉,遷為大司農。閩越、東越相攻,安國及大行王恢將。未至越,越殺其王降,漢兵亦罷。建元六年,武安侯為丞相,安國為御史大夫。

匈奴來請和親,天子下議。大行王恢,燕人也,數為邊吏,習知胡事。議曰:「漢與匈奴和親,率不過數歲即復倍約。不如勿許,興兵擊之。」安國曰:「千里而戰,兵不獲利。今匈奴負戎馬之足,懷禽獸之心,遷徙鳥舉,難得而制也。得其地不足以為廣,有其眾不足以為彊,自上古不屬為人。漢數千里爭利,則人馬罷,虜以全制其敝。且彊弩之極,矢不能穿魯縞;沖風之末,力不能漂鴻毛。非初不勁,末力衰也。擊之不便,不如和親。」群臣議者多附安國,於是上許和親。

其明年,則元光元年,雁門馬邑豪聶翁壹因大行王恢言上曰:「匈奴初和親,親信邊,可誘以利。」陰使聶翁壹為閒,亡入匈奴,謂單于曰:「吾能斬馬邑令丞吏,以城降,財物可盡得。」單于愛信之,以為然,許聶翁壹。聶翁壹乃還,詐斬死罪囚,縣其頭馬邑城,示單于使者為信。曰:「馬邑長吏已死,可急來。」於是單于穿塞將十餘萬騎,入武州塞。

當是時,漢伏兵車騎材官二十餘萬,匿馬邑旁谷中。衛尉李廣為驍騎將軍,太仆公孫賀為輕車將軍,大行王恢為將屯將軍,太中大夫李息為材官將軍。御史大夫韓安國為護軍將軍,諸將皆屬護軍。約單于入馬邑而漢兵縱發。王恢、李息、李廣別從代主擊其輜重。於是單于入漢長城武州塞。未至馬邑百餘里,行掠鹵,徒見畜牧於野,不見一人。單于怪之,攻烽燧,得武州尉史。欲刺問尉史。尉史曰:「漢兵數十萬伏馬邑下。」單于顧謂左右曰:「幾為漢所賣!」乃引兵還。出塞,曰:「吾得尉史,乃天也。」命尉史為「天王」。塞下傳言單于已引去。漢兵追至塞,度弗及,即罷。王恢等兵三萬,聞單于不與漢合,度往擊輜重,必與單于精兵戰,漢兵勢必敗,則以便宜罷兵,皆無功。

天子怒王恢不出擊單于輜重,擅引兵罷也。恢曰:「始約虜入馬邑城,兵與單于接,而臣擊其輜重,可得利。今單于聞,不至而還,臣以三萬人眾不敵,禔取辱耳。臣固知還而斬,然得完陛下士三萬人。」於是下恢廷尉。廷尉當恢逗橈,當斬。恢私行千金丞相蚡。蚡不敢言上,而言於太后曰:「王恢首造馬邑事,今不成而誅恢,是為匈奴報仇也。」上朝太后,太后以丞相言告上。上曰:「首為馬邑事者,恢也,故發天下兵數十萬,從其言,為此。且縱單于不可得,恢所部擊其輜重,猶頗可得,以慰士大夫心。今不誅恢,無以謝天下。」於是恢聞之,乃自殺。

安國為人多大略,智足以當世取合,而出於忠厚焉。貪嗜於財。所推舉皆廉士,賢於己者也。於梁舉壺遂、臧固、郅他,皆天下名士,士亦以此稱慕之,唯天子以為國器。安國為御史大夫四歲餘,丞相田蚡死,安國行丞相事,奉引墮車蹇。天子議置相,欲用安國,使使視之,蹇甚,乃更以平棘侯薛澤為丞相。安國病免數月,蹇愈,上復以安國為中尉。歲餘,徙為衛尉。

車騎將軍衛青擊匈奴,出上谷,破胡蘢城。將軍李廣為匈奴所得,復失之;公孫敖大亡卒:皆當斬,贖為庶人。明年,匈奴大入邊,殺遼西太守,及入鴈門,所殺略數千人。車騎將軍衛青擊之,出鴈門。衛尉安國為材官將軍,屯於漁陽。安國捕生虜,言匈奴遠去。即上書言方田作時,請且罷軍屯。罷軍屯月餘,匈奴大入上谷、漁陽。安國壁乃有七百餘人,出與戰,不勝,復入壁。匈奴虜略千餘人及畜產而去。天子聞之,怒,使使責讓安國。徒安國益東,屯右北平。是時匈奴虜言當入東方。

安國始為御史大夫及護軍,後稍斥疏,下遷;而新幸壯將軍衛青等有功,益貴。安國既疏遠,默默也;將屯又為匈奴所欺,失亡多,甚自愧。幸得罷歸,乃益東徙屯,意忽忽不樂。數月,病歐血死。安國以元朔二年中卒。

太史公曰:余與壺遂定律歷,觀韓長孺之義,壺遂之深中隱厚。世之言梁多長者,不虛哉!壺遂官至詹事,天子方倚以為漢相,會遂卒。不然,壺遂之內廉行修,斯鞠躬君子也。


\end{pinyinscope}