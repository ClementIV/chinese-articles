\article{陳涉世家}

\begin{pinyinscope}
陳勝者,陽城人也,字涉。吳廣者,陽夏人也,字叔。陳涉少時,嘗與人傭耕,輟耕之壟上,悵恨久之,曰:「茍富貴,無相忘。」庸者笑而應曰:「若為庸耕,何富貴也?」陳涉太息曰:「嗟乎,燕雀安知鴻鵠之志哉!」

二世元年七月,發閭左適戍漁陽,九百人屯大澤鄉。陳勝、吳廣皆次當行,為屯長。會天大雨,道不通,度已失期。失期,法皆斬。陳勝、吳廣乃謀曰:「今亡亦死,舉大計亦死,等死,死國可乎?」陳勝曰:「天下苦秦久矣。吾聞二世少子也,不當立,當立者乃公子扶蘇。扶蘇以數諫故,上使外將兵。今或聞無罪,二世殺之。百姓多聞其賢,未知其死也。項燕為楚將,數有功,愛士卒,楚人憐之。或以為死,或以為亡。今誠以吾眾詐自稱公子扶蘇、項燕,為天下唱,宜多應者。」吳廣以為然。乃行卜。卜者知其指意,曰:「足下事皆成,有功。然足下卜之鬼乎!」陳勝、吳廣喜,念鬼,曰:「此教我先威眾耳。」乃丹書帛曰「陳勝王」,置人所罾魚腹中。卒買魚烹食,得魚腹中書,固以怪之矣。又彊令吳廣之次所旁叢祠中,夜篝火,狐鳴呼曰「大楚興,陳勝王」。卒皆夜驚恐。旦日,卒中往往語,皆指目陳勝。

吳廣素愛人,士卒多為用者。將尉醉,廣故數言欲亡,忿恚尉,令辱之,以激怒其眾。尉果笞廣。尉劍挺,廣起,奪而殺尉。陳勝佐之,并殺兩尉。召令徒屬曰:「公等遇雨,皆已失期,失期當斬。藉弟令毋斬,而戍死者固十六七。且壯士不死即已,死即舉大名耳,王侯將相寧有種乎!」徒屬皆曰:「敬受命。」乃詐稱公子扶蘇、項燕,從民欲也。袒右,稱大楚。為壇而盟,祭以尉首。陳勝自立為將軍,吳廣為都尉。攻大澤鄉,收而攻蘄。蘄下,乃令符離人葛嬰將兵徇蘄以東。攻铚、酂、苦、柘、譙皆下之。行收兵。比至陳,車六七百乘,騎千餘,卒數萬人。攻陳,陳守令皆不在,獨守丞與戰譙門中。弗勝,守丞死,乃入據陳。數日,號令召三老、豪傑與皆來會計事。三老、豪傑皆曰:「將軍身被堅執銳,伐無道,誅暴秦,復立楚國之社稷,功宜為王。」陳涉乃立為王,號為張楚。

當此時,諸郡縣苦秦吏者,皆刑其長吏,殺之以應陳涉。乃以吳叔為假王,監諸將以西擊滎陽。令陳人武臣、張耳、陳餘徇趙地,令汝陰人鄧宗徇九江郡。當此時,楚兵數千人為聚者,不可勝數。

葛嬰至東城,立襄彊為楚王。嬰後聞陳王已立,因殺襄彊,還報。至陳,陳王誅殺葛嬰。陳王令魏人周市北徇魏地。吳廣圍滎陽。李由為三川守,守滎陽,吳叔弗能下。陳王徵國之豪傑與計,以上蔡人房君蔡賜為上柱國。

周文,陳之賢人也,嘗為項燕軍視日,事春申君,自言習兵,陳王與之將軍印,西擊秦。行收兵至關,車千乘,卒數十萬,至戲,軍焉。秦令少府章邯免酈山徒、人奴產子生,悉發以擊楚大軍,盡敗之。周文敗,走出關,止次曹陽二三月。章邯追敗之,復走次澠池十餘日。章邯擊,大破之。周文自剄,軍遂不戰。

武臣到邯鄲,自立為趙王,陳餘為大將軍,張耳、召騷為左右丞相。陳王怒,捕系武臣等家室,欲誅之。柱國曰:「秦未亡而誅趙王將相家屬,此生一秦也。不如因而立之。」陳王乃遣使者賀趙,而徙系武臣等家屬宮中,而封耳子張敖為成都君,趣趙兵亟入關。趙王將相相與謀曰:「王王趙,非楚意也。楚已誅秦,必加兵於趙。計莫如毋西兵,使使北徇燕地以自廣也。趙南據大河,北有燕、代,楚雖勝秦,不敢制趙。若楚不勝秦,必重趙。趙乘秦之獘,可以得志於天下。」趙王以為然,因不西兵,而遣故上谷卒史韓廣將兵北徇燕地。

燕故貴人豪傑謂韓廣曰:「楚已立王,趙又已立王。燕雖小,亦萬乘之國也,願將軍立為燕王。」韓廣曰:「廣母在趙,不可。」燕人曰:「趙方西憂秦,南憂楚,其力不能禁我。且以楚之彊,不敢害趙王將相之家,趙獨安敢害將軍之家!」韓廣以為然,乃自立為燕王。居數月,趙奉燕王母及家屬歸之燕。

當此之時,諸將之徇地者,不可勝數。周市北徇地至狄,狄人田儋殺狄令,自立為齊王,以齊反擊周市。市軍散,還至魏地,欲立魏後故甯陵君咎為魏王。時咎在陳王所,不得之魏。魏地已定,欲相與立周市為魏王,周市不肯。使者五反,陳王乃立甯陵君咎為魏王,遣之國。周市卒為相。

將軍田臧等相與謀曰:「周章軍已破矣,秦兵旦暮至,我圍滎陽城弗能下,秦軍至,必大敗。不如少遺兵,足以守(熒)[滎]陽,悉精兵迎秦軍。今假王驕,不知兵權,不可與計,非誅之,事恐敗。」因相與矯王令以誅吳叔,獻其首於陳王。陳王使使賜田臧楚令尹印,使為上將。田臧乃使諸將李歸等守滎陽城,自以精兵西迎秦軍於敖倉。與戰,田臧死,軍破。章邯進兵擊李歸等滎陽下,破之,李歸等死。

陽城人鄧說將兵居郯,章邯別將擊破之,鄧說軍散走陳。铚人伍徐將兵居許,章邯擊破之,伍徐軍皆散走陳。陳王誅鄧說。

陳王初立時,陵人秦嘉、铚人董緶符離人朱雞石、取慮人鄭布、徐人丁疾等皆特起,將兵圍東海守慶於郯。陳王聞,乃使武平君畔為將軍,監郯下軍。秦嘉不受命,嘉自立為大司馬,惡屬武平君。告軍吏曰:「武平君年少,不知兵事,勿聽!」因矯以王命殺武平君畔。

章邯已破伍徐,擊陳,柱國房君死。章邯又進兵擊陳西張賀軍。陳王出監戰,軍破,張賀死。

臘月,陳王之汝陰,還至下城父,其御莊賈殺以降秦。陳勝葬碭,謚曰隱王。

陳王故涓人將軍呂臣為倉頭軍,起新陽,攻陳下之,殺莊賈,復以陳為楚。

初,陳王至陳,令铚人宋留將兵定南陽,入武關。留已徇南陽,聞陳王死,南陽復為秦。宋留不能入武關,乃東至新蔡,遇秦軍,宋留以軍降秦。秦傳留至咸陽,車裂留以徇。

秦嘉等聞陳王軍破出走,乃立景駒為楚王,引兵之方與,欲擊秦軍定陶下。使公孫慶使齊王,欲與并力俱進。齊王曰:「聞陳王戰敗,不知其死生,楚安得不請而立王!」公孫慶曰:「齊不請楚而立王,楚何故請齊而立王!且楚首事,當令於天下。」田儋誅殺公孫慶。

秦左右校復攻陳,下之。呂將軍走,收兵復聚。鄱盜當陽君黥布之兵相收,復擊秦左右校,破之青波,復以陳為楚。會項梁立懷王孫心為楚王。

陳勝王凡六月。已為王,王陳。其故人嘗與庸耕者聞之,之陳,扣宮門曰:「吾欲見涉。」宮門令欲縛之。自辯數,乃置,不肯為通。陳王出,遮道而呼涉。陳王聞之,乃召見,載與俱歸。入宮,見殿屋帷帳,客曰:「夥頤!涉之為王沈沈者!」楚人謂多為夥,故天下傳之,夥涉為王,由陳涉始。客出入愈益發舒,言陳王故情。或說陳王曰:「客愚無知,顓妄言,輕威。」陳王斬之。諸陳王故人皆自引去,由是無親陳王者。陳王以朱房為中正,胡武為司過,主司群臣。諸將徇地,至,令之不是者,系而罪之,以苛察為忠。其所不善者,弗下吏,輒自治之。陳王信用之。諸將以其故不親附,此其所以敗也。

陳勝雖已死,其所置遣侯王將相竟亡秦,由涉首事也。高祖時為陳涉置守冢三十家碭,至今血食。

褚先生曰:地形險阻,所以為固也;兵革刑法,所以為治也。猶未足恃也。夫先王以仁義為本,而以固塞文法為枝葉,豈不然哉!吾聞賈生之稱曰:

秦孝公據殽函之固,擁雍州之地,君臣固守,以窺周室。有席卷天下,包舉宇內,囊括四海之意,并吞八荒之心。當是時也,商君佐之,內立法度,務耕織,修守戰之備;外連衡而鬬諸侯。於是秦人拱手而取西河之外。

孝公既沒,惠文王、武王、昭王蒙故業,因遺策,南取漢中,西舉巴蜀,東割膏腴之地,收要害之郡。諸侯恐懼,會盟而謀弱秦。不愛珍器重寶肥饒之地,以致天下之士。合從締交,相與為一。當此之時,齊有孟嘗,趙有平原,楚有春申,魏有信陵:此四君者,皆明知而忠信,寬厚而愛人,尊賢而重士。約從連衡,兼韓、魏、燕、趙、宋、衛、中山之眾。於是六國之士有甯越、徐尚、蘇秦、杜赫之屬為之謀,齊明、周聚、陳軫、邵滑、樓緩、翟景、蘇厲、樂毅之徒通其意,吳起、孫臏、帶他、兒良、王廖、田忌、廉頗、趙奢之倫制其兵。嘗以什倍之地,百萬之師,仰關而攻秦。秦人開關而延敵,九國之師遁逃而不敢進。秦無亡矢遺鏃之費,而天下固已困矣。於是從散約敗,爭割地而賂秦。秦有餘力而制其獘,追亡逐北,伏尸百萬,流血漂櫓,因利乘便,宰割天下,分裂山河,彊國請服,弱國入朝。

施及孝文王、莊襄王,享國之日淺,國家無事。及至始皇,奮六世之餘烈,振長策而御宇內,吞二周而亡諸侯,履至尊而制六合,執敲樸以鞭笞天下,威振四海。南取百越之地,以為桂林、象郡,百越之君俛首系頸,委命下吏。乃使蒙恬北筑長城而守藩籬,卻匈奴七百餘里,胡人不敢南下而牧馬,士亦不敢貫弓而報怨。於是廢先王之道,燔百家之言,以愚黔首。墮名城,殺豪俊,收天下之兵聚之咸陽,銷鋒鍉,鑄以為金人十二,以弱天下之民。然後踐華為城,因河為池,據億丈之城,臨不測之谿以為固。良將勁弩,守要害之處,信臣精卒,陳利兵而誰何。天下已定,始皇之心,自以為關中之固,金城千里,子孫帝王萬世之業也。

始皇既沒,餘威振於殊俗。然而陳涉甕牖繩樞之子,甿隸之人,而遷徙之徒也。材能不及中人,非有仲尼、墨翟之賢,陶朱、猗頓之富也。躡足行伍之閒,俛仰仟佰之中,率罷散之卒,將數百之眾,轉而攻秦。斬木為兵,揭竿為旗,天下雲會響應,贏糧而景從,山東豪俊遂并起而亡秦族矣。

且天下非小弱也;雍州之地,殽函之固自若也。陳涉之位,非尊於齊、楚、燕、趙、韓、魏、宋、衛、中山之君也;鉏耰棘矜,非铦於句戟長鎩也;適戍之眾,非儔於九國之師也;深謀遠慮,行軍用兵之道,非及鄉時之士也。然而成敗異變,功業相反也。嘗試使山東之國與陳涉度長絜大,比權量力,則不可同年而語矣。然而秦以區區之地。致萬乘之權,抑八州而朝同列,百有餘年矣。然後以六合為家,殽函為宮。一夫作難而七廟墮,身死人手,為天下笑者,何也?仁義不施,而攻守之勢異也。


\end{pinyinscope}