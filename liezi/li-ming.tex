\article{力命}

\begin{pinyinscope}
力謂命曰:「若之功奚若我哉?」命曰:「汝奚功於物,而欲比朕?」力曰:「壽夭、窮達、貴賤、貧富,我力之所能也。」命曰:「彭祖之智不出堯、舜之上,而壽八百;顏淵之才不出眾人之下,而壽十八。仲尼之德。不出諸侯之下,而困於陳,蔡;殷紂之行,不出三仁之上,而居君位。季札无爵於吳,田恆專有齊國。夷、齊餓於首陽,季氏富於展禽。若是汝力之所能,柰何壽彼而夭此,窮聖而達逆,賤賢而貴愚,貧善而富惡邪?」力曰:「若如若言,我固无功於物,而物若此邪,此則若之所制邪?」命曰:「既謂之命,柰何有制之者邪?朕直而推之,曲而任之。自壽自夭,自窮自達,自貴自賤,自富自貧,朕豈能識之哉?朕豈能識之哉?」

北宮子謂西門子曰:「朕與子並世也,而人子達;並族也,而人子敬;並貌也,而人子愛;並言也,而人子庸;並行也,而人子誠;並仕也,而人子貴;並農也,而人子富;並商也,而人子利。朕衣則裋褐,食則粢糲,居則蓬室,出則徒行。子衣則文錦,食則粱肉,居則連欐,出則結駟。在家熙然有棄朕之心,在朝諤然有敖朕之色。請謁不相及,遨游不同行,固有年矣。子自以德過朕邪?」西門子曰:「予无以知其實。汝造事而窮,予造事而達,此厚薄之驗歟?而皆謂與予並,汝之顏厚矣。」北宮子无以應,自失而歸。中途遇東郭先生。先生曰:「汝奚往而反,偊偊而步,有深愧之色邪?」北宮子言其狀。東郭先生曰:『吾將舍汝之愧,與汝更之西門氏而問之。「曰:「汝奚辱北宮子之深乎?固且言之。」西門子曰:「北宮子言世族、年貌、言行與予並,而賤貴、貧富與予異。予語之曰:『予无以知其實。汝造事而窮,予造事而達,此將厚薄之驗歟?而皆謂與予並,汝之顏厚矣。』」東郭先生曰:「汝之言厚薄,不過言才德之差,吾之言厚薄異於是矣。夫北宮子厚於德,薄於命;汝厚於命,薄於德。汝之達,非智得也;北宮子之窮,非愚失也。皆天也,非人也。而汝以命厚自矜,北宮子以德厚自愧,皆不識夫固然之理矣。」西門子曰:「先生止矣!予不敢復言。」北宮子既歸,衣其裋褐,有狐貉之溫;進其茙菽,有稻粱之味;庇其蓬室,若廣廈之蔭;乘其篳輅,若文軒之飾。終身逌然,不知榮辱之在彼也,在我也。東郭先生聞之曰:「北宮子之寐久矣,一言而能寤,易悟也哉!」

管夷吾、鮑叔牙二人相友甚戚,同處於齊。管夷吾事公子糾,鮑叔牙事公子小白。齊公族多寵,嫡庶並行。國人懼亂。管仲與召忽奉公子糾奔魯,鮑叔奉公子小白奔莒。既而公孫无知作亂,齊无君,二公子爭入。管夷吾與小白戰於莒道,射中小白帶鉤。小白既立,脅魯殺子糾,召忽死之,管夷吾被囚。鮑叔牙謂桓公曰:「管夷吾能,可以治國。」桓公曰:『我讎也,願殺之。「鮑叔牙曰:」吾聞賢君无私怨,且人能為其主,亦必能為人君。如欲霸王,非夷吾其弗可。君必舍之!「遂召管仲。魯歸之齊,鮑叔牙郊迎,釋其囚。桓公禮之,而位於高、國之上,鮑叔牙以身下之,任以國政。號曰仲父。桓公遂霸。管仲嘗歎曰:「吾少窮困時,嘗與鮑叔1賈,分財多自與;鮑叔不以我為貪知我貧也。吾嘗為鮑叔謀事而大窮困,鮑叔不以我為愚,知時有利不利也。吾嘗三仕,三見逐於君,鮑叔不以我為不肖,知我不遭時也。吾嘗三戰三北,鮑叔不以我為怯,知我有老母也。公子糾敗,召忽死之,吾幽囚受辱;鮑叔不以我為无恥,知我不羞小節,而恥名不顯於天下也。生我者父母,知我者鮑叔也!」此世稱管、鮑善交者,小白善用能者。然實无善交,實无用能也。實无善交實无用能者,非更有善交、更有善用能也。召忽非能死,不得不死;鮑叔非能舉賢,不得不舉;小白非能用讎,不得不用。及管夷吾有病,小白問之曰:「仲父之病病矣,可不諱云,至於大病,則寡人惡乎屬國而可?」夷吾曰:「公誰欲歟?」小白曰:「鮑叔牙可。」曰:「不可。其為人潔廉善士也,其於不己若者不比之人,一聞人之過,終身不忘。使之理國,上且鉤乎君,下且逆乎民。其得罪於君也,將弗久矣。」小白曰:「然則孰可?」對曰:「勿已,則隰朋可。其為人也,上忘而下不叛,愧其不若黃帝,而哀不己若者。以德分人,謂之聖人;以財分人,謂之賢人。以賢臨人,未有得人者也;以賢下人者,未有不得人者也。其於國有不聞也,其於家有不見也。勿已,則隰朋可。」然則管夷吾非薄鮑叔也,不得不薄;非厚隰朋也,不得不厚。厚之於始,或薄之於終;薄之於終,或厚之於始。厚薄之去來,弗由我也。1. 時,嘗與鮑叔 : 原作「□□□□□」。據《道臧》補

鄧析操兩可之說,設无窮之辭,當子產執政,作《竹刑》。鄭國用之,數難子產之治。子產屈之。子產執而戮之,俄而誅之。然則子產非能用《竹刑》,不得不用;鄧析非能屈子產,不得不屈;子產非能誅鄧析,不得不誅也。

可以生而生,天福也;可以死而死,天福也。可以生而不生,天罰也;可以死而不死,天罰也。可以生,可以死,得生得死,有矣;不可以生,不可以死,或死或生,有矣。然而生生死死,非物非我,皆命也,智之所无柰何。故曰:「窈然无際,天道自會,漠然无分,天道自運。天地不能犯,聖智不能干,鬼魅不能欺。自然者,默之成之,平之寧之,將之迎之。

楊朱之友曰季梁。季梁得疾,十日大漸。其子環而泣之,請醫。季梁謂楊朱曰:「吾子不肖如此之甚,汝奚不為我歌以曉之?」楊朱歌曰:「天其弗識,人胡能覺?匪祐自天,弗孽由人。我乎汝乎!其弗知乎!醫乎巫乎!其知之乎?」其子弗曉終謁三醫。一曰矯氏,二曰俞氏,三曰盧氏,診其所疾。矯氏謂季梁曰:「汝寒溫不節,虛實失度,病由飢飽色欲。精慮煩散,非天非鬼,雖漸,可攻也。」季梁曰:「眾醫也,亟屏之!」俞氏曰:「女始則胎氣不足,乳湩有餘。病非一朝一夕之故,其所由來漸矣,弗可已也。」季梁曰:「良醫也,且食之!」盧氏曰:「汝疾不由天,亦不由人,亦不由鬼。稟生受形,既有制之者矣,亦有知之者矣,藥石其如汝何?」季梁曰:「神醫也,重貺遣之!」俄而季梁之疾自瘳。

生非貴之所能存,身非愛之所能厚;生亦非賤之所能夭,身亦非輕之所能薄。故貴之或不生,賤之或不死;愛之亦不厚,輕之或不薄。此似反也,非反也;此自生自死,自厚自薄。或貴之而生,或賤之而死;或愛之而厚,或輕之而薄。此似順也,非順也,此亦自生自死,自厚自薄。鬻熊語文王曰:「自長非所增,自短非所損。算之所亡若何?」老聃語關尹曰:「天之所惡,孰知其故?」言迎天意,揣利害,不如其已。

楊布問曰:「有人於此,年兄弟也,言兄弟也,才兄弟也,貌兄弟也;而壽夭父子也,貴賤父子也,名譽父子也,愛憎父子也。吾惑之。」楊子曰:「古之人有言,吾嘗識之,將以告若。不知所以然而然,命也。今昏昏昧昧,紛紛若若,隨所為,隨所不為。日去日來,孰能知其故?皆命也。夫信命者亡壽夭,信理者亡是非;信心者亡逆順信性者亡安危。則謂之都亡所信,亡所不信。真矣,愨矣,奚去奚就?奚哀奚樂?奚為奚不為?黃帝之書云:『至人居若死,動若械。』亦不知所以居,亦不知所以不居;亦不知所以動,亦不知所以不動。亦不以眾人之觀易其情貌,亦不謂眾人之不觀不易其情貌。獨往獨來,獨出獨入,孰能礙之?」

墨杘、單至、嘽咺、憋懯四人相與游於世,胥如志也;窮年不相知情,自以智之深也。巧佞、愚直、婩斫、便辟四人相與游於世,胥如志也;窮年而不相語術,自以巧之微也。㺒㤉、情露、謇極、凌誶四人相與游於世,胥如志也;窮年不相曉悟,自以為才之得也。眠娗、諈諉、勇敢、怯疑四人相與游於世,胥如志也;窮年不相讁發,自以行无戾也。多偶、自專、乘權、隻立四人相與游於世,胥如志也;窮年不相顧眄,自以時之適也。此眾態也。貌不一,而咸之於道,命所歸也。

佹佹成者,俏成也,初非成也。佹佹敗者,俏敗者也,初非敗也。故迷生於俏,俏之際昧然。於俏而不昧然,則不駭外禍,不喜內福;隨時動,隨時止,智不能知也。信命者,於彼我无二心。於彼我而有二心者,不若揜目塞耳,背坂面隍,亦不墜仆也。故曰:死生自命也,貧窮自時也。怨夭折者,不知命者也;怨貧窮者,不知時者也。當死不懼,在窮不戚,知命安時也。其使多智之人,量利害,料虛實,度人情,得亦中,亡亦中。其少智之人,不量利害,不料虛實,不度人情,得亦中,亡亦中。量與不量,料與不料,度與不度,奚以異?唯亡所量,亡所不量,則全而亡喪。亦非知全,亦非知喪,自全也,自亡也,自喪也。

齊景公游於牛山,北臨其國城而流涕曰:「美哉國乎!鬱鬱芊芊,若何滴滴去此國而死乎?使古无死者,寡人將去斯而之何?」史孔、梁丘據皆從而泣曰:「臣賴君之賜,跪食惡肉,可得而食,駑馬稜車,可得而乘也,且猶不欲死,而況吾君乎?」晏子獨笑於旁。公雪涕而顧晏子曰:「寡人今日之游悲。孔與據皆從寡人而泣,子之獨笑,何也?」晏子對曰:「使賢者常守之,則太公、桓公將常守之矣;使有勇者而常守之,則莊公、靈公將常守之矣。數君者將守之,吾君方將被蓑笠而立乎畎畝之中,唯事之恤,行假念死乎?則吾君又安得此位而立焉?以其迭處之,迭去之,至於君也,而獨為之流涕,是不仁也。見不仁之君,見諂諛之臣;臣見此二者,臣之所為獨竊笑也。」景公慚焉,舉觴自罰;罰二臣者,各二觴焉。

魏人有東門吳者,其子死而不憂。其相室曰:「公之愛子,天下无有。今子死不憂,何也?」東門吳曰:「吾常无子,无子之時不憂。今子死,乃與嚮无子同,臣奚憂焉?」

農赴時,商趣利,工追術,仕逐勢,勢使然也。然農有水旱,商有得失,工有成敗,仕有遇否,命使然也。


\end{pinyinscope}