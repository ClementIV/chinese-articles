\article{楊朱}

\begin{pinyinscope}
楊朱游於魯,舍於孟氏。孟氏問曰:「人而已矣,奚以名為?」曰:「以名者為富。」既富矣,奚不已焉?「曰:「為貴」。「既貴矣,奚不已焉?」曰:「為死」。「既死矣,奚為焉?」曰:「為子孫。」「名奚益於子孫?」曰:「名乃苦其身,燋其心。乘其名者澤及宗族,利兼鄉黨;況子孫乎?」「凡為名者必廉廉斯貧;為名者必讓,讓斯賤。」曰:「管仲之相齊也,君淫亦淫,君奢亦奢,志合言從,道行國霸,死之後,管氏而已。田氏之相齊也,君盈則己降,君歛則己施,民皆歸之,因有齊國;子孫享之,至今不絕。」「若實名貧,偽名富。」曰:「實無名,名無實;名者,偽而已矣。昔者堯舜偽以天下讓許由、善卷,而不失天下,享祚百年。伯夷、叔齊實以孤竹君讓,而終亡其國,餓死於首陽之山。實偽之辯,如此其省也。」

楊朱曰:「百年壽之大齊;得百年者,千无一焉。設有一者,孩抱以逮昏老,幾居其半矣。夜眠之所弭,晝覺之所遺又幾居其半矣。痛疾哀苦,亡失憂懼,又幾居其半矣。量十數年之中,逌然而自得,亡介焉之慮者,亦亡一時之中爾。則人之生也奚為哉?奚樂哉?為美厚爾,為聲色爾。而美厚復不可常猒足,聲色不可常翫聞。乃復為刑賞之所禁勸,名法之所進退;遑遑爾競一時之虛譽,規死後之餘榮;偊偊爾順耳目之觀聽,惜身意之是非;徒失當年之至樂,不能自肆於一時。重囚纍梏,何以异哉?太古之人,知生之暫來,知死之暫往,故從心而動,不違自然所好,當身之娛,非所去也,故不為名所觀。從性而游,不逆萬物所好,死後之名,非所取也,故不為刑所及。名譽先後,年命多少,非所量也。」

楊朱曰:「萬物所異者生也,所同者死也;生則有賢愚貴賤,是所異也;死則有臭腐消滅,是所同也。雖然,賢愚貴賤,非所能也;臭腐消滅,亦非所能也。故生非所生,死非所死,賢非所賢,愚非所愚,貴非所貴,賤非所賤。然而萬物齊生齊死,齊賢齊愚,齊貴齊賤。十年亦死,百年亦死,仁聖亦死凶愚亦死。生則堯舜,死則腐骨;生則桀紂,死則腐骨。腐骨一矣,孰知其異?且趣當生,奚遑死後?」

楊朱曰:「伯夷非亡欲,矜清之卸,以放餓死。展李非亡情,矜貞之卸,以放寡宗。清貞之誤善之若此。」

楊朱曰:「原憲窶於魯,子貢殖於衛。原憲之窶損生,子貢之殖累身。」「然則窶亦不可,殖亦不可,其可焉在?」曰:「可在樂生,可在逸身。故善樂生者不窶,善逸身者不殖。」

楊朱曰:「古語有之:『生相憐,死相捐。』此語至矣。相憐之道,非唯情也;勤能使逸,饑能使飽,寒能使溫,窮能使達也。相捐之道,非不相哀也;不含珠玉,不服文錦,不陳犧牲,不設明器也。」

晏平仲問養生於管夷吾。管夷吾曰:「肆之而已,勿壅勿閼。」晏平仲曰:「其目柰何?」夷吾曰:「恣耳之所欲聽,恣目之所欲視,恣鼻之所欲向,恣口之所欲言,恣體之所欲安,恣意之所欲行。夫耳之所欲聞者音聲,而不得聽,謂之閼聰;目之所欲見者美色,而不得視,謂之閼明;鼻之所欲向者椒蘭,而不得嗅,謂之閼顫;口之所欲道者是非,而不得言,謂之閼智;體之所欲安者美厚,而不得從,謂之閼適;意之所欲為者放逸,而不得行,謂之閼性。凡此諸閼,廢虐之主。去廢虐之主,熙熙然以俟死,一日一月,一年十年,吾所謂養。拘此廢虐之主,錄而不舍,戚戚然以至久生,百年千年萬年,非吾所謂養。」管夷吾曰:「吾既告子養生矣,送死柰何?」晏平仲曰:「送死略矣,將何以告焉?」管夷吾曰:「吾固欲聞之。」平仲曰:「既死,豈在我哉?焚之亦可,沈之亦可,瘞之亦可,露之亦可,衣薪而棄諸溝壑亦可,袞文繡裳而納諸石椁亦可,唯所遇焉。」管夷吾顧謂鮑叔黃子曰:「生死之道,吾二人進之矣。」

子產相鄭,專國之政三年,善者服其化,惡者畏其禁,鄭國以治。諸侯憚之。而有兄曰公孫朝,有弟曰公孫穆。朝好酒,穆好色。朝之室也,聚酒千鐘,積麴成封,望門百步,糟漿之氣逆於人鼻。方其荒於酒也,不知世道之安危,人理之悔吝,室內之有亡,九族之親踈,存亡之哀樂也。雖水火兵刃交於前,弗知也。穆之後庭,比房數十,皆擇稚齒婑媠者以盈之。方其聃於色也,屏親昵,絕交游,逃於後庭,以晝足夜;三月一出,意猶未愜。鄉有處子之娥姣者,必賄而招之,媒而挑之,弗獲而後已。子產日夜以為戚,密造鄧析而謀之曰:「喬聞治身以及家,治家以及國,此言自於近至於遠也。喬為國則治矣,而家則亂矣!其道逆邪?將奚方以救二子?子其詔之!」鄧析曰:「吾怪之久矣!未敢先言。子奚不時其治也,喻以性命之重,誘以禮義之尊乎?」子產用鄧析之言,因閒以謁其兄弟而告之曰:「人之所以貴於禽獸者智慮,智慮之所將者禮義。禮義成則名位至矣。若觸情而動,聃於嗜慾,則性命危矣。子納喬之言,則朝自悔而夕食祿矣。」朝、穆曰:「吾知之久矣,擇之亦久矣,豈待若言而後識之哉!凡生之難遇,而死之易及;以難遇之生,俟易及之死,可孰念哉?而欲尊禮義以夸人,矯情性以招名,吾以此為弗若死矣。為欲盡一生之歡,窮當年之樂,唯患腹溢而不得恣口之飲,力憊而不得肆情於色,不遑憂名聲之醜,性命之危也。且若以治國之能夸物,欲以說辭亂我之心,榮祿喜我之意,不亦鄙而可憐哉!我又欲與若別之。夫善治外者,物未必治,而身交苦;善治內者,物未必亂,而性交逸。以若之治外,其法可蹔行於一國,未合於人心;以我之治內,可推之於天下,君臣之道息矣。吾常欲以此術而喻之,若反以彼術而教我哉?」子產忙然无以應之。他日以告鄧析。鄧析曰:「子與真人居而不知也,孰謂子智者乎?鄭國之治偶耳,非子之功也。」

衛端木叔者,子貢之世也。藉其先貲,家累萬金。不治世故,放意所好。其生民之所欲為,人意之所欲玩者,无不為也,无不玩也。牆屋臺榭,園囿池沼,飲食車服,聲樂嬪御,擬齊楚之君焉。至其情所欲好,耳所欲聽,目所欲視,口所欲嘗,雖殊方偏國,非齊土之所產育者,无不必致之,猶藩牆之物也。及其游也,雖山川阻險,塗逕脩遠,无不必之,猶人之行咫步也。賓客在庭者日百住,庖廚之下,不絕煙火;堂廡之上,不絕聲樂。奉養之餘,先散之宗族;宗族之餘,次散之邑里;邑里之餘,乃散之一國。行年六十,氣幹將衰,棄其家事,都散其庫藏、珍寶、車服、妾媵,一年之中盡焉,不為子孫留財。及其病也,无藥石之儲;及其死也;无瘞埋之資。一國之人,受其施者,相與賦而藏之,反其子孫之財焉。禽骨釐聞之曰:「端木叔狂人也,辱其祖矣。」段干生聞之曰:「木叔達人也,德過其祖矣。其所行也,其所為也,聚意所經,而誠理所取。衛之君子多以禮教自持,固未足以得此人之心也。」

孟孫陽問楊子曰:「有人於此,貴生愛身,以蘄不死,可乎?」曰:「理无不死。」「以蘄久生,可乎?」曰:「理无久生。生非貴之所能存,身非愛之所能厚。且久生奚為?五情好惡,古猶今也;四體安危,古猶今也;世事苦樂,古猶今也;變易治亂,古猶今也。既聞之矣,既見之矣,既更之矣,百年猶厭其多,況久生之苦也乎?」孟孫陽曰:『若然,速亡愈於久生;則踐鋒刃,入湯火,得所志矣。「楊子曰:「不然。既生,則廢而任之,究其所欲,以俟於死。將死則廢而任之,究其所之,以放於盡。无不廢,无不任,何遽遟速於其閒乎?」

楊朱曰:「伯成子高不以一毫利物,舍國而隱耕。大禹不以一身自利,一體偏枯。古之人,損一毫利天下,不與也,悉天下奉一身,不取也。人人不損一毫,人人不利天下,天下治矣。」禽子問楊朱曰:「去子體之一毛,以濟一世,汝為之乎?」楊子曰:「世固非一毛之所濟。」禽子曰:「假濟,為之乎?」楊子弗應。禽子出,語孟孫陽。孟孫陽曰:「子不達夫子之心,吾請言之。有侵若肌膚獲萬金者,若為之乎?」曰:「為之。」孟孫陽曰:「有斷若一節得一國。子為之乎?」禽子默然有閒。孟孫陽曰:「一毛微於肌膚,肌膚微於一節,省矣。然則積一毛以成肌膚,積肌膚以成一節。一毛固一體萬分中之一物,柰何輕之乎?」禽子曰:「吾不能所以荅子。然則以子之言問老聃、關尹,則子言當矣;以吾言問大禹、墨翟,則吾言當矣。」孟孫陽因顧與其徙說他事。

楊朱曰:「天下之美歸之舜、禹、周、孔,天下之惡歸之桀、紂。然而舜耕於河陽,陶於雷澤,四體不得蹔安,口腹不得美厚;父母之所不愛,弟妹之所不親。行年三十,不告而娶。及受堯之禪,年已長,智已衰。商鈞不才,禪位於禹,戚戚然以至於死:此天人窮毒者也。鯀治水土,績用不就,殛諸羽山。禹纂業事讎,惟荒土功,子產不字,過門不入;身體偏枯,手足胼胝。及受舜禪,卑宮室,美紱冕,戚戚然以至於死:此天人之憂苦者也。武王既終,成王幼弱,周公攝天子之政。邵公不悅,四國流言。居東三年,誅兄放弟,僅免其身,戚戚然以至於死:此天人之危懼者也。孔子明帝王之道,應時君之聘,伐樹於宋,削迹於衛,窮於商周,圍於陳蔡,受屈於季氏,見辱於陽虎,戚戚然以至於死:此天民之遑遽者也。凡彼四聖者,生无一日之歡,死有萬世之名。名者,固非實之所取也。雖稱之弗知,雖賞之不知,與株塊无以異矣。桀藉累世之資,居南面之尊,智足以距群下,威足以震海內;恣耳目之所娛,窮意慮之所為,熙熙然以至於死:此天民之逸蕩者也。紂亦藉累世之資,居南面之尊;威无不行,志无不從;肆情於傾宮,縱欲於長夜;不以禮義自苦,熙熙然以至於誅:此天民之放縱者也。彼二凶也,生有從欲之歡,死被愚暴之名。實者固非名之所與也,雖毀之不知,雖稱之弗知,此與株塊奚以異矣。彼四聖雖美之所歸,苦以至終,同歸於死矣。彼二凶雖惡之所歸,樂以至終,亦同歸於死矣。」

楊朱見梁王,言治天下如運諸掌。梁王曰:「先生有一妻一妾,而不能治;三畝之園,而不能芸,而言治天下如運諸掌,何也?」對曰:「君見其牧羊者乎?百羊而群,使五尺童子荷箠而隨之,欲東而東,欲西而西。使堯牽一羊,舜荷箠而隨之,則不能前矣。且臣聞之:吞舟之魚,不游枝流;鴻鵠高飛,不集汙池。何則?其極遠也。黃鐘大呂,不可從煩奏之舞,何則?其音䟽也。將治大者不治細,成大功者不成小,此之謂矣。」

楊朱曰:「太古之事滅矣,孰誌之哉?三皇之事,若存若亡;五帝之事,若覺若夢;三王之事,或隱或顯,億不識一。當身之事,或聞或見,萬不識一。目前之事或存或廢,千不識一。太古至于今日,年數固不可勝紀。但伏羲已來三十餘萬歲,賢愚、好醜、成敗、是非,无不消滅,但遟速之閒耳。矜一時之毀譽,以焦苦其神形,要死後數百年中餘名,豈足潤枯骨?何生之樂哉?」

楊朱曰:「人肖天地之類,懷五常之性,有生之最靈者人也。人者,爪牙不足以供守衛,肌膚不足以自捍禦,趨走不足以逃利害,无毛羽以禦寒暑,必將資物以為養,性任智而不恃力。故智之所貴,存我為貴;力之所賤,侵物為賤。然身非我有也,既生不得不全之;物非我有也,既有不得而1去之。身固生之主,物亦養之主。雖全生身,不可有其身;雖不去物,不可有其物。有其物有其身,是橫私天下之身,橫私天下之物不橫私天下之身,不橫私天下物者,2。其唯聖人乎!公天下之身,公天下之物,其唯至人矣!此之謂至至者也。」1. 而 : 原作「不」。2. 不橫私天下之身,不橫私天下物者, : 舊脫。 據《莊子集釋》:「各本無此十四字,今從敦煌殘卷增。」

楊朱曰:「生民之不得休息,為四事故:一為壽,二為名,三為位,四為貨。有此四者,畏鬼,畏人,畏威,畏刑,此謂之遁人也。可殺可活,制命在外。不逆命,何羨壽?不矜貴,何羨名?不要勢,何羨位?不貪富,何羨貨?此之謂順民也。天下无對,制命在內,故語有之曰:人不婚宦,情欲失半;人不衣食,君臣道息。周諺曰:「田父可坐殺。晨出夜入,自以性之恆;啜菽茹藿,自以味之極;肌肉麤厚,筋節腃急,一朝處以柔毛綈幕,薦以粱肉蘭橘,心㾓體煩,內熱生病矣。商魯之君與田父侔地,則亦不盈一時而憊矣。故野人之所安,野人之所美,謂天下无過者。昔者宋國有田夫,常衣縕黂,僅以過冬。暨春東作,自曝於日,不知天下之有廣廈隩室,綿纊狐狢。顧謂其妻曰:『負日之煊,人莫知者;以獻吾君,將有重賞。』里之富室告之曰:『昔人有美戎菽,甘枲莖芹萍子者,對鄉豪稱之。鄉豪取而嘗之,蜇於口,慘於腹,眾哂而怨之,其人大慚。子此類也。』」

楊朱曰:「豐屋美服,厚味姣色,有此四者,何求於外?有此而求外者,无猒之性。无猒之性,陰陽之蠹也。忠不足以安君,適足以危身;義不足以利物,適足以害生。安上不由於忠,而忠名滅焉;利物不由於義,而義名絕焉。君臣皆安,物我兼利,古之道也。鬻子曰:『去名者无憂。』老子曰:『名者實之賓。』而悠悠者趨名不已。名固不可去?名固不可賓邪?今有名則尊榮,亡名則卑辱;尊榮則逸樂,卑辱則憂苦。憂苦,犯性者也;逸樂,順性者也,斯實之所係矣。名胡可去?名胡可賓?但惡夫守名而累實。守名而累實,將恤危亡之不救,豈徒逸樂憂苦之閒哉?」


\end{pinyinscope}