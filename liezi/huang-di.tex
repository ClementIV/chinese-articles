\article{黃帝}

\begin{pinyinscope}
黃帝即位十有五年,喜天下戴己,養正命,娛耳目,供鼻口,焦然肌色皯黣,昏然五情爽惑。又十有五年,憂天下之不治,竭聰明,進智力,營百姓,焦然肌色皯黣,昏然五情爽惑。黃帝乃喟然讚曰:「朕之過淫矣。養一己其患如此,治萬物其患如此。」於是放萬機,舍宮寢,去直侍,徹鐘懸。減廚膳,退而間居大庭之館,齋心服形,三月不親政事。晝寢而夢,遊於華胥氏之國。華胥氏之國在弇州之西,台州之北,不知斯齊國幾千萬里;蓋非舟車足力之所及,神游而已。其國无師長,自然而已。其民无嗜慾,自然而已。不知樂生,不知惡死,故无夭殤;不知親己,不知踈物,故无愛憎;不知背逆,不知向順,故无利害;都无所愛惜,都无所畏忌。入水不溺,入火不熱。斫撻无傷痛,指擿无痟癢。乘空如履實,寢虛若處床。雲霧不硋其視,雷霆不亂其聽,美惡不滑其心,山谷不躓其步,神行而已。黃帝既寤,怡然自得,召天老、力牧、太山稽,告之曰:「朕閒居三月,齋心服形,思有以養身治物之道,弗獲其術。疲而睡,所夢若此。今知至道不可以情求矣。朕知之矣!朕得之矣!而不能以告若矣。」又二十有八年,天下大治,幾若華胥氏之國,而帝登假,百姓號之,二百餘年不輟。

列姑射山在海河洲中,山上有神人焉,吸風飲露,不食五穀;心如淵泉,形如處女,不偎不愛,仙聖為之臣;不畏不怒,愿愨為之使;不施不惠,而物自足;不聚不歛,而己无愆。陰陽常調,日月常明,四時常若,風雨常均,字育常時,年穀常豐;而土无札傷,人无夭惡,物无疵厲,鬼无靈響焉。

列子師老商氏,友伯高子;進二子之道,乘風而歸。尹生聞之,從列子居,數月不省舍。因間請蘄其術者,十反而十不告。尹生懟而請辭,列子又不命。尹生退。數月,意不已,又往從之。列子曰:「汝何去來之頻?」尹生曰:「曩章戴有請於子,子不我告,固有憾於子。今復脫然,是以又來。」列子曰:「曩吾以汝為達,今汝之鄙至此乎。姬!將告汝所學於夫子者矣。自吾之事夫子友若人也,三年之後,心不敢念是非,口不敢言利害,始得夫子一眄而已。五年之後,心庚念是非,口庚言利害,夫子始一解顏而笑。七年之後,從心之所念,庚无是非;從口之所言,庚无利害,夫子始一引吾並席而坐。九年之後,橫心之所念,橫口之所言,亦不知我之是非利害歟,亦不知彼之是非利害歟;亦不知夫子之為我師,若人之為我友:內外進矣。而後眼如耳,耳如鼻,鼻如口,无不同也。心凝形釋,骨肉都融;不覺形之所倚,足之所履,隨風東西,猶木葉幹殼。竟不知風乘我邪?我乘風乎?今女居先生之門,曾未浹時,而懟憾者再三。女之片體將氣所不受,汝之一節將地所不載。履虛乘風,其可幾乎?」尹生甚怍,屏息良久,不敢復言。

列子問關尹曰:「至人潛行不空,蹈火不熱,行乎萬物之上而不慄。請問何以至於此?」關尹曰:「是純氣之守也,非智巧果敢之列。姬!魚語汝。凡有貌像聲色者,皆物也。物與物何以相遠也?天奚足以至乎先?是色而已。則物之造乎不形,而止乎无所化。夫得是而窮之者,焉得為正焉?彼將處乎不深之度,而藏乎无端之紀,游乎萬物之所終始。壹其性,養其氣,含其德,以通乎物之所造。夫若是者,其天守全,其神无郤,物奚自入焉?夫醉者之墜於車也,雖疾不死。骨節與人同,而犯害與人異,其神全也。乘亦弗知也,墜亦弗知也。死生驚懼,不入乎其胸,是故遌物而不慴。彼得全於酒,而猶若是,而況得全於天乎?聖人藏於天,故物莫之能傷也。」

列禦寇為伯昏无人射,引之盈貫,措杯水其肘上,發之,鏑矢復沓,方矢復寓。當是時也,猶象人也。伯昏无人曰:「是射之射,非不射之射也。當與汝登高山,履危石,臨百仞之淵,若能射乎?」於是无人遂登高山,履危石,臨百仞之淵,背逡巡,足二分垂在外,揖禦寇而進之。禦寇伏地,汗流至踵。伯昏无人曰:「夫至人者,上闚青天,下潛黃泉,揮斥八極。神氣不變。今汝怵然有恂目之志,爾於中也殆矣夫!」

范氏有子曰子華,善養私名,舉國服之;有寵於晉君,不仕而居三卿之右。目所偏視,晉國爵之;口所偏肥,晉國黜之。游其庭者侔於朝。子華使其俠客,以智鄙相攻,彊弱相凌。雖傷破於前,不用介意。終日夜以此為戲樂,國殆成俗。禾生、子伯、范氏之上客。出行經坰外,宿於田更商丘開之舍。中夜,禾生、子伯二人相與言子華之名勢,能使存者亡,亡者存;富者貧,貧者富。商丘開先窘於飢寒,潛於牖北聽之。因假糧荷畚之子華之門。子華之門徒皆世族也,縞衣乘軒,緩步闊視。顧見商丘開年老力弱,面目黎黑,衣冠不檢,莫不眲之。既而狎侮欺詒,攩㧙挨抌,二所不為。商丘開常无慍容,而諸客之技單,憊於戲笑。遂與商丘開俱乘高臺,於眾中漫言曰:「有能自投下者賞百金。」眾皆競應。商丘開以為信然,遂先投下,形若飛鳥,揚於地,肌骨无毀。范氏之黨以為偶然,末詎怪也。因復指河曲之淫隅曰:「彼中有寶珠,泳可得也。」商丘開復從而泳之,既出,果得珠焉。眾昉同疑。子華昉令豫肉食衣帛之次。俄而范氏之藏大火。子華曰:「若能入火取錦者,從所得多少賞若。」商丘開往,无難色,大火往還,埃不漫,身不焦。范氏之黨以為有道,乃共謝之曰:「吾不知子之有道而誕子,吾不知子之神人而辱子。子其愚我也,子其聾我也,子其盲我也,敢問其道。」商丘開曰:『吾亡道。雖吾之心,亦不知所以。雖然,有一於此,試與子言之。曩子二客之宿吾舍也,聞譽范氏之勢,能使存者亡,亡者存;富者貧,貧者富。吾誠之无二心,故不遠而來。及來,以子黨之言皆實也,唯恐誠之之不至,行之之不及,不知形體之所措,利害之所存也。心一而已。物亡迕者,如斯而已。今昉知子黨之誕我,我內藏猜慮,外矜觀聽,追幸昔日之不焦溺也,怛然內熱,惕然震悸矣。水火豈復可近哉?」自此之後,范氏門徒路遇乞兒馬醫,弗敢辱也,必下車而揖之。宰我聞之,以告仲尼。仲尼曰:「汝弗知乎?夫至信之人,可以感物也。動天地,感鬼神,橫六合而无逆者,豈但履危險,入水火而已哉?商丘開信偽物猶不逆,況彼我皆誠哉?小子識之!」

周宣王之牧正,有役人梁鴦者,能養野禽獸,委食於園庭之內,雖虎狼鵰鶚之類,无不柔者。雄雌在前,孳尾成群,異類雜居,不相搏噬也。王慮其術終於其身,令毛丘園傳之。梁鴦曰:「鴦,賤役也,何術以告爾?懼王之謂隱於爾也,旦一言我養虎之法。凡順之則喜,逆之則怒,此有血氣者之性也。然喜怒豈妄發哉?皆逆之所犯也。夫食虎者,不敢以生物與之,為其殺之之怒也;不敢以全物與之,為其碎之之怒也。時其饑飽,達其怒心。虎之與人異類,而媚養己者,順也;故其殺之,逆也。然則吾豈敢逆之使怒哉?亦不順之使喜也。夫喜之復也必怒,怒之復也常喜,皆不中也。今吾心无逆順者也,則鳥獸之視吾,猶其儕也。故游吾園者,不思高林曠澤;寢吾庭者,不願深山幽谷,理使然也。」

顏回問乎仲尼曰:「吾嘗濟乎觴深之淵矣,津人操舟若神。吾問焉,曰:『操舟可學邪?』曰:『可。能游者可教也,善游者數能。乃若夫沒人,則未嘗見舟而謖操之者也。』吾問焉而不告。敢問何謂也?」仲尼曰:『𧮒!吾與若玩其文也久矣,而未達其實,而固且道與。能游者可教也,輕水也;善游者之散能也,忘水也。乃若夫沒人之未嘗見舟也而謖操之也,彼視淵者陵,視舟之覆猶其車卻也。覆卻萬物方陳乎前,而不得入其舍,惡往而不暇?以瓦摳者巧,以鉤摳者憚,以黃金摳者惛。巧一也,而有所矜,則重外也。凡重外者撰內。「

孔子觀於呂梁,懸水三十仞,流沫三十里,黿鼉魚鱉之所不能游也。見一丈夫游之,以為有苦而欲死者也,使弟子益流而承之。數百步而出,被髮行歌,而游於棠行。孔子從而問之曰:「呂梁懸水三十仞,流沫三十里,黿鼉魚鱉所不能游,向吾見子道之,以為有苦而欲死者,使弟子並流將承子。子出而被髮行歌,吾以子為鬼也。察子則人也。請問蹈水有道乎?」曰:「亡,吾无道。吾始乎故,長乎性,成乎命,與齎俱入,與汩偕出,從水之道而不為私焉。此吾所以道之也。」孔子曰:「何謂始乎故,長乎性,成乎命也?」曰:「吾生於陵而安於陵,故也;長於水而安於水,性也;不知吾所以然而然,命也。」

仲尼適楚,出於林中,見痀僂者承蜩,猶掇之也。仲尼曰:「子巧乎!有道邪?」曰:「我有道也。五六月纍垸二而不墜,則失者錙銖;纍三而不墜,則失者十一;纍五而不墜,猶掇之也。吾處也,若橛株駒,吾執臂若槁木之枝。天地之大、萬物之多,而唯蜩翼之知。吾不反側,不以萬物易蜩之翼,何為而不得?」孔子顧謂弟子曰:「用志不分,乃疑於神。其痀僂丈人之謂乎!」丈人曰:「汝逢衣徒也,亦何知問是乎?脩汝所以,而後載言其上。」

海上之人有好漚鳥者,每旦之海上,從漚鳥游,漚鳥之至者百住而不止。其父曰:「吾聞漚鳥皆從汝游,汝取來,吾玩之。」明日之海上,漚鳥舞而不下也。故曰:至言去言,至為無為;齊智之所知,則淺矣。

趙襄子率徒十萬,狩於中山,藉芿燔林,扇赫百里,有一人從石壁中出,隨煙燼上下,眾謂鬼物。火過,徐行而出,若無所經涉者。襄子怪而留之,徐而察之:形色七竅,人也;氣息音聲,人也。問奚道而處石?奚道而入火?其人曰:「奚物而謂石?奚物而謂火?」襄子曰:「而嚮之所出者,石也;而嚮之所涉者,火也。」其人曰:「不知也。」魏文侯聞之,問子夏曰:「彼何人哉?」子夏曰:「以商所聞夫子之言,和者大同於物,物無得傷閡者,游金石,蹈水火,皆可也。」文侯曰:「吾子奚不為之?」子夏曰:「刳心去智,商未之能。雖然,試語之有暇矣。」文侯曰:「夫子奚不為之?」子夏曰:「夫子能之而能不為者也。」文侯大說。

有神巫自齊來處於鄭,命曰季咸,知人死生、存亡、禍福、壽夭,期以歲、月、旬、日,如神。鄭人見之,皆避而走。列子見之而心醉,而歸以告壺丘子,曰:「始吾以夫子之道為至矣,則又有至焉者矣。」壺子曰:「吾與汝無其文,未既其實,而固得道與?眾雌而无雄,而又奚卵焉?而以道與世抗,必信矣。夫故使人得而相汝。嘗試與來,以予示之。」明日,列子與之見壺子。出而謂列子曰:「譆!子之先生死矣,弗活矣,不可以旬數矣。吾見怪焉,見濕灰焉。」列子入,涕泣沾衾,以告壺子。子曰:「向吾示之以地文,罪乎不誫不止,是殆見吾杜德幾也。嘗又與來!」明日,又與之見壺子。出而謂列子曰:「幸矣,子之先生遇我也,有瘳矣。灰然有生矣,吾見杜權矣。」列子入告壺子。壺子曰:「向吾示之以天壤,名實不入,而機發於踵,此為杜權。是殆見吾善者幾也。嘗又與來!」明日,又與之見壺子。出而謂列子曰:「子之先生,坐不齋,吾无得而相焉。試齋,將且復相之。」列子入告壺子。壺子曰:「向吾示之以太沖莫眹,是殆見吾衡氣幾也。鯢旋之潘為淵,止水之潘為淵,流水之潘為淵,濫水之潘為淵,沃水之潘為淵,氿水之潘為淵,雍水之潘為淵,汧水之潘為淵,肥水之潘為淵,是為九淵焉,嘗又與來!」明日,又與之見壺子。立末定,自失而走。壺子曰:「追之!」列子追之而不及,反以報壺子,曰:「已滅矣,已夫矣,吾不及也。「壺子曰:」向吾示之以未始出吾宗。吾與之虛而猗移,不知其誰何,因以為茅靡,因以為波流,故逃也。」然後列子自以為未始學而歸,三年不出,為其妻爨,食狶如食人,於事无親,雕瑑復朴,塊然獨以其形立,㤋然而封戎,壹以是終。

子列子之齊,中道而反,遇伯昏瞀人。伯昏瞀人曰:「奚方而反?」曰:「吾驚焉。」「惡乎驚?」「吾食於十漿,而五漿先饋。」伯昏瞀人曰:「若是則汝何為驚己?」曰:「夫內誠不解,形諜成光,以外鎮人心,使人輕乎貴老,而𩐋其所患。夫漿人特為食羹之貨,多餘之嬴;其為利也薄,其為權也輕,而猶若是。而況萬乘之主,身勞於國,而智盡於事;彼將任我以事,而效我以功,吾是以驚。」伯昏瞀人曰:「善哉觀乎!汝處己,人將保汝矣。」无幾何而往,則戶外之屨滿矣。伯昏瞀人北面而立,敦杖蹙之乎頤。立有閒,不言而出。賓者以告列子。列子提履徒跣而走,暨乎門,問曰:「先生既來,曾不廢藥乎?」曰:「已矣。吾固告汝曰,人將保汝,果保汝矣。非汝能使人保汝,而汝不能使人无汝保也,而焉用之感也?感豫出異。且必有感也,搖而本身,又无謂也。與汝遊者,莫汝告也。彼所小言,盡人毒也。莫覺莫悟,何相孰也。」

楊朱南之沛,老聃西遊於秦。邀於郊。至梁而遇老子。老子中道仰天而歎曰:「始以汝為可教,今不可教也。」楊子不荅。至舍,進涫漱巾櫛,脫履戶外,膝行而前曰:「向者夫子仰天而歎曰:『始以汝為可教,今不可教。』弟子欲請夫子辭,行不閒,是以不敢。今夫子閒矣,請問其過。」老子曰:「而睢睢,而盱盱,而誰與居?大白若辱,盛德若不足。」楊朱蹴然變容曰:「敬聞命矣!」其往也,舍迎將家,公執席,妻執巾櫛,舍者避席,煬者避竈。其反也,舍者與之爭席矣。

楊朱過宋東之於逆旅。逆旅人有妾二人,其一人美,其一人惡;惡者貴而美者賤。楊子問其故。逆旅小子對曰:「其美者自美,吾不知其美也;其惡者自惡,吾不知其惡也。」楊子曰:「弟子記之!行賢而去自賢之行,安往而不愛哉!」

天下有常勝之道,有不常勝之道。常勝之道曰柔,常不勝之道曰彊。二者亦知,而人未之知。故上古之言:彊,先不己若者;柔先出於己者。先不己若者,至於若己,則殆矣。先出於己者,亡所殆矣。以此勝一身若徒,以此任天下若徒,謂不勝而自勝,不任而自任也。粥子曰:「欲剛,必以柔守之;欲彊,必以弱保之。積於柔必剛,積於弱必彊。觀其所積,以知禍福之鄉。彊勝不若己,至於若己者剛;柔勝出於己者,其力不可量。」老聃曰:「兵彊則滅。木彊則折。柔弱者生之徒,堅彊者化之徒。」

狀不必童,而智童;智不必童,而狀童。聖人取童智而遺童狀,眾人近童狀而䟽童智。狀與我童者,近而愛之;狀與我異者,䟽而畏之。有七尺之骸,手足之異,戴髮含齒,倚而趣者,謂之人。而人未必无獸心;雖有獸心,以狀而見親矣。傅翼戴角,分牙布爪,仰飛伏走,謂之禽獸。而禽獸未必无人心;雖有人心,以狀而見䟽矣。庖犧氏、女媧氏、神農氏、夏后氏,蛇身人面,牛首虎鼻;此有非人之狀,而有大聖之德。夏桀、殷紂、魯桓、楚穆,狀貌七竅,皆同於人,而有禽獸之心。而眾人守一狀以求至智,未可幾也。黃帝與炎帝戰於阪泉之野,帥熊、羆、狼、豹、貙、虎為前驅,鵰、鶡、鷹、鳶為旗幟,此以力使禽獸者也。堯使夔典樂,擊石拊石,百獸率舞;簫韶九成,鳳皇來儀:此以聲致禽獸者也。然則禽獸之心,奚為異人?形音與人異,而不知接之之道焉。聖人无所不知1,无所不通,故得引而使之焉。禽獸之智有自然與人童2者,其齊欲攝生,亦不假智於人也。牝牡相偶,母子相親,避平依險,違寒就溫;居則有群,行則有列;小者居內,壯者居外;飲則相攜,食則鳴群。太古之時,則與人同處,與人並行。帝王之時,始驚駭散亂矣。逮於末世,隱伏逃竄,以避患害。今東方介氏之國,其國人數數解六畜之語者,蓋偏知之所得。太古神聖之人,備知萬物情態,悉解異類音聲。會而聚之,訓而受之,同於人民。故先會劉神魑魅,次達八方人民,末聚禽獸蟲蛾。言血氣之類,心智不殊遠也。神聖知其如此,故其所教訓者无所遺逸焉。1. 知 : 原作「□」。底本空一字,據《正統道臧》本補。2. 童 : 原作「□」。底本空一字,據《正統道臧》本補。

宋有狙公者,愛狙,養之成群,能解狙之意;狙亦得公之心。損其家口,充狙之欲。俄而匱焉,將限其食。恐眾狙之不馴於己也,先誑之曰:「與若茅,朝三而暮四,足乎?」眾狙皆起而怒。俄而曰:「與若茅,朝四而暮三,足乎?」眾狙皆伏而喜。物之以能鄙相籠,皆猶此也。聖人以智籠群愚,亦猶狙公之以智籠眾狙也。若實不虧,使其喜怒哉!

紀渻子為周宣王養鬭鷄,十日而問:「鷄可鬭已乎?「曰:「未也,方虛驕而恃氣。」十日又問。曰:「未也,猶應影嚮。」十日又問。「未也,猶疾視而盛氣。」十日又問。曰:「幾矣。鷄雖有鳴者,已无變矣。望之似木鷄矣,其德全矣。異鷄无敢應者,反走耳。」

惠盎見宋康王。康王蹀足謦欬,疾言曰:「寡人之所說者,勇有力也,不說為仁義者也。客將何以教寡人?」惠盎對曰:「臣有道於此,使人雖勇,刺之不入;雖有力,擊之弗中。大王獨无意邪?」宋王曰:「善,此寡人之所欲聞也。」惠盎曰:「夫刺之不入,擊之不中,此猶辱也。臣有道於此,使人雖有勇弗敢刺;雖有力弗敢擊。夫弗敢,非无其志也。臣有道於此,使人本无其志也。夫无其志也,未有愛利之心也。臣有道於此,使天下丈夫女子,莫不驩然皆欲愛利之。此其賢於勇有力也,四累之上也。大王獨无意邪?」宋王曰:「此寡人之所欲得也。」惠盎對曰:「孔、墨是已。孔丘、墨翟,无地而為君,無官而為長;天下丈夫女子,莫不延頸舉踵而願安利之。今大王,萬乘之主也,誠有其志,則四境之內,皆得其利矣。其賢於孔、墨也遠矣。」宋王無以應。惠盎趨而出。宋王謂左右曰:「辯矣,客之以說服寡人也!」


\end{pinyinscope}