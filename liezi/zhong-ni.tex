\article{仲尼}

\begin{pinyinscope}
仲尼閒居,子貢入侍,而有憂色。子貢不敢問,出告顏回。顏回援琴而歌。孔子聞之,果召回入,問曰:「若奚獨樂?」回曰:「夫子奚獨憂?」孔子曰:「先言爾志。」曰:「吾昔聞之夫子曰:『樂天知命故不憂』,回所以樂也。「孔子愀然有閒曰:」有是言哉?汝之意失矣。此吾昔日之言爾,請以今言為正也。汝徒知樂天知命之无憂,未知樂天知命有憂之大也。今告若其實。脩一身,任窮達,知去來之非我,止變亂於心慮,爾之所謂樂天知命之无憂也。曩吾脩《詩》《書》,正禮樂,將以治天下,遺來世;非但脩一身治魯國而已。而魯之君臣日失其序,仁義益衰,情性益薄。此道不行一國與當年,其如天下與來世矣?吾始知《詩》《書》禮樂无救於治亂,而未知所以革之之方:此樂天知命者之所憂。雖然,吾得之矣。夫樂而知者,非古人之所謂1樂知也。无樂无知,是真樂真知;故无所不樂,无所不知,无所不憂,无所不為。《詩》《書》禮樂,何棄之有?革之何為?「顏回北面拜手曰:「回亦得之矣。」出告子貢。子貢茫然自失,歸家淫思七日,不寢不食,以至骨立。顏回重往喻之,乃反丘門,絃歌誦書,終身不輟。1. 所謂 : 原作「謂所」。(根據CHANT)楊伯峻列子集釋116

陳大夫聘魯,私見叔孫氏。叔孫氏曰:「吾國有聖人。」曰:「非孔丘邪?」曰:「是也。」「何以知其聖乎?」叔孫氏曰:「吾常聞之顏回,曰:『孔丘能廢心而用形。』」陳大夫曰:「吾國亦有聖人,子弗知乎?」曰:「聖人孰謂?」曰:「老聃之弟子,有亢倉子者,得聃之道,能以耳視而目聽。」魯侯聞之大驚,使上卿厚禮而致之。亢倉子應聘而至。魯侯卑辭請問之。亢倉子曰:「傳之者妄。我能視聽不用耳目,不能易耳目之用。」魯侯曰:「此增異矣。其道奈何?寡人終願聞之。」亢倉子曰:「我體合於心,心合於氣,氣合於神,神合於无。其有介然之有,唯然之音,雖遠在八荒之外,近在眉睫之內,來干我者,我必知之。乃不知是我七孔四支之所覺,心腹六藏之所知,其自知而已矣。」魯侯大悅。他日以告仲尼,仲尼笑而不荅。

商太宰見孔子曰:「丘聖者歟?」孔子曰:「聖則丘何敢,然則丘博學多識者也。」商太宰曰:「三王聖者歟?」孔子曰:「三王善任智勇者,聖則丘不知。」曰:「五帝聖者歟?」孔子曰:「五帝善任仁義者,聖則丘弗知。」曰:「三皇聖者歟?」孔子曰:「三皇善任因時者,聖則丘弗知。」商太宰大駭,曰:「然則孰者為聖?」孔子動容有閒,曰:「西方之人,有聖者焉,不治而不亂,不言而自信,不化而自行,蕩蕩乎民无能名焉。丘疑其為聖。弗知真為聖歟?真不聖歟?」商太宰嘿然心計曰:「孔丘欺我哉!」

子夏問孔子曰:「顏回之為人奚若?」子曰:「回之仁賢於丘也。」曰:「子貢之為人奚若?」子曰:「賜之辯賢於丘也。」曰:「子路之為人奚若?」子曰:「由之勇賢於丘也。」曰:「子張之為人奚若?」子曰:「師之莊賢於丘也。」子夏避席而問曰:「然則四子者何為事天子?」曰:「居!吾語汝。夫回能仁而不能反。賜能辯而不能訥,由能勇而不能怯,師能莊而不能同。兼四子之有以易吾,吾弗許也,此其所以事吾而不貳也。」

子列子既師壺丘子林,友伯昏瞀人,乃居南郭。從之處者,日數而不及。雖然,子列子亦微焉,朝朝相與辯,无不聞。而與南郭子連牆二十年,不相謁請;相遇於道,目若不相見者。門之徒役,以為子列子與南郭子有敵不疑。有自楚來者,問子列子曰:「先生與南郭子奚敵?」子列子曰:「南郭子貌充心虛,耳无聞,目无見,口无言,心无知,形无惕。往將奚為?雖然,試與汝偕往。」閱弟子四十人同行。見南郭子,果若欺魄焉而不可與接。顧視子列子,形神不相偶,而不可與群。南郭子俄而指子列子之弟子末行者與言,衎衎然若專直而在雄者。子列子之徒駭之。反舍咸有疑色。子列子曰:「得意者无言,進知者亦无言。用无言為言亦言,无知為知亦知。无言與不言,无知與不知,亦言亦知。亦无所不言,亦无所不知;亦无所言,亦无所知。如斯而已。汝奚妄駭哉?」

子列子學也,三年之後,心不敢念是非,口不敢言利害,始得老商一眄而已。五年之後,心更念是非,口更言利害,老商始一解顏而笑。七年之後,從心之所念,更无是非;從口之所言,更无利害。夫子始一引吾並席而坐。九年之後,橫心之所念,橫口之所言,亦不知我之是非利害歟,亦不知彼之是非利害歟,外內進矣。而後眼如耳,耳如鼻,鼻如口,口无不同。心凝形釋骨肉都融;不覺形之所倚,足之所履,心之所念,言之所藏。如斯而已。則理无所隱矣。

初子列子好游。壺丘子曰:「禦寇好游,游何所好?」列子曰:「游之樂,所玩无故。人之游也,觀其所見;我之游也,觀其所變。游乎游乎!未有能辨其游者。」壺丘子曰:「禦寇之游固與人同歟,而曰固與人異歟?凡所見,亦恆見其變。玩彼物之无故,不知我亦无故。務外游,不知務內觀。外游者,求備於物;內觀者,取足於身。取足於身,游之至也;求備於物,游之不至也。」於是列子終身不出,自以為不知游。壺丘子曰:「游其至乎!至游者不知所適;至觀者不知所眡,物物皆游矣,物物皆觀矣,是我之所謂游,是我之所謂觀也。故曰:游其至矣乎!游其至矣乎!」

龍叔謂文摯曰:「子之術微矣。吾有疾,子能已乎?」文摯曰:「唯命所聽。然先言子所病之證。」龍叔曰:「吾鄉譽不以為榮,國毀不以為辱;得而不喜,失而弗憂;視生如死,視富如貧,視人如豕,視吾如人。處吾之家,如逆旅之舍;觀吾之鄉,如戎蠻之國。凡此眾庶,爵賞不能勸,刑罰不能威,盛衰利害不能易,哀樂不能移。固不可事國君,交親友,御妻子,制僕隸。此奚疾哉?奚方能已之乎?」文摯乃命龍叔背明而立。文摯自後向明而望之,既而曰:「嘻!吾見子之心矣,方寸之地虛矣,幾聖人也!子心六孔流通,一孔不達。今以聖智為疾者,或由此乎!非吾淺術所能已也。」

无所由而常生者道也。由生而生,故雖終而不亡,常也。由生而亡,不幸也。有所由而常死者,亦道也。由死而死,故雖未終而自亡者,亦常。由死而生,幸也。故无用而生謂之道,用道得終謂之常;有所用而死者亦謂之道,用道而得死者亦謂之常。

季梁之死,楊朱望其門而歌。隨梧之死,楊朱撫其尸而哭。隸人之生,隸人之死,眾人且歌,眾人且哭。

目將眇者先睹秋毫;耳將聾者先聞蚋飛;口將爽者先辨淄澠;鼻將窒者先覺焦朽;體將僵者先亟犇佚;心將迷者先識是非:故物不至者則不反。

鄭之圃澤多賢,東里多才。圃澤之役有伯豐子者,行過東里,遇鄧析。鄧析顧其徒而笑曰:「為若舞彼來者奚若?」其徒曰:「所願知也。」鄧析謂伯豐子曰:「汝知養養之義乎?受人養而不能自養者,犬豕之類也;養物而物為我用者,人之力也。使汝之徒,食而飽,衣而息,執政之功也。長幼群聚,而為牢藉庖廚之物,奚異犬豕之類乎?」伯豐子不應。伯豐子之從者越次而進曰:「大夫不聞齊魯之多機乎?有善治土木者,有善治金革者,有善治聲樂者,有善治書數者,有善治軍旅者,有善治宗廟者,群才備也。而无相位者,无能相使者。而位之者无知,使之者无能,而知之與能,為之使焉。執政者迺吾之所使,子奚矜焉?」鄧析无以應,目其徒而退。

公儀伯以力聞諸侯,堂谿公言之於周宣王,王備禮以聘之。公儀伯至,觀形,懦夫也。宣王心惑而疑曰:「女之力何如?」公儀伯曰:「臣之力能折春螽之股,堪秋蟬之翼。」王作色曰:「吾之力者能裂犀兕之革。曳九牛之尾,猶憾其弱。女折春螽之股,堪秋蟬之翼,而力聞天下,何也?」公儀伯長息退席曰:「善哉,王之問也!臣敢以實對。臣之師有商丘子者,力无敵於天下,而六親不知,以未嘗用其力故也。臣以死事之。乃告臣曰:『人欲見其所不見,視人所不窺;欲得其所不得,修人所不為。故學眎者先見輿薪,學聽者先聞撞鍾。夫有易於內者,无難於外。於外无難,故名不出其一道。』今臣之名聞於諸侯,是臣違師之教,顯臣之能者也。然則臣之名不以負其力者也,以能用其力者也,不猶愈於負其力者乎?」

中山公子牟者,魏國之賢公子也。好與賢人游,不恤國事,而悅趙人公孫龍。樂正子輿之徒笑之。公子牟曰:「子何笑牟之悅公孫龍也?」子輿曰:「公孫龍之為人也,行无師,學无友,佞給而不中,漫衍而无家,好怪而妄言。欲惑人之心,屈人之口,與韓檀等肄之。」公子牟變容曰:「何子狀公孫龍之過歟?請聞其實。」子輿曰:「吾笑龍之詒孔穿,言『善射者,能令後鏃中前括,發發相及,矢矢相屬;前矢造準,而无絕落,後矢之括猶銜弦,視之若一焉。』孔穿駭之。龍曰:『此未其妙者。逢蒙之弟子曰鴻超,怒其妻而怖之。引烏號之弓,綦衛之箭,射其目。矢來注眸子,而眶不睫,矢隧地而塵不揚。』是豈智者之言與?「公子牟曰:」智者之言,固非愚者之所曉。後鏃中前括,鈞後於前。矢注眸子而眶不睫,盡矢之勢也。子何疑焉?「樂正子輿曰:『子龍之徒,焉得不飾其闕?吾又言其尤者。』龍誑魏王曰:『有意不心。有指不至。有物不盡。有影不移。髮引千鈞。白馬非馬。孤犢未嘗有母。』其負類反倫,不可勝言也。」公子牟曰:』子不諭至言而以為尤也。尤其在子矣。夫无意則心同。无指則皆至。盡物者常有。影不移者,說在改也。髮引千鈞,勢至等也。白馬非馬,形名離也。孤犢未嘗有母非孤犢也。「樂正子輿曰:「子以公孫龍於馬皆條也。設令發於餘竅,子亦將承之。」公子牟默然良久告退曰:「請待餘日,更謁子論。」

堯治天下五十年,不知天下治歟,不治歟?不知億兆之願戴己歟,不願戴己歟?顧問左右,左右不知。問外朝,外朝不知。問在野,在野不知。堯乃微服游於康衢,聞兒童謠曰:「立我蒸民,莫匪爾極。不識不知,順帝之則。」堯喜問曰:「誰教爾為此言?」童兒曰:「我聞之大夫。」問大夫,大夫曰:「古詩也。」堯還宮,召舜,因禪以天下。舜不辭而受之。

關尹喜曰:「在己无居,形物其箸,其動若水,其靜若鏡,其應若響。故其道若物者也。物自違道,道不違物。善若道者,亦不用耳,亦不用目,亦不用力,亦不用心。欲若道而用視聽形智以求之,弗當矣。瞻之在前,忽焉在後;用之彌滿六虛,廢之莫知其所。亦非有心者所能得遠,亦非无心者所能得近。唯默而得之而性成之者得之。知而亡情,能而不為,真知真能也。發无知,何能情?發不能,何能為?聚塊也,積塵也,雖无為1而非理也。」1. 為 : 舊脫。 據《正統道臧》本補。


\end{pinyinscope}