\article{天瑞}

\begin{pinyinscope}
子列子居鄭圃,四十年人无識者。國君卿大夫眎之,猶眾庶也。國不足,將嫁於衛。弟子曰:「先生往无反期,弟子敢有所謁;先生將何以教?先生不聞壺丘子林之言乎?」子列子笑曰:「壺子何言哉?雖然,夫子嘗語伯昏瞀人,吾側聞之,試以告女。其言曰:有生不生,有化不化。不生者能生生,不化者能化化。生者不能不生,化者不能不化,故常生常化。常生常化者,无時不生,无時不化。陰陽爾,四時爾,不生者疑獨,不化者往復。往復,其際不可終;疑獨,其道不可窮。《黃帝書》曰:「谷神不死,是謂玄牝。玄牝之門,是謂天地之根。綿綿若存,用之不勤。」故生物者不生,化物者不化。自生自化,自形自色,自智自力,自消自息。謂之生化、形色、智力、消息者,非也。」

子列子曰:「昔者聖人因陰陽以統天地。夫有形者生於无形,則天地安從生?故曰:有太易,有太初,有太始,有太素。太易者,未見氣也:太初者,氣之始也;太始者,形之始也;太素者,質之始也。氣形質具而未相離,故曰渾淪。渾淪者,言萬物相渾淪而未相離也。視之不見,聽之不聞,循之不得,故曰易也。易无形埒,易變而為一,一變而為七,七變而為九。九變者,究也,乃復變而為一。一者,形變之始也。清輕者上為天,濁重者下為地,沖和氣者為人;故天地含精,萬物化生。」

子列子曰:「天地无全功,聖人无全能,萬物无全用。故天職生覆,地職形載,聖職教化,物職所宜。然則天有所短,地有所長,聖有所否,物有所通。何則?生覆者不能形載,形載者不能教化,教化者不能違所宜,宜定者不出所位。故天地之道,非陰則陽;聖人之教,非仁則義;萬物之宜,非柔則剛:此皆隨所宜而不能出所位者也。故有生者,有生生者;有形者,有形形者;有聲者,有聲聲者;有色者,有色色者;有味者,有味味者。生之所生者死矣,而生生者未嘗終;形之所形者實矣,而形形者未嘗有;聲之所聲者聞矣,而聲聲者未嘗發;色之所色者彰矣,而色色者未嘗顯;味之所味者嘗矣,而味味者未嘗呈:皆无為之識也。能陰能陽,能柔能剛,能短能長,能員能方,能生能死,能暑能涼,能浮能沈,能宮能商,能出能沒,能玄能黃,能甘能苦,能羶能香。无知也,无能也,而无不知也,而无不能也。」

子列子適衛,食於道,從者見百歲髑髏,攓蓬而指,顧謂弟子百豐曰:「唯予與彼知而未嘗生未嘗死也。此過養乎?此過歡乎?種有幾:若䵷為鶉,得水為藚,得水土之際,則為䵷蠙之衣。生於陵屯,則為陵舄。陵舄得鬱栖,則為烏足。烏足之根為蠐螬,其葉為胡蝶。胡蝶胥也,化而為蟲,生竈下,其狀若脫,其名曰鴝掇。鴝掇千日,化而為鳥,其名曰乾餘1骨。乾餘骨之沫為斯彌。斯彌為食醯頤輅。食醯頤輅生乎食醯黃軦,食醯黃軦生乎九猷。九猷生乎瞀芮,瞀芮生乎腐蠸。羊肝化為地皋,馬血之為轉鄰也,人血之為野火也。鷂之為鸇,鸇之為布穀,布穀久復為鷂也。鷰之為蛤也,田鼠之為鶉也,朽瓜之為魚也,老韭之為莧也。老羭之為猨也,魚卵之為蟲。亶爰之獸,自孕而生,曰類。河澤之鳥,視而生,曰鶂。純雌其名大腰,純雄其名稺蜂。思士不妻而感,思女不夫而孕。后稷生乎巨跡,伊尹生乎空桑。厥昭生乎濕,醯雞生乎酒。羊奚比乎不筍,久竹生青寧,青寧生程,程生馬,馬生人。人久入於機。萬物皆出於機,皆入於機。」1. 餘 : 原作「徐」。據下文及今本《莊子》改。

《黃帝書》曰:「形動不生形而生影,聲動不生聲而生響,无動不生无而生有。」形,必終者也;天地終乎?與我偕終。終進乎?不知也。道終乎本无始,進乎本不久。有生則復於不生,有形則復於无形。不生者,非本不生者也;无形者,非本无形者也。生者,理之必終者也。終者不得不終,亦如生者之不得不生。而欲恆其生,盡其終,惑於數也。精神者,天之分;骨骸者,地之分。屬天清而散,屬地濁而聚。精神離形,各歸其真,故謂之鬼。鬼,歸也,歸其真宅。黃帝曰:「精神入其門,骨骸反其根,我尚何存?」

人自生至終,大化有四:嬰孩也,少壯也,老耄也,死亡也。其在嬰孩,氣專志一,和之至也;物不傷焉,德莫加焉。其在少壯,則血氣飄溢,欲慮充起;物所攻焉,德故衰焉。其在老耄,則欲慮柔焉;禮將休焉,物莫先焉;雖未及嬰孩之全,方於少壯,間矣。其在死亡也,則之於息焉,反其極矣。

孔子遊於大山,見榮啟期行乎郕之野,鹿裘帶索,鼓琴而歌。孔子問曰:「先生所以樂,何也?」對曰:「吾樂甚多。天生萬物,唯人為貴。而吾得為人,是一樂也。男女之別,男尊女卑,故以男為貴;吾既得為男矣,是二樂也。人生有不見日月、不免襁褓者,吾既已行年九十矣,是三樂也。貧者士之常也,死者人之終也,處常得終,當何憂哉?」孔子曰:「善乎!能自寬者也。」

林類年且百歲,底春被裘,拾遺穗於故畦,並歌並進。孔子適衛,望之於野。顧謂弟子曰:「彼叟可與言者,試往訊之!」子貢請行。逆之壠端,面之而歎曰:「先生曾不悔乎,而行歌拾穗?」林類行不留。歌不輟。子貢叩之不已,乃仰而應,曰:「吾何悔邪?」子貢曰:「先生少不勤行,長不競時,老无妻子,死期將至,亦有何樂而拾穗行歌乎?」林類笑曰:「吾之所以為樂,人皆有之,而反以為憂。少不勤行,長不競時,故能壽若此。老无妻子,死期將至,故能樂若此。」子貢曰:「壽者人之情,死者人之惡。子以死為樂,何也?」林類曰:「死之與生,一往一反。故死於是者,安知不生於彼?故吾知其不相若矣。吾又安知營營而求生非惑乎?亦又安知吾今之死不愈昔之生乎?」子貢聞之,不喻其意,還以告夫子。夫子曰:「吾知其可與言,果然;然彼得之而不盡者也。」

子貢倦於學,告仲尼曰:「願有所息,」仲尼曰:「生无所息。」子貢曰:「然則賜息无所乎?」仲尼曰:「有焉耳,望其壙,睪如也,宰如也,墳如也,鬲如也,則知所息矣。」子貢曰:「大哉死乎!君子息焉,小人伏焉。」仲尼曰:「賜!汝知之矣。人胥知生之樂,未知生之苦;知老之憊,未知老之佚;知死之惡,未知死之息也。

晏子曰:『善哉,古之有死也!仁者息焉,不仁者伏焉。』死也者,德之徼也。古者謂死人為歸人。夫言死人為歸人,則生人為行人矣。行而不知歸,失家者也。一人失家,一世非之;天下失家,莫知非焉。有人去鄉土,離六親,廢家業,遊於四方而不歸者,何人哉?世必謂之為狂蕩之人矣。又有人鍾賢世,矜巧能,脩名譽,誇張於世,而不知己者,亦何人哉?世必以為智謀之士。此二者,胥失者也。而世與一不與一,唯聖人知所與,知所去。」

或謂子列子曰:「子奚貴虛?」列子曰:「虛者无貴也。」子列子曰:「非其名也,莫如靜,莫如虛。靜也虛也,得其居矣;取也與也,失其所矣。事之破䃣,而後有舞仁義者,弗能復也。」

粥熊曰:「運轉亡已,天地密移,疇覺之哉?故物損於彼者盈於此,成於此者虧於彼。損盈成虧,隨世隨死。往來相接,閒不可省,疇覺之哉?凡一氣不頓進,一形不頓虧;亦不覺其成,不覺其虧。亦如人自世至老,貌色智態,亡日不異;皮膚爪髮,隨世隨落,非嬰孩時有停而不易也。閒不可覺,俟至後知。」

杞國有人,憂天地崩墜,身亡所寄,廢寢食者。又有憂彼之所憂者,因往曉之,曰:「天,積氣耳,亡處亡氣。若屈伸呼吸,終日在天中行止,奈何憂崩墜乎?」其人曰:「天果積氣,日月星宿不當墜邪?」曉之者曰:「日月星宿,亦積氣中之有光耀者,只使墜,亦不能有所中傷。」其人曰:「奈地壞何?」曉者曰:「地積塊耳,充塞四虛,亡處亡塊。若躇步跐蹈,終日在地上行止,奈何憂其壞?」其人舍然大喜,曉之者亦舍然大喜。長廬子聞而笑之曰:「虹蜺也,雲霧也,風雨也,四時也,此積氣之成乎天者也。山岳也,河海也;金石也,火木也,此積形之成乎地者也。知積氣也,知積塊也,奚謂不壞?夫天地,空中之一細物,有中之最巨者。難終難窮,此固然矣;難測難識,此固然矣。憂其壞者,誠為大遠;言其不壞者,亦為未是。天地不得不壞,則會歸於壞。遇其壞時,奚為不憂哉?」子列子聞而笑曰:「言天地壞者亦謬,言天地不壞者亦謬。壞與不壞,吾所不能知也。雖然,彼一也,此一也。故生不知死,死不知生;來不知去,去不知來。壞與不壞,吾何容心哉?」

舜問乎烝曰:「道可得而有乎?」曰:「汝身非汝有也,汝何得有夫道?」舜曰:「吾身非吾有,孰有之哉?」曰:「是天地之委形也。生非汝有,是天地之委和也。性命非汝有,是天地之委順也。孫子非汝有,是天地之委蛻也。故行不知所往,處不知所持,食不知所以。天地,強陽氣也,又胡可得而有邪?」

齊之國氏大富,宋之向氏大貧;自宋之齊,請其術。國氏告之曰:「吾善為盜。始吾為盜也,一年而給,二年而足,三年大壤。自此以往,施及州閭。」向氏大喜,喻其為盜之言,而不喻其為盜之道,遂踰垣鑿室,手目所及,亡不探也。未及時,以贓獲罪,沒其先居之財。向氏以國氏之謬己也,往而怨之。國氏曰:「若為盜若何?」向氏言其狀。國氏曰:「嘻!若失為盜之道至此乎?今將告若矣。吾聞天有時,地有利。吾盜天地之時利,雲雨之滂潤,山澤之產育,以生吾禾,殖吾稼,築吾垣,建吾舍。陸盜禽獸,水盜魚鱉,亡非盜也。夫禾稼、土木、禽獸、魚鱉,皆天之所生,豈吾之所有?然吾盜天而亡殃。夫金玉珍寶穀帛財貨,人之所聚,豈天之所與?若盜之而獲罪,孰怨哉?」向氏大惑,以為國氏之重罔己也,遇東郭先生問焉。東郭先生曰:「若一身庸非盜乎?盜陰陽之和以成若生,載若形;況外物而非盜哉?誠然,天地萬物不相離也;仞而有之,皆惑也。國氏之盜,公道也,故亡殃;若之盜,私心也,故得罪。有公私者,亦盜也;亡公私者,亦盜也。公公私私,天地之德。知天地之德者,孰為盜邪?孰為不盜邪?」


\end{pinyinscope}