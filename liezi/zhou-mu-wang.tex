\article{周穆王}

\begin{pinyinscope}
周穆王時,西極之國,有化人來,入水火,貫金石;反山川,移城邑;乘虛不墜,觸實不硋。千變萬化,不可窮極。既已變物之形,又且易人之慮。穆王敬之若神,事之若君。推路寢以居之,引三牲以進之,選女樂以娛之。化人以為王之宮室卑陋而不可處,王之廚饌腥螻而不可饗,王之嬪御膻惡而不可親。穆王乃為之改築。土木之功。赭堊之色,无遺巧焉。五府為虛,而臺始成。其高千仞,臨終南之上,號曰中天之臺。𥳑鄭、衛之處子娥媌靡曼者,施芳澤,正蛾眉,設笄珥,衣阿錫。曳齊紈。粉白黛黑,珮玉環。雜芷若以滿之,奏《承雲》、《六瑩》、《九韶》、《晨露》以樂之。月月獻玉衣,旦旦薦玉食。化人猶不舍然,不得已而臨之。居亡幾何,謁王同游。王執化人之袪,騰而上者中天迺止。暨及化人之宮。化人之宮構以金銀,絡以珠玉;出雲雨之上而不知下之據,望之若屯雲焉。耳目所觀聽,鼻口所納嘗,皆非人閒之有。王實以為清都、紫微、鈞天、廣樂,帝之所居。王俯而視之,其宮榭若累塊積蘇焉。王自以居數十年不思其國也。化人復謁王同游,所及之處,仰不見日月,俯不見河海。光影所照,王目眩不能得視;音響所來,王耳亂不能得聽。百骸六藏,悸而不凝。意迷精喪,請化人求還。化人移之,王若殞虛焉。既寤,所坐猶嚮者之處,侍御猶嚮者之人。視其前,則酒未清,肴未昲。王問所從來。左右曰:「王默存耳。」由此穆王自失者三月而復。更問化人。化人曰:「吾與王神游也,形奚動哉?且曩之所居,奚異王之宮?曩之所游,奚異王之圃?王閒恆疑蹔亡。變化之極,徐疾之閒,可盡模哉?」王大悅。不恤國事,不樂臣妾,肆意遠游。命駕八駿之乘,右服驊,騮而左綠耳,右驂赤驥而左白𣚘,主車則造父為御,𧮼𠜦為右,次車之乘,右服渠黃而左踰輪,左驂盜驪而右山子,柏天主車,參百為御,奔戎為右。馳驅千里,至於巨蒐氏之國。巨蒐氏乃獻白鵠之血以飲王,具牛馬之湩以洗王之足,及二乘之人。已飲而行,遂宿于崑崙之阿,赤水之陽。別日升于崑崙之丘1,以觀黃帝之宮,而封之,以詒後世。遂賓于西王母觴于瑤池之上。西王母為王謠,王和之,其辭哀焉。迺觀日之所入,一日行萬里。王乃歎曰:「於乎!予一人不盈于德而諧於樂,後世其追數吾過乎!」穆王幾神人哉!能窮當身之樂,猶百年乃徂,世以為登假焉。1. 丘 : 原作「□」。底本空一字,據《正統道臧》本補。

老成子學幻於尹文先生,三年不告。老成子請其過而求退。尹文先生揖而進之於室,屏左右而與之言曰:「昔老聃之徂西也,顧而告予曰:有生之氣,有形之狀,盡幻也。造化之所始,陰陽之所變者,謂之生,謂之死。窮數達變,因形移易者,謂之化,謂之幻。造物者其巧妙,其功深,固難窮難終。因形者其巧顯。其功淺,故隨起隨滅。知幻化之不異生死也,始可與學幻矣。吾與汝亦幻也,奚須學哉?」老成子歸,用尹文先生之言,深思三月,遂能存亡自在,憣校四時;冬起雷,夏造冰;飛者走,走者飛。終身不箸其術,固世莫傳焉。子列子曰:「善為化者,其道密庸,其功同人。五帝之德,三王之功,未必盡智勇之力,或由化而成。孰測之哉?」

覺有八徵,夢有六候。奚謂八徵?一曰故,二曰為,三曰得,四曰喪,五曰哀,六曰樂,七曰生,八曰死。此者八徵,形所接也。奚謂六候?一曰正夢,二曰蘁夢,三曰思夢,四曰寤夢,五曰喜夢,六曰懼夢。此六者,神所交也。不識感變之所起者,事至則惑其所由然,識感變之所起者,事至則知其所由然。知其所由然則無所怛1。一體之盈虛消息,皆通於天地,應於物類。故陰氣壯,則夢涉大水而恐懼;陽氣壯,則夢涉大火而燔焫;陰陽俱壯,則夢生殺。甚飽則夢與,甚饑則夢取。是以以浮虛為疾者,則夢揚;以沈實為疾者,則夢溺。藉帶而寢,則夢蛇;飛鳥銜髮,則夢飛。將陰夢火,將疾夢食。飲酒者憂,歌儛者哭。子列子曰:「神遇為夢,形接為事。故晝想夜夢,神形所遇。故神凝者想夢自消。信覺不語,信夢不達,物化之往來者也。古之真人,其覺自忘,其寢不夢,幾虛語哉?」1. 怛 : 原作「□」。底本該字不完整,據《正統道臧》本補。

西極之南隅有國焉,不知境界之所接,名古莽之國。陰陽之氣所不交,故寒暑亡辨;日月之光所不照,故晝夜亡辨。其民不食不衣而多眠。五旬一覺,以夢中所為者實,覺之所見者妄。四海之齊謂中央之國,跨河南北,越岱東西,萬有餘里。其陰陽之審度,故一寒一暑;昏明之分察,故一晝一夜。其民有智有愚。萬物滋殖,才藝多方。有君臣相臨,禮法相持。其所云為,不可稱計。一覺一寐,以為覺之所為者實,夢之所見者妄。東極之北隅有國,曰阜落之國。其土氣常燠,日月餘光之照其土,不生嘉苗。其民食草根水實,不知火食。性剛悍,彊弱相藉,貴勝而不尚義;多馳步,少休息,常覺而不眠。

周之尹氏大治產,其下趣役者,侵晨昏而弗息。有老役夫,筋力竭矣,而使之彌勤。晝則呻呼而即事,夜則昏憊而熟寐。精神荒散,昔昔夢為國君。居人民之上,總一國之事。遊燕宮觀,恣意所欲,其樂无比。覺則復役。人有慰喻其懃者,役夫曰:「人生百年,晝夜各分。吾晝為僕虜,苦則苦矣;夜為人君,其樂无比。何所怨哉?」尹氏心營世事,慮鍾家業,心形俱疲,夜亦昏憊而寐。昔昔夢為人僕,趨走作役,无不為也;數罵杖撻,无不至也。眠中啽囈呻呼,徹且息焉。尹氏病之,以訪其友。友曰:「若位足榮身,資財有餘,勝人遠矣。夜夢為僕,苦逸之復,數之常也。若欲覺夢兼之,豈可得邪?」尹氏聞其友言,寬其役夫之程,減己思慮之事,疾並少閒。

鄭人有薪於野者,遇駭鹿,御而擊之,斃之。恐人見之也,遽而藏諸隍中,覆之以蕉,不勝其喜。俄而遺其所藏之處,遂以為夢焉。順塗而詠其事。傍人有聞者,用其言而取之。既歸,告其室人曰:「向薪者夢得鹿而不知其處;吾今得之,彼直真夢者矣。?」室人曰:「若將是夢見薪者之得鹿邪?詎有薪者邪?今真得鹿,是若之夢真邪?」夫曰:「吾據得鹿,何用知彼夢我夢邪?」薪者之歸,不厭失鹿,其夜真夢藏之之處,又夢得之之主。爽旦,案所夢而尋得之。遂訟而爭之,歸之士師。士師曰:「若初真得鹿,妄謂之夢;真夢得鹿,妄謂之實。彼真取若鹿,而與若爭鹿。室人又謂夢仞人鹿,无人得鹿。今據有此鹿,請二分之。」以聞鄭君。鄭君曰:「嘻!士師將復夢分人鹿乎?」訪之國相。國相曰:「夢與不夢,臣所不能辨也。欲辨覺夢,唯黃帝、孔丘。今亡黃帝、孔丘,孰辨之哉?且恂士師之言可也。」

宋陽里華子,中年病忘,朝取而夕忘,夕與而朝忘;在塗則忘行,在室則忘坐;今不識先,後不識今。闔室毒之。謁史而卜之,弗占;謁巫而禱之,弗禁;謁醫而攻之,弗已。魯有儒生,自媒能治之,華子之妻子以居產之半請其方。儒生曰:「此固非卦兆之所占,非祈請之所禱,非藥石之所攻。吾試化其心,變其慮,庶幾其瘳乎!」於是試露之而求衣;饑之而求食;幽之而求明。儒生欣然告其子曰:「疾可已也。然吾之方密傳世,不以告人。試屏左右,獨與居室七日。」從之。莫知其所施為也,而積年之疾,一朝都除。華子既悟,迺大怒,黜妻罰子,操戈逐儒生。宋人執而問其以。華子曰:「曩吾忘也,蕩蕩然不覺天地之有无。今頓識,既往數十年來,存亡得失、哀樂好惡,擾擾萬緒起矣。吾恐將來之存亡得失哀樂好惡之亂吾心如此也,須臾之忘,可復得乎?」子貢聞而怪之,以告孔子。孔子曰:「此非汝所及乎!」顧謂顏回記之。

秦人逢氏有子,少而惠,及壯而有迷罔之疾。聞歌以為哭,視白以為黑,饗香以為朽,常1甘以為苦,行非以為是。意之所之,天地四方水火寒暑,无不倒錯者焉。楊氏告其父曰:「魯之君子多術藝,將能已乎?汝奚不訪焉。?」其父之魯,過陳,遇老聃,因告其子之證。老聃曰:「汝庸知汝子之迷乎?今天下之人,皆惑於是非,昏於利害。同疾者多,固莫有覺者。且一身之迷,不足傾一家;一家之迷,不足傾一鄉;一鄉之迷,不足傾一國;一國之迷,不足傾天下;天下盡迷,孰傾之哉?向使天下之人,其心盡如汝子,汝則反迷矣。哀樂聲色臭味是非,孰能正之?且吾之此言未必非迷,而況魯之君子,迷之郵者,焉能解人之迷哉?榮汝之糧,不若遄歸也。」1. 常 : 或作「嘗」。《太平御覽》引作「嘗」。

燕人生於燕,長於楚,及老而還本國。過晉國,同行者誑之,指城曰:「此燕國之城。」其人愀然變容。指社曰:「此若里之社。」乃喟然而歎。指舍曰:「此若先人之廬。」乃涓然而泣。指壠曰:「此若先人之冢。」其人哭不自禁。同行者啞然大笑,曰:「予昔紿若,此晉國耳。」其人大慚。及至燕,真見燕國之城社,真見先人之廬冢,悲心更微。


\end{pinyinscope}