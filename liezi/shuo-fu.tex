\article{說符}

\begin{pinyinscope}
子列子學於壺丘子林。壺丘子林曰:「子知持後,則可言持身矣。」列子曰:「願聞持後。」曰:「顧若影,則知之。」列子顧而觀影:形枉則影曲,形直則影正。然則枉直隨形而不在影,屈申任物而不在我,此之謂持後而處先。

關尹謂子列子曰:「言美則響美,言惡則響惡;身長則影長,身短則影短。名也者,響也;身也者,影也。故曰:慎爾言,將有知之;慎爾行,將有隨之,是故聖人見出以知入,觀往以知來,此其所以先知之理也。度在身,稽在人。人愛我,我必愛之;人惡我,我必惡之。湯武愛天下,茲王;桀、紂惡天下,故亡,此所稽也。稽度皆明而不道也,譬之出不由門,行不從徑也。以是求利,不亦難乎?嘗觀之神農、有炎之德,稽之虞、夏、商、周之書,度諸法士賢人之言,所以存亡廢興而非由此道者,未之有也。」

嚴恢曰:「所為問道者為富,今1得珠亦富矣,安用道?」子列子曰:「桀、紂唯重利而輕道,是以亡。幸哉余未汝語也!人而无義,唯食而已,是雞狗也。彊食靡角,勝者為制,是禽獸也。為雞狗禽獸矣,而欲人之尊己,不可得也。人不尊己,則危辱及之矣。」1. 今 : 原作「令」。CHANT據《道藏》本改。

列子學射,中矣,請於關尹子。尹子曰:「子知子之所以中者乎?」對曰:「弗知也。」關尹子曰:「未可。」退而習之。三年,又以報關尹子。尹子曰:「子知子之所以中乎?」列子曰:「知之矣。」關尹子曰:「可矣,守而勿失也。非獨射也,為國與身,亦皆如之。故聖人不察存亡,而察其所以然。」

列子曰:「色盛者驕,力盛者奮,未可以語道也。故不班白語道矣,而況行之乎?故自奮則人莫之告。人莫之告,則孤而無輔矣1。賢者任人,故年老而不衰,智盡而不亂。故治國之難,在於知賢而不在自賢。」1. 矣 : 或作「失」。楊柏峻《列子集釋》云:「《釋文》云:為句。失一本作矣,恐誤。」

宋人有為其君以玉為楮葉者,三年而成。鋒殺莖柯,毫芒繁澤,亂之楮葉中,而不可別也。此人遂以巧食宋國。子列子聞之曰:「使天地之生物,三年而成一葉,則物之有葉者寡矣。故聖人恃道化而不恃智巧。」

子列子窮,容貌有飢色。客有言之鄭子陽者,曰:「列禦寇蓋有道之士也,居君之國而窮。君无乃為不好士乎?」鄭子陽即令官遺之粟。子列子出,見使者,再拜而辭。使者去。子列子入,其妻望之而拊心曰:「妾聞為有道者之妻子,皆得佚樂,今有饑色,君遇而遺先生食。先生不受,豈不命也哉?」子列子笑謂之曰:「君非自知我也。以人之言而遺我粟,至1其罪我也,又且以人之言,此吾所以不受也。」其卒,民果作難,而殺子陽。1. 至 : 原作「室」。據《南華真經》改。

魯施氏有二子,其一好學,其一好兵。好學者以術干齊侯;齊侯納之以為諸公子之傅。好兵者之楚,以法干楚王;王悅之,以為軍正。祿富其家,爵榮其親。施氏之鄰人孟氏,同有二子,所業亦同,而窘於貧。羨施氏之有,因從謂進趣之方。二子以實告孟氏。孟氏之一子之秦,以愆干秦王。秦王曰:「當今諸侯力爭,所務兵食而已。若用仁義治吾國,是滅亡之道。」遂官而放之。其一子之衛,以法干衛侯。衛侯曰:「吾弱國也,而攝乎大國之間。大國吾事之,小國吾撫之,是求家之道。者賴兵權,滅亡可待矣。若全而歸之,適於他國。為吾之患不輕矣。」遂則1之而還諸魯。既反,孟氏之父子叩胸而讓施氏。施氏曰:「凡得時者昌,失時者亡。子道與吾同,而功與吾異,失時者也,非行之謬也。且天下理无常是,事无常非。先日所用,今或棄之;今之所棄,後或用之。此用與不用,无定是非也。投隙抵時,應事无方,屬乎智,智苟不2足,使若博如孔丘,術如呂尚,焉往而不窮哉?」孟氏父子舍然无慍容,曰:「吾知之矣,子勿重言!」1. 則 : 原作「州」。CHANT據《道藏》本改。2. 不 : 舊脫。 CHANT據《道藏》本增。

晉文公出,會欲伐衛,公子鋤仰天而笑。公問何笑。曰:「臣笑鄰之人有送其妻適私家者,道見桑婦,悅而與言。然顧視其妻,亦有招之者矣。臣竊笑此也。「公寤其言乃止。引師而還,未至而有伐其北鄙者矣。

晉國苦盜,有郄雍者,能視盜之眼,察其眉睫之閒而得其情。晉侯使視盜,千百无遺一焉。晉侯大喜,告趙文子曰:「吾得一人,而一國盜為盡矣,奚用多為?」文子曰:「吾君恃伺察而得盜,盜不盡矣,且郄雍必不得其死焉。」俄而群盜謀曰:『吾所窮者郄雍也。「遂共盜而殘之。晉侯聞而大駭,立召文子而告之曰:「果如子言,郄雍死矣!然取盜何方?」文子曰:「周諺有言:察見淵魚者不祥,智料隱匿者有殃。』君欲无盜,若莫舉賢而任之;使教明於上,化行於下,民有恥心,則何盜之為?」於是用隨會知政,而群盜奔秦焉。

孔子自衛反魯,息駕乎河梁而觀焉。有懸水三十仞,圜流九十里,魚鱉弗能游,黿鼉弗能居,有一丈夫,方將厲之。孔子使人並涯止之曰:「此懸水三十仞,圜流九十里,魚鱉弗能游,黿鼉弗能居也。意者難可以濟乎?」丈夫不以錯意,遂度而出。孔子問之曰:「巧乎?有道術乎?所以能入而出者何也?」丈夫對曰:『始吾之入也,先以忠信;及吾之出也,又從以忠信。忠信錯吾軀於波流,而吾不敢用私,所以能入而復出者,以此也。孔子謂弟子曰:「二三子識之!水且猶可以忠信誠身親之,而況人乎?」

白公問孔子曰:「人可與微言乎?」孔子不應。白公問曰:「若以石投水何如?」孔子曰:「吳之善沒者能取之。」曰:「若以水投水何如?」孔子曰:「淄、澠之合,易牙嘗而知之。」白公曰:「人故不可與微言乎?」孔子曰:「何為不可?唯知言之謂者乎!夫知言之謂者,不以言言也。爭魚者濡,逐獸者趨,非樂之也。故至言去言,至為无為。夫淺知之所爭者,末矣。」白公不得已,遂死於浴室。

趙襄子使新穉穆子攻翟,勝之,取左人中人;使遽人來謁之。襄子方食而有憂色。左右曰:「一朝而兩城下,此人之所喜也;今君有憂色,何也?」襄子曰:「夫江河之大也,不過三日;飄風暴雨不終朝,日中不須臾。今趙氏之德行,无所施於積,一朝而兩城下,亡其及我哉!」孔子聞之曰:「趙氏其昌乎!夫憂者所以為昌也,喜者所以為亡也。勝非其難者也;持之其難者也。賢主以此持勝,故其福及後世。齊、楚、吳、越皆嘗勝矣,然卒取亡焉,不達乎持勝也。唯有道之主為能持勝。」孔子之勁,能拓國門之關,而不肯以力聞。墨子為守攻,公輸般服,而不肯以兵知。故善持勝者,以彊為弱。

宋人有好行仁義者,三世不懈。家无故黑牛生白犢,以問孔子。孔子曰:「此吉祥也,以薦上帝。」居一年,其父无故而盲,其牛又復生白犢。其父又復令其子問孔子。其子曰:「前問之而失明,又何問乎?」父曰:「聖人之言先迕後合。其事未究,姑復問之。」其子又復問孔子。孔子曰:「吉祥也。」復教以祭。其子歸致命。其父曰:「行孔子之言也。」居一年,其子又无故而盲。其後楚攻宋,國其城。民易子而食之,析骸而炊之;丁壯者皆乘城而戰,死者太半。此人以父子有疾,皆免。及圍解而疾俱復。

宋有蘭子者,以技干宋元。宋元召而使見其技,以雙枝長倍其身,屬其脛,並趨並馳,弄七劍,迭而躍之,五劍常在空中。元君大驚,立賜金帛。又有蘭子又能燕戲者,聞之,復以干元君。元君大怒曰:「昔有異技干寡人者,技无庸,適值寡人有歡心,故賜金帛。彼必聞此而進,復望吾賞。」拘而戮之,經月乃放。

秦穆公謂伯樂曰:「子之年長矣,子姓有可使求馬者乎?」伯樂對曰:「良馬,可形容筋骨相也。天下之馬者,若滅若沒,若亡若失,若此者絕塵弭轍。臣之子皆下才也,可告以良馬,不可告以天下之馬也。臣有所與共擔纆薪菜者,有九方皋,比其於馬,非臣之下也。請1見之。」穆公見之,使行求馬。三月而反,報曰:「已得之矣,在沙丘。」穆公曰:「何馬也?」對曰:「牝而黃。」使人往取之,牡而驪。穆公不說,召伯樂而謂之曰:「敗矣,子所使求馬者!色物、牝牡尚弗能知,又何馬之能知也?」伯樂喟然太息曰:「一至於此乎!是乃其所以千萬臣而无數者也。若皋之所觀,天機也,得其精忘其麤,在其內而忘其外;見其所見,不見其所不見;視其所視,而遺其所不視。若皋之相者,乃有貴乎馬者也。」馬至,果天下之馬也。1. 請 : 原作「謂」。據《正統道臧》改。

楚莊王問詹何曰:「治國柰何?」詹何對曰:「臣明於治身而不明於治國也。」楚莊王曰:「寡人得奉宗廟社稷,願學所以守之。」詹何對曰:「臣未嘗聞身治而國亂者也,又未嘗聞身亂而國治者也。故本在身,不敢對以末。」楚王曰:「善!」

狐丘丈人謂孫叔敖曰:「人有三怨,子知之乎?」孫叔敖曰:「何謂也?」對曰:「爵高者人妬之,官大者主惡之,祿厚者怨遠之。」孫叔敖曰:「吾爵益高,吾志益下;吾官益大,吾心益小;吾祿益厚,吾施益博。以是免於三怨,可乎?」

孫叔敖疾將死,戒其子曰:「王亟封我矣,吾不受也,為我死,王則封汝。汝必无受利地!楚、越之閒,有寢丘者,此地不利而名甚惡。楚人鬼而越人禨,可長有者唯此也。」孫叔敖死,王果以美地封其子。子辭而不受,請寢丘。與之,至今不失。

牛缺者,上地之大儒也,下之邯鄲,遇盜於耦沙之中,盡取其衣裝車,牛步而去。視之,歡然无憂𠫤之色。盜追而問其故。曰:「君子不以所養害其所養。」盜曰:「嘻!賢矣夫!」既而相謂曰:「以彼之賢,往見趙君。便以我為,必困我。不如殺之。」乃相與追而殺之。燕人聞之,聚族相戒,曰:「遇盜莫如上地之牛缺也!」皆受教。俄而其弟適秦,至闕下,果遇盜。憶其兄之戒,因與盜力爭;既而不如,又追而以卑辭請物。盜怒曰:「吾活汝弘矣,而追吾不已,迹將箸焉。既為盜矣,仁將焉在?」遂殺之,又傍害其黨四五人焉。

虞氏者,梁之富人也,家充殷盛,錢帛无量,財貨无訾。登高樓,臨大路,設樂陳酒,擊博樓上,俠客相隨而行,樓上博者射,明瓊張中,反兩㯓魚而笑。飛鳶適墜其腐鼠而中之。俠客相與言曰:「虞氏富樂之日久矣,而常有輕易人之志。吾不侵犯之,而乃辱我以腐鼠。此而不報,无以立慬於天下。請與若等戮力一志,率徒屬,必滅其家為等倫。」皆許諾。至期日之夜,聚眾積兵,以攻虞氏,大滅其家。

東方有人焉,曰爰旌目,將有適也,而餓於道。狐父之盜曰丘,見而下壺餐以餔之。爰旌目三餔而後能視,曰:「子何為者也?」曰:「我狐父之人丘也。」爰旌目曰:「譆!汝非盜邪?胡為而餐我?吾義不食子之食也。」兩手據地而歐之,不出,喀喀然遂伏而死。狐父之人則盜矣,而食非盜也。以人之盜,因謂食為盜而不敢食,是失名實者也。

柱厲叔事莒敖公,自為不知己,去,居海上。夏日則食菱芰,冬日則食橡栗。莒敖公有難,柱厲叔辭其友而往死之。其友曰:「子自以為不知己,故去;今往死之,是知與不知无辨也。」柱厲叔曰:「不然。自以為不知。故去;今死,是果不知我也。吾將死之,以醜後世之人主不知其臣者也。」凡知則死之,不知則弗死,此直道而行者也。柱厲叔可謂懟以忘其身者也。

楊朱曰:「利出者實及,怨往者害來。發於此而應於外者唯請,是故賢者慎所出。」

楊子之鄰人亡羊,既率其黨,又請楊子之豎追之。楊子曰:「嘻!亡一羊何追者之眾?」鄰人曰:「多岐路。」既反,問:「獲羊乎?」曰:「亡之矣。」曰:「奚亡之?」曰:「岐路之中又有岐焉。吾不知所之,所以反也。」楊子戚然變容,不言者移時,不笑者竟日。門人怪之,請曰:「羊賤畜,又非夫子之有,而損言笑者何哉?」揚子不荅。門人不獲所命。弟子孟孫陽出,以告心都子。心都子他日與孟孫陽偕入而問曰:『昔有昆弟三人,游齊、魯之閒,同師而學,進仁義之道而歸。其父曰:『仁義之道若何?』伯曰:『仁義使我愛身而後名。』仲曰:『仁義使我殺身以成名。』叔曰:『仁義使我身名並全。』彼三術相反,而同出於儒。孰是孰非邪?「楊子曰:「人有濱河而居者,習於水,勇於泅,操舟鬻渡,利供百口,裹糧就學者成徒,而溺死者幾半。本學泅不學溺,而利害如此。若以為孰是孰非?」心都子嘿然而出。孟孫陽讓之曰:「何吾子問之迂,夫子荅之僻?吾惑愈甚。」心都子曰:「大道以多岐亡羊,學者以多方喪生。學非本不同,非本不一,而末異若是。唯歸同反一,為亡得喪。子長先生之門,習先生之道,而不達先生之況也,哀哉!」

楊朱之弟曰布,衣素衣而出。天雨,解素衣,衣緇衣而反。其狗不知,迎而吠之。楊布怒將扑之。楊朱曰:「子無扑矣!子亦猶是也。嚮者使汝狗白而往黑而來,豈能无怪哉?」

楊朱曰:「行善不以為名而名從之;名不與利期而利歸之;利不與爭期而爭及之:故君子必慎為善。」

昔人言有知不死之道者,燕君使人受之,不捷,而言者死。燕君甚怒其使者,將加誅焉。幸臣諫曰:「人所憂者莫急乎死,己所重者莫過乎生。彼自喪其生,安能令君不死也?」乃不誅。有齊子亦欲學其道,聞言者之死,乃撫膺而恨。富子聞而笑之曰:「夫所欲學不死,其人已死,而猶恨之,是不知所以為學。」胡子曰:「富子之言非也。凡人有術不能行者有矣,能行而无其術者亦有矣。衛人有善數者,臨死,以決喻其子。志其言而不能行也。他人問之,以其父所言告之。問者用其言而行其術,與其父无差焉。若然,死者奚為不能言生術哉?」

邯鄲之民,以正月之旦獻鳩於𥳑子,𥳑子大悅,厚賞之。客問其故。𥳑子曰:「正旦放生,示有恩也。」客曰:「民知君之欲放之,故競而捕之,死者眾矣。君如欲生之,不若禁民勿捕。捕而放之,恩過不相補矣。」𥳑子曰:「然。」

齊田氏祖於庭,食客千人。中坐有獻魚鴈者。田氏視之,乃歎曰:「天之於民厚矣!殖五穀,生魚鳥,以為之用。眾客和之如響。鮑氏之子年十二,預於次,進曰:「不如君言。天地萬物,與我並生類也。類无貴賤,徒以小大智力而相制,迭相食;非相為而生之。人取可食者而食之,豈天本為人生之?且蚊蚋噆膚,虎狼食肉,非天本為蚊蚋生人、虎狠生肉者哉?」

齊有貧者,常乞於城市。城市患其亟也,眾莫之與。遂適田氏之廄,從馬醫作役,而假食郭中。人戲之曰:「從馬醫而食,不以辱乎?」乞兒曰:「天下之辱莫過於乞。乞猶不辱,豈辱馬醫哉?」

宋人有游於道,得人遺契者,歸而藏之,密數其齒。告鄰人曰:「吾富可待矣。」

人有枯梧樹者,其鄰父言枯梧之樹不祥。其鄰人遽而伐之。鄰人父因請以為薪。其人乃不悅曰:「鄰人之父徒欲為薪,而教吾伐之也。與我鄰若此,其險豈可哉?」

人有亡鈇者,意其鄰之子。視其行步,竊鈇也;顏色,竊鈇也;言語,竊鈇也;作動態度,无為而不竊鈇也。俄而抇其谷而得其鈇,他日復見其鄰人之子,動作態度,无似竊鈇者。

白公勝慮亂,罷朝而立,倒杖策,錣上貫頤,血流至地而弗知也。鄭人聞之曰:「頭之忘,將何不忘哉?」意之所屬箸,其行足躓株埳,頭抵植木,而不自知也。

昔齊人有欲金者,清旦衣冠而之市,適鬻金者之所,因攫其金而去。吏捕得之,問曰:「人皆在焉,子攫人之金何?」對曰:「取金之時,不見人,徒見金。」


\end{pinyinscope}