\article{湯問}

\begin{pinyinscope}
殷湯問於夏革曰:「古初有物乎?」夏革曰:「古初1无物,今惡得物?後之人將謂今之无物可乎?」殷湯曰:「然則物无先後乎?」夏革曰:「物之終始,初无極已。始或為終,終或為始,惡知其紀?然自物之外,自事之先,朕所不知也。」殷湯曰:「然則上下八方有極盡乎?」革曰:「不知也。」湯固問。革曰:「无則无極,有則有盡;朕何以知之?然无極之外,復无无極,无盡之中,復无无盡。无極復无无極,无盡復无无盡。朕以是知其无極无盡也,而不知其有極有盡也。」湯又問曰:「四海之外奚有?」革曰:「猶齊州也。」湯曰:「汝奚以實之?」革曰:「朕東行至營,人民猶是也。問營之東,復猶營也。西行至豳,人民猶是也。問豳之西,復猶豳也。朕以是知四海四荒四極之不異是也。故大小相含,无窮極也。含萬物者亦如含天地;含萬物也故不窮,含天地也故无極。朕亦焉知天地之表不有大天地者乎?亦吾所不知也。然則天地亦物也。物有不足,故昔者女媧氏練五色石以補其闕;斷鼇之足以立四極。其後共工氏與顓頊爭為帝,怒而觸不周之山,折天柱,絕地維,故天傾西北,日月星辰就焉;地不滿東南,故百川水潦歸焉。」1. 曰:「古初 : 原作「□□□」。底本空三字,據《正統道臧》本補。

湯又問:「物有巨細乎?有脩短乎?有同異乎?」革曰:「渤海之東不知幾億萬里,有大壑焉,實惟无底之谷,其下无底,名曰歸墟。八絃九野之水,天漢之流,莫不注之,而无增无減焉。其中有五山焉:一曰岱輿,二曰員嶠,三曰方壺,四曰瀛洲,五曰蓬萊。其山高下周旋三萬里,其頂平處九千里。山之中閒相去七萬里,以為鄰居焉。其上臺觀皆金玉,其上禽獸皆純縞。珠玕之樹皆叢生,華實皆有滋味,食之皆不老不死。所居之人皆仙聖之種;一日一夕飛相往來者,不可數焉。而五山之根,无所連箸,常隨潮波上下往還,不得蹔峙焉。仙聖毒之,訴之於帝。帝恐流於西極,失群仙聖之居,乃命禺彊使巨鼇十五舉首而戴之。迭為三番,六萬歲一交焉。五山始峙而不動。而龍伯之國,有大人,舉足不盈數千而暨五山之所,一釣而連六鼇,合負而趣,歸其國,灼其骨以數焉。於是岱輿員嶠二山流於北極,沈於大海,仙聖之播遷者巨億計。帝憑怒,侵減龍伯之國使阨。侵小龍伯之民使短。至伏羲神農時,其國人猶數十丈。從中州以東四十萬里,得僬僥國。人長一尺五寸。東北極有人名曰諍人,長九寸。荊之南有冥靈者,以五百歲為春,五百歲為秋。上古有大椿者,以八千歲為春,八千歲為秋。朽壤之上有菌芝者,生於朝,死於晦。春夏之月有蠓蚋者,因雨而生,見陽而死。終北之北有溟海者,天池也,有魚焉。其廣數千里,其長稱焉,其名為鯤。有鳥焉。其名為鵬,翼若垂天之雲,其體稱焉。世豈知有此物哉?大禹行而見之,伯益知而名之,夷堅聞而志之。江浦之閒生麼蟲,其名曰焦螟,群飛而集於蚊睫,弗相觸也。栖宿去來,蚊弗覺也。離朱子羽,方晝拭眥揚眉而望之,弗見其形;𧣾俞師曠方夜擿耳俛首而聽之,弗聞其聲。唯黃帝與容成子居空桐之上,同齋三月,心死形廢;徐以神視,塊然見之,若嵩山之阿;徐以氣聽,砰然聞之若電霆之聲。吳、楚之國有大木焉,其名為櫾,碧樹而冬生,實丹而味酸;食其皮汁,已憤厥之疾。齊州珍之,渡淮而北,而化為枳焉。鸜鵒不踰濟,貉踰汶則死矣。地氣然也。雖然形氣異也,性鈞已,无相易已。生皆全已,分皆足已。吾何以識其巨細?何以識其脩短?何以識其同異哉?」

太形、王屋二山,方七百里,高萬仞。本在冀州之南,河陽之北。北山愚公者,年且九十,面山而居。懲山北之塞,出入之迂也。聚室而謀曰:「吾與汝畢力平險,指通豫南,達于漢陰,可乎?」雜然相許。其妻獻疑曰:「以君之力,曾不能損魁父之丘,如太形、王屋何?且焉置土石?」雜曰:「投諸渤海之尾,隱土之北。」遂率子孫荷擔者三夫,叩石墾壤,箕畚運於渤海之尾。鄰人京城氏之孀妻,有遺男,始齓,跳往助之。寒暑易節,始一反焉。河曲智叟笑而止之,曰:「甚矣汝之不惠!以殘年餘力,曾不能毀山之一毛,其如土石何?」北山愚公長息曰:「汝心之固,固不可徹,曾不若孀妻弱子。雖我之死,有子存焉;子又生孫,孫又生子;子又有子,子又有孫;子子孫孫,无窮匱也,而山不加增,何苦而不平?」河曲智叟亡以應。操蛇之神聞之,懼其不已也,告之於帝。帝感其誠,命夸蛾氏二子負二山,一厝朔東,一厝雍南。自此冀之南,漢之陰,无隴斷焉。夸父不量力,欲追日影,逐之於隅谷之際。渴欲得飲,赴飲河渭。河渭不足,將走北飲大澤。未至道,渴而死。棄其杖,尸膏肉所浸,生鄧林。鄧林彌廣數千里焉。

大禹曰:「六合之閒,四海之內,照之以日月,經之以星辰,紀之以四時,要之以太歲。神靈所生,其物異形;或夭或壽,唯聖人能通其道。」夏革曰:「然則亦有不待神靈而生,不待陰陽而形,不待日月而明,不待殺戮而夭,不待將迎而壽,不持五穀而食,不待繒纊而衣,不待舟車而行。其道自然,非聖人之所通也。」

禹之治水上也,迷而失塗,謬之一國。濱北海之北,不知距齊州幾千萬里,其國名曰終北,不知際畔之所齊限。无風雨霜露,不生鳥、獸、蟲、魚、草、木之類。四方悉平,周以喬陟。當國之中有山,山名壺領,狀若甔甄。頂有口,狀若員環,名曰滋穴。有水湧出,名曰神瀵,臭過蘭椒,味過醪醴。一源分為四埒,注於山下;經營一國,亡不悉徧。土氣和,亡札厲。人性婉而從,物不競不爭。柔心而弱骨,不驕不忌;長幼儕居,不君不臣;男女雜游,不媒不聘;緣水而居,不耕不稼;土氣溫適,不織不衣;百年而死,不夭不病。其民孳阜亡數,有喜樂,亡衰老哀苦。其俗好聲,相攜而迭謠,終日不輟音。饑惓則飲神瀵,力志和平。過則醉經旬乃醒。沐浴神瀵,膚色脂澤,香氣經旬乃歇。周穆王北遊,過其國,三年忘歸。既反周室,慕其國,惝然自失。不進酒肉,不召嬪御者數月,乃復。管仲勉齊桓公,因遊遼口,俱之其國。幾剋舉,隰朋諫曰:「君舍齊國之廣,人民之眾,山川之觀,殖物之阜,禮義之盛,章服之美,妖靡盈庭,忠良滿朝,肆咤則徒卒百萬,視撝則諸侯從命,亦奚羨於彼,而棄齊國之社稷,從戎夷之國乎?此仲父之耄,柰何從之?」桓公乃止,以隰朋之言告管仲,仲曰:「此固非朋之所及也。臣恐彼國之不可知之也。齊國之富奚戀?隰朋之言奚顧?」

南國之人,被髮而裸;北國之人,鞨巾而裘;中國之人,冠冕而裳。九土所資,或農或商或田或漁,如冬裘夏葛,水舟陸車,默而得之,性而成之。越之東有輒沐之國,其長子生,則鮮而食之,謂之宜弟。其大父死,負其大母而棄之,曰:「鬼妻不可與同居處。」楚之南有炎人之國,其親戚死,㱙其肉而棄之,然後埋其骨,迺成為孝子。秦之西有儀渠之國者,其親戚死。聚祡積而焚之。燻則煙上,謂之登遐,然後成為孝子。此上以為政,下以為俗。而未足為異也。

孔子東游,見兩小兒辯鬭。問其故,一兒曰:「我以日始出時去人近,而日中時遠也。」一兒以日初出遠,而日中時近也。一兒曰:「日初出大如車蓋,及日中,則如盤盂,此不為遠者小而近者大乎?」一兒曰:「日初出滄滄涼涼,及其日中,如探湯,此不為近者熱而遠者涼乎?」孔子不能決也。兩小兒笑曰:「孰為汝多知乎?」

均,天下之至理也,連於形物亦然。均髮均縣輕重而髮絕,髮不均也。均也,其絕也,莫絕。人以為不然,自有知其然者也。詹何以獨繭絲為綸,芒鍼為鉤,荊篠為竿,剖粒為餌,引盈車之魚於百仞之淵、汩流之中,綸不絕,鉤不伸,竿不撓。楚王聞而異之,召問其故。詹何曰:「臣聞先大夫之言。蒲且子之弋也,弱弓纖繳,乘風振之,連雙鶬於青雲之際。用心專,動手均也。臣因其事,放而學釣,五年始盡其道。當臣之臨河持竿,心无雜慮,唯魚之念;投綸沈鉤,手无輕重,物莫能亂。魚見臣之釣餌,猶沈埃聚沫,吞之不疑。所以能以弱制彊,以輕致重也。大王治國誠能若此,則天下可運於一握,將亦奚事哉?」楚王曰:「善!」

魯公扈、趙齊嬰二人有疾,同請扁鵲求治,扁鵲治之。既同愈。謂公扈、齊嬰曰:「汝曩之所疾,自外而干府藏者,固藥石之所已。今有偕生之疾,與體偕長,今為汝攻之,何如?」二人曰:「願先聞其驗。」扁鵲謂公扈曰:「汝志彊而氣弱,故足於謀而寡於斷。齊嬰志弱而氣彊,故少於慮而傷於專。若換汝之心,則均於善矣。」扁鵲遂飲二人毒酒,迷死三日,剖胸探心,易而置之;投以神藥,既悟,如初。二人辭歸。於是公扈反齊嬰之室,而有其妻子,妻子弗識。齊嬰亦反公扈之室,有其妻子,妻子亦弗識。二室因相與訟,求辨於扁鵲。扁鵲辨其所由,訟乃已。

匏巴鼓琴,而鳥舞魚躍,鄭師文聞之,棄家從師襄游。柱指鈞弦,三年不成章。師襄曰:「子可以歸矣。」師文舍其琴歎曰:「文非弦之不能鈞,非章之不能成。文所存者不在弦,所志者不在聲。內不得於心,外不應於器,故不敢發手而動弦。且小假之以觀其後。」无幾何,復見師襄。師襄曰:「子之琴何如?」師文曰:「得之矣。請嘗試之。」於是當春而叩商弦,以召南呂,涼風揔至,草木成實。及秋而叩角弦,以激夾鐘,溫風徐迴,草木發榮。當夏而叩羽弦,以召黃鐘,霜雪交下,川池暴沍。及冬而叩徵弦,以激蕤賓,陽光熾烈,堅冰立散。將終命宮而揔四弦。則景風翔,慶雲浮,甘露降,澧泉涌。師襄乃撫心高蹈曰:「微矣,子之彈也!雖師曠之清角,鄒衍之吹律,亡以加之。彼將挾琴執管而從子之後耳。」

薛譚學謳於秦青,未窮青之技,自謂盡之,遂辭歸。秦青弗止。餞於郊衢,撫節悲歌,聲振林木,響遏行雲。薛譚乃謝求反,終身不敢言歸。秦青顧謂其友曰:「昔韓娥東之齊,匱糧,過雍門,鬻歌假食。既去,而餘音繞梁欐,三日不絕,左右以其人弗去。過逆旅,逆旅人辱之。韓娥因曼聲哀哭,一里老幼。悲愁垂涕相對,三日不食。遽而追之。娥還復為曼聲長歌,一里長幼,喜躍抃舞,弗能自禁,忘向之悲也。乃厚賂發之。故雍門之人至今善歌哭,效娥之遺聲。」

伯牙善鼓琴,鍾子期善聽。伯牙鼓琴,志在登高山。鍾子期曰:「善哉!峨峨兮若泰山!」志在流水。鍾子期曰:「善哉!洋洋兮若江河!」伯牙所念,鍾子期必得之。伯牙游於泰山之陰,卒逢暴雨,止於巖下;心悲,乃援琴而鼓之。初為霖雨之操,更造崩山之音,曲每奏,鍾子期輒窮其趣。伯牙乃舍琴而歎曰:「善哉善哉!子之聽夫志,想象猶吾心也。吾於何逃聲哉?」

周穆王西巡狩,越崑崙,不至弇山。反還,未及中國,道有獻工人名偃師,穆王薦之,問曰:「若有何能?」偃師曰:「臣唯命所試。然臣已有所造,願王先觀之。」穆王曰:「日以俱來,吾與若俱觀之。」越日,偃師謁見王。王薦之曰:「若與偕來者何人邪?」對曰:「臣之所造能倡者。」穆王驚視之,趣步俯仰,信人也。巧夫,顉其頤,則歌合律;捧其手,則舞應節。千變萬化,惟意所適。王以為實人也。與盛姬內御並觀之。技將終,倡者瞬其目而招王之左右侍妾。王大怒,立欲誅偃師。偃師大懾,立剖散倡者以示王,皆傅會革、木、膠、漆、白、黑、丹、青之所為。王諦料之,內則肝、膽、心、肺、脾、腎、腸、胃,外則筋骨、支節、、皮毛、齒髮,皆假物也,而无不畢具者。合會復如初見。王試廢其心,則口不能言;廢其肝,則目不能視;廢其腎,則足不能步。穆王始悅而歎曰:「人之巧乃可與造化者同功乎?」詔貳車載之以歸。夫班輸之雲梯,墨翟之飛鳶,自謂能之極也。弟子東門賈、禽滑釐,聞偃師之巧,以告二子,二子終身不敢語藝,而時執規矩。

甘蠅,古之善射者,彀弓而獸伏鳥下。弟子名飛衛,學射於甘蠅,而巧過其師。紀昌者,又學射於飛衛。飛衛曰:「爾先學不瞬,而後可言射矣。」紀昌歸,偃臥其妻之機下,以目承牽挺。二年之後,雖錐末倒眥而不瞬也。以告飛衛。飛衛曰:「未也,必學視而後可。視小如大,視微如著,而後告我。」昌以氂懸虱於牖。南面而望之。旬日之閒,浸大也;三年之後,如車輪焉。以覩餘物,皆丘山也。乃以燕角之弧,朔蓬之簳,射之,貫虱之心,而懸不絕。以告飛衛。飛衛高蹈拊膺曰:「汝得之矣!「紀昌既盡衛之術,計天下之敵己者一人而已,乃謀殺飛衛。相遇於野,二人交射;中路矢鋒相觸,而墜於地,而塵不揚。飛衛之矢先窮。紀昌遺一矢,既發,飛衛以棘刺之端扞之,而无差焉。於是二子泣而投弓,相拜於塗,請為父子。剋臂以誓,不得告術於人。

造父之師曰泰豆氏。造父之始從習御也,執禮甚卑,泰豆三年不告。造父執禮愈謹乃告之曰:「古詩言:『良弓之子,必先為箕,良冶之子,必先為裘。』汝先觀吾趣。趣如吾,然後六轡可持,六馬可御。」造父曰:「唯命所從。」泰豆乃立木為塗,僅可容足;計步而置。履之而行。趣走往還,无跌失也。造父學之,三曰盡其巧。泰豆歎曰:「子何其敏也,得之捷乎?凡所御者,亦如此也。曩汝之行,得之於足,應之於心。推於御也,齊輯乎轡銜之際,而急緩乎脣吻之和;正度乎胸臆之中,而執節乎掌握之閒。內得於中心,而外合於馬志,是故能進退履繩,而旋曲中規矩,取道致遠,而氣力有餘,誠得其術也。得之於銜,應之於轡;得之於轡,應之於手;得之於手,應之於心。則不以目視,不以策驅;心閑體正,六轡不亂,而二十四蹄所投无差;廻旋進退,莫不中節。然後輿輪之外,可使无餘轍;馬蹄之外,可使无餘地。未嘗覺山谷之嶮。原隰之夷,視之一也。吾術窮矣。汝其識之!」

魏黑卵以暱嫌殺丘邴章。丘邴章之子來丹謀報父之讎。丹氣甚猛,形甚露,計粒而食,順風而趨。雖怒,不能稱兵以報之。恥假力於人,誓手劍以屠黑卵。黑卵悍志絕眾,力抗百夫,筋骨皮肉,非人類也。延頸承刀,披胸受矢,鋩鍔摧屈,而體无痕撻。負其材力,視來丹猶雛鷇也。來丹之友申他曰:「子怨黑卵至矣,黑卵之易子過矣,將奚謀焉?」來丹垂涕曰:「願子為我謀。」申他曰:『吾聞衛孔周其祖得殷帝之寶劍,一童子服之,卻三軍之眾,奚不請焉?「來丹遂適衛,見孔周,執僕御之禮請先納妻子,後言所欲。孔周曰:「吾有三劍,唯子所擇;皆不能殺人,且先言其狀。一曰含光,視之不可見,運之不知有。其所觸也,泯然无際,經物而物不覺。二曰承影,將旦昧爽之交,日夕昏明之際,北面而察之,淡淡焉若有物存,莫識其狀。其所觸也,竊竊然有聲,經物而物不疾也。三曰宵練,方晝則見影而不見光,方夜見光而不見形。其觸物也,騞然而過,隨過隨合,覺疾而不血刃焉。此三寶者,傳之十三世矣,而无施於事。匣而藏之,未嘗啟封,」來丹曰:「雖然,吾必請其下者。」孔周乃歸其妻子,與齋七日。晏陰之閒,跪而授其下劍,來丹再拜受之以歸。來丹遂執劍從黑卵。時黑卵之醉,偃於牖下,自頸至腰三斬之。黑卵不覺。來丹以黑卵之死,趣而退。遇黑卵之子於門,擊之三下,如投虛。黑卵之子方笑曰:「汝何蚩而三招予?」來丹知劍之不能殺人也,歎而歸。黑卵既醒,怒其妻曰:「醉而露我,使我嗌疾而腰急。」其子曰:「疇昔來丹之來。遇我於門,三招我,亦使我體疾而支彊,彼其厭我哉!」

周穆王大征西戎,西戎獻錕鋙之劍,火浣之布。其劍長尺有咫,練鋼赤刃,用之切玉如切泥焉。火浣之布,浣之必投於火;布則火色,垢則布色;出火而振之,皓然疑乎雪。皇子以為无此物,傳之者妄。蕭叔曰:「皇子果於自信,果於誣理哉!」


\end{pinyinscope}