\article{平帝紀}

\begin{pinyinscope}
孝平皇帝,元帝庶孫,中山孝王子也。母曰衛姬。年三歲嗣立為王。元壽二年六月,哀帝崩,太皇太后詔曰:「大司馬賢年少,不合眾心。其上印綬,罷。」賢即日自殺。新都侯王莽為大司馬,領尚書事。秋七月,遣車騎將軍王舜、大鴻臚左咸使持節迎中山王。辛卯,貶皇太后趙氏為孝成皇后,退居北宮,哀帝皇后傅氏退居桂宮。孔鄉侯傅晏、少府董恭等皆免官爵,徙合浦。九月辛酉,中山王即皇帝位,謁高廟,大赦天下。

帝年九歲,太皇太后臨朝,大司馬莽秉政,百官總己以聽於莽。詔曰:「夫赦令者,將與天下更始,誠欲令百姓改行絜己,全其性命也。性者有司多舉奏赦前事,累增罪過,誅陷亡辜,殆非重信慎刑,洒心自新之意也。及選舉者,其歷職更事有名之士,則以為難保,廢而弗舉,甚謬於赦小過舉賢材之義。對諸有臧及內惡未發而薦舉者,皆勿案驗。令士厲精鄉進,不以小疵妨大材。自今以來,有司無得陳赦前事置奏上。有不如詔書為虧恩,以不道論。定著令,布告天下,使明知之。」

元始元年春正月,越裳氏重譯獻白雉一,黑雉二,詔使三公以薦宗廟。

群臣奏言大司馬莽功德比周公,賜號安漢公,及太師孔光等皆益封。語在莽傳。賜天下民爵一級,吏在位二百石以上,一切滿秩如真。

立故東平王雲太子開明為王,故桃鄉頃侯子成都為中山王。封宣帝耳孫信等三十六人皆為列侯。太僕王惲等二十五人前議定陶傅太后尊號,守經法,不阿指從邪,右將軍孫建爪牙大臣,大鴻臚咸前正議不阿,後奉節使迎中山王,及宗正劉不惡、執金吾任岑、中郎將孔永、尚書令烑恂、沛郡太守石詡,皆以前與建策,東迎即位,奉事周密勤勞,賜爵關內侯,食邑各有差。賜帝徵即位前所過縣邑吏二千石以下至佐史爵,各有差。又令諸侯王、公、列侯、關內侯亡子而有孫若子同產子者,皆得以為嗣。公、列侯嗣子有罪,耐以上先請。宗室屬未盡而以罪絕者,復其屬。其為吏舉廉佐史,補四百石。天下吏比二千石以上年老致仕者,參分故祿,以一與之,終其身。遣諫大夫行三輔,舉籍吏民,以元壽二年倉卒時橫賦斂者,償其直。義陵民冢不妨殿中者勿發。天下吏舍亡得置什器儲偫。

二月,置羲和官,秩二千石;外史、閭師,秩六百石。班教化,禁淫祀,放鄭聲。

乙未,義陵寑神衣在柙中,丙申旦,衣在外床上,寑令以急變聞。用太牢祠。

夏五月丁巳朔,日有蝕之。大赦天下。公卿、將軍、中二千石舉敦厚能直言者各一人。

六月,使少傅左將軍豐賜帝母中山孝王姬璽書,拜為中山孝王后。賜帝舅衛寶、寶弟玄爵關內侯。賜帝女弟四人號皆曰君,食邑各二千戶。

封周公後公孫相如為褒魯侯,孔子後孔均為褒成侯,奉其祀。追諡孔子曰褒成宣尼公。

罷明光宮及三輔馳道。

天下女徒已論,歸家,顧山錢月三百。復貞婦,鄉一人。置少府海丞、果丞各一人;大司農部丞十三人,人部一州,勸農桑。

太皇太后省所食湯沐邑十縣,屬大司農,常別計其租入,以贍貧民。

秋九月,赦天下徒。

以中山苦陘縣為中山孝王后湯沐邑。

二年春,黃支國獻犀牛。

詔曰:「皇帝二名,通于器物,今更名,合於古制。使太師光奉太牢告祠高廟。」

夏四月,立代孝王玄孫之子如意為廣宗王,江都易王孫盱台侯宮為廣川王,廣川惠王曾孫倫為廣德王。封故大司馬博陸侯霍光從父昆弟曾孫陽、宣平侯張敖玄孫慶忌、絳侯周勃玄孫共、舞陽侯樊噲玄孫之子章皆為列侯,復爵。賜故曲周侯酈商等後玄孫酈明友等百一十三人爵關內侯,食邑各有差。

郡國大旱,蝗,青州尤甚,民流亡。安漢公、四輔、三公、卿大夫、吏民為百困乏獻其田宅者二百三十人,以口賦貧民。遣使者捕蝗,民捕蝗詣吏,以石砀受錢。天下民貲不滿二萬,及被災之郡不滿十萬,勿租稅。民疾疫者,舍空邸第,為置醫藥。賜死者一家六尸以上葬錢五千,四尸以上三千,二尸以上二千。罷安定呼池苑,以為安民縣,起官寺巿里,募徙貧民,縣次給食。至徙所,賜田宅什器,假與犁、牛、種、食。又起五里於長安城中,宅二百區,以居貧民。

秋,舉勇武有節明兵法,郡一人,詣公車。

九月戊申晦,日有蝕之。赦天下徒。

使謁者大司馬掾四十四人持節行邊兵。

遣執金吾候陳茂假以鉦鼓,募汝南、南陽勇敢吏士三百人,諭說江湖賊成重等二百餘人皆自出,送家在所收事。重徙雲陽,賜公田宅。

冬,中二千石舉治獄平,歲一人。

三年春,詔有司為皇帝納采安漢公莽女。語在莽傳。又詔光祿大夫劉歆等雜定婚禮。四輔、公卿、大夫、博士、郎、吏家屬皆以禮娶,親迎立軺併馬。

夏,安漢公奏車服制度,吏民養生、送終、嫁娶、奴婢、田宅、器械之品。立官稷及學官。郡國曰學,縣、道、邑、侯國曰校。校、學置經師一人。鄉曰庠,聚曰序。序、庠置孝經師一人。

陽陵任橫等自稱將軍,盜庫兵,攻官寺,出囚徒。大司徒掾督逐,皆伏辜。

安漢公世子宇與帝外家衛氏有謀。宇下獄死,誅衛氏。

四年春正月,郊祀高祖以配天,宗祀孝文以配上帝。

改殷紹嘉公曰宋公,周承休公曰鄭公。

詔曰:「蓋夫婦正則父子親,人倫定矣。前詔有司復貞婦,歸女徒,誠欲以防邪辟,全貞信。及眊悼之人刑罰所不加,聖王之所制也。惟苛暴吏多拘繫犯法者親屬,婦女老弱,搆怨傷化,百姓苦之。其明敕百寮,婦女非身犯法,及男子年八十以上七歲以下,家非坐不道,詔所名捕,它皆無得繫。其當驗者,即驗問。定著令。」

二月丁未,立皇后王氏,大赦天下。

遣太僕王惲等八人置副,假節,分行天下,覽觀風俗。

賜九卿已下至六百石、宗室有屬籍者爵,自五大夫以上各有差。賜天下民爵一級,鰥寡孤獨高年帛。

夏,皇后見于高廟。加安漢公號曰「宰衡」。賜公太夫人號曰功顯君。封公子安、臨皆為列侯。

安漢公奏立明堂、辟廱。尊孝宣廟為中宗,孝元廟為高宗,天子世世獻祭。

置西海郡,徙天下犯禁者處之。

梁王立有罪,自殺。

分京師置前煇光、後丞烈二郡。更公卿、大夫、八十一元士官名位次及十二州名。分界郡國所屬,罷置改易,天下多事,吏不能紀。

冬,大風吹長安城東門屋瓦且盡。

五年春正月,祫祭明堂。諸侯王二十八人、列侯百二十人、宗室子九百餘人徵助祭。禮畢,皆益戶,賜爵及金帛,增秩補吏,各有差。

詔曰:「蓋聞帝王以德撫民,其次親親以相及也。昔堯睦九族,舜惇敘之。朕以皇帝幼年,且統國政,惟宗室子皆太祖高皇帝子孫及兄弟吳頃、楚元之後,漢元至今,十有餘萬人,雖有王侯之屬,莫能相糾,或陷入刑罪,教訓不至之咎也。傳不云乎?『君子篤於親,則民興於仁。』其為宗室自太上皇以來族親,各以世氏,郡國置宗師以糾之,致教訓焉。二千石選有德義者以為宗師。考察不從教令有冤失職者,宗師得因郵亭書言宗伯,請以聞。常以歲正月賜宗師帛各十匹。」

羲和劉歆等四人使治明堂、辟廱,令漢與文王靈臺、周公作洛同符。太僕王惲等八人使行風俗,宣明德化,萬國齊同。皆封為列侯。

徵天下通知逸經、古記、天文、曆算、鍾律、小學、史篇、方術、本草及以五經、論語、孝經、爾雅教授者,在所為駕一封軺傳,遣詣京師。至者數千人。

閏月,立梁孝王玄孫之耳孫音為王。

冬十二月丙午,帝崩于未央宮。大赦天下。有司議曰:「

禮,臣不殤君。皇帝年十有四歲,宜以禮斂,加元服。」奏可。葬康陵。詔曰:「皇帝仁惠,無不顧哀,每疾一發,氣輒上逆,害於言語,故不及有遺詔。其出媵妾,皆歸家得嫁,如孝文時故事。」

贊曰:孝平之世,政自莽出,褒善顯功,以自尊盛。觀其文辭,方外百蠻,亡思不服;休徵嘉應,頌聲並作。至乎變異見於上,民怨於下,莽亦不能文也。


\end{pinyinscope}