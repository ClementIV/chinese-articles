\article{景十三王傳}

\begin{pinyinscope}
孝景皇帝十四男。王皇后生孝武皇帝。栗姬生臨江閔王榮、河間獻王德、臨江哀王閼。程姬生魯共王餘、江都易王非、膠西于王端。賈夫人生趙敬肅王彭祖、中山靖王勝。唐姬生長沙定王發。王夫人生廣川惠王越、膠東康王寄、清河哀王乘、常山憲王舜。

河間獻王德以孝景前二年立,修學好古,實事求是。從民得善書,必為好寫與之,留其真,加金帛賜以招之。繇是四方道術之人不遠千里,或有先祖舊書,多奉以奏獻王者,故得書多,與漢朝等。是時,淮南王安亦好書,所招致率多浮辯。獻王所得書皆古文先秦舊書,周官、尚書、禮、禮記、孟子、老子之屬,皆經傳說記,七十子之徒所論。其學舉六藝,立毛氏詩、左氏春秋博士。修禮樂,被服儒術,造次必於儒者。山東諸儒者從而游。

武帝時,獻王來朝,獻雅樂,對三雍宮及詔策所問三十餘事。其對推道術而言,得事之中,文約指明。

立二十六年薨。中尉常麗以聞,曰:「王身端行治,溫仁恭儉,篤敬愛下,明知深察,惠于鰥寡。」大行令奏:「諡法曰『聰明睿知曰獻』,宜諡曰獻王。」子共王不害嗣,四年薨。子剛王堪嗣,十二年薨。子頃王授嗣,十七年薨。子孝王慶嗣,四十三年薨。子元嗣。

元取故廣陵厲王、厲王太子及中山懷王故姬廉等以為姬。甘露中,冀州刺史敞奏元,事下廷尉,逮召廉等。元迫脅凡七人,令自殺。有司奏請誅元,有詔削二縣,萬一千戶。後元怒少史留貴,留貴踰垣出,欲告元,元使人殺留貴母。有司奏元殘賊不改,不可君國子民。廢勿王,處漢中房陵。居數年,坐與妻若共乘朱輪車,怒若,又笞擊,令自髡。漢中太守請治元,病死。立十七年,國除。

絕五歲,成帝建始元年,復立元弟上郡庫令良,是為河間惠王。良修獻王之行,母太后薨,服喪如禮。哀帝下詔褒揚曰:「河間王良,喪太后三年,為宗室儀表,其益封萬戶。」二十七年薨。子尚嗣,王莽時絕。

臨江哀王閼以孝景前二年立,三年薨。無子,國除為郡。

臨江閔王榮以孝景前四年為皇太子,四歲廢為臨江王。三歲,坐侵廟壖地為宮,上徵榮。榮行,祖於江陵北門,既上車,軸折車廢。江陵父老流涕竊言曰:「吾王不反矣!」榮至,詣中尉府對簿。中尉郅都簿責訊王,王恐,自殺。葬藍田,燕數萬銜土置冢上。百姓憐之。

榮最長,亡子,國除。地入于漢,為南郡。

魯恭王餘以孝景前二年立為淮陽王。吳楚反破後,以孝景前三年徙王魯。好治宮室苑囿狗馬,季年好音,不喜辭。為人口吃難言。

二十八年薨。子安王光嗣,初好音樂輿馬,晚節遴,唯恐不足於財。四十年薨。子孝王慶忌嗣,三十七年薨。子頃王勁嗣,二十八年薨。子文王晳嗣,十八年薨,亡子,國除。哀帝建平三年,復立頃王子晳弟郚鄉侯閔為王。王莽時絕。

恭王初好治宮室,壞孔子舊宅以廣其宮,聞鐘磬琴瑟之聲,遂不敢復壞,於其壁中得古文經傳。

江都易王非以孝景前二年立為汝南王。吳楚反時,非年十五,有材氣,上書自請擊吳。景帝賜非將軍印,擊吳。吳已破,徙王江都,治故吳國,以軍功賜天子旗。元光中,匈奴大入漢邊,非上書願擊匈奴,上不許。非好氣力,治宮館,招四方豪桀,驕奢甚。二十七年薨,子建嗣。

建為太子時,邯鄲人梁蚡持女欲獻之易王,建聞其美,私呼之,因留不出。蚡宣言曰:「子乃與其公爭妻!」建使人殺蚡。蚡家上書,下廷尉考,會赦,不治。易王薨未葬,建居服舍,召易王所愛美人淖姬等凡十人與姦。建女弟徵臣為蓋侯子婦,以易王喪來歸,建復與姦。建異母弟定國為淮陽侯,易王最小子也,其母幸立之,具知建事,行錢使男子荼恬上書告建淫亂,不當為後。事下廷尉,廷尉治恬受人錢財為上書,論棄市。建罪不治。後數使使至長安迎徵臣,魯恭王太后聞之,遺徵臣書曰:「

國中口語籍籍,慎無復至江都。」後建使謁者吉請問共太后,太后泣謂吉:「歸以吾言謂而王,王前事漫漫,今當自謹,獨不聞燕齊事乎?言吾為而王泣也。」吉歸,致共太后語,建大怒,擊吉,斥之。

建游章臺宮,令四女子乘小船,建以足蹈覆其船,四人皆溺,二人死。後游雷波,天大風,建使郎二人乘小船入波中。船覆,兩郎溺,攀船,乍見乍沒。建臨觀大笑,令皆死。

宮人姬八子有過者,輒令臝立擊鼓,或置樹上,久者三十日乃得衣;或髡鉗以鈆杵舂,不中程,輒掠;或縱狼令齧殺之,建觀而大笑;或閉不食,令餓死。凡殺不辜三十五人。建欲令人與禽獸交而生子,彊令宮人臝而四據,與羝羊及狗交。

」

專為淫虐,自知罪多,國中多欲告言者,建恐誅,心內不安,與其后成光共使越婢下神,祝詛上。與郎中令等語怨望:「漢廷使者即復來覆我,我決不獨死!」

建亦頗聞淮南、衡山陰謀,恐一日發,為所并,遂作兵器。號王后父胡應為將軍。中大夫疾有材力,善騎射,號曰靈武君。作治黃屋蓋;刻皇帝璽,鑄將軍、都尉金銀印;作漢使節二十,綬千餘;具置軍官品員,及拜爵封侯之賞;具天下之輿地及軍陳圖。遣人通越繇王閩侯,遺以錦帛奇珍,繇王閩侯亦遺建荃、葛、珠璣、犀甲、翠羽、蝯熊奇獸,數通使往來,約有急相助。及淮南事發,治黨與,頗連及建,建使人多推金錢絕其獄。

後復謂近臣曰:「我為王,詔獄歲至,生又無驩怡日,壯士不坐死,欲為人所不能為耳。」建時佩其父所賜將軍印,載天子旗出。積數歲,事發覺,漢遣丞相長史與江都相雜案,索得兵器璽綬節反具,有司請捕誅建。制曰:「與列侯吏二千石博士議。」議皆曰:「建失臣子道,積久,輒蒙不忍,遂謀反逆。所行無道,雖桀紂惡不至於此。天誅所不赦,當以謀反法誅。」有詔宗正、廷尉即問建。建自殺,后成光等皆棄市。六年國除,地入于漢,為廣陵郡。

絕百二十一年,平帝時新都侯王莽秉政,興滅繼絕,立建弟盱眙侯子宮為廣陵王,奉易王後。莽篡,國絕。

膠西于王端,孝景前三年立。為人賊盭,又陰痿,一近婦人,病數月。有所愛幸少年,以為郎。郎與後宮亂,端禽滅之,及殺其子母。數犯法,漢公卿數請誅端,天子弗忍,而端所為滋甚。有司比再請,削其國,去太半。端心慍,遂為無訾省。府庫壞漏,盡腐財物,以鉅萬計,終不得收徙。令吏毋得收租賦。端皆去衛,封其宮門,從一門出入。數變名姓,為布衣,之它國。

相二千石至者,奉漢法以治,端輒求其罪告之,亡罪者詐藥殺之。所以設詐究變,彊足以距諫,知足以飾非。相二千石從王治,則漢繩以法。故膠西小國,而所殺傷二千石甚眾。

立四十七年薨,無子,國除。地入于漢,為膠西郡。

趙敬肅王彭祖以孝景前二年立為廣川王。趙王遂反破後,徙王趙。彭祖為人巧佞,卑諂足共,而心刻深,好法律,持詭辯以中人。多內寵姬及子孫。相二千石欲奉漢法以治,則害於王家。是以每相二千石至,彭祖衣帛布單衣,自行迎除舍,多設疑事以詐動之,得二千石失言,中忌諱,輒書之。二千石欲治者,則以此迫劫;不聽,乃上書告之,及汙以姦利事。彭祖立六十餘年,相二千石無能滿二歲,輒以罪去,大者死,小者刑。以故二千石莫敢治,而趙王擅權。使使即縣為賈人榷會,入多於國租稅。以是趙王家多金錢,然所賜姬諸子,亦盡之矣。

彭祖不好治宮室禨祥,好為吏。上書願督國中盜賊。常夜從走卒行徼邯鄲中。諸使過客,以彭祖險陂,莫敢留邯鄲。

久之,太子丹與其女弟及同產姊姦。江充告丹淫亂,又使人椎埋攻剽,為姦甚眾。武帝遣使者發吏卒捕丹,下魏郡詔獄,治罪至死。彭祖上書冤訟丹,願從國中勇敢擊匈奴,贖丹罪,上不許。久之,竟赦出。後彭祖入朝,因帝姊平陽隆慮公主,求復立丹為太子,上不許。

彭祖取江都易王寵姬,王建所姦淖姬者,甚愛之,生一男,號淖子。彭祖以征和元年薨,諡敬肅王。彭祖薨時,淖姬兄為漢宦者,上召問:「淖子何如?」對曰:「為人多欲。」上曰:「多欲不宜君國子民。」問武始侯昌,曰:「無咎無譽。」上曰:「如是可矣。」遣使者立昌,是為頃王,十九年薨。子懷王尊嗣,五年薨。無子,絕二歲。宣帝立尊弟高,是為哀王,數月薨。子共王充嗣,五十六年薨。子隱嗣,王莽時絕。

初,武帝復以親親故,立敬肅王小子偃為平干王,是為頃王,十一年薨。子繆王元嗣,二十五年薨。大鴻臚禹奏:「元前以刃賊殺奴婢,子男殺謁者,為刺史所舉奏,罪名明白。病先令,令能為樂奴婢從死,迫脅自殺者凡十六人,暴虐不道。故春秋之義,誅君之子不宜立。元雖未伏誅,不宜立嗣。」奏可,國除。

中山靖王勝以孝景前三年立。武帝初即位,大臣懲吳楚七國行事,議者勿冤晁錯之策,皆以諸侯連城數十,泰強,欲稍侵削,數奏暴其過惡。諸侯王自以骨肉至親,先帝所以廣封連城,犬牙相錯者,為盤石宗也。今或無罪,為臣下所侵辱,有司吹毛求疵,笞服其臣,使證其君,多自以侵冤。

建元三年,代王登、長沙王發、中山王勝、濟川王明來朝,天子置酒,勝聞樂聲而泣。問其故,勝對曰:

臣聞悲者不可為絫欷,思者不可為歎息。故高漸離擊筑易水之上,荊軻為之低而不食;雍門子壹微吟,孟嘗君為之於邑。今臣心結日久,每聞幼眇之聲,不知涕泣之橫集也。

夫眾喣漂山,聚蚊成雷,朋黨執虎,十夫橈椎。是以文王拘於牖里,孔子阨於陳、蔡。此乃烝庶之成風,增積之生害也。臣身遠與寡,莫為之先,眾口鑠金,積毀銷骨,叢輕折軸,羽翮飛肉,紛驚逢羅,潸然出涕。

臣聞白日曬光,幽隱皆照;明月曜夜,蚊虻宵見。然雲蒸列布,杳冥晝昏;塵埃抪覆,昧不泰山。何則?物有蔽之也。今臣雍閼不得聞,讒言之徒蹒生道遼路遠,曾莫為臣聞,臣竊自悲也。

臣聞社鼷不灌,屋鼠不熏。何則?所託者然也。臣雖薄也,得蒙肺附;位雖卑也,得為東藩,屬又稱兄。今群臣非有葭莩之親,鴻毛之重,群居黨議,朋友相為,使夫宗室擯卻,骨肉冰釋。斯伯奇所以流離,比干所以橫分也。《詩》云「我心憂傷,惄焉如擣;假寐永歎,唯憂用老;心之憂矣,疢如疾首」,臣之謂也。

具以吏所侵聞。於是上乃厚諸侯之禮,省有司所奏諸侯事,加親親之恩焉。其後更用主父偃謀,令諸侯以私恩自裂地分其子弟,而漢為定制封號,輒別屬漢郡。漢有厚恩,而諸侯地稍自分析弱小云。

勝為人樂酒好內,有子百二十餘人。常與趙王彭祖相非曰:「兄為王,專代吏治事。王者當日聽音樂,御聲色。」趙王亦曰:「中山王但奢淫,不佐天子拊循百姓,何以稱為藩臣!」

四十三年薨。子哀王昌嗣,一年薨。子康王昆侈嗣,二十一年薨。子頃王輔嗣,四年薨。子憲王福嗣,十七年薨。子懷王循嗣,十五年薨,無子,絕四十五歲。成帝鴻嘉二年復立憲王弟孫利鄉侯子雲客,是為廣德夷王。三年薨,無子,絕十四歲。哀帝復立雲客弟廣漢為廣平王。薨,無後。平帝元始二年復立廣川惠王曾孫倫為廣德王,奉靖王後。王莽時絕。

長沙定王發,母唐姬,故程姬侍者。景帝召程姬,程姬有所避,不願進,而飾侍者唐兒使夜進。上醉,不知,以為程姬而幸之,遂有身。已乃覺非程姬也。及生子,因名曰發。以孝景前二年立。以其母微無寵,故王卑溼貧國。

二十八年薨。子戴王庸嗣,二十七年薨。子頃王鮒鮈嗣,十七年薨。子剌王建德嗣,宣帝時坐獵縱火燔民九十六家,殺二人,又以縣官事怨內史,教人誣告以棄市罪,削八縣,罷中尉官。三十四年薨。子煬王旦嗣,二年薨。無子,絕歲餘。元帝初元三年復立旦弟宗,是為孝王,五年薨。子魯人嗣,王莽時絕。

廣川惠王越以孝景中二年立,十三年薨。子繆王齊嗣,四十四年薨。初齊有幸臣乘距,已而有罪,欲誅距。距亡,齊因禽其宗族。距怨王,乃上書告齊與同產姦。是後,齊數告言漢公卿及幸臣所忠等,又告中尉蔡彭祖捕子明,罵曰:「吾盡汝種矣!」有司案驗,不如王言,劾齊誣罔,大不敬,請繫治。齊恐,上書願與廣川勇士奮擊匈奴,上許之。未發,病薨。有司請除國,奏可。

後數月,下詔曰:「廣川惠王於朕為兄,朕不忍絕其宗廟,其以惠王孫去為廣川王。」去即繆王齊太子也,師受易、論語、孝經皆通,好文辭方技博弈倡優。其殿門有成慶畫,短衣大恊長劍,去好之,作七尺五寸劍,被服皆效焉。有幸姬王昭平、王地餘,許以為后。去嘗疾,姬陽成昭信侍視甚謹,更愛之。去與地餘戲,得袖中刀,笞問狀,服欲與昭平共殺昭信。笞問昭平,不服,以鐵鍼鍼之,彊服。乃會諸姬,去以劍自擊地餘,令昭信擊昭平,皆死。昭信曰:「兩姬婢且泄口。」復絞殺從婢三人。後昭信病,夢見昭平等以狀告去。去曰:「虜乃復見畏我!獨可燔燒耳。」掘出尸,皆燒為灰。

後去立昭信為后;幸姬陶望卿為脩靡夫人,主繒帛;崔脩成為明貞夫人,主永巷。昭信復譖望卿曰:「與我無禮,衣服常鮮於我,盡取善繒饨諸宮人。」去曰:「若數惡望卿,不能減我愛;設聞其淫,我亨之矣。」後昭信謂去曰:「前畫工畫望卿舍,望卿袒裼傅粉其傍。又數出入南戶窺郎吏,疑有姦。」去曰:「善司之。」以故益不愛望卿。後與昭信等飲,諸姬皆侍,去為望卿作歌曰:「背尊章,嫖以忽,謀屈奇,起自絕。行周流,自生患,諒非望,今誰怨!」使美人相和歌之。去曰:「

是中當有自知者。」昭信知去已怒,即誣言望卿歷指郎吏臥處,具知其主名,又言郎中令錦被,疑有姦。去即與昭信從諸姬至望卿所,臝其身,更擊之。令諸姬各持燒鐵共灼望卿。望卿走,自投井死。昭信出之,椓杙其陰中,割其鼻脣,斷其舌。謂去曰:「前殺昭平,反來畏我,今欲靡爛望卿,使不能神。」與去共支解,置大鑊中,取桃灰毒藥并煮之,召諸姬皆臨觀,連日夜靡盡。復共殺其女弟都。

後去數召姬榮愛與飲,昭信復譖之,曰:「榮姬視瞻,意態不善,疑有私。」時愛為去刺方領繡,去取燒之。愛恐,自投井。出之未死,笞問愛,自誣與醫姦。去縛繫柱,燒刀灼潰兩目,生割兩股,銷鈆灌其口中。愛死,支解以棘埋之。諸幸於去者,昭信輒譖殺之,凡十四人,皆埋太后所居長壽宮中。宮人畏之,莫敢復迕。

昭信欲擅愛,曰:「王使明貞夫人主諸姬,淫亂難禁。請閉諸姬舍門,無令出敖。」使其大婢為僕射,主永巷,盡封閉諸舍,上籥於后,非大置酒召,不得見。去憐之,為作歌曰:「愁莫愁,居無聊。心重結,意不舒。內茀鬱,憂哀積。上不見天,生何益!日崔隤,時不再。願棄軀,死無悔。」令昭信聲鼓為節,以教諸姬歌之,歌罷輒歸永巷,封門。獨昭信兄子初為乘華夫人,得朝夕見。昭信與去從十餘奴博飲游敖。

初去年十四五,事師受易,師數諫正去,去益大,逐之。內史請以為掾,師數令內史禁切王家。去使奴殺師父子,不發覺。後去數置酒,令倡俳臝戲坐中以為樂。相彊劾繫倡,闌入殿門,奏狀。事下考案,倡辭,本為王教脩靡夫人望卿弟都歌舞。使者召望卿、都,去對皆淫亂自殺。會赦不治。望卿前亨煮,即取他死人與都死并付其母。母曰:「都是,望卿非也。」數號哭求死,昭信令奴殺之。奴得,辭服。本始三年,相內史奏狀,具言赦前所犯。天子遣大鴻臚、丞相長史、御史丞、廷尉正雜治鉅鹿詔獄,奏請逮捕去及后昭信。制曰:「王后昭信、諸姬奴婢證者皆下獄。」辭服。有司復請誅王。制曰:「與列侯、中二千石、二千石、博士議。」議者皆以為去悖虐,聽后昭信讒言,燔燒亨煮,生割剝人,距師之諫,殺其父子。凡殺無辜十六人,至一家母子三人,逆節絕理。其十五人在赦前,大惡仍重,當伏顯戮以示眾。制曰:「朕不忍致王於法,議其罰。」有司請廢勿王,與妻子徙上庸。奏可。與湯沐邑百戶。去道自殺,昭信棄市。

立二十二年,國除。後四歲,宣帝地節四年,復立去兄文,是為戴王。文素正直,數諫王去,故上立焉,二年薨。子海陽嗣,十五年,坐畫屋為男女臝交接,置酒請諸父姊妺飲,令仰視畫;又海陽女弟為人妻,而使與幸臣姦;又與從弟調等謀殺一家三人,已殺。甘露四年坐廢,徙房陵,國除。後十五年,平帝元始二年,復立戴王弟襄隄侯子瘉為廣德王,奉惠王後,二年薨。子赤嗣,王莽時絕。

膠東康王寄以孝景中二年立,二十八年薨。淮南王謀反時,寄微聞其事,私作兵車鏃矢,戰守備,備淮南之起。及吏治淮南事,辭出之。寄於上最親,意自傷,發病而死,不敢置後。於是上聞寄有長子賢,母無寵,少子慶,母愛幸,寄常欲立之,為非次,因有過,遂無所言。上憐之,立賢為膠東王,奉康王祀,而封慶為六安王,王故衡山地。膠東王賢立十五年薨,諡為哀王。子戴王通平嗣,二十四年薨。子頃王音嗣,五十四年薨。子共王授嗣,十四年薨。子殷嗣,王莽時絕。

六安共王慶立三十八年薨。子夷王祿嗣,十年薨。子繆王定嗣,二十二年薨。子頃王光嗣,二十七年薨。子育嗣,王莽時絕。

清河哀王乘以孝景中三年立,十二年薨。無子,國除。

常山憲王舜以孝景中五年立。舜,帝少子,驕淫,數犯禁,上常寬之。三十三年薨,子勃嗣為王。

初,憲王有不愛姬生長男梲,梲以母無寵故,亦不得幸於王。王后脩生太子勃。王內多,所幸姬生子平、子商,王后稀得幸。及憲王疾甚,諸幸姬侍病,王后以妒媢不常在,輒歸舍。醫進藥,太子勃不自嘗藥,又不宿留侍疾。及王薨,王后、太子乃至。憲王雅不以梲為子數,不分與財物。郎或說太子、王后,令分梲財,皆不聽。太子代立,又不收恤梲。梲怨王后及太子。漢使者視憲王喪,梲自言憲王病時,王后、太子不侍,及薨,六日出舍,太子勃私姦、飲酒、博戲、擊筑,與女子載馳,環城過市,入獄視囚。天子遣大行騫驗問,逮諸證者,王又匿之。吏求捕,勃使人致擊笞掠,擅出漢所疑囚。有司請誅勃及憲王后脩。上曰:「脩素無行,使梲陷之罪。勃無良師傅,不忍致誅。」有司請廢勿王,徙王勃以家屬處房陵,上許之。

勃王數月,廢,國除。月餘,天子為最親,詔有司曰:「常山憲王早夭,后妾不和,適孽誣爭,陷于不誼以滅國,朕甚閔焉。其封憲王子平三萬戶,為真定王;子商三萬戶,為泗水王。」頃王平立二十五年薨。子烈王偃嗣,十八年薨。子孝王由嗣,二十二年薨。子安王雍嗣,二十六年薨。子共王普嗣,十五年薨。子陽嗣,王莽時絕。

泗水思王商立十年薨。子哀王安世嗣,一年薨,無子。於是武帝憐泗水王絕,復立安世弟賀,是為戴王。立二十二年薨,有遺腹子煖,相內史不以聞。太后上書,昭帝閔之,抵相內史罪,立煖,是為勤王。立三十九年薨。子戾王駿嗣,三十一年薨。子靖嗣,王莽時絕。

贊曰:昔魯哀公有言:「寡人生於深宮之中,長於婦人之手,未嘗知憂,未嘗知懼。」信哉斯言也!雖欲不危亡,不可得已。是故古人以宴安為鴆毒,亡德而富貴,謂之不幸。漢興,至于孝平,諸侯王以百數,率多驕淫失道。何則?沈溺放恣之中,居勢使然也。自凡人猶繫于習俗,而況哀公之倫乎!夫唯大雅,卓爾不群,河間獻王近之矣。


\end{pinyinscope}