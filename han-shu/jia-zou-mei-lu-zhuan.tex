\article{賈鄒枚路傳}

\begin{pinyinscope}
賈山,潁川人也。祖父祛,故魏王時博士弟子也。山受學祛,所言涉獵書記,不能為醇儒。嘗給事潁陰侯為騎。

孝文時,言治亂之道,借秦為諭,名曰至言。其辭曰:

臣聞為人臣者,盡忠竭愚,以直諫主,不避死亡之誅者,臣山是也。臣不敢以久遠諭,願借秦以為諭,唯陛下少加意焉。

夫布衣韋帶之士,修身於內,成名於外,而使後世不絕息。至秦則不然。貴為天子,富有天下,賦斂重數,百姓任罷,赭衣半道,群盜滿山,使天下之人戴目而視,傾耳而聽。一夫大謼,天下嚮應者,陳勝是也。秦非徒如此也,起咸陽而西至雍,離宮三百,鍾鼓帷帳,不移而具。又為阿房之殿,殿高數十仞,東西五里,南北千步,從車羅騎,四馬騖馳,旌旗不橈。為宮室之麗至於此,使其後世曾不得聚廬而託處焉。為馳道於天下,東窮燕齊,南極吳楚,江湖之上,瀕海之觀畢至。道廣五十步,三丈而樹,厚築其外,隱以金椎,樹以青松。為馳道之麗至於此,使其後世曾不得邪徑而託足焉。死葬乎驪山,吏徒數十萬人,曠日十年。下徹三泉合采金石,冶銅錮其內,桼塗其外,被以珠玉,飾以翡翠,中成觀游,上成山林。為葬薶之侈至於此,使其後世曾不得蓬顆蔽冢而託葬焉。秦以熊羆之力,虎狼之心,蠶食諸侯,并吞海內,而不篤禮義,故天殃已加矣。臣昧死以聞,願陛下少留意而詳擇其中。

臣聞忠臣之事君也,言切直則不用而身危,不切直則不可以明道,故切直之言,明主所欲急聞,忠臣之所以蒙死而竭知也。地之磽者,雖有善種,不能生焉;江皋河瀕,雖有惡種,無不猥大。昔者夏商之季世,雖關龍逢、箕子、比干之賢,身死亡而道不用。文王之時,豪俊之士皆得竭其智,芻蕘採薪之人皆得盡其力,此周之所以興也。故地之美者善養禾,君之仁者善養士。雷霆之所擊,無不摧折者;萬鈞之所壓,無不糜滅者。今人主之威,非特雷霆也;勢重,非特萬鈞也。開道而求諫,和顏色而受之,用其言而顯其身,士猶恐懼而不敢自盡,又乃況於縱欲恣行暴虐,惡聞其過乎!震之以威,壓之以重,則雖有堯舜之智,孟賁之勇,豈有不摧折者哉?如此,則人主不得聞其過失矣;弗聞,則社稷危矣。古者聖王之制,史在前書過失,工誦箴諫,瞽誦詩諫,公卿比諫,士傳言諫過,庶人謗於道,商旅議於市,然後君得聞其過失也。聞其過失而改之,見義而從之,所以永有天下也。天子之尊,四海之內,其義莫不為臣。然而養三老於大學,親執醬而餽,執爵而酳,祝医在前,祝鯁在後,公卿奉杖,大夫進履,舉賢以自輔弼,求修正之士使直諫。故以天子之尊,尊養三老,視孝也;立輔弼之臣者,恐驕也;置直諫之士者,恐不得聞其過也;學問至於芻蕘者,求善無饜也;商人庶人誹謗己而改之,從善無不聽也。

昔者,秦政力并萬國,富有天下,破六國以為郡縣,築長城以為關塞。秦地之固,大小之勢,輕重之權,其與一家之富,一夫之彊,胡可勝計也!然而兵破於陳涉,地奪於劉氏者,何也?秦王貪狼暴虐,殘賊天下,窮困萬民,以適其欲也。昔者,周蓋千八百國,以九州之民養千八百國之君,用民之力不過歲三日,什一而籍,君有餘財,民有餘力,而頌聲作。秦皇帝以千八百國之民自養,力罷不能勝其役,財盡不能勝其求。一君之身耳,所以自養者馳騁弋獵之娛,天下弗能供也。勞罷者不得休息,飢寒者不得衣食,亡罪而死刑者無所告訴,人與之為怨,家與之為讎,故天下壞也。秦皇帝身在之時,天下已壞矣,而弗自知也。秦皇帝東巡狩,至會稽、琅邪,刻石著其功,自以為過堯舜統;縣石鑄鍾虡,篩土築阿房之宮,自以為萬世有天下也。古者聖王作諡,三四十世耳,雖堯舜禹湯文武絫世廣德以為子孫基業,無過二三十世者也。秦皇帝曰死而以諡法,是父子名號有時相襲也,以一至萬,則世世不相復也,故死而號曰始皇帝,其次曰二世皇帝者,欲以一至萬也。秦皇帝計其功德,度其後嗣,世世無窮,然身死纔數月耳,天下四面而攻之,宗廟滅絕矣。

秦皇帝居滅絕之中而不自知者何也?天下莫敢告也。其所以莫敢告者何也?亡養老之義,亡輔弼之臣,亡進諫之士,縱恣行誅,退誹謗之人,殺直諫之士,是以道諛媮合苟容,比其德則賢於堯舜,課其功則賢於湯武,天下已潰而莫之告也。《詩》曰:「匪言不能,胡此畏忌,聽言則對,譖言則退。」此之謂也。又曰:「濟濟多士,文王以寧。」天下未嘗亡士也,然而文王獨言以寧者何也?文王好仁則仁興,得士而敬之則士用,用之有禮義。

故不致其愛敬,則不能盡其心;不能盡其心,則不能盡其力;不能盡其力,則不能成其功。故古之賢君於其臣也,尊其爵祿而親之;疾則臨視之亡數,死則往弔哭之,臨其小斂大斂,已棺塗而後為之服錫衰麻絰,而三臨其喪;未斂不飲酒食肉,未葬不舉樂,當宗廟之祭而死,為之廢樂。故古之君人者於其臣也,可謂盡禮矣;服法服,端容貌,正顏色,然後見之。故臣下莫敢不竭力盡死以報其上,功德立於後世,而令聞不忘也。

今陛下念思祖考,術追厥功,圖所以昭光洪業休德,使天下舉賢良方正之士,天下皆訢訢焉,曰將興堯舜之道,三王之功矣。天下之士莫不精白以承休德。今方正之士皆在朝廷矣,又選其賢者使為常侍諸吏,與之馳敺射獵,一日再三出。臣恐朝廷之解弛,百官之墮於事也,諸侯聞之,又必怠於政矣。

陛下即位,親自勉以厚天下,損食膳,不聽樂,減外徭衛卒,止歲貢;省廄馬以賦縣傳,去諸苑以賦農夫,出帛十萬餘匹以振貧民;禮高年,九十者一子不事,八十者二算不事;賜天下男子爵,大臣皆至公卿;發御府金賜大臣宗族,亡不被澤者;赦罪人,憐其亡髮,賜之巾,憐其衣赭書其背,父子兄弟相見也而賜之衣。平獄緩刑,天下莫不說喜。是以元年膏雨降,五穀登,此天之所以相陛下也。刑輕於它時而犯法者寡,衣食多於前年而盜賊少,此天下之所以順陛下也。臣聞山東吏布詔令,民雖老羸缮疾,扶杖而往聽之,願少須臾毋死,思見德化之成也。今功業方就,名聞方昭,四方鄉風,今從豪俊之臣,方正之士,直與之日日獵射,擊兔伐狐,以傷大業,絕天下之望,臣竊悼之。《詩》曰:「靡不有初,鮮克有終。」臣不勝大願,願少衰射獵,以夏歲二月,定明堂,造太學,修先王之道。風行俗成,萬世之基定,然後唯陛下所幸耳。古者大臣不媟,故君子不常見其齊嚴之色,肅敬之容。大臣不得與宴游,方正修潔之士不得從射獵,使皆務其方以高其節,則群臣莫敢不正身修行,盡心以稱大禮。如此,則陛下之道尊敬,功業施於四海,垂於萬世子孫矣。誠不如此,則行日壞而榮日滅矣。夫士修之於家,而壞之於天子之廷,臣竊愍之。陛下與眾臣宴游,與大臣方正朝廷論議。夫游不失樂,朝不失禮,議不失計,軌事之大者也。

其後文帝除鑄錢令,山復上書諫,以為變先帝法,非是。又訟淮南王無大罪,宜急令反國。又言柴唐子為不善,足以戒。章下詰責,對以為「錢者,亡用器也,而可以易富貴。富貴者,人主之操柄也,令民為之,是與人主共操柄,不可長也。」其言多激切,善指事意,然終不加罰,所以廣諫爭之路也。其後復禁鑄錢云。

鄒陽,齊人也。漢興,諸侯王皆自治民聘賢。吳王濞招致四方游士,陽與吳嚴忌、枚乘等俱仕吳,皆以文辯著名。久之,吳王以太子事怨望,稱疾不朝,陰有邪謀,陽奏書諫。為其事尚隱,惡指斥言,故先引秦為諭,因道胡、越、齊、趙、淮南之難,然後乃致其意。其辭曰:

臣聞秦倚曲臺之宮,懸衡天下,畫地而不犯,兵加胡越;至其晚節末路,張耳、陳勝連從兵之據,以叩函谷,咸陽遂危。何則?列郡不相親,萬室不相救也。今胡數涉北河之外,上覆飛鳥,下不見伏菟,鬥城不休,救兵不止,死者相隨,輦車相屬,轉粟流輸,千里不絕。何則?彊趙責於河間,六齊望於惠后,城陽顧於盧博,三淮南之心思墳墓。大王不憂,臣恐救兵之不專,胡馬遂進窺於邯鄲,越水長沙,還舟青陽。雖使梁并淮陽之兵,下淮東,越廣陵,以遏越人之糧,漢亦折西河而下,北守漳水,以輔大國,胡亦益進,越亦益深。此臣之所為大王患也。

臣聞交龍襄首奮翼,則浮雲出流,霧雨咸集。聖王底節修德,則游談之士歸義思名。今臣盡智畢議,易精極慮,則無國不可奸;飾固陋之心,則何王之門不可曳長裾乎?然臣所以歷數王之朝,背淮千里而自致者,非惡臣國而樂吳民也,竊高下風之行,尤說大王之義。故願大王之無忽,察聽其志。

臣聞鷙鳥絫百,不如一鶚。夫全趙之時,武力鼎士袨服叢臺之下者一旦成市,而不能止幽王之湛患。淮南連山東之俠,死士盈朝,不能還厲王之西也。然而計議不得,雖諸、賁不能安其位,亦明矣。故願大王審畫而已。

始孝文皇帝據關入立,寒心銷志,不明求衣。自立天子之後,使東牟朱虛東褒義父之後,深割嬰兒王之。壤子王梁、代,益以淮陽。卒仆濟北,囚弟於雍者,豈非象新垣平等哉!今天子新據先帝之遺業,左規山東,右制關中,變權易勢,大臣難知。大王弗察,臣恐周鼎復起於漢,新垣過計於朝,則我吳遺嗣,不可期於世矣。高皇帝燒棧道,水章邯,兵不留行,收弊民之倦,東馳函谷,西楚大破。水攻則章邯以亡其城,陸擊則荊王以失其地,此皆國家之不幾者也。願大王孰察之。

吳王不內其言。

是時,景帝少弟梁孝王貴盛,亦待士。於是鄒陽、枚乘、嚴忌知吳不可說,皆去之梁,從孝王游。

陽為人有智略,忼慨不苟合,介於羊勝、公孫詭之間。勝等疾陽,惡之孝王。孝王怒,下陽吏,將殺之。陽客游以讒見禽,恐死而負絫,乃從獄中上書曰:

臣聞忠無不報,信不見疑,臣常以為然,徒虛語耳。昔荊軻慕燕丹之義,白虹貫日,太子畏之;衛先生為秦畫長平之事,太白食昴,昭王疑之。夫精誠變天地而信不諭兩主,豈不哀哉!今臣盡忠竭誠,畢議願知,左右不明,卒從吏訊,為世所疑。是使荊軻、衛先生復起,而燕、秦不寤也。願大王孰察之。

昔玉人獻寶,楚王誅之;李斯謁忠,胡亥極刑。是以箕子陽狂,接輿避世,恐遭此患也。願大王察玉人、李斯之意,而後楚王、胡亥之聽,毋使臣為箕子、接輿所笑。臣聞比干剖心,子胥鴟夷,臣始不信,乃今知之。願大王孰察,少加憐焉!

語曰「有白頭如新,傾蓋如故」。何則?知與不知也。故樊於期逃秦之燕,藉荊軻首以奉丹事;王奢去齊之魏,臨城自剄以卻齊而存魏。夫王奢、樊於期非新於齊、秦而故於燕、魏也,所以去二國死兩君者,行合於志,慕義無窮也。是以蘇秦不信於天下,為燕尾生;白圭戰亡六城,為魏取中山。何則?誠有以相知也。蘇秦相燕,人惡之燕王,燕王按劍而怒,食以駃騠;白圭顯於中山,人惡之於魏文侯,文侯賜以夜光之璧。何則?兩主二臣,剖心析肝相信,豈移於浮辭哉!

故女無美惡,入宮見妒;士無賢不肖,入朝見嫉。昔司馬喜臏腳於宋,卒相中山;范睢拉脅折齒於魏,卒為應侯。此二人者,皆信必然之畫,捐朋黨之私,挾孤獨之交,故不能自免於嫉妒之人也。是以申徒狄蹈雍之河,徐衍負石入海。不容於世,義不苟取比周於朝以移主上之心。故百里奚乞食於道路,繆公委之以政;甯戚飯牛車下,桓公任之以國。此二人者,豈素宦於朝,借譽於左右,然後二主用之哉?感於心,合於行,堅如膠桼,昆弟不能離,豈惑於眾口哉?故偏聽生姦,獨任成亂。昔魯聽季孫之說逐孔子,宋任子冉之計囚墨翟。夫以孔、墨之辯,不能自免於讒諛,而二國以危。何則?眾口鑠金,積毀銷骨也。秦用戎人由余而伯中國,齊用越人子臧而彊威、宣。此二國豈係於俗,牽於世,繫奇偏之浮辭哉?公聽並觀,垂明當世。故意合則胡越為兄弟,由余、子臧是矣;不合則骨肉為讎敵,朱、象、管、蔡是矣。今人主誠能用齊、秦之明,後宋、魯之聽,則五伯不足侔,而三王易為也。

是以聖王覺寤,損子之之心,而不說田常之賢,封比干之後,修孕婦之墓,故功業覆於天下。何則?欲善亡厭也。夫晉文親其讎,彊伯諸侯;齊桓用其仇,而一匡天下。何則?慈仁殷勤,誠加於心,不可以虛辭借也。

至夫秦用商鞅之法,東弱韓、魏,立彊天下,卒車裂之。越用大夫種之謀,禽勁吳而伯中國,遂誅其身。是以孫叔敖三去相而不悔,於陵子仲辭三公為人灌園。今人主誠能去驕傲之心,懷可報之意,披心腹,見情素,墮肝膽,施德厚,終與之窮達,無愛於士,則桀之犬可使吠堯,跖之客可使刺由,何況因萬乘之權,假聖王之資乎!然則軻湛七族,要離燔妻子,豈足為大王道哉!

臣聞明月之珠,夜光之璧,以闇投人於道,眾莫不按劍相眄者。何則?無因而至前也。蟠木根柢,輪囷離奇,而為萬乘器者,以左右先為之容也。故無因而至前,雖出隨珠和璧,祗怨結而不見德;有人先游,則枯木朽株,樹功而不忘。今夫天下布衣窮居之士,身在貧羸,雖蒙堯、舜之術,挾伊、管之辯,懷龍逢、比干之意,而素無根柢之容,雖竭精神,欲開忠於當世之君,則人主必襲按劍相眄之跡矣。是使布衣之士不得為枯木巧株之資也。

是以聖王制世御俗,獨化於陶鈞之上,而不牽乎卑辭之語,不奪乎眾多之口。故秦皇帝任中庶子蒙之言,以信荊軻,而匕首竊發;周文王獵涇渭,載呂尚歸,以王天下。秦信左右而亡,周用烏集而王。何則?以其能越攣拘之語,馳域外之議,獨觀乎昭曠之道也。

今人主沈諂諛之辭,牽帷廧之制,使不羈之士與牛驥同皁,此鮑焦所以憤於世也。

臣聞盛飾入朝者不以私汙義,底厲名號者不以利傷行。故里名勝母,曾子不入;邑號朝歌,墨子回車。今欲使天下寥廓之士籠於威重之權,脅於位勢之貴,回面汙行,以事諂諛之人,而求親近於左右,則士有伏死堀穴巖藪之中耳,安有盡忠信而趨闕下者哉!

書奏孝王,孝王立出之,卒為上客。

初,勝、詭欲使王求為漢嗣,王又嘗上書,願賜容車之地徑至長樂宮,自使梁國士眾築作甬道朝太后。爰盎等皆建以為不可。天子不許。梁王怒,令人刺殺盎。上疑梁殺之,使者冠蓋相望責梁王。梁王始與勝、詭有謀,陽爭以為不可,故見讒。枚先生、嚴夫子皆不敢諫。

及梁事敗,勝、詭死,孝王恐誅,乃思陽言,深辭謝之,齎以千金,令求方略解罪於上者。陽素知齊人王先生,年八十餘,多奇計,即往見,語以其事。王先生曰:「難哉!人主有私怨深怒,欲施必行之誅,誠難解也。以太后之尊,骨肉之親,猶不能止,況臣下乎?昔秦始皇有伏怒於太后,群臣諫而死者以十數。得茅焦為廓大義,始皇非能說其言也,乃自強從之耳。茅焦亦廑脫死如毛镒耳,故事所以難者也。今子欲安之乎?」陽曰:「

鄒魯守經學,齊楚多辯知,韓魏時有奇節,吾將歷問之。」王先生曰:「子行矣。還,過我而西。」

鄒陽行月餘,莫能為謀,還過王先生,曰:「臣將西矣,為如何?」王先生曰:「吾先日欲獻愚計,以為眾不可蓋,竊自薄陋不敢道也。若子行,必往見王長君,士無過此者矣。」鄒陽發寤於心,曰:「敬諾。」辭去,不過梁,徑至長安,因客見王長君。長君者,王美人兄也,後封為蓋侯。鄒陽留數日,乘間而請曰:「臣非為長君無使令於前,故來侍也;愚戇竊不自料,願有謁也。」長君跪曰:「幸甚。」陽曰:「竊聞長君弟得幸後宮,天下無有,而長君行跡多不循道理者。今爰盎事即窮竟,梁王恐誅。如此,則太后怫鬱泣血,無所發怒,切齒側目於貴臣矣。臣恐長君危於絫卵,竊為足下憂之。」長君懼然曰:「將為之柰何?」陽曰:「長君誠能精為上言之,得毋竟梁事,長君必固自結於太后。太后厚德長君,入於骨髓,而長君之弟幸於兩宮,金城之固也。又有存亡繼絕之功,德布天下,名施無窮,願長君深自計之。昔者,舜之弟象日以殺舜為事,及舜立為天子,封之於有卑。夫仁人之於兄弟,無臧怒,無宿怨,厚親愛而已,是以後世稱之。魯公子慶父使僕人殺子般,獄有所歸,季友不探其情而誅焉;慶父親殺閔公,季子緩追免賊,春秋以為親親之道也。魯哀姜薨於夷,孔子曰『齊桓公法而不譎』,以為過也。以是說天子,徼幸梁事不奏。」長君曰:「諾。」乘間入而言之。及韓安國亦見長公主,事果得不治。

初,吳王濞與七國謀反,及發,齊、濟北兩國城守不行。漢既破吳,齊王自殺,不得立嗣。濟北王亦欲自殺,幸全其妻子。齊人公孫玃謂濟北王曰:「臣請試為大王明說梁王,通意天子,說而不用,死未晚也。」公孫玃遂見梁王,曰:「夫濟北之地,東接彊齊,南牽吳越,北脅燕趙,此四分五裂之國,權不足以自守,勁不足以扞寇,又非有奇怪云以待難也,雖墜言於吳,非其正計也。昔者鄭祭仲許宋人立公子突以活其君,非義也,春秋記之,為其以生易死,以存易亡也。鄉使濟北見情實,示不從之端,則吳必先歷齊畢濟北,招燕、趙而總之。如此,則山東之從結而無隙矣。今吳楚之王練諸侯之兵,敺白徒之眾,西與天子爭衡,濟北獨底節堅守不下。使吳失與而無助,跬步獨進,瓦解土崩,破敗而不救者,未必非濟北之力也。夫以區區之濟北而與諸侯爭彊,是以羔犢之弱而扞虎狼之敵也。守職不橈,可謂誠一矣。功義如此,尚見疑於上,脅肩低首,絫足撫衿,使有自悔不前之心,非社稷之利也。臣恐藩臣守職者疑之。臣竊料之,能歷西山,徑長樂,抵未央,攘袂而正議者,獨大王耳。上有全亡之功,下有安百姓之名,德淪於骨髓,恩加於無窮,願大王留意詳惟之。」孝王大說,使人馳以聞。濟北王得不坐,徙封於淄川。

枚乘字叔,淮陰人也,為吳王濞郎中。吳王之初怨望謀為逆也,乘奏書諫曰:

臣聞得全者全昌,失全者全亡。舜無立錐之地,以有天下;禹無十戶之聚,以王諸侯。湯、武之土不過百里,上不絕三光之明,下不傷百姓之心者,有王術也。故父子之道,天性也;忠臣不避重誅以直諫,則事無遺策,功流萬世。臣乘願披腹心而效愚忠,唯大王少加意念惻怛之心於臣乘言。

夫以一縷之任係千鈞之重,上縣無極之高,下垂不測之淵,雖甚愚之人猶知哀其將絕也。馬方駭鼓而驚之,係方絕又重鎮之;係絕於天不可復結,隊入深淵難以復出。其出不出,間不容髮。能聽忠臣之言,百舉必脫。必若所欲為,危於絫卵,難於上天;變所欲為,易於反掌,安於太山。今欲極天命之壽,敝無窮之樂,究萬乘之勢,不出反掌之易,以居泰山之安,而欲乘絫卵之危,走上天之難,此愚臣之所以為大王惑也。

人性有畏其景而惡其跡者,卻背而走,跡愈多,景愈疾,不知就陰而止,景滅跡絕。欲人勿聞,莫若勿言;欲人勿知,莫若勿為。欲湯之凔,一人炊之,百人揚之,無益也,不如絕薪止火而已。不絕之於彼,而救之於此,譬猶抱薪而救火也。養由基,楚之善射者也,去楊葉百步,百發百中。楊葉之大,加百中焉,可謂善射矣。然其所止,乃百步之內耳,比於臣乘,未知操弓持矢也。

福生有基,禍生有胎;納其基,絕其胎,禍何自來?泰山之霤穿石,單極之镐斷幹。水非石之鑽,索非木之鋸,漸靡使之然也。夫銖銖而稱之,至石必差;寸寸而度之,至丈必過。石稱丈量,徑而寡失。夫十圍之木,始生如櫱,足可搔而絕,手可擢而拔,據其未生,先其未形也。磨礱底厲,不見其損,有時而盡;種樹畜養,不見其益,有時而大;積德絫行,不知其善,有時而用;棄義背理,不知其惡,有時而亡。臣願大王孰計而身行之,此百世不易之道也。

吳王不納。乘等去而之梁,從孝王游。

景帝即位,御史大夫晁錯為漢定制度,損削諸侯,吳王遂與六國謀反,舉兵西鄉,以誅錯為名。漢聞之,斬錯以謝諸侯。枚乘復說吳王曰:

昔者,秦西舉胡戎之難,北備榆中之關,南距羌筰之塞,東當六國之從。六國乘信陵之籍,明蘇秦之約,厲荊軻之威,并力一心以備秦。然秦卒禽六國,滅其社稷,而并天下,是何也?則地利不同,而民輕重不等也。今漢據全秦之地,兼六國之眾,修戎狄之義,而南朝羌筰,此其與秦,地相什而民相百,大王之所明知也。今夫讒諛之臣為大王計者,不論骨肉之義,民之輕重,國之大小,以為吳禍,此臣所以為大王患也。

夫舉吳兵以訾於漢,譬猶蠅蚋之附群牛,腐肉之齒利劍,鋒接必無事矣。天子聞吳率失職諸侯,願責先帝之遺約,今漢親誅其三公,以謝前過,是大王之威加於天下,而功越於湯武也。夫吳有諸侯之位,而實富於天子;有隱匿之名,而居過於中國。夫漢并二十四郡,十七諸侯,方輸錯出,運行數千里不絕於道,其珍怪不如東山之府。轉粟西鄉,陸行不絕,水行滿河,不如海陵之倉。修治上林,雜以離宮,積聚玩好,圈守禽獸,不如長洲之苑。游曲臺,臨上路,不如朝夕之池。深壁高壘,副以關城,不如江淮之險。此臣之所以為大王樂也。

今大王還兵疾歸,尚得十半。不然,漢知吳之有吞天下之心也,赫然加怒,遣羽林黃頭循江而下,襲大王之都;魯東海絕吳之饟道;梁王飭車騎,習戰射,積粟固守,以備滎陽,待吳之飢。大王雖欲反都,亦不得已。夫三淮南之計不負其約,齊王殺身以滅其跡,四國不得出兵其郡,趙囚邯鄲,此不可掩,亦已明矣。大王已去千里之國,而制於十里之內矣。張、韓將北地,弓高宿左右,兵不得下壁,軍不得大息,臣竊哀之。願大王孰察焉。

吳王不用乘策,卒見禽滅。

漢既平七國,乘由是知名。景帝召拜乘為弘農都尉。乘久為大國上賓,與英俊並游,得其所好,不樂郡吏,以病去官。

復游梁,梁客皆善屬辭賦,乘尤高。孝王薨,乘歸淮陰。

武帝自為太子聞乘名,及即位,乘年老,乃以安車蒲輪徵乘,道死。詔問乘子,無能為文者,後乃得其孽子皋。

皋字少孺。乘在梁時,取皋母為小妻。乘之東歸也,皋母不肯隨乘,乘怒,分皋數千錢,留與母居。年十七,上書梁共王,得召為郎。三年,為王使,與冗從爭,見讒惡遇罪,家室沒入。皋亡至長安。會赦,上書北闕,自陳枚乘之子。上得之大喜,召入見待詔,皋因賦殿中。詔使賦平樂館,善之。拜為郎,使匈奴。皋不通經術,詼笑類俳倡,為賦頌,好嫚戲,以故得媟黷貴幸,比東方朔、郭舍人等,而不得比嚴助等得尊官。

武帝春秋二十九乃得皇子,群臣喜,故皋與東方朔作皇太子生賦及立皇子禖祝,受詔所為,皆不從故事,重皇子也。

初,衛皇后立,皋奏賦以戒終。皋為賦善於朔也。

從行至甘泉、雍、河東,東巡狩,封泰山,塞決河宣房,游觀三輔離宮館,臨山澤,弋獵射馭狗馬蹴鞠刻鏤,上有所感,輒使賦之。為文疾,受詔輒成,故所賦者多。司馬相如善為文而遲,故所作少而善於皋。皋賦辭中自言為賦不如相如,又言為賦乃非。見視如倡,自悔類倡也。故其賦有詆娸東方朔,又自詆娸。其文骫骳,曲隨其事,皆得其意,頗詼笑,不甚閒靡。凡可讀者百二十篇,其尤嫚戲不可讀者尚數十篇。

路溫舒字長君,鉅鹿東里人也。父為里監門。使溫舒牧羊,溫舒取澤中蒲,截以為牒,編用寫書。稍習善,求為獄小吏,因學律令,轉為獄史,縣中疑事皆問焉。太守行縣,見而異之,署決曹史。又受春秋,通大義。舉孝廉,為山邑丞,坐法免,復為郡吏。

元鳳中,廷尉光以治詔獄,請溫舒署奏曹掾,守廷尉史。會昭帝崩,昌邑王賀廢,宣帝初即位,溫舒上書,言宜尚德緩刑。其辭曰:

臣聞齊有無知之禍,而桓公以興;晉有驪姬之難,而文公用伯。近世趙王不終,諸呂作難,而孝文為大宗。繇是觀之,禍亂之作,將以開聖人也。故桓文扶微興壞,尊文武之業,澤加百姓,功潤諸侯,雖不及三王,天下歸仁焉。文帝永思至德,以承天心,崇仁義,省刑罰,通關梁,一遠近,敬賢如大賓,愛民如赤子,內恕情之所安,而施之於海內,是以囹圄空虛,天下太平。夫繼變化之後,必有異舊之恩,此賢聖所以昭天命也。往者,昭帝即世而無嗣,大臣憂戚,焦心合謀,皆以昌邑尊親,援而立之。然天不授命,淫亂其心,遂以自亡。深察禍變之故,乃皇天之所以開至聖也。故大將軍受命武帝,股肱漢國,披肝膽,決大計,黜亡義,立有德,輔天而行,然後宗廟以安,天下咸寧。

臣聞春秋正即位,大一統而慎始也。陛下初登至尊,與天合符,宜改前世之失,正始受命之統,滌煩文,除民疾,存亡繼絕,以應天意。

臣聞秦有十失,其一尚存,治獄之吏是也。秦之時,羞文學,好武勇,賤仁義之士,貴治獄之吏;正言者謂之誹謗,遏過者謂之妖言。故盛服先生不用於世,忠良切言皆鬱於胸,譽諛之聲日滿於耳;虛美熏心,實禍蔽塞。此乃秦之所以亡天下也。方今天下賴陛下恩厚,亡金革之危,飢寒之患,父子夫妻戮力安家,然太平未洽者,獄亂之也。夫獄者,天下之大命也,死者不可復生,镇者不可復屬。書曰:「與其殺不辜,寧失不經。」今治獄吏則不然,上下相敺,以刻為明;深者獲公名,平者多後患。故治獄之吏皆欲人死,非憎人也,自安之道在人之死。是以死人之血流離於市,被刑之徒比肩而立,大辟之計歲以萬數,此仁聖之所以傷也。太平之未洽,凡以此也。夫人情安則樂生,痛則思死。棰楚之下,何求而不得?故囚人不勝痛,則飾辭以視之;吏治者利其然,則指道以明之;上奏畏卻,則鍛練而周內之。蓋奏當之成,雖咎繇聽之,猶以為死有餘辜。何則?成練者眾,文致之罪明也。是以獄吏專為深刻,殘賊而亡極,媮為一切,不顧國患,此世之大賊也。故俗語曰:「畫地為獄,議不入;刻木為吏,期不對。」此皆疾吏之風,悲痛之辭也。故天下之患,莫深於獄;敗法亂正,離親塞道,莫甚乎治獄之吏。此所謂一尚存者也。

臣聞烏鳶之卵不毀,而後鳳凰集;誹謗之罪不誅,而後良言進。故古人有言:「山藪藏疾,川澤納汙,瑾瑜匿惡,國君含詬。」唯陛下除誹謗以招切言,開天下之口,廣箴諫之路,掃亡秦之失,尊文武之德,省法制,寬刑罰,以廢治獄,則太平之風可興於世,永履和樂,與天亡極,天下幸甚。

上善其言,遷廣陽私府長。

內史舉溫舒文學高第,遷右扶風丞。時,詔書令公卿選可使匈奴者,溫舒上書,願給冢養,暴骨方外,以盡臣節。事下度遼將軍范明友、太僕杜延年問狀,罷歸故官。久之,遷臨淮太守,治有異跡,卒於官。

溫舒從祖父受曆數天文,以為漢厄三七之間,上封事以豫戒。成帝時,谷永亦言如此。及王莽篡位,欲章代漢之符,著其語焉。溫舒子及孫皆至牧守大官。

贊曰:春秋魯臧孫達以禮諫君,君子以為有後。賈山自下劘上,鄒陽、枚乘游於危國,然卒免刑戮者,以其言正也。路溫舒辭順而意篤,遂為世家,宜哉!


\end{pinyinscope}