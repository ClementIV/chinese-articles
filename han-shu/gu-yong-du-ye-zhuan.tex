\article{谷永杜鄴傳}

\begin{pinyinscope}
谷永字子雲,長安人也。父吉,為衛司馬,使送郅支單于侍子,為郅支所殺,語在陳湯傳。永少為長安小史,後博學經書。建昭中,御史大夫繁延壽聞其有茂材,除補屬,舉為太常丞,數上疏言得失。

建始三年冬,日食地震同日俱發,詔舉方正直言極諫之士,太常陽城侯劉慶忌舉永待詔公車。對曰:

陛下秉至聖之純德,懼天地之戒異,飭身修政,納問公卿,又下明詔,帥舉直言,燕見紬繹,以求咎愆,使臣等得造明朝,承聖問。臣材朽學淺,不通政事。竊聞明王即位,正五事,建大中,以承天心,則庶徵序於下,日月理於上;如人君淫溺後宮,般樂游田,五事失於躬,大中之道不立,則咎徵降而六極至。凡災異之發,各象過失,以類告人。乃十二月朔戊申,日食婺女之分,地震蕭牆之內,二者同日俱發,以丁寧陛下,厥咎不遠,宜厚求諸身。意豈陛下志在閨門,未卹政事,不慎舉錯,婁失中與?內寵大盛,女不遵道,嫉妒專上,妨繼嗣與?古之王者廢五事之中,失夫婦之紀,妻妾得意,謁行於內,勢行於外,至覆傾國家,或亂陰陽。昔褒姒用國,宗周以喪;閻妻驕扇,日以不臧。此其效也。經曰:「皇極,皇建其有極。」傳曰:「皇之不極,是謂不建,時則有日月亂行。」

陛下踐至尊之祚為天下主,奉帝王之職以統群生,方內之治亂,在陛下所執。誠留意於正身,勉強於力行,損燕私之閒以勞天下,放去淫溺之樂,罷歸倡優之笑,絕卻不享之義,慎節游田之虞,起居有常,循禮而動,躬親政事,致行無倦,安服若性。經曰:「繼自今嗣王,其毋淫于酒,毋逸于游田,惟正之共。」未有身治正而臣下邪者也。

夫妻之際,王事綱紀,安危之機,聖王所致慎也。昔舜飭正二女,以崇至德;楚莊忍絕丹姬,以成伯功;幽王惑於褒姒,周德降亡;魯桓脅於齊女,社稷以傾。誠修後宮之政,明尊卑之序,貴者不得嫉妒專寵,以絕驕嫚之端,抑褒、閻之亂,賤者咸得秩進,各得厥職,以廣繼嗣之統,息白華之怨,後宮親屬,饒之以財,勿與政事,以遠皇父之類,損妻黨之權,未有閨門治而天下亂者也。

治遠自近始,習善在左右。昔龍筦納言,而帝命惟允;四輔既備,成王靡有過事。誠敕正左右齊栗之臣,戴金貂之飾執常伯之職者皆使學先王之道,知君臣之義,濟濟謹孚,無敖戲驕恣之過,則左右肅艾,群僚仰法,化流四方。經曰:「亦惟先正克左右。」未有左右正而百官枉者也。

治天下者尊賢考功則治,簡賢違功則亂。誠審思治人之術,歡樂得賢之福,論材選士,必試於職,明度量以程能,考功實以定德,無用比周之虛譽,毋聽寖潤之譖愬,則抱功修職之吏無蔽傷之憂,比周邪偽之徒不得即工,小人日銷,俊艾日隆。經曰:「三載考績,三考黜陟幽明。」又曰:「

九德咸事,俊艾在官。」未有功賞得於前眾賢布於官而不治者也。

堯遭洪水之災,天下分絕為十二州,制遠之道微而無乖畔之難者,德厚恩深,無怨於下也。秦居平土,一夫大呼而海內崩析者,刑罰深酷,吏行殘賊也。夫違天害德,為上取怨於下,莫甚乎殘賊之吏。誠放退殘賊酷暴之吏一廢勿用,益選溫良上德之士以親萬姓,平刑釋冤以理民命,務省繇役,毋奪民時,薄收賦稅,毋殫民財,使天下黎元咸安家樂業,不苦踰時之役,不患苛暴之政,不疾酷烈之吏,雖有唐堯之大災,民無離上之心。經曰:「懷保小人,惠于鰥寡。」未有德厚吏良而民畔者也。

臣聞災異,皇天所以譴告人君過失,猶嚴父之明誡。畏懼敬改,則禍銷福降;忽然簡易,則咎罰不除。經曰:「饗用五福,畏用六極。」傳曰:「六沴作見,若不共御,六罰既侵,六極其下。」今三年之間,災異鋒起,小大畢具,所行不享上帝,上帝不豫,炳然甚著。不求之身,無所改正,疏舉廣謀,又不用其言,是循不享之跡,無謝過之實也,天責愈深。此五者,王事之綱紀,南面之急務,唯陛下留神。

對奏,天子異焉,特召見永。

其夏,皆令諸方正對策,語在杜欽傳。永對畢,因曰:「臣前幸得條對災異之效,禍亂所極,言關於聖聰。書陳於前,陛下委棄不納,而更使方正對策,背可懼之大異,問不急之常論,廢承天之至言,角無用之虛文,欲末殺災異,滿讕誣天,是故皇天勃然發怒,甲己之間暴風三溱,拔樹折木,此天至明不可欺之效也。」上特復問永,永對曰:「日食地震,皇后貴妾專寵所致。」語在五行志。

是時,上初即位,謙讓委政元舅大將軍王鳳,議者多歸咎焉。永知鳳方見柄用,陰欲自託,乃復曰:

方今四夷賓服,皆為臣妾,北無薰粥冒頓之患,南無趙佗、呂嘉之難,三垂晏然,靡有兵革之警。諸侯大者乃食數縣,漢吏制其權柄,不得有為,亡吳、楚、燕、梁之勢。百官盤互,親疏相錯,骨肉大臣有申伯之忠,洞洞屬屬,小心畏忌,無重合、安陽、博陸之亂。三者無毛髮之辜,不可歸咎諸舅。此欲以政事過差丞相父子、中尚書宦官,檻塞大異,皆瞽說欺天者也。竊恐陛下舍昭昭之白過,忽天地之明戒,聽晻昧之瞽說,歸咎乎無辜,倚異乎政事,重失天心,不可之大者也。

陛下即位,委任遵舊,未有過政。元年正月,白氣較然起乎東方,至其四月,黃濁四塞,覆冒京師,申以大水,著以震蝕。各有占應,相為表裏,百官庶事無所歸倚,陛下獨不怪與?白氣起東方,賤人將興之表也;黃濁冒京師,王道微絕之應也。夫賤人當起而京師道微,二者已醜。陛下誠深察愚臣之言,致懼天地之異,長思宗廟之計,改往反過,抗湛溺之意,解偏駮之愛,奮乾剛之威,平天覆之施,使列妾得人人更進,猶尚未足也,急復益納宜子婦人,毋擇好醜,毋避嘗字,毋論年齒。推法言之,陛下得繼嗣於微賤之間,乃反為福。得繼嗣而已,母非有賤也。後宮女史使令有直意者,廣求於微賤之間,以遇天所開右,慰釋皇太后之憂慍,解謝上帝之譴怒,則繼嗣蕃滋,災異訖息。陛下則不深察愚臣之言,忽於天地之戒,咎根不除,水雨之災,山石之異,將發不久;發則災異已極,天變成形,臣雖欲捐身關策,不及事已。

疏賤之臣,至敢直陳天意,斥譏帷幄之私,欲間離貴后盛妾,自知忤心逆耳,必不免於湯鑊之誅。此天保右漢家,使臣敢直言也。三上封事,然後得召;待詔一旬,然後得見。夫由疏賤納至忠,甚苦;由至尊聞天意,甚難。語不可露,願具書所言,因侍中奏陛下,以示腹心大臣。腹心大臣以為非天意,臣當伏妄言之誅;即以為誠天意也,奈何忘國家大本,背天意而從欲!唯陛下省察熟念,厚為宗廟計。

時對者數十人,永與杜欽為上第焉。上皆以其書示後宮。後上嘗賜許皇后書,采永言以責之,語在外戚傳。

永既陰為大將軍鳳說矣,能實最高,由是擢為光祿大夫。永奏書謝鳳曰:「永斗筲之材,質薄學朽,無一日之雅,左右之介,將軍說其狂言,擢之皂衣之吏,廁之爭臣之末,不聽浸潤之譖,不食膚受之愬,雖齊桓晉文用士篤密,察父悊兄覆育子弟,誠無以加!昔豫子吞炭壞形以奉見異,齊客隕首公門以報恩施,知氏、孟嘗猶有死士,何況將軍之門!」鳳遂厚之。

數年,出為安定太守。時上諸舅皆修經書,任政事。平阿侯譚年次當繼大將軍鳳輔政,尤與永善。陽朔中,鳳薨。鳳病困,薦從弟御史大夫音以自代。上從之,以音為大司馬車騎將軍,領尚書事,而平阿侯譚位特進,領城門兵。永聞之,與譚書曰:「君侯躬周召之德,執管晏之操,敬賢下士,樂善不倦,宜在上將久矣,以大將軍在,故抑鬱於家,不得舒憤。今大將軍不幸蚤薨,絫親疏,序材能,宜在君侯。拜吏之日,京師士大夫悵然失望。此皆永等愚劣,不能褒揚萬一。屬聞以特進領城門兵,是則車騎將軍秉政雍容于內,而至戚賢舅執管籥於外也。愚竊不為君侯喜。宜深辭職,自陳淺薄不足以固城門之守,收太伯之讓,保謙謙之路,闔門高枕,為知者首。願君侯與博覽者參之,小子為君侯安此。」譚得其書大感,遂辭讓不受領城門職。由是譚、音相與不平。

永遠為郡吏,恐為音所危,病滿三月免。音奏請永補營軍司馬,永數謝罪自陳,得轉為長史。

音用從舅越親輔政,威權損於鳳時。永復說音曰:「將軍履上將之位,食膏腴之都,任周召之職,擁天下之樞,可謂富貴之極,人臣無二,天下之責四面至矣,將何以居之?宜夙夜孳孳,執伊尹之彊德,以守職匡上,誅惡不避親愛,舉善不避仇讎,以章至公,立信四方。篤行三者,乃可以長堪重任,久享盛寵。太白出西方六十日,法當參天,今已過期,尚在桑榆之間,質弱而行遲,形小而光微。熒惑角怒明大,逆行守尾。其逆,常也;守尾,變也。意豈將軍忘湛漸之義,委曲從順,所執不彊,不廣用士,尚有好惡之忌,蕩蕩之德未純,方與將相大臣乖離之萌也?何故始襲司馬之號,俄而金火並有此變?上天至明,不虛見異,唯將軍畏之慎之,深思其故,改求其路,以享天意。」音猶不平,薦永為護菀使者。

音薨,成都侯商代為大司馬衛將軍,永乃遷為涼州刺史。奏事京師訖,當之部,時有黑龍見東萊,上使尚書問永,受所欲言。永對曰:

輒上聞,則商周不易姓而迭興,三正不變改而更用。夏商之將亡也,行道之人皆知之,晏然自以若天有日莫能危,是故惡日廣而不自知,大命傾而不寤。《易》曰:「危者有其安者也,亡者保其存者也。」陛下誠垂寬明之聽,無忌諱之誅,使芻蕘之臣得盡所聞於前,不懼於後患,直言之路開,則四方眾賢不遠千里,輻湊陳忠,群臣之上願,社稷之長福也。

漢家行夏正,夏正色黑,黑龍,同姓之象也。龍陽德,由小之大,故為王者瑞應。未知同姓有見本朝無繼嗣之慶,多危殆之隙,欲因擾亂舉兵而起者邪?將動心冀為後者,殘賊不仁,若廣陵、昌邑之類?臣愚不能處也。元年九月黑龍見,其晦,日有食之。今年二月己未夜星隕,乙酉,日有食之。六月之間,大異四發,二而同月,三代之末,春秋之亂,未嘗有也。臣聞三代所以隕社稷喪宗廟者,皆由婦人與群惡沈湎於酒。書曰:「乃用婦人之言,自絕于天;」「四方之述逃多罪。是宗是長,是信是使。」《詩》云:「燎之方陽,寧或滅之?赫赫宗周,褒姒滅之!」《易》曰:「濡其首,有孚失是。」秦所以二世十六年而亡者,養生泰奢,奉終泰厚也。二者陛下兼而有之,臣請略陳其效。

《易》曰「在中餽,無攸遂」,言婦人不得與事也。《詩》曰:「懿厥悊婦,為梟為鴟;」「匪降自天,生自婦人。」建始、河平之際,許、班之貴,頃動前朝,熏灼四方,賞賜無量,空虛內臧,女寵至極,不可上矣;今之後起,天所不饗,什倍於前。廢先帝法度,聽用其言,官秩不當,縱釋王誅,驕其親屬,假之威權,從橫亂政,刺舉之吏,莫敢奉憲。又以掖庭獄大為亂阱,榜箠钇於炮格,絕滅人命,主為趙、李報德復怨,反除白罪,建治正吏,多繫無辜,掠立迫恐,至為人起責,分利受謝。生入死出者,不可勝數。是以日食再既,以昭其辜。

王者必先自絕,然后天絕之。陛下棄萬乘之至貴,樂家人之賤事,厭高美之尊號,好匹夫之卑字,崇聚僄輕無義小人以為私客,數離深宮之固,挺身晨夜,與群小相隨,烏集雜會,飲醉吏民之家,亂服共坐,流湎媟嫚,溷殽無別,閔免遁樂,晝夜在路。典門戶奉宿衛之臣執干戈而守空宮,公卿百僚不知陛下所在,積數年矣。

王者以民為基,民以財為本,財竭則下畔,下畔則上亡。是以明王愛養基本,不敢窮極,使民如承大祭。今陛下輕奪民財,不愛民力,聽邪臣之計,去高敞初陵,捐十年功緒,改作昌陵,反天地之性,因下為高,積土為山,發徒起邑,並治宮館,大興繇役,重增賦斂,徵法如雨,役百乾谿,費疑驪山,靡敝天下,五年不成而後反故,又廣盱營表,發人冢墓,斷截骸骨,暴揚尸柩。百姓財竭力盡,愁恨感天,災異婁降,饑饉仍臻。流散冗食,餧死於道,以百萬數。公家無一年之畜,百姓無旬日之儲,上下俱匱,無以相救。《詩》云:「殷監不遠,在夏后之世。」願陛下追觀夏、商、周、秦所以失之,以鏡考己行。有不合者,臣當伏妄言之誅!

漢興九世,百九十餘載,繼體之主七,皆承天順道,遵先祖法度,或以中興,或以治安。至於陛下,獨違道縱欲,輕身妄行,當盛壯之隆,無繼嗣之福,有危亡之憂,積失君道,不合天意,亦已多矣。為人後嗣,守人功業,如此,豈不負哉!方今社稷宗廟禍福安危之機在於陛下,陛下誠肯發明聖之德,昭然遠寤,畏此上天之威怒,深懼危亡之徵兆,蕩滌邪辟之惡志,厲精致政,專心反道,絕群小之私客,免不正之詔除,悉罷北宮私奴車馬惰出之具,克己復禮,毋貳微行出飲之過,以防迫切之禍,深惟日食再既之意,抑損椒房玉堂之盛寵,毋聽後宮之請謁,除掖庭之亂獄,出炮格之陷阱,誅戮邪佞之臣及左右執左道以事上者以塞天下之望,且寑初陵之作,止諸繕治宮室,闕更減賦,盡休力役,存卹振捄困乏之人以弭遠方,厲崇忠直,放退殘賊,無使素餐之吏久尸厚祿,以次貫行,固執無違,夙夜孳孳,婁省無怠,舊衍畢改,新德既章,纖介之邪不復載心,則赫赫大異庶幾可銷,天命去就庶幾可復,社稷宗廟庶幾可保。唯陛下留神反覆,熟省臣言。臣幸得備邊部之吏,不知本朝失得,瞽言觸忌諱,罪當萬死。

成帝性寬而好文辭,又久無繼嗣,數為微行,多近幸小臣,趙、李從微賤專寵,皆皇太后與諸舅夙夜所常憂。至親難數言,故推永等使因天變而切諫,勸上納用之。永自知有內應,展意無所依違,每言事輒見答禮。至上此對,上大怒。衛將軍商密擿永令發去。上使侍御史收永,敕過交道廄者勿追。御史不及永,還,上意亦解,自悔。明年,徵永為太中大夫,遷光祿大夫給事中。

元延元年,為北地太守。時災異尤數,永當之官,上使衛尉淳于長受永所欲言。永對曰:

臣永幸得以愚朽之材為太中大夫,備拾遺之臣,從朝者之後,進不能盡思納忠輔宣聖德,退無被堅執銳討不義之功,猥蒙厚恩,仍遷至北地太守。絕命隕首,身膏草野,不足以報塞萬分。陛下聖德寬仁,不遺易忘之臣,垂周文之聽,下及芻蕘之愚,有詔使衛尉受臣永所欲言。臣聞事君之義,有言責者盡其忠,有官守者修其職。臣永幸得免於言責之辜,有官守之任,當畢力遵職,養綏百姓而已,不宜復關得失之辭。忠臣之於上,志在過厚,是故遠不違君,死不忘國。昔史魚既沒,餘忠未訖,委柩後寑,以屍達誠;汲黯身外思內,發憤舒憂,遺言李息。經曰:「雖爾身在外,乃心無不在王室。」臣永幸得給事中出入三年,雖執干戈守邊垂,思慕之心常存於省闥,是以敢越郡吏之職,陳累年之憂。

臣聞天生蒸民,不能相治,為立王者以統理之,方制海內非為天子,列土封疆非為諸侯,皆以為民也。垂三統,列三正,去無道,開有德,不私一姓,明天下乃天下之天下,非一人之天下也。王者躬行道德,承順天地,博愛仁恕,恩及行葦,籍稅取民不過常法,宮室車服不踰制度,事節財足,黎庶和睦,則卦氣理效,五徵時序,百姓壽考,庶屮蕃滋,符瑞並降,以昭保右。失道妄行,逆天暴物,窮奢極欲,湛湎荒淫,婦言是從,誅逐仁賢,離逖骨肉,群小用事,峻刑重賦,百姓愁怨,則卦氣悖亂,咎徵著郵,上天震怒,災異婁降,日月薄食,五星失行,山崩川潰,水泉踊出,妖孽並見,茀星耀光,饑饉荐臻,百姓短折,萬物夭傷。終不改寤,惡洽變備,不復譴告,更命有德。《詩》云:「乃眷西顧,此惟予宅。」

夫去惡奪弱,遷命賢聖,天地之常經,百王之所同也。加以功德有厚薄,期質有修短,時世有中季,天道有盛衰。陛下承八世之功業,當陽數之標季,涉三七之節紀,遭无妄之卦運,直百六之災阨。三難異科,雜焉同會。建始元年以來二十載間,群災大異,交錯鋒起,多於春秋所書。八世著記,久不塞除,重以今年正月己亥朔日有食之,三朝之會,四月丁酉四方眾星白晝流隕,七月辛未彗星橫天。乘三難之際會,畜眾多之災異,因之以饑饉,接之以不贍。彗星,極異也,土精所生,流隕之應出於飢變之後,兵亂作矣,厥期不久,隆德積善,懼不克濟。內則為深宮後庭將有驕臣悍妾醉酒狂悖卒起之敗,北宮苑囿街巷之中臣妾之家幽閒之處徵舒、崔杼之亂;外則為諸夏下土將有樊並、蘇令、陳勝、項梁奮臂之禍。內亂朝暮,日戒諸夏,舉兵以火角為期。安危之分界,宗廟之至憂,臣永所以破膽寒心,豫言之累年。下有其萌,然後變見於上,可不致慎!

禍起細微,姦生所易。願陛下正君臣之義,無復與群小媟黷燕飲;中黃門後庭素驕慢不謹嘗以醉酒失臣禮者,悉出勿留。勤三綱之嚴,修後宮之政,抑遠驕妒之寵,崇近婉順之行,加惠失志之人,懷柔怨恨之心。保至尊之重,秉帝王之威,朝覲法出而後駕,陳兵清道而後行,無復輕身獨出,飲食臣妾之家。三者既除,內亂之路塞矣。

諸夏舉兵,萌在民饑饉而吏不卹,興於百姓困而賦斂重,發於下怨離而上不知。《易》曰:「屯其膏,小貞吉,大貞凶。」傳曰:「飢而不損茲謂泰,厥災水,厥咎亡。」訞辭曰:「關動牡飛,辟為無道,臣為非,厥咎亂臣謀篡。」王者遭衰難之世,有飢饉之災,不損用而大自潤,故凶;百姓困貧無以共求,愁悲怨恨,故水;城關守國之固,固將去焉,故牡飛。往年郡國二十一傷於水災,禾黍不入。今年蠶麥咸惡。百川沸騰,江河溢決,大水泛濫郡國十五有餘。比年喪稼,時過無宿麥。百姓失業流散,群輩守關。大異較炳如彼,水災浩浩,黎庶窮困如此,宜損常稅小自潤之時,而有司奏請加賦,甚繆經義,逆於民心,布怨趨禍之道也。牡飛之狀,殆為此發。古者穀不登虧膳,災婁至損服,凶年不塈塗,明王之制也。《詩》云:「凡民有喪,扶服捄之。」論語曰:「百姓不足,君孰予足?」臣願陛下勿許加賦之奏,益減大官、導官、中御府、均官、掌畜、廩犧用度,止尚方、織室、京師郡國工服官發輸造作,以助大司農。流恩廣施,振贍困乏,開關梁,內流民,恣所欲之,以救其急。立春,遣使者循行風俗,宣布聖德,存卹孤寡,問民所苦,勞二千石,敕勸耕桑,毋奪農時,以慰綏元元之心,防塞大姦之侽。諸夏之亂,庶幾可息。

臣聞上主可與為善而不可與為惡,下主可與為惡而不可與為善。陛下天然之性,疏通聰敏,上主之姿也。少省愚臣之言,感寤三難,深畏大異,定心為善,捐忘邪志,毋貳舊愆,厲精致改,至誠應天,則積異塞於上,禍亂伏於下,何憂患之有?竊恐陛下公志未專,私好頗存,尚愛群小,不肯為耳!

對奏,天子甚感其言。

永於經書,汎為疏達,與杜欽、杜鄴略等,不能洽浹如劉向父子及揚雄也。其於天官、京氏易最密,故善言災異,前後所上四十餘事,略相反覆,專攻上身與後宮而已。黨於王氏,上亦知之,不甚親信也。

永所居任職,為北地太守歲餘,衛將軍商薨,曲陽侯根為票騎將軍,薦永,徵入為大司農。歲餘,永病,三月,有司奏請免。故事,公卿病,輒賜告,至永獨即時免。數月,卒於家。本名並,以尉氏樊並反,更名永云。

杜鄴字子夏,本魏郡繁陽人也。祖父及父積功勞皆至郡守,武帝時徙茂陵。鄴少孤,其母張敞女。鄴壯,從敞子吉學問,得其家書。以孝廉為郎。

與車騎將軍王音善。平阿侯譚不受城門職,後薨,上閔悔之,乃復令譚弟成都侯商位特進,領城門兵,得舉吏如將軍府。鄴見音前與平阿有隙,即說音曰:「鄴聞人情,恩深者其養謹,愛至者其求詳。夫戚而不見殊,孰能無怨?此棠棣、角弓之詩所為作也。昔秦伯有千乘之國,而不能容其母弟,春秋亦書而譏焉。周召則不然,忠以相輔,義以相匡,同己之親,等己之尊,不以聖德獨兼國寵,又不為長專受榮任,分職於陝,並為弼疑。故內無感恨之隙,外無侵侮之羞,俱享天祐,兩荷高名者,蓋以此也。竊見成都侯以特進領城門兵,復有詔得舉吏如五府,此明詔所欲寵也。將軍宜承順聖意,加異往時,每事凡議,必與及之,指為誠發,出於將軍,則孰敢不說諭?昔文侯寤大鴈之獻而父子益親,陳平共壹飯之篹而將相加驩,所接雖在楹階俎豆之間,其於為國折衝厭難,豈不遠哉!竊慕倉唐、陸子之義,所白声內,唯深察焉。」音甚嘉其言,由是與成都侯商親密,二人皆重鄴。後以病去郎。商為大司馬衛將軍,除鄴主簿,以為腹心,舉侍御史。哀帝即位,遷為涼州刺史。鄴居職寬舒,少威嚴,數年以病免。

是時,帝祖母定陶傅太后稱皇太太后,帝母丁姬稱帝太后,而皇后即傅太后從弟子也。傅氏侯者三人,丁氏侯者二人。又封傅太后同母弟子鄭業為陽信侯。傅太后尤與政專權。元壽元年正月朔,上以皇后父孔鄉侯傅晏為大司馬衛將軍,而帝舅陽安侯丁明為大司馬票騎將軍。臨拜,日食,詔舉方正直言。扶陽侯韋育舉鄴方正,鄴對曰:

臣聞禽息憂國,碎首不恨;卞和獻寶,刖足願之。臣幸得奉直言之詔,無二者之危,敢不極陳!臣聞陽尊陰卑,卑者隨尊,尊者兼卑,天之道也。是以男雖賤,各為其家陽;女雖貴,猶為其國陰。故禮明三從之義,雖有文母之德,必繫於子。春秋不書紀侯之母,陰義殺也。昔鄭伯隨姜氏之欲,終有叔段篡國之禍;周襄王內迫惠后之難,而遭居鄭之危。漢興,呂太后權私親屬,又以外孫為孝惠后,是時繼嗣不明,凡事多晻,晝昏冬雷之變,不可勝載。竊見陛下行不偏之政,每事約儉,非禮不動,誠欲正身與天下更始也。然嘉瑞未應,而日食地震,民訛言行籌,傳相驚恐。案春秋災異,以指象為言語,故在於得一類而達之也。日食,明陽為陰所臨,坤卦乘離,明夷之象也。坤以法地,為土為母,以安靜為德。震,不陰之效也。占象甚明,臣敢不直言其事!

昔曾子問從令之義,孔子曰:「是何言與!」善閔子騫守禮不苟,從親所行,無非理者,故無可間也。前大司馬新都侯莽退伏弟家,以詔策決,復遣就國。高昌侯宏去蕃自絕,猶受封土。制書侍中駙馬都尉遷不忠巧佞,免歸故郡,間未旬月,則有詔還,大臣奏正其罰,卒不得遣,而反兼官奉使,顯寵過故。及陽信侯業,皆緣私君國,非功義所止。諸外家昆弟無賢不肖,並侍帷幄,布在列位,或典兵衛,或將軍屯,寵意并於一家,積貴之勢,世所希見所希聞也。至乃并置大司馬將軍之官。皇甫雖盛,三桓雖隆,魯為作三軍,無以甚此。當拜之日,晻然日食。不在前後,臨事而發者,明陛下謙遜無專,承指非一,所言輒聽,所欲輒隨,有罪惡者不坐辜罰,無功能者畢受官爵,流漸積猥,正尤在是,欲令昭昭以覺聖朝。昔詩人所刺,春秋所譏,指象如此,殆不在它。由後視前,忿邑非之,逮身所行,不自鏡見,則以為可,計之過者。疏賤獨偏見,疑內亦有此類。天變不空,保右世主如此之至,奈何不應!

臣聞野雞著怪,高宗深動;大風暴過,成王怛然。願陛下加致精誠,思承始初,事稽諸古,以厭下心,則黎庶群生無不說喜,上帝百神收還威怒,禎祥福祿何嫌不報!

鄴未拜,病卒。鄴言民訛言行籌,及谷永言王者買私田,彗星隕石牡飛之占,語在五行志。

初,鄴從張吉學,吉子竦又幼孤,從鄴學問,亦著於世,尤長小學。鄴子林,清靜好古,亦有雅材,建武中歷位列卿,至大司空。其正文字過於鄴、竦,故世言小學者由杜公。

贊曰:孝成之世,委政外家,諸舅持權,重於丁、傅在孝哀時。故杜鄴敢譏丁、傅,而欽、永不敢言王氏,其勢然也。及欽欲挹損鳳權,而鄴附會音、商。永陳三七之戒,斯為忠焉,至其引申伯以阿鳳,隙平阿於車騎,指金火以求合,可謂諒不足而談有餘者。孔子稱「友多聞」,三人近之矣。


\end{pinyinscope}