\article{薛宣朱博傳}

\begin{pinyinscope}
薛宣字贛君,東海郯人也。少為廷尉書佐都船獄史。後以大司農斗食屬察廉,補不其丞。琅邪太守趙貢行縣,見宣,甚說其能。從宣歷行屬縣,還至府,令妻子與相見,戒曰:「贛君至丞相,我兩子亦中丞相史。」察宣廉,遷樂浪都尉丞。幽州刺史舉茂材,為宛句令。大將軍王鳳聞其能,薦宣為長安令,治果有名,以明習文法詔補御史中丞。

是時,成帝初即位,宣為中丞,執法殿中,外總部刺史,上疏曰:「陛下至德仁厚,哀閔元元,躬有日仄之勞,而亡佚豫之樂,允執聖道,刑罰惟中,然而嘉氣尚凝,陰陽不和,是臣下未稱,而聖化獨有不洽者也。臣竊伏思其一端,殆吏多苛政,政教煩碎,大率咎在部刺史,或不循守條職,舉錯各以其意,多與郡縣事,至開私門,聽讒佞,以求吏民過失,譴呵及細微,責義不量力。郡縣相迫促,亦內相刻,流至眾庶。是故鄉黨闕於嘉賓之懽,九族忘其親親之恩,飲食周急之厚彌衰,送往勞來之禮不行。夫人道不通,則陰陽否鬲,和氣不興,未必不由此也。《詩》云:『民之失德,乾餱以愆。』鄙語曰:『苛政親,煩苦傷恩。』方刺史奏事時,宜明申敕,使昭然知本朝之要務。臣愚不知治道唯明主察焉。」上嘉納之。

宣數言政事便宜,舉奏部刺史郡國二千石,所貶退稱進,白黑分明,繇是知名。出為臨淮太守,政教大行。會陳留郡有大賊廢亂,上徙宣為陳留太守,盜賊禁止,吏民敬其威信。入守左馮翊,滿歲稱職為真。

始高陵令陽湛、櫟陽令謝游皆貪猾不遜,持郡短長,前二千石數案不能竟。及宣視事,詣府謁,宣設酒飯與相對,接待甚備。已而陰求其罪臧,具得所受取。宣察湛有改節敬宣之效,乃手自牒書,條其姦臧,封與湛曰:「吏民條言君如牒,或議以為疑於主守盜。馮翊敬重令,又念十金法重,不忍相暴章。故密以手書相曉,欲君自圖進退,可復伸眉於後。即無其事,復封還記,得為君分明之。」湛自知罪臧皆應記,而宣辭語溫潤,無傷害意。湛即時解印綬付吏,為記謝宣,終無怨言。而櫟陽令游自以大儒有名,輕宣。宣獨移書顯責之曰:「告櫟陽令:吏民言令治行煩苛,適罰作使千人以上;賊取錢財數十萬,給為非法;賣買聽任富吏,賈數不可知。證驗以明白,欲遣吏考案,恐負舉者,恥辱儒士,故使掾平鐫令。孔子曰:『陳力就列,不能者止。』令詳思之,方調守。」游得檄,亦解印綬去。

又頻陽縣北當上郡、西河,為數郡湊,多盜賊。其令平陵薛恭本縣孝者,功次稍遷,未嘗治民,職不辦。而粟邑縣小,辟在山中,民謹樸易治。令鉅鹿尹賞久郡用事吏,為樓煩長,舉茂材,遷在粟。宣即以令奏賞與恭換縣。二人視事數月,而兩縣皆治。宣因移書勞勉之曰:「昔孟公綽優於趙魏而不宜滕薛,故或以德顯,或以功舉,『君子之道,焉可憮也!』屬縣各有賢君,馮翊垂拱蒙成。願勉所職,卒功業。」

宣得郡中吏民罪名,輒召告其縣長吏,使自行罰。曉曰:「府所以不自發舉者,不欲代縣治,奪賢令長名也。」長吏莫不喜懼,免冠謝宣歸恩受戒者。

宣為吏賞罰明,用法平而必行,所居皆有條教可紀,多仁恕愛利。池陽令舉廉吏獄掾王立,府未及召,聞立受囚家錢。宣責讓縣,縣案驗獄掾,乃其妻獨受繫者錢萬六千,受之再宿,獄掾實不知。掾慚恐自殺。宣聞之,移書池陽曰:「縣所舉廉吏獄掾王立,家私受賕,而立不知,殺身以自明。立誠廉士,甚可閔惜!其以府決曹掾書立之柩,以顯其魂。府掾史素與立相知者,皆予送葬。」

及日至休吏,賊曹掾張扶獨不肯休,坐曹治事。宣出教曰:「蓋禮貴和,人道尚通。日至,吏以令休,所繇來久。曹雖有公職事,家亦望私恩意。掾宜從眾,歸對妻子,設酒肴,請鄰里,壹笑相樂,斯亦可矣!」扶慚愧。官屬善之。

宣為人好威儀,進止雍容,甚可觀也。性密靜有思,思省吏職,求其便安。下至財用筆研,皆為設方略,利用而省費。吏民稱之,郡中清靜。遷為少府,共張職辦。

月餘,御史大夫于永卒,谷永上疏曰:「帝王之德莫大於知人,知人則百僚任職,天工不曠。故皋陶曰:『知人則哲,能官人。』御史大夫內承本朝之風化,外佐丞相統理天下,任重職大,非庸材所能堪。今當選於群卿,以充其缺。得其人則萬姓欣喜,百僚說服;不得其人則大職墮斁,王功不興。虞帝之明,在茲壹舉,可不致詳!竊見少府宣,材茂行絜,達於從政,前為御史中丞,執憲轂下,不吐剛茹柔,舉錯時當;出守臨淮、陳留,二郡稱治;為左馮翊,崇教養善,威德並行,眾職修理,姦軌絕息,辭訟者歷年不至丞相府,赦後餘盜賊什分三輔之一。功效卓爾,自左內史初置以來未嘗有也。孔子曰:『如有所譽,其有所試。』宣考績功課,簡在兩府,不敢過稱以奸欺誣之罪。臣聞賢材莫大於治人,宣已有效。其法律任廷尉有餘,經術文雅足以謀王體,斷國論;身兼數器,有『退食自公』之節。宣無私黨游說之助,臣恐陛下忽於羔羊之詩,舍公實之臣,任華虛之譽,是用越職,陳宣行能,唯陛下留神考察。」上然之,遂以宣為御史大夫。

數月,代張禹為丞相,封高陽侯,食邑千戶。宣除趙貢兩子為史。貢者,趙廣漢之兄子也,為吏亦有能名。宣為相,府辭訟例不滿萬錢不為移書,後皆遵用薛侯故事。然官屬譏其煩碎無大體,不稱賢也。時天子好儒雅,宣經術又淺,上亦輕焉。

久之,廣漢郡盜賊群起,丞相御史遣掾史逐捕不能克。上乃拜河東都尉趙護為廣漢太守,以軍法從事。數月,斬其渠帥鄭躬,降者數千人,乃平。會邛成太后崩,喪事倉卒,吏賦斂以趨辦。其後上聞之,以過丞相御史,遂冊免宣曰:「君為丞相,出入六年,忠孝之行,率先百僚,朕無聞焉。朕既不明,變異數見,歲比不登,倉廩空虛,百姓飢饉,流離道路,疾疫死者以萬數,人至相食,盜賊並興,群職曠廢,是朕之不德而股肱不良也。乃者廣漢群盜橫恣,殘賊吏民,朕惻然傷之,數以問君,君對輒不如其實。西州鬲絕,幾不為郡。三輔賦斂無度,酷吏並緣為姦,侵擾百姓,詔君案驗,復無欲得事實之意。九卿以下,咸承風指,同時陷于謾欺之辜,咎繇君焉!有司法君領職解嫚,開謾欺之路,傷薄風化,無以帥示四方。不忍致君于理,其上丞相高陽侯印綬,罷歸。」

初,宣為丞相,而翟方進為司直。宣知方進名儒,有宰相器,深結厚焉。後方進竟代為丞相,思宣舊恩,宣免後二歲,薦宣明習文法,練國制度,前所坐過薄,可復進用。上徵宣,復爵高陽侯,加寵特進,位次師安昌侯,給事中,視尚書事。宣復尊重。任政數年,後坐善定陵侯淳于長罷就第。

初,宣有兩弟,明、修。明至南陽太守。修歷郡守、京兆尹、少府,善交接,得州里之稱。後母常從修居官。宣為丞相時,修為臨菑令,宣迎後母,修不遣。後母病死,修去官持服。宣謂修三年服少能行之者,兄弟相駮不可,修遂竟服,繇是兄弟不和。

久之,哀帝初即位,博士申咸給事中,亦東海人也,毀宣不供養行喪服,薄於骨肉,前以不忠孝免,不宜復列封侯在朝省。宣子況為右曹侍郎,數聞其語,賕客楊明,欲令創咸面目,使不居位。會司隸缺,況恐咸為之,遂令明遮斫咸宮門外,斷鼻脣,身八創。

事下有司,御史中丞眾等奏:「況朝臣,父故宰相,再封列侯,不相敕丞化,而骨肉相疑,疑咸受修言以謗毀宣。咸所言皆宣行跡,眾人所共見,公家所宜聞。況知咸給事中,恐為司隸舉奏宣,而公令明等迫切宮闕,要遮創戮近臣於大道人眾中,欲以鬲塞聰明,杜絕論議之端。桀黠無所畏忌,萬眾讙譁,流聞四方,不與凡民忿怒爭鬥者同。臣聞敬近臣,為近主也。禮,下公門,式路馬,君畜產且猶敬之。春秋之義,意惡功遂,不免於誅,上浸之源不可長也。況首為惡,明手傷,功意俱惡,皆大不敬。明當以重論,及況皆棄市。」廷尉直以為「律曰『鬥以刃傷人,完為城旦,其賊加罪一等,與謀者同罪。』詔書無以詆欺成罪。傳曰:『遇人不以義而見疻者,與痏人之罪鈞,惡不直也。』咸厚善修,而數稱宣惡,流聞不誼,不可謂直。況以故傷咸,計謀已定,後聞置司隸,因前謀而趣明,非以恐咸為司隸故造謀也。本爭私變,雖於掖門外傷咸道中,與凡民爭鬥無異。殺人者死,傷人者刑,古今之通道,三代所不易也。孔子曰:『必也正名。』名不正,則至於刑罰不中;刑罰不中,而民無所錯手足。今以況為首惡,明手傷為大不敬,公私無差。春秋之義,原心定罪。原況以父見謗發忿怒,無它大惡。加詆欺,輯小過成大辟,陷死刑,違明詔,恐非法意,不可施行。聖王不以怒增刑。明當以賊傷人不直,況與謀者皆爵減完為城旦。」上以問公卿議臣。丞相孔光、大司空師丹以中丞議是,自將軍以下至博士議郎皆是廷尉。況竟減罪一等,徙敦煌。宣坐免為庶人,歸故郡,卒於家。

宣子惠亦至二千石。始惠為彭城令,宣從臨淮遷至陳留,過其縣,橋梁郵亭不修。宣心知惠不能,留彭城數日,案行舍中,處置什器,觀視園菜,終不問惠以吏事。惠自知治縣不稱宣意,遣門下掾送宣至陳留,令掾進見,自從其所問宣不教戒惠吏職之意。宣笑曰:「吏道以法令為師,可問而知。及能與不能,自有資材,何可學也?」眾人傳稱,以宣言為然。

初,宣後封為侯時,妻死,而敬武長公主寡居,上令宣尚焉。及宣免歸故郡,公主留京師。後宣卒,主上書願還宣葬延陵,奏可。況私從敦煌歸長安,會赦,因留與主私亂。哀帝外家丁、傅貴,主附事之,而疏王氏。元始中,莽自尊為安漢公,主又出言非莽。而況與呂寬相善,及寬事覺時,莽并治況,發揚其罪,使使者以太皇太后詔賜主藥。主怒曰:「劉氏孤弱,王氏擅朝,排擠宗室,且嫂何與取妹披抉其閨門而殺之?」使者迫守主,遂飲藥死。況梟首於市。白太后云主暴病薨。太后欲臨其喪,莽固爭,乃止。

朱博字子元,杜陵人也。家貧,少時給事縣為亭長,好客少年,捕搏敢行。稍遷為功曹,伉俠好交,隨從士大夫,不避風雨。是時,前將軍望之子蕭育、御史大夫萬年子陳咸以公卿子著材知名,博皆友之矣。時諸陵縣屬太常,博以太常掾察廉,補安陵丞。後去官入京兆,歷曹史列掾,出為督郵書掾,所部職辦,郡中稱之。

而陳咸為御史中丞,坐漏泄省中語下獄。博去吏,間步至廷尉中,候伺咸事。咸掠治困篤,博詐得為醫入獄,得見咸,具知其所坐罪。博出獄,又變姓名,為咸驗治數百,卒免咸死罪。咸得論出,而博以此顯名,為郡功曹。

久之,成帝即位,大將軍王鳳秉政,奏請陳咸為長史。咸薦蕭育、朱博除莫府屬,鳳甚奇之,舉博櫟陽令,徙雲陽、平陵三縣,以高弟入為長安令。京師治理,遷冀州刺史。

博本武吏,不更文法,及為刺史行部,吏民數百人遮道自言,官寺盡滿。從事白請且留此縣錄見諸自言者,事畢乃發,欲以觀試博。博心知之,告外趣駕。既白駕辦,博出就車見自言者,使從事明敕告吏民:「欲言縣丞尉者,刺史不察黃綬,各自詣郡。欲言二千石墨綬長吏者,使者行部還,詣治所。其民為吏所冤,及言盜賊辭訟事,各使屬其部從事。」博駐車決遣,四五百人皆罷去,如神。吏民大驚,不意博應事變乃至於此。後博徐問,果老從事教民聚會。博殺此吏,州郡畏博威嚴。徙為并州刺史,護漕都尉,遷琅邪太守。

齊郡舒緩養名,博新視事,右曹掾史皆移病臥。博問其故,對言「惶恐!故事二千石新到,輒遣吏存問致意,乃敢起就職。」博奮髯抵几曰:「觀齊兒欲以此為俗邪!」乃召見諸曹史書佐及縣大吏,選視其可用者,出教置之。皆斥罷諸病吏,白巾走出府門。郡中大驚。頃之,門下掾贛遂耆老大儒,教授數百人,拜起舒遲。博出教主簿:「贛老生不習吏禮,主簿且教拜起,閑習乃止。」又敕功曹:「官屬多褒衣大袑,不中節度,自今掾史衣皆令去地三寸。」博尤不愛諸生,所至郡輒罷去議曹,曰:「豈可復置謀曹邪!」文學儒吏時有奏記稱說云云,博見謂曰:「如太守漢吏,奉三尺律令以從事耳,亡奈生所言聖人道何也!且持此道歸,堯舜君出,為陳說之。」其折逆人如此。視事數年,大改其俗,掾史禮節如楚、趙吏。

博治郡,常令屬縣各用其豪桀以為大吏,文武從宜。縣有劇賊及它非常,博輒移書以詭責之。其盡力有效,必加厚賞;懷詐不稱,誅罰輒行。以是豪強慹服。姑幕縣有群輩八人報仇廷中,皆不得。長吏自繫書言府,賊曹掾史自白請至姑幕。事留不出。功曹諸掾即皆自白,復不出。於是府丞詣閤,博乃見丞掾曰:「以為縣自有長吏,府未嘗與也,丞掾謂府當與之邪?」閤下書佐入,博口占檄文曰:「府告姑幕令丞:言賊發不得,有書。檄到,令丞就職,游徼王卿力有餘,如律令!」王卿得敕惶怖,親屬失色,晝夜馳騖,十餘日間捕得五人。博復移書曰:「王卿憂公甚效!檄到,齎伐閱詣府。部掾以下亦可用,漸盡其餘矣。」其操持下,皆此類也。

以高弟入守左馮翊,滿歲為真。其治左馮翊,文理聰明殊不及薛宣,而多武譎,網絡張設,少愛利,敢誅殺。然亦縱舍,時有大貸,下吏以此為盡力。

長陵大姓尚方禁少時嘗盜人妻,見斫,創著其頰。府功曹受賂,白除禁調守尉。博聞知,以它事召見,視其面,果有瘢。博辟左右問禁:「是何等創也?」禁自知情得,叩頭服狀。博笑曰:「大丈夫固時有是。馮翊欲洒卿恥,抆拭用禁,能自效不?」禁且喜且懼,對曰:「必死!」博因敕禁:「毋得泄語,有便宜,輒記言。」因親信之以為耳目。禁晨夜發起部中盜賊及它伏姦,有功效。博擢禁連守縣令。久之,召見功曹,閉閤數責以禁等事,與筆札便自記,「積受取一錢以上,無得有所匿。欺謾半言,斷頭矣!」功曹惶怖,具自疏姦臧,大小不敢隱。博知其對以實,乃令就席,受敕自改而已。投刀使削所記,遣出就職。功曹後常戰栗,不敢蹉跌,博遂成就之。

遷為大司農。歲餘,坐小法,左遷犍為太守。先是南蠻若兒數為寇盜,博厚結其昆弟,使為反間,襲殺之,郡中清。

徙為山陽太守,病免官。復徵為光祿大夫,遷廷尉,職典決疑,當讞平天下獄。博恐為官屬所誣,視事,召見正監典法掾史,謂曰:「廷尉本起於武吏,不通法律,幸有眾賢,亦何憂!然廷尉治郡斷獄以來且二十年,亦獨耳剽日久,三尺律令,人事出其中。掾史試與正監共撰前世決事吏議難知者數十事,持以問廷尉,得諸君覆意之。」正監以為博苟強,意未必能然,即共條白焉。博皆召掾史,並坐而問,為平處其輕重,十中八九。官屬咸服博之疏略,材過人也。每遷徙易官,所到輒出奇譎如此,以明示下為不可欺者。

久之,遷後將軍,與紅陽侯立相善。立有罪就國,有司奏立黨友,博坐免。後歲餘,哀帝即位,以博名臣,召見,起家復為光祿大夫,遷為京兆尹,數月超為大司空。

初,漢興襲秦官,置丞相、御史大夫、太尉。至武帝罷太尉,始置大司馬以冠將軍之號,非有印綬官屬也。及成帝時,何武為九卿,建言「古者民樸事約,國之輔佐必得賢聖,然猶則天三光,備三公官,各有分職。今末俗文弊,政事煩多,宰相之材不能及古,而丞相獨兼三公之事,所以久廢而不治也。宜建三公官,定卿大夫之任,分職授政,以考功效。」其後上以問師安昌侯張禹,禹以為然。時曲陽侯王根為大司馬票騎將軍,而何武為御史大夫。於是上賜曲陽侯根大司馬印綬,置官屬,罷票騎將軍官,以御史大夫何武為大司空,封列侯,皆增奉如丞相,以備三公官焉。議者多以為古今異制,漢自天下之號下至佐史皆不同於古,而獨改三公,職事難分明,無益於治亂。是時御史府吏舍百餘區井水皆竭;又其府中列柏樹,常有野烏數千棲宿其上,晨去暮來,號曰「朝夕烏」,烏去不來者數月,長老異之。後二歲餘,朱博為大司空,奏言「帝王之道不必相襲,各繇時務。高皇帝以聖德受命,建立鴻業,置御史大夫,位次丞相,典正法度,以職相參,總領百官,上下相監臨,歷載二百年,天下安寧。今更為大司空,與丞相同位,未獲嘉祐。故事,選郡國守相高第為中二千石,選中二千石為御史大夫,任職者為丞相,位次有序,所以尊聖德,重國相也。今中二千石未更御史大夫而為丞相,權輕,非所以重國政也。臣愚以為大司空官可罷,復置御史大夫,遵奉舊制。臣願盡力,以御史大夫為百僚率。」哀帝從之,乃更拜博為御史大夫。會大司馬喜免,以陽安侯丁明為大司馬衛將軍,置官屬,大司馬冠號如故事。後四歲,哀帝遂改丞相為大司徒,復置大司空、大司馬焉。

初,何武為大司空,又與丞相方進共奏言:「古選諸侯賢者以為州伯,書曰『咨十有二牧』,所以廣聰明,燭幽隱也。今部刺史居牧伯之位,秉一州之統,選第大吏,所薦位高至九卿,所惡立退,任重職大。春秋之義,用貴治賤,不以卑臨尊。刺史位下大夫,而臨二千石,輕重不相準,失位次之序。臣請罷刺史,更置州牧,以應古制。」奏可,及博奏復御史大夫官,又奏言:「漢家至德溥大,宇內萬里,立置郡縣。部刺史奉使典州,督察郡國吏民安寧,故事居部九歲舉為守相,其有異材功效著者輒登擢,秩卑而賞厚,咸勸功樂進。前丞相方進奏罷刺史,更置州牧,秩真二千石,位次九卿。九卿缺,以高弟補,其中材則苟自守而已,恐功效陵夷,姦軌不禁。臣請罷州牧,置刺史如故。」奏可。

博為人廉儉,不好酒色游宴。自微賤至富貴,食不重味,案上不過三桮。夜寑早起,妻希見其面。有一女,無男。然好樂士大夫,為郡守九卿,賓客滿門,欲仕宦者薦舉之,欲報仇怨者解劍以帶之。其趨事待士如是,博以此自立,然終用敗。

初,哀帝祖母定陶太后欲求稱尊號,太后從弟高武侯傅喜為大司馬,與丞相孔光、大司空師丹共持正議。孔鄉侯傅晏亦太后從弟,諂諛欲順指,會博新徵用為京兆尹,與交結,謀成尊號,以廣孝道。繇是師丹先免,博代為大司空,數燕見奏封事,言「丞相光志在自守,不能憂國;大司馬喜至尊至親,阿黨大臣,無益政治。」上遂罷喜遣就國,免光為庶人,以博代光為丞相,封陽鄉侯,食邑二千戶。博上書讓曰:「故事封丞相不滿千戶,而獨臣過制,誠慚懼,願還千戶。」上許焉。傅太后怨傅喜不已,使孔鄉侯晏風丞相,令奏免喜侯。博受詔,與御史大夫趙玄議,玄言「事已前決,得無不宜?」博曰:「已許孔鄉侯有指。匹夫相要,尚相得死,何況至尊?博唯有死耳!」玄即許可。博惡獨斥奏喜,以故大司空氾鄉侯何武前亦坐過免就國,事與喜相似,即并奏:「喜、武前在位,皆無益於治,雖已退免,爵土之封非所當得也。請皆免為庶人。」上知傅太后素常怨喜,疑博、玄承指,即召玄詣尚書問狀。玄辭服,有詔左將軍彭宣與中朝者雜問。宣等劾奏:「博宰相,玄上卿,晏以外親封位特進,股肱大臣,上所信任,不思竭誠奉公,務廣恩化,為百寮先,皆知喜、武前已蒙恩詔決,事更三赦,博執左道,虧損上恩,以結信貴戚,背君鄉臣,傾亂政治,姦人之雄,附下罔上,為臣不忠不道;玄知博所言非法,枉義附從,大不敬;晏與博議免喜,失禮不敬。臣請詔謁者召博、玄、晏詣廷尉詔獄。」制曰:「

將軍、中二千石、二千石、諸大夫、博士、議郎議。」右將軍蟜望等四十四人以為「如宣等言,可許。」諫大夫龔勝等十四人以為「春秋之義,姦以事君,常刑不舍。魯大夫叔孫僑如欲顓公室,譖其族兄季孫行父於晉,晉執囚行父以亂魯國,春秋重而書之。今晏放命圮族,干亂朝政,要大臣以罔上,本造計謀,職為亂階,宜與博、玄同罪,罪皆不道。」上減玄死罪三等,削晏戶四分之一,假謁者節召丞相詣廷尉詔獄。博自殺,國除。

初博以御史為丞相,封陽鄉侯,玄以少府為御史大夫,並拜於前殿,延登受策,有音如鍾聲。語在五行志。

贊曰:薛宣、朱博皆起佐史,歷位以登宰相。宣所在而治,為世吏師,及居大位,以苛察失名,器誠有極也。博馳騁進取,不思道德,已亡可言,又見孝成之世委任大臣,假借用權。世主已更,好惡異前,復附丁、傅,稱順孔鄉。事發見詰,遂陷誣罔,辭窮情得,仰藥飲鴆。孔子曰:「久矣哉,由之行詐也!」博亦然哉!


\end{pinyinscope}