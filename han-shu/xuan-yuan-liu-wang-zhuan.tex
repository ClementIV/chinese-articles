\article{宣元六王傳}

\begin{pinyinscope}
孝宣皇帝五男。許皇后生孝元帝,張婕妤生淮陽憲王欽,衛婕妤生楚孝王囂,公孫婕妤生東平思王宇,戎婕妤生中山哀王竟。

淮陽憲王欽,元康三年立,母張婕妤有寵於宣帝。霍皇后廢後,上欲立張婕妤為后。久之,懲艾霍氏欲害皇太子,乃更選後宮無子而謹慎者,乃立長陵王婕妤為后,令母養太子。后無寵,希御見,唯張婕妤最幸。而憲王壯大,好經書法律,聰達有材,帝甚愛之。太子寬仁,喜儒術,上數嗟歎憲王,曰:「真我子也!」常有意欲立張婕妤與憲王,然用太子起於微細,上少依倚許氏,及即位而許后以殺死,太子蚤失母,故弗忍也。久之,上以故丞相韋賢子玄成陽狂讓侯兄,經明行高,稱於朝廷,乃召拜玄成為淮陽中尉,欲感諭憲王,輔以推讓之臣,由是太子遂安。宣帝崩,元帝即位,乃遣憲王之國。

時張婕妤已卒,憲王有外祖母,舅張博兄弟三人歲至淮陽見親,輒受王賜。後王上書:請徙外家張氏於國,博上書:願留守墳墓,獨不徙。王恨之。後博至淮陽,王賜之少。博言:「負責數百萬,願王為償。」王不許。博辭去,令弟光恐王云王遇大人益解,博欲上書為大人乞骸骨去。王乃遣人持黃金五十斤送博。博喜,還書謝,為諂語盛稱譽王,因言:「當今朝廷無賢臣,災變數見,足為寒心。萬姓咸歸望於大王,大王奈何恬然不求入朝見,輔助主上乎?」使弟光數說王宜聽博計,令於京師說用事貴人為王求朝。王不納其言。

後光欲至長安,辭王,復言「願盡力與博共為王求朝。王即日至長安,可因平陽侯。」光得王欲求朝語,馳使人語博。博知王意動,復遺王書曰:「博幸得肺腑,數進愚策,未見省察。北游燕趙,欲循行郡國求幽隱之士,聞齊有駟先生者,善為司馬兵法,大將之材也,博得謁見,承間進問五帝三王究竟要道,卓爾非世俗之所知。今邊境不安,天下騷動,微此人其莫能安也。又聞北海之瀕有賢人焉,累世不可逮,然難致也。得此二人而薦之,功亦不細矣。博願馳西以此赴助漢急,無財幣以通顯之。趙王使謁者持牛酒,黃金三十斤勞博,博不受;復使人願尚女,聘金二百斤,博未許。會得光書云大王已遣光西,與博并力求朝。博自以棄捐,不意大王還意反義,結以朱顏,願殺身報德。朝事何足言!大王誠賜咳唾,使得盡死,湯禹所以成大功也。駟先生蓄積道術,書無不有,願知大王所好,請得輒上。」王得書喜說,報博書曰:「子高乃幸左顧存恤,發心惻隱,顯至誠,納以嘉謀,語以至事,雖亦不敏,敢不諭意!今遣有司為子高償責二百萬。」

是時,博女婿京房以明易陰陽得幸於上,數召見言事。自謂為石顯、五鹿充宗所排,謀不得用,數為博道之。博常欲誑耀淮陽王,即具記房諸所說災異及召見密語,持予淮陽王以為信驗,詐言「已見中書令石君求朝,許以金五百斤。賢聖制事,蓋慮功而不計費。昔禹治鴻水,百姓罷勞,成功既立,萬世賴之。今聞陛下春秋未滿四十,髮齒墮落,太子幼弱,佞人用事,陰陽不調,百姓疾疫飢饉死者且半,鴻水之害殆不過此。大王緒欲救世,將比功德,何可以忽?博已與大儒知道者為大王為便宜奏,陳安危,指災異,大王朝見,先口陳其意而後奏之,上必大說。事成功立,大王即有周、邵之名,邪臣散亡,公卿變節,功德亡比,而梁、趙之寵必歸大王,外家亦將富貴,何復望大王之金錢?」王喜說,報博書曰:「乃者詔下,止諸侯朝者,寡人憯然不知所出。子高素有顏冉之資,臧武之智,子貢之辯,卞莊子之勇,兼此四者,世之所鮮。既開端緒,願卒成之。求朝,義事也,奈何行金錢乎!」博報曰:「已許石君,須以成事。」王以金五百斤予博。

會房出為郡守,離左右,顯具得此事告之。房漏泄省中語,博兄弟詿誤諸侯王,誹謗政治,狡猾不道,皆下獄。有司奏請逮捕欽,上不忍致法,遣諫大夫王駿賜欽璽書曰:「皇帝問淮陽王。有司奏王,王舅張博數遺王書,非毀政治,謗訕天子,褒舉諸侯,稱引周、湯。以諂惑王,所言尤惡,悖逆無道。王不舉奏而多與金錢,報以好言,罪至不赦,朕惻焉不忍聞,為王傷之。推原厥本,不祥自博,惟王之心,匪同于凶。已詔有司勿治王事,遣諫大夫駿申諭朕意。詩不云乎?『靖恭爾位,正直是與。』王其勉之!」

駿諭指曰:「禮為諸侯制相朝聘之義,蓋以考禮壹德,尊事天子也。且王不學詩乎?《詩》云:『俾侯於魯,為周室輔。』今王舅博數遺王書,所言悖逆。王幸受詔策,通經術,知諸侯名譽不當出竟。天子普覆,德布於朝,而恬有博言,多予金錢,與相報應,不忠莫大焉。故事,諸侯王獲罪京師,罪惡輕重,縱不伏誅,必蒙遷削貶黜之罪,未有但已者也。今聖主赦王之罪,又憐王失計忘本,為博所惑,加賜璽書,使諫大夫申諭至意,殷勤之恩,豈有量哉!博等所犯罪惡大,群下之所共攻,王法之所不赦也。自今以來,王毋復以博等累心,務與眾棄之。春秋之義,大能變改。《易》曰『藉用白茅,无咎』,言臣子之道,改過自新,絜己以承上,然後免於咎也。王其留意慎戒,惟思所以悔過易行,塞重責,稱厚恩者。如此,則長有富貴,社稷安矣。」

於是淮陽王欽免冠稽首謝曰:「奉藩無狀,過惡暴列,陛下不忍致法,加大恩,遣使者申諭道術守藩之義。伏念博罪惡尤深,當伏重誅。臣欽願悉心自新,奉承詔策。頓首死罪。」

京房及博兄弟三人皆棄市,妻子徙邊。

至成帝即位,以淮陽王屬為叔父,敬寵之,異於它國。王上書自陳舅張博時事,頗為石顯等所侵,因為博家屬徙者求還。丞相御史復劾欽:「前與博相遺私書,指意非諸侯王所宜,蒙恩勿治,事在赦前。不悔過而復稱引,自以為直,失藩臣體,不敬。」上加恩,許王還徙者。

三十六年薨。子文王玄嗣,二十六年薨。子縯嗣,王莽時絕。

楚孝王囂,甘露二年立為定陶王,三年徙楚。成帝河平中入朝,時被疾,天子閔之,下詔曰:「蓋聞『天地之性人為貴,人之行莫大於孝』。楚王囂素行孝順仁慈,之國以來二十餘年,孅介之過未嘗聞,朕甚嘉之。今乃遭命,離于惡疾,夫子所痛,曰:『

蔑之,命矣夫,斯人也而有斯疾也!』朕甚閔焉。夫行純茂而不顯異,則有國者將何勗哉?書不云乎?『用德章厥善。』今王朝正月,詔與子男一人俱,其以廣戚縣戶四千三百封其子勳為廣戚侯。」明年,囂薨。子懷王文嗣,一年薨,無子,絕。明年,成帝復立文弟平陸侯衍,是為思王。二十一年薨,子紆嗣,王莽時絕。

初,成帝時又立紆弟景為定陶王。廣戚侯勳薨,諡曰煬侯,子顯嗣。平帝崩,無子,王莽立顯子嬰為孺子,奉平帝後。莽篡位,以嬰為定安公。漢既誅莽,更始時嬰在長安,平陵方望等頗知天文,以為更始必敗,嬰本統當立者也,共起兵將嬰至臨涇,立為天子。更始遣丞相李松擊破殺嬰云。

東平思王宇,甘露二年立。元帝即位,就國。壯大,通姦犯法,上以至親貰弗罪,傅相連坐。

久之,事太后,內不相得,太后上書言之,求守杜陵園。上於是遣太中大夫張子蟜奉璽書敕諭之,曰:「皇帝問東平王。蓋聞親親之恩莫重於孝,尊尊之義莫大於忠,故諸侯在位不驕以致孝道,制節謹度以翼天子,然後富貴不離於身,而社稷可保。今聞王自修有闕,本朝不和,流言紛紛,謗自內興,朕甚憯焉,為王懼之。詩不云乎?『毋念爾祖,述修厥德,永言配命,自求多褔。』朕惟王之春秋方剛,忽於道德,意有所移,忠言未納,故臨遣太中大夫子蟜諭王朕意。孔子曰:「過而不改,是謂過矣。』王其深惟孰思之,無違朕意。」

又特以璽書賜王太后,曰:「皇帝使諸吏宦者令承問東平王太后。朕有聞,王太后少加意焉。夫褔善之門莫美於和睦,患咎之首莫大於內離。今東平王出繈褓之中而託于南面之位,加以年齒方剛,涉學日寡,驁忽臣下,不自它於太后,以是之間,能無失禮義者,其唯聖人乎!傳曰:『父為子隱,直在其中矣。』王太后明察此意,不可不詳。閨門之內,母子之間,同氣異息,骨肉之恩,豈可忽哉!豈可忽哉!昔周公戒伯禽曰:『故舊無大故,則不可棄也,毋求備於一人。』夫以故舊之恩,猶忍小惡,而況此乎!已遣使者諭王,王既悔過服罪,太后寬忍以貰之,後宜不敢。王太后強餐,止思念,慎疾自愛。」

宇慚懼,因使者頓首謝死罪,願洒心自改。詔書又敕傅相曰:「夫人之性皆有五常,及其少長,耳目牽於耆欲,故五常銷而邪心作,情亂其性,利勝其義,而不失厥家者,未之有也。今王富於春秋,氣力勇武,獲師傅之教淺,加以少所聞見,自今以來,非五經之正術,敢以游獵非禮道王者,輒以名聞。」

宇立二十年,元帝崩。宇謂中謁者信等曰:「漢大臣議天子少弱,未能治天下,以為我知文法,建欲使我輔佐天子。我見尚書晨夜極苦,使我為之,不能也。今暑熱,縣官年少,持服恐無處所,我危得之!」比至下,宇凡三哭,飲酒食肉,妻妾不離側。又姬朐臑故親幸,後疏遠,數歎息呼天。宇聞,斥朐臑為家人子,掃除永巷,數笞擊之。朐臑私疏宇過失,數令家告之。宇覺知,絞殺朐臑。有詔奏請逮捕,有詔削樊、亢父二縣。後三歲,天子詔有司曰:「蓋聞仁以親親,古之道也。前東平王有闕,有司請廢,朕不忍。又請削,朕不敢專。惟王之至親,未嘗忘於心。今聞王改行自新,尊修經術,親近仁人,非法之求,不以奸吏,朕甚嘉焉。傳不云乎?朝過夕改,君子與之。其復前所削縣如故。」

後年來朝,上疏求諸子及太史公書,上以問大將軍王鳳,對曰:「臣聞諸侯朝聘,考文章,正法度,非禮不言。今東平王幸得來朝,不思制節謹度,以防危失,而求諸書,非朝聘之義也。諸子書或反經術,非聖人,或明鬼神,信物怪;太史公書有戰國從橫權譎之謀,漢興之初謀臣奇策,天官災異,地形阨塞:皆不宜在諸侯王。不可予。不許之辭宜曰:『五經聖人所制,萬事靡不畢載。王審樂道,傅相皆儒者,旦夕講誦,足以正身虞意。夫小辯破義,小道不通,致遠恐泥,皆不足以留意。諸益於經術者,不愛於王。』」對奏,天子如鳳言,遂不與。

立三十三年薨,子煬王雲嗣。哀帝時,無鹽危山土自起覆草,如馳道狀,又瓠山石轉立。雲及后謁自之石所祭,治石象瓠山立石,束倍草,并祠之。建平三年,息夫躬、孫寵等共因幸臣董賢告之。是時,哀帝被疾,多所惡,事下有司,逮王、后謁下獄驗治,言使巫傅恭、婢合歡等祠祭詛祝上,為雲求為天子。雲又與知災異者高尚等指星宿,言上疾必不愈,雲當得天下。石立,宣帝起之表也。有司請誅王,有詔廢徙房陵。雲自殺,謁棄市。立十七年,國除。

元始元年,王莽欲反哀帝政,白太皇太后,立雲太子開明為東平王,又立思王孫成都為中山王。開明立三年,薨,無子。復立開明兄嚴鄉侯信子匡為東平王,奉開明後。王莽居攝,東郡太守翟義與嚴鄉侯信謀舉兵誅莽,立信為天子。兵敗,皆為莽所滅。

中山哀王竟,初元二年立為清河王。三年,徙中山,以幼少未之國。建昭四年,薨邸,葬杜陵,無子,絕。太后歸居外家戎氏。

孝元皇帝三男。王皇后生孝成帝,傅昭儀生定陶共王康,馮昭儀生中山孝王興。

定陶共王康,永光三年立為濟陽王。八年,徙為山陽王。八年,徙定陶。王少而愛,長多材藝,習知音聲,上奇器之。母昭儀又幸,幾代皇后太子。語在元后及史丹傳。

成帝即位,緣先帝意,厚遇異於它王。十九年薨,子欣嗣。十五年,成帝無子,徵入為皇太子。上以太子奉大宗後,不得顧私親,乃立楚思王子景為定陶王,奉共王後。成帝崩,太子即位,是為孝哀帝。即位二年,追尊共王為共皇,置寢廟京師,序昭穆,儀如孝元帝。徙定陶王景為信都王云。

中山孝王興,建昭二年王為信都王。十四年,徙中山。成帝之議立太子也,御史大夫孔光以為尚書有殷及王,兄終弟及,中山王元帝之子,宜為後。成帝以中山王不材,又兄弟,不得相入廟。外家王氏與趙昭儀皆欲用哀帝為太子,故遂立焉。上乃封孝王舅馮參為宜鄉侯,而益封孝王萬戶,以尉其意。三十年,薨,子衎嗣。七年,哀帝崩,無子,徵中山王衎入即位,是為平帝。太皇太后以帝為成帝後,故立東平思王孫桃鄉頃侯子成都為中山王,奉孝王後。王莽時絕。

贊曰:孝元之後,遍有天下,然而世絕於孫,豈非天哉!淮陽憲王於時諸侯為聰察矣,張博誘之,幾陷無道。《詩》云「貪人敗類」,古今一也。


\end{pinyinscope}