\article{趙尹韓張兩王傳}

\begin{pinyinscope}
趙廣漢字子都,涿郡蠡吾人也,故屬河間。少為郡吏、州從事,以廉絜通敏下士為名。舉茂材,平準令。察廉為陽翟令。以治行尤異,遷京輔都尉,守京兆尹。會昭帝崩,而新豐杜建為京兆掾,護作平陵方上。建素豪俠,賓客為姦利,廣漢聞之,先風告。建不改,於是收案致法。中貴人豪長者為請無不至,終無所聽。宗族賓客謀欲篡取,廣漢盡知其計議主名起居,使吏告曰:「若計如此,且并滅家。」令數吏將建棄巿,莫敢近者。京師稱之。

是時,昌邑王徵即位,行淫亂,大將軍霍光與群臣共廢王,尊立宣帝。廣漢以與議定策,賜爵關內侯。

遷潁川太守。郡大姓原、褚宗族橫恣,賓客犯為盜賊,前二千石莫能禽制。廣漢既至數月,誅原、褚首惡,郡中震栗。

先是,潁川豪桀大姓相與為婚姻,吏俗朋黨。廣漢患之,厲使其中可用者受記,出有案問,既得罪名,行法罰之,廣漢故漏泄其語,令相怨咎。又教吏為缿筩,及得投書,削其主名,而託以為豪桀大姓子弟所言。其後彊宗大族家家結為仇讎,姦黨散落,風俗大改。吏民相告訐,廣漢得以為耳目,盜賊以故不發,發又輒得。壹切治理,威名流聞,及匈奴降者言匈奴中皆聞廣漢。

本始二年,漢發五將軍擊匈奴,徵廣漢以太守將兵,屬蒲類將軍趙充國。從軍還,復用守京兆尹,滿歲為真。

廣漢為二千石,以和顏接士,其尉薦待遇吏,殷勤甚備。事推功善,歸之於下,曰:「某掾卿所為,非二千石所及。」行之發於至誠。吏見者皆輸寫心腹,無所隱匿,咸願為用,僵仆無所避。廣漢聰明,皆知其能之所宜,盡力與否。其或負者,輒先聞知,風諭不改,乃收捕之,無所逃,按之罪立具,即時伏辜。

廣漢為人彊力,天性精於吏職。見吏民,或夜不寢至旦。尤善為鉤距,以得事情。鉤距者,設欲知馬賈,則先問狗,已問羊,又問牛,然後及馬,參伍其賈,以類相準,則知馬之貴賤不失實矣。唯廣漢至精能行之,它人效者莫能及也。郡中盜賊,閭里輕俠,其根株窟穴所在,及吏受取請求銖兩之姦,皆知之。長安少年數人會窮里空舍謀共劫人,坐語未訖,廣漢使吏捕治具服。富人蘇回為郎,二人劫之。有頃,廣漢將吏到家,自立庭下,使長安丞襲奢叩堂戶曉賊,曰:「京兆尹趙君謝兩卿,無得殺質,此宿衛臣也。釋質,束手,得善相遇,幸逢赦令,或時解脫。」二人驚愕,又素聞廣漢名,即開戶出,下堂叩頭,廣漢跪謝曰:「幸全活郎,甚厚!」送獄,敕吏謹遇,給酒肉。至冬當出死,豫為調棺,給斂葬具,告語之,皆曰:「死無所恨!」

廣漢嘗記召湖都亭長,湖都亭長西至界上,界上亭長戲曰:「至府,為我多謝問趙君。」亭長既至,廣漢與語,問事畢,謂曰:「界上亭長寄聲謝我,何以不為致問?」亭長叩頭服實有之。廣漢因曰:「還為吾謝界上亭長,勉思職事,有以自效,京兆不忘卿厚意。」其發姦擿伏如神,皆此類也。

廣漢奏請,令長安游徼獄吏秩百石,其後百石吏皆差自重,不敢枉法妄繫留人。京兆政清,吏民稱之不容口。長老傳以為自漢興以來治京兆者莫能及。左馮翊、右扶風皆治長安中,犯法者從跡喜過京兆界。廣漢歎曰:「亂吾治者,常二輔也!誠令廣漢得兼治之,直差易耳。」

初,大將軍霍光秉政,廣漢事光。及光薨後,廣漢心知微指,發長安吏自將,與俱至光子博陸侯禹第,直突入其門,廋索私屠酤,椎破盧罌,斧斬其門關而去。時光女為皇后,聞之,對帝涕泣。帝心善之,以召問廣漢。廣漢由是侵犯貴戚大臣。所居好用世吏子孫新進年少者,專厲彊壯鋒氣,見事風生,無所回避,率多果敢之計,莫為持難。廣漢終以此敗。

初,廣漢客私酤酒長安巿,丞相史逐去客。客疑男子蘇賢言之,以語廣漢。廣漢使長安丞按賢,尉史禹故劾賢為騎士屯霸上,不詣屯所,乏軍興。賢父上書訟罪,告廣漢,事下有司覆治。禹坐要斬,請逮捕廣漢。有詔即訊,辭服,會赦,貶秩一等。廣漢疑其邑子榮畜教令,後以它法論殺畜。人上書言之,事下丞相御史,案驗甚急。廣漢使所親信長安人為丞相府門卒,令微司丞相門內不法事。地節三年七月中,丞相傅婢有過,自絞死。廣漢聞之,疑丞相夫人妒殺之府舍。而丞相奉齋酎入廟祠,廣漢得此,使中郎趙奉壽風曉丞相,欲以脅之,毋令窮正己事。丞相不聽,按驗愈急。廣漢欲告之,先問太史知星氣者,言今年當有戮死大臣,廣漢即上書告丞相罪。制曰:「下京兆尹治。」廣漢知事迫切,遂自將吏卒突入丞相府,召其夫人跪庭下受辭,收奴婢十餘人去,責以殺婢事。丞相魏相上書自陳:「妻實不殺婢。廣漢數犯罪法不伏辜,以詐巧迫脅臣相,幸臣相寬不奏。願下明使者治廣漢所驗臣相家事。」事下廷尉治罪,實丞相自以過譴笞傅婢,出至外弟乃死,不如廣漢言。司直蕭望之劾奏:「廣漢摧辱大臣,欲以劫持奉公,逆節傷化,不道。」宣帝惡之,下廣漢廷尉獄,又坐賊殺不辜,鞠獄故不以實,擅斥除騎士乏軍興數罪。天子可其奏。吏民守闕號泣者數萬人,或言「臣生無益縣官,願代趙京兆死,使得牧養小民。」廣漢竟坐要斬。

廣漢雖坐法誅,為京兆尹廉明,威制豪彊,小民得職。百姓追思,歌之至今。

尹翁歸字子兄,河東平陽人也,徙杜陵。翁歸少孤,與季父居。為獄小吏,曉習文法。喜擊劍,人莫能當。是時大將軍霍光秉政,諸霍在平陽,奴客持刀兵入巿鬥變,吏不能禁,及翁歸為巿吏,莫敢犯者。公廉不受餽,百賈畏之。

後去吏居家。會田延年為河東太守,行縣至平陽,悉召故吏五六十人,延年親臨見,令有文者東,有武者西。閱數十人,次到翁歸,獨伏不肯起,對曰:「翁歸文武兼備,唯所施設。」功曹以為此吏倨敖不遜,延年曰:「何傷?」遂召上辭問,甚奇其對,除補卒史,便從歸府。案事發姦,窮竟事情,延年大重之,自以能不及翁歸,徙署督郵。河東二十八縣,分為兩部,閎孺部汾北,翁歸部汾南。所舉應法,得其罪辜,屬縣長吏雖中傷,莫有怨者。舉廉為緱氏尉,歷守郡中,所居治理,遷補都內令,舉廉為弘農都尉。

徵拜東海太守,過辭廷尉于定國。定國家在東海,欲屬託邑子兩人,令坐後堂待見。定國與翁歸語終日,不敢見其邑子。既去,定國乃謂邑子曰:「此賢將,汝不任事也,又不可干以私。」

翁歸治東海明察,郡中吏民賢不肖,及姦邪罪名盡知之。縣縣各有記籍。自聽其政,有急名則少緩之;吏民小解,輒披籍。縣縣收取黠吏豪民,案致其罪,高至於死。收取人必於秋冬課吏大會中,及出行縣,不以無事時。其有所取也,以一警百,吏民皆服,恐懼改行自新。東海大豪郯許仲孫為姦猾,亂吏治,郡中苦之。二千石欲捕者,輒以力勢變詐自解,終莫能制。翁歸至,論棄仲孫巿,一郡怖栗,莫敢犯禁。東海大治。

以高第入守右扶風,滿歲為真。選用廉平疾姦吏以為右職,接待以禮,好惡與同之;其負翁歸,罰亦必行。治如在東海故跡,姦邪罪名亦縣縣有名籍。盜賊發其比伍中,翁歸輒召其縣長吏,曉告以姦黠主名,教使用類推跡盜賊所過抵,類常如翁歸言,無有遺託。緩於小弱,急於豪彊。豪彊有論罪,輸掌畜官,使斫莝,責以員程,不得取代。不中程,輒笞督,極者至以鈇自剄而死。京師畏其威嚴,扶風大治,盜賊課常為三輔最。

翁歸為政雖任刑,其在公卿之間清絜自守,語不及私,然溫良嗛退,不以行能驕人,甚得名譽於朝廷。視事數歲,元康四年病卒。家無餘財,天子賢之,制詔御史:「朕夙興夜寐,以求賢為右,不異親疏近遠,務在安民而已。扶風翁歸廉平鄉正,治民異等,早夭不遂,不得終其功業,朕甚憐之。其賜翁歸子黃金百斤,以奉其祭祠。」

翁歸三子皆為郡守。少子岑歷位九卿,至後將軍。而閎孺亦至廣陵相,有治名。由是世稱田延年為知人。

韓延壽字長公,燕人也,徙杜陵。少為郡文學。父義為燕郎中。剌王之謀逆也,義諫而死,燕人閔之。是時昭帝富於春秋,大將軍霍光持政,徵郡國賢良文學,問以得失。時魏相以文學對策,以為「賞罰所以勸善禁惡,政之本也。日者燕王為無道,韓義出身彊諫,為王所殺。義無比干之親而蹈比干之節,宜顯賞其子,以示天下,明為人臣之義。」光納其言,因擢延壽為諫大夫,遷淮陽太守。治甚有名,徙潁川。

潁川多豪彊,難治,國家常為選良二千石。先是,趙廣漢為太守,患其俗多朋黨,故構會吏民,令相告訐,一切以為聰明,潁川由是以為俗,民多怨讎。延壽欲更改之,教以禮讓,恐百姓不從,乃歷召郡中長老為鄉里所信向者數十人,設酒具食,親與相對,接以禮意,人人問以謠俗,民所疾苦,為陳和睦親愛銷除怨咎之路。長老皆以為便,可施行,因與議定嫁娶喪祭儀品,略依古禮,不得過法。延壽於是令文學校官諸生皮弁執俎豆,為吏民行喪嫁娶禮。百姓遵用其教,賣偶車馬下里偽物者,棄之巿道。數年,徙為東郡太守,黃霸代延壽居潁川,霸因其跡而大治。

延壽為吏,上禮義,好古教化,所至必聘其賢士,以禮待用,廣謀議,納諫爭;舉行喪讓財,表孝弟有行;修治學官,春秋鄉社,陳鍾鼓管弦,盛升降揖讓,及都試講武,設斧鉞旌旗,習射御之事。治城郭,收賦租,先明布告其日,以期會為大事,吏民敬畏趨鄉之。又置正、五長,相率以孝弟,不得舍姦人。閭里仟佰有非常,吏輒聞知,姦人莫敢入界。其始若煩,後吏無追捕之苦,民無箠楚之憂,皆便安之。接待下吏,恩施甚厚而約誓明。或欺負之者,延壽痛自刻責:「豈其負之,何以至此?」吏聞者自傷悔,其縣尉至自刺死。及門下掾自剄,人救不殊,因瘖不能言。延壽聞之,對掾史涕泣,遣吏毉治視,厚復其家。

延壽嘗出,臨上車,騎吏一人後至,敕功曹議罰白。還至府門,門卒當車,願有所言。延壽止車問之,卒曰:「孝經曰:『資於事父以事君,而敬同,故母取其愛,而君取其敬,兼之者父也。』今旦明府早駕,久駐未出,騎吏父來至府門,不敢入。騎吏聞之,趨走出謁,適會明府登車。以敬父而見罰,得毋虧大化乎?」延壽舉手輿中曰:「微子,太守不自知過。」歸舍,召見門卒。卒本諸生,聞延壽賢,無因自達,故代卒,延壽遂待用之。其納善聽諫,皆此類也。在東郡三歲,令行禁止,斷獄大減,為天下最。

入守左馮翊,滿歲稱職為真。歲餘,不肯出行縣。丞掾數白:「宜循行郡中,覽觀民俗,考長吏治跡。」延壽曰:「縣皆有賢令長,督郵分明善惡於外,行縣恐無所益,重為煩擾。」丞掾皆以為方春月,可壹出勸耕桑。延壽不得已,行縣至高陵,民有昆弟相與訟田自言,延壽大傷之,曰:「幸得備位,為郡表率,不能宣明教化,至令民有骨肉爭訟,既傷風化,重使賢長吏、嗇夫、三老、孝弟受其恥,咎在馮翊,當先退。」是日移病不聽事,因入臥傳舍,閉閤思過。一縣莫知所為,令丞、嗇夫、三老亦皆自繫待罪。於是訟者宗族傳相責讓,此兩昆弟深自悔,皆自髡肉袒謝,願以田相移,終死不敢復爭。延壽大喜,開閤延見,內酒肉與相對飲食,厲勉以意告鄉部,有以表勸悔過從善之民。延壽乃起聽事,勞謝令丞以下,引見尉薦。郡中歙然,莫不傳相敕厲,不敢犯。延壽恩信周遍二十四縣,莫復以辭訟自言者。推其至誠,吏民不忍欺紿。

延壽代蕭望之為左馮翊,而望之遷御史大夫。侍謁者福為望之道延壽在東郡時放散官錢千餘萬。望之與丞相丙吉議,吉以為更大赦,不須考。會御史當問事東郡,望之因令并問之。延壽聞知,即部吏案校望之在馮翊時廩犧官錢放散百餘萬。廩犧吏掠治急,自引與望之為姦。延壽劾奏,移殿門禁止望之。望之自奏「職在總領天下,聞事不敢不問,而為延壽所拘持。」上由是不直延壽,各令窮竟所考。望之卒無事實,而望之遣御史案東郡,具得其事。延壽在東郡時,試騎士,治飾兵車,畫龍虎朱爵。延壽衣黃紈方領,駕四馬,傅總,建幢棨,植羽葆,鼓車歌車。功曹引車,皆駕四馬,載棨戟。五騎為伍,分左右部,軍假司馬、千人持幢旁轂。歌者先居射室,望見延壽車,噭咷楚歌。延壽坐射室,騎吏持戟夾陛列立,騎士從者帶弓鞬羅後。令騎士兵車四面營陳,被甲鞮鞪居馬上,抱弩負籣。又使騎士戲車弄馬盜驂。延壽又取官銅物,候月蝕鑄作刀劍鉤鐔,放效尚方事。及取官錢帛,私假繇使吏。及治飾車甲三百萬以上。

於是望之劾奏延壽上僭不道,又自陳:「前為延壽所奏,今復舉延壽罪,眾庶皆以臣懷不正之心,侵冤延壽。願下丞相、中二千石、博士議其罪。」事下公卿,皆以延壽前既無狀,後復誣愬典法大臣,欲以解罪,狡猾不道。天子惡之,延壽竟坐棄市。吏民數千人送至渭城,老小扶持車轂,爭奏酒炙。延壽不忍距逆,人人為飲,計飲酒石餘。使掾史分謝送者:「遠苦吏民,延壽死無所恨。」百姓莫不流涕。

延壽三子皆為郎吏。且死,屬其子勿為吏,以己為戒。子皆以父言去官不仕。至孫威,乃復為吏至將軍。威亦多恩信,能拊眾,得士死力。威又坐奢僭誅,延壽之風類也。

張敞字子高,本河東平陽人也。祖父孺為上谷太守,徙茂陵。敞父福事孝武帝,官至光祿大夫。敞後隨宣帝徙杜陵。敞本以鄉有秩補太守卒史,察廉為甘泉倉長,稍遷太僕丞,杜延年甚奇之。會昌邑王徵即位,動作不由法度,敞上書諫曰:「孝昭皇帝蚤崩無嗣,大臣憂懼,選賢聖承宗廟,東迎之日,唯恐屬車之行遲。今天子以盛年初即位,天下莫不拭目傾耳,觀化聽風。國輔大臣未褒,而昌邑小輦先遷,此過之大者也。」後十餘日王賀廢,敞以切諫顯名,擢為豫州刺史。以數上事有忠言,宣帝徵敞為太中大夫,與于定國並平尚書事。以正違忤大將軍霍光,而使主兵車出軍省減用度,復出為函谷關都尉。宣帝初即位,廢王賀在昌邑,上心憚之,徙敞為山陽太守。

久之,大將軍霍光薨,宣帝始親政事,封光兄孫山、雲皆為列侯,以光子禹為大司馬。頃之,山、雲以過歸第,霍氏諸婿親屬頗出補吏。敞聞之,上封事曰:「臣聞公子季友有功於魯,大夫趙衰有功於晉,大夫田完有功於齊,皆疇其官邑,延及子孫,終後田氏篡齊,趙氏分晉,季氏顓魯。故仲尼作春秋,跡盛衰,譏世卿最甚。乃者大將軍決大計,安宗廟,定天下,功亦不細矣。夫周公七年耳,而大將軍二十歲,海內之命,斷於掌握。方其隆時,感動天地,侵迫陰陽,月朓日蝕,晝冥宵光,地大震裂,火生地中,天文失度,祅祥變怪,不可勝記,皆陰類盛長,臣下顓制之所生也。朝臣宜有明言,曰陛下褒寵故大將軍以報功德足矣。間者輔臣顓政,貴戚太盛,君臣之分不明,請罷霍氏三侯皆就弟。及衛將軍張安世,宜賜几杖歸休,時存問召見,以列侯為天子師。明詔以恩不聽,群臣以義固爭而後許,天子必以陛下為不忘功德,而朝臣為知禮,霍氏世世無所患苦。今朝廷不聞直聲,而令明詔自親其文,非策之得者也。今兩侯以出,人情不相遠,以臣心度之,大司馬及其枝屬必有畏懼之心。夫近臣自危,非完計也,臣敞願於廣朝白發其端,直守遠郡,其路無由。夫心之精微口不能言也,言之微眇書不能文也,故伊尹五就桀,五就湯,蕭相國薦淮陰累歲乃得通,況乎千里之外,因書文諭事指哉!唯陛下省察。」上甚善其計,然不徵也。

久之,勃海、膠東盜賊並起,敞上書自請治之,曰:「臣聞忠孝之道,退家則盡心於親,進宦則竭力於君。夫小國中君猶有奮不顧身之臣,況於明天子乎!今陛下遊意於太平,勞精於政事,亹亹不舍晝夜。群臣有司宜各竭力致身。山陽郡戶九萬三千,口五十萬以上,訖計盜賊未得者七十七人,它課諸事亦略如此。臣敞愚駑,既無以佐思慮,久處閒郡,身逸樂而忘國事,非忠孝之節也。伏聞膠東、勃海左右郡歲數不登,盜賊並起,至攻官寺,篡囚徒,搜市朝,劫列侯。吏失綱紀,姦軌不禁。臣敞不敢愛身避死,唯明詔之所處,願盡力摧挫其暴虐,存撫其孤弱。事即有業,所至郡條奏其所由廢及所以興之狀。」書奏,天子徵敞,拜膠東相,賜黃金三十斤。敞辭之官,自請治劇郡非賞罰無以勸善懲惡,吏追捕有功效者,願得壹切比三輔尤異。天子許之。

敞到膠東,明設購賞,開群盜令相捕斬除罪。吏追捕有功,上名尚書調補縣令者數十人。由是盜賊解散,傳相捕斬。吏民歙然,國中遂平。

居頃之,王太后數出游獵,敞奏書諫曰:「臣聞秦王好淫聲,葉陽后為不聽鄭衛之樂;楚嚴好田獵,樊姬為之不食鳥獸之肉。口非惡旨甘,耳非憎絲竹也,所以抑心意,絕耆欲者,將以率二君而全宗祀也。禮,君母出門則乘輜軿,下堂則從傅母,進退則鳴玉佩,內飾則結綢繆。此言尊貴所以自斂制,不從恣之義也。今太后資質淑美,慈愛寬仁,諸侯莫不聞,而少以田獵縱欲為名,於以上聞,亦未宜也。唯觀覽於往古,全行乎來今,令后姬得有所法則,下臣有所稱誦,臣敞幸甚!」書奏,太后止不復出。

是時潁川太守黃霸以治行第一入守京兆尹。霸視事數月,不稱,罷歸潁川。於是制詔御史:「其以膠東相敞守京兆尹。」自趙廣漢誅後,比更守尹,如霸等數人,皆不稱職。京師寖廢,長安市偷盜尤多,百賈苦之。上以問敞,敞以為可禁。敞既視事,求問長安父老,偷盜酋長數人,居皆溫厚,出從童騎,閭里以為長者。敞皆召見責問,因貰其罪,把其宿負,令致諸偷以自贖。偷長曰:「今一旦召詣府,恐諸偷驚駭,願一切受署。」敞皆以為吏,遣歸休。置酒,小偷悉來賀,且飲醉,偷長以赭汙其衣裾。吏坐里閭閱出者,汙赭輒收縛之,一日捕得數百人。窮治所犯,或一人百餘發,盡行法罰。由是枹鼓稀鳴,市無偷盜,天子嘉之。

敞為人敏疾,賞罰分明,見惡輒取,時時越法縱舍,有足大者。其治京兆,略循趙廣漢之跡。方略耳目,發伏禁姦,不如廣漢,然敞本治春秋,以經術自輔,其政頗雜儒雅,往往表賢顯善,不醇用誅罰,以此能自全,竟免於刑戮。

京兆典京師,長安中浩穰,於三輔尤為劇。郡國二千石以高弟入守,及為真,久者不過二三年,近者數月一歲,輒毀傷失名,以罪過罷。唯廣漢及敞為久任職。敞為京兆,朝廷每有大議,引古今,處便宜,公卿皆服,天子數從之。然敞無威儀,時罷朝會,過走馬章臺街,使御吏驅,自以便面拊馬。又為婦畫眉,長安中傳張京兆眉憮。有司以奏敞。上問之,對曰:「臣聞閨房之內,夫婦之私,有過於畫眉者。」上愛其能,弗備責也。然終不得大位。

敞與蕭望之、于定國相善。始敞與定國俱以諫昌邑王超遷。定國為大夫平尚書事,敞出為刺史,時望之為大行丞。後望之先至御史大夫,定國後至丞相,敞終不過郡守。為京兆九歲,坐與光祿勳楊惲厚善,後惲坐大逆誅,公卿奏惲黨友,不宜處位,等比皆免,而敞奏獨寢不下。敞使卒捕掾絮舜有所案驗。舜以敞劾奏當免,不肯為敞竟事,私歸其家。人或諫舜,舜曰:「吾為是公盡力多矣,今五日京兆耳,安能復案事?」敞聞舜語,即部吏收舜繫獄。是時冬月未盡數日,案事吏晝夜驗治舜,竟致其死事。舜當出死,敞使主簿持教告舜曰:「五日京兆竟何如?冬月已盡,延命乎?」乃棄舜市。會立春,行冤獄使者出,舜家載尸,并編敞教,自言使者。使者奏敞賊殺不辜。天子薄其罪,欲令敞得自便利,即先下敞前坐楊惲不宜處位奏,免為庶人。敞免奏既下,詣闕上印綬,便從闕下亡命。

數月,京師吏民解弛,枹鼓數起,而冀州部中有大賊。天子思敞功效,使使者即家在所召敞。敞身被重劾,及使者至,妻子家室皆泣惶懼,而敞獨笑曰:「吾身亡命為民,郡吏當就捕,今使者來,此天子欲用我也。」即裝隨使者詣公書上車曰:「臣前幸得備位列卿,待罪京兆,坐殺賊捕掾絮舜。舜本臣敞素所厚吏,數蒙恩貸,以臣有章劾當免,受記考事,便歸臥家,謂臣『五日京兆』,背恩忘義,傷化薄俗。臣竊以舜無狀,枉法以誅之。臣敞賊殺無辜,鞠獄故不直,雖伏明法,死無所恨。」天子引見敞,拜為冀州刺史。敞起亡命,復奉使典州。既到部,而廣川王國群輩不道,賊連發,不得。敞以耳目發起賊主名區處,誅其渠帥。廣川王姬昆弟及王同族宗室劉調等通行為之囊橐,吏逐捕窮窘,蹤跡皆入王宮。敞自將郡國吏,車數百兩,圍守王宮,搜索調等,果得之殿屋重轑中。敞傅吏皆捕格斷頭,縣其頭王宮門外。因劾奏廣川王。天子不忍致法,削其戶。敞居部歲餘,冀州盜賊禁止。守太原太守,滿歲為真,太原郡清。

頃之,宣帝崩。元帝初即位,待詔鄭朋薦敞先帝名臣,宜傅輔皇太子。上以問前將軍蕭望之,望之以為敞能吏,任治煩亂,材輕非師傅之器。天子使使者徵敞,欲以為左馮翊。會病卒。敞所誅殺太原吏吏家怨敞,隨至杜陵刺殺敞中子璜。敞三子官皆至都尉。

初,敞為京兆尹,而敞弟武拜為梁相。是時梁王驕貴,民多豪彊,號為難治。敞問武:「欲何以治梁?」武敬憚兄,謙不肯言。敞使吏送至關,戒吏自問武。武應曰:「馭黠馬者利其銜策,梁國大都,吏民凋敝,且當以柱後惠文彈治之耳。」秦時獄法吏冠柱後惠文,武意欲以刑法治梁。吏還道之,敞笑曰:「審如掾言,武必辨治梁矣。」武既到官,其治有跡,亦能吏也。

敞孫竦,王莽時至郡守,封侯,博學文雅過於敞,然政事不及也。竦死,敞無後。

王尊字子贛,涿郡高陽人也。少孤,歸諸父,使牧羊澤中。尊竊學問,能史書。年十三,求為獄小吏。數歲,給事太守府,問詔書行事,尊無不對。太守奇之,除補書佐,署守屬監獄。久之,尊稱病去,事師郡文學官,治尚書、論語,略通大義。復召署守屬治獄,為郡決曹史。數歲,以令舉幽州刺史從事。而太守察尊廉,補遼西鹽官長。數上書言便宜事,事下丞相御史。

初元中,舉直言,遷虢令,轉守槐里,兼行美陽令事。春正月,美陽女子告假子不孝,曰:「兒常以我為妻,妒笞我。」尊聞之,遣吏收捕驗問,辭服。尊曰:「律無妻母之法,聖人所不忍書,此經所謂造獄者也。」尊於是出坐廷上,取不孝子縣磔著樹,使騎吏五人張弓射殺之,吏民驚駭。

後上行幸雍,過虢,尊供張如法而辦。以高弟擢為安定太守。到官,出教告屬縣曰:「令長丞尉奉法守城,為民父母,抑彊扶弱,宣恩廣澤,甚勞苦矣。太守以今日至府,願諸君卿勉力正身以率下。故行貪鄙,能變更者與為治。明慎所職,毋以身試法。」又出教敕掾功曹「各自底厲,助太守為治。其不中用,趣自避退,毋久妨賢。夫羽翮不修,則不可以致千里;闑內不理,無以整外。府丞悉署吏行能,分別白之。賢為上,毋以富。賈人百萬,不足與計事。昔孔子治魯,七日誅少正卯,今太守視事已一月矣,五官掾張輔懷虎狼之心,貪汙不軌,一郡之錢盡入輔家,然適足以葬矣。今將輔送獄,直符史詣閤下,從太守受其事。丞戒之戒之!相隨入獄矣!」輔繫獄數日死,盡得其狡猾不道,百萬姦臧。威震郡中,盜賊分散,入傍郡界。豪彊多誅傷伏辜者。坐殘賊免。

起家,復為護羌將軍轉校尉,護送軍糧委輸。而羌人反,絕轉道,兵數萬圍尊。尊以千餘騎奔突羌賊。功未列上,坐擅離部署,會赦,免歸家。

涿郡太守徐明薦尊不宜久在閭巷,上以尊為郿令,遷益州刺史。先是,琅邪王陽為益州刺史,行部至邛郲九折阪,歎曰:「奉先人遺體,柰何數乘此險!」後以病去。及尊為刺史,至其阪,問吏曰:「此非王陽所畏道邪?」吏對曰:「是。」尊叱其馭曰:「驅之!王陽為孝子,王尊為忠臣。」尊居部二歲,懷來徼外,蠻夷歸附其威信。博士鄭寬中使行風俗,舉奏尊治狀,遷為東平相。

是時,東平王以至親驕奢不奉法度,傅相連坐。及尊視事,奉璽書至庭中,王未及出受詔,尊持璽書歸舍,食已乃還。致詔後,謁見王,太傅在前說相鼠之詩。尊曰:「毋持布鼓過雷門!」王怒,起入後宮。尊亦直趨出就舍。先是王數私出入,驅馳國中,與后姬家交通。尊到官,召敕廄長:「大王當從官屬,鳴和鸞乃出,自今有令駕小車,叩頭爭之,言相教不得。」後尊朝王,王復延請登堂。尊謂王曰:「尊來為相,人皆弔尊也,以尊不容朝廷,故見使相王耳。天下皆言王勇,顧但負貴,安能勇?如尊乃勇耳。」王變色視尊,意欲格殺之,即好謂尊曰:「願觀相君佩刀。」尊舉掖,顧謂傍侍郎:「前引佩刀視王,王欲誣相拔刀向王邪?」王情得,又雅聞尊高名,大為尊屈,酌酒具食,相對極驩。太后徵史奏尊「為相倨慢不臣,王血氣未定,不能忍。愚誠恐母子俱死。今妾不得使王復見尊。陛下不留意,妾願先自殺,不忍見王之失義也。」尊竟坐免為庶人。大將軍王鳳奏請尊補軍中司馬,擢為司隸校尉。

初,中書謁者令石顯貴幸,專權為姦邪。丞相匡衡、御史大夫張譚皆阿附畏事顯,不敢言。久之,元帝崩,成帝初即位,顯徙為中太僕,不復典權。衡、譚乃奏顯舊惡,請免顯等。尊於是劾奏:「丞相衡、御史大夫譚位三公,典五常九德,以總方略,壹統類,廣教化,美風俗為職。知中書謁者令顯等專權擅勢,大作威福,縱恣不制,無所畏忌,為海內患害,不以時皆奏行罰,而阿諛曲從,附下罔上,懷邪迷國,無大臣輔政之義,皆不道,在赦令前。赦後,衡、譚舉奏顯,不自陳不忠之罪,而反揚著先帝任用傾覆之徒,妄言百官畏之,甚於主上。卑君尊臣,非所宜稱,失大臣體。又正月行幸曲臺,臨饗罷衛士,衡與中二千石大鴻臚賞等會坐殿門下,衡南鄉,賞等西鄉。衡更為賞布東鄉席,起立延賞坐,私語如食頃。衡知行臨,百官共職,萬眾會聚,而設不正之席,使下坐上,相比為小惠於公門之下,動不中禮,亂朝廷爵秩之位。衡又使官大奴入殿中,問行起居,還言漏上十四刻行臨到,衡安坐,不變色改容。無怵惕肅敬之心,驕慢不謹。皆不敬。」有詔勿治。於是衡慚懼,免冠謝罪,上丞相、侯印綬。天子以新即位,重傷大臣,乃下御史丞問狀。劾奏尊「妄詆欺非謗赦前事,猥歷奏大臣,無正法,飾成小過,以塗汙宰相,摧辱公卿,輕薄國家,奉使不敬。」有詔左遷尊為高陵令,數月,以病免。

會南山群盜傰宗等數百人為吏民害,拜故弘農太守傅剛為校尉,將跡射士千人逐捕,歲餘不能禽。或說大將軍鳳:「賊數百人在轂下,發軍擊之不能得,難以視四夷。獨選賢京兆尹乃可。」於是鳳薦尊,徵為諫大夫,守京輔都尉,行京兆尹事。旬月間盜賊清。遷光祿大夫,守京兆尹,後為真,凡三歲。坐遇使者無禮。司隸遣假佐放奉詔書白尊發吏捕人,放謂尊:「詔書所捕宜密。」尊曰:「治所公正,京兆善漏泄人事。」放曰:「所捕宜今發吏。」尊又曰:「詔書無京兆文,不當發吏。」及長安繫者三月間千人以上。尊出行縣,男子郭賜自言尊:「

許仲家十餘人共殺賜兄賞,公歸舍。」吏不敢捕。尊行縣還,上奏曰:「彊不陵弱,各得其所,寬大之政行,和平之氣通。」御史大夫中奏尊暴虐不改,外為大言,倨嫚姍嫌,威信日廢,不宜備位九卿。尊坐先,吏民多稱惜之。

湖三老公乘興等上書訟尊治京兆功效日著。「往者南山盜賊阻山橫行,剽劫良民,殺奉法吏,道路不通,城門至以警戒。步兵校尉使逐捕,暴師露眾,曠日煩費,不能禽制。二卿坐黜,群盜寖強,吏氣傷沮,流聞四方,為國家憂。當此之時,有能捕斬,不愛金爵重賞。關內侯寬中使問所徵故司隸校尉王尊捕群盜方略,拜為諫大夫,守京輔都尉,行京兆尹事。尊盡節勞心,夙夜思職,卑體下士,厲奔北之吏,起沮傷之氣,二旬之間,大黨震壞,渠率效首。賊亂蠲除,民反農業,拊循貧弱,鉏耘豪彊。長安宿豪大猾東市賈萬、城西萬章、翦張禁、酒趙放、杜陵楊章等皆通邪結黨,挾養姦軌,上干王法,下亂吏治,并兼役使,侵漁小民,為百姓豺狼。更數二千石,二十年莫能禽討,尊以正法案誅,皆伏其辜。姦邪銷釋,吏民說服。尊撥劇整亂,誅暴禁邪,皆前所稀有,名將所不及。雖拜為真,未有殊絕褒賞加於尊身。今御史大夫奏尊『傷害陰陽,為國家憂,無承用詔書之意,靖言庸違,象龔滔天。』原其所以,出御史丞楊輔,故為尊書佐,素行陰賊,惡口不信,好以刀筆陷人於法。輔常醉過尊大奴利家,利家捽搏其頰,兄子閎拔刀欲剄之。輔以故深怨疾毒,欲傷害尊。疑輔內懷怨恨,外依公事,建畫為此議,傅致奏文,浸潤加誣,以復私怨。昔白起為秦將,東破韓、魏,南拔郢都,應侯譖之,賜死杜郵;吳起為魏守西河,而秦、韓不敢犯,讒人間焉,斥逐奔楚。秦聽浸潤以誅良將,魏信讒言以逐賢守,此皆偏聽不聰,失人之患也。臣等竊痛傷尊修身絜己,砥節首公,刺譏不憚將相,誅惡不避豪彊,誅不制之賊,解國家之憂,功岩職修,威信不廢,誠國家爪牙之吏,折衝之臣,今一旦無辜制於仇人之手,傷於詆欺之文,上不得以功除罪,下不得蒙棘木之聽,獨掩怨讎之偏奏,被共工之大惡,無所陳怨愬罪。尊以京師廢亂,群盜並興,選賢徵用,起家為卿,賊亂既除,豪猾伏辜,即以佞巧廢黜。一尊之身,三期之間,乍賢乍佞,豈不甚哉!孔子曰:『愛之欲其生,惡之欲其死,是惑也。』『浸潤之譖不行焉,可謂明矣。』願下公卿大夫博士議郎,定尊素行。夫人臣而傷害陰陽,死誅之罪也;靖言庸違,放殛之刑也。審如御史章,尊乃當伏觀闕之誅,放於無人之域,不得苟免。及任舉尊者,當獲選舉之辜,不可但已。即不如章,飾文深詆以愬無罪,亦宜有誅,以懲讒賊之口,絕詐欺之俗。唯明主參詳,使白黑分別。」書奏,天子復以尊為徐州刺史,遷東郡太守。

久之,河水盛溢,泛浸瓠子金隄,老弱奔走,恐水大決為害。尊躬率吏民,投沈白馬,祀水神河伯。尊親執圭璧,使巫策祝,請以身填金隄,因止宿,廬居隄上。吏民數千萬人爭叩頭救止尊,尊終不肯去。及水盛隄壞,吏民皆奔走,唯一主簿泣在尊旁,立不動。而水波稍卻迴還。吏民嘉壯尊之勇節,白馬三老朱英等奏其狀。下有司考,皆如言。於是制詔御史:「東郡河水盛長,毀壞金隄,未決三尺,百姓惶恐奔走。太守身當水衝,履咫尺之難,不避危殆,以安眾心,吏民復還就作,水不為災,朕甚嘉之。秩尊中二千石,加賜黃金二十斤。」

數歲,卒官,吏民紀之。尊子伯亦為京兆尹,坐耎弱不勝任免。

王章字仲卿,泰山鉅平人也。少以文學為官,稍遷至諫大夫,在朝廷名敢直言。元帝初,擢為左曹中郎將,與御史中丞陳咸相善,共毀中書令石顯,為顯所陷,咸減死髡,章免官。成帝立,徵章為諫大夫,遷司隸校尉,大臣貴戚敬憚之。王尊免後,代者不稱職,章以選為京兆尹。時帝舅大將軍王鳳輔政,章雖為鳳所舉,非鳳專權,不親附鳳。會日有蝕之,章奏封事,召見,言鳳不可任用,宜更選忠賢。上初納受章言,後不忍退鳳。章由是見疑,遂為鳳所陷,罪至大逆。語在元后傳。

初,章為諸生學長安,獨與妻居。章疾病,無被,臥牛衣中,與妻決,涕泣。其妻呵怒之曰:「仲卿!京師尊貴在朝廷人誰踰仲卿者?今疾病困厄,不自激卬,乃反涕泣,何鄙也!」

後章仕宦歷位,及為京兆,欲上封事,妻又止之曰:「人當知足,獨不念牛衣中涕泣時耶?」章曰:「非女子所知也。」書遂上,果下廷尉獄,妻子皆收繫。章小女年可十二,夜起號哭曰:「平生獄上呼囚,素常至九,今八而止。我君數剛,先死者必君。」明日問之,章果死。妻子皆徙合浦。

大將軍鳳薨後,弟成都侯商復為大將軍輔政,白上還章妻子故郡。其家屬皆完具,采珠致產數百萬,時蕭育為泰山太守,皆令贖還故田宅。

章為京兆二歲,死不以其罪,眾庶冤紀之,號為三王。王駿自有傳,駿即王陽子也。

贊曰:自孝武置左馮翊、右扶風、京兆尹,而吏民為之語曰:「

前有趙、張,後有三王。」然劉向獨序趙廣漢、尹翁歸、韓延壽,馮商傳王尊,揚雄亦如之。廣漢聰明,下不能欺,延壽厲善,所居移風,然皆訐上不信,以失身墮功。翁歸抱公絜己,為近世表。張敞衎衎,履忠進言,緣飾儒雅,刑罰必行,縱赦有度,條教可觀,然被輕惰之名。王尊文武自將,所在必發,譎詭不經,好為大言。王章剛直守節,不量輕重,以陷刑戮,妻子流遷,哀哉!


\end{pinyinscope}