\article{文帝紀}

\begin{pinyinscope}
孝文皇帝,高祖中子也,母曰薄姬。高祖十一年,誅陳豨,定代地,立為代王,都中都。十七年秋,高后崩,諸呂謀為亂,欲危劉氏。丞相陳平、太尉周勃、朱虛侯劉章等共誅之,謀立代王。語在高后紀、高五王傳。

大臣遂使人迎代王。郎中令張武等議,皆曰:「漢大臣皆故高帝時將,習兵事,多謀詐,其屬意非止此也,特畏高帝、呂太后威耳。今已誅諸呂,新喋血京師,以迎大王為名,實不可信。願稱疾無往,以觀其變。」中尉宋昌進曰:「群臣之議皆非也。夫秦失其政,豪傑並起,人人自以為得之者以萬數,然卒踐天子位者,劉氏也,天下絕望,一矣。高帝王子弟,地犬牙相制,所謂盤石之宗也,天下服其彊,二矣。漢興,除秦煩苛,約法令,施德惠,人人自安,難動搖,三矣。夫以呂太后之嚴,立諸呂為三王,擅權專制,然而太尉以一節入北軍,一呼士皆袒左,為劉氏,畔諸呂,卒以滅之。此乃天授,非人力也。今大臣雖欲為變,百姓弗為使,其黨寧能專一邪?內有朱虛、東牟之親,外畏吳、楚、淮南、琅邪、齊、代之彊。方今高帝子獨淮南王與大王,大王又長,賢聖仁孝,聞於天下,故大臣因天下之心而欲迎立大王,大王勿疑也。」代王報太后,計猶豫未定。卜之,兆得大橫。占曰:「大橫庚庚,余為天王,夏啟以光。」代王曰:「寡人固已為王,又何王乎?」卜人曰:「所謂天王者,乃天子也。」於是代王乃遣太后弟薄昭見太尉勃,勃等具言所以迎立王者。昭還報曰:「信矣,無可疑者。」代王笑謂宋昌曰:「果如公言。」乃令宋昌驂乘,張武等六人乘六乘傳詣長安。至高陵止,而使宋昌先之長安觀變。

昌至渭橋,丞相已下皆迎。昌還報,代王乃進至渭橋。群臣拜謁稱臣,代王下拜。太尉勃進曰:「願請間。」宋昌曰:「所言公,公言之;所言私,王者無私。」太尉勃乃跪上天子璽。代王謝曰:「至邸而議之。」

閏月己酉,入代邸。群臣從至,上議曰:「丞相臣平、太尉臣勃、大將軍臣武、御史大夫臣蒼、宗正臣郢、朱虛侯臣章、東牟侯臣興居、典客臣揭再拜言大王足下:子弘等皆非孝惠皇帝子,不當奉宗廟。臣謹請陰安侯、頃王后、琅邪王、列侯、吏二千石議,大王高皇帝子,宜為嗣。願大王即天子位。」代王曰:「奉高帝宗廟,重事也。寡人不佞,不足以稱。願請楚王計宜者,寡人弗敢當。」群臣皆伏,固請。代王西鄉讓者三,南鄉讓者再。丞相平等皆曰:「臣伏計之,大王奉高祖宗廟最宜稱,雖天下諸侯萬民皆以為宜。臣等為宗廟社稷計,不敢忽。願大王幸聽臣等。臣謹奉天子璽符再拜上。」代王曰:「宗室將相王列侯以為其宜寡人,寡人不敢辭。」遂即天子位。群臣以次侍。使太僕嬰、東牟侯興居先清宮,奉天子法駕迎代邸。皇帝即日夕入未央宮。夜拜宋昌為衛將軍,領南北軍,張武為郎中令,行殿中。還坐前殿,下詔曰:「制詔丞相、太尉、御史大夫:間者諸呂用事擅權,謀為大逆,欲危劉氏宗廟,賴將相列侯宗室大臣誅之,皆伏其辜。朕初即位,其赦天下,賜民爵一級,女子百戶牛酒,酺五日。」

元年冬十月辛亥,皇帝見于高廟。遣車騎將軍薄昭迎皇太后于代。詔曰:「前呂產自置為相國,呂祿為上將軍,擅遣將軍灌嬰將兵擊齊,欲代劉氏。嬰留滎陽,與諸侯合謀以誅呂氏。呂產欲為不善,丞相平與太尉勃等謀奪產等軍。朱虛侯章首先捕斬產。太尉勃身率襄平侯通持節承詔入北軍。典客揭奪呂祿印。其益封太尉勃邑萬戶,賜金五千斤。丞相平、將軍嬰邑各三千戶,金二千斤。朱虛侯章、襄平侯通邑各二千戶,金千斤。封典客揭為陽信侯,賜金千斤。」

十二月,立趙幽王子遂為趙王,徙琅邪王澤為燕王。呂氏所奪齊楚地皆歸之。盡除收帑相坐律令。

正月,有司請蚤建太子,所以尊宗廟也。詔曰:「朕既不德,上帝神明未歆饗也,天下人民未有轺志。今縱不能博求天下賢聖有德之人而嬗天下焉,而曰豫建太子,是重吾不德也。謂天下何?其安之。」有司曰:「豫建太子,所以重宗廟社稷,不忘天下也。」上曰:「楚王,季父也,春秋高,閱天下之義理多矣,明於國家之體。吳王於朕,兄也;淮南王,弟也:皆秉德以陪朕,豈為不豫哉!諸侯王宗室昆弟有功臣,多賢及有德義者,若舉有德以陪朕之不能終,是社稷之靈,天下之福也。今不選舉焉,而曰必子,人其以朕為忘賢有德者而專於子,非所以憂天下也。朕甚不取。」有司固請曰:「古者殷周有國,治安皆且千歲,有天下者莫長焉,用此道也。立嗣必子,所從來遠矣。高帝始平天下,建諸侯,為帝者太祖。諸侯王列侯始受國者亦皆為其國祖。子孫繼嗣,世世不絕,天下之大義也。故高帝設之以撫海內。今釋宜建而更選於諸侯宗室,非高帝之志也。更議不宜。子啟最長,敦厚慈仁,請建以為太子。」上乃許之。因賜天下民當為父後者爵一級。封將軍薄昭為軹侯。

三月,有司請立皇后。皇太后曰:「立太子母竇氏為皇后。」

詔曰:「方春和時,草木群生之物皆有以自樂,而吾百姓鰥寡孤獨窮困之人或阽於死亡,而莫之省憂。為民父母將何如?其議所以振貸之。」又曰:「老者非帛不煖,非肉不飽。今歲首,不時使人存問長老,又無布帛酒肉之賜,將何以佐天下子孫孝養其親?今聞吏稟當受鬻者,或以陳粟,豈稱養老之意哉!具為令。」有司請令縣道,年八十已上,賜米人月一石,肉二十斤,酒五斗。其九十已上,又賜帛人二疋,絮三斤。賜物及當稟鬻米者,長吏閱視,丞若尉致。不滿九十,嗇夫、令史致。二千石遣都吏循行,不稱者督之。刑者及有罪耐以上,不用此令。

楚元王交薨。

四月,齊楚地震,二十九山同日崩,大水潰出。

六月,令郡國無來獻。施惠天下,諸侯四夷遠近驩洽。乃脩代來功。詔曰:「方大臣誅諸呂迎朕,朕狐疑,皆止朕,唯中尉宋昌勸朕,朕已得保宗廟。已尊昌為衛將軍,其封昌為壯武侯。諸從朕六人,官皆至九卿。」又曰:「列侯從高帝入蜀漢者六十八人益邑各三百戶。吏二千石以上從高帝潁川守尊等十人食邑六百戶,淮陽守申屠嘉等十人五百戶,衛尉足等十人四百戶。」封淮南王舅趙兼為周陽侯,齊王舅駟鈞為靖郭侯,故常山丞相蔡兼為樊侯。

二年冬十月,丞相陳平薨。詔曰:「朕聞古者諸侯建國千餘,各守其地,以時入貢,民不勞苦,上下驩欣,靡有違德。今列侯多居長安,邑遠,吏卒給輸費苦,而列侯亦無繇教訓其民。其令列侯之國,為吏及詔所止者,遣太子。」

十一月癸卯晦,日有食之。詔曰:「朕聞之,天生民,為之置君以養治之。人主不德,布政不均,則天示之災以戒不治。乃十一月晦,日有食之,適見于天,災孰大焉!朕獲保宗廟,以微眇之身託于士民君王之上,天下治亂,在予一人,唯二三執政猶吾股肱也。朕下不能治育群生,上以累三光之明,其不德大矣。令至,其悉思朕之過失,及知見之所不及,饨以啟告朕。及舉賢良方正能直言極諫者,以匡朕之不逮。因各敕以職任,務省繇費以便民。朕既不能遠德,故贳然念外人之有非,是以設備未息。今縱不能罷邊屯戍,又飭兵厚衛,其罷衛將軍軍。太僕見馬遺財足,餘皆以給傳置。」

春正月丁亥,詔曰:「夫農,天下之本也,其開藉田,朕親率耕,以給宗廟粢盛。民謫作縣官及貸種食未入、入未備者,皆赦之。」

三月,有司請立皇子為諸侯王。詔曰:「前趙幽王幽死,朕甚憐之,已立其太子遂為趙王。遂弟辟彊及齊悼惠王子朱虛侯章、東牟侯興居有功,可王。」乃遂立辟彊為河間王,章為城陽王,興居為濟北王。因立皇子武為代王,參為太原王,揖為梁王。

五月,詔曰:「古之治天下,朝有進善之旌,誹謗之木,所以通治道而來諫者也。今法有誹謗訞言之罪,是使眾臣不敢盡情,而上無由聞過失也。將何以來遠方之賢良?其除之。民或祝詛上,以相約而後相謾,吏以為大逆,其有他言,吏又以為誹謗。此細民之愚,無知抵死,朕甚不取。自今以來,有犯此者勿聽治。」

九月,初與郡守為銅虎符、竹使符。

詔曰:「農,天下之大本也,民所恃以生也,而民或不務本而事末,故生不遂。朕憂其然,故今茲親率群臣農以勸之。其賜天下民今年田租之半。」

三年冬十月丁酉晦,日有食之。十一月丁卯晦,日有蝕之。

詔曰:「前日詔遣列侯之國,辭未行。丞相朕之所重,其為遂率列侯之國。」遂免丞相勃,遣就國。十二月,太尉潁陰侯灌嬰為丞相。罷太尉官,屬丞相。

夏四月,城陽王章薨。淮南王長殺辟陽侯審食其。

五月,匈奴入居北地、河南為寇。上幸甘泉,遣丞相灌嬰擊匈奴,匈奴去。發中尉材官屬衛將軍,軍長安。

上自甘泉之高奴,因幸太原,見故群臣,皆賜之。舉功行賞,諸民里賜牛酒。復晉陽、中都民三歲租。留游太原十餘日。

濟北王興居聞帝之代,欲自擊匈奴,乃反,發兵欲襲滎陽。於是詔罷丞相兵,以棘蒲侯柴武為大將軍,將四將軍十萬眾擊之。祁侯繒賀為將軍,軍滎陽。秋七月,上自太原至長安。詔曰:「濟北王背德反上,詿誤吏民,為大逆。濟北吏民兵未至先自定及以軍城邑降者,皆赦之,復官爵。與王興居去來者,亦赦之。」八月,虜濟北王興居,自殺。赦諸與興居反者。

四年冬十二月,丞相灌嬰薨。

夏五月,復諸劉有屬籍,家無所與。賜諸侯王子邑各二千戶。

秋九月,封齊悼惠王子七人為列侯。

絳侯周勃有罪,逮詣廷尉詔獄。

作顧成廟。

五年春二月,地震。

夏四月,除盜鑄錢令。更造四銖錢。

六年冬十月,桃李華。

十一月,淮南王長謀反,廢遷蜀嚴道,死雍。

七年冬十月,令列侯太夫人、夫人、諸侯王子及吏二千石無得擅徵捕。

六月癸酉,未央宮東闕罘罳災。

八年夏,封淮南厲王長子四人為列侯。

有長星出于東方。

九年春,大旱。

十年冬,行幸甘泉。

將軍薄昭死。

十一年冬十一月,行幸代。春正月,上自代還。

夏六月,梁王揖薨。

匈奴寇狄道。

十二年冬十二月,河決東郡。

春正月,賜諸侯王女邑各二千戶。

二月,出孝惠皇帝後宮美人,令得嫁。

三月,除關無用傳。

詔曰:「道民之路,在於務本。朕親率天下農,十年于今,而野不加辟,歲一不登,民有飢色,是從事焉尚寡,而吏未加務也。吾詔書數下,歲勸民種樹,而功未興,是吏奉吾詔不勤,而勸民不明也。且吾農民甚苦,而吏莫之省,將何以勸焉?其賜農民今年租稅之半。」

又曰:「孝悌,天下之大順也。力田,為生之本也。三老,眾民之師也。廉吏,民之表也。朕甚嘉此二三大夫之行。今萬家之縣,云無應令,豈實人情?是吏舉賢之道未備也。其遣謁者勞賜三老、孝者帛人五匹,悌者、力田二匹,廉吏二百石以上率百石者三匹。及問民所不便安,而以戶口率置三老孝悌力田常員,令各率其意以道民焉。」

十三年春二月甲寅,詔曰:「朕親率天下農耕以供粢盛,皇后親桑以奉祭服,其具禮儀。」

夏,除祕祝,語在郊祀志。五月,除肉刑法,語在刑法志。

六月,詔曰:「農,天下之本,務莫大焉。今廑身從事,而有租稅之賦,是謂本末者無以異也,其於勸農之道未備。其除田之租稅。賜天下孤寡布帛絮各有數。」

十四年冬,匈奴寇邊,殺北地都尉卬。遣三將軍軍隴西、北地、上郡,中尉周舍為衛將軍,郎中令張武為車騎將軍,軍渭北,車千乘,騎卒十萬人。上親勞軍,勒兵,申教令,賜吏卒。自欲征匈奴,群臣諫,不聽。皇太后固要上,乃止。於是以東陽侯張相如為大將軍,建成侯董赫、內史欒布皆為將軍,擊匈奴。匈奴走。

春,詔曰:「朕獲執犧牲珪幣以事上帝宗廟,十四年于今。歷日彌長,以不敏不明而久撫臨天下,朕甚自媿。其廣增諸祀壇場珪幣。昔先王遠施不求其報,望祀不祈其福,右賢左戚,先民後己,至明之極也。今吾聞祠官祝釐,皆歸福於朕躬,不為百姓,朕甚媿之。夫以朕之不德,而專鄉獨美其福,百姓不與焉,是重吾不德也。其令祠官致敬,無有所祈。」

十五年春,黃龍見於成紀。上乃下詔議郊祀。公孫臣明服色,新垣平設五廟。語在郊祀志。夏四月,上幸雍,始郊見五帝,赦天下,修名山大川嘗祀而絕者,有司以歲時致禮。

九月,詔諸侯王公卿郡守舉賢良能直言極諫者,上親策之,傅納以言。語在晁錯傳。

十六年夏四月,上郊祀五帝于渭陽。

五月,立齊悼惠王子六人、淮南厲王子三人皆為王。

秋九月,得玉杯,刻曰「人主延壽」。令天下大酺,明年改元。

後元年冬十月,新垣平詐覺,謀反,夷三族。

春三月,孝惠皇后張氏薨。

詔曰:「間者數年比不登,又有水旱疾疫之災,朕甚憂之。愚而不明,未達其咎。意者朕之政有所失而行有過與?乃天道有不順,地利或不得,人事多失和,鬼神廢不享與?何以致此?將百官之奉養或費,無用之事或多與?何其民食之寡乏也!夫度田非益寡,而計民未加益,以口量地,其於古猶有餘,而食之甚不足者,其咎安在?無乃百姓之從事於末以害農者蕃,為酒醪以靡穀者多,六畜之食焉者眾與?細大之義,吾未能得其中。其與丞相列侯吏二千石博士議之,有可以佐百姓者,率意遠思,無有所隱。」

二年夏,行幸雍棫陽宮。

六月,代王參薨。匈奴和親。詔曰:「朕既不明,不能遠德,使方外之國或不寧息。夫四荒之外不安其生,封圻之內勤勞不處,二者之咎,皆自於朕之德薄而不能達遠也。間者累年,匈奴並暴邊境,多殺吏民,邊臣兵吏入不能諭其內志,以重吾不德。夫久結難連兵,中外之國將何以自寧?今朕夙興夜寐,勤勞天下,憂苦萬民,為之惻怛不安,未嘗一日忘於心,故遣使者冠蓋相望,結徹於道,以諭朕志於單于。今單于反古之道,計社稷之安,便萬民之利,新與朕俱棄細過,偕之大道,結兄弟之義,以全天下元元之民。和親以定,始于今年。」

三年春二月,行幸代。

四年夏四月丙寅晦,日有蝕之。五月,赦天下。免官奴婢為庶人。行幸雍。

五年春正月,行幸隴西。三月,行幸雍。秋七月,行幸代。

六年冬,匈奴三萬騎入上郡,三萬騎入雲中。以中大夫令免為車騎將軍屯飛狐,故楚相蘇意為將軍屯句注,將軍張武屯北地,河內太守周亞夫為將軍次細柳,宗正劉禮為將軍次霸上,祝茲侯徐厲為將軍次棘門,以備胡。

夏四月,大旱,蝗。令諸侯無入貢。弛山澤。減諸服御。損郎吏員。發倉庾以振民。民得賣爵。

七年夏六月己亥,帝崩于未央宮。遺詔曰:「朕聞之,蓋天下萬物之萌生,靡不有死。死者天地之理,物之自然,奚可甚哀!當今之世,咸嘉生而惡死,厚葬以破業,重服以傷生,吾甚不取。且朕既不德,無以佐百姓;今崩,又使重服久臨,以罹寒暑之數,哀人父子,傷長老之志,損其飲食,絕鬼神之祭祀,以重吾不德,謂天下何!朕獲保宗廟,以眇眇之身託于天下君王之上,二十有餘年矣。賴天之靈,社稷之福,方內安寧,靡有兵革。朕既不敏,常畏過行,以羞先帝之遺德;惟年之久長,懼于不終。今乃幸以天年得復供養于高廟,朕之不明與嘉之,其奚哀念之有!其令天下吏民,令到出臨三日,皆釋服。無禁取婦嫁女祠祀飲酒食肉。自當給喪事服臨者,皆無踐。姪帶無過三寸。無布車及兵器。無發民哭臨宮殿中。殿中當臨者,皆以旦夕各十五舉音,禮畢罷。非旦夕臨時,禁無得擅哭臨。以下,服大紅十五日,小紅十四日,纖七日,釋服。它不在令中者,皆以此令比類從事。布告天下,使明知朕意。霸陵山川因其故,無有所改。歸夫人以下至少使。」令中尉亞夫為車騎將軍,屬國悍為將屯將軍,郎中令張武為復土將軍,發近縣卒萬六千人,發內史卒萬五千人,臧郭穿復土屬將軍武。賜諸侯王以下至孝悌力田金錢帛各有數。乙巳,葬霸陵。

贊曰:孝文皇帝即位二十三年,宮室苑囿車騎服御無所增益。有不便,輒弛以利民。嘗欲作露臺,召匠計之,直百金。上曰:「百金,中人十家之產也。吾奉先帝宮室,常恐羞之,何以臺為!」身衣弋綈,所幸慎夫人衣不曳地,帷帳無文繡,以示敦朴,為天下先。治霸陵,皆瓦器,不得以金銀銅錫為飾,因其山,不起墳。南越尉佗自立為帝,召貴佗兄弟,以德懷之,佗遂稱臣。與匈奴結和親,後而背約入盜,令邊備守,不發兵深入,恐煩百姓。吳王詐病不朝,賜以几杖。群臣袁盎等諫說雖切,常假借納用焉。張武等受賂金錢,覺,更加賞賜,以媿其心。專務以德化民,是以海內殷富,興於禮義,斷獄數百,幾致刑措。嗚呼,仁哉!


\end{pinyinscope}