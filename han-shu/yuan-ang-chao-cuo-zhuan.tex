\article{爰盎晁錯傳}

\begin{pinyinscope}
爰盎字絲。其父楚人也,故為群盜,徙安陵。高后時,盎為呂祿舍人。孝文即位,盎兄噲任盎為郎中。

絳侯為丞相,朝罷趨出,意得甚。上禮之恭,常目送之。盎進曰:「丞相何如人也?」上曰:「社稷臣。」盎曰:「絳侯所謂功臣,非社稷臣。社稷臣主在與在,主亡與亡。方呂后時,諸呂用事,擅相王,劉氏不絕如帶。是時絳侯為太尉,本兵柄,弗能正。呂后崩,大臣相與共誅諸呂,太尉主兵,適會其成功,所謂功臣,非社稷臣。丞相如有驕主色,陛下謙讓,臣主失禮,竊為陛下弗取也。」後朝,上益莊,丞相益畏。已而絳侯望盎曰:「吾與汝兄善,今兒乃毀我!」盎遂不謝。

及絳侯就國,人上書告以為反,徵擊請室,諸公莫敢為言,唯盎明絳侯無罪。絳侯得釋,盎頗有力。絳侯乃大與盎結交。

淮南厲王朝,殺辟陽侯,居處驕甚。盎諫曰:「諸侯太驕必生患,可適削地。」上弗許。淮南王益橫。謀反發覺,上徵淮南王,遷之蜀,檻車傳送。盎時為中郎將,諫曰:「陛下素驕之,弗稍禁,以至此,今又暴摧折之。淮南王為人剛,有如遇霜露行道死,陛下竟為以天下大弗能容,有殺弟名,柰何?」上不聽,遂行之。

淮南王至雍,病死,聞,上輟食,哭甚哀。盎入,頓首請罪。上曰:「以不用公言至此。」盎曰:「上自寬,此往事,豈可悔哉!且陛下有高世行三,此不足以毀名。」上曰:「吾高世三者何事?」盎曰:「陛下居代時,太后嘗病,三年,陛下不交睫解衣,湯藥非陛下口所嘗弗進。夫曾參以布衣猶難之,今陛下親以王者修之,過曾參遠矣。諸呂用事,大臣顓制,然陛下從代乘六乘傳,馳不測淵,雖賁育之勇不及陛下。陛下至代邸,西鄉讓天子者三,南鄉讓天子者再。夫許由一讓,陛下五以天下讓,過許由四矣。且陛下遷淮南王,欲以苦其志,使改過,有司宿衛不謹,故病死。」於是上乃解,盎繇此名重朝廷。

盎常引大體慷慨。宦者趙談以數幸,常害盎,盎患之。盎兄子種為常侍騎,諫盎曰:「君眾辱之,後雖惡君,上不復信。」於是上朝東宮,趙談驂乘,盎伏車前曰:「臣聞天子所與共六尺輿者,皆天下豪英。今漢雖乏人,陛下獨柰何與刀鋸之餘共載!」於是上笑,下趙談。談泣下車。

上從霸陵上,欲西馳下峻阪,盎髓轡。上曰:「將軍怯邪?」盎言曰:「臣聞千金之子不垂堂,百金之子不騎衡,聖主不乘危,不徼幸。今陛下騁六飛,馳不測山,有如馬驚車敗,陛下縱自輕,柰高廟、太后何?」上乃止。

上幸上林,皇后、慎夫人從。其在禁中,常同坐。及坐,郎署長布席,盎引卻慎夫人坐。慎夫人怒,不肯坐。上亦怒,起。盎因前說曰:「臣聞尊卑有序則上下和,今陛下既以立后,慎夫人乃妾,妾主豈可以同坐哉!且陛下幸之,則厚賜之。陛下所以為慎夫人,適所以禍之也。獨不見『人豕』乎?」於是上乃說,入語慎夫人。慎夫人賜盎金五十斤。

然盎亦以數直諫,不得久居中。調為隴西都尉,仁愛士卒,士卒皆爭為死。遷齊相,徙為吳相。辭行,種謂盎曰:「吳王驕日久,國多姦,今絲欲刻治,彼不上書告君,則利劍刺君矣。南方卑溼,絲能日飲,亡何,說王毋反而已。如此幸得脫。」盎用種之計,吳王厚遇盎。

盎告歸,道逢丞相申屠嘉,下車拜謁,丞相從車上謝。盎還,媿其吏,乃之丞相舍上謁,求見丞相。丞相良久乃見。因跪曰:「願請間。」丞相曰:「使君所言公事,之曹與長史掾議之,吾且奏之;則私,吾不受私語。」盎即起說曰:「君為相,自度孰與陳平、絳侯?」丞相曰:「不如。」盎曰:「善,君自謂弗如。夫陳平、絳侯輔翼高帝,定天下,為將相,而誅諸呂,存劉氏;君乃為材官蹶張,遷為隊帥,積功至淮陽守,非有奇計攻城野戰之功。且陛下從代來,每朝,郎官者上書疏,未嘗不止輦受。其言不可用,置之;言可采,未嘗不稱善。何也?欲以致天下賢英士大夫,日聞所不聞,以益聖。而君自閉箝天下之口,而日益愚。夫以聖主責愚相,君受禍不久矣。」丞相乃再拜曰:「嘉鄙人,乃不知,將軍幸教。」引與入坐,為上客。

盎素不好晁錯,錯所居坐,盎輒避;盎所居坐,錯亦避:兩人未嘗同堂語。及孝景即位,晁錯為御史大夫,使吏案盎受吳王財物,抵罪,詔赦以為庶人。吳楚反聞,錯謂丞史曰:「爰盎多受吳王金錢,專為蔽匿,言不反。今果反,欲請治盎,宜知其計謀。」丞史曰:「事未發,治之有絕。今兵西向,治之何益!且盎不宜有謀。」錯猶與未決。人有告盎,盎恐,夜見竇嬰,為言吳所以反,願致前,口對狀。嬰入言,上乃召盎。盎入見,竟言吳所以反,獨急斬錯以謝吳,吳可罷。上拜盎為泰常,竇嬰為大將軍。兩人素相善。是時,諸陵長安中賢大夫爭附兩人,車騎者日數百乘。

及晁錯已誅,盎以泰常使吳。吳王欲使將,不肯。欲殺之,使一都尉以五百人圍守盎軍中。初,盎為吳相時,從史盜私盎侍兒。盎知之,弗泄,遇之如故。人有告從史,「君知女與侍者通」,乃亡去。盎驅自追之,遂以侍者賜之,復為從史。及盎使吳見守,從史適在守盎校為司馬,乃悉以其裝齎買二石醇醪,會天寒,士卒飢渴,飲醉西南陬卒,卒皆臥。司馬夜引盎起,曰:「君可以去矣,吳王期旦日斬君。」盎弗信,曰:「何為者?」司馬曰:「臣故為君從史盜侍兒者也。」盎乃驚,謝曰:「公幸有親,吾不足絫公。」司馬曰:「君弟去,臣亦且亡,辟吾親,君何患!」乃以刀決帳,道從醉卒直出。司馬與分背。盎解節旄懷之,屐步行七十里,明,見梁騎,馳去,遂歸報。

吳楚已破,上更以元王子平陸侯禮為楚王,以盎為楚相。嘗上書,不用。盎病免家居,與閭里浮湛,相隨行鬥雞走狗。雒陽劇孟嘗過盎,盎善待之。安陵富人有謂盎曰:「吾聞劇孟博徒,將軍何自通之?」盎曰:「劇孟雖博徒,然母死,客送喪車千餘乘,此亦有過人者。且緩急人所有。夫一旦叩門,不以親為解,不以在亡為辭,天下所望者,獨季心、劇孟。今公陽從數騎,一旦有緩急,寧足恃乎!」遂罵富人,弗與通。諸公聞之,皆多盎。

盎雖居家,景帝時時使人問籌策。梁王欲求為嗣,盎進說,其後語塞。梁王以此怨盎,使人刺盎。刺者至關中,問盎,稱之皆不容口。乃見盎曰:「臣受梁王金刺君,君長者,不忍刺君。然後刺者十餘曹,備之!」盎心不樂,家多怪,乃之棓生所問占。還,梁刺客後曹果遮刺殺盎安陵郭門外。

晁錯,潁川人也。學申商刑名於軹張恢生所,與雒陽宋孟及劉帶同師。以文學為太常掌故。

錯為人峭直刻深。孝文時,天下亡治尚書者,獨聞齊有伏生,故秦博士,治尚書,年九十餘,老不可徵。乃詔太常,使人受之。太常遣錯受尚書伏生所,還,因上書稱說。詔以為太子舍人,門大夫,遷博士。又上書言:「人主所以尊顯功名揚於萬世之後者,以知術數也。故人主知所以臨制臣下而治其眾,則群臣畏服矣;知所以聽言受事,則不欺蔽矣;知所以安利萬民,則海內必從矣;知所以忠孝事上,則臣子之行備矣:此四者,臣竊為皇太子急之。人臣之議或曰皇太子亡以知事為也,臣之愚,誠以為不然。竊觀上世之君,不能奉其宗廟而劫殺於其臣者,皆不知術數者也。皇太子所讀書多矣,而未深知術數者也。皇太子所讀書多矣,而未深知術數者,不問書說也。夫多誦而不知其說,所謂勞苦而不為功。臣竊觀皇太子材智高奇,馭射伎藝過人絕遠,然於術數未有所守者,以陛下為心也。竊願陛下幸擇聖人之術可用今世者,以賜皇太子,因時使太子陳明於前。唯陛下裁察。」上善之,於是拜錯為太子家令。以其辯得幸太子,太子家號曰「智囊」。

是時匈奴彊,數寇邊,上發兵以禦之。錯上言兵事,曰:

臣聞漢興以來,胡虜數入邊地,小入則小利,大入則大利;高后時再入隴西,攻城屠邑,敺略畜產;其後復入隴西,殺吏卒,大寇盜。竊聞戰勝之威,民氣百倍;敗兵之卒,沒世不復。自高后以來,隴西三困於匈奴矣,民氣破傷,亡有勝意。今茲隴西之吏,賴社稷之神靈,奉陛下之明詔,和輯士卒,底厲其節,起破傷之民以當乘勝之匈奴,用少擊眾,殺一王,敗其眾而

法曰大有利。非隴西之民有勇怯,乃將吏之制巧拙異也。故兵法曰:「有必勝之將,無必勝之民。」繇此觀之,安邊境,立功名,在於良將,不可不擇也。

臣又聞用兵,臨戰合刃之急者三:一曰得地形,二曰卒服習,三曰器用利。兵法曰:丈五之溝,漸車之水,山林積石,經川丘阜,屮木所在,此步兵之地也,車騎二不當一。土山丘陵,曼衍相屬,平原廣野,此車騎之地,步兵十不當一。平陵相遠,川谷居間,仰高臨下,此弓弩之地也,短兵百不當一。兩陳相近,平地淺草,可前可後,此長戟之地也,劍楯三不當一。雚葦竹蕭,屮木蒙蘢,支葉茂接,此矛鋋之地也,長戟二不當一。曲道相伏,險阨相薄,此劍楯之地也,弓弩三不當一。士不選練,卒不服習,起居不精,動靜不集,趨利弗及,避難不畢,前擊後解,與金鼓之

音相失,此不習勒卒之過也,百不當十。兵不完利,與空手同;甲不堅密,與袒裼同;弩不可以及遠,與短兵同;射不能中,與亡矢同;中不能入,與亡鏃同:此將不省兵之禍也,五不當一。故兵法曰:器械不利,以其卒予敵也;卒不可用,以其將予敵也;將不知兵,以其主予敵也;君不擇將,以其國予敵也。四者,國之至要也。

臣又聞小大異形,彊弱異勢,險易異備。夫卑身以事彊,小國之形也;合小以攻大,敵國之形也;以蠻夷攻蠻夷,中國之形也。今匈奴地形技藝與中國異。上下山阪,出入溪澗,中國之馬弗與也;險道傾仄,且馳且射,中國之騎弗與也;風雨罷勞,飢渴不困,中國之人弗與也:此匈奴之長技也。若夫平原易地,輕車突騎,則匈奴之眾易撓亂也;勁弩長戟,射疏及遠,則匈奴之弓弗能格也;堅甲利刃,長短相雜,遊弩往來,什伍俱前,則匈奴之兵弗能當也;材官騶發,矢道同的,則匈奴之革笥木薦弗能支也;下馬地鬥,劍戟相接,去就相薄,則匈奴之足弗能給也:此中國之長技也。以此觀之,匈奴之長技三,中國之長技五。陛下又興數十萬之眾,以誅數萬之匈奴,眾寡之計,以一擊十之術也。

雖然,兵,凶器;戰,危事也。以大為小,以彊為弱,在俛卬之間耳。夫以人之死爭勝,跌而不振,則悔之亡及也。帝王之道,出於萬全。今降胡義渠蠻夷之屬來歸誼者,其眾數千,飲食長技與匈奴同,可賜之堅甲絮衣,勁弓利矢,益以邊郡之良騎。令明將能知其習俗和輯其心者,以陛下之明約將之。即有險阻,以此當之;平地通道,則以輕車材官制之。兩軍相為表裏,各用其長技,衡加之以眾,此萬全之術也。

傳曰:「狂夫之言,而明主擇焉。」臣錯愚陋,昧死上狂言,唯陛下財擇。

文帝嘉之,乃賜錯璽書寵答焉,曰:「皇帝問太子家令:上書言兵體三章,聞之。書言『狂夫之言,而明主擇焉』。今則不然。言者不狂,而擇者不明,國之大患,故在於此。使夫不明擇於不狂,是以萬聽而萬不當也。」

錯復言守邊備塞,勸農力本,當世急務二事,曰:

臣聞秦時北攻胡貉,築塞河上,南攻楊粵,置戍卒焉。其起兵而攻胡、粵者,非以衛邊地而救民死也,貪戾而欲廣大也,故功未立而天下亂。且夫起兵而不知其勢,戰則為人禽,屯則卒積死。夫胡貉之地,積陰之處也,木皮三寸,冰厚六尺,食肉而飲酪,其人密理,鳥獸毳毛,其性能寒。楊粵之地少陰多陽,其人疏理,鳥獸希毛,其性能暑。秦之戍卒不能其水土,戍者死於邊,輸者僨於道。秦民見行,如往棄市,因以謫發之,名曰「謫戍」。先發吏有謫及贅婿、賈人,後以嘗有市籍者,又後以大父母、父母嘗有市籍者,後入閭,取其左。發之不順,行者深怨,有背畔之心。凡民守戰至死而不降北者,以計為之也。故戰勝守固則有拜爵之賞,攻城屠邑則得其財鹵以富家室,故能使其眾蒙矢石,赴湯火,視死如生。今秦之發卒也,有萬死之害,而亡銖兩之報,死事之後不得一算之復,天下明知禍烈及己也。陳勝行戍,至於大澤,為天下先倡,天下從之如流水者,秦以威劫而行之之敝也。

胡人衣食之業不著於地,其勢易以擾亂邊竟。何以明之?胡人食肉飲酪,衣皮毛,非有城郭田宅之歸居,如飛鳥走獸於廣野,美草甘水則止,草盡水竭則移。以是觀之,往來轉徙,時至時去,此胡人之生業,而中國之所以離南畝也。今使胡人數處轉牧行獵於塞下,或當燕代,或當上郡、北地、隴西,以候備塞之卒,卒少則入。陛下不救,則邊民絕望而有降敵之心;救之,少發則不足,多發,遠縣纔至,則胡又已去。聚而不罷,為費甚大;罷之,則胡復入。如此連年,則中國貧苦而民不安矣。

陛下幸憂邊境,遣將吏發卒以治塞,甚大惠也。然令遠方之卒守塞,一歲而更,不知胡人之能,不如選常居者,家室田作,且以備之。以便為之高城深塹,具藺石,布渠答,復為一城其內,城間百五十步。要害之處,通川之道,調立城邑,毋下千家,為中周虎落。先為室屋,具田器,乃募罪人及免徒復作令居之;不足,募以丁奴婢贖罪及輸奴婢欲以拜爵者;不足,乃募民之欲往者。皆賜高爵,復其家。予冬夏衣,廩食,能自給而止。郡縣之民得買其爵,以自增至卿。其亡夫若妻者,縣官買予之。人情非有匹敵,不能久安其處。塞下之民,祿利不厚,不可使久居危難之地。胡人入驅而能止其所驅者,以其半予之,縣官為贖其民。如是,則邑里相救助,赴胡不避死。非以德上也,欲全親戚而利其財也。此與東方之戎卒不習地勢而心畏胡者,功相萬也。以陛下之時,徙民實邊,使遠方無屯戍之事,塞下之民父子相保,亡係虜之患,利施後世,名稱聖明,其與秦之行怨民,相去遠矣。

上從其言,募民徙塞下。錯復言:

陛下幸募民相徙以實塞下,使屯戍之事益省,輸將之費益寡,甚大惠也。下吏誠能稱厚惠,奉明法,存卹所徙之老弱,善遇其壯士,和輯其心而勿侵刻,使先至者安樂而不思故鄉,則貧民相募而勸往矣。臣聞古之徙遠方以實廣虛也,相其陰陽之和,嘗其水泉之味,審其土地之宜,觀其魇木之饒,然後營邑立城,製里割宅,通田作之道,正阡陌之界,先為築室,家有一堂二內,門戶之閉,置器物焉,民至有所居,作有所用,此民所以輕去故鄉而勸之新色也。為置醫巫,以救疾病,以脩祭祀,男女有昏,生死相卹,墳墓相從,種樹畜長,室屋完安,此所以使民樂其處而有長居之心也。

臣又聞古之制邊縣以備敵也,使五家為伍,伍有長;十長一里,里有假士;四里一連,連有假五百;十連一邑,邑有假候:皆擇其邑之賢材有護,習地形知民心者,居則習民於射法,出則教民於應敵。故卒伍成於內,則軍正定於外。服習以成,勿令遷徙,幼則同游,長則共事。夜戰聲相知,則足以相救;晝戰目相見,則足以相識;驩愛之心,足以相死。如此而勸以厚賞,威以重罰,則前死不還踵矣。所徙之民非壯有材力,但費衣糧,不可用也;雖有材力,不得良吏,猶亡功也。

陛下絕匈奴不與和親,臣竊意其冬來南也,壹大治,則終身創矣。欲立威者,始於折膠,來而不能困,使得氣去,後未易服也。愚臣亡識,唯陛下財察。

後詔有司舉賢良文學士,錯在選中。上親策詔之,曰:

惟十有五年九月壬子,皇帝曰:昔者大禹勤求賢士,施及方外,四極之內,舟車所至,人跡所及,靡不聞命,以輔其不逮;近者獻其明,遠者通厥聰,比善戮力,以翼天子。是以大禹能亡失德,夏以長楙。高皇帝親除大害,去亂從,並建豪英,以為官師,為諫爭,輔天子之闕,而翼戴漢宗也。賴天之靈,宗廟之福,方內以安,澤及四夷。今朕獲執天子之正,以承宗廟之祀,朕既不德,又不敏,明弗能燭,而智不能治,此大夫之所著聞也。故詔有司、諸侯王、三公、九卿及主郡吏,各帥其志,以選賢良明於國家之大體,通於人事之終始,及能直言極諫者,各有人數,將以匡朕之不逮。二三大夫之行當此三道,朕甚嘉之,故登大夫于朝,親諭朕志。大夫其上三道之要,及永惟朕之不德,吏之不平,政之不宣,民之不寧,四者之闕,悉陳其志,毋有所隱。上以薦先帝之宗廟,下以興愚民之休利,著之于篇,朕親覽焉,觀大夫所以佐朕,至與不至。書之,周之密之,重之閉之。興自朕躬,大夫其正論,毋枉執事。烏虖,戒之!二三大夫其帥志毋怠!

錯對曰:

平陽侯臣窋、汝陰侯臣灶、潁陰侯臣何、廷尉臣宜昌、隴西太守臣昆邪所選賢良太子家令臣錯昧死再拜言:臣竊聞古之賢主莫不求賢以為輔翼,故黃帝得力牧而為五帝,大禹得咎繇而為三王祖,齊桓得筦子而為五伯長。今陛下講于大禹及高皇帝之建豪英也,退託於不明,以求賢良,讓之至也。臣竊觀上世之傳,若高皇帝之建功業,陛下之德厚而得賢佐,皆有司之所覽,刻於玉版,藏於金匱,歷之春秋,紀之後世,為帝者祖宗,與天地相終。今臣窋等乃以臣錯充賦,甚不稱明詔求賢之意。臣錯魇茅臣,亡識知,昧死上愚對,曰:

詔策曰「明於國家大體」,愚臣竊以古之五帝明之。臣聞五帝神聖,其臣莫能及,故自親事,處于法宮之中,明堂之上;動靜上配天,下順地,中得人。故眾生之類亡不覆也,根著之徒亡不載也;燭以光明,亡偏異也;德上及飛鳥,下至水蟲草木諸產,皆被其澤。然後陰陽調,四時節,日月光,風雨時,膏露降,五穀孰,祅孽滅,賊氣息,民不疾疫,河出圖,洛出書,神龍至,鳳鳥翔,德澤滿天下,靈光施四海。此謂配天地,治國大體之功也。

詔策曰「通於人事終始」,愚臣竊以古之三王明之。臣聞三王臣主俱賢,故合謀相輔,計安天下,莫不本於人情。人情莫不欲壽,三王生而不傷也;人情莫不欲富,三王厚而不困也;人情莫不欲安,三王扶而不危也;人情莫不欲逸,三王節其力而不盡也。其為法令也,合於人情而後行之;其動眾使民也,本於人事然後為之。取人以己,內恕及人。情之所惡,不以彊人;情之所欲,不以禁民。是以天下樂其政,歸其德,望之若父母,從之若流水;百姓和親,國家安寧,名位不失,施及後世。此明於人情終始之功也。

詔策曰「直言極諫」,愚臣竊以五伯之臣明之。臣聞五伯不及其臣,故屬之以國,任之以事。五伯之佐之為人臣也,察身而不敢誣,奉法令不容私,盡心力不敢矜,遭患難不避死,見賢不居其上,受祿不過其量,不以亡能居尊顯之位。自行若此,可謂方正之士矣。其立法也,非以苦民傷眾而為之機陷也,以之興利除害,尊主安民而救暴亂也。其行賞也,非虛取民財妄予人也,以勸天下之忠孝而明其功也。故功多者賞厚,功少者賞薄。如此,斂民財以顧其功,而民不恨者,知與而安己也。其行罰也,非以忿怒妄誅而從暴心也,以禁天下不忠不孝而害國者也。故罪大者罰重,罪小者罰輕。如此,民雖伏罪至死而不怨者,知罪罰之至,自取之也。立法若此,可謂平正之吏矣。法之逆者,請而更之,不以傷民;主行之暴者,逆而復之,不以傷國。救主之失,補主之過,揚主之美,明主之功,使主內亡邪辟之行,外亡騫汙之名。事君若此,可謂直言極諫之士矣。此五伯之所以德匡天下,威正諸侯,功業甚美,名聲章明。舉天下之賢主,五伯與焉,此身不及其臣而使得直言極諫補其不逮之功也。今陛下人民之眾,威武之重,德惠之厚,令行禁止之勢,萬萬於五伯,而賜愚臣策曰「匡朕之不逮」,愚臣何足以識陛下之高明而奉承之!

詔策曰「吏之不平,政之不宣,民之不寧」,愚臣竊以秦事明之。臣聞秦始并天下之時,其主不及三王,而臣不及其佐,然功力不遲者,何也?地形便,山川利,財用足,民利戰。其所與並者六國,六國者,臣主皆不肖,謀不輯,民不用,故當此之時,秦最富彊。夫國富彊而鄰國亂者,帝王之資也,故秦能兼六國,立為天子。當此之時,三王之功不能進焉。及其末塗之衰也,任不肖而信讒賊;宮室過度,耆慾亡極,民力罷盡,賦斂不節;矜奮自賢,群臣恐諛,驕溢縱恣,不顧患禍;妄賞以隨善意,妄誅以快怒心,法令煩憯,刑罰暴酷,輕絕人命,身自射殺;天下寒心,莫安其處。姦邪之吏,乘其亂法,以成其威,獄官主斷,生殺自恣。上下瓦解,各自為制。秦始亂之時,吏之所先侵者,貧人賤民也;至其中節,所侵者富人吏家也;及其末塗,所侵者宗室大臣也。是故親疏皆危,外內咸怨,離散逋逃,人有走心。陳勝先倡,天下大潰,絕祀亡世,為異姓福。此吏不平,政不宣,民不寧之禍也。今陛下配天象地,覆露萬民,絕秦之跡,除其亂法;躬親本事,廢去淫末;除苛解嬈,寬大愛人;肉刑不用,罪人亡帑;非謗不治,鑄錢者除;通關去塞,不孽諸侯;賓禮長老,愛卹少孤;罪人有期,後宮出嫁;尊賜孝悌,農民不租;明詔軍師,愛士大夫;求進方正,廢退姦邪;除去陰刑,害民者誅;憂勞百姓,列侯就都;親耕節用,視民不奢。所為天下興利除害,變法易故,以安海內者,大功數十,皆上世之所難及,陛下行之,道純德厚,元元之民幸矣。

詔策曰「永惟朕之不德」,愚臣不足以當之。

詔策曰「悉陳其志,毋有所隱」,愚臣竊以五帝之賢臣明之。臣聞五帝其臣莫能及,則自親之;三王臣主俱賢,則共憂之;五伯不及其臣,則任使之。此所以神明不遺,而聖賢不廢也,故各當其世而立功德焉。傳曰「往者不可及,來者猶可待,能明其世者謂之天子」,此之謂也。竊聞戰不勝者易其地,民貧窮者變其業。今以陛下神明德厚,資財不下五帝,臨制天下,至今十有六年,民不益富,盜賊不衰,邊竟未安,其所以然,意者陛下未之躬親,而待群臣也。今執事之臣皆天下之選已,然莫能望陛下清光,譬之猶五帝之佐也。陛下不自躬親,而待不望清光之臣,臣竊恐神明之遺也。日損一日,歲亡一歲,日月益暮,盛德不及究於天下,以傳萬世,愚臣不自度量,竊為陛下惜之。昧死上狂惑魇茅之愚,臣言唯陛下財擇。

時賈誼已死,對策者百餘人,唯錯為高第,繇是遷中大夫。

錯又言宜削諸侯事,及法令可更定者,書凡三十篇。孝文雖不盡聽,然奇其材。當是時,太子善錯計策,爰盎諸大功臣多不好錯。

景帝即位,以錯為內史。錯數請間言事,輒聽,幸傾九卿,法令多所更定。丞相申屠嘉心弗便,力未有以傷。內史府居太上廟堧中,門東出,不便,錯乃穿門南出,鑿廟堧垣。丞相大怒,欲因此過為奏請誅錯。錯聞之,即請間為上言之。丞相奏事,因言錯擅鑿廟垣為門,請下廷尉誅。上曰:「此非廟垣,乃堧中垣,不致於法。」丞相謝。罷朝,因怒謂長史曰:「吾當先斬以聞,乃先請,固誤。」丞相遂發病死。錯以此愈貴。

遷為御史大夫,請諸侯之罪過,削其支郡。奏上,上公卿列侯宗室,莫敢難,獨竇嬰爭之,繇此與錯有隙。錯所更令三十章,諸侯讙譁。錯父聞之,從潁川來,謂錯曰:「上初即位,公為政用事,侵削諸侯,疏人骨肉,口讓多怨,公何為也!」錯曰:「固也。不如此,天子不尊,宗廟不安。」父曰:「劉氏安矣,而晁氏危,吾去公歸矣!」遂飲藥死,曰:「吾不忍見禍逮身。」

後十餘日,吳楚七國俱反,以誅錯為名。上與錯議出軍事,錯欲令上自將兵,而身居守。會竇嬰言爰盎,詔召入見,上方與錯調兵食。上問盎曰:「君嘗為吳相,知吳臣田祿伯為人虖?今吳楚反,於公意何如?」對曰:「不足憂也,今破矣。」上曰:「吳王即山鑄錢,煮海為鹽,誘天下豪桀,白頭舉事,此其計不百全,豈發虖?何以言其無能為也?」盎對曰:「吳銅鹽之利則有之,安得豪桀而誘之!誠令吳得豪桀,亦且輔而為誼,不反矣。吳所誘,皆亡賴子弟,亡命鑄錢姦人,故相誘以亂。」錯曰:「盎策之善。」上問曰:「計安出?」盎對曰:「願屏左右。」上屏人,獨錯在。盎曰:「

臣所言,人臣不得知。」乃屏錯。錯趨避東箱,甚恨。上卒問盎,對曰:「吳楚相遺書,言高皇帝王子弟各有分地,今賊臣晁錯擅適諸侯,削奪之地,以故反名為西共誅錯,復故地而罷。方今計,獨有斬錯,發使赦吳楚七國,復其故地,則兵可毋血刃而俱罷。」於是上默然,良久曰:「顧誠何如,吾不愛一人謝天下。」盎曰:「愚計出此,唯上孰計之。」乃拜盎為太常,密裝治行。

後十餘日,丞相青翟、中尉嘉、廷尉蓝劾奏錯曰:「吳王反逆亡道,欲危宗廟,天下所當共誅。今御史大夫錯議曰:『兵數百萬,獨屬群臣,不可信,陛下不如自出臨兵,使錯居守。徐、僮之旁吳所未下者可以予吳。』錯不稱陛下德信,欲疏群臣百姓,又欲以城邑予吳,亡臣子禮,大逆無道。錯當要斬,父母妻子同產無少長皆棄巿。臣請論如法。」制曰:「可。」錯殊不知。乃使中尉召錯,紿載行巿。錯衣朝衣斬東巿。

錯已死,謁者僕射鄧公為校尉,擊吳楚為將。還,上書言軍事,見上。上問曰:「道軍所來,聞晁錯死,吳楚罷不?」鄧公曰:「吳為反數十歲矣,發怒削地,以誅錯為名,其意不在錯也。且臣恐天下之士拑口不敢復言矣。」上曰:「何哉?」鄧公曰:「

夫晁錯患諸侯彊大不可制,故請削之,以尊京師,萬世之利也。計畫始行,卒受大戮,內杜忠臣之口,外為諸侯報仇,臣竊為陛下不取也。」於景帝喟然長息,曰:「公言善,吾亦恨之。」乃拜鄧公為城陽中尉。

鄧公,成固人也,多奇計。建元年中,上招賢良,公卿言鄧先。鄧先時免,起家為九卿。一年,復謝病免歸。其子章,以修黃老言顯諸公間。

贊曰:爰盎雖不好學,亦善傅會,仁心為質,引義慷慨。遭孝文初立,資適逢世。時已變易,及吳壹說,果於用辯,身亦不遂。晁錯銳於為國遠慮,而不見身害。其父睹之,經於溝瀆,亡益救敗,不如趙母指括,以全其宗。悲夫!錯雖不終,世哀其忠。故論其施行之語著于篇。


\end{pinyinscope}