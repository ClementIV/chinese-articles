\article{藝文志}

\begin{pinyinscope}
昔仲尼沒而微言絕,七十子喪而大義乖。故春秋分為五,詩分為四,易有數家之傳。戰國從衡,真偽分爭,諸子之言紛然殽亂。至秦患之,乃燔滅文章,以愚黔首。漢興,改秦之敗,大收篇籍,廣開獻書之路。迄孝武世,書缺簡脫,禮壞樂崩,聖上喟然而稱曰:「朕甚閔焉!」於是建藏書之策,置寫書之官,下及諸子傳說,皆充祕府。至成帝時,以書頗散亡,使謁者陳農求遺書於天下。詔光祿大夫劉向校經傳諸子詩賦,步兵校尉任宏校兵書,太史令尹咸校數術,侍醫李柱國校方技。每一書已,向輒條其篇目,撮其指意,錄而奏之。會向卒,哀帝復使向子侍中奉車都尉歆卒父業。歆於是總群書而奏其七略,故有輯略,有六藝略,有諸子略,有詩賦略,有兵書略,有術數略,有方技略。今刪其要,以備篇籍。

易經十二篇,施、孟、梁丘三家。

易傳周氏二篇。

服氏二篇。

楊氏二篇。

蔡公二篇。

韓氏二篇。

王氏二篇。

丁氏八篇。

古五子十八篇。

淮南道訓二篇。

古雜八十篇,雜災異三十五篇,神輸五篇,圖一。

孟氏京房十一篇,災異孟氏京房六十六篇,五鹿充宗略說三篇,京氏段嘉十二篇。

章句施、孟、梁丘氏各二篇。

凡易十三家,二百九十四篇。

《易》曰:「宓戲氏仰觀象於天,俯觀法於地,觀鳥獸之文,與地之宜,近取諸身,遠取諸物,於是始作八卦,以通神明之德,以類萬物之情。」至於殷、周之際,紂在上位,逆天暴物,文王以諸侯順命而行道,天人之占可得而效,於是重易六爻,作上下篇。孔氏為之彖、象、繫辭、文言、序卦之屬十篇。故曰易道深矣,人更三聖,世歷三古。及秦燔書,而易為筮卜之事,傳者不絕。漢興,田和傳之。訖于宣、元,有施、孟、梁丘、京氏列於學官,而民間有費、高二家之說。劉向以中古文易經校施、孟、梁丘經,或脫去「無咎」、「悔亡」,唯費氏經與古文同。

尚書古文經四十六卷。

經二十九卷。

傳四十一篇。

歐陽章句三十一卷。

大、小夏侯章句各二十九卷。

大、小夏侯解故二十九篇。

歐陽說義二篇。

劉向五行傳記十一卷。

許商五行傳記一篇。

周書七十一篇。

議奏四十二篇。

凡書九家,四百一十二篇。

《易》曰:「河出圖,雒出書,聖人則之。」故書之所起遠矣,至孔子篹焉,上斷於堯,下訖于秦,凡百篇,而為之序,言其作意。秦燔書禁學,濟南伏生獨壁藏之。漢興亡失,求得二十九篇,以教齊魯之間。訖孝宣世,有歐陽、大小夏侯氏,立於學官。古文尚書者,出孔子壁中。武帝末,魯共王壞孔子宅,欲以廣其宮,而得古文尚書及禮記、論語、孝經凡數十篇,皆古字也。共王往入其宅,聞鼓琴瑟鍾磬之音,於是懼,乃止不壞。孔安國者,孔子後也,悉得其書,以考二十九篇,得多十六篇。安國獻之。遭巫蠱事,未列于學官。劉向以中古文校歐陽、大小夏侯三家經文,酒誥脫簡一,召誥脫簡二。率簡二十五字者,脫亦二十五字,簡二十二字者,脫亦二十二字,文字異者七百有餘,脫字數十。書者,古之號令,號令於眾,其言不立具,則聽受施行者弗曉。古文讀應爾雅,故解古今語而可知也。

詩經二十八卷,魯、齊、韓三家。

魯故二十五卷。

魯說二十八卷。

齊后氏故二十卷。

齊孫氏故二十七卷。

齊后氏傳三十九卷。

齊孫氏傳二十八卷。

齊雜記十八卷。

韓故三十六卷。

韓內傳四卷。

韓外傳六卷。

韓說四十一卷。

毛詩二十九卷。

毛詩故訓傳三十卷。

凡詩六家,四百一十六卷。

書曰:「詩言志,哥詠言。」故哀樂之心感,而哥詠之聲發。誦其言謂之詩,詠其聲謂之哥。故古有采詩之官,王者所以觀風俗,知得失,自考正也。孔子純取周詩,上采殷,下取魯,凡三百五篇,遭秦而全者,以其諷誦,不獨在竹帛故也。漢興,魯申公為詩訓故,而齊轅固、燕韓生皆為之傳。或取春秋,采雜說,咸非其本義。與不得已,魯最為近之。三家皆列於學官。又有毛公之學,自謂子夏所傳,而河間獻王好之,未得立。

禮古經五十六卷,經七十篇。

記百三十一篇。

明堂陰陽三十三篇。

王史氏二十一篇。

曲臺后倉九篇。

中庸說二篇。

明堂陰陽說五篇。

周官經六篇。

周官傳四篇。

軍禮司馬法百五十五篇。

古封禪群祀二十二篇。

封禪議對十九篇。

漢封禪群祀三十六篇。

議奏三十八篇。

凡禮十三家,五百五十五篇。

《易》曰:「有夫婦父子君臣上下,禮義有所錯。」而帝王質文世有損益,至周曲為之防,事為之制,故曰:「禮經三百,威儀三千。」及周之衰,諸侯將踰法度,惡其害己,皆滅去其籍,自孔子時而不具,至秦大壞。漢興,魯高堂生傳士禮十七篇。訖孝宣世,后倉最明。戴德、戴聖、慶普皆其弟子,三家立於學官。禮古經者,出於魯淹中及孔氏,學七十篇文相似,多三十九篇。及明堂陰陽、王史氏記所見,多天子諸侯卿大夫之制,雖不能備,猶瘉倉等推士禮而致於天子之說。

樂記二十三篇。

樂記二十三篇。

王禹記二十四篇。

雅歌詩四篇。

雅琴趙氏七篇。

雅琴師氏八篇。

雅琴龍氏九十九篇。

凡樂六家,百六十五篇。

《易》曰:「先王作樂崇德,殷薦之上帝,以享祖考。」故自黃帝下至三代,樂各有名。孔子曰:「安上治民,莫善於禮;移風易俗,莫善於樂。」二者相與並行。周衰俱壞,樂尤微眇,以音律為節,又為鄭衛所亂故無遺法。漢興,制氏以雅樂聲律,世在樂官,頗能紀其鏗鏘鼓舞,而不能言其義。六國之君,魏文侯最為好古,孝文時得其樂人竇公,獻其書,乃周官大宗伯之大司樂章也。武帝時,河間獻王好儒,與毛生等共采周官及諸子言樂事者,以作樂記,獻八佾之舞,與制氏不相遠。其內史丞王定傳之,以授常山王禹。禹,成帝時為謁者,數言其義,獻二十四卷記。劉向校書,得樂記二十三篇,與禹不同,其道浸以益微。

春秋古經十二篇,經十一卷。

左氏傳三十卷。

公羊傳十一卷。

穀梁傳十一卷。

鄒氏傳十一卷。

夾氏傳十一卷。

左氏微二篇。

鐸氏微三篇。

張氏微十篇。

虞氏微傳二篇。

公羊外傳五十篇。

穀梁外傳二十篇。

公羊章句三十八篇。

穀梁章句三十三篇。

公羊雜記八十三篇。

公羊顏氏記十一篇。

公羊董仲舒治獄十六篇。

議奏三十九篇。

國語二十一篇。

新國語五十四篇。

世本十五篇。

戰國策二十三篇。

奏事二十篇。

楚漢春秋九篇。

太史公百三十篇。

馮商所續太史公七篇。

太古以來年紀二篇。

漢著記百九十卷。

漢大年紀五篇。

凡春秋二十三家,九百四十八篇。

古之王者世有史官,君舉必書,所以慎言行,昭法式也。左史記言,右史記事,事為春秋,言為尚書,帝王靡不同之。周室既微,載籍殘缺,仲尼思存前聖之業,乃稱曰:「夏禮吾能言之,杞不足徵也;殷禮吾能言之,宋不足徵也。文獻不足故也,足則吾能徵之矣。」以魯周公之國,禮文備物,史官有法,故與左丘明觀其史記,據行事,仍人道,因興以立功,就敗以成罰,假日月以定曆數,藉朝聘以正禮樂。有所褒諱貶損,不可書見,口授弟子,弟子退而異言。丘明恐弟子各安其意,以失其真,故論本事而作傳,明夫子不以空言說經也。春秋所貶損大人當世君臣,有威權勢力,其事實皆形於傳,是以隱其書而不宣,所以免時難也。及末世口說流行,故有公羊、穀梁、鄒、夾之傳。四家之中,公羊、穀梁立於學官,鄒氏無師,夾氏未有書。

論語古二十一篇。

齊二十二篇。

魯二十篇,傳十九篇。

齊說二十九篇。

魯夏侯說二十一篇。

魯安昌侯說二十一篇。

魯王駿說二十篇。

燕傳說三卷。

議奏十八篇。

孔子家語二十七卷。

孔子三朝七篇。

孔子徒人圖法二卷。

凡論語十二家,二百二十九篇。

論語者,孔子應答弟子時人及弟子相與言而接聞於夫子之語也。當時弟子各有所記。夫子既卒,門人相與輯而論篹,故謂之論語。漢興,有齊、魯之說。傳齊論者,昌邑中尉王吉、少府宋畸、御史大夫貢禹、尚書令五鹿充宗、膠東庸生,唯王陽名家。傳魯論語者,常山都尉龔奮、長信少府夏侯勝、丞相韋賢、魯扶卿、前將軍蕭望之、安昌侯張禹,皆名家。張氏最後而行於世。

孝經古孔氏一篇。

孝經古孔氏一篇。

孝經一篇。

長孫氏說二篇。

江氏說一篇。

翼氏說一篇。

后氏說一篇。

雜傳四篇。

安昌侯說一篇。

五經雜議十八篇。

爾雅三卷二十篇。

小爾雅一篇,古今字一卷。

弟子職一篇。

說三篇。

凡孝經十一家,五十九篇。

孝經者,孔子為曾子陳孝道也。夫孝,天之經,地之義,民之行也。舉大者言,故曰孝經。漢興,長孫氏、博士江翁、少府后倉、諫大夫翼奉、安昌侯張禹傳之,各自名家。經文皆同,唯孔氏壁中古文為異。「父母生之,續莫大焉」,「故親生之膝下」,諸家說不安處,古文字讀皆異。

史籀十五篇。

八體六技。

蒼頡一篇。

凡將一篇。

急就一篇。

元尚一篇。

訓纂一篇。

別字十三篇。

蒼頡傳一篇。

揚雄蒼頡訓纂一篇。

杜林蒼頡訓纂一篇。

杜林蒼頡故一篇。

凡小學十家,四十五篇。

《易》曰:「上古結繩以治,後世聖人易之以書契,百官以治,萬民以察,蓋取諸夬。」「夬,揚於王庭」,言其宣揚於王者朝廷,其用最大也。古者八歲入小學,故周官保氏掌養國子,教之六書,謂象形、象事、象意、象聲、轉注、假借,造字之本也。漢興,蕭何草律,亦著其法,曰:「太史試學童,能諷書九千字以上,乃得為史。又以六體試之,課最者以為尚書御史史書令史。吏民上書,字或不正,輒舉劾。」六體者,古文、奇字、篆書、隸書、繆篆、蟲書,皆所以通知古今文字,摹印章,書幡信也。古制,書必同文,不知則闕,問諸故老,至於衰世,是非無正,人用其私。故孔子曰:「吾猶及史之闕文也,今亡矣夫!」蓋傷其浸不正。史籀篇者,周時史官教學童書也,與孔氏壁中古文異體。蒼頡七章者,秦丞相李斯所作也;爰歷六章者,車府令趙高所作也;博學七章者,太史令胡母敬所作也:文字多取史籀篇,而篆體復頗異,所謂秦篆者也。是時始造隸書矣,起於官獄多事,苟趨省易,施之於徒隸也。漢書,閭里書師合蒼頡、爰歷、博學三篇,斷六十字以為一章,凡五十五章,并為蒼頡篇。武帝時司馬相如作凡將篇,無復字。元帝時黃門令史游作急就篇,成帝時將作大匠李長作元尚篇,皆蒼頡中正字也。凡將則頗有出矣。至元始中,徵天下通小學者以百數,各令記字於庭中。揚雄取其有用者以作訓纂篇,順續蒼頡,又易蒼頡中重復之字,凡八十九章。臣復續揚雄作十二章,凡一百二章,無復字,六藝群書所載略備矣。蒼頡多古字,俗師失其讀,宣帝時徵齊人能正讀者,張敞從受之,傳至外孫之子杜林,為作訓故,并列焉。

凡六藝一百三家,三千一百二十三篇。

六藝之文:樂以和神,仁之表也;詩以正言,義之用也;禮以明體,明者著見,故無訓也;書以廣聽,知之術也;春秋以斷事,信之符也。五者,蓋五常之道,相須而備,而易為之原。故曰「易不可見,則乾坤或幾乎息矣」,言與天地為終始也。至於五學,世有變改,猶五行之更用事焉。古之學者耕且養,三年而通一藝,存其大體,玩經文而已,是故用日少而畜德多,三十而五經立也。後世經傳既已乖離,博學者又不思多聞闕疑之義,而務碎義逃難,便辭巧說,破壞形體;說五字之文,至於二三萬言。後進彌以馳逐,故幼童而守一藝,白首而後能言;安其所習,毀所不見,終以自蔽。此學者之大患也。序六藝為九種。

晏子八篇。

子思二十三篇。

曾子十八篇。

漆雕子十三篇。

宓子十六篇。

景子三篇。

世子二十一篇。

魏文侯六篇。

李克七篇。

公孫尼子二十八篇。

孟子十一篇。

孫卿子三十三篇。

羋子十八篇。

內業十五篇。

周史六弢六篇。

周政六篇。

周法九篇。

河間周制十八篇。

讕言十一篇。

功議四篇。

甯越一篇。

王孫子一篇。

公孫固一篇。

李氏春秋二篇。

羊子四篇。

董子一篇。

侯子一篇。

徐子四十二篇。

魯仲連子十四篇。

平原君七篇。

虞氏春秋十五篇。

高祖傳十三篇。

陸賈二十三篇。

劉敬三篇。

孝文傳十一篇。

賈山八篇。

太常蓼侯孔臧十篇。

賈誼五十八篇。

河間獻王對上下三雍宮三篇。

董仲舒百二十三篇。

兒寬九篇。

公孫弘十篇。

終軍八篇。

吾丘壽王六篇。

虞丘說一篇。

莊助四篇。

臣彭四篇。

鉤盾冗從李步昌八篇。

儒家言十八篇。

桓寬鹽鐵論六十篇。

劉向所序六十七篇。

揚雄所序三十八篇。

右儒五十三家,八百三十六篇。

儒家者流,蓋出於司徒之官,助人君順陰陽明教化者也。游文於六經之中,留意於仁義之際,祖述堯舜,憲章文武,宗師仲尼,以重其言,於道最為高。孔子曰:「如有所譽,其有所試。」唐虞之隆,殷周之盛,仲尼之業,已試之效者也。然惑者既失精微,而辟者又隨時抑揚,違離道本,苟以譁眾取寵。後進循之,是以五經乖析,儒學浸衰,此辟儒之患。

伊尹五十一篇。

太公二百三十七篇。謀八十一篇,言七十一篇,兵八十五篇。

辛甲二十九篇。

鬻子二十二篇。

筦子八十六篇。

老子鄰氏經傳四篇。

老子傅氏經說三十七篇。

老子徐氏經說六篇。

劉向說老子四篇。

文子九篇。

蜎子十三篇。

關尹子九篇。

莊子五十二篇。

列子八篇。

老成子十八篇。

長盧子九篇。

王狄子一篇。

公子牟四篇。

田子二十五篇。

老萊子十六篇。

黔婁子四篇。

宮孫子二篇。

鶡冠子一篇。

周訓十四篇。

黃帝四經四篇。

黃帝銘六篇。

黃帝君臣十篇。

雜黃帝五十八篇。

力牧二十二篇。

孫子十六篇。

捷子二篇。

曹羽二篇。

郎中嬰齊十二篇。

臣君子二篇。

鄭長者一篇。

楚子三篇。

道家言二篇。

右道三十七家,九百九十三篇。

道家者流,蓋出於史官,歷記成敗存亡禍福古今之道,然後知秉要執本,清虛以自守,卑弱以自持,此君人南面之術也。合於堯之克攘,易之嗛嗛,一謙而四益,此其所長也。及放者為之,則欲絕去禮學,兼棄仁義,曰獨任清虛可以為治。

宋司星子韋三篇。

公檮生終始十四篇。

公孫發二十二篇。

鄒子四十九篇。

鄒子終始五十六篇。

乘丘子五篇。

杜文公五篇。

黃帝泰素二十篇。

南公三十一篇。

容成子十四篇。

張蒼十六篇。

鄒奭子十二篇。

閭丘子十三篇。

馮促十三篇。

將鉅子五篇。

五曹官制五篇。

周伯十一篇。

衛侯官十二篇。

于長天下忠臣九篇。

公孫渾邪十五篇。

雜陰陽三十八篇。

右陰陽二十一家,三百六十九篇。

陰陽家者流,蓋出於羲和之官,敬順昊天,歷象日月星辰,敬授民時,此其所長也。及拘者為之,則牽於禁忌,泥於小數,舍人事而任鬼神。

李子三十二篇。

商君二十九篇。

申子六篇。

處子九篇。

慎子四十二篇。

韓子五十五篇。

游棣子一篇。

晁錯三十一篇。

燕十事十篇。

法家言二篇。

右法十家,二百一十七篇。

法家者流,蓋出於理官,信賞必罰,以輔禮制。《易》曰「先王以明罰飭法」,此其所長也。及刻者為之,則無教化,去仁愛,專任刑法而欲以致治,至於殘害至親,傷恩薄厚。

鄧析二篇。

尹文子一篇。

公孫龍子十四篇。

成公生五篇。

惠子一篇。

黃公四篇。

毛公九篇。

右名七家,三十六篇。

名家者流,蓋出於禮官。古者名位不同,禮亦異數。孔子曰:「必也正名乎!名不正則言不順,言不順則事不成。」此其所長也。及譥者為之,則苟鉤鈲析亂而已。

尹佚二篇。

田俅子三篇。

我子一篇。

隨巢子六篇。

胡非子三篇。

墨子七十一篇。

右墨六家,八十六篇。

墨家者流,蓋出於清廟之守。茅屋采椽,是以貴儉;養三老五更,是以兼愛;選士大射,是以上賢;宗祀嚴父,是以右鬼;順四時而行,是以非命;以孝視天下,是以上同:此其所長也。及蔽者為之,見儉之利,因以非禮,推兼愛之意,而不知別親疏。

蘇子三十一篇。

張子十篇。

龐煖二篇。

闕子一篇。

國筮子十七篇。

秦零陵令信一篇。

蒯子五篇。

鄒陽七篇。

主父偃二十八篇。

徐樂一篇。

莊安一篇。

待詔金馬聊蒼三篇。

右從橫十二家,百七篇。

從橫家者流,蓋出於行人之官。孔子曰:「誦詩三百,使於四方,不能專對,雖多亦奚以為?」又曰:「使乎,使乎!」言其當權事制宜,受命而不受辭,此其所長也。及邪人為之,則上詐諼而棄其信。

孔甲盤盂二十六篇。

大𢁰三十七篇。

五子胥八篇。

子晚子三十五篇。

由余三篇。

尉繚子二十九篇。

尸子二十篇。

呂氏春秋二十六篇。

淮南內二十一篇。

淮南外三十三篇。

東方朔二十篇。

伯象先生一篇。

荊軻論五篇。

吳子一篇。

公孫尼一篇。

博士臣賢對一篇。

臣說三篇。

解子簿書三十五篇。

推雜書八十七篇。

雜家言一篇。

右雜二十家,四百三篇。

雜家者流,蓋出於議官。兼儒、墨,合名、法,知國體之有此,見王治之無不貫,此其所長也。及盪者為之,則漫羨而無所歸心。

神農二十篇。

野老十七篇。

宰氏十七篇。

董安國十六篇。

尹都尉十四篇。

趙氏五篇。

氾勝之十八篇。

王氏六篇。

蔡癸一篇。

右農九家,百一十四篇。

農家者流,蓋出於農稷之官。播百穀,勸耕桑,以足衣食,故八政一曰食,二曰貨。孔子曰「所重民食」,此其所長也。及鄙者為之,以為無所事聖王,欲使君臣並耕,誖上下之序。

伊尹說二十七篇。

鬻子說十九篇。

周考七十六篇。

青史子五十七篇。

師曠六篇。

務成子十一篇。

宋子十八篇。

天乙三篇。

黃帝說四十篇。

封禪方說十八篇。

待詔臣饒心術二十五篇。

待詔臣安成未央術一篇。

臣壽周紀七篇。

虞初周說九百四十三篇。

百家百三十九卷。

右小說十五家,千三百八十篇。

小說家者流,蓋出於稗官。街談巷語,道聽塗說者之所造也。孔子曰:「雖小道,必有可觀者焉,致遠恐泥,是以君子弗為也。」然亦弗滅也。閭里小知者之所及,亦使綴而不忘。如或一言可采,此亦芻蕘狂夫之議也。

凡諸子百八十九家,四千三百二十四篇。

諸子十家,其可觀者九家而已。皆起於王道既微,諸侯力政,時君世主,好惡殊方,是以九家之術蠭出並作,各引一端,崇其所善,以此馳說,取合諸侯。其言雖殊,辟猶水火,相滅亦相生也。仁之與義,敬之與和,相反而皆相成也。《易》曰:「天下同歸而殊塗,一致而百慮。」今異家者各推所長,窮知究慮,以明其指,雖有蔽短,合其要歸,亦六經之支與流裔。使其人遭明王聖主,得其所折中,皆股肱之材已。仲尼有言:「禮失而求諸野。」方今去聖久遠,道術缺廢,無所更索,彼九家者,不猶瘉於野乎?若能修六藝之術,而觀此九家之言,舍短取長,則可以通萬方之略矣。

屈原賦二十五篇。

唐勒賦四篇。

宋玉賦十六篇。

趙幽王賦一篇。

莊夫子賦二十四篇。

賈誼賦七篇。

枚乘賦九篇。

司馬相如賦二十九篇。

淮南王賦八十二篇。

淮南王群臣賦四十四篇。

太常蓼侯孔臧賦二十篇。

陽丘侯劉双賦十九篇。

吾丘壽王賦十五篇。

蔡甲賦一篇。

上所自造賦二篇。

兒寬賦二篇。

光祿大夫張子僑賦三篇。

陽成侯劉德賦九篇。

劉向賦三十三篇。

王褒賦十六篇。

右賦二十家,三百六十一篇。

陸賈賦三篇。

枚皋賦百二十篇。

朱建賦二篇。

常侍郎莊璴奇賦十一篇。

嚴助賦三十五篇。

朱買臣賦三篇。

宗正劉辟彊賦八篇。

司馬遷賦八篇。

郎中臣嬰齊賦十篇。

臣說賦九篇。

臣吾賦十八篇。

遼東太守蘇季賦一篇。

蕭望之賦四篇。

河內太守徐明賦三篇。

給事黃門侍郎李息賦九篇。

淮陽憲王賦二篇。

揚雄賦十二篇。

待詔馮商賦九篇。

博士弟子杜參賦二篇。

車郎張豐賦三篇。

驃騎將軍朱宇賦三篇。

右賦二十一家,二百七十四篇。

孫卿賦十篇。

秦時雜賦九篇。

李思孝景皇帝頌十五篇。

廣川惠王越賦五篇。

長沙王群臣賦三篇。

魏內史賦二篇。

東暆令延年賦七篇。

衛士令李忠賦二篇。

張偃賦二篇。

賈充賦四篇。

張仁賦六篇。

秦充賦二篇。

李步昌賦二篇。

侍郎謝多賦十篇。

平陽公主舍人周長孺賦二篇。

雒陽錡華賦九篇。

眭弘賦一篇。

別栩陽賦五篇。

臣昌市賦六篇。

臣義賦二篇。

黃門書者假史王商賦十三篇。

侍中徐博賦四篇。

黃門書者王廣呂嘉賦五篇。

漢中都尉丞華龍賦二篇。

左馮翊史路恭賦八篇。

右賦二十五家,百三十六篇。

客主賦十八篇。

雜行出及頌德賦二十四篇。

雜四夷及兵賦二十篇。

雜中賢失意賦十二篇。

雜思慕悲哀死賦十六篇。

雜鼓琴劍戲賦十三篇。

雜山陵水泡雲氣雨旱賦十六篇。

雜禽獸六畜昆蟲賦十八篇。

雜器械草木賦三十三篇。

文雜賦三十四篇。

成相雜辭十一篇。

隱書十八篇。

右雜賦十二家,二百三十三篇。

高祖歌詩二篇。

泰一雜甘泉壽宮歌詩十四篇。

宗廟歌詩五篇。

漢興以來兵所誅滅歌詩十四篇。

出行巡狩及游歌詩十篇。

臨江王及愁思節士歌詩四篇。

李夫人及幸貴人歌詩三篇。

詔賜中山靖王子噲及孺子妾冰未央材人歌詩四篇。

吳楚汝南歌詩十五篇。

燕代謳雁門雲中隴西歌詩九篇。

邯鄲河間歌詩四篇。

齊鄭歌詩四篇。

淮南歌詩四篇。

左馮翊秦歌詩三篇。

京兆尹秦歌詩五篇。

河東蒲反歌詩一篇。

黃門倡車忠等歌詩十五篇。

雜各有主名歌詩十篇。

雜歌詩九篇。

雒陽歌詩四篇。

河南周歌詩七篇。

河南周歌聲曲折七篇。

周謠歌詩七十五篇。

周謠歌詩聲曲折七十五篇。

諸神歌詩三篇。

送迎靈頌歌詩三篇。

周歌詩二篇。

南郡歌詩五篇。

右歌詩二十八家,三百一十四篇。

凡詩賦百六家,千三百一十八篇。

傳曰:「不歌而誦謂之賦,登高能賦可以為大夫。」言感物造耑,材知深美,可與圖事,故可以為列大夫也。古者諸侯卿大夫交接鄰國,以微言相感,當揖讓之時,必稱詩以諭其志,蓋以別賢不肖而觀盛衰焉。故孔子曰「不學詩,無以言」也。春秋之後,周道浸壞,聘問歌詠不行於列國,學詩之士逸在布衣,而賢人失志之賦作矣。大儒孫卿及楚臣屈原離讒憂國,皆作賦以風,咸有惻隱古詩之義。其後宋玉、唐勒,漢興枚乘、司馬相如,下及揚子雲,競為侈麗閎衍之詞,沒其風諭之義。是以揚子悔之,曰:「詩人之賦麗以則,辭人之賦麗以淫。如孔氏之門人用賦也,則賈誼登堂,相如入室矣,如其不用何!」自孝武立樂府而采歌謠,於是有代趙之謳,秦楚之風,皆感於哀樂,緣事而發,亦可以觀風俗,知薄厚云。詩賦為五種。

吳孫子兵法八十二篇。

齊孫子八十九篇。

公孫鞅二十七篇。

吳起四十八篇。

范蠡二篇。

大夫種二篇。

季子十篇。

娷一篇。

兵春秋一篇。

龐煖三篇。

兒良一篇。

廣武君一篇。

韓信三篇。

右兵權謀十三家,二百五十九篇。

權謀者,以正守國,以奇用兵,先計而後戰,兼形勢,包陰陽,用技巧者也。

楚兵法七篇。

蚩尤二篇。

孫軫五篇。

繇敘二篇。

王孫十六篇。

尉繚三十一篇。

魏公子二十一篇。

景子十三篇。

李良三篇。

丁子一篇。

項王一篇。

右兵形勢十一家,九十二篇。圖十八卷。形勢者,雷動風舉,後發而先至,離合背鄉,變化無常,以輕疾制敵者也。

太壹兵法一篇。

天一兵法三十五篇。

神農兵法一篇。

黃帝十六篇。

封胡五篇。

風后十三篇。

力牧十五篇。

鵊冶子一篇。

鬼容區三篇。

地典六篇。

孟子一篇。

東父三十一篇。

師曠八篇。

萇弘十五篇。

別成子望軍氣六篇。

辟兵威勝方七十篇。

右陰陽十六家,二百四十九篇,圖十卷。

陰陽者,順時而發,推刑德,隨斗擊,因五勝,假鬼神而為助者也。

鮑子兵法十篇。

五子胥十篇。

公勝子五篇。

苗子五篇。

逢門射法二篇。

陰通成射法十一篇。

李將軍射法三篇。

魏氏射法六篇。

彊弩將軍王圍射法五卷。

望遠連弩射法具十五篇。

護軍射師王賀射書五篇。

蒲苴子弋法四篇。

劍道三十八篇。

手搏六篇。

雜家兵法五十七篇。

蹴雏二十五篇。

右兵技巧十三家,百九十九篇。

技巧者,習手足,便器械,積機關,以立攻守之勝者也。

凡兵書五十三家,七百九十篇,圖四十三卷。

兵家者,蓋出古司馬之職,王官之武備也。洪範八政,八曰師。孔子曰為國者「足食足兵」,「以不教民戰,是謂棄之」,明兵之重也。《易》曰「古者弦木為弧,剡木為矢,弧矢之利,以威天下」,其用上矣。後世燿金為刃,割革為甲,器械甚備。下及湯武受命,以師克亂而濟百姓,動之以仁義,行之以禮讓,司馬法是其遺事也。自春秋至於戰國,出奇設伏,變詐之兵並作。漢興,張良、韓信序次兵法,凡百八十二家,刪取要用,定著三十五家。諸呂用事而盜取之。武帝時,軍政楊僕捃摭遺逸,紀奏兵錄,猶未能備。至于孝成,命任宏論次兵書為四種。

泰壹雜子星二十八卷。

五殘雜變星二十一卷。

黃帝雜子氣三十三篇。

常從日月星氣二十一卷。

皇公雜子星二十二卷。

淮南雜子星十九卷。

泰壹雜子雲雨三十四卷。

國章觀霓雲雨三十四卷。

泰階六符一卷。

金度玉衡漢五星客流出入八篇。

漢五星彗客行事占驗八卷。

漢日旁氣行事占驗三卷。

漢流星行事占驗八卷。

漢日旁氣行占驗十三卷。

漢日食月暈雜變行事占驗十三卷。

海中星占驗十二卷。

海中五星經雜事二十二卷。

海中五星順逆二十八卷。

海中二十八宿國分二十八卷。

海中二十八宿臣分二十八卷。

海中日月彗虹雜占十八卷。

圖書祕記十七篇。

右天文二十一家,四百四十五卷。

天文者,序二十八宿,步五星日月,以紀吉凶之象,聖王所以參政也。《易》曰:「觀乎天文,以察時變。」然星事杂悍,非湛密者弗能由也。夫觀景以譴形,非明王亦不能服聽也。以不能由之臣,諫不能聽之王,此所以兩有患也。

黃帝五家曆三十三卷。

顓頊曆二十一卷。

顓頊五星曆十四卷。

日月宿曆十三卷。

夏殷周魯曆十四卷。

天曆大曆十八卷。

漢元殷周諜曆十七卷。

耿昌月行帛圖二百三十二卷。

耿昌月行度二卷。

傳周五星行度三十九卷。

律曆數法三卷。

自古五星宿紀三十卷。

太歲謀日晷二十九卷。

帝王諸侯世譜二十卷。

古來帝王年譜五卷。

日晷書三十四卷。

許商算術二十六卷。

杜忠算術十六卷。

右曆譜十八家,六百六卷。曆譜者,序四時之位,正分至之節,會日月五星之辰,以考寒暑殺生之實。故聖王必正曆數,以定三統服色之制,又以探知五星日月之會。凶阨之患,吉隆之喜,其術皆出焉。此聖人知命之術也,非天下之至材,其孰與焉!道之亂也,患出於小人而強欲知天道者,壞大以為小,削遠以為近,是以道術破碎而難知也。

泰一陰陽二十三卷。

黃帝陰陽二十五卷。

黃帝諸子論陰陽二十五卷。

諸王子論陰陽二十五卷。

太元陰陽二十六卷。

三典陰陽談論二十七卷。

神農大幽五行二十七卷。

四時五行經二十六卷。

猛子閭昭二十五卷。

陰陽五行時令十九卷。

堪輿金匱十四卷。

務成子災異應十四卷。

十二典災異應十二卷。

鍾律災異二十六卷。

鍾律叢辰日苑二十三卷。

鍾律消息二十九卷。

黃鍾七卷。

天一六卷。

泰一二十二九卷。

刑德七卷。

風鼓六甲二十四卷。

風后孤虛二十卷。

六合隨典二十五卷。

轉位十二神二十五卷。

羨門式法二十卷。

羨門式二十卷。

文解六甲十八卷。

文解二十八宿二十八卷。

五音奇胲用兵二十三卷。

五音奇胲刑德二十一卷。

五音定名十五卷。

右五行三十一家,六百五十二卷。

五行者,五常之形氣也。《書》云「初一曰五行,次二曰羞用五事」,言進用五事以順五行也。貌、言、視、聽、思心失,而五行之序亂,五星之變作,皆出於律曆之數而分為一者也。其法亦起五德終始,推其極則無不至。而小數家因此以為吉凶,而行於世,浸以相亂。

龜書五十二卷。

夏龜二十六卷。

南龜書二十八卷。

巨龜三十六卷。

雜龜十六卷。

蓍書二十八卷。

周易三十八卷。

周易明堂二十六卷。

周易隨曲射匿五十卷。

大筮衍易二十八卷。

大次雜易三十卷。

鼠序卜黃二十五卷。

於陵欽易吉凶二十三卷。

任良易旗七十一卷。

易卦八具。

右蓍龜十五家,四百一卷。蓍龜者,聖人之所用也。《書》曰:「女則有大疑,謀及卜筮。」《易》曰:「定天下之吉凶,成天下之亹亹者,莫善於蓍龜。」「是故君子將有為也,將有行也,問焉而以言,其受命也如嚮,無有遠近幽深,遂知來物。非天下之至精,其孰能與於此!」及至衰世,解於齊戒,而婁煩卜筮,神明不應。故筮瀆不告,易以為忌;龜厭不告,詩以為刺。

黃帝長柳占夢十一卷。

甘德長柳占夢二十卷。

武禁相衣器十四卷。

嚏耳鳴雜占十六卷。

禎祥變怪二十一卷。

人鬼精物六畜變怪二十一卷。

變怪誥咎十三卷。

執不祥劾鬼物八卷。

請官除訞祥十九卷。

禳祀天文十八卷。

請禱致福十九卷。

請雨止雨二十六卷。

泰壹雜子候歲二十二卷。

子贛雜子候歲二十六卷。

五法積貯寶臧二十三卷。

神農教田相土耕種十四卷。

昭明子釣種生魚鱉八卷。

種樹臧果相蠶十三卷。

右雜占十八家,三百一十三卷。

雜占者,紀百事之象,候善惡之徵。《易》曰:「占事知來。」眾占非一,而夢為大,故周有其官。而詩載熊羆虺蛇眾魚旐旟之夢,著明大人之占,以考吉凶,蓋參卜筮。春秋之說訞也,曰:「人之所忌,其氣炎以取之,訞由人興也。人失常則訞興,人無釁焉,訞不自作。」故曰:「德勝不祥,義厭不惠。」桑穀共生,大戊以興;鴝雉登鼎,武丁為宗。然惑者不稽諸躬,而忌訞之見,是以詩刺「召彼故老,訊之占夢」,傷其舍本而憂末,不能勝凶咎也。

山海經十三篇。

國朝七卷。

宮宅地形二十卷。

相人二十四卷。

相寶劍刀二十卷。

相六畜三十八卷。

右形法六家,百二十二卷。形法者,大舉九州之勢以立城郭室舍形,人及六畜骨法之度數、器物之形容以求其聲氣貴賤吉凶。猶律有長短,而各徵其聲,非有鬼神,數自然也。然形與氣相首尾,亦有有其形而無其氣,有其氣而無其形,此精微之獨異也。

凡數術百九十家,二千五百二十八卷。數術者,皆明堂羲和史卜之職也。史官之廢久矣,其書既不能具,雖有其書而無其人。《易》曰:「苟非其人,道不虛行。」春秋時魯有梓慎,鄭有屓灶,晉有卜偃,宋有子韋。六國時楚有甘公,魏有石申夫。漢有唐都,庶得麤觕。蓋有因而成易,無因而成難,故因舊書以序數術為六種。

黃帝內經十八卷。

外經三十九卷。

扁鵲內經九卷。

外經十二卷。

白氏內經三十八卷。

外經三十六卷。

旁篇二十五卷。

右醫經七家,二百一十六卷。醫經者,原人血脈經絡骨髓陰陽表裏,以起百病之本,死生之分,而用度箴石湯火所施,調百藥齊和之所宜。至齊之得,猶慈石取鐵,以物相使。拙者失理,以瘉為劇,以死為生。

五藏六府痺十二病方三十卷。

五藏六府疝十六病方四十卷。

五藏六府癉十二病方四十卷。

風寒熱十六病方二十六卷。

泰始黃帝扁鵲俞拊方二十三卷。

五藏傷中十一病方三十一卷。

客疾五藏狂顛病方十七卷。

金創瘲瘛方三十卷。

婦人嬰兒方十九卷。

湯液經法三十二卷。

神農黃帝食禁七卷。

右經方十一家,二百七十四卷。

經方者,本草石之寒溫,量疾病之淺深,假藥味之滋,因氣感之宜,辯五苦六辛,致水火之齊,以通閉解結,反之於平。及失其宜者,以熱益熱,以寒增寒,精氣內傷,不見於外,是所獨失也。故諺曰:「有病不治,常得中醫。」

容成陰道二十六卷。

務成子陰道三十六卷。

堯舜陰道二十三卷。

湯盤庚陰道二十卷。

天老雜子陰道二十五卷。

天一陰道二十四卷。

黃帝三王養陽方二十卷。

三家內房有子方十七卷。

右房中八家,百八十六卷。房中者,性情之極,至道之際,是以聖王制外樂以禁內情,而為之節文。傳曰:「先王之作樂,所以節百事也。」樂而有節,則和平壽考。及迷者弗顧,以生疾而隕性命。

宓戲雜子道二十篇。

上聖雜子道二十六卷。

道要雜子十八卷。

黃帝雜子步引十二卷。

黃帝岐伯按摩十卷。

黃帝雜子芝菌十八卷。

黃帝雜子十九家方二十一卷。

泰壹雜子十五家方二十二卷。

神農雜子技道二十三卷。

泰壹雜子黃冶三十一卷。

右神僊十家,二百五卷。

神僊者,所以保性命之真,而游求於其外者也。聊以盪意平心,同死生之域,而無怵惕於胸中。然而或者專以為務,則誕欺怪迂之文彌以益多,非聖王之所以教也。孔子曰:「索隱行怪,後世有述焉,吾不為之矣。」

凡方技三十六家,八百六十八卷。方技者,皆生生之具,王官之一守也。太古有岐伯、俞拊,中世有扁鵲、秦和,蓋論病以及國,原診以知政。漢興有倉公。今其技術晻味,故論其書,以序方技為四種。

大凡書,六略三十八種,五百九十六家,萬三千二百六十九卷。


\end{pinyinscope}