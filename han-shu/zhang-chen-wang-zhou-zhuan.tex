\article{張陳王周傳}

\begin{pinyinscope}
張良字子房,其先韓人也。大父開地,相韓昭侯、宣惠王、襄哀王。父平,相釐王、悼惠王。悼惠王二十三年,平卒。卒二十歲,秦滅韓。良年少,未宦事韓。韓破,良家僮三百人,弟死不葬,悉以家財求客刺秦王,為韓報仇,以五世相韓故。

良嘗學禮淮陽,東見倉海君,得力士,為鐵椎重百二十斤。秦皇帝東游,至博狼沙中,良與客狙擊秦皇帝,誤中副車。秦皇帝大怒,大索天下,求賊急甚。良乃更名姓,亡匿下邳。

良嘗閒從容步游下邳圯上,有一老父,衣褐,至良所,直墮其履圯下,顧謂良曰:「孺子下取履!」良愕然,欲歐之。為其老,乃彊忍,下取履,因跪進。父以足受之,笑去。良殊大驚。父去里所,復還,曰:「孺子可教矣。後五日平明,與我期此。」良因怪之,跪曰:「諾。」五日平明,良往。父已先在,怒曰:「與老人期,後,何也?去,後五日蚤會。」五日,雞鳴往。父又先在,復怒曰:「後,何也?去,後五日復蚤來。」五日,良夜半往。有頃,父亦來,喜曰:「當如是。」出一編書,曰:「讀是則為王者師。後十年興。十三年,孺子見我,濟北穀城山下黃石即我已。」遂去不見。旦日視其書,乃太公兵法。良因異之,常習誦。

居下邳,為任俠。項伯嘗殺人,從良匿。

後十年,陳涉等起,良亦聚少年百餘人。景駒自立為楚假王,在留。良欲往從之,行道遇沛公。沛公將數千人略地下邳,遂屬焉。沛公拜良為廄將。良數以太公兵法說沛公,沛公喜,常用其策。良為它人言,皆不省。良曰:「沛公殆天授。」故遂從不去。

沛公之薛,見項梁,共立楚懷王。良乃說項梁曰:「君已立楚後,韓諸公子橫陽君成賢,可立為王,益樹黨。」項梁使良求韓成,立為韓王。以良為韓司徒,與韓王將千餘人西略韓地,得數城,秦輒復取之,往來為游兵潁川。

沛公之從雒陽南出轘轅,良引兵從沛公,下韓十餘城,擊楊熊軍。沛公乃令韓王成留守陽翟,與良俱南,攻下宛,西入武關。沛公欲以二萬人擊秦嶢關下軍,良曰:「秦兵尚彊,未可輕。臣聞其將屠者子,賈豎易動以利。願沛公且留壁,使人先行,為五萬人具食,益張旗幟諸山上,為疑兵,令酈食其持重寶啗秦將。」秦將果欲連和俱西襲咸陽,沛公欲聽之。良曰:「此獨其將欲叛,士卒恐不從。不從必危,不如因其解擊之。」沛公乃引兵擊秦軍,大破之。逐北至藍田,再戰,秦兵竟敗。遂至咸陽,秦王子嬰降沛公。

沛公入秦,宮室帷帳狗馬重寶婦女以千數,意欲留居之。樊噲諫,沛公不聽。良曰:「夫秦為無道,故沛公得至此。為天下除殘去賊,宜縞素為資。今始入秦,即安其樂,此所謂『助桀為虐』。且『忠言逆耳利於行,毒藥苦口利於病』,願沛公聽樊噲言。」沛公乃還軍霸上。

項羽至鴻門,欲擊沛公,項伯夜馳至沛公軍,私見良,欲與俱去。良曰:「臣為韓王送沛公,今有事急,亡去不義。」乃具語沛公。沛公大驚,曰:「為之柰何?」良曰:「沛公誠欲背項王邪?」沛公曰:「鯫生說我距關毋內諸侯,秦地可王也,故聽之。」良曰:「沛公自度能卻項王乎?」沛公默然,曰:「

今為柰何?」良因要項伯見沛公。沛公與伯飲,為壽,結婚,令伯具言沛公不敢背項王,所以距關者,備它盜也。項羽後解,語在羽傳。

漢元年,沛公為漢王,王巴蜀,賜良金百溢,珠二斗,良具以獻項伯。漢王亦因令良厚遺項伯,使請漢中地。項王許之。漢王之國,良送至褒中,遣良歸韓。良因說漢王燒絕棧道,示天下無還心,以固項王意。乃使良還。行,燒絕棧道。

良歸至韓,聞項羽以良從漢王故,不遣韓王成之國,與俱東,至彭城殺之。時漢王還定三秦,良乃遺項羽書曰:「漢王失職,欲得關中,如約即止,不敢復東。」又以齊反書遺羽,曰:「齊與趙欲并滅楚。」項羽以故北擊齊。

良乃間行歸漢。漢王以良為成信侯,從東擊楚。至彭城,漢王兵敗而還。至下邑,漢王下馬踞鞍而問曰:「吾欲捐關已東等棄之,誰可與共功者?」良曰:「九江王布,楚梟將,與項王有隙,彭越與齊王田榮反梁地,此兩人可急使。而漢王之將獨韓信可屬大事,當一面。即欲捐之,捐之此三人,楚可破也。」漢王乃遣隨何說九江王布,而使人連彭越。及魏王豹反,使韓信特將北擊之,因舉燕、伐、齊、趙。然卒破楚者,此三人力也。

良多病,未嘗特將兵,常為畫策臣,時時從。

漢三年,項羽急圍漢王於滎陽,漢王憂恐,與酈食其謀橈楚權。酈生曰:「昔湯伐桀,封其後杞;武王誅紂,封其後宋。今秦無道,伐滅六國,無立錐之地。陛下誠復立六國後,此皆爭戴陛下德義,願為臣妾。德義已行,南面稱伯,楚必斂衽而朝。」漢王曰:「善。趣刻印,先生因行佩之。」

酈生未行,良從外來謁漢王。漢王方食,曰:「客有為我計橈楚權者。」具以酈生計告良曰:「於子房何如?」良曰:「誰為陛下畫此計者?陛下事去矣。」漢王曰:「何哉?」良曰:「臣請借前箸以籌之。昔湯武伐桀紂封其後者,度能制其死命也。今陛下能制項籍死命乎?其不可一矣。武王入殷,表商容閭,式箕子門,封比干墓,今陛下能乎?其不可二矣。發鉅橋之粟,散鹿臺之財,以賜貧窮,今陛下能乎?其不可三矣。殷事以畢,偃革為軒,倒載干戈,示不復用,今陛下能乎?其不可四矣。休馬華山之陽,示無所為,今陛下能乎?其不可五矣。息牛桃林之野,示天下不復輸積,今陛下能乎?其不可六矣。且夫天下游士,左親戚,棄墳墓,去故舊,從陛下者,但日夜望咫尺之地。今乃立六國後,唯無復立者,游士各歸事其主,從親戚,反故舊,陛下誰與取天下乎?其不可七矣。且楚唯毋彊,六國復橈而從之,陛下焉得而臣之?其不可八矣。誠用此謀,陛下事去矣。」漢王輟食吐哺,罵曰:「豎儒,幾敗乃公事!」令趣銷印。

後韓信破齊欲自立為齊王,漢王怒。良說漢王,漢王使良授齊王信印。語在信傳。

五年冬,漢王追楚至陽夏南,戰不利,壁固陵,諸侯期不至。良說漢王,漢王用其計,諸侯皆至。語在高紀。

漢六年,封功臣。良未嘗有戰鬥功,高帝曰:「運籌策帷幄中,決勝千里外,子房功也。自擇齊三萬戶。」良曰:「始臣起下邳,與上會留,此天以臣授陛下。陛下用臣計,幸而時中,臣願封留足矣,不敢當三萬戶。」乃封良為留侯,與蕭何等俱封。

上已封大功臣二十餘人,其餘日夜爭功而不決,未得行封。上居雒陽南宮,從復道望見諸將往往數人偶語。上曰:「此何語?」良曰:「陛下不知乎?此謀反耳。」上曰:「天下屬安定,何故而反?」良曰:「陛下起布衣,與此屬取天下,今陛下已為天子,而所封皆蕭、曹故人所親愛,而所誅者皆平生仇怨。今軍吏計功,天下不足以遍封,此屬畏陛下不能盡封,又恐見疑過失及誅,故相聚而謀反耳。」上乃憂曰:「為將柰何?」良曰:「上平生所憎,群臣所共知,誰最甚者?」上曰:「雍齒與我有故怨,數窘辱我,我欲殺之,為功多,不忍。」良曰:「今急先封雍齒,以示群臣,群臣見雍齒先封,則人人自堅矣。」於是上置酒,封雍齒為什方侯,而急趣丞相御史定功行封。群臣罷酒,皆喜曰:「雍齒且侯,我屬無患矣。」

劉敬說上都關中,上疑之。左右大臣皆山東人,多勸上都雒陽:「雒陽東有成皋,西有殽黽,背河鄉雒,其固亦足恃。」良曰:「雒陽雖有此固,其中小,不過數百里,田地薄,四面受敵,此非用武之國。夫關中左殽函,右隴蜀,沃野千里,南有巴蜀之饒,北有胡苑之利,阻三面而固守,獨以一面東制諸侯。諸侯安定,河、渭漕輓天下,西給京師;諸侯有變,順流而下,足以委輸。此所謂金城千里,天府之國。劉敬說是也。」於是上即日駕,西都關中。

良從入關,性多疾,即道引不食穀,閉門不出歲餘。

上欲廢太子,立戚夫人子趙王如意。大臣多爭,未能得堅決也。呂后恐,不知所為。或謂呂后曰:「留侯善畫計,上信用之。」呂后乃使建成侯呂澤劫良,曰:「君常為上謀臣,今上日欲易太子,君安得高枕而臥?」良曰:「始上數在急困之中,幸用臣策;今天下安定,以愛欲易太子,骨肉之間,雖臣等百人何益!」呂澤彊要曰:「為我畫計。」良曰:「此難以口舌爭也。顧上有所不能致者四人。四人年老矣,皆以上嫚厉士,故逃匿山中,義不為漢臣。然上高此四人。今公誠能毋愛金玉璧帛,令太子為書,卑辭安車,因使辨士固請,宜來。來,以為客,時從入朝,令上見之,則一助也。」於是呂后令呂澤使人奉太子書,卑辭厚禮,迎此四人。四人至,客建成侯所。

漢十一年,黥布反,上疾,欲使太子往擊之。四人相謂曰:「凡來者,將以存太子。太子將兵,事危矣。」乃說建成侯曰:「太子將兵,有功即位不益,無功則從此受禍。且太子所與俱諸將,皆與上定天下梟將也,今乃使太子將之,此無異使羊將狼,皆不肯為用,其無功必矣。臣聞『母愛者子抱』,今戚夫人日夜侍御,趙王常居前,上『終不使不肖子居愛子上』,明代太子位必矣。君何不急請呂后承間為上泣言:『黥布,天下猛將,善用兵,今諸將皆陛下故等夷,乃令太子將,此屬莫肯為用,且布聞之,鼓行而西耳。上雖疾,彊載輜車,臥而護之,諸將不敢不盡力。上雖苦,彊為妻子計。』」於是呂澤夜見呂后。呂后承間為上泣而言,如四人意。上曰:「吾惟之,豎子固不足遣,乃公自行耳。」於是上自將而東,群臣居守,皆送至霸上。良疾,強起至曲郵,見上曰:「臣宜從,疾甚。楚人剽疾,願上慎毋與楚爭鋒。」因說上令太子為將軍監關中兵。上謂「子房雖疾,彊臥傅太子」。是時叔孫通已為太傅,良行少傅事。

漢十二年,上從破布歸,疾益甚,愈欲易太子。良諫不聽,因疾不視事。叔孫太傅稱說引古,以死爭太子。上陽許之,猶欲易之。及宴,置酒,太子侍。四人者從太子,年皆八十有餘,須眉皓白,衣冠甚偉。上怪,問曰:「何為者?」四人前對,各言其姓名。上乃驚曰:「吾求公,避逃我,今公何自從吾兒游乎?」四人曰:「陛下輕士善罵,臣等義不辱,故恐而亡匿。今聞太子仁孝,恭敬愛士,天下莫不延頸願為太子死者,故臣等來。」上曰:「煩公幸卒調護太子。」

四人為壽已畢,趨去。上目送之,召戚夫人指視曰:「我欲易之,彼四人為之輔,羽翼已成,難動矣。呂氏真乃主矣。」戚夫人泣涕,上曰:「為我楚舞,吾為若楚歌。」歌曰:「鴻鵠高飛,一舉千里。羽翼以就,橫絕四海。橫絕四海,又可奈何!雖有矰繳,尚安所施!」歌數闋,戚夫人歔欷流涕,上起去,罷酒。竟不易太子者,良本招此四人之力也。

良從上擊代,出奇計下馬邑,及立蕭相國,所與從容言天下事甚眾,非天下所以存亡,故不著。良乃稱曰:「家世相韓,及韓滅,不愛萬金之資,為韓報仇彊秦,天下震動。今以三寸舌為帝者師,封萬戶,位列侯,此布衣之極,於良足矣。願棄人間事,欲從赤松子游耳。」乃學道,欲輕舉。高帝崩,呂后德良,乃彊食之,曰:「人生一世,如白駒之過隙,何自苦如此!」良不得已,彊聽食。後六歲薨。諡曰文成侯。

良始所見下邳圯上老父與書者,後十三歲從高帝過濟北,果得穀城山下黃石,取而寶祠之。及良死,并葬黃石。每上冢伏臘祠黃石。

子不疑嗣侯。孝文三年坐不敬,國除。

陳平,陽武戶牖鄉人也。少時家貧,好讀書,治黃帝、老子之術。有田三十畝,與兄伯居。伯常耕田,縱平使游學。平為人長大美色,人或謂平:「貧何食而肥若是?」其嫂疾平之不親家生產,曰:「亦食糠覈耳。有叔如此,不如無有!」伯聞之,逐其婦棄之。

及平長,可取婦,富人莫與者,貧者平亦媿之。久之,戶牖富人張負有女孫,五嫁夫輒死,人莫敢取,平欲得之。邑中有大喪,平家貧侍喪,以先往後罷為助。張負既見之喪所,獨視偉平,平亦以故後去。負隨平至其家,家乃負郭窮巷,以席為門,然門外多長者車轍。張負歸,謂其子仲曰:「吾欲以女孫予陳平。」仲曰:「平貧不事事,一縣中盡笑其所為,獨柰何予之女?」負曰:「固有美如陳平長貧者乎?」卒與女。為平貧,乃假貸幣以聘,予酒肉之資以內婦。負戒其孫曰:「毋以貧故,事人不謹。事兄伯如事乃父,事嫂如事乃母。」平既取張氏女,資用益饒,游道日廣。

里中社,平為宰,分肉甚均。里父老曰:「善,陳孺子之為宰!」平曰:「嗟乎,使平得宰天下,亦如此肉矣!」

陳涉起王,使周市略地,立魏咎為魏王,與秦軍相攻於臨濟。平已前謝兄伯,從少年往事魏王咎,為太僕。說魏王,王不聽。人或讒之,平亡去。

項羽略地至河上,平往歸之,從入破秦,賜爵卿。項羽之東王彭城也,漢王還定三秦而東。殷王反楚,項羽乃以平為信武君,將魏王客在楚者往擊,殷降而還。項王使項悍拜平為都尉,賜金二十溢。居無何,漢攻下殷。項王怒,將誅定殷者。平懼誅,乃封其金與印,使使歸項王,而平身間行杖劍亡。度河,船人見其美丈夫,獨行,疑其亡將,要下當有寶器金玉,目之,欲殺平。平心恐,乃解衣臝而佐刺船。船人知其無有,乃止。

平遂至脩武降漢,因魏無知求見漢王,漢王召入。是時,萬石君石奮為中涓,受平謁。平等十人俱進,賜食。王曰:「罷,就舍矣。」平曰:「臣為事來,所言不可以過今日。」於是漢王與語而說之,問曰:「子居楚何官?」平曰:「為都尉。」是日拜平為都尉,使參乘,典護軍。諸將盡讙,曰:「大王一日得楚之亡卒,未知高下,而即與共載,使監護長者!」漢王聞之,愈益幸平,遂與東伐項王。至彭城,為楚所敗,引師而還。收散兵至滎陽,以平為亞將,屬韓王信,軍廣武。

絳、灌等或讒平曰:「平雖美丈夫,如冠玉耳,其中未必有也。聞平居家時盜其嫂;事魏王不容,亡而歸楚;歸楚不中,又亡歸漢。今大王尊官之,令護軍。臣聞平使諸將,金多者得善處,金少者得惡處。平,反覆亂臣也,願王察之。」漢王疑之,以讓無知,問曰:「有之乎?」無知曰:「有。」漢王曰:「

公言其賢人何也?」對曰:「臣之所言者,能也;陛下所問者,行也。今有尾生、孝已之行,而無益於勝敗之數,陛下何暇用之乎?今楚漢相距,臣進奇謀之士,顧其計誠足以利國家耳。盜嫂受金又安足疑乎?」漢王召平而問曰:「吾聞先生事魏不遂,事楚而去,今又從吾游,信者固多心乎?」平曰:「臣事魏王,魏王不能用臣說,故去事項王。項王不信人,其所任愛,非諸項即妻之昆弟,雖有奇士不能用。臣居楚聞漢王之能用人,故歸大王。臝身來,不受金無以為資。誠臣計畫有可采者,願大王用之;使無可用者,大王所賜金具在,請封輸官,得請骸骨。」漢王乃謝,厚賜,拜以為護軍中尉,盡護諸將。諸將乃不敢復言。

其後,楚急擊,絕漢甬道,圍漢王於滎陽城。漢王患之,請割滎陽以西和。項王弗聽。漢王謂平曰:「天下紛紛,何時定乎?」平曰:「項王為人,恭敬愛人,士之廉節好禮者多歸之。至於行功賞爵邑,重之,士亦以此不附。今大王嫚而少禮,士之廉節者不來;然大王能饒人以爵邑,士之頑頓耆利無恥者亦多歸漢。誠各去兩短,集兩長,天下指麾即定矣。然大王資侮人,不能得廉節之士。顧楚有可亂者,彼項王骨鯁之臣亞父、鍾離眛、龍且、周殷之屬,不過數人耳。大王能出捐數萬斤金,行反間,間其君臣,以疑其心,項王為人意忌信讒,必內相誅。漢因舉兵而攻之,破楚必矣。」漢王以為然,乃出黃金四萬斤予平,恣所為,不問出入。

平既多以金縱反間於楚軍,宣言諸將鍾離眛等為項王將,功多矣,然終不得列地而王,欲與漢為一,以滅項氏,分王其地。項王果疑之,使使至漢。漢為太牢之具,舉進,見楚使,即陽驚曰:「

以為亞父使,乃項王使也!」復持去,以惡草具進楚使。使歸,具以報項王,果大疑亞父。亞父欲急擊下滎陽城,項王不信,不肯聽亞父。亞父聞項王疑之,乃大怒曰:「天下事大定矣,君王自為之!願乞骸骨歸!」歸未至彭城,疽發背而死。

平乃夜出女子二千人滎陽東門,楚因擊之。平乃與漢王從城西門出去。遂入關,收聚兵而復東。

明年,淮陰侯信破齊,自立為假齊王,使使言之漢王。漢王怒而罵,平躡漢王。漢王寤,乃厚遇齊使,使張良往立信為齊王。於是封平以戶牖鄉。用其計策,卒滅楚。

漢六年,人有上書告楚王韓信反。高帝問諸將,諸將曰:「亟發兵阬豎子耳。」高帝默然。以問平,平固辭謝,曰:「諸將云何?」上具告之。平曰:「人之上書言信反,人有聞知者乎?」曰:「未有。」曰:「信知之乎?」曰:「弗知。」平曰:「陛下兵精孰與楚?」上曰:「不能過也。」平曰:「陛下將用兵有能敵韓信者乎?」上曰:「莫及也。」平曰:「今兵不如楚精,將弗及,而舉兵擊之,是趣之戰也,竊為陛下危之。」上曰:「為之柰何?」平曰:「古者天子巡狩,會諸侯。南方有雲夢,陛下弟出偽游雲夢,會諸侯於陳。陳,楚之西界,信聞天子以好出游,其勢必郊迎謁。而陛下因禽之,特一力士之事耳。」高帝以為然,乃發使告諸侯會陳,「吾將南游雲夢」。上因隨以行。行至陳,楚王信果郊迎道中。高帝豫具武士,見信,即執縛之。語在信傳。

遂會諸侯於陳。還至雒陽,與功臣剖符定封,封平為戶牖侯,世世勿絕。平辭曰:「此非臣之功也。」上曰:「吾用先生計謀,戰勝克敵,非功而何?」平曰:「非魏無知臣安得進?」上曰:「若子可謂不背本矣!」乃復賞魏無知。

其明年,平從擊韓王信於代。至平城,為匈奴圍,七日不得食。高帝用平奇計,使單于閼氏解,圍以得開。高帝既出,其計祕,世莫得聞。高帝南過曲逆,上其城,望室屋甚大,曰:「壯哉縣!吾行天下,獨見雒陽與是耳。」顧問御史:「曲逆戶口幾何?」對曰:「始秦時三萬餘戶,間者兵數起,多亡匿,今見五千餘戶。」於是召御史,更封平為曲逆侯,盡食之,除前所食戶牖。

平自初從,至天下定後,常以護軍中尉從擊臧荼、陳豨、黥布。凡六出奇計,輒益邑封。奇計或頗祕,世莫得聞也。

高帝從擊布軍還,病創,徐行至長安。燕王盧綰反,上使樊噲以相國將兵擊之。既行,人有短惡噲者。高帝怒曰:「噲見吾病,乃幾我死也!」用平計,召絳侯周勃受詔床下,曰:「平乘馳傳載勃代噲將,平至軍中即斬噲頭!」二人既受詔,馳傳未至軍,行計曰:「樊噲,帝之故人,功多,又呂后女弟呂須夫,有親且貴,帝以忿怒故欲斬之,即恐後悔。寧囚而致上,令上自誅之。」未至軍,為壇,以節召樊噲。噲受詔,即反接,載檻車詣長安,而令周勃代將兵定燕。

平行聞高帝崩,平恐呂后及呂須怒,乃馳傳先去。逢使者詔平與灌嬰屯於滎陽。平受詔,立復馳至宮,哭殊悲,因奏事喪前。呂后哀之,曰:「君出休矣!」平畏讒之就,因固請之,得宿衛中。太后乃以為郎中令,日傅教帝。是後呂須讒乃不得行。樊噲至,即赦復爵邑。

惠帝六年,相國曹參薨,安國侯王陵為右丞相,平為左丞相。

王陵,沛人也。始為縣豪,高祖微時兄事陵。及高祖起沛,入咸陽,陵亦聚黨數千人,居南陽,不肯從沛公。及漢王之還擊項籍,陵乃以兵屬漢。項羽取陵母置軍中,陵使至,則東鄉坐陵母,欲以招陵。陵母既私送使者,泣曰:「願為老妾語陵,善事漢王。漢王長者,毋以老妾故持二心。妾以死送使者。」遂伏劍而死。項王怒,亨陵母。陵卒從漢王定天下。以善雍齒,雍齒,高祖之仇,陵又本無從漢之意,以故後封陵,為安國侯。

陵為人少文任氣,好直言。為右丞相二歲,惠帝崩。高后欲立諸呂為王,問陵。陵曰:「高皇帝刑白馬而盟曰:『非劉氏而王者,天下共擊之。』今王呂氏,非約也。」太后不說。問丞相平及絳侯周勃等,皆曰:「高帝定天下,王子弟;今太后稱制,欲王昆弟諸呂,無所不可。」太后喜。罷朝,陵讓平、勃曰:「始與高帝唼血而盟,諸君不在邪?今高帝崩,太后女主,欲王呂氏,諸君縱欲阿意背約,何面目見高帝於地下乎!」平曰:「於面折廷爭,臣不如君;全社稷,定劉氏後,君亦不如臣。」陵無以應之。於是呂太后欲廢陵,乃陽遷陵為帝太傅,實奪之相權。陵怒,謝病免,杜門竟不朝請,十年而薨。

陵之免,呂太后徙平為右丞相,以辟陽侯審食其為左丞相。食其亦沛人也。漢王之敗彭城西,楚取太上皇、呂后為質,食其以舍人侍呂后。其後從破項籍為侯,幸於呂太后。及為相,不治,監宮中,如郎中令,公卿百官皆因決事。

呂須常以平前為高帝謀執樊噲,數讒平曰:「為丞相不治事,日飲醇酒,戲婦人。」平聞,日益甚。呂太后聞之,私喜。面質呂須於平前,曰:「鄙語曰『兒婦人口不可用』,顧君與我何如耳,無畏呂須之譖。」

呂太后多立諸呂為王,平偽聽之。及呂太后崩,平與太尉勃合謀,卒誅諸呂,立文帝,平本謀也。審食其免相,文帝立,舉以為相。

太尉勃親以兵誅呂氏,功多;平欲讓勃位,乃謝病。文帝初立,怪平病,問之。平曰:「高帝時,勃功不如臣;及誅諸呂,臣功亦不如勃。願以相讓勃。」於是乃以太尉勃為右丞相,位第一;平徙為左丞相,位第二。賜平金千斤,益封三千戶。

居頃之,上益明習國家事,朝而問右丞相勃曰:「天下一歲決獄幾何?」勃謝不知。問「天下錢穀一歲出入幾何?」勃又謝不知。汗出洽背,媿不能對。上亦問左丞相平。平曰:「各有主者。」上曰:「主者為誰乎?」平曰:「陛下即問決獄,責廷尉;問錢穀,責治粟內史。」上曰:「苟各有主者,而君所主何事也?」平謝曰:「主臣!陛下不知其駑下,使待罪宰相。宰相者,上佐天子理陰陽,順四時,下遂萬物之宜,外填撫四夷諸侯,內親附百姓,使卿大夫各得任其職也。」上稱善。勃大慚,出而讓平曰:「君獨不素教我乎!」平笑曰:「君居其位,獨不知其任邪?且陛下即問長安盜賊數,又欲彊對邪?」於是絳侯自知其能弗如平遠矣。居頃之,勃謝病請免相,而平顓為丞相。

孝文二年,平薨,諡曰獻侯。傳子至曾孫何,坐略人妻棄主。王陵亦至玄孫,坐酎金國除。辟陽侯食其免後三歲而為淮南王所殺,文帝令其子平嗣侯。淄川王反,辟陽近淄川,平降之,國除。

始平曰:「我多陰謀,道家之所禁。吾世即廢,亦已矣,終不能復起,以吾多陰禍也。」其後曾孫陳掌以衛氏親戚貴,願得續封,然終不得也。

周勃,沛人。其先卷人也,徙沛。勃以織薄曲為生,常以吹簫給喪事,材官引強。

高祖為沛公初起,勃以中涓從攻胡陵,下方與。方與反,與戰,卻敵。攻豐。擊秦軍碭東。還軍留及蕭。復攻碭,破之。下下邑,先登。賜爵五大夫。攻蘭、虞,取之。擊章邯車騎殿。略定魏地。攻轅戚、東嬢,以往至栗,取之。攻齧桑,先登。擊秦軍阿下,破之。追至濮陽,下蘄城。攻都關、定陶,襲取宛朐,得單父令。夜襲取臨濟,攻壽張,以前至卷,破李由雍丘下。攻開封,先至城下為多。後章邯破項梁,沛公與項羽引兵東如碭。自初起沛還至碭,一歲二月。楚懷王封沛公號武安侯,為碭郡長。沛公拜勃為襄賁令。從沛公定魏地,攻東郡尉於成武,破之。攻長社,先登。攻潁陽、緱氏,絕河津。擊趙賁軍尸北。南攻南陽守齮,破武關、嶢關。攻秦軍於藍田。至咸陽,滅秦。

項羽至,以沛公為漢王。漢王賜勃爵為威武侯。從入漢中,拜為將軍。還定三秦,賜食邑懷德。攻槐里、好畤,最。北擊趙賁、內史保於咸陽,最。北救漆。擊章平、姚卬軍。西定汧還下郿、頻陽。圍章邯廢丘,破之。西擊益已軍,破之。攻上邽。東守嶢關。擊項籍。攻曲遇,最。還守敖倉,追籍。籍已死,因東定楚地泗水、東海郡,凡得二十二縣。還守雒陽、櫟陽,賜與潁陰侯共食鍾離。以將軍從高祖擊燕王臧荼,破之易下。所將卒當馳道為多。賜爵列侯,剖符世世不絕。食絳八千二百八十戶。

以將軍從高帝擊韓王信於代,降下霍人。以前至武泉,擊胡騎,破之武泉北。轉攻韓信軍銅鞮,破之。還,降太原六城。擊韓信胡騎晉陽下,破之,下晉陽。後擊韓信軍於硰石,破之,追北八十里。還攻樓煩三城,因擊胡騎平城下,所將卒當馳道為多。勃遷為太尉。

陳豨,屠馬邑。所將卒斬豨將軍乘馬降。轉擊韓信、陳豨、趙利軍於樓煩,破之。得豨將宋最、鴈門守圂。因轉攻得雲中守遫、丞相箕肄、將軍博。定鴈門郡十七縣,雲中郡十二縣。因復擊豨靈丘,破之,斬豨丞相程縱、將軍陳武、都尉高肄。定代郡九縣。

燕王盧綰反,勃以相國代樊噲將,擊下薊,得綰大將抵、丞相偃、守陘、太尉弱、御史大夫施屠渾都。破綰軍上蘭,後擊綰軍沮陽。追至長城,定上谷十二縣,右北平十六縣,遼東二十九縣,漁陽二十二縣。最從高帝得相國一人,丞相二人,將軍、二千石各三人;別破軍二,下城三,定郡五,縣七十九,得丞相、大將各一人。

勃為人木強敦厚,高帝以為可屬大事。勃不好文學,每召諸生說士,東鄉坐責之:「趣為我語。」其椎少文如此。

勃既定燕而歸,高帝已崩矣,以列侯事惠帝。惠帝六年,置太尉官,以勃為太尉。十年,高后崩。呂祿以趙王為漢上將軍,呂產以呂王為相國,秉權,欲危劉氏。勃與丞相平、朱虛侯章共誅諸呂。語在高后紀。

於是陰謀乃為「少帝及濟川、淮陽、恆山王皆非惠帝子,呂太后以計詐名它人子,殺其母,養之後宮,令孝惠子之,立以為後,用彊呂氏。今已滅諸呂,少帝即長用事,吾屬類無矣,不如視諸侯賢者立之。」遂迎立代王,是為孝文皇帝。

東牟侯興居,朱虛侯章弟也,曰:「誅諸呂,臣無功,請得除宮。」乃與太僕汝陰滕公入宮。滕公前謂少帝曰:「足下非劉氏,不當立。」乃顧麾左右執戟,皆仆兵罷。有數人不肯去,官者令張釋諭告,亦去。滕公召乘輿車載少帝出。少帝曰:「欲持我安之乎?」滕公曰:「就舍少府。」乃奉天子法駕,迎皇帝代邸,報曰:「宮謹除。」皇帝入未央宮,有謁者十人持戟衛端門,曰:「天子在也,足下何為者?」不得入。太尉往喻,乃引兵去,皇帝遂入。是夜,有司分部誅濟川、淮陽、常山王及少帝於邸。

文帝即位,以勃為右丞相,賜金五千斤,邑萬戶。居十餘月,人或說勃曰:「君既誅諸呂,立代王,威震天下,而君受厚賞處尊位以厭之,則禍及身矣。」勃懼,亦自危,乃謝請歸相印。上許之。歲餘,陳丞相平卒,上復用勃為丞相。十餘月,上曰:「前日吾詔列侯就國,或頗未能行,丞相朕所重,其為朕率列侯之國。」乃免相就國。

歲餘,每河東守尉行縣至絳,絳侯勃自畏恐誅,常被甲,令家人持兵以見。其後人有上書告勃欲反,下廷尉,逮捕勃治之。勃恐,不知置辭。吏稍侵辱之。勃以千金與獄吏,獄吏乃書牘背示之,曰「以公主為證」。公主者,孝文帝女也,勃太子勝之尚之,故獄吏教引為證。初,勃之益封,盡以予薄昭。及繫急,薄昭為言薄太后,太后亦以為無反事。文帝朝,太后以冒絮提文帝,曰:「絳侯綰皇帝璽,將兵於北軍,不以此時反,今居一小縣,顧欲反邪!」文帝既見勃獄辭,乃謝曰:「吏方驗而出之。」於是使使持節赦勃,復爵邑。勃既出,曰:「吾嘗將百萬軍,安知獄吏之貴也!」

勃復就國,孝文十一年薨,諡曰武侯。子勝之嗣,尚公主不相中,坐殺人,死,國絕。一年,弟亞夫復為侯。

亞夫為河內守時,許負相之:「君後三歲而侯。侯八歲,為將相,持國秉,貴重矣,於人臣無二。後九年而餓死。」亞夫笑曰:「臣之兄以代父侯矣,有如卒,子當代,我何說侯乎?然既已貴如負言,又何說餓死?指視我。」負指其口曰:「從理入口,此餓死法也。」居三歲,兄絳侯勝之有罪,文帝擇勃子賢者,皆推亞夫,乃封為條侯。

文帝後六年,匈奴大入邊。以宗正劉禮為將軍軍霸上,祝茲侯徐厲為將軍軍棘門,以河內守亞夫為將軍軍細柳,以備胡。上自勞軍,至霸上及棘門軍,直馳入,將以下騎出入送迎。已而之細柳軍,軍士吏被甲,銳兵刃,彀弓弩,持滿。天子先驅至,不得入。先驅曰:「天子且至!」軍門都尉曰:「軍中聞將軍之令,不聞天子之詔。」有頃,上至,又不得入。於是上使使持節詔將軍曰:「吾欲勞軍。」亞夫乃傳言開壁門。壁門士請車騎曰:「將軍約,軍中不得驅馳。」於是天子乃按轡徐行。至中營,將軍亞夫揖,曰:「介冑之士不拜,請以軍禮見。」天子為動,改容式車。使人稱謝:「皇帝敬勞將軍。」成禮而去。既出軍門,群臣皆驚。文帝曰:「嗟乎,此真將軍矣!鄉者霸上、棘門如兒戲耳,其將固可襲而虜也。至於亞夫,可得而犯邪!」稱善者久之。月餘,三軍皆罷。乃拜亞夫為中尉。

文帝且崩時,戒太子曰:「即有緩急,周亞夫真可任將兵。」文帝崩,亞夫為車騎將軍。

孝景帝三年,吳楚反。亞夫以中尉為太尉,東擊吳楚。因自請上曰:「楚兵剽輕,難與爭鋒。願以梁委之,絕其食道,乃可制也。」上許之。

亞夫既發,至霸上,趙涉遮說亞夫曰:「將軍東誅吳楚,勝則宗廟安,不勝則天下危,能用臣之言乎?」亞夫下車,禮而問之。涉曰:「吳王素富,懷輯死士久矣。此知將軍且行,必置間人於殽黽阨骥之間。且兵事上神密,將軍何不從此右去,走藍田,出武關,抵雒陽,間不過差一二日,直入武庫,擊鳴鼓。諸侯聞之,以為將軍從天而下也。」太尉如其計。至雒陽,使吏搜殽黽間,果得吳伏兵。乃請涉為護軍。

亞夫至,會兵滎陽。吳方攻梁,梁急,請救。亞夫引兵東北走昌邑,深壁而守。梁王使使請亞夫,亞夫守便宜,不往。梁上書言景帝,景帝詔使救梁。亞夫不奉詔,堅壁不出,而使輕騎兵弓高侯等絕吳楚兵後食道。吳楚兵乏糧,飢,欲退,數挑戰,終不出。夜,軍中驚,內相攻擊擾亂,至於帳下。亞夫堅臥不起。頃之,復定。吳奔壁東南陬,亞夫使備西北。已而其精兵果奔西北,不得入。吳楚既餓,乃引而去。亞夫出精兵追擊,大破吳王濞。吳王濞棄其軍,與壯士數千人亡走,保於江南丹徒。漢兵因乘勝,遂盡虜之,降其縣,購吳王千金。月餘,越人斬吳王頭以告。凡相守攻三月,而吳楚破平。於是諸將乃以太尉計謀為是。由此梁孝王與亞夫有隙。

歸,復置太尉官。五歲,遷為丞相,景帝甚重之。上廢栗太子,亞夫固爭之,不待。上由此疏之。而梁孝王每朝,常與太后言亞夫之短。

竇太后曰:「皇后兄王信可侯也。」上讓曰:「始南皮及章武先帝不侯,及臣即位,乃侯之,信未得封也。」竇太后曰:「人生各以時行耳。竇長君在時,竟不得封侯,死後,乃其子彭祖顧得侯。吾甚恨之。帝趣侯信也!」上曰:「請得與丞相計之。」亞夫曰:「高帝約『非劉氏不得王,非有功不得侯。不如約,天下共擊之』。今信雖皇后兄,無功,侯之,非約也。」上默然而沮。

其後匈奴王徐盧等五人降漢,上欲侯之以勸後。亞夫曰:「彼背其主降陛下,陛下侯之,即何以責人臣不守節者乎?」上曰:「丞相議不可用。」乃悉封徐盧等為列侯。亞夫因謝病免相。

頃之,上居禁中,召亞夫賜食。獨置大胾,無切肉,又不置箸。亞夫心不平,顧謂尚席取箸。上視而笑曰:「此非不足君所乎?」亞夫免冠謝上。上曰:「起。」亞夫因趨出。上目送之,曰:「此鞅鞅,非少主臣也!」

居無何,亞夫子為父買工官尚方甲楯五百被可以葬者。取庸苦之,不與錢。庸知其盜買縣官器,怨而上變告子,事連汙亞夫。書既聞,上下吏。吏簿責亞夫,亞夫不對。上罵之曰:「吾不用也。」召詣廷尉。廷尉責問曰:「君侯欲反何?」亞夫曰:「臣所買器,乃葬器也,何謂反乎?」吏曰:「君縱不欲反地上,即欲反地下耳。」吏侵之益急。初,吏捕亞夫,亞夫欲自殺,其夫人止之,以故不得死,遂入廷尉,因不食五日,歐血而死。國絕。

一歲,上乃更封絳侯勃它子堅為平曲侯,續絳侯後。傳子建德,為太子太傅,坐酎金免官。後有罪,國除。

亞夫果餓死。死後,上乃封王信為蓋侯。至平帝元始二年,繼絕世,復封勃玄孫之子恭為絳侯,千戶。

贊曰:聞張良之智勇,以為其貌魁梧奇偉,反若婦人女子。故孔子稱「以貌取人,失之子羽」。學者多疑於鬼神,如良受書老父,亦異矣。高祖數離困阨,良常有力,豈可謂非天乎!陳平之志,見於社下,傾側擾攘楚、魏之間,卒歸於漢,而為謀臣。及呂后時,事多故矣,平竟自免,以智終。王陵廷爭,杜門自絕,亦各其志也。周勃為布衣時,鄙樸庸人,至登輔佐,匡國家難,誅諸呂,立孝文,為漢伊周,何其盛也!始呂后問宰相,高祖曰:「陳平智有餘,王陵少戇,可以佐之;安劉氏者必勃也。」又問其次,云「過此以後,非乃所及」。終皆如言,聖矣夫!


\end{pinyinscope}