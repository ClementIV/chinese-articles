\article{竇田灌韓傳}

\begin{pinyinscope}
竇嬰字王孫,孝文皇后從兄子也。父世觀津人也。喜賓客。孝文時為吳相,病免。孝景即位,為詹事。

帝弟梁孝王,母竇太后愛之。孝王朝,因燕昆弟飲。是時上未立太子,酒酣,上從容曰:「千秋萬歲後傳王。」太后驩。嬰引卮酒進上曰:「天下者,高祖天下,父子相傳,漢之約也,上何以得傳梁王!」太后由此憎嬰。嬰亦薄其官,因病免。太后除嬰門籍,不得朝請。

孝景三年,吳楚反,上察宗室諸竇無如嬰賢,召入見,固讓謝,稱病不足任。太后亦慚。於是上曰:「天下方有急,王孫寧可以讓邪?」乃拜嬰為大將軍,賜金千斤。嬰言爰盎、欒布諸名將賢士在家者進之。所賜金,陳廊廡下,軍吏過,輒令財取為用,金無入家者。嬰守滎陽,監齊趙兵。七國破,封為魏其侯。游士賓客爭歸之。每朝議大事,條侯、魏其,列侯莫敢與亢禮。

四年,立栗太子,以嬰為傅。七年,栗太子廢,嬰爭弗能得,謝病,屏居藍田南山下數月,諸竇賓客辯士說,莫能來。梁人高遂乃說嬰曰:「能富貴將軍者,上也;能親將軍者,太后也。今將軍傅太子,太子廢,爭不能拔,又不能死,自引謝病,擁趙女屏閒處而不朝,袛加懟自明,揚主之過。有如兩宮奭將軍,則妻子無類矣。」嬰然之,乃起,朝請如故。

桃侯免相,竇太后數言魏其。景帝曰:「太后豈以臣有愛相魏其者?魏其沾沾自喜耳,多易,難以為相持重。」遂不用,用建陵侯衛綰為丞相。

田蚡,孝景王皇后同母弟也,生長陵。竇嬰已為大將軍,方盛,蚡為諸曹郎,未貴,往來侍酒嬰所,跪起如子姓。及孝景晚節,蚡益貴幸,為中大夫。辯有口,學盤盂諸書,王皇后賢之。

孝景崩,武帝初即位,蚡以舅封為武安侯,弟勝為周陽侯。

蚡新用事,卑下賓客,進名士家居者貴之,欲以傾諸將相。上所填撫,多蚡賓客計策。會丞相綰病免,上議置丞相、太尉。藉福說蚡曰:「魏其侯貴久矣,素天下士歸之。今將軍初興,未如,即上以將軍為相,必讓魏其。魏其為相,將軍必為太尉。太尉、相尊等耳,有讓賢名。」蚡乃微言太后風上,於是乃以嬰為丞相,蚡為太尉。藉福賀嬰,因弔曰:「君侯資性喜善疾惡,方今善人譽君侯,故至丞相;然惡人眾,亦且毀君侯。君侯能兼容,則幸久;不能,今以毀去矣。」嬰不聽。

嬰、蚡俱好儒術,推轂趙綰為御史大夫,王臧為郎中令。迎魯申公,欲設明堂,令列侯就國,除關,以禮為服制,以興太平。舉謫諸竇宗室無行者,除其屬藉。諸外家為列侯,列侯多尚公主,皆不欲就國,以故毀日至竇太后。太后好黃老言,而嬰、蚡、趙綰等務隆推儒術,貶道家言,是以竇太后滋不說。二年,御史大夫趙綰請毋奏事東宮。竇太后大怒,曰:「此欲復為新垣平邪!」乃罷逐趙綰、王臧,而免丞相嬰、太尉蚡,以柏至侯許昌為丞相,武彊侯莊青翟為御史大夫。嬰、蚡以侯家居。

蚡雖不任職,以王太后故親幸,數言事,多效,士吏趨勢利者皆去嬰而歸蚡。蚡日益橫。六年,竇太后崩,丞相昌、御史大夫青翟坐喪事不辦,免。上以蚡為丞相,大司農韓安國為御史大夫。天下士郡諸侯愈益附蚡。

蚡為人貌侵,生貴甚。又以為諸侯王多長,上初即位,富於春秋,蚡以肺附為相,非痛折節以禮屈之,天下不肅。當是時,丞相入奏事,語移日,所言皆聽。薦人或起家至二千石,權移主上。上乃曰:「君除吏盡未?吾亦欲除吏。」嘗請考工地益宅,上怒曰:「遂取武庫!」是後乃退。召客飲,坐其兄蓋侯北鄉,自坐東鄉,以為漢相尊,不可以兄故私橈。由此滋驕,治宅甲諸第,田園極膏腴,市買郡縣器物相屬於道。前堂羅鐘鼓,立曲旃;後房婦女以百數。諸奏珍物狗馬玩好,不可勝數。

而嬰失竇太后,益疏不用,無勢,諸公稍自引而怠怀,唯灌夫獨否。故嬰墨墨不得意,而厚遇夫也。

灌夫字仲孺,潁陰人也。父張孟,常為潁陰侯灌嬰舍人,得幸,因進之,至二千石,故蒙灌氏姓為灌孟。吳楚反時,潁陰侯灌嬰為將軍,屬太尉,請孟為校尉。夫以千人與父俱。孟年老,潁陰侯彊請之,鬱鬱不得意,故戰常陷堅,遂死吳軍中。漢法,父子俱,有死事,得與喪歸。夫不肯隨喪歸,奮曰:「願取吳王若將軍頭以報父仇。」於是夫被甲持戟,募軍中壯士所善願從數十人。及出壁門,莫敢前。獨兩人及從奴十餘騎馳入吳軍,至戲下,所殺傷數十人。不得前,復還走漢壁,亡其奴,獨與一騎歸。夫身中大創十餘,適有萬金良藥,故得無死。創少瘳,又復請將軍曰:「吾益知吳壁曲折,請復往。」將軍壯而義之,恐亡夫,乃言太尉,太尉召固止之。吳軍敗,夫以此名聞天下。

潁陰侯言夫,夫為郎中將。數歲,坐法去。家居長安中,諸公莫不稱,由是復為代相。

武帝即位,以為淮陽天下郊,勁兵處,故徙夫為淮陽太守。入為太僕。二年,夫與長樂衛尉竇甫飲,輕重不得,夫醉,搏甫。甫,竇太后昆弟。上恐太后誅夫,徙夫為燕相。數歲,坐法免,家居長安。

夫為人剛直,使酒,不好面諛。貴戚諸勢在己之右,欲必陵之;士在己左,愈貧賤,尤益禮敬,與鈞。稠人廣眾,薦寵下輩。士亦以此多之。

夫不好文學,喜任俠,已然諾。諸所與交通,無非豪桀大猾。家累數千萬,食客日數十百人。波池田園,宗族賓客為權利,橫潁川。潁川兒歌之曰:「潁水清,灌氏寧;潁水濁,灌氏族。」

夫家居,卿相侍中賓客益衰。及竇嬰失勢,亦欲倚夫引繩排根生平慕之後棄者。夫亦得嬰通列侯宗室為名高。兩人相為引重,其游如父子然,相得驩甚,無厭,恨相知之晚。

夫嘗有服,過丞相蚡。蚡從容曰:「吾欲與仲孺過魏其侯,會仲孺有服。」夫曰:「將軍乃肯幸臨況魏其侯,夫安敢以服為解!請語魏其具,將軍旦日蚤臨。」蚡許諾。夫以語嬰。嬰與夫人益巿牛酒,夜洒埽張具至旦。平明,令門下候司。至日中,蚡不來。嬰謂夫曰:「丞相豈忘之哉?」夫不懌,曰:「夫以服請,不宜。」乃駕,自往迎蚡。蚡特前戲許夫,殊無意往。夫至門,蚡尚臥也。於是夫見,曰:「將軍昨日幸許過魏其,魏其夫妻治具,至今未敢嘗食。」蚡悟,謝曰:「吾醉,忘與仲孺言。」乃駕往。往又徐行,夫愈益怒。及飲酒酣,夫起舞屬蚡,蚡不起。夫徙坐,語侵之。嬰乃扶夫去,謝蚡。蚡卒飲至夜,極驩而去。

後蚡使藉福請嬰城南田,嬰大望曰:「老僕雖棄,將軍雖貴,寧可以勢相奪乎!」不許。夫聞,怒罵福。福惡兩人有隙,乃謾好謝蚡曰:「魏其老且死,易忍,且待之。」已而蚡聞嬰、夫實怒不予,亦怒曰:「魏其子嘗殺人,蚡活之。蚡事魏其無所不可,愛數頃田?且灌夫何與也?吾不敢復求田。」由此大怒。

元光四年春,蚡言灌夫家在潁川,橫甚,民苦之。請案之。上曰:「此丞相事,何請?」夫亦持蚡陰事,為姦利,受淮南王金與語言。賓客居間,遂已,俱解。

夏,蚡取燕王女為夫人,太后詔召列侯宗室皆往賀。嬰過夫,欲與俱。夫謝曰:「夫數以酒失過丞相,丞相今者又與夫有隙。」嬰曰:「事已解。」彊與俱。酒酣,蚡起為壽,坐皆避席伏。已嬰為壽,獨故人避席,餘半膝席。夫行酒,至蚡,蚡膝席曰:「不能滿觴。」夫怒,因嘻笑曰:「將軍貴人也,畢之!」時蚡不肯。行酒次至臨汝侯灌賢,賢方與程不識耳語,又不避席。夫無所發怒,乃罵賢曰:「平生毀程不識不直一錢,今日長者為壽,乃效女曹兒呫囁耳語!」蚡謂夫曰:「程、李俱東西宮衛尉,今眾辱程將軍,仲孺獨不為李將軍地乎?」夫曰:「今日斬頭穴匈,何知程、李!」坐乃起更衣,稍稍去。嬰去,戲夫。夫出,蚡遂怒曰:「此吾驕灌夫罪也。」乃令騎留夫,夫不得出。藉福起為謝,案夫項令謝。夫愈怒,不肯順。蚡乃戲騎縛夫置傳舍,召長史曰:「今日召宗室,有詔。」劾灌夫罵坐不敬,繫居室。遂其前事,遣吏分曹逐捕諸灌氏支屬,皆得棄巿罪。嬰愧,為資使賓客請,莫能解。蚡吏皆為耳目,諸灌氏皆亡匿,夫繫,遂不得告言蚡陰事。

嬰銳為救夫,嬰夫人諫曰:「灌將軍得罪丞相,與太后家迕,寧可救邪?」嬰曰:「侯自我得之,自我捐之,無所恨。且終不令灌仲孺獨死,嬰獨生。」乃匿其家,竊出上書。立召入,具告言灌夫醉飽事,不足誅。上然之,賜嬰食,曰:「東朝廷辯之。」

嬰東朝,盛推夫善,言其醉飽得過,乃丞相以它事誣罪之。蚡盛毀夫所為橫恣,罪逆不道。嬰度無可奈何,因言蚡短。蚡曰:「天下幸而安樂無事,蚡得為胏附,所好音樂狗馬田宅,所愛倡優巧匠之屬,不如魏其、灌夫日夜招聚天下豪桀壯士與論議,腹誹而心謗,卬視天,俛畫地,辟睨兩宮間,幸天下有變,而欲有大功。臣乃不如魏其等所為。」上問朝臣:「兩人孰是?」御史大夫韓安國曰:「魏其言灌夫父死事,身荷戟馳不測之吳軍,身被數十創,名冠三軍,此天下壯士,非有大惡,爭柸酒,不足引它過以誅也。魏其言是。丞相亦言灌夫通姦猾,侵細民,家累巨萬,橫恣潁川,輘轢宗室,侵犯骨肉,此所謂『

支大於幹,脛大於股,不折必披』。丞相言亦是。唯明主裁之。」主爵都尉汲黯是魏其。內史鄭當時是魏其,後不堅。餘皆莫敢對。上怒內史曰:「公平生數言魏其、武安長短,今日廷論,局趣效轅下駒,吾并斬若屬矣!」即罷起入,上食太后。太后亦已使人候司,具以語太后。太后怒,不食,曰:「我在也,而人皆藉吾弟,令我百歲後,皆魚肉之乎!且帝寧能為石人邪!此特帝在,即錄錄,設百歲後,是屬寧有可信者乎?」上謝曰:「俱外家,故廷辨之。不然,此一獄吏所決耳。」是時郎中令石建為上分別言兩人。

蚡已罷朝,出止車門,召御史大夫安國載,怒曰:「與長孺共一禿翁,何為首鼠兩端?」安國良久謂蚡曰:「君何不自喜!夫魏其毀君,君當免冠解印綬歸,曰『臣以胏附幸得待罪,固非其任,魏其言皆是。』如此,上必多君有讓,不廢君。魏其必媿,杜門齰舌自殺。今人毀君,君亦毀之,譬如賈豎女子爭言,何其無大體也!」蚡謝曰:「爭時急,不知出此。」

於是上使御史簿責嬰所言灌夫頗不讎,劾繫都司空。孝景時,嬰嘗受遺詔,曰「事有不便,以便宜論上。」及繫,灌夫罪至族,事日急,諸公莫敢復明言於上。嬰乃使昆弟子上書言之,幸得召見。書奏,案尚書,大行無遺詔。詔書獨臧嬰家,嬰家丞封。乃劾嬰矯先帝詔害,罪當棄市。五年十月,悉論灌夫支屬。嬰良久乃聞有劾,即陽病痱,不食欲死。或聞上無意殺嬰,復食,治病,議定不死矣。乃有飛語為惡言聞上,故以十二月晦論棄市渭城。

春,蚡疾,一身盡痛,若有擊者,謼服謝罪。上使視鬼者瞻之,曰:「魏其侯與灌夫共守,笞欲殺之。」竟死。子恬嗣,元朔中有罪免。

後淮南王安謀反,覺。始安入朝時,蚡為太尉,迎安霸上,謂安曰:「上未有太子,大王最賢,高祖孫,即公車晏駕,非大王立,尚誰立哉?」淮南王大喜,厚遺金錢財物。上自嬰、夫事時不直蚡,特為太后故。及聞淮南事,上曰:「使武安侯在者,族矣。」

韓安國字長孺,梁成安人也,後徙睢陽。嘗受韓子、雜說鄒田生所。事梁孝王,為中大夫。吳楚反時,孝王使安國及張羽為將,扞吳兵於東界。張羽力戰,安國持重,以故吳不能過梁。吳楚破,安國、張羽名由此顯梁。

梁王以至親故,得自置相、二千石,出入游戲,僭於天子。天子聞之,心不善。太后知帝弗善,乃怒梁使者,弗見,案責王所為。安國為梁使,見大長公主而泣曰:「何梁王為人子之孝,為人臣之忠,而太后曾不省也?夫前日吳、楚、齊、趙七國反,自關以東皆合從而西嚮,唯梁最親,為限難。梁王念太后、帝在中,而諸侯擾亂,壹言泣數行而下,跪送臣等六人將兵擊卻吳楚,吳楚以故兵不敢西,而卒破亡,梁之力也。今太后以小苛禮責望梁王。梁王父兄皆帝王,而所見者大,故出稱旧,入言警,車旗皆帝所賜,即以嫮鄙小縣,驅馳國中,欲夸諸侯,令天下知太后、帝愛之也。今梁使來,輒案責之,梁王恐,日夜涕泣思慕,不知所為。何梁王之忠孝而太后不卹也?」長公主具以告太后,太后喜曰:「為帝言之。」言之,帝心乃解,而免冠謝太后曰:「兄弟不能相教,乃為太后遺憂。」悉見梁使,厚賜之。其後,梁王益親驩。太后、長公主更賜安國直千餘金。由此顯,結於漢。

其後,安國坐法抵罪,蒙獄吏田甲辱安國。安國曰:「死灰獨不復然乎?」甲曰:「然即溺之。」居無幾,梁內史缺,漢使使者拜安國為梁內史,起徙中為二千石。田甲亡。安國曰:「甲不就官,我滅而宗。」四甲肉袒謝,安國笑曰:「公等足與治乎?」卒善遇之。

內史之缺也,王新得齊人公孫詭,說之,欲請為內史。竇太后所,乃詔王以安國為內史。

公孫詭、羊勝說王求為帝太子及益地事,恐漢大臣不聽,乃陰使人刺漢用事謀臣。及殺故吳相爰盎,景帝遂聞詭、勝等計畫,乃遣使捕詭、勝,必得。漢使十輩至梁,相以下舉國大索,月餘弗得。安國聞詭、勝匿王所,乃入見王而泣曰:「主辱者臣死。大王無良臣,故紛紛至此。今勝、詭不得,請辭賜死。」王曰:「何至此?」安國泣數行下,曰:「大王自度於皇帝,孰與太上皇之與高帝及皇帝與臨江王親?」王曰:「弗如也。」安國曰:「夫太上皇、臨江親父子間,然高帝曰『提三尺取天下者朕也』,故太上終不得制事,居于櫟陽。臨江,適長太子,以一言過,廢王臨江;用宮垣事,卒自殺中尉府。何則?治天下終不用私亂公。語曰:『雖有親父,安知不為虎?雖有親兄,安知不為狼?』今大王列在諸侯,訹邪臣浮說,犯上禁,橈明法。天子以太后故,不忍致法於大王。太后日夜涕泣,幸大王自改,大王終不覺寤。有如太后宮車即晏駕,大王尚誰攀乎?」語未卒,王泣數行而下,謝安國曰:「吾今出之。」即日詭、勝自殺。漢使還報,梁事皆得釋,安國力也。景帝、太后益重安國。

孝王薨,共王即位,安國坐法失官,家居。武帝即位,武安侯田蚡為太尉,親貴用事。安國以五百金遺蚡,蚡言安國太后,上素聞安國賢,即召以為北地都尉,遷為太司農。閩、東越相攻,遣安國、大行王恢將兵。未至越,越殺其王降,漢兵亦罷。其年,田蚡為丞相,安國為御史大夫。

匈奴來請和親,上下其議。大行王恢,燕人,數為邊吏,習胡事,議曰:「漢與匈奴和親,率不過數歲即背約。不如勿許,舉兵擊之。」安國曰:「千里而戰,即兵不獲利。今匈奴負戎馬足,懷鳥獸心,遷徙鳥集,難得而制。得其地不足為廣,有其眾不足為彊,自上古弗屬。漢數千里爭利,則人馬罷,虜以全制其敝,勢必危殆。臣故以為不如和親。」群臣議多附安國,於是上許和親。

明年,雁門馬邑豪聶壹因大行王恢言:「匈奴初和親,親信邊,可誘以利致之,伏兵襲擊,必破之道也。」上乃召問公卿曰:「朕飾子女以配單于,幣帛文錦,賂之甚厚。單于待命加嫚,侵盜無已,邊竟數驚,朕甚閔之。今欲舉兵攻之,何如?」

大行恢對曰:「陛下雖未言,臣固願效之。臣聞全代之時,北有彊胡之敵,內連中國之兵,然尚得養老長幼,種樹以時,倉廩常實,匈奴不輕侵也。今以陛下之威,海內為一,天下同任,又遣子弟乘邊守塞,轉粟輓輸,以為之備,然匈奴侵盜不已者,無它,以不恐之故耳。臣竊以為擊之便。」

御史大夫安國曰:「不然。臣聞高皇帝嘗圍於平城,匈奴至者投鞍高如城者數所。平城之飢,七日不食,天下歌之,及解圍反位,而無忿怒之心。夫聖人以天下為度者也,不以己私怒傷天下之功,故乃遣劉敬奉金千斤,以結和親,至今為五世利。孝文皇帝又嘗壹擁天下之精兵聚之廣武常谿,然終無尺寸之功,而天下黔首無不憂者。孝文寤於兵之不可宿,故復合和親之約。此二聖之跡,足以為效矣。臣竊以為勿擊便。」

恢曰:「不然。臣聞五帝不相襲禮,三王不相復樂,非故相反也,各因世宜也。且高帝身被堅執銳,蒙霧露,沐霜雪,行幾十年,所以不報平城之怨者,非力不能,所以休天下之心也。今邊竟數驚,士卒傷死,中國槥車相望,此仁人之所隱也。臣故曰擊之便。」

安國曰:「不然。臣聞利不十者不易業,功不百者不變常,是以古之人君謀事必就祖,發政占古語,重作事也。且自三代之盛,夷狄不與正朔服色,非威不能制,彊弗能服也,以為遠方絕地不牧之民,不足煩中國也。且匈奴,輕疾悍亟之兵也,至如猋風,去如收電,畜牧為業,弧弓射獵,逐獸隨草,居處無常,難得而制。今使邊郡久廢耕織,以支胡之常事,其勢不相權也。臣故曰勿擊便。」

恢曰:「不然。臣聞鳳鳥乘於風,聖人因於時。昔秦繆公都雍,地方三百里,知時宜之變,攻取西戎,辟地千里,并國十四,隴西、北地是也。及後蒙恬為秦侵胡,辟數千里,以河為竟,累石為城,樹榆為塞,匈奴不敢飲馬於河,置餍锚然後敢牧馬。夫匈奴獨可以威服,不可以仁畜也。今以中國之盛,萬倍之資,遣百分之一以攻匈奴,譬猶以彊弩射且潰之癰也,必不留行矣。若是,則北發月氏可得而臣也。臣故曰擊之便。」

安國曰:「不然。臣聞用兵者以飽待饑,正治以待其亂,定舍以待其勞。故接兵覆眾,伐國墮城,常坐而役敵國,此聖人之兵也。且臣聞之,衝風之衰,不能起毛羽;彊弩之末,力不能入魯縞。夫盛之有衰,猶朝之必莫也。今將卷甲輕舉,深入長敺,難以為功;從行則迫脅,衡行則中絕,疾則糧乏,徐則後利,不至千里,人馬乏食。兵法曰:『遺人獲也。』意者有它繆巧可以禽之,則臣不知也;不然,則未見深入之利也。臣故曰勿擊便。」

恢曰:「不然。夫草木遭霜者不可以風過,清水明鏡不可以形逃,通方之士,不可以文亂。今臣言擊之者,固非發而深入也,將順因單于之欲,誘而致之邊,吾選梟騎或絕其後,單于可禽,百全必取。」

上曰:「善。」乃從恢議。陰使聶壹為間,亡入匈奴,謂單于曰:「吾能斬馬邑令丞,以城降,財物可盡得。」單于愛信,以為然而許之。聶壹乃詐斬死罪囚,縣其頭馬邑城下,視單于使者為信,曰:「馬邑長吏已死,可急來。」於是單于穿塞,將十萬騎入武州塞。

當是時,漢伏兵車騎材官三十餘萬,匿馬邑旁谷中。衛尉李廣為驍騎將軍,太僕公孫賀為輕車將軍,大行王恢為將屯將軍,太中大夫李息為材官將軍。御史大夫安國為護軍將軍,諸將皆屬。約單于入馬邑縱兵。王恢、李息別從代主擊輜重。於是單于入塞,未至馬邑百餘里,覺之,還去。語在匈奴傳。塞下傳言單于已去,漢兵追至塞,度弗及,王恢等皆罷兵。

上怒恢不出擊單于輜重也,恢曰:「始約為入馬邑城,兵與單于接,而臣擊其輜重,可得利。今單于不至而還,臣以三萬人眾不敵,祗取辱。固知還而斬,然完陛下士三萬人。」於是下恢廷尉,廷尉當恢逗橈,當斬。恢行千金丞相蚡。蚡不敢言上,而言於太后曰:「王恢首為馬邑事,今不成而誅恢,是為匈奴報仇也。」上朝太后,太后以蚡言告上。上曰:「首為馬邑事者恢,故發天下兵數十萬,從其言,為此。且縱單于不可得,恢所部擊,猶頗可得,以尉士大夫心。今不誅恢,無以謝天下。」於是恢聞,乃自殺。

安國為人多大略,知足以當世取舍,而出於忠厚。貪耆財利,然所推舉皆廉士賢於己者。於梁舉壺遂、臧固,至它,皆天下名士,士亦以此稱慕之,唯天子以為國器。安國為御史大夫五年,丞相蚡薨。安國行丞相事,引墮車,蹇。上欲用安國為丞相,使使視,蹇甚,乃更以平棘侯薛澤為丞相。安國病免,數月,瘉,復為中尉。

歲餘,徙為衛尉。而將軍衛青等擊匈奴,破龍城。明年,匈奴大入邊。語在青傳。安國為材官將軍,屯漁陽,捕生口虜,言匈奴遠去。即上言方佃作時,請且罷屯。罷屯月餘,匈奴大入上谷、漁陽。安國壁乃有七百餘人,出與戰,安國傷,入壁。匈奴虜略千餘人及畜產去。上怒,使使責讓安國。徙益東,屯右北平。是時虜言當入東方。

安國始為御史大夫及護軍,後稍下遷。新壯將軍衛青等有功,益貴。安國既斥疏,將屯又失亡多,甚自媿。幸得罷歸,乃益東徙,意忽忽不樂,數月,病歐血死。

壺遂與太史遷等定漢律曆,官至詹事,其人深中篤行君子。上方倚欲以為相,會其病卒。

贊曰:竇嬰、田蚡皆以外戚重,灌夫用一時決策,而各名顯,並位卿相,大業定矣。然嬰不知時變,夫亡術而不遜,蚡負貴而驕溢。凶德參會,待時而發,藉福區區其間,惡能救斯敗哉!以韓安國之見器,臨其摯而顛墜,陵夷以憂死,遇合有命,悲夫!若王恢為兵首而受其咎,豈命也虖?


\end{pinyinscope}