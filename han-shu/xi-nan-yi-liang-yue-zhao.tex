\article{西南夷兩粵朝鮮傳}

\begin{pinyinscope}
西夷君長以十數,夜郎最大。其西,靡莫之屬以十數,滇最大。自滇以北,君長以十數,邛都最大。此皆椎結,耕田,有邑聚。其外,西至桐師以東,北至葉榆,外為巂、昆明,編髮,隨畜移徙,亡常處,亡君長,地方可數千里。自巂以東北,君長以十數,徙、莋都最大。自莋以東北,君長以十數,冉駹最大。其俗,或土著,或移徙。在蜀之西。自駹以東北,君長以十數,白馬最大,皆氐類也。此皆巴蜀西南外蠻夷也。

始楚威王時,使將軍莊蹻將兵循江上,略巴、黔中以西。莊蹻者,楚莊王苗裔也。蹻至滇池,方三百里,旁平地肥饒數千里,以兵威定屬楚。欲歸報,會秦擊奪楚巴、黔中郡,道塞不通,因乃以其眾王滇,變服,從其俗,以長之。秦時嘗破,略通五尺道,諸此國頗置吏焉。十餘歲,秦滅。及漢興,皆棄此國而關蜀故徼。巴屬民或竊出商賈,取其莋馬、僰僮、旄牛,以此巴蜀殷富。

建元六年,大行王恢擊東粵,東粵殺王郢以報。恢因兵威使番陽令唐蒙風曉南粵。南粵食蒙蜀枸醬,蒙問所從來,曰:「道西北牂柯江,江廣數里,出番禺城下。」蒙歸至長安,問蜀賈人,獨蜀出枸醬,多持竊出市夜郎。夜郎者,臨牂柯江,江廣百餘步,足以行船。南粵以財物役屬夜郎,西至桐師,然亦不能臣使也。蒙乃上書說上曰:「南粵王黃屋左纛,地東西萬餘里,名為外臣,實一州主。今以長沙、豫章往,水道多絕,難行。竊聞夜郎所有精兵可得十萬,浮船牂柯,出不意,此制粵一奇也。誠以漢之強,巴蜀之饒,通夜郎道,為置吏,甚易。」上許之。乃拜蒙以郎中將,將千人,食重萬餘人,從巴莋關入,遂見夜郎侯多同。厚賜,諭以威德,約為置吏,使其子為令。夜郎旁小邑皆貪漢繒帛,以為漢道險,終不能有也,乃且聽蒙約。還報,乃以為犍為郡。發巴蜀卒治道,自僰道指牂柯江。蜀人司馬相如亦言西夷邛、莋可置郡。使相如以郎中將往諭,皆如南夷,為置一都尉,十餘縣,屬蜀。

當是時,巴蜀四郡通西南夷道,載轉相饟。數歲,道不通,士罷餓餧,離暑溼,死者甚眾。西南夷又數反,發兵興擊,耗費亡功。上患之,使公孫弘往視問焉。還報,言其不便。及弘為御史大夫,時方築朔方,據河逐胡,弘等因言西南夷為害,可且罷,專力事匈奴。上許之,罷西夷,獨置南夷兩縣一都尉,稍令犍為自保就。

及元狩元年,博望侯張騫言使大夏時,見蜀布、邛竹杖,問所從來,曰「從東南身毒國,可數千里,得蜀賈人市。」或聞邛西可二千里有身毒國。騫因盛言大夏在漢西南,慕中國,患匈奴隔其道,誠通蜀,身毒國道便近,又亡害。於是天子乃令王然于、柏始昌、呂越人等十餘輩間出西南夷,指求身毒國。至滇,滇王當羌乃留為求道。四歲餘,皆閉昆明,莫能通。滇王與漢使言:「漢孰與我大?」及夜郎侯亦然。各自以一州王,不知漢廣大。使者還,因盛言滇大國,足事親附。天子注意焉。

及至南粵反,上使馳義侯因犍為發南夷兵。且蘭君恐遠行,旁國虜其老弱,乃與其眾反,殺使者及犍為太守。漢乃發巴蜀罪人當擊南粵者八校尉擊之。會越已破,漢八校尉不下,中郎將郭昌、衛廣引兵還,行誅隔滇道者且蘭,斬首數萬,遂平南夷為牂柯郡。夜郎侯始倚南粵,南粵已滅,還誅反者,夜郎遂入朝,上以為夜郎王。南粵破後,及漢誅且蘭、邛君,并殺莋侯,冉駹皆震恐,請臣置吏。以邛都為粵巂郡,莋都為沈黎郡,冉駹為文山郡,廣漢西白馬為武都郡。

使王然于以粵破及誅南夷兵威風諭滇王入朝。滇王者,其眾數萬人,其旁東北勞深、靡莫皆同姓相杖,未肯聽。勞、莫數侵犯使者吏卒。元封二年,天子發巴蜀兵擊滅勞深、靡莫,以兵臨滇。滇王始首善,以故弗誅。滇王離西夷,滇舉國降,請置吏入朝。於是以為益州郡,賜滇王王印,復長其民。西南夷君長以百數,獨夜郎、滇受王印。滇,小邑也,最寵焉。

後二十三歲,孝昭始元元年,益州廉頭、姑繒民反,殺長吏。牂柯、談指、同並等二十四邑,凡三萬餘人皆反。遣水衡都尉發蜀郡、犍為奔命萬餘人擊牂柯,大破之。後三歲,姑繒、葉榆復反,遣水衡都尉呂辟胡將郡兵擊之。辟胡不進,蠻夷遂殺益州太守,乘勝與辟胡戰,士戰及溺死者四千餘人。明年,復遣軍正王平與大鴻臚田廣明等並進,大破益州,斬首捕虜五萬餘級,獲畜產十餘萬。上曰:「鉤町侯亡波率其邑君長人民擊反者,斬首捕虜有功,其立亡波為鉤町王。大鴻臚廣明賜爵關內侯,食邑三百戶。」後間歲,武都氐人反,遣執金吾馬適建、龍哣侯韓增與大鴻臚廣明將兵擊之。

至成帝河平中,夜郎王興與鉤町王禹、漏臥侯俞更舉兵相攻。牂柯太守請發兵誅興等,議者以為道遠不可擊,乃遣太中大夫蜀郡張匡持節和解。興等不從命,刻木象漢吏,立道旁射之。杜欽說大將軍王鳳曰:「太中大夫匡使和解蠻夷王侯,王侯受詔,已復相攻,輕易漢使,不憚國威,其效可見。恐議者選耎,復守和解,太守察動靜,有變乃以聞。如此,則復曠一時,王侯得收獵其眾,申固其謀,黨助眾多,各不勝忿,必相殄滅。自知罪成,狂犯守尉,遠臧溫暑毒草之地,雖有孫吳將,賁育士,若入水火,往必焦沒,知勇亡所施。屯田守之,費不可勝量。宜因其罪惡未成,未疑漢家加誅,陰敕旁郡守尉練士馬,大司農豫調穀積要害處,選任職太守往,以秋涼時入,誅其王侯尤不軌者。即以為不毛之地,亡用之民,聖王不以勞中國,宜罷郡,放棄其民,絕其王侯勿復通。如以先帝所立累世之功不可墮壞,亦宜因其萌牙,早斷絕之,及已成形然後戰師,則萬姓被害。」

大將軍鳳於是薦金城司馬陳立為牂柯太守。立者,臨邛人,前為連然長,不韋令,蠻夷畏之。及至牂柯,諭告夜郎王興,興不從命,立請誅之。未報,乃從吏數十人出行縣,至興國且同亭,召興。興將數千人往至亭,從邑君數十人入見立。立數責,因斷頭。邑君曰:「將軍誅亡狀,為民除害,願出曉士眾。」以興頭示之,皆釋兵降。鉤町王禹、漏臥侯俞震恐,入粟千斛,牛羊勞吏士。立還歸郡,興妻父翁指與興子邪務收餘兵,迫脅旁二十二邑反。至冬,立奏募諸夷與都尉長史分將攻翁指等。翁指據阨為壘,立使奇兵絕其饟道,縱反間以誘其眾。都尉萬年曰:「兵久不決,費不可共。」引兵獨進,敗走,趨立營。立怒,叱戲下令格之。都尉復還戰,立引兵救之。時天大旱,立攻絕其水道。蠻夷共斬翁指,持首出降。立已平定西夷,徵詣京師。會巴郡有盜賊,復以立為巴郡太守,秩中二千石居,賜爵左庶長。徙為天水太守,勸民農桑為天下最,賜金四十斤。入為左曹衛將軍、護軍都尉,卒官。

王莽篡位,改漢制,貶鉤町王以為侯。王邯怨恨,牂柯大尹周欽詐殺邯。邯弟承攻殺欽,州郡擊之,不能服。三邊蠻夷愁擾盡反,復殺益州大尹程隆。莽遣平蠻將軍馮茂發巴、蜀、犍為吏士,賦斂取足於民,以擊益州。出入三年,疾疫死者什七,巴、蜀騷動。莽徵茂還,誅之。更遣寧始將軍廉丹與庸部牧史熊大發天水、隴西騎士,廣漢、巴、蜀、犍為吏民十萬人,轉輸者合二十萬人,擊之。始至,頗斬首數千,其後軍糧前後不相及,士卒飢疫,三歲餘死者數萬。而粵嶲蠻夷任貴亦殺太守枚根,自立為邛穀王。會莽敗漢興,誅貴,復舊號云。

南粵王趙佗,真定人也。秦并天下,略定揚粵,置桂林、南海、象郡,以適徙民與粵雜處。十三歲,至二世時,南海尉任囂病且死,召龍川令趙佗語曰「聞陳勝等作亂,豪桀叛秦相立,南海辟遠,恐盜兵侵此。吾欲興兵絕新道,自備待諸侯變,會疾甚。且番禺負山險阻,南北東西數千里,頗有中國人相輔,此亦一州之主,可為國。郡中長吏亡足與謀者,故召公告之。」即被佗書,行南海尉事。囂死,佗即移檄告橫浦、陽山、湟谿關曰:「盜兵且至,急絕道聚兵自守。」因稍以法誅秦所置吏,以其黨為守假。秦已滅,佗即擊并桂林、象郡,自立為南粵武王。

高帝已定天下,為中國勞苦,故釋佗不誅。十一年,遣陸賈立佗為南粵王,與剖符通使,使和輯百粵,毋為南邊害,與長沙接境。

高后時,有司請禁粵關巿鐵器。佗曰:「高皇帝立我,通使物,今高后聽讒臣,別異蠻夷,鬲絕器物,此必長沙王計,欲倚中國,擊滅南海并王之,自為功也。」於是佗乃自尊號為南武帝,發兵攻長沙邊,敗數縣焉。高后遣將軍隆慮侯灶擊之,會暑溼,士卒大疫,兵不能隃領。歲餘,高后崩,即罷兵。佗因此以兵威財物賂遺閩粵、西甌駱,锁屬焉。東西萬餘里。乃乘黃屋左纛,稱制,與中國侔。

文帝元年,初鎮撫天下,使告諸侯四夷從代來即位意,諭盛德焉。乃為佗親冢在真定置守邑,歲時奉祀。召其從昆弟,尊官厚賜寵之。詔丞相平舉可使粵者,平言陸賈先帝時使粵。上召賈為太中大夫,謁者一人為副使,賜佗書曰:「皇帝謹問南粵王,甚苦心勞意。朕,高皇帝側室之子,棄外奉北藩于代,道里遼遠,壅蔽樸愚,未嘗致書。高皇帝棄群臣,孝惠皇帝即世,高后

白臨事,不幸有疾,日進不衰,以故誖暴乎治。諸呂為變故亂法,不能獨制,乃取它姓子為孝惠皇帝嗣。賴宗廟之靈,功臣之力,誅之已畢。朕以王侯吏不釋之故,不得不立,今即位。乃者聞王遺將軍隆慮侯書,求親昆弟,請罷長沙兩將軍。朕以王書罷將軍博陽侯,親昆弟在真定者,已遣人存問。脩治先人冢。前日聞王發兵於邊,為寇災不止。當其時長沙苦之,南郡尤甚,雖王之國,庸獨利乎!必多殺士卒,傷良將吏,寡人之妻,孤人之子,獨人父母,得一亡十,朕不忍為也。朕欲定地犬牙相入者,以問吏,吏曰『高皇帝所以介長沙土也』,朕不得擅變焉。吏曰:『得王之地不足以為大,得王之財不足以為富,服領以南,王自治之。』雖然,王之號為帝。兩帝並立,亡一乘之使以通其道,是爭也;爭而不讓,仁者不為也。願與王分棄前患,終今以來,通使如故。故使賈馳諭告王朕意,王亦受之,毋為寇災矣。上褚五十衣,中褚三十衣,下褚二十衣,遺王。願王聽樂娛憂,存問鄰國。」

陸賈至,南粵王恐,乃頓首謝,願奉明詔,長為藩臣,奉貢職。於是下令國中曰:「吾聞兩雄不俱立,兩賢不並世。漢皇帝賢天子。自今以來,去帝制黃屋左纛。」因為書稱:「蠻夷大長老夫臣佗昧死再拜上書皇帝陛下:老夫故粵吏也,高皇帝幸賜臣佗璽,以為南粵王,使為外臣,時內貢職。孝惠皇帝即位,義不忍絕,所以賜老夫者厚甚。高后自臨用事,近細士,信讒臣,別異蠻夷,出令曰:『毋予蠻夷外粵金鐵田器;馬牛羊即予,予牡,毋與牝。』老夫處辟,馬牛羊齒已長,自以祭祀不脩,有死罪,使內史藩、中尉高、御史平凡三輩上書謝過,皆不反。又風聞老夫父母墳墓已壞削,兄弟宗族已誅論。吏相與議曰:『今內不得振於漢,外亡以自高異。』故更號為帝,自帝其國,非敢有害於天下也。高皇后聞之大怒,削去南粵之籍,使使不通。老夫竊疑長沙王讒臣,故敢發兵以伐其邊。且南方卑溼,蠻夷中西有西甌,其眾半羸,南面稱王;東有閩粵,其眾數千人,亦稱王;西北有長沙,其半蠻夷,亦稱王。老夫故敢妄竊帝號,聊以自娛。老夫身定百邑之地,東西南北數千萬里,帶甲百萬有餘,然北面而臣事漢,何也?不敢背先人之故。老夫處粵四十九年,于今抱孫焉。然夙興夜寐,寢不安席,食不甘味,目不視靡曼之色,耳不聽鍾鼓之音者,以不得事漢也。今陛下幸哀憐,復故號,通使漢如故,老夫死骨不腐,改號不敢為帝矣!謹北面因使者獻白璧一雙,翠鳥千,犀角十,紫貝五百,桂蠹一器,生翠四十雙,孔雀二雙。昧死再拜,以聞皇帝陛下。」

陸賈還報,文帝大說。遂至孝景時,稱臣遣使入朝請。然其居國,竊如故號;其使天子,稱王朝命如諸侯。

至武帝建元四年,佗孫胡為南粵王。立三年,閩粵王郢興兵南擊邊邑。粵使人上書曰:「兩粵俱為藩臣,毋擅興兵相攻擊。今東粵擅興兵侵臣,臣不敢興兵,唯天子詔之。」於是天子多南粵義,守職約,為興師,遣兩將軍往討閩粵。兵未隃領,閩粵王弟餘善殺郢以降,於是罷兵。

天子使嚴助往諭意,南粵王胡頓首曰:「天子乃興兵誅閩粵,死亡以報德!」遣太子嬰齊入宿衛。謂助曰:「國新被寇,使者行矣。胡方日夜裝入見天子。」助去後,其大臣諫胡曰:「漢興兵誅郢,亦行以驚動南粵。且先王言事天子期毋失禮,要之不可以怵好語入見。入見則不得復歸,亡國之勢也。」於是胡稱病,竟不入見。後十餘歲,胡實病甚,太子嬰齊請歸。胡薨,諡曰文王。

嬰齊嗣立,即臧其先武帝、文帝璽。嬰齊在長安時,取邯鄲摎氏女,生子興。及即位,上書請立摎氏女為后,興為嗣。漢數使使者風諭,嬰齊猶尚樂擅殺生自恣,懼入見,要以用漢法,比內諸侯,固稱病,遂不入見。遣子次公入宿衛,嬰齊薨,諡為明王。

太子興嗣立,其母為太后,太后自未為嬰齊妻時,曾與霸陵人安國少季通。及嬰齊薨後,元鼎四年,漢使安國少季諭王、王太后入朝,令辯士諫大夫終軍等宣其辭,勇士魏臣等輔其決,衛尉路博德將兵屯桂陽,待使者。王年少,太后中國人,安國少季往,復與私通,國人頗知之,多不附太后。太后恐亂起,亦欲倚漢威,勸王及幸臣求內屬。即因使者上書,請比內諸侯,三歲壹朝,除邊關。於是天子許之,賜其丞相呂嘉銀印,及內史、中尉、太傅印,餘得自置。除其故黥劓刑,用漢法。諸使者皆留填撫之。王、王太后飭治行裝重資,為入朝具。

相呂嘉年長矣,相三王,宗族官貴為長吏七十餘人,男盡尚王女,女盡嫁王子弟宗室,及蒼梧秦王有連。其居國中甚重,粵人信之,多為耳目者,得眾心愈於王。王之上書,數諫止王,王不聽。有畔心,數稱病不見漢使者。使者注意嘉,勢未能誅。王、王太后亦恐嘉等先事發,欲介使者權,謀誅嘉等。置酒請使者,大臣皆侍坐飲。嘉弟為將,將卒居宮外。酒行,太后謂嘉:「南粵內屬,國之利,而相君苦不便者,何也?」以激怒使者。使者狐疑相杖,遂不敢發。嘉見耳目非是,即趨出。太后怒,欲鏦嘉以矛,王止太后。嘉遂出,介弟兵就舍,稱病,不肯見王及使者。乃陰謀作亂,王素亡意誅嘉,嘉知之,以故數月不發。太后獨欲誅嘉等,力又不能。

天子聞之,罪使者怯亡決。又以為王、王太后已附漢,獨呂嘉為亂,不足以興兵,欲使莊參以二千人往。參曰:「以好往,數人足;以武往,二千人亡足以為也。」辭不可,天子罷參兵。郟壯士故濟北相韓千秋奮曰:「以區區粵,又有王應,獨呂嘉為害,願得勇士三百人,必斬嘉以報。」於是天子遣千秋與王太后弟摎樂將二千人往。入粵境,呂嘉乃遂反,下令國中曰:「王年少。太后中國人,又與使者亂,專欲內屬,盡持先王寶入獻天子以自媚,多從人,行至長安,虜賣以為僮。取自脫一時利,亡顧趙氏社稷為萬世慮之意。」乃與其弟將卒攻殺太后、王,盡殺漢使者。遣人告蒼梧秦王及其諸郡縣,立明王長男粵妻子術陽侯建德為王。而韓千秋兵之入也,破數小邑。其後粵直開道給食,未至番禺四十里,粵以兵擊千秋等,滅之。使人函封漢使節置塞上,好為謾辭謝罪,發兵守要害處。於是天子曰:「韓千秋雖亡成功,亦軍鋒之冠。封其子延年為成安侯。摎樂,其姊為王太后,首願屬漢,封其子廣德為龒侯。」乃赦天下,曰:「天子微弱,諸侯力政,譏臣不討賊。呂嘉、建德等反,自立晏如,令粵人及江淮以南樓船十萬師往討之。」

元鼎五年秋,衛尉路博德為伏波將軍,出桂陽,下湟水;主爵都尉楊僕為樓船將軍,出豫章,下橫浦;故歸義粵侯二人為戈船、下瀨將軍,出零陵,或下離水,或抵蒼梧;使馳義侯因巴蜀罪人,發夜郎兵,下牂柯江:咸會番禺。

六年冬,樓船將軍將精卒先陷尋骥,破石門,得粵船粟,因推而前,挫粵鋒,以粵數萬人待伏波將軍。伏波將軍將罪人,道遠後期,與樓船會乃有千餘人,遂俱進。樓船居前,至番禺,建德、嘉皆城守。樓船自擇便處,居東南面,伏波居西北面。會暮,樓船攻敗粵人,縱火燒城。粵素聞伏波,莫,不知其兵多少。伏波乃為營,遣使招降者,賜印綬,復縱令相招。樓船力攻燒敵,反敺而入伏波營中。遲旦,城中皆降伏波。呂嘉、建德以夜與其屬數百人亡入海。伏波又問降者,知嘉所之,遣人追。故其校司馬蘇弘得建德,為海常侯;粵郎都稽得嘉,為臨蔡侯。

蒼梧王趙光與粵王同姓,聞漢兵至,降,為隨桃侯。又粵揭陽令史定降漢,為安道侯。粵將畢取以軍降,為膫侯。粵桂林監居翁諭告甌駱四十餘萬口降,為湘城侯。戈船、下瀨將軍兵及馳義侯所發夜郎兵未下,南粵已平。遂以其地為儋耳、珠崖、南海、蒼梧、鬱林、合浦、交阯、九真、日南九郡。伏波將軍益封。樓船將軍以推鋒陷堅為將梁侯。

自尉佗王凡五世,九十三歲而亡。

閩粵王無諸及粵東海王搖,其先皆粵王句踐之後也,姓騶氏。秦并天下,廢為君長,以其地為閩中郡。及諸侯畔秦,無諸、搖率粵歸番陽令吳芮,所謂番君者也,從諸侯滅秦。當是時,項羽主命,不王也,以故不佐楚。漢擊項籍,無諸、搖帥粵人佐漢。漢五年,復立無諸為閩粵王,王閩中故地,都冶。孝惠三年,舉高帝時粵功,曰閩君搖功多,其民便附,乃立搖為東海王,都東甌,世號曰東甌王。

后數世,孝景三年,吳王濞反,欲從閩粵,閩粵未肯行,獨東甌從。及吳破,東甌受漢購,殺吳王丹徒,以故得不誅。

吳王子駒亡走閩粵,怨東甌殺其父,常勸閩粵擊東甌。建元三年,閩粵發兵圍東甌,東甌使人告急天子。天子問太尉田蚡,蚡對曰:「粵人相攻擊,固其常,不足以煩中國往救也。」中大夫嚴助詰蚡,言當救。天子遣助發會稽郡兵浮海救之,語具在助傳。漢兵未至,閩粵引兵去。東粵請舉國徙中國,乃悉與眾處江淮之間。

六年,閩粵擊南粵,南粵守天子約,不敢擅發兵,而以聞。上遣大行王恢出豫章,大司農韓安國出會稽,皆為將軍。兵未隃領,閩粵王郢發兵距險。其弟餘善與宗族謀曰:「王以擅發兵,不請,故天子兵來誅。漢兵眾強,即幸勝之,後來益多,滅國乃止。今殺王以謝天子,天子罷兵,固國完。不聽乃力戰,不勝即亡入海。」皆曰:「善。」即鏦殺王,使使奉其頭致大行。大行曰:「所為來者,誅王。王頭至,不戰而殞,利莫大焉。」乃以便宜案兵告大司農軍,而使使奉王頭馳報天子。詔罷兩將軍兵,曰:「郢等首惡,獨無諸孫繇君丑不與謀。」乃使郎中將立丑為粵繇王,奉閩粵祭祀。

餘善以殺郢,威行國中,民多屬,竊自立為王,繇王不能制。上聞之,為餘善不足復興師,曰:「餘善首誅郢,師得不勞。」因立餘善為東粵王,與繇王並處。

至元鼎五年,南粵反,餘善上書請以卒八十從樓船擊呂嘉等。兵至揭陽,以海風波為解,不行,持兩端,陰使南粵。及漢破番禺,樓船將軍僕上書願請引兵擊東粵。上以士卒勞倦,不許。罷兵,令諸校留屯豫章梅領待命。

明年秋,餘善聞樓船請誅之,漢兵留境,且往,乃遂發兵距漢道,號將軍騶力等為「吞漢將軍」,入白沙、武林、梅領,殺漢三校尉。是時,漢使大司農張成、故山州侯齒將屯,不敢擊,卻就便處,皆坐畏懦誅。餘善刻「武帝」璽自立,詐其民,為妄言。上遣橫海將軍韓說出句章,浮海從東方往;樓船將軍僕出武林,中尉王溫舒出梅領,粵侯為戈船、下瀨將軍出如邪、白沙,元封元年冬,咸入東粵。東粵素發兵距嶮,使徇北將軍守武林,敗樓船軍數校尉,殺長史。樓船軍卒錢唐榬終古斬徇北將軍,為語兒侯。自兵未往。

故粵衍侯吳陽前在漢,漢使歸諭餘善,不聽。及橫海軍至,陽以其邑七百人反,攻粵軍於漢陽。及故粵建成侯敖與繇王居股謀,俱殺餘善,以其眾降橫海軍。封居股為東成侯,萬戶;封敖為開陵侯;封陽為卯石侯,橫海將軍說為按道侯,橫海校尉福為繚嫈侯。福者,城陽王子,故為海常侯,坐法失爵,從軍亡功,以宗室故侯。及東粵將多軍,漢兵至,棄軍降,封為無錫侯。故甌駱將左黃同斬西于王,封為下鄜侯。

於是天子曰「東粵骥多阻,閩粵悍,數反覆」,詔軍吏皆將其民徙處江淮之間。東粵地遂虛。

朝鮮王滿,燕人。自始燕時,嘗略屬真番、朝鮮,為置吏築障。秦滅燕,屬遼東外徼。漢興,為遠難守,復修遼東故塞,至浿水為界,屬燕。燕王盧綰反,入匈奴,滿亡命,聚黨千餘人,椎結蠻夷服而東走出塞,度浿水,居秦故空地上下障,稍锁屬真番、朝鮮蠻夷及故燕、齊亡在者王之,都王險。

會孝惠、高后天下初定,遼東太守即約滿為外臣,保塞外蠻夷,毋使盜邊;蠻夷君長欲入見天子,勿得禁止。以聞,上許之,以故滿得以兵威財物侵降其旁小邑,真番、臨屯皆來服屬,方數千里。

傳子至孫右渠,所誘漢亡人滋多,又未嘗入見;真番、辰國欲上書見天子,又雍閼弗通。元封二年,漢使涉何譙諭右渠,終不肯奉詔。何去至界,臨浿水,使馭刺殺送何者朝鮮裨王長,即渡水,馳入塞,遂歸報天子曰「殺朝鮮將」。上為其名美,弗詰,拜何為遼東東部都尉。朝鮮怨何,發兵攻襲,殺何。

天子募罪人擊朝鮮。其秋,遣樓船將軍楊僕從齊浮勃海,兵五萬,左將軍荀彘出遼東,誅右渠。右渠發兵距險。左將軍卒多率遼東士兵先縱,敗散。多還走,坐法斬。樓船將齊兵七千人先至王險。右渠城守,窺知樓船軍少,即出擊樓船,樓船軍敗走。將軍僕失其眾,遁山中十餘日,稍求收散卒,復聚。左將軍擊朝鮮浿水西軍,未能破。

天子為兩將未有利,乃使衛山因兵威往諭右渠。右渠見使者,頓首謝:「願降,恐將詐殺臣;今見信節,請服降。」遣太子入謝,獻馬五千匹,及餽軍糧。人眾萬餘持兵,方度浿水,使者及左將軍疑其為變,謂太子已服降,宜令人毋持兵。太子亦疑使者左將軍詐之,遂不度浿水,復引歸。山報,天子誅山。

左將軍破浿水上軍,乃前至城下,圍其西北。樓船亦往會,居城南。右渠遂堅城守,數月未能下。

左將軍素侍中,幸,將燕代卒,悍,乘勝,軍多驕。樓船將齊卒,入海已多敗亡,其先與右渠戰,困辱亡卒,卒皆恐,將心慚,其圍右渠,常持和節。左將軍急擊之,朝鮮大臣乃陰間使人私約降樓船,往來言,尚未肯決。左將軍數與樓船期戰,樓船欲就其約,不會左將軍亦使人求間隙降下朝鮮,不肯,心附樓船。以故兩將不相得。左將軍心意樓船前有失軍罪,今與朝鮮和善而又不降,疑其有反計,未敢發。天子曰:「將率不能前,乃使衛山諭降右渠,不能顓決,與左將軍相誤,卒沮約。今兩將圍城又乖異,以故久不決。」使故濟南太守公孫遂往正之,有便宜得以從事。遂至,左將軍曰:「朝鮮當下久矣,不下者,樓船數期不會。」具以素所意告遂曰:「今如此不取,恐為大害,非獨樓船,又且與朝鮮共滅吾軍。」遂亦以為然,而以節召樓船將軍入左將軍軍計事,即今左將軍戲下執縛樓船將軍,并其軍。以報,天子許遂。

左將軍已并兩軍,即急擊朝鮮。朝鮮相路人、相韓陶、尼谿相參、將軍王唊相與謀曰:「始欲降樓船,樓船今執,獨左將軍并將,戰益急,恐不能與,王又不肯降。」陶、唊路人皆亡降漢。路人道死。元封三年夏,尼谿相參乃使人殺朝鮮王右渠來降。王險城未下,故右渠之大臣成已又反,復攻吏。左將軍使右渠子長、降相路人子最,告諭其民,誅成已。故遂定朝鮮為真番、臨屯、樂浪、玄菟四郡。封參為澅清侯,陶為秋苴侯,唊為平州侯,長為幾侯。最以父死頗有功,為沮陽侯。左將軍徵至,坐爭功相嫉乖計,棄市。樓船將軍亦坐兵至列口當待左將軍,擅先縱,失亡多,當誅,贖為庶人。

贊曰:楚、粵之先,歷世有土。及周之衰,楚地方五千里,而句踐亦以粵伯。秦滅諸侯,唯楚尚有滇王。漢誅西南夷,獨滇復寵。及東粵滅國遷眾,繇王居股等猶為萬戶侯。三方之開,皆自好事之臣。故西南夷發於唐蒙、司馬相如,兩粵起嚴助、朱買臣,朝鮮由涉何。遭世富盛,能成功,然已勤矣。追觀太宗填撫尉佗,豈古所謂「招攜以禮,懷遠以德」者哉!


\end{pinyinscope}