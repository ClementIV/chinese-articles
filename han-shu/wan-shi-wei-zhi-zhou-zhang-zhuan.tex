\article{萬石衛直周張傳}

\begin{pinyinscope}
萬石君石奮,其父趙人也。趙亡,徙溫。高祖東擊項籍,過河內,時奮年十五,為小吏,侍高祖。高祖與語,愛其恭敬,問曰:「若何有?」對曰;「有母,不幸失明。家貧。有姊,能鼓瑟。」高祖曰:「若能從我乎?」曰:「願盡力。」於是高祖召其姊為美人,以奮為中涓,受書謁。徙其家長安中戚里,以姊為美人故也。

奮積功勞,孝文時官至太中大夫。無文學,恭謹,舉無與比。東陽侯張相如為太子太傅,免。選可為傅者,皆推奮太子太傅。及孝景即位,以奮為九卿。迫近,憚之,徙奮為諸侯相。奮長子建,次甲,次乙,次慶,皆以馴行孝謹,官至二千石。於是景帝曰:「石君及四子皆二千石,人臣尊寵乃舉集其門。」凡號奮為萬石君。

孝景季年,萬石君以上大夫祿歸老于家,以歲時為朝臣。過宮門闕必下車趨,見路馬必軾焉。子孫謂小吏,來歸謁,萬石君必朝服見之,不名。子孫有過失,不誚讓,側,雖燕必冠,申申如也。僮僕訢訢如也,唯謹。上時賜食於家,必稽首俯伏而食,如在上前。其執喪,哀戚甚。子孫遵教,亦如之。萬石君家以孝謹聞乎郡國,雖齊魯諸儒質行,皆自以為不及也。

建元二年,郎中令王臧以文學獲罪皇太后。太后以為儒者文多質少,今萬石君家不言而躬行,乃以長子建為郎中令,少子慶為內史。

建老白首,萬石君尚無恙。每五日洗沐歸謁親,入子舍,竊問侍者,取親中茕廁牏,身自澣洒,復與侍者,不敢令萬石君知之,以為常。建奏事於上前,即有可言,屏人乃言極切;至廷見,如不能言者。上以是親而禮之。

萬石君徙居陵里。內史慶醉歸,入外門不下車。萬石君聞之,不食。慶恐,肉袒謝請罪,不許。舉宗及兄建肉袒,萬石君讓曰:「內史貴人,入閭里,里中長老皆走匿,而內史坐車中自如,固當!」乃謝罷慶。慶及諸子入里門,趨至家。

萬石君元朔五年卒,建哭泣哀思,杖乃能行。歲餘,建亦死。諸子孫咸孝,然建最甚,甚於萬石君。

建為郎中令,奏事下,建讀之,驚恐曰:「書『馬』者與尾而五,今乃四,不足一,獲譴死矣!」其為謹慎,雖他皆如是。

慶為太僕,御出,上問車中幾馬,慶以策數馬畢,舉手曰:「六馬。」慶於兄弟最為簡易矣,然猶如此。出為齊相,齊國慕其家行,不治而齊國大治,為立石相祠。

元狩元年,上立太子,選群臣可傅者,慶自沛守為太子太傅,七歲遷御史大夫。元鼎五年,丞相趙周坐酎金免,制詔御史:「萬石君先帝尊之,子孫至孝,其以御史大夫慶為丞相,封牧丘侯。」是時漢方南誅兩越,東擊朝鮮,北逐匈奴,西伐大宛,中國多事。天子巡狩海內,修古神祠,封禪,興禮樂。公家用少,桑弘羊等致利,王溫舒之屬峻法,兒寬等推文學,九卿更進用事,事不關決於慶,慶醇謹而已。在位九歲,無能有所匡言。嘗欲請治上近臣所忠、九卿咸宣,不能服,反受其過,贖罪。

元封四年,關東流民二百萬口,無名數者四十萬,公卿議欲請徙流民於邊以適之。上以為慶老謹,不能與其議,乃賜丞相告歸,而案御史大夫以下議為請者。慶慚不任職,上書曰:「臣幸得待罪丞相,疲駑無以輔治。城郭倉廩空虛,民多流亡,罪當伏斧質,上不忍致法。願歸丞相侯印,乞骸骨歸,避賢者路。」

上報曰:「間者,河水滔陸,泛濫十餘郡,隄防勤勞,弗能讯塞,朕甚憂之。是故巡方州,禮嵩嶽,通八神,以合宣房。濟淮江,歷山濱海,問百年民所疾苦。惟吏多私,徵求無已,去者便,居者擾,故為流民法,以禁重賦。乃者封泰山,皇天嘉況,神物並見。朕方答氣應,未能承意,是以切比閭里,知吏姦邪。委任有司,然則官曠民愁,盜賊公行。往年覲明堂,赦殊死,無禁錮,咸自新,與更始。今流民愈多,計文不改,君不繩責長吏,而請以興徙四十萬口,搖蕩百姓,孤兒幼年未滿十歲,無罪而坐率,朕失望焉。今君上書言倉庫城郭不充實,民多貧,盜賊眾,請入粟為庶人。夫懷知民貧而請益賦,動危之而辭位,欲安歸難乎?君其反室!」

慶素質,見詔報反室,自以為得許,欲上印綬。掾史以為見責甚深,而終以反室者,醜惡之辭也。或勸慶宜引決。慶甚懼,不知所出,遂復起視事。

慶為丞相,文深審謹,無他大略。後三歲餘薨,諡曰恬侯。中子德,慶愛之。上以德嗣,後為太常,坐法免,國除。慶方為丞相時,諸子孫為小吏至二千石者十三人。及慶死後,稍以罪去,孝謹衰矣。

衛綰,代大陵人也,以戲車為郎,事文帝,功次遷中郎將,醇謹無它。孝景為太子時,召上左右飲,而綰稱病不行。文帝且崩時,屬孝景曰:「綰長者,善遇之。」及景帝立,歲餘,不孰何綰,綰日以謹力。

景帝幸上林,詔中郎將參乘,還而問曰:「君知所以得驂乘乎?」綰曰:「臣代戲車士,幸得功次遷,待罪中郎將,不知也。」上問曰:「吾為太子時召君,君不肯來,何也?」對曰:「死罪,病。」上賜之劍,綰曰:「先帝賜臣劍凡六,不敢奉詔。」上曰:「劍,人之所施易,獨至今乎?」綰曰:「具在。」上使取六劍,劍常盛,未嘗服也。

郎官有譴,常蒙其罪,不與它將爭;有功,常讓它將。上以為廉,忠實無它腸,乃拜綰為河間王太傅。吳楚反,詔綰為將,將河間兵擊吳楚有功,拜為中尉。三歲,以軍功封綰為建陵侯。

明年,上廢太子,誅栗卿之屬。上以綰為長者,不忍,乃賜綰告歸,而使郅都治捕栗氏。既已,上立膠東王為太子,召綰拜為太子太傅,遷為御史大夫。五歲,代桃侯舍為丞相,朝奏事如職所奏。然自初宦以至相,終無可言。上以為敦厚可相少主,尊寵之,賞賜甚多。

為丞相三歲,景帝崩,武帝立。建元中,丞相以景帝病時諸官囚多坐不辜者,而君不任職,免之。後薨,諡曰哀侯。子信嗣,坐酎金,國除。

直不疑,南陽人也。為郎,事文帝。其同舍有告歸,誤持其同舍郎金去。已而同舍郎覺,亡意不疑,不疑謝有之,買金償。後告歸者至而歸金,亡金郎大慚,以此稱為長者。稍遷至中大夫。朝,廷見,人或毀不疑曰:「不疑狀貌甚美,然特毋柰其善盜嫂何也!」不疑聞,曰:「我乃無兄。」然終不自明也。

吳楚反時,不疑以二千石將擊之。景帝後元年,拜為御史大夫。天子修吳楚時功,封不疑為塞侯。武帝即位,與丞相綰俱以過免。

不疑學老子言。其所臨,為官如故,唯恐人之知其為吏跡也。不好立名,稱為長者。薨,諡曰信侯。傳子至孫彭祖,坐酎金,國除。

周仁,其先任城人也。以毉見。景帝為太子時,為舍人,積功遷至太中大夫。景帝初立,拜仁為郎中令。

仁為人陰重不泄。常衣弊補衣溺苎,期為不潔清,以是得幸,入臥內。於後宮祕戲,仁常在旁,終無所言。上時問人,仁曰:「上自察之。」然亦無所毀,如此。景帝再自幸其家。家徙陽陵。上所賜甚多,然終常讓,不敢受也。諸侯群臣賂遺,終無所受。武帝立,為先帝臣重之。仁乃病免,以二千石祿歸老,子孫咸至大官。

張敺字叔,高祖功臣安丘侯說少子也。敺孝文時以治刑名侍太子,然其人長者。景帝時尊重,常為九卿。至武帝元朔中,代韓安國為御史大夫。敺為吏,未嘗言按人,剸以誠長者處官。官屬以為長者,亦不敢大欺。上具獄事,有可卻,卻之;不可者,不得已,為涕泣,面而封之。其愛人如此。

老篤,請免,天子亦寵以上大夫祿,歸老于家。家陽陵。子孫咸至大官。

贊曰:仲尼有言「君子欲訥於言而敏於行」,其萬石君、建陵侯、塞侯、張叔之謂與?是以其教不肅而成,不嚴而治。至石建之澣衣,周仁為垢汙,君子譏之。


\end{pinyinscope}