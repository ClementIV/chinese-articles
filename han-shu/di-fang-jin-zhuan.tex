\article{翟方進傳}

\begin{pinyinscope}
翟方進字子威,汝南上蔡人也。家世微賤,至方進父翟公,好學,為郡文學。方進年十二三,失父孤學,給事太守府為小史,號遲頓不及事,數為掾史所詈辱。方進自傷,乃從汝南蔡父相問己能所宜。蔡父大奇其形貌,謂曰:「小史有封侯骨,當以經術進,努力為諸生學問。」方進既厭為小史,聞蔡父言,心喜,因病歸家,辭其後母,欲西至京師受經。母憐其幼,隨之長安,織屨以給方進讀,經博士受春秋。積十餘年,經學明習,徒眾日廣,諸儒稱之。以射策甲科為郎。二三歲,舉明經,遷議郎。

是時宿儒有清河胡常,與方進同經。常為先進,名譽出方進下,心害其能,論議不右方進。方進知之,候伺常大都授時,遣門下諸生至常所問大義疑難,因記其說。如是者久之,常知方進之宗讓己,內不自得,其後居士大夫之間未嘗不稱述方進,遂相親友。

河平中,方進轉為博士。數年,遷朔方刺史,居官不煩苛,所察應條輒舉,甚有威名。再三奏事,遷為丞相司直。從上甘泉,行馳道中,司隸校尉陳慶劾奏方進,沒入車馬。既至甘泉宮,會殿中,慶與廷尉范延壽語,時慶有章劾,自道:「行事以贖論,今尚書持我事來,當於此決。前我為尚書時,嘗有所奏事,忽忘之,留月餘。」方進於是舉劾慶曰:「案慶奉使刺舉大臣,故為尚書,知機事周密壹統,明主躬親不解。慶有罪未伏誅,無恐懼心,豫自設不坐之比。又暴揚尚書事,言遲疾無所在,虧損聖德之聰明,奉詔不謹,皆不敬,臣謹以劾。」慶坐免官。

會北地浩商為義渠長所捕,亡,長取其母,與豭豬連繫都亭下。商兄弟會賓客,自稱司隸掾、長安縣尉,殺義渠長妻子六人,亡。丞相、御史請遣掾史與司隸校尉、部刺史并力逐捕,察無狀者,奏可。司隸校尉涓勳奏言:「春秋之義,王人微者序乎諸侯之上,尊王命也。臣幸得奉使,以督察公卿以下為職,今丞相宣請遣掾史,以宰士督察天子奉使命大夫,甚誖逆順之理。宣本不師受經術,因事以立姦威。案浩商所犯,一家之禍耳,而宣欲專權作威,乃害於乃國,不可之大者。願下中朝特進列侯、將軍以下,正國法度。」議者以為丞相掾不宜移書督趣司隸。會浩商捕得伏誅,家屬徙合浦。

故事,司隸校尉位在司直下,初除,謁兩府,其有所會,居中二千石前,與司直並迎丞相、御史。初,方進新視事,而涓勳亦初拜為司隸,不肯謁丞相、御史大夫,後朝會相見,禮節又倨。方進陰察之,勳私過光祿勳辛慶忌,又出逢帝舅成都侯商道路,下車立,鹭過,乃就車。於是方進舉奏其狀,因曰:「臣聞國家之興,尊尊而敬長,爵位上下之禮,王道綱紀。春秋之義,尊上公謂之宰,海內無不統焉。丞相進見聖主,御坐為起,在輿為下。群臣宜皆承順聖化,以視四方。勳吏二千石,幸得奉使,不遵禮儀,輕謾宰相,賤易上卿,而又詘節失度,邪諂無常,色厲內荏。墮國體,亂朝廷之序,不宜處位。臣請下丞相免勳。」

時太中大夫平當給事中,奏言「方進國之司直,不自敕正以先群下,前親犯令行馳道中,司隸慶平心舉劾,方進不自責悔而內挾私恨,伺記慶之從容語言,以詆欺成罪。後丞相宣以一不道賊,請遣掾督趣司隸校尉,司隸校尉勳自奏暴於朝廷,今方進復舉奏勳。議者以為方進不以道德輔正丞相,苟阿助大臣,欲必勝立威,宜抑絕其原。勳素行公直,姦人所惡,可少寬假,使遂其功名。」上以方進所舉應科,不得用逆詐廢正法,遂貶勳為昌陵令。方進旬歲間免兩司隸,朝廷由是憚之。丞相宣甚器重焉,常誡掾史:「謹事司直,翟君必在相位,不久。」

是時起昌陵,營作陵邑,貴戚近臣子弟賓客多辜榷為姦利者,方進部掾史覆案,發大姦贓數千萬。上以為任公卿,欲試以治民,徙方進為京兆尹,博擊豪彊,京師畏之。時胡常為青州刺史,聞之,與方進書曰:「竊聞政令甚明,為京兆能,則恐有所不宜。」方進心知所謂,其後少弛威嚴。

居官三歲,永始二年遷御史大夫。數月,會丞相薛宣坐廣漢盜賊群起及太皇太后喪時三輔吏並徵發為姦,免為庶人。方進亦坐為京兆尹時奉喪事煩擾百姓,左遷執金吾。二十餘日,丞相官缺,群臣多舉方進,上亦器其能,遂擢方進為丞相,封高陵侯,食邑千戶。身既富貴,而後母尚在,方進內行修飾,供養甚篤。及後母終,既葬三十六日,除服起視事,以為身備漢相,不敢踰國家之制。為相公絜,請託不行郡國。持法刻深,舉奏牧守九卿,峻文深詆,中傷者尤多。如陳咸、朱博、蕭育、逢信、孫閎之屬,皆京師世家,以材能少歷牧守列卿,知名當世,而方進特立後起,十餘年間至宰相,據法以彈咸等,皆罷退之。

初咸最先進,自元帝初為御史中丞顯名朝廷矣。成帝初即位,擢為部刺史,歷楚國、北海、東郡太守。陽朔中,京兆尹王章譏切大臣,而薦琅邪太守馮野王可代大將軍王鳳輔政,東郡太守陳咸可御史大夫。是時方進甫從博士為刺史云。後方進為京兆尹,咸從南陽太守入為少府,與方進厚善。先是逢信已從高弟郡守歷京兆、太僕為衛尉矣,官簿皆在方進之右。及御史大夫缺,三人皆名卿,俱在選中,而方進得之。會丞相宣有事與方進相連,上使五二千石雜問丞相、御史,咸詰責方進,冀得其處,方進心恨。初大將軍鳳奏除陳湯為中郎,與從事。鳳薨後,從弟車騎將軍音代鳳輔政,亦厚湯。逢信、陳咸皆與湯善,湯數稱之於鳳、音所。久之,音薨,鳳弟成都侯商復為大司馬衛將軍輔政。商素憎陳湯,白其罪過,下有司案驗,遂免湯,徙敦煌。時方進新為丞相,陳咸內懼不安,乃令小冠杜子夏往觀其意,微自解說。子夏既過方進,揣知其指,不敢發言。居亡何,方進奏咸與逢信「邪枉貪汙,營私多欲。皆知陳湯姦佞傾覆,利口不軌,而親交賂遺,以求薦舉。後為少府,數饋遺湯。信、咸幸得備九卿,不思盡忠正身,內自知行辟亡功效,而官媚邪臣,欲以徼幸,苟得亡恥。孔子曰:『鄙夫可與事君也與哉!』咸、信之謂也。過惡暴見,不宜處位,臣請免以示天下。」奏可。

後二歲餘,詔舉方正直言之士,紅陽侯立舉咸對策,拜為光祿大夫給事中。方進復奏:「咸前為九卿,坐為貪邪免,自知罪惡暴陳,依託紅陽侯立徼幸,有司莫敢舉奏。冒濁苟容,不顧恥辱,不當蒙方正舉,備內朝臣。」并劾紅陽侯立選舉故不以實。有詔免咸,勿劾立。

後數年,皇太后姊子侍中衛尉定陵侯淳于長有罪,上以太后故,免官勿治罪。有司奏請遣長就國,長以金錢與立,立上封事為長求留曰:「陛下既託文以皇太后故,誠不可更有它計。」後長陰事發,遂下獄。方進劾立「懷姦邪,亂朝政,欲傾誤要主上,狡猾不道,請下獄。」上曰:「紅陽侯,朕之舅,不忍致法,遣就國。」於是方進復奏立黨友曰:「立素行積為不善,眾人所共知。邪臣自結,附託為黨,庶幾立與政事,欲獲其利。今立斥逐就國,所交結尤著者,不宜備大臣,為郡守。案後將軍朱博、鉅鹿太守孫閎、故光祿大夫陳咸與立交通厚善,相與為腹心,有背公死黨之信,欲相攀援,死而後已;皆內有不仁之性,而外有雋材,過絕於人,勇猛果敢,處事不疑,所居皆尚殘賊酷虐,苛刻慘毒以立威,而亡纖介愛利之風。天下所共知,愚者猶惑。孔子曰:『人而不仁如禮何!人而不仁如樂何!』言不仁之人,亡所施用;不仁而多材,國之患也。此三人皆內懷姦猾,國之所患,而深相與結,信於貴戚姦臣,此國家大憂,大臣所宜沒身而爭也。昔季孫行父有言曰:『見有善於君者愛之,若孝子之養父母也;見不善者誅之,若鷹鸇之逐鳥爵也。』翅翼雖傷,不避也。貴戚彊黨之眾誠難犯,犯之,眾敵並怨,善惡相冒。臣幸得備宰相,不敢不盡死。請免博、閎、咸歸故郡,以銷姦雄之黨,絕群邪之望。」奏可。咸既廢錮,復徙故郡,以憂發疾而死。

方進知能有餘,兼通文法吏事,以儒雅緣飭法律,號為通明相,天子甚器重之,奏事亡不當意,內求人主微指以固其位。初,定陵侯淳于長雖外戚,然以能謀議為九卿,新用事,方進獨與長交,稱薦之。及長坐大逆誅,諸所厚善皆坐長免,上以方進大臣,又素重之,為隱諱。方進內慚,上疏謝罪乞骸骨。上報曰:「定陵侯長已伏其辜,君雖交通,傳不云乎,朝過夕改,君子與之,君何疑焉?其專心壹意毋怠,近醫藥以自持。」方進乃起視事,條奏長所厚善京兆尹孫寶、右扶風蕭育,刺史二千石以上免二十餘人,其見任如此。

方進雖受穀梁,然好左氏傳、天文星曆,其左氏則國師劉歆,星曆則長安令田終術師也。厚李尋,以為議曹。為相九歲,綏和二年春熒惑守心,尋奏記言:「應變之權,君侯所自明。往者數白,三光垂象,變動見端,山川水泉,反理視患,民人訛謠,斥事感名。三者既效,可為寒心。今提揚眉,矢貫中,狼奮角,弓且張,金歷庫,土逆度,輔湛沒,火守舍,萬歲之期,近慎朝暮。上無惻怛濟世之功,下無推讓避賢之效,欲當大位,為具臣以全身,難矣!大責日加,安得但保斥逐之戮?闔府三百餘人,唯君侯擇其中,與盡節轉凶。」

方進憂之,不知所出。會郎賁麗善為星,言大臣宜當之。上乃召見方進。還歸,未及引決,上遂賜冊曰:「皇帝問丞相:君有孔子之慮,孟賁之勇,朕嘉與君同心一意,庶幾有成。惟君登位,於今十年,災害並臻,民被飢餓,加以疾疫溺死,關門牡開,失國守備,盜賊黨輩。吏民殘賊,毆殺良民,斷獄歲歲多前。上書言事,交錯道路,懷姦朋黨,相為隱蔽,皆亡忠慮,群下兇兇,更相嫉妒,其咎安在?觀君之治,無欲輔朕富民便安元元之念。間者郡國穀雖頗孰,百姓不足者尚眾,前去城郭,未能盡還,夙夜未嘗忘焉。朕惟往時之用,與今一也,百僚用度各有數。君不量多少,一聽群下言,用度不足,奏請一切增賦,稅城郭堧及園田,過更,算馬牛羊,增益鹽鐵,變更無常。朕既不明,隨奏許可,使議者以為不便,制詔下君,君云賣酒醪。後請止,未盡月復奏議令賣酒醪。朕誠怪君,何持容容之計,無忠固意,將何以輔朕帥道群下?而欲久蒙顯尊之位,豈不難哉!傳曰:『高而不危,所以長守貴也。』欲退君位,尚未忍。君其孰念詳計,塞絕姦原,憂國如家,務便百姓以輔朕。朕既已改,君其自思,強食慎職。使尚書令賜君上尊酒十石,養牛一,君審處焉。」

方進即日自殺。上祕之,遣九卿冊贈以丞相高陵侯印綬,賜乘輿祕器,少府供張,柱檻皆衣素。天子親臨弔者數至,禮賜異於它相故事。諡曰恭侯。長子宣嗣。

宣字太伯,亦明經篤行,君子人也。及方進在,為關都尉、南郡太守。

少子曰義。義字文仲,少以父任為郎,稍遷諸曹,年二十出為南陽都尉。宛令劉立與曲陽侯為婚,又素著名州郡,輕義年少。義行太守事,行縣至宛,丞相史在傳舍。立持酒肴謁丞相史,對飲未訖,會義亦往,外吏白都尉方至,立語言自若。須臾義至,內謁徑入,立乃走下。義既還,大怒,陽以他事召立至,以主守盜十金,賊殺不辜,部掾夏恢等收縛立,傳送鄧獄。恢亦以宛大縣,恐見篡奪,白義可因隨後行縣送鄧。義曰:「欲令都尉自送,則如勿收邪!」載環宛市乃送,吏民不敢動,威震南陽。

立家輕騎馳從武關入語曲陽侯,曲陽侯白成帝,帝以問丞相。方進遣吏敕義出宛令。宛令已出,吏還白狀。方進曰:「小兒未知為吏也,其意以為入獄當輒死矣。」

後義坐法免,起家而為弘農太守,遷河南太守,青州牧。所居著名,有父風烈。徙為東郡太守。

數歲,平帝崩,王莽居攝,義心惡之,乃謂姊子上蔡陳豐曰:「

新都侯攝天子位,號令天下,故擇宗室幼稚者以為孺子,依託周公輔成王之義,且以觀望,必代漢家,其漸可見。方今宗室衰弱,外無彊蕃,天下傾首服從,莫能亢扞國難。吾幸得備宰相子,身守大郡,父子受漢厚恩,義當為國討賊,以安社稷。欲舉兵西誅不當攝者,選宗室子孫輔而立之。設令時命不成,死國埋名,猶可以不慚於先帝。今欲發之,乃肯從我乎?」豐年十八,勇壯,許諾。

義遂與東郡都尉劉宇、嚴鄉侯劉信、信弟武平侯劉璜結謀。及東郡王孫慶素有勇略,以明兵法,徵在京師,義乃詐移書以重罪傳逮慶。於是以九月都試日斬觀令,因勒其車騎材官士,募郡中勇敢,部署將帥。嚴鄉侯信者,東平王雲子也。雲誅死,信兄開明嗣為王,薨,無子,而信子匡復立為王,故義舉兵并東平,立信為天子。義自號大司馬柱天大將軍,以東平王傅蘇隆為丞相,中尉皋丹為御史大夫,移檄郡國,言莽鴆殺孝平皇帝,矯攝尊號,今天子已立,共行天罰。郡國皆震,比至山陽,眾十餘萬。

莽聞之,大懼,乃拜其黨親輕車將軍成武侯孫建為奮武將軍,光祿勳成都侯王邑為虎牙將軍,明義侯王駿為強弩將軍,春王城門校尉王況為震威將軍,宗伯忠孝侯劉宏為奮衝將軍,中少府建威侯王昌為中堅將軍,中郎將震羌侯竇兄為奮威將軍,凡七人,自擇除關西人為校尉軍吏,將關東甲卒,發奔命以擊義焉。復以太僕武讓為積弩將軍屯函谷關,將作大匠蒙鄉侯逯並為橫野將軍屯武關,羲和紅休侯劉歆為揚武將軍屯宛,太保後丞丞陽侯甄邯為大將軍屯霸上,常鄉侯王惲為車騎將軍屯平樂館,騎都尉王晏為建威將軍屯城北,城門校尉趙恢為城門將軍,皆勒兵自備。

莽日抱孺子謂群臣而稱曰:「昔成王幼,周公攝政,而管蔡挾祿父以畔,今翟義亦挾劉信而作亂。自古大聖猶懼此,況臣莽之斗筲!」群臣皆曰:「不遭此變,不章聖德。」莽於是依周書作大誥,曰:

惟居攝二年十月甲子,攝皇帝若曰:大誥道諸侯王三公列侯于汝卿大夫元士御事。不弔,天降喪于趙、傅、丁、董。洪惟我幼沖孺子,當承繼嗣無疆大歷服事,予未遭其明悊能道民於安,況其能往知天命!熙!我念孺子,若涉淵水,予惟往求朕所濟度,奔走以傅近奉承高皇帝所受命,予豈敢自比於前人乎!天降威明,用寧帝室,遺我居攝寶龜。太皇太后以丹石之符,乃紹天明意,詔予即命居攝踐祚,如周公故事。

反虜故東郡太守翟義擅興師動眾,曰「有大難于西土,西土人亦不靖。」於是動嚴鄉侯信,誕敢犯祖亂宗之序。天降威遺我寶龜,固知我國有呰災,使民不安,是天反復右我漢國也。粵其聞日,宗室之俊有四百人,民獻儀九萬夫,予敬以終於此謀繼嗣圖功。我有大事,休,予卜并吉,故我出大將告郡太守諸侯相令長曰:「予得吉卜,予惟以汝于伐東郡嚴鄉逋播臣。」爾國君或者無不反曰:「難大,民亦不靜,亦惟在帝宮諸侯宗室,於小子族父,敬不可征。」帝不違卜,故予為沖人長思厥難曰:「烏虖!義、信所犯,誠動鰥寡,哀哉!」予遭天役遺,大解難於予身,以為孺子,不身自卹。

予義彼國君泉陵侯上書曰:「成王幼弱,周公踐天子位以治天下,六年,朝諸侯於明堂,制禮樂,班度量,而天下大服。太皇太后承順天心,成居攝之義。皇太子為孝平皇帝子,年在襁褓,宜且為子,知為人子道,令皇太后得加慈母恩。畜養成就,加元服,然後復予明辟。」

熙!為我孺子之故,予惟趙、傅、丁、董之亂,遏絕繼嗣,變剝適庶,危亂漢朝,以成三鹞,隊極厥命。烏虖!害其可不旅力同心戒之哉!予不敢僭上帝命。天休於安帝室,興我漢國,惟卜用克綏受茲命。今天其相民,況亦惟卜用!

太皇太后肇有元城沙鹿之右,陰精女主聖明之祥,配元生成,以興我天下之符,遂獲西王母之應,神靈之徵,以祐我帝室,以安我大宗,以紹我後嗣,以繼我漢功。厥害適統不宗元緒者,辟不違親,辜不避戚。夫豈不愛?亦惟帝室。是以廣立王侯,並建曾玄,俾屏我京師,綏撫宇內;傅徵儒生,講道於廷,論序乖繆,制禮作樂,同律度量,混壹風俗;正天地之位,昭郊宗之禮,定五畤廟祧,咸秩亡文;建靈臺,立明堂,設辟雍,張太學,尊中宗、高宗之號。昔我高宗崇德建武,克綏西域,以受白虎威勝之瑞,天地判合,乾坤序德。太皇太后臨政,有龜龍麟鳳之應,五德嘉符,相因而備。河圖雒書遠自昆侖,出於重野。古讖著言,肆今享實。此乃皇天上帝所以安我帝室,俾我成就洪烈也。烏虖!天用威輔漢始而大大矣。爾有惟舊人泉陵侯之言,爾不克遠省,爾豈知太皇太后若此勤哉!

天毖勞我成功所,予不敢不極卒安皇帝之所圖事。肆予告我諸侯王公列侯卿大夫元士御事:天輔誠辭,天其累我以民,予害敢不於祖宗安人圖功所終?天亦惟勞我民,若有疾,予害敢不於祖宗所受休輔?予聞孝子善繼人之意,忠臣善成人之事。予思若考作室,厥子堂而構之;厥父菑,厥子播而穫之。予害敢不於身撫祖宗之所受大命?若祖宗乃有效湯武伐厥子,民長其勸弗救。烏虖肆哉!諸侯王公列侯卿大夫元士御事,其勉助國道明!亦惟宗室之俊,民之表儀,迪知上帝命。況今天降定于漢國,惟大艱人翟義、劉信大逆,欲相伐於厥室,豈亦知命之不易乎?予永念曰天惟喪翟義、劉信,若嗇夫,予害敢不終予畝?天亦惟休於祖宗,予害其極卜,害敢不卜從?率寧人有旨疆土,況今卜并吉!故予大以爾東征,命不僭差,卜陳惟若此。

乃遣大夫桓譚等班行諭告當反位孺子之意。還,封譚為明告里附城。

諸將東破陳留菑,與義會戰,破之,斬劉璜首。莽大喜,復下詔曰:「太皇太后遭家不造,國統三絕,絕輒復續,恩莫厚焉,信莫立焉。孝平皇帝短命蚤崩,幼嗣孺沖,詔予居攝。予承明詔,奉社稷之任,持大宗之重,養六尺之託,受天下之寄,戰戰兢兢,不敢安息。伏念太皇太后惟經藝分析,王道離散,漢家制作之業獨未成就,故博徵儒士,大興典制,備物致用,立功成器,以為天下利。王道粲然,基業既著,千載之廢,百世之遺,於今乃成,道德庶幾於唐虞,功烈比齊於殷周。今翟義、劉信等謀反大逆,流言惑眾,欲以篡位,賊害我孺子,罪深於管蔡,惡甚於禽獸。信父故東平王雲,不孝不謹,親毒殺其父思王,名曰鉅鼠,後雲竟坐大逆誅死。義父故丞相方進,險詖陰賊,兄宣靜言令色,外巧內嫉,所殺鄉邑汝南者數十人。今積惡二家,迷惑相得,此時命當殄,天所滅也。義始發兵,上書言宇、信等與東平相輔謀反,執捕械繫,欲以威民,先自相被以反逆大惡,轉相捕械,此其破殄之明證也。已捕斬斷信二子穀鄉侯章、德廣侯鮪,義母練、兄宣、親屬二十四人皆磔暴于長安都巿四通之衢。當其斬時,觀者重疊,天氣和清,可謂當矣。命遣大將軍共行皇天之罰,討海內之讎,功效著焉,予甚嘉之。司馬法不云乎?『賞不踰時。』欲民速睹為善之利也。今先封車騎都尉孫賢等五十五人皆為列侯,戶邑之數別下。遣使者持黃金印、赤韍縌、朱輪車,即軍中拜授。」因大赦天下。

於是吏士精銳遂攻圍義於圉城,破之,義與劉信棄軍庸亡。至固始界中捕得義,尸磔陳都巿。卒不得信。

初,三輔聞翟義起,自茂陵以西至汧二十三縣盜賊並發,趙明、霍鴻等自稱將軍,攻燒官寺,殺右輔都尉及斄令,劫略吏民,眾十餘萬,火見未央宮前殿。莽晝夜抱孺子禱宗廟。復拜衛尉王級為虎賁將軍,大鴻臚望鄉侯閻遷為折衝將軍,與甄邯、王晏西擊趙明等。正月,虎牙將軍王邑等自關東還,便引兵西。彊弩將軍王駿以無功免,揚武將軍劉歆歸故官。復以邑弟侍中王奇為揚武將軍,城門將軍趙恢為彊弩將軍,中郎將李棽為厭難將軍,復將兵西。二月,明等殄滅,諸縣悉平,還師振旅。莽乃置酒白虎殿,勞饗將帥,大封拜。先是益州蠻夷及金城塞外羌反畔,時州郡擊破之。莽乃并錄,以小大為差,封侯伯子男凡三百九十五人,曰「皆以奮怒,東指西擊,羌寇蠻盜,反虜逆賊,不得旋踵,應時殄滅,天子咸服」之功封云。莽於是自謂大得天人之助,至其年十二月,遂即真矣。

初,義所收宛令劉立聞義舉兵,上書願備軍吏為國討賊,內報私怨。莽擢立為陳留太守,封明德侯。

始,義兄宣居長安,先義未發,家數有怪,夜聞哭聲,聽之不知所在。宣教授諸生滿堂,有狗從外入,齧其中庭群鴈數十,比驚救之,已皆斷頭。狗走出門,求不知處。宣大惡之,謂後母曰:「東郡太守文仲素俶儻,今數有惡怪,恐有妄為而大禍至也。大夫人可歸,為棄去宣家者以避害。」母不肯去,後數月敗。

莽盡壞義第宅,汙池之。發父方進及先祖冢在汝南者,燒其棺柩,夷滅三族,誅及種嗣,至皆同坑,以棘五毒并葬之。而下詔曰:「蓋聞古者伐不敬,取其鯨鯢築武軍,封以為大戮,於是乎有京觀以懲淫慝。乃者反虜劉信、翟義誖逆作亂於東,而芒竹群盜趙明、霍鴻造逆西土,遣武將征討,咸伏其辜。惟信、義等始發自濮陽,結姦無鹽,殄滅於圉。趙明依阻槐里環隄,霍鴻負倚盩厔芒竹,咸用破碎,亡有餘類。其取反虜逆賊之鯨鯢,聚之通路之旁,濮陽、無鹽、圉、槐里、盩厔凡五所,各方六丈,高六尺,築為武軍,封以為大戮,薦樹之棘。建表木,高丈六尺。書曰『反虜逆賊鯨鯢』,在所長吏常以秋循行,勿令壞敗,以懲淫慝焉。」

初,汝南舊有鴻隙大陂,郡以為饒,成帝時,關東數水,陂溢為害。方進為相,與御史大夫孔光共遣掾行事,以為決去陂水,其地肥美,省隄防費而無水憂,遂奏罷之。及翟氏滅,鄉里歸惡,言方進請陂下良田不得而奏罷陂云。王莽時常枯旱,郡中追怨方進,童謠曰:「壞陂誰?翟子威。飯我豆食羹芋魁。反乎覆,陂當復。誰云者?兩黃鵠。」

司徒掾班彪曰:「丞相方進以孤童攜老母,羈旅入京師,身為儒宗,致位宰相,盛矣。當莽之起,蓋乘天威,雖有賁育,奚益於敵?義不量力,懷忠憤發,以隕其宗,悲夫!」


\end{pinyinscope}