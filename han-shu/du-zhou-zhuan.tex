\article{杜周傳}

\begin{pinyinscope}
杜周,南陽杜衍人也。義縱為南陽太守,以周為爪牙,薦之張湯,為廷尉史。使案邊失亡,所論殺甚多。奏事中意,任用,與減宣更為中丞者十餘歲。

周少言重遲,而內深次骨。宣為左內史,周為廷尉,其治大抵放張湯,而善候司。上所欲擠者,因而陷之;上所欲釋,久繫待問而微見其冤狀。客有謂周曰:「

君為天下決平,不循三尺法,專以人主意指為獄,獄者固如是乎?」周曰:「三尺安出哉?前主所是著為律,後主所是疏為令;當時為是,何古之法乎!」

至周為廷尉,詔獄亦益多矣。二千石繫者新故相因,不減百餘人。郡吏大府舉之廷尉,一歲至千餘章。章大者連逮證案數百,小者數十人;遠者數千里,近者數百里。會獄,吏因責如章告劾,不服,以掠笞定之。於是聞有逮證,皆亡匿。獄久者至更數赦十餘歲而相告言,大氐盡詆以不道,以上廷尉及中都官,詔獄逮至六七萬人,吏所增加十有餘萬。

周中廢,後為執金吾,逐捕桑弘羊、衛皇后昆弟子刻深,上以為盡力無私,遷為御史大夫。

始周為廷史,有一馬,及久任事,列三公,而兩子夾河為郡守,家訾累巨萬矣。治皆酷暴,唯少子延年行寬厚云。

延年字幼公,亦明法律。昭帝初立,大將軍霍光秉政,以延年三公子,吏材有餘,補軍司空。始元四年,益州蠻夷反,延年以校尉將南陽士擊益州,還,為諫大夫。左將軍上官桀父子與蓋主、燕王謀為逆亂,假稻田使者燕倉知其謀,以告大司農楊敞。敞惶懼,移病,以語延年。延年以聞,桀等伏辜。延年封為建平侯。

延年本大將軍霍光吏,首發大姦,有忠節,由是擢為太僕右曹給事中。光持刑罰嚴,延年輔之以寬。治燕王獄時,御史大夫桑弘羊子遷亡,過父故吏侯史吳。後遷捕得,伏法。會赦,侯史吳自出繫獄,廷尉王平與少府徐仁雜治反事,皆以為桑遷坐父謀反而侯史吳臧之,非匿反者,乃匿為隨者也。即以赦令除吳罪。後侍御史治實,以桑遷通經術,知父謀反而不諫爭,與反者身無異;侯史吳故三百石吏,首匿遷,不與庶人匿隨從者等,吳不得赦。奏請覆治,劾廷尉、少府縱反者。少府徐仁即丞相車千秋女婿也,故千秋數為侯史吳言。恐光不聽,千秋即召中二千石、博士會公車門,議問吳法。議者知大將軍指,皆執吳為不道。明日,千秋封上眾議,光於是以千秋擅召中二千石以下,外內異言,遂下廷尉平、少府仁獄。朝廷皆恐丞相坐之。延年乃奏記光爭,以為「吏縱罪人,有常法,今更詆吳為不道,恐於法深。又丞相素無所守持,而為好言於下,盡其素行也。至擅召中二千石,甚無狀。延年愚,以為丞相久故,及先帝用事,非有大故,不可棄也。間者民頗言獄深,吏為峻詆,今丞相所議,又獄事也,如是以及丞相,恐不合眾心。群下讙譁,庶人私議,流言四布,延年竊重將軍失此名於天下也!」光以廷尉、少府弄法輕重,皆論棄市,而不以及丞相,終與相竟。延年論議持平,合和朝廷,皆此類也。

見國家承武帝奢侈師旅之後,數為大將軍光言:「年歲比不登,流民未盡還,宜修孝文時政,示以儉約寬和,順天心,說民意,年歲宜應。」光納其言,舉賢良,議罷酒榷鹽鐵,皆自延年發之。吏民上書言便宜,有異,輒下延年平處復奏。言可官試者,至為縣令,或丞相、御史除用,滿歲以狀聞,或抵其罪法,常與兩府及廷尉分章。

昭帝末,寢疾,徵天下名醫,延年典領方藥。帝崩,昌邑王即位,廢,大將軍光、車騎將軍張安世與大臣議所立。時宣帝養於掖廷,號皇曾孫,與延年中子佗相愛善,延年知曾孫德美,勸光、安世立焉。宣帝即位,褒賞大臣,延年以定策安宗廟,益戶二千三百,與始封所食邑凡四千三百戶。詔有司論定策功,大司馬大將軍光功德過太尉絳侯周勃,車騎將軍安世、丞相楊敞功比丞相陳平,前將軍韓增、御史大夫蔡誼功比潁陰侯灌嬰,太僕杜延年功比朱虛侯劉章,後將軍趙充國、大司農田延年、少府史樂成功比典客劉揭,皆封侯益土。

延年為人安和,備於諸事,久典朝政,上任信之,出即奉駕,人給事中,居九卿位十餘年,賞賜賂遺,訾數千萬。

霍光薨後,子禹與宗族謀反,誅。上以延年霍氏舊人,欲退之,而丞相魏相奏延年素貴用事,官職多姦。遣吏考案,但得苑馬多死,官奴婢乏衣食,延年坐免官,削戶二千。後數月,復召拜為北地太守。延年以故九卿外為邊吏,治郡不進,上以璽書讓延年。延年乃選用良吏,捕繫豪強,郡中清靜。居歲餘,上使謁者賜延年璽書,黃金二十斤,徙為西河太守,治甚有名。五鳳中,徵入為御史大夫。延年居父官府,不敢當舊位,坐臥皆易其處。是時四夷和,海內平,延年視事三歲,以老病乞骸骨,天子優之,使光祿大夫持節賜延年黃金百斤、牛酒,加致醫藥。延年遂稱

疾篤。賜安車駟馬,罷就第。後數月薨,諡曰敬侯,子緩嗣。

緩少為郎,本始中以校尉從蒲類將軍擊匈奴,還為諫大夫,遷上谷都尉,雁門太守。父延年薨,徵視喪事,拜為太常,治諸陵縣,每冬月封具獄日,常去酒省食,官屬稱其有恩。元帝初即位,穀貴民流,永光中西羌反,緩輒上書入錢穀以助用,前後數百萬。

緩六弟,五人至大官,少弟熊歷五郡二千石,三州牧刺史,有能名,唯中弟欽官不至而最知名。

欽字子夏,少好經書,家富而目偏盲,故不好為吏。茂陵杜鄴與欽同姓字,俱以材能稱京師,故衣冠謂欽為「盲杜子夏」以相別。欽惡以疾見詆,乃為小冠,高廣財二寸,由是京師更謂欽為「小冠杜子夏」,而鄴為「大冠杜子夏」云。

時帝舅大將軍王鳳以外戚輔政,求賢知自助。鳳父頃侯禁與欽兄緩相善,故鳳深知欽能,奏請欽為大將軍軍武庫令。職閒無事,欽所好也。

欽為人深博有謀。自上為太子時,以好色聞,及即位,皇太后詔采良家女。欽因是說大將軍鳳曰:「禮壹娶九女,所以極陽數,廣嗣重祖也;必鄉舉求窈窕,不問華色,所以助德理內也;娣姪雖缺不復補,所以養壽塞爭也。故后妃有貞淑之行,則胤嗣有賢聖之君;制度有威儀之節,則人君有壽考之福。廢而不由,則女德不厭;女德不厭,則壽命不究於高年。《書》云『或四三年』,言失欲之生害也。男子五十,好色未衰;婦人四十,容貌改前。以改前之容侍於未衰之年,而不以禮為制,則其原不可救而後徠異態;後徠異態,則正后自疑而支庶有間適之心。是以晉獻被納讒之謗,申生蒙無罪之辜。今聖主富於春秋,未有適嗣,方鄉術入學,未親后妃之議。將軍輔政,宜因始初之隆,建九女之制,詳擇有行義之家,求淑女之質,毋必有聲色音技能,為萬世大法。夫少,戒之在色,小卞之作,可為寒心。唯將軍常以為憂。」

鳳白之太后,太后以為故事無有。欽復重言:「《詩》云『殷監不遠,在夏后氏之世』。刺戒者至迫近,而省聽者常怠忽,可不慎哉!前言九女,略陳其禍福,甚可悼懼,竊恐將軍不深留意。后妃之制,夭壽治亂存亡之端也。跡三代之季世,覽宗、宣之饗國,察近屬之符驗,禍敗曷常不由女德?是以佩玉晏鳴,關雎歎之,知好色之伐性短年,離制度之生無厭,天下將蒙化,陵夷而成俗也。故詠淑女,幾以配上,忠孝之篤,仁厚之作也。夫君親壽尊,國家治安,誠臣子之至願,所當勉之也。《易》曰:『正其本,萬物理。』凡事論有疑未可立行者,求之往古則典刑無,考之來今則吉凶同,卒搖易之則民心惑,若是者誠難施也。今九女之制,合於往古,無害於今,不逆於民心,至易行也,行之至有福也,將軍輔政而不蚤定,非天下之所望也。唯將軍信臣子之願,念關雎之思,逮委政之隆,及始初清明,為漢家建無窮之基,誠難以忽,不可以遴。」鳳不能自立法度,循故事而已。會皇太后女弟司馬君力與欽兄子私通,事上聞,欽慚懼,乞骸骨去。

後有日蝕地震之變,詔舉賢良方正能直言士,合陽侯梁放舉欽。欽上對曰:「陛下畏天命,悼變異,延見公卿,舉直言之士,將以求天心,跡得失也。臣欽愚戇,經術淺薄,不足以奉大對。臣聞日蝕地震,陽微陰盛也。臣者,君之陰也;子者,父之陰也;妻者,夫之陰也;夷狄者,中國之陰也。春秋日蝕三十六,地震五,或夷狄侵中國,或政權在臣下,或婦乘夫,或臣子背君父,事雖不同,其類一也。臣竊觀人事以考變異,則本朝大臣無不自安之人,外戚親屬無乖剌之心,關東諸侯無強大之國,三垂蠻夷無逆理之節;殆為後宮。何以言之?日以戊申蝕,時加未。戊夫,土也。土者,中宮之部也。其夜地震未央宮殿中,此必適妾將有爭寵相害而為患者,唯陛下深戒之。變感以類相應,人事失於下,變象見於上。能應之以德,則異咎消亡;不能應之以善,則禍敗至。高宗遭雊雉之戒,飭己正事,享百年之壽,殷道復興,要在所以應之。應之非誠不立,非信不行。宋景公小國之諸侯耳,有不忍移禍之誠,出人君之言三,熒惑為之退舍。以陛下聖明,內推至誠,深思天變,何應而不感?何搖而不動?孔子曰:『仁遠乎哉!』唯陛下正后妾,抑女寵,防奢泰,去佚游,躬節儉,親萬事,數御安車,由輦道,親二宮之饔膳,致晨昏之定省。如此,即堯舜不足與比隆,咎異何足消滅!如不留聽於庶事,不論材而授位,殫天下之財以奉淫侈,匱萬姓之力以從耳目,近諂諛之人而遠公方,信讒賊之臣以誅忠良,賢俊失在巖穴,大臣怨於不以,雖無變異,社稷之憂也。天下至大,萬事至眾,祖業至重,誠不可以佚豫為,不可以奢泰持也。唯陛下忍無益之欲,以全眾庶之命。臣欽愚戇,言不足采。」

其夏,上盡召直言之士詣白虎殿對策,策曰:「天地之道何貴?王者之法何如?六經之義何上?人之行何先?取人之術何以?當世之治何務?各以經對。」

欽對曰:「臣聞天道貴信,地道貴貞;不信不貞,萬物不生。生,天地之所貴也。王者承天地之所生,理而成之,昆蟲草木靡不得其所。王者法天地,非仁無以廣施,非義無以正身;克己就義,恕以及人,六經之所上也。不孝,則事君不忠,蒞官不敬,戰陳無勇,朋友不信。孔子曰:『孝無終始,而患不及者,未之有也。』孝,人行之所先也。觀本行於鄉黨,考功能於官職,達觀其所舉,富觀其所予,窮觀其所不為,乏觀其所不取,近觀其所為,遠觀其所主。孔子曰:『視其所以,觀其所由,察其所安,人焉廋哉?』取人之術也。殷因於夏尚質,周因於殷尚文,今漢家承周秦之敝,宜抑文尚質,廢奢長儉,表實去偽。孔子曰『惡紫之奪朱』,當世治之所務也。臣竊有所憂,言之則拂心逆指,不言則漸日長,為禍不細,然小臣不敢廢道而求從,違忠而耦意。臣聞玩色無厭,必生好憎之心;好憎之心生,則愛寵偏於一人;愛寵偏於一人,則繼嗣之路不廣,而嫉妒之心興矣。如此,則匹婦之說,不可勝也。唯陛下純德普施,無欲是從,此則眾庶咸說,繼嗣日廣,而海內長安。萬事之是非何足備言!」

欽以前事病,賜帛罷,後為議郎,復以病免。

徵詣大將軍莫府,國家政謀,鳳常與欽慮之。數稱達名士王駿、韋安世、王延世等,救解馮野王、王尊、胡常之罪過,及繼功臣絕世,填撫四夷,當世善政,多出於欽者。見鳳專政泰重,戒之曰:「昔周公身有至聖之德,屬有叔父之親,而成王有獨見之明,無信讒之聽,然管蔡流言而周公懼。穰侯,昭王之舅也,權重於秦,威震鄰敵,有旦莫偃伏之愛,心不介然有間,然范雎起徒步,由異國,無雅信,開一朝之說,而穰侯就封。及近者武安侯之見退,三事之跡,相去各數百歲,若合符節,甚不可不察。願將軍由周公之謙懼,損穰侯之威,放武安之欲,毋使范雎之徒得間其說。」

頃之,復日蝕,京兆尹王章上封事求見,果言鳳專權蔽主之過,宜廢勿用,以應天變。於是天子感悟,召見章,與議,欲退鳳。鳳甚憂懼,欽令鳳上疏謝罪,乞骸骨,文指甚哀。太后涕泣為不食。上少而親倚鳳,亦不忍廢,復起鳳就位。鳳心慚,稱病篤,欲遂退。欽復說之曰:「將軍深悼輔政十年,變異不已,故乞骸骨,歸咎於身,刻己自責,至誠動眾,愚知莫不感傷。雖然,是無屬之臣,執進退之分,絜其去就之節者耳,非主上所以待將軍,非將軍所以報主上也。昔周公雖老,猶在京師,明不離成周,示不忘王室也。仲山父異姓之臣,無親於宣,就封於齊,猶歎息永懷,宿夜徘徊,不忍遠去,況將軍之於主上,主上之與將軍哉!夫欲天下治安變異之意,莫有將軍,主上照然知之,故攀援不遣,書稱『

公毋困我!』唯將軍不為四國流言自疑於成王,以固至忠。」鳳復起視事。上令尚書劾奏京兆尹章,章死詔獄。語在元后傳。

章既死,眾庶冤之,以譏朝廷。欽欲救其過,復說鳳曰:「京兆尹章所坐事密,吏民見章素好言事,以為不坐官職,疑其以日蝕見對有所言也。假令章內有所犯,雖陷正法,事不暴揚,自京師不曉,況於遠方。恐天下不知章實有罪,而以為坐言事也。如是,塞爭引之原,損寬明之德。欽愚以為宜因章事舉直言極諫,並見郎從官展盡其意,加於往前,以明示四方,使天下咸知主上聖明,不以言罪下也。若此,則流言消釋,疑惑著明。」鳳白行其策。欽之補過將美,皆此類也。

優游不仕,以壽終。欽子及昆弟支屬至二千石者且十人。欽兄緩前免太常,以列侯奉朝請,成帝時乃薨,子業嗣。

業有材能,以列侯選,復為太常。數言得失,不事權貴,與丞相翟方進、衛尉定陵侯淳于長不平。後業坐法免官,復為函谷關都尉。會定陵侯長有罪,當就國,長舅紅陽侯立與業書曰:「誠哀老姊垂白,隨無狀子出關,願勿復用前事相侵。」定陵侯既出關,伏罪復發,下雒陽獄。丞相史搜得紅陽侯書,奏業聽請,不敬,坐免就國。

其春,丞相方進薨,業上書言:「方進本與長深結厚,更相稱薦,長陷大惡,獨得不坐,苟欲障塞前過,不為陛下廣持平例,又無恐懼之心,反因時信其邪辟,報睚眥怨。故事,大逆朋友坐免官,無歸故郡者,今在長者歸故郡,已深一等;紅陽侯立坐子受長貨賂故就國耳,非大逆也,而方進復奏立黨友後將軍朱博、鉅鹿太守孫宏、故少府陳咸,皆免官,歸咸故郡。刑罰無平,在方進之筆端,眾庶莫不疑惑,皆言孫宏不與紅陽侯相愛。宏前為中丞時,方進為御史大夫,舉掾隆可侍御史,宏奏隆前奉使欺謾,不宜執法近侍,方進以此怨宏。又方進為京兆尹時,陳咸為少府,在九卿高弟,陛下所自知也。方進素與司直師丹相善,臨御史大夫缺,使丹奏咸為姦利,請案驗,卒不能有所得,而方進果自得御史大夫。為丞相,即時詆欺,奏免咸,復因紅陽侯事歸咸故郡。眾人皆言國家假方進權太甚。案師丹行能無異,及光祿勳許商被病殘人,皆但以附從方進,嘗獲尊官。丹前親屬邑子丞相史能使巫下神,為國求福,幾獲大利。幸賴陛下至明,遣使者毛莫如先考驗,卒得其姦,皆坐死。假令丹知而白之,此誣罔罪也;不知而白之,是背經術惑左道也:二者皆在大辟,重於朱博、孫宏、陳咸所坐。方進終不舉白,專作威福,阿黨所厚,排擠英俊,託公報私,橫厲無所畏忌,欲以熏轑天下。天下莫不望風而靡,自尚書近臣皆結舌杜口,骨肉親屬莫不股栗。威權泰盛而不忠信,非所以安國家也。今聞方進卒病死,不以尉示天下,反復賞賜厚葬,唯陛下深思往事,以戒來今。」

會成帝崩,哀帝即位,業復上書言:「王氏世權日久,朝無骨骾之臣,宗室諸侯微弱,與繫囚無異,自佐史以上至於大吏皆權臣之黨。曲陽侯根前為三公輔政,知趙昭儀殺皇子,不輒白奏,反與趙氏比周,恣意妄行,譖愬故許后,被加以非罪,誅破諸許族,敗元帝外家。內嫉妒同產兄姊紅陽侯立及淳于氏,皆老被放棄。新喋血京師,威權可畏。高陽侯薛宣有不養母之名,安昌侯張禹姦人之雄,惑亂朝廷,使先帝負謗於海內,尤不可不慎。陛下初即位,謙讓未皇,孤獨特立,莫可據杖,權臣易世,意若探湯。宜蚤以義割恩,安百姓心。竊見朱博忠信勇猛,材略不世出,誠國家雄俊之寶臣也,宜徵博置左右,以填天下。此人在朝,則陛下可高枕而臥矣。昔諸呂欲危劉氏,賴有高祖遺臣周勃、陳平尚存,不者,幾為姦臣笑。」

業又言宜為恭王立廟京師,以章孝道。時高昌侯董宏亦言宜尊帝母定陶王丁后為帝太后。大司空師丹等劾宏誤朝不道,坐免為庶人,業復上書訟宏。前後所言皆合指施行,朱博果見拔用。業由是徵,復為太常。歲餘,左遷上黨都尉。會司隸奏業為太常選舉不實,業坐免官,復就國。

哀帝崩,王莽秉政,諸前議立廟尊號者皆免,徙合浦。業以前罷黜,故見闊略,憂恐,發病死。業成帝初尚帝妹潁邑公主,主無子,薨,業家上書求還京師與主合葬,不許,而賜諡曰荒侯,傳子至孫絕。初,杜周武帝時徙茂陵,至延年徙杜陵云。

贊曰:張湯、杜周並起文墨小吏,致位三公,列於酷吏。而俱有良子,德器自過,爵位尊顯,繼世立朝,相與提衡,至於建武,杜氏爵乃獨絕。跡其福祚,元功儒林之後莫能及也。自謂唐杜苗裔,豈其然乎?及欽浮沈當世,好謀而成,以建始之初深陳女戒,終如其言,庶幾乎關雎之見微,非夫浮華博習之徒所能規也。業因勢而抵稳,稱朱博,毀師丹,愛憎之議可不畏哉!


\end{pinyinscope}