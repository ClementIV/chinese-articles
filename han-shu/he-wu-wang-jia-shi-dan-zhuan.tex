\article{何武王嘉師丹傳}

\begin{pinyinscope}
何武字君公,蜀郡郫縣人也。宣帝時,天下和平,四夷賓服,神爵、五鳳之間婁蒙瑞應。而益州刺史王襄使辯士王褒頌漢德,作中和、樂職、宣布詩三篇。武年十四五,與成都楊覆眾等共習歌之。是時,宣帝循武帝故事,求通達茂異士,召見武等於宣室。上曰:「此盛德之事,吾何足以當之哉!」以褒為待詔,武等賜帛罷。

武詣博士受業,治易。以射策甲科為郎,與翟方進交志相友。光祿勳舉四行,遷為鄠令,坐法免歸。

武兄弟五人,皆為郡吏,郡縣敬憚之。武弟顯家有巿籍,租常不入,縣數負其課。巿嗇夫求商捕辱顯家,顯怒,欲以吏事中商。武曰:「以吾家租賦繇役不為眾先,奉公吏不亦宜乎!」武卒白太守,召商為卒吏,州里聞之皆服焉。

久之,太僕王音舉武賢良方正,徵對策,拜為諫大夫,遷揚州刺史。所舉奏二千石長吏必先露章,服罪者為虧除,免之而已;不服,極法奏之,抵罪或至死。

九江太守戴聖,禮經號小戴者也,行治多不法,前刺史以其大儒,優容之。及武為刺史,行部錄囚徒,有所舉以屬郡。聖曰:「後進生何知,乃欲亂人治!」皆無所決。武使從事廉得其罪,聖懼,自免。後為博士,毀武於朝廷。武聞之,終不揚其惡。而聖子賓客為群盜,得,繫廬江,聖自以子必死。武平心決之,卒得不死。自是後,聖慚服。武每奏事至京師,聖未嘗不造門謝恩。

武為刺史,二千石有罪,應時舉奏,其餘賢與不肖敬之如一,是以郡國各重其守相,州中清平。行部必先即學官見諸生,試其誦論,問以得失,然後入傳舍,出記問墾田頃畝,五穀美惡,已乃見二千石,以為常。

初,武為郡吏時,事太守何壽。壽知武有宰相器,以其同姓故厚之。後壽為大司農,其兄子為廬江長史。時武奏事在邸,壽兄子適在長安,壽為具召武弟顯及故人楊覆眾等,酒酣,見其兄子,曰:「此子揚州長史,材能駑下,未嘗省見。」顯等甚慚,退以謂武,武曰:「刺史古之方伯,上所委任,一州表率也,職在進善退惡。吏治行有茂異,民有隱逸,乃當召見,不可有所私問。」顯、覆眾強之,不得已召見,賜卮酒。歲中,廬江太守舉之。其守法見憚如此。

為刺史五歲,入為丞相司直,丞相薛宣敬重之。出為清河太守,數歲,坐郡中被災害什四以上免。久之,大司馬曲陽侯王根薦武,徵為諫大夫。遷兗州刺史,入為司隸校尉,徙京兆尹。二歲,坐舉方正所舉者召見槃辟雅拜,有司以為詭眾虛偽。武坐左遷楚內史,遷沛郡太守,復入為廷尉。綏和三年,御史大夫孔光左遷廷尉,武為御史大夫。成帝欲修辟雍,通三公官,即改御史大夫為大司空。武更為大司空,封氾鄉侯,食邑千戶。氾鄉在琅邪不其,哀帝初即位,褒賞大臣,更以南陽犨之博望鄉為氾鄉侯國,增邑千戶。

武為人仁厚,好進士,獎稱人之善。為楚內史厚兩龔,在沛郡厚兩唐,及為公卿,薦之朝廷。此人顯於世者,何侯力也,世以此多焉。然疾朋黨,問文吏必於儒者,問儒者必於文吏,以相參檢。欲除吏,先為科例以防請託。其所居亦無赫赫名,去後常見思。

及為御史大夫司空,與丞相方進共奏言:「往者諸侯王斷獄治政,內史典獄事,相總綱紀輔王,中尉備盜賊。今王不斷獄與政,中尉官罷,職并內史,郡國守相委任,所以壹統信,安百姓也。今內史位卑而權重,威職相踰,不統尊者,難以為治。臣請相如太守,內史如都尉,以順尊卑之序,平輕重之權。」制曰:「可。」以內史為中尉。初武為九卿時,奏言宜置三公官,又與方進共奏罷刺史,更置州牧,後皆復復故,語在朱博傳。唯內史事施行。

多所舉奏,號為煩碎,不稱賢公。功名略比薛宣,其材不及也,而經術正直過之。武後母在郡,遣吏歸迎。會成帝崩,吏恐道路有盜賊,後母留止,左右或譏武事親不篤。哀帝亦欲改易大臣,遂策免武曰:「君舉錯煩苛,不合眾心,孝聲不聞,惡名流行,無以率示四方。其上大司空印綬,罷歸就國。」後五歲,諫大夫鮑宣數稱冤之,天子感丞相王嘉之對,而高安侯董賢亦薦武,武由是復徵為御史大夫。月餘,徙為前將軍。

先是,新都侯王莽就國,數年,上以太皇太后故徵莽還京師。莽從弟成都侯王邑為侍中,矯稱太皇太后指白哀帝,為莽求特進給事中。哀帝復請之,事發覺。太后為謝,上以太后故不忍誅之,左遷邑為西河屬國都尉,削千戶。後有詔舉大常,莽私從武求舉,武不敢舉。後數月,哀帝崩,太后即日引莽入,收大司馬董賢印綬,詔有司舉可大司馬者。莽故大司馬,辭位辟丁、傅,眾庶稱以為賢,又太后近親,自大司徒孔光以下舉朝皆舉莽。武為前將軍,素與左將軍公孫祿相善,二人獨謀,以為往時孝惠、孝昭少主之世,外戚呂、霍、上官持權,幾危社稷,今孝成、孝哀比世無嗣,方當選立親近輔幼主,不宜令異姓大臣持權,親疏相錯,為國計便。於是武舉公孫祿可大司馬,而祿亦舉武。太后竟自用莽為大司馬。莽風有司劾奏武、公孫祿互相稱舉,皆免。

武就國後,莽寖盛,為宰衡,陰誅不附己者。元始三年,呂寬等事起。時大司空甄豐承莽風指,遣使者乘傳案治黨與,連引諸所欲誅,上黨鮑宣,南陽彭偉、杜公子,郡國豪桀坐死者數百人。武在見誣中,大理正檻車徵武,武自殺。眾人多冤武者,莽欲厭眾意,令武子況嗣為侯,諡武曰剌侯。莽篡位,免況為庶人。

王嘉字公仲,平陵人也。以明經射策甲科為郎,坐戶殿門失闌免。光祿勳于永除為掾,察廉為南陵丞,復察廉為長陵尉。鴻嘉中,舉敦朴能直言,召見宣室,對政事得失,超遷太中大夫。出為九江、河南太守,治甚有聲。徵入為大鴻臚,徙京兆尹,遷御史大夫。建平三年代平當為丞相,封新甫侯,加食邑千一百戶。

嘉為人剛直嚴毅有威重,上甚敬之。哀帝初立,欲匡成帝之政,多所變動,嘉上疏曰:

臣聞聖王之功在於得人。孔子曰:「材難,不其然與!」「故繼世立諸侯,象賢也。」雖不能盡賢,天子為擇臣,立命卿以輔之。居是國也,累世尊重,然後士民之眾附焉,是以教化行而治功立。今之郡守重於古諸侯,往者致選賢材,賢材難得,拔擢可用者,或起於囚徒。昔魏尚坐事繫,文帝感馮唐之言,遣使持節赦其罪,拜為雲中太守,匈奴忌之。武帝擢韓安國於徒中,拜為梁內史,骨肉以安。張敞為京兆尹,有罪當免,黠吏知而犯敞,敞收殺之,其家自冤,使者覆獄,劾敞賊殺人,上逮捕不下,會免,亡命數十日,宣帝徵敞拜為冀州刺史,卒獲其用。前世非私此三人,貪其材器有益於公家也。

孝文時,吏居官者或長子孫,以官為氏,倉氏、庫氏則倉庫吏之後也。其二千石長吏亦安官樂職,然後上下相望,莫有苟且之意。其後稍稍變易,公卿以下傳相促急,又數改更政事,司隸、部刺史察過悉劾,發揚陰私,吏或居官數月而退,送故迎新,交錯道路。中材苟容求全,下材懷危內顧,壹切營私者多。二千石益輕賤,吏民慢易之。或持其微過,增加成罪,言於刺史、司隸,或至上書章下;眾庶知其易危,小失意則有離畔之心。前山陽亡徒蘇令等從橫,吏士臨難,莫肯伏節死義,以守相威權素奪也。孝成皇帝悔之,下詔書,二千石不為縱,遣使者賜金,尉厚其意,誠以為國家有急,取辦於二千石,二千石尊重難危,乃能使下。

孝宣皇帝愛其良民吏,有章劾,事留中,會赦壹解。故事,尚書希下章,為煩擾百姓,證驗繫治,或死獄中,章文必有「敢告之」字乃下。唯陛下留神於擇賢,記善忘過,容忍臣子,勿責以備。二千石、部刺史、三輔縣令有材任職者,人情不能不有過差,宜可闊略,令盡力者有所勸。此方今急務,國家之利也。前蘇令發,欲遣大夫使逐問狀,時見大夫無可使者,召盩厔令尹逢拜為諫大夫遣之。令諸大夫有材能者甚少,宜豫畜養可成就者,則士赴難不愛其死;臨事倉卒乃求,非所以明朝廷也。

嘉因薦儒者公孫光、滿昌及能吏蕭咸、薛修等,皆故二千石有名稱。天子納而用之。

會息夫躬、孫寵等因中常侍宋弘上書告東平王雲祝詛,又與后舅伍宏謀弒上為逆,雲等伏誅,躬、寵擢為吏二千石。是時,侍中董賢愛幸於上,上欲侯之而未有所緣,傅嘉勸上因東平事以封賢。上於是定躬、寵告東平本章,掇去宋弘,更言因董賢以聞,欲以其功侯之,皆先賜爵關內侯。頃之,欲封賢等,上心憚嘉,乃先使皇后父孔鄉侯傅晏持詔書視丞相御史。於是嘉與御史大夫賈延上封事言:「竊見董賢等三人始賜爵,眾庶匈匈,咸曰賢貴,其餘并蒙恩,至今流言未解。陛下仁恩於賢等不已,宜暴賢等本奏語言,延問公卿大夫博士議郎,考合古今,明正其義,然後乃加爵土;不然,恐大失眾心,海內引領而議。暴平其事,必有言當封者,在陛下所從;天下雖不說,咎有所分,不獨在陛下。前定陵侯淳于長初封,其事亦議。大司農谷永以長當封,眾人歸咎於永,先帝不獨蒙其譏。臣嘉、臣延材駑不稱,死有餘責。知順指不迕,可得容身須臾,所以不敢者,思報厚恩也。」上感其言,止,數月,遂下詔封賢等,因以切責公卿曰:「朕居位以來,寢疾未瘳,反逆之謀相連不絕,賊亂之臣近侍帷幄。前東平王雲與后謁祝詛朕,使侍醫伍宏等內侍案脈,幾危社稷,殆莫甚焉!昔楚有子玉得臣,晉文為之側席而坐;近事,汲黯折淮南之謀。今雲等至有圖弒天子逆亂之謀者,是公卿股肱莫能悉心務聰明以銷厭未萌之故。賴宗廟之靈,侍中駙馬都尉賢等發覺以聞,咸伏厥辜。書不云乎?『用德章厥善。』其封賢為高安侯、南陽太守寵為方陽侯、左曹光祿大夫躬為宜陵侯。」

後數月,日食,舉直言,嘉復奏封事曰:

臣聞咎繇戒帝舜曰:「亡敖佚欲有國,兢兢業業,一日二日萬機。」箕子戒武王曰:「臣無有作威作福,亡有玉食;臣之有作威作福玉食,害于而家,凶于而國,人用側頗辟,民用僭慝。」言如此則逆尊卑之序,亂陰陽之統,而害及王者,其國極危。國人傾仄不正,民用僭差不壹,此君不由法度,上下失序之敗也。武王躬履此道,隆至成康。自是以後,縱心恣欲,法度陵遲,至於臣弒君,子弒父。父子至親,失禮患生,何況異姓之臣?孔子曰:「道千乘之國,敬事而信,節用而愛人,使民以時。」孝文皇帝備行此道,海內蒙恩,為漢太宗。孝宣皇帝賞罰信明,施與有節,記人之功,忽於小過,以致治平。孝元皇帝奉承大業,溫恭少欲,都內錢四十萬萬,水衡錢二十五萬萬,少府錢十八萬萬。嘗幸上林,後宮馮貴人從臨獸圈,猛獸驚出,貴人前當之,元帝嘉美其義,賜錢五萬。掖庭見親,有加賞賜,屬其人勿眾謝。示平惡偏,重失人心,賞賜節約。是時外戚貲千萬者少耳,故少府水衡見錢多也。雖遭初元、永光凶年飢饉,加有西羌之變,外奉師旅,內振貧民,終無傾危之憂,以府臧內充實也。孝成皇帝時,諫臣多言燕出之害,及女寵專愛,耽於酒色,損德傷年,其言甚切,然終不怨怒也。寵臣淳于長、張放、史育,育數貶退,家貲不滿千萬,放斥逐就國,長榜死於獄。不以私愛害公義,故雖多內譏,朝廷安平,傳業陛下。

陛下在國之時,好詩書,上儉節,徵來所過道上稱誦德美,此天下所以回心也。初即位,易帷帳,去錦繡,乘輿席緣綈繒而已。共皇寢廟比比當作,憂閔元元,惟用度不足,以義割恩,輒且止息,今始作治。而駙馬都尉董賢亦起官寺上林中,又為賢治大第,開門鄉北闕,引王渠灌園池,使者護作,賞賜吏卒,甚於治宗廟。賢母病,長安廚給祠具,道中過者皆飲食。為賢治器,器成,奏御乃行,或物好,特賜其工,自貢獻宗廟三宮,猶不至此。賢家有賓婚及見親,諸官並共,賜及倉頭奴婢,人十萬錢。使者護視,發取市物,百賈震動,道路讙譁,群臣惶惑。詔書罷菀,而以賜賢二千餘頃,均田之制從此墮壞。奢僭放縱,變亂陰陽,災異眾多,百姓訛言,持籌相驚,被髮徒跣而走,乘馬者馳,天惑其意,不能自止。或以為籌者策失之戒也。陛下素仁智慎事,今而有此大譏。

孔子曰:「危而不持,顛而不扶,則將安用彼相矣!」臣嘉幸得備位,竊內悲傷不能通愚忠之信;身死有益於國,不敢自惜。唯陛下慎己之所獨鄉,察眾人之所共疑。往者寵臣鄧通、韓嫣驕貴失度,逸豫無厭,小人不勝情欲,卒陷罪辜。亂國亡驅,不終其祿,所謂愛之適足以害之者也。宜深覽前世,以節賢↨,全安其命。

於是上寖不說,而愈愛賢,不能自勝。

會祖母傅太后薨,上因託傅太后遺詔,令成帝母王太后下丞相御史,益封賢二千戶,及賜孔鄉侯、汝昌侯、陽新侯國。嘉封還詔書,因奏封事諫上及太后曰:「臣聞爵祿土地,天之有也。《書》云:『天命有德,五服五章哉!』王者代天爵人,尤宜慎之。裂地而封,不得其宜,則眾庶不服,感動陰陽,其害疾自深。今聖體久不平,此臣嘉所內懼也。高安侯賢,佞幸之臣,陛下傾爵位以貴之,單貨財以富之,損至尊以寵之,主威已黜,府藏已竭,唯恐不足。財皆民力所為,孝文皇帝欲起露臺,重百金之費,克己不作。今賢散公賦以施私惠,一家至受千金,往古以來貴臣未嘗有此,流聞四方,皆同怨之。里諺曰:『千人所指,無病而死。』臣常為之寒心。今太皇太后以永信太后遺詔,詔丞相御史益賢戶,賜三侯國,臣嘉竊惑。山崩地動,日食於三朝,皆陰侵陽之戒也。前賢已再封,晏、商再易邑,業緣私橫求,恩已過厚,求索自恣,不知厭足,甚傷尊卑之義,不可以示天下,為害痛矣!臣驕侵罔,陰陽失節,氣感相動,害及身體。陛下寢疾久不平,繼嗣未立,宜思正萬事,順天人之心,以求福祐,柰何輕身肆意,不念高祖之勤苦垂立制度欲傳之於無窮哉!孝經曰:『天子有爭臣七人,雖無道,不失其天下。』臣謹封上詔書,不敢露見,非愛死而不自法,恐天下聞之,故不敢自劾。愚贛數犯忌諱,唯陛下省察。」

初,廷尉梁相與丞相長史、御史中丞及五二千石雜治東平王雲獄,時冬月未盡二旬,而相心疑雲冤,獄有飾辭,奏欲傳之長安,更下公卿覆治。尚書令鞫譚、僕射宗伯鳳以為可許。天子以相等皆見上體不平,外內顧望,操持兩心,幸雲踰冬,無討賊疾惡主讎之意,制詔免相等皆為庶人。後數月大赦,嘉奏封事薦相等明習治獄,「相計謀深沈,譚頗知雅文,鳳經明行修,聖王有計功除過,臣竊為朝廷惜此三人。」書奏,上不能平。後二十餘日,嘉封還益董賢戶事,上乃發怒,召嘉詣尚書,責問以「

相等前坐在位不盡忠誠,外附諸侯,操持兩心,背人臣之義,今所稱相等材美,足以相計除罪。君以道德,位在三公,以總方略一統萬類分明善惡為職,知相等罪惡陳列,著聞天下,時輒以自劾,今又稱譽相等,云為朝廷惜之。大臣舉錯,恣心自在,迷國罔上,近由君始,將謂遠者何!對狀。」嘉免冠謝罪。

事下將軍中朝者。光祿大夫孔光、左將軍公孫祿、右將軍王安、光祿勳馬宮、光祿大夫龔勝劾嘉迷國罔上不道,請與廷尉雜治。勝獨以為嘉備宰相,諸事並廢,咎由嘉生;嘉坐薦相等,微薄,以應迷國罔上不道,恐不可以示天下。遂可光等奏。

光等請謁者召嘉詣廷尉詔獄,制曰:「票騎將軍、御史大夫、中二千石、二千石、諸大夫、博士、議郎議。」衛尉雲等五十人以為「

如光等言可許」。議郎龔等以為「嘉言事前後相違,無所執守,不任宰相之職,宜奪爵土,免為庶人。」永信少府猛等十人以為「聖王斷獄,必先原心定罪,探意立情,故死者不抱恨而入地,生者不銜怨而受罪。明主躬聖德,重大臣刑辟,廣延有司議,欲使海內咸服。嘉罪名雖應法,聖王之於大臣,在輿為下,御坐則起,疾病視之無數,死則臨弔之,廢宗廟之祭,進之以禮,退之以義,誄之以行。案嘉本以相等為罪,罪惡雖著,大臣括髮關械、裸躬就笞,非所以重國褒宗廟也。今春月寒氣錯繆,霜露數降,宜示天下以寬和。臣等不知大義,唯陛下察焉。」有詔假謁者節,召丞相詣廷尉詔獄。

使者既到府,掾史涕泣,共和藥進嘉,嘉不肯服。主簿曰:「將相不對理陳冤,相踵以為故事,君侯宜引決。」使者危坐府門上。主簿復前進藥,嘉引藥杯以擊地,謂官屬曰:「丞相幸得備位三公,奉職負國,當伏刑都市以示萬眾。丞相豈兒女子邪,何謂咀藥而死!」嘉遂裝出,見使者再拜受詔,乘吏小車,去蓋不冠,隨使者詣廷尉。廷尉收嘉丞相新甫侯印綬,縛嘉載致都船詔獄。

上聞嘉生自詣吏,大怒,使將軍以下與五二千石雜治。吏詰問嘉,嘉對曰:「案事者思得實。竊見相等前治東平王獄,不以雲為不當死,欲關公卿示重慎;置驛馬傳囚,勢不得踰冬月,誠不見其外內顧望阿附為雲驗。復幸得蒙大赦,相等皆良善吏,臣竊為國惜賢,不私此三人。」獄吏曰:「苟如此,則君何以為罪猶當?有以負國,不空入獄矣。」吏稍侵辱嘉,嘉喟然卬天嘆曰:「幸得充備宰相,不能進賢退不肖,以是負國,死有餘責。」吏問賢不肖主名,嘉曰:「賢,故丞相孔光、故大司空何武,不能進;惡,高安侯董賢父子,佞邪亂朝,而不能退。罪當死,死無所恨。」嘉繫獄二十餘日,不食歐血而死。帝舅大司馬票騎將軍丁明素重嘉而憐之,上遂免明,以董賢代之,語在賢傳。

嘉為相三年誅,國除。死後上覽其對而思嘉言,復以孔光代嘉為丞相,徵用何武為御史大夫。元始四年,詔書追錄忠臣,封嘉子崇為新甫侯,追諡嘉為忠侯。

師丹字仲公,琅邪東武人也。治詩,事匡衡。舉孝廉為郎。元帝末,為博士,免。建始中,州舉茂材,復補博士,出為東平王太傅。丞相方進、御史大夫孔光舉丹論議深博,廉正守道,徵入為光祿大夫、丞相司直。數月,復以光祿大夫給事中,由是為少府、光祿勳、侍中,甚見尊重。成帝末年,立定陶王為皇太子,以丹為太子太傅。哀帝即位,為左將軍,賜爵關內侯,食邑,領尚書事,遂代王莽為大司馬,封高樂侯。月餘,徙為大司空。

上少在國,見成帝委政外家,王氏僭盛,常內邑邑。即位,多欲有所匡正。封拜丁、傅,奪王氏權。丹自以師傅居三公位,得信於上,上書言:「古者諒闇不言,聽於冢宰,三年無改於父之道。前大行尸柩在堂,而官爵臣等以及親屬,赫然皆貴寵。封舅為陽安侯,皇后尊號未定,豫封父為孔鄉侯。出侍中王邑、射聲校尉王邯等。詔書比下,變動政事,卒暴無漸。臣縱不能明陳大義,復曾不能牢讓爵位,相隨空受封侯,增益陛下之過。問者郡國多地動,水出流殺人民,日月不明,五星失行,此皆舉錯失中,號令不定,法度失理,陰陽溷濁之患也。臣伏惟人情無子,年雖六七十,猶博取而廣求。孝成皇帝深見天命,燭知至德,以壯年克己,立陛下為嗣。先帝暴棄天下而陛下繼體,四海安寧,百姓不懼,此先帝聖德當合天人之功也。臣聞天威不違顏咫尺,願陛下深思先帝所以建立陛下之意,且克己躬行以觀群下之從化。天下者,陛下之家也,胏附何患不富貴,不宜倉卒。先帝不量臣愚,以為太傅,陛下以臣託師傅,故亡功德而備鼎足,封大國,加賜黃金,位為三公,職在左右,不能盡忠補過,而令庶人竊議,災異數見,此臣之大罪也。臣不敢言乞骸骨歸於海濱,恐嫌於偽。誠慚負重責,義不得不盡死。」書數十上,多切直之言。

初,哀帝即位,成帝母稱太皇太后,成帝趙皇后稱皇太后,而上祖母傅太后與母丁后皆在國邸,自以定陶共王為稱。高昌侯董宏上書言:「秦莊襄王母本夏氏,而為華陽夫人所子,及即位後,俱稱太后。宜立定陶共王后為皇太后。」事下有司,時丹以左將軍與大司馬王莽共劾奏宏「知皇太后至尊之號,天下一統,而稱引亡秦以為比喻,詿誤聖朝,非所宜言,大不道。」上新立,謙讓,納用莽、丹言,免宏為庶人。傅太后大怒。要上欲必稱尊號,上於是追尊定陶共王為共皇,尊傅太后為共皇太后,丁后為共皇后。郎中令泠褒、黃門郎段猶等復奏言:「定陶共皇太后、共皇后皆不宜復引定陶蕃國之名以冠大號,車馬衣服宜皆稱皇之意,置吏二千石以下各供厥職,又宜為共皇立廟京師。」上復下其議,有司皆以為宜如褒、猶言。丹議獨曰:「聖王制禮取法於天地,故尊卑之禮明則人倫之序正,人倫之序正則乾坤得其位而陰陽順其節,人主與萬民俱蒙祐福,尊卑者,所以正天地之位,不可亂也。今定陶共皇太后、共皇后以定陶共為號者,母從子妻從夫之義也。欲立官置吏,車服與太皇太后並,非所以明尊卑亡二上之義也。定陶共皇號諡已前定,義不得復改。禮:『父為士,子為天子,祭以天子,其尸服以士服。』子亡爵父之義,尊父母也。為人後者為之子,故為所後服斬衰三年,而降其父母期,明尊本祖而重正統也。孝成皇帝聖恩深遠,故為共王立後,奉承祭祀,今共皇長為一國太祖,萬世不毀,恩義已備。陛下既繼體先帝,持重大宗,承宗廟天地社稷之祀,義不得復奉定陶共皇祭入其廟。今欲立廟於京師,而使臣下祭之,是無主也。又親盡當毀,空去一國太祖不墮之祀,而就無主當毀不正之禮,非所以尊厚共皇也。」丹由是浸不合上意。

會有上書言古者以龜貝為貨,今以錢易之,民以故貧,宜可改幣。上以問丹,丹對言可改。章下有司議,皆以為行錢以來久,難卒變易。丹老人,忘其前語,後從公卿議。又丹使吏書奏,吏私寫其草,丁、傅子弟聞之,使人上書告丹上封事行道人遍持其書。上以問將軍中朝臣,皆對曰:「忠臣不顯諫,大臣奏事不宜漏泄,令吏民傳寫流聞四方。『臣不密則失身』,宜下廷尉治。」事下廷尉,廷尉劾丹大不敬。事未決,給事中博士申咸、炔欽上書,言「丹經行無比,自近世大臣能若丹者少。發憤懣,奏封事,不及深思遠慮,使主簿書,漏泄之過不在丹。以此貶黜,恐不厭眾心。」尚書劾咸、欽:「幸得以儒官選擢備腹心,上所折中定疑,知丹社稷重臣,議罪處罰,國之所慎,咸、欽初傅經義以為當治,事以暴列,乃復上書妄稱譽丹,前後相違,不敬。」上貶咸、欽秩各二等,遂策免丹曰:「夫三公者,朕之腹心也,輔善相過,匡率百僚,和合天下者也。朕既不明,委政於公,間者陰陽不調,寒暑失常,變異婁臻,山崩地震,河決泉涌,流殺人民,百姓流連,無所歸心,司空之職尤廢焉。君在位出入三年,未聞忠言嘉謀,而反有朋黨相進不公之名。乃者以挺力田議改幣章示君,君內為朕建可改不疑;以君之言博考朝臣,君乃希眾雷同,外以為不便,令觀聽者歸非於朕。朕隱忍不宣,為君受愆。朕疾夫比周之徒虛偽壞化,寖以成俗,故屢以書飭君,幾君省過求己,而反不受,退有後言。及君奏封事,傳於道路,布聞朝市,言事者以為大臣不忠,辜陷重辟,獲虛采名,謗譏匈匈,流於四方。腹心如此,謂疏者何?殆謬於二人同心之利焉,將何以率示群下,附親遠方?朕惟君位尊任重,慮不周密,懷諼迷國,進退違命,反覆異言,甚為君恥之,非所以共承天地,永保國家之意。以君嘗託傅位,未忍考於理,已詔有司赦君勿治。其上大司空高樂侯印綬,罷歸。」

尚書令唐林上疏曰:「竊見免大司空丹策書,泰深痛切,君子作文,為賢者諱。丹經為世儒宗,德為國黃耇,親傅聖躬,位在三公,所坐者微,海內未見其大過,事既已往,免爵大重,京師識者咸以為宜復丹邑爵,使奉朝請,四方所瞻卬也。惟陛下財覽眾心,有以尉復師傅之臣。」上從林言,下詔賜丹爵關內侯,食邑三百戶。

丹既免數月,上用朱博議,尊傅太后為皇太太后,丁后為帝太后,與太皇太后及皇太后同尊,又為共皇立廟京師,儀如孝元皇帝。博遷為丞相,復與御史大夫趙玄奏言:「前高昌侯宏首建尊號之議,而為丹所劾奏,免為庶人。時天下衰麤,委政於丹。丹不深惟褒廣尊親之義而妄稱說,抑貶尊號,虧損孝道,不忠莫大焉。陛下聖仁,昭然定尊號,宏以忠孝復封高昌侯。丹惡逆暴著,雖蒙赦令,不宜有爵邑,請免為庶人。」奏可。丹於是廢歸鄉里者數年。

平帝即位,新都侯王莽白太皇太后發掘傅太后、丁太后冢,奪其璽綬,更以民葬之,定陶隳廢共皇廟。諸造議泠褒、段猶等皆徙合浦,復免高昌侯宏為庶人。徵丹詣公車,賜爵關內侯,食故邑。數月,太皇太后詔大司徒、大司空曰:「夫褒有德,賞元功,先聖之制,百王不易之道也。故定陶太后造稱僭號,甚悖義理。關內侯師丹端誠於國,不顧患難,執忠節,據聖法,分明尊卑之制,確然有柱石之固,臨大節而不可奪,可謂社稷之臣矣。有司條奏邪臣建定稱號者已放退,而丹功賞未加,殆繆乎先賞後罰之義,非所以章有德報厥功也。其以厚丘之中鄉戶二千一百封丹為義陽侯。」月餘薨,諡曰節侯。子業嗣,王莽敗乃絕。

贊曰:何武之舉,王嘉之爭,師丹之議,考其禍福,乃效於後。當王莽之作,外內咸服,董賢之愛,疑於親戚,武、嘉區區,以一蕢障江河,用沒其身。丹與董宏更受賞罰,哀哉!故曰「依世則廢道,違俗則免殆」,此古人所以難受爵位者也。


\end{pinyinscope}