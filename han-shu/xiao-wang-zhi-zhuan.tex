\article{蕭望之傳}

\begin{pinyinscope}
蕭望之字長倩,東海蘭陵人也,徙杜陵。家世以田為業,至望之,好學,治齊詩,事同縣后倉且十年。以令詣太常受業,復事同學博士白奇,又從夏侯勝問論語、禮服。京師諸儒稱述焉。

是時大將軍霍光秉政,長史丙吉薦儒生王仲翁與望之等數人,皆召見。先是左將軍上官桀與蓋主謀殺光,光既誅桀等,後出入自備。吏民當見者,露索去刀兵,兩吏挾持。望之獨不肯聽,自引出閤曰:「不願見。」吏牽持匈匈。光聞之,告吏勿持。望之既至前,說光曰:「將軍以功德輔幼主,將以流大化,致於洽平,是以天下之士延頸企踵,爭願自劾,以輔高明。今士見者皆先露索挾持,恐非周公相成王躬吐握之禮,致白屋之意。」於是光獨不除用望之,而仲翁等皆補大將軍史。三歲間,仲翁至光祿大夫給事中,望之以射策甲科為郎,署小苑東門候。仲翁出入從倉頭廬兒,下車趨門,傳呼甚寵,顧謂望之曰:「

不肯錄錄,反抱關為。」望之曰:「各從其志。」

後數年,坐弟犯法,不得宿衛,免歸為郡吏。及御史大夫魏相除望之為屬,察廉為大行治禮丞。

時大將軍光薨,子禹復為大司馬,兄子山領尚書,親屬皆宿衛內侍。地節三年夏,京師雨雹,望之因是上疏,願賜清閒之宴,口陳災異之意。宣帝自在民間聞望之名,曰:「此東海蕭生邪?下少府宋畸問狀,無有所諱。」望之對,以為「春秋昭公三年大雨雹,是時季氏專權,卒逐昭公。鄉使魯君察於天變,宜亡此害。今陛下以聖德居位,思政求賢,堯舜之用心也。然而善祥未臻,陰陽不和,是大臣任政,一姓擅勢之所致也。附枝大者賊本心,私家盛者公室危。唯明主躬萬機,選同姓,舉賢材,以為腹心,與參政謀,令公卿大臣朝見奏事,明陳其職,以考功能。如是,則庶事理,公道立,姦邪塞,私權廢矣。」對奏,天子拜望之為謁者。時上初即位,思進賢良,多上書言便宜,輒下望之問狀,高者請丞相御史,次者中二千石試事,滿歲以狀聞,下者報聞,或罷歸田里,所白處奏皆可。累遷諫大夫,丞相司直,歲中三遷,官至二千石。其後霍氏竟謀反誅,望之寖益任用。

是時選博士諫大夫通政事者補郡國守相,以望之為平原太守。望之雅意在本朝,遠為郡守,內不自得,乃上疏曰:「陛下哀愍百姓,恐德化之不究,悉出諫官以補郡吏,所謂憂其末而忘其本者也。朝無爭臣則不知過,國無達士則不聞善。願陛下選明經術,溫故知新,通於幾微謀慮之士以為內臣,與參政事。諸侯聞之,則知國家納諫憂政,亡有闕遺。若此不怠,成康之道其庶幾乎!外郡不治,豈足憂哉?」書聞,徵入守少府。宣帝察望之經明持重,論議有餘,材任宰相,欲詳試其政事,復以為左馮翊。望之從少府出為左遷,恐有不合意,即移病。上聞之,使侍中成都侯金安上諭意曰:「所用皆更治民以考功。君前為平原太守日淺,故復試之於三輔,非有所聞也。」望之即視事。

是歲西羌反,漢遣後將軍征之。京兆尹張敞上書言:「國兵在外,軍以夏發,隴西以北,安定以西,吏民並給轉輸,田事頗廢,素無餘積,雖羌虜以破,來春民食必乏。窮辟之處,買亡所得,縣官穀度不足以振之。願令諸有罪,非盜受財殺人及犯法不得赦者,皆得以差入穀此八郡贖罪。務益致穀以豫備百姓之急。」事下有司,望之與少府李彊議,以為「民函陰陽之氣,有仁義欲利之心,在教化之所助。堯在上,不能去民欲利之心,而能令其欲利不勝其好義也;雖桀在上,不能去民好義之心,而能令其好義不勝其欲利也。故堯、桀之分,在於義利而已,道民不可不慎也。今欲令民量粟以贖罪,如此則富者得生,貧者獨死,是貧富異刑而法不壹也。人情,貧窮,父兄囚執,聞出財得以生活,為人子弟者將不顧死亡之患,敗亂之行,以赴財利,求救親戚。一人得生,十人以喪,如此,伯夷之行壞,公綽之名滅。政教壹傾,雖有周召之佐,恐不能復。古者臧於民,不足則取,有餘則予。《詩》曰『爰及矜人,哀此鰥寡』,上惠下也。又曰『雨我公田,遂及我私』,下急上也。今有西邊之役,民失作業,雖戶賦口斂以贍其困乏,古之通義,百姓莫以為非。以死救生,恐未可也。陛下布德施教,教化既成,堯舜亡以加也。今議開利路以傷既成之化,臣竊痛之。」

於是天子復下其議兩府,丞相、御史以難問張敞。敞曰:「少府左馮翊所言,常人之所守耳。昔先帝征四夷,兵行三十餘年,百姓猶不加賦,而軍用給。今羌虜一隅小夷,跳梁於山谷間,漢但令罪人出財減罪以誅之,其名賢於煩擾良民橫興賦斂也。又諸盜及殺人犯不道者,百姓所疾苦也,皆不得贖;首匿、見知縱、所不當得為之屬,議者或頗言其法可蠲除,今因此令贖,其便明甚,何化之所亂?甫刑之罰,小過赦,薄罪贖,有金選之品,所從來久矣,何賊之所生?敞備皁衣二十餘年,嘗聞罪人贖矣,未聞盜賊起也。竊憐涼州被寇,方秋饒時,民尚有飢乏,病死於道路,況至來春將大困乎!不早慮所以振救之策,而引常經以難,恐後為重責。常人可與守經,未可與權也。敞幸得備列卿,以輔兩府為職,不敢不盡愚。」

望之、彊復對曰:「先帝聖德,賢良在位,作憲垂法,為無窮之規,永惟邊竟之不贍,故金布令甲曰『邊郡數被兵,離飢寒,夭絕天年,父子相失,令天下共給其費』,固為軍旅卒暴之事也。聞天漢四年,常使死罪人入五十萬錢減死罪一等,豪彊吏民請奪假貣,至為盜賊以贖罪。其後姦邪橫暴,群盜並起,至攻城邑,殺郡守,充滿山谷,吏不能禁,明詔遣繡衣使者以興兵擊之,誅者過半,然後衰止。愚以為此使死罪贖之敗也,故曰不便。」時丞相魏相、御史大夫丙吉亦以為羌虜且破,轉輸略足相給,遂不施敞議。望之為左馮翊三年,京師稱之,遷大鴻臚。

先是烏孫昆彌翁歸靡因長羅侯常惠上一書,願以漢外孫元貴靡為嗣,得復尚少主,結婚內附,畔去匈奴。詔下公卿議,望之以為烏孫絕域,信其美言,萬里結婚,非長策也。天子不聽。神爵二年,遣長羅侯惠使送公主配元貴靡。未出塞,翁歸靡死,其兄子狂王背約自立。惠從塞下上書,願留少主敦煌郡。惠至烏孫,責以負約,因立元貴靡,還迎少主。詔下公卿議,望之復以為「不可。烏孫持兩端,亡堅約,其效可見。前少主在烏孫四十餘年,恩愛不親密,邊境未以安,此已事之驗也。今少主以元貴靡不得立而還,信無負於四夷,此中國之大福也。少主不止,繇役將興,其原起此。」天子從其議,徵少主還。後烏孫雖分國兩立,以元貴靡為大昆彌,漢遂不復與結婚。

三年,代丙吉為御史大夫。五鳳中匈奴大亂,議者多曰匈奴為害日久,可因其壞亂舉兵滅之。詔遣中朝大司馬車騎將軍韓增、諸吏富平侯張延壽、光祿勳楊惲、太僕戴長樂問望之計策,望之對曰:「春秋晉士饨帥師侵齊,聞齊侯卒,引師而還,君子大其不伐喪,以為恩足以服孝子,誼足以動諸侯。前單于慕化鄉善稱弟,遣使請求和親,海內欣然,夷狄莫不聞。未終奉約,不幸為賊臣所殺,今而伐之,是乘亂而幸災也,彼必奔走遠遁。不以義動兵,恐勞而無功。宜遣使者弔問,輔其微弱,救其災患,四夷聞之,咸貴中國之仁義。如遂蒙恩得復其位,必稱臣服從,此德之盛也。」上從其議,後竟遣兵護輔呼韓邪單于定其國。

是時大司農中丞耿壽昌奏設常平倉,上善之,望之非壽昌。丞相丙吉年老,上重焉,望之又奏言:「百姓或乏困,盜賊未止,二千石多材下不任職。三公非其人,則三光為之不明,今首歲日月少光,咎在臣等。」上以望之意輕丞相,乃下侍中建章衛尉金安上、光祿勳楊惲、御史中丞王忠,并詰問望之。望之免冠置對,天子繇是不說。

後丞相司直茇延壽奏:「侍中謁者良使丞制詔望之,望之再拜已。良與望之言,望之不起,因故下手,而謂御史曰『良禮不備』。故事丞相病,明日御史大夫輒問病;朝奏事會庭中,差居丞相後,丞相謝,大夫少進,揖。今丞相數病,望之不問病;會庭中,與丞相鈞禮。時議事不合意,望之曰:『侯年寧能父我邪!』知御史有令不得擅使,望之多使守史自給車馬,之杜陵護視家事。少史冠法冠,為妻先引,又使賣買,私所附益凡十萬三千。案望之大臣,通經術,居九卿之右,本朝所仰,至不奉法自修,踞慢不遜攘,受所監臧二百五十以上,請逮捕繫治。」上於是策望之曰:「有司奏君責使者禮,遇丞相亡禮,廉聲不聞,敖慢不遜,亡以扶政,帥先百僚。君不深思,陷于茲穢,朕不忍致君于理,使光祿勳惲策詔,左遷君為太子太傅,授印。其上故印使者,便道之官。君其秉道明孝,正直是與,帥意亡伥,靡有後言。」

望之既左遷,而黃霸代為御史大夫。數月間,丙吉薨,霸為丞相。霸薨,于定國復代焉。望之遂見廢,不得相。為太傅,以論語、禮服授皇太子。

初,匈奴呼韓邪單于來朝,詔公卿議其儀,丞相霸、御史大夫定國議曰:「聖王之制,施德行禮,先京師而後諸夏,先諸夏而後夷狄。《詩》云:『率禮不越,遂視既發;相土烈烈,海外有截。』陛下聖德充塞天地,光被四表,匈奴單于鄉風慕化,奉珍朝賀,自古未之有也。其禮儀宜如諸侯王,位次在下。」望之以為「單于非正朔所加,故稱敵國,宜待以不臣之禮,位在諸侯王上。外夷稽首稱藩,中國讓而不臣,此則羈縻之誼,謙亨之福也。書曰『戎狄荒服』,言其來,荒忽亡常。如使匈奴後嗣卒有鳥竄鼠伏,闕於朝享,不為畔臣。信讓行乎蠻貉,福祚流于亡窮,萬世之長策也。」天子采之,下詔曰:「蓋聞五帝三王教化所不施,不及以政。今匈奴單于稱北藩,朝正朔,朕之不逮,德不能弘覆。其以客禮待之,令單于位在諸侯王上,贊謁稱臣而不名。」

及宣帝寢疾,選大臣可屬者,引外屬侍中樂陵侯史高、太子太傅望之、少傅周堪至禁中,拜高為大司馬車騎將軍,望之為前將軍光祿勳,堪為光祿大夫,皆受遺詔輔政,領尚書事。宣帝崩,太子襲尊號,是為孝元帝。望之、堪本以師傅見尊重,上即位,數宴見,言治亂,陳王事。望之選白宗室明經達學散騎諫大夫劉更生給事中,與侍中金敞並拾遺左右。四人同心謀議,勸道上以古制,多所欲匡正,上甚鄉納之。

初,宣帝不甚從儒術,任用法律,而中書宦官用事。中書令弘恭、石顯久典樞機,明習文法,亦與車騎將軍高為表裏,論議常獨持故事,不從望之等。恭、顯又時傾仄見詘。望之以為中書政本,宜以賢明之選,自武帝游宴後庭,故用宦者,非國舊制,又違古不近刑人之義,白欲更置士人,繇是大與高、恭、顯忤。上初即位,謙讓重改作,議久不定,出劉更生為宗正。

望之、堪數薦名儒茂材以備諫官。會稽鄭朋陰欲附望之,上疏言車騎將軍高遣客為姦利郡國,及言許、史子弟罪過。章視周堪,堪白令朋待詔金馬門。朋奏記望之曰:「將軍體周召之德,秉公綽之質,有卞莊之威。至乎耳順之年,履折衝之位,號至將軍,誠士之高致也。窟穴黎庶莫不懽喜,咸曰將軍其人也。今將軍橱跻云若管晏而休,遂行日仄至周召乃留乎?若管晏而休,則下走將歸延陵之皋,修農圃之疇,畜雞種黍,俟見二子,沒齒而已矣。如將軍昭然度行,積思塞邪枉之險蹊,宣中庸之常政,興周召之遺業,親日仄之兼聽,則下走其庶幾願竭區區,底厲鋒鍔,奉萬分之一。」望之見納朋,接待以意。朋數稱述望之,短車騎將軍。言許、史過失。

後朋行傾邪,望之絕不與通。朋與大司農史李宮俱待詔,堪獨白宮為黃門郎。朋,楚士,怨恨,更求入許、史,推所言許、史事曰:「皆周堪、劉更生教我,我關東人,何以知此?」於是侍中許章白見朋。朋出揚言曰:「我見,言前將軍小過五,大罪一。中書令在旁,知我言狀。」望之聞之,以問弘恭、石顯。顯、恭恐望之自訟,下於它吏,即挾朋及待詔華龍。龍者,宣帝時與張子蟜等待詔,以行汙濊不進,欲入堪等,堪等不納,故與朋相結。恭、顯令二人告望之等謀欲罷車騎將軍疏退許、史狀,候望之出休日,令朋、龍上之。事下弘恭問狀,望之對曰:「外戚在位多奢淫,欲以匡正國家,非為邪也。」恭、顯奏「望之、堪、更生朋黨相稱舉,數譖訴大臣,毀離親戚,欲以專擅權勢,為臣不忠,誣上不道,請謁者召致廷尉。」時上初即位,不省「謁者召致廷尉」為下獄也,可其奏。後上召堪、更生,曰繫獄。上大驚曰:「非但廷尉問邪?」以責恭、顯,皆叩頭謝。上曰:「令出視事。」恭、顯因使高言:「上新即位,未以德化聞於天下,而先驗師傅,既下九卿大夫獄,宜因決免。」於是制詔丞相御史:「前將軍望之傅朕八年,亡它罪過,今事久遠,識忘難明。其赦望之罪,收前將軍光祿勳印綬,及堪、更生皆免為庶人。」而朋為黃門郎。

後數月,制詔御史:「國之將興,尊師而重傅。故前將軍望之傅朕八年,道以經術,厥功茂焉。其賜望之爵關內侯,食邑六百戶,給事中,朝朔望,坐次將軍。」天子方倚欲以為丞相,會望之子散騎中郎伋上書訟望之前事,事下有司,復奏「望之前所坐明白,無譖訴者,而教子上書,稱引亡辜之詩,失大臣體,不敬,請逮捕。」弘恭、石顯等知望之素高節,不詘辱,建白「望之前為將軍輔政,欲排退許、史,專權擅朝。幸得不坐,復賜爵邑,與聞政事,不悔過服罪,深懷怨望,教子上書,歸非於上,自以託師傅,懷終不坐。非頗詘望之於牢獄,塞其怏怏心,則聖朝亡以施恩厚。」上曰:「蕭太傅素剛,安肯就吏?」顯等曰:「人命至重,望之所坐,語言薄罪,必亡所憂。」上乃可其奏。

顯等封以付謁者,敕令召望之手付,因令太常急發執金吾車騎馳圍其第。使者至,召望之。望之欲自殺,其夫人止之,以為非天子意。望之以問門下生朱雲。雲者好節士,勸望之自裁。於是望之卬天歎曰:「吾嘗備位將相,年踰六十矣,老入牢獄,苟求生活,不亦鄙乎!」字謂雲曰:「游,趣和藥來,無久留我死!」竟飲鴆自殺。天子聞之驚,拊手曰:「曩固疑其不就牢獄,果然殺吾賢傅!」是時太官方上晝食,上乃卻食,為之涕泣,哀慟左右。於是召顯等責問以議不詳。皆免冠謝,良久然後已。

望之有罪死,有司請絕其爵邑。有詔加恩,長子伋嗣為關內侯。天子追念望之不忘,每歲時遣使者祠祭望之冢,終元帝世。望之八子,至大官者育、咸、由。

育字次君,少以父任為太子庶子。元帝即位,為郎,病免,後為御史。大將軍王鳳以育名父子,著材能,除為功曹,遷謁者,使匈奴副校尉。後為茂陵令,會課,育第六。而漆令郭舜殿,見責問,育為之請,扶風怒曰:「君課第六,裁自脫,何暇欲為左右言?」及罷出,傳召茂陵令詣後曹,當以職事對。育徑出曹,書佐隨牽育,育案佩刀曰:「蕭育杜陵男子,何詣曹也!」遂趨出,欲去官。明旦,詔召入,拜為司隸校尉。育過扶風府門,官屬掾史數百人拜謁車下。後坐失大將軍指免官。復為中郎將使匈奴。歷冀州、青州兩郡刺史,長水校尉,泰山太守,入守大鴻臚。以鄠名賊梁子政阻山為害,久不伏辜,育為右扶風數月,盡誅子政等。坐與定陵侯淳于長厚善免官。

哀帝時,南郡江中多盜賊,拜育為南郡太守。上以育耆舊名臣,乃以三公使車載育入殿中受策,曰:「南郡盜賊群輩為害,朕甚憂之。以太守威信素著,故委南郡太守,之官,其於為民除害,安元元而已,亡拘於小文。」加賜黃金二十斤。育至南郡,盜賊靜。病去官,起家復為光祿大夫執金吾,以壽終於官。

育為人嚴猛尚威,居官數免,稀遷。少與陳咸、朱博為友,著聞當世。往者有王陽、貢公,故長安語曰「蕭、朱結綬,王、貢彈冠」,言其相薦達也。始育與陳咸俱以公卿子顯名,咸最先進,年十八為左曹,二十餘御史中丞。時朱博尚為杜陵亭長,為咸、育所攀援,入王氏。後遂並歷刺史郡守相,及為九卿,而博先至將軍上卿,歷位多於咸、育,遂至丞相。育與博後有隙,不能終,故世以交為難。

咸字仲,為丞相史,舉茂材,好畤令,遷淮陽、泗水內史,張掖、弘農、河東太守。所居有跡,數增秩賜金。後免官,復為越騎校尉、護軍都尉、中郎將,使匈奴,至大司農,終官。

由字子驕,為丞相西曹衛將軍掾,遷謁者,使匈奴副校尉。後舉賢良,為定陶令,遷太原都尉,安定太守。治郡有聲,多稱薦者。初,哀帝為定陶王時,由為定陶令,失王指,頃之,制書免由為庶人。哀帝崩,為復土校尉、京輔左輔都尉,遷江夏太守。平江賊成重等有功,增秩為陳留太守。元始中,作明堂辟雍,大朝諸侯,徵由為大鴻臚,會病,不及賓贊,還歸故官,病免。復為中散大夫,終官。家至吏二千石者六七人。

贊曰:蕭望之歷位將相,籍師傅之恩,可謂親昵亡間。及至謀泄隙開,讒邪搆之,卒為便嬖宦豎所圖,哀哉!,望之堂堂,折而不橈,身為儒宗,有輔佐之能,近古社稷臣也。


\end{pinyinscope}