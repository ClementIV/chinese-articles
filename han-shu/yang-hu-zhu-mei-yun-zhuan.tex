\article{楊胡朱梅云傳}

\begin{pinyinscope}
楊王孫者,孝武時人也。學黃老之術,家業千金,厚自奉養生,亡所不致。及病且終,先令其子,曰:「吾欲臝葬,以反吾真,必亡易吾意。死則為布囊盛尸,入地七尺,既下,從足引脫其囊,以身親土。」其子欲默而不從,重廢父命,欲從其,心又不忍,乃往見王孫友人祁侯。

祁侯與王孫書曰:「王孫苦疾,僕迫從上祠雍,未得詣前。願存精神,省思慮,進醫藥,厚自持。竊閒王孫先令臝葬,令死者亡知則已,若其有知,是戮尸地下,將臝見先人,竊為王孫不取也。且孝經曰『為之棺槨衣衾』,是亦聖人之遺制,何必區區獨守所聞?願王孫察焉。」

王孫報曰:「蓋聞古之聖王,緣人情不忍其親,故為制禮,今則越之,吾是以臝葬,將以矯世也。夫厚葬誠亡益於死者,而俗人競以相高,靡財單幣,腐之地下。或乃今日入而明日發,此真與暴骸於中野何異!且夫死者,終生之化,而物之歸者也。歸者得至,化者得變,是物各反其真也。反真冥冥,亡形亡聲,乃合道情。夫飾外以華眾,厚葬以鬲真,使歸者不得至,化者不得變,是使物各失其所也。且吾聞之,精神者天之有也,形骸者地之有也。精神離形,各歸其真,故謂之鬼,鬼之為言歸也。其尸塊然獨處,豈有知哉?裹以幣帛,鬲以棺槨,支體絡束,口含玉石,欲化不得,鬱為枯腊,千載之後,棺槨朽腐,乃得歸土,就其真宅。繇是言之,焉用久客!昔帝堯之葬也,窾木為櫝,葛藟為緘,其穿下不亂泉,上不泄殠。故聖王生易尚,死易葬也。不加功於亡用,不損財於亡謂。今費財厚葬,留歸鬲至,死者不知,生者不得,是謂重惑。於戲!吾不為也。」

祁侯曰:「善。」遂臝葬。

胡建字子孟,河東人也。孝武天漢中,守軍正丞,貧亡車馬,常步與走卒起居,所以尉薦走卒,甚得其心。時監軍御史為姦,穿北軍壘垣以為賈區,建欲誅之,乃約其走卒曰:「我欲與公有所誅,吾言取之則取,斬之則斬。」於是當選士馬日,監御史與護軍諸校列坐堂皇上,建從走卒趨至堂皇下拜謁,因上堂,走卒皆上,建指監御史曰:「取彼。」走卒前曳下堂皇。建曰:「斬之。」遂斬御史。護軍諸校皆愕驚,不知所以。建亦已有成奏在其懷中,遂上奏曰:「臣聞軍法,立武以威眾,誅惡以禁邪。今監御史公穿軍垣以求賈利,私買賣以與士巿,不立剛毅之心,勇猛之節,亡以帥先士大夫,尤失理不公。用文吏議,不至重法。黃帝李法曰:『壁壘已定,穿窬不繇路,是謂姦人,姦人者殺。』臣謹按軍法曰:『正亡屬將軍,將軍有罪以聞,二千石以下行法焉。』丞於用法疑,執事不諉上,臣謹以斬,昧死以聞。」制曰:「司馬法曰『國容不入軍,軍容不入國』,何文吏也?三王或誓於軍中,欲民先成其慮也;或誓於軍門之外,欲民先意以待事也;或將交刃而誓,致民志也。』建又何疑焉?」建繇是顯名。

後為渭城令,治甚有聲。值昭帝幼,皇后父上官將軍安與帝姊蓋主私夫丁外人相善。外人矯恣,怨故京兆尹樊福,使客射殺之。客臧公主廬,吏不敢捕。渭城令建將吏卒圍捕。蓋主聞之,與外人、上官將軍多從奴客往,奔射追吏,吏散走。主使僕射劾渭城令游徼傷主家奴。建報亡它坐。蓋主怒,使人上書告建侵辱長公主,射甲舍門。知吏賊傷奴,辟報故不窮審。大將軍霍光寢其奏。後光病,上官氏代聽事,下吏捕建,建自殺。吏民稱冤,至今渭城立其祠。

朱雲字游,魯人也,徙平陵。少時通輕俠,借客報仇。長八尺餘,容貌甚壯,以勇力聞。年四十,乃變節從博士白子友受易,又事前將軍蕭望之受論語,皆能傳其業。好倜儻大節,當世以是高之。

元帝時,琅邪貢禹為御史大夫,而華陰守丞嘉上封事,言『治道在於得賢,御史之官,宰相之副,九卿之右,不可不選。平陵朱雲,兼資文武,忠正有智略,可使以六百石秩試守御史大夫,以盡其能。」上乃下其事問公卿。太子少傅匡衡對,以為「大臣者,國家之股肱,萬姓所瞻仰,明王所慎擇也。傳曰下輕其上爵,賤人圖柄臣,則國家搖動而民不靜矣。今嘉從守丞而圖大臣之位,欲以匹夫徒走之人而超九卿之右,非所以重國家而尊社稷也。自堯之用舜,文王於太公,猶試然後爵之,又況朱雲者乎?雲素好勇,數犯法亡命,受易頗有師道,其行義未有以異。今御史大夫禹絜白廉正,經術通明,有伯夷、史魚之風,海內莫不聞知,而嘉泸稱雲,欲令為御史大夫,妄相稱舉,疑有姦心,漸不可長,宜下有司案驗以明好惡。」嘉竟坐之。

是時,少府五鹿充宗貴幸,為梁丘易。自宣帝時善梁丘氏說,元帝好之,欲考其異同,令充宗與諸易家論。充宗乘貴辯口,諸儒莫能與抗,皆稱疾不敢會。有薦雲者,召入,攝沥登堂,抗首而請,音動左右。既論難,連拄五鹿君,故諸儒為之語曰:「五鹿嶽嶽,朱雲折其角。」繇是為博士。

遷杜陵令,坐故縱亡命,會赦,舉方正,為槐里令。時中書令石顯用事,與充宗為黨,百僚畏之。唯御史中丞陳咸年少抗節,不附顯等,而與雲相結。雲數上疏,言丞相韋玄成容身保位,亡能往來,而咸數毀石顯。久之,有司考雲,疑風吏殺人。群臣朝見,上問丞相以雲治行。丞相玄成言雲暴虐亡狀。時陳咸在前,聞之,以語雲。雲上書自訟,咸為定奏草,求下御史中丞。事下丞相,丞相部吏考立其殺人罪。雲亡入長安,復與咸計議。丞相具發其事,奏「咸宿衛執法之臣,幸得進見,漏泄所聞,以私語雲,為定奏草,欲令自下治,後知雲亡命罪人,而與交通,雲以故不得。」上於是下咸、雲獄,減死為城旦。咸、雲遂廢錮,終元帝世。

至成帝時,丞相故安昌侯張禹以帝師位特進,甚尊重。雲上書求見,公卿在前。雲曰:「今朝廷大臣上不能匡主,下亡以益民,皆尸位素餐,孔子所謂『鄙夫不可與事君』,『苟患失之,亡所不至』者也。臣願賜尚方斬馬劍,斷佞臣一人以厲其餘。」上問:「誰也?」對曰「安昌侯張禹。」上大怒,曰:「小臣居下訕上,廷辱師傅,罪死不赦。」御史將雲下,雲攀殿檻,檻折。雲呼曰:「臣得下從龍逄、比干遊於地下,足矣!未知聖朝何如耳?」御史遂將雲去。於是左將軍辛慶忌免冠解印綬,叩頭殿下曰:「此臣素著狂直於世。使其言是,不可誅;其言非,固當容之。臣敢以死爭。」慶忌叩頭流血。上意解,然後得已。及後當治檻,上曰:「勿易!因而輯之,以旌直臣。」

雲自是之後不復仕,常居鄠田,時出乘牛車從諸生,所過皆敬事焉。薛宣為丞相,雲往見之。宣備賓主禮,因留雲宿,從容謂雲曰:「在田野亡事,且留我東閤,可以觀四方奇士。」雲曰:「小生乃欲相吏邪?」宣不敢復言。

其教授,擇諸生,然後為弟子。九江嚴望及望兄子元,字仲,能傳雲學,皆為博士。望至泰山太守。

雲年七十餘,終於家。病不呼醫飲藥。遺言以身服斂,棺周於身,土周於槨,為丈五墳,葬平陵東郭外。

梅福字子真,九江壽春人也。少學長安,明尚書、穀梁春秋,為郡文學,補南昌尉。後去官歸壽春,數因縣道上言變事,求假軺傳,詣行在所條對急政,輒報罷。

是時成帝委任大將軍王鳳,鳳專勢擅朝,而京兆尹王章素忠直,譏刺鳳,為鳳所誅。王氏浸盛,災異數見,群下莫敢正言。福復上書曰:

臣聞箕子佯狂於殷,而為周陳洪範;叔孫通遁秦歸漢,制作儀品。夫叔孫先非不忠也,箕子非疏其家而畔親也,不可為言也。昔高祖納善若不及,從諫若轉圜,聽言不求其能,舉功不考其素。陳平起於亡命而為謀主,韓信拔於行陳而建上將。故天下之士雲合歸漢,爭進奇異,知者竭其策,愚者盡其慮,勇士極其節,怯夫勉其死。合天下之知,并天下之威,是以舉秦如鴻毛,取楚若拾遺,此高祖所以亡敵於天下也。孝文皇帝起於代谷,非有周召之師,伊呂之佐也,循高祖之法,加以恭儉。當此之時,天下幾平。繇是言之,循高祖之法則治,不循則亂。何者?秦為亡道,削仲尼之跡,滅周公之軌,壞井田,除五等,禮廢樂崩,王道不通,故欲行王道者莫能致其功也。孝文皇帝好忠諫,說至言,出爵不待廉茂,慶賜不須顯功,是以天下布衣各厲志竭精以赴闕廷自衒鬻者不可勝數。漢家得賢,於此為盛。使孝武皇帝聽用其計,升平可致。於是積尸暴骨,快心胡越,故淮南

安王緣間而起。所以計慮不成而謀議泄者,以眾賢聚於本朝,故其大臣勢陵不敢和從也。方今布衣乃窺國家之隙,見間而起者,蜀郡是也。及山陽亡徒蘇令之群,蹈藉名都大郡,求黨與,索隨和,而亡逃匿之意。此皆輕量大臣,亡所畏忌,國家之權輕,故匹夫欲與上爭衡也。

士者,國之重器;得士則重,失士則輕。《詩》云:「濟濟多士,文王以寧。」廟堂之議,非草茅所當言也。臣誠恐身塗野草,尸并卒伍,故數上書求見,輒報罷。臣聞齊桓之時有以九九見者,桓公不逆,欲以致大也。今臣所言非特九九也,陛下距臣者三矣,此天下士所以不至也。昔秦武王好力,任鄙叩關自鬻;繆公行伯,繇余歸德。今欲致天下之士,民有上書求見者,輒使詣尚書問其所言,言可采取者,秩以升斗之祿,賜以一束之帛。若此,則天下之士發憤懣,吐忠言,嘉謀日聞於上,天下條貫,國家表裏,爛然可睹矣。夫以四海之廣,士民之數,能言之類至眾多也。然其俊桀指世陳政,言成文章,質之先聖而不繆,施之當世合時務,若此者,亦亡幾人。故爵祿束帛者,天下之厎石,高祖所以厲世摩鈍也。孔子曰:「工欲善其事,必先利其器。」至秦則不然,張誹謗之罔,以為漢驅除,倒持泰阿,授楚其柄。故誠能勿失其柄,天下雖有不順,莫敢觸其鋒,此孝武皇帝所以辟地建功為漢世宗也。今不循伯者之道,乃欲以三代選舉之法取當時之士,猶察伯樂之圖,求騏驥於市,而不可得,亦已明矣。故高祖棄陳平之過而獲其謀,晉文召天王,齊桓用其讎,亡益於時,不顧逆順,此所謂伯道者也。一色成體謂之醇,白黑雜合謂之駮。欲以承平之法治暴秦之緒,猶以鄉飲酒之禮理軍市也。

今陛下既不納天下之言,又加戮焉。夫觏鵲遭害,則仁鳥增逝;愚者蒙戮,則知士深退。間者愚民上疏,多觸不急之法,或下廷尉,而死者眾。自陽朔以來,天下以言為諱,朝廷尤甚,群臣皆承順上指,莫有執正。何以明其然也?取民所上書,陛下之所善,試下之廷尉,廷尉必曰「非所宜言,大不敬。」以此卜之,一矣。故京兆尹王章資質忠直,敢面引廷爭,孝元皇帝擢之,以厲具臣而矯曲朝。及至陛下,戮及妻子。且惡惡止其身,王章非有反畔之辜,而殃及家。折直士之節,結諫臣之舌,群臣皆知其非,然不敢爭,天下以言為戒,最國家之大患也。願陛下循高祖之軌,杜亡秦之路,數御十月之歌,留意亡逸之戒,除不急之法,下亡諱之詔,博覽兼聽,謀及疏賤,令深者不隱,遠者不塞,所謂「辟四門,明四目」也。且不急之法,誹謗之微者也。「往者不可及,來者猶可追。」方今君命犯而主威奪,外戚之權日以益隆,陛下不見其形,願察其景。建始以來,日食地震,以率言之,三倍春秋,水災亡與比數。陰盛陽微,金鐵為飛,此何景也!漢興以來,社稷三危。呂、霍、上官皆母后之家也,親親之道,全之為右,當與之賢師良傅,教以忠孝之道。今乃尊寵其位,授以魁柄,使之驕逆,至於夷滅,此失親親之大者也。自霍光之賢,不能為子孫慮,故權臣易世則危。書曰:「毋若火,始庸庸。」勢陵於君,權隆於主,然後防之,亦亡及已。

上遂不納。

成帝久亡繼嗣,福以為宜建三統,封孔子之世以為殷後,復上書曰:

臣聞「不在其位,不謀其政」。政者職也,位卑而言高者罪也。越職觸罪,危言世患,雖伏質橫分,臣之願也。守職不言,沒齒身全,死之日,尸未腐而名滅,雖有景公之位,伏歷千駟,臣不貪也。故願壹登文石之陛,涉赤墀之塗,當戶牖之法坐,盡平生之愚慮。亡益於時,有遺於世,此臣寢所以不安,食所以忘味也。願陛下深省臣言。

臣聞存人所以自立也,壅人所以自塞也。善惡之報,各如其事。昔者秦滅二周,夷六國,隱士不顯,佚民不舉,絕三統,滅天道,是以身危子殺,厥孫不嗣,所謂壅人以自塞者也。故武王克殷,未下車,存五帝之後,封殷於宋,紹夏於杞,明著三統,示不獨有也。是以姬姓半天下,遷廟之主,流出於戶,所謂存人以自立者也。今成湯不祀,殷人亡後。陛下繼嗣久微,殆為此也。春秋經曰:「宋殺其大夫。」穀梁傳曰:「其不稱名姓,以其在祖位,尊之也。」此言孔子故殷後也,雖不正統,封其子孫以為殷後,禮亦宜之。何者?諸侯奪宗,聖庶奪適。傳曰「賢者子孫宜有土」,而況聖人,又殷之後哉!昔成王以諸侯禮葬周公,而皇天動威,雷風著災。今仲尼之廟不出闕里,孔氏子孫不免編戶,以聖人而歆匹夫之祀,非皇天之意也。今陛下誠能據仲尼之素功,以封其子孫,則國家必獲其福,又陛下之名與天亡極。何者?追聖人素功,封其子孫,未有法也,後聖必以為則。不滅之名,可不勉哉!

福孤遠,又譏切王氏,故終不見納。

,武帝時,始封周後姬嘉為周子南君,至元帝時,尊周子南君為周承休侯,位次諸侯王。使諸大夫博士求殷後,分散為十餘姓,郡國往往得其大家,推求子孫,絕不能紀。時匡衡議,以為「王者存二王後,所以尊其先王而通三統也。其犯誅絕之罪者絕,而更封他親為始封君,上承其王者之始祖。春秋之義,諸侯不能守其社稷者絕。今宋國已不守其統而失國矣,則宜更立殷後為始封君,而上承湯統,非當繼宋之絕侯也,宜明得殷後而已。今之故宋,推求其嫡,久遠不可得;雖得其嫡,嫡之先已絕,不當得立。禮記孔子曰:『

丘,殷人也。』先師所共傳,宜以孔子世為湯後。」上以其語不經,遂見寢。至成帝時,梅福復言宜封孔子後以奉湯祀。綏和元年,立二王後,推跡古文,以左氏、穀梁、世本、禮記相明,遂下詔封孔子世為殷紹嘉公。語在成紀。是時,福居家,常以讀書養性為事。

至元始中,王莽顓政,福一朝棄妻子,去九江,至今傳以為仙。其後,人有見福於會稽者,變名姓,為吳市門卒云。

云敞字幼儒,平陵人也。師事同縣吳章,章治尚書經為博士。平帝以中山王即帝位,年幼,莽秉政,自號安漢公。以平帝為成帝後,不得顧私親,帝母及外家衛氏皆留中山,不得至京師。莽長子宇,非莽鬲絕衛氏,恐帝長大後見怨。宇與吳章謀,夜以血塗莽門,若鬼神之戒,冀以懼莽。章欲因對其咎。事發覺,莽殺宇,誅滅衛氏,謀所聯及,死者百餘人。章坐要斬,磔尸東市門。初,章為當世名儒,教授尤盛,弟子千餘人,莽以為惡人黨,皆當禁固,不得仕宦。門人盡更名他師。敞時為大司徒掾,自劾吳章第子,收抱章尸歸,棺斂葬之,京師稱焉。車騎將軍王舜高其志節,比之欒布,表奏以為掾,薦為中郎諫大夫。莽篡位,王舜為太師。復薦敞可輔職。以病免。唐林言敞可典郡,擢為魯郡大尹。更始時,安車徵敞為御史大夫,復病免去,卒于家。

贊曰:昔仲尼稱不得中行,則思狂狷。觀楊王孫之志,賢於秦始皇遠矣。世稱朱雲多過其實,「蓋有不知而作之者,我亡是也。」胡建臨敵敢斷,武昭於外。斬伐姦隙,軍旅不隊。梅福之辭,合於大雅雖無老成,尚有典刑;殷監不遠,夏后所聞。遂從所好,全性市門。云敞之義,著於吳章,為仁由己,再入大府,清則濯纓,何遠之有?


\end{pinyinscope}