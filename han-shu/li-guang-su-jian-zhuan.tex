\article{李廣蘇建傳}

\begin{pinyinscope}
李廣,隴西成紀人也。其先曰李信,秦時為將,逐得燕太子丹者也。廣世世受射。孝文十四年,匈奴大入蕭關,而廣以良家子從軍擊胡,用善射,殺首虜多,為郎,騎常侍。數從射獵,格殺猛獸,文帝曰:「惜廣不逢時,令當高祖世,萬戶侯豈足道哉!」

景帝即位,為騎郎將。吳楚反時,為驍騎都尉,從太尉亞夫戰昌邑下,顯名。以梁王授廣將軍印,故還,賞不行。為上谷太守,數與匈奴戰。典屬國公孫昆邪為上泣曰:「李廣材氣,天下亡雙,自負其能,數與虜确,恐亡之。」上乃徙廣為上郡太守。

匈奴入上郡,上使中貴人從廣勒習兵擊匈奴。中貴人者將數十騎從,見匈奴三人,與戰。射傷中貴人,殺其騎且盡。中貴人走廣,廣曰:「是必射鵰者也。」廣乃從百騎往馳三人。三人亡馬步行,行數十里。廣令其騎張左右翼,而廣身自射彼三人者,殺其二人,生得一人,果匈奴射鵰者也。已縛之上山,望匈奴數千騎,見廣,以為誘騎,驚,上山陳。廣之百騎皆大恐,欲馳還走。廣曰:「我去大軍數十里,今如此走,匈奴追射,我立盡。今我留,匈奴必以我為大軍之誘,不我擊。」廣令曰:「前!」未到匈奴陳二里所,止,令曰:「皆下馬解鞍!」騎曰:「虜多如是,解鞍,即急,柰何?」廣曰:「彼虜以我為走,今解鞍以示不去,用堅其意。」有白馬將出護兵。廣上馬,與十餘騎奔射殺白馬將,而復還至其百騎中,解鞍,縱馬臥。時會暮,胡兵終怪之,弗敢擊。夜半,胡兵以為漢有伏軍於傍欲夜取之,即引去。平旦,廣乃歸其大軍。後徙為隴西、北地、雁門、雲中太守。

武帝即位,左右言廣名將也,由是入為未央衛尉,而程不識時亦為長樂衛尉。程不識故與廣俱以邊太守將屯。及出擊胡,而廣行無部曲行陳,就善水草頓舍,人人自便,不擊刀斗自衛,莫府省文書,然亦遠斥候,未嘗遇害。程不識正部曲行伍營陳,擊刀斗,吏治軍簿至明,軍不得自便。不識曰:「李將軍極簡易,然虜卒犯之,無以禁;而其士亦佚樂,為之死。我軍雖煩擾,虜亦不得犯我。」是時漢邊郡李廣、程不識為名將,然匈奴畏廣,士卒多樂從,而苦程不識。不識孝景時以數直諫為太中大夫,為人廉,謹於文法。

後漢誘單于以馬邑城,使大軍伏馬邑傍,而廣為驍騎將軍,屬護軍將軍。單于覺之,去,漢軍皆無功。後四歲,廣以衛尉為將軍,出雁門擊匈奴。匈奴兵多,破廣軍,生得廣。單于素聞廣賢,令曰:「得李廣必生致之。」胡騎得廣,廣時傷,置兩馬間,絡而盛

之臥。行十餘里,廣陽死,睨其傍有一兒騎善馬,暫騰而上胡兒馬,因抱兒鞭馬南馳數十里,得其餘軍。匈奴騎數百追之,廣行取兒弓射殺追騎,以故得脫。於是至漢,漢下廣吏。吏當廣亡失多,為虜所生得,當斬,贖為庶人。

數歲,與故潁陰侯屏居藍田南山中射獵。嘗夜從一騎出,從人田間飲。還至亭,霸陵尉醉,呵止廣,廣騎曰:「故李將軍。」尉曰:「今將軍尚不得夜行,何故也!」宿廣亭下。居無何,匈奴入遼西,殺太守,敗韓將軍。韓將軍後徙居右北平,死。於是上乃召拜廣為右北平太守。廣請霸陵尉與俱,至軍而斬之,上書自陳謝罪。上報曰:「將軍者,國之爪牙也。司馬法曰:『登車不式,遭喪不服,振旅撫師,以征不服;率三軍之心,同戰士之力,故怒形則千里竦,威振則萬物伏;是以名聲暴於夷貉,威稜憺乎鄰國。』夫報忿除害,捐殘去殺,朕之所圖於將軍也;若乃免冠徒跣,稽顙請罪,豈朕之指哉!將軍其率師東轅,彌節白檀,以臨右北平盛秋。」廣在郡,匈奴號曰「漢飛將軍」,避之,數歲不入界。

廣出獵,見草中石,以為虎而射之,中石沒矢,視之,石也。他日射之,終不能入矣。廣所居郡聞有虎,常自射之。及居右北平射虎,虎騰傷廣,廣亦射殺之。

石建卒,上召廣代為郎中令。元朔六年,廣復為將軍,從大將軍出定襄。諸將多中首虜率為侯者,而廣軍無功。後三歲,廣以郎中令將四千騎出右北平,博望侯張騫將萬騎與廣俱,異道。行數百里,匈奴左賢王將四萬騎圍廣,廣軍士皆恐,廣乃使其子敢往馳之。敢從數十騎直貫胡騎,出其左右而還,報廣曰:「胡虜易與耳。」軍士乃安。為圜陳外鄉,胡急擊,矢下如雨。漢兵死者過半,漢矢且盡。廣乃令持滿毋發,而廣身自以大黃射其裨將,殺數人,胡虜益解。會暮,吏士無人色,而廣意氣自如,益治軍。軍中服其勇也。明日,復力戰,而博望侯軍亦至,匈奴乃解去。漢軍罷,弗能追。是時廣軍幾沒,罷歸。漢法,博望侯後期,當死,贖為庶人。廣軍自當,亡賞。

初,廣與從弟李蔡俱為郎,事文帝。景帝時,蔡積功至二千石。武帝元朔中,為輕車將軍,從大將軍擊右賢王,有功中率,封為樂安侯。元狩二年,代公孫弘為丞相。蔡為人在下中,名聲出廣下遠甚,然廣不得爵邑,官不過九卿。廣之軍吏及士卒或取封侯。廣與望氣王朔語云:「自漢擊匈奴,廣未嘗不在其中,而諸妄校尉已下,材能不及中,以軍功取侯者數十人。廣不為後人,然終無尺寸功以得封邑者,何也?豈吾相不當侯邪?」朔曰:「將軍自念,豈嘗有恨者乎?」廣曰:「吾為隴西守,羌嘗反,吾誘降者八百餘人,詐而同日殺之,至今恨獨此耳。」朔曰:「禍莫大於殺已降,此乃將軍所以不得侯者也。」

廣歷七郡太守,前後四十餘年,得賞賜,輒分其戲下,飲食與士卒共之。家無餘財,終不言生產事。為人長,爰臂,其善射亦天性,雖子孫他人學者莫能及。廣吶口少言,與人居,則畫地為軍陳,射闊狹以飲。專以射為戲。將兵乏絕處見水,士卒不盡飲,不近水,不盡餐,不嘗食。寬緩不苛,士以此愛樂為用。其射,見敵,非在數十步之內,度不中不發,發即應弦而倒。用此,其將數困辱,及射猛獸,亦數為所傷云。

元狩四年,大將軍票騎將軍大擊匈奴,廣數自請行。上以為老,不許;良久乃許之,以為前將軍。

大將軍青出塞,捕虜知單于所居,乃自以精兵走之,而令廣并於右將軍軍,出東道。東道少回遠,大軍行,水草少,甚勢不屯行。廣辭曰:「臣部為前將軍,今大將軍乃徙臣出東道,且臣結髮而與匈奴戰,乃今一得當單于,臣願居前,先死單于。」大將軍陰受上指,以為李廣數奇,毋令當單于,恐不得所欲。是時公孫敖新失侯,為中將軍,大將軍亦欲使敖與俱當單于,故徙廣。廣知之,固辭。大將軍弗聽,令長史封書與廣之莫府,曰:「急詣部,如書。」廣不謝大將軍而起行,意象慍怒而就部,引兵與右將軍食其合軍出東道。惑失道,後大將軍。大將軍與單于接戰,單于遁走,弗能得而還。南絕幕,乃遇兩將軍。廣已見大將軍,還入軍。大將軍使長史持糒醪遺廣,因問廣、食其失道狀,曰:「青欲上書報天子失軍曲折。」廣未對。大將軍長史急責廣之莫府上簿。廣曰:「諸校尉亡罪,乃我自失道。吾今自上簿。」

至莫府,謂其麾下曰:「廣結髮與匈奴大小七十餘戰,今幸從大將軍出接單于兵,而大將軍徙廣部行回遠,又迷失道,豈非天哉!且廣年六十餘,終不能復對刀筆之吏矣!」遂引刀自剄。百姓聞之,知與不知,老壯皆為垂泣。而右將軍獨下吏,當死,贖為庶人。

廣三子,曰當戶、椒、敢,皆為郎。上與韓嫣戲,嫣少不遜,當戶擊嫣,嫣走,於是上以為能。當戶蚤死,乃拜椒為代郡太守,皆先廣死。廣死軍中時,敢從票騎將軍。廣死明年,李蔡以丞相坐詔賜冢地陽陵當得二十畝,蔡盜取三頃,頗賣得四十餘萬,又盜取神道外壖地一畝葬其中,當下獄,自殺。敢以校尉從票騎將軍擊胡左賢王,力戰,奪左賢王旗鼓,斬首多,賜爵關內侯,食邑二百戶,代廣為郎中令。頃之,怨大將軍青之恨其父,乃擊傷大將軍,大將軍匿諱之。居無何,敢從上雍,至甘泉宮獵,票騎將軍去病怨敢傷青,射殺敢。去病時方貴幸,上為諱,云鹿觸殺之。居歲餘,去病死。

敢有女為太子中人,愛幸。敢男禹有寵於太子,然好利,亦有勇。嘗與侍中貴人飲,侵陵之,莫敢應。後愬之上,上召禹,使刺虎,縣下圈中,未至地,有詔引出之。禹從落中以劍斫絕纍,欲刺虎。上壯之,遂救止焉。而當戶有遺腹子陵,將兵擊胡,兵敗,降匈奴。後人告禹謀欲亡從陵,下吏死。

陵字少卿,少為侍中建章監。善騎射,愛人,謙讓下士,甚得名譽。武帝以為有廣之風,使將八百騎,深入匈奴二千餘里,過居延視地形,不見虜,還。拜為騎都尉,將勇敢五千人,教射酒泉、張掖以備胡。數年,漢遣貳師將軍伐大宛,使陵將五校兵隨後。行至塞,會貳師還。上賜陵書,陵留吏士,與輕騎五百出敦煌,至鹽水,迎貳師還,復留屯張掖。

天漢二年,貳師將三萬騎出酒泉,擊右賢王於天山。召陵,欲使為貳師將輜重。陵召見武臺,叩頭自請曰:「臣所將屯邊者,皆荊楚勇士奇材劍客也,力扼虎,射命中,願得自當一隊,到蘭干山南以分單于兵,毋令專鄉貳師軍。」上曰:「將惡相屬邪!吾發軍多,毋騎予女。」陵對:「對所事騎,臣願以少擊眾,步兵五千人涉單于庭。」上壯而許之,因詔彊弩都尉路博德將兵半道迎陵軍。博德故伏波將軍,亦羞為陵後距,奏言:「方秋匈奴馬肥,未可與戰,臣願留陵至春,俱將酒泉、張掖騎各五千人並擊東西浚稽,可必禽也。」書奏,上怒,疑陵悔不欲出而教博德上書,乃詔博德:「吾欲予李陵騎,云『欲以少擊眾』。今虜入西河,其引兵走西河,遮鉤營之道。」詔陵:「以九月發,出遮虜鄣,至東浚稽山南龍勒水上,徘徊觀虜,即亡所見,從浞野侯趙破奴故道抵受降城休士,因騎置以聞。所與博德言者云何?具以書對。」陵於是將其步卒五千人出居延,北行三十日,至浚稽山止營,舉圖所過山川地形,使麾下騎陳步樂還以聞。步樂召見,道陵將率得士死力,上甚說,拜步樂為郎。

陵至浚稽山,與單于相直,騎可三萬圍陵軍。軍居兩山間,以大車為營。陵引士出營外為陳,前行持戟盾,後行持弓弩,令曰:「聞鼓聲而縱,聞金聲而止。」虜見漢軍少,直前就營。陵搏戰攻之,千弩俱發,應弦而倒。虜還走上山,漢軍追擊,殺數千人。單于大驚,召左右地兵八萬餘騎攻陵。陵且戰且引,南行數日,抵山谷中。連戰,士卒中矢傷,三創者載輦,兩創者將車,一創者持兵戰。陵曰:「吾士氣少衰而鼓不起者,何也?軍中豈有女子乎?」始軍出時,關東群盜妻子徙邊者隨軍為卒妻婦,大匿車中。陵搜得,皆劍斬之。明日復戰,斬首三千餘級。引兵東南,循故龍城道行,四五日,抵大澤葭葦中,虜從上風縱火,陵亦令軍中縱火以自救。南行至山下,單于在南山上,使其子將騎擊陵。陵軍步鬥樹木間,復殺數千人,因發連弩射單于,單于下走。是日捕得虜,言「單于曰:『此漢精兵,擊之不能下,日夜引吾南近塞,得毋有伏兵乎?』諸當戶君長皆言『單于自將數萬騎擊漢數千人不能滅,後無以復使邊臣,令漢益輕匈奴。復力戰山谷間,尚四五十里得平地,不能破,乃還。』」

是時陵軍益急,匈奴騎多,戰一日數十合,復傷殺虜二千餘人。虜不利,欲去,會陵軍候管敢為校尉所辱,亡降匈奴,具言「陵軍無後救,射矢且盡,獨將軍麾下及成安侯校各八百人為前行,以黃與白為幟,當使精騎射之即破矣。」成安侯者,穎川人,父韓千秋,故濟南相,奮擊南越戰死,武帝封子延年為侯,以校尉隨陵。單于得敢大喜,使騎並攻漢軍,疾呼曰:「李陵、韓延年趣降!」遂遮道急攻陵。陵居谷中,虜在山上,四面射,矢如雨下。漢軍南行,未至鞮汗山,一日五十萬矢皆盡,即棄車去。士尚三千餘人,徒斬車輻而持之,軍吏持尺刀,抵山入骥谷。單于遮其後,乘隅下壘石,士卒多死,不得行。昏後,陵便衣獨步出營,止左右:「毋隨我,丈夫一取單于耳!」良久,陵還,大息曰:「兵敗,死矣!」軍吏或曰:「將軍威震匈奴,天命不遂,後求道徑還歸,如浞野侯為虜所得,後亡還,天子客遇之,況於將軍乎!」陵曰:「公止!吾不死,非壯士也。」於是盡斬旌旗,及珍寶埋地中,陵歎曰:「復得數十矢,足以脫矣。今無兵復戰,天明坐受縛矣!各鳥獸散,猶有得脫歸報天子者。」令軍士人持二升糒,一半冰,期至遮虜鄣者相待。夜半時,擊鼓起士,鼓不鳴。陵與韓延年俱上馬,壯士從者十餘人。虜騎數千追之,韓延年戰死。陵曰:「無面目報陛下!」遂降。軍人分散,脫至塞者四百餘人。

陵敗處去塞百餘里,邊塞以聞。上欲陵死戰,召陵母及婦,使相者視之,無死喪色。後聞陵降,上怒甚,責問陳步樂,步樂自殺。群臣皆罪陵,上以問太史令司馬遷,遷盛言:「陵事親孝,與士信,常奮不顧身以殉國家之急。其素所畜積也,有國士之風。今舉事一不幸,全軀保妻子之臣隨而媒糱其短,誠可痛也!且陵提步卒不滿五千,深輮戎馬之地,抑數萬之師,虜救死扶傷不暇,悉舉引弓之民共攻圍之。轉鬥千里,矢盡道窮,士張空拳,冒白刃,北首爭死敵,得人之死力,雖古名將不過也。身雖陷敗,然其所摧敗亦足暴於天下。彼之不死,宜欲得當以報漢也。」初,上遣貳師大軍出,財令陵為助兵,及陵與單于相值,而貳師功少。上以遷誣罔,欲沮貳師,為陵游說,下遷腐刑。

久之,上悔陵無救,曰:「陵當發出塞,乃詔彊弩都尉令迎軍。坐預詔之,得令老將生姦詐。」乃遣使勞賜陵餘軍得脫者。

陵在匈奴歲餘,上遣因杅將軍公孫敖將兵深入匈奴迎陵。敖軍無功還,曰:「捕得生口,李陵教單于為兵以備漢軍,故臣無所得。」上聞,於是族陵家,母弟妻子皆伏誅。隴西士大夫以李氏為愧。其後,漢遣使使匈奴,陵謂使者曰:「吾為漢將步卒五千人橫行匈奴,以亡救而敗,何負於漢而誅吾家?」使者曰:「漢聞李少卿教匈奴為兵。」陵曰:「乃李緒,非我也。」李緒本漢塞外都尉,居奚侯城,匈奴攻之,緒降,而單于客遇緒,常坐陵上。陵痛其家以李緒而誅,使人刺殺緒。大閼氏欲殺陵,單于匿之北方,大閼氏死乃還。

單于壯陵,以女妻之,立為右校王,衛律為丁靈王,皆貴用事。衛律者,父本長水胡人。律生長漢,善協律都尉李延年,延年薦言律使匈奴。使還,會延年家收,律懼并誅,亡還降匈奴。匈奴愛之,常在單于左右。陵居外,有大事,乃入議。

昭帝立,大將軍霍光、左將軍上官桀輔政,素與陵善,遣陵故人隴西任立政等三人俱至匈奴招陵。立政等至,單于置酒賜漢使者,李陵、衛律皆侍坐。立政等見陵,未得私語,即目視陵,而數數自循其刀環,握其足,陰諭之,言可還歸漢也。後陵、律持牛酒勞漢使,博飲,兩人皆胡服椎結。立政大言曰:「漢已大赦,中國安樂,主上富於春秋,霍子孟、上官少叔用事。」以此言微動之。陵墨不應,孰視而自循其髮,答曰:「吾已胡服矣!」有頃,律起更衣,立政曰:「咄,少卿良苦!霍子孟、上官少叔謝女。」陵曰:「霍與上官無恙乎?」立政曰:「請少卿來歸故鄉,毋憂富貴。」陵字立政曰:「少公,歸易耳,恐再辱,柰何!」語未卒,衛律還,頗聞餘語,曰:「李少卿賢者,不獨居一國。范蠡遍遊天下,由余去戎入秦,今何語之親也!」因罷去。立政隨謂陵曰:「亦有意乎?」陵曰:「丈夫不能再辱。」

陵在匈奴二十餘年,元平元年病死。

蘇建,杜陵人也。以校尉從大將軍青擊匈奴,封平陵侯。以將軍築朔方。後以衛尉為遊擊將軍,從大將軍出朔方。後一歲,以右將軍再從大將軍出定襄,亡翕侯,失軍當斬,贖為庶人。其後為代郡太守,卒官。有三子:嘉為奉車都尉,賢為騎都尉,中子武最知名。

武字子卿,少以父任,兄弟並為郎,稍遷至栘中廄監。時漢連伐胡,數通使相窺觀,匈奴留漢使郭吉、路充國等,前後十餘輩。匈奴使來,漢亦留之以相當。天漢元年,且鞮侯單于初立,恐漢襲之,乃曰:「漢天子我丈人行也。」盡歸漢使路充國等。武帝嘉其義,乃遣武以中郎將使持節送匈奴使留在漢者,因厚輅單于,答其善意。武與副中郎將張勝及假吏常惠等募士斥候百餘人俱。既至匈奴,置幣遺單于。單于益驕,非漢所望也。

方欲發使送武等,會緱王與長水虞常等謀反匈奴中。緱王者,昆邪王姊子也,與昆邪王俱降漢,後隨浞野侯沒胡中。及衛律所將降者,陰相與謀劫單于母閼氏歸漢。會武等至匈奴,虞常在漢時素與副張勝相知,私候勝曰:「聞漢天子甚怨衛律,常能為漢伏弩射殺之。吾母與弟在漢,幸蒙其賞賜。」張勝許之,以貨物與常。後月餘,單于出獵,獨閼氏子弟在。虞常等七十餘人欲發,其一人夜亡,告之。單于子弟發兵與戰。緱王等皆死,虞常生得。

單于使衛律治其事。張勝聞之,恐前語發,以狀語武。武曰:「

事如此,此必及我。見犯乃死,重負國。」欲自殺,勝、惠共止之。虞常果引張勝。單于怒,召諸貴人議,欲殺漢使者。左伊秩訾曰:「即謀單于,何以復加?宜皆降之。」單于使衛律召武受辭,武謂惠等:「屈節辱命,雖生,何面目以歸漢!」引佩刀自刺。衛律驚,自抱持武,馳召毉。鑿地為坎,置熅火,覆武其上,蹈其背以出血。武氣絕,半日復息。惠等哭,輿歸營。單于壯其節,朝夕遣人候問武,而收繫張勝。

武益愈,單于使使曉武。會論虞常,欲因此時降武。劍斬虞常已,律曰:「漢使張勝謀殺單于近臣,當死,單于募降者赦罪。」舉劍欲擊之,勝請降。律謂武曰:「副有罪,當相坐。」武曰:「本無謀,又非親屬,何謂相坐?」復舉劍擬之,武不動。律曰:「蘇君,律前負漢歸匈奴,幸蒙大恩,賜號稱王,擁眾數萬,馬畜彌山,富貴如此。蘇君今日降,明日復然。空以身膏草野,誰復知之!」武不應。律曰:「君因我降,與君為兄弟,今不聽吾計,後雖欲復見我,尚可得乎?」武罵律曰:「女為人臣子,不顧恩義,畔主背親,為降虜於蠻夷,何以女為見?且單于信女,使決人死生,不平心持正,反欲鬥兩主,觀禍敗。南越殺漢使者,屠為九郡;宛王殺漢使者,頭縣北闕;朝鮮殺漢使者,即時誅滅。獨匈奴未耳。若知我不降明,欲令兩國相攻,匈奴之禍從我始矣。」

律知武終不可脅,白單于。單于愈益欲降之,乃幽武置大窖中,絕不飲食。天雨雪,武臥齧雪與旃毛并咽之,數日不死,匈奴以為神,乃徙武北海上無人處,使牧羝,羝乳乃得歸。別其官屬常惠等,各置他所。

武既至海上,廩食不至,掘野鼠去屮實而食之。杖漢節牧羊,臥起操持,節旄盡落。積五六年,單于弟於靬王弋射海上。武能網紡繳,檠弓弩,於靬王愛之,給其衣食。三歲餘,王病,賜武馬畜服匿穹廬。王死後,人眾徙去。其冬,丁令盜武牛羊,武復窮厄。

初,武與李陵俱為侍中,武使匈奴明年,陵降,不敢求武。久之,單于使陵至海上,為武置酒設樂,因謂武曰:「單于聞陵與子卿素厚,故使陵來說足下,虛心欲相待。終不得歸漢,空自苦亡人之地,信義安所見乎?前長君為奉車,從至雍棫陽宮,扶輦下除,觸柱折轅,劾大不敬,伏劍自刎,賜錢二百萬以葬。孺卿從祠河東后土,宦騎與黃門駙馬爭船,推墮駙馬河中溺死,宦騎亡,詔使孺卿逐捕不得,惶恐飲藥而死。來時,大夫人已不幸,陵送葬至陽陵。子卿婦年少,聞已更嫁矣。獨有女弟二人,兩女一男,今復十餘年,存亡不可知。人生如朝露,何久自苦如此!陵始降時,忽忽如狂,自痛負漢,加以老母繫保宮,子卿不欲降,何以過陵?且陛下春秋高,法令亡常,大臣亡罪夷滅者數十家,安危不可知,子卿尚復誰為乎?願聽陵計,勿復有云。」武曰:「武父子亡功德,皆為陛下所成就,位列將,爵通侯,兄弟親近,常願肝腦塗地。今得殺身自效,雖蒙斧鉞湯鑊,誠甘樂之。臣事君,猶子事父也,子為父死亡所恨。願勿復再言。」陵與武飲數日,復曰:「子卿壹聽陵言。」武曰:「自分已死久矣!王必欲降武,請畢今日之驩,效死於前!」陵見其至誠,喟然歎曰:「

嗟乎,義士!陵與衛律之罪上通於天。」因泣下霑衿,與武決去。

陵惡自賜武,使其妻賜武牛羊數十頭。後陵復至北海上,語武:「區脫捕得雲中生口,言太守以下吏民皆白服,曰上崩。」武聞之,南鄉號哭,歐血,旦夕臨。

數月,昭帝即位。數年,匈奴與漢和親。漢求武等,匈奴詭言武死。後漢使復至匈奴,常惠請其守者與俱,得夜見漢使,具自陳道。教使者謂單于,言天子射上林中,得雁,足有係帛書,言武等在某澤中。使者大喜,如惠語以讓單于。單于視左右而驚,謝漢使曰:「武等實在。」於是李陵置酒賀武曰:「今足下還歸,揚名於匈奴,功顯於漢室,雖古竹帛所載,丹青所畫,何以過子卿!陵雖駑怯,令漢且貰陵罪,全其老母,使得奮大辱之積志,庶幾乎曹柯之盟,此陵宿昔之所不忘也。收族陵家,為世大戮,陵尚復何顧乎?已矣!令子卿知吾心耳。異域之人,壹別長絕!」陵起舞,歌曰:「徑萬里兮度沙幕,為君將兮奮匈奴。路窮絕兮矢刃摧,士眾滅兮名已隤。老母已死,雖欲報恩將安歸!」陵泣下數行,因與武決。單于召會武官屬,前以降及物故,凡隨武還者九人。

武以元始六年春至京師。詔武奉一太牢謁武帝園廟,拜為典屬國,秩中二千石,賜錢二百萬,公田二頃,宅一區。常惠、徐聖、趙終根皆拜為中郎,賜帛各二百匹。其餘六人老歸家,賜錢人十萬,復終身。常惠後至右將軍,封列侯,自有傳。武留匈奴凡十九歲,始以彊壯出,及還,須髮盡白。

武來歸明年,上官桀子安與桑弘羊及燕王、蓋王謀反。武子男元與安有謀,坐死。

初桀、安與大將軍霍光爭權,數疏光過失予燕王,令上書告之。又言蘇武使匈奴二十年不降,還乃為典屬國,大將軍長史無功勞,為搜粟都尉,光顓權自恣。及燕王等反誅,窮治黨與,武素與桀、弘羊有舊,數為燕王所訟,子又在謀中,廷尉奏請逮捕武。霍光寢其奏,免武官。

數年,昭帝崩,武以故二千石與計謀立宣帝,賜爵關內侯,食邑三百戶。久之,衛將軍張安世薦武明習故事,奉使不辱命,先帝以為遺言。宣帝即時召武待詔宦者署,數進見,復為右曹典屬國。以武著節老臣,令朝朔望,號稱祭酒,甚優寵之。

武所得賞賜,盡以施予昆弟故人,家不餘財。皇后父平恩侯、帝舅平昌侯、樂昌侯、車騎將軍韓增、丞相魏相、御史大夫丙吉皆敬重武。武年老,子前坐事死,上閔之,問左右:「武在匈奴久,豈有子乎?」武因平恩侯自白:「前發匈奴時,胡婦適產一子通國,有聲問來,願因使者致金帛贖之。」上許焉。後通國隨使者至,上以為郎。又以武弟子為右曹。武年八十餘,神爵二年病卒。

甘露三年,單于始入朝。上思股肱之美,乃圖畫其人於麒麟閣,法其形貌,署其官爵姓名。唯霍光不名,曰大司馬大將軍博陸侯姓霍氏,次曰衛將軍富平侯張安世,次曰車騎將軍龍镪侯韓增,次曰後將軍營平侯趙充國,次曰丞相高平侯魏相,次曰丞相博陽侯丙吉,次曰御史大夫建平侯杜延年,次曰宗正陽城侯劉德,次曰少府梁丘賀,次曰太子太傅蕭望之,次曰典屬國蘇武。皆有功德,知名當世,是以表而揚之,明著中興輔佐,列於方叔、召虎、仲山甫焉。凡十一人,皆有傳。自丞相黃霸、廷尉于定國、大司農朱邑、京兆尹張敞、右扶風尹翁歸及儒者夏侯勝等,皆以善終,著名宣帝之世,然不得列於名臣之圖,以此知其選矣。

贊曰:李將軍恂恂如鄙人,口不能出辭,及死之日,天下知與不知皆為流涕,彼其中心誠信於士大夫也。諺曰:「桃李不言,下自成蹊。」此言雖小,可以喻大。然三代之將,道家所忌,自廣至陵,遂亡其宗,哀哉!孔子稱「志士仁人,有殺身以成仁,無求生以害仁」,「使於四方,不辱君命」,蘇武有之矣。


\end{pinyinscope}