\article{馮奉世傳}

\begin{pinyinscope}
馮奉世字子明,上黨潞人也,徙杜陵。其先馮亭,為韓上黨守。奏攻上黨,絕太行道,韓不能守,馮亭乃入上黨城守於趙。趙封馮亭為華陽君,與趙將括距秦,戰死於長平。宗族繇是分散,或留潞,或在趙。在趙者為官帥將,官帥將子為代相。及秦滅六國,而馮亭之後馮毋擇、馮去疾、馮劫皆為秦將相焉。

漢興,文帝時馮唐顯名,即代相子也。至武帝末,奉世以良家子選為郎。昭帝時,以功次補武安長。失官,年三十餘矣,乃學春秋涉大義,讀兵法明習,前將軍韓增奏以為軍司空令。本始中,從軍擊匈奴。軍罷,復為郎。

先是時,漢數出使西域,多辱命不稱,或貪汙,為外國所苦。是時烏孫大有擊匈奴之功,而西域諸國新輯,漢方善遇,欲以安之,選可使外國者。前將軍增舉奉世以衛候使持節送大宛諸國客。至伊脩城,都尉宋將言莎車與旁國共攻殺漢所置莎車王萬年,并殺漢使者奚充國。時匈奴又發兵攻車師城,不能下而去。莎車遣使揚言北道諸國已屬匈奴矣,於是攻劫南道,與歃盟畔漢,從鄯善以西皆絕不通。都護鄭吉、校尉司馬意皆在北道諸國間。奉世與其副嚴昌計,以為不亟擊之則莎車日彊,其勢難制,必危西域。遂以節諭告諸國王,因發其兵,南北道合萬五千人進擊莎車,攻拔其城。莎車王自殺,傳其首詣長安。諸國悉平,威振西域。奉世乃罷兵以聞。宣帝召見韓增,曰:「賀將軍所舉得其人。」奉世遂西至大宛。大宛聞其斬莎車王,敬之異於它使。得其名馬象龍而還。上甚說,下議封奉世。丞相、將軍皆曰:「春秋之義,大夫出疆,有可以安國家,則顓之可也。奉世功效尤著,宜加爵土之賞。」少府蕭望之獨以奉世奉使有指,而擅矯制違命,發諸國兵,雖有功效,不可以為後法。即封奉世,開後奉使者利,以奉世為比,爭逐發兵,要功萬里之外,為國家生事於夷狄。漸不可長,奉世不宜受封。上善望之議,以奉世為光祿大夫、水衡都尉。

元帝即位,為執金吾。上郡屬國歸義降胡萬餘人反去。初,昭帝末,西河屬國胡伊酋若王亦將眾數千人畔,奉世輒持節將兵追擊。右將軍典屬國常惠薨,奉世代為右將軍典屬國,加諸吏之號。數歲,為光祿勳。

永光二年秋,隴西羌彡姐旁種反,詔召丞相韋玄成、御史大夫鄭弘、大司馬車騎將軍王接、左將軍許嘉、右將軍奉世入議。是歲時比不登,京師穀石二百餘,邊郡四百,關東五百。四方饑饉,朝廷方以為憂,而遭羌變。玄成等漠然莫有對者。奉世曰:「羌虜近在竟內背畔,不以時誅,亡以威制遠蠻。臣願帥師討之。」上問用兵之數,對曰:「臣聞善用兵者,役不再興,糧不三載,故師不久暴而天誅亟決。往者數不料敵,而師至於折傷;再三發軵,則曠日煩費,威武虧矣。今反虜無慮三萬人,法當倍用六萬人。然羌戎弓矛之兵耳,器不犀利,可用四萬人,一月足以決。」丞相、御史、兩將軍皆以為民方收斂時,未可多發;萬人屯守之,且足。奉世曰:「

不可。天下被饑饉,士馬羸秏,守戰之備久廢不簡,夷狄皆有輕邊吏之心,而羌首難。今以萬人分屯數處,虜見兵少,必不畏懼,戰則挫兵病師,守則百姓不救。如此,怯弱之形見,羌人乘利,諸種並和,相扇而起,臣恐中國之役不得止於四萬,非財幣所能解也。故少發師而曠日,與一舉而疾決,利害相萬也。」固爭之,不能得。有詔益二千人。

於是遣奉世將萬二千人騎,以將屯為名。典屬國任立、護軍都尉韓昌為偏裨,到隴西,分屯三處。典屬國為右軍,屯白石;護軍都尉為前軍,屯臨洮;奉世為中軍,屯首陽西極上。前軍到降同阪,先遣校尉在前與羌爭地利,又別遣校尉救民於廣陽谷。羌虜盛多,皆為所破,殺兩校尉。奉世具上地形部眾多少之計,願益三萬六千人乃足以決事。書奏,天子大為發兵六萬餘人,拜太常弋陽侯任千秋為奮武將軍以助焉。奉世上言:「願得其眾,不須復煩大將。」因陳轉輸之費。

上於是以璽書勞奉世,且讓之,曰:「皇帝問將兵右將軍,甚苦暴露。羌虜侵邊境,殺吏民,甚逆天道,故遣將軍帥士大夫行天誅。以將軍材質之美,奮精兵,誅不軌,百下百全之道也。今乃有畔敵之名,大為中國羞。以昔不閑習之故邪?以恩厚未洽,信約不明也?朕甚怪之。上書言羌虜依深山,多徑道,不得不多分部遮要害,須得後發營士,足以決事,部署已定,勢不可復置大將,聞之。前為將軍兵少,不足自守,故發近所騎,日夜詣,非為擊也。今發三輔、河東、弘農越騎、跡射、佽飛、彀者、羽林孤兒及呼速絫、嗕種,方急遣。且兵,凶器也,必有成敗者,患策不豫定,料敵不審也,故復遣奮武將軍。兵法曰大將軍出必有偏裨,所以揚威武,參計策,將軍又何疑焉?夫愛吏士,得眾心,舉而無悔,禽敵必全,將軍之職也。若乃轉輸之費,則有司存,將軍勿憂。須奮武將軍兵到,合擊羌虜。」

十月,兵畢至隴西。十一月,並進。羌虜大破,斬首數千級,餘皆走出塞。兵未決間,漢復發募士萬人,拜定襄太守韓安國為建威將軍。未進,聞羌破,還。上曰:「羌虜破散創艾,亡出塞,其罷吏士,頗留屯田,備要害處。」

明年二月,奉世還京師,更為左將軍,光祿勳如故。其後錄功拜爵,下詔曰:「羌虜桀黠,賊害吏民,攻隴西府寺,燔燒置亭,絕道橋,甚逆天道。左將軍光祿勳奉世前將兵征討,斬捕首虜八千餘級,鹵馬牛羊以萬數。賜奉世爵關內侯,食邑五百戶,黃金六十斤。」裨將、校尉三十餘人,皆拜。

後歲餘,奉世病卒。居爪牙官前後十年,為折衝宿將,功名次趙充國。

奮武將軍任千秋者,其父宮,昭帝時以丞相徵事捕斬反者左將軍上官桀,封侯,宣帝時為太常,薨。千秋嗣後,復為太常。成帝時,樂昌侯王商代奉世為左將軍,而千秋為右將軍,後亦為左將軍。子孫傳國,至王莽乃絕云。

奉世死後二年,西域都護甘延壽以誅郅支單于封為列侯。時丞相匡衡亦用延壽矯制生事,據蕭望之前議,以為不當封,而議者咸美其功,上從眾而侯之。於是杜欽上疏,追訟奉世前功曰:「前莎車王殺漢使者,約諸國背畔。左將軍奉世以衛候便宜發兵誅莎車王,策定城郭,功施邊境。議者以奉世奉使有指,春秋之義亡遂事,漢家之法有矯制,故不得侯。今匈奴郅支單于殺漢使者,亡保康居,都護延壽發城郭兵屯田吏士四萬餘人以誅斬之,封為列侯。臣愚以為比罪則郅支薄,量敵則莎車眾,用師則奉世寡,計勝則奉世為功於邊境安,慮敗則延壽為禍於國家深。其違命而擅生事同,延壽割地封,而奉世獨不錄。臣聞功同賞異則勞臣疑,罪鈞刑殊則百姓惑;疑生無常,惑生不知所從;亡常則節趨不立,不知所從則百姓無所措手足。奉世圖難忘死,信命殊俗,威功白著,為世使表,獨抑厭而不揚,非聖主所以塞疑厲節之意也。願下有司議。」上以先帝時事,不復錄。

奉世有子男九人,女四人。長女媛以選充後宮,為元帝昭儀,產中山孝王。元帝崩,媛為中山太后,隨王就國。奉世長子譚,太常舉孝廉為郎,功次補天水司馬。奉世擊西羌,譚為校尉,隨父從軍有功,未拜病死。譚弟野王、逡、立、參至大官。

野王字君卿,受業博士,通詩。少以父任為太子中庶子。年十八,上書願試守長安令。宣帝奇其志,問丞相魏相,相以為不可許。後以功次補當陽長,遷為櫟陽令,徙夏陽令。元帝時,遷隴西太守,以治行高,入為左馮翊。歲餘,而池陽令並素行貪汙,輕野王外戚年少,治行不改。野王部督郵掾祋祤趙都案驗,得其主守盜十金罪,收捕。並不首吏,都格殺。並家上書陳冤,事下廷尉。都詣吏自殺以明野王,京師稱其威信,遷為大鴻臚。

數年,御史大夫李延壽病卒,在位多舉野王。上使尚書選第中二千石,而野王行能第一。上曰:「吾用野王為三公,後世必謂我私後宮親屬,以野王為比。」乃下詔曰:「剛彊堅固,確然亡欲,大鴻臚野王是也。心辨善辭,可使四方,少府五鹿充宗是也。廉絜節儉,太子少傅張譚是也。其以少傅為御史大夫。」上繇下第而用譚,越次避嫌不用野王,以昭儀兄故也。野王乃歎曰:「人皆以女寵貴,我兄弟獨以賤!」野王雖不為三公,甚見器重,有名當世。

成帝立,有司奏野王王舅,不宜備九卿。以秩出為上郡太守,加賜黃金百斤。朔方剌史蕭育奏封事,薦言「野王行能高妙,內足與圖身,外足以慮化。竊惜野王懷國之寶,而不得陪朝廷與朝者並。野王前以王舅出,以賢復入,明國家樂進賢也。」上自為太子時聞知野王。會其病免,復以故二千石使行河隄,因拜為琅邪太守。是時,成帝長舅陽平侯王鳳為大司馬大將軍,輔政八九年矣,時數有災異,京兆尹王章譏鳳顓權不可任用,薦野王代鳳。上初納其言,而後誅章,語在元后傳。於是野王懼不自安,遂病,滿三月賜告,與妻子歸杜陵就醫藥。大將軍鳳風御史中丞劾奏野王賜告養病而私自便,持虎符出界歸家,奉詔不敬。杜欽時在大將軍莫府,欽素高野王父子行能,奏記於鳳,為野王言曰:「竊見令曰,吏二千石告,過長安謁,不分別予賜。今有司以為予告得歸,賜告不得,是一律兩科,失省刑之意。夫三最予告,令也;病滿三月賜告,詔恩也。令告則得,詔恩則不得,失輕重之差。又二千石病賜告得歸有故事,不得去郡亡著令。傳曰:『賞疑從予,所以廣恩勸功也;罰疑從去,所以慎刑,闕難知也。』今釋令與故事而假不敬之法,甚違闕疑從去之意。即以二千石守千里之地,任兵馬之重,不宜去郡,將以制刑為後法者,則野王之罪,在未制令前也。刑賞大信,不可不慎。」鳳不聽,竟免野王。郡國二千石病賜告不得歸家,自此始。

初,野王嗣父爵為關內侯,免歸。數年,年老,終于家。子座嗣爵,至孫坐中山太后事絕。

逡字子產,通易。太常察孝廉為郎,補謁者。建昭中,選為復土校尉。光祿勳于永舉茂材,為美陽令。功次遷長樂屯衛司馬,清河都尉,隴西太守。治行廉平,年四十餘卒。為都尉時,言河隄方略,在溝洫志。

立字聖卿,通春秋。以父任為郎,稍遷諸曹。竟寧中,以王舅出為五原屬國都尉。數年,遷五原太守,徙西河、上郡。立居職公廉,治行略與野王相似,而多知有恩貸,好為條教。吏民嘉美野王、立相代為太守,歌之曰:「大馮君,小馮君,兄弟繼踵相因循,聰明賢知惠吏民,政如魯、衛德化鈞,周公、康叔猶二君。」後遷為東海太守,下溼病痺。天子聞之,徙立為太原太守。更歷五郡,所居有跡。年老卒官。

參字叔平,學通尚書。少為黃門郎給事中,宿衛十餘年。參為人矜嚴,好修容儀,進退恂恂,甚可觀也。參,昭儀少弟,行又敕備,以嚴見憚,終不得親近侍帷幄。竟寧中,以王舅出補渭陵食官令。以數病徙為寢中郎,有詔勿事。陽朔中,中山王來朝,參擢為上河農都尉。病免官,復為渭陵寢中郎。永始中,超遷代郡太守。以邊郡道遠,徙為安定太守。數歲,病免,復為諫大夫,使領護左馮翊都水。綏和中,立定陶王為皇太子,以中山王見廢,故封王舅參為宜鄉侯,以慰王意。參之國,上書願至中山見王、太后。行未到而王薨。王病時,上奏願貶參爵以關內侯食邑留長安。上憐之,下詔曰:「中山孝王短命早薨,願以舅宜鄉侯參為關內侯,歸家,朕甚愍之。其還參京師,以列侯奉朝請。」五侯皆敬憚之。丞相翟方進亦甚重焉,數謂參:「物禁太甚。君侯以王舅見廢,不得在公卿位,今五侯至尊貴也,與之並列,宜少詘節卑體,視有所宗。而君侯盛修容貌以威嚴加之,此非所以下五侯而自益者也。」參性好禮儀,終不改其恆操。頃之,哀帝即位,帝祖母傅太后用事,追怨參姊中山太后,陷以祝詛大逆大罪,語在外戚傳。參以同產當相坐,謁者承制召參詣廷尉,參自殺。且死,仰天歎曰:「參父子兄弟皆備大位,身至封侯,今被惡名而死,姊弟不敢自惜,傷無以見先人於地下!」死者十七人,眾莫不憐之。宗族徙歸故郡。

贊曰:詩稱「抑抑威儀,惟德之隅」。宜鄉侯參鞠躬履方,擇地而行,可謂淑人君子,然卒死於非罪,不能自免,哀哉!讒邪交亂,貞良被害,自古而然。故伯奇放流,孟子宮刑,申生雉經,屈原赴湘,小弁之詩作,離騷之辭興。經曰:「心之憂矣,涕既隕之。」馮參姊弟,亦云悲矣!


\end{pinyinscope}