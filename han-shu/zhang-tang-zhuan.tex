\article{張湯傳}

\begin{pinyinscope}
張湯,杜陵人也。父為長安丞,出,湯為兒守舍。還,鼠盜肉,父怒,笞湯。湯掘熏得鼠及餘肉,劾鼠掠治,傳爰書,訊鞫論報,并取鼠與肉,具獄磔堂下。父見之,視文辭如老獄吏,大驚,遂使書獄。

父死後,湯為長安吏。周陽侯為諸卿時,嘗繫長安,湯傾身事之。及出為侯,大與湯交,遍見貴人。湯給事內史,為甯成掾,以湯為無害,言大府,調茂陵尉,治方中。

武安侯為丞相,徵湯為史,薦補侍御史。治陳皇后巫蠱獄,深竟黨與,上以為能,遷太中大夫。與趙禹共定諸律令,務在深文,拘守職之吏。已而禹至少府,湯為廷尉,兩人交驩,兄事禹。禹志在奉公孤立,而湯舞知以御人。始為小吏,乾沒,與長安富賈田甲、魚翁叔之屬交私。及列九卿,收接天下名士大夫,己心內雖不合,然陽浮道與之。

是時,上方鄉文學,湯決大獄,欲傅古義,乃請博士弟子治尚書、春秋,補廷尉史,平亭疑法。奏讞疑,必奏先為上分別其原,上所是,受而著讞法廷尉挈令,揚主之明。奏事即譴,湯摧謝,鄉上意所便,必引正監掾史賢者,曰:「固為臣議,如此上責臣,臣弗用,愚抵此。」罪常釋。間即奏事,上善之,曰:「臣非知為此奏,乃監、掾、史某所為。」其欲薦吏,揚人之善解人之過如此。所治即上意所欲罪,予監吏深刻者;即上意所欲釋,予監吏輕平者。所治即豪,必舞文巧詆;即下戶羸弱,時口言「雖文致法,上裁察。」於是往往釋湯所言。湯至於大吏,內行修,交通賓客飲食,於故人子弟為吏及貧昆弟,調護之尤厚。其造請諸公,不避寒暑。是以湯雖文深意忌不專平,然得此聲譽。而深刻吏多為爪牙用者,依於文學之士。丞相弘數稱其美。

及治淮南、衡山、江都反獄,皆窮根本。嚴助、伍被,上欲釋之,湯爭曰:「伍被本造反謀,而助親幸出入禁闥腹心之臣,乃交私諸侯,如此弗誅,後不可治。」上可論之。其治獄所巧排大臣自以為功,多此類。繇是益尊任,遷御史大夫。

會渾邪等降漢,大興兵伐匈奴,山東水旱,貧民流徙,皆卬給縣官,縣官空虛。湯承上指,請造白金及五銖錢,籠天下鹽鐵,排富商大賈,出告緡令,鉏豪彊并兼之家,舞文巧詆以輔法。湯每朝奏事,語國家用,日旰,天子忘食。丞相取充位,天下事皆決湯。百姓不安其生,騷動,縣官所興未獲其利,姦吏並侵漁,於是痛繩以罪。自公卿以下至於庶人咸指湯。湯嘗病,上自至舍視,其隆貴如此。

匈奴求和親,群臣議前,博士狄山曰:「和親便。」上問其便,山曰:「兵,凶器,未易數動。高帝欲伐匈奴,大困平城,乃遂結和親。孝惠、高后時,天下安樂,及文帝欲事匈奴,北邊蕭然苦兵。孝景時,吳楚七國反,景帝往來東宮間,天下寒心數月。吳楚已破,竟景帝不言兵,天下富實。今自陛下興兵擊匈奴,中國以空虛,邊大困貧。由是觀之,不如和親。」上問湯,湯曰:「此愚儒無知。」狄山曰:「臣固愚忠,若御史大夫湯,乃詐忠。湯之治淮南、江都,以深文痛詆諸侯,別疏骨肉,使藩臣不自安,臣固知湯之為詐忠。」於是上作色曰:「吾使生居一郡,能無使虜入盜乎?」山曰:「不能。」曰:「居一縣?」曰:「不能。」復曰:「居一鄣間?」山自度辯窮且下吏,曰:「能。」乃遣山乘鄣。至月餘,匈奴斬山頭而去。是後群臣震讋。

湯客田甲雖賈人,有賢操,始湯為小吏,與錢通,及為大吏,而甲所以責湯行義,有烈士之風。

湯為御史大夫七歲,敗。

河東人李文,故嘗與湯有隙,已而為御史中丞,薦數從中文事有可以傷湯者,不能為地。湯有所愛史魯謁居,知湯弗平,使人上飛變告文姦事。事下湯,湯治論殺文,而湯心知謁居為之。上問:「變事從跡安起?」湯陽驚曰:「此殆文故人怨之。」謁居病臥閭里主人,湯自往視病,為謁居摩足。趙國以冶鑄為業,王數訟鐵官事,湯常排趙王。趙王求湯陰事。謁居嘗案趙王,趙王怨之,并上書告:「湯大臣也,史謁居有病,湯至為摩足,疑與為大姦。」事下廷尉。謁居病死,事連其弟,弟繫導官。湯亦治它囚導官,見謁居弟,欲陰為之,而陽不省。謁居弟不知而怨湯,使人上書,告湯與謁居謀,兵變李文。事下減宣。宣嘗與湯有隙,及得此事,窮竟其事,未奏也。會人有盜發孝文園瘞錢,丞相青翟朝,與湯約俱謝,至前,湯念獨丞相以四時行園,當謝,湯無與也,不謝。丞相謝,上使御史案其事。湯欲致其文丞相見知,丞相患之。三長史皆害湯,欲陷之。

始,長史朱買臣素怨湯,語在其傳。王朝,齊人,以術至右內史。邊通學短長,剛暴人也,官至濟南相。故皆居湯右,已而失官,守長史,詘體於湯。湯數行丞相事,知此三長史素貴,常陵折之,故三長史合謀曰:「始湯約與君謝,已而賣君;今欲劾君以宗廟事,此欲代君耳。吾知湯陰事。」使吏捕案湯左田信等,曰湯且欲為請奏,信輒先知之,居物致富,與湯分之。及它姦事。事辭頗聞。上問湯曰:「吾所為,賈人輒知,益居其物,是類有以吾謀告之者。」湯不謝,又陽驚曰:「

固宜有。」減宣亦奏謁居事。上以湯懷詐面欺,使使八輩簿責湯。湯具自道無此,不服。於是上使趙禹責湯。禹至,讓湯曰:「君何不知分也!君所治,夷滅者幾何人矣!今人言君皆有狀,天子重致君獄,欲令君自為計,何多以對為?」湯乃為書謝曰:「湯無尺寸之功,起刀筆吏,陛下幸致位三公,無以塞責。然謀陷湯者,三長史也。」遂自殺。

湯死,家產直不過五百金,皆所得奉賜,無它贏。昆弟諸子欲厚葬湯,湯母曰:「湯為天子大臣,被惡言而死,何厚葬為!」載以牛車,有棺而無槨。上聞之,曰:「非此母不生此子。」乃盡按誅三長史。丞相青翟自殺。出田信。上惜湯,復稍進其子安世。

安世字子孺,少以父任為郎。用善書給事尚書,精力於職,休沐未嘗出。上行幸河東,嘗亡書三篋,詔問莫能知,唯安世識之,具作其事。後購求得書,以相校無所遺失。上奇其材,擢為尚書令,遷光祿大夫。

昭帝即位,大將軍霍光秉政,以安世篤行,光親重之。會左將軍上官桀父子及御史大夫桑弘羊皆與燕王、蓋主謀反誅,光以朝無舊臣,白用安世為右將軍光祿勳,以自副焉。久之,天子下詔曰:「右將軍光祿勳安世輔政宿衛,肅敬不怠,十有三年,咸以康寧。夫親親任賢,唐虞之道也,其封安世為富平侯。」

明年,昭帝崩,未葬,大將軍光白太后,徙安世為車騎將軍,與共徵立昌邑王。王行淫亂,光復與安世謀廢王,尊立宣帝。帝初即位,褒賞大臣,詔曰:「夫褒有德,賞有功,古今之通義也。車騎將軍光祿勳富平侯安世,宿衛忠正,宣德明恩,勤勞國家,守職秉義,以安宗廟,其益封萬六百戶,功次大將軍光。」安世子千秋、延壽、彭祖,皆中郎將侍中。

大將軍光薨後數月,御史大夫魏相上封事曰:「聖王褒有德以懷萬方,顯有功以勸百寮,是以朝廷尊榮,天下鄉風。國家承祖宗之業,制諸侯之重,新失大將軍,宜宣章盛德以示天下,顯明功臣以填藩國。毋空大位,以塞爭權,所以安社稷絕未萌也。車騎將軍安世事孝武皇帝三十餘年,忠信謹厚,勤勞政事,夙夜不怠,與大將軍定策,天下受其福,國家重臣也,宜尊其位,以為大將軍,毋令領光祿勳事,使專精神,憂念天下,思惟得失。安世子延壽重厚,可以為光祿勳,領宿衛臣。」上亦欲用之。安世聞指,懼不敢當,請間求見,免冠頓首曰:「老臣耳妄聞,言之為先事,不言情不達,誠自量不足以居大位,繼大將軍後。唯天子財哀,以全老臣之命。」上笑曰:「君言泰謙。君而不可,尚誰可者!」安世深辭弗能得。後數日,竟拜為大司馬車騎將軍,領尚書事。數月,罷車騎將軍屯兵,更為衛將軍,兩宮衛尉,城門、北軍兵屬焉。

時霍光子禹為右將軍,上亦以禹為大司馬,罷其右將軍屯兵,以虛尊加之,而實奪其眾。後歲餘,禹謀反,夷宗族,安世素小心畏忌,已內憂矣。其女孫敬為霍氏外屬婦,當相坐,安世瘦懼,形於顏色。上怪而憐之,以問左右,乃赦敬,以慰其意。安世浸恐。職典樞機,以謹慎周密自著,外內無間。每定大政,已決,輒移病出,聞有詔令,乃驚,使吏之丞相府問焉。自朝廷大臣莫知其與議也。

嘗有所薦,其人來謝,安世大恨,以為舉賢達能,豈有私謝邪?絕勿復為通。有郎功高不調,自言,安世應曰:「君之功高,明主所知。人臣執事,何長短而自言乎!」絕不許。已而郎果遷。莫府長史遷,辭去之官,安世問以過失。長史曰:「將軍為明主股肱,而士無所進,論者以為譏。」安世曰:「明主在上,賢不肖較然,臣下自修而已,何知士而薦之?」其欲匿名跡遠權勢如此。

為光祿勳,郎有醉小便殿上,主事白行法,安世曰:「何以知其不反水漿邪?如何以小過成罪!」郎淫官婢,婢兄自言,安世曰:「奴以恚怒,誣汙衣冠。」自署適奴。其隱人過失,皆此類也。

安世自見父子尊顯,懷不自安,為子延壽求出補吏,上以為北地太守。歲餘,上閔安世年老,復徵延壽為左曹太僕。

初,安世兄賀幸於衛太子,太子敗,賓客皆誅,安世為賀上書,得下蠶室。後為掖庭令,而宣帝以皇曾孫收養掖庭。賀內傷太子無辜,而曾孫孤幼,所以視養拊循,恩甚密焉。及曾孫壯大,賀教書,令受詩,為取許妃,以家財聘之。曾孫數有徵怪,語在宣紀。賀聞知,為安世道之,稱其材美。安世輒絕止,以為少主在上,不宜稱述曾孫。及宣帝即位,而賀已死。上謂安世曰:「掖廷令平生稱我,將軍止之,是也。」上追思賀恩,欲封其冢為恩德侯,置守冢二百家。賀有一子蚤死,無子,子安世小男彭祖。彭祖又小與上同席研書,指欲封之,先賜爵關內侯。故安世深辭賀封,又求損守冢戶數,稍減至三十戶。上曰:「吾自為掖廷令,非為將軍也。」安世乃止,不敢復言。遂下詔曰:「其為故掖廷令張賀置守冢三十家。」上自處置其里,居冢西鬥雞翁舍南,上少時所嘗游處也。明年,復下詔曰:「朕微眇時,故掖廷令張賀輔道朕躬,修文學經術,恩惠卓異,厥功茂焉。《詩》云:『無言不讎,無德不報。』其封賀弟子侍中關內侯彭祖為陽都侯,賜賀諡曰陽都哀侯。」時賀有孤孫霸,年七歲,拜為散騎中郎將,賜爵關內侯,食邑三百戶。安世以父子封侯,在位大盛,乃辭祿。詔都內別臧張氏無名錢以百萬數。

安世尊為公侯,食邑萬戶,然身衣弋綈,夫人自紡績,家童七百人,皆有手技作事,內治產業,累積纖微,是以能殖其貨,富於大將軍光。天子甚尊憚大將軍,然內親安世,心密於光焉。

元康四年春,安世病,上疏歸侯,乞骸骨。天子報曰:「將軍年老被病,朕甚閔之。雖不能視事,折衝萬里,君先帝大臣,明於治亂,朕所不及,得數問焉,何感而上書歸衛將軍富平侯印?薄朕忘故,非所望也!願將軍強餐食,近醫藥,專精神,以輔天年。」安世復強起視事,至秋薨。天子贈印綬,送以輕車介士,諡曰敬侯。賜塋杜東,將作穿復土,起冢祠堂。子延壽嗣。

延壽已歷位九卿,既嗣侯,國在陳留,別邑在魏郡,租入歲千餘萬。延壽自以身無功德,何以能久堪先人大國,數上書讓減戶邑,又因弟陽都侯彭祖口陳至誠。天子以為有讓,乃徙封平原,并一國,戶口如故,而租稅減半。薨,諡曰愛侯。子勃嗣,為散騎諫大夫。

元帝初即位,詔列侯舉茂材,勃舉太官獻丞陳湯。湯有罪,勃坐削戶二百,會薨,故賜諡曰繆侯。後湯立功西域,世以勃為知人。子臨嗣。

臨亦謙儉,每登閣殿,常歎曰:「桑、霍為我戒,豈不厚哉!」且死,分施宗族故舊,薄葬不起墳。臨尚敬武公主。薨,子放嗣。

鴻嘉中,上欲遵武帝故事,與近臣游宴,放以公主子開敏得幸。放取皇后弟平恩侯許嘉女,上為放供張,賜甲第,充以乘輿服飾,號為天子取婦,皇后嫁女。大官私官並供具第,兩宮使者冠蓋不絕,賞賜以千萬數。放為侍中中郎將,監平樂屯兵,置莫府,儀比將軍。與上臥起,寵愛殊絕,常從為微行出游,北至甘泉,南至長楊、五莋,鬥雞走馬長安中,積數年。

是時上諸舅皆害其寵,白太后。太后以上春秋富,動作不節,甚以過放。時數有災異,議者歸咎放等。於是丞相宣、御史大夫方進奏:「放驕蹇縱恣,奢淫不制。前侍御史修等四人奉使至放家逐名捕賊,時放見在,奴從者閉門設兵弩射吏,距使者不肯內。知男子李游君欲獻女,使樂府音監景武強求不得,使奴康等之其家,賊傷三人。又以縣官事怨樂府游徼莽,而使大奴駿等四十餘人群黨盛兵弩,白晝入樂府攻射官寺,縛束長吏子弟,斫破器物,宮中皆奔走伏匿。莽自髡鉗,衣赭衣,及守令史調等皆徒跣叩頭謝放,放乃止。奴從者支屬並乘權勢為暴虐,至求吏妻不得,殺其夫,或恚一人,妄殺其親屬,輒亡入放弟,不得,幸得勿治。放行輕薄,連犯大惡,有感動陰陽之咎,為臣不忠首,罪名雖顯,前蒙恩。驕逸悖理,與背畔無異,臣子之惡,莫大於是,不宜宿衛在位。臣請免放歸國,以銷眾邪之萌,厭海內之心。」

上不得已,左遷放為北地都尉。數月,復徵入侍中。太后以放為言,出放為天水屬國都尉。永始、元延間,比年日蝕,故久不還放,璽書勞問不絕。居歲餘,徵放歸第視母公主疾。數月,主有瘳,出放為河東都尉。上雖愛放,然上迫太后,下用大臣,故常涕泣而遣之。後復徵放為侍中光祿大夫,秩中二千石。歲餘,丞相方進復奏放,上不得已,免放,賜錢五百萬,遣就國。數月,成帝崩,放思慕哭泣而死。

初,安世長子千秋與霍光子禹俱為中郎將,將兵隨度遼將軍范明友擊烏桓。還,謁大將軍光,問千秋戰鬥方略,山川形勢,千秋口對兵事,畫地成圖,無所忘失。光復問禹,禹不能記,曰:「皆有文書。」光由是賢千秋,以禹為不材,歎曰:「霍氏世衰,張氏興矣!」及禹誅滅,而安世子孫相繼,自宣、元以來為侍中、中常侍、諸曹散騎、列校尉者凡十餘人。功臣之世,唯有金氏、張氏,親近寵貴,比於外戚。

放子純嗣侯,恭儉自修,明習漢家制度故事,有敬侯遺風。王莽時不失爵,建武中歷位至大司空,更封富平之別鄉為武始侯。

張湯本居杜陵,安世武、昭、宣世輒隨陵,凡三徙,復還杜陵。

贊曰:馮商稱張湯之先與留侯同祖,而司馬遷不言,故闕焉。漢興以來,侯者百數,保國持寵,未有若富平者也。湯雖酷烈,及身蒙咎,其推賢揚善,固宜有後。安世履道,滿而不溢。賀之陰德,亦有助云。


\end{pinyinscope}