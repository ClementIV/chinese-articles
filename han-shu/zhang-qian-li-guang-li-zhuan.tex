\article{張騫李廣利傳}

\begin{pinyinscope}
張騫,漢中人也,建元中為郎。時匈奴降者言匈奴破月氏王,以其頭為飲器,月氏遁而怨匈奴,無與共擊之。漢方欲事滅胡,聞此言,欲通使,道必更匈奴中,乃募能使者。騫以郎應募,使月氏,與堂邑氏奴甘父俱出隴西。徑匈奴,匈奴得之,傳詣單于。單于曰:「月氏在吾北,漢何以得往使?吾欲使越,漢肯聽我乎?」留騫十餘歲,予妻,有子,然騫持漢節不失。

居匈奴西,騫因與其屬亡鄉月氏,西走數十日至大宛。大宛聞漢之饒財,欲通不得,見騫,喜,問欲何之。騫曰:「為漢使月氏而為匈奴所閉道,今亡,唯王使人道送我。誠得至,反漢,漢之賂遺王財物不可勝言。」大宛以為然,遣騫,為發譯道,抵康居。康居傳致大月氏。大月氏王已為胡所殺,立其夫人為王。既臣大夏而君之,地肥饒,少寇,志安樂,又自以遠遠漢,殊無報胡之心。騫從月氏至大夏,竟不能得月氏要領。

留歲餘,還,並南山,欲從羌中歸,復為匈奴所得。留歲餘,單于死,國內亂,騫與胡妻及堂邑父俱亡歸漢。拜騫太中大夫,堂邑父為奉使君。

騫為人彊力,寬大信人,蠻夷愛之。堂邑父胡人,善射,窮急射禽獸給食。初,騫行時百餘人,去十三歲,唯二人得還。

騫身所至者,大宛、大月氏、大夏、康居,而傳聞其旁大國五六,具為天子言其地形,所有語皆在西域傳。

騫曰:「臣在大夏時,見邛竹杖、蜀布,問安得此,大夏國人曰:『吾賈人往市之身毒國。身毒國在大夏東南可數千里。其俗土著,與大夏同,同卑溼暑熱。其民乘象以戰。其國臨大水焉。』以騫度之,大夏去漢萬二千里,居西南。今身毒又居大夏東南數千里,有蜀物,此其去蜀不遠矣。今使大夏,從羌中,險,羌人惡之;少北,則為匈奴所得;從蜀,宜徑,又無寇。」天子既聞大宛及大夏、安息之屬皆大國,多奇物,土著,頗與中國同俗,而兵弱,貴漢財物;其北則大月氏、康居之屬,兵彊,可以賂遺設利朝也。誠得而以義屬之,則廣地萬里,重九譯,致殊俗,威德遍於四海。天子欣欣以騫言為然。乃令因蜀犍為發間使,四道並出:出駹,出莋,出徙、邛,出僰,皆各行一二千里。其北方閉氐、莋,南方閉巂、昆明。昆明之屬無君長,善寇盜,輒殺略漢使,終莫得通。然聞其西可千餘里,有乘象國,名滇越,而蜀賈間出物者或至焉,於是漢以求大夏道始通滇國。初,漢欲通西南夷,費多,罷之。及騫言可以通大夏,乃復事西南夷。

騫以校尉從大將軍擊匈奴,知水草處,軍得以不乏,乃封騫為博望侯。是歲元朔六年也。後二年,騫為衛尉,與李廣俱出右北平擊匈奴。匈奴圍李將軍,軍失亡多,而騫後期當斬,贖為庶人。是歲驃騎將軍破匈奴西邊,殺數萬人,至祁連山。其秋,渾邪王率眾降漢,而金城、河西西並南山至鹽澤,空無匈奴。匈奴時有候者到,而希矣。後二年,漢擊走單于於幕北。

天子數問騫大夏之屬。騫既失侯,因曰:「臣居匈奴中,聞烏孫王號昆莫。昆莫父難兜靡本與大月氏俱在祁連、焞煌間,小國也。大月氏攻殺難兜靡,奪其地,人民亡走匈奴。子昆莫新生,傅父布就翎侯抱亡置草中,為求食,還,見狼乳之,又烏銜肉翔其旁,以為神,遂持歸匈奴,單于愛養之。及壯,以其父民眾與昆莫,使將兵,數有功。時,月氏已為匈奴所破,西擊塞王。塞王南走遠徙,月氏居其地。昆莫既健,自請單于報父怨,遂西攻破大月氏。大月氏復西走,徙大夏地。昆莫略其眾,因留居,兵稍彊,會單于死,不肯復朝事匈奴。匈奴遣兵擊之,不勝,益以為神而遠之。今單于新困於漢,而昆莫地空。蠻夷戀故地,又貪漢物,誠以此時厚賂烏孫,招以東居故地,漢遣公主為夫人,結昆弟,其勢宜聽,則是斷匈奴右臂也。既連烏孫,自其西大夏之屬皆可招來而為外臣。」天子以為然,拜騫為中郎將,將三百人,馬各二匹,牛羊以萬數,齎金幣帛直數千鉅萬,多持節副使,道可便遣之旁國。騫既至烏孫,致賜諭指,未能得其決。語在西域傳。騫即分遣副使使大宛、康居、月氏、大夏。烏孫發譯道送騫,與烏孫使數十人,馬數十匹,報謝,因令窺漢,知其廣大。

騫還,拜為大行。歲餘,騫卒。後歲餘,其所遣副使通大夏之屬者皆頗與其人俱來,於是西北國始通於漢矣。然騫鑿空,諸後使往者皆稱博望侯,以為質於外國,外國由是信之。其後,烏孫竟與漢結婚。

初,天子發書易,曰「神馬當從西北來」。得烏孫馬好,名曰「天馬」。及得宛汗血馬,益壯,更名烏孫馬曰「西極馬」,宛馬曰「天馬」云。而漢始築令居以西,初置酒泉郡,以通西北國。因益發使抵安息、奄蔡、犛靬、條支、身毒國。而天子好宛馬,使者相望於道,一輩大者數百,少者百餘人,所齎操,大放博望侯時。其後益習而衰少焉。漢率一歲中使者多者十餘,少者五六輩,遠者八九歲,近者數歲而反。

是時,漢既滅越,蜀所通西南夷皆震,請吏。置牂柯、越巂、益州、沈黎、文山郡,欲地接以前通大夏。乃遣使歲十餘輩,出此初郡,皆復閉昆明,為所殺,奪幣物。於是漢發兵擊昆明,斬首數萬。後復遣使,竟不得通。語在西南夷傳。

自騫開外國道以尊貴,其吏士爭上書言外國奇怪利害,求使。天子為其絕遠,非人所樂,聽其言,予節,募吏民無問所從來,為具備人眾遣之,以廣其道。來還不能無侵盜幣物,及使失指,天子為其習之,輒覆按致重罪,以激怒令贖,復求使。使端無窮,而輕犯法。其吏卒亦輒復盛推外國所有,言大者予節,言小者為副,故妄言無行之徒皆爭相效。其使皆私縣官齎物,欲賤市以私其利。外國亦厭漢使人人有言輕重,度漢兵遠,不能至,而禁其食物,以苦漢使。漢使乏絕,責怨,至相攻擊。樓蘭、姑師小國,當空道,攻劫漢使王恢等尤甚。而匈奴奇兵又時時遮擊之。使者爭言外國利害,皆有城邑,兵弱易擊。於是天子遣從票侯破奴將屬國騎及郡兵數萬以擊胡,胡皆去。明年,擊破姑師,虜樓蘭王。酒泉列亭鄣至玉門矣。

而大宛諸國發使隨漢使來,觀漢廣大,以大鳥卵及犛靬眩人獻於漢,天子大說。而漢使窮河源,其山多玉石,采來,天子案古圖書,名河所出山曰昆侖云。

是時,上方數巡狩海上,乃悉從外國客,大都多人則過之,散財帛賞賜,厚具饒給之,以覽視漢富厚焉。大角氐,出奇戲諸怪物,多聚觀者,行賞賜,酒池肉林,令外國客遍觀各倉庫府臧之積,欲以見漢廣大,傾駭之。及加其眩者之工,而角氐奇戲歲增變,其益興,自此始。而外國使更來更去。大宛以西皆自恃遠,尚驕恣,未可詘以禮羈縻而使也。

漢使往既多,其少從率進孰於天子,言大宛有善馬在貳師城,匿不肯示漢使。天子既好宛馬,聞之甘心,使壯士車令等持千金及金馬以請宛王貳師城善馬。宛國饒漢物,相與謀曰:「漢去我遠,而鹽水中數有敗,出其北有胡寇,出其南乏水草,又且往往而絕邑,乏食者多。漢使數百人為輩來,常乏食,死者過半,是安能致大軍乎?且貳師馬,宛寶馬也。」遂不肯予漢使。漢使怒,妄言,椎金馬而去。宛中貴人怒曰:「漢使至輕我!」遣漢使去,令其東邊郁成王遮攻,殺漢使,取其財物。天子大怒。諸嘗使宛姚定漢等言:「宛兵弱,誠以漢兵不過三千人,強弩射之,即破宛矣。」天子以嘗使浞野侯攻樓蘭,以七百騎先至,虜其王,以定漢等言為然,而欲侯寵姬李氏,乃以李廣利為將軍,伐宛。

騫孫猛,字子游,有俊才,元帝時為光祿大夫,使匈奴,給事中,為石顯所譖,自殺。

李廣利,女弟李夫人有寵於上,產昌邑哀王。太初元年,以廣利為貳師將軍,發屬國六千騎及郡國惡少年數萬人以往,期至貳師城取善馬,故號「貳師將軍」。故浩侯王恢使道軍。既西過鹽水,當道小國各堅城守,不肯給食,攻之不能下。下者得食,不下者數日則去。比至郁成,士財有數千,皆飢罷攻郁成城,郁成距之,所殺傷甚眾。貳師將軍與左右計:「至郁成尚不能舉,況至其王都乎?」引而還。往來二歲,至敦煌,士不過什一二。使使上書言:「道遠,多乏食,且士卒不患戰而患飢。人少,不足以拔宛。願且罷兵,益發而復往。」天子聞之,大怒,使使遮玉門關,曰:「軍有敢入,斬之。」貳師恐,因留屯敦煌。

其夏,漢亡浞野之兵二萬餘於匈奴,公卿議者皆願罷宛軍,專力攻胡。天子業出兵誅宛,宛小國而不能下,則大夏之屬漸輕漢,而宛善馬絕不來,烏孫、輪臺易苦漢使,為外國笑。乃案言伐宛尤不便者鄧光等。赦囚徒扞寇盜,發惡少年及邊騎,歲餘而出敦煌六萬人,負私從者不與。牛十萬,馬三萬匹,驢橐駝以萬數齎糧,兵弩甚設。天下騷動,轉相奉伐宛,五十餘校尉。宛城中無井,汲城外流水,於是遣水工徙其城下水空以穴其城。益發戍甲卒十八萬酒泉、張掖北,置居延、休屠以衛酒泉。而發天下七科適,及載糒給貳師,轉車人徒相連屬至敦煌。而拜習馬者二人為執驅馬校尉,備破宛擇取其善馬云。

於是貳師後復行,兵多,所至小國莫不迎,出食給軍。至輪臺,輪臺不下,攻數日,屠之。自此而西,平行至宛城,兵到者三萬。宛兵迎擊漢兵,漢兵射敗之,宛兵走入保其城。貳師欲攻郁成城,恐留行而令宛益生詐,乃先至宛,決其水原,移之,則宛固已憂困。圍其城,攻之四十餘日。宛貴人謀曰:「王毋寡匿善馬,殺漢使。今殺王而出善馬,漢兵宜解;即不,乃力戰而死,未晚也。」宛貴人皆以為然,共殺王。其外城壞,虜宛貴人勇將煎靡。宛大怨,走入中城,相與謀曰:「漢所為攻宛,以王毋寡。」持其頭,遣人使貳師,約曰:「漢無攻我,我盡出善馬,恣所取,而給漢軍食。即不聽我,我盡殺善馬,康居之救又且至。至,我居內,康居居外,與漢軍戰。孰計之,何從?」是時,康居候視漢兵尚盛,不敢進。貳師聞宛城中新得漢人知穿井,而其內食尚多。計以為來誅首惡者毋寡,毋寡頭已至,如此不許,則堅守,而康居候漢兵罷來救宛,破漢軍必矣。軍吏皆以為然,許宛之約。宛乃出其馬,令漢自擇之,而多出食食漢軍。漢軍取其善馬數十匹,中馬以下牝牡三千餘匹,而立宛貴人之故時遇漢善者名昧蔡為宛王,與盟而罷兵。終不得入中城,罷而引歸。

初,貳師起敦煌西,為人多,道上國不能食,分為數軍,從南北道。校尉王申生、故鴻臚壺充國等千餘人別至郁成,城守不肯給食。申生去大軍二百里,負而輕之,攻郁成急。郁成窺知申生軍少,晨用三千人攻殺申生等,數人脫亡,走貳師。貳師令搜粟都尉上官桀往攻破郁成,郁成降。其王亡走康居,桀追至康居。康居聞漢已破宛,出郁成王與桀。桀令四騎士縛守詣大將軍。四人相謂:「郁成,漢所毒,今生將,卒失大事。」欲殺,莫適先擊。上邽騎士趙弟拔劍擊斬郁成王。桀等遂追及大將軍。

初,貳師後行,天子使使告烏孫大發兵擊宛。烏孫發二千騎往,持兩端,不肯前。貳師將軍之東,諸所過小國聞宛破,皆使其子弟從入貢獻,見天子,因為質焉。軍還,入玉門者萬餘人,馬千餘匹。後行,非乏食,戰死不甚多,而將吏貪,不愛卒,侵牟之,以此物故者眾。天子為萬里而伐,不錄其過,乃下詔曰:「匈奴為害久矣,今雖徙幕北,與帝國謀共要絕大月氏使,遮殺中郎將江、故雁門守攘。危須以西及大宛皆合約殺期門車令、中郎將朝及身毒國使,隔東西道。貳師將軍廣利征討厥罪,伐勝大宛。賴天之靈,從泝河山,涉流沙,通西海,山雪不積,士大夫徑度,獲王首虜,珍怪之物畢陳於闕。其封廣利為海西侯,食邑八千戶。」又封斬郁成王者趙弟為新畤侯;軍正趙始成功最多,為光祿大夫;上官桀敢深入,為少府;李哆有計謀,為上黨太守。軍官吏為九卿者三人,諸侯相、郡守、二千石百餘人,千石以下千餘人。奮行者官過其望,以適過行者皆黜其勞。士卒賜直四萬錢。伐宛再反,凡四歲而得罷焉。

後十一歲,征和三年,貳師復將七萬騎出五原,擊匈奴,度郅居水。兵敗,降匈奴,為單于所殺。語在匈奴傳。

贊曰:「禹本紀言河出昆侖,昆侖高二千五百里餘,日月所相避隱為光明也。自張騫使大夏之後,窮河原,惡睹所謂昆侖者乎?故言九州山川,尚書近之矣。至禹本紀、山經所有,放哉!


\end{pinyinscope}