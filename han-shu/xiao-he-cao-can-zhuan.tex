\article{蕭何曹參傳}

\begin{pinyinscope}
蕭何,沛人也。以文毋害為沛主吏掾。高祖為布衣時,數以吏事護高祖。高祖為亭長,常佑之。高祖以吏繇咸陽,吏皆送奉錢三,何獨以五。秦御史監郡者,與從事辨之。何乃給泗水卒史事,第一。秦御史欲入言徵何,何固請,得毋行。

及高祖起為沛公,何嘗為丞督事。沛公至咸陽,諸將皆爭走金帛財物之府分之,何獨先入收秦丞相御史律令圖書臧之。沛公具知天下阨塞,戶口多少,彊弱處,民所疾苦者,以何得秦圖書也。

初,諸侯相與約,先入關破秦者王其地。沛公既先定秦,項羽後至,欲攻沛公,沛公謝之得解。羽遂屠燒咸陽,與范增謀曰:「巴蜀道險,秦之遷民皆居蜀。」乃曰:「蜀漢亦關中地也。」故立沛公為漢王,而三分關中地,王秦降將以距漢王。漢王怒,欲謀攻項羽。周勃、灌嬰、樊噲皆勸之,何諫之曰:「雖王漢中之惡,不猶愈於死乎?」漢王曰:「何為乃死也?」何曰:「今眾弗如,百戰百敗,不死何為?周書曰『天予不取,反受其咎』。語曰『天漢』,其稱甚美。夫能詘於一人之下,而信於萬乘之上者,湯武是也。臣願大王王漢中,養其民以致賢人,收用巴蜀,還定三秦,天下可圖也。」漢王曰:「善。」乃遂就國,以何為丞相。何進韓信,漢王以為大將軍,說漢王令引兵東定三秦。語在信傳。

何以丞相留收巴蜀,填撫諭告,使給軍食。漢二年,漢王與諸侯擊楚,何守關中,侍太子,治櫟陽。為令約束,立宗廟、社稷、宮室、縣邑,輒奏,上可許以從事;即不及奏,輒以便宜施行,上來以聞。計戶轉漕給軍,漢王數失軍遯去,何常興關中卒,輒補缺。上以此剸屬任何關中事。

漢三年,與項羽相距京、索間,上數使使勞苦丞相。鮑生謂何曰:「今王暴衣露蓋,數勞苦君者,有疑君心。為君計,莫若遣君子孫昆弟能勝兵者悉詣軍所,上益信君。」於是何從其計,漢王大說。

漢五年,已殺項羽,即皇帝位,論功行封,群臣爭功,歲餘不決。上以何功最盛,先封為酇侯,食邑八千戶。功臣皆曰:「臣等身被堅執兵,多者百餘戰,少者數十合,攻城略地,大小各有差。今蕭何未有汗馬之勞,徒持文墨議論,不戰,顧居臣等上,何也?」上曰:「諸君知獵乎?」曰:「知之。」「知獵狗乎?」曰:「知之。」上曰:「夫獵,追殺獸者狗也,而發縱指示獸處者人也。今諸君徒能走得獸耳,功狗也;至如蕭何,發縱指示,功人也。且諸君獨以身從我,多者三兩人;蕭何舉宗數十人皆隨我,功不可忘也!」群臣後皆莫敢言。

列侯畢已受封,奏位次,皆曰:「平陽侯曹參身被七十創,攻城略地,功最多,宜第一。」上已橈功臣多封何,至位次未有以復難之,然心欲何第一。關內侯鄂千秋時為謁者,進曰:「群臣議皆誤。夫曹參雖有野戰略地之功,此特一時之事。夫上與楚相距五歲,失軍亡眾,跳身遯者數矣,然蕭何常從關中遣軍補其處。非上所詔令召,而數萬眾會上乏絕者數矣。夫漢與楚相守滎陽數年,軍無見糧,蕭何轉漕關中,給食不乏。陛下雖數亡山東,蕭何常全關中待陛下,此萬世功也。今雖無曹參等百數,何缺於漢?漢得之不必待以全。柰何欲以一旦之功而加萬世之功哉!蕭何當第一,曹參次之。」上曰:「善。」於是乃令何第一,賜帶劍履上殿,入朝不趨。上曰:「吾聞進賢受上賞,蕭何功雖高,待鄂君乃得明。」於是因鄂千秋故所食關內侯邑二千戶,封為安平侯。是日,悉封何父母兄弟十餘人,皆食邑。乃益封何二千戶,「以嘗繇咸陽時何送我獨贏錢二也」。

陳豨反,上自將,至邯鄲。而韓信謀反關中。呂后用何計誅信。語在信傳。上已聞誅信,使使拜丞相為相國,益封五千戶,令卒五百人一都尉為相國衛。諸君皆賀,召平獨弔。召平者,故秦東陵侯。秦破,為布衣,貧,種瓜長安城東,瓜美,故世謂「東陵瓜」,從召平始也。平謂何曰:「禍自此始矣。上暴露於外,而君守於內,非被矢石之難,而益君封置衛者,以今者淮陰新反於中,有疑君心。夫置衛衛君,非以寵君也。願君讓封勿受,悉以家私財佐軍。」何從其計,上說。

其秋,黥布反,上自將擊之,數使使問相國何為。曰:「

為上在軍,拊循勉百姓,悉所有佐軍,如陳豨時。」客又說何曰:「君滅族不久矣。夫君位為相國,功第一,不可復加。然君初入關,本得百姓心,十餘年矣。皆附君,尚復孳孳得民和。上所謂數問君,畏君傾動關中。今君胡不多買田地,賤貰貣以自汙?上心必安。」於是何從其計,上乃大說。

上罷布軍歸,民道遮行,上書言相國彊賤買民田宅數千人。上至,何謁。上笑曰:「今相國乃利民!」民所上書皆以與何,曰:「君自謝民。」後何為民請曰:「長安地骥,上林中多空地,棄,願令民得入田,毋收稿為獸食。」上大怒曰:「相國多受賈人財物,為請吾苑!」乃下何廷尉,械繫之。數日,王衛尉侍,前問曰:「相國胡大罪,陛下繫之暴也?」上曰:「吾聞李斯相秦皇帝,有善歸主,有惡自予。今相國多受賈豎金,為請吾苑,以自媚於民。故繫治之。」王衛尉曰:「夫職事苟有便於民而請之,真宰相事也。陛下柰何乃疑相國受賈人錢乎!且陛下距楚數歲,陳豨、黥布反時,陛下自將往,當是時相國守關中,關中搖足則關西非陛下有也。相國不以此時為利,乃利賈人之金乎?且秦以不聞其過亡天下,夫李斯之分過,又何足法哉!陛下何疑宰相之淺也!」上不懌。是日,使使持節赦出何。何年老,素恭謹,徒跣入謝。上曰:「相國休矣!相國為民請吾苑不許,我不過為桀紂主,而相國為賢相。吾故繫相國,欲令百姓聞吾過。」

高祖崩,何事惠帝。何病,上親自臨視何疾,因問曰:「君即百歲後,誰可代君?」對曰:「知臣莫如主。」帝曰:「曹參何如?」何頓首曰:「帝得之矣。何死不恨矣!」

何買田宅必居窮辟處,為家不治垣屋。曰:「令後世賢,師吾儉;不賢,毋為勢家所奪。」

孝惠二年,何薨,諡曰文終侯。子祿嗣,薨,無子。高后乃封何夫人同為酇侯,小子延為筑陽侯。孝文元年,罷同,更封延為酇侯。薨,子遺嗣。薨,無子。文帝復以遺弟則嗣,有罪免。景帝二年,制詔御史:「故相國蕭何,高皇帝大功臣,所與為天下也。今其祀絕,朕甚憐之。其以武陽縣戶二千封何孫嘉為列侯。」嘉,則弟也。薨,子勝嗣,後有罪免。武帝元狩中,復下詔御史:「以酇戶二千四百封何曾孫慶為酇侯,布告天下,令明知朕報蕭相國德也。」慶,則子也。薨,子壽成嗣,坐為太常儀牲瘦免。宣帝時,詔丞相御史求問蕭相國後在者,得玄孫建世等十二人,復下詔以酇戶二千封建世為酇侯。傳子至孫獲,坐使奴殺人減死論。成帝時,復封何玄孫之子南讀長喜為酇侯。傳子至曾孫,王莽敗乃絕。

曹參,沛人也。秦時為獄掾,而蕭何為主吏,居縣為豪吏矣。高祖為沛公也,參以中涓從。擊胡陵、方與,攻秦監公軍,大破之。東下薛,擊泗水守軍薛郭西。復攻胡陵,取之。徙守方與。方與反為魏,擊之。豐反為魏,攻之。賜爵七大夫。北擊司馬欣軍碭東,取狐父、祁善置。又攻下邑以西,至虞,擊秦將章邯車騎。攻轅戚及亢父,先登。遷為五大夫。北救東阿,擊章邯軍,陷陳,追至濮陽。攻定陶,取臨濟。南救雍丘,擊李由軍,破之,殺李由,虜秦候一人。章邯破殺項梁也,沛公與項羽引兵而東。楚懷王以沛公為碭郡長,將碭郡兵。於是乃封參執帛,號曰建成君。遷為戚公,屬碭郡。

其後從攻東郡尉軍,破之成武南。擊王離軍成陽南,又攻杠里,大破之。追北,西至開封,擊趙賁軍,破之,圍趙賁開封城中。西擊秦將楊熊軍於曲遇,破之,虜秦司馬及御史各一人。遷為執珪。從西攻陽武,下轘轅、緱氏,絕河津。擊趙賁軍尸北,破之。從南攻犨,與南陽守齮戰陽城郭東,陷陳,取宛,虜齮,定南陽郡。從西攻武關、嶢關,取之。前攻秦軍藍田南,又夜擊其北軍,大破之,遂至咸陽,破秦。

項羽至,以沛公為漢王。漢王封參為建成侯。從至漢中,遷為將軍。從還定三秦,攻下辨、故道、雍、斄。擊章平軍於好畤南,破之,圍好畤,取壤鄉。擊三秦軍壤東及高櫟,破之。復圍章平,平出好畤走。因擊趙賁、內史保軍,破之。東取咸陽,更名曰新城。參將兵守景陵二十三日,三秦使章平等攻參,參出擊,大破之。賜食邑於寧秦。以將軍引兵圍章邯廢丘;以中尉從漢王出臨晉關。至河內,下脩武,度圍津,東擊龍且、項佗定陶,破之。東取碭、蕭、彭城。擊項籍軍,漢軍大敗走。參以中尉圍取雍丘。王武反於外黃,程處反於燕,往擊,盡破之。柱天侯反於衍氏,進破取衍氏。擊羽嬰於昆陽,追至葉。還攻武彊,因至滎陽。參自漢中為將軍中尉,從擊諸侯,及項王敗,還至滎陽。

漢二年,拜為假左丞相,入屯兵關中。月餘,魏王豹反,以假丞相別與韓信東攻魏將孫速東張,大破之。因攻安邑,得魏將王襄。擊魏王於曲陽,追至東垣,生獲魏王豹。取平陽,得豹母妻子,盡定魏地,凡五十二縣。賜食邑平陽。因從韓信擊趙相國夏說軍於鄔東,大破之,斬夏說。韓信與故常山王張耳引兵下井陘,擊成安君陳餘,而令參還圍趙別將戚公於鄔城中。戚公出走,追斬之。乃引兵詣漢王在所。韓信已破趙,為相國,東擊齊,參以左丞相屬焉。攻破齊歷下軍,遂取臨淄。還定濟北郡,收著、漯陰、平原、鬲、盧。已而從韓信擊龍且軍於上假密,大破之,斬龍且,虜亞將周蘭。定齊郡,凡得七十縣。得故齊王田廣相田光,其守相許章,及故將軍田既。韓信立為齊王,引兵東詣陳,與漢王共破項羽,而參留平齊未服者。

漢王即皇帝位,韓信徙為楚王。參歸相印焉。高祖以長子肥為齊王,而以參為相國。高祖六年,與諸侯剖符,賜參爵列侯,食邑平陽萬六百三十戶,世世勿絕。

參以齊相國擊陳豨將張春,破之。黥布反,參從悼惠王將車騎十二萬,與高祖會擊黥布軍,大破之。南至蘄,還定竹邑、相、蕭、留。

參功:凡下二國,縣百二十二;得王二人,相三人,將軍六人,大莫囂、郡守、司馬、候、御史各一人。

孝惠元年,除諸侯相國法,更以參為齊丞相。參之相齊,齊七十城。天下初定,悼惠王富於春秋,參盡召長老諸先生,問所以安集百姓。而齊故諸儒以百數,言人人殊,參未知所定。聞膠西有蓋公,善治黃老言,使人厚幣請之。既見蓋公,蓋公為言治道貴清靜而民自定,推此類具言之。參於是避正堂,舍蓋公焉。其治要用黃老術,故相齊九年,齊國安集,大稱賢相。

蕭何薨,參聞之,告舍人趣治行,「吾且入相。」居無何,使者果召參。參去,屬其後相曰:「以齊獄市為寄,慎勿擾也。」後相曰:「治無大於此者乎?」參曰:「不然。夫獄市者,所以并容也,今君擾之,姦人安所容乎?吾是以先之。」

始參微時,與蕭何善,及為宰相,有隙。至何且死,所推賢唯參。參代何為相國,舉事無所變更,壹遵何之約束。擇郡國吏長大,訥於文辭,謹厚長者,即召除為丞相史。吏言文刻深,欲務聲名,輒斥去之。日夜飲酒。卿大夫以下吏及賓客見參不事事,來者皆欲有言。至者,參輒飲以醇酒,度之欲有言,復飲酒,醉而後去,終莫得開說,以為常。

相舍後園近吏舍,吏舍日飲歌呼。從吏患之,無如何,乃請參遊後園。聞吏醉歌呼,從吏幸相國召按之。乃反取酒張坐飲,大歌呼與相和。

參見人之有細過,掩匿覆蓋之,府中無事。

參子窋為中大夫。惠帝怪相國不治事,以為「豈少朕與?」乃謂窋曰:「女歸,試私從容問乃父曰:『高帝新棄群臣,帝富於春秋,君為相國,日飲,無所請事,何以憂天下?』然無言吾告女也。」窋既洗沐歸,時間,自從其所諫參。參怒而笞之二百,曰:「趣入侍,天下事非乃所當言也。」至朝時,帝讓參曰:「與窋胡治乎?乃者我使諫君也。」參免冠謝曰:「陛下自察聖武孰與高皇帝?」上曰:「朕乃安敢望先帝!」參曰:「陛下觀參孰與蕭何賢?」上曰:「君似不及也。」參曰:「陛下言之是也。且高皇帝與蕭何定天下,法令既明具,陛下垂拱,參等守職,遵而勿失,不亦可乎?」惠帝曰:「善。君休矣!」

參為相國三年,薨,諡曰懿侯。百姓歌之曰:「蕭何為法,講若畫一;曹參代之,守而勿失。載其清靖,民以寧壹。」

窋嗣侯,高后時至御史大夫。傳國至曾孫襄,武帝時為將軍,擊匈奴,薨。子宗嗣,有罪,完為城旦。至哀帝時,乃封參玄孫之孫本始為平陽侯,二千戶,王莽時薨。子宏嗣,建武中先降河北,封平陽侯。至今八侯。

贊曰:蕭何、曹參皆起秦刀筆吏,當時錄錄未有奇節。漢興,依日月之末光,何以信謹守管籥,參與韓信俱征伐。天下既定,因民之疾秦法,順流與之更始,二人同心,遂安海內。淮陰、黥布等已滅,唯何、參擅功名,位冠群臣,聲施後世,為一代之宗臣,慶流曲裔,盛矣哉!


\end{pinyinscope}