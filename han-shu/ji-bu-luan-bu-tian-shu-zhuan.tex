\article{季布欒布田叔傳}

\begin{pinyinscope}
季布,楚人也,為任俠有名。項籍使將兵,數窘漢王。項籍滅,高祖購求布千金,敢有舍匿,罪三族。布匿濮陽周氏,周氏曰:「漢求將軍急,跡且至臣家,能聽臣,臣敢進計;即否,願先自剄。」布許之。乃髡鉗布,衣褐,置廣柳車中,并與其家僮數十人,之魯朱家所賣之。朱家心知其季布也,買置田舍。乃之雒陽見汝陰侯滕公,說曰:「季布何罪?臣各為其主用,職耳。項氏臣豈可盡誅邪?今上始得天下,而以私怨求一人,何示不廣也!且以季布之賢,漢求之急如此,此不北走胡,南走越耳。夫忌壯士以資敵國,此伍子胥所以鞭荊平之墓也。君何不從容為上言之?」滕公心知朱家大俠,意布匿其所,乃許諾。侍間,果言如朱家指。上乃赦布。當是時,諸公皆多布能摧剛為柔,朱家亦以此名聞當世。布召見,謝,拜郎中。

孝惠時,為中郎將。單于嘗為書嫚呂太后,太后怒,召諸將議之。上將軍樊噲曰:「臣願得十萬眾,橫行匈奴中。」諸將皆阿呂太后,以噲言為然。布曰:「樊噲可斬也。夫以高帝兵三十餘萬,困於平城,噲時亦在其中。今噲奈何以十萬眾橫行匈奴中,面謾!且秦以事胡,陳勝等起。今瘡痍未瘳,噲又面諛,欲搖動天下。」是時殿上皆恐,太后罷朝,遂不復議擊匈奴事。

布為河東守。孝文時,人有言其賢,召欲以為御史大夫。人又言其勇,使酒難近。至,留邸一月,見罷。布進曰:「臣待罪河東,陛下無故召臣,此人必有以臣欺陛下者。今臣至,無所受事,罷去,此人必有毀臣者。夫陛下以一人譽召臣,一人毀去臣,臣恐天下有識者聞之,有以窺陛下。」上默然,慚曰:「河東吾股肱郡,故特召君耳。」布之官。

辯士曹丘生數招權顧金錢,事貴人趙談等,與竇長君善。布聞,寄書諫長君曰:「吾聞曹丘生非長者,勿與通。」及曹丘生歸,欲得書請布。竇長君曰:「季將軍不說足下,足下無往。」固請書,遂行。使人先發書,布果大怒,待曹丘。曹丘至,則揖布曰:「楚人諺曰『得黃金百,不如得季布諾』,足下何以得此聲梁楚之間哉?且僕與足下俱楚人,使僕游揚足下名於天下,顧不美乎?何足下距僕之深也!」布乃大說。引入,留數月,為上客,厚送之。布名所以益聞者,曹丘揚之也。

布弟季心氣蓋關中,遇人恭謹,為任俠,方數千里,士爭為死。嘗殺人,亡吳,從爰絲匿,長事爰絲,弟畜灌夫、籍福之屬。嘗為中司馬,中尉郅都不敢加。少年多時時竊借其名以行。當是時,季心以勇,布以諾,聞關中。

布母弟丁公,為項羽將,逐窘高祖彭城西。短兵接,漢王急,顧謂丁公曰:「兩賢豈相厄哉!」丁公引兵而還。及項王滅,丁公謁見高祖,以丁公徇軍中,曰:「丁公為項王臣不忠,使項王失天下者也。」遂斬之,曰:「使後為人臣無傚丁公也!」

欒布,梁人也。彭越為家人時,嘗與布游,窮困,賣庸於齊,為酒家保。數歲別去,而布為人所略,賣為奴於燕。為其主家報仇,燕將臧荼舉以為都尉。荼為燕王,布為將。及荼反,漢擊燕,虜布。梁王彭越聞之,乃言上,請贖布為梁大夫。使於齊,未反,漢召彭越責以謀反,夷三族,梟首雒陽,下詔有收視者輒捕之。布還,奏事彭越頭下,祠而哭之。吏捕以聞。上召布罵曰:「若與彭越反邪?吾禁人勿收,若獨祠而哭之,與反明矣。趣亨之。」方提趨湯,顧曰:「願一言而死。」上曰:「何言?」布曰:「方上之困彭城,敗滎陽、成皋間,項王所以不能遂西,徒以彭王居梁地,與漢合從苦楚也。當是之時,彭王壹顧,與楚則漢破,與漢則楚破。且垓下之會,微彭王,項氏不亡。天下已定,彭王剖符受封,亦欲傳之萬世。今漢壹徵兵於梁,彭王病不行,而疑以為反。反形未見,以苛細誅之,臣恐功臣人人自危也。今彭王已死,臣生不如死,請就亨。」上乃釋布,拜為都尉。

孝文時,為燕相,至將軍。布稱曰:「窮困不能辱身,非人也;富貴不能快意,非賢也。」於是嘗有德,厚報之;有怨,必以法滅之。吳楚反時,以功封為鄃侯,復為燕相。燕齊之間皆為立社,號曰欒公社。

布薨,子賁嗣侯,孝武時坐為太常犧牲不如令,國除。

田叔,趙陘城人也。其先,齊田氏也。叔好劍,學黃老術於樂鉅公。為人廉直,喜任俠。游諸公,趙人舉之趙相趙午,言之趙王張敖,以為郎中。數歲,趙王賢之,未及遷。

會趙午、貫高等謀弒上,事發覺,漢下詔捕趙王及群臣反者。趙有敢隨王,罪三族。唯田叔、孟舒等十餘人赭衣自髡鉗,隨王至長安。趙王敖事白,得出,廢王為宣平侯,乃進言叔等十人。上召見,與語,漢廷臣無能出其右者。上說,盡拜為郡守、諸侯相。叔為漢中守十餘年。

孝文帝初立,召叔問曰:「公知天下長者乎?」對曰:「臣何足以知之!」上曰:「公長者,宜知之。」叔頓首曰:「故雲中守孟舒,長者也。」是時孟舒坐虜大入雲中免。上曰:「先帝置孟舒雲中十餘年矣,虜常一入,孟舒不能堅守,無故士卒戰死者數百人。長者固殺人乎?」叔叩頭曰:「夫貫高等謀反,天子下明詔,趙有敢隨張王者罪三族,然孟舒自髡鉗,隨張王,以身死之,豈自知為雲中守哉!漢與楚相距,士卒罷敝,而匈奴冒頓新服北夷,來為邊寇,孟舒知士卒罷敝,不忍出言,士爭臨城死敵,如子為父,以故死者數百人,孟舒豈敺之哉!是乃孟舒所以為長者。」於是上曰:「賢哉孟舒!」復召以為雲中守。

後數歲,叔坐法失官。梁孝王使人殺漢議臣爰盎,景帝召叔案梁,具得其事。還報,上曰:「梁有之乎?」對曰:「有之。」「事安在?」叔曰:「上無以梁事為問也。今梁王不伏誅,是廢漢法也;如其伏誅,太后食不甘味,臥不安席,此憂在陛下。」於是上大賢之,以為魯相。

相初至官,民以王取其財物自言者百餘人。叔取其渠率二千人笞,怒之曰:「王非汝主邪?何敢自言主!」魯王聞之,大慚,發中府錢,使相償之。相曰:「王自使人償之,不爾,是王為惡而相為善也。」

魯王好獵,相常從入苑中,王輒休相就館。相常暴坐苑外,終不休,曰:「吾王暴露,獨何為舍?」王以故不大出遊。

數年以官卒,魯以百金祠,少子仁不受,曰:「義不傷先人名。」

仁以壯勇為衛將軍舍人,數從擊匈奴。衛將軍進言仁為郎中,至二千石、丞相長史,失官。後使刺三河,還,奏事稱意,拜為京輔都尉。月餘,遷司直。數歲,戾太子舉兵,仁部閉城門,令太子得亡,坐縱反者族。

贊曰:以項羽之氣,而季布以勇顯名楚,身履軍搴旗者數矣,可謂壯士。及至困厄奴僇,苟活而不變,何也?彼自負其材,受辱不羞,欲有所用其未足也,故終為漢名將。賢者誠重其死。夫婢妾賤人,感概而自殺,非能勇也,其畫無俚之至耳。欒布哭彭越,田叔隨張敖,赴死如歸,彼誠知所處,雖古烈士,何以加哉!


\end{pinyinscope}