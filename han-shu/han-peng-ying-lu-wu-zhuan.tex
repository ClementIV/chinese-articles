\article{韓彭英盧吳傳}

\begin{pinyinscope}
韓信,淮陰人也。家貧無行,不得推擇為吏,又不能治生為商賈,常從人寄食。其母死無以葬,乃行營高燥地,令傍可置萬家者。信從下鄉南昌亭長食,亭長妻苦之,乃晨炊蓐食。食時信往,不為具食。信亦知其意,自絕去。至城下釣,有一漂母哀之,飯信,竟漂數十日。信謂漂母曰:「

吾必重報母。」母怒曰:「大丈夫不能自食,吾哀王孫而進食,豈望報乎!」淮陰少年又侮信曰:「雖長大,好帶刀劍,怯耳。」眾辱信曰:「能死,刺我;不能,出跨下。」於是信孰視,俛出跨下。一巿皆笑信,以為怯。

及項梁度淮,信乃杖劍從之,居戲下,無所知名。梁敗,又屬項羽,為郎中。信數以策干項羽,羽弗用。漢王之入蜀,信亡楚歸漢,未得知名,為連敖。坐法當斬,其疇十三人皆已斬,至信,信乃仰視,適見滕公,曰:「上不欲就天

子乎?而斬壯士!」滕公奇其言,壯其貌,釋弗斬。與語,大說之,言於漢王。漢王以為治粟都尉,上未奇之也。

數與蕭何語,何奇之。至南鄭,諸將道亡者數十人。信度何等已數言上,不我用,即亡。何聞信亡,不及以聞,自追之。人有言上曰:「丞相何亡。」上怒,如失左右手。居一二日,何來謁。上且怒且喜,罵何曰:「若亡,何也?」何曰:「臣非敢亡,追亡者耳。」上曰:「所追者誰也?」曰:「韓信。」上復罵曰:「諸將亡者已數十,公無所追;追信,詐也。」何曰:「諸將易得,至如信,國士無雙。王必欲長王漢中,無所事信;必欲爭天下,非信無可與計事者。顧王策安決。」王曰:「吾亦欲東耳,安能鬱鬱久居此乎?」何曰:「王計必東,能用信,信即留;不能用信,信終亡耳。」王曰:「吾為公以為將。」何曰:「雖為將,信不留。」王曰:「以為大將。」何曰:「幸甚。」於是王欲召信拜之。何曰:「王素嫚無禮,今拜大將如召小兒,此乃信所以去也。必欲拜之,擇日齋戒,設壇場具禮,乃可。」王許之。諸將皆喜,人人各自以為得大將。至拜,乃韓信也,一軍皆驚。

信以拜,上坐。王曰:「丞相數言將軍,將軍何以教寡人計策?」信謝,因問王曰:「今東鄉爭權天下,豈非項王邪?」上曰:「然。」信曰:「大王自料勇悍仁彊孰與項王?」漢王默然良久,曰:「弗如也。」信再拜賀曰:「唯信亦以為大王弗如也。然臣嘗事項王,請言項王為人也。項王意烏猝嗟,千人皆廢,然不能任屬賢將,此特匹夫之勇也。項王見人恭謹,言語姁姁,人有病疾,涕泣分食飲,至使人有功,當封爵,刻印刓,忍不能予,此所謂婦人之仁也。項王雖霸天下而臣諸侯,不居關中而都彭城;又背義帝約,而以親愛王,諸侯不平。諸侯之見項王逐義帝江南,亦皆歸逐其主,自王善地。項王所過亡不殘滅,多怨百姓,百姓不附,特劫於威,彊服耳。名雖為霸,實失天下心,故曰其彊易弱。今大王誠能反其道,任天下武勇,何不誅!以天下城邑封功臣,何不服!以義兵從思東歸之士,何不散!且三秦王為秦將,將秦子弟數歲,而所殺亡不可勝計,又欺其眾降諸侯。至新安,項王詐阬秦降卒二十餘萬人,唯獨邯、欣、翳脫。秦父兄怨此三人,痛於骨髓。今楚強以威王此三人,秦民莫愛也。大王之入武關,秋豪亡所害,除秦苛法,與民約,法三章耳,秦民亡不欲得大王王秦者。於諸侯之約,大王當王關中,關中民戶知之。王失職之蜀,民亡不恨者。今王舉而東,三秦可傳檄而定也。」於是漢王大喜,自以為得信晚。遂聽信計,部署諸將所擊。

漢王舉兵東出陳倉,定三秦。二年,出關,收魏、河南,韓、殷王皆降。令齊、趙共擊楚彭城,漢兵敗散而還。信復發兵與漢王會滎陽,復擊破楚京、索間,以故楚兵不能西。

漢之敗卻彭城,塞王欣、翟王翳亡漢降楚,齊、趙、魏亦皆反,與楚和。漢王使酈生往說魏王豹,豹不聽,乃以信為左丞相擊魏。信問酈生:「魏得毋用周叔為大將乎?」曰:「跷直也。」信曰:「豎子耳。」遂進兵擊魏。魏盛兵蒲阪,塞臨晉。信乃益為疑兵,陳船欲度臨晉,而伏兵從夏陽以木罌缶度軍,襲安邑。魏王豹驚,引兵迎信。信遂虜豹,定河東,使人請漢王:「願益兵三萬人,臣請以北舉燕、趙,東擊齊,南絕楚之糧道,西與大王會於滎陽。」漢王與兵三萬人,遣張耳與俱,進擊趙、代。破代,禽夏說閼與。信之下魏、代,漢輒使人收其精兵,詣滎陽以距楚。

信、耳以兵數萬,欲東下井陘擊趙。趙王、成安君陳餘聞漢且襲之,聚兵井陘口,號稱二十萬。廣武君李左車說成安君曰:「聞漢將韓信涉西河,虜魏王,禽夏說,新喋血閼與。今乃輔以張耳,議欲以下趙,此乘勝而去國遠鬥,其鋒不可當。臣聞『千里餽糧,士有飢色;樵蘇後爨,師不宿飽。』今井陘之道,車不得方軌,騎不得成列,行數百里,其勢糧食必在後。願足下假臣奇兵三萬人,從間道絕其輜重;足下深溝高壘勿與戰。彼前不得鬥,退不得還,吾奇兵絕其後,野無所掠鹵,不至十日,兩將之頭可致戲下。願君留意臣之計,必不為二子所禽矣。」成安君,儒者,常稱義兵不用詐謀奇計,謂曰:「吾聞兵法『什則圍之,倍則戰。』今韓信兵號數萬,其實不能,千里襲我,亦以罷矣。今如此避弗擊,後有大者,何以距之?諸侯謂吾怯,而輕來伐我。」不聽廣武君策。

信使間人窺知其不用,還報,則大喜,乃敢引兵遂下。未至井陘口三十里,止舍。夜半傳發,選輕騎二千人,人持一赤幟,從間道萆山而望趙軍,戒曰:「趙見我走,必空壁逐我,若疾入,拔趙幟,立漢幟。」令其裨將傳餐,曰:「今日破趙會食。」諸將皆嘸然,陽應曰:「諾。」信謂軍吏曰:「趙已先據便地壁,且彼未見大將旗鼓,未肯擊前行,恐吾阻險而還。」乃使萬人先行,出,背水陳。趙兵望見大笑。平旦,信建大將旗鼓,鼓行出井陘口,趙開壁擊之,大戰良久。於是信、張耳棄鼓旗,走水上軍,復疾戰。趙空壁爭漢鼓旗,逐信、耳。信、耳已入水上軍,軍皆殊死戰,不可敗。信所出奇兵二千騎者,候趙空壁逐利,即馳入趙壁,皆拔趙旗幟,立漢赤幟二千。趙軍已不能得信、耳等,欲還歸壁,壁皆漢赤幟,大驚,以漢為皆已破趙王將矣,遂亂,遁走。趙將雖斬之,弗能禁。於是漢兵夾擊,破虜趙軍,斬成安君泜水上,禽趙王歇。

信乃令軍毋斬廣武君,有生得之者,購千金。頃之,有縛而至戲下者,信解其縛,東鄉坐,西鄉對,而師事之。

諸校劾首虜休,皆賀,因問信曰:「兵法有『

右背山陵,前左水澤』,今者將軍令臣等反背水陳,曰破趙會食,臣等不服。然竟以勝,此何術也?」信曰:「此在兵法,顧諸君弗察耳。兵法不曰『陷之死地而後生,投之亡地而後存』乎?且信非得素拊循士大夫,經所謂『敺市人而戰之』也,其勢非置死地,人人自為戰;今即予生地,皆走,寧尚得而用之乎!」諸將皆服曰:「非所及也。」

於是問廣武君曰:「僕欲北攻燕,東伐齊,何若有功?」廣武君辭曰:「臣聞『亡國之大夫不可以圖存,敗軍之將不可以語勇。』若臣者,何足以權大事乎!」信曰:「僕聞之,百里奚居虞而虞亡,之秦而秦伯,非愚於虞而智於秦也,用與不用,聽與不聽耳。向使成安君聽子計,僕亦禽矣。僕委心歸計,願子勿辭。」廣武君曰:「臣聞『智者千慮,必有一失;愚者千慮,亦有一得。』故曰『狂夫之言,聖人擇焉。』故恐臣計未足用,願效愚忠。故成安君有百戰百勝之計,一日而失之,軍敗鄗下,身死泜水上。今足下虜魏王,禽夏說,不旬朝破趙二十萬眾,誅成安君。名聞海內,威震諸侯,眾庶莫不輟作怠惰,靡衣媮食,傾耳以待禽者。然而眾勞卒罷,其實難用也。今足下舉倦敝之兵,頓之燕堅城之下,情見力屈,欲戰不拔,曠日持久,糧食單竭。若燕不破,齊必距境而以自彊。二國相持,則劉項之權未有所分也。臣愚,竊以為亦過矣。」信曰:「然則何由?」廣武君對曰:「當今之計,不如按甲休兵,百里之內,牛酒日至,以饗士大夫,北首燕路,然後發一乘之使,奉咫尺之書,以使燕,燕必不敢不聽。從燕而東臨齊,雖有智者,亦不知為齊計矣。如是,則天下事可圖也。兵故有先聲而後實者,此之謂也。」信曰:「善。敬奉教。」於是用廣武君策,發使燕,燕從風而靡。乃遣使報漢,因請立張耳王趙以撫其國。漢王許之。

楚數使奇兵度河擊趙,王耳、信往來救趙,因行定趙城邑,發卒佐漢。楚方急圍漢王滎陽,漢王出,南之宛、葉,得九江王布,入成皋,楚復急圍之。四年,漢王出成皋,度河,獨與滕公從張耳軍修武。至,宿傳舍。晨自稱漢使,馳入壁。張耳、韓信未起,即其臥,奪其印符,麾召諸將易置之。信、耳起,乃知獨漢王來,大驚。漢王奪兩人軍,即令張耳備守趙地,拜信為相國,發趙兵未發者擊齊。

信引兵東,未度平原,聞漢王使酈食其已說下齊。信欲止,蒯通說信令擊齊。語在通傳。信然其計,遂渡河,襲歷下軍,至臨菑。齊王走高密,使使於楚請救。信已定臨菑,東追至高密西。楚使龍且將,號稱二十萬,救齊。

齊王、龍且并軍與信戰,未合。或說龍且曰:「漢兵遠鬥,窮寇戰,鋒不可當也。齊、楚自居其地戰,兵易敗散。不如深壁,令齊王使其信臣招所亡城,城聞王在,楚來救,必反漢。漢二千里客居齊,齊城皆反之,其勢無所得食,可毋戰而降也。」龍且曰:「吾平生知韓信為人,易與耳。寄食於漂母,無資身之策;受辱於跨下,無兼人之勇,不足畏也。且救齊而降之,吾何功?今戰而勝之,齊半可得,何為而止!」遂戰,與信夾濰水陳。信乃夜令人為萬餘囊,沙以壅水上流,引兵半度,擊龍且。陽不勝,還走。龍且果喜曰:「固知信怯。」遂追度水。信使人決壅囊,水大至。龍且軍太半不得度,即急擊,殺龍且。龍且水東軍散走,齊王廣亡去。信追北至城陽,虜廣。楚卒皆降,遂平齊。

使人言漢王曰:「齊夸詐多變,反覆之國,南邊荒,不為假王以填之,其勢不定。今權輕,不足以安之,臣請自立為假王。」當是時,楚方急圍漢王於滎陽,使者至,發書,漢王大怒,罵曰:「吾困於此,旦暮望而來佐我,乃欲自立為王!」張良、陳平伏後躡漢王足,因附耳語曰:「漢方不利,寧能禁信之自王乎?不如因立,善遇之,使自為守。不然,變生。」漢王亦寤,因復罵曰:「大丈夫定諸侯,即為真王耳,何以假為!」遣張良立信為齊王,徵其兵使擊楚。

楚以亡龍且,項王恐,使盱台人武涉往說信曰:「足下何不反漢與楚?楚王與足下有舊故。且漢王不可必,身居項王掌握中數矣,然得脫,背約,復擊項王,其不可親信如此。今足下雖自以為與漢王為金石交,然終為漢王所禽矣。足下所以得須臾至今者,以項王在。項王即亡,次取足下。何不與楚連和,三分天下而王齊?今釋此時,自必於漢王以擊楚,且為智者固若此邪!」信謝曰:「臣得事項王數年,官不過郎中,位不過執戟,言不聽,畫策不用,故背楚歸漢。漢王授我上將軍印,數萬之眾,解衣衣我,推食食我,言聽計用,吾得至於此。夫人深親信我,背之不祥。幸為信謝項王。」武涉已去,蒯通知天下權在於信,深說以三分天下,之計。語在通傳。信不忍背漢,又自以功大,漢王不奪我齊,遂不聽。

漢王之敗固陵,用張良計,徵信將兵會陔下。項羽死,高祖襲奪信軍,徙信為楚王,都下邳。

信至國,召所從食漂母,賜千金。及下鄉亭長,錢百,曰:「公,小人,為德不竟。」召辱己少年令出跨下者,以為中尉,告諸將相曰:「此壯士也。方辱我時,寧不能死?死之無名,故忍而就此。」

項王亡將鍾離辚家在伊廬,素與信善。項王敗,辚亡歸信。漢怨辚,聞在楚,詔楚捕之。信初之國,行縣邑,陳兵出入。有變告信欲反,書聞,上患之。用陳平謀,偽游於雲夢者,實欲襲信,信弗知。高祖且至楚,信欲發兵,自度無罪;欲謁上,恐見禽。人或說信曰:「斬辚謁上,上必喜,亡患。」信見辚計事,辚曰:「漢所以不擊取楚,以辚在。公若欲捕我自媚漢,吾今死,公隨手亡矣。」乃罵信曰:「公非長者!」卒自剄。信持其首謁於陳。高祖令武士縛信,載後車。信曰:「果若人言,『

狡兔死,良狗亨。』」上曰:「人告公反。」遂械信。至雒陽,赦以為淮陰侯。

信知漢王畏惡其能,稱疾不朝從。由此日怨望,居常鞅鞅,羞與絳、灌等列。嘗過樊將軍噲,噲趨拜送迎,言稱臣,曰:「大王乃肯臨臣。」信出門,笑曰:「生乃與噲等為伍!」

上嘗從容與信言諸將能各有差。上問曰:「如我,能將幾何?」信曰:「陛下不過能將十萬。」上曰:「如公何如?」曰:「

如臣,多多益辦耳。」上笑曰:「多多益辦,何為為我禽?」信曰:「陛下不能將兵,而善將將,此乃信之為陛下禽也。且陛下所謂天授,非人力也。」

後陳豨為代相監邊,辭信,信挈其手,與步於庭數匝,仰天而嘆曰:「子可與言乎?吾欲與子有言。」豨因曰:「唯將軍命。」信曰:「公之所居,天下精兵處也,而公,陛下之信幸臣也。人言公反,陛下必不信;再至,陛下乃疑;三至,必怒而自將。吾為公從中起,天下可圖也。」陳豨素知其能,信之,曰:「謹奉教!」

漢十年,豨果反,高帝自將而往,信病不從。陰使人之豨所,而與家臣謀,夜詐赦諸官徒奴,欲發兵襲呂后、太子。部署已定,待豨報。其舍人得罪信,信囚,欲殺之。舍人弟上書變告信欲反狀於呂后。呂后欲召,恐其黨不亂,乃與蕭相國謀,詐令人從帝所來,稱豨已死,群臣皆賀。相國紿信曰:「雖病,強入賀。」信入,呂后使武士縛信,斬之長樂鍾室。信方斬,曰:「吾不用蒯通計,反為女子所詐,豈非天哉!」遂夷信三族。

高祖已破豨歸,至,聞信死,且喜且哀之,問曰:「信死亦何言?」呂后道其語。高祖曰:「此齊辯士蒯通也。」召欲亨之。通至自說,釋弗誅。語在通傳。

彭越字仲,昌邑人也。常漁鉅野澤中,為盜。陳勝起,或謂越曰:「豪桀相立畔秦,仲可效之。」越曰:「兩龍方鬥,且待之。」

居歲餘,澤間少年相聚百餘人,往從越,「請仲為長」,越謝不願也。少年強請,乃許。與期旦日日出時,後會者斬。旦日日出,十餘人後,後者至日中。於是越謝曰:「臣老,諸君強以為長。今期而多後,不可盡誅,誅最後者一人。」令校長斬之。皆笑曰:「

何至是!請後不敢。」於是越乃引一人斬之,設壇祭,令徒屬。徒屬皆驚,畏越,不敢仰視。乃行略地,收諸侯散卒,得千餘人。

沛公之從碭北擊昌邑,越助之。昌邑未下,沛公引兵西。越亦將其眾居鉅野澤中,收魏敗散卒。項籍入關,王諸侯,還歸,越眾萬餘人無所屬。齊王田榮叛項王,漢乃使人賜越將軍印,使下濟陰以擊楚。楚令蕭公角將兵擊越,越大破楚軍。漢二年春,與魏豹及諸侯東擊楚,越將其兵三萬餘人,歸漢外黃。漢王曰:「彭將軍收魏地,得十餘城,欲急立魏後。今西魏王豹,魏咎從弟,真魏也。」乃拜越為魏相國,擅將兵,略定梁地。

漢王之敗彭城解而西也,越皆亡其所下城,獨將其兵北居河上。漢三年,越常往來為漢游兵擊楚,絕其糧於梁地。項王與漢王相距滎陽,越攻下睢陽、外黃十七城。項王聞之,乃使曹咎守成皋,自東收越所下城邑,皆復為楚。越將其兵北走穀城。項王南走陽夏,越復下昌邑旁二十餘城,得粟十餘萬斛,以給漢食。

漢王敗,使使召越并力擊楚,越曰:「魏地初定,尚畏楚,未可去。」漢王追楚,為項籍所敗固陵。乃謂留侯曰:「諸侯兵不從,為之奈何?」留侯曰:「彭越本定梁地,功多,始君王以魏豹故,拜越為相國。今豹死亡後,且越亦欲王,而君王不蚤定。今取睢陽以北至穀城,皆許以王彭越。」又言所以許韓信。語在高紀。於是漢王發使使越,如留侯策。使者至,越乃引兵會垓下。項籍死,立越為梁王,都定陶。

六年,朝陳。九年,十年,皆來朝長安。

陳豨反代地,高帝自往擊之,至邯鄲,徵兵梁。梁王稱病,使使將兵詣邯鄲。高帝怒,使人讓梁王。梁王恐,欲自往謝。其將扈輒曰:「王始不往,見讓而往,往即為禽,不如遂發兵反。」梁王不聽,稱病。梁太僕有罪,亡走漢,告梁王與扈輒謀反。於是上使使掩捕梁王,囚之雒陽。有司治反形已具,請論如法。上赦以為庶人,徙蜀青衣。西至鄭,逢呂后從長安東,欲之雒陽,道見越。越為呂后泣涕,自言亡罪,願處故昌邑。呂后許諾,詔與俱東。至雒陽,呂后言上曰:「彭越壯士也,今徙之蜀,此自遺患,不如遂誅之。妾謹與俱來。」於是呂后令其舍人告越復謀反。廷尉奏請,遂夷越宗族。

黥布,六人也,姓英氏。少時客相之,當刑而王。及壯,坐法黥,布欣然笑曰:「人相我當刑而王,幾是乎?」人有聞者,共戲笑之。布以論輸驪山,驪山之徒數十萬人,布皆與其徒長豪桀交通,乃率其曹耦,亡之江中為群盜。

陳勝之起也,布乃見番君,其眾數千人。番君以女妻之。章邯之滅陳勝,破呂臣軍,布引兵北擊秦左右校,破之青波,引兵而東。聞項梁定會稽,西度淮,布以兵屬梁。梁西擊景駒、秦嘉等,布常冠軍。項梁聞陳涉死,立楚懷王,以布為當陽君。項梁敗死,懷王與布及諸侯將皆聚彭城。當是時,秦急圍趙,趙數使人請救懷王。懷王使宋義為上將,項籍與布皆屬之,北救趙。及籍殺宋義河上,自立為上將軍,使布先涉河,擊秦軍,數有利。籍乃悉引兵從之,遂破秦軍,降章邯等。楚兵常勝,功冠諸侯。諸侯兵皆服屬楚者,以布數以少敗眾也。

項籍之引兵西至新安,又使布等夜擊阬章邯秦卒二十餘萬人。至關,不得入,又使布等先從間道破關下軍,遂得入。至咸陽,布為前鋒。項王封諸將,立布為九江王,都六。尊懷王為義帝,徙都長沙,乃陰令布擊之。布使將追殺之郴。

齊王田榮叛楚,項王往擊齊,徵兵九江,布稱病不往,遣將將數千人行。漢之敗楚彭城,布又稱病不佐楚。項王由此怨布,數使使者譙讓召布,布愈恐,不敢往。項王方北憂齊、趙,西患漢,所與者獨布,又多其材,欲親用之,以故未擊。

漢王與楚大戰彭城,不利,出梁地,至虞,謂左右曰:「

如彼等者,無足與計天下事者。」謁者隨何進曰:「不審陛下所謂。」漢王曰:「孰能為我使淮南,使之發兵背楚,留項王於齊數月,我之取天下可以萬全。」隨何曰:「臣請使之。」乃與二十人俱使淮南。至,太宰主之,三日不得見。隨何因說太宰曰:「王之不見何,必以楚為彊,以漢為弱,此臣之所為使。使何得見,言之而是邪,是大王所欲聞也;言之而非邪,使何等二十人伏斧質淮南巿,以明背漢而與楚也。」太宰乃言之王,王見之。隨何曰:「漢王使使臣敬進書大王御者,竊怪大王與楚何親也。」淮南王曰:「寡人北鄉而臣事之。」隨何曰:「大王與項王俱列為諸侯,北鄉而臣事之,必以楚為彊,可以託國也。項王伐齊,身負版築,以為士卒先。大王宜悉淮南之眾,身自將,為楚軍前鋒,今乃發四千人以助楚。夫北面而臣事人者,固若是乎?夫漢王戰於彭城,項王未出齊也,大王宜埽淮南之眾,日夜會戰彭城下。今撫萬人之眾,無一人渡淮者,陰拱而觀其孰勝。夫託國於人者,固若是乎?大王提空名以鄉楚,而欲厚自託,臣竊為大王不取也。然大王不背楚者,以漢為弱也。夫楚兵雖彊,天下負之以不義之名,以其背明約而殺義帝也。然而楚王特以戰勝自彊。漢王收諸侯,還守成皋、滎陽,下蜀、漢之粟,深溝壁壘,分卒守徼乘塞。楚人還兵,間以梁地,深入敵國八九百里,欲戰則不得,攻城則力不能,老弱轉糧千里之外。楚兵至滎陽、成皋,漢堅守而不動,進則不得攻,退則不能解,故楚兵不足罷也。使楚兵勝漢,則諸侯自危懼而相救。夫楚之彊,適足以致天下之兵耳。故楚不如漢,其勢易見也。今大王不與萬全之漢,而自託於危亡之楚,臣竊為大王或之。臣非以淮南之兵足以亡楚也。夫大王發兵而背楚,項王必留;留數月,漢之取天下可以萬全。臣請與大王杖劍而歸漢王,漢王必裂地而分大王,又況淮南,必大王有也。故漢王敬使使臣進愚計,願大王之留意也。」淮南王曰:「請奉命。」陰許叛楚與漢,未敢泄。

楚使者在,方急責布發兵,隨何直入曰:「九江王已歸漢,楚何以得發兵!」布愕然。楚使者起,何因說布曰:「事已搆,獨可遂殺楚使,毋使歸,而疾走漢并力。」布曰:「如使者教。」因起兵而攻楚。楚使項聲、龍且攻淮南,項王留而攻下邑。數月,龍且攻淮南,破布軍。布欲引兵走漢,恐項王擊之,故間行與隨何俱歸漢。

至,漢王方踞床洗,而召布入見。布大怒,悔來,欲自殺。出就舍,張御食飲從官如漢王居,布又大喜過望。於是乃使人之九江。楚已使項伯收九江兵,盡殺布妻子。布使者頗得故人幸臣,將眾數千人歸漢。漢益分布兵而與俱北,收兵至成皋。四年秋七月,立布為淮南王,與擊項籍。布使人之九江,得數縣。五年,布與劉賈入九江,誘大司馬周殷,殷反楚。遂舉九江兵與漢擊楚,破垓下。

項籍死,上置酒對眾折隨何曰腐儒,「為天下安用腐儒哉!」隨何跪曰:「夫陛下引兵攻彭城,楚王未去齊也,陛下發步卒五萬人,騎五千,能以取淮南乎?」曰:「不能。」隨何曰:「

陛下使何與二十人使淮南,如陛下之意,是何之功賢於步卒數萬,騎五千也。然陛下謂何腐儒,『為天下安用腐儒』,何也?」上曰:「

吾方圖子之功。」乃以隨何為護軍中尉。布遂剖符為淮南王,都六,九江、廬江、衡山、豫章郡皆屬焉。

六年,朝陳。七年,朝雒陽。九年,朝長安。

十一年,高后誅淮陰侯,布因心恐。夏,漢誅梁王彭越,盛其醢以遍賜諸侯。至淮南,淮南王方獵,見醢,因大恐,陰令人部聚兵,候伺旁郡警急。

布有所幸姬病,就醫。醫家與中大夫賁赫對門,赫乃厚餽遺,從姬飲醫家。姬侍王,從容語次,譽赫長者也。王怒曰:「女安從知之?」具道,王疑與亂。赫恐,稱病。王愈怒,欲捕赫。赫上變事,乘傳詣長安。布使人追,不及。赫至,上變,言布謀反有端,可先未發誅也。上以其書語蕭相國,蕭相國曰:「布不宜有此,恐仇怨妄誣之。請繫赫,使人微驗淮南王。」布見赫以罪亡上變,已疑其言國陰事,漢使又來,頗有所驗,遂族赫家,發兵反。

反書聞,上乃赦赫,以為將軍。召諸侯問:「布反,為之柰何?」皆曰:「發兵阬豎子耳,何能為!」汝陰侯滕公以問其客薛公,薛公曰:「是固當反。」滕公曰:「上裂地而封之,疏爵而貴之,南面而立萬乘之主,其反何也?」薛公曰:「前年殺彭越,往年殺韓信,三人皆同功一體之人也。自疑禍及身,故反耳。」滕公言之上曰:「臣客故楚令尹薛公,其人有籌策,可問。」上乃見問薛公,對曰:「布反不足怪也。使布出於上計,山東非漢之有也;出於中計,勝負之數未可知也;出於下計,陛下安枕而臥矣。」上曰:「

何謂上計?」薛公對曰:「東取吳,西取楚,并齊取魯,傳檄燕、趙,固守其所,山東非漢之有也。」「何謂中計?」「東取吳,西取楚,并韓取魏,據敖倉之粟,塞成皋之險,勝敗之數未可知也。」「何謂下計?」「東取吳,西取下蔡,歸重於越,身歸長沙,陛下安枕而臥,漢無事矣。」上曰:「是計將安出?」薛公曰:「

出下計。」上曰:「胡為廢上計而出下計?」薛公曰:「布故驪山之徒也,致萬乘之主,此皆為身,不顧後為百姓萬世慮者也,故出下計。」上曰:「善。」封薛公千戶。遂發兵自將東擊布。

布之初反,謂其將曰:「上老矣,厭兵,必不能來。使諸將,諸將獨患淮陰、彭越,今已死,餘不足畏。」故遂反。果如薛公揣之,東擊荊,荊王劉賈走死富陵。盡劫其兵,度淮擊楚。楚發兵與戰徐、僮間,為三軍,欲以相救為奇。或說楚將曰:「布善用兵,民素畏之。且兵法,諸侯自戰其地為散地。今別為三,彼敗吾一,餘皆走,安能相救!」不聽。布果破其一軍,二軍散走。

遂西,與上兵遇蘄西,會诓。布兵精甚,上乃壁庸城,望布軍置陳如項籍軍。上惡之,與布相望見,隃謂布「何苦而反?」布曰:「欲為帝耳。」上怒罵之,遂戰,破布軍。布走度淮,數止戰,不利,與百餘人走江南。布舊與番君婚,故長沙哀王使人誘布,偽與俱亡,走越,布信而隨至番陽。番陽人殺布茲鄉,遂滅之。封賁赫為列侯,將率封者六人。

盧綰,豐人也,與高祖同里。綰親與高祖太上皇相愛,及生男,高祖、綰同日生,里中持羊酒賀兩家。及高祖、綰壯,學書,又相愛也。里中嘉兩家親相愛,生子同日,壯又相愛,復賀羊酒。高祖為布衣時,有吏事避宅,綰常隨上下。及高祖初起沛,綰以客從,入漢為將軍,常侍中。從東擊項籍,以太尉常從,出入臥內,衣被食飲賞賜,群臣莫敢望。雖蕭、曹等,特以事見禮,至其親幸,莫及綰者。封為長安侯。長安,故咸陽也。

項籍死,使綰別將,與劉賈擊臨江王共尉,還,從擊燕王臧荼,皆破平。時諸侯非劉氏而王者七人。上欲王綰,為群臣觖望。及虜臧荼,乃下詔,詔諸將相列侯擇群臣有功者以為燕王。群臣知上欲王綰,皆曰:「太尉長安侯盧綰常從平定天下,功最多,可王。」上乃立綰為燕王。諸侯得幸莫如燕王者。綰立六年,以陳豨事見疑而敗。

豨者,宛句人也,不知始所以得從。及韓王信反入匈奴,上至平城還,豨以郎中封為列侯,以趙相國將監趙、代邊,邊兵皆屬焉。豨少時,常稱慕魏公子,及將守邊,招致賓客。常告過趙,賓客隨之者千餘乘,邯鄲官舍皆滿。豨所以待客,如布衣交,皆出客下。趙相周昌乃求入見上,具言豨賓客盛,擅兵於外,恐有變。上令人覆案豨客居代者諸為不法事,多連引豨。豨恐,陰令客通使王黃、曼丘臣所。漢十年秋,太上皇崩,上因是召豨。豨稱病,遂與王黃等反,自立為代王,劫略趙、代。上聞,乃赦吏民為豨所詿誤劫略者。上自擊豨,破之。語在高紀。

初,上如邯鄲擊豨,燕王綰亦擊其東北。豨使王黃求救匈奴,綰亦使其臣張勝使匈奴,言豨等軍破。勝至胡,故燕王臧荼子衍亡在胡,見勝曰:「公所以重於燕者,以習胡事也。燕所以久存者,以諸侯數反,兵連不決也。今公為燕欲急滅豨等,豨等已盡,次亦至燕,公等亦且為虜矣。公何不令燕且緩豨,而與胡連和?事寬,得長王燕,即有漢急,可以安國。」勝以為然,乃私令匈奴兵擊燕。綰疑勝與胡反,上書請族勝。勝還報,具道所以為者。綰寤,乃詐論他人,以脫勝家屬,使得為匈奴間。而陰使范齊之豨所,欲令久連兵毋決。

漢既斬豨,其裨將降,言燕王綰使范齊通計謀豨所。上使使召綰,綰稱病。又使辟陽侯審食其、御史大夫趙堯往迎綰,因驗問其左右。綰愈恐,閟匿,謂其幸臣曰:「非劉氏而王者,獨我與長沙耳。往年漢族淮陰,誅彭越,皆呂后計。今上病,屬任呂后。呂后婦人,專欲以事誅異姓王者及大功臣。」乃稱病不行。其左右皆亡匿。語頗泄,辟陽侯聞之,歸具報,上益怒。又得匈奴降者,言張勝亡在匈奴,為燕使。於是上曰:「綰果反矣!」使樊噲擊綰。綰悉將其宮人家屬,騎數千,居長城下候伺,幸上病瘉,自入謝。高祖崩,綰遂將其眾亡入匈奴,匈奴以為東胡盧王。為蠻夷所侵奪,常思復歸。居歲餘,死胡中。

高后時,綰妻與其子亡降,會高后病,不能見,舍燕邸,為欲置酒見之。高后竟崩,綰妻亦病死。

孝景帝時,綰孫它人以東胡王降,封為惡谷侯。傳至曾孫,有罪,國除。

吳芮,秦時番陽令也,甚得江湖間民心,號曰番君。天下之初叛秦也,黥布歸芮,芮妻之,因率越人舉兵以應諸侯。沛公攻南陽,乃遇芮之將梅鋗,與偕攻析、酈,降之。及項羽相王,以芮率百越佐諸侯,從入關,故立芮為衡山王,都邾。其將梅鋗功多,封十萬戶,為列侯。項籍死,上以鋗有功,從入武關,故德芮,徙為長沙王,都臨湘,一年薨,諡曰文王,子成王臣嗣。薨,子哀王回嗣。薨,子共王右嗣。薨,子靖王差嗣。孝文後七年薨,無子,國除。初,文王芮,高祖賢之,制詔御史:「長沙王忠,其定著令。」至孝惠、高后時,封芮庶子二人為列侯,傳國數世絕。

贊曰:昔高祖定天下,功臣異姓而王者八國。張耳、吳芮、彭越、黥布、臧荼、盧綰與兩韓信,皆徼一時之權變,以詐力成功,咸得裂土,南面稱孤。見疑強大,懷不自安,事窮勢迫,卒謀叛逆,終於滅亡。張耳以智全,至子亦失國。唯吳芮之起,不失正道,故能傳號五世,以無嗣絕,慶流支庶。有以矣夫,著于甲令而稱忠也!


\end{pinyinscope}