\article{溝洫志}

\begin{pinyinscope}
夏書:禹堙洪水十三年,過家不入門。陸行載車,水行乘舟,泥行乘毳,山行則梮,以別九州;隨山浚川,任土作貢;通九道,陂九澤,度九山。然河災之羨溢,害中國也尤甚。唯是為務,故道河自積石,歷龍門,南到華陰,東下底柱,及盟津、雒內,至于大伾。於是禹以為河所從來者高,水湍悍,難以行平地,數為敗,乃釃二渠以引其河,北載之高地,過洚水,至於大陸,播為九河,同為迎河,入于勃海。九川既疏,九澤既陂,諸夏乂安,功施乎三代。

自是之後,滎陽下引河東南為鴻溝,以通宋、鄭、陳、蔡、曹、衛,與濟、汝、淮、泗會。於楚,西方則通渠漢川、雲夢之際,東方則通溝江淮之間。於吳,則通渠三江、五湖。於齊,則通淄濟之間。於蜀,則蜀守李冰鑿離馈,避沫水之害,穿二江成都中。此渠皆可行舟,有餘則用溉,百姓饗其利。至於它,往往引其水,用溉田,溝渠甚多,然莫足數也。

魏文侯時,西門豹為鄴令,有令名。至文侯曾孫襄王時,與群臣飲酒,王為群臣祝曰:「今吾臣皆西門豹之為人臣也!」史起進曰:「魏氏之行田也以百畝,鄴獨二百畝,是田惡也。漳水在其旁,西門豹不知用,是不智也。知而不興,是不仁也。仁智豹未之盡,何足法也!」於是以史起為鄴令,遂引漳水溉鄴,以富魏之河內。民歌之曰:「鄴有賢令兮為史公,決漳水兮灌鄴旁,終古秽鹵兮生稻梁。」

其後韓聞秦之好興事,欲罷之,無令東伐。乃使水工鄭國間說秦,令鑿涇水,自中山西邸瓠口為渠,並北山,東注洛,三百餘里,欲以溉田。中作而覺,秦欲殺鄭國。鄭國曰:「始臣為間,然渠成亦秦之利也。臣為韓延數歲之命,而為秦建萬世之功。」秦以為然,卒使就渠。渠成而用溉注填閼之水,溉秽鹵之地四萬餘頃,收皆畝一鍾。於是關中為沃野,無凶年,秦以富彊,卒并諸侯,因名曰鄭國渠。

漢興三十有九年,孝文時河決酸棗,東潰金隄,於是東郡大興卒塞之。

其後三十六歲,孝武元光中,河決於瓠子,東南注鉅野,通於淮、泗。上使汲黯、鄭當時興人徒塞之,輒復壞。是時武安侯田蚡為丞相,其奉邑食鄃。鄃居河北,河決而南則鄃無水災,邑收入多。蚡言於上曰:「江河之決皆天事,未易以人力彊塞,彊塞之未必順天。」而望氣用數者亦以為然,是以久不復塞也。

時鄭當時為大司農,言「異時關東漕粟從渭上,度六月罷,而渭水道九百餘里,時有難處。引渭穿渠起長安,旁南山下,至河三百餘里,徑,易遭,度可令三月罷;罷而渠下民田萬餘頃又可得以溉。此捐漕省卒,而益肥關中之地,得穀。」上以為然,令齊人水工徐伯表,發卒數萬人穿漕渠,三歲而通。以漕,大便利。其後漕稍多,而渠下之民頗得以溉矣。

後河東守番係言:「漕從山東西,歲百餘萬石,更底柱之艱,敗亡甚多而煩費。穿渠引汾溉皮氏、汾陰下,引河溉汾陰、蒲阪下,度可得五千頃。故盡河堧棄地,民茭牧其中耳,今溉田之,度可得穀二百萬石以上。穀從渭上,與關中無異,而底柱之東可毋復漕。」上以為然,發卒數萬人作渠田。數歲,河移徙,渠不利,田者不能償種。久之,河東渠田廢,予越人,令少府以為稍入。

其後人有上書,欲通褒斜道及漕,事下御史大夫張湯。湯問之,言「抵蜀從故道,故道多阪,回遠。今穿褒斜道,少阪,近四百里;而褒水通沔,斜水通渭,皆可以行船漕。漕從南陽上沔入褒,褒絕水至斜,間百餘里,以車轉,從斜下渭。如此,漢中穀可致,而山東從沔無限,便於底柱之漕。且褒斜材木竹箭之饒,儗於巴蜀。」上以為然。拜湯子卬為漢中守,發數萬人作褒斜道五百餘里。道果便近,而水多湍石,不可漕。

其後嚴熊言「臨晉民願穿洛以溉重泉以東萬餘頃故惡地。誠即得水,可令畝十石。」於是為發卒萬人穿渠,自徵引洛水至商顏下。岸善崩,乃鑿井,深者四十餘丈。往往為井,井下相通行水。水隤以絕商顏,東至山領十餘里間。井渠之生自此始。穿得龍骨,故名曰龍首渠。作之十餘歲,渠頗通,猶未得其饒。

自河決瓠子後二十餘歲,歲因以數不登,而梁楚之地尤甚。上既封禪,巡祭山川,其明年,乾封少雨。上乃使汲仁、郭昌發卒數萬人塞瓠子決河。於是上以用事萬里沙,則還自臨決河,湛白馬玉璧,令群臣從官自將軍以下皆負薪寘決河。是時東郡燒草,以故薪柴少,而下淇園之竹以為揵。上既臨河決,悼功之不成,乃作歌曰:

瓠子決兮將奈何?浩浩洋洋,慮殫為河。殫為河兮地不得寧,功無已時兮吾山平。吾山平兮鉅野溢,魚弗鬱兮柏冬日。正道弛兮離常流,蛟龍騁兮放遠游。歸舊川兮神哉沛,不封禪兮安知外!皇謂河公兮何不仁,泛濫不止兮愁吾人!齧桑浮兮淮、泗滿,久不反兮水維緩。

一曰:

河湯湯兮激潺湲,北渡回兮迅流難。搴長茭兮湛美玉,河公許兮薪不屬。薪不屬兮衛人罪,燒蕭條兮噫乎何以御水!隤林竹兮揵石菑,宣防塞兮萬福來。

於是卒塞瓠子,築宮其上,名曰宣防。而道河北行二渠,復禹舊跡,而梁、楚之地復寧,無水災。

自是之後,用事者爭言水利。朔方、西河、河西、酒泉皆引河及川谷以溉田。而關中靈軹、成國、湋渠引諸川,汝南、九江引淮,東海引鉅定,泰山下引汶水,皆穿渠為溉田,各萬餘頃。它小渠及陂山通道者,不可勝言也。

自鄭國渠起,至元鼎六年,百三十六歲,而兒寬為左內史,奏請穿鑿六輔渠,以益溉鄭國傍高卬之田。上曰:「農,天下之本也。泉流灌寖,所以育五穀也。左、右內史地,名山川原甚眾,細民未知其利,故為通溝瀆,畜陂澤,所以備旱也。今內史稻田租挈重,不與郡同,其議減。令吏民勉農,盡地利,平繇行水,勿使失時。」

後十六歲,太始二年,趙中大夫白公復奏穿渠。引涇水,首起谷口,尾入櫟陽,注渭中,袤二百里,溉田四千五百餘頃,因名曰白渠。民得其饒,歌之曰:「田於何所?池陽、谷口。鄭國在前,白渠起後。舉臿為雲,決渠為雨。涇水一石,其泥數斗。且溉且糞,長我禾黍。衣食京師,億萬之口。」言此兩渠饒也。

是時方事匈奴,興功利,言便宜者甚眾。齊人延年上書言:「河出昆侖,經中國,注勃海,是其地勢西北高而東南下也。可案圖書,觀地形,令水工準高下,開大河上領,出之胡中,東注之海。如此,關東長無水災,北邊不憂匈奴,可以省隄防備塞,士卒轉輸,胡寇侵盜,覆軍殺將,暴骨原野之患。天下常備匈奴而不憂百越者,以其水絕壤斷也。此功壹成,萬世大利。」書奏,上壯之,報曰:「延年計議甚深。然河乃大禹之所道也,聖人作事,為萬世功,通於神明,恐難改更。」

自塞宣房後,河復北決於館陶,分為屯氏河,東北經魏郡、清河、信都、勃海入海,廣深與大河等,故因其自然,不隄塞也。此開通後,館陶東北四五郡雖時小被水害,而兗州以南六郡無水憂。宣帝地節中,光祿大夫郭昌使行河。北曲三所水流之勢皆邪直貝丘縣。恐水盛,隄防不能禁,乃各更穿渠,直東,經東郡界中,不令北曲。渠通利,百姓安之。元帝永光五年,河決清河靈鳴犢口,而屯氏河絕。

成帝初,清河都尉馮逡奏言:「郡承河下流,與兗州東郡分水為界,城郭所居尤卑下,土壤輕脆易傷。頃所以闊無大害者,以屯氏河通,兩川分流也。今屯氏河塞,靈鳴犢口又益不利,獨一川兼受數河之任,雖高增隄防,終不能泄。如有霖雨,旬日不霽,必盈溢。靈鳴犢口在清河東界,所在處下,雖令通利,猶不能為魏郡、清河減損水害。禹非不愛民力,以地形有勢,故穿九河,今既滅難明,屯氏河不流行七十餘年,新絕未久,其處易浚。又其口所居高,於以分殺水力,道里便宜,可復浚以助大河泄暴水,備非常。又地節時郭昌穿直渠,後三歲,河水更從故第二曲間北可六里,復南合。今其曲勢復邪直貝丘,百姓寒心,宜復穿渠東行。不豫修治,北決病四五郡,南決病十餘郡,然後憂之,晚矣。」事下丞相、御史,白博士許商治尚書,善為算,能度功用。遣行視,以為屯氏河盈溢所為,方用度不足,可且勿浚。

後三歲,河果決於館陶及東郡金隄,泛溢兗、豫,入平原、千乘、濟南,凡灌四郡三十二縣,水居地十五萬餘頃,深者三丈,壞敗官亭室廬且四萬所。御史大夫尹忠對方略疏闊,上切責之,忠自殺。遣大司農非調調均錢穀河決所灌之郡,謁者二人發河南以東漕船五百馏,徙民避水居丘陵,九萬七千餘口。河隄使者王延世使塞,以竹落長四丈,大九圍,盛以小石,兩船夾載而下之。三十六日,河隄成。上曰:「東郡河決,流漂二州,校尉延世隄防三旬立塞。其以五年為河平元年。卒治河者為著外繇六月。惟延世長於計策,功費約省,用力日寡,朕甚嘉之。其以延世為光祿大夫,秩中二千石,賜爵關內侯,黃金百斤。」

後二歲,河復決平原,流入濟南、千乘,所壞敗者半建始時,復遣王延世治之。杜欽說大將軍王鳳,以為「前河決,丞相史楊焉言延世受焉術以塞之,蔽不肯見。今獨任延世,延世見前塞之易,恐其慮害不深。又審如焉言,延世之巧,反不如焉。且水勢各異,不博議利害而任一人,如使不及今冬成,來春桃華水盛,必羨溢,有填淤反壤之害。如此,數郡種不得下,民人流散,盜賊將生,雖重誅延世,無益於事。宜遣焉及將作大匠許商、諫大夫乘馬延年雜作。延世與焉必相破壞,深論便宜,以相難極。商、延年皆明計算,能商功利,足以分別是非,擇其善而從之,必有成功。」鳳如欽言,白遣焉等作治,六月乃成。復賜延世黃金百斤。治河卒非受平賈者,為著外繇六月。

後九歲,鴻嘉四年,楊焉言「從河上下,患底柱隘,可鐫廣之。」上從其言,使焉鐫之。鐫之裁沒水中,不能去,而令水益湍怒,為害甚於故。

是歲,勃海、清河、信都河水湓溢,灌縣邑三十一,敗官亭民舍四萬餘所。河隄都尉許商與丞相史孫禁共行視,圖方略。禁以為「今河溢之害數倍於前決平原時。今可決平原金隄間,開通大河,令入故篤馬河。至海五百餘里,水道浚利,又乾三郡水地,得美田且二十餘萬頃,足以償所開傷民田廬處,又省吏卒治隄救水,歲三萬人以上。」許商以為「古說九河之名,有徒駭、胡蘇、鬲津,今見在成平、東光、鬲界中。自鬲以北至徒駭間,相去二百餘里,今河雖數移徙,不離此域。孫禁所欲開者,在九河南篤馬河,失水之跡,處勢平夷,旱則淤絕,水則為敗,不可許。」公卿皆從商言。先是,谷永以為「河,中國之經瀆,聖王興則出圖書,王道廢則竭絕。今潰溢橫流,漂沒陵阜,異之大者也。修政以應之,災變自除。」是時李尋、解光亦言「陰氣盛則水為之長,故一日之間,晝減夜增,江河滿溢,所謂水不潤下,雖常於卑下之地,猶日月變見於朔望,明天道有因而作也。眾庶見王延世蒙重賞,競言便巧,不可用。議者常欲求索九河故跡而穿之,今因其自決,可且勿塞,以觀水勢。河欲居之,當稍自成川,跳出沙土,然後順天心而圖之,必有成功,而用財力寡。」於是遂止不塞。滿昌、師丹等數言百姓可哀,上數遣使者處業振贍之。

哀帝初,平當使領河隄,奏言「九河今皆寘滅,按經義治水,有決河深川,而無隄防雍塞之文。河從魏郡以東,北多溢決,水跡難以分明。四海之眾不可誣,宜博求能浚川疏河者。」下丞相孔光、大司空何武,奏請部刺史、三輔、三河、弘農太守舉吏民能者,莫有應書。待詔賈讓奏言:

治河有上中下策。古者立國居民,疆理土地,必遺川澤之分,度水勢所不及。大川無防,小水得入,陂障卑下,以為汙澤,使秋水多,得有所休息,左右游波,寬緩而不迫。夫土之有川,猶人之有口也。治土而防其川,猶止兒啼而塞其口,豈不遽止,然其死可立而待也。故曰:「善為川者,決之使道;善為民者,宣之使言。」蓋隄防之作,近起戰國,雍防百川,各以自利。齊與趙、魏,以河為竟。趙、魏瀕山,齊地卑下,作隄去河二十五里。河水東抵齊隄,則西泛趙、魏,趙、魏亦為隄去河二十五里。雖非其正,水尚有所游盪。時至而去,則填淤肥美,民耕田之。或久無害,稍築室宅,遂成聚落。大水時至漂沒,則更起隄防以自救,稍去其城郭,排水澤而居之,湛溺自其宜也。今隄防骥者去水數百步,遠者數里。近黎陽南故大金隄,從河西西北行,至西山南頭,乃折東,與東山相屬。民居金隄東,為廬舍,住十餘歲更起隄,從東山南頭直南與故大隄會。又內黃界中有澤,方數十里,環之有隄,往十餘歲太守以賦民,民今起廬舍其中,此臣親所見者也。東郡白馬故大隄亦復數重,民皆居其間。從黎陽北盡魏界,故大隄去河遠者數十里,內亦數重,此皆前世所排也。河從河內北至黎陽為石隄,激使東抵東郡平剛;又為石隄,使西北抵黎陽、觀下;又為石隄,使東北抵東郡津北;又為石隄,使西北抵魏郡昭陽;又為石隄,激使東北。百餘里間,河再西三東,迫阨如此,不得安息。

今行上策,徙冀州之民當水衝者,決黎陽遮害亭,放河使北入海。河西薄大山,東薄金隄,勢不能遠泛濫,期月自定。難者將曰:「若如此,敗壞城郭田廬冢墓以萬數,百姓怨恨。」昔大禹治水,山陵當路者毀之,故鑿龍門,辟伊闕,析底柱,破碣石,墮斷天地之性。此乃人功所造,何足言也!今瀕河十郡治隄歲費且萬萬,及其大決,所殘無數。如出數年治河之費,以業所徙之民,遵古聖之法,定山川之位,使神人各處其所,而不相奸。且以大漢方制萬里,豈其與水爭咫尺之地哉?此功一立,河定民安,千載無患,故謂之上策。

若乃多穿漕渠於冀州地,使民得以溉田,分殺水怒,雖非聖人法,然亦救敗術也。難者將曰:「河水高於平地,歲增隄防,猶尚決溢,不可以開渠。」臣竊按視遮害亭西十八里,至淇水口,乃有金隄,高一丈。自是東,地稍下,隄稍高,至遮害亭,高四五丈。往六七歲,河水大盛,增丈七尺,壞黎陽南郭門,入至隄下。水未踰隄二尺所,從隄上北望,河高出民屋,百姓皆走上山。水留十三日,隄潰二所,吏民塞之。臣循隄上,行視水勢,南七十餘里,至淇口,水適至隄半,計出地上五尺所。今可從淇口以東為石隄,多張水門。初元中,遮害亭下河去隄足數十步,至今四十餘歲,適至隄足。由是言之,其地堅矣。恐議者疑河大川難禁制,滎陽漕渠足以下之,其水門但用木與土耳,今據堅地作石隄,勢必完安。冀州渠首盡當卬此水門。治渠非穿地也,但為東方一隄,北行三百餘里,入漳水中,其西因山足高地,諸渠皆往往股引取之;旱則開東方下水門溉冀州,水則開西方高門分河流。通渠有三利,不通有三害。民常罷於救水,半失作業;水行地上,湊潤上徹,民則病溼氣,木皆立枯,鹵不生穀;決溢有敗,為魚鱉食:此三害也。若有渠溉,則鹽鹵下溼,填淤加肥;故種禾麥,更為岻稻,高田五倍,下田十倍;轉漕舟船之便:此三利也。今瀕河隄吏卒郡數千人,伐買薪石之費歲數千萬,足以通渠成水門;又民利其溉灌,相率治渠,雖勞不罷。民田適治,河隄亦成,此誠富國安民,興利除害,支數百歲,故謂之中策。

若乃繕完故隄,增卑倍薄,勞費無已,數逢其害,此最下策也。

王莽時,徵能治河者以百數,其大略異者,長水校尉平陵關並言:「河決率常於平原、東郡左右,其地形下而土疏惡。聞禹治河時,本空此地,以為水猥,盛則放溢,少稍自索,雖時易處,猶不能離此。上古難識,近察秦漢以來,河決曹、衛之域,其南北不過百八十里者,可空此地,勿以為官亭民室而已。」大司馬史長安張戎言:「水性就下,行疾則自刮除成空而稍深。河水重濁,號為一石水而六斗泥。今西方諸郡,以至京師東行,民皆引河、渭山川水溉田。春夏乾燥,少水時也,故使河流遲,貯淤而稍淺;雨多水暴至,則溢決。而國家數隄塞之,稍益高於平地,猶築垣而居水也。可各順從其性,毋復灌溉,則百川流行,水道自利,無溢決之害矣。」御史臨淮韓牧以為「可略於禹貢九河處穿之,縱不能為九,但為四五,宜有益。」大司空掾王橫言:「河入勃海,勃海地高於韓牧所欲穿處。往者天嘗連雨,東北風,海水溢,西南出,浸數百里,九河之地已為海所漸矣。禹之行河水,本隨西山下東北去。周譜云定王五年河徙,則今所行非禹之所穿也。又秦攻魏,決河灌其都,決處遂大,不可復補。宜卻徙完平處,更開空,使緣西山足乘高地而東北入海,乃無水災。」沛郡桓譚為司空掾,典其議,為甄豐言:「凡此數者,必有一是。宜詳考驗,皆可豫見,計定然後舉事,費不過數億萬,亦可以事諸浮食無產業民。空居與行役,同當衣食;衣食縣官,而為之作,乃兩便,可以上繼禹功,下除民疾。」王莽時,但崇空語,無施行者。

贊曰:古人有言:「微禹之功,吾其魚乎!」中國川原以百數,莫著於四瀆,而河為宗。孔子曰:「多聞而志之,知之次也。」國之利害,故備論其事。


\end{pinyinscope}