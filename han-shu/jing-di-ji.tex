\article{景帝紀}

\begin{pinyinscope}
孝景皇帝,文帝太子也。母曰竇皇后。後七年六月,文帝崩。丁未,太子即皇帝位,尊皇太后薄氏曰太皇太后,皇后曰皇太后。

九月,有星孛于西方。

元年冬十月,詔曰:「蓋聞古者祖有功而宗有德,制禮樂各有由。歌者,所以發德也;舞者,所以明功也。高廟酎,奏武德、文始、五行之舞。孝惠廟酎,奏文始、五行之舞。孝文皇帝臨天下,通關梁,不異遠方;除誹謗,去肉刑,賞賜長老,收恤孤獨,以遂群生;減耆欲,不受獻,罪人不帑,不誅亡罪,不私其利也;除宮刑,出美人,重絕人之世也。朕既不敏,弗能勝識。此皆上世之所不及,而孝文皇帝親行之。德厚侔天地,利澤施四海,靡不獲福。明象乎日月,而廟樂不稱,朕甚懼焉。其為孝文皇帝廟為昭德之舞,以明休德。然后祖宗之功德,施于萬世,永永無窮,朕甚嘉之。其與丞相、列侯、中二千石、禮官具禮儀奏。」丞相臣嘉等奏曰:「陛下永思孝道,立昭德之舞以明孝文皇帝之盛德,皆臣嘉等愚所不及。臣謹議:世功莫大於高皇帝,德莫盛於孝文皇帝。高皇帝廟宜為帝者太祖之廟,孝文皇帝廟宜為帝諸太宗之廟。天子宜世世獻祖宗之廟。郡國諸侯宜各為孝文皇帝立太宗之廟。諸侯王列侯使者侍祠天子所獻祖宗之廟。請宣布天下。」制曰「可」。

春正月,詔曰:「間者歲比不登,民多乏食,夭絕天年,朕甚痛之。郡國或磽骥,無所農桑诒畜;或地饒廣,薦草莽,水泉利,而不得徙。其議民欲徙寬大地者,聽之。」

夏四月,赦天下。賜民爵一級。

遣御史大夫青翟至代下與匈奴和親。

五月,令田半租。

秋七月,詔曰:「吏受所監臨,以飲食免,重;受財物,賤買貴賣,論輕。廷尉與丞相更議著令。」廷尉信謹與丞相議曰:「吏及諸有秩受其官屬所監、所治、所行、所將,其與飲食計償費,勿論。它物,若買故賤,賣故貴,皆坐臧為盜,沒入臧縣官。吏遷徙免罷,受其故官屬所將監治送財物,奪爵為士伍,免之。無爵,罰金二斤,令沒入所受。有能捕告,畀其所受臧。」

二年冬十二月,有星孛于西南。

令天下男子年二十始傅。

春三月,立皇子德為河間王,閼為臨江王,餘為淮陽王,非為汝南王,彭祖為廣川王,發為長沙王。

夏四月壬午,太皇太后崩。

六月,丞相嘉薨。

封故相國蕭何孫係為列侯。

秋,與匈奴和親。

三年冬十二月,詔曰:「襄平侯嘉子恢說不孝,謀反,欲以殺嘉,大逆無道。其赦嘉為襄平侯,及妻子當坐者復故爵。論恢說及妻子如法。」

春正月,淮陽王宮正殿災。

吳王濞、膠西王卬、楚王戊、趙王遂、濟南王辟光、菑川王賢、膠東王雄渠皆舉兵反。大赦天下。遣太尉亞夫、大將軍竇嬰將兵擊之。斬御史大夫晁錯以謝七國。

二月壬子晦,日有食之。

諸將破七國,斬首十餘萬級。追斬吳王濞於丹徒。膠西王卬、楚王戊、趙王遂、濟南王辟光、菑川王賢、膠東王雄渠皆自殺。夏六月,詔曰:「乃者吳王濞等為逆,起兵相脅,詿誤吏民,吏民不得已。今濞等已滅,吏民當坐濞等及逋逃亡軍者,皆赦之。楚元王子蓺等與濞等為逆,朕不忍加法,除其籍,毋令汙宗室。」立平陸侯劉禮為楚王,續元王後。立皇子端為膠西王,勝為中山王。賜民爵一級。

四年春,復置諸關用傳出入。

夏四月己巳,立皇子榮為皇太子,徹為膠東王。

六月,赦天下,賜民爵一級。

秋七月,臨江王閼薨。

十月戊戌晦,日有蝕之。

五年春正月,作陽陵邑。夏,募民徙陽陵,賜錢二十萬。

遣公主嫁匈奴單于。

六年冬十二月,雷,霖雨。

秋九月,皇后薄氏廢。

七年冬十一月庚寅晦,日有蝕之。

春正月,廢皇太子榮為臨江王。

二月,罷太尉官。

夏四月乙巳,立皇后王氏。

丁巳,立膠東王徹為皇太子。賜民為父後者爵一級。

中元年夏四月,赦天下,賜民爵一級。封故御史大夫周苛、周昌孫子為列侯。

二年春二月,令諸侯王薨、列侯初封及之國,大鴻臚奏諡、誄、策。列侯薨及諸侯太傅初除之官,大行奏諡、誄、策。王薨,遣光祿大夫弔襚祠賵,視喪事,因立嗣子。列侯薨,遣大中大夫弔祠,視喪事,因立嗣。其薨葬,國得發民輓喪,穿復土,治墳無過三百人畢事。

匈奴入燕。

改磔曰棄市,勿復磔。

三月,臨江王榮坐侵太宗廟地,徵詣中尉,自殺。

夏四月,有星孛于西北。

立皇子越為廣川王,寄為膠東王。

秋七月,更郡守為太守,郡尉為都尉。

九月,封故楚、趙傅相內史前死事者四人子皆為列侯。

甲戌晦,日有蝕之。

三年冬十一月,罷諸侯御史大夫官。

春正月,皇太后崩。

夏旱,禁酤酒。秋九月,蝗。有星孛于西北。戊戌晦,日有蝕之。

立皇子乘為清河王。

四年春三月,起德陽宮。

立皇子乘為清河王。

御史大夫綰奏禁馬高五尺九寸以上,齒未平,不得出關。

夏,蝗。

秋,赦徒作陽陵者死罪;欲腐者,許之。

十月戊午,日有蝕之。

五年夏,立皇子舜為常山王。六月,赦天下,賜民爵一級。

秋八月己酉,未央宮東闕災。

更名諸侯丞相為相。

九月,詔曰:「法令度量,所以禁暴止邪也。獄,人之大命,死者不可復生。吏或不奉法令,以貨賂為市,朋黨比周,以苛為察,以刻為明,令亡罪者失職,朕甚憐之。有罪者不伏罪,姦法為暴,甚亡謂也。諸獄疑,若雖文致於法而於人心不厭者,輒讞之。」

六年冬十月,行幸雍,郊五畤。

十二月,改諸官名。定鑄錢偽黃金棄市律。

春三月,雨雪。

夏四月,梁王薨,分梁為五國,立孝王子五人皆為王。

五月,詔曰:「夫吏者,民之師也,車駕衣服宜稱。吏六百石以上,皆長吏也,亡度者或不吏服,出入閭里,與民亡異。令長吏二千石車朱兩轓,千石至六百石朱左轓。車騎從者不稱其官衣服,下吏出入閭巷亡吏體者,二千石上其官屬,三輔舉不如法令者,皆上丞相御史請之。」先是吏多軍功,車服尚輕,故為設禁。又惟酷吏奉憲失中,乃詔有司減笞法,定箠令。語在刑法志。

六月,匈奴入鴈門,至武泉,入上郡,取苑馬。吏卒戰死者二千人。

秋七月辛亥晦,日有蝕之。

後元年春正月,詔曰:「獄,重事也。人有智愚,官有上下。獄疑者讞有司。有司所不能決,移廷尉。有令讞而後不當,讞者不為失。欲令治獄者務先寬。」三月,赦天下,賜民爵一級,中二千石諸侯相爵右庶長。夏,大酺五日,民得酤酒。

五月,地震。秋七月乙巳晦,日有蝕之。

條侯周亞夫下獄死。

二年冬十月,省徹侯之國。

春,匈奴入鴈門,太守馮敬與戰死。發車騎材官屯。

春,以歲不登,禁內郡食馬粟,沒入之。

夏四月,詔曰:「雕文刻鏤,傷農事者也;錦繡纂組,害女紅者也。農事傷則飢之本也,女紅害則寒之原也。夫飢寒並至,而能亡為非者寡矣。朕親耕,后親桑,以奉宗廟粢盛祭服,為天下先;不受獻,減太官,省繇賦,欲天下務農蠶,素有畜積,以備災害。彊毋攘弱,眾毋暴寡,老耆以壽終,幼孤得遂長。今歲或不登,民食頗寡,其咎安在?或詐偽為吏,吏以貨賂為市,漁奪百姓,侵牟萬民。縣丞,長吏也,奸法與盜盜,甚無謂也。其令二千石修其職;不事官職耗亂者,丞相以聞,請其罪。布告天下,使明知朕意。」

五月,詔曰:「人不患其不知,患其為詐也;不患其不勇,患其為暴也;不患其不富,患其亡厭也。其唯廉士,寡欲易足。今訾算十以上乃得宦,廉士算不必眾。有市籍不得宦,無訾又不得宦,朕甚愍之。訾算四得宦,亡令廉士久失職,貪夫長利。」

秋,大旱。

三年春正月,詔曰:「農,天下之本也。黃金珠玉,飢不可食,寒不可衣,以為幣用,不識其終始。間歲或不登,意為末者眾,農民寡也。其令郡國務勸農桑,益種樹,可得衣食物。吏發民若取庸采黃金珠玉者,坐臧為盜。二千石聽者,與同罪。」

皇太子冠,賜民為父後者爵一級。

甲子,帝崩于未央宮。遺詔賜諸侯王列侯馬二駟,吏二千石黃金二斤,吏民戶百錢。出宮人歸其家,復終身。二月癸酉,葬陽陵。

贊曰:孔子稱「斯民,三代之所以直道而行也」,信哉!周秦之敝,罔密文峻,而姦軌不勝。漢興,掃除煩苛,與民休息。至于孝文,加之以恭儉,孝景遵業,五六十載之間,至於移風易俗,黎民醇厚。周云成康,漢言文景,美矣!


\end{pinyinscope}