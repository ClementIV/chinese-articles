\article{元帝紀}

\begin{pinyinscope}
孝元皇帝,宣帝太子也。母曰共哀許皇后,宣帝微時生民間。年二歲,宣帝即位。八歲,立為太子。壯大,柔仁好儒。見宣帝所用多文法吏,以刑名繩下,大臣楊惲、盍寬饒等坐刺譏辭語為罪而誅,嘗侍燕從容言:「

陛下持刑太深,宜用儒生。」宣帝作色曰:「漢家自有制度,本以霸王道雜之,奈何純住德教,用周政乎!且俗儒不達時宜,好是古非今,使人眩於名實,不知所守,何足委任!」乃歎曰:「亂我家者,太子也!」繇是疏太子而愛淮陽王,曰:「淮陽王明察好法,宜為吾子。」而王母張婕妤尤幸。上有意欲用淮陽王代太子,然以少依許氏,俱從微起,故終不背焉。

黃龍元年十二月,宣帝崩。癸巳,太子即皇帝位,謁高廟。尊皇太后曰太皇太后,皇后曰皇太后。

初元元年春正月辛丑,孝宣皇帝葬杜陵。賜諸侯王、公主、列侯黃金,吏二千石以下錢帛,各有差。大赦天下。三月,封皇太后兄侍中中郎將王舜為安平侯。丙午,立皇后王氏。以三輔、太常、郡國公田及苑可省者振業貧民,貲不滿千錢者賦貸種、食。封外祖父平恩戴侯同產弟子中常侍許嘉為平恩侯,奉戴侯後。

夏四月,詔曰:「朕承先帝之聖緒,獲奉宗廟,戰戰兢兢。間者地數動而未靜,懼於天地之戒,不知所繇。方田作時,朕憂蒸庶之失業,臨遣光祿大夫褒等十二人循行天下,存問耆老鰥寡孤獨困乏失職之民,延登賢俊,招顯側陋,因覽風俗之化。相守二千石誠能正躬勞力,宣明教化,以親萬姓,則六合之內和親,庶幾虖無憂矣。書不云乎?『股肱良哉,庶事康哉!』布告天下,使明知朕意。」又曰:「關東今年穀不登,民多困乏。其令郡國被災害甚者毋出租賦。江海陂湖園池屬少府者以假貧民,勿租賦。賜宗室有屬籍者馬一匹至二駟,三老、孝者帛五匹,弟者、力田三匹,鰥寡孤獨二匹,吏民五十戶牛酒。」

六月,以民疾疫,令大官損膳,減樂府員,省苑馬,以振困乏。

秋八月,上郡屬國降胡萬餘人亡入匈奴。

九月,關東郡國十一大水,饑,或人相食,轉旁郡錢穀以相救。詔曰:「間者陰陽不調,黎民饑寒,無以保治,惟德淺薄,不足以充入舊貫之居。其令諸宮館希御幸者勿繕治,太僕減穀食馬,水衡省肉食獸。」

二年春正月,行幸甘泉,郊泰畤。賜雲陽民爵一級,女子百戶牛酒。

立弟竟為清河王。

三月,立廣陵厲王太子霸為王。

詔罷黃門乘輿狗馬,水衡禁囿、宜春下苑、少府佽飛外池、嚴烃池田假與貧民。詔曰:「蓋聞賢聖在位,陰陽和,風雨時,日月光,星辰靜,黎庶康寧,考終厥命。今朕恭承天地,託于公侯之上,明不能燭,德不能綏,災異並臻,連年不息。乃二月戊午,地震于隴西郡,毀落太上皇廟殿壁木飾,壞敗豲道縣城郭官寺及民室屋,壓殺人眾。山崩地裂,水泉湧出。天惟降災,震驚朕師。治有大虧,咎至於斯。夙夜兢兢,不通大變,深惟鬱悼,未知其序。間者歲數不登,元元困乏,不勝饑寒,以陷刑辟,朕甚閔之。郡國被地動災甚者無出租賦。赦天下。有可蠲除減省以便萬姓者,條奏,毋有所諱。丞相、御史、中二千石舉茂材異等直言極諫之士,朕將親覽焉。」

夏四月丁巳,立皇太子。賜御史大夫爵關內侯,中二千石右庶長,天下當為父後者爵一級,列侯錢各二十萬,五大夫十萬。

六月,關東饑,齊地人相食。秋七月,詔曰:「歲比災害,民有菜色,慘怛於心。已詔吏虛倉廩,開府庫振救,賜寒者衣。今秋禾麥頗傷。一年中地再動。北海水溢,流殺人民。陰陽不和,其咎安在?公卿將何以憂之?其悉意陳朕過,靡有所諱。」

冬,詔曰:「國之將興,尊師而重傅。故前將軍望之傅朕八年,道以經書,厥功茂焉。其賜爵關內侯,食邑八百戶,朝朔望。」

十二月,中書令弘恭、石顯等譖望之,令自殺。

三年春,令諸侯相位在郡守下。

珠崖郡山南縣反,博謀群臣。待詔賈捐之以為宜棄珠崖,救民饑饉。乃罷珠崖。

夏四月乙未晦,茂陵白鶴館災。詔曰:「乃者火災降於孝武園館,朕戰栗恐懼。不燭變異,咎在朕躬。群司又未肯極言朕過,以至於斯,將何以寤焉!百姓仍遭凶阨,無以相振,加以煩擾虖苛吏,拘牽乎微文,不得永終性命,朕甚閔焉。其赦天下。」

夏,旱。立長沙煬王弟宗為王。封故海昏侯賀子代宗為侯。

六月,詔曰:「蓋聞安民之道,本繇陰陽。間者陰陽錯謬,風雨不時。朕之不德,庶幾群公有敢言朕之過者,今則不然。媮合苟從,未肯極言,朕甚閔焉。永惟烝庶之饑寒,遠離父母妻子,勞於非業之作,衛於不居之宮,恐非所以佐陰陽之道也。其罷甘泉、建章宮衛,令就農。百官各省費。條奏毋有所諱。有司勉之,毋犯四時之禁。丞相御史舉天下明陰陽災異者各三人。」於是言事者眾,或進擢召見,人人自以得上意。

四年春正月,行幸甘泉,郊泰畤。三月,行幸河東,祠后土。赦汾陰徒。賜民爵一級,女子百戶牛酒,鰥寡高年帛。行所過無出租賦。

五年春正月,以周子南君為周承休侯,位次諸侯王。

三月,行幸雍,祠五畤。

夏四月,有星孛于參。詔曰:「朕之不逮,序位不明,眾僚久崃,未得其人。元元失望,上感皇天,陰陽為變,咎流萬民,朕甚懼之。乃者關東連遭災害,饑寒疾疫,夭不終命。詩不云乎?『凡民有喪,匍匐救之。』其令太官毋日殺,所具各減半。乘輿秣馬,無乏正事而已。罷角抵、上林宮館希御幸者、齊三服官、北假田官、鹽鐵官、常平倉。博士弟子毋置員,以廣學者。賜宗室子有屬籍者馬一匹至二駟,三老、孝者帛,人五匹,弟者、力田三匹,鰥寡孤獨二匹,吏民五十戶牛酒。」省刑罰七十餘事。除光祿大夫以下至郎中保父母同產之令。令從官給事宮司馬中者,得為大父母父母兄弟通籍。

冬十二月丁未,御史大夫貢禹卒。

衛司馬谷吉使匈奴,不還。

永光元年春正月,行幸甘泉,郊泰畤。赦雲陽徒。賜民爵一級,女子百戶牛酒,高年帛。行所過毋出租賦。

二月,詔丞相、御史舉質樸敦厚遜讓有行者,光祿歲以此科第郎、從官。

三月,詔曰:「五帝三王任賢使能,以登至平,而今不治者,豈斯民異哉?咎在朕之不明,亡以知賢也。是故壬人在位,而吉士雍蔽。重以周秦之弊,民漸薄俗,去禮義,觸刑法,豈不哀哉!繇此觀之,元元何辜?其赦天下,令厲精自新,各務農畝。無田者皆假之,貨種、食如貧民。賜吏六百石以上爵五大夫,勤事吏二級,為父後者民一級,女子百戶牛酒,鰥寡孤獨高年帛。」是月雨雪,隕霜傷麥稼,秋罷。

二年春二月,詔曰:「蓋聞唐虞象刑而民不犯,殷周法行而姦軌服。今朕獲承高祖之洪業,託位公侯之上,夙夜戰栗,永惟百姓之急,未嘗有忘焉。然而陰陽未調,三光晻昧。元元大困,流散道路,盜賊並興。有司又長殘賊,失牧民之術。是皆朕之不明,政有所虧。咎至於此,朕甚自恥。為民父母,若是之薄,謂百姓何!其大赦天下,賜民爵一級,女子百戶牛酒,鰥寡孤獨高年、三老、孝弟力田帛。」又賜諸侯王、公主、列侯黃金,中二千石以下至中都官長吏各有差,吏六百石以上爵五大夫,勤事吏各二級。

三月壬戌朔,日有蝕之。詔曰:「朕戰戰栗栗,夙夜思過失,不敢荒寧。惟陰陽不調,未燭其咎。婁敕公卿,日望有效。至今有司執政,未得其中,施與禁切,未合民心。暴猛之俗彌長,和睦之道日衰,百姓愁苦,靡所錯躬。是以氛邪歲增,侵犯太陽,正氣湛掩,日久奪光。乃壬戌,日有蝕之。天見大異,以戒朕躬,朕甚悼焉。其令內郡國舉茂材異等賢良直言之士各一人。」

夏六月,詔曰:「間者連年不收,四方咸困。元元之民,勞於耕耘,又亡成功,困於饑饉,亡以相救。朕為民父母,德不能覆,而有其刑,甚自傷焉。其赦天下。」

秋七月,西羌反,遣右將軍馮奉世擊之。八月,以太常任千秋為奮威將軍,別將五校並進。

三年春,西羌平,軍罷。

三月,立皇子康為濟陽王。

夏四月癸未,大司馬車騎將軍接薨。

冬十一月,詔曰:「乃者己丑地動,中冬雨水,大霧,盜賊並起。吏何不以時禁?各悉意對。」

冬,復鹽鐵官、博士弟子員。以用度不足,民多復除,無以給中外繇役。

四年春二月,詔曰:「朕承至尊之重,不能燭理百姓,婁遭凶咎。加以邊竟不安,師旅在外,賦斂轉輸,元元騷動,窮困亡聊,犯法抵罪。夫上失其道而繩下以深刑,朕甚痛之。其赦天下,所貸貧民勿收責。」

三月,行幸雍,祠五畤。

夏六月甲戌,孝宣園東闕災。

戊寅晦,日有蝕之。詔曰:「蓋聞明王在上,忠賢布職,則群生和樂,方外蒙澤。今朕晻于王道,夙夜憂勞,不通其理,靡瞻不眩,靡聽不惑,是以政令多還,民心未得,邪說空進,事亡成功。此天下所著聞也。公卿大夫好惡不同,或緣姦作邪,侵削細民,元元安所歸命哉!乃六月晦,日有蝕之。詩不云虖?『今此下民,亦孔之哀!』自今以來,公卿大夫其勉思天戒,慎身修永,以輔朕之不逮。直言盡意,無有所諱。」

九月戊子,罷衛思后園及戾園。冬十月乙丑,罷祖宗廟在郡國者。諸陵分屬三輔。以渭城壽陵亭部原上為初陵。詔曰:「安土重遷,黎民之性;骨肉相附,人情所願也。頃者有司緣臣子之義,奏徙郡國民以奉園陵,令百姓遠棄先祖墳墓,破業失產,親戚別離,人懷思慕之心,家有不安之意。是以東垂被虛耗之害,關中有無聊之民,非久長之策也。詩不云虖?『民亦勞止,迄可小康,惠此中國,以綏四方。』今所為初陵者,勿置縣邑,使天下咸安土樂業,亡有動搖之心。布告天下,令明知之。」又罷先后父母奉邑。

五年春正月,行幸甘泉,郊泰畤。三月,上幸河東,祠后土。

秋,潁川水出,流殺人民。吏、從官縣被害者與告。士卒遣歸。

冬,上幸長楊射熊館,布車騎,大獵。

十二月乙酉,毀太上皇、孝惠皇帝寢廟園。

建昭元年春三月,上幸雍,祠五畤。

秋八月,有白蛾群飛蔽日,從東都門至枳道。

冬,河間王元有罪,廢遷房陵。罷孝文太后、孝昭太后寢園。

二年春正月,行幸甘泉,郊泰畤。三月,行幸河東,祠后土。益三河郡太守秩。戶十二萬為大郡。

夏四月,赦天下。

六月,立皇子興為信都王。閏月丁酉,太皇太后上官氏崩。

冬十一月,齊楚地震,大雨雪,樹折屋壞。

淮陽王舅張博、魏郡太守京房坐窺道諸侯王以邪意,漏泄省中語,博要斬,房棄市。

三年夏,令三輔都尉、大郡都尉秩皆二千石。

六月甲辰,丞相玄成薨。

秋,使護西域騎都尉甘延壽、副校尉陳湯撟發戊己校尉屯田吏士及西域胡兵攻郅支單于。冬,斬其首,傳詣京師,縣蠻夷邸門。

四年春正月,以誅郅支單于告祠郊廟。赦天下。群臣上壽置酒,以其圖書示後宮貴人。

夏四月,詔曰:「朕承先帝之休烈,夙夜栗栗,懼不克任。間者陰陽不調,五行失序,百姓饑饉。惟烝庶之失業,臨遣諫大夫博士賞等二十一人循行天下,存問耆老鰥寡孤獨乏困失職之人,舉茂材特立之士。相將九卿,其帥意毋怠,使朕獲觀教化之流焉。」

六月甲申,中山王竟薨。

藍田地沙石雍霸水,安陵岸崩雍涇水,水逆流。

五年春三月,詔曰:「蓋聞明王之治國也,明好惡而定去就,崇敬讓而民興行,故法設而民不犯,令施而民從。今朕獲保宗廟,兢兢業業,匪敢解怠,德薄明晻,教化淺微。傳不云虖?『

百姓有過,在予一人。』其赦天下,賜民爵一級,女子百戶牛酒,三老、孝弟力田帛。」又曰:「方春農桑興,百姓戮力自盡之時也,故是月勞農勸民,無使後時。今不良之吏,覆案小罪,徵召證案,興不急之事,以妨百姓,使失一時之作,亡終歲之功,公卿其明察申敕之。」

夏六月庚申,復戾園。

壬申晦,日有蝕之。

秋七月庚子,復太上皇寢廟園、原廟、昭靈后、武哀王、昭哀后、衛思后園。

竟寧元年春正月,匈奴虖韓邪單于來朝。詔曰:「匈奴郅支單于背叛禮義,既伏其辜,虖韓邪單于不忘恩德,鄉慕禮義,復修朝賀之禮,願保塞傳之無窮,邊垂長無兵革之事。其改元為竟寧,賜單于待詔掖庭王檣為閼氏。」

皇太子冠。賜列侯嗣子爵五大夫,天下為父後者爵一級。

二月,御史大夫延壽卒。

三月癸未,復孝惠皇帝寢廟園、孝文太后、孝昭太后寢園。

夏,封騎都尉甘延壽為列侯。賜副校尉陳湯爵關內侯,黃金百斤。

五月壬辰,帝崩于未央宮。

毀太上皇、孝惠、孝景皇帝廟。罷孝文、孝昭太后、昭靈后、武哀王、昭哀后寢園。

秋七月丙戌,葬渭陵。

贊曰:臣外祖兄弟為元帝侍中,語臣曰元帝多材藝,善史書。鼓琴瑟,吹洞簫,自度曲,被歌聲,分刌節度,窮極幼眇。少而好儒,及即位,徵用儒生,委之以政,貢、薛、韋、匡迭為宰相。而上牽制文義,優游不斷,孝宣之業衰焉。然寬弘盡下,出於恭儉,號令溫雅,有古之風烈。


\end{pinyinscope}