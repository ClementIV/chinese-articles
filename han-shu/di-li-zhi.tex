\article{地理志}

\begin{pinyinscope}
昔在黃帝,作舟車以濟不通,旁行天下,方制萬里,畫野分州,得百里之國萬區。是故易稱「先王以建萬國,親諸侯」,《書》云「協和萬國」,此之謂也。堯遭洪水,褱山襄陵,天下分絕,為十二州,使禹治之。水土既平,更制九州,列五服,任土作貢。

曰:禹敷土,隨山赞木,奠高山大川。

冀州既載,壺口治梁及岐。既脩太原,至于嶽陽。覃懷底績,至于衡章。厥土惟白壤。厥賦上上錯,厥田中中。恆、衛既從,大陸既作。鳥夷皮服。夾右碣石,入于河。

泲、河惟兗州。九河既道,雷夏既澤,雍、沮會同,桑土既蠶,是降丘宅土。厥土黑墳,屮繇木條。厥田中下,賦貞,作十有三年乃同。厥貢漆絲,厥棐織文。浮于泲、漯,通于河。

海、岱惟青州。嵎夷既略,惟、甾其道。厥土白墳,海瀕廣潟。田上下,賦中上。貢鹽、絺,海物惟錯,岱畎絲、枲、鈆、松、怪石,萊夷作牧,厥棐览絲。浮于汶,達于泲。

海、岱及淮惟徐州。淮、沂其乂,蒙、羽其藝。大野既豬,東原厎平。厥土赤埴墳,草木漸包。田上中,賦中中。貢土五色,羽畎夏狄,嶧陽孤桐,泗瀕浮磬,淮夷蠙珠臮魚,厥棐玄纖縞。浮于淮、泗,達于河。

淮、海惟揚州。彭蠡既豬,陽鳥逌居。三江既入,震澤厎定。篠簜既敷,魇夭木喬。厥土塗泥。田下下,賦下上錯。貢金三品,瑤、鲥、篠簜,齒、革、羽毛,鳥夷卉服,厥棐織貝,厥包橘、柚,錫貢。均江海,通于淮、泗。

荊及衡陽惟荊州。江、漢朝宗于海。九江孔殷,沱、灊既道,雲夢,土作乂。厥土塗泥。田下中,賦上下。貢羽旄、齒、革,金三品,杶、幹、栝、柏,厲、砥、砮、丹,惟箘廐、楛,三國厎貢厥名,包匭菁茅,厥棐玄纁璣組,九江納錫大龜。浮于江、沱、灊、漢,逾于洛,至于南河。

荊、河惟豫州。伊、雒、瀍、澗既入于河,滎、波既豬,道荷澤,被盟豬。厥土惟壤,下土墳壚。田中上,賦錯上中。貢漆、枲、絺、紵、棐纖纊,錫貢磬錯。浮于洛,入于河。

華陽、黑水惟梁州。岷、嶓既藝,沱、灊既道,蔡、蒙旅平,和夷厎績。厥土青黎。田下上,賦下中三錯。貢璆、鐵、銀、鏤、砮、磬,熊、羆、狐、貍、織皮。西頃因桓是來,浮于灊,逾于沔,入于渭,亂于河。

黑水、西河惟雍州。弱水既西,涇屬渭汭。漆、沮既從,酆水逌同。荊、岐既旅,終南、惇物,至于鳥鼠。原隰厎績,至于豬野。三危既宅,三苗丕敘。厥土黃壤。田上上,賦中下。貢球、琳、琅玕。浮于積石,至于龍門西河,會于渭汭。織皮昆崙、析支、渠叟,西戎即敘。

道钐及岐,至于荊山,逾于河;壺口、雷首,至于大嶽;厎柱、析城,至于王屋;太行、恆山,至于碣石,入于海。西傾、朱圉、鳥鼠,至于太華;熊耳、外方、桐柏,至于倍尾。道嶓冢,至于荊山;內方,至于大別;崏山之陽,至于衡山,過九江,至于敷淺原。

道弱水,至于合藜,餘波入于流沙。道黑水,至于三危,入于南海。道河積石,至于龍門,南至于華陰,東至于厎柱,又東至于盟津,東過洛汭,至于大伾,北過降水,至于大陸,又北播為九河,同為逆河,入于海。嶓冢道漾,東流為漢,又東為滄浪之水,過三澨,至于大別,南入于江,東匯澤為彭蠡,東為北江,入于海。崏山道江,東別為沱,又東至于醴,過九江,至于東陵,東迆北會于匯,東為中江,入于海。道沇水,東流為泲,入于河,軼為滎,東出于陶丘北,又東至于荷,又東北會於汶,又北東入于海。道淮自桐柏,東會于泗、沂,東入于海。道渭自鳥鼠同穴,東會于酆,又東至于涇,又東過漆、沮,入于河。道洛自熊耳,東北會于澗、瀍,又東會于伊,又東北入于河。

九州逌同,四奧既宅,九山刊旅,九川滌原,九澤既陂,四海會同。六府孔修,庶土交正,厎慎財賦,咸則三壤,成賦中國。錫土姓:「祗台德先,不距朕行。」

五百里甸服:百里賦內總,二百里納銍,三百里內戛服,四百里粟,五百里米。五百里侯服:百里采,二百里男國,三百里諸侯。五百里綏服:三百里揆文教,二百里奮武衛。五百里要服:三百里夷,二百里蔡。五百里荒服:三百里蠻,二百里流。東漸于海,西被于流沙,朔、南臮,聲教訖于四海。

禹錫玄圭,告厥成功。

後受禪於虞,為夏后氏。

殷因於夏,亡所變改。周既克殷,監於二代而損益之,定官分職,改禹徐、梁二州合之於雍、青,分冀州之地以為幽、并。故周官有職方氏,掌天下之地,辯九州之國。

東南曰揚州:其山曰會稽,藪曰具區,川曰三江,寖曰五湖;其利金、錫、竹箭;民二男五女;畜宜鳥獸,穀宜稻。

正南曰荊州:其山曰衡,藪曰雲夢,川曰江、漢,寖曰潁、湛;其利丹、銀、齒、革;民一男二女;畜及穀宜,與揚州同。

河南曰豫州:其山曰華,藪曰圃田,川曰滎、雒,簧曰波、溠;其利林、漆、絲枲;民二男三女;畜宜六擾,其穀宜五種。

正東曰青州:其山曰沂,藪曰孟諸,川曰淮、泗,寖曰沂、沭;其利蒲、魚;民二男三女;其畜宜雞、狗,穀宜稻、麥。

河東曰兗州:其山曰岱,藪曰泰野,其川曰河、泲,浸曰盧、濰;其利蒲、魚;民二男三女;其畜宜六擾,穀宜四種。

正西曰雍州:其山曰嶽,藪曰弦蒲,川曰涇、汭,其浸曰渭、洛;其利玉、石;其民三男二女;畜宜牛、馬,穀宜黍、稷。

東北曰幽州:其山曰醫無閭,藪曰豯養,川曰河、泲,浸曰菑、時;其利魚、鹽;民一男三女;畜宜四擾,穀宜三種。

河內曰冀州:其山曰霍,藪曰揚紆,川曰漳,浸曰汾、潞;其利松、柏;民五男三女;畜宜牛、羊,穀宜黍、稷。

正北曰并州:其山曰恆山,藪曰昭餘祁,川曰虖池、嘔夷,浸曰淶、易;其利布帛;民二男三女;畜宜五擾,穀宜五種。

而保章氏掌天文,以星土辯九州之地,所封封域皆有分星,以視吉凶。

周爵五等,而土三等:公、侯百里,伯七十里,子、男五十里。不滿為附庸,蓋千八百國。而太昊、黃帝之後,唐、虞侯伯猶存,帝王圖籍相踵而可知。周室既衰,禮樂征伐自諸侯出,轉相吞滅,數百年間,列國秏盡。至春秋時,尚有數十國,五伯迭興,總其盟會。陵夷至於戰國,天下分而為七,合從連衡,經數十年。秦遂并兼四海。以為周制微弱,終為諸侯所喪,故不立尺土之封,分天下為郡縣,盪滅前聖之苗裔,靡有孑遺者矣。

漢興,因秦制度,崇恩德,行簡易,以撫海內。至武帝攘卻胡、越,開地斥境,南置交阯,北置朔方之州,兼徐、梁、幽、并夏、周之制,改雍曰涼,改梁曰益,凡十三郡,置刺史。先王之跡既遠,地名又數改易,是以采獲舊聞,考跡詩書,推表山川,以綴禹貢、周官、春秋,下及戰國、秦、漢焉。

京兆尹,元始二年戶十九萬五千七百二,口六十八萬二千四百六十八。縣十二:長安,新豐,船司空,藍田,華陰,鄭,湖,下邽,南陵,奉明,霸陵,杜陵。

左馮翊,戶二十三萬五千一百一,口九十一萬七千八百二十二。縣二十四:高陵,櫟陽,翟道,池陽,夏陽,衙,粟邑,谷口,

蓮勺,頻陽,臨晉,重泉,郃陽,祋祤,武城,沈陽,褱德,徵,

雲陵,萬年,長陵,陽陵,雲陽。

右扶風,戶二十一萬六千三百七十七,口八十三萬六千七十。縣二十一:渭城,槐里,鄠,盩厔,斄,郁夷,美陽,郿,雍,漆,栒邑,隃麋,陳倉,杜陽,汧,好畤,

虢,安陵,茂陵,平陵,武功,

弘農郡,戶十一萬八千九十一,口四十七萬五千九百五十四。縣十一:弘農,盧氏,陝,宜陽,黽池,丹水,新安,商,析,陸渾,上雒。

河東郡,戶二十三萬六千八百九十六,口九十六萬二千九百一十二。縣二十四:安邑,大陽,猗氏,解,蒲反,河北,左邑,汾陰,聞喜,濩澤,端氏,臨汾,垣,皮氏,長脩,平陽,襄陵,彘,楊,北屈,蒲子,絳,狐讘,騏。

太原郡,戶十六萬九千八百六十三,口六十八萬四百八十八。

縣二十一:晉陽,葰人,界休,榆次,中都,于離,茲氏,狼孟,鄔,盂,平陶,汾陽,京陵,陽曲,大陵,原平,祁,上艾,慮虒,陽邑,廣武。

上黨郡,戶七萬三千七百九十八,口三十三萬七千七百六十六。縣十四:長子,屯留,余吾,銅鞮,沾,

涅氏,襄垣,壺關,泫氏,高都,潞,陭氏,陽阿,穀遠。

河內郡,戶二十四萬一千二百四十六,口百六萬七千九十七。縣十八:懷,汲,武德,波,山陽,河陽,州,共,平皋,朝歌,脩武,溫,野王,獲嘉,軹,沁水,隆慮,蕩陰。

河南郡,戶二十七萬六千四百四十四,口一百七十四萬二百七十九。縣二十二:雒陽,滎陽,偃師,京,平陰,中牟,平,陽武,河南,緱氏,卷,原武,鞏,穀成,故市,密,

新成,開封,成皋,苑陵。梁,新鄭。

東郡,戶四十萬一千二百九十七,口百六十五萬九千二十八。縣二十二:濮陽,畔觀,聊城,頓丘,發干,范,茬平,東武陽,

博平,黎,清,東阿,離狐,臨邑,利苗,須昌,壽良,樂昌,陽平,白馬,南燕,廩丘。

陳留郡,戶二十九萬六千二百八十四,口一百五十萬九千五十。縣十七:陳留,小黃,成安,寧陵,雍丘,酸棗,東昏,襄邑,外黃,封丘,長羅,尉氏,傿,長垣,平丘,濟陽,浚儀。

潁川郡,戶四十三萬二千四百九十一,口二百二十一萬九百七十三。縣二十:陽翟,昆陽,潁陽,定陵,長社,新汲,襄城,郾,郟,舞陽,潁陰,崇高,許,傿陵,臨潁,父城,成安,周承休,陽城,綸氏。

汝南郡,戶四十六萬一千五百八十七,口二百五十九萬六千一百四十八。縣三十七:平輿,陽安,陽城,郦強,富波,女陽,鮦陽,吳房,安成,南頓,朗陵,細陽,宜春,女陰,新蔡,新息,灈陽,期思,慎陽,慎,召陵,弋陽,西平,上蔡,浸,西華,長平,宜祿,項,新郪,歸德,新陽,

安昌,安陽,

博陽,成陽,定陵。

南陽郡,戶三十五萬九千一百一十六,口一百九十四萬二千五十一。縣三十六:宛,犨,杜衍,酇,育陽,博山,涅陽,陰,堵陽,雉,山都,蔡陽,新野,筑陽,棘陽,武當,舞陰,西鄂,穰,酈,安眾,冠軍,比陽,平氏,

隨,葉,鄧,朝陽,魯陽,舂陵,新都,湖陽,紅陽,樂成,博望,復陽。

南郡,戶十二萬五千五百七十九,口七十一萬八千五百四十。縣十八:江陵,臨沮,夷陵,華容,宜城,郢,踬,當陽,中盧,枝江,襄陽,編,秭歸,夷道,州陵,若,巫,高成。

江夏郡,戶五萬六千八百四十四,口二十一萬九千二百一十八。縣十四:西陵,竟陵,西陽,襄,邾,軑,鄂,安陸,沙羨,蘄春,鄳,雲杜,下雉,鍾武。

廬江郡,戶十二萬四千三百八十三,口四十五萬七千三百三十三。縣十二:舒,居巢,龍舒,臨湖,雩婁,襄安,樅陽,尋陽,灊,睆,

湖陵邑,松茲。

九江郡,戶十五萬五十二,口七十八萬五百二十五。縣十五:壽春邑,浚遒,成德,橐皋,陰陵,歷陽,當塗,鍾離,合肥,東城,博鄉,曲陽,建陽,全椒,阜陵。

山陽郡,戶十七萬二千八百四十七,口八十萬一千二百八十八。縣二十三:昌邑,南平陽,成武,湖陵,東嬢,方與,橐,鉅野,單父,薄,都關,城都,黃,爰戚,郜成,中鄉,平樂,鄭,瑕丘,甾鄉,

栗鄉,曲鄉,

西陽。

濟陰郡,戶二十九萬二千五,口百三十八萬六千二百七十八。縣九:定陶,冤句,呂都,葭密,成陽,鄄城,句陽,秺,乘氏。

沛郡,戶四十萬九千七十九,口二百三萬四百八十。縣三十七:相,龍亢,竹,穀陽,蕭,向,

銍,廣戚,下蔡,豐,

鄲,譙,蘄,颛,輒與,山桑,公丘,符離,敬丘,夏丘,洨,沛,芒,建成,城父,建平,酇,栗,扶陽,高,高柴,漂陽,平阿,東鄉,臨都,義成,祈鄉。

魏郡,戶二十一萬二千八百四十九,口九十萬九千六百五十五。縣十八:鄴,館陶,斥丘,

沙,內黃,清淵,魏,繁陽,元城,梁期,黎陽,即裴,武始,邯會,陰安,平恩,邯溝,武安。

鉅鹿郡,戶十五萬五千九百五十一,口八十二萬七千一百七十七。縣二十:鉅鹿,南讀,廣阿,象氏,廮陶,宋子,楊氏,臨平,下曲陽,貰,郻,

新巿,堂陽,

安定,敬武,歷鄉,

樂信,武陶,柏鄉,安鄉。

常山郡,戶十四萬一千七百四十一,口六十七萬七千九百五十六。縣十八:元氏,石邑,桑中,靈壽,蒲吾,上曲陽,九門,井陘,房子,中丘,封斯,關,平棘,鄗,樂陽,平臺,都鄉,南行唐。

清河郡,戶二十萬一千七百七十四,口八十七萬五千四百二十二。縣十四:清陽,東武城,繹幕,靈,厝,鄃,貝丘,信成,衬題,東陽,信鄉,繚,棗彊,復陽。

涿郡,戶十九萬五千六百七,口七十八萬二千七百六十四。縣二十九:涿,

迺,穀丘,故安,南深澤,范陽,蠡吾,容城,易,廣望,鄚,高陽,州鄉,安平,樊輿,

成,良鄉,利鄉,臨鄉,益昌,陽鄉,西鄉,饒陽,中水,武垣,阿陵,阿武,高郭,

新昌。

勃海郡,戶二十五萬六千三百七十七,口九十萬五千一百一十九。縣二十六:浮陽,陽信,東光,阜城,千童,重合,南皮,定,章武,中邑,高成,高樂,

參戶,成平,柳,臨樂,東平舒,重平,安次,脩市,文安,景成,束州,建成,章鄉,蒲領。

平原郡,戶十五萬四千三百八十七,口六十六萬四千五百四十三。縣十九:平原,鬲,高唐,重丘,平昌,羽,般,樂陵,祝阿,瑗,阿陽,漯陰,朸,富平,安德,合陽,樓虛,龍哣,安。

千乘郡,戶十一萬六千七百二十七,口四十九萬七百二十。縣十五:千乘,東鄒,溼沃,平安,博昌,蓼城,建信,狄,琅槐,樂安,被陽,高昌,繁安,高宛,延鄉。

濟南郡,戶十四萬七百六十一,口六十四萬二千八百八十四。縣十四:東平陵,鄒平,臺,梁鄒,土鼓,於陵,陽丘,般陽,菅,朝陽,歷城,猇,著,宜成。

泰山郡,戶十七萬二千八十六,口七十二萬六千六百四。縣二十四:奉高,博,茬,盧,肥成,蛇丘,剛,柴,蓋,梁父,東平陽,南武陽,萊蕪,鉅平,嬴,牟,蒙陰,華,寧陽,乘丘,富陽,桃山,桃鄉,式。

齊郡,戶十五萬四千八百二十六,口五十五萬四千四百四十四。縣十二:臨淄,昌國,利,西安,鉅定,廣,廣饒,昭南,臨朐,北鄉,平廣,臺鄉。

北海郡,戶十二萬七千,口五十九萬三千一百五十九。縣二十六:營陵,劇魁,安丘,

瓡,淳于,益,平壽,劇,都昌,平望,平的,柳泉,

壽光,樂望,饒,斟,桑犢,平城,密鄉,羊石,樂都,石鄉,上鄉,新成,成鄉,膠陽。

東萊郡,戶十萬三千二百九十二,口五十萬二千六百九十三。縣十七:掖,腄,

平度,黃,臨朐,曲成,牟平,東牟,脏,育犁,昌陽,不夜,當利,盧鄉,陽樂,陽石,徐鄉。

琅邪郡,戶二十二萬八千九百六十,口一百七萬九千一百。縣五十一:東武,不其,海曲,贛榆,朱虛,諸,梧成,靈門,姑幕,虛水,臨原,琅邪,祓,柜,缾,邞,雩段,黔陬,雲,計斤,稻,皋虞,平昌,長廣,橫,東莞,魏其,昌,茲鄉,箕,

椑,高廣,高鄉,柔,即來,麗,武鄉,伊鄉,新山,高陽,昆山,參封,折泉,博石,房山,

慎鄉,駟望,安丘,高陵,臨安,石山。

東海郡,戶三十五萬八千四百一十四,口百五十五萬九千三百五十七。縣三十八:郯,蘭陵,襄賁,下邳,良成,平曲,戚,朐,開陽,費,利成,海曲,蘭祺,繒,南成,山鄉,建鄉,即丘,祝其,臨沂,厚丘,容丘,東安,合鄉,承,建陽,曲陽,司吾,于鄉,平曲,都陽,陰平,郚鄉,武陽,新陽,建陵,昌慮,都平。

臨淮郡,戶二十六萬八千二百八十三,口百二十三萬七千七百六十四。縣二十九:徐,取慮,淮浦,盱眙,厹猶,僮,射陽,開陽,贅其,高山,睢陵,鹽瀆,淮陰,淮陵,下相,富陵,

東陽,播旌,西平,

高平,開陵,昌陽,廣平,蘭陽,襄平,海陵,輿,堂邑,樂陵。

會稽郡,戶二十二萬三千三十八,口百三萬二千六百四。縣二十六:吳,曲阿,烏傷,毗陵,餘暨,陽羨,諸暨,無錫,山陰,丹徒,餘姚,婁,上虞,海鹽,剡,由拳,大末,烏程,句章,餘杭,鄞,錢唐,鄮,富春,冶,回浦。

丹揚郡,戶十萬七千五百四十一,口四十萬五千一百七十一。縣十七:宛陵,於骄,江乘,春穀,秣陵,

故鄣,句容,涇,丹陽,石城,胡孰,陵陽,蕪湖,黝,溧陽,歙,宣城。

豫章郡,戶六萬七千四百六十二,口三十五萬一千九百六十五。縣十八:南昌,廬陵,彭澤,鄱陽,歷陵,餘汗,柴桑,

艾,贛,新淦,南城,建成,宜春,海昏,雩都,鄡陽,南野,安平,

桂陽郡,戶二萬八千一百一十九,口十五萬六千四百八十八。縣十一:郴,

臨武,

便,南平,耒陽,桂陽,

陽山,曲江,

含洭,湞陽,

陰山。

武陵郡,戶三萬四千一百七十七,口十八萬五千七百五十八。縣十三:索,孱陵,臨沅,沅陵,鐔成,無陽,遷陵,辰陽,酉陽,義陵,佷山,零陽,充。

零陵郡,戶二萬一千九十二,口十三萬九千三百七十八。縣十:零陵,營道,始安,夫夷,營浦,都梁,泠道,泉陵,洮陽,鍾武。

漢中郡,戶十萬一千五百七十,口三十萬六百一十四。縣十二:西城,旬陽,南鄭,褒中,房陵,

安陽,成固,沔陽,鍚,武陵,上庸,長利。

廣漢郡,戶十六萬七千四百九十九,口六十六萬二千二百四十九。縣十三:梓潼,汁方,涪,雒,綿竹,廣漢,葭明,郪,新都,甸氐道,白水,剛氐道,陰平道。

蜀郡,戶二十六萬八千二百七十九,口百二十四萬五千九百二十九。縣十五:成都,郫,繁,廣都,臨邛,青衣,江原,嚴道,

綿虒,旄牛,徙,湔氐道,汶江,廣柔,蠶陵。

犍為郡,戶十萬九千四百一十九,口四十八萬九千四百八十六。縣十二:僰道,江陽,武陽,南安,資中,符,牛鞞,南廣,漢陽,瘾癣,朱提,堂琅。

越嶲郡,戶六萬一千二百八,口四十萬八千四百五。縣十五:邛都,遂久,靈關道,臺登,定莋,會無,莋秦,大莋,姑復,三絳,蘇示,闌,卑水,绀街,青蛉。

益州郡,戶八萬一千九百四十六,口五十八萬四百六十三。縣二十四:滇池,雙柏,同勞,銅瀨,連然,俞元,

收靡,穀昌,秦臧,邪龍,味,昆澤,葉榆,律高,不韋,雲南,嶲唐,弄棟,比蘇,賁古,毋棳,勝休,健伶,來唯。

牂柯郡,戶二萬四千二百一十九,口十五萬三千三百六十。縣十七:故且蘭,鐔封,鄨,漏臥,平夷,同並,談指,宛溫,毋斂,夜郎,毋單,漏江,西隨,都夢,談稿,進桑,句町。

巴郡,戶十五萬八千六百四十三,口七十萬八千一百四十八。縣十一:江州,臨江,枳,閬中,墊江,朐忍,安漢,宕渠,魚復,充國,涪陵。

武都郡,戶五萬一千三百七十六,口二十三萬五千五百六十。縣九:武都,上祿,故道,河池,平樂道,沮,嘉陵道,循成道,下辨道。

隴西郡,戶五萬三千九百六十四,口二十三萬六千八百二十四。縣十一:狄道,上邽,安故,氐道,首陽,予道,大夏,羌道,襄武,臨洮,西。

金城郡,戶三萬八千四百七十,口十四萬九千六百四十八。縣十三:允吾,浩亹,令居,枝陽,金城,榆中,枹罕,白石,河關,破羌,安夷,允街,臨羌。

天水郡,戶六萬三百七十,口二十六萬一千三百四十八。縣十六:平襄,街泉,戎邑道,望垣,罕幵,綿諸道,阿陽,略陽道,冀,勇士,

成紀,清水,奉捷,隴,豲道,蘭干。

武威郡,戶萬七千五百八十一,口七萬六千四百一十九。縣十:姑臧,張掖,武威,休屠,揟次,鸞鳥,撲峦,媼圍,蒼巅,宣威。

張掖郡,戶二萬四千三百五十二,口八萬八千七百三十一。縣十:觻得,昭武,刪丹,氐池,屋蘭,曰勒,驪靬,番和,居延,顯美。

酒泉郡,戶萬八千一百三十七,口七萬六千七百二十六。縣九:祿福,表是,樂涫,天孪,玉門,會水,池頭,綏彌,乾齊。

敦煌郡,戶萬一千二百,口三萬八千三百三十五。縣六:敦煌,冥安,效穀,淵泉,廣至,龍勒。

安定郡,戶四萬二千七百二十五,口十四萬三千二百九十四。縣二十一:高平,復累,安俾,撫夷,朝那,

涇陽,臨涇,鹵,烏氏,陰密,安定,參讀,三水,陰槃,安武,祖厲,爰得,眴卷,彭陽,鶉陰,月支道。

北地郡,戶六萬四千四百六十一,口二十一萬六百八十八。縣十九:馬領,直路,靈武,富平,靈州,昫衍,方渠,除道,五街,鶉孤,歸德,回獲,略畔道,泥陽,郁郅,義渠道,弋居,大呓,廉。

上郡,戶十萬三千六百八十三,口六十萬六千六百五十八。縣二十三:膚施,獨樂,陽周,木禾,平都,淺水,京室,洛都,白土,襄洛,原都,漆垣,奢延,雕陰,推邪,楨林,高望,雕陰道,龜茲,定陽,高奴,望松,宜都。

西河郡,戶十三萬六千三百九十,口六十九萬八千八百三十六。縣三十六:富昌,騶虞,鵠澤,平定,美稷,中陽,樂街,徒經,皋狼,大成,廣田,圜陰,益闌,平周,鴻門,藺,宣武,千章,增山,圜陽,廣衍,武車,虎猛,離石,穀羅,饒,方利,隰成,臨水,土軍,西都,平陸,陰山,觬是,博陵,鹽官。

朔方郡,戶三萬四千三百三十八,口十三萬六千六百二十八。縣十:三封,朔方,修都,臨河,

呼遒,窳渾,

渠搜,沃野,廣牧,臨戎。

五原郡,戶三萬九千三百二十二,口二十三萬一千三百二十八。縣十六:九原,固陵,五原,臨沃,文國,河陰,蒱澤,南興,武都,宜梁,曼柏,成宜,稒陽,莫庞,西安陽,河目。

雲中郡,戶三萬八千三百三,口十七萬三千二百七十。縣十一:雲中,咸陽,陶林,楨陵,犢和,沙陵,原陽,沙南,北輿,武泉,陽壽。

定襄郡,戶三萬八千五百五十九,口十六萬三千一百四十四。縣一十二:成樂,桐過,都武,武進,襄陰,武皋,駱,定陶,武城,武要,定襄,復陸。莽曰聞武。

鴈門郡,戶七萬三千一百三十八,口二十九萬三千四百五十四。縣十四:善無,沃陽,繁畤,中陵,陰館,樓煩,武州,鬓陶,劇陽,崞,平城,埒,馬邑,

彊陰。

代郡,戶五萬六千七百七十一,口二十七萬八千七百五十四。縣十八:桑乾,道人,當城,高柳,馬城,班氏,延陵,狋氏,且如,平邑,陽原,東安陽,參合,平舒,代,靈丘,廣昌,鹵城。

上谷郡,戶三萬六千八,口十一萬七千七百六十二。縣十五:沮陽,

泉上,潘,軍都,居庸,雊瞀,夷輿,寧,昌平,廣寧,涿鹿,且居,茹,女祈,下落。

漁陽郡,戶六萬八千八百二,口二十六萬四千一百一十六。縣十二:漁陽,狐奴,路,雍奴,泉州,平谷,安樂,厗奚,獷平,要陽,白檀,滑鹽。

右北平郡,戶六萬六千六百八十九,口三十二萬七百八十。縣十六:平剛,無終,石成,廷陵,俊靡,薋,徐無,字,土垠,白狼,夕陽,昌城,驪成,廣成,聚陽,平明。

遼西郡,戶七萬二千六百五十四,口三十五萬二千三百二十五。縣十四:且慮,海陽,新安平,柳城,令支,肥如,賓從,交黎,陽樂,狐蘇,徒河,文成,臨渝,絫。

遼東郡,戶五萬五千九百七十二,口二十七萬二千五百三十九。縣十八:襄平,新昌,無慮,

望平,房,候城,遼隊,遼陽,險瀆,居就,高顯,安市,武次,平郭,西安平,文,番汗,沓氏。

玄菟郡,戶四萬五千六,口二十二萬一千八百四十五。縣三:高句驪,上殷台,西蓋馬。

樂浪郡,戶六萬二千八百一十二,口四十萬六千七百四十八。縣二十五:朝鮮,俨邯,浿水,含資,黏蟬,遂成,增地,帶方,駟望,海冥,列口,長岑,屯有,昭明,鏤方,提奚,渾彌,吞列,東傥,不而,蠶台,華麗,邪頭昧,前莫,夫租。

南海郡,戶萬九千六百一十三,口九萬四千二百五十三。縣六:番禺,博羅,中宿,龍川,四會,揭陽。

鬱林郡,戶萬二千四百一十五,口七萬一千一百六十二。縣十二:布山,安廣,阿林,廣鬱,中留,桂林,潭中,臨塵,定周,增食,

領方。雍雞。

蒼梧郡,戶二萬四千三百七十九,口十四萬六千一百六十。縣十:廣信,謝沐,高要,封陽,臨賀,端谿,馮乘,富川,荔蒲,猛陵。

交趾郡,戶九萬二千四百四十,口七十四萬六千二百三七。縣十:羸啮,安定,苟龈,麊泠,曲昜,北帶,稽徐,西于,龍編,朱觏。

合浦郡,戶萬五千三百九十八,口七萬八千九百八十。縣五:徐聞,高涼,合浦,臨允,朱盧。

九真郡,戶三萬五千七百四十三,口十六萬六千一十三。縣七:胥浦,居風,都龐,餘發,咸驩,無切,無編。

日南郡,戶萬五千四百六十,口六萬九千四百八十五。縣五:朱吾,比景,盧容,西捲,象林。

趙國,戶八萬四千二百二,口三十四萬九千九百五十二。縣四:邯鄲,易陽,柏人,襄國。

廣平國,戶二萬七千九百八十四,口十九萬八千五百五十八。縣十六:廣平,張,朝平,南和,列人,斥章,任,曲周,南曲,曲梁,廣鄉,平利,平鄉,陽臺,廣年,城鄉。

真定國,戶三萬七千一百二十六,口十七萬八千六百一十六。縣四:真定,稿城,肥纍,綿曼。

中山國,戶十六萬八百七十三,口六十六萬八千八十。縣十四:盧奴,北平,

北新成,唐,深澤,苦陘,安國,曲逆,望都,新市,新處,毋極,陸成,安險。

信都國,戶六萬五千五百五十六,口三十萬四千三百八十四。縣十七:信都,歷,扶柳,辟陽,南宮,下博,

武邑,觀津,高隄,廣川,樂鄉,平隄,桃,西梁,昌成,東昌,脩。

河間國,戶四萬五千四十三,口十八萬七千六百六十二。縣四:樂成,候井,武隧,弓高。

廣陽國,戶二萬七百四十,口七萬六百五十八。縣四:薊,方城,廣陽,陰鄉。

甾川國,戶五萬二百八十九,口二十二萬七千三十一。縣三:劇,東安平,樓鄉。

廣陽國,戶二萬七百四

膠東國,戶七萬二千二,口三十二萬三千三百三十一。縣八:即墨,昌武,下密,壯武,郁秩,挺,觀陽,鄒盧。

高密國,戶四萬五百三十一,口十九萬二千五百三十六。縣五:高密,昌安,石泉,夷安,

成鄉。

城陽國,戶五萬六千六百四十二,口二十萬五千七百八十四。縣四:莒,陽都,東安,慮。

淮陽國,戶十三萬五千五百四十四,口九十八萬一千四百二十三。縣九:陳,苦,陽夏,寧平,扶溝,

固始,圉,新平,柘。

梁國,戶三萬八千七百九,口十萬六千七百五十二。縣八:碭,甾,杼秋,蒙,已氏,虞,下邑,睢陽。

東平國,戶十三萬一千七百五十三,口六十萬七千九百七十六。縣七:無鹽,任城,東平陸,富城,

章,亢父,樊。

魯國,戶十一萬八千四十五,口六十萬七千三百八十一。縣六:魯,卞,汶陽,

蕃,騶,薛。

楚國,戶十一萬四千七百三十八,口四十九萬七千八百四。縣七:彭城,留,梧,傅陽,呂,武原,甾丘。

泗水國,戶二萬五千二十五,口十一萬九千一百一十四。縣三:淩,泗陽,于。

廣陵國,戶三萬六千七百七十三,口十四萬七百二十二。縣四:廣陵,江都,高郵,平安。

六安國,戶三萬八千三百四十五,口十七萬八千六百一十六。縣五:六,蓼,安豐,安風,陽泉。

長沙國,戶四萬三千四百七十,口二十三萬五千八百二十五。縣十三:臨湘,羅,連道,益陽,下雋,收,酃,承陽,湘南,昭陵,荼陵,容陵,安成。

本秦京師為內史,分天下作三十六郡。漢興,以其郡大大,稍復開置,又立諸侯王國。武帝開廣三邊。故自高祖增二十六,文、景各六,武帝二十八,昭帝一,訖於孝平,凡郡國一百三,縣邑千三百一十四,道三十二,侯國二百四十一。地東西九千三百二里,南北萬三千三百六十八里。提封田一萬萬四千五百一十三萬六千四百五頃,其一萬萬二百五十二萬八千八百八十九頃,邑居道路,山川林澤,群不可墾,其三千二百二十九萬九百四十七頃,可墾不可墾,定墾田八百二十七萬五百三十六頃。民戶千二百二十三萬三千六十二,口五千九百五十九萬四千九百七十八。漢極盛矣。

凡民函五常之性,而其剛柔緩急,音聲不同,繫水土之風氣,故謂之風;好惡取舍,動靜亡常,隨君上之情欲,故謂之俗。孔子曰:「移風易俗,莫善於樂。」言聖王在上,統理人倫,必移其本,而易其末,此混同天下一之虖中和,然後王教成也。漢承百年之末,國土變改,民人遷徙,成帝時劉向略言其域分,丞相張禹使屬潁川朱贛條其風俗,猶未宣究,故輯而論之,終其本末著於篇。

秦地,於天官東井、輿鬼之分野也。其界自弘農故關以西,京兆、扶風、馮翊、北地、上郡、西河、安定、天水、隴西,南有巴、蜀、廣漢、犍為、武都,西有金城、武威、張掖、酒泉、敦煌,又西南有牂柯、越巂、益州,皆宜屬焉。

秦之先曰柏益,出自帝顓頊,堯時助禹治水,為舜朕虞,養育草木鳥獸,賜姓嬴氏,歷夏、殷為諸侯。至周有造父,善馭習馬,得華騮、綠耳之乘,幸於穆王,封於趙城,故更為趙氏。後有非子,為周孝王養馬汧、渭之間。孝王曰:「昔伯益知禽獸,子孫不絕。」乃封為附庸,邑之於秦,今隴西秦亭秦谷是也。至玄孫,氏為莊公,破西戎,有其地。子襄公時,幽王為犬戎所敗,平王東遷雒邑。襄公將兵救周有功,賜受廄、酆之地,列為諸侯。後八世,穆公稱伯,以河為竟。十餘世,孝公用商君,制轅田,開仟伯,東雄諸侯。子惠公初稱王,得上郡、西河。孫昭王開巴蜀,滅周,取九鼎。昭王曾孫政并六國,稱皇帝,負力怙威,燔書阬儒,自任私智。至子胡亥,天下畔之。

天水、隴西,山多林木,民以板為室屋。及安定、北地、上郡、西河,皆迫近戎狄,修習戰備,高上氣力,以射獵為先。故《秦詩》曰「在其板屋」;又曰「王于興師,修我甲兵,與子偕行」。及車轔、四臷、小戎之篇,皆言車馬田狩之事。漢興,六郡良家子選給羽林、期門,以材力為官,名將多出焉。孔子曰:「君子有勇而亡誼則為亂,小人有勇而亡誼則為盜。」故此數郡,民俗質木,不恥寇盜。

故秦地於禹貢時跨雍、梁二州,詩風兼秦、豳兩國。昔后稷封鹅,公劉處豳,大王徙廄,文王作酆,武王治鎬,其民有先王遺風,好稼穡,務本業,故豳詩言農桑衣食之本甚備。有鄠、杜竹林,南山檀柘,號稱陸海,為九州膏腴。始皇之初,鄭國穿渠,引涇水溉田,沃野千里,民以富饒。漢興,立都長安,徙齊諸田,楚昭、屈、景及諸功臣家於長陵。後世世徙吏二千石、高訾富人及豪桀并兼之家於諸陵。蓋亦以彊幹弱支,非獨為奉山園也。是故五方雜厝,風俗不純。其世家則好禮文,富人則商賈為利,豪桀則游俠通姦。瀕南山,近夏陽,多阻險輕薄,易為盜賊,常為天下劇。又郡國輻湊,浮食者多,民去本就末,列侯貴人車服僭上,眾庶放效,羞不相及,嫁娶尤崇侈靡,送死過度。

自武威以西,本匈奴昆邪王、休屠王地,武帝時攘之,初置四郡,以通西域,鬲絕南羌、匈奴。其民或以關東下貧,或以報怨過當,或以誖逆亡道,家屬徙焉。習俗頗殊,地廣民稀,水屮宜畜牧,古涼州之畜為天下饒。保邊塞,二千石治之,咸以兵馬為務;酒禮之會,上下通焉,吏民相親。是以其俗風雨時節,穀糴常賤,少盜賊,有和氣之應,賢於內郡。此政寬厚,吏不苛刻之所致也。

巴、蜀、廣漢本南夷,秦并以為郡,土地肥美,有江水沃野,山林竹木疏食果實之饒。南賈滇、僰僮,西近邛、莋馬旄牛。民食稻魚,亡凶年憂,俗不愁苦,而輕易淫泆,柔弱褊阨。景、武間,文翁為蜀守,教民讀書法令,未能篤信道德,反以好文刺譏,貴慕權勢。及司馬相如游宦京師諸侯,以文辭顯於世,鄉黨慕循其跡。後有王褒、嚴遵、揚雄之徒,文章冠天下。繇文翁倡其教,相如為之師,故孔子曰:「有教亡類。」

武都地雜氐、羌,及犍為、牂柯、越巂,皆西南外夷,武帝初開置。民俗略與巴、蜀同,而武都近天水,俗頗似焉。

故秦地天下三分之一,而人眾不過什三,然量其富居什六。秦豳吳札觀樂,為之歌秦,曰:「此之謂夏聲。夫能夏則大,大之至也,其周舊乎?」

自井十度至柳三度,謂之鶉首之次,秦之分也。

魏地,觜觿、參之分野也。其界自高陵以東,盡河東、河內,南有陳留及汝南之召陵、郦彊、新汲、西華、長平,潁川之舞陽、郾、許、傿陵,河南之開封、中牟、陽武、酸棗、卷,皆魏分也。

河內本殷之舊都,周既滅殷,分其畿內為三國,詩風邶、庸、衛國是也。鄁,以封紂子武庚;庸,管叔尹之;衛,蔡叔尹之:以監殷民,謂之三監。故《書序》曰「武王崩,三監畔」,周公誅之,盡以其地封弟康叔,號曰孟侯,以夾輔周室;遷邶、庸之民于雒邑,故邶、庸、衛三國之詩相與同風。《邶詩》曰「在浚之下」,《庸》曰「在浚之郊」;《邶》又曰「亦流于淇」,「河水洋洋」,《庸》曰「送我淇上」,「在彼中河」,《衛》曰「瞻彼淇奧」,「河水洋洋」。故吳公子札聘魯觀周樂,聞邶、庸、衛之歌,曰:「美哉淵乎!吾聞康叔之德如是,是其衛風乎?」至十六世,懿公亡道,為狄所滅。齊桓公帥諸侯伐狄,而更封衛於河南曹、楚丘,是為文公。而河內殷虛,更屬于晉。康叔之風既歇,而紂之化猶存,故俗剛彊,多豪桀侵奪,薄恩禮,好生分。

河東土地平易,有鹽鐵之饒,本唐堯所居,詩風唐、魏之國也。周武王子唐叔在母未生,武王夢帝謂己曰:「余名而子曰虞,將與之唐,屬之參。」及生,名之曰虞。至成王滅唐,而封叔虞。唐有晉水,及叔虞子燮為晉侯云,故參為晉星。其民有先王遺教,君子深思,小人儉陋。故唐詩蟋蟀、山樞、葛生之篇曰「今我不樂,日月其邁」;「宛其死矣,它人是媮」;「百歲之後,歸于其居」。皆思奢儉之中,念死生之慮。吳札聞唐之歌,曰:「思深哉!其有陶唐氏之遺民乎?」

魏國,亦姬姓也,在晉之南河曲,故其《詩》曰「彼汾一曲」;「寘諸河之側」。自唐叔十六世至獻公,滅魏以封大夫畢萬,滅耿以封大夫趙夙,及大夫韓武子食采於韓原,晉於是始大。至於文公,伯諸侯,尊周室,始有河內之土。吳札聞魏之歌,曰:「美哉渢渢乎!以德輔此,則明主也。」文公後十六世為韓、魏、趙所滅,三家皆自立為諸侯,是為三晉。趙與秦同祖,韓、趙皆姬姓也。自畢萬後十世稱侯,至孫稱王,徙都大梁,故魏一號為梁,七世為秦所滅。

周地,柳、七星、張之分野也。今之河南雒陽、穀成、平陰、偃師、鞏、緱氏,是其分也。

昔周公營雒邑,以為在于土中,諸侯蕃屏四方,故立京師。至幽王淫褒姒,以滅宗周,子平王東居雒邑。其後五伯更帥諸侯以尊周室,故周於三代最為長久。八百餘年至於赧王,乃為秦所兼。初雒邑與宗周通封畿,東西長而南北短,短長相覆為千里。至襄王以河內賜晉文公,又為諸侯所侵,故其分墬小。

周人之失,巧偽趨利,貴財賤義,高富下貧,憙為商賈,不好仕宦。

自柳三度至張十二度,謂之鶉火之次,周之分也。

韓地,角、亢、氐之分野也。韓分晉得南陽郡及潁川之父城、定陵、襄城、潁陽、潁陰、長社、陽翟、郟,東接汝南,西接弘農得新安、宜陽,皆韓分也。及詩風陳、鄭之國,與韓同星分焉。

鄭國,今河南之新鄭,本高辛氏火正祝融之虛也。及成皋、滎陽,潁川之崇高、陽城,皆鄭分也。本周宣王弟友為周司徒,食采於宗周畿內,是為鄭。鄭桓公問於史伯曰:「王室多故,何所可以逃死?」史伯曰:「四方之國,非王母弟甥舅則夷狄,不可入也,其濟、洛、河、潁之間乎!子男之國,虢、會為大,恃勢與險,镯侈貪冒,君若寄帑與賄,周亂而敝,必將背君;君以成周之眾,奉辭伐罪,亡不克矣。」公曰:「南方不可乎?」對曰:「夫楚,重黎之後也,黎為高辛氏火正,昭顯天地,以生鲫嘉之材。姜、嬴、荊、羋,實與諸姬代相干也。姜,伯夷之後也;嬴,伯益之後也。伯夷能禮於神以佐堯,伯益能儀百物以佐舜,其後皆不失祠,而未有興者,周衰將起,不可偪也。」桓公從其言,乃東寄帑與賄,虢、會受之。後三年,幽王敗,威公死,其子武公與平王東遷,卒定虢、會之地,右雒左沛,食溱、洧焉。土骥而險,山居谷汲,男女亟聚會,故其俗淫。《鄭詩》曰:「出其東門,有女如雲。」又曰:「溱與洧方灌灌兮,士與女方秉菅兮。」「恂盱且樂,惟士與女,伊其相謔。」此其風也。吳札聞鄭之歌,曰:「美哉!其細已甚,民弗堪也。是其先亡乎?」自武公後二十三世,為韓所滅。

陳國,今淮陽之地。陳本太昊之虛,周武王封舜後媯滿於陳,是為胡公,妻以元女大姬。婦人尊貴,好祭祀,用史巫,故其俗巫鬼。《陳詩》曰:「坎其擊鼓,宛丘之下,亡冬亡夏,值其鷺羽。」又曰:「東門之枌,宛丘之栩,子仲之子,婆娑其下。」此其風也。吳札聞陳之歌,曰:「國亡主,其能久乎!」自胡公後二十三世為楚所滅。陳雖屬楚,於天文自若其故。

潁川、南陽,本夏禹之國。夏人上忠,其敝鄙朴。韓自武子後七世稱侯,六世稱王,五世而為秦所滅。秦既滅韓,徙天下不軌之民於南陽,故其俗夸奢,上氣力,好商賈漁獵,藏匿難制御也。宛,西通武關,東受江、淮,一都之會也。宣帝時,鄭弘、召信臣為南陽太守,治皆見紀。信臣勸民農桑,去末歸本,郡以殷富。潁川,韓都。士有申子、韓非,刻害餘烈,高士宦,好文法,民以貪遴爭訟生分為失。韓延壽為太守,先之以敬讓;黃霸繼之,教化大行,獄或八年亡重罪囚。南陽好商賈,召父富以本業;潁川好爭訟分異,黃、韓化以篤厚。「君子之德風也,小人之德草也」,信矣。

自東井六度至亢六度,謂之壽星之次,鄭之分野,與韓同分。

趙地,昴、畢之分野。趙分晉,得趙國。北有信都、真定、常山、中山,又得涿郡之高陽、鄚、州鄉;東有廣平、鉅鹿、清河、河間,又得渤海郡之東平舒、中邑、文安、束州、成平、章武,河以北也;南至浮水、繁陽、內黃、斥丘;西有太原、定襄、雲中、五原、上黨。上黨,本韓之別郡也,遠韓近趙,後卒降趙,皆趙分也。

自趙夙後九世稱侯,四世敬侯徙都邯鄲,至曾孫武靈王稱王,五世為秦所滅。

趙、中山地薄人眾,猶有沙丘紂淫亂餘民。丈夫相聚游戲,悲歌慷慨,起則椎剽掘冢,作姦巧,多弄物,為倡優。女子彈弦跕硔,游媚富貴,遍諸侯之後宮。

邯鄲北通燕、涿,南有鄭、衛,漳、河之間一都會也。其土廣俗雜,大率精急,高氣勢,輕為姦。

太原、上黨又多晉公族子孫,以詐力相傾,矜夸功名,報仇過直,嫁取送死奢靡。漢興,號為難治,常擇嚴猛之將,或任殺伐為威。父兄被誅,子弟怨憤,至告訐刺史二千石,或報殺其親屬。

鍾、代、石、北,迫近胡寇,民俗懻忮,好氣為姦,不事農商,自全晉時,已患其剽悍,而武靈王又益厲之。故冀州之部,盜賊常為它州劇。

定襄、雲中、五原,本戎狄地,頗有趙、齊、衛、楚之徙。其民鄙朴,少禮文,好射獵。雁門亦同俗,於天文別屬燕。

燕地,尾、箕分野也。武王定殷,封召公於燕,其後三十六世與六國俱稱王。東有漁陽、右北平、遼西,遼東,西有上谷、代郡、雁門,南得涿郡之易、容城、范陽、北新城、故安、涿縣、良鄉、新昌,及勃海之安次,皆燕分也。樂浪、玄菟,亦宜屬焉。

燕稱王十世,秦欲滅六國,燕王太子丹遣勇士荊軻西刺秦王,不成而誅,秦遂舉兵滅燕。

薊,南通齊、趙,勃、碣之間一都會也。初太子丹賓養勇士,不愛後宮美女,民化以為俗,至今猶然。賓客相過,以婦侍宿,嫁取之夕,男女無別,反以為榮。後稍頗止,然終未改。其俗愚悍少慮,輕薄無威,亦有所長,敢於急人,燕丹遺風也。

上谷至遼東,地廣民希,數被胡寇,俗與趙、代相類,有魚鹽棗栗之饒。北隙烏丸、夫餘,東賈真番之利。

玄菟、樂浪,武帝時置,皆朝鮮、濊貉、句驪蠻夷。殷道衰,箕子去之朝鮮,教其民以禮義,田蠶織作。樂浪朝鮮民犯禁八條:相殺以當時償殺;相傷以穀償;相盜者男沒入為其家奴,女子為婢,欲自贖者,人五十萬。雖免為民,俗猶羞之,嫁取無所讎,是以其民終不相盜,無門戶之閉,婦人貞信不淫辟。其田民飲食以籩豆,都邑頗放效吏及內郡賈人,往往以杯器食。郡初取吏於遼東,吏見民無閉臧,及賈人往者,夜則為盜,俗稍益薄。今於犯禁浸多,至六十餘條。可貴哉,仁賢之化也!然東夷天性柔順,異於三方之外,故孔子悼道不行,設浮於海,欲居九夷,有以也夫!樂浪海中有倭人,分為百餘國,以歲時來獻見云。

自危四度至斗六度,謂之析木之次,燕之分也。

齊地,虛、危之分野也。東有甾川、東萊、琅邪、高密、膠東,南有泰山、城陽,北有千乘,清河以南,勃海之高樂、高城、重合、陽信,西有濟南、平原,皆齊分也。

少昊之世有爽鳩氏,虞、夏時有季崱,湯時有逢公柏陵,殷末有薄姑氏,皆為諸侯,國此地。至周成王時,薄姑氏與四國共作亂,成王滅之,以封師尚父,是為太公。詩風齊國是也。臨甾名營丘,故《齊詩》曰:「子之營兮,遭我虖嶩之間兮。」又曰:「俟我於著乎而。」此亦其舒緩之體也。吳札聞齊之歌,曰:「泱泱乎,大風也哉!其太公乎?國未可量也。」

古有分土,亡分民。太公以齊地負海舄鹵,少五穀而人民寡,乃勸以女工之業,通魚鹽之利,而人物輻湊。後十四世,桓公用管仲,設輕重以富國,合諸侯成伯功,身在陪臣而取三歸。故其俗彌侈,織作冰紈綺繡純麗之物,號為冠帶衣履天下。

初太公治齊,修道術,尊賢智,賞有功,故至今其土多好經術,矜功名,舒緩闊達而足智。其失夸奢朋黨,言與行繆,虛詐不情,急之則離散,緩之則放縱。始桓公兄襄公淫亂,姑姊妹不嫁,於是令國中民家長女不得嫁,名曰「巫兒」,為家主祠,嫁者不利其家,民至今以為俗。痛乎,道民之道,可不慎哉!

昔太公始封,周公問「何以治齊?」太公曰:「舉賢而上功。」周公曰:「後世必有篡殺之臣。」其後二十九世為彊臣田和所滅,而和自立為齊侯。初,和之先陳公子完有罪來奔齊,齊桓公以為大夫,更稱田氏。九世至和而篡齊,至孫威王稱王,五世為秦所滅。

臨甾,海、岱之間一都會也,其中具五民云。

魯地,奎、婁之分野也。東至東海,南有泗水,至淮,得臨淮之下相、睢陵、僮、取慮,皆魯分也。

周興,以少昊之虛曲阜封周公子伯禽為魯侯,以為周公主。其民有聖人之教化,故孔子曰「齊一變至於魯,魯一變至於道」,言近正也。瀕洙泗之水,其民涉度,幼者扶老而代其任。俗既益薄,長老不自安,與幼少相讓,故曰:「魯道衰,洙泗之間齗齗如也。」孔子閔王道將廢,乃修六經,以述唐虞三代之道,弟子受業而通者七十有七人。是以其民好學,上禮義,重廉恥。周公始封,太公問「何以治魯?」周公曰:「尊尊而親親。」太公曰:「後世浸弱矣。」故魯自文公以後,祿去公室,政在大夫,季氏逐昭公,陵夷微弱,三十四世而為楚所滅。然本大國,故自為分野。

今去聖久遠,周公遺化銷微,孔氏庠序衰壞。地骥民眾,頗有桑麻之業,亡林澤之饒。俗儉嗇愛財,趨商賈,好訾毀,多巧偽,喪祭之禮文備實寡,然其好學猶愈於它俗。

漢興以來,魯東海多至卿相。東平、須昌、壽良,皆在濟東,屬魯,非宋地也,當考。

宋地,房、心之分野也。今之沛、梁、楚、山陽、濟陰、東平及東郡之須昌、壽張,皆宋分也。

周封微子於宋,今之睢陽是也,本陶唐氏火正閼伯之虛也。濟陰定陶,詩風曹國也。武王封弟叔振鐸於曹,其後稍大,得山陽、陳留,二十餘世為宋所滅。

昔堯作游成陽,舜漁剨澤,湯止于亳,故其民猶有先王遺風,重厚多君子,好稼穡,惡衣食,以致畜藏。

宋自微子二十餘世,至景公滅曹,滅曹後五世亦為齊、楚、魏所滅,參分其地。魏得其梁、陳留,齊得其濟陰、東平,楚得其沛。故今之楚彭城,本宋也,春秋經曰「圍宋彭城」。宋雖滅,本大國,故自為分野。

沛楚之失,急疾顓己,地薄民貧,而山陽好為姦盜。

衛地,營室、東壁之分野也。今之東郡及魏郡黎陽,河內之野王、朝歌,皆衛分也。

衛本國既為狄所滅,文公徙封楚丘,三十餘年,子成公徙於帝丘。故春秋經曰「衛俣于帝丘」,今之濮陽是也。本顓頊之虛,故謂之帝丘。夏后之世,昆吾氏居之。成公後十餘世,為韓、魏所侵,盡亡其旁邑,獨有濮陽。後秦滅濮陽,置東郡,徙之於野王。始皇既并天下,猶獨置衛君,二世時乃廢為庶人。凡四十世,九百年,最後絕,故獨為分野。

衛地有桑間濮上之阻,男女亦亟聚會,聲色生焉,故俗稱鄭衛之音。周末有子路、夏育,民人慕之,故其俗剛武,上氣力。漢興,二千石治者亦以殺戮為威。宣帝時韓延壽為東郡太守,承聖恩,崇禮義,尊諫爭,至今東郡號善為吏,延壽之化也。其失頗奢靡,嫁取送死過度,而野王好氣任俠,有濮上風。

楚地,翼、軫之分野也。今之南郡、江夏、零陵、桂陽、武陵、長沙及漢中、汝南郡,盡楚分也。

周成王時,封文、武先師鬻熊之曾孫熊繹於荊蠻,為楚子,居丹陽。後十餘世至熊達,是為武王,簧以彊大。後五世至嚴王,總帥諸侯,觀兵周室,并吞江、漢之間,內滅陳、魯之國。後十餘世,頃襄王東徙于陳。

楚有江漢川澤山林之饒;江南地廣,或火耕水耨。民食魚稻,以漁獵山伐為業,果蓏蠃蛤,食物常足。故髁窳媮生,而亡積聚,飲食還給,不憂凍餓,亦亡千金之家。信巫鬼,重淫祀。而漢中淫失枝柱,與巴蜀同俗。汝南之別,皆急疾有氣勢。江陵,故郢都,西通巫、巴,東有雲夢之饒,亦一都會也。

吳地,斗分野也。今之會稽、九江、丹陽、豫章、廬江、廣陵、六安、臨淮郡,盡吳分也。

殷道既衰,周大王亶父興廄梁之地,長子大伯,次曰仲雍,少曰公季。公季有聖子昌,大王欲傳國焉。大伯、仲雍辭行采藥,遂奔荊蠻。公季嗣位,至昌為西伯,受命而王。故孔子美而稱曰:「大伯,可謂至德也已矣!三以天下讓,民無得而稱焉。」謂「虞仲夷逸,隱居放言,身中清,廢中權。」大伯初奔荊蠻,荊蠻歸之,號曰句吳。大伯卒,仲雍立,至曾孫周章,而武王克殷,因而封之。又封周章弟中於河北,是為北吳,後世謂之虞,十二世為晉所滅。後二世而荊蠻之吳子壽夢盛大稱王。其少子則季札,有賢材。兄弟欲傳國,札讓而不受。自大伯壽夢稱王六世,闔廬舉伍子胥、孫武為將,戰勝攻取,興伯名於諸侯。至子夫差,誅子胥,用宰嚭,為粵王句踐所滅。

吳、粵之君皆好勇,故其民至今好用劍,輕死易發。

粵既并吳,後六世為楚所滅。後秦又擊楚,徙壽春,至子為秦所滅。

壽春、合肥受南北湖皮革、鮑、木之輸,亦一都會也。始楚賢臣屈原被讒放流,作離騷諸賦以自傷悼。後有宋玉、唐勒之屬慕而述之,皆以顯名。漢興,高祖王兄子濞於吳,招致天下之娛游子弟,枚乘、鄒陽、嚴夫子之徒興於文、景之際。而淮南王安亦都壽春,招賓客著書。而吳有嚴助、朱賈臣,貴顯漢朝,文辭並發,故世傳楚辭。其失巧而少信。初淮南王異國中民家有女者,以待游士而妻之,故至今多女而少男。本吳粵與楚接比,數相并兼,故民俗略同。

吳東有海鹽章山之銅,三江五湖之利,亦江東之一都會也。豫章出黃金,然菫菫物之所有,取之不足以更費。江南卑溼,丈夫多夭。

會稽海外有東鯷人,分為二十餘國,以歲時來獻見云。

粵地,牽牛、婺女之分野也。今之蒼梧、鬱林、合浦、交阯、九真、南海、日南,皆粵分也。

其君禹後,帝少康之庶子云,封於會稽,文身斷髮,以避蛟龍之害。後二十世,至句踐稱王,與吳王闔廬戰,敗之雋李。夫差立,句踐乘勝復伐吳,吳大破之,棲會稽,臣服請平。後用范蠡、大夫種計,遂伐滅吳,兼并其地。度淮與齊、晉諸侯會,致貢於周。周元王使使賜命為伯,諸侯畢賀。後五世為楚所滅,子孫分散,君服於楚。後十世,至閩君搖,佐諸侯平秦。漢興,復立搖為越王。是時,秦南海尉趙佗亦自王,傳國至武帝時,盡滅以為郡云。

處近海,多犀、象、毒冒、珠璣、銀、銅、果、布之湊,中國往商賈者多取富焉。番禺,其一都會也。

自合浦徐聞南入海,得大州,東西南北方千里,武帝元封元年略以為儋耳、珠崖郡。民皆服布如單被,穿中央為貫頭。男子耕農,種禾稻紵麻,女子桑蠶織績。亡馬與虎,民有五畜,山多麈嗷。兵則矛、盾、刀,木弓弩,竹矢,或骨為鏃。自初為郡縣,吏卒中國人多侵陵之,故率數歲壹反。元帝時,遂罷棄之。

自日南障塞、徐聞、合浦船行可五月,有都元國;又船行可四月,有邑盧沒國;又船行可二十餘日,有諶離國;步行可十餘日,有夫甘都盧國。自夫甘都盧國船行可二月餘,有黃支國,民俗略與珠劯相類。其州廣大,戶口多,多異物,自武帝以來皆獻見。有譯長,屬黃門,與應募者俱入海巿明珠、璧流離、奇石異物,齎黃金雜繒而往。所至國皆稟食為耦,蠻夷賈船,轉送致之。亦利交易,剽殺人。又苦逢風波溺死,不者數年來還。大珠至圍二寸以下。平帝元始中,王莽輔政,欲燿威德,厚遺黃支王,令遣使獻生犀牛。自黃支船行可八月,到皮宗;船行可八月,到日南、象林界云。黃支之南,有已程不國,漢之譯使自此還矣。


\end{pinyinscope}