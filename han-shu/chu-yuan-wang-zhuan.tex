\article{楚元王傳}

\begin{pinyinscope}
楚元王交字游,高祖同父少弟也。好書,多材藝。少時嘗與魯穆生、白生、申公俱受詩於浮丘伯。伯者,孫卿門人也。及秦焚書,各別去。

高祖兄弟四人,長兄伯,次仲,伯蚤卒。高祖既為沛公,景駒自立為楚王。高祖使仲與審食其留侍太上皇,交與蕭、曹等俱從高祖見景駒,遇項梁,共立楚懷王。因西攻南陽,入武關,與秦戰於藍田。至霸上,封交為文信君,從入蜀漢,還定三秦,誅項籍。即帝位,交與盧綰常侍上,出入臥內,傳言語諸內事隱謀。而上從父兄劉賈數別將。

漢六年,既廢楚王信,分其地為二國,立賈為荊王,交為楚王,王薛郡、東海、彭城三十六縣,先有功也。後封次兄仲為代王,長子肥為齊王。

初,高祖微時,常避事,時時與賓客過其丘嫂食。嫂厭叔與客來,陽為羹盡,轑釜,客以故去。已而視釜中有羹,繇是怨嫂。及立齊、代王,而伯子獨不得侯。太上皇以為言,高祖曰:「某非敢忘封之也,為其母不長者。」七年十月,封其子信為羹頡侯。

元王既至楚,以穆生、白生、申公為中大夫。高后時,浮丘伯在長安,元王遣子郢客與申公俱卒業。文帝時,聞申公為詩最精,以為博士。元王好詩,諸子皆讀詩,申公始為詩傳,號魯詩。元王亦次之詩傳,號曰元王詩,世或有之。

高后時,以元王子郢客為宗正,封上邳侯。元王立二十三年薨,太子辟非先卒,文帝乃以宗正上邳侯郢客嗣,是為夷王。申公為博士,失官,隨郢客歸,復以為中大夫。立四年薨,子戊嗣。文帝尊寵元王,子生,爵比皇子。景帝即位,以親親封元王寵子五人:子禮為平陸侯,富為休侯,歲為沈猶侯,埶為宛朐侯,調為棘樂侯。

初,元王敬禮申公等,穆生不耆酒,元王每置酒,常為穆生設醴。及王戊即位,常設,後忘設焉。穆生退曰:「可以逝矣!醴酒不設,王之意怠,不去,楚人將鉗我於市。」稱疾臥。申公、白生強起之曰:「獨不念先王之德與?今王一旦失小禮,何足至此!」穆生曰:「易稱『知幾其神乎!幾者動之微,吉凶之先見者也。君子見幾而作,不俟終日。』先王之所以禮吾三人者,為道之存故也;今而忽之,是忘道也。忘道之人,胡可與久處!豈為區區之禮哉?」遂謝病去。申公、白生獨留。

王戊稍淫暴,二十年,為薄太后服私姦,削東海、薛郡,乃與吳通謀。二人諫,不聽,胥靡之,衣之赭衣,使杵臼雅舂於市。休侯使人諫王,王曰:「季父不吾與,我起,先取季父矣。」休侯懼,乃與母太夫人奔京師。二十一年春,景帝之三年也,削書到,遂應吳王反。其相張尚、太傅趙夷吾諫,不聽。遂殺尚、夷吾,起兵會吳西攻梁,破棘壁,至昌邑南,與漢將周亞夫戰。漢絕吳楚糧道,士饑,吳王走,戊自殺,軍遂降漢。

漢已平吳楚,景帝乃立宗正平陸侯禮為楚王,奉元王後,是為文王。四年薨,子安王道嗣。二十二年薨,子襄王注嗣。十四年薨,子節王純嗣。十六年薨,子延壽嗣。宣帝即位,延壽以為廣陵王胥武帝子,天下有變必得立,陰欲附倚輔助之,故為其後母弟趙何齊取廣陵王女為妻。與何齊謀曰:「我與廣陵王相結,天下不安,發兵助之,使廣陵王立,何齊尚公主,列侯可得也。」因使何齊奉書遺廣陵王曰:「願長耳目,毋後人有天下。」何齊父長年上書告之。事下有司,考驗辭服,延壽自殺。立三十二年,國除。

初,休侯富既奔京師,而王戊反,富等皆坐免侯,削屬籍。後聞其數諫戊,乃更封為紅侯。太夫人與竇太后有親,懲山東之寇,求留京師,詔許之。富子辟彊等四人供養,仕於朝。太夫人薨,賜塋,葬靈戶。富傳國至曾孫,無子,絕。

辟彊字少卿,亦好讀詩,能屬文。武帝時,以宗室子隨二千石論議,冠諸宗室。清靜少欲,常以書自娛,不肯仕。昭帝即位,或說大將軍霍光曰:「將軍不見諸呂之事乎?處伊尹、周公之位,攝政擅權,而背宗室,不與共職,是以天下不信,卒至於滅亡。今將軍當盛位,帝春秋富,宜納宗室,又多與大臣共事,反諸呂道,如是則可以免患。」光然之,乃擇宗室可用者。辟彊子德待詔丞相府,年三十餘,欲用之。或言父見在,亦先帝之所寵也。遂拜辟彊為光祿大夫,守長樂衛尉,時年已八十矣。徙為宗正,數月卒。

德字路叔少,修黃老術,有智略。少時數言事,召見甘泉宮,武帝謂之「千里駒」。昭帝初,為宗正丞,雜治劉澤詔獄。父為宗正,徙大鴻臚丞,遷太中大夫,後復為宗正,雜案上官氏、蓋主事。德常持老子知足之計。妻死,大將軍光欲以女妻之,德不敢取,畏盛滿也。蓋長公主孫譚遮德自言,德數責以公主起居無狀。侍御史以為光望不受女,承指劾德誹謗詔獄,免為庶人,屏居山田。光聞而恨之,復白召德守青州刺史。歲餘,復為宗正,與立宣帝,以定策賜爵關內侯。地節中,以親親行謹厚封為陽城侯。子安民為郎中右曹,宗家以德得官宿衛者二十餘人。

德寬厚,好施生,每行京兆尹事,多所平反罪人。家產過百萬,則以振昆弟賓客食飲,曰:「富,民之怨也。」立十一年,子向坐鑄偽黃金,當伏法,德上書訟罪。會薨,大鴻臚奏德訟子罪,失大臣體,不宜賜諡置嗣。制曰:「賜諡繆侯,為置嗣。」傳至孫慶忌,復為宗正太常。薨,子岑嗣,為諸曹中郎將,列校尉,至太常。薨,傳子,至王莽敗,乃絕。

向字子政,本名更生。年十二,以父德任為輦郎。既冠,以行修飭擢為諫大夫。是時,宣帝循武帝故事,招選名儒俊材置左右。更生以通達能屬文辭,與王褒、張子僑等並進對,獻賦頌凡數十篇。上復興神僊方術之事,而淮南有枕中鴻寶苑秘書。書言神僊使鬼物為金之術,及鄒衍重道延命方,世人莫見,而更生父德武帝時治淮南獄得其書。更生幼而讀誦,以為奇,獻之,言黃金可成。上令典尚方鑄作事,費甚多,方不驗。上乃下更生吏,吏劾更生鑄偽黃金,繫當死。更生兄陽城侯安民上書,入國戶半,贖更生罪。上亦奇其材,得踰冬減死論。會初立穀梁春秋,徵更生受穀梁,講論五經於石渠。復拜為郎中給事黃門,遷散騎諫大夫給事中。

元帝初即位,太傅蕭望之為前將軍,少傅周堪為諸吏光祿大夫,皆領尚書事,甚見尊任。更生年少於望之、堪,然二人重之,薦更生宗室忠直,明經有行,擢為散騎宗正給事中,與侍中金敞拾遺於左右。四人同心輔政,患苦外戚許、史在位放縱,而中書宦官弘恭、石顯弄權。望之、堪、更生議,欲白罷退之。未白而語泄,遂為許、史及恭、顯所譖愬,堪、更生下獄,及望之皆免官。語在望之傳。其春地震,夏,客星見昴、卷舌間。上感悟,下詔賜望之爵關內侯,奉朝請。秋,徵堪、向,欲以為諫大夫,恭、顯白皆為中郎。冬,地復震。時恭、顯、許、史子弟侍中諸曹,皆側目於望之等,更生懼焉,乃使其外親上變事,言:

竊聞故前將軍蕭望之等,皆忠正無私,欲致大治,忤於貴戚尚書。今道路人聞望之等復進,以為且復見毀讒,必曰嘗有過之臣不宜復用,是大不然。臣聞春秋地震,為在位執政太盛也,不為三獨夫動,亦已明矣。且往者高皇帝時,季布有罪,至於夷滅,後赦以為將軍,高后、孝文之間卒為名臣。孝武帝時兒寬有重罪繫,按道侯韓說諫曰:「前吾丘壽王死,陛下至今恨之;今殺寬,後將復大恨矣!」上感其言,遂貰寬,復用之,位至御史大夫,御史大夫未有及寬者也。又董仲舒坐私為災異書,主父偃取奏之,下吏,罪至不道,幸蒙不誅,復為太中大夫,膠西相,以老病免歸。漢有所欲興,常有詔問。仲舒為世儒宗,定議有益天下。孝宣皇帝時,夏侯勝坐誹謗繫獄,三年免為庶人。宣帝復用勝,至長信少府,太子太傅,名敢直言,天下美之。若乃群臣,多此比類,難一二記。有過之臣,無負國家,有益天下,此四臣者,足以觀矣。

前弘恭奏望之等獄決,三月,地大震。恭移病出,後復視事,天陰雨雪。由是言之,地動殆為恭等。

臣愚以為宜退恭、顯以章蔽善之罰,進望之等以通賢者之路。如此,太平之門開,災異之原塞矣。

書奏,恭、顯疑其更生所為,白請考姦詐。辭果服,遂逮更生繫獄,下太傅韋玄成、諫大夫貢禹,與廷尉雜考。劾更生前為九卿,坐與望之、堪謀排車騎將軍高、許、史氏侍中者,毀離親戚,欲退去之,而獨專權。為臣不忠,幸不伏誅,復蒙恩徵用,不悔前過,而教令人言變事,誣罔不道。更生坐免為庶人。而望之亦坐使子上書自冤前事,恭、顯白令詣獄置對。望之自殺。天子甚悼恨之,乃擢周堪為光祿勳,堪弟子張猛光祿大夫給事中,大見信任。恭、顯憚之,數譖毀焉。更生見堪、猛在位,幾己得復進,懼其傾危,乃上封事諫曰:

臣前幸得以骨肉備九卿,奉法不謹,乃復蒙恩。竊見災異並起,天地失常,徵表為國。欲終不言,念忠臣雖在甽畝,猶不忘君,惓惓之義也。況重以骨肉之親,又加以舊恩未報乎!欲竭愚誠,又恐越職,然惟二恩未報,忠臣之義,一杼愚意,退就農畝,死無所恨。

臣聞舜命九官,濟濟相讓,和之至也。眾賢和於朝,則萬物和於野。故簫韶九成,而鳳皇來儀;擊石拊石,百獸率舞。四海之內,靡不和寧。及至周文,開基西郊,雜遝眾賢,罔不肅和,崇推讓之風,以銷分爭之訟。文王既沒,周公思慕,歌詠文王之德,其《詩》曰:「於穆清廟,肅雍顯相;濟濟多士,秉文之德。」當此之時,武王、周公繼政,朝臣和於內,萬國驩於外,故盡得其驩心,以事其先祖。其《詩》曰:「有來雍雍,至止肅肅,相維辟公,天子穆穆。」言四方皆以和來也。諸侯和於下,天應報於上,故周頌曰「降福穰穰」,又曰「飴我釐麰」。釐麰,麥也,始自天降。此皆以和致和,獲天助也。

下至幽、厲之際,朝廷不和,轉相非怨,詩人疾而憂之曰:「民之無良,相怨一方。」眾小在位而從邪議,歙歙相是而背君子,故其《詩》曰:「歙歙訿訿,亦孔之哀!謀之其臧,則具是違;謀之不臧,則具是依!」君子獨處守正,不橈眾枉,勉彊以從王事則反見憎毒讒愬,故其《詩》曰:「密勿從事,不敢告勞,無罪無辜,讒口嗷嗷!」當是之時,日月薄蝕而無光,其《詩》曰:「朔日辛卯,日有蝕之,亦孔之醜!」又曰:「彼月而微,此日而微,今此下民,亦孔之哀!」又曰:「日月鞠凶,不用其行;四國無政,不用其良!」天變見於上,地變動於下,水泉沸騰,山谷易處。其《詩》曰:「百川沸騰,山冢卒崩,高岸為谷,深谷為陵。哀今之人,胡憯莫懲!」霜降失節,不以其時,其《詩》曰:「正月繁霜,我心憂傷;民之訛言,亦孔之將!」言民以是為非,甚眾大也。此皆不和,賢不肖易位之所致也。

自此之後,天下大亂,篡殺殃禍並作,厲王奔彘,幽王見殺。至乎平王末年,魯隱之始即位也,周大夫祭伯乖離不和,出奔於魯,而春秋為諱,不言來奔,傷其禍殃自此始也。是後尹氏世卿而專恣,諸侯背畔而不朝,周室卑微。二百四十二年之間,日食三十六,地震五,山陵崩阤二,彗星三見,夜常星不見,夜中星隕如雨一,火災十四。長狄入三國,五石隕墜,六鶂退飛,多麋,有蜮、蜚,鴝鵒來巢者,皆一見。晝冥晦。雨木冰。李梅冬實。七月霜降,草木不死。八月殺菽。大雨雹。雨雪雷霆失序相乘。水、旱、饑,蝝、螽、螟螽午並起。當是時,禍亂輒應,弒君三十六,亡國五十二,諸侯奔走,不得保其社稷者,不可勝數也。周室多禍:晉敗其師於貿戎;伐其郊;鄭傷桓王;戎執其使;衛侯朔召不往,齊逆命而助朔;五大夫爭權,三君更立,莫能正理。遂至陵夷不能復興。

由此觀之,和氣致祥,乖氣致異;祥多者其國安,異眾者其國危,天地之常經,古今之通義也。今陛下開三代之業,招文學之士,優游寬容,使得並進。今賢不肖渾殽,白黑不分,邪正雜糅,忠讒並進。章交公車,人滿北軍。朝臣舛午,膠戾乘剌,更相讒愬,轉相是非。傳授增加,文書紛糾,前後錯繆,毀譽渾亂。所以營或耳目,感移心意,不可勝載。分曹為黨,往往群朋,將同心以陷正臣。正臣進者,治之表也;正臣陷者,亂之機也。乘治亂之機,未知孰任,而災異數見,此臣所以寒心者也。夫乘權藉勢之人,子弟鱗集於朝,羽翼陰附者眾,輻湊於前,毀譽將必用,以終乖離之咎。是以日月無光,雪霜夏隕,海水沸出,陵谷易處,列星失行,皆怨氣之所致也。夫遵衰周之軌跡,循詩人之所刺,而欲以成太平,致雅頌,猶卻行而求及前人也。初元以來六年矣,案春秋六年之中,災異未有稠如今者也。夫有春秋之異,無孔子之救,猶不能解紛,況甚於春秋乎?

原其所以然者,讒邪並進也。讒邪之所以並進者,由上多疑心,既已用賢人而行善政,如或譖之,則賢人退而善政還。夫執狐疑之心者,來讒賊之口;持不斷之意者,開群枉之門。讒邪進則眾賢退,群枉盛則正士消。故易有否泰。小人道長,君子道消,君子道消,則政日亂,故為否。否者,閉而亂也。君子道長,小人道消,小人道消,則政日治,故為泰。泰者,通而治也。《詩》又云「雨雪麃麃,見晛聿消」,與易同義。昔者鯀、共工、驩兜與舜、禹雜處堯朝,周公與管、蔡並居周位,當是時,迭進相毀,流言相謗,豈可勝道哉!帝堯、成王能賢舜、禹、周公而消共工、管、蔡,故以大治,榮華至今。孔子與季、孟偕仕於魯,李斯與叔孫俱宦於秦,定公、始皇賢季、孟、李斯而消孔子、叔孫,故以大亂,污辱至今。故治亂榮辱之端,在所信任;信任既賢,在於堅固而不移。《詩》云「我心匪石,不可轉也」。言守善篤也。《易》曰「渙汗其大號」。言號令如汗,汗出而不反者也。今出善令,未能踰時而反,是反汗也;用賢未能三旬而退,是轉石也。《論語》曰:「見不善如探湯。」今二府奏佞諂不當在位,歷年而不去。故出令則如反汗,用賢則如轉石,去佞則如拔山,如此望陰陽之調,不亦難乎!

是以群小窺見間隙,緣飾文字,巧言醜詆,流言飛文,譁於民間。故《詩》云:「憂心悄悄,慍于群小。」小人成群,誠足慍也。昔孔子與顏淵、子貢更相稱譽,不為朋黨;禹、稷與皋陶傳相汲引,不為比周。何則?忠於為國,無邪心也。故賢人在上位,則引其類而聚之於朝,《易》曰「飛龍在天,大人聚也」;在下位,則思與其類俱進,《易》曰「拔茅茹以其彙,征吉」。在上則引其類,在下則推其類,故湯用伊尹,不仁者遠,而眾賢至,類相致也。今佞邪與賢臣並在交戟之內,合黨共謀,違善依惡,歙歙訿訿,數設危險之言,欲以傾移主上。如忽然用之,此天地之所以先戒,災異之所以重至者也。

自古明聖,未有無誅而治者也,故舜有四放之罰,而孔子有兩觀之誅,然後聖化可得而行也。今以陛下明知,誠深思天地之心,跡察兩觀之誅,覽否泰之卦,觀雨雪之詩,歷周、唐之所進以為法,原秦、魯之所消以為戒,考祥應之福,省災異之禍,以揆當世之變,放遠佞邪之黨,壞散險詖之聚,杜閉群枉之門,廣開眾正之路,決斷狐疑,分別猶豫,使是非炳然可知,則百異消滅,而眾祥並至,太平之基,萬世之利也。

臣幸得託肺附,誠見陰陽不調,不敢不通所聞。竊推春秋災異,以效今事一二,條其所以,不宜宣泄。臣謹重封昧死上。

恭、顯見其書,愈與許、史比而怨更生等。堪性公方,自見孤立,遂直道而不曲。是歲夏寒,日青無光,恭、顯及許、史皆言堪、猛用事之咎。上內重堪,又患眾口之寖潤,無所取信。時長安令楊興以材能幸,常稱譽堪。上欲以為助,乃見問興:「朝臣齗齗不可光祿勳,何也?」興者傾巧士,謂上疑堪,因順指曰:「堪非獨不可於朝廷,自州里亦不可也。臣見眾人聞堪前與劉更生等謀毀骨肉,以為當誅,故臣前言堪不可誅傷,為國養恩也。」上曰:「然此何罪而誅?今宜奈何?」興曰:「臣愚以為可賜爵關內侯,食邑三百戶,勿令典事。明主不失師傅之恩,此最策之得者也。」上於是疑。會城門校尉諸葛豐亦言堪、猛短,上因發怒免豐。語在其傳。又曰:「豐言堪、猛貞信不立,朕閔而不治,又惜其材能未有所效,其左遷堪為河東太守,猛槐里令。」

顯等專權日甚。後三歲餘,孝宣廟闕災,其晦,日有蝕之。於是上召諸前言日變在堪、猛者責問,皆稽首謝。乃因下詔曰:「河東太守堪,先帝賢之,命而傅朕。資質淑茂,道術通明,論議正直,秉心有常,發憤悃愊,信有憂國之心。以不能阿尊事貴,孤特寡助,抑厭遂退,卒不克明。往者眾臣見異,不務自修,深惟其故,而反晻昧說天,託咎此人。朕不得已,出而試之,以彰其材。堪出之後,大變仍臻,眾亦嘿然。堪治未期年,而三老官屬有識之士詠頌其美,使者過郡,靡人不稱。此固足以彰先帝之知人,而朕有以自明也。俗人乃造端作基,非議詆欺,或引幽隱,非所宜明,意疑以類,欲以陷之,朕亦不取也。朕迫于俗,不得專心,乃者天著大異,朕甚懼焉。今堪年衰歲暮,恐不得自信,排於異人,將安究之哉?其徵堪詣行在所。」拜為光祿大夫,秩中二千石,領尚書事。猛復為太中大夫給事中。顯幹尚書,尚書五人,皆其黨也。堪希得見,常因顯白事,事決顯口。會堪疾瘖,不能言而卒。顯誣譖猛,令自殺於公車。更生傷之,乃著疾讒、擿要、救危及世頌,凡八篇,依興古事,悼己及同類也。遂廢十餘年。

成帝即位,顯等伏辜,更生乃復進用,更名向。向以故九卿召拜為中郎,使領護三輔都水。數奏封事,遷光祿大夫。是時帝元舅陽平侯王鳳為大將軍秉政,倚太后,專國權,兄弟七人皆封為列侯。時數有大異,向以為外戚貴盛,鳳兄弟用事之咎。而上方精於詩書,觀古文,詔向領校中五經祕書。向見尚書洪範,箕子為武王陳五行陰陽休咎之應。向乃集合上古以來歷春秋六國至秦漢符瑞災異之記,推跡行事,連傳禍福,著其占驗,比類相從,各有條目,凡十一篇,號曰洪範五行傳論,奏之。天子心知向忠精,故為鳳兄弟起此論也,然終不能奪王氏權。

久之,營起昌陵,數年不成,復還歸延陵,制度泰奢。向上疏諫曰:

臣聞易曰:「安不忘危,存不忘亡,是以身安而國家可保也。」故賢聖之君,博觀終始,窮極事情,而是非分明。王者必通三統,明天命所授者博,非獨一姓也。孔子論詩,至於「殷士膚敏,祼將于京」,喟然歎曰:「大哉天命!善不可不傳于子孫,是以富貴無常;不如是,則王公其何以戒慎,民萌何以勸勉?」蓋傷微子之事周,而痛殷之亡也。雖有堯舜之聖,不能化丹朱之子;雖有禹湯之德,不能訓末孫之桀紂。自古及今,未有不亡之國也。昔高皇帝既滅秦,將都雒陽,感寤劉敬之言,自以德不及周,而賢於秦,遂徙都關中,依周之德,因秦之阻。世之長短,以德為效,故常戰栗,不敢諱亡。孔子所謂「富貴無常」,蓋謂此也。

孝文皇帝居霸陵,北臨廁,意悽愴悲懷,顧謂群臣曰:「嗟乎!以北山石為槨,用紵絮斮陳漆其間,豈可動哉!」張釋之進曰:「使其中有可欲,雖錮南山猶有隙;使其中無可欲,雖無石槨,又何慼焉?」夫死者無終極,而國家有廢興,故釋之之言,為無窮計也。孝文寤焉,遂薄葬,不起山墳。

《易》曰:「古之葬者,厚衣之以薪,臧之中野,不封不樹。後世聖人易之以棺槨。」棺槨之作,自黃帝始。黃帝葬於橋山,堯葬濟陰,丘壟皆小,葬具甚微。舜葬蒼梧,二妃不從。禹葬會稽,不改其列。殷湯無葬處。文、武、周公葬於畢,秦穆公葬於雍橐泉宮祈年館下,樗里子葬於武庫,皆無丘隴之處。此聖帝明王賢君智士遠覽獨慮無窮之計也。其賢臣孝子亦承命順意而薄葬之,此誠奉安君父,忠孝之至也。

夫周公,武王弟也,葬兄甚微。孔子葬母於防,稱古墓而不墳,曰:「丘,東西南北之人也,不可不識也。」為四尺墳,遇雨而崩。弟子修之,以告孔子,孔子流涕曰:「吾聞之,古不修墓。」蓋非之也。延陵季子適齊而反,其子死,葬於嬴、博之間,穿不及泉,斂以時服,封墳掩坎,其高可隱,而號曰:「骨肉歸復於土,命也,魂氣則無不之也。」夫嬴、博去吳千有餘里,季子不歸葬。孔子往觀曰:「延陵季子於禮合矣。」故仲尼孝子,而延陵慈父,舜禹忠臣,周公弟弟,其葬君親骨肉,皆微薄矣;非苟為儉,誠便於體也。宋桓司馬為石槨,仲尼曰「不如速朽。」秦相呂不韋集知略之士而造春秋,亦言薄葬之義,皆明於事情者也。

逮至吳王闔閭,違禮厚葬,十有餘年,越人發之。及秦惠文、武、昭、嚴襄五王,皆大作丘隴,多其瘞臧,咸盡發掘暴露,甚足悲也。秦始皇帝葬於驪山之阿,下錮三泉,上崇山墳,其高五十餘丈,周回五里有餘;石槨為游館,人膏為燈燭,水銀為江海,黃金為鳧雁。珍寶之臧,機械之變,棺槨之麗,宮館之盛,不可勝原。又多殺宮人,生薶工匠,計以萬數。天下苦其役而反之,驪山之作未成,而周章百萬之師至其下矣。項籍燔其宮室營宇,往者咸見發掘。其後牧兒亡羊,羊入其鑿,牧者持火照求羊,失火燒其臧槨。自古至今,葬未有盛如始皇者也,數年之間,外被項籍之災,內離牧豎之禍,豈不哀哉!

是故德彌厚者葬彌薄,知愈深者葬愈微。無德寡知,其葬愈厚,丘隴彌高,宮廟甚麗,發掘必速。由是觀之,明暗之效,葬之吉凶,昭然可見矣。周德既衰而奢侈,宣王賢而中興,更為儉宮室,小寢廟。詩人美之,斯干之詩是也,上章道宮室之如制,下章言子孫之眾多也。及魯嚴公刻飾宗廟,多築臺囿,後嗣再絕,春秋刺焉。周宣如彼而昌,魯、秦如此而絕,是則奢儉之得失也。

陛下即位,躬親節儉,始營初陵,其制約小,天下莫不稱賢明。及徙昌陵,增埤為高,積土為山,發民墳墓,積以萬數,營起邑居,期日迫卒,功費大萬百餘。死者恨於下,生者愁於上,怨氣感動陰陽,因之以饑饉,物故流離以十萬數,臣甚惛焉。以死者為有知,發人之墓,其害多矣;若其無知,又安用大?謀之賢知則不說,以示眾庶則苦之;若苟以說愚夫淫侈之人,又何為哉!陛下慈仁篤美甚厚,聰明疏達蓋世,宜弘漢家之德,崇劉氏之美,光昭五帝、三王,而顧與暴秦亂君競為奢侈,比方丘隴,說愚夫之目,隆一時之觀,違賢知之心,亡萬世之安,臣竊為陛下羞之。唯陛下上覽明聖黃帝、堯、舜、禹、湯、文、武、周公、仲尼之制,下觀賢知穆公、延陵、樗里、張釋之之意。孝文皇帝去墳薄葬,以儉安神,可以為則;秦昭、始皇增山厚臧,以侈生害,足以為戒。初陵之跻,宜從公卿大臣之議,以息眾庶。

書奏,上甚感向言,而不能從其計。

向睹俗彌奢淫,而趙、衛之屬起微賤,踰禮制。向以為王教由內及外,自近者始。故採取詩書所載賢妃貞婦,興國顯家可法則,及孽嬖亂亡者,序次為列女傳,凡八篇,以戒天子。及采傳記行事,著新序、說苑凡五十篇奏之。數上疏言得失,陳法戒。書數十上,以助觀覽,補遺闕。上雖不能盡用,然內嘉其言,常嗟歎之。

時上無繼嗣,政由王氏出,災異阊甚。向雅奇陳湯智謀,與相親友,獨謂湯曰:「災異如此,而外家日甚,其漸必危劉氏。吾幸得同姓末屬,絫世蒙漢厚恩,身為宗室遺老,歷事三主。上以我先帝舊臣,每進見常加優禮,吾而不言,孰當言者?」向遂上封事極諫曰:

臣聞人君莫不欲安,然而常危,莫不欲存,然而常亡,失御臣之術也。夫大臣操權柄,持國政,未有不為害者也。昔晉有六卿,齊有田、崔,衛有孫、甯,魯有季、孟,常掌國事,世執朝柄。終後田氏取齊;六卿分晉;崔杼弒其君光;孫林父、甯殖出其君衎,弒其君剽;季氏八佾舞於庭,三家者以雍徹,並專國政,卒逐昭公。周大夫尹氏筦朝事,濁亂王室,子朝、子猛更立,連年乃定。故經曰「王室亂」,又曰「尹氏殺王子克」,甚之也。春秋舉成敗,錄禍福,如此類甚眾,皆陰盛而陽微,下失臣道之所致也。故書曰:「臣之有作威作福,害于而家,凶于而國。」孔子曰「祿去公室,政逮大夫」,危亡之兆。秦昭王舅穰侯及涇陽、葉陽君專國擅勢,上假太后之威,三人者權重於昭王,家富於秦國,國甚危殆,賴寤范睢之言,而秦復存。二世委任趙高,專權自恣,壅蔽大臣,終有閻樂望夷之禍,秦遂以亡。近事不遠,即漢所代也。

漢興,諸呂無道,擅相尊王。呂產、呂祿席太后之寵,據將相之位,兼南北軍之眾,擁梁、趙王之尊,驕盈無厭,欲危劉氏。賴忠正大臣絳侯、朱虛侯等竭誠盡節以誅滅之,然後劉氏復安。今王氏一姓乘朱輪華轂者二十三人,青紫貂蟬充盈幄內,魚鱗左右。大將軍秉事用權,五侯驕奢僭盛,並作威福,擊斷自恣,行汙而寄治,身私而託公,依東宮之尊,假甥舅之親,以為威重。尚書九卿州牧郡守皆出其門,筦執樞機,朋黨比周。稱譽者登進,忤恨者誅傷;游談者助之說,執政者為之言。排擯宗室,孤弱公族,其有智能者,尤非毀而不進。遠絕宗室之任,不令得給事朝省,恐其與己分權;數稱燕王、蓋主以疑上心,避諱呂、霍而弗肯稱。內有管、蔡之萌,外假周公之論,兄弟據重,宗族磐互。歷上古至秦漢,外戚僭貴未有如王氏者也。雖周皇甫、秦穰侯、漢武安、呂、霍、上官之屬,皆不及也。

物盛必有非常之變先見,為其人微象。孝昭帝時,冠石立於泰山,仆柳起於上林。而孝宣帝即位,今王氏先祖墳墓在濟南者,其梓柱生枝葉,扶疏上出屋,根艺地中,雖立石起柳,無以過此之明也。事勢不兩大,王氏與劉氏亦且不並立,如下有泰山之安,則上有累卵之危。陛下為人子孫,守持宗廟,而令國祚移於外親,降為皁隸,縱不為身,奈宗廟何!婦人內夫家,外父母家,此亦非皇太后之福也。孝宣皇帝不與舅平昌、樂昌侯權,所以安全之也。

夫明者起福於無形,銷患於未然。宜發明詔,吐德音,援近宗室,親而納信,黜遠外戚,毋授以政,皆罷令就弟,以則效先帝之所行,厚安外戚,全其宗族,誠東宮之意,外家之福也。王氏永存,保其爵祿,劉氏長安,不失社稷,所以褒睦外內之姓,子子孫孫無疆之計也。如不行此策,田氏復見於今,六卿必起於漢,為後嗣憂,昭昭甚明,不可不深圖,不可不蚤慮。《易》曰:「君不密,則失臣;臣不密,則失身;幾事不密,則害成。」唯陛下深留聖思,審固幾密,覽往事之戒,以折中取信,居萬安之實,用保宗廟,久承皇太后,天下幸甚。

書奏,天子召見向,歎息悲傷其意,謂曰:「君且休矣,吾將思之。」以向為中壘校尉。

向為人簡易無威儀,廉靖樂道,不交接世俗,專積思於經術,晝誦書傳,夜觀星宿,或不寐達旦。元延中,星孛東井,蜀郡岷山崩雍江。向惡此異,語在五行志。懷不能已,復上奏,其辭曰:

臣聞帝舜戒伯禹,毋若丹朱敖;周公戒成王,毋若殷王紂。《詩》曰「殷監不遠,在夏后之世」,亦言湯以桀為戒也。聖帝明王常以敗亂自戒,不諱廢興,故臣敢極陳其愚,唯陛下留神察焉。

謹案春秋二百四十二年,日蝕三十六,襄公尤數,率三歲五月有奇而壹食。漢興訖竟寧,孝景帝尤數,率三歲一月而一食。臣向前數言日當食,今連三年比食。自建始以來,二十歲間而八食,率二歲六月而一發,古今罕有。異有小大希稠,占有舒疾緩急,而聖人所以斷疑也。《易》曰:「觀乎天文,以察時變。」昔孔子對魯哀公,並言夏桀、殷紂暴虐天下,故曆失則攝提失方,孟陬無紀,此皆易姓之變也。秦始皇之末至二世時,日月薄食,山陵淪亡,辰星出於四孟,太白經天而行,無雲而雷,枉矢夜光,熒惑襲月,谪火燒宮,野禽戲廷,都門內崩,長人見臨洮,石隕于東郡,星孛大角,大角以亡。觀孔子之言,考暴秦之異,天命信可畏也。及項籍之敗,亦孛大角。漢之入秦,五星聚于東井,得天下之象也。孝惠時,有雨血,日食於衝,滅光星見之異。孝昭時,有泰山臥石自立,上林僵柳復起,大星如月西行,眾星隨之,此為特異。孝宣興起之表,天狗夾漢而西,久陰不雨者二十餘日,昌邑不終之異也。皆著於漢紀。觀秦、漢之易世,覽惠、昭之無後,察昌邑之不終,視孝宣之紹起,天之去就,豈不昭昭然哉!高宗、成王亦有雊雉拔木之變,能思其故,故高宗有百年之福,成王有復風之報。神明之應,應若景嚮,世所同聞也。

臣幸得託末屬,誠見陛下有寬明之德,冀銷大異,而興高宗、成王之聲,以崇劉氏,故豤豤數奸死亡之誅。今日食尤屢,星孛東井,攝提炎及紫宮,有識長老莫不震動,此變之大者也。其事難一二記,故《易》曰「書不盡言,言不盡意」,是以設卦指爻,而復說義。《書》曰「伻來以圖」,天文難以相曉,臣雖圖上,猶須口說,然後可知,願賜清燕之閒,指圖陳狀。

上輒入之,然終不能用也。向每召見,數言公族者國之枝葉,枝葉落則本根無所庇廕;方今同姓疏遠,母黨專政,祿去公室,權在外家,非所以彊漢宗,卑私門,保守社稷,安固後嗣也。

向自見得信於上,故常顯訟宗室,譏刺王氏及在位大臣,其言多痛切,發於至誠。上數欲用向為九卿,輒不為王氏居位者及丞相御史所持,故終不遷。居列大夫官前後三十餘年,年七十二卒。卒後十三歲而王氏代漢。向三子皆好學:長子伋,以易教授,官至郡守;中子賜,九卿丞,蚤卒;少子歆,最知名。

歆字子駿,少以通詩書能屬文召見成帝,待詔宦者署,為黃門郎。河平中,受詔與父向領校祕書,講六藝傳記,諸子、詩賦、數術、方技,無所不究。向死後,歆復為中壘校尉。

哀帝初即位,大司馬王莽舉歆宗室有材行,為侍中太中大夫,遷騎都尉、奉車光祿大夫,貴幸。復領五經,卒父前業。歆乃集六藝群書,種別為七略。語在藝文志。

歆及向始皆治易,宣帝時,詔向受穀梁春秋,十餘年,大明習。及歆校秘書,見古文春秋左氏傳,歆大好之。時丞相史尹咸以能治左氏,與歆共校經傳。歆略從咸及丞相翟方進受,質問大義。初左氏傳多古字古言,學者傳訓故而已,及歆治左氏,引傳文以解經,轉相發明,由是章句義理備焉。歆亦湛靖有謀,父子俱好古,博見彊志,過絕於人。歆以為左丘明好惡與聖人同,親見夫子,而公羊、穀梁在七十子後,傳聞之與親見之,其詳略不同。歆數以難向,向不能非間也,然猶自持其穀梁義。及歆親近,欲建立左氏春秋及毛詩、逸禮、古文尚書皆列於學官。哀帝令歆與五經博士講論其義,諸博士或不肯置對,歆因移書太常博士,責讓之曰:

昔唐虞既衰,而三代迭興,聖帝明王,累起相襲,其道甚著。周室既微而禮樂不正,道之難全也如此。是故孔子憂道之不行,歷國應聘。自衛反魯,然後樂正,雅頌乃得其所;修易,序書,制作春秋,以紀帝王之道。及夫子沒而微言絕,七十子終而大義乖。重遭戰國,棄籩豆之禮,理軍旅之陳,孔氏之道抑,而孫吳之術興。陵夷至于暴秦,燔經書,殺儒士,設挾書之法,行是古之罪,道術由是遂滅。漢興,去聖帝明王遐遠,仲尼之道又絕,法度無所因襲。時獨有一叔孫通略定禮儀,天下唯有易卜,未有它書。至孝惠之世,乃除挾書之律,然公卿大臣絳、灌之屬咸介冑武夫,莫以為意。至孝文皇帝,始使掌故朝錯從伏生受尚書。尚書初出于屋壁,朽折散絕,今其書見在,時師傳讀而已。詩始萌牙。天下眾書往往頗出,皆諸子傳說,猶廣立於學官,為置博士。在漢朝之儒,唯賈生而已。至孝武皇帝,然後鄒、魯、梁、趙頗有詩、禮、春秋先師,皆起於建元之間。當此之時,一人不能獨盡其經,或為雅,或為頌,相合而成。泰誓後得,博士集而讀之。故詔書稱曰:「禮壞樂崩,書缺簡脫,朕甚閔焉。」時漢興已七八十年,離於全經,固已遠矣。

及魯恭王壞孔子宅,欲以為宮,而得古文於壞壁之中,逸禮有三十九,書十六篇。天漢之後,孔安國獻之,遭巫蠱倉卒之難,未及施行。及春秋左氏丘明所修,皆古文舊書,多者二十餘通,臧於祕府,伏而未發。孝成皇帝閔學殘文缺,稍離其真,乃陳發祕臧,校理舊文,得此三事,以考學官所傳,經或脫簡,傳或間編。傳問民間,則有魯國柏公、趙國貫公、膠東庸生之遺學與此同,抑而未施。此乃有識者之所惜閔,士君子之所嗟痛也。往者綴學之士不思廢絕之闕,苟因陋就寡,分文析字,煩言碎辭,學者罷老且不能究其一藝。信口說而背傳記,是末師而非往古,至於國家將有大事,若立辟雍封禪巡狩之儀,則幽冥而莫知其原。猶欲保殘守缺,挾恐見破之私意,而無從善服義之公心,或懷妒嫉,不考情實,雷同相從,隨聲是非,抑此三學,以尚書為備,謂左氏為不傳春秋,豈不哀哉!

今聖上德通神明,繼統揚業,亦閔文學錯亂,學士若茲,雖昭其情,猶依違謙讓,樂與士君子同之。故下明詔,試左氏可立不,遣近臣奉指銜命,將以輔弱扶微,與二三君子比意同力,冀得廢遺。今則不然,深閉固距,而不肯試,猥以不誦絕之,欲以杜塞餘道,絕滅微學。夫可與樂成,難與慮始,此乃眾庶之所為耳,非所望士君子也。且此數家之事,皆先帝所親論,今上所考視,其古文舊書,皆有徵驗,外內相應,豈苟而已哉!

夫禮失求之於野,古文不猶愈於野乎?往者博士書有歐陽,春秋公羊,易則施、孟,然孝宣皇帝猶復廣立穀梁春秋,梁丘易,大小夏侯尚書,義雖相反,猶並置之。何則?與其過而廢之也,寧過而立之。傳曰:「文武之道未墜於地,在人;賢者志其大者,不賢者志其小者。」今此數家之言所以兼包大小之義,豈可偏絕哉!若必專己守殘,黨同門,妒道真,違明詔,失聖意,以陷於文吏之議,甚為二三君子不取也。

其言甚切,諸儒皆怨恨。是時名儒光祿大夫龔勝以歆移書上疏深自罪責,願乞骸骨罷。及儒者師丹為大司空,亦大怒,奏歆改亂舊章,非毀先帝所立。上曰:「歆欲廣道術,亦何以為非毀哉?」歆由是忤執政大臣,為眾儒所訕,懼誅,求出補吏,為河內太守。以宗室不宜典三河,徙守五原,後復轉在涿郡,歷三郡守。數年,以病免官,起家復為安定屬國都尉。會哀帝崩,王莽持政,莽少與歆俱為黃門郎,重之,白太后。太后留歆為右曹太中大夫,遷中壘校尉,羲和,京兆尹,使治明堂辟雍,封紅休侯。典儒林史卜之官,考定律曆,著三統曆譜。

初,歆以建平元年改名秀,字穎叔云。及王莽篡位,歆為國師,後事皆在莽傳。

贊曰:仲尼稱「材難不其然與!」自孔子後,綴文之士眾矣,唯孟軻、孫況,董仲舒、司馬遷、劉向、揚雄。此數公者,皆博物洽聞,通達古今,其言有補於世。傳曰「聖人不出,其間必有命世者焉」,豈近是乎?劉氏洪範論發明大傳,著天人之應;七略剖判藝文,總百家之緒;三統曆譜考步日月五星之度。有意其推本之也。嗚虖!向言山陵之戒,于今察之,哀哉!指明梓柱以推廢興,昭矣!豈非直諒多聞,古之益友與!


\end{pinyinscope}