\article{公孫弘卜式兒寬傳}

\begin{pinyinscope}
公孫弘,菑川薛人也。少時為獄吏,有罪,免。家貧,牧豕海上。年四十餘,乃學春秋雜說。

武帝初即位,招賢良文學士,是時弘年六十,以賢良徵為博士。使匈奴,還報,不合意,上怒,以為不能,弘乃移病免歸。

元光五年,復徵賢良文學,菑川國復推上弘。弘謝曰:「前已嘗西,用不能罷,願更選。」國人固推弘,弘至太常。上策詔諸儒:

制曰:蓋聞上古至治,畫衣冠,異章服,而民不犯;陰陽和,五穀登,六畜蕃,甘露降,風雨時,嘉禾興,朱屮生,山不童,澤不涸;麟鳳在郊藪,龜龍游於沼,河洛出圖書;父不喪子,兄不哭弟;北發渠搜,南撫交阯,舟車所至,人跡所及,跂行喙息,咸得其宜。朕甚嘉之,今何道而臻乎此?子大夫修先聖之術,明君臣之義,講論洽聞,有聲乎當世,問子大夫:天人之道,何所本始?吉凶之效,安所期焉?禹湯水旱,厥咎何由?仁義禮知四者之宜,當安設施?屬統垂業,物鬼變化,天命之符,廢興何如?天文地理人事之紀,子大夫習焉。其悉意正議,詳具其對,著之于篇,朕將親覽焉,靡有所隱。

弘對曰:

臣聞上古堯舜之時,不貴爵賞而民勸善,不重刑罰而民不犯,躬率以正而遇民信也;末世貴爵厚賞而民不勸,深刑重罰而姦不止,其上不正,遇民不信也。夫厚當重刑未足以勸善而禁非,必信而已矣。是故因能任官,則分職治;去無用之言,則事情得;不作無用之器,即賦斂省;不奪民時,不妨民力,則百姓富;有德者進,無德者退,則朝廷尊;有功者上,無功者下,則群臣逡;罰當罪,則姦邪止;賞當賢,則臣下勸:凡此八者,治之本也。故民者,業之即不爭,理得則不怨;有禮則不暴,愛之則親上,此有天下之急者也。故法不遠義,則民服而不離;和不遠禮,則民親而不暴。故法之所罰,義之所去也;和之所賞,禮之所取也。禮義者,民之所服也,而賞罰順之,則民不犯禁矣。故畫衣冠,異章服,而民不犯者,此道素行也。

臣聞之,氣同則從,聲比則應。今人主和德於上,百姓和合於下,故心和則氣和,氣和則形和,形和則聲和,聲和則天地之和應矣。故陰陽和,風雨時,甘露降,五穀登,六畜蕃,嘉禾興,朱草生,山不童,澤不涸,此和之至也。故形和則無疾,無疾則不夭,故父不喪子,兄不哭弟。德配天地,明並日月,則麟鳳至,龜龍在郊,河出圖,洛出書,遠方之君莫不說義,奉幣而來朝,此和之極也。

臣聞之,仁者愛也,義者宜也,禮者所履也,智者術之原也。致利除害,兼愛無私,謂之仁;明是非,立可否,謂之義;進退有度,尊卑有分,謂之禮;擅殺生之柄,通塞之塗,權輕重之數,論得失之道,使遠近情偽必見於上,謂之術:凡此四者,治之本,道之用也,皆當設施,不可廢也。得其要,則天下安樂,法設而不用;不得其術,則主蔽於上,官亂於下。此事之情,屬統垂業之本也。

臣聞堯遭鴻水,使禹治之,未聞禹之有水也。若湯之旱,則桀之餘烈也。桀紂行惡,受天之罰;禹湯積德,以王天下。因此觀之,天德無私親,順之和起,逆之害生。此天文地理人事之紀。臣弘愚戇,不足以奉大對。

時對者百餘人,太常奏弘第居下。策奏,天子擢弘對為第一。召入見,容貌甚麗,拜為博士,待詔金馬門。

弘復上疏曰:「陛下有先聖之位而無先聖之名,有先聖之名而無先聖之吏,是以勢同而治異。先世之吏正,故其民篤;今世之吏邪,故其民薄。政弊而不行,令倦而不聽。夫使邪吏行弊政,用倦令治薄民,民不可得而化,此治之所以異也。臣聞周公旦治天下,期年而變,三年而化,五年而定。唯陛下之所志。」書奏,天子以冊書答曰:「問:弘稱周公之治,弘之材能自視孰與周公賢?」弘對曰:「愚臣淺薄,安敢比材於周公!雖然,愚心曉然見治道之可以然也。夫虎豹馬牛,禽獸之不可制者也,及其教馴服習之,至可牽持駕服,唯人之從。臣聞揉曲木者不累日,銷金石者不累月,夫人之於利害好惡,豈比禽獸木石之類哉?期年而變,臣弘尚竊遲之。」上異其言。

時方通西南夷,巴蜀苦之,詔使弘視焉。還奏事,盛毀西南夷無所用,上不聽。每朝會議,開陳其端,使人主自擇,不肯面折庭爭。於是上察其行慎厚,辯論有餘,習文法吏事,緣飾以儒術,上說之,一歲中至左內史。

弘奏事,有所不可,不肯庭辯。常與主爵都尉汲黯請間,黯先發之,弘推其後,上常說,所言皆聽,以此日益親貴。嘗與公卿約議,至上前,皆背其約以順上指。汲黯庭詰弘曰:「齊人多詐而無情,始為與臣等建此議,今皆背之,不忠。」上問弘,弘謝曰:「夫知臣者以臣為忠,不知臣者以臣為不忠。」上然弘言。左右幸臣每毀弘,上益厚遇之。

弘為人談笑多聞,常稱以為人主病不廣大,人臣病不儉節。養後母孝謹,後母卒,服喪三年。

為內史數年,遷御史大夫。時又東置蒼海,北築朔方之郡。弘數諫,以為罷弊中國以奉無用之地,願罷之。於是上乃使朱買臣等難弘置朔方之便。發十策,弘不得一。弘乃謝曰:「山東鄙人,不知其便若是,願罷西南夷、蒼海,專奉朔方。」上乃許之。

汲黯曰:「弘位在三公,奉祿甚多,然為布被,此詐也。」上問弘,弘謝曰:「有之。夫九卿與臣善者無過黯,然今日庭詰弘,誠中弘之病。夫以三公為布被,誠飾詐欲以釣名。且臣聞管仲相齊,有三歸,侈擬於君,桓公以霸,亦上僭於君。晏嬰相景公,食不重肉,妾不衣絲,齊國亦治,亦下比於民。今臣弘位為御史大夫,為布被,自九卿以下至於小吏無差,誠如黯言。且無黯,陛下安聞此言?」上以為有讓,愈益賢之。

元朔中,代薛澤為丞相。先是,漢常以列侯為丞相,唯弘無爵,上於是下詔曰:「朕嘉先聖之道,開廣門路,宣招四方之士,蓋古者任賢而序位,量能以授官,勞大者厥祿厚,德盛者獲爵尊,故武功以顯重,而文德以行褒。其以高成之平津鄉戶六百五十封丞相弘為平津侯。」其後以為故事,至丞相封,自弘始也。

時上方興功業,婁舉賢良。弘自見為舉首,起徒步,數年至宰相封侯,於是起客館,開東閣以延賢人,與參謀議。弘身食一肉,脫粟飯,故人賓客仰衣食,奉祿皆以給之,家無所餘。然其性意忌,外寬內深。諸常與弘有隙,無近遠,雖陽與善,後竟報其過。殺主父偃,徙董仲舒膠西,皆弘力也。

後淮南、衡山謀反,治黨與方急,弘病甚,自以為無功而封侯,居宰相位,宜佐明主填撫國家,使人由臣子之道。今諸侯有畔逆之計,此大臣奉職不稱也。恐病死無以塞責,乃上書曰:「臣聞天下通道五,所以行之者三。君臣、父子、夫婦、長幼、朋友之交,五者天下之通道也;仁、知、勇三者,所以行之也。故曰『好問近乎知,力行近乎仁,知恥近乎勇:知此三者,知所以自治;知所以自治,然後知所以治人。』未有不能自治而能治人者也。陛下躬孝弟,監三王,建周道,兼文武,招來四方之士,任賢序位,量能授官,將以厲百姓勸賢材也。今臣愚駑,無汗馬之勞,陛下下過意擢臣弘卒伍之中,封為列侯,致位三公。臣弘行能不足以稱,加有負薪之疾,恐先狗馬填溝壑,終無以報德塞責。願歸侯,乞骸骨,避賢者路。」上報曰:「古者賞有功,褒有德,守成文,遭遇右武,未有易此者也。朕夙夜庶幾,獲承至尊,懼不能寧,惟所與共為治者,君宜知之。蓋君子善善及後世,若茲行,常在朕躬。君不幸罹霜露之疾,何恙不已,乃上書歸侯,乞骸骨,是章朕之不德也。今事少閒,君其存精神,止念慮,輔助醫藥以自持。」因賜告牛酒雜帛。居數月,有瘳,視事。

凡為丞相御史六歲,年八十,終丞相位。其後李蔡、嚴青翟、趙周、石慶、公孫賀、劉屈镒繼踵為丞相。自蔡至慶,丞相府客館丘虛而已,至賀、屈镒時壞以為馬廄車庫奴婢室矣,唯慶以惇謹,復終相位,其餘盡伏誅云。

弘子度嗣侯,為山陽太守十餘歲,詔徵鉅野令史成詣公車,度留不遣,坐論為城旦。

元始中,修功臣後,下詔曰:「漢興以來,股肱在位,身行儉約,輕財重義,未有若公孫弘者也。位在宰相封侯,而為布被脫粟之飯,奉祿以給故人賓客,無有所餘,可謂減於制度,而率下篤俗者也,與內富厚而外為詭服以釣虛譽者殊科。夫表德章義,所以率世厲俗,聖王之制也。其賜弘後子孫之次見為適者,爵關內侯,食邑三百戶。」

卜式,河南人也。以田畜為事。有少弟,弟壯,式脫身出,獨取畜羊百餘,田宅財物盡與弟。式入山牧,十餘年,羊致千餘頭,買田宅。而弟盡破其產,式輒復分與弟者數矣。

時漢方事匈奴,式上書,願輸家財半助邊。上使使問式:「欲為官乎?」式曰:「自少牧羊,不習仕宦,不願也。」使者曰:「家豈有冤,欲言事乎?」式曰:「臣生與人亡所爭,邑人貧者貸之,不善者教之,所居,人皆從式,式何故見冤!」使者曰:「苟,子何欲?」式曰:「天子誅匈奴,愚以為賢者宜死節,有財者宜輸之,如此而匈奴可滅也。」使者以聞。上以語丞相弘。弘曰:「此非人情。不軌之臣不可以為化而亂法,願陛下勿許。」上不報,數歲乃罷式。式歸,復田牧。

歲餘,會渾邪等降,縣官費眾,倉府空,貧民大徙,皆卬給縣官,無以盡贍。式復持錢二十萬與河南太守,以給徙民。河南上富人助貧民者,上識式姓名,曰:「是固前欲輸其家半財助邊。」乃賜式外繇四百人,式又盡復與官。是時富豪皆爭匿財,唯式尤欲助費。上於是以式終長者,乃召拜式為中郎,賜爵左庶長,田十頃,布告天下,尊顯以風百姓。

初式不願為郎,上曰:「吾有羊在上林中,欲令子牧之。」式既為郎,布衣屮蹻而牧羊。歲餘,羊肥息。上過其羊所,善之。式曰:「非獨羊也,治民亦猶是矣。以時起居,惡者輒去,毋令敗群。」上奇其言,欲試使治民。拜式緱氏令,緱氏便之;遷成皋令,將漕最。上以式朴忠,拜為齊王太傅,轉為相。

會呂嘉反,式上書曰:「臣聞主媿臣死。群臣宜盡死節,其駑下者宜出財以佐軍,如是則強國不犯之道也。臣願與子男及臨菑習弩博昌習船者請行死之,以盡臣節。」上賢之,下詔曰:「朕聞報德以德,報怨以直。今天下不幸有事,郡縣諸侯未有奮繇直道者也。齊相雅行躬耕,隨牧蓄番,輒分昆弟,更造,不為利惑。日者北邊有興,上書助官。往年西河歲惡,率齊人入粟。今又首奮,雖未戰,可謂義形於內矣。其賜式爵關內侯,黃金四百斤,田十頃,布告天下,使明知之。」

元鼎中,徵式代石慶為御史大夫。式既在位,言郡國不便鹽鐵而船有算,可罷。上由是不說式。明年當封禪,式又不習文章,貶秩為太子太傅,以兒寬代之。式以壽終。

兒寬,千乘人也。治尚書,事歐陽生。以郡國選詣博士,受業孔安國。貧無資用,嘗為弟子都養。時行賃作,帶經而鉏,休息輒讀誦,其精如此。以射策為掌故,功次,補廷尉文學卒史。

寬為人溫良,有廉知自將,善屬文,然懦於武,口弗能發明也。時張湯為廷尉,廷尉府盡用文史法律之吏,而寬以儒生在其間,見謂不習事,不署曹,除為從史,之北地視畜數年。還至府,上畜簿,會廷尉時有疑奏,已再見卻矣,掾史莫知所為。寬為言其意,掾史因使寬為奏。奏成,讀之皆服,以白廷尉湯。湯大驚,召寬與語,乃奇其材,以為掾。上寬所作奏,即時得可。異日,湯見上。問曰:「前奏非俗吏所及,誰為之者?湯言兒寬。上曰:「吾固聞之久矣。」湯由是鄉學,以寬為奏讞掾,以古法義決疑獄,甚重之。及湯為御史大夫,以寬為掾,舉侍御史。見上,語經學。上說之,從問尚書一篇。擢為中大夫,遷左內史。

寬既治民,勸農業,緩刑罰,理獄訟,卑體下士,務在於得人心;擇用仁厚士,推情與下,不求名聲,吏民大信愛之。寬表奏開六輔渠,定水令以廣溉田。收租稅,時裁闊狹,與民相假貸,以故租多不入。後有軍發,左內史以負租課殿,當免。民聞當免,皆恐失之,大家牛車,小家擔負,輸租繈屬不絕,課更以最。上由此愈奇寬。

及議欲放古巡狩封禪之事,諸儒對者五十餘人,未能有所定。先是,司馬相如病死,有遺書,頌功德,言符瑞,足以封泰山。上奇其書,以問寬,寬對曰:「陛下躬發聖德,統楫群元,宗祀天地,薦禮百神,精神所鄉,徵兆必報,天地並應,符瑞昭明。其封泰山,禪梁父,昭姓考瑞,帝王之盛節也。然享薦之義,不著于經,以為封禪告成,合祛於天地神祇,祗戒精專以接神明。總百官之職,各稱事宜而為之節文。唯聖主所由,制定其當,非群臣之所能列。今將舉大事,優游數年,使群臣得人自盡,終莫能成。唯天子建中和之極,兼總條貫,金聲而玉振之,以順成天慶,垂萬世之基。」上然之,乃自制儀,采儒術以文焉。

既成,將用事,拜寬為御史大夫,從東封泰山,還登明堂。寬上壽曰:「臣聞三代改制,屬象相因。間者聖統廢絕,陛下發憤,合指天地,祖立明堂辟雍,宗祀泰一,六律五聲,幽贊聖意,神樂四合,各有方象,以丞嘉祀,為萬世則,天下幸甚。將建大元本瑞,登告岱宗,發祉闓門,以候景至。癸亥宗祀,日宣重光;上元甲子,肅邕永享。光輝充塞,天文粲然,充象日昭,報降符應。臣寬奉觴再拜,上千萬歲壽。」制曰:「敬舉君之觴。」

後太史令司馬遷等言:「曆紀壞廢,漢興未改正朔,宜可正。」上乃詔寬與遷等共定漢太初曆。語在律曆志。

初梁相褚大通五經,為博士,時寬為弟子。及御史大夫缺,徵褚大,大自以為得御史大夫。至洛陽,聞兒寬為之,褚大笑。及至,與寬議封禪於上前,大不能及,退而服曰:「上誠知人。」寬為御史大夫,以稱意任職,故久無有所匡諫於上,官屬易之。居位九歲,以官卒。

贊曰:公孫弘、卜式、兒寬皆以鴻漸之翼困於燕爵,遠跡羊豕之間,非遇其時,焉能致此位乎?是時,漢興六十餘載,海內艾安,府庫充實,而四夷未賓,制度多闕。上方欲用文武,求之如弗及,始以蒲輪迎枚生,見主父而歎息。群士慕嚮,異人並出。卜式拔於芻牧,弘羊擢於賈豎,衛青奮於奴僕,日磾出於降虜,斯亦曩時版築飯牛之明已。漢之得人,於茲為盛,儒雅則公孫弘、董仲舒、兒寬,篤行則石建、石慶,質直則汲黯、卜式,推賢則韓安國、鄭當時,定令則趙禹、張湯,文章則司馬遷、相如,滑稽則東方朔、枚皋,應對則嚴助、朱買臣,曆數則唐都、洛下閎,協律則李延年,運籌則桑弘羊,奉使則張騫、蘇武,將率則衛青、霍去病,受遺則霍光、金日磾,其餘不可勝紀。是以興造功業,制度遺文,後世莫及。孝宣承統,纂修洪業,亦講論六藝,招選茂異,而蕭望之、梁丘賀、夏侯勝、韋玄成、嚴彭祖、尹更始以儒術進,劉向、王褒以文章顯,將相則張安世、趙充國、魏相、丙吉、于定國、杜延年,治民則黃霸、王成、龔遂、鄭弘、召信臣、韓延壽、尹翁歸、趙廣漢、嚴延年、張敞之屬,皆有功跡見述於世。參其名臣,亦其次也。


\end{pinyinscope}