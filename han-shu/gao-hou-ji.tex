\article{高后紀}

\begin{pinyinscope}
高皇后呂氏,生惠帝。佐高祖定天下,父兄及高祖而侯者三人。惠帝即位,尊呂后為太后。太后立帝姊魯元公主女為皇后,無子,取後宮美人子名之以為太子。惠帝崩,太子立為皇帝,年幼,太后臨朝稱制,大赦天下。乃立兄子呂台、產、祿、台子通四人為王,封諸呂六人為列侯。語在外戚傳。

元年春正月,詔曰:「前日孝惠皇帝言欲除三族罪、妖言令,議未決而崩,今除之。」二月,賜民爵,戶一級。初置孝弟力田二千石者一人。夏五月丙申,趙王宮叢臺災。立孝惠後宮子強為淮陽王,不疑為恆山王,弘為襄城侯,朝為軹侯,武為壺關侯。秋,桃李華。

二年春,詔曰:「高皇帝匡飭天下,諸有功者皆受分地為列侯,萬民大安,莫不受休德。朕思念至於久遠而功名不著,亡以尊大誼,施後世。今欲差次列侯功以定朝位,臧于高廟,世世勿絕,嗣子各襲其功位。其與列侯議定奏之。」丞相臣平言:「謹與絳侯臣勃、曲周侯臣商、潁陰侯臣嬰、安國侯臣陵等議,列侯幸得賜餐錢奉邑,陛下加惠,以功次定朝位,臣請臧高廟。」奏可。春正月乙卯,地震,羌道、武都道山崩。夏六月丙戌晦,日有蝕之。秋七月,恆山王不疑薨。行八銖錢。

三年夏,江水、溢,流民四千餘家。秋,星晝見。

四年夏,少帝自知非皇后子,出怨言,皇太后幽之永巷。詔曰:「凡有天下治萬民者,蓋之如天,容之如地;上有驩心以使百姓,百姓欣然以事其上,驩欣交通而天下治。今皇帝疾久不已,乃失惑昏亂,不能繼嗣奉宗廟,守祭祀,不可屬天下。其議代之。」群臣皆曰:「皇太后為天下計,所以安宗廟社稷甚深。頓首奉詔。」五月丙辰,立恆山王弘為皇帝。

五年春,南粵王尉佗自稱南武帝。秋八月,淮陽王彊薨。九月,發河東、上黨騎屯北地。

六年春,星晝見。夏四月,赦天下。秩長陵令二千石。六月,城長陵。匈奴寇狄道,攻阿陽。行五分錢。

七年冬十二月,匈奴寇狄道,略二千餘人。春正月丁丑,趙王友幽死于邸。己丑晦,日有蝕之,既。以梁王呂產為相國,趙王祿為上將軍。立營陵侯劉澤為琅邪王。夏五月辛未,詔曰:「昭靈夫人,太上皇妃也;武哀侯、宣夫人,高皇帝兄姊也。號諡不稱,其議尊號。」丞相臣平等請尊昭靈夫人曰昭靈后,武哀侯曰武哀王,宣夫人曰昭哀后。六月,趙王恢自殺。秋九月,燕王建薨。南越侵盜長沙,遣隆慮侯灶將兵擊之。

八年春,封中謁者張釋卿為列侯。諸中官、宦者令丞皆賜爵關內侯,食邑。夏,江水、漢水溢,流萬餘家。

秋七月辛巳,皇太后崩于未央宮。遺詔賜諸侯王各千金,將相列侯下至郎吏各有差。大赦天下。

上將軍祿、相國產顓兵秉政,自知背高皇帝約,恐為大臣諸侯王所誅,因謀作亂。時齊悼惠王子朱虛侯章在京師,以祿女為婦,知其謀,乃使人告兄齊王,令發兵西。章欲與太尉勃、丞相平為內應,以誅諸呂。齊王遂發兵,又詐琅邪王澤發其國兵,并將而西。產、祿等遣大將軍灌嬰將兵擊之。嬰至滎陽,使人諭齊王與連和,待呂氏變而共誅之。

太尉勃與丞相平謀,以曲周侯酈商子寄與祿善,使人劫商令寄紿說祿曰:「高帝與呂后共定天下,劉氏所立九王,呂氏所立三王,皆大臣之議。事以布告諸侯王,諸侯王以為宜。今太后崩,帝少,足下不急之國守藩,乃為上將將兵留此,為大臣諸侯所疑。何不速歸將軍印,以兵屬太尉,請梁王亦歸相國印,與大臣盟而之國?齊兵必罷,大臣得安,足下高枕而王千里,此萬世之利也。」祿然其計,使人報產及諸呂老人。或以為不便,計猶豫未有所決。祿信寄,與俱出遊,過其姑呂嬃。嬃怒曰:「奴為將而棄軍,呂氏今無處矣!」乃悉出珠玉寶器散堂下,曰:「無為它人守也!」

八月庚申,平陽侯窋行御史大夫事,見相國產計事。郎中令賈壽使從齊來,因數產曰:「王不早之國,今雖欲行,尚可得邪?」具以灌嬰與齊楚合從狀告產。平陽侯窋聞其語,馳告丞相平、太尉勃。勃欲入北軍,不得入。襄平侯紀通尚符節,乃令持節矯內勃北軍。勃復令酈寄、典客劉揭說祿,曰:「帝使太尉守北軍,欲令足下之國,急歸將軍印辭去。不然,禍且起。」祿遂解印屬典客,而以兵授太尉勃。勃入軍門,行令軍中曰:「為呂氏右袒,為劉氏左袒。」軍皆左袒。勃遂將北軍。然尚有南軍,丞相平召朱虛侯章佐勃。勃令章監軍門,令平陽侯告衛尉,毋內相國產殿門。產不知祿已去北軍,入未央宮欲為亂。殿門弗內,徘徊往來。平陽侯馳語太尉勃,勃尚恐不勝,未敢誦言誅之,乃謂朱虛侯章曰:「急入宮衛帝。」章從勃請卒千人,入未央宮掖門,見產廷中。日餔時,遂擊產。產走。天大風,從官亂,莫敢鬥者。逐產,殺之郎中府吏舍廁中。

章已殺產,帝令謁者持節勞章。章欲奪節,謁者不肯,章乃從與載,因節信馳斬長樂衛尉呂更始。還入北軍,復報太尉勃。勃起拜賀章,曰:「所患獨產,今已誅,天下定矣。」辛酉,

殺呂祿,笞殺呂嬃。分部悉捕諸呂男女,無少長皆斬之。

大臣相與陰謀,以為少帝及三弟為王者皆非孝惠子,復共誅之,尊立文帝。語在周勃、高五王傳。

贊曰:孝惠、高后之時,海內得離戰國之苦,君臣俱欲無為,故惠帝拱己,高后女主制政,不出房闥,而天下晏然,刑罰罕用,民務稼穡,衣食滋殖。


\end{pinyinscope}