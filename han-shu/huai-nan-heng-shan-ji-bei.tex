\article{淮南衡山濟北王傳}

\begin{pinyinscope}
淮南厲王長,高帝少子也,其母故趙王張敖美人。高帝八年,從東垣過趙,趙王獻美人,厲王母也,幸,有身。趙王不敢內宮,為築外宮舍之。及貫高等謀反事覺,并逮治王,盡捕王母兄弟美人,繫之河內。厲王母亦繫,告吏曰:「日得幸上,有子。」吏以聞,上方怒趙,未及理厲王母,厲王母弟趙兼因辟陽侯言呂后,呂后妒,不肯白,辟陽侯不強爭。厲王母已生厲王,恚。即自殺。吏奉厲王詣上,上悔,令呂后母之,而葬其母真定。真定,厲王母家縣也。

十一年,淮南王布反,上自將擊滅布,即立子長為淮南王。王早失母,常附呂后,孝惠、呂后時以故得幸無患,然常心怨辟陽侯,不敢發。及孝文初即位,自以為最親,驕蹇,數不奉法。上寬赦之。三年,入朝,甚橫。從上入苑獵,與上同輦,常謂上「大兄」。厲王有材力,力扛鼎,乃往請辟陽侯。辟陽侯出見之,即自袖金椎椎之,命從者刑之。馳詣闕下,肉袒而謝曰:「臣母不當坐趙時事,辟陽侯力能得之呂后,不爭,罪一也。趙王如意子母無罪,呂后殺之,辟陽侯不爭,罪二也。呂后王諸呂,欲以危劉氏,辟陽侯不爭,罪三也。臣謹為天下誅賊,報母之仇,伏闕下請罪。」文帝傷其志,為親故不治,赦之。

當是時,自薄太后及太子諸大臣皆憚厲王。厲王以此歸國益恣,不用漢法,出入警蹕,稱制,自作法令,數上書不遜順。文帝重自切責之。時帝舅薄昭為將軍,尊重,上令昭予厲王書諫數之,曰:

竊聞大王剛直而勇,慈惠而厚,貞信多斷,是天以聖人之資奉大王也甚盛,不可不察。今大王所行,不稱天資。皇帝初即位,易侯邑在淮南者,大王不肯。皇帝卒易之,使大王得三縣之實,甚厚。大王以未嘗與皇帝相見,求入朝見,未畢昆弟之歡,而殺列侯以自為名。皇帝不使吏與其間,赦大王,甚厚。漢法,二千石缺,輒言漢補,大王逐漢所置,而請自置相、二千石。皇帝骫天下正法而許大王,甚厚。大王欲屬國為布衣,守冢真定。皇帝不許,使大王毋失南面之尊,甚厚。大王宜日夜奉法度,修貢職,以稱皇帝之厚德,今乃輕言恣行,以負謗於天下,甚非計也。

夫大王以千里為宅居,以萬民為臣妾,此高皇帝之厚德也。高帝蒙霜露,沬風雨,赴矢石,野戰次城,身被創痍,以為子孫成萬世之業,艱難危苦甚矣。大王不思先帝之艱苦,日夜怵惕,修身正行,養犧牲,豐潔粢盛,奉祭祀,以無忘先帝之功德,而欲屬國為布衣,甚過。且夫貪讓國土之名,輕廢先帝之業,不可以言孝。父為之基,而不能守,不賢。不求守長陵,而求之真定,先母後父,不誼。數逆天子之令,不順。言節行以高兄,無禮。幸臣有罪,大者立斷,小者肉刑,不仁。貴布衣一劍之任,賤王侯之位,不知。不好學問大道,觸情妄行,不詳。此八者,危亡之路也,而大王行之。棄南面之位,奮諸、賁之勇,常出入危亡之路,臣之所見,高皇帝之神必不廟食於大王之手,明白。

昔者,周公誅管叔,放蔡叔,以安周;齊桓殺其弟,以反國;秦始皇殺兩弟,遷其母,以安秦;頃王亡代,高帝奪之國,以便事;濟北舉兵,皇帝誅之,以安漢。故周、齊行之於古,秦、漢用之於今,大王不察古今之所以安國便事,而欲以親戚之意望於太上,不可得也。亡之諸侯,游宦事人,及舍匿者,論皆有法。其在王所,吏主者坐。今諸侯子為吏者,御史主;為軍吏者,中尉主;客出入殿門者,衛尉大行主;諸從蠻夷來歸誼及以亡名數自古者,內史縣令主。相欲委下吏,無與其禍,不可得也。王若不改,漢繫大王邸,論相以下,為之奈何?夫墮父大業,退為布衣所哀,幸臣皆伏法而誅,為天下笑,以羞先帝之德,甚為大王不取也。

宜急改操易行,上書謝罪,曰:「臣不幸早失先帝,少孤,呂氏之世,未嘗忘死。陛下即位,臣怙恩德驕盈,行多不軌。追念罪過,恐懼,伏地待誅不敢起。」皇帝聞之必喜。大王昆弟歡欣於上,群臣皆得延壽於下;上下得宜,海內常安。願孰計而疾行之。行之有疑,禍如發矢,不可追已。

王得書不說。六年,令男子但等七十人與棘蒲侯柴武太子奇謀,以輦車四十乘反谷口,令人使閩越、匈奴。事覺,治之,乃使使召淮南王。

王至長安,丞相張蒼,典客馮敬行御史大夫事,與宗正、廷尉雜奏:「長廢先帝法,不聽天子詔,居處無度,為黃屋蓋儗天子,擅為法令,不用漢法。及所置吏,以其郎中春為丞相,收聚漢諸侯人及有罪亡者,匿與居,為治家室,賜與財物爵祿田宅,爵或至關內侯,奉以二千石所當得。大夫但、士伍開章等七十人與棘蒲侯太子奇謀反,欲以危宗廟社稷,謀使閩越及匈奴發其兵。事覺,長安尉奇等往捕開章,長匿不予,與故中尉蕑忌謀,殺以閉口,為棺槨衣衾,葬之肥陵,謾吏曰『不知安在』。又陽聚土,樹表其上曰『開章死,葬此下』。及長身自賊殺無罪者一人;令吏論殺無罪者六人;為亡命棄市詐捕命者以除罪;擅罪人,無告劾繫治城旦以上十四人;敗免罪人死罪十八人,城旦舂以下五十八人;賜人爵關內侯以下九十四人。前日長病,陛下心憂之,使使者賜棗脯,長不肯見拜使者。南海民處廬江界中者反,淮南吏卒擊之。陛下遣使者齎帛五十匹,以賜吏卒勞苦者。長不欲受賜,謾曰『無勞苦者』。南海王織上書獻璧帛皇帝,忌擅燔其書,不以聞。吏請召治忌,長不遣,謾曰『忌病』。長所犯不軌,當棄市,臣請論如法。」

制曰:「朕不忍置法於王,其與列侯吏二千石議。」列侯吏二千石臣嬰等四十三人議,皆曰:「宜論如法。」制曰:「其赦長死罪,廢勿王。」有司奏:「請處蜀嚴道邛郵,遣其子、子母從居,縣為築蓋家室,皆日三食,給薪菜鹽炊食器席蓐。」制曰:「食長,給肉日五斤,酒二斗。令故美人材人得幸者十人從居。」於是盡誅所與謀者。乃遣長,載以輜車,令縣次傳。

爰盎諫曰:「上素驕淮南王,不為置嚴相傅,以故至此。且淮南王為人剛,今暴摧折之,臣恐其逢霧露病死,陛下有殺弟之名,奈何!」上曰:「吾特苦之耳,令復之。」淮南王謂侍者曰:「誰謂乃公勇者?吾以驕不聞過,故至此。」乃不食而死。縣傳者不敢發車封。至雍,雍令發之,以死聞。上悲哭,謂爰盎曰:「吾不從公言,卒亡淮南王。」盎曰:「淮南王不可奈何,願陛下自寬。」上曰:「為之奈何?」曰:「獨斬丞相、御史以謝天下乃可。」上即令丞相、御史逮諸縣傳淮南王不發封餽侍者,皆棄市。乃以列侯葬淮南王于雍,置守冢三十家。

孝文八年,憐淮南王,王有子四人。年皆七八歲,乃封子安為阜陵侯,子勃為安陽侯,子賜為陽周侯,子良為東城侯。

十二年,民有作歌歌淮南王曰:「一尺布,尚可縫;一斗粟,尚可舂。兄弟二人,不相容!」上聞之曰:「昔堯舜放逐骨肉,周公殺管蔡,天下稱聖,不以私害公。天下豈以為我貪淮南地邪?」乃徙城陽王王淮南故地,而追尊諡淮南王為厲王,置園如諸侯儀。

十六年,上憐淮南王廢法不軌,自使失國早夭,乃徙淮南王喜復王故城陽,而立厲王三子王淮南故地,三分之:阜陵侯安為淮南王,安陽侯勃為衡山王,陽周侯賜為廬江王。東城侯良前薨,無後。

孝景三年,吳楚七國反,吳使者至淮南,淮南王欲發兵應之。其相曰:「王必欲應吳,臣願為將。」王乃屬之。相已將兵,因城守,不聽王而為漢。漢亦使曲城侯將兵救淮南,淮南以故得完。吳使者至廬江,廬江王不應,而往來使越;至衡山,衡山王堅守無二心。孝景四年,吳楚已破,衡山王朝,上以為貞信,乃勞苦之曰:「南方卑溼。」徙王王於濟北以褒之。及薨,遂賜諡為貞王。廬江王以邊越,數使使相交,徙為衡山王,王江北。

淮南王安為人好書,鼓琴,不喜弋獵狗馬馳騁,亦欲以行陰德拊循百姓,流名譽。招致賓客方術之士數千人,作為內書二十一篇,外書甚眾,又有中篇八卷,言神仙黃白之術,亦二十餘萬言。時武帝方好藝文,以安屬為諸父,辯博善為文辭,甚尊重之。每為報書及賜,常召司馬相如等視草乃遣。初,安入朝,獻所作內篇,新出,上愛祕之。使為離騷傳,旦受詔,日食時上。又獻頌德及長安都國頌。每宴見,談說得失及方技賦頌,昏莫然後罷。

安初入朝,雅善太尉武安侯,武安侯迎之霸上,與語曰:「方今上無太子,王親高皇帝孫,行仁義,天下莫不聞。宮車一日晏駕,非王尚誰立者!」淮南王大喜,厚遺武安侯寶賂。其群臣賓客,江淮間多輕薄,以厲王遷死感激安。建元六年,彗星見,淮南王心怪之。或說王曰:「先吳軍時,彗星出,長數尺,然尚流血千里。今彗星竟天,天下兵當大起。」王心以為上無太子,天下有變,諸侯並爭,愈益治攻戰具,積金錢賂遺郡國。遊士妄作妖言阿諛王,王喜,多賜予之。

王有女陵,慧有口。王愛陵,多予金錢,為中詗長安,約結上左右。元朔二年,上賜淮南王几杖,不朝。后荼愛幸,生子遷為太子,取皇太后外孫修成君女為太子妃。王謀為反具,畏太子妃知而內泄事,乃與太子謀,令詐不愛,三月不同席。王陽怒太子,閉使與妃同內,終不近妃。妃求去,王乃上書謝歸之。后荼、太子遷及女陵擅國權,奪民田宅,妄致繫人。

太子學用劍,自以為人莫及,聞郎中雷被巧,召與戲。被壹再辭讓,誤中太子。太子怒,被恐。此時有欲從軍者輒詣長安,被即願奮擊匈奴。太子數惡被,王使郎中令斥免,欲以禁後。元朔五年,被遂亡之長安,上書自明。事下廷尉、河南。河南治,逮淮南太子。王、王后計欲毋遣太子,遂發兵。計未定,猶與十餘日。會有詔即訊太子,淮南相怒壽春丞留太子逮不遣,劾不敬。王請相,相不聽。王使人上書告相,事下廷尉治。從跡連王,王使人候司。漢公卿請逮捕治王,王恐,欲發兵。太子遷謀曰:「漢使即逮王,令人衣衛士衣,持戟居王旁,有非是者,即刺殺之,臣亦使人刺殺淮南中尉,乃舉兵,未晚也。」是時上不許公卿,而遣漢中尉宏即訊驗王。王視漢中尉顏色和,問斥雷被事耳,自度無何,不發。中尉還,以聞。公卿治者曰:「淮南王安雍閼求奮擊匈奴者雷被等,格明詔,當棄市。」詔不許。請廢勿王,上不許。請削五縣,可二縣。使中尉宏赦其罪,罰以削地。中尉入淮南界,宣言赦王。王初聞公卿請誅之,未知得削地,聞漢使來,恐其捕之,乃與太子謀如前計。中尉至,即賀王,王以故不發。其後自傷曰:「吾行仁義見削地,寡人甚恥之。」為反謀益甚。諸使者道長安來,為妄言,言上無男,即喜;言漢廷治,有男,即怒,以為妄言,非也。

日夜與左吳等按輿地圖,部署兵所從入。王曰:「上無太子,宮車即晏駕,大臣必徵膠東王,不即常山王,諸侯並爭,吾可以無備乎!且吾高帝孫,親行仁義,陛下遇我厚,吾能忍之;萬世之後,吾寧能北面事豎子乎!」

王有孽子不害,最長,王不愛,后、太子皆不以為子兄數。不害子建,材高有氣,常怨望太子不省其父。時諸侯皆得分子弟為侯,淮南王有兩子,一子為太子,而建父不得為侯。陰結交,欲害太子,以其父代之。太子知之,數捕繫笞建。建具知太子之欲謀殺漢中尉,即使所善壽春嚴正上書天子曰:「毒藥苦口利病,忠言逆耳利行。今淮南王孫建材能高,淮南王后荼、荼子遷常疾害建。建父不害無罪,擅數繫,欲殺之。今建在,可徵問,具知淮南王陰事。」書既聞,上以其事下廷尉、河南治。是歲元朔六年也。故辟陽侯孫審卿善丞相公孫弘,怨淮南厲王殺其大父,陰求淮南事而搆之於弘。弘乃疑淮南有畔逆計,深探其獄。河南治建,辭引太子及黨與。

初,王數以舉兵謀問伍被,被常諫之,以吳楚七國為效。王引陳勝、吳廣,被復言形勢不同,必敗亡。及建見治,王恐國陰事泄,欲發,復問被,被為言發兵權變。語在被傳。於是王銳欲發,乃令官奴入宮中,作皇帝璽,丞相、御史大夫、將軍、吏中二千石、都官令、丞印,及旁近郡太守、都尉印,漢使節法冠。欲如伍被計,使人為得罪而西,事大將軍、丞相;一日發兵,即刺大將軍衛青,而說丞相弘下之,如發蒙耳。欲發國中兵,恐相、二千石不聽,王乃與伍被謀,為失火宮中,相、二千石救火,因殺之。又欲令人衣求盜衣,持羽檄從南方來,呼言曰「南越兵入」,欲因以發兵。乃使人之廬江、會稽為求盜,未決。

廷尉以建辭連太子遷聞,上遣廷尉監與淮南中尉逮捕太子。至,淮南王聞,與太子謀召相、二千石,欲殺而發兵。召相,相至;內史以出為解。中尉曰:「臣受詔使,不得見王。」王念獨殺相而內史、中尉不來,無益也,即罷相。計猶與未決。太子念所坐者謀殺漢中尉,所與謀殺者已死,以為口絕,乃謂王曰:「群臣可用者皆前繫,今無足與舉事者。王以非時發,恐無功,臣願會逮。」王亦愈欲休,即許太子。太子自刑,不殊。伍被自詣吏,具告與淮南王謀反。吏因捕太子、王后,圍王宮,盡捕王賓客在國中者,索得反具以聞。上下公卿治,所連引與淮南王謀反列侯、二千石、豪桀數千人,皆以罪輕重受誅。

衡山王賜,淮南王弟,當坐收。有司請逮捕衡山王,上曰:「諸侯各以其國為本,不當相坐。與諸侯王列侯議。」趙王彭祖、列侯讓等四十三人皆曰:「淮南王安大逆無道,謀反明白,當伏誅。」膠西王端議曰:「安廢法度,行邪辟,有詐偽心,以亂天下,營惑百姓,背畔宗廟,妄作妖言。春秋曰『臣毋將,將而誅。』安罪重於將,謀反形已定。臣端所見其書印圖及它逆亡道事驗明白,當伏法。論國吏二百石以上及比者,宗室近幸臣不在法中者,不能相教,皆當免,削爵為士伍,毋得官為吏。其非吏,它贖死金二斤八兩,以章安之罪,使天下明知臣子之道,毋敢復有邪僻背畔之意。」丞相弘、廷尉湯等以聞,上使宗正以符節治王。未至,安自刑殺。后、太子諸所與謀皆收夷。國除為九江郡。

衡山王賜,后乘舒生子三人,長男爽為太子,次女無采,少男孝。姬徐來生子男女四人,美人厥姬生子二人。淮南、衡山相責望禮節,間不相能。衡山王聞淮南王作為畔逆具,亦心結賓客以應之,恐為所并。

元光六年入朝,謁者衛慶有方術,欲上書事天子,王怒,故劾慶死罪,強榜服之。內史以為非是,卻其獄。王使人上書告內史,內史治,言王不直。又數侵奪人田,壞人冢以為田。有司請逮治衡山王,上不許,為置吏二百石以上。衡山王以此恚,與奚慈、張廣昌謀,求能為兵法候星氣者,日夜縱臾王謀反事。

后乘舒死,立徐來為后,厥姬俱幸。兩人相妒,厥姬乃惡徐來於太子,曰「徐來使婢蠱殺太子母。」太子心怨徐來。徐來兄至衡山,太子與飲,以刃刑傷之。后以此怨太子,數惡之於王。女弟無采嫁,棄歸,與客姦。太子數以數讓之,無采怒,不與太子通。后聞之,即善遇無采及孝。孝少失母,附后,后以計愛之,與共毀太子,王以故數繫笞太子。元朔四年中,人有賊傷后假母者,王疑太子使人傷之,笞太子。後王病,太子時稱病不侍。孝、無采惡太子:「實不病,自言,有喜色。」王於是大怒,欲廢太子而立弟孝。后知王決廢太子,又欲并廢孝。后有侍者善舞,王幸之,后欲令與孝亂以污之,欲并廢二子而以己子廣代之。太子知之,念后數惡己無已時,欲與亂以止其口。后飲太子,太子前為壽,因據后股求與臥。后怒,以告王。王乃召,欲縛笞之。太子知王常欲廢己而立孝,乃謂王曰:「孝與王御者姦,無采與奴姦,王強食,請上書。」即背王去。王使人止之,莫能禁,王乃自追捕太子。太子妄惡言,王械繫宮中。

孝日益以親幸。王奇孝材能,乃佩之王印,號曰將軍,今居外家,多給金錢,招致賓客。賓客來者,微知淮南、衡山有逆計,皆將養勸之。王乃使孝客江都人枚赫、陳喜作輣車鍛矢,刻天子璽,將、相、軍吏印。王日夜求壯士如周丘等,數稱引吳楚反時計畫約束。衡山王非敢效淮南王求即天子位,畏淮南起并其國,以為淮南已西,發兵定江淮間而有之,望如是。

元朔五年秋,當朝,六年,過淮南。淮南王乃昆弟語,除前隙,約束反具。衡山王即上書謝病,上賜不朝。乃使人上書請廢太子爽,立孝為太子。爽聞,即使所善白嬴之長安上書,言衡山王與子謀逆,言孝作兵車鍛矢,與王御者姦。至長安未及上書,即吏捕嬴,以淮南事繫。王聞之,恐其言國陰事,即上書告太子,以為不道。事下沛郡治。元狩元年冬,有司求捕與淮南王謀反者,得陳喜於孝家。吏劾孝首匿喜。孝以為陳喜雅數與王計反,恐其發之,聞律先自告除其罪,又疑太子使白嬴上書發其事,即先自告所與謀反者枚赫、陳喜等。廷尉治,事驗,請逮捕衡山王治。上曰:「勿捕。」遣中尉安、大行息即問王,王具以情實對。吏皆圍王宮守之。中尉、大行還,以聞。公卿請遣宗正、大行與沛郡雜治王。王聞,即自殺。孝先自告反,告除其罪。孝坐與王御婢姦,及后徐來坐蠱前后乘舒,及太子爽坐告王父不孝,皆棄市。諸坐與王謀反者皆誅。國除為郡。

濟北貞王勃者,景帝四年徙。徙二年,因前王衡山,凡十四年薨。子式王胡嗣,五十四年薨。子寬嗣。十二年,寬坐與父式王后光、姬孝兒姦,誖人倫,又祠祭祝詛上,有司請誅。上遣大鴻臚利召王,王以刃自剄死。國除為北安縣,屬泰山郡。

贊曰:《詩》云「戎狄是膺,荊舒是懲」,信哉是言也!淮南、衡山親為骨肉,疆土千里,列在諸侯,不務遵蕃臣職,以丞輔天子,而剸懷邪辟之計,謀為畔逆,仍父子再亡國,各不終其身。此非獨王也,亦其俗薄,臣下漸靡使然。夫荊楚剽輕,好作亂,乃自古記之矣。


\end{pinyinscope}