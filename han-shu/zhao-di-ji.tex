\article{昭帝紀}

\begin{pinyinscope}
孝昭皇帝,武帝少子也。母曰趙婕妤,本以有奇異得幸,及生帝,亦奇異。語在外戚傳。武帝末,戾太子敗,燕王旦、廣陵王胥行驕嫚,後元二年二月上疾病,遂立昭帝為太子,年八歲。以侍中奉車都尉霍光為大司馬大將軍,受遺詔輔少主。明日,武帝崩。戊辰,太子即皇帝位,謁高廟。帝姊鄂邑公主益湯沐邑,為長公主,共養省中。大將軍光秉政,領尚書事,車騎將軍金日磾、左將軍上官桀副焉。

夏六月,赦天下。

秋七月,有星孛于東方。

濟北王寬有罪,自殺。

賜長公主及宗室昆弟各有差。追尊趙婕妤為皇太后,起雲陵。

冬,匈奴入朔方,殺略吏民。發軍屯西河,左將軍桀行北邊。

始元元年春二月,黃鵠下建章宮太液池中。公卿上壽。賜諸侯王、列侯、宗室金錢各有差。

己亥,上耕于鉤盾弄田。

益封燕王、廣陵王及鄂邑長公主各萬三千戶。

夏,為太后起園廟雲陵。

益州廉頭、姑繒、牂柯談指、同並二十四邑皆反。遣水衡都尉呂破胡募吏民及發犍為、蜀郡奔命擊益州,大破之。

有司請河內屬冀州,河東屬并州。

秋七月,赦天下,賜民百戶牛酒。大雨,渭橋絕。

八月,齊孝王孫劉澤謀反,欲殺青州刺史雋不疑,發覺,皆伏誅。遷不疑為京兆尹,賜錢百萬。

九月丙子,車騎將軍日磾薨。

閏月,遣故廷尉王平等五人持節行郡國,舉賢良,問民所疾苦、冤、失職者。

冬,無冰。

二年春正月,大將軍光、左將軍桀皆以前捕斬反虜重合侯馬通功封,光為博陸侯,桀為安陽侯。

以宗室毋在位者,舉茂才劉辟彊、劉長樂皆為光祿大夫,辟彊守長樂衛尉。

三月,遣使者振貸貧民毋種、食者。秋八月,詔曰:「往年災害多,今年蠶麥傷,所振貸種、食勿收責,毋令民出今年田租。」

冬,發習戰射士詣朔方,調故吏將屯田張掖郡。

三年春二月,有星孛于西北。

秋,募民徙雲陵,賜錢田宅。

冬十月,鳳皇集東海,遣使者祠其處。

十一月壬辰朔,日有蝕之。

四年春三月甲寅,立皇后上官氏。赦天下。辭訟在後二年前,皆勿聽治。夏六月,皇后見高廟。賜長公主、丞相、將軍、列侯、中二千石以下及郎吏宗室錢帛各有差。

徙三輔富人雲陵,賜錢,戶十萬。

秋七月,詔曰:「比歲不登,民匱於食,流庸未盡還,往時令民共出馬,其止勿出。諸給中都官者,且減之。」

冬,遣大鴻臚田廣明擊益州。

廷尉李种坐故縱死罪棄市。

五年春正月,追尊皇太后父為順成侯。

夏陽男子張延年詣北闕,自稱衛太子,誣罔,要斬。

夏,罷天下亭母馬及馬弩關。

六月,封皇后父驃騎將軍上官安為桑樂侯。

詔曰:「朕以眇身獲保宗廟,戰戰栗栗,夙興夜寐,修古帝王之事,通保傅,傳孝經、論語、尚書,未云有明。其令三輔、太常舉賢良各二人,郡國文學高第各一人。賜中二千石以下至吏民爵各有差。」

罷儋耳、真番郡。

秋,大鴻臚廣明、軍正王平擊益州,斬首捕虜三萬餘人,獲畜產五萬餘頭。

六月春正月,上耕于上林。

二月,詔有司問郡國所舉賢良文學民所疾苦。議罷鹽鐵榷酤。

栘中監蘇武前使匈奴,留單于庭十九歲乃還,奉使全節,以武為典屬國,賜錢百萬。

夏,旱,大雩,大得舉火。

秋七月,罷榷酤官,令民得以律占租,賣酒升四錢。以邊塞闊遠,取天水、隴西、張掖郡各二縣置金城郡。

詔曰:「鉤町侯毋波率其君長人民擊反者,斬首捕虜有功。其立毋波為鉤町王。大鴻臚廣明將率有功,賜爵關內侯,食邑。」

元鳳元年春,長公主共養勞苦,復以藍田益長公主湯沐邑。

泗水戴王前薨,以毋嗣,國除。後宮有遺腹子煖,相、內史不奏言,上聞而憐之,立煖為泗水王。相、內史皆下獄。

三月,賜郡國所選有行義者涿郡韓福等五人帛,人五十匹,遣歸。詔曰:「朕閔勞以官職之事,其務修孝弟以教鄉里。令郡縣常以正月賜羊酒。有不幸者賜衣被一襲,祠以中牢。」

武都氐人反,遣執金吾馬適建、龍镪侯韓增、大鴻臚廣明將三輔、太常徒,皆免刑擊之。

夏六月,赦天下。

秋七月乙亥晦,日有蝕之,既。

八月,改始元為元鳳。

九月,鄂邑長公主、燕王旦與左將軍上官桀、桀子票騎將軍安、御史大夫桑弘羊皆謀反,伏誅。初,桀、安父子與大將軍光爭權,欲害之,詐使人為燕王旦上書言光罪。時上年十四,覺其詐。後有譖光者,上輒怒曰:「大將軍國家忠臣,先帝所屬,敢有譖毀者,坐之。」光由是得盡忠。語在燕王、霍光傳。

冬十月,詔曰:「左將軍安陽侯桀、票騎將軍桑樂侯安、御史大夫弘羊皆數以邪枉干輔政,大將軍不聽,而懷怨望,與燕王通謀,置驛往來相約結。燕王遣壽西長、孫縱之等賂遺長公主、丁外人、謁者杜延年、大將軍長史公孫遺等,交通私書,共謀令長公主置酒,伏兵殺大將軍光,徵立燕王為天子,大逆毋道。故稻田使者燕倉先發覺,以告大司農敞,敞告諫大夫延年,延年以聞。丞相徵事任宮手捕斬桀,丞相少史王壽誘將安入府門,皆已伏誅,吏民得以安。封延年、倉、宮、壽皆為列侯。」又曰:「燕王迷惑失道,前與齊王子劉澤等為逆,抑而不揚,望王反道自新,今乃與長公主及左將軍桀等謀危宗廟。王及公主皆自伏辜。其赦王太子建、公主子文信及宗室子與燕王、上官桀等謀反父母同產當坐者,皆免為庶人。其吏為桀等所詿誤,未發覺在吏者,除其罪。」

二年夏四月,上自建章宮徙未央宮,大置酒。賜郎從官帛,及宗室子錢,人二十萬。吏民獻牛酒者賜帛,人一匹。

六月,赦天下。詔曰:「朕閔百姓未贍,前年減漕三百萬石。頗省乘輿馬及菀馬,以補邊郡三輔傳馬。其令郡國毋斂今年馬口錢,三輔、太常郡得以叔粟當賦。」

三年春正月,泰山有大石自起立,上林有柳樹枯僵自起生。

罷中牟苑賦貧民。詔曰:「乃者民被水災,頗匱於食,朕虛倉廩,使使者振困乏。其止四年毋漕。三年以前所振貸,非丞相御史所請,邊郡受牛者勿收責。」

夏四月,少府徐仁、廷尉王平、左馮翊賈勝胡皆坐縱反者,仁自殺,平、勝胡皆要斬。

冬,遼東烏桓反,以中郎將范明友為度遼將軍,將北邊七郡郡二千騎擊之。

四年春正月丁亥,帝加元服,見于高廟。賜諸侯王、丞相、大將軍、列侯、宗室下至吏民金帛牛酒各有差。賜中二千石以下及天下民爵。毋收四年、五年口賦。三年以前逋更賦未入者,皆勿收。令天下酺五日。

甲戌,丞相千秋薨。

夏四月,詔曰:「度遼將軍明友前以羌騎校尉將羌王侯君長以下擊益州反虜,後復率擊武都反氐,今破烏桓,斬虜獲生,有功。其封明友為平陵侯。平樂監傅介子持節使,誅斬樓蘭王安,歸首縣北闕,封義陽侯。」

五月丁丑,孝文廟正殿火,上及群臣皆素服。發中二千石將五校作治,六日成。太常及廟令丞郎吏皆劾大不敬,會赦,太常轑陽侯德免為庶人。

六月,赦天下。

五年春正月,廣陵王來朝,益國萬一千戶,賜錢二千萬,黃金二百斤,劍二,安車一,乘馬二駟。

夏,大旱。

六月,發三輔及郡國惡少年吏有告劾亡者,屯遼東。

秋,罷象郡,分屬鬱林、牂柯。

冬十一月,大雷。

十二月庚戌,丞相訢薨。

六年春正月,募郡國徒築遼東玄菟城。夏,赦天下。詔曰:「夫穀賤傷農,今三輔、太常穀減賤,其令以叔粟當今年賦。」

右將軍張安世宿衛忠謹,封富平侯。

烏桓復犯塞,遣度遼將軍范明友擊之。

元平元年春二月,詔曰:「天下以農桑為本。日者省用,罷不急官,減外繇,耕桑者益眾,而百姓未能家給,朕甚愍焉。其減口賦錢。」有司奏請減什三,上許之。

甲申,晨有流星,大如月,眾星皆隨西行。

夏四月癸未,帝崩于未央宮。六月壬申,葬平陵。

贊曰:昔周成以孺子繼統,而有管、蔡四國流言之變。孝昭幼年即位,亦有燕、盍、上官逆亂之謀。成王不疑周公,孝昭委任霍光,各因其時以成名,大矣哉!承孝武奢侈餘敝師旅之後,海內虛耗,戶口減半,光知時務之要,輕繇薄賦,與民休息。至始元、元鳳之間,匈奴和親,百姓充實。舉賢良文學,問民所疾苦,議鹽鐵而罷榷酤,尊號曰「昭」,不亦宜乎!


\end{pinyinscope}