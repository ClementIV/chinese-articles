\article{元后傳}

\begin{pinyinscope}
孝元皇后,王莽之姑也。莽自謂黃帝之後,其自本曰:黃帝姓姚氏,八世生虞舜。舜起媯汭,以媯為姓。至周武王封舜後媯滿於陳,是為胡公,十三世生完。完字敬仲,奔齊,齊桓公以為卿,姓田氏。十一世,田和有齊國,三世稱王,至王建為秦所滅。項羽起,封建孫安為濟北王。至漢興,安失國,齊人謂之「王家」,因以為氏。

文、景間,安孫遂字伯紀,處東平陵,生賀,字翁孺。為武帝繡衣御史,逐捕魏郡群盜堅盧等黨與,及吏畏懦逗遛當坐者,翁孺皆縱不誅。它部御史暴勝之等奏殺二千石,誅千石以下,及通行飲食坐連及者,大部至斬萬餘人,語見酷吏傳。翁孺以奉使不稱免,嘆曰:「吾聞活千人有封子孫,吾所活者萬餘人,後世其興乎!」

翁孺既免,而與東平陵終氏為怨,乃徙魏郡元城委粟里,為三老,魏郡人德之。元城建公曰:「昔春秋沙麓崩,晉史卜之,曰:『陰為陽雄,土火相乘,故有沙麓崩。後六百四十五年,宜有聖女興。』其齊田乎!今王翁孺徙,正直其地,日月當之。元城郭東有五鹿之虛,即沙鹿地也。後八十年,當有貴女興天下」云。

王翁孺生禁,字稚君,少學法律長安,為廷尉史。本始三年,生女政君,即元后也。禁有大志,不修廉隅,好酒色,多取傍妻,凡有四女八男:長女君俠,次即元后政君,次君力,次君弟;長男鳳孝卿,次曼元卿,譚子元,崇少子,商子夏,立子叔,根稚卿,逢時季卿。唯鳳、崇與元后政君同母。母,適妻,魏郡李氏女也。後以妒去,更嫁為河內苟賓妻。

初,李親任政君在身,夢月入其懷。及壯大,婉順得婦人道。嘗許嫁未行,所許者死。後東平王聘政君為姬,未入,王薨。禁獨怪之,使卜數者相政君,「當大貴,不可言。」禁心以為然,乃教書,學鼓琴。五鳳中,獻政君,年十八矣,入掖庭為家人子。

歲餘,會皇太子所愛幸司馬良娣病,且死,謂太子曰:「妾死非天命,乃諸娣妾良人更祝詛殺我。」太子憐之,且以為然。及司馬良娣死,太子悲恚發病,忽忽不樂,因以過怒諸娣妾,莫得進見者。久之,宣帝聞太子恨過諸娣妾,欲順適其意,乃令皇后擇後宮家人子可以虞侍太子者,政君與在其中。及太子朝,皇后乃見政君等五人,微令旁長御問知太子所欲。太子殊無意於五人者,不得已於皇后,彊應曰:「此中一人可。」是時政君坐近太子,又獨衣絳緣諸于,長御即以為。皇后使侍中杜輔、掖庭令濁賢交送政君太子宮,見丙殿。得御幸,有身。先是者,太子後宮娣妾以十數,御幸久者七八年,莫有子,及王妃壹幸而有身。甘露三年,生成帝於甲館畫堂,為世適皇孫。宣帝愛之,自名曰驁,字太孫,常置左右。

後三年,宣帝崩,太子即位,是為孝元帝。立太孫為太子,以母王妃為婕妤,封父禁為陽平侯。後三日,婕妤立為皇后,禁位特進,禁弟弘至長樂衛尉。永光二年,禁薨,諡曰頃侯。長子鳳嗣侯,為衛尉侍中。皇后自有子後,希復進見。太子壯大,寬博恭慎,語在成紀。其後幸酒,樂燕樂,元帝不以為能。而傅昭儀有寵於上,生定陶共王。王多材藝,上甚愛之,坐則側席,行則同輦,常有意欲廢太子而立共王。時鳳在位,與皇后、太子同心憂懼,賴侍中史丹擁右太子,語在丹傳。上亦以皇后素謹慎,而太子先帝所常留意,故得不廢。

元帝崩,太子立,是為孝成帝。尊皇后為皇太后,以鳳為大司馬大將軍領尚書事,益封五千戶。王氏之興自鳳始。又封太后同母弟崇為安成侯,食邑萬戶。鳳庶弟譚等皆賜爵關內侯,食邑。

其夏,黃霧四塞終日。天子以問諫大夫楊興、博士駟勝等,對皆以為「陰盛侵陽之氣也。高祖之約也,非功臣不侯,今太后諸弟皆以無功為侯,非高祖之約,外戚未曾有也,故天為見異。」言事者多以為然。鳳於是懼,上書辭謝曰:「陛下即位,思慕諒闇,故詔臣鳳典領尚書事,上無以明聖德,下無以益政治。今有茀星天地赤黃之異,咎在臣鳳,當伏顯戮,以謝天下。今諒闇已畢,大義皆舉,宜躬親萬機,以承天心。」因乞骸骨辭職。上報曰:「朕承先帝聖緒,涉道未深,不明事情,是以陰陽錯繆,日月無光,赤黃之氣,充塞天下。咎在朕躬,今大將軍乃引過自予,欲上尚書事,歸大將軍印綬,罷大司馬官,是明朕之不德也。朕委將軍以事,誠欲庶幾有成,顯先祖之功德。將軍其專心固意,輔朕之不逮,毋有所疑。」

後五年,諸吏散騎安成侯崇薨,謚曰共侯。有遺腹子奉世嗣侯,太后甚哀之。明年,河平二年,上悉封舅譚為平阿侯,商成都侯,立紅陽侯,根曲陽侯,逢時高平侯。五人同日封,故世謂之「五侯」。太后同產唯曼蚤卒,餘畢侯矣。太后母李親,苟氏妻,生一男名參,寡居。頃侯禁在時,太后令禁還李親。太后憐參,欲以田蚡為比而封之。上曰:「封田氏,非正也。」以參為侍中水衡都尉。王氏子弟皆卿大夫侍中諸曹,分據勢官滿朝廷。

大將軍鳳用事,上遂謙讓無所顓。左右常薦光祿大夫劉向少子歆通達有異材。上召見歆,誦讀詩賦,甚說之,欲以為中常侍,召取衣冠。臨當拜,左右皆曰:「未曉大將軍。」上曰:「此小事,何須關大將軍?」左右叩頭爭之。上於是語鳳,鳳以為不可,乃止。其見憚如此。

上即位數年,無繼嗣,體常不平。定陶共王來朝,太后與上承先帝意,遇共王甚厚,賞賜十倍於它王,不以往事為纖介。共王之來朝也,天子留,不遣歸國。上謂共王:「我未有子,人命不諱,一朝有它,且不復相見。爾長留侍我矣!」其後天子疾益有瘳,共王因留國邸,旦夕侍上,上甚親重。大將軍鳳心不便共王在京師,會日蝕,鳳因言「日蝕陰盛之象,為非常異。定陶王雖親,於禮當奉藩在國。今留侍京師,詭正非常,故天見戒。宜遣王之國。」上不得已於鳳而許之。共王辭去,上與相對泣而決。

京兆尹王章素剛直敢言,以為鳳建遣共王之國非是,乃奏封事言日蝕之咎矣。天子召見章,延問以事,章對曰:「天道聰明,佐善而災惡,以瑞異為符效。今陛下以未有繼嗣,引近定陶王,所以承宗廟,重社稷,上順天心,下安百姓。此正義善事,當有祥瑞,何故致災異?災異之發,為大臣顓政者也。今聞大將軍猥歸日蝕之咎於定陶王,建遣之國,苟欲使天子孤立於上,顓擅朝事以便其私,非忠臣也。且日蝕,陰侵陽臣顓君之咎,今政事大小皆自鳳出,天子曾不一舉手,鳳不內省責,反歸咎善人,推遠定陶王。且鳳誣罔不忠,非一事也。前丞相樂昌侯商本以先帝外屬,內行篤,有威重,位歷將相,國家柱石臣也,其人守正,不肯詘節隨鳳委曲,卒用閨門之事為鳳所罷,身以憂死,眾庶愍之。又鳳知其小婦弟張美人已嘗適人,於禮不宜配御至尊,託以為宜子,內之後宮,苟以私其妻弟。聞張美人未嘗任身就館也。且羌胡尚殺首子以盪腸正世,況於天子而近已出之女也!此三者皆大事,陛下所自見,足以知其餘,及它所不見者。鳳不可令久典事,宜退使就第,選忠賢以代之。」

自鳳之白罷商後遣定陶王也,上不能平。及聞章言,天子感寤,納之,謂章曰:「微京兆尹直言,吾不聞社稷計!且唯賢知賢,君試為朕求可以自輔者。」於是章奏封事,薦中山孝王舅琅邪太守馮野王「先帝時歷二卿,忠信質直,知謀有餘。野王以王舅出,以賢復入,明聖主樂進賢也。」上自為太子時數聞野王先帝名卿,聲譽出鳳遠甚,方倚欲以代鳳。

初,章每召見,上輒辟左右。時太后從弟長樂衛尉弘子侍中音獨側聽,具知章言,以語鳳。鳳聞之,稱病出就第,上疏乞骸骨,謝上曰:「臣材駑愚戇,得以外屬兄弟七人封為列侯,宗族蒙恩,賞賜無量。輔政出入七年,國家委任臣鳳,所言輒聽,薦士常用。無一功善,陰陽不調,災異數見,咎在臣鳳奉職無狀,此臣一當退也。五經傳記,師所誦說,咸以日蝕之咎在於大臣非其人,《易》曰『折其右肱』,此臣二當退也。

河平以來,臣久病連年,數出在外,曠職素餐,此臣三當退也。陛下以皇太后故不忍誅廢,臣猶自知當遠流放,又重自念,兄弟宗族所蒙不測,當殺身靡骨死輦轂下,不當以無益之故有離寑門之心。誠歲餘以來,所苦加侵,日日益甚,不勝大願,願乞骸骨,歸自治養,冀賴陛下神靈,未理髮齒,期月之間,幸得瘳愈,復望帷幄,不然,必寘溝壑。臣以非材見私,天下知臣受恩深也;以病得全骸骨歸,天下知臣被恩見哀,重巍巍也。進退於國為厚,萬無纖介之議。唯陛下哀憐!」其辭指甚哀,太后聞之為垂涕,不御食。

上少而親倚鳳,弗忍廢,乃報鳳曰:「朕秉事不明,政事多闕,故天變屢臻,咸在朕躬。將軍乃深引過自予,欲乞骸骨而退,則朕將何嚮焉!書不云乎?『公毋困我。』務專精神,安心自持,期於亟瘳,稱朕意焉。」於是鳳起視事。上使尚書劾奏章「知野王前以王舅出補吏,而私薦之,欲令在朝阿附諸侯;又知張美人體御至尊,而妄稱引羌胡殺子蕩腸,非所宜言。」遂下章吏。廷尉致其大逆罪,以為「比上夷狄,欲絕繼嗣之端;背畔天子,私為定陶王。」章死獄中,妻子徙合浦。

自是公卿見鳳,側目而視,郡國守相刺史皆出其門。又以侍中太僕音為御史大夫,列于三公。而五侯群弟,爭為奢侈,賂遺珍寶,四面而至;後庭姬妾,各數十人,僮奴以千百數,羅鐘磬,舞鄭女,作倡優,狗馬馳逐;大治第室,起土山漸臺,洞門高廊閣道,連屬彌望。百姓歌之曰:「五侯初起,曲陽最怒,壞決高都,連竟外杜,土山漸臺西白虎。」奢僭如此。然皆通敏人事,好士養賢,傾財施予,以相高尚。

鳳輔政凡十一歲。陽朔三年秋,鳳病,天子數自臨問,親執其手,涕泣曰:「將軍病,如有不可言,平阿侯譚次將軍矣。」鳳頓首泣曰:「譚等雖與臣至親,行皆奢僭,無以率導百姓,不如御史大夫音謹敕,臣敢以死保之。」及鳳且死,上疏謝上,復固薦音自代,譚等五人必不可用。天子然之。

初,譚倨,不肯事鳳,而音敬鳳,卑恭如子,故薦之。鳳薨,天子臨弔贈寵,送以輕車介士,軍陳自長安至渭陵,諡曰敬成侯。子襄嗣侯,為衛尉。御史大夫音竟代鳳為大司馬車騎將軍,而平阿侯譚位特進,領城門兵。谷永說譚,令讓不受城門職,由是與音不平,語在永傳。

音既以從舅越親用事,小心親職,歲餘,上下詔曰:「車騎將軍音宿衛忠正,勤勞國家,前為御史大夫,以外親宜典兵馬,入為將軍,不獲宰相之封,朕甚慊焉!其封音為安陽侯,食邑與五侯等,俱三千戶。」

初,成都侯商嘗病,欲避暑,從上借明光宮。後又穿長安城,引內灃水注第中大陂以行船,立羽蓋,張周帷,輯濯越歌。上幸商第,見穿城引水,意恨,內銜之,未言。後微行出,過曲陽侯第,又見園中土山漸臺似類白虎殿。於是上怒,以讓車騎將軍音。商、根兄弟欲自黥劓謝太后。上聞之大怒,乃使尚書責問司隸校尉、京兆尹「知成都侯商擅穿帝城,決引灃水,曲陽侯根驕奢僭上,赤墀青瑣,紅陽侯立父子臧匿姦猾亡命,賓客為群盜,司隸、京兆皆阿縱不舉奏正法。」二人頓首省戶下。又賜車騎將軍音策書曰:「外家何甘樂禍敗,而欲自黥劓,相戮辱於太后前,傷慈母之心,以危亂國!外家宗族彊,上一身浸弱日久,今將一施之。君其召諸侯,令待府舍。」是日,詔尚書奏文帝時誅將軍薄昭故事。車騎將軍音藉槁請罪,商、立、根皆負斧質謝。上不忍誅,然後得已。

久之,平阿侯譚薨,諡曰安侯,子仁嗣侯。太后憐弟曼蚤死,獨不封,曼寡婦渠供養東宮,子莽幼孤不及等比,常以為語。平阿侯譚、成都侯商及在位多稱莽者。久之,上復下詔追封曼為新都哀侯,而子莽嗣爵為新都侯。後又封太后姊子淳于長為定陵侯。王氏親屬,侯者凡十人。

上悔廢平阿侯譚不輔政而薨也,乃復進成都侯商以特進,領城門兵,置幕府,得舉吏如將軍。杜鄴說車騎將軍音令親附商,語在鄴傳。王氏爵位日盛,唯音為修整,數諫正,有忠節,輔政八年,薨。弔贈如大將軍,諡曰敬侯。子舜嗣侯,為太僕侍中。特進成都侯商代音為大司馬衛將軍,而紅陽侯立位特進,領城門兵。商輔政四歲,病乞骸骨,天子憫之,更以為大將軍,益封二千戶,賜錢百萬。商薨,弔贈如大將軍故事,諡曰景成侯,子況嗣侯。紅陽侯立次當輔政,有罪過,語在孫寶傳。上乃廢立而用光祿勳曲陽侯根為大司馬票騎將軍,歲餘益封千七百戶。高平侯逢時無材能名稱,是歲薨,諡曰戴侯,子買之嗣侯。

綏和元年,上即位二十餘年無繼嗣,而定陶共王已薨,子嗣立為王。王祖母定陶傅太后重賂遺票騎將軍根,為王求漢嗣,根為言,上亦欲立之,遂徵定陶王為太子。時根輔政五歲矣,乞骸骨,上乃益封根五千戶,賜安車駟馬,黃金五百斤,罷就第。

先是定陵侯淳于長以外屬能謀議,為衛尉侍中,在輔政之次。是歲,新都侯莽告長伏罪與紅陽侯立相連,長下獄死,立就國,語在長傳。故曲陽侯根薦莽以自代,上亦以為莽有忠直節,遂擢莽從侍中騎都尉光祿大夫為大司馬。

歲餘,成帝崩,哀帝即位。太后詔莽就第,避帝外家。哀帝初優莽,不聽。莽上書固乞骸骨而退。上乃下詔曰:「曲陽侯根前在位,建社稷策。侍中太僕安陽侯舜往時護太子家,導朕,忠誠專壹,有舊恩。新都侯莽憂勞國家,執義堅固,庶幾與為治,太皇太后詔休就第,朕甚閔焉。其益封根二千戶,舜五百戶,莽三百五十戶。以莽為特進,朝朔望。」又還紅陽侯立京師。哀帝少而聞知王氏驕盛,心不能善,以初立,故優之。

後月餘,司隸校尉解光奏:「曲陽根宗重身尊,三世據權,五將秉政,天下輻湊自效。根行貪邪,臧累鉅萬,縱橫恣意,大治第宅,第中起土山,立兩市,殿上赤墀,戶青瑣;遊觀射獵,使奴從者被甲持弓弩,陳為步兵;止宿離宮,水衡共張,發民治道,百姓苦其役。內懷姦邪,欲筦朝政,推親近吏主簿張業以為尚書,蔽上壅下,內塞王路,外交藩臣,驕奢僭上,壞亂制度。案根骨肉至親,社稷大臣,先帝棄天下,根不悲哀思慕,山陵未成,公聘取故掖庭女樂五官殷嚴、王飛君等,置酒歌舞,捐忘先帝厚恩,背臣子義。及根兄子成都侯況幸得以外親繼父為列侯侍中,不思報厚恩,亦聘取故掖庭貴人以為妻,皆無人臣禮,大不敬不道。」於是天子曰:「先帝遇根、況父子,至厚也,今乃背忘恩義!」以根嘗建社稷之策,遣就國。免況為庶人,歸故郡。根及況父商所薦舉為官者,皆罷。

後二歲,傅太后、帝母丁姬皆稱尊號。有司奏「新都侯莽前為大司馬,貶抑尊號之議,虧損孝道,及平阿侯仁臧匿趙昭儀親屬,皆就國。」天下多冤王氏。

諫大夫楊宣上封事言:「孝成皇帝深惟宗廟之重,稱述陛下至德以承天序,聖策深遠,恩德至厚。惟念先帝之意,豈不欲以陛下自代,奉承東宮哉!太皇太后春秋七十,數更憂傷,敕令親屬引領以避丁、傅。行道之人為之隕涕,況於陛下,時登高遠望,獨不慚於延陵乎!」哀帝深感其言,復封商中子邑為成都侯。

元壽元年,日蝕。賢良對策多訟新都侯莽者,上於是徵莽及平阿侯仁還京師侍太后。曲陽侯根薨,國除。

明年,哀帝崩,無子,太皇太后以莽為大司馬,與共徵立中山王奉哀帝後,是為平帝。帝年九歲,常年被疾,太后臨朝,委政於莽,莽顓威福。紅陽侯立莽諸父,平阿侯仁素剛直,莽內憚之,令大臣以罪過奏遣立、仁就國。莽日誑燿太后,言輔政致太平,群臣奏請尊莽為安漢公。後遂遣使者迫守立、仁令自殺,賜立諡曰荒侯,子柱嗣,仁諡曰刺侯,子術嗣。是歲,元始三年也。明年,莽風群臣奏立莽女為皇后。又奏尊莽為宰衡,莽母及兩子皆封為列侯,語在莽傳。

莽既外壹群臣,令稱己功德,又內媚事旁側長御以下,賂遺以千萬數。白尊太后姊妹君俠為廣恩君,君力為廣惠君,君弟為廣施君,皆食湯沐邑,日夜共譽莽。莽又知太后婦人厭居深宮中,莽欲虞樂以市其權,乃令太后四時車駕巡狩四郊,存見孤寡貞婦。春幸繭館,率皇后列侯夫人桑,遵霸水而祓除;夏遊篽宿、鄠、杜之間;秋歷東館,望昆明,集黃山宮;冬饗飲飛羽,校獵上蘭,登長平館,臨涇水而覽焉。太后所至屬縣,輒施恩惠,賜民錢帛牛酒,歲以為常。太后從容言曰:「我始入太子家時,見於丙殿,至今五六十歲尚頗識之。」莽因曰:「太子宮幸近,可壹往遊觀,不足以為勞。」於是太后幸太子宮,甚說。太后旁弄兒病在外舍,莽自親候之。其欲得太后意如此。

平帝崩,無子,莽徵宣帝玄孫選最少者廣戚侯子劉嬰,年二歲,託以卜相為最吉。乃風公卿奏請立嬰為孺子,令宰衡安漢公莽踐祚居攝,如周公傅成王故事。太后不以為可,力不能禁,於是莽遂為攝皇帝,改元稱制焉。俄而宗室安眾侯劉崇及東郡太守翟義等惡之,更舉兵欲誅莽。太后聞之,曰:「人心不相遠也。我雖婦人,亦知莽必以是自危,不可。」其後,莽遂以符命自立為真皇帝,先奉諸符瑞以白太后,太后大驚。

初,漢高祖入咸陽至霸上,秦王子嬰降於軹道,奉上始皇璽。及高祖誅項籍,即天子位,因御服其璽,世世傳受,號曰漢傳國璽。以孺子未立,璽臧長樂宮。及莽即位,請璽,太后不肯授莽。莽使安陽侯舜諭指。舜素謹敕,太后雅愛信之。舜既見,太后知其為莽求璽,怒罵之曰:「而屬父子宗族蒙漢家力,富貴累世,既無以報,受人孤寄,乘便利時,奪取其國,不復顧恩義。人如此者,狗豬不食其餘,天子豈有而兄弟邪!且若自以金匱符命為新皇帝,變更正朔服制,亦當自更作璽,傳之萬世,何用此亡國不祥璽為,而欲求之?我漢家老寡婦,旦暮且死,欲與此璽俱葬,終不可得!」太后因涕泣而言,旁側長御以下皆垂涕。舜亦悲不能自止,良久乃仰謂太后:「臣等已無可言者。莽必欲得傳國璽,太后寧能終不與邪!」太后聞舜語切,恐莽欲脅之,乃出漢傳國璽,投之地以授舜,曰:「我老已死,知而兄弟,今族滅也!」舜既得傳國璽,奏之,莽大說,乃為太后置酒未央宮漸臺,大縱眾樂。

莽又欲改太后漢家舊號,易其璽綬,恐不見聽,而莽疏屬王諫欲諂莽,上書言:「皇天廢去漢而命立新室,太皇太后不宜稱尊號,當隨漢廢,以奉天命。」莽乃車駕至東宮,親以其書白太后。太后曰:「此言是也!」莽因曰:「此誖德之臣也,罪當誅!」於是冠軍張永獻符命銅璧,文言「太皇太后當為新室文母太皇太后」。莽乃下詔曰:「予視群公,咸曰『休哉!其文字非刻非畫,厥性自然。』予伏念皇天命予為子,更命太皇太后為『新室文母太皇太后』,協于新室故交代之際,信於漢氏。哀帝之代,世傳行詔籌,為西王母共具之祥,當為歷代為母,昭然著明。予祗畏天命,敢不欽承!謹以令月吉日,親率群公諸侯卿士,奉上皇太后璽紱,以當順天心,光于四海焉。」太后聽許。莽於是鴆殺王諫,而封張永為貢符子。

初,莽為安漢公時,又諂太后,奏尊元帝廟為高宗,太后晏駕後當以禮配食云。及莽改太后為新室文母,絕之於漢,不令得體元帝。墮壞孝元廟,更為文母太后起廟,獨置孝元廟故殿以為文母篹食堂,既成,名曰長壽宮。以太后在,故未謂之廟。莽以太后好出遊觀,乃車駕置酒長壽宮,請太后。既至,見孝元廟廢徹塗地,太后驚,泣曰:「此漢家宗廟,皆有神靈,與何治而壞之!且使鬼神無知,又何用廟為!如令有知,我乃人之妃妾,豈宜辱帝之堂以陳饋食哉!」私謂左右曰:「此人嫚神多矣,能久得祐乎!」飲酒不樂而罷。

自莽篡位後,知太后怨恨,求所以媚太后無不為,然愈不說。莽更漢家黑貂,著黃貂,又改漢正朔伏臘日。太后令其官屬黑貂,至漢家正臘日,獨與其左右相對飲酒食。

太后年八十四,建國五年二月癸丑崩。三月乙酉,合葬渭陵。莽詔大夫揚雄作誄曰:「太陰之精,沙麓之靈,作合於漢,配元生成。」著其協於元城沙麓。泰陰精者,謂夢月也。太后崩後十年,漢兵誅莽。

初,紅陽侯立就國南陽,與諸劉結恩,立少子丹為中山太守。世祖初起,丹降為將軍,戰死。上閔之,封丹子泓為武桓侯,至今。

司徒掾班彪曰:三代以來,春秋所記,王公國君,與其失世,稀不以女寵。漢興,后妃之家呂、霍、上官,幾危國者數矣。及王莽之興,由孝元后歷漢四世為天下母,饗國六十餘載,群弟世權,更持國柄,五將十侯,卒成新都。位號已移於天下,而元后卷卷猶握一璽,不欲以授莽,婦人之仁,悲夫!


\end{pinyinscope}