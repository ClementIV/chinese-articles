\article{外戚傳}

\begin{pinyinscope}
自古受命帝王及繼體守文之君,非獨內德茂也,蓋亦有外戚之助焉。夏之興也以塗山,而桀之放也用末喜;殷之興也以有娀又有榇,而紂之滅也嬖妲己;周之興也以姜嫄及太任、太姒,而幽王之禽也淫褒姒。故易基乾坤,詩首關雎,書美釐降,春秋譏不親迎。夫婦之際,人道之大倫也。禮之用,唯昏姻為兢兢。夫樂調而四時和,陰陽之變,萬物之統也,可不慎與!人能弘道,末如命何。甚哉妃匹之愛,君不能得之臣,父不能得之子,況卑下乎!既驩合矣,或不能成子姓,成子姓矣,而不能要其終,豈非命也哉!孔子罕言命,蓋難言之。非通幽明之變,惡能識乎性命!

漢興,因秦之稱號,帝母稱皇太后,祖母稱太皇太后,適稱皇后,妾皆稱夫人。又有美人、良人、八子、七子、長使、少使之號焉。至武帝制婕妤、娙娥、傛華、充依,各有爵位,而元帝加昭儀之號,凡十四等云。昭儀位視丞相,爵比諸侯王。婕妤視上卿,比列侯。娙娥視中二千石,比關內侯。傛華視真二千石,比大上造。美人視二千石,比少上造。八子視千石,比中更。充依視千石,比左更。七子視八百石,比右庶長。良人視八百石,比左庶長。長使視六百石,比五大夫。少使視四百石,比公乘。五官視三百石。順常視二百石。無涓、共和、娛靈、保林、良使、夜者皆視百石。上家人子、中家人子視有秩斗食云。五官以下,葬司馬門外。

高祖呂皇后,父呂公,單父人也,好相人。高祖微時,呂公見而異之,乃以女妻高祖,生惠帝、魯元公主。高祖為漢王,元年封呂公為臨泗侯,二年立孝惠為太子。

後漢王得定陶戚姬,愛幸,生趙隱王如意。太子為人仁弱,高祖以為不類己,常欲廢之而立如意,「如意類我」。戚姬常從上之關東,日夜啼泣,欲立其子。呂后年長,常留守,希見,益疏。如意且立為趙王,留長安,幾代太子者數。賴公卿大臣爭之,及叔孫通諫,用留侯之策,得無易。

呂后為人剛毅,佐高帝定天下,兄二人皆為列將,從征伐。長兄澤為周呂侯,次兄釋之為建成侯,逮高祖而侯者三人。高祖四年,臨泗侯呂公薨。

高祖崩,惠帝立,呂后為皇太后,乃令永巷囚戚夫人,髡鉗衣赭衣,令舂。戚夫人舂且歌曰:「子為王,母為虜,終日舂薄暮,常與死為伍!相離三千里,當誰使告女?」太后聞之大怒,曰:「乃欲倚女子邪?」乃召趙王誅之。使者三反,趙相周昌不遣。太后召趙相,相徵至長安。使人復召趙王,王來。惠帝慈仁,知太后怒,自迎趙王霸上,入宮,挾與起居飲食。數月,帝晨出射,趙王不能蚤起,太后伺其獨居,使人持鴆飲之。遲帝還,趙王死。太后遂斷戚夫人手足,去眼熏耳,飲瘖藥,使居鞠域中,名曰「人彘」。居數月,乃召惠帝視「人彘」。帝視而問知其戚夫人,乃大哭,因病,歲餘不能起。使人請太后曰:「此非人所為。臣為太后子,終不能復治天下!」以此日飲為淫樂,不聽政,七年而崩。

太后發喪,哭而泣不下。留侯子張辟彊為侍中,年十五,謂丞相陳平曰:「太后獨有帝,今哭而不悲,君知其解未?」陳平曰:「何解?」辟彊曰:「帝無壯子,太后畏君等。今請拜呂台、呂產為將,將兵居南北軍,及諸呂皆官,居中用事。如此則太后心安,君等幸脫禍矣!」丞相如辟彊計請之,太后說,其哭乃哀。呂氏權由此起。乃立孝惠後宮子為帝,太后臨朝稱制。復殺高祖子趙幽王友、共王恢及燕靈王建。遂立周呂侯子台為呂王,台弟產為梁王,建城侯釋之子祿為趙王,台子通為燕王,又封諸呂凡六人皆為列侯,追尊父呂公為呂宣王,兄周呂侯為悼武王。

太后持天下八年,病犬禍而崩,語在五行志。病困,以趙王祿為上將軍居北軍,梁王產為相國居南軍,戒產、祿曰:「高祖與大臣約,非劉氏王者天下共擊之,今王呂氏,大臣不平。我即崩,恐其為變,必據兵衛宮,慎毋送喪,為人所制。」太后崩,太尉周勃、丞相陳平、朱虛侯劉章等共誅產、祿,悉捕諸呂男女,無少長皆斬之。而迎立代王,是為孝文皇帝。

孝惠張皇后。宣平侯敖尚帝姊魯元公主,有女。惠帝即位,呂太后欲為重親,以公主女配帝為皇后。欲其生子,萬方終無子,乃使陽為有身,取後宮美人子名之,殺其母,立所名子為太子。

惠帝崩,太子立為帝,四年,乃自知非皇后子,出言曰:「太后安能殺吾母而名我!我壯即為所為。」太后聞而患之,恐其作亂,乃幽之永巷,言帝病甚,左右莫得見。太后下詔廢之,語在高后紀。遂幽死,更立恆山王弘為皇帝,而以呂祿女為皇后。欲連根固本牢甚,然而無益也。呂太后崩,大臣正之,卒滅呂氏。少帝恆山、淮南、濟川王,皆以非孝惠子誅。獨置孝惠皇后,廢處北宮,孝文後元年薨,葬安陵,不起墳。

高祖薄姬,文帝母也。父吳人,秦時與故魏王宗女魏媼通,生薄姬。而薄姬父死山陰,因葬焉。及諸侯畔秦,魏豹立為王,而魏媼內其女於魏宮。許負相薄姬,當生天子。是時項羽方與漢王相距滎陽,天下未有所定。豹初與漢擊楚,及聞許負言,心喜,因背漢而中立,與楚連和。漢使曹參等虜魏王豹,以其國為郡,而薄姬輸織室。豹已死,漢王入織室,見薄姬,有詔內後宮,歲餘不得幸。

始姬少時,與管夫人、趙子兒相愛,約曰:「先貴毋相忘!」已而管夫人、趙子兒先幸漢王。漢王四年,坐河南成皋靈臺,此兩美人侍,相與笑薄姬初時約。漢王問其故,兩人俱以實告。漢王心悽然憐薄姬,是日召欲幸之。對曰:「昨暮夢龍據妾胸。」上曰:「是貴徵也,吾為汝成之。」遂幸,有身。歲中生文帝,年八歲立為代王。自有子後,希見。高祖崩,諸幸姬戚夫人之屬,呂后怒,皆幽之不得出宮。而薄姬以希見故,得出從子之代,為代太后。太后弟薄昭從如代。

代王立十七年,高后崩。大臣議立後,疾外家呂氏彊暴,皆稱薄氏仁善,故迎立代王為皇帝,尊太后為皇太后,封弟昭為軹侯。太后母亦前死,葬櫟陽北。乃追尊太后父為靈文侯,會稽郡致園邑三百家,長丞以下使奉守寢廟,上食祠如法。櫟陽亦置靈文夫人園,令如靈文侯園儀。太后蚤失父,其奉太后外家魏氏有力,乃召復魏氏,賞賜各以親疏受之。薄氏侯者一人。

太后後文帝二歲,孝景前二年崩,葬南陵。用呂后不合葬長陵,故特自起陵,近文帝。

孝文竇皇后,景帝母也,呂太后時以良家子選入宮。太后出宮人以賜諸王各五人,竇姬與在行中。家在清河,願如趙,近家,請其主遣宦者吏「必置我籍趙之伍乃」。宦者忘之,誤置籍代伍中。籍奏,詔可。當行,竇姬涕泣,怨其宦者,不欲往,相彊乃肯行。至代,代王獨幸竇姬,生女嫖。孝惠七年,生景帝。

代王王后生四男,先代王未入立為帝而王后卒,及代王為帝後,王后所生四男更病死。文帝立數月,公卿請立太子,而竇姬男最長,立為太子。竇姬為皇后,女為館陶長公主。明年,封少子武為代王,後徙梁,是為梁孝王。

竇皇后親蚤卒,葬觀津。於是薄太后乃詔有司追封竇后父為安成侯,母曰安成夫人,令清河置園邑二百家,長丞奉守,比靈文園法。

竇后兄長君。弟廣國字少君,年四五歲時,家貧,為人所略賣,其家不知處。傳十餘家至宜陽,為其主人入山作炭。暮臥岸下百餘人,岸崩,盡厭殺臥者,少君獨脫不死。自卜,數日當為侯。從其家之長安,聞皇后新立,家在觀津,姓竇氏。廣國去時雖少,識其縣名及姓,又嘗與其姊采桑,墮,用為符信,上書自陳。皇后言帝,召見問之,具言其故,果是。復問其所識,曰:「姊去我西時,與我決傳舍中,饨沐沐我,已,飯我,乃去。」於是竇皇后持之而泣,侍御左右皆悲。乃厚賜之,家於長安。絳侯、灌將軍等曰:「吾屬不死,命乃且縣此兩人。此兩人所出微,不可不為擇師傅,又復放呂氏大事也。」於是乃選長者之有節行者與居。竇長君、少君由此為退讓君子,不敢以富貴驕人。

竇皇后疾,失明。文帝幸邯鄲慎夫人、尹姬,皆無子。文帝崩,景帝立,皇后為皇太后,乃封廣國為章武侯。長君先死,封其子彭祖為南皮侯。吳楚反時,太后從昆弟子竇嬰俠,喜士,為大將軍,破吳楚,封魏其侯。竇氏侯者凡三人。

竇太后好黃帝、老子言,景帝及諸竇不得不讀老子尊其術。太后後景帝六歲,凡立五十一年,元光六年崩,合葬霸陵。遺詔盡以東宮金錢財物賜長公主嫖。至武帝時,魏其侯竇嬰為丞相,後誅。

孝景薄皇后,孝文薄太后家女也。景帝為太子時,薄太后取以為太子妃。景帝立,立薄妃為皇后,無子無寵。立六年,薄太后崩,皇后廢。廢後四年薨,葬長安城東平望亭南。

孝景王皇后,武帝母也。父王仲,槐里人也。母臧兒,故燕王臧荼孫也,為仲妻,生男信與兩女。而仲死,臧兒更嫁為長陵田氏婦,生男蚡、勝。臧兒長女嫁為金王孫婦,生一女矣,而臧兒卜筮曰兩女當貴,欲倚兩女,奪金氏。金氏怒,不肯與決,乃內太子宮。太子幸愛之,生三女一男。男方在身時,王夫人夢日入其懷,以告太子,太子曰:「此貴徵也。」未生而文帝崩,景帝即位,王夫人生男。是時,薄皇后無子。後數歲,景帝立齊栗姬男為太子,而王夫人男為膠東王。

長公主嫖有女,欲與太子為妃,栗姬妒,而景帝諸美人皆因長公主見得貴幸,栗姬日怨怒,謝長主,不許。長主欲與王夫人,王夫人許之。會薄皇后廢,長公主日譖栗姬短。景帝嘗屬諸姬子,曰:「吾百歲後,善視之。」栗姬怒不肯應,言不遜,景帝心銜之而未發也。

長公主日譽王夫人男之美,帝亦自賢之。又耳曩者所夢日符,計未有所定。王夫人又陰使人趣大臣立栗姬為皇后。大行奏事,文曰:「『子以母貴,母以子貴。』今太子母號宜為皇后。」帝怒曰:「是乃所當言邪!」遂案誅大行,而廢太子為臨江王。栗姬愈恚,不得見,以憂死。卒立王夫人為皇后,男為太子。封皇后兄信為蓋侯。

初,皇后始入太子家,後女弟兒姁亦復入,生四男。兒姁蚤卒,四子皆為王。皇后長女為平陽公主,次南宮公主,次隆慮公主。

皇后立九年,景帝崩。武帝即位,為皇太后,尊太后母臧兒為平原君,封田蚡為武安侯,勝為周陽侯。王氏、田氏侯者凡三人。蓋侯信好酒,田蚡、勝貪,巧於文辭。蚡至丞相,追尊王仲為共侯,槐里起園邑二百家,長丞奉守。及平原君薨,從田氏葬長陵,亦置園邑如共侯法。

初,皇太后微時所謂金王孫生女俗,在民間,蓋諱之也。武帝始立,韓嫣白之。帝曰:「何為不蚤言?」乃車駕自往迎之。其家在長陵小市,直至其門,使左右入求之。家人驚恐,女逃匿。扶將出拜,帝下車立曰:「大姊,何藏之深也?」載至長樂宮,與俱謁太后,太后垂涕,女亦悲泣。帝奉酒,前為壽。錢千萬,奴婢三百人,公田百頃,甲第,以賜姊。太后謝曰:「為帝費。」因賜湯沐邑,號修成君。男女各一人,女嫁諸侯,男號修成子仲,以太后故,橫於京師。太后凡立二十五年,後景帝十五歲,元朔三年崩,合葬陽陵。

孝武陳皇后,長公主嫖女也。曾祖父陳嬰與項羽俱起,後歸漢,為堂邑侯。傳子至孫午,午尚長公主,生女。

初,武帝得立為太子,長主有力,取主女為妃。及帝即位,立為皇后,擅寵驕貴,十餘年而無子,聞衛子夫得幸,幾死者數焉。上愈怒。后又挾婦人媚道,頗覺。元光五年,上遂窮治之,女子楚服等坐為皇后巫蠱祠祭祝詛,大逆無道,相連及誅者三百餘人。楚服梟首於市。使有司賜皇后策曰:「皇后失序,惑於巫祝,不可以承天命。其上璽綬,罷退居長門宮。」

明年,堂邑侯午薨,主男須嗣侯。主寡居,私近董偃。十餘年,主薨。須坐淫亂,兄弟爭財,當死,自殺,國除。後數年,廢后乃薨,葬霸陵郎官亭東。

孝武衛皇后字子夫,生微也。其家號曰衛氏,出平陽侯邑。子夫為平陽主謳者。武帝即位,數年無子。平陽主求良家女十餘人,飾置家。帝祓霸上,還過平陽主。主見所偫美人,帝不說。既飲,謳者進,帝獨說子夫。帝起更衣,子夫侍尚衣軒中,得幸。還坐驩甚,賜平陽主金千斤。主因奏子夫送入宮。子夫上車,主拊其背曰:「行矣!強飯勉之。即貴,願無相忘!」入宮歲餘,不復幸。武帝擇宮人不中用者斥出之,子夫得見,涕泣請出。上憐之,復幸,遂有身,尊寵。召其兄衛長君、弟青侍中。而子夫生三女,元朔元年生男據,遂立為皇后。

先是衛長君死,乃以青為將軍,擊匈奴有功,封長平侯。青三子皆襁褓中,皆為列侯。及皇后姊子霍去病亦以軍功為冠軍侯,至大司馬票騎將軍。青為大司馬大將軍。衛氏支屬侯者五人。青還,尚平陽主。

皇后立七年,而男立為太子。後色衰,趙之王夫人、中山李夫人有寵,皆蚤卒。後有尹婕妤、鉤弋夫人更幸。衛后立三十八年,遭巫蠱事起,江充為姦,太子懼不能自明,遂與皇后共誅充,發兵,兵敗,太子亡走。詔遣宗正劉長樂、執金吾劉敢奉策收皇后璽綬,自殺。黃門蘇文、姚定漢輿置公車令空舍,盛以小棺,瘞之城南桐柏。衛氏悉滅。宣帝立,及改葬衛后,追諡曰思后,置園邑三百家,長丞周衛奉守焉。

孝武李夫人,本以倡進。初,夫人兄延年性知音,善歌舞,武帝愛之。每為新聲變曲,聞者莫不感動。延年侍上起舞,歌曰:「北方有佳人,絕世而獨立,一顧傾人城,再顧傾人國。寧不知傾城與傾國,佳人難再得!」上嘆息曰:「善!世豈有此人乎?」平陽主因言延年有女弟,上乃召見之,實妙麗善舞。由是得幸,生一男,是為昌邑哀王。李夫人少而蚤卒,上憐閔焉,圖畫其形於甘泉宮。及衛思后廢後四年,武帝崩,大將軍霍光緣上雅意,以李夫人配食,追上尊號曰孝武皇后。

初,李夫人病篤,上自臨候之,夫人蒙被謝曰:「妾久寢病,形貌毀壞,不可以見帝。願以王及兄弟為託。」上曰:「夫人病甚,殆將不起,一見我屬託王及兄弟,豈不快哉?」夫人曰:「婦人貌不修飾,不見君父。妾不敢以燕惰見帝。」上曰:「夫人弟一見我,將加賜千金,而予兄弟尊官。」夫人曰:「尊官在帝,不在一見。」上復言欲必見之,夫人遂轉鄉歔欷而不復言。於是上不說而起。夫人姊妹讓之曰:「貴人獨不可一見上屬託兄弟邪?何為恨上如此?」夫人曰:「所以不欲見帝者,乃欲以深託兄弟也。我以容貌之好,得從微賤愛幸於上。夫以色事人者,色衰而愛弛,愛弛則恩絕。上所以攣攣顧念我者,乃以平生容貌也。今見我毀壞,顏色非故,必畏惡吐棄我,意尚肯復追思閔錄其兄弟哉!」及夫人卒,上以后禮葬焉。其後,上以夫人兄李廣利為貳師將軍,封海西侯,延年為協律都尉。

上思念李夫人不已,方士齊人少翁言能致其神。乃夜張燈燭,設帷帳,陳酒肉,而令上居他帳,遙望好女如李夫人之貌,還幄坐而步。又不得就視,上愈益相思悲感,為作詩曰:「是邪,非邪?立而望之,偏何姍姍其來遲!」令樂府諸音家絃歌之。上又自為作賦,以傷悼夫人,其辭曰:

美連娟以脩嫮兮,命樔絕而不長,飾新宮以延貯兮,泯不歸乎故鄉。慘鬱鬱其蕪穢兮,隱處幽而懷傷,釋輿馬於山椒兮,奄修夜之不陽。秋氣潛以淒淚兮,桂枝落而銷亡,神煢煢以遙思兮,精浮游而出谗。託沈陰以壙久兮,惜蕃華之未央,念窮極之不還兮,惟幼眇之相羊。函荾荴以俟風兮,芳雜襲以彌章,的容與以猗靡兮,縹飄姚虖愈莊。燕淫衍而撫楹兮,連流視而娥揚,既激感而心逐兮,包紅顏而弗明。驩接狎以離別兮,宵寤夢之芒芒,忽遷化而不反兮,魄放逸以飛揚。何靈魂之紛紛兮,哀裴回以躊躇,勢路日以遠兮,遂荒忽而辭去。超兮西征,屑兮不見。寢淫敞胧,寂兮無音,思若流波,怛兮在心。

亂曰:佳俠函光,隕朱榮兮,嫉妒闟葺,將安程兮!方時隆盛,年夭傷兮,弟子增欷,洿沬悵兮。悲愁於邑,喧不可止兮。嚮不虛應,亦云己兮。嫶妍太息,嘆稚子兮,懰慄不言,倚所恃兮。仁者不誓,豈約親兮?既往不來,申以信兮。去彼昭昭,就冥冥兮,既下新宮,不復故庭兮。嗚呼哀哉,想魂靈兮!

其後李延年弟季坐姦亂後宮,廣利降匈奴,家族滅矣。

孝武鉤弋趙婕妤,昭帝母也,家在河間。武帝巡狩過河間,望氣者言此有奇女,天子亟使使召之。既至,女兩手皆拳,上自披之,手即時伸。由是得幸,號曰拳夫人。先是其父坐法宮刑,為中黃門,死長安,葬雍門。

拳夫人進為婕妤,居鉤弋宮,大有寵,元始三年生昭帝,號鉤弋子。任身十四月乃生,上曰:「聞昔堯十四月而生,今鉤弋亦然。」乃命其所生門曰堯母門。後衛太子敗,而燕王旦、廣陵王胥多過失,寵姬王夫人男齊懷王、李夫人男昌邑哀王皆蚤薨,鉤弋子年五六歲,壯大多知,上常言「類我」,又感其生與眾異,甚奇愛之,心欲立焉,以其年稚母少,恐女主顓恣亂國家,猶與久之。

鉤弋婕妤從幸甘泉,有過見譴,以憂死,因葬雲陽。後上疾病,乃立鉤弋子為皇太子。拜奉車都尉霍光為大司馬大將軍,輔少主。明日,帝崩。昭帝即位,追尊鉤弋婕妤為皇太后,發卒二萬人起雲陵,邑三千戶。追尊外祖趙父為順成侯,詔右扶風置園邑二百家,長丞奉守如法。順成侯有姊君姁,賜錢二百萬,奴婢第宅以充實焉。諸昆弟各以親疏受賞賜。趙氏無在位者,唯趙父追封。

孝昭上官皇后。祖父桀,隴西上邽人也。少時為羽林期門郎,從武帝上甘泉,天大風,車不得行,解蓋授桀。桀奉蓋,雖風常屬車;雨下,蓋輒御。上奇其材力,遷未央廄令。上嘗體不安,及愈,見馬,馬多瘦,上大怒:「令以我不復見馬邪!」欲下吏,桀頓首曰:「臣聞聖體不安,日夜憂懼,意誠不在馬。」言未卒,泣數行下。上以為忠,由是親近,為侍中,稍遷至太僕。武帝疾病,以霍光為大將軍,太僕桀為左將軍,皆受遺詔輔少主。以前捕斬反者莽通功,封桀為安陽侯。

初,桀子安取霍光女,結婚相親,光每休沐出,桀常代光入決事。昭帝始立,年八歲,帝長姊鄂邑蓋長公主居禁中,共養帝。蓋主私近子客河間丁外人。上與大將軍聞之,不絕主驩,有詔外人侍長主。長主內周陽氏女,令配耦帝。時上官安有女,即霍光外孫,安因光欲內之。光以為尚幼,不聽。安素與丁外人善,說外人曰:「聞長主內女,安子容貌端正,誠因長主時得入為后,以臣父子在朝而有椒房之重,成之在於足下,漢家故事常以列侯尚主,足下何憂不封侯乎?」外人喜,言於長主。長主以為然,詔召安女入為婕妤,安為騎都尉。月餘,遂立為皇后,年甫六歲。

安以后父封桑樂侯,食邑千五百戶,遷車騎將軍,日以驕淫。受賜殿中,出對賓客言:「與我婿飲,大樂!」見其服飾,使人歸,欲自燒物。安醉則裸行內,與後母及父諸良人、侍御皆亂。子病死,仰而罵天。數守大將軍光,為丁外人求侯,及桀欲妄官祿外人,光執正,皆不聽。又桀妻父所幸充國為太醫監,闌入殿中,下獄當死。冬月且盡,蓋主為充國入馬二十匹贖罪,乃得減死論。於是桀、安父子深怨光而重德蓋主。知燕王旦帝兄,不得立,亦怨望,桀、安即記光過失予燕王,令上書告之,又為丁外人求侯。燕王大喜,上書稱:「子路喪姊,期而不除,孔子非之。子路曰:『由不幸寡兄弟,不忍除之。』故曰:『觀過知仁』。今臣與陛下獨有長公主為姊,陛下幸使丁外人侍之,外人宜蒙爵號。」書奏,上以問光,光執不許。及告光罪過,上又疑之,愈親光而疏桀、安。桀、安寖恚,遂結黨與謀殺光,誘徵燕王至而誅之,因廢帝而立桀。或曰:「當如皇后何?」安曰:「逐麋之狗,當顧菟邪!且用皇后為尊,一旦人主意有所移,雖欲為家人亦不可得,此百世之一時也。」事發覺,燕王、蓋主皆自殺。語在霍光傳。桀、安宗族既滅,皇后以年少不與謀,亦光外孫,故得不廢。皇后母前死,葬茂陵郭東,追尊曰敬夫人,置園邑二百家,長丞奉守如法。皇后自使私奴婢守桀、安冢。

光欲皇后擅寵有子,帝時體不安,左右及醫皆阿意,言宜禁內,雖宮人使令皆為窮恊,多其帶,後宮莫有進者。

皇后立十歲而昭帝崩,后年十四五云。昌邑王賀徵即位,尊皇后為皇太后。光與太后共廢王賀,立孝宣帝。宣帝即位,為太皇太后。凡立四十七年,年五十二,建昭二年崩,合葬平陵。

衛太子史良娣,宣帝祖母也。太子有妃,有良娣,有孺子,妻妾凡三等,子皆稱皇孫。史良娣家本魯國,有母貞君,兄恭。以元鼎四年入為良娣,生男進,號史皇孫。

武帝末,巫蠱事起,衛太子及良娣、史皇孫皆遭害。史皇孫有一男,號皇曾孫,時生數月,猶坐太子繫獄,積五歲乃遭赦。治獄使者邴吉憐皇曾孫無所歸,載以附史恭。恭母貞君年老,見孫孤,甚哀之,自養視焉。

後曾孫收養於掖庭,遂登至尊位,是為宣帝。而貞君及恭已死,恭三子皆以舊恩封。長子高為樂陵侯,曾為將陵侯,玄為平臺侯,及高子丹以功德封武陽侯,侯者凡四人。高至大司馬車騎將軍,丹左將軍,自有傳。

史皇孫王夫人,宣帝母也,名翁須,太始中得幸於史皇孫。皇孫妻妾無號位,皆稱家人子。征和二年,生宣帝。帝生數月,衛太子、皇孫敗,家人子皆坐誅,莫有收葬者,唯宣帝得全。即尊位後,追尊母王夫人諡曰悼后,祖母史良娣曰戾后,皆改葬,起園邑,長丞奉守。語在戾太子傳。地節三年,求得外祖母王媼,媼男無故,無故弟武皆隨使者詣闕。時乘黃牛車,故百姓謂之黃牛嫗。

初,上即位,數遣使者求外家,久遠,多似類而非是。既得王媼,令太中大夫任宣與丞相御史屬雜考問鄉里識知者,皆曰王嫗。嫗言名妄人,家本涿郡蠡吾平鄉。年十四嫁為同鄉王更得妻。更得死,嫁為廣望王迺始婦,產子男無故、武,女翁須。翁須年八九歲時,寄居廣望節侯子劉仲卿宅,仲卿謂迺始曰:「予我翁須,自養長之。」媼為翁須作縑單衣,送仲卿家。仲卿教翁須歌舞,往來歸取冬夏衣。居四五歲,翁須來言「邯鄲賈長兒求歌舞者,仲卿欲以我與之。」媼即與翁須逃走,之平鄉。仲卿載迺始共求媼,媼惶急,將翁須歸,曰:「兒居君家,非受一錢也,奈何欲予它人?」仲卿詐曰:「不也。」後數日,翁須乘長兒車馬過門,呼曰:「我果見行,當之柳宿。」媼與迺始之柳宿,見翁須相對涕泣,謂曰:「我欲為汝自言。」翁須曰:「母置之,何家不可以居?自言無益也。」媼與迺始還求錢用,隨逐至中山盧奴,見翁須與歌舞等比五人同處,媼與翁須共宿。明日,迺始留視翁須,媼還求錢,欲隨至邯鄲。媼歸,糶買未具,迺始來歸曰:「翁須已去,我無錢用隨也。」因絕至今,不聞其問。賈長兒妻貞及從者師遂辭:「往二十歲,太子舍人侯明從長安來求歌舞者,請翁須等五人。長兒使遂送至長安,皆入太子家。」及廣望三老更始、劉仲卿妻其等四十五人辭,皆驗。宣奏王媼悼后母明白,上皆召見,賜無故、武爵關內侯,旬月間,賞賜以鉅萬計。頃之,制詔御史賜外祖母號為博平君,以博平、蠡吾兩縣戶萬一千為湯沐邑。封舅無故為平昌侯,武為樂昌侯,食邑各六千戶。

初,迺始以本始四年病死,後三歲,家乃富貴,追賜諡曰思成侯。詔涿郡治冢室,置園邑四百家,長丞奉守如法。歲餘,博平君薨,諡曰思成夫人。詔徙思成侯合葬奉明顧成廟南,置園邑長丞,罷涿郡思成園。王氏侯者二人,無故子接為大司馬車騎將軍,而武子商至丞相,自有傳。

孝宣許皇后,元帝母也。父廣漢,昌邑人,少時為昌邑王郎。從武帝上甘泉,誤取它郎鞍以被其馬,發覺,吏劾從行而盜,當死,有詔募下蠶室。後為宦者丞。上官桀謀反時,廣漢部索,其殿中廬有索長數尺可以縛人者數千杖,滿一篋緘封,廣漢索不得,它吏往得之。廣漢坐論為鬼薪,輸掖庭,後為暴室嗇夫。時宣帝養於掖庭,號皇曾孫,與廣漢同寺居。時掖庭令張賀,本衛太子家吏,及太子敗,賀坐下刑,以舊恩養視皇曾孫甚厚。及曾孫壯大,賀欲以女孫妻之。是時,昭帝始冠,長八尺二寸。賀弟安世為右將軍,與霍將軍同心輔政,聞賀稱譽皇曾孫,欲妻以女,安世怒曰:「曾孫乃衛太子後也,幸得以庶人衣食縣官,足矣,勿復言予女事。」於是賀止。時許廣漢有女平君,年十四五,當為內者令歐侯氏子婦。臨當入,歐侯氏子死。其母將行卜相,言當大貴,母獨喜。賀聞許嗇夫有女,乃置酒請之,酒酣,為言「曾孫體近,下人,乃關內侯,可妻也。」廣漢許諾。明日嫗聞之,怒。廣漢重令為介,遂與曾孫,一歲生元帝。數月,曾孫立為帝,平君為婕妤。是時,霍將軍有小女,與皇太后有親。公卿議更立皇后,皆心儀霍將軍女,亦未有言。上乃詔求微時故劍,大臣知指,白立許婕妤為皇后。既立,霍光以后父廣漢刑人不宜君國,歲餘乃封為昌成君。

霍光夫人顯欲貴其小女,道無從。明年,許皇后當娠,病。女醫淳于衍者,霍氏所愛,嘗入宮侍皇后疾。衍夫賞為掖庭戶衛,謂衍「可過辭霍夫人行,為我求安池監。」衍如言報顯。顯因生心,辟左右,字謂衍:「少夫幸報我以事,我亦欲報少夫,可乎?」衍曰:「夫人所言,何等不可者!」顯曰:「將軍素愛小女成君,欲奇貴之,願以累少夫。」衍曰:「何謂邪?」顯曰:「婦人免乳大故,十死一生。今皇后當免身,可因投毒藥去也,成君即得為皇后矣。如蒙力事成,富貴與少夫共之。」衍曰:「藥雜治,當先嘗,安可?」顯曰:「在少夫為之耳。將軍領天下,誰敢言者?緩急相護,但恐少夫無意耳!」衍良久曰:「願盡力。」即擣附子,齎入長定宮。皇后免身後,衍取附子并合大醫大丸以飲皇后。有頃曰:「我頭岑岑也,藥中得無有毒?」對曰:「無有。」遂加煩懣,崩。衍出,過見顯,相勞問,亦未敢重謝衍。後人有上書告諸醫侍疾無狀者,皆收繫詔獄,劾不道。顯恐事急,即以狀具語光,因曰:「既失計為之,無令吏急衍!」光驚鄂,默然不應。其後奏上,署衍勿論。

許后立三年而崩,諡曰恭哀皇后,葬杜南,是為杜陵南園。後五年,立皇太子,乃封太子外祖父昌成君廣漢為平恩侯,位特進。後四年,復封廣漢兩弟,舜為博望侯,延壽為樂成侯。許氏侯者凡三人。廣漢薨,諡曰戴侯,無子,絕。葬南園旁,置邑三百家,長丞奉守如法。宣帝以延壽為大司馬車騎將軍,輔政。元帝即位,復封延壽中子嘉為平恩侯,奉戴侯後,亦為大司馬車騎將軍。

孝宣霍皇后,大司馬大將軍博陸侯光女也。母顯,既使淳于衍陰殺許后,顯因為成君衣補,治入宮具,勸光內之,果立為皇后。

初許后起微賤,登至尊日淺,從官車服甚節儉,五日一朝皇太后於長樂宮,親奉案上食,以婦道共養。及霍后立,亦修許后故事。而皇太后親霍后之姊子,故常竦體,敬而禮之。皇后轝駕侍從甚盛,賞賜官屬以千萬計,與許后時縣絕矣。上亦寵之,顓房燕。立三歲而光薨。後一歲,上立許后男為太子,昌成君者為平恩侯。顯怒恚不食,歐血,曰:「此乃民間時子,安得立?即后有子,反為王邪!」復教皇后令毒太子。皇后數召太子賜食,保阿輒先嘗之,后挾毒不得行。後殺許后事頗泄,顯遂與諸婿昆弟謀反,發覺,皆誅滅。使有司賜皇后策曰:「皇后熒惑失道,懷不德,挾毒與母博陸宣成侯夫人顯謀欲危太子,無人母之恩,不宜奉宗廟衣服,不可以承天命。烏呼傷哉!其退避宮,上璽綬有司。」霍后立五年,廢處昭臺宮。後十二歲,徙雲林館,乃自殺,葬昆吾亭東。

初,霍光及兄驃騎將軍去病皆自以功伐封侯居位,宣帝以光故,封去病孫山、山弟雲皆為列侯,侯者前後四人。

孝宣王皇后。其先高祖時有功賜爵關內侯,自沛徙長陵,傳爵至后父奉光。奉光少時好鬥雞,宣帝在民間數與奉光會,相識。奉光有女年十餘歲,每當適人,所當適輒死,故久不行。及宣帝即位,召入後宮,稍進為婕妤。是時,館陶主母華婕妤及淮陽憲王母張婕妤、楚孝王母衛婕妤皆愛幸。

霍皇后廢後,上憐許太子蚤失母,幾為霍氏所害,於是乃選後宮素謹慎而無子者,遂立王婕妤為皇后,令母養太子。自為后後,希見無寵。封父奉光為邛成侯。立十六年,宣帝崩,元帝即位,為皇太后。封太后兄舜為安平侯。後二年,奉光薨,諡曰共侯,葬長門南,置園邑二百家,長丞奉守如法。元帝崩,成帝即位,為太皇太后。復爵太皇太后弟駿為關內侯,食邑千戶。王氏列侯二人,關內侯一人。舜子章,章從弟咸,皆至左右將軍。時成帝母亦姓王氏,故世號太皇太后為邛成太后。

邛成太后凡立四十九年,年七十餘,永始元年崩,合葬杜陵,稱東園。奉光孫勳坐法免。元始中,成帝太后下詔曰:「孝宣王皇后,朕之姑,深念奉質共脩之義,恩結于心。惟邛成共侯國廢祀絕,朕甚閔焉。其封共侯曾孫堅固為邛成侯。」至王莽乃絕。

孝元王皇后,成帝母也。家凡十侯,五大司馬,外戚莫盛焉。自有傳。

孝成許皇后,大司馬車騎將軍平恩侯嘉女也。元帝悼傷母恭哀后居位日淺而遭霍氏之辜,故選嘉女以配皇太子。初入太子家,上令中常侍黃門親近者侍送,還白太子懽說狀,元帝喜謂左右:「酌酒賀我!」左右皆稱萬歲。久之,有一男,失之。及成帝即位,立許妃為皇后,復生一女,失之。

初后父嘉自元帝時為大司馬車騎將軍輔政,已八九年矣。及成帝立,復以元舅陽平侯王鳳為大司馬大將軍,與嘉並。杜欽以為故事后父重於帝舅,乃說鳳曰:「車騎將軍至貴,將軍宜尊重之敬之,無失其意。蓋輕細微眇之漸,必生乖忤之患,不可不慎。衛將軍之日盛於蓋侯,近世之事,語尚在於長老之耳,唯將軍察焉。」久之,上欲專委任鳳,乃策嘉曰:「將軍家重身尊,不宜以吏職自絫。賜黃金二百斤,以特進侯就朝位。」後歲餘薨,諡曰恭侯。

后聰慧,善史書,自為妃至即位,常寵於上,後宮希得進見。皇太后及帝舅憂上無繼嗣,時又數有災異,劉向、谷永等皆陳其咎在於後宮。上然其言。於是省減椒房掖廷用度。皇后乃上疏曰:

妾誇布服糲食,加以幼稚愚惑,不明義理,幸得免離茅屋之下,備後宮埽除。蒙過誤之寵,居非命所當託,洿穢不修,曠職尸官,數逆至法,踰越制度,當伏放流之誅,不足以塞責。乃壬寅日大長秋受詔:「椒房儀法,御服輿駕,所發諸官署,及所造作,遺賜外家群臣妾,皆如竟寧以前故事。」妾伏自念,入椒房以來,遺賜外家未嘗踰故事,每輒決上,可覆問也。今誠時世異制,長短相補,不出漢制而已,纖微之間,未必可同。若竟寧前與黃龍前,豈相放哉?家吏不曉,今壹受詔如此,且使妾搖手不得。今言無得發取諸官,殆謂未央宮不屬妾,不宜獨取也。言妾家府亦不當得,妾竊惑焉。幸得賜湯沐邑以自奉養,亦小發取其中,何害於誼而不可哉?又詔書言服御所造,皆如竟寧前,吏誠不能揆其意,即且令妾被服所為不得不如前。設妾欲作某屏風張於某所,曰故事無有,或不能得,則必繩妾以詔書矣。此二事誠不可行,唯陛下省察。

官吏忮佷,必欲自勝。幸妾尚貴時,猶以不急事操人,況今日日益侵,又獲此詔,其操約人,豈有所訴?陛下見妾在椒房,終不肯給妾纖微內邪?若不私府小取,將安所仰乎?舊故,中宮乃私奪左右之賤繒,及發乘輿服繒,言為待詔補,已而伛易其中。左右多竊怨者,甚恥為之。又故事以特牛祠大父母,戴侯、敬侯皆得蒙恩以太牢祠,今當率如故事,唯陛下哀之!

今吏甫受詔讀記,直豫言使后知之,非可復若私府有所取也。其萌牙所以約制妾者,恐失人理。今但損車駕,及毋若未央宮有所發,遺賜衣服如故事,則可矣。其餘誠太迫急,奈何?妾薄命,端遇竟寧前。竟寧前於今世而比之,豈可耶?故時酒肉有所賜外家,輒上表乃決。又故杜陵梁美人歲時遺酒一石,肉百斤耳。妾甚少之,遺田八子誠不可若是。事率眾多,不可勝以文陳。俟自見,索言之,唯陛下深察焉!

上於是采劉向、谷永之言以報曰:

皇帝問皇后,所言事聞之。夫日者眾陽之宗,天光之貴,王者之象,人君之位也。夫以陰而侵陽,虧其正體,是非下陵上,妻乘夫,賤踰貴之變與?春秋二百四十二年,變異為眾,莫若日蝕大。自漢興,日蝕亦為呂、霍之屬見。以今揆之,豈有此等之效與?諸侯拘迫漢制,牧相執持之也,又安獲齊、趙七國之難?將相大臣褢誠秉忠,唯義是從,又惡有上官、博陸、宣成之謀?若乃徒步豪桀,非有陳勝、項梁之群也;匈奴、夷狄,非有冒頓、郅支之倫也。方外內鄉,百蠻賓服,殊俗慕義,八州懷德,雖使其懷挾邪意,猶不足憂,又況其無乎?求於夷狄無有,求於臣下無有,微後宮也當,何以塞之?

日者,建始元年正月,白氣出於營室。營室者,天子之後宮也。正月於尚書為皇極。皇極者,王氣之極也。白者西方之氣,其於春當廢。今正於王極之月,興廢氣於後宮,視后妾無能懷任保全者,以著繼嗣之微,賤人將起也。至其九月,流星如瓜,出於文昌,貫紫宮,尾委曲如龍,臨於鉤陳,此又章顯前尤,著在內也。其後則有北宮井溢,南流逆理,數郡水出,流殺人民。後則訛言傳相驚震,女童入殿,咸莫覺知。夫河者水陰,四瀆之長,今乃大決,沒漂陵邑,斯昭陰盛盈溢,違經絕紀之應也。乃昔之月,鼠巢于樹,野鵲變色。五月庚子,鳥焚其巢太山之域。《易》曰:「鳥焚其巢,旅人先笑後號咷。喪牛于易,凶。」言王者處民上,如鳥之處巢也,不顧卹百姓,百姓畔而去之,若鳥之自焚也,雖先快意說笑,其後必號而無及也。百姓喪其君,若牛亡其毛也,故稱凶。泰山,王者易姓告代之處,今正於岱宗之山,甚可懼也。三月癸未,大風自西搖祖宗寢廟,揚裂帷席,折拔樹木,頓僵車輦,毀壞檻屋,災及宗廟,足為寒心!四月己亥,日蝕東井,轉旋且索,與既無異。己猶戊也,亥復水也,明陰盛,咎在內。於戊己,虧君體,著絕世於皇極,顯禍敗及京都。於東井,變怪眾備,末重益大,來數益甚。成形之禍月以迫切,不救之患日寖婁深,咎敗灼灼若此,豈可以忽哉!

《書》云「高宗肜日,粵有雊雉。祖己曰:『惟先假王正厥事。』」又曰「雖休勿休,惟敬五刑,以成三德。」即飭椒房及掖庭耳。今皇后有所疑,便不便,其條刺,使大長秋來白之。吏拘於法,亦安足過?蓋矯枉者過宜,古今同之。且財帛之省,特牛之祠,其於皇后,所以扶助德美,為華寵也。咎根不除,災變相襲,祖宗且不血食,何戴侯也!傳不云乎?「以約失之者鮮。」審皇后欲從其奢與?朕亦當法孝武皇帝也,如此則甘泉、建章可復興矣。世俗歲殊,時變日化,遭事制宜,因時而移,舊之非者,何可放焉!君子之道,樂因循而重改作。昔魯人為長府,閔子騫曰:「仍舊貫如之何?何必改作!」蓋惡之也。《詩》云:「雖無老成人,尚有典刑,曾是莫聽,大命以傾。」孝文皇帝,朕之師也。皇太后,皇后成法也。假使太后在彼時不如職,今見親厚,又惡可以踰乎!皇后其刻心秉德,毋違先后之制度,力誼勉行,稱順婦道,減省群事,謙約為右。其孝東宮,毋闕朔望,推誠永究,爰何不臧!養名顯行,以息眾讙,垂則列妾,使有法焉。皇后深惟毋忽!

是時大將軍鳳用事,威權尤盛。其後,比三年日蝕,言事者頗歸咎於鳳矣。而谷永等遂著之許氏,許氏自知為鳳所不佑。久之,皇后寵亦益衰,而後宮多新愛。后姊平安剛侯夫人謁等為媚道祝謯後宮有身者王美人及鳳等,事發覺,太后大怒,下吏考問,謁等誅死,許后坐廢處昭臺宮,親屬皆歸故郡山陽,后弟子平恩侯旦就國。凡立十四年而廢,在昭臺歲餘,還徙長定宮。

後九年,上憐許氏,下詔曰:「蓋聞仁不遺遠,誼不忘親。前平安剛侯夫人謁坐大逆罪,家屬幸蒙赦令,歸故郡。朕惟平恩戴侯,先帝外祖,魂神廢棄,莫奉祭祀,念之未嘗忘于心。其還平恩侯旦及親屬在山陽郡者。」是歲,廢后敗。先是廢后姊缅寡居,與定陵侯淳于長私通,因為之小妻。長紿之曰:「我能白東宮,復立許后為左皇后。」廢后因缅私賂遺長,數通書記相報謝。長書有誖謾,發覺,天子使廷尉孔光持節賜廢后藥,自殺,葬延陵交道廄西。

孝成班婕妤,帝初即位選入後宮。始為少使,蛾而大幸,為婕妤,居增成舍,再就館,有男,數月失之。成帝遊於後庭,嘗欲與婕妤同輦載,婕妤辭曰:「觀古圖畫,賢聖之君皆有名臣在側,三代末主乃有嬖女,今欲同輦,得無近似之乎?」上善其言而止。太后聞之,喜曰:「古有樊姬,今有班婕妤。」婕妤誦詩及窈窕、德象、女師之篇。每進見上疏,依則古禮。

自鴻嘉後,上稍隆於內寵。婕妤進侍者李平,平得幸,立為婕妤。上曰:「始衛皇后亦從微起。」乃賜平姓曰衛,所謂衛婕妤也。其後趙飛燕姊弟亦從自微賤興,踰越禮制,寖盛於前。班婕妤及許皇后皆失寵,稀復進見。鴻嘉三年,趙飛燕譖告許皇后、班婕妤挾媚道,祝詛後宮,詈及主上。許皇后坐廢。考問班婕妤,婕妤對曰:「妾聞『死生有命,富貴在天。』修正尚未蒙福,為邪欲以何望?使鬼神有知,不受不臣之愬;如其無知,愬之何益?故不為也。」上善其對,憐憫之,賜黃金百斤。

趙氏姊弟驕妒,婕妤恐久見危,求共養太后長信宮,上許焉。婕妤退處東宮,作賦自傷悼,其辭曰:

承祖考之遺德兮,何性命之淑靈,登薄軀於宮闕兮,充下陳於後庭。蒙聖皇之渥惠兮,當日月之盛明,揚光烈之翕赫兮,奉隆寵於增成。既過幸於非位兮,竊庶幾乎嘉時,每寤寐而絫息兮,申佩離以自思,陳女圖以鏡監兮,顧女史而問詩。悲晨婦之作戒兮,哀褒、閻之為郵;美皇、英之女虞兮,榮任、姒之母周。雖愚陋其靡及兮,敢舍心而忘茲?歷年歲而悼懼兮,閔蕃華之不滋。痛陽祿與柘館兮,仍繈褓而離災,豈妾人之殃咎兮?將天命之不可求。

白日忽已移光兮,遂晻莫而昧幽,猶被覆載之厚德兮,不廢捐於罪郵。奉共養于東宮兮,託長信之末流,共洒埽於帷幄兮,永終死以為期。願歸骨於山足兮,依松柏之餘休。

重曰:潛玄宮兮幽以清,應門閉兮禁闥扃。華殿塵兮玉階労,中庭萋兮綠草生。廣室陰兮帷幄暗,房櫳虛兮風泠泠。感帷裳兮發紅羅,紛綷縩兮紈素聲。神眇眇兮密靚處,君不御兮誰為榮?俯視兮丹墀,思君兮履綦。仰視兮雲屋,雙涕兮橫流。顧左右兮和顏,酌羽觴兮銷憂。惟人生兮一世,忽一過兮若浮。已獨享兮高明,處生民兮極休。勉虞精兮極樂,與福祿兮無期。綠衣兮白華,自古兮有之。

至成帝崩,婕妤充奉園陵,薨,因葬園中。

孝成趙皇后,本長安宮人。初生時,父母不舉,三日不死,乃收養之。及壯,屬陽阿主家,學歌舞,號曰飛燕。成帝嘗微行出,過陽阿主,作樂。上見飛燕而說之,召入宮,大幸。有女弟復召入,俱為婕妤,貴傾後宮。

許后之廢也,上欲立趙婕妤。皇太后嫌其所出微甚,難之。太后姊子淳于長為侍中,數往來傳語,得太后指,上立封趙婕妤父臨為成陽侯。後月餘,乃立婕妤為皇后。追以長前白罷昌陵功,封為定陵侯。

皇后既立,後寵少衰,而弟絕幸,為昭儀。居昭陽舍,其中庭彤朱,而殿上觋漆,切皆銅沓冒黃金塗,白玉階,壁帶往往為黃金釭,函藍田璧,明珠、翠羽飾之,自後宮未嘗有焉。姊弟顓寵十餘年,卒皆無子。

末年,定陶王來朝,王祖母傅太后私賂遺趙皇后、昭儀,定陶王竟為太子。

明年春,成帝崩。帝素彊,無疾病。是時楚思王衍、梁王立來朝,明旦當辭去,上宿供張白虎殿。又欲拜左將軍孔光為丞相,已刻侯印書贊。昏夜平善,鄉晨,傅恊拦欲起,因失衣,不能言,晝漏上十刻而崩。民間歸罪趙昭儀,皇太后詔大司馬莽、丞相大司空曰:「皇帝暴崩,群眾讙譁怪之。掖庭令輔等在後庭左右,侍燕迫近,雜與御史、丞相、廷尉治問皇帝起居發病狀。」趙昭儀自殺。

哀帝既立,尊趙皇后為皇太后,封太后弟侍中駙馬都尉欽為新成侯。趙氏侯者凡二人。後數月,司隸解光奏言:

臣聞許美人及故中宮史曹宮皆御幸孝成皇帝,產子,子隱不見。

臣遣從事掾業、史望驗問知狀者掖庭獄丞籍武,故中黃門王舜、吳恭、靳嚴,官婢曹曉、道房、張棄,故趙昭儀御者于客子、王偏、臧兼等,皆曰宮即曉子女,前屬中宮,為學事史,通詩,授皇后。房與宮對食,元延元年中宮語房曰:「陛下幸宮。」後數月,曉入殿中,見宮腹大,問宮。宮曰:「御幸有身。」其十月中,宮乳掖庭牛官令舍,有婢六人。中黃門田客持詔記,盛綠綈方底,封御史中丞印,予武曰:「取牛官令舍婦人新產兒,婢六人,盡置暴室獄,毋問兒男女,誰兒也!」武迎置獄。宮曰:「

善臧我兒胞,丞知是何等兒也!」後三日,客持詔記與武,問「兒死未?手書對牘背。」武即書對:「兒見在,未死。」有頃,客出曰:「上與昭儀大怒,奈何不殺?」武叩頭啼曰:「

不殺兒,自知當死;殺之,亦死!」即因客奏封事,曰:「陛下未有繼嗣,子無貴賤,唯留意!」奏入,客復持詔記予武曰:「今夜漏上五刻,持兒與舜,會東交掖門。」武因問客:「陛下得武書,意何如?」曰:「储也。」武以兒付舜。舜受詔,內兒殿中,為擇乳母,告「善養兒,且有賞。毋令漏泄!」舜擇棄為乳母,時兒生八九日。後三日,客復持詔記,封如前予武,中有封小綠篋,記曰:「告武以篋中物書予獄中婦人,武自臨飲之。」武發篋中有裹藥二枚,赫蹄書,曰「告偉能:努力飲此藥,不可復入。女自知之!」偉能即宮。宮讀書已,曰:「果也,欲姊弟擅天下!我兒男也,額上有壯髮,類孝元皇帝。今兒安在?危殺之矣!奈何令長信得聞之?」宮飲藥死。後宮婢六人召入,出語武曰:「昭儀言『女無過。寧自殺邪,若外家也?』我曹言願自殺。」即自繆死。武皆表奏狀。棄所養兒十一日,宮長李南以詔書取兒去,不知所置。

許美人前在上林涿沐館,數召入飾室中若舍,一歲再三召,留數月或半歲御幸。元延二年褱子,其十一月乳。詔使嚴持乳毉及五種和藥丸三,送美人所。後客子、偏、兼聞昭儀謂成帝曰:「常紿我言從宮中來,即從中宮來,許美人兒何從生中?許氏竟當復立邪!」懟,以手自擣,以頭擊壁戶柱,從床上自投地,啼泣不肯食,曰:「今當安置我,欲歸耳!」帝曰:「今故告之,反怒為!殊不可曉也。」帝亦不食。昭儀曰:「陛下自知是,不食為何?陛下常自言『約不負女』,今美人有子,竟負約,謂何?」帝曰:「約以趙氏,故不立許氏。使天下無出趙氏上者,毋憂也!」後詔使嚴持綠囊書予許美人,告嚴曰:「美人當有以予女,受來,置飾室中簾南。」美人以葦篋一合盛所生兒,緘封,及綠囊報書予嚴。嚴持篋書,置飾室簾南去。帝與昭儀坐,使客子解篋緘。未已,帝使客子、偏、兼皆出,自閉戶,獨與昭儀在。須臾開戶,呼客子、偏、兼,使緘封篋及綠綈方底,推置屏風東。恭受詔,持篋方底予武,皆封以御史中丞印,曰:「告武:篋中有死兒,埋屏處,勿令人知。」武穿獄樓垣下為坎,埋其中。

故長定許貴人及故成都、平阿侯家婢王業、任孋、公孫習前免為庶人,詔召入,屬昭儀為私婢。成帝崩,未幸梓宮,倉卒悲哀之時,昭儀自知罪惡大,知業等故許氏、王氏婢,恐事泄,而以大婢羊子等賜予業等各且十人,以慰其意,屬無道我家過失。

元延二年五月,故掖庭令吾丘遵謂武曰:「掖庭丞吏以下皆與昭儀合通,無可與語者,獨欲與武有所言。我無子,武有子,是家輕族人,得無不敢乎?掖庭中御幸生子者輒死,又飲藥傷墯者無數,欲與武共言之大臣,票騎將軍貪耆錢,不足計事,奈何令長信得聞之?」遵後病困,謂武:「今我已死,前所語事,武不能獨為也,慎語!」

皆在今年四月丙辰赦令前。臣謹案永光三年男子忠等發長陵傅夫人冢。事更大赦,孝元皇帝下詔曰:「比朕不當所得赦也。」窮治,盡伏辜,天下以為當。魯嚴公夫人殺世子,齊桓召而誅焉,春秋予之。趙昭儀傾亂聖朝,親滅繼嗣,家屬當伏天誅。前平安剛侯夫人謁坐大逆,同產當坐,以蒙赦令,歸故郡。今昭儀所犯尤誖逆,罪重於謁,而同產親屬皆在尊貴之位,迫近幃幄,群下寒心,非所以懲惡崇誼示四方也。請事窮竟,丞相以下議正法。

哀帝於是免新成侯趙欽、欽兄子成陽侯訢,皆為庶人,將家屬徙遼西郡。時議郎耿育上疏言:

臣聞繼嗣失統,廢適立庶,聖人法禁,古今至戒。然大伯見歷知適,逡循固讓,委身吳粵,權變所設,不計常法,致位王季,以崇聖嗣,卒有天下,子孫承業,七八百載,功冠三王,道德最備,是以尊號追及大王。故世必有非常之變,然後乃有非常之謀。孝成皇帝自知繼嗣不以時立,念雖末有皇子,萬歲之後未能持國,權柄之重,制於女主,女主驕盛則耆欲無極,少主幼弱則大臣不使,世無周公抱負之輔,恐危社稷,傾亂天下。知陛下有賢聖通明之德,仁孝子愛之恩,懷獨見之明,內斷於身,故廢後宮就館之漸,絕微嗣禍亂之根,乃欲致位陛下以安宗廟。愚臣既不能深援安危,定金匱之計,又不知推演聖德,述先帝之志,乃反覆校省內,暴露私燕,誣汙先帝傾惑之過,成結寵妾妒媚之誅,甚失賢聖遠見之明,逆負先帝憂國之意。

夫論大德不拘俗,立大功不合眾,此乃孝成皇帝至思所以萬萬於眾臣,陛下聖德盛茂所以符合於皇天也,豈當世庸庸斗筲之臣所能及哉!且褒廣將順君父之美,匡捄銷滅既往之過,古今通義也。事不當時固爭,防禍於未然,各隨指阿從,以求容媚,晏駕之後,尊號已定,萬事已訖,乃探追不及之事,訐揚幽昧之過,此臣所深痛也!

願下有司議,即如臣言,宜宣布天下,使咸曉知先帝聖意所起。不然,空使謗議上及山陵,下流後世,遠聞百蠻,近布海內,甚非先帝託後之意也。蓋孝子善述父之志,善成人之事,唯陛下省察!

哀帝為太子,亦頗得趙太后力,遂不竟其事。傅太后恩趙太后,趙太后亦歸心,故成帝母及王氏皆怨之。

哀帝崩,王莽白太后詔有司曰:「前皇太后與昭儀俱侍帷幄,姊弟專寵錮寢,執賊亂之謀,殘滅繼嗣以危宗廟,誖天犯祖,無為天下母之義。貶皇太后為孝成皇后,徙居北宮。」後月餘,復下詔曰:「皇后自知罪惡深大,朝請希闊,失婦道,無共養之禮,而有狼虎之毒,宗室所怨,海內之讎也,而尚在小君之位,誠非皇天之心。夫小不忍亂大謀,恩之所不能已者義之所割也,今廢皇后為庶人,就其園。」是日自殺。凡立十六年而誅。先是有童謠曰:「燕燕,尾龚龚,張公子,時相見。木門倉琅根,燕飛來,啄皇孫。皇孫死,燕啄矢。」成帝每微行出,常與張放俱,而稱富平侯家,故曰張公子。倉琅根,宮門銅鍰也。

孝元傅昭儀,哀帝祖母也。父河內溫人,蚤卒,母更嫁為魏郡鄭翁妻,生男惲。昭儀少為上官太后才人,自元帝為太子,得進幸。元帝即位,立為婕妤,甚有寵。為人有材略,善事人,下至宮人左右,飲酒酹地,皆祝延之。產一男一女,女為平都公主,男為定陶恭王。恭王有材藝,尤愛於上。元帝既重傅婕妤,及馮婕妤亦幸,生中山孝王,上欲殊之於後宮,以二人皆有子為王,上尚在,未得稱太后,乃更號曰昭儀,賜以印綬,在婕妤上。昭其儀,尊之也。至成、哀時,趙昭儀、董昭儀皆無子,猶稱焉。

元帝崩,傅昭儀隨王歸國,稱定陶太后。後十年,恭王薨,子代為王。王母曰丁姬。傅太后躬自養視,既壯大,成帝無繼嗣。時中山孝王在。元延四年,孝王及定陶王皆入朝。傅太后多以珍寶賂遺趙昭儀及帝舅票騎將軍王根,陰為王求漢嗣。皆見上無子,欲豫自結為久長計,更稱譽定陶王。上亦自器之,明年,遂徵定陶王立為太子,語在哀紀。月餘,天子立楚孝王孫景為定陶王,奉恭王後。太子議欲謝,少傅閻崇以為「春秋不以父命廢王父命,為人後之禮不得顧私親,不當謝。」太傅趙玄以為當謝,太子從之。詔問所以謝狀,尚書劾奏玄,左遷少府,以光祿勳師丹為太傅。詔傅太后與太子母丁姬自居定陶國邸。下有司議皇太子得與傅太后、丁姬相見不,有司奏議不得相見。頃之,成帝母王太后欲令傅太后、丁姬十日一至太子家,成帝曰:「太子丞正統,當共養陛下,不得復顧私親。」王太后曰:「太子小,而傅太后抱養之,今至太子家,以乳母恩耳,不足有所妨。」於是令傅太后得至太子家。丁姬以不小養太子,獨不得。

成帝崩,哀帝即位。王太后詔令傅太后、丁姬十日一至未央宮。高昌侯董宏希指,上書言宜立丁姬為帝太后。師丹劾奏「宏懷邪誤朝,不道。」上初即位,謙讓,從師丹言止。後乃白令王太后下詔,尊定陶恭王為恭皇。哀帝因是曰:「春秋『母以子貴』,尊傅太后為恭皇太后,丁姬為恭皇后,各置左右詹事,食邑如長信宮、中宮。追尊恭皇太后父為崇祖侯,恭皇后父為褒德侯。」後歲餘,遂下詔曰:「漢家之制,推親親以顯尊尊,定陶恭皇之號不宜復稱定陶。其尊恭皇太后為帝太太后,丁后為帝太后。」後又更號帝太太后為皇太太后,稱永信宮,帝太后稱中安宮,而成帝母太皇太后本稱長信宮,成帝趙后為皇太后,並四太后,各置少府、太僕,秩皆中二千石。為恭皇立寢廟於京師,比宣帝父悼皇考制度,序昭穆於前殿。

傅太后父同產弟四人,曰子孟、中叔、子元、幼君。子孟子喜至大司馬,封高武侯。中叔子晏亦大司馬,封孔鄉侯。幼君子商封汝昌侯,為太后父崇祖侯後,更號崇祖曰汝昌哀侯。太后同母弟鄭惲前死,以惲子業為陽信侯,追尊惲為陽信節侯。鄭氏、傅氏侯者凡六人,大司馬二人,九卿二千石六人,侍中諸曹十餘人。

傅太后既尊,後尤驕,與成帝母語,至謂之嫗。與中山孝王母馮太后並事元帝,追怨之,陷以祝詛罪,令自殺。元壽元年崩,合葬渭陵,稱孝元傅皇后云。

定陶丁姬,哀帝母也,易祖師丁將軍之玄孫。家在山陽瑕丘,父至廬江太守。始定陶恭王先為山陽王,而丁氏內其女為姬。王后姓張氏,其母鄭禮,即傅太后同母弟也。太后以親戚故,欲其有子,然終無有。唯丁姬河平四年生哀帝。丁姬為帝太后,兩兄忠、明。明以帝舅封陽安侯。忠蚤死,封忠子滿為平周侯。太后叔父憲、望。望為左將軍,憲為太僕。明為大司馬票騎將軍輔政。丁氏侯者凡二人,大司馬一人,將軍、九卿、二千石六人,等中諸曹亦十餘人。丁、傅以一二年間暴興尤盛。然哀帝不甚假以權勢,權勢不如王氏在成帝世也。

建平二年,丁太后崩。上曰:「《詩》云『穀則異室,死則同穴』。昔季武子成寢,杜氏之墓在西階下,請合葬而許之。附葬之禮,自周興焉。孝子事亡如事存,帝太后宜起陵恭皇之園。」遣大司馬票騎將軍明東送葬于定陶,貴震山東。

哀帝崩,王莽秉政,使有司舉奏丁、傅罪惡。莽以太皇太后詔皆免官爵,丁氏徙歸故郡。莽奏貶傅太后號為定陶共王母,丁太后號曰丁姬。

元始五年,莽復言「共王母、丁姬前不臣妾,至葬渭陵,冢高與元帝山齊,懷帝太后、皇太太后璽綬以葬,不應禮。禮有改葬,請發共王母及丁姬冢,取其璽綬消滅,徙共王母及丁姬歸定陶,葬共王冢次,而葬丁姬復其故。」太后以為既已之事,不須復發。莽固爭之,太后詔曰:「因故棺為致槨作冢,祠以太牢。」謁者護既發傅太后冢,崩壓殺數百人;開丁姬槨戶,火出炎四五丈,吏卒以水沃滅乃得入,燒燔槨中器物。

莽復奏言:「前共王母生,僭居桂宮,皇天震怒,災其正殿;丁姬死,葬踰制度,今火焚其槨。此天見變以告,當改如媵妾也。臣前奏請葬丁姬復故,非是。共王母及丁姬棺皆名梓宮,珠玉之衣非藩妾服,請更以木棺代,去珠玉衣,葬丁姬媵妾之次。」奏可。既開傅太后棺,臭聞數里。公卿在位皆阿莽指,入錢帛,遣子弟及諸生四夷,凡十餘萬人,操持作具,助將作掘平共王母、丁姬故冢,二旬間皆平。莽又周棘其處以為世戒云。時有群燕數千,銜土投丁姬穿中。丁、傅既敗,孔鄉侯晏將家屬徙合浦,宗族皆歸故郡。唯高武侯喜得全,自有傳。

孝哀傅皇后,定陶太后從弟子也。哀帝為定陶王時,傅太后欲重親,取以配王。王入為漢太子,傅氏女為妃。哀帝即位,成帝大行尚在前殿,而傅太后封傅妃父晏為孔鄉侯,與帝舅陽安侯丁明同日俱封。時師丹諫,以為「天下自王者所有,親戚何患不富貴?而倉卒若是,其不久長矣!」晏封後月餘,傅妃立為皇后。傅氏既盛,晏最尊重。哀帝崩,王莽白太皇太后詔曰:「定陶共王太后與孔鄉侯晏同心合謀,背恩忘本,專恣不軌,與至尊同稱號,終沒,至乃配食於左坐,誖逆無道。今令孝哀皇后退就桂宮。」後月餘,復與孝成趙皇后俱廢為庶人,就其園自殺。

孝元馮昭儀,平帝祖母也。元帝即位二年,以選入後宮。時父奉世為執金吾。昭儀始為長使,數月至美人,後五年就館生男,拜為婕妤。時父奉世為右將軍光祿勳,奉世長男野王為左馮翊,父子並居朝廷,議者以為器能當其位,非用女寵故也。而馮婕妤內寵與傅昭儀等。

建昭中,上幸虎圈鬥獸,後宮皆坐。熊佚出圈,攀檻欲上殿。左右貴人傅昭儀等皆驚走,馮婕妤直前當熊而立,左右格殺熊。上問:「人情驚懼,何故前當熊?」婕妤對曰:「猛獸得人而止,妾恐熊至御坐,故以身當之。」元帝嗟嘆,以此倍敬重焉。傅昭儀等皆慚。明年夏,馮婕妤男立為信都王,尊婕妤為昭儀。元帝崩,為信都太后,與王俱居儲元宮。河平中,隨王之國。後徙中山,是為孝王。

後徵定陶王為太子,封中山王舅參為宜鄉侯。參,馮太后少弟也。是歲,孝王薨,有一男,嗣為王,時未滿歲,有眚病,太后自養視,數禱祠解。

哀帝即位,遣中郎謁者張由將毉治中山小王。由素有狂易病,病發怒去,西歸長安。尚書簿責擅去狀,由恐,因誣言中山太后祝詛上及太后。太后即傅昭儀也,素常怨馮太后,因是遣御史丁玄案驗,盡收御者官吏及馮氏昆弟在國者百餘人,分繫雒陽、魏郡、鉅鹿。數十日無所得,更使中謁者令史立與丞相長史大鴻臚丞雜治。立受傅太后指,幾得封侯,治馮太后女弟習及寡弟婦君之,死者數十人。巫劉吾服祝詛。毉徐遂成言習、君之曰「武帝時毉修氏刺治武帝得二千萬耳,今愈上,不得封侯,不如殺上,令中山王代,可得封。」立等劾奏祝詛謀反,大逆。責問馮太后,無服辭。立曰:「熊之上殿何其勇,今何怯哉!」太后還謂左右:「此乃中語,前世事,吏何用知之?是欲陷我效也!」乃飲藥自殺。

先未死,有司請誅之,上不忍致法,廢為庶人,徙雲陽宮。既死,有司復奏「太后死在未廢前。」有詔以諸侯王太后儀葬之。宜鄉侯參、君之、習夫及子當相坐者,或自殺,或伏法。參女弁為孝王后,有兩女,有司奏免為庶人,與馮氏宗族徙歸故郡。張由以先告賜爵關內侯,史立遷中太僕。

哀帝崩,大司徒孔光奏「由前誣告骨肉,立陷人入大辟,為國家結怨於天下,以取秩遷,獲爵邑,幸蒙赦令,請免為庶人,徙合浦」云。

中山衛姬,平帝母也。父子豪,中山盧奴人,官至衛尉。子豪女弟為宣帝婕妤,生楚孝王;長女又為元帝婕妤,生平陽公主。成帝時,中山孝王無子,上以衛氏吉祥,以子豪少女配孝王。元延四年,生平帝。

年二歲,孝王薨,代為王。哀帝崩,無嗣,太皇太后與新都侯莽迎中山王立為帝。莽欲顓國權,懲丁、傅行事,以帝為成帝後,母衛姬及外家不當得至京師。乃更立宗室桃鄉侯子成都為中山王,奉孝王後,遣少傅左將軍甄豐賜衛姬璽綬,即拜為中山孝王后,以苦陘縣為湯沐邑。又賜帝舅衛寶、寶弟玄爵關內侯。賜帝三妹,謁臣號修義君,哉皮為承禮君,鬲子為尊德君,食邑各二千戶。莽長子宇非莽隔絕衛氏,恐久後受禍,即私與衛寶通書記,教衛后上書謝恩,因陳丁、傅舊惡,幾得至京師。莽白太皇太后詔有司曰:「中山孝王后深分明為人後之義,條陳故定陶傅太后、丁姬誖天逆理,上僭位號,徙定陶王於信都,為共王立廟於京師,如天子制,不畏天命,侮聖人言,壞亂法度,居非其制,稱非其號。是以皇天震怒,火燒其殿,六年之間大命不遂,禍殃仍重,竟令孝哀帝受其餘災,大失天心,夭命暴崩,又令共王祭祀絕廢,精魂無所依歸。朕惟孝王后深說經義,明鏡聖法,懼古人之禍敗,近事之咎殃,畏天命,奉聖言,是乃久保一國,長獲天祿,而令孝王永享無彊之祀,福祥之大者也。朕甚嘉之。夫褒義賞善,聖王之制,其以中山故安戶七千益中山后湯沐邑,加賜及中山王黃金各百斤,增傅相以下秩。」

衛后日夜啼泣,思見帝,而但益戶邑。宇復教令上書求至京師。會事發覺,莽殺宇,盡誅衛氏支屬。衛寶女為中山王后,免后,徙合浦。唯衛后在,王莽篡國,廢為家人,後歲餘卒,葬孝王旁。

孝平王皇后,安漢公太傅大司馬莽女也。平帝即位,年九歲,成帝母太皇太后稱制,而莽秉政。莽欲依霍光故事,以女配帝,太后意不欲也。莽設變詐,令女必入,因以自重,事在莽傳。太后不得已而許之,遣長樂少府夏侯藩、宗正劉宏、少府宗伯鳳、尚書令平晏納采,太師光、大司徒馬宮、大司空甄豐、左將軍孫建、執金吾尹賞、行太常事太中大夫劉歆及太卜、太史令以下四十九人賜皮弁素績,以禮雜卜筮,太牢祠宗廟,待吉月日。明年春,遣大司徒宮、大司空豐、左將軍建、右將軍甄邯、光祿大夫歆奉乘輿法駕,迎皇后於安漢公第。宮、豐、歆授皇后璽紱,登車稱警蹕,便時上林延壽門,入未央宮前殿。群臣就位行禮,大赦天下。益封父安漢公地滿百里,賜迎皇后及行禮者,自三公以下至騶宰執事長樂、未央宮、安漢公第者,皆增秩,賜金帛各有差。皇后立三月,以禮見高廟。尊父安漢公號曰宰衡,位在諸侯王上。賜公夫人號曰功顯君,食邑。封公子安為褒新侯,臨為賞都侯。

后立歲餘,平帝崩。莽立孝宣帝玄孫嬰為孺子,莽攝帝位,尊皇后為皇太后。三年,莽即真,以嬰為定安公,改皇太后號為定安公太后。太后時年十八矣,為人婉瘱有節操。自劉氏廢,常稱疾不朝會。莽敬憚傷哀,欲嫁之,乃更號為黃皇室主,令立國將軍成新公孫建世子襐飾將毉往問疾。后大怒,笞鞭其旁侍御。因發病,不肯起,莽遂不復彊也。及漢兵誅莽,燔燒未央宮,后曰:「

何面目以見漢家!」自投火中而死。

贊曰:易著吉凶而言謙盈之效,天地鬼神至于人道靡不同之。夫女寵之興,繇至微而體至尊,窮富貴而不以功,此固道家所畏,禍福之宗也。序自漢興,終于孝平,外戚後庭色寵著聞二十有餘人,然其保位全家者,唯文、景、武帝太后及邛成后四人而已。至如史良娣、王悼后、許恭哀后身皆夭折不辜,而家依託舊恩,不敢縱恣,是以能全。其餘大者夷滅,小者放流,烏呼!鑒茲行事,變亦備矣。


\end{pinyinscope}