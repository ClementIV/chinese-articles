\article{高五王傳}

\begin{pinyinscope}
高皇帝八男:呂后生孝惠帝,曹夫人生齊悼惠王肥,薄姬生孝文帝,戚夫人生趙隱王如意,趙姬生淮南厲王長,諸姬生趙幽王友、趙共王恢、燕靈王建。淮南厲王長自有傳。

齊悼惠王肥,其母高祖微時外婦也。高祖六年立,食七十餘城。諸民能齊言者皆與齊。孝惠二年,入朝。帝與齊王燕飲太后前,置齊王上坐,如家人禮。太后怒,乃令人酌兩卮鴆酒置前,令齊王為壽。齊王起,帝亦起,欲俱為壽。太后恐,自起反卮。齊王怪之,因不敢飲,陽醉去。問知其鴆,乃憂,自以為不得脫長安。內史士曰:「太后獨有帝與魯元公主,今王有七十餘城,而公主乃食數城。王誠以一郡上太后為公主湯沐邑,太后必喜,王無患矣。」於是齊王獻城陽郡以尊公主為王太后。呂太后喜而許之。乃置酒齊邸,樂飲,遣王歸國。後十三年薨,子襄嗣。

趙隱王如意,九年立。四年,高祖崩,呂太后徵王到長安,鴆殺之。無子,絕。

趙幽王友,十一年立為淮陽王。趙隱王如意死,孝惠元年,徙友王趙,凡立十四年。友以諸呂女為后,不愛,愛它姬。諸呂女怒去,讒之於太后曰:「王曰『呂氏安得王?太后百歲後,吾必擊之。』」太后怒,以故召趙王。趙王至,置邸不見,令衛圍守之,不得食。其群臣或竊饋之,輒捕論之。趙王餓,乃歌曰:「諸呂用事兮,劉氏微;迫脅王侯兮,彊授我妃。我妃既妒兮,誣我以惡;讒女亂國兮,上曾不寤。我無忠臣兮,何故棄國?自快中野兮,蒼天與直!于嗟不可悔兮,寧早自賊!為王餓死兮,誰者憐之?呂氏絕理兮,託天報仇!」遂幽死。以民禮葬之長安。

高后崩,孝文即位,立幽王子遂為趙王。二年,有司請立皇子為王。上曰:「趙幽王幽死,朕甚憐之。已立其長子遂為趙王。遂弟辟彊及齊悼惠王子朱虛侯章、東牟侯興居有功,皆可王。」於是取趙之河間立辟彊,是為河間文王。文王立十三年薨,子哀王福嗣。一年薨。無子,國除。

趙王遂立二十六年,孝景時晁錯以過削趙常山郡,諸侯怨,吳楚反,遂與合謀起兵。其相建德、內史王悍諫,不聽。遂燒殺德、悍,發兵住其西界,欲待吳楚俱進,北使匈奴與連和。漢使曲周侯酈寄擊之,趙王城守邯鄲,相距七月。吳楚敗,匈奴聞之,亦不肯入邊。欒布自破齊還,并兵引水灌趙城。城壞,王遂自殺,國除。景帝憐趙相、內史守正死,皆封其子為列侯。

趙共王恢。十一年,梁王彭越誅,立恢為梁王。十六年,趙幽王死,呂后徙恢王趙,恢心不樂。太后以呂產女為趙王后,王后從官皆諸呂也,內擅權,微司趙王,王不得自恣。王有愛姬,王后鴆殺之。王乃為歌詩四章,令樂人歌之。王悲思,六月自殺。太后聞之,以為用婦人故自殺,無思奉宗廟禮,廢其嗣。

齊悼惠王子,前後凡九人為王:太子襄為齊哀王,次子章為城陽景王,興居為濟北王,將閭為齊王,志為濟北王,辟光為濟南王,賢為菑川王,卬為膠西王,雄渠為膠東王。

齊哀王襄,孝惠六年嗣立。明年,惠帝崩,呂太后稱制。元年,以其兄子鄜侯呂台為呂王,割齊之濟南郡為呂王奉邑。明年,哀王弟章入宿衛於漢,高后封為朱虛侯,以呂祿女妻之。後四年,封章弟興居為東牟侯,皆宿衛長安。高后七年,割齊琅邪郡,立營陵侯劉澤為琅邪王。是歲,趙王友幽死于邸。三趙王既廢,高后立諸呂為三王,擅權用事。

章年二十,有氣力,忿劉氏不得職。嘗入侍燕飲,高后令章為酒吏。章自請曰:「臣,將種也,請得以軍法行酒。」高后曰:「可。」酒酣,章進歌舞,已而曰:「請為太后言耕田。」高后兒子畜之,笑曰:「顧乃父知田耳,若生而為王子,安知田乎?」章曰:「臣知之。」太后曰:「試為我言田意。」章曰:「深耕穊種,立苗欲疏;非其種者,鉏而去之。」太后默然。頃之,諸呂有一人醉,亡酒,章追,拔劍斬之,而還報曰:「有亡酒一人,臣謹行軍法斬之。」太后左右大驚。業已許其軍法,亡以罪也。因罷酒。自是後,諸呂憚章,雖大臣皆依朱虛侯。劉氏為彊。

其明年,高后崩。趙王呂祿為上將軍,呂王產為相國,皆居長安中,聚兵以威大臣,欲為亂。章以呂祿女為婦,知其謀,乃使人陰出告其兄齊王,欲令發兵西,朱虛侯、東牟侯欲從中與大臣為內應,以誅諸呂,因立齊王為帝。

齊王聞此計,與其舅駟鈞、郎中令祝午、中尉魏勃陰謀發兵。齊相召平聞之,乃發兵入衛王宮。魏勃紿平曰:「王欲發兵,非有漢虎符驗也。而相君圍王,固善。勃請為君將兵衛衛王。」召平信之,乃使魏勃將。勃既將,以兵圍相府。召平曰:「嗟乎!道家之言『當斷不斷,反受其亂』。」遂自殺。於是齊王以駟鈞為相,魏勃為將軍,祝午為內史,悉發國中兵。使祝午紿琅邪王曰:「呂氏為亂,齊王發兵欲西誅之。齊王自以兒子,年少,不習兵革之事,願舉國委大王。大王自高帝將也,習戰事。齊王不敢離兵,使臣請大王幸之臨菑見齊王計事,并將齊兵以西平關中之亂。」琅邪王信之,以為然,乃馳見齊王。齊王與魏勃等因留琅邪王,而使祝午盡發琅邪國而并將其兵。

琅邪王劉澤既欺,不得反國,乃說齊王曰:「齊悼惠王,高皇帝長子也,推本言之,大王高皇帝適長孫也,當立。今諸大臣狐疑未有所定,而澤於劉氏最為長年,大臣固待澤決計。今大王留臣無為也,不如使我入關計事。」齊王以為然,乃益具車送琅邪王。

琅邪王既行,齊遂舉兵西攻呂國之濟南。於是齊王遺諸侯王書曰:「高帝平定天下,王諸子弟。悼惠王薨,惠帝使留侯張良立臣為齊王。惠帝崩,高后用事,春秋高,聽諸呂擅廢帝更立,又殺三趙王,滅梁、趙、燕,以王諸呂,分齊國為四。忠臣進諫,上或亂不聽。今高后崩,皇帝春秋富,未能治天下,固待大臣諸侯。今諸呂又擅自尊官,聚官嚴威,劫列侯忠臣,撟制以令天下,宗廟以危。寡人帥兵入誅不當為王者。」

漢聞之,相國呂產等遣大將軍潁陰侯灌嬰將兵擊之。嬰至滎陽,乃謀曰:「諸呂舉兵關中,欲危劉氏而自立,今我破齊還報,是益呂氏資也。」乃留兵屯滎陽,使人諭齊王及諸侯,與連和,以待呂氏之變而共誅之。齊王聞之,乃屯兵西界待約。

呂祿、呂產欲作亂,朱虛侯章與太尉勃、丞相平等誅之。章首先斬呂產,太尉勃等乃盡誅諸呂。而琅邪王亦從齊至長安。

大臣議欲立齊王,皆曰:「母家駟鈞惡戾,虎而冠者也。訪以呂氏故,幾亂天下,今又立齊王,是欲復為呂氏也。代王母家薄氏,君子長者,且代王,高帝子,於今見在,最為長。以子則順,以善人則大臣安。」於是大臣乃謀迎代王,而遣章以誅呂氏事告齊王,令罷兵。

灌嬰在滎陽,聞魏勃本教齊王反,既誅呂氏,罷齊兵,使使召責問魏勃。勃曰:「失火之家,豈暇先言丈人後救火乎!」因退立,股戰而栗。恐不能言者,終無他語。灌將軍孰視,笑曰:「人謂魏勃勇,妄庸人耳,何能為乎!」乃罷勃。勃父以善鼓琴見秦皇帝。及勃少時,欲求見齊相曹參,家貧無以自通,乃常獨早埽齊相舍人門外。舍人怪之,以為物而司之,得勃。勃曰:「

願見相君無因,故為子埽,欲以求見。」於是舍人見勃,曹參因以為舍人。壹為參御言事,以為賢,言之悼惠王。王召見,拜為內史。始悼惠王得自置二千石。及悼惠王薨,哀王嗣,勃用事重於相。

齊王既罷兵歸,而代王立,是為孝文帝。

文帝元年,盡以高后時所割齊之城陽、琅邪、濟南郡復予齊,而徙琅邪王王燕。益封朱虛侯、東牟侯各二千戶,黃金千斤。

是歲,齊哀王薨,子文王則嗣。十四年薨,無子,國除。

城陽景王章,孝文二年以朱虛侯與東牟侯興居俱立,二年薨。子共王喜嗣。孝文十二年,徙王淮南,五年,復還王城陽,凡立三十三年薨。子頃王延嗣,二十六年薨。子敬王義嗣,九年薨。子惠王武嗣,十一年薨。子荒王順嗣,四十六年薨。子戴王恢嗣,八年薨。子孝王景嗣,二十四年薨。子哀王雲嗣,一年薨,無子,國絕。成帝復立雲兄俚為城陽王,王莽時絕。

濟北王興居初以東牟侯與大臣共立文帝於代邸,曰:「誅呂氏,臣無功,請與太僕滕公俱入清宮。」遂將少帝出,迎皇帝入宮。

始誅諸呂時,朱虛侯章功尤大,大臣許盡以趙地王章,盡以梁地王興居。及文帝立,聞朱虛、東牟之初欲立齊王,故黜其功。二年,王諸子,乃割齊二郡以王章、興居。章、興居意自以失職奪功。歲餘,章薨,而匈奴大入邊,漢多發兵,丞相灌嬰將擊之,文帝親幸太原。興居以為天子自擊胡,遂發兵反。上聞之,罷兵歸長安,使棘蒲侯柴將軍擊破,虜濟北王。王自殺,國除。

文帝憫濟北王逆亂以自滅,明年,盡封悼惠王諸子罷軍等七人為列侯。至十五年,齊文王又薨,無子。時悼惠王後尚有城陽王在,文帝憐悼惠王適嗣之絕,於是乃分齊為六國,盡立前所封悼惠王子列侯見在者六人為王。齊孝王將閭以楊虛侯立,濟北王志以安都侯立,菑川王賢以武成侯立,膠東王雄渠以白石侯立,膠西王卬以平昌侯立,濟南王辟光以扐侯立。孝文十六年,六王同日俱立。

立十一年,孝景三年,吳楚反,膠東、膠西、菑川、濟南王皆發兵應吳楚。欲與齊,齊孝王狐疑,城守不聽。三國兵共圍齊,齊王使路中大夫告於天子。天子復令路中大夫還報,告齊王堅守,漢兵今破吳楚矣。路中大夫至,三國兵圍臨菑數重,無從入。三國將與路中大夫盟曰:「若反言漢已破矣,齊趣下三國,不且見屠。」路中大夫既許,至城下,望見齊王,曰:「漢已發兵百萬,使太尉亞夫擊破吳楚,方引兵救齊,齊必堅守無下!」三國將誅路中大夫。

齊初圍急,陰與三國通謀,約未定,會路中大夫從漢來,其大臣乃復勸王無下三國。會漢將欒布、平陽侯等兵至齊,擊破三國兵,解圍。已後聞齊初與三國有謀,將欲移兵伐齊。齊孝王懼,飲藥自殺。而膠東、膠西、濟南、菑川王皆伏誅,國除。獨濟北王在。

齊孝王之自殺也,景帝聞之,以為齊首善,以迫劫有謀,非其罪也,召立孝王太子壽,是為懿王。二十三年薨,子厲王次昌嗣。

其母曰紀太后。太后取其弟紀氏女為王后,王不愛。紀太后欲其家重寵,令其長女紀翁主入王宮正其後宮無令得近王,欲令愛紀氏女。王因與其姊翁主姦。

齊有宦者徐甲,入事漢皇太后。皇太后有愛女曰脩成君,脩成君非劉氏子,太后憐之。脩成君有女娥,太后欲嫁之於諸侯。宦者甲乃請使齊,必令王上書請娥。皇太后大喜,使甲之齊。時主父偃知甲之使齊以取后事,亦因謂甲:「即事成,幸言偃女願得充王後宮。」甲至齊,風以此事。紀太后怒曰:「王有后,後宮備具。且甲,齊貧人,及為宦者入事漢,初無補益,乃欲亂吾王家!且主父偃何為者?乃欲以女充後宮!」甲大窮,還報皇太后曰:「王已願尚娥,然事有所害,恐如燕王。」燕王者,與其子昆弟姦,坐死。故以燕感太后。太后曰:「毋復言嫁女齊事。」事寖淫聞於上。主父偃由此與齊有隙。

偃方幸用事,因言:「齊臨菑十萬戶,巿租千金,人眾殷富,鉅於長安,非天子親弟愛子不得王此。今齊王於親屬益疏。」乃從容言呂太后時齊欲反,及吳楚時孝王幾為亂。今聞齊王與其姊亂。於是武帝拜偃為齊相,且正其事。偃至齊,急治王後宮宦者為王通於姊翁主所者,辭及王。王年少,懼以罪為吏所執誅,乃飲藥自殺。

是時趙王懼主父偃壹出敗齊,恐其漸疏骨肉,乃上書言偃受金及輕重之短,天子亦因囚偃。公孫弘曰:「齊王以憂死,無後,非誅偃無以塞天下之望。」偃遂坐誅。

厲王立五年,國除。

濟北王志,吳楚反時初亦與通謀,後堅守不發兵,故得不誅,徙王菑川。元朔中,齊國絕。

悼惠王後唯有二國:城陽、菑川。菑川地比齊,武帝為悼惠王冢園在齊,乃割臨菑東圜悼惠王冢園邑盡以予菑川,令奉祭祀。

志立三十五年薨,是為懿王。子靖王建嗣,二十年薨。子頃王遺嗣,三十五年薨。子思王終古嗣。五鳳中,青州刺史奏終古使所愛奴與八子及諸御婢姦,終古或參與被席,或白晝使羸伏,犬馬交接,終古親臨觀。產子,輒曰:「亂不可知,使去其子。」事下丞相御史,奏終古位諸侯王,以令置八子,秩比六百石,所以廣嗣重祖也。而終古禽獸行,亂君臣夫婦之別,悖逆人倫,請逮捕。有詔削四縣。二十八年薨。子考王尚嗣,五年薨。子孝王橫嗣,三十一年薨。子懷王交嗣,六年薨。子永嗣,王莽時絕。

贊曰:悼惠之王齊,最為大國。以海內初定,子弟少,激秦孤立亡藩輔,故大封同姓,以填天下。時諸侯得自除御史大夫群卿以下眾官,如漢朝,漢獨為置丞相。自吳楚誅後,稍奪諸侯權,左官附益阿黨之法設。其後諸侯唯得衣食租稅,貧者或乘牛車。


\end{pinyinscope}