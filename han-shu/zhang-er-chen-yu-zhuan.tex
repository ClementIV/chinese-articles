\article{張耳陳餘傳}

\begin{pinyinscope}
張耳,大梁人也,少時及魏公子毋忌為客。嘗亡命遊外黃,外黃富人女甚美,庸奴其夫,亡邸父客。父客謂曰:「必欲求賢夫,從張耳。」女聽,為請決,嫁之。女家厚奉給耳,耳以故致千里客,宦為外黃令。

陳餘,亦大梁人,好儒術。遊趙苦陘,富人公乘氏以其女妻之。餘年少,父事耳,相與為刎頸交。

高祖為布衣時,嘗從耳遊。秦滅魏,購求耳千金,餘五百金。兩人變名姓,俱之陳,為里監門。吏嘗以過笞餘,餘欲起,耳攝使受笞。吏去,耳數之曰:「始吾與公言何如?今見小辱而欲死一吏乎?」餘謝罪。

陳涉起蘄至陳,耳、餘上謁涉。涉及左右生平數聞耳、餘賢,見,大喜。

陳豪桀說涉曰:「將軍被堅執銳,帥士卒以誅暴秦,復立楚社稷,功德宜為王。」陳涉問兩人,兩人對曰:「將軍瞋目張膽,出萬死不顧之計,為天下除殘。今始至陳而王之,視天下私。願將軍毋王,急引兵而西,遣人立六國後,自為樹黨。如此,野無交兵,誅暴秦,據咸陽以令諸侯,則帝業成矣。今獨王陳,恐天下解矣。」涉不聽,遂立為王。

耳、餘復說陳王曰:「大王興梁、楚,務在入關,未及收河北也。臣嘗遊趙,知其豪桀,願請奇兵略趙地。」於是陳王許之,以所善陳人武臣為將軍,耳、餘為左右校尉,與卒三千人,從白馬渡河。至諸縣,說其豪桀曰:「秦為亂政虐刑,殘滅天下,北為長城之役,南有五領之戍,外內騷動,百姓罷敝,頭會箕斂以供軍費,財匱力盡,重以苛法,使天下父子不相聊。今陳王奮臂為天下倡始,莫不嚮應,家自為怒,各報其怨,縣殺其令丞,郡殺其守尉。今以張大楚,王陳,使吳廣、周文將卒百萬西擊秦。於此時而不成封侯之業者,非人豪也。夫因天下之力而攻無道之君,報父兄之怨而成割地之業,此一時也。」豪桀皆然其言。乃行收兵,得數萬人,號武信君。下趙十餘城,餘皆城守莫肯下。乃引兵東北擊范陽。范陽人蒯通說其令徐公降武信君,又說武信君以侯印封范陽令。語在通傳。趙地聞之,不戰下者三十餘城。

至邯鄲,耳、餘聞周章軍入關,至戲卻;又聞諸將為陳王徇地,多以讒毀得罪誅。怨陳王不以為將軍而以為校尉,乃說武臣曰:「陳王非必立六國後。今將軍下趙數十城,獨介居河北,不王無以填之。且陳王聽讒,還報,恐不得脫於禍。願將軍毋失時。」武臣乃聽,遂立為趙王。以餘為大將軍,耳為丞相。

使人報陳王,陳王大怒,欲盡族武臣等家,而發兵擊趙。相國房君諫曰:「秦未亡,今又誅武臣等家,此生一秦也。不如因而賀之,使急引兵西擊秦。」陳王從其計,徙繫武臣等家宮中,封耳子敖為成都君。使使者賀趙,趣兵西入關。耳、餘說武臣曰:「王王趙,非楚意,特以計賀王。楚已滅秦,必加兵於趙。願王毋西兵,北徇燕、代,南收河內,以自廣。趙南據大河,北有燕、代,楚雖勝秦,必不敢制趙。」趙王以為然,因不西兵,而使韓廣略燕,李良略常山,張黶略上黨。

韓廣至燕,燕人因立廣為燕王。趙王乃與陳、餘北略地燕界。趙王間出,為燕軍所得。燕囚之,欲與分地。使者往,燕輒殺之,以固求地。耳、餘患之。有廝養卒謝其舍曰:「吾為二公說燕,與趙王載歸。」舍中人皆笑曰:「使者往十輩皆死,若何以能得王?」乃走燕壁。燕將見之,問曰:「知臣何欲?」燕將曰:「若欲得王耳。」曰:「君知張耳、陳餘何如人也?」燕將曰:「賢人也。」曰:「其志何欲?」燕將曰:「欲得其王耳。」趙卒笑曰;「君未知兩人所欲也。夫武臣、張耳、陳餘,杖馬箠下趙數十城,亦各欲南面而王。夫臣之與主,豈可同日道哉!顧其勢初定,且以長少先立武臣,以持趙心。今趙地已服,兩人亦欲分趙而王,時未可耳。今君囚趙王,念此兩人名為求王,實欲燕殺之,此兩人分趙而王。夫以一趙尚易燕,況以兩賢王左提右挈,而責殺王,滅燕易矣。」燕以為然,乃歸趙王。養卒為御而歸。

李良已定常山,還報趙王,趙王復使良略太原。至石邑,秦兵塞井陘,未能前。秦將詐稱二世使使遺良書,不封,曰:「良嘗事我,得顯幸,誠能反趙為秦,赦良罪,貴良。」良得書,疑不信,之邯鄲益請兵。未至,道逢趙王姊,從百餘騎。良望見,以為王,伏謁道旁。王姊醉,不知其將,使騎謝良。良素貴,起,慚其從官。從官有一人曰:「天下叛秦,能者先立。且趙王素出將軍下,今女兒乃不為將軍下車,請追殺之。」良以得秦書,欲反趙,未決,因此怒,遣人追殺王姊,遂襲邯鄲。邯鄲不知,竟殺武臣。趙人多為耳、餘耳目者,故得脫出。收兵得數萬人。客有說耳、餘曰:「兩君羈旅,而欲附趙,難可獨立;趙後,輔以誼,可就功。」及求得趙歇,立為趙王,居信都。

李良進兵擊餘,餘敗良。良走歸章邯。章邯引兵至邯鄲,皆徙其民河內,夷其城郭。耳與趙王歇走入鉅鹿城,王離圍之。餘北收常山兵,得數萬人,軍鉅鹿北。章邯軍鉅鹿南棘原,築甬道屬河,饟王離。王離兵食多,急攻鉅鹿。鉅鹿城中食盡,耳數使人召餘,餘自度兵少,不能敵秦,不敢前。數月,耳大怒,怨餘,使張黶、陳釋往讓餘曰:「始吾與公為刎頸交,今王與耳旦暮死,而公擁兵數萬,不肯相救,相不赴奏俱死?且什一二相全。」餘曰:「所以不俱死,欲為趙王、張君報秦。今俱死,如以肉餧虎,何益?」張黶、陳釋曰:「事已急,要以俱死立信,安知後慮!」餘曰:「吾顧以無益。」乃使五千人令張黶、陳釋先嘗秦軍,至皆沒。

當是時,燕、齊、楚聞趙急,皆來救。張敖亦北收代,得萬餘人來,皆壁餘旁。項羽兵數絕章邯甬道,王離軍乏食。項羽悉引兵渡河,破章邯軍。諸侯軍及敢擊秦軍,遂虜王離。於是趙王歇、張耳得出鉅鹿。與餘相見,責讓餘,問張黶、陳釋所在。餘曰:「黶、釋以必死責臣,臣使將五千人先嘗秦軍,皆沒。」耳不信,以為殺之,數問餘。餘怒曰:「不意君之望臣深也!豈以臣重去將哉?」乃脫解印綬與耳,耳不敢受。餘起如廁,客有說耳曰:「天予不取,反受其咎。今陳將軍與君印綬,不受,反天不祥。急取之。」耳乃佩其印,收其麾下。餘還,亦望耳不讓,趨出。耳遂收其兵。餘獨與麾下數百人之河上澤中漁獵。由此有隙。

趙王歇復居信都。耳從項羽入關。項羽立諸侯,耳雅遊,多為人所稱。項羽素亦聞耳賢,乃分趙立耳為常山王,治信都。信都更名襄國。

餘客多說項羽:「陳餘、張耳一體有功於趙。」羽以餘不從入關,聞其在南皮,即以南皮旁三縣封之。而徙趙王歇王代。

耳之國,餘愈怒曰:「耳與餘功等也,今耳王,餘獨侯。」及齊王田榮叛楚,餘乃使夏說說田榮曰:「項羽為天下宰不平,盡王諸將善地,徙故王王惡地,今趙王乃居代!願王假臣兵,請以南皮為扞蔽。」田榮欲樹黨,乃遣兵從餘。餘悉三縣兵,襲常山王耳。耳敗走,曰:「漢王與我有故,而項王彊,立我,我欲之楚。」甘公曰:「漢王之入關,五星聚東井。東井者,秦分也。先至必王。楚雖彊,後必屬漢。」耳走漢。漢亦還定三秦,方圍章邯廢丘。耳謁漢王,漢王厚遇之。

餘已敗耳,皆收趙地,迎趙王於代,復為趙王。趙王德餘,立以為代王。餘為趙王弱,國初定,留傅趙王,而使夏說以相國守代。

漢二年,東擊楚,使告趙,欲與俱。餘曰:「漢殺張耳乃從。」於是漢求人類耳者,斬其頭遺餘,餘乃遣兵助漢。漢敗於彭城西,餘亦聞耳詐死,即背漢。漢遣耳與韓信擊破趙井陘,斬餘泜水上,追殺趙王歇襄國。

四年夏,立耳為趙王。五年秋,耳薨,諡曰景王。子敖嗣立為王,尚高祖長女魯元公主,為王后。

七年,高祖從平城過趙,趙王旦暮自上食,體甚卑,有子婿禮。高祖箕踞罵詈,甚慢之。趙相貫高、趙午年六十餘,故耳客也,怒曰:「吾王孱王也!」說敖曰:「天下豪桀並起,能者先立,今王事皇帝甚恭,皇帝遇王無禮,請為王殺之。」敖齧其指出血,曰:「君何言之誤!且先王亡國,賴皇帝得復國,德流子孫,秋豪皆帝力也。願君無復出口。」貫高等十餘人相謂曰:「

吾等非也。吾王長者,不背德。且吾等義不辱,今帝辱我王,故欲殺之,何乃汙王為?事成歸王,事敗獨身坐耳。」

八年,上從東垣過。貫高等乃壁人柏人,要之置廁。上過欲宿,心動,問曰:「縣名為何?」曰:「柏人。」「柏人者,迫於人!」不宿去。

九年,貫高怨家知其謀,告之。於是上逮捕趙王諸反者。趙午等十餘人皆爭自剄,貫高獨怒罵曰:「誰令公等為之?今王實無謀,而并捕王;公等死,誰當白王不反者?」乃檻車與王詣長安。高對獄曰:「獨吾屬為之,王不知也。」吏榜笞數千,刺爇,身無完者,終不復言。呂后數言張王以魯元故,不宜有此。上怒曰:「使長敖據天下,豈少乃女虖!」廷尉以貫高辭聞,上曰:「壯士!誰知者,以私問之。」中大夫泄公曰:「臣素知之,此固趙國立名義不侵為然諾者也。」上使泄公持節問之箯輿前。卬視泄公,勞若如平生歡。與語,問張王果有謀不。高曰:「人情豈不各愛其父母妻子哉?今吾三族皆以論死,豈以王易吾親哉!顧為王實不反,獨吾等為之。」具道本根所以,王不知狀。於是泄公具以報上,上乃赦趙王。

上賢高能自立然諾,使泄公赦之,告曰:「張王已出,上多足下,故赦足下。」高曰:「所以不死,白張王不反耳。今王已出,吾責塞矣。且人臣有篡弒之名,豈有面目復事上哉!」乃仰絕亢而死。

敖已出,尚魯元公主如故,封為宣平侯。於是上賢張王諸客,皆以為諸侯相、郡守。語在田叔傳。及孝惠、高后、文、景時,張王客子孫皆為二千石。

初,孝惠時,齊悼惠王獻城陽郡,尊魯元公主為太后。高后元年,魯元太后薨。後六年,宣平侯敖復薨。呂太后立敖子偃為魯王,以母為太后故也。又憐其年少孤弱,乃封敖前婦子二人:壽為樂昌侯,侈為信都侯。高后崩,大臣誅諸呂,廢魯王及二侯。孝文即位,復封故魯王偃為南宮侯。薨,子生嗣。武帝時,生有罪免,國除。元光中,復封偃孫廣國為睢陵侯。薨,子昌嗣。太初中,昌坐不敬免,國除。孝平元始二年,繼絕世,封敖玄孫慶忌為宣平侯,食千戶。

贊曰:張耳、陳餘,世所稱賢,其賓客廝役皆天下俊桀,所居國無不取卿相者。然耳、餘始居約時,相然信死,豈顧問哉!及據國爭權,卒相滅亡,何鄉者慕用之誠,後相背之盭也!勢利之交,古人羞之,蓋謂是矣。


\end{pinyinscope}