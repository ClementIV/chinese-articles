\article{武五子傳}

\begin{pinyinscope}
孝武皇帝六男。衛皇后生戾太子,趙婕妤生孝昭帝,王夫人生齊懷王閎,李姬生燕剌王旦、廣陵厲王胥,李夫人生昌邑哀王髆。

戾太子據,元狩元年立為皇太子,年七歲矣。初,上年二十九乃得太子,甚喜,為立禖,使東方朔、枚皋作禖祝。少壯,詔受公羊春秋,又從瑕丘江公受穀梁。及冠就宮,上為立博望苑,使通賓客,從其所好,故多以異端進者。元鼎四年,納史良娣,產子男進,號曰史皇孫。

武帝末,衛后寵衰,江充用事。充與太子及衛氏有隙,恐上晏駕後為太子所誅,會巫蠱事起,充因此為姦。是時,上春秋高,意多所惡,以為左右皆為蠱道祝詛,窮治其事。丞相公孫賀父子,陽石、諸邑公主,及皇后弟子長平侯衛伉皆坐誅。語在公孫賀、江充傳。

充典治巫蠱,既知上意,白言宮中有蠱氣,入宮至省中,壞御座掘地。上使按道侯韓說、御史章贛、黃門蘇文等助充。充遂至太子宮掘蠱,得桐木人。時上疾,辟暑甘泉宮,獨皇后、太子在。太子召問少傅石德,德懼為師傅并誅,因謂太子曰:「前丞相父子、兩公主及衛氏皆坐此,今巫與使者掘地得徵驗,不知巫置之邪,將實有也,無以自明,可矯以節收捕充等繫獄,窮治其姦詐。且上疾在甘泉,皇后及家吏請問皆不報,上存亡未可知,而姦臣如此,太子將不念秦扶蘇事耶?」太子急,然德言。

征和二年七月壬午,乃使客為使者收捕充等。按道侯說疑使者有詔,不肯受詔,客格殺說。御史章贛被創突亡,自歸甘泉。太子使舍人無且持節夜入未央宮殿長秋門,因長御倚華具白皇后,發中廄車載射士,出武庫兵,發長樂宮衛,告令百官曰江充反。乃斬充以徇,炙胡巫上林中。遂部賓客為將率,與丞相劉屈氂等戰。長安中擾亂,言太子反,以故眾不肯附。太子兵敗,亡,不得。

上怒甚,群下憂懼,不知所出。壺關三老茂上書曰:「臣聞父者猶天,母者猶地,子猶萬物也。故天平地安,陰陽和調,物乃茂成;父慈母愛室家之中,子乃孝順。陰陽不和則萬物夭傷,父子不和則室家散亡。故父不父則子不子,君不君則臣不臣,雖有粟,吾豈得而食諸!昔者虞舜,孝之至也,而不中於瞽叟;孝己被謗,伯奇放流,骨肉至親,父子相疑。何者?積毀之所生也。由是觀之,子無不孝,而父有不察。令皇太子為漢適嗣,承萬世之業,體祖宗之重,親則皇帝之宗子也。江充,布衣之人,閭閻之隸臣耳,陛下顯而用之,銜至尊之命以迫蹴皇太子,造飾姦詐,群邪錯謬,是以親戚之路鬲塞而不通。太子進則不得上見,退則困於亂臣,獨冤結而亡告,不忍忿忿之心,起而殺充,恐懼逋逃,子盜父兵以救難自免耳,臣竊以為無邪心。《詩》云:『營營青蠅,止于藩;愷悌君子,無信讒言;讒言罔極,交亂四國。』往者江充讒殺趙太子,天下莫不聞,其罪固宜。陛下不省察,深過太子,發盛怒,舉大兵而求之,三公自將,智者不敢言,辯士不敢說,臣竊痛之。臣聞子胥盡忠而忘其號,比干盡仁而遺其身,忠臣竭誠不顧鈇鉞之誅以陳其愚,志在匡君安社稷也。《詩》云:『取彼譖人,投畀豺虎。』唯陛下寬心慰意,少察所親,毋患太子之非,亟罷甲兵,無令太子久亡。臣不勝惓惓,出一旦之命,待罪建章闕下。」書奏,天子感寤。

太子之亡也,東至湖,臧匿泉鳩里。主人家貧,常賣屨以給太子。太子有故人在湖,聞其富贍,使人呼之而發覺。吏圍捕太子,太子自度不得脫,即入室距戶自經。山陽男子張富昌為卒,足蹋開戶,新安令史李壽趨抱解太子,主人公遂格鬥死,皇孫二人皆并遇害。上既傷太子,乃下詔曰:「蓋行疑賞,所以申信也。其封李壽為邘侯,張富昌為題侯。」

久之,巫蠱事多不信。上知太子惶恐無他意,而車千秋復訟太子冤,上遂擢千秋為丞相,而族滅江充家,焚蘇文於橫橋上,及泉鳩里加兵刃於太子者,初為北地太守,後族。上憐太子無辜,乃作思子宮,為歸來望思之臺於湖。天下聞而悲之。

初,太子有三男一女,女者平輿侯嗣子尚焉。及太子敗,皆同時遇害。衛侯、史良娣葬長安城南。史皇孫、皇孫妃王夫人及皇女孫葬廣明。皇孫二人隨太子者,與太子并葬湖。

太子有遺孫一人,史皇孫子,王夫人男,年十八即尊位,是為孝宣帝。帝初即位,帝詔曰:「故皇太子在湖,未有號諡,歲時祠,其議諡,置園邑。」有司奏請:「禮『為人後者,為之子也』,故降其父母不得祭,尊祖之義也。陛下為孝昭帝後,承祖宗之祀,制禮不踰閑。謹行視孝昭帝所為故皇太子起位在湖,史良娣冢在博望苑北,親史皇孫位在廣明郭北。諡法曰『諡者,行之跡也』,愚以為親諡宜曰悼皇,母曰悼后,比諸侯王園,置奉邑三百家。故皇太子諡曰戾,置奉邑二百家。史良娣曰戾夫人,置守冢三十家。園置長丞,周衛奉守如法。」以湖閿鄉邪里聚為戾園,長安白亭東為戾后園,廣明成鄉為悼園。皆改葬焉。

後八歲,有司復言:「禮『父為士,子為天子,祭以天子』。悼園宜稱尊號曰皇考,立廟:因園為寢,以時薦享焉。益奉園民滿千六百家,以為奉明縣。尊戾夫人曰戾后,置園奉邑,及益戾園各滿三百家。」

齊懷王閎與燕王旦、廣陵王胥同日立,皆賜策,各以國土風俗申戒焉,曰:「惟元狩六年四月乙巳,皇帝使御史大夫湯廟立子閎為齊王,曰:烏呼!小子閎,受茲青社。朕承天序,惟稽古,建爾國家,封于東土,世為漢藩輔。烏呼!念哉,共朕之詔。惟命不于常,人之好德,克明顯光;義之不圖,俾君子怠。悉爾心,允執其中,天祿永終;厥有愆不臧,乃凶于乃國,而害于爾躬。嗚呼!保國乂民,可不敬與!王其戒之!」閎母王夫人有寵,閎尤愛幸,立八年,薨,無子,國除。

燕剌王旦賜策曰:「嗚呼!小子旦,受茲玄社,建爾國家,封于北土,世為漢藩輔。嗚呼!薰鬻氏虐老獸心,以姦巧邊甿。朕命將率,徂征厥罪。萬夫長,千夫長,三十有二帥,降旗奔師。薰鬻徙域,北州以妥。悉爾心,毋作怨,毋作棐德,毋乃廢備。非教士不得從徵。王其戒之!」

旦壯大就國,為人辯略,博學經書雜說,好星曆數術倡優射獵之事,招致游士。及衛太子敗,齊懷王又薨,旦自以次第當立,上書求入宿衛。上怒,下其使獄。後坐臧匿亡命,削良鄉、安次、文安三縣。武帝由是惡旦,後遂立少子為太子。

帝崩,太子立,是為孝昭帝,賜諸侯王璽書。旦得書,不肯哭,曰:「璽書封小。京師疑有變。」遣幸臣壽西長、孫縱之、王孺等之長安,以問禮儀為名。王孺見執金吾廣義,問帝崩所病,立者誰子,年幾歲。廣意言待詔五莋宮,宮中讙言帝崩,諸將軍共立太子為帝,年八九歲,葬時不出臨。歸以報王。王曰:「上棄群臣,無語言,蓋主又不得見,甚可怪也。」復遣中大夫至京師上書言:「竊見孝武皇帝躬聖道,孝宗廟,慈愛骨肉,和集兆民,德配天地,明並日月,威武洋溢,遠方執寶而朝,增郡數十,斥地且倍,封泰山,禪梁父,巡狩天下,遠方珍物陳于太廟,德甚休盛,請立廟郡國。」奏報聞。時大將霍光秉政,褒賜燕王錢三千萬,益封萬三千戶。旦怒曰:「

我當為帝,何賜也!」遂與宗室中山哀王子劉長、齊孝王孫劉澤等結謀,詐言以武帝時受詔,得職吏事,修武備,備非常。

長於是為旦命令群臣曰:「寡人賴先帝休德,獲奉北藩,親受明詔,職吏事,領庫兵,飭武備,任重職大,夙夜兢兢,子大夫將何以規佐寡人?且燕國雖小,成周之建國也,上自召公,下及昭、襄,于今千載,豈可謂無賢哉?寡人束帶聽朝三十餘年,曾無聞焉。其者寡人之不及與?意亦子大夫之思有所不至乎?其咎安在?方今寡人欲撟邪防非,章聞揚和,撫慰百姓,移風易俗,厥路何由?子大夫其各悉心以對,寡人將察焉。」

群臣皆免冠謝。郎中成軫謂旦曰:「大王失職,獨可起而索,不可坐而得也。大王壹起,國中雖女子皆奮臂隨大王。」旦曰:「前高后時,偽立子弘為皇帝,諸侯交手事之八年。呂太后崩,大臣誅諸呂,迎立文帝,天下乃知非孝惠子也。我親武帝長子,反不得立,上書請立廟,又不聽。立者疑非劉氏。」

即與劉澤謀為姦書,言少帝非武帝子,大臣所共立,天下宜共伐之。使人傳行郡國,以搖動百姓。澤謀歸發兵臨淄,與燕王俱起。旦遂招來郡國姦人,賦斂銅鐵作甲兵,數閱其車騎材官卒,建旌旗鼓車,旄頭先敺,郎中侍從者著貂羽,黃金附蟬,皆號侍中。旦從相、中尉以下,勒車騎,發民會圍,大獵文安縣,以講士馬,須期日。郎中韓義等數諫旦,旦殺義等凡十五人。會缾侯劉成知澤等謀,告之青州刺史雋不疑,不疑收捕澤以聞。天子遣大鴻臚丞治,連引燕王。有詔弗治,而劉澤等皆伏誅。益封缾侯。

久之,旦姊鄂邑蓋長公主、左將軍上官桀父子與霍光爭權有隙,皆知旦怨光,即私與燕交通。旦遣孫縱之等前後十餘輩,多齎金寶走馬,賂遺蓋主。上官桀及御史大夫桑弘羊等皆與交通,數記疏光過失與旦,令上書告之。桀欲從中下其章。旦聞之,喜,上疏曰:「昔秦據南面之位,制一世之命,威服四夷,輕弱骨肉,顯重異族,廢道任刑,無恩宗室。其後尉佗入南夷,陳涉呼楚澤,近狎作亂,內外俱發,趙氏無炊火焉。高皇帝覽蹤跡,觀得失,見秦建本非是,故改其路,規土連城,布王子孫,是以支葉扶疏,異姓不得間也。今陛下承明繼成,委任公卿,群臣連與成朋,非毀宗室,膚受之愬,日騁於廷,惡吏廢法立威,主恩不及下究。臣聞武帝使中郎將蘇武使匈奴,見留二十年不降,還亶為典屬國。今大將軍長史敞無勞,為搜粟都尉。又將軍都郎羽林,道上移蹕,太官先置。臣旦願歸符璽,入宿衛,察姦臣之變。」

是時昭帝年十四,覺其有詐,遂親信霍光,而疏上官桀等。桀等因謀共殺光,廢帝,迎立燕王為天子。旦置驛書,往來相報,許立桀為王,外連郡國豪桀以千數。旦以語相平,平曰:「大王前與劉澤結謀,事未成而發覺者,以劉澤素夸,好侵陵也。平聞左將軍素輕易,車騎將軍少而驕,臣恐其如劉澤時不能成,又恐既成,反大王也。」旦曰:「前日一男子詣闕,自謂故太子,長安中民趣鄉之,正讙不可止,大將軍恐,出兵陳之,以自備耳。我帝長子,天下所信,何憂見反?」後謂群臣:「蓋主報言,獨患大將軍與右將軍王莽。今右將軍物故,丞相病,幸事必成,徵不久。」令群臣皆裝。

是時天雨,虹下屬宮中飲井水,水泉竭。廁中豕群出,壞大官灶。烏鵲鬥死。鼠舞殿端門中。殿上戶自閉,不可開。天火燒城門。大風壞宮城樓,折拔樹木。流星下墯。后姬以下皆恐。王驚病,使人祠葭水、台水。王客呂廣等知星,為王言「當有兵圍城,期在九月十月,漢當有大臣戮死者。」語具在五行志。

王愈憂恐,謂廣等曰:「謀事不成,妖祥數見,兵氣且至,柰何?」會蓋主舍人父燕倉知其謀,告之,由是發覺。丞相賜璽書,部中二千石逐捕孫縱之及左將軍桀等,皆伏誅。旦聞之,召相平曰:「事敗,遂發兵乎?」平曰:「左將軍已死,百姓皆知之,不可發也。」王憂懣,置酒萬載宮,會賓客群臣妃妾坐飲。王自歌曰:「歸空城兮,狗不吠,雞不鳴,橫術何廣廣兮,固知國中之無人!」華容夫人起舞曰:「髮紛紛兮寘渠,骨籍籍兮亡居。母求死子兮,妻求死夫。裴回兩渠間兮,君子獨安居!」坐者皆泣。

有赦令到,王讀之,曰:「嗟乎!獨赦吏民,不赦我。」因迎后姬諸夫人之明光殿,王曰:「老虜曹為事當族!」欲自殺。左右曰:「黨得削國,幸不死。」后妃夫人共啼泣止王。會天子使使者賜燕王璽書曰:「昔高皇帝王天下,建立子弟以藩屏社稷。先日諸呂陰謀大逆,劉氏不絕若髮,賴絳侯等誅討賊亂,尊立孝文,以安宗廟,非以中外有人,表裏相應故邪?樊、酈、曹、灌,攜劍推鋒,從高帝墾菑除害,耘鉏海內,勤苦至矣,然其賞不過諸侯。今宗室子孫曾無暴衣露冠之勞,裂地而王之,分財而賜之,父死子繼,兄終弟及。今王骨肉至親,敵吾一體,乃與他姓異族謀害社稷,親其所疏,疏其所親,有逆悖之心,無忠愛之義。如使古人有知,當何面目復舉齊酎見高祖之廟乎!」

旦得書,以符璽屬醫工長,謝相二千石:「奉事不謹,死矣。」即以綬自絞。后夫人隨旦自殺者二十餘人。天子加恩,赦王太子建為庶人,賜旦諡曰剌王。旦立三十八年而誅,國除。

後六年,宣帝即位,封旦兩子,慶為新昌侯,賢為定安侯,又立故太子建,是為廣陽頃王,二十九年薨。子穆王舜嗣,二十一年薨。子思王璜嗣,二十年薨。子嘉嗣。王莽時,皆廢漢藩王為家人,嘉獨以獻符命封扶美侯,賜姓王氏。

廣陵厲王胥賜策曰:「嗚呼!小子胥,受茲赤社,建爾國家,封于南土,世世為漢藩輔。古人有言曰:『大江之南,五湖之間,其人輕心。揚州保彊,三代要服,不及以正。』嗚呼!悉爾心,祗祗兢兢,乃惠乃順,毋桐好逸,毋邇宵人,惟法惟則!《書》云『臣不作福,不作威』,靡有後羞。王其戒之!」

胥壯大,好倡樂逸游,力扛鼎,空手搏熊彘猛獸。動作無法度,故終不得為漢嗣。

昭帝初立,益封胥萬三千戶,元鳳中入朝,復益萬戶,賜錢二千萬,黃金二千斤,安車駟馬寶劍。及宣帝即位,封胥四子聖、曾、寶、昌皆為列侯,又立胥小子弘為高密王。所以褒賞甚厚。

始,昭帝時,胥見上年少無子,有覬欲心。而楚地巫鬼,胥迎女巫李女須,使下神祝詛。女須泣曰:「孝武帝下我。」左右皆服。言「吾必令胥為天子。」胥多賜女須錢,使禱巫山。會昭帝崩,胥曰:「女須良巫也!」殺牛塞禱。及昌邑王徵,復使巫祝詛之。後王廢,胥寖信女須等,數賜予錢物。宣帝即位,胥曰:「太子孫何以反得立?」復令女須祝詛如前。又胥女為楚王延壽后弟婦,數相餽遺,通私書。後延壽坐謀反誅,辭連及胥。有詔勿治,賜胥黃金前後五千斤,它器物甚眾。胥又聞漢立太子,謂姬南等曰:「我終不得立矣。」乃止不詛。後胥子南利侯寶坐殺人奪爵,還歸廣陵,與胥姬左修姦。事發覺,繫獄,棄巿。相勝之奏奪王射陂草田以賦貧民,奏可。胥復使巫祝詛如前。

胥宮園中棗樹生十餘莖,莖正赤,葉白如素。池水變赤,魚死。有鼠晝立舞王后廷中。胥謂姬南等曰:「棗水魚鼠之怪甚可惡也。」居數月,祝詛事發覺,有司按驗,胥惶恐,藥殺巫及宮人二十餘人以絕口。公卿請誅胥,天子遣廷尉、大鴻臚即訊。胥謝曰:「罪死有餘,誠皆有之。事久遠,請歸思念具對。」胥既見使者還,置酒顯陽殿,召太子霸及子女董訾、胡生等夜飲,使所幸八子郭昭君、家人子趙左君等鼓瑟歌舞。王自歌曰:「欲久生兮無終,長不樂兮安窮!奉天期兮不得須臾,千里馬兮駐待路。黃泉下兮幽深,人生要死,何為苦心!何用為樂心所喜,出入無悰為樂亟。蒿里召兮郭門閱,死不得取代庸,身自逝。」左右悉更涕泣奏酒,至雞鳴時罷。胥謂太子霸曰:「上遇我厚,今負之甚。我死,骸骨當暴。幸而得葬,薄之,無厚也。」即以綬自絞死。及八子郭昭君等二人皆自殺。天子加恩,赦王諸子皆為庶人,賜諡曰厲王。立六十四年而誅,國除。

後七年,元帝復立胥太子霸,是為孝王,十三年薨。子共王意嗣,三年薨。子哀王護嗣,十六年薨,無子,絕。後六年,成帝復立孝王子守,是為靖王,立二十年薨。子宏嗣,王莽時絕。

初,高密哀王弘本始元年以廣陵王胥少子立,九年薨。子頃王章嗣,三十三年薨。子懷王寬嗣,十一年薨。子慎嗣,王莽時絕。

昌邑哀王髆天漢四年立,十一年薨,子賀嗣。立十三年,昭帝崩,無嗣,大將軍霍光徵王賀典喪。璽書曰:「制詔昌邑王:使行大鴻臚事少府樂成、宗正德、光祿大夫吉、中郎將利漢徵王,乘七乘傳詣長安邸。」夜漏未盡一刻,以火發書。其日中,賀發,晡時至定陶,行百三十五里,侍從者馬死相望於道。郎中令龔遂諫王,令還郎謁者五十餘人。賀到濟陽,求長鳴雞,道買積竹杖。過弘農,使大奴善以衣車載女子。至湖,使者以讓相安樂。安樂告遂,遂入問賀,賀曰:「無有。」遂曰:「即無有,何愛一善以毀行義!請收屬吏,以湔洒大王。」即捽善,屬衛士長行法。

賀到霸上,大鴻臚郊迎,騶奉乘輿車。王使僕壽成御,郎中令遂參乘。旦至廣明東都門,遂曰:「禮,奔喪望見國都哭。此長安東郭門也。」賀曰:「我嗌痛,不能哭。」至城門,遂復言,賀曰:「城門與郭門等耳。」且至未央宮東闕,遂曰:「昌邑帳在是闕外馳道北,未至帳所,有南北行道,馬足未至數步,大王宜下車,鄉闕西面伏,哭盡哀止。」王曰:「諾。」到,哭如儀。

王受皇帝璽綬,襲尊號。即位二十七日,行淫亂。大將軍光與群臣議,白孝昭皇后,廢賀歸故國,賜湯沐邑二千戶,故王家財物皆與賀。及哀王女四人各賜湯沐邑千戶。語在霍光傳。國除,為山陽郡。

初賀在國時,數有怪。嘗見白犬,高三尺,無頭,其頸以下似人,而冠方山冠。後見熊,左右皆莫見。又大鳥飛集宮中。王知,惡之,輒以問郎中令遂。遂為言其故,語在五行志。王卬天歎曰:「不祥何為數來!」遂叩頭曰:「臣不敢隱忠,數言危亡之戒,大王不說。夫國之存亡,豈在臣言哉?願王內自揆度。大王誦詩三百五篇,人事浹,王道備,王之所行中詩一篇何等也?大王位為諸侯王,行汙於庶人,以存難,以亡易,宜深察之。」後又血汙王坐席,王問遂,遂叫然號曰:「宮空不久,祅祥數至。血者,陰憂象也。宜畏慎自省。」賀終不改節。居無何,徵。既即位,後王夢青蠅之矢積西階東,可五六石,以屋版瓦覆,發視之,青蠅矢也。以問遂,遂曰:「陛下之詩不云乎?『營營青蠅,至于藩;愷悌君子,毋信讒言。』陛下左側讒人眾多,如是青蠅惡矣。宜進先帝大臣子孫親近以為左右。如不忍昌邑故人,信用讒諛,必有凶咎。願詭禍為福,皆放逐之。臣當先逐矣。」賀不用其言,卒至於廢。

大將軍光更尊立武帝曾孫,是為孝宣帝。即位,心內忌賀,元康二年遣使者賜山陽太守張敞璽書曰:「制詔山陽太守:其謹備盜賊,察往來過客。毋下所賜書!」敞於是條奏賀居處,著其廢亡之效,曰:「臣敞地節三年五月視事,故昌邑王居故宮,奴婢在中者百八十三人,閉大門,開小門,廉吏一人為領錢物市買,朝內食物,它不得出入。督盜一人別主徼循,察往來者,以王家錢取卒,迾宮清中備盜賊。臣敞數遣丞吏行察。四年九月中,臣敞入視居處狀,故王年二十六七,為人青黑色,小目,鼻末銳卑,少須眉,身體長大,疾痿,行步不便。衣短衣大恊,冠惠文冠,佩玉環,簪筆持牘趨謁。臣敞與坐語中庭,閱妻子奴婢。臣敞欲動觀其意,即以惡鳥感之,曰:『昌邑多梟。』故王應曰:『然前賀西至長安,殊無梟。復來,東至濟陽,乃復聞梟聲。』臣敞閱至子女持轡,故王跪曰:『持轡母,嚴長孫女也。』臣敞故知執金吾嚴延年字長孫,女羅紨,前為故王妻。察故王衣服言語跪起,清狂不惠。妻十六人,子二十二人,其十一人男,十一人女。昧死奏名籍及奴婢財物簿。臣敞前書言:『昌邑哀王歌舞者張修等十人,無子,又非姬,但良人,無官名,王薨當罷歸。太傅豹等擅留,以為哀王園中人,所不當得為,請罷歸。』故王聞之曰:『中人守園,疾者當勿治,相殺傷者當勿法,欲令亟死,太守柰何而欲罷之?』其天資喜由亂亡,終不見仁義,如此。後丞相御史以臣敞書聞,奏可。皆以遣。」上由此知賀不足忌。

其明年春,乃下詔曰:「蓋聞象有罪,舜封之,骨肉之親,析而不殊。其封故昌邑王賀為海昏侯,食邑四千戶。」侍中衛尉金安上上書言:「賀天之所棄,陛下至仁,復封為列侯。賀嚚頑放廢之人,不宜得奉宗廟朝聘之禮。」奏可。賀就國豫章。

數年,揚州刺史柯奏賀與故太守卒史孫萬世交通,萬世問賀:「前見廢時,何不堅守毋出宮,斬大將軍,而聽人奪璽綬乎?」賀曰:「然。失之。」萬世又以賀且王豫章,不久為列侯。賀曰:「

且然,非所宜言。」有司案驗,請逮捕。制曰:「削戶三千。」後薨。

豫章太守廖奏言:「舜封象於有鼻,死不為置後,以為暴亂之人不宜為太祖。海昏侯賀死,上當為後者子充國;充國死,復上弟奉親;奉親復死,是天絕之也。陛下聖仁,於賀甚厚,雖舜於象無以加也。宜以禮絕賀,以奉天意。願下有司議。」議皆以為不宜為立嗣,國除。

元帝即位,復封賀子代宗為海昏侯,傳子至孫,今見為。

贊曰:巫蠱之禍,豈不哀哉!此不唯一江充之辜,亦有天時,非人力所致焉。建元六年,蚩尤之旗見,其長竟天。後遂命將出征,略取河南,建置朔方。其春,戾太子生。自是之後,師行三十年,兵所誅屠夷滅死者不可勝數。及巫蠱事起,京師流血,僵尸數萬,太子子父皆敗。故太子生長於兵,與之終始,何獨一嬖臣哉!秦始皇即位三十九年,內平六國,外攘四夷,死人如亂麻,暴骨長城之下,頭盧相屬於道,不一日而無兵。由是山東之難興,四方潰而逆秦。秦將吏外畔,賊臣內發,亂作蕭牆,禍成二世。故曰「兵猶火也,弗戢必自焚」,信矣。是以倉頡作書,「止」「戈」為「武」。聖人以武禁暴整亂,止息干戈,非以為殘而興縱之也。《易》曰:「天之所助者順也,人之所助者信也;君子履信思順,自天祐之,吉無不利也。」故車千秋指明蠱情,章太子之冤。千秋材知未必能過人也,以其銷惡運,遏亂原,因衰激極,道迎善氣,傳得天人之祐助云。


\end{pinyinscope}