\article{雋疏于薛平彭傳}

\begin{pinyinscope}
雋不疑字曼倩,勃海人也。治春秋,為郡文學,進退必以禮,名聞州郡。

武帝末,郡國盜賊群起,暴勝之為直指使者,衣繡衣,持斧,逐捕盜賊,督課郡國,東至海,以軍興誅不從命者,威振州郡。勝之素聞不疑賢,至勃海,遣吏請與相見。不疑冠進賢冠,帶櫑具劍,佩環玦,褒衣博帶,盛服至門上謁。門下欲使解劍,不疑曰:「劍者君子武備,所以衛身,不可解。請退。」吏白勝之。勝之開閤延請,望見不疑容貌尊嚴,衣冠甚偉,勝之硔履起迎。登堂坐定,不疑據地曰:「竊伏海瀕,聞暴公子威名舊矣,今乃承顏接辭。凡為吏,太剛則折,太柔則廢,威行施之以恩,然後樹功揚名,永終天祿。」勝之知不疑非庸人,敬納其戒,深接以禮意,問當世所施行。門下諸從事皆州郡選吏,側聽不疑,莫不驚駭。至昏夜,罷去。勝之遂表薦不疑,徵詣公車,拜為青州刺史。

久之,武帝崩,昭帝即位,而齊孝王孫劉澤交結郡國豪傑謀反,欲先殺青州刺史。不疑發覺,收捕,皆伏其辜。擢為京兆尹,賜錢百萬。京師吏民敬其威信。每行縣錄囚徒還,其母輒問不疑:「

有所平反,活幾何人?」即不疑多有所平反,母喜笑,為飲食語言異於他時;或亡所出,母怒,為之不食。故不疑為吏,嚴而不殘。

始元五年,有一男子乘黃犢車,建黃旐,衣黃襜褕,著黃冒,詣北闕,自謂衛太子。公車以聞,詔使公卿將軍中二千石雜識視。長安中吏民聚觀者數萬人。右將軍勒兵闕下,以備非常。丞相御史中二千石至者立莫敢發言。京兆尹不疑後到,叱從吏收縛。或曰:「是非未可知,且安之。」不疑曰:「諸君何患於衛太子!昔蒯聵違命出奔,輒距而不納,春秋是之。衛太子得罪先帝,亡不即死,今來自詣,此罪人也。」遂送詔獄。

天子與大將軍霍光聞而嘉之,曰:「公卿大臣當用經術明於大誼。」繇是名聲重於朝廷,在位者皆自以不及也。大將軍光欲以女妻之,不疑固辭,不肯當。久之,以病免,終於家。京師紀之。後趙廣漢為京兆尹,言「我禁姦止邪,行於吏民,至於朝廷事,不及不疑遠甚。」廷尉驗治何人,竟得姦詐。本夏陽人,姓成名方遂,居湖,以卜筮為事。有故太子舍人嘗從方遂卜,謂曰:「子狀貌甚似衛太子。」方遂心利其言,幾得以富貴,即詐自稱詣闕。廷尉逮召鄉里識知者張宗祿等,方遂坐誣罔不道,要斬東巿。一姓張名延年。

疏廣字仲翁,東海蘭陵人也。少好學,明春秋,家居教授,學者自遠方至。徵為博士太中大夫。地節三年,立皇太子,選丙吉為太傅,廣為少傅。數月,吉遷御史大夫,廣徙為太傅,廣兄子受字公子,亦以賢良舉為太子家令。受好禮恭謹,敏而有辭。宣帝幸太子宮,受迎謁應對,及置酒宴,奉觴上壽,辭禮閑雅,上甚讙說。頃之,拜受為少傅。

太子外祖父特進平恩侯許伯以為太子少,白使其弟中郎將舜監護太子家。上以問廣,廣對曰:「太子國儲副君,師友必於天下英俊,不宜獨親外家許氏。且太子自有太傅少傅,官屬已備,今復使舜護太子家,視陋,非所以廣太子德於天下也。」上善其言,以語丞相魏相,相免冠謝曰:「此非臣等所能及。」廣繇是見器重,數受賞賜。太子每朝,因進見,太傅在前,少傅在後。父子並為師傅,朝廷以為榮。

在位五歲,皇太子年十二,通論語、孝經。廣謂受曰:「吾聞『

知足不辱,知止不殆』,『功遂身退,天之道』也。今仕宦至二千石,宦成名立,如此不去,懼有後悔,豈如父子相隨出關,歸老故鄉,以壽命終,不亦善乎?」受叩頭曰:「從大人議。」即日父子俱移病。滿三月賜告,廣遂稱篤,上疏乞骸骨。上以其年篤老,皆許之,加賜黃金二十斤,皇太子贈以五十斤。公卿大夫故人邑子設祖道,供張東都門外,送者車數百兩,辭決而去。及道路觀者皆曰:「賢哉二大夫!」或歎息為之下泣。

廣既歸鄉里,日令家共具設酒食,請族人故舊賓客,與相娛樂。數問其家金餘尚有幾所,趣賣以共具。居歲餘,廣子孫竊謂其昆弟老人廣所愛信者曰:「子孫幾及君時頗立產業基阯,今日飲食廢且盡。宜從丈人所,勸說君買田宅。」老人即以閒暇時為廣言此計,廣曰:「吾豈老誖不念子孫哉?顧自有舊田廬,令子孫勤力其中,足以共衣食,與凡人齊。今復增益之以為贏餘,但教子孫怠墯耳。賢而多財,則損其志;愚而多財,則益其過。且夫富者,眾人之怨也;吾既亡以教化子孫,不欲益其過而生怨。又此金者,聖主所以惠養老臣也,故樂與鄉黨宗族共饗其賜,以盡吾餘日,不亦可乎!」於是族人說服。皆以壽終。

于定國字曼倩,東海郯人也。其父于公為縣獄史,郡決曹,決獄平,羅文法者于公所決皆不恨。郡中為之生立祠,號曰于公祠。

東海有孝婦,少寡,亡子,養姑甚謹,姑欲嫁之,終不肯。姑謂鄰人曰:「孝婦事我勤苦,哀其亡子守寡。我老,久絫丁壯,柰何?」其後姑自經死,姑女告吏:「婦殺我母。」吏捕孝婦,孝婦辭不殺姑。吏驗治,孝婦自誣服。具獄上府,于公以為此婦養姑十餘年,以孝聞,必不殺也。太守不聽,于公爭之,弗能得,乃抱其具獄,哭於府上,因辭疾去。太守竟論殺孝婦。郡中枯旱三年。後太守至,卜筮其故,于公曰:「孝婦不當死,前太守彊斷之,咎黨在是乎?」於是太守殺牛自祭孝婦冢,因表其墓,天立大雨,歲孰。郡中以此大敬重于公。

定國少學法于父,父死,後定國亦為獄史,郡決曹,補廷尉史,以選與御史中丞從事治反者獄,以材高舉侍御史,遷御史中丞。會昭帝崩,昌邑王徵即位,行淫亂,定國上書諫。後王廢,宣帝立,大將軍光領尚書事,條奏群臣諫昌邑王者皆超遷。定國繇是為光祿大夫,平尚書事,甚見任用。數年,遷水衡都尉,超為廷尉。

定國乃迎師學春秋,身執經,北面備弟子禮。為人謙恭,尤重經術士,雖卑賤徒步往過,定國皆與鈞禮,恩敬甚備,學士咸

聲焉。其決疑平法,務在哀鰥寡,罪疑從輕,加審慎之心。朝廷稱之曰:「張釋之為廷尉,天下無冤民;于定國為廷尉,民自以不冤。」定國食酒至數石不亂,冬月請治讞,飲酒益精明。為廷尉十八歲,遷御史大夫。

甘露中,代黃霸為丞相,封西平侯。三年,宣帝崩,元帝立,以定國任職舊臣,敬重之。時陳萬年為御史大夫,與定國並位八年,論議無所拂。後貢禹代為御史大夫,數處駮議,定國明習政事,率常丞相議可。然上始即位,關東連年被災害,民流入關,言事者歸咎於大臣。上於是數以朝日引見丞相、御史,入受詔,條責以職事,曰:「惡吏負賊,妄意良民,至亡辜死。或盜賊發,吏不亟追而反繫亡家,後不敢復告,以故寖廣。民多冤結,州郡不理,連上書者交於闕廷。二千石選舉不實,是以在位多不任職。民田有災害,吏不肯除,收趣其租,以故重困。關東流民飢寒疾疫,已詔吏轉漕,虛食廩開府臧相振救,賜寒者衣,至春猶恐不贍。今丞相、御史將欲何施以塞此咎?悉意條狀,陳朕過失。」定國上書謝罪。

永光元年,春霜夏寒,日青亡光,上復以詔條責曰:「郎有從東方來者,言民父子相棄。丞相、御史案事之吏匿不言邪?將從東方來者加增之也?何以錯繆至是?欲知其實。方今年歲未可預知也,即有水旱,其憂不細。公卿有可以防其未然,救其已然者不?各以誠對,毋有所諱。」定國惶恐,上書自劾,歸侯印,乞骸骨。上報曰:「君相朕躬,不敢怠息,萬方之事,大錄于君。能毋過者,其唯聖人。方今承周秦之敝,俗化陵夷,民寡禮誼,陰陽不調,災咎之發,不為一端而作,自聖人推類以記,不敢專也,況於非聖者乎!日夜惟思所以,未能盡明。經曰:『萬方有罪,罪在朕躬。』君雖任職,何必顓焉?其勉察郡國守相郡牧,非其人者毋令久賊民。永執綱紀,務悉聰明,強食慎疾。」定國遂稱篤,固辭。上乃賜安車駟馬、黃金六十斤,罷就第。數歲,七十餘薨,諡曰安侯。

子永嗣。少時,耆酒多過失,年且三十,乃折節修行,以父任為侍中中郎將、長水校尉。定國死,居喪如禮,孝行聞。由是以列侯為散騎光祿勳,至御史大夫。尚館陶公主施。施者,宣帝長女,成帝姑也,賢有行,永以選尚焉。上方欲相之,會永薨。子恬嗣。恬不肖,薄於行。

始定國父于公,其閭門壞,父老方共治之。于公謂曰:「

少高大閭門,令容駟馬高蓋車。我治獄多陰德,未嘗有所冤,子孫必有興者。」至定國為丞相,永為御史大夫,封侯傳世云。

薛廣德字長卿,沛郡相人也。以魯詩教授楚國,龔勝、舍師事焉。蕭望之為御史大夫,除廣德為屬,數與論議,器之,薦廣德經行宜充本朝。為博士,論石渠,遷諫大夫,代貢禹為長信少府、御史大夫。

廣德為人溫雅有醞藉。及為三公,直言諫爭。始拜旬日間,上幸甘泉,郊泰畤,禮畢,因留射獵。廣德上書曰:「竊見關東困極,人民流離。陛下日撞亡秦之鐘,聽鄭衛之樂,臣誠悼之。今士卒暴露,從官勞倦,願陛下亟反宮,思與百姓同憂樂,天下幸甚。」上即日還。其秋,上酎祭宗廟,出便門,欲御樓船,廣德當乘輿車,免冠頓首曰:「宜從橋。」詔曰:「大夫冠。」廣德曰:「陛下不聽臣,臣自刎,以血汙車輪,陛下不得入廟矣!」上不說。先蓝光祿大夫張猛進曰:「臣聞主聖臣直。乘船危,就橋安,聖主不乘危。御史大夫言可聽。」上曰:「曉人不當如是邪!」乃從橋。

後月餘,以歲惡民流,與丞相定國、大司馬車騎將軍史高俱乞骸骨,皆賜安車駟馬、黃金六十斤,罷。廣德為御史大夫,凡十月免。東歸沛,太守迎之界上。沛以為榮,縣其安車傳子孫。

平當字子思,祖父以訾百萬,自下邑徙平陵。當少為大行治禮丞,功次補大鴻臚文學,察廉為順陽長,栒邑令,以明經為博士,公卿薦當論議通明,給事中。每有災異,當輒傅經術,言得失。文雅雖不能及蕭望之、匡衡,然指意略同。

自元帝時,韋玄成為丞相,奏罷太上皇寢廟園,當上書言:「臣聞孔子曰:『如有王者,必世而後仁。』三十年之間,道德和洽,制禮興樂,災害不生,禍亂不作。今聖漢受命而王,繼體承業二百餘年,孜孜不怠,政令清矣。然風俗未和,陰陽未調,災害數見,意者大本有不立與?何德化休徵不應之久也!禍福不虛,必有因而至者焉。宜深跡其道而務修其本。昔者帝堯南面而治,先『克明俊德,以親九族』,而化及萬國。孝經曰:『天地之性人為貴,人之行莫大於孝,孝莫大於嚴父,嚴父莫大於配天,則周公其人也。』夫孝子善述人之志,周公既成文武之業而制作禮樂,修嚴父配天之事,知文王不欲以子臨父,故推而序之,上極於后稷而以配天。此聖人之德,亡以加於孝也。高皇帝聖德受命,有天下,尊太上皇,猶周文武之追王太王、王季也。此漢之始祖,後嗣所宜尊奉以廣盛德,孝之至也。《書》云:『正稽古建功立事,可以永年,傳於亡窮。』」上納其言,下詔復太上皇寢廟園。

頃之,使行流民幽州,舉奏刺史二千石勞來有意者,言勃海鹽池可且勿禁,以救民急。所過見稱,奉使者十一人為最,遷丞相司直。坐法,左遷朔方刺史,復徵入為太中大夫給事中,絫遷長信少府、大鴻臚、光祿勳。

先是太后姊子衛尉淳于長白言昌陵不可成,下有司議。當以為作治連年,可遂就。上既罷昌陵,以長首建忠策,復下公卿議封長。當又以為長雖有善言,不應封爵之科。坐前議不正,左遷鉅鹿太守。後上遂封長。當以經明禹貢,使行河,為騎都尉,領河隄。

哀帝即位,徵當為光祿大夫諸吏散騎,復為光祿勳,御史大夫,至丞相。以冬月,賜爵關內侯。明年春,上使使者召,欲封當。當病篤,不應召。室家或謂當:「不可強起受侯印為子孫邪?」當曰:「吾居大位,已負素餐之責矣,起受侯印,還臥而死,死有餘罪。今不起者,所以為子孫也。」遂上書乞骸骨。上報曰:「朕選於眾,以君為相,視事日寡,輔政未久,陰陽不調,冬無大雪,旱氣為災,朕之不德,何必君罪?君何疑而上書乞骸骨,歸關內侯爵邑?使尚書令譚賜君養牛一,上尊酒十石。君其勉致醫藥以自持。」後月餘,卒。子晏以明經歷位大司徒,封防鄉侯。漢興,唯韋、平父子至宰相。

彭宣字子佩,淮陽陽夏人也。治易,事張禹,舉為博士,遷東平太傅。禹以帝師見尊信,薦宣經明有威重,可任政事,繇是入為右扶風,遷廷尉,以王國人出為太原太守。數年,復入為大司農、光祿勳、右將軍。哀帝即位,徙為左將軍。歲餘,上欲令丁、傅處爪牙官,乃策宣曰:「有司數奏言諸侯國人不得宿衛,將軍不宜典兵馬,處大位。朕唯將軍任漢將之重,而子又前取淮陽王女,婚姻不絕,非國之制。使光祿大夫曼賜將軍黃金五十斤、安車駟馬,其上左將軍印綬,以關內侯歸家。」

宣罷數歲,諫大夫鮑宣數薦宣。會元壽元年正月朔日蝕,鮑宣復上言乃召宣為光祿大夫,遷御史大夫,轉為大司空,封長平侯。

會哀帝崩,新都侯王莽為大司馬,秉政專權。宣上書言:「三公鼎足承君,一足不任,則覆亂美實。臣資性淺薄,年齒老眊,數伏疾病,昏亂遺忘,願上大司空、長平侯印綬,乞骸骨歸鄉里,俟寘溝壑。」莽白太后,策宣曰:「惟君視事日寡,功德未效,迫于老眊昏亂,非所以輔國家,綏海內也。使光祿勳豐冊詔君,其上大司空印綬,便就國。」莽恨宣求退,故不賜黃金安車駟馬。宣居國數年,薨,諡曰頃侯。傳子至孫,王莽敗,乃絕。

贊曰:雋不疑學以從政,臨事不惑,遂立名跡,終始可述。疏廣行止足之計,免辱殆之絫,亦其次也。于定國父子哀鰥哲獄,為任職臣。薛廣德保縣車之榮,平當逡遁有恥,彭宣見險而止,異乎「苟患失之」者矣。


\end{pinyinscope}