\article{司馬相如傳}

\begin{pinyinscope}
司馬相如字長卿,蜀郡成都人也。少時好讀書,學擊劍,名犬子。相如既學,慕藺相如之為人也,更名相如。以訾為郎,事孝景帝,為武騎常侍,非其好也。會景帝不好辭賦,是時梁孝王來朝,從游說之士齊人鄒陽、淮陰枚乘、吳嚴忌夫子之徒,相如見而說之,因病免,客游梁,得與諸侯游士居,數歲,乃著子虛之賦。

會梁孝王薨,相如歸,而家貧無以自業。素與臨邛令王吉相善,吉曰:「長卿久宦游,不遂而困,來過我。」於是相如往舍都亭。臨邛令繆為恭敬,日往朝相如。相如初尚見之,後稱病,使從者謝吉,吉愈益謹肅。

臨邛多富人,卓王孫僮客八百人,程鄭亦數百人,乃相謂曰:「令有貴客,為具召之。并召令。」令既至,卓氏客以百數,至日中請司馬長卿,長卿謝病不能臨。臨邛令不敢嘗食,身自迎相如,相如為不得已而強往,一坐盡傾。酒酣,臨邛令前奏琴曰:「竊聞長卿好之,願以自娛。」相如辭謝,為鼓一再行。是時,卓王孫有女文君新寡,好音,故相如繆與令相重而以琴心挑之。相如時從車騎,雍容閒雅,甚都。及飲卓氏弄琴,文君竊從戶窺,心說而好之,恐不得當也。既罷,相如乃令侍人重賜文君侍者通殷勤。文君夜亡奔相如,相如與馳歸成都。家徒四壁立。卓王孫大怒曰:「女不材,我不忍殺,一錢不分也!」人或謂王孫,王孫終不聽。文君久之不樂,謂長卿曰:「弟俱如臨邛,從昆弟假貣,猶足以為生,何至自苦如此!」相如與俱之臨邛,盡賣車騎,買酒舍,乃令文君當盧。相如身自著犢鼻褌,與庸保雜作,滌器於市中。卓王孫恥之,為杜門不出。昆弟諸公更謂王孫曰:「有一男兩女,所不足者非財也。今文君既失身於司馬長卿,長卿故倦游,雖貧,其人材足依也。且又令客,奈何相辱如此!」卓王孫不得已,分與文君僮百人,錢百萬,及其嫁時衣被財物。文君乃與相如歸成都,買田宅,為富人。

居久之,蜀人楊得意為狗監,侍上。上讀子虛賦而善之,曰:「朕獨不得與此人同時哉!」得意曰:「臣邑人司馬相如自言為此賦。」上驚,乃召問相如。相如曰:「有是。然此乃詩侯之事,未足觀,請為天子游獵之賦。」上令尚書給筆札,相如以「子虛」,虛言也,為楚稱;「烏有先生」者,烏有此事也,為齊難;「亡是公」者,亡是人也,欲明天子之義。故虛藉此三人為辭,以推天子諸侯之苑囿。其卒章歸之於節儉,因以風諫。奏之天子,天子大說。其辭曰:

楚使子虛使於齊,齊王悉發車騎與使者出田。田罷,子虛過嚯烏有先生,亡是公存焉。坐定,烏有先生問曰:「今日田樂乎?」子虛曰:「樂。」「獲多乎?」曰:「少。」「然則何樂?」對曰:「僕樂王之欲夸僕以車騎之眾,而僕對以雲夢之事也。」曰:「可得聞乎?」

子虛曰:「可。王駕車千乘,選徒萬騎,田於海濱,列卒滿澤,罘罔彌山。掩菟轔鹿,射麋格麟,騖於鹽浦,割鮮染輪。射中獲多,矜而自功,顧謂僕曰:『楚亦有平原廣澤遊獵之地饒樂若此者乎?楚王之獵孰與寡人?』僕下車對曰:『臣,楚國之鄙人也,幸得宿衛十有餘年,時從出遊,遊於後園,覽於有無,然猶未能遍睹也。又烏足以言其外澤乎?』齊王曰:『雖然,略以子之所聞見言之。』

「僕對曰:『唯唯。臣聞楚有七澤,嘗見其一,未睹其餘也。臣之所見,蓋特其小小者耳,名曰雲夢。雲夢者,方九百里,其中有山焉。其山則盤紆岪鬱,隆崇律崒;岑崟參差,日月蔽虧;交錯糾紛,上干青雲;罷池陂鲶,下屬江河。其土則丹青赭堊,雌黃白匮,錫碧金銀,眾色炫燿,照爛龍鱗。其石則赤玉玫瑰,琳禄昆吾,瑊栎玄厲,礝石武夫。其東則有蕙圃,衡蘭芷若,穹窮昌蒲,江離蘪蕪,諸柘巴且。其南則有平原廣澤,登降阤靡,案衍壇曼,緣以大江,限以巫山。其高燥則生葴析苞荔,薜莎青薠。其埤溼則生藏莨蒹葭,東蘠彫胡,蓮藕觚盧,奄閭軒于。眾物居之,不可勝圖。其西則有涌泉清池,激水推移,外發夫容粮華,內隱鉅石白沙。其中則有神龜蛟鼉,毒冒鱉黿。其北則有陰林巨樹,楩柟豫章,桂椒木蘭,檗離朱楊,樝梨梬栗,橘柚芬芳。其上則有宛雛孔鸞,騰遠射干。其下則有白虎玄豹,蟃蜒貙豻。

「『於是乎乃使剸諸之倫,手格此獸。楚王乃駕馴駮之駟,乘彫玉之輿,靡魚須之橈旃,曳明月之珠旗,建干將之雄戟,左烏號之彫弓,右夏服之勁箭;陽子驂乘,孅阿為御;案節未舒,即陵狡獸,蹴蛩蛩,轔距虛,軼野馬,阌騊駼;乘遺風,射遊騏,儵胂倩浰,雷動焱至,星流電擊,弓不虛發,中必決眥,洞胸達掖,絕乎心繫,獲若雨獸,揜屮蔽地。於是楚王乃弭節徘徊,翱翔容與,覽乎陰林,觀壯士之暴怒,與猛獸之恐懼,徼镁受詘,殫睹眾物之變態。

「『於是鄭女曼姬,被阿錫,揄紵縞,雜纖羅,垂霧縠,襞積褰縐,鬱橈谿谷;衯衯裶裶,揚衪戌削,蜚襳垂髾;扶輿猗靡,翕呷萃蔡,下摩蘭蕙,上拂羽蓋;錯翡翠之葳蕤,繆繞玉綏;眇眇忽忽,若神之髣彿。

「『於是乃群相與獠於蕙圃,媻姍勃窣,上金隄,揜翡翠,射鵔鸃,微矰出,孅繳施,弋白鵠,連鴐鵝,雙鶬下,玄鶴加。怠而後游於清池,浮文鷁,揚旌枻,張翠帷,建羽蓋。罔毒冒,釣紫貝,摐金鼓,吹鳴籟,榜人歌,聲流喝,水蟲駭,波鴻沸,涌泉起,奔揚會,礧石相擊,琅琅磕磕,若雷霆之聲,聞乎數百里外。

「『將息獠者,擊靈鼓,起烽燧,車案行,騎就隊,纚乎淫淫,般乎裔裔。於是楚王乃登陽雲之臺,泊乎無為,澹乎自持,勺藥之和具而後御之。不若大王終日馳騁,曾不下輿,脟割輪焠,自以為娛。臣竊觀之,齊殆不如。』於是王無以應僕也。」

烏有先生曰:「是何言之過也!足下不遠千里,來況齊國,王悉境內之士,備車騎之眾,與使者出田,乃欲戮力致獲,以娛左右也,何名為夸哉!問楚地之有無者,願聞大國之風烈,先生之餘論也。今足下不稱楚王之德厚,而盛推雲夢以為驕,奢言淫樂而顯侈靡,竊為足下不取也。必若所言,固非楚國之美也。有而言之,是章君之惡也;無而言之,是害足下之信也。章君惡,傷私義,二者無一可,而先生行之,必且輕於齊而累於楚矣。且齊東陼鉅海,南有琅邪,觀乎成山,射乎之罘,浮勃澥,游孟諸,邪與肅慎為鄰,右以湯谷為界。秋田乎青丘,仿偟乎海外,吞若雲夢者八九,其於匈中曾不蔕芥。若乃俶儻瑰瑋,異方殊類,珍怪鳥獸,萬端鱗崒,充仞其中者,不可勝記,禹不能名,党不能計。然在諸侯之位,不敢言游戲之樂,苑囿之大;先生又見客,是以王辭不復,何為無以應哉!」

亡是公听然而笑曰:「楚則失矣,而齊亦未為得也。夫使諸侯納貢者,非為財幣,所以述職也;封彊畫界者,非為守禦,所以禁淫也。今齊列為東蕃,而外私肅慎,捐國隃限,越海而田,其於義固未可也。且二君之論,不務明君臣之義,正諸侯之禮,徒事爭於游戲之樂,苑囿之大,欲以奢侈相勝,荒淫相越,此不可以揚名發譽,而適足以貶君自損也。

「且夫齊楚之事又烏足道乎!君未睹夫巨麗也,獨不聞天子之上林乎?左蒼梧,右西極,丹水更其南,紫淵徑其北。終始霸產,出入涇渭,酆鎬潦潏,紆餘委蛇,經營其內。蕩蕩乎八川分流,相背異態,東西南北,馳騖往來,出乎椒丘之闕,行乎州淤之浦,徑乎桂林之中,過乎泱莽之野,汨乎混流,順阿而下,赴隘骥之口,觸穹石,激堆埼,沸乎暴怒,洶涌彭湃,滭弗宓汨,偪側泌瀄,潢流逆折,轉騰潎洌,滂濞沆溉,穹隆雲橈,宛潬膠盭,踰波趨浥,蒞蒞下瀨,批巖衝擁,奔揚滯沛,臨坻注壑,瀺灂霣隊,沈沈隱隱,砰磅訇磕,潏潏淈淈,湁潗鼎沸,馳波跳沫,汨莸漂疾,悠遠長懷,寂漻無聲,肆乎永歸。然後灝溔潢漾,安翔徐佪,翯乎滈滈,東注大湖,衍溢陂池。於是蛟龍赤螭,鳃蒇漸離,鰅鰫鰬魠,禺禺魼鰨,揵鰭掉尾,振鱗奮翼,潛處乎深巖。魚鱉讙聲,萬物眾夥。明月珠子,的皪江靡,蜀石黃浍,水玉磊砢,磷磷爛爛,采色澔汗,叢積乎其中。骚鷫鵠鴇,鴐鵝屬玉,交精旋目,煩鶩庸渠,箴疵鵁盧,群浮乎其上。汎淫氾濫,隨風澹淡,與波搖蕩,奄薄水陼,唼喋菁藻,咀嚼菱藕。

「於是乎崇山矗矗,巃嵷崔巍,深林巨木,嶄巖參差。九嵕巀嶭,南山峨峨,巖阤甗錡,骞崣崛崎,振溪通谷,蹇產溝瀆,锇呀豁閜,阜陵別闯,崴磈饽廆,丘虛堀礨,隱轔鬱饥,登降施靡,陂池貏豸。允溶淫鬻,散渙夷陸,亭皋千里,靡不被築。揜以綠蕙,被以江離,糅以蘼蕪,雜以留夷。布結縷,攢戾莎,揭車衡蘭,稿本射干,茈薑蘘荷,葴持若蓀,鮮支黃礫,蔣芧青薠,布濩閎澤,延曼太原,離靡廣衍,應風披靡,吐芳揚烈,郁郁菲菲,眾香發越,肸蠁布寫,晻薆咇茀。

「於是乎周覽氾觀,縝紛軋芴,芒芒怳忽,視之無端,察之無涯。日出東沼,入虖西陂。其南則隆冬生長,涌水躍波;其獸則庸旄貘犛,沈牛麈麋,赤首圜題,窮奇象犀。其北則盛夏含凍裂地,涉冰揭河;其獸則麒麟角端,騊駼橐駝,蛩蛩驒騱,駃騠驢鏽。

「於是乎離宮別館,彌山跨谷,高廊四注,重坐曲閣,華榱璧璫,輦道纚屬,步锒周流,長途中宿。夷嵕築堂,絫臺增成,巖突洞房。頫杳眇而無見,仰攀橑而捫天,奔星更於閨闥,宛虹拖於楯軒。青龍蚴蟉於東箱,象輿婉僤於西清,靈圉燕於閒館,偓佺之倫暴於南榮,醴泉涌於清室,通川過於中庭。磐石裖崖,嶔巖倚傾,嵯峨趸嶪,刻削崢嶸,玫瑰碧琳,珊瑚叢生,禄玉旁唐,玢豳文磷,赤瑕駁犖,雜释其間,晁采琬琰,和氏出焉。

「於是乎盧橘夏孰,黃甘橙楱,枇杷橪柿,亭柰厚朴,梬棗楊梅,櫻桃蒲陶,隱夫薁棣,荅遝離支,羅乎後宮,列乎北園,貤丘陵,下平原,揚翠葉,扤紫莖,發紅華,垂朱榮,煌煌扈扈,照曜鉅野。沙棠櫟櫧,華楓枰櫨,留落胥邪,仁頻并閭,欃檀木蘭,豫章女貞,長千仞,大連抱,夸條直暢,實葉葰楙,攢立叢倚,連卷欐佹,崔錯癹骫,坑衡閜砢,垂條扶疏,落英幡纚,紛溶萷蔘,猗柅從風,藰蒞芔歙,蓋象金石之聲,管籥之音。柴池茈虒,旋還乎後宮,雜襲絫輯,被山緣谷,循阪下隰,視之無端,究之亡窮。

「於是乎玄猿素雌,蜼玃飛蠝,蛭蜩玃蝚,獑胡豰蛫,棲息乎其間。長嘯哀鳴,翩幡互經,夭蟜枝格,偃蹇杪顛,隃絕梁,騰殊榛,捷垂條,掉希間,牢落陸離,爛漫遠遷。

「若此者數百千處,娛游往來,宮宿館舍,庖廚不徙,後宮不移,百官備具。

「於是乎背秋涉冬,天子校獵。乘鏤象,六玉虯,拖蜺旌,靡雲旗,前皮軒,後道游;孫叔奉轡,衛公參乘,扈從橫行,出乎四校之中。鼓嚴簿,縱獵者,江河為阹,泰山為櫓,車騎雷起,殷天動地,先後陸離,離散別追,淫淫裔裔,緣陵流澤,雲布雨施。生貔豹,搏豺狼,手熊羆,足野羊。蒙鶡蘇,恊白虎,被斑文,跨野馬,陵三嵕之危,下磧歷之坻,徑峻赴險,越壑厲水。推蜚廉,弄解廌,格蝦蛤,鋋猛氏,砺要褭,射封豕。箭不苟害,解脰陷腦;弓不虛發,應聲而倒。

「於是乘輿弭節徘徊,翱翔往來,睨部曲之進退,覽將帥之變態。然後侵淫促節,儵夐遠去,流離輕禽,蹴履狡獸,阌白鹿,捷狡菟。軼赤電,遺光耀,追怪物,出宇宙,彎蕃弱,滿白羽,射游梟,櫟蜚遽。擇肉而后發,先中而命處,弦矢分,蓺殪仆。

「然後揚節而上浮,陵驚風,歷駭猋,乘虛亡,與神俱,藺玄鶴,亂昆雞,遒孔鸞,促鵔鸃,拂翳鳥,捎鳳凰,捷鵷鶵,揜焦明。

「道盡塗殫,迴車而還。消译乎襄羊,降集乎北紘,率乎直指,揜乎反鄉,蹶石關,歷封巒,過璁鵲,望露寒,下堂梨,息宜春,西馳宣曲,濯鷁牛首,登龍臺,掩細柳,觀士大夫之勤略,鈞獵者之所得獲。徒車之所閵轢,騎之所蹂若,人之所蹈藉,與其窮極倦镁,驚憚讋伏,不被創刃而死者,它它藉藉,填阬滿谷,掩平彌澤。

「於是乎游戲懈怠,置酒乎顥天之臺,張樂乎膠葛之宇,撞千石之鐘,立萬石之虡,建翠華之旗,樹靈鼉之鼓,奏陶唐氏之舞,聽葛天氏之歌,千人倡,萬人和山陵為之震動,川谷為之蕩波。巴俞宋蔡,淮南干遮,文成顛歌,族居遞奏,金鼓迭起,鏗鎗闛鞈,洞心駭耳。荊吳鄭衛之聲,韶濩武象之樂,陰淫案衍之音,鄢郢繽紛,激楚結風,俳優侏儒,狄鞮之倡,所以娛耳目樂心意者,麗靡爛漫於前,靡曼美色於後。

「若夫青琴虙妃之徒,絕殊離俗,妖冶閑都,靚莊刻飾,便嬛銭約,柔橈議議,嫵媚孅弱,曳獨繭之褕涝,眇閻易以恤削,便姍嫳屑,與世殊服,芬芳漚鬱,酷烈淑郁,皓齒粲爛,宜笑的皪,長眉連娟,微睇綿藐,色授魂予,心愉於側。

「於是酒中樂酣,天子芒然而思,似若有亡,曰:『嗟乎,此大奢侈!朕以覽聽餘閒,無事棄日,順天道以殺伐,時休息以於此,恐後世靡麗,遂往而不返,非所以為繼嗣創業垂統也。』於是乎乃解酒罷獵,而命有司曰:『地可墾辟,悉為農郊,以贍氓隸,隤牆填塹,使山澤之民得至焉。實陂池而勿禁,虛宮館而勿仞。發倉廩以救貧窮,補不足,恤鰥寡,存孤獨。出德號,省刑罰,改制度,易服色,革正朔,與天下為始。』

「於是歷吉日以齋戒,襲朝服,乘法駕,建華旗,鳴玉鸞,游于六藝之囿,馳騖乎仁義之塗,覽觀春秋之林,射貍首,兼騶虞,弋玄鶴,舞干戚,戴雲罕,揜群雅,悲伐檀,樂樂胥,修容乎禮園,翱翔乎書圃,述易道,放怪獸,登明堂,坐清廟,恣群臣,奏得失,四海之內,靡不受獲。於斯之時,天下大說,鄉風而聽,隨流而化,芔然興道而遷義,刑錯而不用,德隆於三皇,功羡於五帝。若此,故獵乃可喜也。

「若夫終日馳騁,勞神苦形,罷車馬之用,抏士卒之精,費府庫之財,而無德厚之恩,務在獨樂,不顧眾庶,忘國家之政,貪雉菟之獲,則仁者不繇也。從此觀之,齊楚之事,豈不哀哉!地方不過千里,而囿居九百,是草木不得墾辟,而民無所食也。夫以諸侯之細,而樂萬乘之所侈,僕恐百姓被其尤也。」

於是二子愀然改容,超若自失,逡巡避席,曰:「鄙人固陋,不知忌諱,乃今日見教,謹受命矣。」

賦奏,天子以為郎。亡是公言上林廣大,山谷水泉萬物,及子虛言雲夢所有甚眾,侈靡多過其實,且非義理所止,故刪取其要,歸正道而論之。

相如為郎數歲,會唐蒙使略通夜郎、僰中,發巴蜀吏卒千人,郡又多為發轉漕萬餘人,用軍興法誅其渠率。巴蜀民大驚恐。上聞之,乃遣相如責唐蒙等,因諭告巴蜀民以非上意。檄曰:

告巴蜀太守:蠻夷自擅,不討之日久矣,時侵犯邊境,勞士大夫。陛下即位,存撫天下,集安中國,然後興師出兵,北征匈奴,單于怖駭,交臂受事,屈膝請和。康居西域,重譯納貢,稽首來享。移師東指,閩越相誅;右弔番禺,太子入朝。南夷之君,西僰之長,常效貢職,不敢惰怠,延頸舉踵,喁喁然,皆鄉風慕義,欲為臣妾,道里遼遠,山川阻深,不能自致。夫不順者已誅,而為善者未賞,故遣中郎將往賓之,發巴蜀之士各五百人以奉幣,衛使者不然,靡有兵革之事,戰鬥之患。今聞其乃發軍興制,驚懼子弟,憂患長老,郡又擅為轉粟運輸,皆非陛下之意也。當行者或亡逃自賊殺,亦非人臣之節也。

夫邊郡之士,聞烽舉燧燔,皆攝弓而馳,荷兵而走,流汗相屬,惟恐居後,觸白刃,冒流矢,議不反顧,計不旋踵,人懷怒心,如報私讎。彼豈樂死惡生,非編列之民,而與巴蜀異主哉?計深慮遠,急國家之難,而樂盡人臣之道也。故有剖符之封,析圭而爵,位為通侯,居列東第。終則遺顯號於後世,傳土地於子孫,事行甚忠敬,居位甚安佚,名聲施於無窮,功業著而不滅。是以賢人君子。肝腦塗中原,膏液潤埜屮而不辭也。今奉幣使至南夷,即自賊殺,或亡逃抵誅,身死無名,諡為至愚,恥及父母,為天下笑。人之度量相越,豈不遠哉!然此非獨行者之罪也,父兄之教不先,子弟之率不謹,寡廉鮮恥,而俗不長厚也。其被刑戮,不亦宜乎!

陛下患使者有司之若彼,悼不肖愚民之如此,故遣信使,曉諭百姓以發卒之事,因數之以不忠死亡之罪,讓三老孝弟以不教誨之過。方今田時,重煩百姓,已親見近縣,恐遠所谿谷山澤之民不遍聞,檄到,亟下縣道,咸喻陛下意,毋忽!

相如還報。唐蒙已略通夜郎,因通西南夷道,發巴蜀廣漢卒,作者數萬人。治道二歲,道不成,士卒多物故,費以億萬計。蜀民及漢用事者多言其不便。是時邛、莋之君長聞南夷與漢通,得賞賜多,多欲願為內臣妾,請吏,比南夷。上問相如,相如曰:「邛、莋、冉、駹者近蜀,道易通,異時嘗通為郡縣矣,至漢興而罷。今誠復通,為置縣,愈於南夷。」上以為然,乃拜相如為中郎將,建節往使。副使者王然于、壺充國、呂越人,馳四乘之傳,因巴蜀吏幣物以賂西南夷。至蜀,太守以下郊迎,縣令負弩矢先驅,蜀人以為寵。於是卓王孫、臨邛諸公皆因門下獻牛酒以交驩。卓王孫喟然而歎,自以得使女尚司馬長卿晚,乃厚分與其女財,與男等。相如使略定西南夷,邛、莋、冉、駹、斯榆之君皆請為臣妾,除邊關,益斥,西至沬、若水,南至牂牁為徼,通靈山道,橋孫水,以通邛、莋。還報,天子大說。

相如使時,蜀長老多言通西南夷之不為用,大臣亦以為然。相如欲諫,業已建之,不敢,乃著書,藉蜀父老為辭,而己詰難之,以風天子,且因宣其使詣,令百姓皆知天子意。其辭曰:

漢興七十有八載,德茂存乎六世,威武紛云,湛恩汪濊,群生霑濡,洋溢乎方外。於是乃命使西征,隨流而攘,風之所被,罔不披靡。因朝冉從駹,定莋存邛,略斯榆,舉苞蒲,結軌還轅,東鄉將報,至于蜀都。

耆老大夫搢紳先生之徒二十有七人,儼然造焉。辭畢,進曰:「蓋聞天子之於夷狄也,其義羈縻勿絕而已。今罷三郡之士,通夜郎之塗,三年於茲,而功不竟,士卒勞倦,萬民不贍;今又接之以西夷,百姓力屈,恐不能卒業,此亦使者之累也,竊為左右患之。且夫邛、莋、西僰之與中國並也,歷年茲多,不可記已。仁者不以德來,強者不以力并,意者殆不可乎!今割齊民以附夷狄,弊所恃以事無用,鄙人固陋,不識所謂。」

使者曰:「烏謂此乎?必若所云,則是蜀不變服而巴不化俗也,僕尚惡聞若說。然斯事體大,固非觀者之所覯也。余之行急,其詳不可得聞已。請為大夫粗陳其略:

「蓋世必有非常之人,然後有非常之事;有非常之事,然後有非常之功。非常者,固常人之所異也。故曰非常之元,黎民懼焉;及臻厥成,天下晏如也。

「昔者,洪水沸出,氾濫衍溢,民人升降移徙,崎嶇而不安。夏后氏戚之,乃堙洪原,決江疏河,灑沈澹災,東歸之於海,而天下永寧。當斯之勤,豈惟民哉?心煩於慮,而身親其勞,躬傶骿胝無胈,膚不生毛,故休烈顯乎無窮,聲稱浹乎于茲。

「且夫賢君之踐位也,豈特委瑣握愦,拘文牽俗,循誦習傳,當世取說云爾哉!必將崇論谹議,創業垂統,為萬世規。故馳騖乎兼容并包,而勤思乎參天貳地。且詩不云乎?『普天之下,莫非王土;率土之濱,莫非王臣。』是以六合之內,八方之外,浸淫衍溢,懷生之物有不浸潤於澤者,賢君恥之。今封疆之內,冠帶之倫,咸獲嘉祉,靡有闕遺矣。而夷狄殊俗之國,遼絕異黨之域,舟車不通,人跡罕至,政教未加,流風猶微,內之則犯義侵禮於邊境,外之則邪行橫作,放殺其上,君臣易位,尊卑失序,父兄不辜,幼孤為奴虜,係絫號泣。內鄉而怨,曰:『蓋聞中國有至仁焉,德洋恩普,物靡不得其所,今獨曷為遺己!』舉踵思慕,若枯旱之望雨,盭夫為之垂涕,況乎上聖,又烏能已?故北出師以討強胡,南馳使以誚勁越。四面風德,二方之君鱗集仰流,願得受號者以億計。故乃關沬、若,徼牂牁,鏤靈山,梁孫原,創道德之塗,垂仁義之統,將博恩廣施,遠撫長駕,使疏逖不閉,曶爽闇昧得燿乎光明,以偃甲兵於此,而息討伐於彼。遐邇一體,中外禔福,不亦康乎?夫拯民於沈溺,奉至尊之休德,反衰世之陵夷,繼周氏之絕業,天子之急務也。百姓雖勞,又惡可以已哉?

「且夫王者固未有不始於憂勤,而終於佚樂者也。然則受命之符合在於此。方將增太山之封,加梁父之事,鳴和鸞,揚樂頌,上咸五,下登三。觀者未睹指,聽者未聞音,猶焦朋已翔乎寥廓,而羅者猶視乎藪澤,悲夫!」

於是諸大夫茫然喪其所懷來,失厥所以進,喟然並稱曰:「允哉漢德,此鄙人之所願聞也。百姓雖勞,請以身先之。」敞罔靡徙,遷延而辭避。

其後人有上書言相如使時受金,失官。居歲餘,復召為郎。

相如口吃而善著書。常有消渴病。與卓氏婚,饒於財。故其事宦,未嘗肯與公卿國家之事,常稱疾閒居,不慕官爵。嘗從上至長楊獵。是時天子方好自擊熊豕,馳逐野獸,相如因上疏諫。其辭曰:

臣聞物有同類而殊能者,故力稱烏獲,捷言慶忌,勇期賁育。臣之愚,竊以為人誠有之,獸亦宜然。今陛下好陵阻險,射猛獸,卒然遇逸材之獸,駭不存之地,犯屬車之清塵,輿不及還轅,人不暇施巧,雖有烏獲、逢蒙之技不能用,枯木朽株盡為難矣。是胡越起於轂下,而羌夷接軫也,豈不殆哉!雖萬全而無患,然本非天子之所宜近也。

且夫清道而後行,中路而馳,猶時有銜庑之變。況乎涉豐草,騁丘虛,前有利獸之樂,而內無存變之意,其為害也不難矣!夫輕萬乘之重不以為安,樂出萬有一危之塗以為娛,臣竊為陛下不取。

蓋明者遠見於未萌,而知者避危於無形,禍固多藏於隱微而發於人之所忽者也。故鄙諺曰:「家絫千金,坐不垂堂。」此言雖小,可以諭大。臣願陛下留意幸察。

上善之。還過宜春宮,相如奏賦以哀二世行失。其辭曰:

登陂鲶之長阪兮,坌入會宮之嵯峨。臨曲江之隑州兮,望南山之參差。巖巖深山之谾谾兮,通谷崂乎锇苹。汨淢靸以永逝兮,注平皋之廣衍。觀眾樹之蓊薆兮,覽竹林之榛榛。東馳土山兮,北揭石瀨。弭節容與兮,歷弔二世。持身不謹兮,亡國失勢;信讒不寤兮,宗廟滅絕。烏乎!操行之不得,墓蕪穢而不修兮,魂亡歸而不食。

相如拜為孝文園令。上既美子虛之事,相如見上好僊,因曰:「

上林之事未足美也,尚有靡者。臣嘗為大人賦,未就,請具而奏之。」相如以為列僊之儒居山澤間,形容甚臞,此非帝王之僊意也,乃遂奏大人賦。其辭曰:

世有大人兮,在乎中州。宅彌萬里兮,曾不足以少留。悲世俗之迫隘兮,朅輕舉而遠游。乘絳幡之素蜺兮,載雲氣而上浮。建格澤之修竿兮,總光燿之采旄。垂旬始以為幓兮,曳彗星而為髾。掉指橋以偃寋兮,又猗抳以招搖。髓攙搶以為旌兮,靡屈虹而為綢。紅杳眇以玄湣兮,猋風涌而雲浮。駕應龍象輿之蠖略委麗兮,驂赤螭青虯之蚴蟉宛蜒。低卬夭蟜裾以驕驁兮,詘折隆窮躩以連卷。沛艾赳螑仡以佁儗兮,放散畔岸驤以孱顏。跮踱輵螛容以骫麗兮,蜩蟉偃寋怵铨以梁倚。糾蓼叫奡踏以芦路兮,薎蒙踊躍騰而狂趭。蒞颯芔歙焱至電過兮,煥然霧除,霍然雲消。

邪絕少陽而登太陰兮,與真人乎相求。互折窈窕以右轉兮,橫厲飛泉以正東。悉徵靈圉而選之兮,部署眾神於搖光。使五帝先導兮,反大壹而從陵陽。左玄冥而右黔雷兮,前長離而後矞皇。廝征伯僑而役羨門兮,詔岐伯使尚方。祝融警而蹕御兮,清氣氛而后行。屯余車而萬乘兮,綷雲蓋而樹華旗。使句芒其將行兮,吾欲往乎南娭。

歷唐堯於崇山兮,過虞舜於九疑。紛湛湛其差錯兮,雜遝膠輵以方馳。騷擾衝蓯其相紛挐兮,滂濞泱軋麗以林離。攢羅列聚叢以蘢茸兮,衍曼流爛痑以陸離。徑入雷室之砰磷鬱律兮,洞出鬼谷之堀礨崴魁。遍覽八紘而觀四海兮,朅度九江越五河。經營炎火而浮弱水兮,杭絕浮渚涉流沙。奄息蔥極氾濫水娭兮,使靈媧鼓琴而舞馮夷。時若曖曖將混濁兮,召屏翳誅風伯,刑雨師。西望崑崙之軋沕荒忽兮,直徑馳乎三危。排閶闔而入帝宮兮,載玉女而與之歸。登閬風而遙集兮,亢鳥騰而壹止。低徊陰山翔以紆曲兮,吾乃今日睹西王母。暠然白首戴勝而穴處兮,亦幸有三足烏為之使。必長生若此而不死兮,雖濟萬世不足以喜。

回車朅來兮,絕道不周,會食幽都。呼吸沆瀣兮餐朝霞,咀唣芝英兮嘰瓊華。僸祲尋而高縱兮,紛鴻溶而上厲。貫列缺之倒景兮,涉豐隆之滂濞。騁游道而脩降兮,騖遺霧而遠逝。迫區中之隘陝兮,舒節出乎北垠。遺屯騎於玄闕兮,軼先驅於寒門。下崢嶸而無地兮,上嵺廓而無天。視眩泯而亡見兮,聽敞怳而亡聞。乘虛亡而上遐兮,超無友而獨存。

相如既奏大人賦,天子大說,飄飄有陵雲氣游天地之閒意。

相如既病免,家居茂陵。天子曰:「司馬相如病甚,可往從悉取其書,若後之矣。」使所忠往,而相如已死,家無遺書。問其妻,對曰:「長卿未嘗有書也。時時著書,人又取去。長卿未死時,為一卷書,曰有使來求書,奏之。」其遺札書言封禪事,所忠奏焉,天子異之。其辭曰:

伊上古之初肇,自顥穹生民。歷選列辟,以迄乎秦。率邇者踵武,聽逖者風聲。紛輪威蕤,堙滅而不稱者,不可勝數也。繼昭夏,崇號諡,略可道者七十有二君。罔若淑而不昌,疇逆失而能存?

軒轅之前,遐哉邈乎,其詳不可得聞已。五三六經載籍之傳,維見可觀也。書曰:「元首明哉!股肱良哉!」因斯以談,君莫盛於堯,臣莫賢於后稷。后稷創業於唐,公劉發跡於西戎,文王改制,爰周郅隆,大行越成,而后陵夷衰微,千載亡聲,豈不善始善終哉!然無異端,慎所由於前,謹遺教於後耳。故軌跡夷易,易遵也;湛恩厖洪,易豐也;憲度著明,易則也;垂統理順,易繼也。是以業隆於繈保而崇冠乎二后。揆厥所元,終都攸卒,未有殊尤絕跡可考於今者也。然猶躡梁甫,登大山,建顯號,施尊名。大漢之德,逢涌原泉,沕潏曼羨,旁魄四塞,雲布霧散,上暢九垓,下泝八埏。懷生之類,沾濡浸潤,協氣橫流,武節焱逝,爾骥游原,迥闊泳末,首惡鬱沒,闇昧昭晰,昆蟲闓懌,回首面內。然后囿騶虞之珍群,徼麋鹿之怪獸,導一莖六穗於庖,犧雙觡共抵之獸,獲周餘放龜于岐,招翠黃乘龍於沼。鬼神接靈圉,賓於閒館。奇物譎詭,俶儻窮變。欽哉,符瑞臻茲,猶以為薄,不敢道封禪。蓋周躍魚隕杭,休之以燎。微夫斯之為符也,以登介丘,不亦恧乎!進攘之道,何其爽與?

於是大司馬進曰:「陛下仁育群生,義征不譓,諸夏樂貢,百蠻執贄,德牟往初,功無與二,休烈液洽,符瑞眾變,期應紹至,不特創見。意者太山、梁父設壇場望幸,蓋號以況榮,上帝垂恩儲祉,將以慶成,陛下嗛讓而弗發也。挈三神之歡,缺王道之儀,群臣恧焉。或謂且天為質闇,示珍符固不可辭;若然辭之,是泰山靡記而梁父罔幾也。亦各並時而榮,咸濟厥世而屈,說者尚何稱於後,而云七十二君哉?夫修德以錫符,奉符以行事,不為進越也。故聖王弗替,而修禮以祇,謁款天神,勒功中岳,以章至尊,舒盛德,發號榮,受厚福,以浸黎民。皇皇哉斯事,天下之壯觀,王者之卒業,不可貶也。願陛下全之。而后因雜縉紳先生之略術,使獲曜日月之末光絕炎,以展采錯事。猶兼正列其義,祓飾厥文,作春秋一藝。將襲舊六為七,攄之無窮,俾萬世得激清流,揚微波,蜚英聲,騰茂實。前聖之所以永保鴻名而常為稱首者用此。宜命掌故悉奏其儀而覽焉。」

於是天子沛然改容,曰:「俞乎,朕其試哉!」乃遷思回慮,總公卿之議,詢封禪之事,詩大澤之博,廣符瑞之富。遂作頌曰:

自我天覆,雲之油油。甘露時雨,厥壤可游。滋液滲漉,何生不育!嘉穀六穗,我穡曷蓄?

匪唯雨之,又潤澤之;匪唯偏我,氾布護之;萬物熙熙,懷而慕之。名山顯位,望君之來。君兮君兮,侯不邁哉!

般般之獸,樂我君圃;白質黑章,其儀可喜;旼旼穆穆,君子之態。蓋聞其聲,今視其來。厥塗靡從,天瑞之徵。茲爾於舜,虞氏以興。

濯濯之麟,游彼靈畤。孟冬十月,君徂郊祀。馳我君輿,帝用享祉。三代之前,蓋未嘗有。

宛宛黃龍,興德而升;采色玄耀,炳炳煇煌。正陽顯見,覺寤黎烝。於傳載之,云受命所乘。

厥之有章,不必諄諄。依類託寓,諭以封巒。

披藝觀之,天人之際已交,上下相發允答。聖王之事,兢兢翼翼。故曰於興必慮衰,安必思危。是以湯武至尊嚴,不失肅祗,舜在假典,顧省厥遺:此之謂也。

相如既卒五歲,上始祭后土。八年而遂禮中岳,封于太山,至梁甫,禪肅然。

相如它所著,苦遺平陵侯書、與五公子相難、屮木書篇,不采,采其尤著公卿者云。

贊曰:司馬遷稱「春秋推見至隱,易本隱以之顯,大雅言王公大人,而德逮黎庶,小雅譏小己之得失,其流及上。所言雖殊,其合德一也。相如雖多虛辭濫說,然要其歸引之於節儉,此亦詩之風諫何異?」揚雄以為靡麗之賦,勸百而風一,猶騁鄭衛之聲,曲終而奏雅,不已戲乎!


\end{pinyinscope}