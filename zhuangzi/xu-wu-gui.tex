\article{徐無鬼}

\begin{pinyinscope}
徐無鬼因女商見魏武侯,武侯勞之曰:「先生病矣!苦於山林之勞,故乃肯見於寡人。」徐無鬼曰:「我則勞於君,君有何勞於我?君將盈耆欲,長好惡,則性命之情病矣;君將黜耆欲,掔好惡,則耳目病矣。我將勞君,君有何勞於我?」武侯超然不對。

少焉,徐無鬼曰:「嘗語君,吾相狗也。下之質,執飽而止,是狸德也;中之質,若視日;上之質,若亡其一。吾相狗,又不若吾相馬也。吾相馬,直者中繩,曲者中鉤,方者中矩,圓者中規,是國馬也,而未若天下馬也。天下馬有成材,若卹若失,若喪其一,若是者,超軼絕塵,不知其所。」武侯大悅而笑。

徐無鬼出,女商曰:「先生獨何以說吾君乎?吾所以說吾君者,橫說之則以《詩》、《書》、《禮》、《樂》,從說之則以金板、六弢,奉事而大有功者不可為數,而吾君未嘗啟齒。今先生何以說吾君,使吾君說若此乎?」徐無鬼曰:「吾直告之吾相狗馬耳。」女商曰:「若是乎」?曰:「子不聞夫越之流人乎?去國數日,見其所知而喜;去國旬月,見其所嘗見於國中者喜;及期年也,見似人者而喜矣。不亦去人滋久,思人滋深乎!夫逃虛空者,藜、藋柱乎鼪、鼬之逕,踉位其空,聞人足音跫然而喜矣,而況乎兄弟親戚之謦欬其側者乎!久矣夫!莫以真人之言謦欬吾君之側乎!」

徐無鬼見武侯,武侯曰:「先生居山林,食芧栗,厭蔥韭,以賓寡人,久矣夫!今老邪?其欲干酒肉之味邪?其寡人亦有社稷之福邪?」徐無鬼曰:「無鬼生於貧賤,未嘗敢飲食君之酒肉,將來勞君也。」君曰:「何哉?奚勞寡人?」曰:「勞君之神與形。」武侯曰:「何謂邪?」徐無鬼曰:「天地之養也一,登高不可以為長,居下不可以為短。君獨為萬乘之主,以苦一國之民,以養耳目鼻口,夫神者不自許也。夫神者,好和而惡姦。夫姦,病也,故勞之。唯君所病之,何也?」

武侯曰:「欲見先生久矣。吾欲愛民而為義偃兵,可乎?」徐無鬼曰:「不可。愛民,害民之始也;為義偃兵,造兵之本也。君自此為之,則殆不成。凡成美,惡器也。君雖為仁義,幾且偽哉!形固造形,成固有伐,變固外戰。君亦必無盛鶴列於麗譙之間,無徒驥於錙壇之宮,無藏逆於得,無以巧勝人,無以謀勝人,無以戰勝人。夫殺人之士民,兼人之土地,以養吾私與吾神者,其戰不知孰善?勝之惡乎在?君若勿已矣,修胸中之誠,以應天地之情而勿攖。夫民死已脫矣,君將惡乎用夫偃兵哉!」

黃帝將見大隗乎具茨之山,方明為御,昌宇驂乘,張若、謵朋前馬,昆閽、滑稽後車。至於襄城之野,七聖皆迷,無所問塗。適遇牧馬童子,問塗焉,曰:「若知具茨之山乎?」曰:「然。」「若知大隗之所存乎?」曰:「然。」黃帝曰:「異哉小童!非徒知具茨之山,又知大隗之所存。請問為天下。」小童曰:「夫為天下者,亦若此而已矣,又奚事焉?予少而自遊於六合之內,予適有瞀病,有長者教予曰:『若乘日之車,而遊於襄城之野。』今予病少痊,予又且復遊於六合之外。夫為天下,亦若此而已。予又奚事焉?」黃帝曰:「夫為天下者,則誠非吾子之事。雖然,請問為天下。」小童辭。黃帝又問。小童曰:「夫為天下者,亦奚以異乎牧馬者哉?亦去其害馬者而已矣。」黃帝再拜稽首,稱天師而退。

知士無思慮之變則不樂,辯士無談說之序則不樂,察士無淩誶之事則不樂,皆囿於物者也。招世之士興朝,中民之士榮官,筋力之士矜難,勇敢之士奮患,兵革之士樂戰,枯槁之士宿名,法律之士廣治,禮教之士敬容,仁義之士貴際。農夫無草萊之事則不比,商賈無市井之事則不比。庶人有旦暮之業則勸,百工有器械之巧則壯。錢財不積則貪者憂,權勢不尤則夸者悲。勢物之徒樂變,遭時有所用,不能無為也。此皆順比於歲,不物於易者也,馳其形性,潛之萬物,終身不反,悲夫!

莊子曰:「射者非前期而中,謂之善射,天下皆羿也,可乎?」惠子曰:「可。」莊子曰:「天下非有公是也,而各是其所是,天下皆堯也,可乎?」惠子曰:「可。」

莊子曰:「然則,儒、墨、楊、秉四,與夫子為五,果孰是邪?或者若魯遽者邪?其弟子曰:『我得夫子之道矣,吾能冬爨鼎而夏造冰矣。』魯遽曰:『是直以陽召陽,以陰召陰,非吾所謂道也。吾示子乎吾道。』於是為之調瑟,廢一於堂,廢一於室,鼓宮宮動,鼓角角動,音律同矣。夫或改調一弦,於五音無當也,鼓之二十五弦皆動,未始異於聲,而音之君已。且若是者邪?」

惠子曰:「今夫儒、墨、楊、秉,且方與我以辯,相拂以辭,相鎮以聲,而未始吾非也,則奚若矣?」莊子曰:「齊人蹢子於宋者,其命閽也不以完,其求鈃鍾也以束縛,其求唐子也而未始出域,有遺類矣夫!楚人寄而蹢閽者,夜半於無人之時而與舟人鬥,未始離於岑,而足以造於怨也。」

莊子送葬,過惠子之墓,顧謂從者曰:「郢人堊慢其鼻端若蠅翼,使匠石斲之。匠石運斤成風,聽而斲之,盡堊而鼻不傷,郢人立不失容。宋元君聞之,召匠石曰:『嘗試為寡人為之。』匠石曰:『臣則嘗能斲之。雖然,臣之質死久矣。』自夫子之死也,吾無以為質矣,吾無與言之矣。」

管仲有病,桓公問之曰:「仲父之病病矣,可不謂云,至於大病,則寡人惡乎屬國而可?」管仲曰:「公誰欲與?」公曰:「鮑叔牙。」曰:「不可。其為人,絜廉善士也,其於不己若者不比之;又一聞人之過,終身不忘。使之治國,上且鉤乎君,下且逆乎民。其得罪於君也,將弗久矣。」公曰:「然則孰可?」對曰:「勿已,則隰朋可。其為人也,上忘而下畔,愧不若黃帝而哀不己若者。以德分人謂之聖,以財分人謂之賢。以賢臨人,未有得人者也;以賢下人,未有不得人者也。其於國有不聞也,其於家有不見也。勿已,則隰朋可。」

吳王浮於江,登乎狙之山。眾狙見之,恂然棄而走,逃於深蓁。有一狙焉,委蛇攫搔,見巧乎王王射之,敏給搏捷矢。王命相者趨射,狙執死。王顧謂其友顏不疑曰:「之狙也,伐其巧、恃其便,以敖予,以至此殛也。戒之哉!嗟乎,無以汝色驕人哉!」顏不疑歸而師董梧,以助其色,去樂辭顯,三年而國人稱之。

南伯子綦隱几而坐,仰天而噓。顏成子入見曰:「夫子,物之尤也。形固可使若槁骸,心固可使若死灰乎?」曰:「吾嘗居山穴之中矣。當是時也,田禾一覩我,而齊國之眾三賀之。我必先之,彼故知之;我必賣之,彼故鬻之。若我而不有之,彼惡得而知之?若我而不賣之,彼惡得而鬻之?嗟乎!我悲人之自喪者,吾又悲夫悲人者,吾又悲夫悲人之悲者,其後而日遠矣。」

仲尼之楚,楚王觴之,孫叔敖執爵而立,市南宜僚受酒而祭曰:「古之人乎!於此言已。」曰:「丘也聞不言之言矣,未之嘗言,於此乎言之。市南宜僚弄丸而兩家之難解,孫叔敖甘寢秉羽而郢人投兵。丘願有喙三尺。」

彼之謂不道之道,此之謂不言之辯。故德總乎道之所一,而言休乎知之所不知,至矣。道之所一者,德不能同也;知之所不能知者,辯不能舉也。名若儒、墨而凶矣。故海不辭東流,大之至也。聖人并包天地,澤及天下,而不知其誰氏。是故生無爵,死無諡,實不聚,名不立,此之謂大人。

狗不以善吠為良,人不以善言為賢,而況為大乎!夫為大不足以為大,而況為德乎!夫大備矣,莫若天地;然奚求焉,而大備矣。知大備者,無求、無失、無棄,不以物易己也。反己而不窮,循古而不摩,大人之誠。

子綦有八子,陳諸前,召九方歅曰:「為我相吾子,孰為祥?」九方歅曰:「梱也為祥。」子綦瞿然喜曰:「奚若?」曰:「梱也將與國君同食以終其身。」子綦索然出涕曰:「吾子何為以至於是極也!」九方歅曰:「夫與國君同食,澤及三族,而況父母乎?今夫子聞之而泣,是禦福也。子則祥矣,父則不祥。」

子綦曰:「歅!汝何足以識之?而梱祥邪,盡於酒肉,入於鼻口矣。而何足以知其所自來?吾未嘗為牧而牂生於奧,未嘗好田而鶉生於宎,若勿怪,何邪?吾所與吾子遊者,遊於天地。吾與之邀樂於天,吾與之邀食於地;吾不與之為事,不與之為謀,不與之為怪;吾與之乘天地之誠而不以物與之相攖,吾與之一委蛇而不與之為事所宜。今也然有世俗之償焉!凡有怪徵者,必有怪行。殆乎!非我與吾子之罪,幾天與之也!吾是以泣也。」

無幾何而使梱之於燕,盜得之於道,全而鬻之則難,不若刖之則易,於是乎刖而鬻之於齊,適當渠公之街,然身食肉而終。

齧缺遇許由,曰:「子將奚之?」曰:「將逃堯。」曰:「奚謂邪?」曰:「夫堯,畜畜然仁,吾恐其為天下笑。後世其人與人相食與!夫民不難聚也,愛之則親,利之則至,譽之則勸,致其所惡則散。愛利出乎仁義,捐仁義者寡,利仁義者眾。夫仁義之行,唯且無誠,且假乎禽貪者器。是以一人之斷制利天下,譬之猶一覕也。夫堯知賢人之利天下也,而不知其賊天下也,夫唯外乎賢者知之矣。」

有暖姝者,有濡需者,有卷婁者。

所謂暖姝者,學一先生之言,則暖暖姝姝而私自說也,自以為足矣,而未知未始有物也,是以謂暖姝者也。

濡需者,豕蝨是也。擇疏鬣,自以為廣宮大囿,奎蹄曲隈,乳閒股腳,自以為安室利處,不知屠者之一旦鼓臂、布草、操煙火,而己與豕俱焦也。此以域進,此以域退,此其所謂濡需者也。

卷婁者,舜也。羊肉不慕蟻,蟻慕羊肉,羊肉羶也。舜有羶行,百姓悅之,故三徙成都,至鄧之虛而十有萬家。堯聞舜之賢,舉之童土之地,曰冀得其來之澤。舜舉乎童土之地,年齒長矣,聰明衰矣,而不得休歸,所謂卷婁者也。

是以神人惡眾至,眾至則不比,不比則不利也。故無所甚親,無所甚疏,抱德煬和,以順天下,此謂真人。於蟻棄知,於魚得計,於羊棄意。以目視目,以耳聽耳,以心復心,若然者,其平也繩,其變也循。古之真人,以天待之,不以人入天。

古之真人,得之也生,失之也死;得之也死,失之也生。藥也,其實堇也。桔梗也,雞壅也,豕零也,是時為帝者也,何可勝言!

句踐也以甲楯三千,棲於會稽。唯種也能知亡之所以存,唯種也不知身之所以愁。故曰:鴟目有所適,鶴脛有所節,解之也悲。故曰:風之過河也有損焉,日之過河也有損焉。請只風與日相與守河,而河以為未始其攖也,恃源而往者也。故水之守土也審,影之守人也審,物之守物也審。

故目之於明也殆,耳之於聰也殆,心之於殉也殆。凡能其於府也殆,殆之成也不給改。禍之長也茲萃,其反也緣功,其果也待久。而人以為己寶,不亦悲乎!故有亡國戮民無已,不知問是也。

故足之於地也踐,雖踐,恃其所不蹍而後善博也;人之於知也少,雖少,恃其所不知而後知天之所謂也。知大一,知大陰,知大目,知大均,知大方,知大信,知大定,至矣。大一通之,大陰解之,大目視之,大均緣之,大方體之,大信稽之,大定持之。

盡有天,循有照,冥有樞,始有彼。則其解之也似不解之者,其知之也似不知之也,不知而後知之。其問之也,不可以有崖,而不可以無崖。頡滑有實,古今不代,而不可以虧,則可不謂有大揚搉乎!闔不亦問是已,奚惑然為!以不惑解惑,復於不惑,是尚大不惑。


\end{pinyinscope}