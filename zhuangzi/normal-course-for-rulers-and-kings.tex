\article{應帝王}

\begin{pinyinscope}
齧缺問於王倪,四問而四不知。齧缺因躍而大喜,行以告蒲衣子。蒲衣子曰:「而乃今知之乎?有虞氏不及泰氏。有虞氏,其猶藏仁以要人,亦得人矣,而未始出於非人。泰氏,其臥徐徐,其覺于于,一以己為馬,一以己為牛,其知情信,其德甚真,而未始入於非人。」

肩吾見狂接輿。狂接輿曰:「日中始何以語女?」肩吾曰:「告我:君人者,以己出經式義度,人孰敢不聽而化諸!」狂接輿曰:「是欺德也。其於治天下也,猶涉海鑿河,而使蚉負山也。夫聖人之治也,治外乎?正而後行,確乎能其事者而已矣。且鳥高飛以避矰弋之害,鼷鼠深穴乎神丘之下,以避熏鑿之患,而曾二蟲之無知!」

天根遊於殷陽,至蓼水之上,適遭無名人而問焉,曰:「請問為天下。」無名人曰:「去!汝鄙人也,何問之不豫也!予方將與造物者為人,厭則又乘夫莽眇之鳥,以出六極之外,而遊無何有之鄉,以處壙埌之野。汝又何帠以治天下感予之心為?」又復問。無名人曰:「汝遊心於淡,合氣於漠,順物自然,而無容私焉,而天下治矣。」

陽子居見老聃曰:「有人於此,嚮疾強梁,物徹疏明,學道不倦。如是者,可比明王乎?」老聃曰:「是於聖人也,胥易技係,勞形怵心者也。且也虎豹之文來田,猿狙之便、執嫠之狗來藉。如是者,可比明王乎?」陽子居蹴然曰:「敢問明王之治。」老聃曰:「明王之治,功蓋天下而似不自己,化貸萬物而民弗恃,有莫舉名,使物自喜,立乎不測,而遊於無有者也。」

鄭有神巫曰季咸,知人之生死存亡,禍福壽夭,期以歲月旬日,若神。鄭人見之,皆棄而走。列子見之而心醉,歸以告壺子,曰:「始吾以夫子之道為至矣,則又有至焉者矣。」壺子曰:「吾與汝既其文,未既其實,而固得道與?」眾雌而無雄,而又奚卵焉!而以道與世亢必信,夫故使人得而相女。嘗試與來,以予示之。」明日,列子與之見壺子。出而謂列子曰:「嘻!子之先生死矣,弗活矣,不以旬數矣!吾見怪焉,見溼灰焉。」列子入,泣涕沾襟,以告壺子。壺子曰:「鄉吾示之以地文,萌乎不震不正。是殆見吾杜德機也。嘗又與來。」明日,又與之見壺子。出而謂列子曰:「幸矣!子之先生遇我也。有瘳矣,全然有生矣。吾見其杜權矣。」列子入,以告壺子。壺子曰:「鄉吾示之以天壤,名實不入,而機發於踵。是殆見吾善者機也。嘗又與來。」明日,又與之見壺子。出而謂列子曰:「子之先生不齊,吾無得而相焉。試齊,且復相之。」列子入,以告壺子。壺子曰:「吾鄉示之以太沖莫勝。是殆見吾衡氣機也。鯢桓之審為淵,止水之審為淵,流水之審為淵。淵有九名,此處三焉。嘗又與來。」明日,又與之見壺子。立未定,自失而走。壺子曰:「追之!」列子追之不及,反以報壺子,曰:「已滅矣,已失矣,吾弗及也。」壺子曰:「鄉吾示之以未始出吾宗。吾與之虛而委蛇,不知其誰何,因以為弟靡,因以為波流,故逃也。」然後列子自以為未始學而歸,三年不出。為其妻爨,食豕如食人。於事無與親,彫琢復朴,塊然獨以其形立。紛而封哉,一以是終。

無為名尸,無為謀府,無為事任,無為知主。體盡無窮,而遊無朕,盡其所受於天,而無見得,亦虛而已。至人之用心若鏡,不將不迎,應而不藏,故能勝物而不傷。

南海之帝為儵,北海之帝為忽,中央之帝為渾沌。儵與忽時相與遇於渾沌之地,渾沌待之甚善。儵與忽謀報渾沌之德,曰:「人皆有七竅,以視聽食息,此獨無有,嘗試鑿之。」日鑿一竅,七日而渾沌死。


\end{pinyinscope}