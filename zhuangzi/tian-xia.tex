\article{天下}

\begin{pinyinscope}
天下之治方術者多矣,皆以其有為不可加矣。古之所謂道術者,果惡乎在?曰:「無乎不在。」曰:「神何由降?明何由出?」「聖有所生,王有所成,皆原於一。」

不離於宗,謂之天人。不離於精,謂之神人。不離於真,謂之至人。以天為宗,以德為本,以道為門,兆於變化,謂之聖人。以仁為恩,以義為理,以禮為行,以樂為和,薰然慈仁,謂之君子。以法為分,以名為表,以參為驗,以稽為決,其數一二三四是也。百官以此相齒,以事為常,以衣食為主,蕃息畜藏,老弱孤寡為意,皆有以養,民之理也。

古之人其備乎!配神明,醇天地,育萬物,和天下,澤及百姓,明於本數,係於末度,六通四辟,小大精粗,其運無乎不在。其明而在數度者,舊法世傳之史尚多有之。其在於《詩》、《書》、《禮》、《樂》者,鄒、魯之士、搢紳先生多能明之。《詩》以道志,《書》以道事,《禮》以道行,《樂》以道和,《易》以道陰陽,《春秋》以道名分。其數散於天下而設於中國者,百家之學時或稱而道之。

天下大亂,賢聖不明,道德不一,天下多得一察焉以自好。譬如耳目鼻口,皆有所明,不能相通。猶百家眾技也,皆有所長,時有所用。雖然,不該不遍,一曲之士也。判天地之美,析萬物之理,察古人之全,寡能備於天地之美,稱神明之容。是故內聖外王之道,闇而不明,鬱而不發,天下之人各為其所欲焉以自為方。悲夫!百家往而不反,必不合矣。後世之學者,不幸不見天地之純,古人之大體,道術將為天下裂。

不侈於後世,不靡於萬物,不暉於數度,以繩墨自矯,而備世之急,古之道術有在於是者。墨翟、禽滑釐聞其風而說之。為之大過,己之大循。作為《非樂》,命之曰《節用》,生不歌,死無服。墨子汎愛兼利而非鬥,其道不怒;又好學而博,不異,不與先王同,毀古之禮樂。

黃帝有《咸池》,堯有《大章》,舜有《大韶》,禹有《大夏》,湯有《大濩》,文王有辟雍之樂,武王、周公作《武》。古之喪禮,貴賤有儀,上下有等,天子棺槨七重,諸侯五重,大夫三重,士再重。今墨子獨生不歌,死不服,桐棺三寸而無槨,以為法式。以此教人,恐不愛人;以此自行,固不愛己。未敗墨子道,雖然,歌而非歌,哭而非哭,樂而非樂,是果類乎?其生也勤,其死也薄,其道大觳,使人憂,使人悲,其行難為也,恐其不可以為聖人之道,反天下之心,天下不堪。墨子雖能獨任,奈天下何!離於天下,其去王也遠矣。

墨子稱道曰:「昔者禹之湮洪水,決江河而通四夷九州也,名山三百,支川三千,小者無數。禹親自操稿耜而九雜天下之川,腓無胈,脛無毛,沐甚雨,櫛疾風,置萬國。禹,大聖也,而形勞天下也如此。」使後世之墨者多以裘褐為衣,以跂蹻為服,日夜不休,以自苦為極,曰:「不能如此,非禹之道也,不足謂墨。」相里勤之弟子五侯之徒,南方之墨者苦獲、已齒、鄧陵子之屬,俱誦《墨經》,而倍譎不同,相謂別墨,以堅白、同異之辯相訾,以觭偶不仵之辭相應,以巨子為聖人,皆願為之尸,冀得為其後世,至今不決。

墨翟、禽滑釐之意則是,其行則非也。將使後世之墨者必自苦以腓無胈、脛無毛,相進而已矣。亂之上也,治之下也。雖然,墨子真天下之好也,將求之不得也,雖枯槁不舍也,才士也!

夫不累於俗,不飾於物,不苟於人,不忮於眾,願天下之安寧以活民命,人我之養畢足而止,以此白心,古之道術有在於是者。宋鈃、尹文聞其風而悅之。作為華山之冠以自表,接萬物以別宥為始。語心之容,命之曰心之行,以聏合驩,以調海內,請欲置之以為主。見侮不辱,救民之鬥;禁攻寢兵,救世之戰。以此周行天下,上說下教,雖天下不取,強聒而不舍者也。故曰:「上下見厭而強見也。」雖然,其為人太多,其自為太少,曰:「請欲固置五升之飯足矣,先生恐不得飽,弟子雖飢,不忘天下。」日夜不休,曰:「我必得活哉!」圖傲乎救世之士哉!曰:「君子不為苛察,不以身假物。」以為無益於天下者,明之不如已也。以禁攻寢兵為外,以情欲寡淺為內,其小大精粗,其行適至是而止。

公而不當,易而無私,決然無主,趣物而不兩,不顧於慮,不謀於知,於物無擇,與之俱往,古之道術有在於是者。彭蒙、田駢、慎到聞其風而說之。齊萬物以為首,曰:「天能覆之而不能載之,地能載之而不能覆之,大道能包之而不能辯之。」知萬物皆有所可,有所不可,故曰:「選則不遍,教則不至,道則無遺者矣。」是故慎到,棄知去己,而緣不得已,泠汰於物以為道理,曰:「知不知,將薄知而後鄰傷之者也。」謑髁無任而笑天下之尚賢也,縱脫無行而非天下之大聖,椎拍輐斷,與物宛轉,舍是與非,苟可以免,不師知慮,不知前後,魏然而已矣。推而後行,曳而後往,若飄風之還,若羽之旋,若磨石之隧,全而無非,動靜無過,未嘗有罪。是何故?夫無知之物,無建己之患,無用知之累,動靜不離於理,是以終身無譽。故曰:「至於若無知之物而已,無用賢聖,夫塊不失道。」豪桀相與笑之曰:「慎到之道,非生人之行而至死人之理,適得怪焉。」田駢亦然,學於彭蒙,得不教焉。彭蒙之師曰:「古之道人,至於莫之是、莫之非而已矣。其風窢然,惡可而言?」常反人,不見觀,而不免於鯇斷。其所謂道非道,而所言之韙不免於非。彭蒙、田駢、慎到不知道。雖然,概乎皆嘗有聞者也。

以本為精,以物為粗,以有積為不足,澹然獨與神明居,古之道術有在於是者。關尹、老聃聞其風而悅之。建之以常無有,主之以太一,以濡弱謙下為表,以空虛不毀萬物為實。

關尹曰:「在己無居,形物自著。其動若水,其靜若鏡,其應若響。芴乎若亡,寂乎若清,同焉者和,得焉者失。未嘗先人而常隨人。」

老聃曰:「知其雄,守其雌,為天下谿;知其白,守其辱,為天下谷。」人皆取先,己獨取後,曰:「受天下之垢。」人皆取實,己獨取虛,無藏也故有餘,巋然而有餘。其行身也,徐而不費,無為也而笑巧。人皆求福,己獨曲全,曰:「苟免於咎。」以深為根,以約為紀,曰:「堅則毀矣,銳則拙矣。」常寬容於物,不削於人,可謂至極。關尹、老聃乎!古之博大真人哉!

芴漠無形,變化無常,死與生與!天地並與!神明往與!芒乎何之?忽乎何適?萬物畢羅,莫足以歸,古之道術有在於是者。莊周聞其風而悅之。以謬悠之說,荒唐之言,無端崖之辭,時恣縱而不儻,不以觭見之也。以天下為沈濁,不可與莊語;以卮言為曼衍,以重言為真,以寓言為廣。獨與天地精神往來,而不敖倪於萬物,不譴是非,以與世俗處。其書雖瑰瑋而連犿無傷也,其辭雖參差而諔詭可觀。彼其充實不可以已,上與造物者遊,而下與外死生、無終始者為友。其於本也,宏大而辟,深閎而肆;其於宗也,可謂稠適而上遂矣。雖然,其應於化而解於物也,其理不竭,其來不蛻,芒乎昧乎,未之盡者。

惠施多方,其書五車,其道舛駁,其言也不中。歷物之意,曰:「至大無外,謂之大一;至小無內,謂之小一。無厚不可積也,其大千里。天與地卑,山與澤平。日方中方睨,物方生方死。大同而與小同異,此之謂小同異;萬物畢同畢異,此之謂大同異。南方無窮而有窮,今日適越而昔來。連環可解也。我知天下之中央,燕之北,越之南是也。氾愛萬物,天地一體也。」

惠施以此為大觀於天下而曉辯者,天下之辯者相與樂之。卵有毛,雞三足,郢有天下,犬可以為羊,馬有卵,丁子有尾,火不熱,山出口,輪不蹍地,目不見,指不至,至不絕,龜長於蛇,矩不方,規不可以為圓,鑿不圍枘,飛鳥之景未嘗動也,鏃矢之疾而有不行不止之時,狗非犬,黃馬、驪牛三,白狗黑,孤駒未嘗有母,一尺之捶,日取其半,萬世不竭。辯者以此與惠施相應,終身無窮。

桓團、公孫龍辯者之徒,飾人之心,易人之意,能勝人之口,不能服人之心,辯者之囿也。惠施日以其知,與人之辯,特與天下之辯者為怪,此其柢也。

然惠施之口談,自以為最賢,曰:「天地其壯乎!」施存雄而無術。南方有倚人焉,曰黃繚,問天地所以不墜不陷,風雨雷霆之故。惠施不辭而應,不慮而對,遍為萬物說;說而不休,多而無已,猶以為寡,益之以怪。以反人為實,而欲以勝人為名,是以與眾不適也。弱於德,強於物,其塗隩矣。由天地之道觀惠施之能,其猶一蚉一虻之勞者也,其於物也何庸!夫充一尚可,曰愈貴,道幾矣!惠施不能以此自寧,散於萬物而不厭,卒以善辯為名。惜乎!惠施之才,駘蕩而不得,逐萬物而不反,是窮響以聲,形與影競走也。悲夫!


\end{pinyinscope}