\article{刻意}

\begin{pinyinscope}
刻意尚行,離世異俗,高論怨誹,為亢而已矣,此山谷之士,非世之人,枯槁赴淵者之所好也。語仁義忠信,恭儉推讓,為修而已矣,此平世之士,教誨之人,遊居學者之所好也。語大功,立大名,禮君臣,正上下,為治而已矣,此朝廷之士,尊主強國之人,致功并兼者之所好也。就藪澤,處閒曠,釣魚閒處,無為而已矣,此江海之士,避世之人,閒暇者之所好也。吹呴呼吸,吐故納新,熊經鳥申,為壽而已矣,此道引之士,養形之人,彭祖壽考者之所好也。

若夫不刻意而高,無仁義而修,無功名而治,無江海而閒,不道引而壽,無不忘也,無不有也,澹然無極而眾美從之,此天地之道,聖人之德也。

故曰:夫恬惔寂寞,虛無無為,此天地之平而道德之質也。

故曰:聖人休,休焉則平易矣,平易則恬惔矣。平易恬惔,則憂患不能入,邪氣不能襲,故其德全而神不虧。

故曰:聖人之生也天行,其死也物化;靜而與陰同德,動而與陽同波;不為福先,不為禍始;感而後應,迫而後動,不得已而後起。去知與故,循天之理,故無天災,無物累,無人非,無鬼責。其生若浮,其死若休;不思慮,不豫謀;光矣而不耀,信矣而不期;其寢不夢,其覺無憂;其神純粹,其魂不罷。虛無恬惔,乃合天德。

故曰:悲樂者,德之邪;喜怒者,道之過;好惡者,德之失。故心不憂樂,德之至也;一而不變,靜之至也;無所於忤,虛之至也;不與物交,惔之至也;無所於逆,粹之至也。

故曰:形勞而不休則弊,精用而不已則勞,勞則竭。水之性,不雜則清,莫動則平,鬱閉而不流,亦不能清,天德之象也。

故曰:純粹而不雜,靜一而不變,惔而無為,動而以天行,此養神之道也。

夫有干、越之劍者,柙而藏之,不敢用也,寶之至也。精神四達並流,無所不極,上際於天,下蟠於地,化育萬物,不可為象,其名為同帝。純素之道,惟神是守,守而勿失,與神為一,一之精通,合於天倫。野語有之曰:「眾人重利,廉士重名,賢人尚志,聖人貴精。」故素也者,謂其無所與雜也;純也者,謂其不虧其神也。能體純素,謂之真人。


\end{pinyinscope}