\article{天運}

\begin{pinyinscope}
天其運乎?地其處乎?日月其爭於所乎?孰主張是?孰維綱是?孰居無事推而行是?意者其有機緘而不得已邪?意者其運轉而不能自止邪?雲者為雨乎?雨者為雲乎?孰隆施是?孰居無事淫樂而勸是?風起北方,一西一東,有上彷徨,孰噓吸是?孰居無事而披拂是?敢問何故?巫咸袑曰:「來!吾語女。天有六極五常,帝王順之則治,逆之則凶。九洛之事,治成德備,監照下土,天下戴之,此謂上皇。」

商太宰蕩問仁於莊子。莊子曰:「虎狼,仁也。」曰:「何謂也?」莊子曰:「父子相親,何為不仁?」曰:「請問至仁。」莊子曰:「至仁無親。」太宰曰:「蕩聞之:無親則不愛,不愛則不孝。謂至仁不孝,可乎?」莊子曰:「不然。夫至仁尚矣,孝固不足以言之。此非過孝之言也,不及孝之言也。夫南行者至於郢,北面而不見冥山,是何也?則去之遠也。故曰:以敬孝易,以愛孝難;以愛孝易,以忘親難;忘親易,使親忘我難;使親忘我易,兼忘天下難;兼忘天下易,使天下兼忘我難。夫德遺堯、舜而不為也,利澤施於萬世,天下莫知也,豈直太息而言仁孝乎哉!夫孝悌仁義,忠信貞廉,此皆自勉以役其德者也,不足多也。故曰:至貴,國爵并焉;至富,國財并焉;至願,名譽并焉。是以道不渝。」

北門成問於黃帝曰:帝張咸池之樂於洞庭之野,吾始聞之懼,復聞之怠,卒聞之而惑,蕩蕩默默,乃不自得。」

帝曰:「女殆其然哉!吾奏之以人,徵之以天,行之以禮義,建之以太清。夫至樂者,先應之以人事,順之以天理,行之以五德,應之以自然,然後調理四時,太和萬物。四時迭起,萬物循生;一盛一衰,文武倫經;一清一濁,陰陽調和,流光其聲;蟄蟲始作,吾驚之以雷霆;其卒無尾,其始無首;一死一生,一僨一起;所常無窮,而一不可待。女故懼也。

吾又奏之以陰陽之和,燭之以日月之明;其聲能短能長,能柔能剛;變化齊一,不主故常;在谷滿谷,在阬滿阬;塗郤守神,以物為量。其聲揮綽,其名高明。是故鬼神守其幽,日月星辰行其紀。吾止之於有窮,流之於無止。予欲慮之而不能知也,望之而不能見也,逐之而不能及也,儻然立於四虛之道,倚於槁梧而吟。目知窮乎所欲見,力屈乎所欲逐,吾既不及已夫!形充空虛,乃至委蛇。汝委蛇,故怠。

吾又奏之以無怠之聲,調之以自然之命,故若混逐叢生,林樂而無形;布揮而不曳,幽昏而無聲。動於無方,居於窈冥;或謂之死,或謂之生;或謂之實,或謂之榮;行流散徙,不主常聲。世疑之,稽於聖人。聖也者,達於情而遂於命也。天機不張而五官皆備,此之謂天樂,無言而心說。故有焱氏為之頌曰:『聽之不聞其聲,視之不見其形,充滿天地,苞裏六極。』汝欲聽之而無接焉,而故惑也。

樂也者,始於懼,懼故祟;吾又次之以怠,怠故遁;卒之於惑,惑故愚;愚故道,道可載而與之俱也。」

孔子西遊於衛。顏淵問師金,曰:「以夫子之行為奚如?」師金曰:「惜乎,而夫子其窮哉!」顏淵曰:「何也?」師金曰:「夫芻狗之未陳也,盛以篋衍,巾以文繡,尸祝齊戒以將之;及其已陳也,行者踐其首脊,蘇者取而爨之而已。將復取而盛以篋衍,巾以文繡,遊居寢臥其下,彼不得夢,必且數眯焉。今而夫子,亦取先王已陳芻狗,聚弟子游居寢臥其下。故伐樹於宋,削跡於衛,窮於商、周,是非其夢邪?圍於陳、蔡之間,七日不火食,死生相與鄰,是非其眯邪?

夫水行莫如用舟,而陸行莫如用車。以舟之可行於水也而求推之於陸,則沒世不行尋常。古今非水陸與?周、魯非舟車與?今蘄行周於魯,是猶推舟於陸也,勞而無功,身必有殃。彼未知夫無方之傳,應物而不窮者也。

且子獨不見夫桔槔者乎?引之則俯,舍之則仰。彼,人之所引,非引人也,故俯仰而不得罪於人。故夫三皇、五帝之禮義法度,不矜於同而矜於治。故譬三皇、五帝之禮義法度,其猶柤梨橘柚邪!其味相反,而皆可於口。

故禮義法度者,應時而變者也。今取猨狙而衣以周公之服,彼必齕齧挽裂,盡去而後慊。觀古今之異,猶猨狙之異乎周公也。故西施病心而矉其里,其里之醜人見而美之,歸亦捧心而矉其里。其里之富人見之,堅閉門而不出;貧人見之,挈妻子而去之走。彼知矉美而不知矉之所以美。惜乎!而夫子其窮哉!」

孔子行年五十有一而不聞道,乃南之沛,見老聃。老聃曰:「子來乎?吾聞子北方之賢者也,子亦得道乎?」孔子曰:「未得也。」老子曰:「子惡乎求之哉?」曰:「吾求之於度數,五年而未得也。」老子曰:「子又惡乎求之哉?」曰:「吾求之於陰陽,十有二年而未得。」

老子曰:「然。使道而可獻,則人莫不獻之於其君;使道而可進,則人莫不進之於其親;使道而可以告人,則人莫不告其兄弟;使道而可以與人,則人莫不與其子孫。然而不可者,無佗也,中無主而不止,外無正而不行。由中出者,不受於外,聖人不出;由外入者,無主於中,聖人不隱。名,公器也,不可多取。仁義,先王之蘧廬也,止可以一宿而不可以久處,覯而多責。古之至人,假道於仁,託宿於義,以遊逍遙之虛,食於苟簡之田,立於不貸之圃。逍遙,無為也;苟簡,易養也;不貸,無出也。古者謂是采真之遊。

以富為是者,不能讓祿;以顯為是者,不能讓名;親權者,不能與人柄。操之則慄,舍之則悲,而一無所鑒,以闚其所不休者,是天之戮民也。怨、恩、取、與、諫、教、生、殺,八者,正之器也,唯循大變無所湮者,為能用之。故曰:正者,正也。其心以為不然者,天門弗開矣。」

孔子見老聃而語仁義。老聃曰:「夫播穅眯目,則天地四方易位矣;蚊虻噆膚,則通昔不寐矣。夫仁義憯然,乃憤吾心,亂莫大焉。吾子使天下無失其朴,吾子亦放風而動,總德而立矣,又奚傑然若負建鼓而求亡子者邪?夫鵠不日浴而白,烏不日黔而黑。黑白之朴,不足以為辯;名譽之觀,不足以為廣。泉涸,魚相與處於陸,相呴以溼,相濡以沫,不若相忘於江湖。」

孔子見老聃歸,三日不談。弟子問曰:「夫子見老聃,亦將何歸哉?」孔子曰:「吾乃今於是乎見龍。龍合而成體,散而成章,乘乎雲氣而養乎陰陽。予口張而不能嗋,予又何規老聃哉!」子貢曰:「然則人固有尸居而龍見,雷聲而淵默,發動如天地者乎?賜亦可得而觀乎?」遂以孔子聲見老聃。

老聃方將倨堂而應微曰:「予年運而往矣,子將何以戒我乎?」子貢曰:「夫三王、五帝之治天下不同,其係聲名一也。而先生獨以為非聖人,如何哉?」老聃曰:「小子少進!子何以謂不同?」對曰:「堯授舜,舜授禹,禹用力而湯用兵,文王順紂而不敢逆,武王逆紂而不肯順,故曰不同。」

老聃曰:「小子少進!余語汝三皇、五帝之治天下。黃帝之治天下,使民心一,民有其親死不哭而民不非也。堯之治天下,使民心親,民有為其親殺其殺而民不非也。舜之治天下,使民心競,民孕婦十月生子,子生五月而能言,不至乎孩而始誰,則人始有夭矣。禹之治天下,使民心變,人有心而兵有順,殺盜非殺,人自為種而天下耳,是以天下大駭,儒、墨皆起。其作始有倫,而今乎婦女,何言哉!余語汝:三皇、五帝之治天下,名曰治之,而亂莫甚焉。三皇之知,上悖日月之明,下睽山川之精,中墮四時之施。其知憯於蠣蠆之尾,鮮規之獸,莫得安其性命之情者,而猶自以為聖人,不可恥乎?其無恥也!」子貢蹴蹴然立不安。

孔子謂老聃曰:「丘治《詩》、《書》、《禮》、《樂》、《易》、《春秋》六經,自以為久矣,孰知其故矣,以奸者七十二君,論先王之道而明周、召之跡,一君無所鉤用。甚矣夫!人之難說也,道之難明邪!」

老子曰:「幸矣,子之不遇治世之君也!夫六經,先王之陳跡也,豈其所以跡哉!今子之所言,猶迹也。夫迹,履之所出,而迹豈履哉!夫白鶂之相視,眸子不運而風化;蟲,雄鳴於上風,雌應於下風而風化。類自為雌雄,故風化。性不可易,命不可變,時不可止,道不可壅。苟得其道,無自而不可;失焉者,無自而可。」

孔子不出三月,復見,曰:「丘得之矣。烏鵲孺,魚傅沫,細要者化,有弟而兄啼。久矣夫,丘不與化為人!不與化為人,安能化人!」老子曰:「可。丘得之矣。」


\end{pinyinscope}