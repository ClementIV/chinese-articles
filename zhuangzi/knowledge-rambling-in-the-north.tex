\article{知北遊}

\begin{pinyinscope}
知北遊於玄水之上,登隱弅之丘,而適遭無為謂焉。知謂無為謂曰:「予欲有問乎若:何思何慮則知道?何處何服則安道?何從何道則得道?」三問而無為謂不答也,非不答,不知答也。知不得問,反於白水之南,登狐闋之丘,而睹狂屈焉。知以之言也問乎狂屈。狂屈曰:「唉!予知之,將語若,中欲言而忘其所欲言。」知不得問,反於帝宮,見黃帝而問焉。黃帝曰:「無思無慮始知道,無處無服始安道,無從無道始得道。」

知問黃帝曰:「我與若知之,彼與彼不知也,其孰是邪?」黃帝曰:「彼無為謂真是也,狂屈似之,我與汝終不近也。夫知者不言,言者不知,故聖人行不言之教。道不可致,德不可至。仁可為也,義可虧也,禮相偽也。故曰:『失道而後德,失德而後仁,失仁而後義,失義而後禮。禮者,道之華而亂之首也。』故曰:『為道者日損,損之又損之,以至於無為,無為而無不為也。』今已為物也,欲復歸根,不亦難乎!其易也,其唯大人乎!生也死之徒,死也生之始,孰知其紀!人之生,氣之聚也,聚則為生,散則為死。若死生為徒,吾又何患!故萬物一也,是其所美者為神奇,其所惡者為臭腐;臭腐復化為神奇,神奇復化為臭腐。故曰:『通天下一氣耳。』聖人故貴一。」

知謂黃帝曰:「吾問無為謂,無為謂不應我,非不我應,不知應我也。吾問狂屈,狂屈中欲告我而不我告,非不我告,中欲告而忘之也。今予問乎若,若知之,奚故不近?」黃帝曰:「彼其真是也,以其不知也;此其似之也,以其忘之也;予與若終不近也,以其知之也。」

狂屈聞之,以黃帝為知言。

天地有大美而不言,四時有明法而不議,萬物有成理而不說。聖人者,原天地之美而達萬物之理。是故至人無為,大聖不作,觀於天地之謂也。

今彼神明至精,與彼百化,物已死生方圓,莫知其根也,扁然而萬物自古以固存。六合為巨,未離其內;秋豪為小,待之成體。天下莫不沈浮,終身不故;陰陽四時運行,各得其序。惛然若亡而存,油然不形而神,萬物畜而不知。此之謂本根,可以觀於天矣。

齧缺問道乎被衣,被衣曰:「若正汝形,一汝視,天和將至;攝汝知,一汝度,神將來舍。德將為汝美,道將為汝居,汝瞳焉如新出之犢而無求其故!」言未卒,齧缺睡寐。被衣大說,行歌而去之,曰:「形若槁骸,心若死灰,真其實知,不以故自持。媒媒晦晦,無心而不可與謀。彼何人哉!」

舜問乎丞曰:「道可得而有乎?」曰:「汝身非汝有也,汝何得有夫道?」舜曰:「吾身非吾有也,孰有之哉?」曰:「是天地之委形也;生非汝有,是天地之委和也;性命非汝有,是天地之委順也;孫子非汝有,是天地之委蛻也。故行不知所往,處不知所持,食不知所味。天地之強陽氣也,又胡可得而有邪?」

孔子問於老聃曰:「今日晏閒,敢問至道。」

老聃曰:「汝齊戒,疏𤅢而心,澡雪而精神,掊擊而知!夫道,窅然難言哉!將為汝言其崖略。

夫昭昭生於冥冥,有倫生於無形,精神生於道,形本生於精,而萬物以形相生,故九竅者胎生,八竅者卵生。其來無跡,其往無崖,無門無房,四達之皇皇也。邀於此者,四肢彊,思慮恂達,耳目聰明,其用心不勞,其應物無方。天不得不高,地不得不廣,日月不得不行,萬物不得不昌,此其道與!

且夫博之不必知,辯之不必慧,聖人以斷之矣。若夫益之而不加益,損之而不加損者,聖人之所保也。淵淵乎其若海,魏魏乎其終則復始也,運量萬物而不匱,則君子之道,彼其外與!萬物皆往資焉而不匱,此其道與!

中國有人焉,非陰非陽,處於天地之閒,直且為人,將反於宗。自本觀之,生者,暗醷物也。雖有壽夭,相去幾何?須臾之說也。奚足以為堯、桀之是非?

果蓏有理,人倫雖難,所以相齒。聖人遭之而不違,過之而不守。調而應之,德也;偶而應之,道也。帝之所興,王之所起也。

人生天地之間,若白駒之過郤,忽然而已。注然勃然,莫不出焉;油然漻然,莫不入焉。已化而生,又化而死,生物哀之,人類悲之。解其天弢,墮其天𧙍,紛乎宛乎,魂魄將往,乃身從之,乃大歸乎!

不形之形,形之不形,是人之所同知也,非將至之所務也,此眾人之所同論也。彼至則不論,論則不至。明見無值,辯不若默。道不可聞,聞不若塞。此之謂大得。」

東郭子問於莊子曰:「所謂道,惡乎在?」莊子曰:「無所不在。」東郭子曰:「期而後可。」莊子曰:「在螻蟻。」曰:「何其下邪?」曰:「在稊稗。」曰:「何其愈下邪?」曰:「在瓦甓。」曰:「何其愈甚邪?」曰:「在屎溺。」東郭子不應。

莊子曰:「夫子之問也,固不足質。正獲之問於監市履狶也,每下愈況。汝唯莫必,無乎逃物。至道若是,大言亦然。周、遍、咸三者,異名同實,其指一也。嘗相與游乎無何有之宮,同合而論,無所終窮乎!嘗相與無為乎!澹而靜乎!漠而清乎!調而閒乎!寥已吾志,無往焉而不知其所至;去而來而不知其所止,吾已往來焉而不知其所終;彷徨乎馮閎,大知入焉而不知其所窮。物物者與物無際,而物有際者,所謂物際者也;不際之際,際之不際者也。謂盈虛衰殺,彼為盈虛非盈虛,彼為衰殺非衰殺,彼為本末非本末,彼為積散非積散也。」

婀荷甘與神農同學於老龍吉。神農隱几闔戶晝瞑,婀荷甘日中奓戶而入,曰:「老龍死矣!」神農隱几擁杖而起,嚗然放杖而笑,曰:「天知予僻陋慢訑,故棄予而死。已矣!夫子無所發予之狂言而死矣夫!」

弇堈弔聞之,曰:「夫體道者,天下之君子所繫焉。今於道,秋豪之端,萬分未得處一焉,而猶知藏其狂言而死,又況夫體道者乎!視之無形,聽之無聲,於人之論者,謂之冥冥,所以論道,而非道也。」

於是泰清問乎無窮曰:「子知道乎?」無窮曰:「吾不知。」又問乎無為。無為曰:「吾知道。」曰:「子之知道,亦有數乎?」曰:「有。」曰:「其數若何?」無為曰:「吾知道之可以貴,可以賤,可以約,可以散。此吾所以知道之數也。」泰清以之言也問乎無始,曰:「若是,則無窮之弗知,與無為之知,孰是而孰非乎?」無始曰:「不知深矣,知之淺矣;弗知內矣,知之外矣。」於是泰清中而歎曰:「弗知乃知乎!知乃不知乎!孰知不知之知?」無始曰:「道不可聞,聞而非也;道不可見,見而非也;道不可言,言而非也。知形形之不形乎?道不當名。」

無始曰:「有問道而應之者,不知道也。雖問道者,亦未聞道。道無問,問無應。無問問之,是問窮也;無應應之,是無內也。以無內待問窮,若是者,外不觀乎宇宙,內不知乎太初,是以不過乎崑崙,不遊乎太虛。」

光曜問乎無有曰:「夫子有乎,其無有乎?」光曜不得問,而孰視其狀貌,窅然空然,終日視之而不見,聽之而不聞,搏之而不得也。光曜曰:「至矣!其孰能至此乎!予能有無矣,而未能無無也,及為無有矣,何從至此哉!」

大馬之捶鉤者,年八十矣,而不失豪芒。大馬曰:「子巧與?有道與?」曰:「臣有守也。臣之年二十而好捶鉤,於物無視也,非鉤無察也。是用之者,假不用者也以長得其用,而況乎無不用者乎!物孰不資焉?」

冉求問於仲尼曰:「未有天地可知邪?」仲尼曰:「可。古猶今也。」冉求失問而退,明日復見,曰:「昔者吾問『未有天地可知乎』,夫子曰:『可。古猶今也。』昔者吾昭然,今日吾昧然,敢問何謂也?」仲尼曰:「昔之昭然也,神者先受之;今之昧然也,且又為不神者求邪?無古無今,無始無終。未有子孫而有子孫,可乎?」冉求未對。仲尼曰:「已矣,末應矣!不以生生死,不以死死生。死生有待邪?皆有所一體。有先天地生者物邪?物物者非物。物出不得先物也,猶其有物也。猶其有物也,無已。聖人之愛人也終無已者,亦乃取於是者也。」

顏淵問乎仲尼曰:「回嘗聞諸夫子曰:『無有所將,無有所迎。』回敢問其遊。」仲尼曰:「古之人,外化而內不化;今之人,內化而外不化。與物化者,一不化者也。安化安不化,安與之相靡,必與之莫多。狶韋氏之囿,黃帝之圃,有虞氏之宮,湯、武之室。君子之人,若儒、墨者師,故以是非相𩐋也,而況今之人乎!聖人處物不傷物。不傷物者,物亦不能傷也。唯無所傷者,為能與人相將、迎。山林與!皋壤與!使我欣欣然而樂與!樂未畢也,哀又繼之。哀樂之來,吾不能禦,其去弗能止。悲夫!世人直為物逆旅耳!夫知遇而不知所不遇,知能能而不能所不能。無知無能者,固人之所不免也。夫務免乎人之所不免者,豈不亦悲哉!至言去言,至為去為。齊知之所知,則淺矣。」


\end{pinyinscope}