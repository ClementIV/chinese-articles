\article{寓言}

\begin{pinyinscope}
寓言十九,重言十七,卮言日出,和以天倪。寓言十九,藉外論之。親父不為其子媒。親父譽之,不若非其父者也;非吾罪也,人之罪也。與己同則應,不與己同則反,同於己為是之,異於己為非之。重言十七,所以已言也,是為耆艾。年先矣,而無經緯本末以期年耆者,是非先也。人而無以先人,無人道也;人而無人道,是之謂陳人。

卮言日出,和以天倪,因以曼衍,所以窮年。不言則齊,齊與言不齊,言與齊不齊也,故曰無言。言無言,終身言,未嘗言;終身不言,未嘗不言。有自也而可,有自也而不可;有自也而然,有自也而不然。惡乎然?然於然。惡乎不然?不然於不然。惡乎可?可於可。惡乎不可?不可於不可。物固有所然,物固有所可,無物不然,無物不可。非卮言日出,和以天倪,孰得其久!萬物皆種也,以不同形相禪,始卒若環,莫得其倫,是謂天均。天均者,天倪也。

莊子謂惠子曰:「孔子行年六十而六十化,始時所是,卒而非之,未知今之所謂是之非五十九年非也。」惠子曰:「孔子勤志服知也。」莊子曰:「孔子謝之矣,而其未之嘗言。孔子云:『夫受才乎大本,復靈以生。』鳴而當律,言而當法,利義陳乎前,而好惡是非直服人之口而已矣。使人乃以心服而不敢蘁立,定天下之定。已乎已乎!吾且不得及彼乎!」

曾子再仕而心再化,曰:「吾及親仕,三釜而心樂;後仕,三千鍾而不洎,吾心悲。」弟子問於仲尼曰:「若參者,可謂無所縣其罪乎?」曰:「既已縣矣。夫無所縣者,可以有哀乎?彼視三釜、三千鍾,如觀雀蚊虻相過乎前也。」

顏成子游謂東郭子綦曰:自吾聞子之言,一年而野,二年而從,三年而通,四年而物,五年而來,六年而鬼入,七年而天成,八年而不知死、不知生,九年而大妙。

生有為,死也。勸公:以其死也,有自也;而生陽也,無自也。而果然乎?惡乎其所適?惡乎其所不適?天有曆數,地有人據,吾惡乎求之?莫知其所終,若之何其無命也?莫知其所始,若之何其有命也?有以相應也,若之何其無鬼邪?無以相應也,若之何其有鬼邪?」

眾罔兩問於景曰:「若向也俯而今也仰,向也括而今被髮,向也坐而今也起,向也行而今也止,何也?」景曰:「搜搜也,奚稍問也?予有而不知其所以。予,蜩甲也,蛇蛻也,似之而非也。火與日,吾屯也;陰與夜,吾代也。彼,吾所以有待邪?而況乎以有待者乎!彼來則我與之來,彼往則我與之往,彼強陽則我與之強陽。強陽者,又何以有問乎!」

陽子居南之沛,老聃西遊於秦,邀於郊,至於梁而遇老子。老子中道仰天而歎曰:「始以汝為可教,今不可也。」陽子居不答。至舍,進盥漱巾櫛,脫屨戶外,膝行而前曰:「向者弟子欲請夫子,夫子行不閒,是以不敢。今閒矣,請問其過。」老子曰:「而睢睢盱盱,而誰與居?大白若辱,盛德若不足。」陽子居蹴然變容曰:「敬聞命矣。」其往也,舍者迎將其家,公執席,妻執巾櫛,舍者避席,煬者避灶。其反也,舍者與之爭席矣。


\end{pinyinscope}