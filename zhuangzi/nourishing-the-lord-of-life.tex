\article{養生主}

\begin{pinyinscope}
吾生也有涯,而知也无涯。以有涯隨无涯,殆已;已而為知者,殆而已矣。為善无近名,為惡无近刑。緣督以為經,可以保身,可以全生,可以養親,可以盡年。

庖丁為文惠君解牛,手之所觸,肩之所倚,足之所履,膝之所踦,砉然嚮然,奏刀騞然,莫不中音。合於《桑林》之舞,乃中《經首》之會。文惠君曰:「譆!善哉!技蓋至此乎?」庖丁釋刀對曰:「臣之所好者道也,進乎技矣。始臣之解牛之時,所見无非牛者。三年之後,未嘗見全牛也。方今之時,臣以神遇,而不以目視,官知止而神欲行。依乎天理,批大郤,導大窾,因其固然。技經肯綮之未嘗,而況大軱乎!良庖歲更刀,割也;族庖月更刀,折也。今臣之刀十九年矣,所解數千牛矣,而刀刃若新發於硎。彼節者有間,而刀刃者无厚,以无厚入有間,恢恢乎其於遊刃必有餘地矣,是以十九年而刀刃若新發於硎。雖然,每至於族,吾見其難為,怵然為戒,視為止,行為遲。動刀甚微,謋然已解,如土委地。提刀而立,為之四顧,為之躊躇滿志,善刀而藏之。」文惠君曰:「善哉!吾聞庖丁之言,得養生焉。」

公文軒見右師而驚曰:「是何人也?惡乎介也?天與,其人與?」曰:「天也,非人也。天之生是使獨也,人之貌有與也。以是知其天也,非人也。」

澤雉十步一啄,百步一飲,不蘄畜乎樊中。神雖王,不善也。

老聃死,秦失弔之,三號而出。弟子曰:「非夫子之友邪?」曰:「然。」「然則弔焉若此,可乎?」曰:「然。始也,吾以為其人也,而今非也。向吾入而弔焉,有老者哭之,如哭其子;少者哭之,如哭其母。彼其所以會之,必有不蘄言而言,不蘄哭而哭者。是遁天倍情,忘其所受,古者謂之遁天之刑。適來,夫子時也;適去,夫子順也。安時而處順,哀樂不能入也,古者謂是帝之縣解。」

指窮於為薪,火傳也,不知其盡也。


\end{pinyinscope}