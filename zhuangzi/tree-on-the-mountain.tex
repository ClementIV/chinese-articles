\article{山木}

\begin{pinyinscope}
莊子行於山中,見大木,枝葉盛茂,伐木者止其旁而不取也。問其故。曰:「無所可用。」莊子曰:「此木以不材得終其天年。」夫子出於山,舍於故人之家。故人喜,命豎子殺鴈而烹之。豎子請曰:「其一能鳴,其一不能鳴,請奚殺?」主人曰:「殺不能鳴者。」

明日,弟子問於莊子曰:「昨日山中之木,以不材得終其天年;今主人之鴈,以不材死。先生將何處?」莊子笑曰:「周將處夫材與不材之間。材與不材之間,似之而非也,故未免乎累。若夫乘道德而浮游則不然。無譽無訾,一龍一蛇,與時俱化,而無肯專為;一上一下,以和為量,浮游乎萬物之祖;物物而不物於物,則胡可得而累邪!此黃帝、神農之法則也。若夫萬物之情,人倫之傳,則不然。合則離,成則毀,廉則挫,尊則議,有為則虧,賢則謀,不肖則欺,胡可得而必乎哉?悲夫!弟子志之,其唯道德之鄉乎!」

市南宜僚見魯侯,魯侯有憂色。市南子曰:「君有憂色,何也?」魯侯曰:「吾學先王之道,修先君之業,吾敬鬼尊賢,親而行之,無須臾離居,然不免於患,吾是以憂。」

市南子曰:「君之除患之術淺矣。夫豐狐文豹,棲於山林,伏於巖穴,靜也;夜行晝居,戒也;雖飢渴隱約,猶旦胥疏於江湖之上而求食焉,定也。然且不免於罔羅機辟之患,是何罪之有哉?其皮為之災也。今魯國獨非君之皮邪?吾願君刳形去皮,洒心去欲,而遊於無人之野。南越有邑焉,名為建德之國。其民愚而朴,少私而寡欲;知作而不知藏,與而不求其報;不知義之所適,不知禮之所將;猖狂妄行,乃蹈乎大方;其生可樂,其死可葬。吾願君去國捐俗,與道相輔而行。」

君曰:「彼其道遠而險,又有江山,我無舟車,奈何?」市南子曰:「君無形倨,無留居,以為舟車。」

君曰:「彼其道幽遠而無人,吾誰與為鄰?吾無糧,我無食,安得而至焉?」市南子曰:「少君之費,寡君之欲,雖無糧而乃足。君其涉於江而浮於海,望之而不見其崖,愈往而不知其所窮。送君者皆自崖而反,君自此遠矣。故有人者累,見有於人者憂。故堯非有人,非見有於人也。吾願去君之累,除君之憂,而獨與道遊於大莫之國。方舟而濟於河,有虛船來觸舟,雖有惼心之人不怒;有一人在其上,則呼張歙之;一呼而不聞,再呼而不聞,於是三呼邪,則必以惡聲隨之。向也不怒而今也怒,向也虛而今也實。人能虛己以遊世,其孰能害之!」

北宮奢為衛靈公賦斂以為鐘,為壇乎國門之外,三月而成上下之縣。王子慶忌見而問焉,曰:「子何術之設?」奢曰:「一之間,無敢設也。奢聞之:『既彫既琢,復歸於朴。』侗乎其無識,儻乎其怠疑;萃乎芒乎,其送往而迎來;來者勿禁,往者勿止;從其彊梁,隨其曲傅,因其自窮。故朝夕賦斂而毫毛不挫,而況有大塗者乎!」

孔子圍於陳、蔡之間,七日不火食。大公任往弔之,曰:「子幾死乎?」曰:「然。」「子惡死乎?」曰:「然。」

任曰:「予嘗言不死之道。東海有鳥焉,其名曰意怠。其為鳥也,翂翂翐翐,而似無能;引援而飛,迫脅而棲;進不敢為前,退不敢為後;食不敢先嘗,必取其緒。是故其行列不斥,而外人卒不得害,是以免於患。直木先伐,甘井先竭。子其意者飾知以驚愚,修身以明汙,昭昭乎若揭日月而行,故不免也。昔吾聞之大成之人曰:『自伐者無功,功成者墮,名成者虧。』孰能去功與名而還與眾人!道流而不明居,得行而不名處;純純常常,乃比於狂;削跡捐勢,不為功名。是故無責於人,人亦無責焉。至人不聞,子何喜哉?」

孔子曰:「善哉!」辭其交遊,去其弟子,逃於大澤;衣裘褐,食杼栗;入獸不亂群,入鳥不亂行。鳥獸不惡,而況人乎!

孔子問子桑雽曰:「吾再逐於魯,伐樹於宋,削跡於衛,窮於商、周,圍於陳、蔡之間。吾犯此數患,親交益疏,徒友益散,何與?」

子桑雽曰:「子獨不聞假人之亡與?林回棄千金之璧,負赤子而趨。或曰:『為其布與?赤子之布寡矣。為其累與?赤子之累多矣。棄千金之璧,負赤子而趨,何也?』林回曰:『彼以利合,此以天屬也。』夫以利合者,迫窮禍患害相棄也;以天屬者,迫窮禍患害相收也。夫相收之與相棄亦遠矣。且君子之交淡若水,小人之交甘若醴;君子淡以親,小人甘以絕。彼無故以合者,則無故以離。」

孔子曰:「敬聞命矣。」徐行翔佯而歸,絕學捐書,弟子無挹於前,其愛益加進。

異日,桑雽又曰:「舜之將死,真泠禹曰:『汝戒之哉!形莫若緣,情莫若率。緣則不離,率則不勞;不離不勞,則不求文以待形;不求文以待形,固不待物。』」

莊子衣大布而補之,正緳係履而過魏王。魏王曰:「何先生之憊邪?」莊子曰:「貧也,非憊也。士有道德不能行,憊也。衣弊履穿,貧也,非憊也,此所謂非遭時也。王獨不見夫騰猿乎?其得柟、梓、豫、章也,攬蔓其枝,而王長其間,雖羿、蓬蒙不能眄睨也。及其得柘、棘、枳、枸之閒也,危行側視,振動悼慄,此筋骨非有加急而不柔也,處勢不便,未足以逞其能也。今處昏上亂相之間,而欲無憊,奚可得邪?此比干之見剖心,徵也夫!」

孔子窮於陳、蔡之間,七日不火食,左據槁木,右擊槁枝,而歌猋氏之風,有其具而無其數,有其聲而無宮角,木聲與人聲,犁然有當於人心。

顏回端拱還目而窺之。仲尼恐其廣己而造大也,愛己而造哀也,曰:「回!無受天損易,無受人益難。無始而非卒也,人與天一也。夫今之歌者其誰乎?」

回曰:「敢問無受天損易。」仲尼曰:「飢溺寒暑,窮桎不行,天地之行也,運物之泄也,言與之偕逝之謂也。為人臣者,不敢去之。執臣之道猶若是,而況乎所以待天乎!」

「何謂無受人益難?」仲尼曰:「始用四達,爵祿並至而不窮,物之所利,乃非己也,吾命有在外者也。君子不為盜,賢人不為竊。吾若取之,何哉?故曰:鳥莫知於鷾鴯,目之所不宜處,不給視,雖落其實,棄之而走。其畏人也,而襲諸人間,社稷存焉爾。」

「何謂無始而非卒?」仲尼曰:「化其萬物而不知其禪之者,焉知其所終?焉知其所始?正而待之而已耳。」

「何謂天與人一邪?」仲尼曰:「有人,天也;有天,亦天也。人之不能有天,性也,聖人晏然體逝而終矣。」

莊周遊乎雕陵之樊,睹一異鵲自南方來者,翼廣七尺,目大運寸,感周之顙而集於栗林。莊周曰:「此何鳥哉?翼殷不逝,目大不覩。」蹇裳躩步,執彈而留之。睹一蟬方得美蔭而忘其身;螳蜋執翳而搏之,見得而忘其形;異鵲從而利之,見利而忘其真。莊周怵然曰:「噫!物固相累,二類相召也。」捐彈而反走,虞人逐而誶之。

莊周反入,三月不庭。藺且從而問之:「夫子何為頃間甚不庭乎?」莊周曰:「吾守形而忘身,觀於濁水而迷於清淵。且吾聞諸夫子曰:『入其俗,從其俗。』今吾遊於雕陵而忘吾身,異鵲感吾顙,遊於栗林而忘真,栗林虞人以吾為戮,吾所以不庭也。」

陽子之宋,宿於逆旅。逆旅有妾二人,其一人美,其一人惡,惡者貴而美者賤。陽子問其故,逆旅小子對曰:「其美者自美,吾不知其美也;其惡者自惡,吾不知其惡也。」陽子曰:「弟子記之!行賢而去自賢之行,安往而不愛哉?」


\end{pinyinscope}