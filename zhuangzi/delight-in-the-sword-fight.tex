\article{說劍}

\begin{pinyinscope}
昔趙文王喜劍,劍士夾門而客三千餘人,日夜相擊於前,死傷者歲百餘人,好之不厭。如是三年,國衰,諸侯謀之。太子悝患之,募左右曰:「孰能說王之意止劍士者,賜之千金。」左右曰:「莊子當能。」

太子乃使人以千金奉莊子。莊子弗受,與使者俱往見太子曰:「太子何以教周,賜周千金?」太子曰:「聞夫子明聖,謹奉千金以幣從者。夫子弗受,悝尚何敢言!」莊子曰:「聞太子所欲用周者,欲絕王之喜好也。使臣上說大王而逆王意,下不當太子,則身刑而死,周尚安所事金乎!使臣上說大王,下當太子,趙國何求而不得也?」太子曰:「然。吾王所見,唯劍士也。」莊子曰:「諾。周善為劍。」太子曰:「然吾王所見劍士,皆蓬頭、突鬢、垂冠,曼胡之纓,短後之衣,嗔目而語難,王乃說之。今夫子必儒服而見王,事必大逆。」莊子曰:「請治劍服。」治劍服三日,乃見太子。太子乃與見王,王脫白刃待之。

莊子入殿門不趨,見王不拜。王曰:「子欲何以教寡人,使太子先?」曰:「臣聞大王喜劍,故以劍見王。」王曰:「子之劍何能禁制?」曰:「臣之劍,十步一人,千里不留行。」王大悅之,曰:「天下無敵矣。」莊子曰:「夫為劍者,示之以虛,開之以利,後之以發,先之以至。願得試之。」王曰:「夫子休就舍,待命令設戲請夫子。」王乃校劍士七日,死傷者六十餘人,得五六人,使奉劍於殿下,乃召莊子。王曰:「今日試使士敦劍。」莊子曰:「望之久矣。」王曰:「夫子所御杖,長短何如?」曰:「臣之所奉皆可。然臣有三劍,唯王所用,請先言而後試。」

王曰:「願聞三劍。」曰:「有天子劍,有諸侯劍,有庶人劍。」王曰:「天子之劍何如?」曰:「天子之劍,以燕谿、石城為鋒,齊、岱為鍔,晉、魏為脊,周、宋為鐔,韓、魏為夾,包以四夷,裹以四時,繞以渤海,帶以常山,制以五行,論以刑德,開以陰陽,持以春夏,行以秋冬。此劍直之無前,舉之無上,案之無下,運之無旁,上決浮雲,下絕地紀。此劍一用,匡諸侯,天下服矣。此天子之劍也。」

文王芒然自失,曰:「諸侯之劍何如?」曰:「諸侯之劍,以知勇士為鋒,以清廉士為鍔,以賢良士為脊,以忠聖士為鐔,以豪桀士為夾。此劍值之亦無前,舉之亦無上,案之亦無下,運之亦無旁,上法圓天以順三光,下法方地以順四時,中和民意以安四鄉。此劍一用,如雷霆之震也,四封之內,無不賓服而聽從君命者矣。此諸侯之劍也。」

王曰:「庶人之劍何如?」曰:「庶人之劍,蓬頭、突鬢、垂冠,曼胡之纓,短後之衣,瞋目而語難,相擊於前,上斬頸領,下決肝肺。此庶人之劍,無異於鬥雞,一旦命已絕矣,無所用於國事。今大王有天子之位,而好庶人之劍,臣竊為大王薄之。」

王乃牽而上殿,宰人上食,王三環之。莊子曰:「大王安坐定氣,劍事已畢奏矣。」於是文王不出宮三月,劍士皆服斃其處也。


\end{pinyinscope}