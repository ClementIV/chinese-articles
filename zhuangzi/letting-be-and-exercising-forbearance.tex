\article{在宥}

\begin{pinyinscope}
聞在宥天下,不聞治天下也。在之也者,恐天下之淫其性也;宥之也者,恐天下之遷其德也。天下不淫其性,不遷其德,有治天下者哉!昔堯之治天下也,使天下欣欣焉人樂其性,是不恬也;桀之治天下也,使天下瘁瘁焉人苦其性,是不愉也。夫不恬不愉,非德也。非德也而可長久者,天下無之。人大喜邪,毗於陽。大怒邪,毗於陰。陰陽並毗,四時不至,寒暑之和不成,其反傷人之形乎!使人喜怒失位,居處無常,思慮不自得,中道不成章,於是乎天下始喬詰、卓鷙,而後有盜跖、曾、史之行。故舉天下以賞其善者不足,舉天下以罰其惡者不給,故天下之大不足以賞罰。自三代以下者,匈匈焉終以賞罰為事,彼何暇安其性命之情哉!而且說明邪,是淫於色也;說聰邪,是淫於聲也;說仁邪,是亂於德也;說義邪,是悖於理也;說禮邪,是相於技也;說樂邪,是相於淫也;說聖邪,是相於藝也;說知邪,是相於疵也。天下將安其性命之情,之八者,存可也;亡可也;天下將不安其性命之情,之八者,乃始臠卷、獊囊而亂天下也。而天下乃始尊之惜之,甚矣天下之惑也!豈直過也而去之邪!乃齊戒以言之,跪坐以進之,鼓歌以儛之,吾若是何哉!故君子不得已而臨邪天下,莫若無為。無為也,而後安其性命之情。故貴以身於為天下,則可以託天下;愛以身於為天下,則可以寄天下。故君子苟能無解其五藏,無擢其聰明,尸居而龍見,淵默而雷聲,神動而天隨,從容無為而萬物炊累焉。吾又何暇治天下哉!

崔瞿問於老聃曰:「不治天下,安藏人心?」老聃曰:「汝慎無攖人心。人心排下而進上,上下囚殺,淖約柔乎剛強。廉劌彫琢,其熱焦火,其寒凝冰。其疾俛仰之間,而再撫四海之外,其居也淵而靜,其動也縣而天。僨驕而不可係者,其唯人心乎!昔者黃帝始以仁義攖人之心,堯、舜於是乎股無胈,脛無毛,以養天下之形,愁其五藏以為仁義,矜其血氣以規法度。然猶有不勝也。堯於是放讙兜於崇山,投三苗於三峗,流共工於幽都,此不勝天下也夫!施及三王而天下大駭矣。下有桀、跖,上有曾、史,而儒、墨畢起。於是乎喜怒相疑,愚知相欺,善否相非,誕信相譏,而天下衰矣;大德不同,而性命爛漫矣;天下好知,而百姓求竭矣。於是乎釿鋸制焉,繩墨殺焉,椎鑿決焉。天下脊脊大亂,罪在攖人心。故賢者伏處大山嵁巖之下,而萬乘之君憂慄乎廟堂之上。今世殊死者相枕也,桁楊者相推也,刑戮者相望也,而儒、墨乃始離跂攘臂乎桎梏之間。意!甚矣哉!其無愧而不知恥也甚矣!吾未知聖知之不為桁楊椄槢也,仁義之不為桎梏、鑿枘也,焉知曾、史之不為桀、跖嚆矢也!故曰:『絕聖棄知而天下大治。』」

黃帝立為天子十九年,令行天下,聞廣成子在於空同之上,故往見之,曰:「我聞吾子達於至道,敢問至道之精。吾欲取天地之精,以佐五穀,以養民人;吾又欲官陰陽,以遂群生。為之奈何?」廣成子曰:「而所欲問者,物之質也;而所欲官者,物之殘也。自而治天下,雲氣不待族而雨,草木不待黃而落,日月之光益以荒矣。而佞人之心翦翦者,又奚足以語至道!」黃帝退,捐天下,築特室,席白茅,閒居三月,復往邀之。廣成子南首而臥,黃帝順下風膝行而進,再拜稽首而問曰:「聞吾子達於至道,敢問治身奈何而可以長久?」廣成子蹶然而起,曰:「善哉問乎!來!吾語女至道。至道之精,窈窈冥冥;至道之極,昏昏默默。無視無聽,抱神以靜,形將自正。必靜必清,無勞女形,無搖女精,乃可以長生。目無所見,耳無所聞,心無所知,女神將守形,形乃長生。慎女內,閉女外,多知為敗。我為女遂於大明之上矣,至彼至陽之原也;為女入於窈冥之門矣,至彼至陰之原也。天地有官,陰陽有藏,慎守女身,物將自壯。我守其一,以處其和,故我修身千二百歲矣,吾形未嘗衰。」黃帝再拜稽首曰:「廣成子之謂天矣!」廣成子曰:「來!吾語女。彼其物無窮,而人皆以為有終;彼其物無測,而人皆以為有極。得吾道者,上為皇而下為王;失吾道者,上見光而下為土。今夫百昌,皆生於土而反於土,故余將去女,入無窮之門,以遊無極之野。吾與日月參光,吾與天地為常。當我,緡乎!遠我,昏乎!人其盡死,而我獨存乎!」

雲將東遊,過扶搖之枝,而適遭鴻蒙。鴻蒙方將拊髀雀躍而遊。雲將見之,倘然止,贄然立,曰:「叟何人邪?叟何為此?」鴻蒙拊髀雀躍不輟,對雲將曰:「遊。」雲將曰:「朕願有問也。」鴻蒙仰而視雲將曰:「吁!」雲將曰:「天氣不合,地氣鬱結,六氣不調,四時不節。今我願合六氣之精,以育群生,為之奈何?」鴻蒙拊髀雀躍掉頭曰:「吾弗知,吾弗知。」雲將不得問。又三年,東遊,過有宋之野,而適遭鴻蒙。雲將大喜,行趨而進曰:「天忘朕邪?天忘朕邪?」再拜稽首,願聞於鴻蒙。鴻蒙曰:「浮游不知所求,猖狂不知所往,遊者鞅掌,以觀無妄,朕又何知!」雲將曰:「朕也自以為猖狂,而百姓隨予所往;朕也不得已於民,今則民之放也。願聞一言。」鴻蒙曰:「亂天之經,逆物之情,玄天弗成;解獸之群,而鳥皆夜鳴;災及草木,禍及止蟲。意!治人之過也!」雲將曰:「然則吾奈何?」鴻蒙曰:「意!毒哉!僊僊乎歸矣!」雲將曰:「吾遇天難,願聞一言。」鴻蒙曰:「意!心養。汝徒處無為,而物自化。墮爾形體,吐爾聰明;倫與物忘,大同乎涬溟;解心釋神,莫然無魂。萬物云云,各復其根,各復其根而不知。渾渾沌沌,終身不離;若彼知之,乃是離之。無問其名,無闚其情,物故自生。」雲將曰:「天降朕以德,示朕以默,躬身求之,乃今也得。」再拜稽首,起辭而行。

世俗之人,皆喜人之同乎己,而惡人之異於己也。同於己而欲之、異於己而不欲者,以出乎眾為心也。夫以出於眾為心者,曷嘗出乎眾哉!因眾以寧所聞,不如眾技眾矣。而欲為人之國者,此攬乎三王之利,而不見其患者也。此以人之國僥倖也,幾何僥倖而不喪人之國乎!其存人之國也,無萬分之一;而喪人之國也,一不成而萬有餘喪矣。悲夫!有土者之不知也!

夫有土者,有大物也。有大物者,不可以物物;而不物,故能物物。明乎物物者之非物也,豈獨治天下百姓而已哉!出入六合,遊乎九州,獨往獨來,是謂獨有。獨有之人,是謂至貴。

大人之教,若形之於影,聲之於響。有問而應之,盡其所懷,為天下配。處乎無響,行乎無方。挈汝適復之撓撓,以遊無端,出入無旁,與日無始,頌論形軀,合乎大同,大同而無己。無己,惡乎得有有!睹有者,昔之君子;睹無者,天地之友。

賤而不可不任者,物也;卑而不可不因者,民也;匿而不可不為者,事也;麤而不可不陳者,法也;遠而不可不居者,義也;親而不可不廣者,仁也;節而不可不積者,禮也;中而不可不高者,德也;一而不可不易者,道也;神而不可不為者,天也。故聖人觀於天而不助,成於德而不累,出於道而不謀,會於仁而不恃,薄於義而不積,應於禮而不諱,接於事而不辭,齊於法而不亂,恃於民而不輕,因於物而不去。物者莫足為也,而不可不為。不明於天者,不純於德;不通於道者,無自而可。不明於道者,悲夫!

何謂道?有天道,有人道。無為而尊者,天道也;有為而累者,人道也。主者,天道也;臣者,人道也。天道之與人道也,相去遠矣,不可不察也。


\end{pinyinscope}