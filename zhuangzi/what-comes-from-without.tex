\article{外物}

\begin{pinyinscope}
外物不可必,故龍逢誅,比干戮,箕子狂,惡來死,桀、紂亡。人主莫不欲其臣之忠,而忠未必信,故伍員流於江,萇弘死於蜀,藏其血三年,化而為碧。人親莫不欲其子之孝,而孝未必愛,故孝己憂而曾參悲。

木與木相摩則然,金與火相守則流。陰陽錯行,則天地大絯,於是乎有雷有霆,水中有火,乃焚大槐。有甚憂兩陷而無所逃,螴蜳不得成,心若縣於天地之間,慰睯沈屯,利害相摩,生火甚多,眾人焚和。月固不勝火,於是乎有僓然而道盡。

莊周家貧,故往貸粟於監河侯。監河侯曰:「諾。我將得邑金,將貸子三百金,可乎?」莊周忿然作色曰:「周昨來,有中道而呼者。周顧視車轍中,有鮒魚焉。周問之曰:『鮒魚來!子何為者邪?』對曰:『我,東海之波臣也。君豈有斗升之水而活我哉?』周曰:『諾。我且南遊吳、越之王,激西江之水而迎子,可乎?』鮒魚忿然作色曰:『吾失我常與,我無所處。吾得斗升之水然活耳,君乃言此,曾不如早索我於枯魚之肆!』」

任公子為大鉤巨緇,五十犗以為餌,蹲乎會稽,投竿東海,旦旦而釣,期年不得魚。已而大魚食之,牽巨鉤錎沒而下,騖揚而奮鬐,白波若山,海水震蕩,聲侔鬼神,憚赫千里。任公子得若魚,離而腊之,自制河以東,蒼梧以北,莫不厭若魚者。

已而後世輇才諷說之徒,皆驚而相告也。夫揭竿累,趣灌瀆,守鯢鮒,其於得大魚難矣;飾小說以干縣令,其於大達亦遠矣。是以未嘗聞任氏之風俗,其不可與經於世亦遠矣。

儒以《詩》、《禮》發冢。大儒臚傳曰:「東方作矣,事之何若?」小儒曰:「未解裙襦,口中有珠。《詩》固有之曰:『青青之麥,生於陵陂。生不布施,死何含珠為?』接其鬢,壓其顪,儒以金椎控其頤,徐別其頰,無傷口中珠!」

老萊子之弟子出薪,遇仲尼,反以告曰:「有人於彼,修上而趨下,末僂而後耳,視若營四海,不知其誰氏之子。」老萊子曰:「是丘也,召而來!」仲尼至。曰:「丘!去汝躬矜與汝容知,斯為君子矣。」仲尼揖而退,蹙然改容而問曰:「業可得進乎?」老萊子曰:「夫不忍一世之傷,而驁萬世之患,抑固窶邪?亡其略弗及邪?惠以歡為驁,終身之醜,中民之行進焉耳,相引以名,相結以隱。與其譽堯而非桀,不如兩忘而閉其所譽。反無非傷也,動無非邪也。聖人躊躇以興事,以每成功。奈何哉其載焉終矜爾!」

宋元君夜半而夢人被髮闚阿門,曰:「予自宰路之淵,予為清江使河伯之所,漁者余且得予。」元君覺,使人占之,曰:「此神龜也。」君曰:「漁者有余且乎?」左右曰:「有。」君曰:「令余且會朝。」明日,余且朝。君曰:「漁何得?」對曰:「且之網,得白龜焉,其圓五尺。」君曰:「獻若之龜。」龜至,君再欲殺之,再欲活之,心疑,卜之,曰:「殺龜以卜,吉。」乃刳龜,七十二鑽而無遺筴。

仲尼曰:「神龜能見夢於元君而不能避余且之網;知能七十二鑽而無遺筴,不能避刳腸之患。如是,則知有所困,神有所不及也。雖有至知,萬人謀之。魚不畏網而畏鵜鶘。去小知而大知明,去善而自善矣。」嬰兒生無石師而能言,與能言者處也。

惠子謂莊子曰:「子言無用。」莊子曰:「知無用而始可與言用矣。夫地非不廣且大也,人之所用容足耳。然則廁足而墊之,致黃泉,人尚有用乎?」惠子曰:「無用。」莊子曰:「然則無用之為用也亦明矣。」

莊子曰:「人有能遊,且得不遊乎?人而不能遊,且得遊乎?夫流遁之志,決絕之行,噫!其非至知厚德之任與!覆墜而不反,火馳而不顧,雖相與為君臣,時也,易世而無以相賤。故曰:至人不留行焉。夫尊古而卑今,學者之流也。且以豨韋氏之流觀今之世,夫孰能不波?唯至人乃能遊於世而不僻,順人而不失己,彼教不學,承意不彼。

目徹為明,耳徹為聰,鼻徹為顫,口徹為甘,心徹為知,知徹為德。凡道不欲壅,壅則哽,哽而不止則跈,跈則眾害生。物之有知者恃息,其不殷,非天之罪。天之穿之,日夜無降,人則顧塞其竇。胞有重閬,心有天遊。室無空虛,則婦姑勃谿;心無天遊,則六鑿相攘。大林丘山之善於人也,亦神者不勝。

德溢乎名,名溢乎暴,謀稽乎誸,知出乎爭,柴生乎守,官事果乎眾宜。春雨日時,草木怒生,銚鎒於是乎始修,草木之到植者過半,而不知其然。

靜然可以補病,眥搣可以休老,寧可以止遽。雖然,若是,勞者之務也,非佚者之所未嘗過而問焉。聖人之所以駴天下,神人未嘗過而問焉;賢人所以駴世,聖人未嘗過而問焉;君子所以駴國,賢人未嘗過而問焉;小人所以合時,君子未嘗過而問焉。

演門有親死者,以善毀,爵為官師,其黨人毀而死者半。堯與許由天下,許由逃之;湯與務光天下,務光怒之。紀他聞之,帥弟子而踆於窾水,諸侯弔之三年,申徒狄因以踣河。

荃者所以在魚,得魚而忘荃;蹄者所以在兔,得兔而忘蹄;言者所以在意,得意而忘言。吾安得忘言之人而與之言哉?」


\end{pinyinscope}