\article{齊物論}

\begin{pinyinscope}
南郭子綦隱几而坐,仰天而噓,嗒焉似喪其耦。顏成子游立侍乎前,曰:「何居乎?形固可使如槁木,而心固可使如死灰乎?今之隱几者,非昔之隱几者也。」子綦曰:「偃,不亦善乎而問之也!今者吾喪我,汝知之乎?女聞人籟而未聞地籟,女聞地籟而未聞天籟夫!」子游曰:「敢問其方。」子綦曰:「夫大塊噫氣,其名為風。是唯无作,作則萬竅怒呺。而獨不聞之翏翏乎?山林之畏佳,大木百圍之竅穴,似鼻,似口,似耳,似枅,似圈,似臼,似洼者,似污者;激者,謞者,叱者,吸者,叫者,譹者,宎者,咬者,前者唱于而隨者唱喁。泠風則小和,飄風則大和,厲風濟則眾竅為虛。而獨不見之調調、之刁刁乎?」子游曰:「地籟則眾竅是已,人籟則比竹是已。敢問天籟。」子綦曰:「夫吹萬不同,而使其自已1也,咸其自取,怒者其誰邪!」1. 已 : 或作「己」。王孝魚點校《莊子集釋》作「己」。

大知閑閑,小知閒閒;大言炎炎,小言詹詹。其寐也魂交,其覺也形開,與接為構,日以心鬭。縵者,窖者,密者。小恐惴惴,大恐縵縵。其發若機栝,其司是非之謂也;其留如詛盟,其守勝之謂也;其殺如秋冬,以言其日消也;其溺之所為之,不可使復之也;其厭也如緘,以言其老洫也;近死之心,莫使復陽也。喜怒哀樂,慮嘆變慹,姚佚啟態;樂出虛,蒸成菌。日夜相代乎前,而莫知其所萌。已乎已乎!旦暮得此,其所由以生乎!

非彼無我,非我無所取。是亦近矣,而不知其所為使。若有真宰,而特不得其眹。可行已信,而不見其形,有情而無形。百骸、九竅、六藏,賅而存焉,吾誰與為親?汝皆說之乎?其有私焉?如是皆有,為臣妾乎,其臣妾不足以相治乎。其遞相為君臣乎,其有真君存焉。如求得其情與不得,無益損乎其真。一受其成形,不亡以待盡。與物相刃相靡,其行盡如馳,而莫之能止,不亦悲乎!終身役役而不見其成功,苶然疲役而不知其所歸,可不哀邪!人謂之不死,奚益?其形化,其心與之然,可不謂大哀乎?人之生也,固若是芒乎!其我獨芒,而人亦有不芒者乎!

夫隨其成心而師之,誰獨且無師乎?奚必知代而心自取者有之?愚者與有焉。未成乎心而有是非,是今日適越而昔至也。是以無有為有。無有為有,雖有神禹,且不能知,吾獨且柰何哉!夫言非吹也。言者有言,其所言者特未定也。果有言邪?其未嘗有言邪?其以為異於鷇音,亦有辯乎,其無辯乎?道惡乎隱而有真偽?言惡乎隱而有是非?道惡乎往而不存?言惡乎存而不可?道隱於小成,言隱於榮華。故有儒、墨之是非,以是其所非,而非其所是。欲是其所非而非其所是,則莫若以明。

物無非彼,物無非是。自彼則不見,自知則知之。故曰:彼出於是,是亦因彼。彼是,方生之說也。雖然,方生方死,方死方生;方可方不可,方不可方可;因是因非,因非因是。是以聖人不由,而照之于天,亦因是也。是亦彼也,彼亦是也。彼亦一是非,此亦一是非。果且有彼是乎哉?果且無彼是乎哉?彼是莫得其偶,謂之道樞。樞始得其環中,以應無窮。是亦一無窮,非亦一無窮也。故曰「莫若以明」。

以指喻指之非指,不若以非指喻指之非指也;以馬喻馬之非馬,不若以非馬喻馬之非馬也。天地,一指也;萬物,一馬也。可乎可,不可乎不可。道行之而成,物謂之而然。惡乎然?然於然。惡乎不然?不然於不然。物固有所然,物固有所可。無物不然,無物不可。故為是舉莛與楹,厲與西施,恢恑憰怪,道通為一。其分也,成也;其成也,毀也。凡物無成與毀,復通為一。唯達者知通為一,為是不用而寓諸庸。庸也者,用也;用也者,通也;通也者,得也。適得而幾矣。因是已。已而不知其然,謂之道。勞神明為一,而不知其同也,謂之朝三。何謂朝三?曰狙公賦芧,曰:「朝三而莫四。」眾狙皆怒。曰:「然則朝四而莫三。」眾狙皆悅。名實未虧,而喜怒為用,亦因是也。是以聖人和之以是非,而休乎天鈞,是之謂兩行。

古之人,其知有所至矣。惡乎至?有以為未始有物者,至矣盡矣,不可以加矣。其次以為有物矣,而未始有封也。其次以為有封焉,而未始有是非也。是非之彰也,道之所以虧也。道之所以虧,愛之所以成。果且有成與虧乎哉?果且無成與虧乎哉?有成與虧,故昭氏之鼓琴也;無成與虧,故昭氏之不鼓琴也。昭文之鼓琴也,師曠之枝策也,惠子之據梧也,三子之知幾乎!皆其盛者也,故載之末年。唯其好之也,以異於彼,其好之也,欲以明之彼。非所明而明之,故以堅白之昧終。而其子又以文之綸終,終身無成。若是而可謂成乎,雖我亦成也。若是而不可謂成乎,物與我無成也。是故滑疑之耀,聖人之所圖也。為是不用而寓諸庸,此之謂以明。

今且有言於此,不知其與是類乎?其與是不類乎?類與不類,相與為類,則與彼無以異矣。雖然,請嘗言之。有始也者,有未始有始也者,有未始有夫未始有始也者。有有也者,有無也者,有未始有無也者,有未始有夫未始有無也者。俄而有無矣,而未知有無之果孰有孰無也。今我則已有謂矣,而未知吾所謂之其果有謂乎,其果無謂乎?

天下莫大於秋豪之末,而大山為小;莫壽乎殤子,而彭祖為夭。天地與我並生,而萬物與我為一。既已為一矣,且得有言乎?既已謂之一矣,且得無言乎?一與言為二,二與一為三。自此以往,巧歷不能得,而況其凡乎!故自無適有,以至於三,而況自有適有乎!無適焉,因是已。

夫道未始有封,言未始有常,為是而有畛也。請言其畛:有左,有右,有倫,有義,有分,有辯,有競,有爭,此之謂八德。六合之外,聖人存而不論;六合之內,聖人論而不議。春秋經世,先王之志,聖人議而不辯。故分也者,有不分也;辯也者,有不辯也。曰:何也?聖人懷之,眾人辯之以相示也。故曰:辯也者,有不見也。夫大道不稱,大辯不言,大仁不仁,大廉不嗛,大勇不忮。道昭而不道,言辯而不及,仁常而不成,廉清而不信,勇忮而不成。五者园而幾向方矣。故知止其所不知,至矣。孰知不言之辯,不道之道?若有能知,此之謂天府。注焉而不滿,酌焉而不竭,而不知其所由來,此之謂葆光。故昔者堯問於舜曰:「我欲伐宗、膾、胥敖,南面而不釋然。其故何也?」舜曰:「夫三子者,猶存乎蓬艾之間。若不釋然,何哉?昔者十日並出,萬物皆照,而況德之進乎日者乎!」

齧缺問乎王倪曰:「子知物之所同是乎?」曰:「吾惡乎知之!」「子知子之所不知邪?」曰:「吾惡乎知之!」「然則物無知邪?」曰:「吾惡乎知之!雖然,嘗試言之。庸詎知吾所謂知之非不知邪?庸詎知吾所謂不知之非知邪?且吾嘗試問乎女:民溼寢則腰疾偏死,鰌然乎哉?木處則惴慄恂懼,猨猴然乎哉?三者孰知正處?民食芻豢,麋鹿食薦,蝍且甘帶,鴟鴉耆鼠,四者孰知正味?猨,猵狙以為雌,麋與鹿交,鰌與魚游。毛嬙、麗姬,人之所美也,魚見之深入,鳥見之高飛,麋鹿見之決驟。四者孰知天下之正色哉?自我觀之,仁義之端,是非之塗,樊然殽亂,吾惡能知其辯!」齧缺曰:「子不知利害,則至人固不知利害乎?」王倪曰:「至人神矣:大澤焚而不能熱,河、漢沍而不能寒,疾雷破山、風振海而不能驚。若然者,乘雲氣,騎日月,而遊乎四海之外。死生无變於己,而況利害之端乎!」

瞿鵲子問乎長梧子曰:「吾聞諸夫子,聖人不從事於務,不就利,不違害,不喜求,不緣道,无謂有謂,有謂无謂,而遊乎塵垢之外。夫子以為孟浪之言,而我以為妙道之行也。吾子以為奚若?」長梧子曰:「是黃帝之所聽熒也,而丘也何足以知之!且女亦大早計,見卵而求時夜,見彈而求鴞炙。予嘗為女妄言之,女以妄聽之,奚?旁日月,挾宇宙,為其脗合,置其滑涽,以隸相尊。眾人役役,聖人愚芚,參萬歲而一成純。萬物盡然,而以是相蘊。予惡乎知說生之非惑邪!予惡乎知惡死之非弱喪而不知歸者邪!麗之姬,艾封人之子也。晉國之始得之也,涕泣沾襟;及其至於王所,與王同筐床,食芻豢,而後悔其泣也。予惡乎知夫死者不悔其始之蘄生乎!夢飲酒者,旦而哭泣;夢哭泣者,旦而田獵。方其夢也,不知其夢也。夢之中又占其夢焉,覺而後知其夢也。且有大覺而後知此其大夢也,而愚者自以為覺,竊竊然知之。君乎,牧乎,固哉!丘也,與女皆夢也;予謂女夢,亦夢也。是其言也,其名為弔詭。萬世之後,而一遇大聖知其解者,是旦暮遇之也。既使我與若辯矣,若勝我,我不若勝,若果是也?我果非也邪?我勝若,若不吾勝,我果是也?而果非也邪?其或是也,其或非也邪?其俱是也,其俱非也邪?我與若不能相知也,則人固受其黮闇。吾誰使正之?使同乎若者正之,既與若同矣,惡能正之!使同乎我者正之,既同乎我矣,惡能正之!使異乎我與若者正之,既異乎我與若矣,惡能正之!使同乎我與若者正之,既同乎我與若矣,惡能正之!然則我與若與人俱不能相知也,而待彼也邪?何化聲之相待,若其不相待。和之以天倪,因之以曼衍,所以窮年也。1謂和之以天倪?曰:是不是,然不然。是若果是也,則是之異乎不是也亦無辯;然若果然也,則然之異乎不然也亦無辯。化聲之相待,若其不相待。和之以天倪,因之以曼衍,所以窮年也。2忘年忘義,振於無竟,故寓諸無竟。」1. 化聲之相待,若其不相待。和之以天倪,因之以曼衍,所以窮年也。 : 從第12條移到此處。  2. 化聲之相待,若其不相待。和之以天倪,因之以曼衍,所以窮年也。 : 移到第12條。

罔兩問景曰:「曩子行,今子止,曩子坐,今子起,何其無特操與?」景曰:「吾有待而然者邪!吾所待又有待而然者邪!吾待蛇蚹、蜩翼邪!惡識所以然?惡識所以不然?」

昔者莊周夢為胡蝶,栩栩然胡蝶也,自喻適志與!不知周也。俄然覺,則蘧蘧然周也。不知周之夢為胡蝶與,胡蝶之夢為周與?周與胡蝶,則必有分矣。此之謂物化。


\end{pinyinscope}