\article{讓王}

\begin{pinyinscope}
堯以天下讓許由,許由不受。又讓於子州支父,子州支父曰:「以為我天子,猶之可也。雖然,我適有幽憂之病,方且治之,未暇治天下也。」夫天下至重也,而不以害其生,又況他物乎!唯無以天下為者,可以託天下也。

舜讓天下於子州支伯,子州支伯曰:「予適有幽憂之病,方且治之,未暇治天下也。」故天下大器也,而不以易生,此有道者之所以異乎俗者也。

舜以天下讓善卷,善卷曰:「余立於宇宙之中,冬日衣皮毛,夏日衣葛絺;春耕種,形足以勞動;秋收斂,身足以休息;日出而作,日入而息,逍遙於天地之間而心意自得。吾何以天下為哉?悲夫!子之不知余也!」遂不受。於是去而入深山,莫知其處。

舜以天下讓其友石戶之農,石戶之農曰:「捲捲乎后之為人,葆力之士也。」以舜之德為未至也,於是夫負妻戴,攜子以入於海,終身不反也。

大王亶父居邠,狄人攻之。事之以皮帛而不受,事之以犬馬而不受,事之以珠玉而不受,狄人之所求者土地也。大王亶父曰:「與人之兄居而殺其弟,與人之父居而殺其子,吾不忍也。子皆勉居矣!為吾臣與為狄人臣,奚以異?且吾聞之,不以所用養害所養。」因杖筴而去之。民相連而從之,遂成國於岐山之下。夫大王亶父可謂能尊生矣。能尊生者,雖貴富不以養傷身,雖貧賤不以利累形。今世之人,居高官尊爵者,皆重失之,見利輕亡其身,豈不惑哉!

越人三世弒其君,王子搜患之,逃乎丹穴。而越國無君,求王子搜不得,從之丹穴。王子搜不肯出,越人薰之以艾,乘以王輿。王子搜援綏登車,仰天而呼曰:「君乎君乎!獨不可以舍我乎!」王子搜非惡為君也,惡為君之患也。若王子搜者,可謂不以國傷生矣,此固越人之所欲得為君也。

韓、魏相與爭侵地。子華子見昭僖侯,昭僖侯有憂色。子華子曰:「今使天下書銘於君之前,書之言曰:『左手攫之則右手廢,右手攫之則左手廢,然而攫之者必有天下。』君能攫之乎?」昭僖侯曰:「寡人不攫也。」子華子曰:「甚善!自是觀之,兩臂重於天下也,身亦重於兩臂。韓之輕於天下亦遠矣,今之所爭者,其輕於韓又遠。君固愁身傷生以憂戚不得也!」僖侯曰:「善哉!教寡人者眾矣,未嘗得聞此言也。」子華子可謂知輕重矣。

魯君聞顏闔得道之人也,使人以幣先焉。顏闔守陋閭,苴布之衣而自飯牛。魯君之使者至,顏闔自對之。使者曰:「此顏闔之家與?」顏闔對曰:「此闔之家也。」使者致幣,顏闔曰:「恐聽者謬而遺使者罪,不若審之。」使者還,反審之,復來求之,則不得已。故若顏闔者,真惡富貴也。

故曰:道之真以治身,其緒餘以為國家,其土苴以治天下。由此觀之,帝王之功,聖人之餘事也,非所以完身養生也。今世俗之君子,多為身棄生以殉物,豈不悲哉!凡聖人之動作也,必察其所以之,與其所以為。今且有人於此,以隨侯之珠彈千仞之雀,世必笑之。是何也?則其所用者重而所要者輕也。夫生者,豈特隨侯之重哉!

子列子窮,容貌有飢色。客有言之於鄭子陽者曰:「列御寇,蓋有道之士也,居君之國而窮,君無乃為不好士乎?」鄭子陽即令官遺之粟。子列子見使者,再拜而辭。使者去,子列子入,其妻望之而拊心曰:「妾聞為有道者之妻子,皆得佚樂,今有飢色。君過而遺先生食,先生不受,豈不命邪!」子列子笑謂之曰:「君非自知我也。以人之言而遺我粟,至其罪我也,又且以人之言。此吾所以不受也。」其卒,民果作難而殺子陽。

楚昭王失國,屠羊說走而從於昭王。昭王反國,將賞從者,及屠羊說。屠羊說曰:「大王失國,說失屠羊;大王反國,說亦反屠羊。臣之爵祿已復矣,又何賞之言?」王曰:「強之!」屠羊說曰:「大王失國,非臣之罪,故不敢伏其誅;大王反國,非臣之功,故不敢當其賞。」王曰:「見之!」屠羊說曰:「楚國之法,必有重賞大功而後得見。今臣之知不足以存國,而勇不足以死寇。吳軍入郢,說畏難而避寇,非故隨大王也。今大王欲廢法毀約而見說,此非臣之所以聞於天下也。」王謂司馬子綦曰:「屠羊說居處卑賤而陳義甚高,子綦為我延之以三旌之位。」屠羊說曰:「夫三旌之位,吾知其貴於屠羊之肆也;萬鍾之祿,吾知其富於屠羊之利也。然豈可以食爵祿而使吾君有妄施之名乎!說不敢當,願復反吾屠羊之肆。」遂不受也。

原憲居魯,環堵之室,茨以生草,蓬戶不完,桑以為樞而甕牖,二室,褐以為塞,上漏下溼,匡坐而弦。子貢乘大馬,中紺而表素,軒車不容巷,往見原憲。原憲華冠縰履,杖藜而應門。子貢曰:「嘻!先生何病?」原憲應之曰:「憲聞之:『無財謂之貧,學而不能行謂之病。』今憲,貧也,非病也。」子貢逡巡而有愧色。原憲笑曰:「夫希世而行,比周而友,學以為人,教以為己,仁義之慝,輿馬之飾,憲不忍為也。」

曾子居衛,縕袍無表,顏色腫噲,手足胼胝。三日不舉火,十年不製衣,正冠而纓絕,捉衿而肘見,納履而踵決。曳縰而歌商頌,聲滿天地,若出金石。天子不得臣,諸侯不得友。故養志者忘形,養形者忘利,致道者忘心矣。

孔子謂顏回曰:「回來!家貧居卑,胡不仕乎?」顏回對曰:「不願仕。回有郭外之田五十畝,足以給饘粥;郭內之田十畝,足以為絲麻;鼓琴足以自娛;所學夫子之道者足以自樂也。回不願仕。」孔子愀然變容曰:「善哉回之意!丘聞之:『知足者不以利自累也,審自得者失之而不懼,行修於內者無位而不怍。』丘誦之久矣,今於回而後見之,是丘之得也。」

中山公子牟謂瞻子曰:「身在江海之上,心居乎魏闕之下,奈何?」瞻子曰:「重生。重生則利輕。」中山公子牟曰:「雖知之,未能自勝也。」瞻子曰:「不能自勝則從,神無惡乎?不能自勝而強不從者,此之謂重傷。重傷之人,無壽類矣。」魏牟,萬乘之公子也,其隱巖穴也,難為於布衣之士,雖未至乎道,可謂有其意矣。

孔子窮於陳、蔡之間,七日不火食,藜羹不糝,顏色甚憊,而弦歌於室。顏回擇菜,子路、子貢相與言曰:「夫子再逐於魯,削迹於衛,伐樹於宋,窮於商、周,圍於陳、蔡,殺夫子者無罪,藉夫子者無禁。弦歌鼓琴,未嘗絕音,君子之無恥也若此乎?」顏回無以應,入告孔子。孔子推琴喟然而歎曰:「由與賜,細人也。召而來!吾語之。」

子路、子貢入。子路曰:「如此者可謂窮矣。」孔子曰:「是何言也!君子通於道之謂通,窮於道之謂窮。今丘抱仁義之道,以遭亂世之患,其何窮之為?故內省而不窮於道,臨難而不失其德,天寒既至,霜露既降,吾是以知松柏之茂也。陳、蔡之隘,於丘其幸乎!」孔子削然反琴而弦歌,子路扢然執干而舞。子貢曰:「吾不知天之高也,地之下也。」

古之得道者,窮亦樂,通亦樂。所樂非窮通也,道德於此,則窮通為寒暑風雨之序矣。故許由娛於潁陽,而共伯得乎共首。

舜以天下讓其友北人無擇,北人無擇曰:「異哉!后之為人也,居於甽畝之中,而遊堯之門。不若是而已,又欲以其辱行漫我。吾羞見之。」因自投清泠之淵。

湯將伐桀,因卞隨而謀,卞隨曰:「非吾事也。」湯曰:「孰可?」曰:「吾不知也。」湯又因瞀光而謀,瞀光曰:「非吾事也。」湯曰:「孰可?」曰:「吾不知也。」湯曰:「伊尹何如?」曰:「強力忍垢,吾不知其他也。」湯遂與伊尹謀伐桀。

剋之,以讓卞隨。卞隨辭曰:「后之伐桀也謀乎我,必以我為賊也;勝桀而讓我,必以我為貪也。吾生乎亂世,而無道之人再來漫我以其辱行,吾不忍數聞也。」乃自投稠水而死。

湯又讓瞀光曰:「知者謀之,武者遂之,仁者居之,古之道也。吾子胡不立乎?」瞀光辭曰:「廢上,非義也;殺民,非仁也;人犯其難,我享其利,非廉也。吾聞之曰:『非其義者,不受其祿;無道之世,不踐其土。』況尊我乎!吾不忍久見也。」乃負石而自沈於廬水。

昔周之興,有士二人處於孤竹,曰伯夷、叔齊。二人相謂曰:「吾聞西方有人,似有道者,試往觀焉。」至於岐陽,武王聞之,使叔旦往見之,與盟曰:「加富二等,就官一列。」血牲而埋之。二人相視而笑曰:「嘻!異哉!此非吾所謂道也。昔者神農之有天下也,時祀盡敬而不祈喜;其於人也,忠信盡治而無求焉。樂與政為政,樂與治為治,不以人之壞自成也,不以人之卑自高也,不以遭時自利也。今周見殷之亂而遽為政,上謀而下行貨,阻兵而保威,割牲而盟以為信,揚行以說眾,殺伐以要利,是推亂以易暴也。吾聞古之士遭治世不避其任,遇亂世不為苟存。今天下闇,周德衰,其並乎周以塗吾身也,不如避之以絜吾行。」

二子北至於首陽之山,遂餓而死焉。若伯夷、叔齊者,其於富貴也,苟可得已,則必不賴。高節戾行,獨樂其志,不事於世,此二士之節也。


\end{pinyinscope}