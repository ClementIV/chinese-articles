\article{天道}

\begin{pinyinscope}
天道運而無所積,故萬物成;帝道運而無所積,故天下歸;聖道運而無所積,故海內服。明於天,通於聖,六通四辟於帝王之德者,其自為也,昧然無不靜者矣。聖人之靜也,非曰靜也善,故靜也,萬物無足以鐃心者,故靜也。水靜則明燭鬚眉,平中準,大匠取法焉。水靜猶明,而況精神!聖人之心靜乎,天地之鑑也,萬物之鏡也。夫虛靜恬淡,寂漠無為者,天地之平而道德之至,故帝王聖人休焉。休則虛,虛則實,實者倫矣。虛則靜,靜則動,動則得矣。靜則無為,無為也,則任事者責矣。無為則俞俞,俞俞者憂患不能處,年壽長矣。夫虛靜恬淡,寂寞無為者,萬物之本也。明此以南鄉,堯之為君也;明此以北面,舜之為臣也。以此處上,帝王天子之德也;以此處下,玄聖素王之道也。以此退居而閒游,江海山林之士服;以此進為而撫世,則功大名顯而天下一也。靜而聖,動而王,無為也而尊,樸素而天下莫能與之爭美。夫明白於天地之德者,此之謂大本大宗,與天和者也;所以均調天下,與人和者也。與人和者,謂之人樂;與天和者,謂之天樂。

莊子曰:「吾師乎!吾師乎!虀萬物而不為戾,澤及萬世而不為仁,長於上古而不為壽,覆載天地、刻雕眾形而不為朽,此之謂天樂。故曰:知天樂者,其生也天行,其死也物化;靜而與陰同德,動而與陽同波。故知天樂者,無天怨,無人非,無物累,無鬼責。故曰:其動也天,其靜也地,一心定而王天下;其鬼不祟,其魂不疲,一心定而萬物服。言以虛靜推於天地,通於萬物,此之謂天樂。天樂者,聖人之心,以蓄天下也。」

夫帝王之德,以天地為宗,以道德為主,以無為為常。無為也,則用天下而有餘;有為也,則為天下用而不足。故古之人貴夫無為也。上無為也,下亦無為也,是下與上同德,下與上同德則不臣;下有為也,上亦有為也,是上與下同道,上與下同道則不主。上必無為而用天下,下必有為為天下用,此不易之道也。故古之王天下者,知雖落天地,不自慮也;辯雖彫萬物,不自說也;能雖窮海內,不自為也。天不產而萬物化,地不長而萬物育,帝王無為而天下功。故曰:莫神於天,莫富於地,莫大於帝王。故曰:帝王之德配天地。此乘天地,馳萬物,而用人群之道也。

本在於上,末在於下;要在於主,詳在於臣。三軍、五兵之運,德之末也;賞罰利害,五刑之辟,教之末也;禮法度數,形名比詳,治之末也;鐘鼓之音,羽毛之容,樂之末也;哭泣衰絰,隆殺之服,哀之末也。此五末者,須精神之運,心術之動,然後從之者也。末學者,古人有之,而非所以先也。

君先而臣從,父先而子從,兄先而弟從,長先而少從,男先而女從,夫先而婦從。夫尊卑先後,天地之行也,故聖人取象焉。天尊地卑,神明之位也;春夏先,秋冬後,四時之序也。萬物化作,萌區有狀,盛衰之殺,變化之流也。夫天地至神,而有尊卑先後之序,而況人道乎!宗廟尚親,朝廷尚尊,鄉黨尚齒,行事尚賢,大道之序也。語道而非其序者,非其道也;語道而非其道者,安取道!

是故古之明大道者,先明天而道德次之,道德已明而仁義次之,仁義已明而分守次之,分守已明而形名次之,形名已明而因任次之,因任已明而原省次之,原省已明而是非次之,是非已明而賞罰次之。賞罰已明而愚知處宜,貴賤履位,仁賢不肖襲情,必分其能,必由其名。以此事上,以此畜下,以此治物,以此修身,知謀不用,必歸其天,此之謂太平,治之至也。

故《書》曰:「有形有名。」形名者,古人有之,而非所以先也。古之語大道者,五變而形名可舉,九變而賞罰可言也。驟而語形名,不知其本也;驟而語賞罰,不知其始也。倒道而言,迕道而說者,人之所治也,安能治人!驟而語形名賞罰,此有知治之具,非知治之道;可用於天下,不足以用天下。此之謂辯士,一曲之人也。禮法度數,形名比詳,古人有之,此下之所以事上,非上之所以畜下也。

昔者舜問於堯曰:「天王之用心何如?」堯曰:「吾不敖無告,不廢窮民,苦死者,嘉孺子而哀婦人。此吾所以用心也。」舜曰:「美則美矣,而未大也。」堯曰:「然則何如?」舜曰:「天德而出寧,日月照而四時行,若晝夜之有經,雲行而雨施矣。」堯曰:「膠膠擾擾乎!子,天之合也;我,人之合也。」夫天地者,古之所大也,而黃帝、堯、舜之所共美也。故古之王天下者,奚為哉?天地而已矣。

孔子西藏書於周室,子路謀曰:「由聞周之徵藏史有老聃者,免而歸居。夫子欲藏書,則試往因焉。」孔子曰:「善。」往見老聃,而老聃不許,於是繙十二經以說。老聃中其說,曰:「大謾,願聞其要。」孔子曰:「要在仁義。」老聃曰:「請問:仁義,人之性邪?」孔子曰:「然。君子不仁則不成,不義則不生。仁義,真人之性也,又將奚為矣?」老聃曰:「請問何謂仁義?」孔子曰:「中心物愷,兼愛無私,此仁義之情也。」老聃曰:「意!幾乎後言!夫兼愛,不亦迂乎!無私焉,乃私也。夫子若欲使天下無失其牧乎?則天地固有常矣,日月固有明矣,星辰固有列矣,禽獸固有群矣,樹木固有立矣。夫子亦放德而行,循道而趨,已至矣,又何偈偈乎揭仁義,若擊鼓而求亡子焉?意!夫子亂人之性也!」

士成綺見老子而問曰:「吾聞夫子聖人也,吾固不辭遠道而來,願見,百舍重趼而不敢息。今吾觀子,非聖人也。鼠壤有餘蔬,而棄妹之者,不仁也;生熟不盡於前,而積歛無崖。」老子漠然不應。士成綺明日復見,曰:「昔者吾有刺於子,今吾心正卻矣,何故也?」老子曰:「夫巧知神聖之人,吾自以為脫焉。昔者子呼我牛也而謂之牛,呼我馬也而謂之馬。苟有其實,人與之名而弗受,再受其殃。吾服也恒服,吾非以服有服。」士成綺雁行避影,履行,遂進而問:「修身若何?」老子曰:「而容崖然,而目衝然,而顙頯然,而口闞然,而狀義然,似繫馬而止也。動而持,發也機,察而審,知巧而睹於泰,凡以為不信。邊竟有人焉,其名為竊。」

夫子曰:「夫道,於大不終,於小不遺,故萬物備。廣廣乎其無不容也,淵乎其不可測也。形德仁義,神之末也,非至人孰能定之!夫至人有世,不亦大乎!而不足以為之累。天下奮柄而不與之偕,審乎無假而不與利遷,極物之真,能守其本,故外天地,遺萬物,而神未嘗有所困也。通乎道,合乎德,退仁義,賓禮樂,至人之心有所定矣。」

世之所貴道者,書也,書不過語,語有貴也。語之所貴者,意也,意有所隨。意之所隨者,不可以言傳也,而世因貴言傳書。世雖貴之,我猶不足貴也,為其貴非其貴也。故視而可見者,形與色也;聽而可聞者,名與聲也。悲夫!世人以形色名聲為足以得彼之情!夫形色名聲果不足以得彼之情,則知者不言,言者不知,而世豈識之哉!

桓公讀書於堂上,輪扁斲輪於堂下,釋椎鑿而上,問桓公曰:「敢問公之所讀者何言邪?」公曰:「聖人之言也。」曰:「聖人在乎?」公曰:「已死矣。」曰:「然則君之所讀者,古人之糟魄已夫!」桓公曰:「寡人讀書,輪人安得議乎!有說則可,無說則死。」輪扁曰:「臣也,以臣之事觀之。斲輪,徐則甘而不固,疾則苦而不入。不徐不疾,得之於手而應於心,口不能言,有數存焉於其間。臣不能以喻臣之子,臣之子亦不能受之於臣,是以行年七十而老斲輪。古之人與其不可傳也死矣,然則君之所讀者,古人之糟魄已夫。」


\end{pinyinscope}