\article{則陽}

\begin{pinyinscope}
則陽游於楚,夷節言之於王,王未之見,夷節歸。彭陽見王果曰:「夫子何不譚我於王?」王果曰:「我不若公閱休。」彭陽曰:「公閱休奚為者邪?」曰:「冬則擉鱉於江,夏則休乎山樊。有過而問者,曰:『此予宅也。』夫夷節已不能,而況我乎!吾又不若夷節。夫夷節之為人也,無德而有知,不自許,以之神其交,固顛冥乎富貴之地,非相助以德,相助消也。夫凍者假衣於春,暍者反冬乎冷風。夫楚王之為人也,形尊而嚴,其於罪也,無赦如虎,非夫佞人、正德,其孰能橈焉!故聖人,其窮也使家人忘其貧,其達也使王公忘其爵祿而化卑。其於物也,與之為娛矣;其於人也,樂物之通而保己焉。故或不言而飲人以和,與人並立而使人化。父子之宜,彼其乎歸居,而一閒其所施。其於人心者,若是其遠也。故曰待公閱休。」

聖人達綢繆,周盡一體矣,而不知其然,性也。復命搖作而以天為師,人則從而命之也。憂乎知而所行恆無幾時,其有止也若之何?

生而美者,人與之鑑,不告則不知其美於人也。若知之,若不知之,若聞之,若不聞之,其可喜也終無已,人之好之亦無已,性也。聖人之愛人也,人與之名,不告則不知其愛人也。若知之,若不知之,若聞之,若不聞之,其愛人也終無已,人之安之亦無已,性也。

舊國舊都,望之暢然;雖使丘陵草木之緡,入之者十九,猶之暢然。況見見聞聞者也?以十仞之臺縣眾閒者也!

冉相氏得其環中以隨成,與物無終無始,無幾無時日。與物化者,一不化者也,闔嘗舍之!夫師天而不得師天,與物皆殉,其以為事也若之何?夫聖人未始有天,未始有人,未始有始,未始有物,與世偕行而不替,所行之備而不洫,其合之也若之何?

湯得其司御門尹登恆為之傅之,從師而不囿,得其隨成;為之司其名之名,嬴法得其兩見。仲尼之盡慮,為之傅之。

容成氏曰:「除日無歲,無內無外。」

魏瑩與田侯牟約,田侯牟背之。魏瑩怒,將使人刺之。

犀首聞而恥之,曰:「君為萬乘之君也,而以匹夫從讎!衍請受甲二十萬,為君攻之,虜其人民,係其牛馬,使其君內熱發於背,然後拔其國。忌也出走,然後抶其背,折其脊。」

季子聞而恥之,曰:「築十仞之城,城者既十仞矣,則又壞之,此胥靡之所苦也。今兵不起七年矣,此王之基也。衍亂人,不可聽也。」

華子聞而醜之,曰:「善言伐齊者,亂人也;善言勿伐者,亦亂人也;謂伐之與不伐亂人也者,又亂人也。」王曰:「然則若何?」曰:「君求其道而已矣。」

惠子聞之而見戴晉人。戴晉人曰:「有所謂蝸者,君知之乎?」曰:「然。」「有國於蝸之左角者曰觸氏,有國於蝸之右角者曰蠻氏,時相與爭地而戰,伏尸數萬,逐北旬有五日而後反。」君曰:「噫!其虛言與?」曰:「臣請為君實之。君以意在四方上下有窮乎?」君曰:「無窮。」曰:「知遊心於無窮,而反在通達之國,若存若亡乎?」君曰:「然。」曰:「通達之中有魏,於魏中有梁,於梁中有王。王與蠻氏,有辯乎?」君曰:「無辯。」客出而君惝然若有亡也。

客出,惠子見。君曰:「客,大人也,聖人不足以當之。」惠子曰:「夫吹筦也,猶有嗃也;吹劍首者,吷而已矣。堯、舜,人之所譽也;道堯、舜於戴晉人之前,譬猶一吷也。」

孔子之楚,舍於蟻丘之漿。其鄰有夫妻臣妾登極者,子路曰:「是稯稯何為者邪?」仲尼曰:「是聖人僕也。是自埋於民,自藏於畔。其聲銷,其志無窮,其口雖言,其心未嘗言,方且與世違而心不屑與之俱。是陸沈者也,是其市南宜僚邪?」子路請往召之。孔子曰:「已矣!彼知丘之著於己也,知丘之適楚也,以丘為必使楚王之召己也,彼且以丘為佞人也。夫若然者,其於佞人也羞聞其言,而況親見其身乎!而何以為存?」子路往視之,其室虛矣。

長梧封人問子牢曰:「君為政焉勿鹵莽,治民焉勿滅裂。昔予為禾,耕而鹵莽之,則其實亦鹵莽而報予;芸而滅裂之,其實亦滅裂而報予。予來年變齊,深其耕而熟耰之,其禾蘩以滋,予終年厭飧。」莊子聞之曰:「今人之治其形,理其心,多有似封人之所謂:遁其天,離其性,滅其情,亡其神,以眾為。故鹵莽其性者,欲惡之孽,為性萑葦蒹葭,始萌以扶吾形,尋擢吾性,並潰漏發,不擇所出,漂疽疥癕,內熱溲膏是也。」

柏矩學於老聃,曰:「請之天下遊。」老聃曰:「已矣!天下猶是也。」又請之,老聃曰:「汝將何始?」曰:「始於齊。」至齊,見辜人焉,推而強之,解朝服而幕之,號天而哭之曰:「子乎子乎!天下有大菑,子獨先離之!」曰:「莫為盜!莫為殺人!榮辱立,然後睹所病;貨財聚,然後睹所爭。今立人之所病,聚人之所爭,窮困人之身,使無休時,欲無至此,得乎!古之君人者,以得為在民,以失為在己;以正為在民,以枉為在己。故一形有失其形者,退而自責。今則不然。匿為物而愚不識,大為難而罪不敢,重為任而罰不勝,遠其塗而誅不至。民知力竭,則以偽繼之,日出多偽,士民安得不偽!夫力不足則偽,知不足則欺,財不足則盜。盜竊之行,於誰責而可乎?」

蘧伯玉行年六十而六十化,未嘗不始於是之而卒詘之以非也,未知今之所謂是之非五十九年非也。萬物有乎生而莫見其根,有乎出而莫見其門。人皆尊其知之所知,而莫知恃其知之所不知而後知,可不謂大疑乎!已乎已乎!且無所逃。此所謂然與,然乎?

仲尼問於大史大弢、伯常騫、狶韋曰:「夫衛靈公飲酒湛樂,不聽國家之政;田獵畢弋,不應諸侯之際。其所以為靈公者何邪?」大弢曰:「是因是也。」伯常騫曰:「夫靈公有妻三人,同濫而浴。史鰌奉御而進所,搏幣而扶翼。其慢若彼之甚也,見賢人若此其肅也,是其所以為靈公也。」狶韋曰:「夫靈公也死,卜葬於故墓不吉,卜葬於沙丘而吉。掘之數仞,得石槨焉,洗而視之,有銘焉,曰:『不馮其子,靈公奪而里之。』夫靈公之為靈也久矣,之二人何足以識之?」

少知問於大公調曰:「何謂丘里之言?」大公調曰:「丘里者,合十姓百名而以為風俗也。合異以為同,散同以為異。今指馬之百體而不得馬,而馬係於前者,立其百體而謂之馬也。是故丘山積卑而為高,江河合水而為大,大人合并而為公。是以自外入者,有主而不執;由中出者,有正而不距。四時殊氣,天不賜,故歲成;五官殊職,君不私,故國治;文武大人不賜,故德備;萬物殊理,道不私,故無名。無名故無為,無為而無不為。時有終始,世有變化,禍福淳淳,至有所拂者而有所宜;自殉殊面,有所正者有所差。比於大澤,百材皆度;觀於大山,木石同壇。此之謂丘里之言。」

少知曰:「然則謂之道,足乎?」大公調曰:「不然。今計物之數,不止於萬,而期曰『萬物』者,以數之多者號而讀之也。是故天地者,形之大者也;陰陽者,氣之大者也;道者為之公。因其大而號以讀之,則可也。已有之矣,乃將得比哉!則若以斯辯,譬猶狗馬,其不及遠矣。」

少知曰:「四方之內,六合之裏,萬物之所生惡起?」太公調曰:「陰陽相照、相蓋、相治,四時相代、相生、相殺,欲惡去就於是橋起,雌雄片合於是庸有。安危相易,禍福相生,緩急相摩,聚散以成。此名實之可紀,精微之可志也。隨序之相理,橋運之相使,窮則反,終則始。此物之所有,言之所盡,知之所至,極物而已。覩道之人,不隨其所廢,不原其所起,此議之所止。」

少知曰:「季真之莫為,接子之或使,二家之議,孰正於其情?孰偏於其理?」太公調曰:「雞鳴狗吠,是人之所知,雖有大知,不能以言讀其所自化,又不能以意其所將為。斯而析之,精至於無倫,大至於不可圍,或之使,莫之為,未免於物而終以為過。或使則實,莫為則虛。有名有實,是物之居;無名無實,在物之虛。可言可意,言而愈疏。未生不可忌,已死不可阻。死生非遠也,理不可睹。或之使,莫之為,疑之所假。吾觀之本,其往無窮;吾求之末,其來無止。無窮、無止,言之無也,與物同理;或使、莫為,言之本也,與物終始。道不可有,有不可無。道之為名,所假而行。或使莫為,在物一曲,夫胡為於大方?言而足,則終日言而盡道;言而不足,則終日言而盡物。道、物之極,言、默不足以載;非言非默,議其有極。」


\end{pinyinscope}