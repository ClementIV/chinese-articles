\article{逍遙遊}

\begin{pinyinscope}
北冥有魚,其名為鯤。鯤之大,不知其幾千里也。化而為鳥,其名為鵬。鵬之背,不知其幾千里也;怒而飛,其翼若垂天之雲。是鳥也,海運則將徙於南冥。南冥者,天池也。齊諧者,志怪者也。諧之言曰:「鵬之徙於南冥也,水擊三千里,摶扶搖而上者九萬里,去以六月息者也。」野馬也,塵埃也,生物之以息相吹也。天之蒼蒼,其正色邪?其遠而無所至極邪?其視下也亦若是,則已矣。且夫水之積也不厚,則負大舟也無力。覆杯水於坳堂之上,則芥為之舟,置杯焉則膠,水淺而舟大也。風之積也不厚,則其負大翼也無力。故九萬里則風斯在下矣,而後乃今培風;背負青天而莫之夭閼者,而後乃今將圖南。蜩與學鳩笑之曰:「我決起而飛,槍1榆、枋,時則不至而控於地而已矣,奚以之九萬里而南為?」適莽蒼者三湌而反,腹猶果然;適百里者宿舂糧;適千里者三月聚糧。之二蟲又何知!小知不及大知,小年不及大年。奚以知其然也?朝菌不知晦朔,蟪蛄不知春秋,此小年也。楚之南有冥靈者,以五百歲為春,五百歲為秋;上古有大椿者,以八千歲為春,八千歲為秋。而彭祖乃今以久特聞,眾人匹之,不亦悲乎!1. 槍 : 原作「搶」。據《四部叢刊》本改。

湯之問棘也是已。窮髮之北,有冥海者,天池也。有魚焉,其廣數千里,未有知其脩者,其名為鯤。有鳥焉,其名為鵬,背若泰山,翼若垂天之雲,摶扶搖羊角而上者九萬里,絕雲氣,負青天,然後圖南,且適南冥也。斥鴳笑之曰:「彼且奚適也?我騰躍而上,不過數仞而下,翱翔蓬蒿之間,此亦飛之至也。而彼且奚適也?」此小大之辯也。

故夫知效一官,行比一鄉,德合一君而徵一國者,其自視也亦若此矣。而宋榮子猶然笑之。且舉世而譽之而不加勸,舉世而非之而不加沮,定乎內外之分,辯乎榮辱之竟,斯已矣。彼其於世,未數數然也。雖然,猶有未樹也。夫列子御風而行,泠然善也,旬有五日而後反。彼於致福者,未數數然也。此雖免乎行,猶有所待者也。若夫乘天地之正,而御六氣之辯,以遊無窮者,彼且惡乎待哉!故曰:至人無己,神人無功,聖人無名。

堯讓天下於許由,曰:「日月出矣,而爝火不息,其於光也,不亦難乎!時雨降矣,而猶浸灌,其於澤也,不亦勞乎!夫子立而天下治,而我猶尸之,吾自視缺然,請致天下。」許由曰:「子治天下,天下既已治也。而我猶代子,吾將為名乎?名者,實之賓也,吾將為賓乎?鷦鷯巢於深林,不過一枝;偃鼠飲河,不過滿腹。歸休乎君!予無所用天下為。庖人雖不治庖,尸祝不越樽俎而代之矣。」

肩吾問於連叔曰:「吾聞言於接輿,大而無當,往而不反。吾驚怖其言,猶河漢而無極也,大有逕庭,不近人情焉。」連叔曰:「其言謂何哉?」曰:「藐姑射之山,有神人居焉,肌膚若冰雪,淖約若處子,不食五穀,吸風飲露。乘雲氣,御飛龍,而遊乎四海之外。其神凝,使物不疵癘而年穀熟。吾以是狂而不信也。」連叔曰:「然,瞽者無以與乎文章之觀,聾者無以與乎鍾鼓之聲。豈唯形骸有聾盲哉?夫知亦有之。是其言也,猶時女也。之人也,之德也,將旁礡萬物,以為一世蘄乎亂,孰弊弊焉以天下為事!之人也,物莫之傷,大浸稽天而不溺,大旱、金石流、土山焦而不熱。是其塵垢粃糠,將猶陶鑄堯、舜者也,孰肯以物為事!宋人資章甫而適諸越,越人斷髮文身,無所用之。堯治天下之民,平海內之政,往見四子藐姑射之山,汾水之陽,窅然喪其天下焉。」

惠子謂莊子曰:「魏王貽我大瓠之種,我樹之成而實五石,以盛水漿,其堅不能自舉也。剖之以為瓢,則瓠落無所容。非不呺然大也,吾為其無用而掊之。」莊子曰:「夫子固拙於用大矣。宋人有善為不龜手之藥者,世世以洴澼絖為事。客聞之,請買其方百金。聚族而謀曰:『我世世為洴澼絖,不過數金;今一朝而鬻技百金,請與之。』客得之,以說吳王。越有難,吳王使之將。冬,與越人水戰,大敗越人,裂地而封之。能不龜手一也,或以封,或不免於洴澼絖,則所用之異也。今子有五石之瓠,何不慮以為大樽而浮乎江湖,而憂其瓠落無所容?則夫子猶有蓬之心也夫!」

惠子謂莊子曰:「吾有大樹,人謂之樗。其大本擁腫而不中繩墨,其小枝卷曲而不中規矩,立之塗,匠者不顧。今子之言,大而無用,眾所同去也。」莊子曰:「子獨不見狸狌乎?卑身而伏,以候敖者;東西跳梁,不避高下;中於機辟,死於罔罟。今夫斄牛,其大若垂天之雲。此能為大矣,而不能執鼠。今子有大樹,患其無用,何不樹之於無何有之鄉,廣莫之野,彷徨乎無為其側,逍遙乎寢臥其下?不夭斤斧,物無害者,無所可用,安所困苦哉!」


\end{pinyinscope}