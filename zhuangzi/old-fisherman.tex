\article{漁父}

\begin{pinyinscope}
孔子遊乎緇帷之林,休坐乎杏壇之上。弟子讀書,孔子絃歌鼓琴,奏曲未半。有漁父者下船而來,須眉交白,被髮揄袂,行原以上,距陸而止,左手據膝,右手持頤以聽。曲終而招子貢、子路,二人俱對。客指孔子曰:「彼何為者也?」子路對曰:「魯之君子也。」客問其族。子路對曰:「族孔氏。」客曰:「孔氏者何治也?」子路未應,子貢對曰:「孔氏者,性服忠信,身行仁義,飾禮樂,選人倫,上以忠於世主,下以化於齊民,將以利天下。此孔氏之所治也。」又問曰:「有土之君與?」子貢曰:「非也。」「侯王之佐與?」子貢曰:「非也。」客乃笑而還行,言曰:「仁則仁矣,恐不免其身,苦心勞形以危其真。嗚乎,遠哉其分於道也。」

子貢還,報孔子。孔子推琴而起曰:「其聖人與!」乃下求之,至於澤畔,方將杖拏而引其船,顧見孔子,還鄉而立。孔子反走,再拜而進。客曰:「子將何求?」孔子曰:「曩者先生有緒言而去,丘不肖,未知所謂,竊待於下風,幸聞咳唾之音,以卒相丘也!」客曰:「嘻!甚矣子之好學也!」孔子再拜而起曰:「丘少而修學,以至於今,六十九歲矣,無所得聞至教,敢不虛心!」

客曰:「同類相從,同聲相應,固天之理也。吾請釋吾之所有而經子之所以。子之所以者,人事也。天子、諸侯、大夫、庶人,此四者自正,治之美也,四者離位而亂莫大焉。官治其職,人憂其事,乃無所陵。故田荒室露,衣食不足,徵賦不屬,妻妾不和,長少無序,庶人之憂也;能不勝任,官事不治,行不清白,群下荒怠,功美不有,爵祿不持,大夫之憂也;廷無忠臣,國家昏亂,工技不巧,貢職不美,春秋後倫,不順天子,諸侯之憂也;陰陽不和,寒暑不時,以傷庶物,諸侯暴亂,擅相攘伐,以殘民人,禮樂不節,財用窮匱,人倫不飭,百姓淫亂,天子有司之憂也。今子既上無君侯有司之勢,而下無大臣職事之官,而擅飭禮樂,選人倫,以化齊民,不泰多事乎?且人有八疵,事有四患,不可不察也。非其事而事之,謂之摠;莫之顧而進之,謂之佞;希意道言,謂之諂;不擇是非而言,謂之諛;好言人之惡,謂之讒;析交離親,謂之賊;稱譽詐偽以敗惡人,謂之慝;不擇善否,兩容頰適,偷拔其所欲,謂之險。此八疵者,外以亂人,內以傷身,君子不友,明君不臣。所謂四患者,好經大事,變更易常,以挂功名,謂之叨;專知擅事,侵人自用,謂之貪;見過不更,聞諫愈甚,謂之很;人同於己則可,不同於己,雖善不善,謂之矜。此四患也。能去八疵,無行四患,而始可教已。」

孔子愀然而歎,再拜而起曰:「丘再逐於魯,削跡於衛,伐樹於宋,圍於陳、蔡。丘不知所失,而離此四謗者何也?」客悽然變容曰:「甚矣子之難悟也!人有畏影惡跡而去之走者,舉足愈數而跡愈多,走愈疾而影不離身,自以為尚遲,疾走不休,絕力而死。不知處陰以休影,處靜以息跡,愚亦甚矣!子審仁義之間,察同異之際,觀動靜之變,適受與之度,理好惡之情,和喜怒之節,而幾於不免矣。謹修而身,慎守其真,還以物與人,則無所累矣。今不修之身而求之人,不亦外乎!」

孔子愀然曰:「請問何謂真?」客曰:「真者,精誠之至也。不精不誠,不能動人。故強哭者雖悲不哀,強怒者雖嚴不威,強親者雖笑不和。真悲無聲而哀,真怒未發而威,真親未笑而和。真在內者,神動於外,是所以貴真也。其用於人理也,事親則慈孝,事君則忠貞,飲酒則歡樂,處喪則悲哀。忠貞以功為主,飲酒以樂為主,處喪以哀為主,事親以適為主,功成之美,無一其跡矣。事親以適,不論所以矣;飲酒以樂,不選其具矣;處喪以哀,無問其禮矣。禮者,世俗之所為也;真者,所以受於天也,自然不可易也。故聖人法天貴真,不拘於俗。愚者反此,不能法天而恤於人,不知貴真,祿祿而受變於俗,故不足。惜哉!子之早湛於人偽,而晚聞大道也!」

孔子又再拜而起曰:「今者丘得遇也,若天幸然。先生不羞而比之服役,而身教之。敢問舍所在,請因受業而卒學大道。」客曰:「吾聞之:可與往者與之,至於妙道;不可與往者,不知其道,慎勿與之,身乃無咎。子勉之!吾去子矣,吾去子矣。」乃刺船而去,延緣葦間。

顏淵還車,子路授綏,孔子不顧,待水波定,不聞拏音,而後敢乘。子路旁車而問曰:「由得為役久矣,未嘗見夫子遇人如此其威也。萬乘之主,千乘之君,見夫子未嘗不分庭伉禮,夫子猶有倨敖之容。今漁者杖拏逆立,而夫子曲要磬折,言拜而應,得無太甚乎?門人皆怪夫子矣,漁人何以得此乎?」孔子伏軾而歎曰:「甚矣由之難化也!湛於禮義有間矣,而樸鄙之心至今未去。進!吾語汝。夫遇長不敬,失禮也;見賢不尊,不仁也。彼非至人,不能下人,下人不精,不得其真,故長傷身。惜哉!不仁之於人也,禍莫大焉,而由獨擅之。且道者,萬物之所出也,庶物失之者死,得之者生;為事逆之則敗,順之則成。故道之所在,聖人尊之。今漁父之道,可謂有矣,吾敢不敬乎!」


\end{pinyinscope}