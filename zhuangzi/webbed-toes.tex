\article{駢拇}

\begin{pinyinscope}
駢拇枝指,出乎性哉!而侈於德。附贅縣疣,出乎形哉!而侈於性。多方乎仁義而用之者,列於五藏哉!而非道德之正也。是故駢於足者,連無用之肉也;枝於手者,樹無用之指也;多方駢枝於五藏之情者,淫僻於仁義之行,而多方於聰明之用也。是故駢於明者,亂五色,淫文章,青黃黼黻之煌煌非乎?而離朱是已。多於聰者,亂五聲,淫六律,金石、絲竹,黃鐘、大呂之聲非乎?而師曠是已。枝於仁者,擢德塞性以收名聲,使天下簧鼓以奉不及之法非乎?而曾、史是已。駢於辯者,纍瓦結繩竄句,遊心於堅白同異之間,而敝跬譽無用之言非乎?而楊、墨是已。故此皆多駢旁枝之道,非天下之至正也。彼正正者,不失其性命之情。故合者不為駢,而枝者不為跂;長者不為有餘,短者不為不足。是故鳧脛雖短,續之則憂;鶴脛雖長,斷之則悲。故性長非所斷,性短非所續,無所去憂也。意仁義其非人情乎!彼仁人何其多憂也?且夫駢於拇者,決之則泣;枝於手者,齕之則啼。二者或有餘於數,或不足於數,其於憂一也。今世之仁人,蒿目而憂世之患;不仁之人,決性命之情而饕富貴。故意仁義其非人情乎!自三代以下者,天下何其囂囂也?

且夫待鉤繩規矩而正者,是削其性;待繩約膠漆而固者,是侵其德也;屈折禮樂,呴俞仁義,以慰天下之心者,此失其常然也。天下有常然。常然者,曲者不以鉤,直者不以繩,圓者不以規,方者不以矩,附離不以膠漆,約束不以纆索。故天下誘然皆生,而不知其所以生;同焉皆得,而不知其所以得。故古今不二,不可虧也。則仁義又奚連連如膠漆纆索,而遊乎道德之間為哉?使天下惑也!夫小惑易方,大惑易性。何以知其然邪?自虞氏招仁義以撓天下也,天下莫不奔命於仁義,是非以仁義易其性與?

故嘗試論之,自三代以下者,天下莫不以物易其性矣。小人則以身殉利,士則以身殉名,大夫則以身殉家,聖人則以身殉天下。故此數子者,事業不同,名聲異號,其於傷性以身為殉,一也。臧與穀,二人相與牧羊,而俱亡其羊。問臧奚事,則挾筴讀書;問穀奚事,則博塞以遊。二人者,事業不同,其於亡羊均也。伯夷死名於首陽之下,盜跖死利於東陵之上。二人者,所死不同,其於殘生傷性均也,奚必伯夷之是而盜跖之非乎?天下盡殉也。彼其所殉仁義也,則俗謂之君子;其所殉貨財也,則俗謂之小人。其殉一也,則有君子焉,有小人焉;若其殘生損性,則盜跖亦伯夷已,又惡取君子小人於其間哉?且夫屬其性乎仁義者,雖通如曾、史,非吾所謂臧也;屬其性於五味,雖通如俞兒,非吾所謂臧也;屬其性乎五聲,雖通如師曠,非吾所謂聰也;屬其性乎五色,雖通如離朱,非吾所謂明也。吾所謂臧者,非仁義之謂也,臧於其德而已矣;吾所謂臧者,非所謂仁義之謂也,任其性命之情而已矣;吾所謂聰者,非謂其聞彼也,自聞而已矣;吾所謂明者,非謂其見彼也,自見而已矣。夫不自見而見彼,不自得而得彼者,是得人之得而不自得其得者也,適人之適而不自適其適者也。夫適人之適而不自適其適,雖盜跖與伯夷,是同為淫僻也。余愧乎道德,是以上不敢為仁義之操,而下不敢為淫僻之行也。


\end{pinyinscope}