\article{庚桑楚}

\begin{pinyinscope}
老聃之役,有庚桑楚者,偏得老聃之道,以北居畏壘之山。其臣之畫然知者去之,其妾之挈然仁者遠之,擁腫之與居,鞅掌之為使。居三年,畏壘大壤。畏壘之民相與言曰:「庚桑子之始來,吾洒然異之。今吾日計之而不足,歲計之而有餘。庶幾其聖人乎!子胡不相與尸而祝之,社而稷之乎?」

庚桑子聞之,南面而不釋然。弟子異之。庚桑子曰:「弟子何異於予?夫春氣發而百草生,正得秋而萬寶成。夫春與秋,豈無得而然哉?天道已行矣。吾聞至人尸居環堵之室,而百姓猖狂不知所如往。今以畏壘之細民而竊竊欲俎豆予于賢人之閒,我其杓之人邪?吾是以不釋於老聃之言。」

弟子曰:「不然。夫尋常之溝,巨魚無所還其體,而鯢鰌為之制;步仞之丘陵,巨獸無所隱其軀,而㜸狐為之祥。且夫尊賢授能,先善與利,自古堯、舜以然,而況畏壘之民乎?夫子亦聽矣!」

庚桑子曰:「小子來!夫函車之獸,介而離山,則不免於罔罟之患;吞舟之魚,碭而失水,則蟻能苦之。故鳥獸不厭高,魚鱉不厭深。夫全其形生之人,藏其身也,不厭深眇而已矣。且夫二子者,又何足以稱揚哉!是其於辯也,將妄鑿垣牆而殖蓬蒿也。簡髮而櫛,數米而炊,竊竊乎又何足以濟世哉!舉賢則民相軋,任知則民相盜。之數物者,不足以厚民。民之於利甚勤,子有殺父,臣有殺君,正晝為盜,日中穴杯。吾語女:大亂之本,必生於堯、舜之間,其末存乎千世之後。千世之後,其必有人與人相食者也。」

南榮趎蹴然正坐曰:「若趎之年者已長矣,將惡乎託業以及此言邪?」庚桑子曰:「全汝形,抱汝生,無使汝思慮營營。若此三年,則可以及此言矣。」南榮趎曰:「目之與形,吾不知其異也,而盲者不能自見;耳之與形,吾不知其異也,而聾者不能自聞;心之與形,吾不知其異也,而狂者不能自得。形之與形亦辟矣,而物或閒之邪,欲相求而不能相得?今謂趎曰:『全汝形,抱汝生,勿使汝思慮營營。』趎勉聞道達耳矣。」庚桑子曰:「辭盡矣。曰:『奔蜂不能化藿蠋,越雞不能伏鵠卵,魯雞固能矣。』雞之與雞,其德非不同也,有能有不能者,其才固有巨小也。今吾才小,不足以化子,子胡不南見老子?」

南榮趎贏糧,七日七夜至老子之所。老子曰:「子自楚之所來乎?」南榮趎曰:「唯。」老子曰:「子何與人偕來之眾也?」南榮趎懼然顧其後。老子曰:「子不知吾所謂乎?」南榮趎俯而慚,仰而歎曰:「今者吾忘吾答,因失吾問。」老子曰:「何謂也?」南榮趎曰;「不知乎?人謂我朱愚。知乎?反愁我軀。不仁則害人,仁則反愁我身;不義則傷彼,義則反愁我已。我安逃此而可?此三言者,趎之所患也,願因楚而問之。」老子曰:「向吾見若眉睫之間,吾因以得汝矣,今汝又言而信之。若規規然若喪父母,揭竿而求諸海也。女亡人哉!惘惘乎汝欲反汝情性而無由入,可憐哉!」

南榮趎請入就舍,召其所好,去其所惡,十日自愁,復見老子。老子曰:「汝自洒濯,熟哉鬱鬱乎!然而其中津津乎猶有惡也。夫外韄者不可繁而捉,將內揵;內韄者不可繆而捉,將外揵。外、內韄者,道德不能持,而況放道而行者乎!」

南榮趎曰:「里人有病,里人問之,病者能言其病,然其病病者猶未病也。若趎之聞大道,譬猶飲藥以加病也,趎願聞衛生之經而已矣。」老子曰:「衛生之經,能抱一乎?能勿失乎?能無卜筮而知吉凶乎?能止乎?能已乎?能舍諸人而求諸己乎?能翛然乎?能侗然乎?能兒子乎?兒子終日嗥而嗌不嗄,和之至也;終日握而手不掜,共其德也;終日視而目不瞚,偏不在外也。行不知所之,居不知所為,與物委蛇,而同其波。是衛生之經已。」

南榮趎曰:「然則是至人之德已乎?」曰:「非也。是乃所謂冰解凍釋者能乎?夫至人者,相與交食乎地而交樂乎天,不以人物利害相攖,不相與為怪,不相與為謀,不相與為事,翛然而往,侗然而來。是謂衛生之經已。」曰:「然則是至乎?」曰:「未也。吾固告汝曰:『能兒子乎?』兒子動不知所為,行不知所之,身若槁木之枝而心若死灰。若是者,禍亦不至,福亦不來。禍福無有,惡有人災也?」

宇泰定者,發乎天光。發乎天光者,人見其人。人有修者,乃今有恆;有恆者,人舍之,天助之。人之所舍,謂之天民;天之所助,謂之天子。學者,學其所不能學也;行者,行其所不能行也;辯者,辯其所不能辯也。知止乎其所不能知,至矣。若有不即是者,天鈞敗之。

備物以將形,藏不虞以生心,敬中以達彼,若是而萬惡至者,皆天也,而非人也,不足以滑成,不可內於靈臺。靈臺者有持,而不知其所持,而不可持者也。不見其誠己而發,每發而不當,業入而不舍,每更為失。為不善乎顯明之中者,人得而誅之;為不善乎幽閒之中者,鬼得而誅之。明乎人、明乎鬼者,然後能獨行。

券內者行乎無名,券外者志乎期費。行乎無名者,唯庸有光;志乎期費者,唯賈人也,人見其跂,猶之魁然。與物窮者,物入焉;與物且者,其身之不能容,焉能容人!不能容人者無親,無親者盡人。兵莫憯於志,鏌鋣為下;寇莫大於陰陽,無所逃於天地之間。非陰陽賊之,心則使之也。

道通,其分也,其成也毀也。所惡乎分者,其分也以備;所以惡乎備者,其有以備。故出而不反,見其鬼;出而得,是謂得死。滅而有實,鬼之一也。以有形者象無形者而定矣。

出無本,入無竅。有實而無乎處,有長而無乎本剽,有所出而無竅者有實。有實而無乎處者,宇也;有長而無本剽者,宙也。有乎生,有乎死,有乎出,有乎入,入出而無見其形,是謂天門。天門者,無有也,萬物出乎無有。有不能以有為有,必出乎無有,而無有一無有。聖人藏乎是。

古之人,其知有所至矣。惡乎至?有以為未始有物者,至矣盡矣,弗可以加矣。其次以為有物矣,將以生為喪也,以死為反也,是以分已。其次曰始無有,既而有生,生俄而死;以無有為首,以生為體,以死為尻。孰知有無死生之一守者,吾與之為友。是三者雖異,公族也,昭、景也,著戴也,甲氏也,著封也。非一也。

有生,黬也,披然曰移是。嘗言移是,非所言也。雖然,不可知者也。臘者之有膍胲,可散而不可散也;觀室者周於寢廟,又適其偃焉,為是舉移是。

請嘗言移是。是以生為本,以知為師,因以乘是非;果有名實,因以己為質;使人以己為節,因以死償節。若然者,以用為知,以不用為愚,以徹為名,以窮為辱。移是,今之人也,是蜩與學鳩同於同也。

蹍市人之足,則辭以放驁,兄則以嫗,大親則已矣。故曰:至禮有不人,至義不物,至知不謀,至仁無親,至信辟金。

徹志之勃,解心之繆,去德之累,達道之塞。富、貴、顯、嚴、名、利六者,勃志也;容、動、色、理、氣、意六者,繆心也;惡、欲、喜、怒、哀、樂六者,累德也;去、就、取、與、知、能六者,塞道也。此四六者不盪胸中則正,正則靜,靜則明,明則虛,虛則無為而無不為也。

道者,德之欽也;生者,德之光也;性者,生之質也。性之動謂之為,為之偽謂之失。

知者,接也;知者,謨也;知者之所不知,猶睨也。

動以不得已之謂德,動無非我之謂治,名相反而實相順也。

羿工乎中微而拙於使人無己譽,聖人工乎天而拙乎人。夫工乎天而俍乎人者,唯全人能之。

唯蟲能蟲,唯蟲能天。全人惡天,惡人之天,而況吾天乎人乎!

一雀適羿,羿必得之,威也;以天下為之籠,則雀無所逃。是故湯以胞人籠伊尹,秦穆公以五羊之皮籠百里奚。是故非以其所好籠之而可得者,無有也。

介者拸畫,外非譽也;胥靡登高而不懼,遺死生也。夫復謵不餽而忘人,忘人,因以為天人矣。故敬之而不喜,侮之而不怒者,唯同乎天和者為然。出怒不怒,則怒出於不怒矣;出為無為,則為出於無為矣。欲靜則平氣,欲神則順心,有為也。欲當則緣於不得已,不得已之類,聖人之道。


\end{pinyinscope}