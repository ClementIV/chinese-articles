\article{盜跖}

\begin{pinyinscope}
孔子與柳下季為友。柳下季之弟名曰盜跖。盜跖從卒九千人,橫行天下,侵暴諸侯,穴室樞戶,驅人牛馬,取人婦女,貪得忘親,不顧父母兄弟,不祭先祖。所過之邑,大國守城,小國入保,萬民苦之。

孔子謂柳下季曰:「夫為人父者,必能詔其子;為人兄者,必能教其弟。若父不能詔其子,兄不能教其弟,則無貴父子兄弟之親矣。今先生,世之才士也,弟為盜跖,為天下害,而弗能教也,丘竊為先生羞之。丘請為先生往說之。」柳下季曰:「先生言『為人父者必能詔其子,為人兄者必能教其弟』,若子不聽父之詔,弟不受兄之教,雖今先生之辯,將奈之何哉?且跖之為人也,心如涌泉,意如飄風,強足以距敵,辯足以飾非,順其心則喜,逆其心則怒,易辱人以言。先生必無往。」

孔子不聽,顏回為御,子貢為右,往見盜跖。盜跖乃方休卒徒太山之陽,膾人肝而餔之。孔子下車而前,見謁者曰:「魯人孔丘,聞將軍高義,敬再拜謁者。」謁者入通,盜跖聞之大怒,目如明星,髮上指冠,曰:「此夫魯國之巧偽人孔丘非邪?為我告之:『爾作言造語,妄稱文、武,冠枝木之冠,帶死牛之脅,多辭繆說,不耕而食,不織而衣,搖脣鼓舌,擅生是非,以迷天下之主,使天下學士不反其本,妄作孝弟而儌倖於封侯富貴者也。子之罪大極重,疾走歸!不然,我將以子肝益晝餔之膳。』」

孔子復通曰:「丘得幸於季,願望履幕下。」謁者復通,盜跖曰:「使來前!」孔子趨而進,避席反走,再拜盜跖。盜跖大怒,兩展其足,案劍瞋目,聲如乳虎,曰:「丘來前!若所言,順吾意則生,逆吾心則死。」

孔子曰:「丘聞之,凡天下有三德:生而長大,美好無雙,少長貴賤見而皆說之,此上德也;知維天地,能辯諸物,此中德也;勇悍果敢,聚眾率兵,此下德也。凡人有此一德者,足以南面稱孤矣。今將軍兼此三者,身長八尺二寸,面目有光,脣如激丹,齒如齊貝,音中黃鐘,而名曰盜跖,丘竊為將軍恥不取焉。將軍有意聽臣,臣請南使吳、越,北使齊、魯,東使宋、衛,西使晉、楚,使為將軍造大城數百里,立數十萬戶之邑,尊將軍為諸侯,與天下更始,罷兵休卒,收養昆弟,共祭先祖。此聖人才士之行,而天下之願也。」

盜跖大怒曰:「丘來前!夫可規以利而可諫以言者,皆愚陋恆民之謂耳。今長大美好,人見而悅之者,此吾父母之遺德也。丘雖不吾譽,吾獨不自知邪?且吾聞之:『好面譽人者,亦好背而毀之。』今丘告我以大城眾民,是欲規我以利而恆民畜我也,安可久長也?城之大者,莫大乎天下矣。堯、舜有天下,子孫無置錐之地,湯、武立為天子而後世絕滅,非以其利大故邪?

且吾聞之:古者禽獸多而人少,於是民皆巢居以避之,晝拾橡栗,暮栖木上,故命之曰有巢氏之民。古者民不知衣服,夏多積薪,冬則煬之,故命之曰知生之民。神農之世,臥則居居,起則于于,民知其母,不知其父,與麋鹿共處,耕而食,織而衣,無有相害之心,此至德之隆也。然而黃帝不能致德,與蚩尤戰於涿鹿之野,流血百里。堯、舜作,立群臣,湯放其主,武王殺紂。自是之後,以強陵弱,以眾暴寡。湯、武以來,皆亂人之徒也。

今子修文、武之道,掌天下之辯,以教後世,縫衣淺帶,矯言偽行,以迷惑天下之主,而欲求富貴焉,盜莫大於子。天下何故不謂子為盜丘而乃謂我為盜跖?子以甘辭說子路而使從之,使子路去其危冠,解其長劍,而受教於子,天下皆曰『孔丘能止暴禁非』。其卒之也,子路欲殺衛君而事不成,身菹於衛東門之上,是子教之不至也。子自謂才士聖人邪!則再逐於魯,削跡於衛,窮於齊,圍於陳、蔡,不容身於天下。子教子路菹此患,上無以為身,下無以為人,子之道豈足貴邪?

世之所高,莫若黃帝,黃帝尚不能全德,而戰涿鹿之野,流血百里。堯不慈,舜不孝,禹偏枯,湯放其主,武王伐紂,文王拘羑里。此六子者,世之所高也,孰論之,皆以利惑其真而強反其情性,其行乃甚可羞也!

世之所謂賢士,伯夷、叔齊,伯夷、叔齊辭孤竹之君,而餓死於首陽之山,骨肉不葬。鮑焦飾行非世,抱木而死。申徒狄諫而不聽,負石自投於河,為魚鱉所食。介子推至忠也,自割其股以食文公,文公後背之,子推怒而去,抱木而燔死。尾生與女子期於梁下,女子不來,水至不去,抱梁柱而死。此六子者,無異於磔犬、流豕、操瓢而乞者,皆離名輕死,不念本養壽命者也。

世之所謂忠臣者,莫若王子比干、伍子胥,子胥沈江,比干剖心。此二子者,世謂忠臣也,然卒為天下笑。自上觀之,至於子胥、比干,皆不足貴也。

丘之所以說我者,若告我以鬼事,則我不能知也;若告我以人事者,不過此矣,皆吾所聞知也。今吾告子以人之情:目欲視色,耳欲聽聲,口欲察味,志氣欲盈。人上壽百歲,中壽八十,下壽六十,除病瘦、死喪、憂患,其中開口而笑者,一月之中不過四五日而已矣。天與地無窮,人死者有時,操有時之具而託於無窮之間,忽然無異騏驥之馳過隙也。不能說其志意,養其壽命者,皆非通道者也。丘之所言,皆吾之所棄也,亟去走歸,無復言之!子之道,狂狂汲汲,詐巧虛偽事也,非可以全真也,奚足論哉?」

孔子再拜趨走,出門上車,執轡三失,目芒然無見,色若死灰,據軾低頭,不能出氣。歸到魯東門外,適遇柳下季。柳下季曰:「今者闕然數日不見,車馬有行色,得微往見跖邪?」孔子仰天而歎曰:「然。」柳下季曰:「跖得無逆汝意若前乎?」孔子曰:「然。丘所謂無病而自灸也,疾走料虎頭,編虎須,幾不免虎口哉!」

子張1問於滿苟得曰:「盍不為行?無行則不信,不信則不任,不任則不利。故觀之名,計之利,而義真是也。若棄名利,反之於心,則夫士之為行,不可一日不為乎?」滿苟得曰:「無恥者富,多信者顯。夫名利之大者,幾在無恥而信。故觀之名,計之利,而信真是也。若棄名利,反之於心,則夫士之為行,抱其天乎!」

子張曰:「昔者桀、紂貴為天子,富有天下,今謂臧聚曰『汝行如桀、紂』,則有怍色,有不服之心者,小人所賤也。仲尼、墨翟,窮為匹夫,今謂宰相曰『子行如仲尼、墨翟』,則變容易色稱不足者,士誠貴也。故勢為天子,未必貴也;窮為匹夫,未必賤也。貴賤之分,在行之美惡。」滿苟得曰:「小盜者拘,大盜者為諸侯,諸侯之門,義士存焉。昔者桓公小白殺兄入嫂而管仲為臣,田成子常殺君竊國而孔子受幣。論則賤之,行則下之,則是言行之情悖戰於胸中也,不亦拂乎!故《書》曰:『孰惡孰美?成者為首,不成者為尾。』」

子張曰:「子不為行,即將疏戚無倫,貴賤無義,長幼無序,五紀六位將何以為別乎?」滿苟得曰:「堯殺長子,舜流母弟,疏戚有倫乎?湯放桀,武王伐紂,貴賤有義乎?王季為適,周公殺兄,長幼有序乎?儒者偽辭,墨者兼愛,五紀六位將有別乎?且子正為名,我正為利。名利之實,不順於理,不監於道。吾日與子訟於無約,曰:『小人殉財,君子殉名。其所以變其情,易其性,則異矣;乃至於棄其所為而殉其所不為,則一也。』故曰:無為小人,反殉而天;無為君子,從天之理。若枉若直,相而天極,面觀四方,與時消息。若是若非,執而圓機,獨成而意,與道徘徊。無轉而行,無成而義,將失而所為。無赴而富,無殉而成,將棄而天。比干剖心,子胥抉眼,忠之禍也;直躬證父,尾生溺死,信之患也;鮑子立乾,申子不自理,廉之害也;孔子不見母,匡子不見父,義之失也。此上世之所傳,下世之所語,以為士者正其言,必其行,故服其殃,離其患也。」1. 子張 : 這裡只是借用他的名字,並不是真的寫子張其人其事。

無足問於知和曰:「人卒未有不興名就利者。彼富則人歸之,歸則下之,下則貴之。夫見下貴者,所以長生、安體、樂意之道也。今子獨無意焉,知不足邪?意知而力不能行邪?故推正不忘邪?」知和曰:「今夫此人以為與己同時而生、同鄉而處者,以為夫絕俗過世之士焉,是專無主正,所以覽古今之時,是非之分也,與俗化世。去至重,棄至尊,以為其所為也,此其所以論長生、安體、樂意之道,不亦遠乎!慘怛之疾,恬愉之安,不監於體;怵惕之恐,欣懽之喜,不監於心。知為為而不知所以為,是以貴為天子,富有天下,而不免於患也。」

無足曰:「夫富之於人,無所不利,窮美究埶,至人之所不得逮,賢人之所不能及,俠人之勇力而不為威強,秉人之知謀以為明察,因人之德以為賢良,非享國而嚴若君父。且夫聲色、滋味、權勢之於人,心不待學而樂之,體不待象而安之。夫欲惡避就,固不待師,此人之性也。天下雖非我,孰能辭之!」知和曰:「知者之為,故動以百姓,不違其度,是以足而不爭,無以為故不求。不足故求之,爭四處而不自以為貪;有餘故辭之,棄天下而不自以為廉。廉貪之實,非以迫外也,反監之度。勢為天子而不以貴驕人,富有天下而不以財戲人。計其患,慮其反,以為害於性,故辭而不受也,非以要名譽也。堯、舜為帝而雍,非仁天下也,不以美害生也;善卷、許由得帝而不受,非虛辭讓也,不以事害己。此皆就其利,辭其害,而天下稱賢焉,則可以有之,彼非以興名譽也。」

無足曰:「必持其名,苦體絕甘,約養以持生,則亦久病長阨而不死者也。」知和曰:「平為福,有餘為害者,物莫不然,而財其甚者也。今富人耳營鐘鼓筦籥之聲,口嗛於芻豢醪醴之味,以感其意,遺忘其業,可謂亂矣;侅溺於馮氣,若負重行而上也,可謂苦矣;貪財而取慰,貪權而取竭,靜居則溺,體澤則馮,可謂疾矣;為欲富就利,故滿若堵耳而不知避,且馮而不舍,可謂辱矣;財積而無用,服膺而不舍,滿心戚醮,求益而不止,可謂憂矣;內則疑劫請之賊,外則畏寇盜之害,內周樓疏,外不敢獨行,可謂畏矣。此六者,天下之至害也,皆遺忘而不知察,及其患至,求盡性竭財,單以反一日之無故而不可得也。故觀之名則不見,求之利則不得,繚意體而爭此,不亦惑乎!」


\end{pinyinscope}