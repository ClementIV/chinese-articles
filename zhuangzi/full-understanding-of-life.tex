\article{達生}

\begin{pinyinscope}
達生之情者,不務生之所無以為;達命之情者,不務知之所無奈何。養形必先之以物,物有餘而形不養者有之矣;有生必先無離形,形不離而生亡者有之矣。生之來不能卻,其去不能止。悲夫!世之人以為養形足以存生,而養形果不足以存生,則世奚足為哉!雖不足為而不可不為者,其為不免矣。

夫欲免為形者,莫如棄世。棄世則無累,無累則正平,正平則與彼更生,更生則幾矣。事奚足棄而生奚足遺?棄事則形不勞,遺生則精不虧。夫形全精復,與天為一。天地者,萬物之父母也,合則成體,散則成始。形精不虧,是謂能移;精而又精,反以相天。

子列子問關尹曰:「至人潛行不窒,蹈火不熱,行乎萬物之上而不慄。請問何以至於此?」關尹曰:「是純氣之守也,非知巧果敢之列。居!吾語女。凡有貌象聲色者,皆物也,物與物何以相遠?夫奚足以至乎先?是色而已。則物之造乎不形,而止乎無所化,夫得是而窮之者,物焉得而止焉!彼將處乎不淫之度,而藏乎無端之紀,遊乎萬物之所終始,壹其性,養其氣,合其德,以通乎物之所造。夫若是者,其天守全,其神無郤,物奚自入焉!夫醉者之墜車,雖疾不死。骨節與人同,而犯害與人異,其神全也,乘亦不知也,墜亦不知也,死生驚懼不入乎其胷中,是故遻物而不慴。彼得全於酒而猶若是,而況得全於天乎!聖人藏於天,故莫之能傷也。」復讎者不折鏌、干,雖有忮心者不怨飄瓦,是以天下平均。故無攻戰之亂,無殺戮之刑者,由此道也。不開人之天,而開天之天,開天者德生,開人者賊生。不厭其天,不忽於人,民幾乎以其真。

仲尼適楚,出於林中,見痀僂者承蜩,猶掇之也。仲尼曰:「子巧乎?有道邪?」曰:「我有道也。五六月累丸,二而不墜,則失者錙銖;累三而不墜,則失者十一;累五而不墜,猶掇之也。吾處身也若厥株拘,吾執臂也若槁木之枝,雖天地之大,萬物之多,而唯蜩翼之知。吾不反不側,不以萬物易蜩之翼,何為而不得!」孔子顧謂弟子曰:「用志不分,乃凝於神,其痀僂丈人之謂乎!」

顏淵問仲尼曰:「吾嘗濟乎觴深之淵,津人操舟若神。吾問焉,曰:『操舟可學邪?』曰:『可。善游者數能。若乃夫沒人,則未嘗見舟而便操之也。』吾問焉而不吾告,敢問何謂也?」仲尼曰:「善游者數能,忘水也。若乃夫沒人之未嘗見舟而便操之也,彼視淵若陵,視舟之覆猶其車卻也。覆卻萬方陳乎前而不得入其舍,惡往而不暇!以瓦注者巧,以鉤注者憚,以黃金注者殙。其巧一也,而有所矜,則重外也。凡外重者內拙。」

田開之見周威公。威公曰:「吾聞祝腎學生。吾子與祝腎游,亦何聞焉?」田開之曰:「開之操拔篲以倚門庭,亦何聞於夫子!」威公曰:「田子無讓!寡人願聞之。」開之曰:「聞之夫子曰:『善養生者,若牧羊然,視其後者而鞭之。』」威公曰:「何謂也?」田開之曰:「魯有單豹者,巖居而水飲,不與民共利,行年七十而猶有嬰兒之色,不幸遇餓虎,餓虎殺而食之。有張毅者,高門、懸薄,無不走也,行年四十而有內熱之病以死。豹養其內而虎食其外,毅養其外而病攻其內,此二子者,皆不鞭其後者也。」

仲尼曰:「無入而藏,無出而陽,柴立其中央。三者若得,其名必極。夫畏塗者,十殺一人,則父子兄弟相戒也,必盛卒徒而後敢出焉,不亦知乎!人之所取畏者,衽席之上,飲食之間,而不知為之戒者,過也。」

祝宗人玄端以臨牢筴,說彘曰:「汝奚惡死?吾將三月豢汝,十日戒,三日齊,藉白茅,加汝肩尻乎彫俎之上,則汝為之乎?」為彘謀曰:「不如食以糠糟,而錯之牢筴之中。」自為謀,則苟生有軒冕之尊,死得於腞、楯之上,聚僂之中,則為之。為彘謀則去之,自為謀則取之,所異彘者何也?

桓公田於澤,管仲御,見鬼焉。公撫管仲之手曰:「仲父何見?」對曰:「臣無所見。」公反,誒詒為病,數日不出。齊士有皇子告敖者曰:「公則自傷,鬼惡能傷公!夫忿滀之氣,散而不反,則為不足;上而不下,則使人善怒;下而不上,則使人善忘;不上不下,中身當心,則為病。」桓公曰:「然則有鬼乎?」曰:「有。沈有履,灶有髻。戶內之煩壤,雷霆處之;東北方之下者,倍阿、鮭蠪躍之;西北方之下者,則泆陽處之。水有罔象,丘有峷,山有夔,野有彷徨,澤有委蛇。」公曰:「請問委蛇之狀何如?」皇子曰:「委蛇,其大如轂,其長如轅,紫衣而朱冠。其為物也惡,聞雷車之聲,則捧其首而立。見之者殆乎霸。」桓公囅然而笑曰:「此寡人之所見者也。」於是正衣冠與之坐,不終日而不知病之去也。

紀渻子為王養鬥雞。十日而問:「雞已乎?」曰:「未也。方虛憍而恃氣。」十日又問。曰:「未也。猶應嚮景。」十日又問。曰:「未也。猶疾視而盛氣。」十日又問。曰:「幾矣。雞雖有鳴者,已無變矣,望之似木雞矣,其德全矣,異雞無敢應者,反走矣。」

孔子觀於呂梁,縣水三十仞,流沫四十里,黿鼉魚龞之所不能游也。見一丈夫游之,以為有苦而欲死也,使弟子並流而拯之。數百步而出,被髮行歌而游於塘下。孔子從而問焉,曰:「吾以子為鬼,察子則人也。請問蹈水有道乎?」曰:「亡,吾無道。吾始乎故,長乎性,成乎命。與齊俱入,與汩偕出,從水之道而不為私焉。此吾所以蹈之也。」孔子曰:「何謂始乎故,長乎性,成乎命?」曰:「吾生於陵而安於陵,故也;長於水而安於水,性也;不知吾所以然而然,命也。」

梓慶削木為鐻,鐻成,見者驚猶鬼神。魯侯見而問焉,曰:「子何術以為焉?」對曰:「臣工人,何術之有!雖然,有一焉。臣將為鐻,未嘗敢以耗氣也,必齊以靜心。齊三日,而不敢懷慶賞爵祿;齊五日,不敢懷非譽巧拙;齊七日,輒然忘吾有四枝形體也。當是時也,無公朝,其巧專而外骨消;然後入山林,觀天性;形軀至矣,然後成見鐻,然後加手焉;不然則已。則以天合天,器之所以疑神者,其是與?」

東野稷以御見莊公,進退中繩,左右旋中規。莊公以為文弗過也,使之鉤百而反。顏闔遇之,入見曰:「稷之馬將敗。」公密而不應。少焉,果敗而反。公曰:「子何以知之?」曰:「其馬力竭矣,而猶求焉,故曰敗。」

工倕旋而蓋規矩,指與物化,而不以心稽,故其靈臺一而不桎。忘足,履之適也;忘要,帶之適也;知忘是非,心之適也;不內變,不外從,事會之適也。始乎適而未嘗不適者,忘適之適也。

有孫休者,踵門而詫子扁慶子曰:「休居鄉不見謂不修,臨難不見謂不勇,然而田原不遇歲,事君不遇世,賓於鄉里,逐於州部,則胡罪乎天哉?休惡遇此命也?」扁子曰:「子獨不聞夫至人之自行邪?忘其肝膽,遺其耳目,芒然彷徨乎塵垢之外,逍遙乎無事之業,是謂『為而不恃,長而不宰』。今汝飾知以驚愚,修身以明汙,昭昭乎若揭日月而行也。汝得全而形軀,具而九竅,無中道夭於聾盲跛蹇而比於人數,亦幸矣,又何暇乎天之怨哉!子往矣!」

孫子出。扁子入坐,有間,仰天而歎。弟子問曰:「先生何為歎乎?」扁子曰:「向者休來,吾告之以至人之德,吾恐其驚而遂至於惑也。」弟子曰:「不然。孫子之所言是邪,先生之所言非邪,非固不能惑是。孫子所言非邪,先生所言是邪,彼固惑而來矣,又奚罪焉?」

扁子曰:「不然。昔者有鳥止於魯郊,魯君說之,為具太牢以饗之,奏九韶以樂之,鳥乃始憂悲眩視,不敢飲食。此之謂以己養養鳥也。若夫以鳥養養鳥者,宜棲之深林,浮之江湖,食之以委蛇,則平陸而已矣。今休,款啟寡聞之民也,吾告以至人之德,譬之若載鼷以車馬,樂鴳以鐘鼓也。彼又奚能無驚乎哉?」


\end{pinyinscope}