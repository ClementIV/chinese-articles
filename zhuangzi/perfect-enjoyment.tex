\article{至樂}

\begin{pinyinscope}
天下有至樂無有哉?有可以活身者無有哉?今奚為奚據?奚避奚處?奚就奚去?奚樂奚惡?

夫天下之所尊者,富貴壽善也;所樂者,身安、厚味、美服、好色、音聲也;所下者,貧賤夭惡也;所苦者,身不得安逸,口不得厚味,形不得美服,目不得好色,耳不得音聲;若不得者,則大憂以懼。其為形也亦愚哉!

夫富者,苦身疾作,多積財而不得盡用,其為形也亦外矣。夫貴者,夜以繼日,思慮善否,其為形也亦疏矣。人之生也,與憂俱生,壽者惛惛,久憂不死,何苦也!其為形也亦遠矣。烈士為天下見善矣,未足以活身。吾未知善之誠善邪,誠不善邪?若以為善矣,不足活身;以為不善矣,足以活人。故曰:「忠諫不聽,蹲循勿爭。」故夫子胥爭之以殘其形,不爭,名亦不成。誠有善無有哉?今俗之所為與其所樂,吾又未知樂之果樂邪,果不樂邪?吾觀夫俗之所樂,舉群趣者,誙誙然如將不得已,而皆曰樂者,吾未之樂也,亦未之不樂也。果有樂無有哉?吾以無為誠樂矣,又俗之所大苦也。故曰:「至樂無樂,至譽無譽。」

天下是非果未可定也。雖然,無為可以定是非。至樂活身,唯無為幾存。請嘗試言之。天無為以之清,地無為以之寧,故兩無為相合,萬物皆化。芒乎芴乎,而無從出乎!芴乎芒乎,而無有象乎!萬物職職,皆從無為殖。故曰:「天地無為也,而無不為也。」人也,孰能得無為哉!

莊子妻死,惠子弔之,莊子則方箕踞鼓盆而歌。惠子曰:「與人居長子,老身死,不哭亦足矣,又鼓盆而歌,不亦甚乎!」莊子曰:「不然。是其始死也,我獨何能無概然!察其始而本無生,非徒無生也,而本無形,非徒無形也,而本無氣。雜乎芒芴之間,變而有氣,氣變而有形,形變而有生,今又變而之死,是相與為春秋冬夏四時行也。人且偃然寢於巨室,而我噭噭然隨而哭之,自以為不通乎命,故止也。」

支離叔與滑介叔觀於冥伯之丘,崑崙之虛,黃帝之所休。俄而柳生其左肘,其意蹶蹶然惡之。支離叔曰:「子惡之乎?」滑介叔曰:「亡。予何惡?生者,假借也;假之而生生者,塵垢也。死生為晝夜。且吾與子觀化而化及我,我又何惡焉?」

莊子之楚,見空髑髏,髐然有形,撽以馬捶,因而問之曰:「夫子貪生失理,而為此乎?將子有亡國之事,斧鉞之誅,而為此乎?將子有不善之行,愧遺父母妻子之醜,而為此乎?將子有凍餒之患,而為此乎?將子之春秋故及此乎?」於是語卒,援髑髏枕而臥。

夜半,髑髏見夢曰:「子之談者似辯士。視子所言,皆生人之累也,死則無此矣。子欲聞死之說乎?」莊子曰:「然。」髑髏曰:「死,無君於上,無臣於下,亦無四時之事,從然以天地為春秋,雖南面王樂,不能過也。」莊子不信,曰:「吾使司命復生子形,為子骨肉肌膚,反子父母妻子、閭里、知識,子欲之乎?」髑髏深矉蹙頞曰:「吾安能棄南面王樂而復為人間之勞乎?」

顏淵東之齊,孔子有憂色。子貢下席而問曰:「小子敢問:回東之齊,夫子有憂色,何邪?」孔子曰:「善哉汝問!昔者管子有言,丘甚善之,曰:『褚小者不可以懷大,綆短者不可以汲深。』夫若是者,以為命有所成而形有所適也,夫不可損益。吾恐回與齊侯言堯、舜、黃帝之道,而重以燧人、神農之言。彼將內求於己而不得,不得則惑,人惑則死。且女獨不聞邪?昔者海鳥止於魯郊,魯侯御而觴之於廟,奏九韶以為樂,具太牢以為善。鳥乃眩視憂悲,不敢食一臠,不敢飲一杯,三日而死。此以己養養鳥也,非以鳥養養鳥也。夫以鳥養養鳥者,宜栖之深林,遊之壇陸,浮之江湖,食之鰍鰷,隨行列而止,委蛇而處。彼唯人言之惡聞,奚以夫譊譊為乎!咸池、九韶之樂,張之洞庭之野,鳥聞之而飛,獸聞之而走,魚聞之而下入,人卒聞之,相與還而觀之。魚處水而生,人處水而死,故必相與異,其好惡故異也。故先聖不一其能,不同其事。名止於實,義設於適,是之謂條達而福持。」

列子行食於道,從見百歲髑髏,攓蓬而指之曰:「唯予與汝知而未嘗死,未嘗生也。若果養乎?予果歡乎?」

種有幾,得水則為㡭,得水土之際則為蛙蠙之衣,生於陵屯則為陵舄,陵舄得鬱棲則為烏足,烏足之根為蠐螬,其葉為蝴蝶。胡蝶,胥也化而為蟲,生於灶下,其狀若脫,其名為鴝掇。鴝掇千日為鳥,其名曰乾餘骨。乾餘骨之沬為斯彌,斯彌為食醯。頤輅生乎食醯,黃軦生乎九猷,瞀芮生乎腐蠸。羊奚比乎不筍,久竹生青寧,青寧生程,程生馬,馬生人,人又反入於機。萬物皆出於機,皆入於機。


\end{pinyinscope}