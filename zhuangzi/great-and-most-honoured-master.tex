\article{大宗師}

\begin{pinyinscope}
知天之所為,知人之所為者,至矣。知天之所為者,天而生也;知人之所為者,以其知之所知,以養其知之所不知,終其天年而不中道夭者,是知之盛也。雖然,有患。夫知有所待而後當,其所待者特未定也。庸詎知吾所謂天之非人乎?所謂人之非天乎?且有真人,而後有真知。

何謂真人?古之真人,不逆寡,不雄成,不謨士。若然者,過而弗悔,當而不自得也。若然者,登高不慄,入水不濡,入火不熱。是知之能登假於道也若此。

古之真人,其寢不夢,其覺無憂,其食不甘,其息深深。真人之息以踵,眾人之息以喉。屈服者,其嗌言若哇。其耆欲深者,其天機淺。

古之真人,不知說生,不知惡死;其出不訢,其入不距;翛然而往,翛然而來而已矣。不忘其所始,不求其所終;受而喜之,忘而復之。是之謂不以心捐道,不以人助天。是之謂真人。若然者,其心志,其容寂,其顙頯,淒然似秋,煖然似春,喜怒通四時,與物有宜,而莫知其極。故聖人之用兵也,亡國而不失人心;利澤施於萬物,不為愛人。故樂通物,非聖人也;有親,非仁也;天時,非賢也;利害不通,非君子也;行名失己,非士也;亡身不真,非役人也。若狐不偕、務光、伯夷、叔齊、箕子胥餘、紀他、申徒狄,是役人之役,適人之適,而不自適其適者也。

古之真人,其狀義而不朋,若不足而不承,與乎其觚而不堅也,張乎其虛而不華也,邴邴乎其似喜乎!崔乎其不得已乎!滀乎進我色也,與乎止我德也,厲乎其似世乎!謷乎其未可制也,連乎其似好閉也,悗乎忘其言也。以刑為體,以禮為翼,以知為時,以德為循。以刑為體者,綽乎其殺也;以禮為翼者,所以行於世也;以知為時者,不得已於事也;以德為循者,言其與有足者至於丘也,而人真以為勤行者也。故其好之也一,其弗好之也一。其一也一,其不一也一。其一,與天為徒;其不一,與人為徒。天與人不相勝也,是之謂真人。

死生,命也,其有夜旦之常,天也。人之有所不得與,皆物之情也。彼特以天為父,而身猶愛之,而況其卓乎!人特以有君為愈乎己,而身猶死之,而況其真乎!泉涸,魚相與處於陸,相呴以溼,相濡以沫,不如相忘於江湖。與其譽堯而非桀,不如兩忘而化其道。夫大塊載我以形,勞我以生,佚我以老,息我以死。故善吾生者,乃所以善吾死也。夫藏舟於壑,藏山於澤,謂之固矣。然而夜半有力者負之而走,昧者不知也。藏大小有宜,猶有所遯。若夫藏天下於天下,而不得所遯,是恆物之大情也。特犯人之形而猶喜之,若人之形者,萬化而未始有極也,其為樂可勝計邪!故聖人將遊於物之所不得遯而皆存。善妖善老,善始善終,人猶效之,又況萬物之所係,而一化之所待乎!

夫道,有情有信,無為無形;可傳而不可受,可得而不可見;自本自根,未有天地,自古以固存;神鬼神帝,生天生地;在太極之先而不為高,在六極之下而不為深;先天地生而不為久,長於上古而不為老。豨韋氏得之,以挈天地;伏犧氏得之,以襲氣母;維斗得之,終古不忒;日月得之,終古不息;堪坏得之,以襲崑崙;馮夷得之,以遊大川;肩吾得之,以處太山;黃帝得之,以登雲天;顓頊得之,以處玄宮;禺強得之,立乎北極;西王母得之,坐乎少廣,莫知其始,莫知其終;彭祖得之,上及有虞,下及五伯;傅說得之,以相武丁,奄有天下,乘東維,騎箕尾,而比於列星。

南伯子葵問乎女偊曰:「子之年長矣,而色若孺子,何也?」曰:「吾聞道矣。」南伯子葵曰:「道可得學邪?」曰:「惡!惡可!子非其人也。夫卜梁倚有聖人之才,而無聖人之道,我有聖人之道,而無聖人之才,吾欲以教之,庶幾其果為聖人乎!不然,以聖人之道告聖人之才,亦易矣。吾猶守而告之,參日而後能外天下;已外天下矣,吾又守之,七日而後能外物;已外物矣,吾又守之,九日而後能外生;已外生矣,而後能朝徹;朝徹,而後能見獨;見獨,而後能無古今;無古今,而後能入於不死不生。殺生者不死,生生者不生。其為物,無不將也,無不迎也;無不毀也,無不成也。其名為攖寧。攖寧也者,攖而後成者也。」南伯子葵曰:「子獨惡乎聞之?」曰:「聞諸副墨之子,副墨之子聞諸洛誦之孫,洛誦之孫聞之瞻明,瞻明聞之聶許,聶許聞之需役,需役聞之於謳,於謳聞之玄冥,玄冥聞之參寥,參寥聞之疑始。」

子祀、子輿、子犁、子來四人相與語曰:「孰能以無為首,以生為脊,以死為尻,孰知生死存亡之一體者,吾與之友矣。」四人相視而笑,莫逆於心,遂相與為友。俄而子輿有病,子祀往問之。曰:「偉哉!夫造物者,將以予為此拘拘也!曲僂發背,上有五管,頤隱於齊,肩高於頂,句贅指天。」陰陽之氣有沴,其心閒而無事,跰足而鑑於井,曰:「嗟乎!夫造物者,又將以予為此拘拘也!」子祀曰:「汝惡之乎?」曰:「亡,予何惡!浸假而化予之左臂以為雞,予因以求時夜;浸假而化予之右臂以為彈,予因以求鴞炙;浸假而化予之尻以為輪,以神為馬,予因以乘之,豈更駕哉!且夫得者時也,失者順也,安時而處順,哀樂不能入也。此古之所謂縣解也,而不能自解者,物有結之。且夫物不勝天久矣,吾又何惡焉?」俄而子來有病,喘喘然將死,其妻子環而泣之。子犁往問之曰:「叱!避!無怛化!」倚其戶與之語曰:「偉哉造物!又將奚以汝為?將奚以汝適?以汝為鼠肝乎?以汝為蟲臂乎?」子來曰:「父母於子,東西南北,唯命之從。陰陽於人,不翅於父母,彼近吾死而我不聽,我則悍矣,彼何罪焉!夫大塊載我以形,勞我以生,佚我以老,息我以死。故善吾生者,乃所以善吾死也。今之大冶鑄金,金踊躍曰『我且必為鏌鋣』,大冶必以為不祥之金。今一犯人之形,而曰『人耳人耳』,夫造化者必以為不祥之人。今一以天地為大鑪,以造化為大冶,惡乎往而不可哉!成然寐,蘧然覺。」

子桑戶、孟子反、子琴張三人相與友,曰:「孰能相與於無相與,相為於無相為?孰能登天遊霧,撓挑無極,相忘以生,無所終窮?」三人相視而笑,莫逆於心,遂相與友。莫然有閒,而子桑戶死,未葬。孔子聞之,使子貢往侍事焉。或編曲,或鼓琴,相和而歌曰:「嗟來桑戶乎!嗟來桑戶乎!而已反其真,而我猶為人猗!」子貢趨而進曰:「敢問臨尸而歌,禮乎?」二人相視而笑,曰:「是惡知禮意!」子貢反,以告孔子曰:「彼何人者邪?修行無有,而外其形骸,臨尸而歌,顏色不變,無以命之。彼何人者邪?」孔子曰:「彼遊方之外者也,而丘游方之內者也。外內不相及,而丘使女往弔之,丘則陋矣。彼方且與造物者為人,而遊乎天地之一氣。彼以生為附贅縣疣,以死為決𤴯潰癰。夫若然者,又惡知死生先後之所在!假於異物,託於同體,忘其肝膽,遺其耳目,反覆終始,不知端倪,芒然彷徨乎塵垢之外,逍遙乎無為之業。彼又惡能憒憒然為世俗之禮,以觀眾人之耳目哉!」子貢曰:「然則夫子何方之依?」孔子曰:「丘,天之戮民也。雖然,吾與汝共之。」子貢曰:「敢問其方。」孔子曰:「魚相造乎水,人相造乎道。相造乎水者,穿池而養給;相造乎道者,無事而生定。故曰:魚相忘乎江湖,人相忘乎道術。」子貢曰:「敢問畸人。」曰:「畸人者,畸於人而侔於天。故曰:天之小人,人之君子;人之君子,天之小人也。」

顏回問仲尼曰:「孟孫才,其母死,哭泣無涕,中心不戚,居喪不哀。無是三者,以善處喪蓋魯國。固有無其實而得其名者乎?回壹怪之。」仲尼曰:「夫孟孫氏盡之矣,進於知矣。唯簡之而不得,夫已有所簡矣。孟孫氏不知所以生,不知所以死,不知就先,不知就後,若化為物,以待其所不知之化已乎!且方將化,惡知不化哉?方將不化,惡知已化哉?吾特與汝其夢未始覺者邪!且彼有駭形而無損心,有旦宅而無情死。孟孫氏特覺,人哭亦哭,是自其所以乃。且也,相與吾之耳矣,庸詎知吾所謂吾之乎?且汝夢為鳥而厲乎天,夢為魚而沒於淵,不識今之言者,其覺者乎,夢者乎?造適不及笑,獻笑不及排,安排而去化,乃入於寥天一。」

意而子見許由,許由曰:「堯何以資汝?」意而子曰:「堯謂我:『汝必躬服仁義,而明言是非。』」許由曰:「而奚為來軹?夫堯既已黥汝以仁義,而劓汝以是非矣,汝將何以遊夫遙蕩、恣睢、轉徙之途乎?」意而子曰:「雖然,吾願遊於其藩。」許由曰:「不然。夫盲者無以與乎眉目顏色之好,瞽者無以與乎青黃黼黻之觀。」意而子曰:「夫無莊之失其美,據梁之失其力,黃帝之亡其知,皆在鑪捶之間耳。庸詎知夫造物者之不息我黥而補我劓,使我乘成以隨先生邪?」許由曰:「噫!未可知也。我為汝言其大略。吾師乎!吾師乎!齏萬物而不為義,澤及萬世而不為仁,長於上古而不為老,覆載天地、刻彫眾形而不為巧。此所遊已。」

顏回曰:「回益矣。」仲尼曰:「何謂也?」曰:「回忘仁義矣。」曰:「可矣,猶未也。」他日復見,曰:「回益矣。」曰:「何謂也?」曰:「回忘禮樂矣。」曰:「可矣,猶未也。」他日復見,曰:「回益矣。」曰:「何謂也?」曰:「回坐忘矣。」仲尼蹴然曰:「何謂坐忘?」顏回曰:「墮肢體,黜聰明,離形去知,同於大通,此謂坐忘。」仲尼曰:「同則無好也,化則無常也。而果其賢乎!丘也請從而後也。」

子輿與子桑友,而霖雨十日。子輿曰:「子桑殆病矣!」裹飯而往食之。至子桑之門,則若歌若哭,鼓琴曰:「父邪母邪!天乎人乎!」有不任其聲,而趨舉其詩焉。子輿入,曰:「子之歌詩,何故若是?」曰:「吾思乎使我至此極者而弗得也。父母豈欲吾貧哉?天無私覆,地無私載,天地豈私貧我哉?求其為之者而不得也。然而至此極者,命也夫!」


\end{pinyinscope}