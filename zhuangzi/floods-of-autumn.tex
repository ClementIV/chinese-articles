\article{秋水}

\begin{pinyinscope}
秋水時至,百川灌河,涇流之大,兩涘渚崖之間,不辯牛馬。於是焉河伯欣然自喜,以天下之美為盡在己。順流而東行,至於北海,東面而視,不見水端,於是焉河伯始旋其面目,望洋向若而歎,曰:「野語有之曰『聞道百,以為莫己若』者,我之謂也。且夫我嘗聞少仲尼之聞而輕伯夷之義者,始吾弗信,今我睹子之難窮也,吾非至於子之門則殆矣,吾長見笑於大方之家。」

北海若曰:「井蛙不可以語於海者,拘於虛也;夏蟲不可以語於冰者,篤於時也;曲士不可以語於道者,束於教也。今爾出於崖涘,觀於大海,乃知爾醜,爾將可與語大理矣。天下之水,莫大於海,萬川歸之,不知何時止而不盈;尾閭泄之,不知何時已而不虛;春秋不變,水旱不知。此其過江河之流,不可為量數。而吾未嘗以此自多者,自以比形於天地而受氣於陰陽,吾在天地之間,猶小石小木之在大山也,方存乎見少,又奚以自多!計四海之在天地之間也,不似礨空之在大澤乎?計中國之在海內,不似稊米之在大倉乎?號物之數謂之萬,人處一焉;人卒九州,穀食之所生,舟車之所通,人處一焉。此其比萬物也,不似豪末之在於馬體乎?五帝之所連,三王之所爭,仁人之所憂,任士之所勞,盡此矣。伯夷辭之以為名,仲尼語之以為博,此其自多也,不似爾向之自多於水乎?」

河伯曰:「然則吾大天地而小毫末可乎?」北海若曰:「否。夫物,量無窮,時無止,分無常,終始無故。是故大知觀於遠近,故小而不寡,大而不多,知量無窮;證曏今故,故遙而不悶,掇而不跂,知時無止;察乎盈虛,故得而不喜,失而不憂,知分之無常也;明乎坦塗,故生而不說,死而不禍,知終始之不可故也。計人之所知,不若其所不知;其生之時,不若未生之時。以其至小,求窮其至大之域,是故迷亂而不能自得也。由此觀之,又何以知毫末之足以定至細之倪!又何以知天地之足以窮至大之域!」

河伯曰:「世之議者皆曰:『至精無形,至大不可圍。』是信情乎?」北海若曰:「夫自細視大者不盡,自大視細者不明。夫精,小之微也,垺,大之殷也,故異便。此勢之有也。夫精粗者,期於有形者也;無形者,數之所不能分也;不可圍者,數之所不能窮也。可以言論者,物之粗也;可以意致者,物之精也;言之所不能論,意之所不能察致者,不期精粗焉。是故大人之行,不出乎害人,不多仁恩;動不為利,不賤門隸;貨財弗爭,不多辭讓;事焉不惜人,不多食乎力,不賤貪污;行殊乎俗,不多辟異;為在從眾,不賤佞諂;世之爵祿不足以為勸,戮恥不足以為辱;知是非之不可為分,細大之不可為倪。聞曰:『道人不聞,至德不得,大人無己,約分之至也。」

河伯曰:「若物之外,若物之內,惡至而倪貴賤?惡至而倪小大?」北海若曰:「以道觀之,物無貴賤;以物觀之,自貴而相賤:以俗觀之,貴賤不在己。以差觀之,因其所大而大之,則萬物莫不大;因其所小而小之,則萬物莫不小。知天地之為稊米也,知豪末之為丘山也,則差數等矣。以功觀之,因其所有而有之,則萬物莫不有;因其所無而無之,則萬物莫不無。知東西之相反,而不可以相無,則功分定矣。以趣觀之,因其所然而然之,則萬物莫不然;因其所非而非之,則萬物莫不非。知堯、桀之自然而相非,則趣操睹矣。昔者堯、舜讓而帝,之、噲讓而絕;湯、武爭而王,白公爭而滅。由此觀之,爭讓之禮,堯、桀之行,貴賤有時,未可以為常也。梁麗可以衝城,而不可以窒穴,言殊器也;騏驥驊騮,一日而馳千里,捕鼠不如狸狌,言殊技也;鴟鵂夜撮蚤,察毫末,晝出瞋目而不見丘山,言殊性也。故曰:蓋師是而無非,師治而無亂乎?是未明天地之理,萬物之情者也。是猶師天而無地,師陰而無陽,其不可行明矣。然且語而不舍,非愚則誣也。帝王殊禪,三代殊繼。差其時,逆其俗者,謂之篡夫;當其時,順其俗者,謂之義徒。默默乎河伯!女惡知貴賤之門,大小之家!」

河伯曰:「然則我何為乎?何不為乎?吾辭受趣舍,吾終奈何?」北海若曰:「以道觀之,何貴何賤,是謂反衍,無拘而志,與道大蹇。何少何多,是謂謝施,無一而行,與道參差。嚴乎若國之有君,其無私德;繇繇乎若祭之有社,其無私福;泛泛乎其1若四方之無窮,其無所畛域。兼懷萬物,其孰承翼?是謂無方。萬物一齊,孰短孰長?道無終始,物有死生,不恃其成;一虛一滿,不位乎其形。年不可舉,時不可止;消息盈虛,終則有始。是所以語大義之方,論萬物之理也。物之生也若驟若馳,無動而不變,無時而不移。何為乎?何不為乎?夫固將自化。」1. 其 : 刪除。

河伯曰:「然則何貴於道邪?」北海若曰:「知道者必達於理,達於理者必明於權,明於權者不以物害己。至德者,火弗能熱,水弗能溺,寒暑弗能害,禽獸弗能賊。非謂其薄之也,言察乎安危,寧於禍福,謹於去就,莫之能害也。故曰:天在內,人在外,德在乎天。知天人之行,本乎天,位乎得。蹢䠱而屈伸,反要而語極。」曰:「何謂天?何謂人?」北海若曰:「牛馬四足,是謂天;落馬首,穿牛鼻,是謂人。故曰:無以人滅天,無以故滅命,無以得殉名。謹守而勿失,是謂反其真。」

夔憐蚿,蚿憐蛇,蛇憐風,風憐目,目憐心。

夔謂蚿曰:「吾以一足趻踔而行,予無如矣。今子之使萬足,獨奈何?」蚿曰:「不然。子不見夫唾者乎?噴則大者如珠,小者如霧,雜而下者不可勝數也。今予動吾天機,而不知其所以然。」

蚿謂蛇曰:「吾以眾足行,而不及子之無足,何也?」蛇曰:「夫天機之所動,何可易邪?吾安用足哉!」

蛇謂風曰:「予動吾脊脅而行,則有似也。今子蓬蓬然起於北海,蓬蓬然入於南海,而似無有,何也?」風曰:「然。予蓬蓬然起於北海而入於南海也,然而指我則勝我,䠓我亦勝我。雖然,夫折大木,蜚大屋者,唯我能也,故以眾小不勝為大勝也。為大勝者,唯聖人能之。」

孔子遊於匡,宋人圍之數匝,而絃歌不惙。子路入見,曰:「何夫子之娛也?」孔子曰:「來!吾語女。我諱窮久矣,而不免,命也;求通久矣,而不得,時也。當堯、舜而天下無窮人,非知得也,當桀,紂而天下無通人,非知失也,時勢適然。夫水行不避蛟龍者,漁父之勇也;陸行不避兕虎者,獵夫之勇也;白刃交於前,視死若生者,烈士之勇也;知窮之有命,知通之有時,臨大難而不懼者,聖人之勇也。由處矣!吾命有所制矣。」無幾何,將甲者進,辭曰:「以為陽虎也,故圍之;今非也,請辭而退。」

公孫龍問於魏牟曰:「龍少學先生之道,長而明仁義之行,合同異,雜堅白,然不然,可不可,困百家之知,窮眾口之辯,吾自以為至達已。今吾聞莊子之言,汒焉異之,不知論之不及與,知之弗若與?今吾無所開吾喙,敢問其方。」

公子牟隱机太息,仰天而笑曰:「子獨不聞夫埳井之鼃乎?謂東海之鱉曰:『吾樂與!出跳梁乎井幹之上,入休乎缺甃之崖,赴水則接腋持頤,蹶泥則沒足滅跗,還虷蟹與科斗,莫吾能若也。且夫擅一壑之水,而跨跱埳井之樂,此亦至矣,夫子奚不時來入觀乎?』東海之鱉左足未入,而右膝已縶矣。於是逡巡而卻,告之海曰:『夫千里之遠,不足以舉其大;千仞之高,不足以極其深。禹之時,十年九潦,而水弗為加益;湯之時,八年七旱,而崖不為加損。夫不為頃久推移,不以多少進退者,此亦東海之大樂也。』於是埳井之鼃聞之,適適然驚,規規然自失也。且夫知不知是非之竟,而猶欲觀於莊子之言,是猶使蚊負山,商蚷馳河也,必不勝任矣。且夫知不知論極妙之言,而自適一時之利者,是非埳井之鼃與?且彼方跐黃泉而登大皇,無南無北,奭然四解,淪於不測;無東無西,始於玄冥,反於大通。子乃規規然而求之以察,索之以辯,是直用管窺天,用錐指地也,不亦小乎!子往矣!且子獨不聞壽陵餘子之學行於邯鄲與?未得國能,又失其故行矣,直匍匐而歸耳。今子不去,將忘子之故,失子之業。」

公孫龍口呿而不合,舌舉而不下,乃逸而走。

莊子釣於濮水,楚王使大夫二人往先焉,曰:「願以境內累矣!」莊子持竿不顧,曰:「吾聞楚有神龜,死已三千歲矣,王巾笥而藏之廟堂之上。此龜者,寧其死為留骨而貴乎,寧其生而曳尾於塗中乎?」二大夫曰:「寧生而曳尾塗中。」莊子曰:「往矣!吾將曳尾於塗中。」

惠子相梁,莊子往見之。或謂惠子曰:「莊子來,欲代子相。」於是惠子恐,搜於國中三日三夜。莊子往見之,曰:「南方有鳥,其名為鵷鶵,子知之乎?夫鵷鶵發於南海而飛於北海,非梧桐不止,非練實不食,非醴泉不飲。於是鴟得腐鼠,鵷鶵過之,仰而視之曰:『嚇!』今子欲以子之梁國而嚇我邪?」

莊子與惠子遊於濠梁之上。莊子曰:「儵魚出遊從容,是魚樂也。」惠子曰:「子非魚,安知魚之樂?」莊子曰:「子非我,安知我不知魚之樂?」惠子曰:「我非子,固不知子矣;子固非魚也,子之不知魚之樂全矣。」莊子曰:「請循其本。子曰『汝安知魚樂』云者,既已知吾知之而問我,我知之濠上也。」


\end{pinyinscope}