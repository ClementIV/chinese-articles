\article{德充符}

\begin{pinyinscope}
魯有兀者王駘,從之遊者,與仲尼相若。常季問於仲尼曰:「王駘,兀者也,從之遊者,與夫子中分魯。立不教,坐不議,虛而往,實而歸。固有不言之教,無形而心成者邪?是何人也?」仲尼曰:「夫子,聖人也。丘也,直後而未往耳。丘將以為師,而況不如丘者乎!奚假魯國!丘將引天下而與從之。」常季曰:「彼兀者也,而王先生,其與庸亦遠矣。若然者,其用心也,獨若之何?」仲尼曰:「死生亦大矣,而不得與之變,雖天地覆墜,亦將不與之遺。審乎無假,而不與物遷,命物之化,而守其宗也。」常季曰:「何謂也?」仲尼曰:「自其異者視之,肝膽楚越也;自其同者視之,萬物皆一也。夫若然者,且不知耳目之所宜,而游心於德之和,物視其所一,而不見其所喪,視喪其足,猶遺土也。」常季曰:「彼為己,以其知得其心,以其心得其常心,物何為最之哉?」仲尼曰:「人莫鑑於流水,而鑑於止水,唯止能止眾止。受命於地,唯松柏獨也在,冬夏青青;受命於天,唯舜獨也正,幸能正生,以正眾生。夫保始之徵,不懼之實。勇士一人,雄入於九軍。將求名而能自要者,而猶若此,而況官天地,府萬物,直寓六骸,象耳目,一知之所知,而心未嘗死者乎!彼且擇日而登假,人則從是也。彼且何肯以物為事乎!」

申徒嘉,兀者也,而與鄭子產同師於伯昏無人。子產謂申徒嘉曰:「我先出,則子止;子先出,則我止。」其明日,又與合堂同席而坐。子產謂申徒嘉曰:「我先出,則子止;子先出,則我止。今我將出,子可以止乎,其未邪?且子見執政而不違,子齊執政乎?」申徒嘉曰:「先生之門,固有執政焉如此哉?子而說子之執政而後人者也!聞之曰:『鑑明則塵垢不止,止則不明也。久與賢人處,則無過。』今子之所取大者,先生也,而猶出言若是,不亦過乎!」子產曰:「子既若是矣,猶與堯爭善,計子之德不足以自反邪?」申徒嘉曰:「自狀其過以不當亡者眾,不狀其過以不當存者寡。知不可奈何而安之若命,惟有德者能之。遊於羿之彀中,中央者,中地也,然而不中者,命也。人以其全足笑吾不全足者多矣。我怫然而怒,而適先生之所,則廢然而反。不知先生之洗我以善邪!吾與夫子遊十九年矣,而未嘗知吾兀者也。今子與我遊於形骸之內,而子索我於形骸之外,不亦過乎!」子產蹴然改容更貌曰:「子無乃稱!」

魯有兀者叔山無趾,踵見仲尼。仲尼曰:「子不謹,前既犯患若是矣。雖今來,何及矣?」無趾曰:「吾唯不知務而輕用吾身,吾是以亡足。今吾來也,猶有尊足者存,吾是以務全之也。夫天無不覆,地無不載,吾以夫子為天地,安知夫子之猶若是也!」孔子曰:「丘則陋矣。夫子胡不入乎?請講以所聞!」無趾出。孔子曰:「弟子勉之!夫無趾,兀者也,猶務學以復補前行之惡,而況全德之人乎!」無趾語老聃曰:「孔丘之於至人,其未邪!彼何賓賓以學子為?彼且蘄以諔詭幻怪之名聞,不知至人之以是為己桎梏邪?」老聃曰:「胡不直使彼以死生為一條,以可不可為一貫者,解其桎梏,其可乎?」無趾曰:「天刑之,安可解?」

魯哀公問於仲尼曰:「衛有惡人焉,曰哀駘它。丈夫與之處者,思而不能去也。婦人見之,請於父母曰『與為人妻,寧為夫子妾』者,十數而未止也。未嘗有聞其唱者也,常和而已矣。無君人之位以濟乎人之死,無聚祿以望人之腹。又以惡駭天下,和而不唱,知不出乎四域,且而雌雄合乎前。是必有異乎人者也。寡人召而觀之,果以惡駭天下。與寡人處,不至以月數,而寡人有意乎其為人也;不至乎期年,而寡人信之。國無宰,寡人傳國焉。悶然而後應,氾而若辭。寡人醜乎,卒授之國。無幾何也,去寡人而行,寡人卹焉若有亡也,若無與樂是國也。是何人者也?」仲尼曰:「丘也,嘗使於楚矣,適見㹠子食於其死母者,少焉眴若,皆棄之而走。不見己焉爾,不得類焉爾。所愛其母者,非愛其形也,愛使其形者也。戰而死者,其人之葬也,不以翣資,刖者之屨,無為愛之,皆無其本矣。為天子之諸御,不爪翦,不穿耳;娶妻者止於外,不得復使。形全猶足以為爾,而況全德之人乎!今哀駘它未言而信,無功而親,使人授己國,唯恐其不受也,是必才全而德不形者也。」哀公曰:「何謂才全?」仲尼曰:「死生存亡,窮達貧富,賢與不肖,毀譽、饑渴、寒暑,是事之變,命之行也;日夜相代乎前,而知不能規乎其始者也。故不足以滑和,不可入於靈府。使之和豫通而不失於兌,使日夜無郤而與物為春,是接而生時於心者也。是之謂才全。」「何謂德不形?」曰:「平者,水停之盛也。其可以為法也,內保之而外不蕩也。德者,成和之修也。德不形者,物不能離也。」哀公異日以告閔子曰:「始也,吾以南面而君天下,執民之紀,而憂其死,吾自以為至通矣。今吾聞至人之言,恐吾無其實,輕用吾身而亡其國。吾與孔丘,非君臣也,德友而已矣。」

闉跂支離無脤說衛靈公,靈公說之,而視全人,其脰肩肩。甕盎大癭說齊桓公,桓公說之,而視全人,其脰肩肩。故德有所長,而形有所忘,人不忘其所忘,而忘其所不忘,此謂誠忘。故聖人有所遊,而知為孽,約為膠,德為接,工為商。聖人不謀,惡用知?不斲,惡用膠?無喪,惡用德?不貨,惡用商?四者,天鬻也。天鬻者,天食也。既受食於天,又惡用人?有人之形,無人之情。有人之形,故群於人;無人之情,故是非不得於身。眇乎小哉!所以屬於人也。謷乎大哉!獨成其天。

惠子謂莊子曰:「人故無情乎?」莊子曰:「然。」惠子曰:「人而無情,何以謂之人?」莊子曰:「道與之貌,天與之形,惡得不謂之人?」惠子曰:「既謂之人,惡得無情?」莊子曰:「是非吾所謂情也。吾所謂無情者,言人之不以好惡內傷其身,常因自然而不益生也。」惠子曰:「不益生,何以有其身?」莊子曰:「道與之貌,天與之形,無以好惡內傷其身。今子外乎子之神,勞乎子之精,倚樹而吟,據槁梧而瞑。天選子之形,子以堅白鳴!」


\end{pinyinscope}