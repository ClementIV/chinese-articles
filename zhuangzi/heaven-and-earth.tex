\article{天地}

\begin{pinyinscope}
天地雖大,其化均也;萬物雖多,其治一也;人卒雖眾,其主君也。君原於德而成於天,故曰:玄古之君天下,無為也,天德而已矣。以道觀言而天下之君正,以道觀分而君臣之義明,以道觀能而天下之官治,以道汎觀而萬物之應備。故通於天地者,德也;行於萬物者,道也;上治人者,事也;能有所藝者,技也。技兼於事,事兼於義,義兼於德,德兼於道,道兼於天。故曰:「古之畜天下者,無欲而天下足,無為而萬物化,淵靜而百姓定。」記曰:「通於一而萬事畢,無心得而鬼神服。」

夫子曰:「夫道,覆載萬物者也,洋洋乎大哉!君子不可以不刳心焉。無為為之之謂天,無為言之之謂德,愛人利物之謂仁,不同同之之謂大,行不崖異之謂寬,有萬不同之謂富。故執德之謂紀,德成之謂立,循於道之謂備,不以物挫志之謂完。君子明於此十者,則韜乎其事心之大也,沛乎其為萬物逝也。若然者,藏金於山,藏珠於淵;不利貨財,不近貴富;不樂壽,不哀夭;不榮通,不醜窮;壽夭俱忘,窮通不足言矣。不拘一世之利以為己私分,不以王天下為己處顯。顯則明,萬物一府,死生同狀。」

夫子曰:「夫道,淵乎其居也,漻乎其清也。金石不得,無以鳴。故金石有聲,不考不鳴。萬物孰能定之!夫王德之人,素逝而恥通於事,立之本原而知通於神。故其德廣,其心之出,有物採之。故形非道不生,生非德不明。存形窮生,立德明道,非王德者邪!蕩蕩乎!忽然出,勃然動,而萬物從之乎!此謂王德之人。視乎冥冥,聽乎無聲。冥冥之中,獨見曉焉;無聲之中,獨聞和焉。故深之又深,而能物焉;神之又神,而能精焉。故其與萬物接也,至無而供其求,時騁而要其宿,大小、長短、修遠。」

黃帝遊乎赤水之北,登乎崑崙之丘而南望,還歸,遺其玄珠,使知索之而不得,使離朱索之而不得,使喫詬索之而不得也。乃使象罔,象罔得之。黃帝曰:「異哉!象罔乃可以得之乎?」

堯之師曰許由,許由之師曰齧缺,齧缺之師曰王倪,王倪之師曰被衣。堯問於許由曰:「齧缺可以配天乎?吾藉王倪以要之。」許由曰:「殆哉圾乎天下!齧缺之為人也,聰明叡知,給數以敏,其性過人,而又乃以人受天。彼審乎禁過,而不知過之所由生。與之配天乎?彼且乘人而無天,方且本身而異形,方且尊知而火馳,方且為緒使,方且為物絯,方且四顧而物應,方且應眾宜,方且與物化而未始有恒。夫何足以配天乎?雖然,有族有祖,可以為眾父,而不可以為眾父父。治亂之率也,北面之禍也,南面之賊也。」

堯觀乎華。華封人曰:「嘻!聖人!請祝聖人:使聖人壽。」堯曰:「辭。」「使聖人富」。堯曰:「辭。」「使聖人多男子」。堯曰:「辭。」封人曰:「壽、富、多男子,人之所欲也。女獨不欲,何邪?」堯曰:「多男子則多懼,富則多事,壽則多辱。是三者,非所以養德也,故辭。」封人曰:「始也我以女為聖人邪,今然君子也。天生萬民,必授之職,多男子而授之職,則何懼之有!富而使人分之,則何事之有!夫聖人鶉居而鷇食,鳥行而無彰;天下有道則與物皆昌,天下無道則修德就閒;千歲厭世,去而上僊,乘彼白雲,至於帝鄉。三患莫至,身常無殃,則何辱之有!」封人去之,堯隨之,曰:「請問。」封人曰:「退已!」

堯治天下,伯成子高立為諸侯。堯授舜,舜授禹,伯成子高辭為諸侯而耕。禹往見之,則耕在野。禹趨就下風,立而問焉,曰:「昔堯治天下,吾子立為諸侯;堯授舜,舜授予,而吾子辭為諸侯而耕。敢問其故何也?」子高曰:「昔堯治天下,不賞而民勸,不罰而民畏。今子賞罰而民且不仁,德自此衰,刑自此立,後世之亂自此始矣。夫子闔行邪?無落吾事!」俋俋乎耕而不顧。

泰初有無,無有無名,一之所起,有一而未形。物得以生,謂之德;未形者有分,且然無間,謂之命;留動而生物,物成生理,謂之形;形體保神,各有儀則,謂之性。性修反德,德至同於初。同乃虛,虛乃大。合喙鳴,喙鳴合,與天地為合。其合緡緡,若愚若昏,是謂玄德,同乎大順。

夫子問於老聃曰:「有人治道若相放,可不可,然不然。辯者有言曰:『離堅白若縣宇。』若是,則可謂聖人乎?」老聃曰:「是胥易技係,勞形怵心者也。執留之狗成思,猿狙之便自山林來。丘!予告若,而所不能聞與而所不能言。凡有首、有趾、無心、無耳者眾,有形者與無形無狀而皆存者盡無。其動,止也;其死,生也;其廢,起也。此又非其所以也。有治在人,忘乎物,忘乎天,其名為忘己。忘己之人,是之謂入於天。」

將閭葂見季徹曰:「魯君謂葂也曰:『請受教。』辭不獲命,既已告矣,未知中否,請嘗薦之。吾謂魯君曰:『必服恭儉,拔出公忠之屬,而無阿私,民孰敢不輯!』」季徹局局然笑曰:「若夫子之言,於帝王之德,猶螳蜋之怒臂以當車軼,則必不勝任矣。且若是,則其自為處危,其觀臺多物,將往投跡者眾。」將閭葂覤覤然驚曰:「葂也汒若於夫子之所言矣。雖然,願先生之言其風也。」季徹曰:「大聖之治天下也,搖蕩民心,使之成教易俗,舉滅其賊心而皆進其獨志,若性之自為,而民不知其所由然。若然者,豈兄堯、舜之教民,溟滓然弟之哉?欲同乎德而心居矣。」

子貢南遊於楚,反於晉,過漢陰,見一丈人方將為圃畦,鑿隧而入井,抱甕而出灌,搰搰然用力甚多而見功寡。子貢曰:「有械於此,一日浸百畦,用力甚寡而見功多,夫子不欲乎?」為圃者卬而視之曰:「奈何?」曰:「鑿木為機,後重前輕,挈水若抽,數如泆湯,其名為槔。」為圃者忿然作色而笑曰:「吾聞之吾師:『有機械者必有機事,有機事者必有機心。』機心存於胸中,則純白不備;純白不備,則神生不定;神生不定者,道之所不載也。吾非不知,羞而不為也。」子貢瞞然慙,俯而不對。

有間,為圃者曰:「子奚為者邪?」曰:「孔丘之徒也。」為圃者曰:「子非夫博學以擬聖,於于以蓋眾,獨弦哀歌以賣名聲於天下者乎?汝方將忘汝神氣,墮汝形骸,而庶幾乎!而身之不能治,而何暇治天下乎?子往矣,無乏吾事!

子貢卑陬失色,頊頊然不自得,行三十里而後愈。其弟子曰:「向之人何為者邪?夫子何故見之變容失色,終日不自反邪?」曰:「始以為天下一人耳,不知復有夫人也。吾聞之夫子:『事求可、功求成、用力少、見功多者,聖人之道。』今徒不然。執道者德全,德全者形全,形全者神全。神全者,聖人之道也。託生與民並行,而不知其所之,汒乎淳備哉!功利、機巧,必忘夫人之心。若夫人者,非其志不之,非其心不為。雖以天下譽之,得其所謂,謷然不顧;以天下非之,失其所謂,儻然不受。天下之非譽,無益損焉,是謂全德之人哉!我之謂風波之民。」反於魯,以告孔子。孔子曰:「彼假修渾沌氏之術者也:識其一,不知其二;治其內,而不治其外。夫明白入素,無為復朴,體性抱神,以遊世俗之間者,汝將固驚邪?且渾沌氏之術,予與汝何足以識之哉!」

諄芒將東之大壑,適遇苑風於東海之濱。苑風曰:「子將奚之?」曰:「將之大壑。」曰:「奚為焉?」曰:「夫大壑之為物也,注焉而不滿,酌焉而不竭,吾將遊焉。」苑風曰:「夫子無意於橫目之民乎?願聞聖治。」諄芒曰:「聖治乎,官施而不失其宜,拔舉而不失其能,畢見其情事而行其所為,行言自為而天下化,手撓顧指,四方之民莫不俱至,此之謂聖治。」「願聞德人。」曰:「德人者,居無思,行無慮,不藏是非美惡。四海之內,共利之之謂悅,共給之之謂安;怊乎若嬰兒之失其母也,儻乎若行而失其道也。財用有餘而不知其所自來,飲食取足而不知其所從。此謂德人之容。」「願聞神人。」曰:「上神乘光,與形滅亡,此謂照曠。天地樂而萬事銷亡,萬物復情,此之謂混冥。」

門無鬼與赤張滿稽,觀於武王之師。赤張滿稽曰:「不及有虞氏乎!故離此患也。」門無鬼曰:「天下均治而有虞氏治之邪,其亂而後治之與?」赤張滿稽曰:「天下均治之為願,而何計以有虞氏為?有虞氏之藥瘍也,禿而施髢,病而求醫。孝子操藥以修慈父,其色燋然,聖人羞之。至德之世,不尚賢,不使能;上如標枝,民如野鹿;端正而不知以為義,相愛而不知以為仁;實而不知以為忠,當而不知以為信;蠢動而相使,不以為賜。是故行而無迹,事而無傳。」

孝子不諛其親,忠臣不諂其君,臣子之盛也。親之所言而然,所行而善,則世俗謂之不肖子;君之所言而然,所行而善,則世俗謂之不肖臣。而未知此其必然邪!世俗之所謂然而然之,所謂善而善之,則不謂之道諛之人也。然則俗固嚴於親而尊於君邪!謂己道人,則勃然作色;謂己諛人,則怫然作色。而終身道人也,終身諛人也,合譬飾辭聚眾也,是始終本末不相坐。垂衣裳,設采色,動容貌,以媚一世,而不自謂道諛,與夫人之為徒,通是非,而不自謂眾人,愚之至也。知其愚者,非大愚也;知其惑者,非大惑也。大惑者,終身不解;大愚者,終身不靈。三人行而一人惑,所適者猶可致也,惑者少也;二人惑則勞而不至,惑者勝也。而今也以天下惑,予雖有祈嚮,不可得也。不亦悲乎!

大聲不入於里耳,《折楊》、《皇荂》,則嗑然而笑。是故高言不止於眾人之心,至言不出,俗言勝也。以二缶鍾惑,而所適不得矣。而今也以天下惑,予雖有祈嚮,其庸可得邪?知其不可得也而強之,又一惑也,故莫若釋之而不推。不推,誰其比憂!厲之人夜半生其子,遽取火而視之,汲汲然惟恐其似己也。

百年之木,破為犧尊,青黃而文之,其斷在溝中。比犧尊於溝中之斷,則美惡有間矣,其於失性一也。跖與曾、史,行義有間矣,然其失性均也。且夫失性有五:一曰五色亂目,使目不明;二曰五聲亂耳,使耳不聰;三曰五臭薰鼻,困惾中顙;四曰五味濁口,使口厲爽;五曰趣舍滑心,使性飛揚。此五者,皆生之害也。而楊、墨乃始離跂自以為得,非吾所謂得也。夫得者困,可以為得乎?則鳩鴞之在於籠也,亦可以為得矣。且夫趣舍聲色以柴其內,皮弁、鷸冠、搢笏、紳修以約其外,內支盈於柴柵,外重纆繳,睆睆然在纆繳之中而自以為得,則是罪人交臂、歷指,而虎豹在於囊檻,亦可以為得矣。


\end{pinyinscope}