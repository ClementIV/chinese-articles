\article{人間世}

\begin{pinyinscope}
顏回見仲尼請行。曰:「奚之?」曰:「將之衛。」曰:「奚為焉?」曰:「回聞衛君,其年壯,其行獨,輕用其國,而不見其過,輕用民死,死者以國量乎澤,若蕉,民其无如矣。回嘗聞之夫子曰:『治國去之,亂國就之,醫門多疾。』願以所聞思其則,庶幾其國有瘳乎!」仲尼曰:「譆!若殆往而刑耳!夫道不欲雜,雜則多,多則擾,擾則憂,憂而不救。古之至人,先存諸己,而後存諸人。所存於己者未定,何暇至於暴人之所行!且若亦知夫德之所蕩,而知之所為出乎哉?德蕩乎名,知出乎爭。名也者,相軋也;知也者,爭之器也。二者凶器,非所以盡行也。且德厚信矼,未達人氣;名聞不爭,未達人心。而彊以仁義繩墨之言術暴人之前者,是以人惡有其美也,命之曰菑人。菑人者,人必反菑之,若殆為人菑夫!且苟為悅賢而惡不肖,惡用而求有以異?若唯无詔,王公必將乘人而鬭其捷。而目將熒之,而色將平之,口將營之,容將形之,心且成之。是以火救火,以水救水,名之曰益多,順始无窮。若殆以不信厚言,必死於暴人之前矣。且昔者桀殺關龍逢,紂殺王子比干,是皆脩其身以下傴拊人之民,以下拂其上者也,故其君因其脩以擠之。是好名者也。昔者堯攻叢枝、胥敖,禹攻有扈,國為虛厲,身為刑戮,其用兵不止,其求實无已。是皆求名、實者也,而獨不聞之乎?名、實者,聖人之所不能勝也,而況若乎!雖然,若必有以也,嘗以語我來!」顏回曰:「端而虛,勉而一,則可乎?」曰:「惡!惡可?夫以陽為充孔揚,采色不定,常人之所不違,因案人之所感,以求容與其心。名之曰日漸之德不成,而況大德乎!將執而不化,外合而內不訾,其庸詎可乎!」「然則我內直而外曲,成而上比。內直者,與天為徒。與天為徒者,知天子之與己皆天之所子,而獨以己言蘄乎而人善之,蘄乎而人不善之邪?若然者,人謂之童子,是之謂與天為徒。外曲者,與人之為徒也。擎、跽、曲拳,人臣之禮也,人皆為之,吾敢不為邪!為人之所為者,人亦无疵焉,是之謂與人為徒。成而上比者,與古為徒。其言雖教,讁之實也。古之有也,非吾有也。若然者,雖直不為病,是之謂與古為徒。若是,則可乎?」仲尼曰:「惡!惡可?大多政,法而不諜,雖固,亦无罪。雖然,止是耳矣,夫胡可以及化!猶師心者也。」

顏回曰:「吾无以進矣,敢問其方。」仲尼曰:「齋,吾將語若!有而為之,其易邪?易之者,皞天不宜。」顏回曰:「回之家貧,唯不飲酒、不茹葷者數月矣。若此,則可以為齋乎?」曰:「是祭祀之齋,非心齋也。」回曰:「敢問心齋。」仲尼曰:「若一志,无聽之以耳而聽之以心,无聽之以心而聽之以氣。聽止於耳,心止於符。氣也者,虛而待物者也。唯道集虛。虛者,心齋也。」顏回曰:「回之未始得使,實自回也;得使之也,未始有回也。可謂虛乎?」夫子曰:「盡矣。吾語若!若能入遊其樊而无感其名,入則鳴,不入則止。无門无毒,一宅而寓於不得已,則幾矣。絕迹易,无行地難。為人使,易以偽;為天使,難以偽。聞以有翼飛者矣,未聞以无翼飛者也;聞以有知知者矣,未聞以无知知者也。瞻彼闋者,虛室生白,吉祥止止。夫且不止,是之謂坐馳。夫徇耳目內通而外於心知,鬼神將來舍,而況人乎!是萬物之化也,禹、舜之所紐也,伏戲、几蘧之所行終,而況散焉者乎!」

葉公子高將使於齊,問於仲尼曰:「王使諸梁也甚重,齊之待使者,蓋將甚敬而不急。匹夫猶未可動,而況諸侯乎!吾甚慄之。子常語諸梁也,曰:『凡事若小若大,寡不道以懽成。事若不成,則必有人道之患;事若成,則必有陰陽之患。若成若不成而後無患者,唯有德者能之。』吾食也,執粗而不臧,爨無欲清之人。今吾朝受命而夕飲冰,我其內熱與!吾未至乎事之情,而既有陰陽之患矣;事若不成,必有人道之患。是兩也,為人臣者不足以任之,子其有以語我來!」仲尼曰:「天下有大戒二:其一,命也;其一,義也。子之愛親,命也,不可解於心;臣之事君,義也,無適而非君也,無所逃於天地之間。是之謂大戒。是以夫事其親者,不擇地而安之,孝之至也;夫事其君者,不擇事而安之,忠之盛也;自事其心者,哀樂不易施乎前,知其不可奈何而安之若命,德之至也。為人臣子者,固有所不得已,行事之情而忘其身,何暇至於悅生而惡死!夫子其行可矣!丘請復以所聞:凡交,近則必相靡以信,遠則必忠之以言,言必或傳之。夫傳兩喜兩怒之言,天下之難者也。夫兩喜必多溢美之言,兩怒必多溢惡之言。凡溢之類妄,妄則其信之也莫,莫則傳言者殃。故法言曰:『傳其常情,無傳其溢言,則幾乎全。』且以巧鬥力者,始乎陽,常卒乎陰,大至則多奇巧;以禮飲酒者,始乎治,常卒乎亂,大至則多奇樂。凡事亦然。始乎諒,常卒乎鄙;其作始也簡,其將畢也必巨。夫言者,風波也;行者,實喪也。風波易以動,實喪易以危。故忿設無由,巧言偏辭。獸死不擇音,氣息茀然,於是並生心厲。剋核大至,則必有不肖之心應之,而不知其然也。苟為不知其然也,孰知其所終!故法言曰:『無遷令,無勸成。』過度,益也。遷令、勸成殆事,美成在久,惡成不及改,可不慎與!且夫乘物以遊心,託不得已以養中,至矣。何作為報也!莫若為致命。此其難者。」

顏闔將傅衛靈公大子,而問於蘧伯玉曰:「有人於此,其德天殺。與之為無方,則危吾國;與之為有方,則危吾身。其知適足以知人之過,而不知其所以過。若然者,吾奈之何?」蘧伯玉曰:「善哉問乎!戒之慎之,正汝身也哉!形莫若就,心莫若和。雖然,之二者有患。就不欲入,和不欲出。形就而入,且為顛為滅,為崩為蹶。心和而出,且為聲為名,為妖為孽。彼且為嬰兒,亦與之為嬰兒;彼且為無町畦,亦與之為無町畦;彼且為無崖,亦與之為無崖。達之,入於無疵。汝不知夫螳蜋乎?怒其臂以當車轍,不知其不勝任也,是其才之美者也。戒之慎之!積伐而美者以犯之,幾矣。汝不知夫養虎者乎?不敢以生物與之,為其殺之之怒也;不敢以全物與之,為其決之之怒也。時其飢飽,達其怒心。虎之與人異類而媚養己者,順也;故其殺者,逆也。夫愛馬者,以筐盛矢,以蜄盛溺。適有蚉虻僕緣,而拊之不時,則缺銜、毀首、碎胸。意有所至,而愛有所亡,可不慎邪!」

匠石之齊,至乎曲轅,見櫟社樹。其大蔽數千牛,絜之百圍,其高臨山十仞而後有枝,其可以為舟者旁十數。觀者如市,匠伯不顧,遂行不輟。弟子厭觀之,走及匠石,曰:「自吾執斧斤以隨夫子,未嘗見材如此其美也。先生不肯視,行不輟,何邪?」曰:「已矣,勿言之矣!散木也,以為舟則沈,以為棺槨則速腐,以為器則速毀,以為門戶則液樠,以為柱則蠹。是不材之木也,無所可用,故能若是之壽。」匠石歸,櫟社見夢曰:「女將惡乎比予哉?若將比予於文木邪?夫柤、梨、橘、柚、果、蓏之屬,實熟則剝,剝則辱,大枝折,小枝泄。此以其能苦其生者也,故不終其天年而中道夭,自掊擊於世俗者也。物莫不若是。且予求無所可用久矣,幾死,乃今得之,為予大用。使予也而有用,且得有此大也邪?且也,若與予也皆物也,奈何哉其相物也?而幾死之散人,又惡知散木!」匠石覺而診其夢。弟子曰:「趣取無用,則為社何邪?」曰:「密!若無言!彼亦直寄焉,以為不知己者詬厲也。不為社者,且幾有翦乎!且也,彼其所保,與眾異,以義譽之,不亦遠乎!」

南伯子綦遊乎商之丘,見大木焉有異,結駟千乘,隱將芘其所藾。子綦曰:「此何木也哉?此必有異材夫!」仰而視其細枝,則拳曲而不可以為棟梁;俯而見其大根,則軸解而不可為棺槨;咶其葉,則口爛而為傷;嗅之,則使人狂酲三日而不已。子綦曰:「此果不材之木也,以至於此其大也。嗟乎!神人以此不材!」宋有荊氏者,宜楸、柏、桑。其拱把而上者,求狙猴之杙者斬之;三圍四圍,求高名之麗者斬之;七圍八圍,貴人富商之家求樿傍者斬之。故未終其天年,而中道已夭於斧斤,此材之患也。故解之以牛之白顙者,與豚之亢鼻者,與人有痔病者,不可以適河。此皆巫祝以知之矣,所以為不祥也,此乃神人之所以為大祥也。

支離疏者,頤隱於臍,肩高於頂,會撮指天,五管在上,兩髀為脅。挫鍼治繲,足以餬口;鼓筴播精,足以食十人。上徵武士,則支離攘臂而遊於其間;上有大役,則支離以有常疾不受功;上與病者粟,則受三鐘與十束薪。夫支離其形者,猶足以養其身,終其天年,又況支離其德者乎!」

孔子適楚,楚狂接輿遊其門曰:「鳳兮鳳兮,何如德之衰也!來世不可待,往世不可追也。天下有道,聖人成焉;天下無道,聖人生焉。方今之時,僅免刑焉。福輕乎羽,莫之知載;禍重乎地,莫之知避。已乎已乎,臨人以德!殆乎殆乎,畫地而趨!迷陽迷陽,無傷吾行!吾行卻曲,無傷吾足!」

山木自寇也,膏火自煎也。桂可食,故伐之;漆可用,故割之。人皆知有用之用,而莫知無用之用也。


\end{pinyinscope}