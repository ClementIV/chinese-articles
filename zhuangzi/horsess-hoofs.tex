\article{馬蹄}

\begin{pinyinscope}
馬,蹄可以踐霜雪,毛可以禦風寒,齕草飲水,翹足而陸。此馬之真性也。雖有義臺、路寢,無所用之。及至伯樂,曰:「我善治馬。」燒之剔之,刻之雒之,連之以羈馽,編之以皁棧,馬之死者十二三矣;飢之渴之,馳之驟之,整之齊之,前有橛飾之患,而後有鞭筴之威,而馬之死者已過半矣。陶者曰:「我善治埴,圓者中規,方者中矩。」匠人曰:「我善治木,曲者中鉤,直者應繩。」夫埴、木之性,豈欲中規矩鉤繩哉?然且世世稱之曰:「伯樂善治馬,而陶、匠善治埴木。」此亦治天下者之過也。

吾意善治天下者不然。彼民有常性,織而衣,耕而食,是謂同德;一而不黨,命曰天放。故至德之世,其行填填,其視顛顛。當是時也,山無蹊隧,澤無舟梁;萬物群生,連屬其鄉;禽獸成群,草木遂長。是故禽獸可係羈而遊,烏鵲之巢可攀援而闚。夫至德之世,同與禽獸居,族與萬物並,惡乎知君子小人哉!同乎無知,其德不離;同乎無欲,是謂素樸。素樸而民性得矣。及至聖人,蹩躠為仁,踶跂為義,而天下始疑矣;澶漫為樂,摘僻為禮,而天下始分矣。故純樸不殘,孰為犧尊!白玉不毀,孰為珪璋!道德不廢,安取仁義!性情不離,安用禮樂!五色不亂,孰為文采!五聲不亂,孰應六律!夫殘樸以為器,工匠之罪也;毀道德以為仁義,聖人之過也。

夫馬,陸居則食草飲水,喜則交頸相靡,怒則分背相踶。馬知已此矣。夫加之以衡扼,齊之以月題,而馬知介倪、闉扼、鷙曼、詭銜、竊轡。故馬之知而態至盜者,伯樂之罪也。夫赫胥氏之時,民居不知所為,行不知所之,含哺而熙,鼓腹而遊,民能以此矣。及至聖人,屈折禮樂以匡天下之形,縣跂仁義以慰天下之心,而民乃始踶跂好知,爭歸於利,不可止也。此亦聖人之過也。


\end{pinyinscope}