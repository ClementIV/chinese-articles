\article{列御寇}

\begin{pinyinscope}
列御寇之齊,中道而反,遇伯昏瞀人。伯昏瞀人曰:「奚方而反?」曰:「吾驚焉。」曰:「惡乎驚?」曰:「吾嘗食於十漿,而五漿先饋。」伯昏瞀人曰:「若是,則汝何為驚已?」曰:「夫內誠不解,形諜成光,以外鎮人心,使人輕乎貴老,而齏其所患。夫漿特為食羹之貨,多餘之贏,其為利也薄,其為權也輕,而猶若是,而況於萬乘之主乎!身勞於國而知盡於事,彼將任我以事而效我以功,吾是以驚。」伯昏瞀人曰:「善哉觀乎!汝處已,人將保汝矣。」

無幾何而往,則戶外之屨滿矣。伯昏瞀人北面而立,敦杖蹙之乎頤,立有間,不言而出。賓者以告列子,列子提屨,跣而走,暨乎門,曰:「先生既來,曾不發藥乎?」曰:「已矣!吾固告汝曰『人將保汝』,果保汝矣。非汝能使人保汝,而汝不能使人無保汝也,而焉用之感豫出異也!必且有感,搖而本才,又無謂也。與汝遊者,又莫汝告也,彼所小言,盡人毒也。莫覺莫悟,何相孰也!巧者勞而知者憂,無能者無所求,飽食而敖遊,汎若不繫之舟,虛而敖遊者也。」

鄭人緩也呻吟裘氏之地。祗三年而緩為儒,潤河九里,澤及三族,使其弟墨。儒、墨相與辯,其父助翟。十年而緩自殺。其父夢之,曰:「使而子為墨者,予也。闔胡嘗視其良,既為秋柏之實矣!」夫造物者之報人也,不報其人而報其人之天。彼故使彼。夫人以己為有以異於人,以賤其親,齊人之井,飲者相捽也。故曰:「今之世皆緩也。」自是,有德者以不知也,而況有道者乎!古者謂之遁天之刑。

聖人安其所安,不安其所不安;眾人安其所不安,不安其所安。

莊子曰:「知道易,勿言難。知而不言,所以之天也;知而言之,所以之人也。古之人,天而不人。」

朱泙漫學屠龍於支離益,單千金之家,三年技成,而無所用其巧。

聖人以必不必,故無兵;眾人以不必必之,故多兵。順於兵,故行有求。兵,恃之則亡。

小夫之知,不離苞苴竿牘,敝精神乎蹇淺,而欲兼濟道物,太一形虛。若是者,迷惑於宇宙,形累不知太初。彼至人者,歸精神乎無始,而甘冥乎無何有之鄉。水流乎無形,發泄乎太清。悲哉乎!汝為知在毫毛,而不知大寧!

宋人有曹商者,為宋王使秦。其往也,得車數乘;王說之,益車百乘。反於宋,見莊子曰:「夫處窮閭阨巷,困窘織屨,槁項黃馘者,商之所短也;一悟萬乘之主,而從車百乘者,商之所長也。」莊子曰:「秦王有病召醫,破癰潰痤者得車一乘,舐痔者得車五乘,所治愈下,得車愈多。子豈治其痔邪?何得車之多也?子行矣!」

魯哀公問於顏闔曰:「吾以仲尼為貞幹,國其有瘳乎?」曰:「殆哉圾乎!仲尼方且飾羽而畫,從事華辭,以支為旨,忍性以視民而不知不信,受乎心,宰乎神,夫何足以上民!彼宜女與?予頤與?誤而可矣。今使民離實學偽,非所以視民也。為後世慮,不若休之,難治也。」

施於人而不忘,非天布也。商賈不齒,雖以事齒之,神者勿齒。為外刑者,金與木也;為內刑者,動與過也。宵人之離外刑者,金木訊之;離內刑者,陰陽食之。夫免乎外內之刑者,唯真人能之。

孔子曰:「凡人心險於山川,難於知天。天猶有春秋冬夏旦暮之期,人者厚貌深情。故有貌愿而益,有長若不肖,有順懁而達,有堅而縵,有緩而釬。故其就義若渴者,其去義若熱。故君子遠使之而觀其忠,近使之而觀其敬,煩使之而觀其能,卒然問焉而觀其知,急與之期而觀其信,委之以財而觀其仁,告之以危而觀其節,醉之以酒而觀其側,雜之以處而觀其色。九徵至,不肖人得矣。

正考父一命而傴,再命而僂,三命而俯,循牆而走,孰敢不軌!如而夫者,一命而呂鉅,再命而於車上舞,三命而名諸父,孰協唐、許!

賊莫大乎德有心而心有眼1,及其有眼2也而內視,內視而敗矣。凶德有五,中德為首。何謂中德?中德也者,有以自好也而吡其所不為者也。

窮有八極,達有三必,形有六府。美、髯、長、大、壯、麗、勇、敢,八者俱過人也,因以是窮。緣循、偃佒、困畏不若人,三者俱通達。知慧外通,勇動多怨,仁義多責。達生之情者傀,達於知者肖;達大命者隨,達小命者遭。1. 眼 : 或作「睫」。《四部叢刊》本作「睫」。2. 眼 : 或作「睫」。《四部叢刊》本作「睫」。

人有見宋王者,錫車十乘,以其十乘驕稚莊子。莊子曰:「河上有家貧恃緯蕭而食者,其子沒於淵,得千金之珠。其父謂其子曰『取石來鍛之!夫千金之珠,必在九重之淵而驪龍頷下,子能得珠者,必遭其睡也。使驪龍而寤,子尚奚微之有哉!』今宋國之深,非直九重之淵也;宋王之猛,非直驪龍也。子能得車者,必遭其睡也。使宋王而寤,子為齏粉夫!」

或聘於莊子,莊子應其使曰:「子見夫犧牛乎?衣以文繡,食以芻叔,及其牽而入於太廟,雖欲為孤犢,其可得乎!」

莊子將死,弟子欲厚葬之。莊子曰:「吾以天地為棺槨,以日月為連璧,星辰為珠璣,萬物為齎送。吾葬具豈不備邪?何以加此!」弟子曰:「吾恐烏鳶之食夫子也。」莊子曰:「在上為烏鳶食,在下為螻蟻食,奪彼與此,何其偏也!」

以不平平,其平也不平;以不徵徵,其徵也不徵。明者唯為之使,神者徵之。夫明之不勝神也久矣,而愚者恃其所見入於人,其功外也,不亦悲乎!


\end{pinyinscope}