\article{胠篋}

\begin{pinyinscope}
將為胠篋、探囊、發匱之盜而為守備,則必攝緘、縢,固扃、鐍,此世俗之所謂知也。然而巨盜至,則負匱、揭篋、擔囊而趨,唯恐緘、縢、扃、鐍之不固也。然則鄉之所謂知者,不乃為大盜積者也?故嘗試論之,世俗之所謂知者,有不為大盜積者乎?所謂聖者,有不為大盜守者乎?何以知其然邪?昔者齊國鄰邑相望,雞狗之音相聞,罔罟之所布,耒耨之所刺,方二千餘里。闔四竟之內,所以立宗廟社稷,治邑、屋、州、閭、鄉曲者,曷嘗不法聖人哉!然而田成子一旦殺齊君而盜其國。所盜者豈獨其國邪?並與其聖知之法而盜之。故田成子有乎盜賊之名,而身處堯、舜之安,小國不敢非,大國不敢誅,十二世有齊國。則是不乃竊齊國,並與其聖知之法,以守其盜賊之身乎?嘗試論之,世俗之所謂至知者,有不為大盜積者乎?所謂至聖者,有不為大盜守者乎?何以知其然邪?昔者龍逢斬,比干剖,萇弘胣,子胥靡,故四子之賢而身不免乎戮。故盜跖之徒問於跖曰:「盜亦有道乎?」跖曰:「何適而無有道邪?夫妄意室中之藏,聖也;入先,勇也;出後,義也;知可否,知也;分均,仁也。五者不備而能成大盜者,天下未之有也。」由是觀之,善人不得聖人之道不立,跖不得聖人之道不行;天下之善人少而不善人多,則聖人之利天下也少而害天下也多。

故曰:「脣竭則齒寒,魯酒薄而邯鄲圍,聖人生而大盜起。」掊擊聖人,縱舍盜賊,而天下始治矣。夫川竭而谷虛,丘夷而淵實。聖人已死,則大盜不起,天下平而無故矣。聖人不死,大盜不止。雖重聖人而治天下,則是重利盜跖也。為之斗斛以量之,則並與斗斛而竊之;為之權衡以稱之,則並與權衡而竊之;為之符璽以信之,則並與符璽而竊之;為之仁義以矯之,則並與仁義而竊之。何以知其然邪?彼竊鉤者誅,竊國者為諸侯,諸侯之門,而仁義存焉,則是非竊仁義聖知邪?故逐於大盜,揭諸侯,竊仁義並斗斛、權衡、符璽之利者,雖有軒冕之賞弗能勸,斧鉞之威弗能禁。此重利盜跖而使不可禁者,是乃聖人之過也。故曰:「魚不可脫於淵,國之利器不可以示人。」彼聖人者,天下之利器也,非所以明天下也。故絕聖棄知,大盜乃止;擿玉毀珠,小盜不起;焚符破璽,而民朴鄙;掊斗折衡,而民不爭;殫殘天下之聖法,而民始可與論議。擢亂六律,鑠絕竽瑟,塞瞽曠之耳,而天下始人含其聰矣;滅文章,散五采,膠離朱之目,而天下始人含其明矣;毀絕鉤繩而棄規矩,攦工倕之指,而天下始人有其巧矣。故曰:「大巧若拙。」削曾、史之行,鉗楊、墨之口,攘棄仁義,而天下之德始玄同矣。彼人含其明,則天下不鑠矣;人含其聰,則天下不累矣;人含其知,則天下不惑矣;人含其德,則天下不僻矣。彼曾、史、楊、墨、師曠、工倕、離朱,皆外立其德,而以爚亂天下者也,法之所無用也。

子獨不知至德之世乎?昔者容成氏、大庭氏、伯皇氏、中央氏、栗陸氏、驪畜氏、軒轅氏、赫胥氏、尊盧氏、祝融氏、伏羲氏、神農氏,當是時也,民結繩而用之,甘其食,美其服,樂其俗,安其居,鄰國相望,雞狗之音相聞,民至老死而不相往來。若此之時,則至治已。今遂至使民延頸舉踵曰「某所有賢者」,贏糧而趣之,則內棄其親而外去其主之事,足跡接乎諸侯之境,車軌結乎千里之外,則是上好知之過也。上誠好知而無道,則天下大亂矣。何以知其然邪?夫弓、弩、畢、弋、機變之知多,則鳥亂於上矣;鉤餌、罔、罟罾笱之知多,則魚亂於水矣;削格、羅落、罝罘之知多,則獸亂於澤矣;知詐漸毒、頡滑堅白、解垢同異之變多,則俗惑於辯矣。故天下每每大亂,罪在於好知。故天下皆知求其所不知而莫知求其所已知者,皆知非其所不善而莫知非其所已善者,是以大亂。故上悖日月之明,下爍山川之精,中墮四時之施,惴耎之蟲,肖翹之物,莫不失其性。甚矣夫好知之亂天下也!自三代以下者是已。舍夫種種之民而悅夫役役之佞,釋夫恬淡無為而悅夫啍啍之意,啍啍已亂天下矣。


\end{pinyinscope}