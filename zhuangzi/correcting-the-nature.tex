\article{繕性}

\begin{pinyinscope}
繕性於俗,俗學以求復其初,滑欲於俗,思以求致其明,謂之蔽蒙之民。

古之治道者,以恬養知;知生而無以知為也,謂之以知養恬。知與恬交相養,而和理出其性。夫德,和也;道,理也。德無不容,仁也;道無不理,義也;義明而物親,忠也;中純實而反乎情,樂也;信行容體而順乎文,禮也。禮樂遍行,則天下亂矣。彼正而蒙己德,德則不冒,冒則物必失其性也。

古之人在混芒之中,與一世而得澹漠焉。當是時也,陰陽和靜,鬼神不擾,四時得節,萬物不傷,群生不夭,人雖有知,無所用之,此之謂至一。當是時也,莫之為而常自然。

逮德下衰,及燧人、伏羲始為天下,是故順而不一。德又下衰,及神農、黃帝始為天下,是故安而不順。德又下衰,及唐、虞始為天下,興治化之流,澆淳散朴,離道以善,險德以行,然後去性而從於心。心與心識知而不足以定天下,然後附之以文,益之以博。文滅質,博溺心,然後民始惑亂,無以反其性情而復其初。

由是觀之,世喪道矣,道喪世矣。世與道交相喪也。道之人何由興乎世,世亦何由興乎道哉!道無以興乎世,世無以興乎道,雖聖人不在山林之中,其德隱矣。隱,故不自隱。古之所謂隱士者,非伏其身而弗見也,非閉其言而不出也,非藏其知而不發也,時命大謬也。當時命而大行乎天下,則反一無跡;不當時命而大窮乎天下,則深根寧極而待。此存身之道也。古之行身者,不以辯飾知,不以知窮天下,不以知窮德,危然處其所而反其性,己又何為哉!道固不小行,德固不小識。小識傷德,小行傷道。故曰:正己而已矣。

樂全之謂得志。古之所謂得志者,非軒冕之謂也,謂其無以益其樂而已矣。今之所謂得志者,軒冕之謂也。軒冕在身,非性命也,物之儻來,寄者也。寄之,其來不可圉,其去不可止。故不為軒冕肆志,不為窮約趨俗,其樂彼與此同,故無憂而已矣。今寄去則不樂,由是觀之,雖樂,未嘗不荒也。故曰:喪己於物,失性於俗者,謂之倒置之民。


\end{pinyinscope}