\article{田子方}

\begin{pinyinscope}
田子方侍坐於魏文侯,數稱谿工。文侯曰:「谿工,子之師邪?」子方曰:「非也。無擇之里人也,稱道數當,故無擇稱之。」文侯曰:「然則子無師邪?」子方曰:「有。」曰:「子之師誰邪?」子方曰:「東郭順子。」文侯曰:「然則夫子何故未嘗稱之?」子方曰:「其為人也真,人貌而天虛,緣而葆真,清而容物。物無道,正容以悟之,使人之意也消。無擇何足以稱之!」

子方出,文侯儻然終日不言,召前立臣,而語之曰:「遠矣全德之君子!始吾以聖知之言、仁義之行為至矣,吾聞子方之師,吾形解而不欲動,口鉗而不欲言。吾所學者直土梗耳,夫魏真為我累耳!」

溫伯雪子適齊,舍於魯。魯人有請見之者,溫伯雪子曰:「不可。吾聞中國之君子,明乎禮義而陋於知人心,吾不欲見也。」至於齊,反舍於魯,是人也又請見。溫伯雪子曰:「往也蘄見我,今也又蘄見我,是必有以振我也。」出而見客,入而歎。明日見客,又入而歎。其僕曰:「每見之客也,必入而歎,何邪?」曰:「吾固告子矣:『中國之民,明乎禮義而陋乎知人心。』昔之見我者,進退一成規,一成矩;從容一若龍,一若虎;其諫我也似子,其道我也似父。是以歎也。」

仲尼見之而不言。子路曰:「吾子欲見溫伯雪子久矣,見之而不言,何邪?」仲尼曰:「若夫人者,目擊而道存矣,亦不可以容聲矣。」

顏淵問於仲尼曰:「夫子步亦步,夫子趨亦趨,夫子馳亦馳,夫子奔逸絕塵,而回瞠若乎後矣。」夫子曰:「回,何謂邪?」曰:「夫子步亦步也,夫子言亦言也,夫子趨亦趨也,夫子辯亦辯也,夫子馳亦馳也,夫子言道,回亦言道也。及奔逸絕塵,而回瞠若乎後者,夫子不言而信,不比而周,無器而民滔乎前,而不知所以然而已矣。」

仲尼曰:「惡!可不察與!夫哀莫大於心死,而人死亦次之。日出東方而入於西極,萬物莫不比方。有目有趾者,待是而後成功,待晝而作。是出則存,是入則亡。萬物亦然,有待也而死,有待也而生。吾一受其成形,而不化以待盡,效物而動,日夜無隙,而不知其所終,薰然其成形,知命不能規乎其前,丘以是日徂。吾終身與汝交一臂而失之,可不哀與!女殆著乎吾所以著也。彼已盡矣,而女求之以為有,是求馬於唐肆也。吾服女也甚忘,女服吾也亦甚忘。雖然,女奚患焉!雖忘乎故吾,吾有不忘者存。」

孔子見老聃,老聃新沐,方將被髮而乾,慹然似非人。孔子便而待之,少焉見曰:「丘也眩與?其信然與?向者先生形體掘若槁木,似遺物離人而立於獨也。」老聃曰:「吾遊心於物之初。」

孔子曰:「何謂邪?」曰:「心困焉而不能知,口辟焉而不能言,嘗為汝議乎其將。至陰肅肅,至陽赫赫;肅肅出乎天,赫赫發乎地;兩者交通成和而物生焉,或為之紀而莫見其形。消息滿虛,一晦一明,日改月化,日有所為,而莫見其功。生有所乎萌,死有所乎歸,始終相反乎無端,而莫知其所窮。非是也,且孰為之宗!」

孔子曰:「請問遊是。」老聃曰:「夫得是,至美至樂也。得至美而遊乎至樂,謂之至人。」孔子曰:「願聞其方。」曰:「草食之獸不疾易藪,水生之蟲不疾易水,行小變而不失其大常也,喜怒哀樂不入於胸次。夫天下也者,萬物之所一也。得其所一而同焉,則四支百體將為塵垢,而死生終始將為晝夜而莫之能滑,而況得喪禍福之所介乎!棄隸者若棄泥塗,知身貴於隸也,貴在於我而不失於變。且萬化而未始有極也,夫孰足以患心!已為道者解乎此。」

孔子曰:「夫子德配天地,而猶假至言以修心,古之君子,孰能脫焉?」老聃曰:「不然。夫水之於汋也,無為而才自然矣。至人之於德也,不修而物不能離焉,若天之自高,地之自厚,日月之自明,夫何修焉!」

孔子出,以告顏回曰:「丘之於道也,其猶醯雞與!微夫子之發吾覆也,吾不知天地之大全也。」

莊子見魯哀公。哀公曰:「魯多儒士,少為先生方者。」莊子曰:「魯少儒。」哀公曰:「舉魯國而儒服,何謂少乎?」莊子曰:「周聞之:儒者冠圜冠者,知天時;履句屨者,知地形;緩佩玦者,事至而斷。君子有其道者,未必為其服也;為其服者,未必知其道也。公固以為不然,何不號於國中曰『無此道而為此服者,其罪死』?」於是哀公號之五日,而魯國無敢儒服者。獨有一丈夫儒服而立乎公門,公即召而問以國事,千轉萬變而不窮。莊子曰:「以魯國而儒者一人耳,可謂多乎?」

百里奚爵祿不入於心,故飯牛而牛肥,使秦穆公忘其賤,與之政也。有虞氏死生不入於心,故足以動人。

宋元君將畫圖。眾史皆至,受揖而立;舐筆和墨,在外者半。有一史後至者,儃儃然不趨,受揖不立,因之舍。公使人視之,則解衣般礡,臝。君曰:「可矣,是真畫者也。」

文王觀於臧,見一丈夫釣,而其釣莫釣,非持其釣,有釣者也,常釣也。

文王欲舉而授之政,而恐大臣父兄之弗安也;欲終而釋之,而不忍百姓之無天也。於是旦而屬之夫夫曰:「昔者寡人夢,見良人黑色而髯,乘駁馬而偏朱蹄,號曰:『寓而政於臧丈人,庶幾乎民有瘳乎!』」諸大夫蹴然曰:「先君王也。」文王曰:「然則卜之。」諸大夫曰:「先君之命王,其無它,又何卜焉!」

遂迎臧丈人而授之政。典洗無更,偏令無出。三年,文王觀於國,則列士壞植散群,長官者不成德,斔斛不敢入於四竟。列士壞植散群,則尚同也;長官者不成德,則同務也;斔斛不敢入於四竟,則諸侯無二心也。文王於是焉以為大師,北面而問曰:「政可以及天下乎?」臧丈人昧然而不應,泛然而辭,朝令而夜遁,終身無聞。

顏淵問於仲尼曰:「文王其猶未邪?又何以夢為乎?」仲尼曰:「默!汝無言!夫文王盡之也,而又何論刺焉!彼直以循斯須也。」

列御寇為伯昏無人射,引之盈貫,措杯水其肘上,發之,適矢復沓,方矢復寓。當是時,猶象人也。伯昏無人曰:「是射之射,非不射之射也。嘗與汝登高山,履危石,臨百仞之淵,若能射乎?」於是無人遂登高山,履危石,臨百仞之淵,背逡巡,足二分垂在外,揖御寇而進之。御寇伏地,汗流至踵。伯昏無人曰:「夫至人者,上闚青天,下潛黃泉,揮斥八極,神氣不變。今汝怵然有恂目之志,爾於中也殆矣夫!」

肩吾問於孫叔敖曰:「子三為令尹而不榮華,三去之而無憂色。吾始也疑子,今視子之鼻間栩栩然,子之用心獨奈何?」孫叔敖曰:「吾何以過人哉!吾以其來不可卻也,其去不可止也,吾以為得失之非我也,而無憂色而已矣。我何以過人哉!且不知其在彼乎,其在我乎?其在彼邪,亡乎我;在我邪,亡乎彼。方將躊躇,方將四顧,何暇至乎人貴人賤哉!」

仲尼聞之曰:「古之真人,知者不得說,美人不得濫,盜人不得劫,伏戲、黃帝不得友。死生亦大矣,而無變乎己,況爵祿乎!若然者,其神經乎大山而無介,入乎淵泉而不濡,處卑細而不憊,充滿天地,既以與人,己愈有。」

楚王與凡君坐,少焉,楚王左右曰「凡亡」者三。凡君曰:「凡之亡也,不足以喪吾存。夫『凡之亡也,不足以喪吾存』,則楚之存不足以存存。由是觀之,則凡未始亡而楚未始存也。」


\end{pinyinscope}