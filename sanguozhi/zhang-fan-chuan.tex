\article{zhang-fan-chuan}

\begin{pinyinscope}
張範,字公儀,河內脩武人也。祖父歆,為漢司徒。父延,為太尉。太傅袁隗欲以女妻範,範辭不受。性恬靜樂道,忽於榮利,徵命無所就。弟承,字公先,亦知名,以方正徵,拜議郎,遷伊闕都尉。董卓作亂,承欲合徒衆與天下共誅卓。承弟昭時為議郎,適從長安來,謂承曰:「今欲誅卓,衆寡不敵,且起一朝之謀,戰阡陌之民,士不素撫,兵不練習,難以成功。卓阻兵而無義,固不能乆;不若擇所歸附,待時而動,然後可以如志。」承然之,乃解印綬間行歸家,與範避地揚州。袁術備禮招請,範稱疾不往,術不彊屈也。遣承與相見,術問曰:「昔周室陵遲,則有桓、文之霸;秦失其政,漢接而用之。今孤以土地之廣,士民之衆,欲徼福齊桓,擬迹高祖,何如?」承對曰:「在德不在彊。夫能用德以同天下之欲,雖由匹夫之資,而興霸王之功,不足為難。若苟僭擬,干時而動,衆之所棄,誰能興之?」術不恱。是時,太祖將征兾州,術復問曰:「今曹公欲以弊兵數千,敵十萬之衆,可謂不量力矣!子以為何如?」承乃曰:「漢德雖衰,天命未改,今曹公挾天子以令天下,雖敵百萬之衆可也。」術作色不懌,承去之。

太祖平兾州,遣使迎範。範以疾留彭城,遣承詣太祖,太祖表以為諫議大夫。範子陵及承子戩為山東賊所得,範直詣賊請二子,賊以陵還範。範謝曰:「諸君相還兒厚矣。夫人情雖愛其子,然吾憐戩之小,請以陵易之。」賊義其言,悉以還範。太祖自荊州還,範得見於陳,以為議郎,參丞相軍事,甚見敬重。太祖征伐,常令範及邴原留,與世子居守。太祖謂文帝:「舉動必諮此二人。」世子執子孫禮。救恤窮乏,家無所餘,中外孤寡皆歸焉。贈遺無所逆,亦終不用,及去,皆以還之。建安十七年卒。魏國初建,承以丞相參軍祭酒領趙郡太守,政化大行。太祖將西征,徵承參軍事,至長安,病卒。

魏書曰:文帝即位,以範子參為郎中。承孫邵,晉中護軍,與舅楊駿俱被誅。事見晉書。


\end{pinyinscope}