\article{dian-wei-chuan}

\begin{pinyinscope}
典韋,陳留己吾人也。形貌魁梧,旅力過人,有志節任俠。襄邑劉氏與睢陽李永為讎,韋為報之。永故富春長,備衞甚謹。韋乘車載雞酒,偽為候者,門開,懷匕首入殺永,并殺其妻,徐出,取車上刀戟,步出。永居近巿,一巿盡駭。追者數百,莫敢近。行四五里,遇其伴,轉戰得脫。由是為豪傑所識。初平中,張邈舉義兵,韋為士,屬司馬趙寵。牙門旗長大,人莫能勝,韋一手建之,寵異其才力。後屬夏侯惇,數斬首有功,拜司馬。太祖討呂布於濮陽。布有別屯在濮陽西四五十里,太祖夜襲,比明破之。未及還,會布救兵至,三面掉戰。時布身自搏戰,自旦至日昳數十合,相持急。太祖募陷陣,韋先占,將應募者數十人,皆重衣兩鎧,棄楯,但持長矛撩戟。時西面又急,韋進當之,賊弓弩亂發,矢至如雨,韋不視,謂等人曰:「虜來十步,乃白之。」等人曰:「十步矣。」又曰:「五步乃白。」等人懼,疾言「虜至矣」!韋手持十餘戟,大呼起,所抵無不應手倒者。布衆退。會日暮,太祖乃得引去。拜韋都尉,引置左右,將親兵數百人,常繞大帳。韋旣壯武,其所將皆選卒,每戰鬬,常先登陷陣。遷為校尉。性忠至謹重,常晝立侍終日,夜宿帳左右,稀歸私寢。好酒食,飲噉兼人,每賜食於前,大飲長歠,左右相屬,數人益乃供,太祖壯之。韋好持大雙戟與長刀等,軍中為之語曰:「帳下壯士有典君,提一雙戟八十斤。」

太祖征荊州,至宛,張繡迎降。太祖甚恱,延繡及其將帥,置酒高會。太祖行酒,韋持大斧立後,刃徑尺,太祖所至之前,韋輒舉斧目之。竟酒,繡及其將帥莫敢仰視。後十餘日,繡反,襲太祖營,太祖出戰不利,輕騎引去。韋戰於門中,賊不得入。兵遂散從他門並入。時韋校尚有十餘人,皆殊死戰,無不一當十。賊前後至稍多,韋以長戟左右擊之,一叉入,輒十餘矛摧。左右死傷者略盡。韋被數十創,短兵接戰,賊前搏之。韋雙挾兩賊擊殺之,餘賊不敢前。韋復前突賊,殺數人,創重發,瞋目大罵而死。賊乃敢前,取其頭,傳觀之,覆軍就視其軀。太祖退住舞陰,聞韋死,為流涕,募閒取其喪,親自臨哭之,遣歸葬襄邑,拜子滿為郎中。車駕每過,常祠以中牢。太祖思韋,拜滿為司馬,引自近。文帝即王位,以滿為都尉,賜爵關內侯。


\end{pinyinscope}