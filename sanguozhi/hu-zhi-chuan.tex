\article{胡質傳}

\begin{pinyinscope}
胡質字文德,楚國壽春人也。少與蔣濟、朱績俱知名於江、淮間,仕州郡。蔣濟為別駕,使見太祖。太祖問曰:「胡通達,長者也,寧有子孫不?」濟曰:「有子曰質,規模大略不及於父,至於精良綜事過之。」

案胡氏譜:通達名敏,以方正徵。太祖即召質為頓丘令。縣民郭政通於從妹,殺其夫程他,郡吏馮諒繫獄為證。政與妹皆耐掠隱抵,諒不勝痛,自誣,當反其罪。質至官,察其情色,更詳其事,檢驗具服。

入為丞相東曹議令史,州請為治中。將軍張遼與其護軍武周有隙。遼見刺史溫恢求請質,質辭以疾。遼出謂質曰:「僕委意於君,何以相辜如此?」質曰:「古人之交也,取多知其不貪,奔北知其不怯,聞流言而不信,故可終也。武伯南身為雅士,往者將軍稱之不容於口,今以睚眦之恨,乃成嫌隙。睚,五賣反。眦,士賣反。況質才薄,豈能終好?是以不願也。」遼感言,復與周平。虞預晉書曰:周字伯南,沛國竹邑人。位至光祿大夫。子陔,字元夏。陔及二弟韶、茂,皆總角見稱,並有器望,雖鄉人諸父,未能覺其多少。時同郡劉公榮,名知人,嘗造周。周謂曰:「卿有知人之明,欲使三兒見卿,卿為目高下,以效郭、許之聽可乎?」公榮乃自詣陔兄弟,與共言語,觀其舉動。出語周曰:「君三子皆國士也。元夏器量最優,有輔佐之風,展力仕宦,可為亞公。叔夏、季夏,不減常伯、納言也。」陔少出仕宦,歷職內外,泰始初為吏部尚書,遷左僕射、右光祿大夫、開府儀同三司,卒於官。陔以在魏已為大臣,本非佐命之數,懷遜讓,不得已而居位,故在官職,無所荷任,夙夜思恭而已。終始全潔,當世以為美談。韶歷二官吏部郎。山濤啟事稱韶清白有誠,終於散騎常侍。茂至侍中、尚書。潁川荀愷,宣帝外孫,世祖姑子,自負貴戚,要與茂交。茂拒而不荅,由是見怒。元康元年,楊駿被誅。愷時為尚書僕射,以茂駿之姨弟,陷為駿黨,遂枉見殺,衆咸冤痛之。

太祖辟為丞相屬。黃初中,徙吏部郎,為常山太守,遷任東莞。士盧顯為人所殺,質曰:「此士無讎而有少妻,所以死乎!」悉見其比居年少,書吏李若見問而色動,遂窮詰情狀。若即自首,罪人斯得。每軍功賞賜,皆散之於衆,無入家者。在郡九年,吏民便安,將士用命。

遷荊州刺史,加振威將軍,賜爵關內侯。吳大將朱然圍樊城,質輕軍赴之。議者皆以為賊盛不可迫,質曰:「樊城卑下,兵少,故當進軍為之外援;不然,危矣。」遂勒兵臨圍,城中乃安。遷征東將軍,假節都督青、徐諸軍。廣農積糓,有兼年之儲,置東征臺,且佃且守。又通渠諸郡,利舟楫,嚴設備以待敵。海邊無事。

性沈實內察,不以其節檢物,所在見思。嘉平二年薨,家無餘財,惟有賜衣書篋而已。軍師以聞,追進封陽陵亭侯,邑百戶,謚曰貞侯。子威嗣。六年,詔書褒述質清行,賜其家錢糓。語在徐邈傳。威,咸熈中官至徐州刺史,晉陽秋曰:威字伯虎。少有志尚,厲操清白。質之為荊州也,威自京都省之。家貧,無車馬童僕,威自驅驢單行,拜見父。停廄中十餘日,告歸。臨辭,質賜絹一匹,為道路糧。威跪曰:「大人清白,不審於何得此絹?」質曰:「是吾俸祿之餘,故以為汝糧耳。」威受之,辭歸。每至客舍,自放驢,取樵炊爨,食畢,復隨旅進道,往還如是。質帳下都督,素不相識,先其將歸,請假還家,陰資裝百餘里要之,因與為伴,每事佐助經營之,又少進飲食,行數百里。威疑之,密誘問,乃知其都督也,因取向所賜絹荅謝而遣之。後因他信,具以白質。質杖其都督一百,除吏名。其父子清慎如此。於是名譽著聞,歷位宰牧。晉武帝賜見,論邊事,語及平生。帝歎其父清,謂威曰:「卿清孰與父清?」威對曰:「臣不如也。」帝曰:「以何為不如?」對曰:「臣父清恐人知,臣清恐人不知,是臣不如者遠也。」官至前將軍、青州刺史。太康元年卒,追贈鎮東將軍。威弟羆,字季象,征南將軍;威子弈,字次孫,平東將軍;並以潔行垂名。有殊績,歷三郡守,所在有名。卒於安定。


\end{pinyinscope}