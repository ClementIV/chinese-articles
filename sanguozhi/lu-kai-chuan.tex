\article{陸凱傳}

\begin{pinyinscope}
陸凱字敬風,吳郡吳人,丞相遜族子也。黃武初為永興、諸曁長,所在有治迹,拜建武都尉,領兵。雖統軍衆,手不釋書。好太玄,論演其意,以筮輒驗。赤烏中,除儋耳太守,討朱崖,斬獲有功,遷為建武校尉。五鳳二年,討山賊陳毖於零陵,斬毖克捷,拜巴丘督、偏將軍,封都鄉侯,轉為武昌右部督。與諸將共赴壽春,還,累遷盪魏、綏遠將軍。孫休即位,拜征北將軍,假節領豫州牧。孫皓立,遷鎮西大將軍,都督巴丘,領荊州牧,進封嘉興侯。孫皓與晉平,使者丁忠自北還,說皓弋陽可襲,凱諫止,語在皓傳。寶鼎元年,遷左丞相。

皓性不好人視己,羣臣侍見,皆莫敢迕。凱說皓曰:「夫君臣無不相識之道,若卒有不虞,不知所赴。」皓聽凱自視。

皓時徙都武昌,揚土百姓泝流供給,以為患苦,又政事多謬,黎元窮匱。凱上疏曰:

臣聞有道之君,以樂樂民;無道之君,以樂樂身。樂民者,其樂弥長;樂身者,不乆而亡。夫民者,國之根也,誠宜重其食,愛其命。民安則君安,民樂則君樂。自頃年以來,君威傷於桀紂,君明闇於姦雄,君惠閉於羣孽。無災而民命盡,無為而國財空,辜無罪,賞無功,使君有謬誤之愆,天為作妖。而諸公卿媚上以求愛,困民以求饒,導君於不義,敗政於淫俗,臣竊為痛心。今鄰國交好,四邊無事,當務息役養士,實其廩庫,以待天時。而更傾動天心,搔擾萬姓,使民不安,大小呼嗟,此非保國養民之術也。

臣聞吉凶在天,猶影之在形,響之在聲也,形動則影動,形止則影止,此分數乃有所繫,非在口之所進退也。昔秦所以亡天下者,但坐賞輕而罰重,政刑錯亂,民力盡於奢侈,目眩於美色,志濁於財寶,邪臣在位,賢哲隱藏,百姓業業,天下苦之,是以遂有覆巢破卵之憂。漢所以彊者,躬行誠信,聽諫納賢,惠及負薪,躬請巖穴,廣采博察,以成其謀。此往事之明證也。

近者漢之衰末,三家鼎立,曹失綱紀,晉有其政。又益州危險,兵多精彊,閉門固守,可保萬世,而劉氏與奪乖錯,賞罰失所,君恣意於奢侈,民力竭於不急,是以為晉所伐,君臣見虜。此目前之明驗也。

臣闇於大理,文不及義,智慧淺劣,無復兾望,竊為陛下惜天下耳。臣謹奏耳目所聞見,百姓所為煩苛,刑政所為錯亂,願陛下息大功,損百役,務寬盪,忽苛政。

又武昌土地,實危險而塉确,非王都安國養民之處,船泊則沈漂,陵居則峻危,且童謠言:「寧飲建業水,不食武昌魚;寧還建業死,不止武昌居。」臣聞翼星為變,熒惑作妖,童謠之言,生於天心,乃以安居而比死,足明天意,知民所苦也。

臣聞國無三年之儲,謂之非國,而今無一年之畜,此臣下之責也。而諸公卿位處人上,祿延子孫,曾無致命之節,匡救之術,苟進小利於君,以求容媚,荼毒百姓,不為君計也。自從孫弘造義兵以來,耕種旣廢,所在無復輸入,而分一家父子異役,廩食日張,畜積日耗,民有離散之怨,國有露根之漸,而莫之恤也。民力困窮,鬻賣兒子,調賦相仍,日以疲極,所在長吏,不加隱括,加有監官,旣不愛民,務行威勢,所在搔擾,更為煩苛,民苦二端,財力再耗,此為無益而有損也。願陛下一息此輩,矜哀孤弱,以鎮撫百姓之心。此猶魚龞得免毒螫之淵,鳥獸得離羅網之綱,四方之民繈負而至矣。如此,民可得保,先王之國存焉。

臣聞五音令人耳不聦,五色令人目不明,此無益於政,有損於事者也。自昔先帝時,後宮列女,及諸織絡,數不滿百,米有畜積,貨財有餘。先帝崩後,幼、景在位,更改奢侈,不蹈先迹。伏聞織絡及諸徒坐,乃有千數,計其所長,不足為國財,然坐食官廩,歲歲相承,此為無益,願陛下料出賦嫁,給與無妻者。如此,上應天心,下合地意,天下幸甚。

臣聞殷湯取士於商賈,齊桓取士於車轅,周武取士於負薪,大漢取士於奴僕。明王聖主取士以賢,不拘卑賤,故其功德洋溢,名流竹素,非求顏色而取好服、捷口、容恱者也。臣伏見當今內寵之臣,位非其人,任非其量,不能輔國匡時,羣黨相扶,害忠隱賢。願陛下簡文武之臣,各勤其官,州牧督將,藩鎮方外,公卿尚書,務脩仁化,上助陛下,下拯黎民,各盡其忠,拾遺萬一,則康哉之歌作,刑錯之理清。願陛下留神思臣愚言。

時殿上列將何定佞巧便辟,貴幸任事,凱面責定曰:「卿見前後事主不忠,傾亂國政,寧有得以壽終者邪!何以專為佞邪,穢塵天聽?宜自改厲。不然,方見卿有不測之禍矣。」定大恨凱,思中傷之,凱終不以為意,乃心公家,義形於色,表疏皆指事不飾,忠懇內發。

建衡元年,疾病,皓遣中書令董朝問所欲言,凱陳:「何定不可任用,宜授外任,不宜委以國事。奚熙小吏,建起浦里田,欲復嚴密故迹,亦不可聽。姚信、樓玄、賀卲、張悌、郭逴、薛瑩、滕脩及族弟喜、抗,或清白忠勤,或姿才卓茂,皆社稷之楨幹,國家之良輔,願陛下重留神思,訪以時務,各盡其忠,拾遺萬一。」遂卒,時年七十二。

子禕,初為黃門侍郎,出領部曲,拜偏將軍。凱亡後,入為太子中庶子。右國史華覈表薦禕曰:「禕體質方剛,器幹彊固,董率之才,魯肅不過。及被召當下,徑還赴都,道由武昌,曾不迴顧,器械軍資,一無所取,在戎果毅,臨財有節。夫夏口,賊之衝要,宜選名將以鎮戍之,臣竊思惟,莫善於禕。」

初,皓常銜凱數犯顏忤旨,加何定譖構非一,旣以重臣,難繩以法,又陸抗時為大將在疆埸,故以計容忍。抗卒後,竟徙凱家於建安。

或曰寶鼎元年十二月,凱與大司馬丁奉、御史大夫丁固謀,因皓謁廟,欲廢皓立孫休子。時左將軍留平領兵先驅,故密語平,平拒而不許,誓以不泄,是以所圖不果。太史郎陳苗奏皓乆陰不雨,風氣迴逆,將有陰謀,皓深警懼云。

吳錄曰:舊拜廟,選兼大將軍領三千兵為衞,凱欲因此兵圖之,令選曹白用丁奉。皓偶不欲,曰:「更選。」凱令執據,雖蹔兼,然宜得其人。皓曰:「用留平。」凱令其子禕謀語平。平素與丁奉有隙,禕未及得宣凱旨,平語禕曰:「聞野豬入丁奉營,此凶徵也。」有喜色。禕乃不敢言,還,因具啟凱,故輟止。

予連從荊、揚來者得凱所諫皓二十事,博問吳人,多云不聞凱有此表。又按其文殊甚切直,恐非皓之所能容忍也。或以為凱藏之篋笥,未敢宣行,病困,皓遣董朝省問欲言,因以付之。虛實難明,故不著于篇,然愛其指擿皓事,足為後戒,故鈔列于凱傳左云。

皓遣親近趙欽口詔報凱前表曰:「孤動必遵先帝,有何不平?君所諫非也。又建業宮不利,故避之,而西宮室宇摧朽,須謀移都,何以不可徙乎?」凱上疏曰:

臣竊見陛下執政以來,陰陽不調,五星失晷,職司不忠,姦黨相扶,是陛下不遵先帝之所致。江表傳載凱此表曰:「臣拜受明詔,心與氣結。陛下何心之難悟,意不聦之甚也!」夫王者之興,受之於天,脩之由德,豈在宮乎?而陛下不諮之公輔,便盛意驅馳,六軍流離悲懼,逆犯天地,天地以災,童歌其謠。縱令陛下一身得安,百姓愁勞,何以用治?此不遵先帝一也。

臣聞有國以賢為本,夏殺龍逄,殷獲伊摯,斯前世之明效,今日之師表也。中常侍王蕃黃中通理,處朝忠謇,斯社稷之重鎮,大吳之龍逄也,而陛下忿其苦辭,惡其直對,梟之殿堂,尸骸暴棄。邦內傷心,有識悲悼,咸以吳國夫差復存。先帝親賢,陛下反之,是陛下不遵先帝二也。

臣聞宰相國之柱也,不可不彊,是故漢有蕭、曹之佐,先帝有顧、步之相。而萬彧瑣才凡庸之質,昔從家隷,超步紫闥,於彧已豐,於器已溢,而陛下愛其細介,不訪大趣,榮以尊輔,越尚舊臣。賢良憤惋,智士赫咤,是不遵先帝三也。

先帝愛民過於嬰孩,民無妻者以妾妻之,見單衣者以帛給之,枯骨不收而取埋之。而陛下反之,是不遵先帝四也。

昔桀紂滅由妖婦,幽厲亂在嬖妾,先帝鑒之,以為身戒,故左右不置淫邪之色,後房無曠積之女。今中宮萬數,不備嬪嬙,外多鰥夫,女吟於中。風雨逆度,正由此起,是不遵先帝五也。

先帝憂勞萬機,猶懼有失。陛下臨阼以來,游戲後宮,眩惑婦女,乃令庶事多曠,下吏容姦,是不遵先帝六也。

先帝篤尚朴素,服不純麗,宮無高臺,物不彫飾,故國富民充,姦盜不作。而陛下徵調州郡,竭民財力,土被玄黃,宮有朱紫,是不遵先帝七也。

先帝外仗顧、陸、朱、張,內近胡綜、薛綜,是以庶績雍熈,邦內清肅。今者外非其任,內非其人,陳聲、曹輔,斗筲小吏,先帝之所棄,而陛下幸之,是不遵先帝八也。

先帝每宴見羣臣,抑損醇醲,臣下終日無失慢之尤,百寮庶尹,並展所陳。而陛下拘以視瞻之敬,懼以不盡之酒。夫酒以成禮,過則敗德,此無異商辛長夜之飲也,是不遵先帝九也。

昔漢之桓、靈,親近宦豎,大失民心。今高通、詹廉、羊度,黃門小人,而陛下賞以重爵,權以戰兵。若江渚有難,烽燧互起,則度等之武不能禦侮明也,是不遵先帝十也。

今宮女曠積,而黃門復走州郡,條牒民女,有錢則舍,無錢則取,怨呼道路,母子死訣,是不遵先帝十一也。

先帝在時,亦養諸王太子,若取乳母,其夫復役,賜與錢財,給其資糧,時遣歸來,視其弱息。今則不然,夫婦生離,夫故作役,兒從後死,家為空戶,是不遵先帝十二也。

先帝歎曰:「國以民為本,民以食為天,衣其次也,三者,孤存之於心。」今則不然,農桑並廢,是不遵先帝十三也。

先帝簡士,不拘卑賤,任之鄉閭,效之於事,舉者不虛,受者不妄。今則不然,浮華者登,朋黨者進,是不遵先帝十四也。

先帝戰士,不給他役,使春惟知農,秋惟收稻,江渚有事,責其死效。今之戰士,供給衆役,廩賜不贍,是不遵先帝十五也。

夫賞以勸功,罰以禁邪,賞罰不中,則士民散失。今江邊將士,死不見哀,勞不見賞,是不遵先帝十六也。

今在所監司,已為煩猥,兼有內使,擾亂其中,一民十吏,何以堪命?昔景帝時,交阯反亂,實由茲起,是為遵景帝之闕,不遵先帝十七也。

夫校事,吏民之仇也。先帝末年,雖有呂壹、錢欽,尋皆誅夷,以謝百姓。今復張立校曹,縱吏言事,是不遵先帝十八也。

先帝時,居官者咸乆於其位,然後考績黜陟。今州縣職司,或莅政無幾,便徵召遷轉,迎新送舊,紛紜道路,傷財害民,於是為甚,是不遵先帝十九也。

先帝每察竟解之奏,當留心推桉,是以獄無冤囚,死者吞聲。今則違之,是不遵先帝二十也。

若臣言可錄,藏之盟府;如其虛妄,治臣之罪。願陛下留意。江表傳曰:皓所行彌暴,凱知其將亡,上表曰:「臣聞惡不可積,過不可長;積惡長過,喪亂之源也。是以古人懼不聞非,故設進善之旌,立敢諫之鼓。武公九十,思聞警戒,詩美其德,士恱其行。臣察陛下無思警戒之義,而有積惡之漸,臣深憂之,此禍兆見矣。故略陳其要,寫盡愚懷。陛下宜克己復禮,述脩前德,不可捐棄臣言,而放奢意。意奢情至,吏日欺民;民離則上不信下,下當疑上,骨肉相克,公子相奔。臣雖愚,闇於天命,以心審之,敗不過二十稔也。臣常忿亡國之人夏桀、殷紂,亦不可使後人復忿陛下也。臣受國恩,奉朝三世,復以餘年,值遇陛下,不能循俗,與衆沈浮。若比干、伍員,以忠見戮,以正見疑,自謂畢足,無所餘恨,灰身泉壤,無負先帝,願陛下九思,社稷存焉。」初,皓始起宮,凱上表諫,不聽,凱重表曰:「臣聞宮功當起,夙夜反側,是以頻煩上事,往往留中,不見省報,於邑歎息,企想應罷。昨食時,被詔曰:『君所諫,誠是大趣,然未合鄙意,如何?此宮殿不利,宜當避之,乃可以妨勞役,長坐不利宮乎?父之不安,子亦何倚?』臣拜紙詔,伏讀一周,不覺氣結於胷,而涕泣雨集也。臣年已六十九,榮祿已重,於臣過望,復何所兾?所以勤勤數進苦言者,臣伏念大皇帝創基立業,勞苦勤至,白髮生於鬢膚,黃耇被於甲冑。天下始靜,晏駕早崩,自含息之類,能言之倫,無不歔欷,如喪考妣。幼主嗣統,柄在臣下,軍有連征之費,民有彫殘之損。賊臣干政,公家空竭。今彊敵當塗,西州傾覆,孤罷之民,宜當畜養,廣力肆業,以備有虞。且始徙都,屬有軍征,戰士流離,州郡搔擾,而大功復起,徵召四方,斯非保國致治之漸也。臣聞為人主者,攘災以德,除咎以義。故湯遭大旱,身禱桑林,熒惑守心,宋景退殿,是以旱魃銷亡,妖星移舍。今宮室之不利,但當克己復禮,篤湯、宋之至道,愍黎庶之困苦,何憂宮之不安,災之不銷乎?陛下不務脩德,而務築宮室,若德之不脩,行之不貴,雖殷辛之瑤臺,秦皇之阿房,何止而不喪身覆國,宗廟作墟乎?夫興土功,高臺榭,旣致水旱,民又多疾,其不疑也?為父長安,使子無倚,此乃子離於父,臣離於陛下之象也。臣子一離,雖念克骨,茅茨不翦,復何益焉?是以大皇帝居于南宮,自謂過於阿房。故先朝大臣,以為宮室宜厚,備衞非常,大皇帝曰:『逆虜游魂,當愛育百姓,何聊趣於不急?』然臣下懇惻,由不獲已,故裁調近郡,苟副衆心,比當就功,猶豫三年。當此之時,寇鈔懾威,不犯我境,師徒奔北,且西阻岷、漢,南州無事,尚猶沖讓,未肯築宮,況陛下危惻之世,又乏大皇帝之德,可不慮哉?願陛下留意,臣不虛言。」

胤字敬宗,凱弟也。始為御史、尚書選曹郎,太子和聞其名,待以殊禮。會全寄、楊笁等阿附魯王霸,與和分爭,陰相譖搆,胤坐收下獄,楚毒備至,終無他辭。吳錄曰:太子自懼黜廢,而魯王覬覦益甚。權時見楊笁,辟左右而論霸之才,笁深述霸有文武英姿,宜為嫡嗣,於是權乃許立焉。有給使伏于牀下,具聞之,以告太子。胤當至武昌,往辭太子。太子不見,而微服至其車上,與共密議,欲令陸遜表諫。旣而遜有表極諫,權疑笁泄之,笁辭不服。權使笁出尋其由,笁白頃惟胤西行,必其所道。又遣問遜何由知之,遜言胤所述。召胤考問,胤為太子隱曰:「楊笁向臣道之。」遂共為獄。笁不勝痛毒,服是所道。初權疑笁泄之,及服,以為果然,乃斬笁。

後為衡陽督軍都尉。赤烏十一年,交阯九真夷賊攻沒城邑,交部搔動。以胤為交州刺史、安南校尉。胤入南界,喻以恩信,務崇招納,高涼渠帥黃吳等支黨三千餘家皆出降。引軍而南,重宣至誠,遺以財幣。賊帥百餘人,民五萬餘家,深幽不羈,莫不稽顙,交域清泰。就加安南將軍。復討蒼梧建陵賊,破之,前後出兵八千餘人,以充軍用。

永安元年,徵為西陵督,封都亭侯,後轉左虎林。中書丞華覈表薦胤曰:「胤天姿聦朗,才通行絜,昔歷選曹,遺跡可紀。還在交州,奉宣朝恩,流民歸附,海隅肅清。蒼梧、南海,歲有舊風瘴氣之害,風則折木,飛砂轉石,氣則霧鬱,飛鳥不經。自胤至州,風氣絕息,商旅平行,民無疾疫,田稼豐稔。州治臨海,海流秋鹹,胤又畜水,民得甘食。惠風橫被,化感人神,遂憑天威,招合遺散。至被詔書當出,民感其恩,以忘戀土,負老攜幼,甘心景從,衆無攜貳,不煩兵衞。自諸將合衆,皆脅之以威,未有如胤結以恩信者也。銜命在州,十有餘年,賔帶殊俗,寶玩所生,而內無粉黛附珠之妾,家無文甲犀象之珍,方之今臣,實難多得。宜在輦轂,股肱王室,以贊唐虞康哉之頌。江邊任輕,不盡其才,虎林選督,堪之者衆。若召還都,寵以上司,則天工畢脩,庶績咸熈矣。」

胤卒,子式嗣,為柴桑督、揚武將軍。天策元年,與從兄禕俱徙建安。天紀二年,召還建業,復將軍、侯。

評曰:潘濬公清割斷,陸凱忠壯質直,皆節槩梗梗,有大丈夫格業。胤身絜事濟,著稱南土,可謂良牧矣。


\end{pinyinscope}