\article{孫登傳}

\begin{pinyinscope}
孫登字子高,權長子也。魏黃初二年,以權為吳王,拜登東中郎將,封萬戶侯,登辭疾不受。是歲,立登為太子,選置師傅,銓簡秀士,以為賔友,於是諸葛恪、張休、顧譚、陳表等以選入,侍講詩書,出從騎射。權欲登讀漢書,習知近代之事,以張昭有師法,重煩勞之,乃令休從昭受讀,還以授登。登待接寮屬,略用布衣之禮,與恪、休、譚等或同輿而載,或共帳而寐。太傅張溫言於權曰:「夫中庶子官最親密,切問近對,宜用儁德。」於是乃用表等為中庶子。後又以庶子禮拘,復令整巾侍坐。黃龍元年,權稱尊號,登為皇太子,以恪為左輔,休右弼,譚為輔正,表為翼正都尉,是為四友,而謝景、范慎、刁玄、羊衜等皆為賔客,

衜音道。於是東宮號為多士。吳錄曰:慎字孝敬,廣陵人,竭忠知己之君,纏綿三益之友,時人榮之。著論二十篇,名曰矯非。後為侍中,出補武昌左部督,治軍整頓。孫皓移都,甚憚之,詔曰:「慎勳德俱茂,朕所敬憑,宜登上公,以副衆望。」以為太尉。慎自恨久為將,遂託老耄。軍士戀之,舉營為之隕涕。鳳凰三年卒,子耀嗣。玄,丹楊人。衜,南陽人。吳書曰:衜初為中庶子,年二十。時廷尉監隱蕃交結豪傑,自衞將軍全琮等皆傾心敬待,惟衜及宣詔郎豫章楊迪拒絕不與通,時人咸怪之。而蕃後叛逆,衆乃服之。江表傳曰:登使侍中胡綜作賔友目曰:「英才卓越,超踰倫匹,則諸葛恪。精識時機,達幽究微,則顧譚。凝辨宏達,言能釋結,則謝景。究學甄微,游夏同科,則范慎。」衜乃私駁綜曰:「元遜才而疏,子嘿精而狠,叔發辯而浮,孝敬深而狹。」所言皆有指趣。而衜卒以此言見咎,不為恪等所親。後四人皆敗,吳人謂衜之言有徵。位至桂陽太守,卒。

權遷都建業,徵上大將軍陸遜輔登鎮武昌,領宮府留事。登或射獵,當由徑道,常遠避良田,不踐苗稼,至所頓息,又擇空閑之地,其不欲煩民如此。嘗乘馬出,有彈丸過,左右求之。有一人操彈佩丸,咸以為是,辭對不服,從者欲捶之,登不聽,使求過丸,比之非類,乃見釋。又失盛水金馬盂,覺得其主,左右所為,不忍致罰,呼責數之,長遣歸家,勑親近勿言。後弟慮卒,權為之降損,登晝夜兼行,到賴鄉,自聞,即時召見。見權悲泣,因諫曰:「慮寢疾不起,此乃命也。方今朔土未一,四海喁喁,天戴陛下,而以下流之念,減損大官殽饌,過於禮制,臣竊憂惶。」權納其言,為之加膳。住十餘日,欲遣西還,深自陳乞,以乆離定省,子道有闕,又陳陸遜忠勤,無所顧憂,權遂留焉。嘉禾三年,權征新城,使登居守,總知留事。時年穀不豐,頗有盜賊,乃表定科令,所以防禦,甚得止姦之要。

初,登所生庶賤,徐夫人少有母養之恩,後徐氏以妬廢處吳,而步夫人最寵。步氏有賜,登不敢辭,拜受而已。徐氏使至,所賜衣服,必沐浴服之。登將拜太子,辭曰:「本立而道生,欲立太子,宜先立后。」權曰:「卿母安在?」對曰:「在吳。」權嘿然。吳書曰:弟和有寵於權,登親敬,待之如兄,常有欲讓之心。

立凡二十一年,年三十三卒。臨終,上疏曰:「臣以無狀,嬰抱篤疾,自省微劣,懼卒隕斃。臣不自惜,念當委離供養,埋胔后土,長不復奉望宮省,朝覲日月,生無益於國,死貽陛下重慼,以此為哽結耳。臣聞死生有命,長短自天,周晉、顏回有上智之才,而尚夭折,況臣愚陋,年過其壽,生為國嗣,沒享榮祚,於臣已多,亦何悲恨哉!方今大事未定,逋寇未討,萬國喁喁,係命陛下,危者望安,亂者仰治。願陛下棄忘臣身,割下流之恩,脩黃老之術,篤養神光,加羞珍膳,廣開神明之慮,以定無窮之業,則率土幸賴,臣死無恨也。皇子和仁孝聦哲,德行清茂,宜早建置,以繫民望。諸葛恪才略博達,器任佐時。張休、顧譚、謝景,皆通敏有識斷,入宜委腹心,出可為爪牙。范慎、華融矯矯壯節,有國士之風。羊衜辯捷,有專對之材。刁玄優弘,志履道真。裴欽博記,翰采足用。蔣脩、虞翻,志節分明。凡此諸臣,或宜廊廟,或任將帥,皆練時事,明習法令,守信固義,有不可奪之志。此皆陛下日月所照,選置臣宮,得與從事,備知情素,敢以陳聞。臣重惟當今方外多虞,師旅未休,當厲六軍,以圖進取。軍以人為衆,衆以財為寶,竊聞郡縣頗有荒殘,民物凋弊,姦亂萌生,是以法令繁滋,刑辟重切。臣聞為政聽民,律令與時推移,誠宜與將相大臣詳擇時宜,博采衆議,寬刑輕賦,均息力役,以順民望。陸遜忠勤於時,出身憂國,謇謇在公,有匪躬之節。諸葛瑾、步隲、朱然、全琮、朱據、呂岱、吾粲、闞澤、嚴畯、張承、孫怡忠於為國,通達治體。可令陳上便宜,蠲除苛煩,愛養士馬,撫循百姓。五年之外,十年之內,遠者歸復,近者盡力,兵不血刃,而大事可定也。臣聞『鳥之將死其鳴也哀,人之將死其言也善』,故子囊臨終,遺言戒時,君子以為忠,豈況臣登,其能已乎?願陛下留意聽采,臣雖死之日,猶生之年也。」旣絕而後書聞,權益以摧感,言則隕涕。是歲,赤烏四年也。謝景時為豫章太守,不勝哀情,棄官奔赴,拜表自劾。權曰:「君與太子從事,異於他吏。」使中使慰勞,聽復本職,發遣還郡。謚登曰宣太子。吳書曰:初葬句容,置園邑,奉守如法,後三年改葬蔣陵。

子璠、希,皆早卒,次子英,封吳侯。五鳳元年,英以大將軍孫峻擅權,謀誅峻,事覺自殺,國除。吳歷曰:孫和以無罪見殺,衆庶皆懷憤歎,前司馬桓慮因此招合將吏,欲共殺峻立英,事覺,皆見殺,英實不知。

謝景者字叔發,南陽宛人。在郡有治迹,吏民稱之,以為前有顧劭,其次即景。數年卒官。


\end{pinyinscope}