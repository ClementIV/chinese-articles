\article{高柔傳}

\begin{pinyinscope}
高柔字文惠,陳留圉人也。父靖,為蜀郡都尉。

陳留耆舊傳曰:靖高祖父固,不仕王莽世,為淮陽太守所害,以烈節垂名。固子慎,字孝甫。敦厚少華,有沈深之量。撫育孤兄子五人,恩義甚篤。琅邪相何英嘉其行履,以女妻焉。英即車騎將軍熈之父也。慎歷二縣令、東萊太守。老病歸家,草屋蓬戶,甕缶無儲。其妻謂之曰:「君累經宰守,積有年歲,何能不少為儲畜以遺子孫乎?」慎曰:「我以勤身清名為之基,以二千石遺之,不亦可乎!」子式,至孝,常盡力供養。永初中,螟蝗為害,獨不食式麥,圉令周彊以表州郡。太守楊舜舉式孝子,讓不行。後以孝廉為郎。次子昌,昌弟賜,並為刺史、郡守。式子弘,孝廉。弘生靖。柔留鄉里,謂邑中曰:「今者英雄並起,陳留四戰之地也。曹將軍雖據兖州,本有四方之圖,未得安坐守也。而張府君先得志於陳留,吾恐變乘間作也,欲與諸君避之。」衆人皆以張邈與太祖善,柔又年少,不然其言。柔從兄幹,袁紹甥也,謝承後漢書曰:幹字元才。才志弘邈,文武秀出。父躬,蜀郡太守。祖賜,司隷校尉。案陳留耆舊傳及謝承書,幹應為柔從父,非從兄也。未知何者為誤。在河北呼柔,柔舉宗從之。會靖卒於西州,時道路艱澁,兵寇縱橫,而柔冒艱險詣蜀迎喪,辛苦荼毒,無所不嘗,三年乃還。

太祖平袁氏,以柔為菅長。縣中素聞其名,姧吏數人皆自引去。柔教曰:「昔邴吉臨政,吏嘗有非,猶尚容之。況此諸吏,於吾未有失乎!其召復之。」咸還,皆自勵,咸為佳吏。高幹旣降,頃之以并州叛。柔自歸太祖,太祖欲因事誅之,以為刺姧令史;處法允當,獄無留滯,辟為丞相倉曹屬。魏氏春秋曰:柔旣處法平允,又夙夜匪懈,至擁膝抱文書而寢。太祖嘗夜微出,觀察諸吏,見柔,哀之,徐解裘覆柔而去。自是辟焉。太祖欲遣鍾繇等討張魯,柔諫,以為今猥遣大兵,西有韓遂、馬超,謂為己舉,將相扇動作逆,宜先招集三輔,三輔苟平,漢中可傳檄而定也。繇入關,遂、超等果反。

魏國初建,為尚書郎。轉拜丞相理曹掾,令曰:「夫治定之化,以禮為首。撥亂之政,以刑為先。是以舜流四凶族,臯陶作士。漢祖除秦苛法,蕭何定律。掾清識平當,明于憲典,勉恤之哉!」鼓吹宋金等在合肥亡逃。舊法,軍征士亡,考竟其妻子。太祖患猶不息,更重其刑。金有母妻及二弟皆給官,主者奏盡殺之。柔啟曰:「士卒亡軍,誠在可疾,然竊聞其中時有悔者。愚謂乃宜貸其妻子,一可使賊中不信,二可使誘其還心。正如前科,固已絕其意望,而猥復重之,柔恐自今在軍之士,見一人亡逃,誅將及己,亦且相隨而走,不可復得殺也。此重刑非所以止亡,乃所以益走耳。」太祖曰:「善。」即止不殺金母、弟,蒙活者甚衆。

遷為潁川太守,復還為法曹掾。時置校事盧洪、趙達等,使察羣下,柔諫曰:「設官分職,各有所司。今置校事,旣非居上信下之旨。又達等數以憎愛擅作威福,宜檢治之。」太祖曰:「卿知達等,恐不如吾也。要能刺舉而辨衆事,使賢人君子為之,則不能也。昔叔孫通用羣盜,良有以也。」達等後姧利發,太祖殺之以謝於柔。

文帝踐阼,以柔為治書侍御史,賜爵關內侯,轉加治書執法。民間數有誹謗妖言,帝疾之,有妖言輒殺,而賞告者。柔上疏曰;「今妖言者必戮,告之者輒賞。旣使過誤無反善之路,又將開凶狡之羣相誣罔之漸,誠非所以息姧省訟,緝熈治道也。昔周公作誥,稱殷之祖宗,咸不顧小人之怨。在漢太宗,亦除妖言誹謗之令。臣愚以為宜除妖謗賞告之法,以隆天父養物之仁。」帝不即從,而相誣告者滋甚。帝乃下詔:「敢以誹謗相告者,以所告者罪罪之。」於是遂絕。校事劉慈等,自黃初初數年之閒,舉吏民姧罪以萬數,柔皆請懲虛實;其餘小小挂法者,不過罰金。四年,遷為廷尉。

魏初,三公無事,又希與朝政。柔上疏曰:「天地以四時成功,元首以輔弼興治;成湯杖阿衡之佐,文、武憑旦、望之力,逮至漢初,蕭、曹之儔並以元勳代作心膂,此皆明王聖主任臣於上,賢相良輔股肱於下也。今公輔之臣,皆國之棟梁,民所具瞻,而置之三事,不使知政,遂各偃息養高,鮮有進納,誠非朝廷崇用大臣之義,大臣獻可替否之謂也。古者刑政有疑,輒議於槐棘之下。自今之後,朝有疑議及刑獄大事,宜數以咨訪三公。三公朝朔望之日,又可特延入,講論得失,博盡事情,庶有裨起天聽,弘益大化。」帝嘉納焉。

帝以宿嫌,欲枉法誅治書執法鮑勛,而柔固執不從詔命。帝怒甚,遂召柔詣臺;遣使者承指至廷尉考竟勛,勛死乃遣柔還寺。

明帝即位,封柔延壽亭侯。時博士執經,柔上疏曰:「臣聞遵道重學,聖人洪訓;褒文崇儒,帝者明義。昔漢末陵遲,禮樂崩壞,雄戰虎爭,以戰陣為務,遂使儒林之羣,幽隱而不顯。太祖初興,愍其如此,在於撥亂之際,並使郡縣立教學之官。高祖即位,遂闡其業,興復辟雍,州立課試,於是天下之士,復聞庠序之教,親俎豆之禮焉。陛下臨政,允迪叡哲,敷弘大猷,光濟先軌,雖夏啟之承基,周成之繼業,誠無以加也。然今博士皆經明行脩,一國清選,而使遷除限不過長,懼非所以崇顯儒術,帥勵怠惰也。孔子稱『舉善而教不能則勸』,故楚禮申公,學士銳精,漢隆卓茂,搢紳競慕。臣以為博士者,道之淵藪,六藝所宗,宜隨學行優劣,待以不次之位。敦崇道教,以勸學者,於化為弘。」帝納之。

後大興殿舍,百姓勞役;廣采衆女,充盈後宮;後宮皇子連夭,繼嗣未育。柔上疏曰:「二虜狡猾,潛自講肄,謀動干戈,未圖束手;宜畜養將士,繕治甲兵,以逸待之。而頃興造殿舍,上下勞擾;若使吳、蜀知人虛實,通謀并勢,復俱送死,甚不易也。昔漢文惜十家之資,不營小臺之娛;去病慮匈奴之害,不遑治第之事。況今所損者非惟百金之費,所憂者非徒北狄之患乎?可粗成見所營立,以充朝宴之儀。訖罷作者,使得就農。二方平定,復可徐興。昔軒轅以二十五子,傳祚彌遠;周室以姬國四十,歷年滋多。陛下聦達,窮理盡性,而頃皇子連多夭逝,熊羆之祥又未感應。羣下之心,莫不悒戚。周禮,天子后妃以下百二十人,嬪嬙之儀,旣以盛矣。竊聞後庭之數,或復過之,聖嗣不昌,殆能由此。臣愚以為可妙簡淑媛,以備內官之數,其餘盡遣還家。且以育精養神,專靜為寶。如此,則螽斯之徵,可庶而致矣。」帝報曰:「知卿忠允,乃心王室,輒克昌言;他復以聞。」

時獵法甚峻。宜陽典農劉龜竊於禁內射兎,其功曹張京詣校事言之。帝匿京名,收龜付獄。柔表請告者名,帝大怒曰:「劉龜當死,乃敢獵吾禁地。送龜廷尉,廷尉便當考掠,何復請告者主名,吾豈妄收龜邪?」柔曰:「廷尉,天下之平也,安得以至尊喜怒而毀法乎?」重復為奏,辭指深切。帝意寤,乃下京名。即還訊,各當其罪。

時制,吏遭大喪者,百日後皆給役。有司徒吏解弘遭父喪,後有軍事,受勑當行,以疾病為辭。詔怒曰:「汝非曾、閔,何言毀邪?」促收考竟。柔見弘信甚羸劣,奏陳其事,宜加寬貸。帝乃詔曰:「孝哉弘也!其原之。」

初,公孫淵兄晃,為叔父恭任內侍,先淵未反,數陳其變。及淵謀逆,帝不忍巿斬,欲就獄殺之。柔上疏曰:「書稱『用罪伐厥死,用德彰厥善』,此王制之明典也。晃及妻子叛逆之類,誠應梟縣,勿使遺育。而臣竊聞晃先數自歸,陳淵禍萌,雖為凶族,原心可恕。夫仲尼亮司馬牛之憂,祁奚明叔向之過,在昔之美義也。臣以為晃信有言,宜貸其死;苟自無言,便當巿斬。今進不赦其命,退不彰其罪,閉著囹圄,使自引分,四方觀國,或疑此舉也。」帝不聽,竟遣使齎金屑飲晃及其妻子,賜以棺、衣,殯歛於宅。孫盛曰:聞五帝無誥誓之文,三王無盟祝之事,然則盟誓之文,始自三季,質任之作,起於周微。夫貞夫之一,則天地可動,機心內萌,則鷗鳥不下。況信不足焉而祈物之必附,猜生於我而望彼之必懷,何異挾冰求溫,抱炭希涼者哉?且夫要功之倫,陵肆之類,莫不背情任計,昧利忘親,縱懷慈孝之愛,或慮傾身之禍。是以周、鄭交惡,漢高請羹,隗嚻捐子,馬超背父,其為酷忍如此之極也,安在其因質委誠,取任永固哉?世主若能遠覽先王閑邪之至道,近鑒狡肆徇利之凶心,勝之以解網之仁,致之以來蘇之惠,燿之以雷霆之威,潤之以時雨之施,則不恭可歛衽於一朝,炰哮可屈膝於象魏矣。何必拘厥親以來其情,逼所愛以制其命乎?苟不能然,而仗夫計術,籠之以權數,檢之以一切,雖覽一室而庶徵於四海,法生鄙局,兾或半之暫益,自不得不有不忍之刑,以遂孥戮之罰,亦猶瀆盟由乎一人,而云俾墜其師,無克遺育之言耳。豈得復引四罪不及之典,司馬牛獲宥之義乎?假令任者皆不保其父兄,輒有二三之言,曲哀其意而悉活之,則長人子危親自存之悖。子弟雖質,必無刑戮之憂,父兄雖逆,終無勦絕之慮。柔不究明此術非盛王之道,宜開張遠義,蠲此近制,而陳法內之刑以申一人之命,可謂心存小善,非王者之體。古者殺人之中,又有仁焉。刑之於獄,未為失也。臣松之以為辨章事理,貴得當時之宜,無為虛唱大言而終歸無用。浮誕之論,不切於實,猶若畫魑魅之象,而躓於犬馬之形也。質任之興,非防近世,況三方鼎峙,遼東偏遠,羈其親屬以防未然,不為非矣。柔謂晃有先言之善,宜蒙原心之宥。而盛責柔不能開張遠理,蠲此近制。不達此言竟為何謂?若云猜防為非,質任宜廢,是謂應大明先王之道,不預任者生死也。晃之為任,歷年已久,豈得於殺活之際,方論至理之本。是何異叢棘旣繁,事須判決,空論刑措之美,無聞當不之實哉?其為迂闊,亦已甚矣,漢高事窮理迫,權以濟親,而總之酷忍之科,旣已大有所誣。且自古已來,未有子弟妄告父兄以圖全身者,自存之悖,未之或聞。晃以兄告弟,而其事果驗。謂晃應殺,將以遏防。若言之亦死,不言亦死,豈不杜歸善之心,失正刑之中哉?若趙括之母,以先請獲免,鍾會之兄,以密言全子,古今此比,蓋為不少。晃之前言,事同斯例,而獨遇否閉,良可哀哉!

是時,殺禁地鹿者身死,財產沒官,有能覺告者厚加賞賜。柔上疏曰:「聖王之御世,莫不以廣農為務,儉用為資。夫農廣則穀積,用儉則財畜,畜財積穀而有憂患之虞者,未之有也。古者,一夫不耕,或為之饑;一婦不織,或為之寒。中閒已來,百姓供給衆役,親田者旣減,加頃復有獵禁,羣鹿犯暴,殘食生苗,處處為害,所傷不貲。民雖障防,力不能禦。至如熒陽左右,周數百里,歲略不收,元元之命,實可矜傷。方今天下生財者甚少,而麋鹿之損者甚多。卒有兵戎之役,凶年之災,將無以待之。惟陛下覽先聖之所念,愍稼穡之艱難,寬放民間,使得捕鹿,遂除其禁,則衆庶久濟,莫不恱預矣。」魏名臣奏載柔上疏曰:「臣深思陛下所以不早取此鹿者,誠欲使極蕃息,然後大取以為軍國之用。然臣竊以為今鹿但有日耗,終無從得多也。何以知之?今禁地廣輪且千餘里,臣下計無慮其中有虎大小六百頭,狼有五百頭,狐萬頭。使大虎一頭三日食一鹿,一虎一歲百二十鹿,是為六百頭虎一歲食七萬二千頭鹿也。使十狼日共食一鹿,是為五百頭狼一歲共食萬八千頭鹿。鹿子始生,未能善走,使十狐一日共食一子,比至健走一月之間,是為萬狐一月共食鹿子三萬頭也。大凡一歲所食十二萬頭。其鵰鶚所害,臣置不計。以此推之,終無從得多,不如早取之為便也。」

頃之,護軍營士竇禮近出不還。營以為亡,表言逐捕,沒其妻盈及男女為官奴婢。盈連至州府,稱冤自訟,莫有省者。乃辭詣廷尉。柔問曰:「汝何以知夫不亡?」盈垂泣對曰:「夫少單特,養一老嫗為母,事甚恭謹,又哀兒女,撫視不離,非是輕狡不顧室家者也。」柔重問曰:「汝夫不與人有怨讎乎?」對曰:「夫良善,與人無讎。」又曰:「汝夫不與人交錢財乎?」對曰:「嘗出錢與同營士焦子文,求不得。」時子文適坐小事繫獄,柔乃見子文,問所坐。言次,曰:「汝頗曾舉人錢不?」子文曰:「自以單貧,初不敢舉人錢物也。」柔察子文色動,遂曰:「汝昔舉竇禮錢,何言不邪?」子文恠知事露,應對不次。柔曰:「汝已殺禮,便宜早服。」子文於是叩頭,具首殺禮本末,埋藏處所。柔便遣吏卒,承子文辭往掘禮,即得其屍。詔書復盈母子為平民。班下天下,以禮為戒。

在官二十三年,轉為太常,旬日遷司空,後徙司徒。太傅司馬宣王奏免曹爽,皇太后詔召柔假節行大將軍事,據爽營。太傅謂柔曰:「君為周勃矣。」爽誅,進封萬歲鄉侯。高貴鄉公即位,進封安國侯,轉為太尉。常道鄉公即位,增邑并前四千,前後封二子亭侯。景元四年,年九十薨,謚曰元侯。孫渾嗣。咸熈中,開建五等,以柔等著勳前朝,改封渾昌陸子。晉諸公贊曰:柔長子儁,大將軍掾,次誕,歷三州刺史、太僕。誕放率不倫,而決烈過人。次光,字宣茂,少習家業,明練法理。晉武帝世,為黃沙御史,與中丞同,遷守廷尉,後即真。兄誕與光異操,謂光小節,常輕侮之,而光事誕愈謹。終於尚書令。追贈司空。


\end{pinyinscope}