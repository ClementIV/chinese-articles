\article{xun-yu-chuan}

\begin{pinyinscope}
荀彧字文若,潁川潁陰人也。祖父淑,字季和,朗陵令。當漢順、桓之間,知名當世。有子八人,號曰八龍。彧父緄,濟南相。叔父爽,司空。

續漢書曰:淑有高才,王暢、李膺皆以為師,為朗陵侯相,號稱神君。張璠漢紀曰:淑博學有高行,與李固、李膺同志友善,拔李昭於小吏,友黃叔度於幼童,以賢良方正徵,對策譏切梁氏,出補朗陵侯相,卒官。八子:儉、緄、靖、燾、詵、爽、肅、旉。音敷。爽字慈明,幼好學,年十二,通春秋、論語,耽思經典,不應徵命,積十數年。董卓秉政,復徵爽,爽欲遁去,吏持之急。詔下郡,即拜平原相。行至苑陵,又追拜光祿勳。視事三日,策拜司空。爽起自布衣,九十五日而至三公。淑舊居西豪里,縣令苑康曰昔高陽氏有才子八人,署其里為高陽里。靖字叔慈,亦有至德,名幾亞爽,隱居終身。皇甫謐逸士傳:或問許子將,靖與爽孰賢?子將曰:「二人皆玉也,慈明外朗,叔慈內潤。」

彧年少時,南陽何顒異之,曰:「王佐才也。」典略曰:中常侍唐衡欲以女妻汝南傅公明,公明不娶,轉以與彧。父緄慕衡勢,為彧娶之。彧為論者所譏。臣松之案:漢紀云唐衡以桓帝延熹七年死,計彧于時年始二歲,則彧婚之日,衡之沒乆矣。慕勢之言為不然也。臣松之又以為緄八龍之一,必非苟得者也,將有逼而然,何云慕勢哉?昔鄭忽以違齊致譏,雋生以拒霍見美,致譏在於失援,見美嘉其慮遠,並無交至之害,故得各全其志耳。至於閹豎用事,四海屏氣;左悺、唐衡殺生在口。故于時諺云「左迴天,唐獨坐」,言威權莫二也。順之則六親以安,忤違則大禍立至;斯誠以存易亡,蒙恥期全之日。昔蔣詡姻于王氏,無損清高之操,緄之此婚,庸何傷乎!永漢元年,舉孝廉,拜守宮令。董卓之亂,求出補吏。除亢父令,遂棄官歸,謂父老曰:「潁川,四戰之地也,天下有變,常為兵衝,宜亟去之,無乆留。」鄉人多懷土猶豫,會兾州牧同郡韓馥遣騎迎之,莫有隨者,彧獨將宗族至兾州。而袁紹已奪馥位,待彧以上賔之禮。彧弟諶及同郡辛評、郭圖,皆為紹所任。彧度紹終不能成大事,時太祖為奮武將軍,在東郡,初平二年,彧去紹從太祖。太祖大恱曰:「吾之子房也。」以為司馬,時年二十九。是時,董卓威陵天下,太祖以問彧,彧曰:「卓暴虐已甚,必以亂終,無能為也。」卓遣李傕等出關東,所過虜略,至潁川、陳留而還。鄉人留者多見殺略。明年,太祖領兖州牧,後為鎮東將軍,彧常以司馬從。興平元年,太祖征陶謙,任彧留事。會張邈、陳宮以兖州反,潛迎呂布。布旣至,邈乃使劉翊告彧曰:「呂將軍來助曹使君擊陶謙,宜亟供其軍食。」衆疑惑。彧知邈為亂,即勒兵設備,馳召東郡太守夏侯惇,而兖州諸城皆應布矣。時太祖悉軍攻謙,留守兵少,而督將大吏多與邈、宮通謀。惇至,其夜誅謀叛者數十人,衆乃定。豫州刺史郭貢帥衆數萬來至城下,或言與呂布同謀,衆甚懼。貢求見彧,彧將往。惇等曰:「君,一州鎮也,往必危,不可。」彧曰:「貢與邈等,分非素結也,今來速,計必未定;及其未定說之,縱不為用,可使中立,若先疑之,彼將怒而成計。」貢見彧無懼意,謂鄄城未易攻,遂引兵去。又與程昱計,使說范、東阿,卒全三城,以待太祖。太祖自徐州還擊布濮陽,布東走。二年夏,太祖軍乘氏,大饑,人相食。

陶謙死,太祖欲遂取徐州,還乃定布。彧曰:「昔高祖保關中,光武據河內,皆深根固本以制天下,進足以勝敵,退足以堅守,故雖有困敗而終濟大業。將軍本以兖州首事,平山東之難,百姓無不歸心恱服。且河、濟,天下之要地也,今雖殘壞,猶易以自保,是亦將軍之關中、河內也,不可以不先定。今以破李封、薛蘭,若分兵東擊陳宮,宮必不敢西顧,以其閒勒兵收熟麥,約食畜穀,一舉而布可破也。破布,然後南結揚州,共討袁術,以臨淮、泗。若舍布而東,多留兵則不足用,少留兵則民皆保城,不得樵採。布乘虛寇暴,民心益危,唯鄄城、范、衞可全,其餘非己之有,是無兖州也。若徐州不定,將軍當安所歸乎?且陶謙雖死,徐州未易亡也。彼懲往年之敗,將懼而結親,相為表裏。今東方皆以收麥,必堅壁清野以待將軍,將軍攻之不拔,略之無獲,不出十日,則十萬之衆未戰而自困耳。臣松之以為于時徐州未平,兖州又叛,而云十萬之衆,雖是抑抗之言,要非寡弱之稱。益知官渡之役,不得云兵不滿萬也。前討徐州,威罰實行,曹瞞傳云:自京師遭董卓之亂,人民流移東出,多依彭城間。遇太祖至,坑殺男女數萬口於泗水,水為不流。陶謙帥其衆軍武原,太祖不得進。引軍從泗南攻取慮、睢陵、夏丘諸縣,皆屠之;雞犬亦盡,墟邑無復行人。其子弟念父兄之恥,必人自為守,無降心,就能破之,尚不可有也。夫事固有棄此取彼者,以大易小可也,以安易危可也,權一時之勢,不患本之不固可也。今三者莫利,願將軍熟慮之。」太祖乃止。大收麥,復與布戰,分兵平諸縣。布敗走,兖州遂平。

建安元年,太祖擊破黃巾。漢獻帝自河東還洛陽。太祖議奉迎都許,或以山東未平,韓暹、楊奉新將天子到洛陽,北連張楊,未可卒制。彧勸太祖曰:「昔晉文公納周襄王而諸侯景從,高祖東伐為義帝縞素而天下歸心。自天子播越,將軍首唱義兵,徒以山東擾亂,未能遠赴關右,然猶分遣將帥,蒙險通使,雖禦難于外,乃心無不在王室,是將軍匡天下之素志也。今車駕旋軫,東京榛蕪,義士有存本之思,百姓感舊而增哀。誠因此時,奉主上以從民望,大順也;秉至公以服雄傑,大略也;扶弘義以致英俊,大德也。天下雖有逆節,必不能為累,明矣。韓暹、楊奉其敢為害!若不時定,四方生心,後雖慮之,無及。」太祖遂至洛陽,奉迎天子都許。天子拜太祖大將軍,進彧為漢侍中,守尚書令。常居中持重,典略曰:彧折節下士,坐不累席。其在臺閣,不以私欲撓意。彧有羣從一人,才行實薄,或謂彧:「以君當事,不可不以某為議郎邪?」彧笑曰:「官者所以表才也,若如來言,衆人其謂我何邪!」其持心平正皆類此。太祖雖征伐在外,軍國事皆與彧籌焉。典略曰:彧為人偉美。又平原禰衡傳曰:衡字正平,建安初,自荊州北游許都,恃才傲逸,臧否過差,見不如己者不與語,人皆以是憎之。唯少府孔融高貴其才,上書薦之曰:「淑質貞亮,英才卓犖。初涉藝文,升堂覩奧;目所一見,輒誦於口,耳所暫聞,不忘於心。性與道合,思若有神。弘羊心計,安世默識,以衡準之,誠不足恠。」衡時年二十四。是時許都雖新建,尚饒人士。衡嘗書一刺懷之,字漫滅而無所適。或問之曰:「何不從陳長文、司馬伯達乎?」衡曰:「卿欲使我從屠沽兒輩也!」又問曰:「當今許中,誰最可者?」衡曰:「大兒有孔文舉,小兒有楊德祖。」又問:「曹公、荀令君、趙盪寇皆足蓋世乎?」衡稱曹公不甚多;又見荀有儀容,趙有腹尺,因荅曰:「文若可借面吊喪,稚長可使監廚請客。」其意以為荀但有貌,趙健啖肉也。於是衆人皆切齒。衡知衆不恱,將南還荊州。裝束臨發,衆人為祖道,先設供帳於城南,自共相誡曰:「衡數不遜,今因其後到,以不起報之。」及衡至,衆人皆坐不起,衡乃號咷大哭。衆人問其故,衡曰:「行屍柩之間,能不悲乎?」衡南見劉表,表甚禮之。將軍黃祖屯夏口,祖子射與衡善,隨到夏口。祖嘉其才,每在坐,席有異賔,介使與衡談。後衡驕蹇,荅祖言徘優饒言,祖以為罵己也,大怒,顧伍伯捉頭出。左右遂扶以去,拉而殺之。臣松之以本傳不稱彧容貌,故載典略與衡傳以見之。又潘勗為彧碑文,稱彧「瓌姿奇表」。張衡文士傳曰:孔融數薦衡於太祖,欲與相見,而衡疾惡之,意常憤懣。因狂疾不肯往,而數有言論。太祖聞其名,圖欲辱之,乃錄為鼓吏。後至八月朝,大宴,賔客並會。時鼓吏擊鼓過,皆當脫其故服,易着新衣。次衡,衡擊為漁陽參檛,容態不常,音節殊妙。坐上賔客聽之,莫不慷慨。過不易衣,吏呵之,衡乃當太祖前,以次脫衣,裸身而立,徐徐乃着褌冐畢,復擊鼓參檛,而顏色不怍。太祖大笑,告四坐曰:「本欲辱衡,衡反辱孤。」至今有漁陽參檛,自衡造也。融深責數衡,并宣太祖意,欲令與太祖相見。衡許之,曰:「當為卿往。」至十月朝,融先見太祖,說「衡欲求見」。至日晏,衡着布單衣,疏巾履,坐太祖營門外,以杖捶地,數罵太祖。太祖勑外廄急具精馬三匹,并騎二人,謂融曰:「禰衡豎子,乃敢爾!孤殺之無異於雀鼠,顧此人素有虛名,遠近所聞,今日殺之,人將謂孤不能容。今送與劉表,視卒當何如?」乃令騎以衡置馬上,兩騎扶送至南陽。傅子曰:衡辯於言而剋于論,見荊州牧劉表日,所以自結於表者甚至,表恱之以為上賔。衡稱表之美盈口,而論表左右不廢繩墨。於是左右因形而譖之,曰:「衡稱將軍之仁,西伯不過也,唯以為不能斷;終不濟者,必由此也。」是言實指表智短,而非衡所言也。表不詳察,遂疏衡而逐之。衡以交絕於劉表,智窮於黃祖,身死名滅,為天下笑者,譖之者有形也。太祖問彧:「誰能代卿為我謀者?」彧言「荀攸、鍾繇」。先是,彧言策謀士,進戲志才。志才卒,又進郭嘉。太祖以彧為知人,諸所進達皆稱職,唯嚴象為楊州,韋康為涼州,後敗亡。三輔決錄曰:象字文則,京兆人。少聦博,有膽智。以督軍御史中丞詣揚州討袁術,會術病卒,因以為揚州刺史。建安五年,為孫策廬江太守李術所殺,時年三十八。象同郡趙岐作三輔決錄,恐時人不盡其意,故隱其書,唯以示象。康字元將,亦京兆人。孔融與康父端書曰:「前日元將來,淵才亮茂,雅度弘毅,偉世之器也。昨日仲將又來,懿性貞實,文愍篤誠,保家之主也。不意雙珠,近出老蚌,甚珍貴之。」端從涼州牧徵為太僕,康代為涼州刺史,時人榮之。後為馬超所圍,堅守歷時,救軍不至,遂為超所殺。仲將名誕,見劉邵傳。

自太祖之迎天子也,袁紹內懷不服。紹旣并河朔,天下畏其彊。太祖方東憂呂布,南拒張繡,而繡敗太祖軍於宛。紹益驕,與太祖書,其辭悖慢。太祖大怒,出入動靜變於常,衆皆謂以失利於張繡故也。鍾繇以問彧,彧曰:「公之聦明,必不追咎往事,殆有他慮。」則見太祖問之,太祖乃以紹書示彧,曰:「今將討不義,而力不敵,何如?」彧曰:「古之成敗者,誠有其才,雖弱必彊,苟非其人,雖彊易弱,劉、項之存亡,足以觀矣。今與公爭天下者,唯袁紹爾。紹貌外寬而內忌,任人而疑其心,公明達不拘,唯才所宜,此度勝也。紹遲重少決,失在後機,公能斷大事,應變無方,此謀勝也。紹御軍寬緩,法令不立,士卒雖衆,其實難用,公法令旣明,賞罰必行,士卒雖寡,皆爭致死,此武勝也。紹憑世資,從容飾智,以收名譽,故士之寡能好問者多歸之,公以至仁待人,推誠心不為虛美,行己謹儉,而與有功者無所恡惜,故天下忠正效實之士咸願為用,此德勝也。夫以四勝輔天子,扶義征伐,誰敢不從?紹之彊其何能為!」太祖恱。彧曰:「不先取呂布,河北亦未易圖也。」太祖曰:「然。吾所惑者,又恐紹侵擾關中,亂羌、胡,南誘蜀漢,是我獨以兖、豫抗天下六分之五也。為將柰何?」彧曰:「關中將帥以十數,莫能相一,唯韓遂、馬超最彊。彼見山東方爭,必各擁衆自保。今若撫以恩德,遣使連和,相持雖不能乆安,比公安定山東,足以不動。鍾繇可屬以西事。則公無憂矣。」

三年,太祖旣破張繡,東禽呂布,定徐州,遂與袁紹相拒。孔融謂彧曰:「紹地廣兵彊;田豐、許攸,智計之士也,為之謀;審配、逢紀,盡忠之臣也,任其事;顏良、文醜,勇冠三軍,統其兵:殆難克乎!」彧曰:「紹兵雖多而法不整。田豐剛而犯上,許攸貪而不治。審配專而無謀,逢紀果而自用,此二人留知後事,若攸家犯其法,必不能縱也,不縱,攸必為變。顏良、文醜,一夫之勇耳,可一戰而禽也。」

五年,與紹連戰。太祖保官渡,紹圍之。太祖軍糧方盡,書與彧,議欲還許以引紹。彧曰:「今軍食雖少,未若楚、漢在滎陽、成皐間也。是時劉、項莫肯先退,先退者勢屈也。公以十分居一之衆,畫地而守之,扼其喉而不得進,已半年矣。情見勢竭,必將有變,此用奇之時,不可失也。」太祖乃住。遂以奇兵襲紹別屯,斬其將淳于瓊等,紹退走。審配以許攸家不法,收其妻子,攸怒叛紹;顏良、文醜臨陣授首;田豐以諫見誅:皆如彧所策。

六年,太祖就穀東平之安民,糧少,不足與河北相支,欲因紹新破,以其間擊討劉表。彧曰:「今紹敗,其衆離心,宜乘其困,遂定之;而背兖、豫,遠師江、漢,若紹收其餘燼,承虛以出人後,則公事去矣。」太祖復次于河上。紹病死。太祖渡河,擊紹子譚、尚,而高幹、郭援侵略河東,關右震動,鍾繇帥馬騰等擊破之。語在繇傳。八年,太祖錄彧前後功,表封彧為萬歲亭侯。彧別傳載太祖表曰:「臣聞慮為功首,謀為賞本,野績不越廟堂,戰多不踰國勳。是故典阜之錫,不後營丘,蕭何之土,先於平陽。珍策重計,古今所尚。侍中守尚書令彧,積德累行,少長無悔,遭世紛擾,懷忠念治。臣自始舉義兵,周游征伐,與彧勠力同心,左右王略,發言授策,無施不效。彧之功業,臣由以濟,用披浮雲,顯光日月。陛下幸許,彧左右機近,忠恪祗順,如履薄冰,研精極銳,以撫庶事。天下之定,彧之功也。宜享高爵,以彰元勳。」彧固辭無野戰之勞,不通太祖表。太祖與彧書曰:「與君共事已來,立朝廷,君之相為匡弼,君之相為舉人,君之相為建計,君之相為密謀,亦以多矣。夫功未必皆野戰也,願君勿讓。」彧乃受。九年,太祖拔鄴,領兾州牧。或說太祖「宜復古置九州,則兾州所制者廣大,天下服矣。」太祖將從之,彧言曰:「若是,則兾州當得河東、馮翊、扶風、西河、幽、并之地,所奪者衆。前日公破袁尚,禽審配,海內震駭,必人人自恐不得保其土地,守其兵衆也;今使分屬兾州,將皆動心。且人多說關右諸將以閉關之計;今聞此,以為必以次見奪。一旦生變,雖有善守者,轉相脅為非,則袁尚得寬其死,而袁譚懷貳,劉表遂保江、漢之閒,天下未易圖也。願公急引兵先定河北,然後脩復舊京,南臨荊州,責貢之不入,則天下咸知公意,人人自安。天下大定,乃議古制,此社稷長乆之利也。」太祖遂寢九州議。

是時荀攸常為謀主。彧兄衍以監軍校尉守鄴,都督河北事。太祖之征袁尚也,高幹密遣兵謀襲鄴,衍逆覺,盡誅之,以功封列侯。荀氏家傳曰:衍字休若,彧第三兄。彧第四兄諶,字友若,事見袁紹傳。陳羣與孔融論汝、潁人物,羣曰:「荀文若、公達、休若、友若、仲豫,當今並無對。」衍子紹,位至太僕。紹子融,字伯雅,與王弼、鍾會俱知名,為洛陽令,參大將軍軍事,與弼、會論易、老義,傳於世。諶子閎,字仲茂,為太子文學掾。時有甲乙疑論,閎與鍾繇、王朗、袁渙議各不同。文帝與繇書曰:「袁、王國士,更為脣齒,荀閎勁悍,往來銳師,真君侯之勍敵,左右之深憂也。」終黃門侍郎。閎從孫惲字景文,太子中庶子,亦知名。與賈充共定音律,又作易集解。仲豫名恱,朗陵長儉之少子,彧從父兄也。張璠漢紀稱恱清虛沈靜,善於著述。建安初為祕書監侍中,被詔刪漢書作漢紀三十篇,因事以明臧否,致有典要;其書大行於世。太祖以女妻彧長子惲,後稱安陽公主。彧及攸並貴重,皆謙冲節儉,祿賜散之宗族知舊,家無餘財。十二年,復增彧邑千戶,合二千戶。彧別傳曰:太祖又表曰:「昔袁紹侵入郊甸,戰於官渡。時兵少糧盡,圖欲還許,書與彧議,彧不聽臣。建宜住之便,恢進討之規,更起臣心,易其愚慮,遂摧大逆,覆取其衆。此彧覩勝敗之機,略不世出也。及紹破敗,臣糧亦盡,以為河北未易圖也,欲南討劉表。彧復止臣,陳其得失,臣用反斾,遂吞凶族,克平四州。向使臣退於官渡,紹必鼓行而前,有傾覆之形,無克捷之勢。後若南征,委棄兖、豫,利旣難要,將失本據。彧之二策,以亡為存,以禍致福,謀殊功異,臣所不及也。是以先帝貴指蹤之功,薄搏獲之賞;古人尚帷幄之規,下攻拔之捷。前所賞錄,未副彧巍巍之勳,乞重平議,疇其戶邑。」彧深辭讓,太祖報之曰:「君之策謀,非但所表二事。前後謙沖,欲慕魯連先生乎?此聖人達節者所不貴也。昔介子推有言『竊人之財,猶謂之盜』。況君密謀安衆,光顯於孤者以百數乎!以二事相還而復辭之,何取謙亮之多邪!」太祖欲表彧為三公,彧使荀攸深讓,至于十數,太祖乃止。

太祖將伐劉表,問彧策安出,彧曰:「今華夏已平,南土知困矣。可顯出宛、葉而間行輕進,以掩其不意。」太祖遂行。會表病死,太祖直趨宛、葉如彧計,表子琮以州逆降。

十七年,董昭等謂太祖宜進爵國公,九錫備物,以彰殊勳,密以諮彧。彧以為太祖本興義兵以匡朝寧國,秉忠貞之誠,守退讓之實;君子愛人以德,不宜如此。太祖由是心不能平。會征孫權,表請彧勞軍于譙,因輒留彧,以侍中光祿大夫持節,參丞相軍事。太祖軍至濡須,彧疾留壽春,以憂薨,時年五十。謚曰敬侯。明年,太祖遂為魏公矣。魏氏春秋曰:太祖饋彧食,發之乃空器也,於是飲藥而卒。咸熈二年,贈彧太尉。彧別傳曰:彧自為尚書令,常以書陳事,臨薨,皆焚毀之,故奇策密謀不得盡聞也。是時征役草創,制度多所興復,彧嘗言於太祖曰:「昔舜分命禹、稷、契、皐陶以揆庶績,教化征伐,並時而用。及高祖之初,金革方殷,猶舉民能善教訓者,叔孫通習禮儀於戎旅之閒,世祖有投戈講藝、息馬論道之事,君子無終食之閒違仁。今公外定武功,內興文學,使干戈戢睦,大道流行,國難方弭,六禮俱治,此姬旦宰周之所以速平也。旣立德立功,而又兼立言,成仲尼述作之意;顯制度於當時,揚名於後世,豈不盛哉!若須武事畢而後制作,以稽治化,於事未敏。宜集天下大才通儒,考論六經,刊定傳記,存古今之學,除其煩重,以一聖真,並隆禮學,漸敦教化,則王道兩濟。」彧從容與太祖論治道,如此之類甚衆,太祖常嘉納之。彧德行周備,非正道不用心,名重天下,莫不以為儀表,海內英儁咸宗焉。司馬宣王常稱書傳遠事,吾自耳目所從聞見,逮百數十年間,賢才未有及荀令君者也。前後所舉者,命世大才,邦邑則荀攸、鍾繇、陳羣,海內則司馬宣王,及引致當世知名郗慮、華歆、王朗、荀恱、杜襲、辛毗、趙儼之儔,終為卿相,以十數人。取士不以一揆,戲志才、郭嘉等有負俗之譏,杜畿簡傲少文,皆以智策舉之,終各顯名。荀攸後為魏尚書令,亦推賢進士。太祖曰:「二荀令之論人,乆而益信,吾沒世不忘。」鍾繇以為顏子旣沒,能備九德,不貳其過,唯荀彧然。或問繇曰:「君雅重荀君,比之顏子,自以不及,可得聞乎?」曰:「夫明君師臣,其次友之。以太祖之聦明,每有大事,常先諮之荀君,是則古師友之義也。吾等受命而行,猶或不盡,相去顧不遠邪!」獻帝春秋曰:董承之誅,伏后與父完書,言司空殺董承,帝方為報怨。完得書以示彧,彧惡之,乆隱而不言。完以示妻弟樊普,普封以呈太祖,太祖陰為之備。彧後恐事覺,欲自發之,因求使至鄴,勸太祖以女配帝。太祖曰:「今朝廷有伏后,吾女何得以配上,吾以微功見錄,位為宰相,豈復賴女寵乎!」彧曰:「伏后無子,性又凶邪,往常與父書,言辭醜惡,可因此廢也。」太祖曰:「卿昔何不道之?」彧陽驚曰:「昔已嘗為公言也。」太祖曰:「此豈小事而吾忘之!」彧又驚曰:「誠未語公邪!昔公在官渡與袁紹相持,恐增內顧之念,故不言爾。」太祖曰:「官渡事後何以不言?」彧無對,謝闕而已。太祖以此恨彧,而外含容之,故世莫得知。至董昭建立魏公之議,彧意不同,欲言之於太祖。及齎璽書犒軍,飲饗禮畢,彧留請閒。太祖知彧欲言封事,揖而遣之,彧遂不得言。彧卒於壽春,壽春亡者告孫權,言太祖使彧殺伏后,彧不從,故自殺。權以露布於蜀,劉備聞之,曰:「老賊不死,禍亂未已。」臣松之案獻帝春秋云彧欲發伏后事而求使至鄴,而方誣太祖云「昔已嘗言」。言旣無徵,迴託以官渡之虞,俛仰之間,辭情頓屈,雖在庸人,猶不至此,何以玷累賢哲哉!凡諸云云,皆出自鄙俚,可謂以吾儕之言而厚誣君子者矣。袁暐虛罔之類,此最為甚也。

子惲,嗣侯,官至虎賁中郎將。初,文帝與平原侯植並有擬論,文帝曲禮事彧。及彧卒,惲又與植善,而與夏侯尚不穆,文帝深恨惲。惲早卒,子甝、霬,音翼。以外甥故猶寵待。惲弟俁,御史中丞,俁弟詵,大將軍從事中郎,皆知名,早卒。荀氏家傳曰:惲字長倩,俁字叔倩,詵字曼倩,俁子寓,字景伯。世語曰:寓少與裴楷、王戎、杜默俱有名京邑,仕晉,位至尚書,名見顯著。子羽嗣,位至尚書。詵弟顗,咸熈中為司空。晉陽秋曰:顗字景倩,幼為姊夫陳羣所異。博學洽聞,意思慎密。司馬宣王見顗,奇之,曰:「荀令君之子也。近見袁偘,亦曜卿之子也。」擢拜散騎侍郎。顗佐命晉室,位至太尉,封臨淮康公。嘗難鍾會「易無互體」,見稱於世。顗弟粲,字奉倩。何劭為粲傳曰:粲字奉倩,粲諸兄並以儒術論議,而粲獨好言道,常以為子貢稱夫子之言性與天道,不可得聞,然則六籍雖存,固聖人之糠粃。粲兄俁難曰:「易亦云聖人立象以盡意,繫辭焉以盡言,則微言胡為不可得而聞見哉?」粲荅曰:「蓋理之微者,非物象之所舉也。今稱立象以盡意,此非通于意外者也。繫辭焉以盡言,此非言乎繫表者也;斯則象外之意,繫表之言,固蘊而不出矣。」及當時能言者不能屈也。又論父彧不如從兄攸。彧立德高整,軌儀以訓物,而攸不治外形,慎密自居而已。粲以此言善攸,諸兄怒而不能迴也。太和初,到京邑與傅嘏談。嘏善名理而粲尚玄遠,宗致雖同,倉卒時或有格而不相得意。裴徽通彼我之懷,為二家騎驛,頃之,粲與嘏善。夏侯玄亦親。常謂嘏、玄曰:「子等在世塗間,功名必勝我,但識劣我耳!」嘏難曰:「能盛功名者,識也。天下孰有本不足而末有餘者邪?」粲曰:「功名者,志局之所獎也。然則志局自一物耳,固非識之所獨濟也。我以能使子等為貴,然未必齊子等所為也。」粲常以婦人者,才智不足論,自宜以色為主。驃騎將軍曹洪女有美色,粲於是娉焉,容服帷帳甚麗,專房歡宴。歷年後,婦病亡,未殯,傅嘏往喭粲;粲不哭而神傷。嘏問曰:「婦人才色並茂為難。子之娶也,遺才而好色。此自易遇,今何哀之甚?」粲曰:「佳人難再得!顧逝者不能有傾國之色,然未可謂之易遇。」痛悼不能已,歲餘亦亡,時年二十九。粲簡貴,不能與常人交接,所交皆一時俊傑。至葬夕,赴者裁十餘人,皆同時知名士也,哭之,感慟路人。惲子甝,嗣為散騎常侍,進爵廣陽鄉侯,年三十薨。子頵嗣。荀氏家傳曰:頵字溫伯,為羽林右監,早卒。頵子崧,字景猷。晉陽秋稱崧少有志操,雅好文學,孝義和愛,在朝恪勤,位至左右光祿大夫、開府儀同三司。崧子羨,字令則,清和有才。尚公主,少歷顯位,年二十八為北中郎將,徐、兖二州刺史,假節都督徐、兖、青三州諸軍事。在任十年,遇疾解職,卒於家,追贈驃騎將軍。羨孫伯子,今御史中丞也。霬官至中領軍,薨,謚曰貞侯,追贈驃騎將軍。子愷嗣。霬妻,司馬景王、文王之妹也,二王皆與親善。咸熈中,開建五等,霬以著勳前朝,改封愷南頓子。荀氏家傳曰:愷,晉武帝時為侍中。于寶晉紀曰:武帝使侍中荀顗、和嶠俱至東宮,觀察太子。顗還稱太子德識進茂,而嶠云聖質如初。孫盛曰「遣荀勗」,其餘語則同。臣松之案和嶠為侍中,荀顗亡沒乆矣。荀勗位亞台司,不與嶠同班,無緣方稱侍中。二書所云,皆為非也。考其時位,愷寔當之。愷位至征西大將軍。愷兄憺,少府。弟悝,護軍將軍,追贈車騎大將軍。


\end{pinyinscope}