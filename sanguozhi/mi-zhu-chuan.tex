\article{麋笁傳}

\begin{pinyinscope}
麋笁字子仲,東海朐人也。祖世貨殖,僮客萬人,貲產鉅億。

搜神記曰:笁甞從洛歸,未達家數十里,路傍見一婦人,從笁求寄載。行可數里,婦謝去,謂笁曰:「我天使也,當往燒東海麋笁家,感君見載,故以相語。」笁因私請之,婦曰:「不可得不燒。如此,君可馳去,我當緩行,日中火當發。」笁乃還家,遽出財物,日中而火大發。後徐州牧陶謙辟為別駕從事。謙卒,笁奉謙遺命,迎先主於小沛。建安元年,呂布乘先主之出拒袁術,襲下邳,虜先主妻子。先主轉軍廣陵海西,笁於是進妹於先主為夫人,奴客二千,金銀貨幣以助軍資;于時困匱,賴此復振。後曹公表笁領嬴郡太守,曹公集載公表曰:「泰山郡界廣遠,舊多輕悍,權時之宜,可分五縣為嬴郡,揀選清廉以為守將。偏將軍麋笁,素履忠貞,文武昭烈,請以笁領嬴郡太守,撫慰吏民。」笁弟芳為彭城相,皆去官,隨先主周旋。先主將適荊州,遣笁先與劉表相聞,以笁為左將軍從事中郎。益州旣平,拜為安漢將軍,班在軍師將軍之右。笁雍容敦雅,而幹翮非所長。是以待之以上賔之禮,未甞有所統御。然賞賜優寵,無與為比。

芳為南郡太守,與關羽共事,而私好攜貳,叛迎孫權,羽因覆敗。笁面縛請罪,先主慰諭以兄弟罪不相及,崇待如初。笁慙恚發病,歲餘卒。子威,官至虎賁中郎將。威子照,虎騎監。自笁至照,皆便弓馬,善射御云。


\end{pinyinscope}