\article{夫餘傳}

\begin{pinyinscope}
夫餘在長城之北,去玄菟千里,南與高句麗,東與挹婁,西與鮮卑接,北有弱水,方可二千里。戶八萬,其民土著,有宮室、倉庫、牢獄。多山陵、廣澤,於東夷之域最平敞。土地宜五穀,不生五果。其人麤大,性彊勇謹厚,不寇鈔。國有君王,皆以六畜名官,有馬加、牛加、豬加、狗加、大使、大使者、使者。邑落有豪民,名下戶皆為奴僕。諸加別主四出,道大者主數千家,小者數百家。食飲皆用俎豆,會同、拜爵、洗爵,揖讓升降。以殷正月祭天,國中大會,連日飲食歌舞,名曰迎鼓,於是時斷刑獄,解囚徒。在國衣尚白,白布大袂,袍、袴,履革鞜。出國則尚繒繡錦𦋺,大人加狐狸、狖白、黑貂之裘,以金銀飾帽。譯人傳辭,皆跪,手據地竊語。用刑嚴急,殺人者死,沒其家人為奴婢。竊盜一責十二。男女淫,婦人妬,皆殺之。尤憎妬,已殺,尸之國南山上,至腐爛。女家欲得,輸牛馬乃與之。兄死妻嫂,與匈奴同俗。其國善養牲,出名馬、赤玉、貂狖、美珠。珠大者如酸棗。以弓矢刀矛為兵,家家自有鎧仗。國之耆老自說古之亡人。作城柵皆員,有似牢獄。行道晝夜無老幼皆歌,通日聲不絕。有軍事亦祭天,殺牛觀蹄以占吉凶,蹄解者為凶,合者為吉。有敵,諸加自戰,下戶俱擔糧飲食之。其死,夏月皆用氷。殺人徇葬,多者百數。厚葬,有槨無棺。

魏畧曰:其俗停喪五月,以乆為榮。其祭亡者,有生有熟。喪主不欲速而他人彊之,常諍引以此為節。其居喪,男女皆純白,婦人著布靣衣,去環珮,大體與中國相彷彿也。

夫餘本屬玄菟。漢末,公孫度雄張海東,威服外夷,夫餘王尉仇台更屬遼東。時句麗、鮮卑彊,度以夫餘在二虜之間,妻以宗女。尉仇台死,簡位居立。無適子,有孽子麻余。位居死,諸加共立麻余。牛加兄子名位居,為大使,輕財善施,國人附之,歲歲遣使詣京都貢獻。正始中,幽州刺史毌丘儉討句麗,遣玄菟太守王頎詣夫餘,位居遣大加郊迎,供軍糧。季父牛加有二心,位居殺季父父子,籍沒財物,遣使簿斂送官。舊夫餘俗,水旱不調,五糓不熟,輙歸咎於王,或言當易,或言當殺。麻余死,其子依慮年六歲,立以為王。漢時,夫餘王葬用玉匣,常豫以付玄菟郡,王死則迎取以葬。公孫淵伏誅,玄菟庫猶有玉匣一具。今夫餘庫有玉璧、珪、瓚數代之物,傳世以為寶,耆老言先代之所賜也。魏畧曰:其國殷富,自先世以來,未甞破壞也。其印文言「濊王之印」,國有故城名濊城,蓋本濊貊之地,而夫餘王其中,自謂「亡人」,抑有似也。魏畧曰:舊志又言,昔北方有高離之國者,其王者侍婢有身,王欲殺之,婢云:「有氣如鷄子來下,我故有身。」後生子,王捐之於溷中,猪以喙噓之,徙至馬閑,馬以氣噓之,不死。王疑以為天子也,乃令其母收畜之,名曰東明,常令牧馬。東明善射,王恐奪其國也,欲殺之。東明走,南至施掩水,以弓擊水,魚鼈浮為橋,東明得渡,魚鼈乃解散,追兵不得渡。東明因都王夫餘之地。


\end{pinyinscope}