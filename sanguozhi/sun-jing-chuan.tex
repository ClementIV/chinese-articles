\article{孫靜傳}

\begin{pinyinscope}
孫靜字幼臺,堅季弟也。堅始舉事,靜糾合鄉曲及宗室五六百人以為保障,衆咸附焉。策破劉繇,定諸縣,進攻會稽,遣人請靜,靜將家屬與策會于錢唐。是時太守王朗拒策於固陵,策數渡水戰,不能克。靜說策曰:「朗負阻城守,難可卒拔。查瀆南去此數十里,而道之要徑也,宜從彼據其內,所謂攻其無備、出其不意者也。吾當自帥衆為軍前隊,破之必矣。」策曰:「善。」乃詐令軍中曰:「頃連雨水濁,兵飲之多腹痛,令促具甖缶數百口澄水。」至昏暮,羅以然火誑朗,便分軍夜投查瀆道,襲高遷屯。

臣松之案:今永興縣有高遷橋。查音祖加反。查音祖加反。朗大驚,遣故丹楊太守周昕等帥兵前戰。策破昕等,斬之,遂定會稽。會稽典錄曰:昕字大明。少游京師,師事太傅陳蕃,博覽羣書,明於風角,善推災異。辟太尉府,舉高第,稍遷丹楊太守。曹公起義兵,昕前後遣兵萬餘人助公征伐。袁術之在淮南也,昕惡其淫虐,絕不與通。獻帝春秋曰:袁術遣吳景攻昕,未拔,景乃募百姓敢從周昕者死不赦。昕曰:「我則不德,百姓何罪?」遂散兵,還本郡。表拜靜為奮武校尉,欲授之重任,靜戀墳墓宗族,不樂出身,求留鎮守。策從之。權統事,就遷昭義中郎將,終於家。有五子,暠、瑜、皎、奐、謙。暠三子:綽、超、恭。超為偏將軍。恭生峻。綽生綝。


\end{pinyinscope}