\article{ren-jun-chuan}

\begin{pinyinscope}
任峻字伯達,河南中牟人也。漢末擾亂,關東皆震。中牟令楊原愁恐,欲棄官走。峻說原曰:「董卓首亂,天下莫不側目,然而未有先發者,非無其心也,勢未敢耳。明府若能唱之,必有和者。」原曰:「為之柰何?」峻曰:「今關東有十餘縣,能勝兵者不減萬人,若權行河南尹事,總而用之,無不濟矣。」原從其計,以峻為主簿。峻乃為原表行尹事,使諸縣堅守,遂發兵。會太祖起關東,入中牟界,衆不知所從,峻獨與同郡張奮議,舉郡以歸太祖。峻又別收宗族及賔客家兵數百人,願從太祖。太祖大恱,表峻為騎都尉,妻以從妹,甚見親信。太祖每征伐,峻常居守以給軍。是時歲饑旱,軍食不足,羽林監潁川棗祗建置屯田,太祖以峻為典農中郎將,數年中所在積粟,倉廩皆滿。官渡之戰,太祖使峻典軍器糧運。賊數寇鈔絕糧道,乃使千乘為一部,十道方行,為複陳以營衞之,賊不敢近。軍國之饒,起於棗祗而成於峻。

魏武故事載令曰:「故陳留太守棗祗,天性忠能。始共舉義兵,周旋征討。後袁紹在兾州,亦貪祗,欲得之。祗深附託於孤,使領東阿令。呂布之亂,兖州皆叛,惟范、東阿完在,由祗以兵據城之力也。後大軍糧乏,得東阿以繼,祗之功也。及破黃巾定許,得賊資業。當興立屯田,時議者皆言當計牛輸穀,佃科以定。施行後,祗白以為僦牛輸穀,大收不增穀,有水旱災除,大不便。反覆來說,孤猶以為當如故,大收不可復改易。祗猶執之,孤不知所從,使與荀令君議之。時故軍祭酒侯聲云:『科取官牛,為官田計。如祗議,於官便,於客不便。』聲懷此云云,以疑令君。祗猶自信,據計畫還白,執分田之術。孤乃然之,使為屯田都尉,施設田業。其時歲則大收,後遂因此大田,豐足軍用,摧滅羣逆,克定天下,以隆王室。祗興其功,不幸早沒,追贈以郡,猶未副之。今重思之,祗宜受封,稽留至今,孤之過也。祗子處中,宜加封爵,以祀祗為不朽之事。」文士傳曰:祗本姓棘,先人避難,易為棗。孫據,字道彥,晉兾州刺史。據子嵩,字臺產,散騎常侍。並有才名,多所著述。嵩兄腆,字玄方,襄城太守,亦有文采。太祖以峻功高,乃表封為都亭侯,邑三百戶,遷長水校尉。

峻寬厚有度而見事理,每有所陳,太祖多善之。於饑荒之際,收卹朋友孤遺,中外貧宗,周急繼乏,信義見稱。建安九年薨,太祖流涕者乆之。子先嗣。先薨,無子,國除。文帝追錄功臣,謚峻曰成侯。復以峻中子覽為關內侯。


\end{pinyinscope}