\article{孫破虜吳夫人傳}

\begin{pinyinscope}
孫破虜吳夫人,吳主權母也。本吳人,徙錢唐,早失父母,與弟景居。孫堅聞其才貌,欲娶之。吳氏親戚嫌堅輕狡,將拒焉,堅甚以慙恨。夫人謂親戚曰:「何愛一女以取禍乎?如有不遇,命也。」於是遂許為婚,生四男一女。

搜神記曰:初,夫人孕而夢月入其懷,旣而生策。及權在孕,又夢日入其懷,以告堅曰:「昔姙策,夢月入我懷,今也又夢日入我懷,何也?」堅曰:「日月者陰陽之精,極貴之象,吾子孫其興乎!」

景常隨堅征伐有功,拜騎都尉。袁術上景領丹楊太守,討故太守周昕,遂據其郡。孫策與孫河、呂範依景,合衆共討涇縣山賊祖郎,郎敗走。會景為劉繇所迫,復北依術,術以為督軍中郎將,與孫賁共討樊能、于麋於橫江,又擊笮融、薛禮於秣陵。時策被創牛渚,降賊復反,景攻討,盡禽之。從討劉繇,繇奔豫章,策遣景、賁到壽春報術。術方與劉備爭徐州,以景為廣陵太守。術後僭號,策以書喻術,術不納,便絕江津,不與通,使人告景。景即委郡東歸,策復以景為丹楊太守。漢遣議郎王誧音普。銜命南行,表景為揚武將軍,領郡如故。

及權少年統業,夫人助治軍國,甚有補益。會稽典錄曰:策功曹魏騰,以迕意見譴,將殺之,士大夫憂恐,計無所出。夫人乃倚大井而謂策曰:「汝新造江南,其事未集,方當優賢禮士,捨過錄功。魏功曹在公盡規,汝今日殺之,則明日人皆叛汝。吾不忍見禍之及,當先投此井中耳。」策大驚,遽釋騰。夫人智略權譎,類皆如此。建安七年,臨薨,引見張昭等,屬以後事,合葬高陵。志林曰:按會稽貢舉簿,建安十二年到十三年闕,無舉者,云府君遭憂,此則吳后以十二年薨也。八年九年皆有貢舉,斯甚分明。

八年,景卒官,子奮授兵為將,封新亭侯,卒。吳書曰:權征荊州,拜奮吳郡都督,以鎮東方。子安嗣,安坐黨魯王霸死。奮弟祺,吳書曰:祺與張溫、顧譚友善,權令關平辭訟事。封都亭侯,卒。子纂嗣。纂妻即滕胤女也,胤被誅,并遇害。


\end{pinyinscope}