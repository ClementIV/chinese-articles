\article{程普傳}

\begin{pinyinscope}
程普字德謀,右北平土垠人也。初為州郡吏,有容貌計略,善於應對。從孫堅征伐,討黃巾於宛、鄧,破董卓於陽人,攻城野戰,身被創夷。

堅薨,復隨孫策在淮南,從攻廬江,拔之,還俱東渡。策到橫江、當利,破張英、于麋等,轉下秣陵、湖熟、句容、曲阿,普皆有功,增兵二千,騎五十匹。進破烏程、石木、波門、陵傳、餘杭,普功為多。策入會稽,以普為吳郡都尉,治錢唐。後徙丹楊都尉,居石城。復討宣城、涇、安吳、陵陽、春穀諸賊,皆破之。策甞攻祖郎,大為所圍,普與一騎共蔽扞策,驅馬疾呼,以矛突賊,賊披,策因隨出。後拜盪寇中郎將,領零陵太守,從討劉勳於尋陽,進攻黃祖於沙羨,還鎮石城。

策薨,與張昭等共輔孫權,遂周旋三郡,平討不服。又從征江夏,還過豫章,別討樂安。樂安平定,代太史慈備海昏,與周瑜為左右督,破曹公於烏林,又進攻南郡,走曹仁。拜裨將軍,領江夏太守,治沙羨,食四縣。

先出諸將,普最年長,時人皆呼程公。性好施與,喜士大夫。周瑜卒,代領南郡太守。權分荊州與劉備,普復還領江夏,遷盪寇將軍,卒。

吳書曰:普殺叛者數百人,皆使投火,即日病癘,百餘日卒。權稱尊號,追論普功,封子咨為亭侯。


\end{pinyinscope}