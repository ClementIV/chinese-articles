\article{闞澤傳}

\begin{pinyinscope}
闞澤字德潤,會稽山陰人也。家世農夫,至澤好學,居貧無資,常為人傭書,以供紙筆,所寫旣畢,誦讀亦遍。追師論講,究覽羣籍,兼通歷數,由是顯名。察孝廉,除錢唐長,遷郴令。孫權為驃騎將軍,辟補西曹掾;及稱尊號,以澤為尚書。嘉禾中,為中書令,加侍中。赤烏五年,拜太子太傅,領中書如故。

澤以經傳文多,難得盡用,乃斟酌諸家,刊約禮文及諸注說以授二宮,為制行出入及見賔儀,又著乾象歷注以正時日。每朝廷大議,經典所疑,輒諮訪之。以儒學勤勞,封都鄉侯。性謙恭篤慎,宮府小吏,呼召對問,皆為抗禮。人有非短,口未嘗及,容貌似不足者,然所聞少窮。權嘗問:「書傳篇賦,何者為美?」澤欲諷喻以明治亂,因對賈誼過秦論最善,權覽讀焉。初,以呂壹姦罪發聞,有司窮治,奏以大辟,或以為宜加焚裂,用彰元惡。權以訪澤,澤曰:「盛明之世,不宜復有此刑。」權從之。又諸官司有所患疾,欲增重科防,以檢御臣下,澤每曰「宜依禮、律」,其和而有正,皆此類也。

吳錄曰:虜翻稱澤曰:「闞生矯傑,蓋蜀之揚雄。」又曰:「闞子儒術德行,亦今之仲舒也。」初,魏文帝即位,權嘗從容問羣臣曰:「曹丕以盛年即位,恐孤不能及之,諸卿以為何如?」羣臣未對,澤曰:「不及十年,丕其沒矣,大王勿憂也。」權曰:「何以知之?」澤曰:「以字言之,不十為丕,此其數也。」文帝果七年而崩。臣松之計孫權年大文帝五歲,其為長幼也微耳。六年冬卒,權痛惜感悼,食不進者數日。

澤州里先輩丹楊唐固亦脩身積學,稱為儒者,著國語、公羊、穀梁傳注,講授常數十人。權為吳王,拜固議郎,自陸遜、張溫、駱統等皆拜之。黃武四年為尚書僕射,卒。吳錄曰:固字子正,卒時年七十餘矣。


\end{pinyinscope}