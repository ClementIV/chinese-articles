\article{郭淮}

\begin{pinyinscope}
郭淮字伯濟,太原陽曲人也。

案郭氏譜:淮祖全,大司農;父縕,鴈門太守。建安中舉孝廉,除平原府丞。文帝為五官將,召淮署為門下賊曹,轉為丞相兵曹議令史,從征漢中。太祖還,留征西將軍夏侯淵拒劉備,以淮為淵司馬。淵與備戰,淮時有疾不出。淵遇害,軍中擾擾,淮收散卒,推盪寇將軍張郃為軍主,諸營乃定。其明日,備欲渡漢水來攻。諸將議衆寡不敵,備便乘勝,欲依水為陣以拒之。淮曰:「此示弱而不足挫敵,非筭也。不如遠水為陣,引而致之,半濟而後擊,備可破也。」旣陣,備疑不渡,淮遂堅守,示無還心。以狀聞,太祖善之,假郃節,復以淮為司馬。文帝即王位,賜爵關內侯,轉為鎮西長史。又行征羌護軍,護左將軍張郃、冠軍將軍楊秋討山賊鄭甘、盧水叛胡,皆破平之。關中始定,民得安業。

黃初元年,奉使賀文帝踐阼,而道路得疾,故計遠近為稽留。及群臣歡會,帝正色責之曰:「昔禹會諸侯於塗山,防風後至,便行大戮。今溥天同慶而卿最留遲,何也?」淮對曰:「臣聞五帝先教導民以德,夏后政衰,始用刑辟。今臣遭唐虞之世,是以自知免於防風之誅也。」帝恱之,擢領雍州刺史,封射陽亭侯,五年為真。安定羗大帥辟蹏反,討破降之。每羌、胡來降,淮輙先使人推問其親理,男女多少,年歲長幼;及見,一二知其款曲,訊問周至,咸稱神明。

太和二年,蜀相諸葛亮出祁山,遣將軍馬謖至街亭,高詳屯列柳城。張郃擊謖,淮攻詳營,皆破之。又破隴西名羌唐蹏於枹罕,加建威將軍。五年,蜀出鹵城。是時,隴右無穀,議欲關中大運,淮以威恩撫循羌、胡,家使出穀,平其輸調,軍食用足,轉揚武將軍。青龍二年,諸葛亮出斜谷,並田于蘭坑。是時司馬宣王屯渭南;淮策亮必爭北原,宜先據之,議者多謂不然。淮曰:「若亮跨渭登原,連兵北山,隔絕隴道,搖盪民、夷,此非國之利也。」宣王善之,淮遂屯北原。塹壘未成,蜀兵大至,淮逆擊之。後數日,亮盛兵西行,諸將皆謂欲攻西圍,淮獨以為此見形於西,欲使官兵重應之,必攻陽遂耳。其夜果攻陽遂,有備不得上。

正始元年,蜀將羌維出隴西。淮遂進軍,追至彊中,維退,遂討羌迷當等,案撫柔氐三千餘落,拔徙以實關中。遷左將軍。涼州休屠胡梁元碧等,率種落二千餘家附雍州。淮奏請使居安定之高平,為民保鄣,其後因置西川都尉。轉拜前將軍,領州如故。

五年,夏侯玄伐蜀,淮督諸軍為前鋒。淮度勢不利,輙拔軍出,故不大敗。還假淮節。八年,隴西、南安、金城、西平諸羌餓何、燒戈、伐同、蛾遮塞等相結叛亂,攻圍城邑,南招蜀兵,涼州名胡治無戴復叛應之。討蜀護軍夏侯霸督諸軍屯為翅。淮軍始到狄道,議者僉謂宜先討定枹罕,內平惡羌,外折賊謀。淮策維必來攻霸,遂入渢中,轉南迎霸。維果攻為翅,會淮軍適至,維遁退。進討叛羌,斬餓何、燒戈,降服者萬餘落。九年,遮塞等屯河關、白土故城,據河拒軍。淮見形上流,密於下渡兵據白土城,擊,大破之。治無戴圍武威,家屬留在西海。淮進軍趨西海,欲掩取其累重,會無戴折還,與戰於龍夷之北,破走之。令居惡虜在石頭山之西,當大道止,斷絕王使。淮還過討,大破之。姜維出石營,從彊川,乃西迎治無戴,留陰平太守廖化於成重山築城,斂破羌保質。淮欲分兵取之。諸將以維衆西接彊胡,化以據險,分軍兩持,兵勢轉弱,進不制維,退不拔化,非計也,不如合而俱西,及胡、蜀未接,絕其內外,此伐交之兵也。淮曰:「今往取化,出賊不意,維必狼顧。比維自致,足以定化,且使維疲於奔命。兵不遠西,而胡交自離,此一舉而兩全之策也。」乃別遣夏侯霸等追維於沓中,淮自率諸軍就攻化等。維果馳還救化,皆如淮計。進封都鄉侯。

嘉平元年,遷征西將軍,都督雍、涼諸軍事。是歲,與雍州刺史陳泰恊策,降蜀牙門將句安等於翅上。二年,詔曰:「昔漢川之役,幾至傾覆。淮臨危濟難,功書王府。在關右三十餘年,外征寇虜,內綏民夷。比歲以來,摧破廖化,禽虜句安,功績顯著,朕甚嘉之。今以淮為車騎將軍、儀同三司,持節、都督如故。」進封陽曲侯,邑凡二千七百八十戶,分三百戶,封一子亭侯。世語曰:淮妻,王淩之妹。淩誅,妹當從坐,御史往收。督將及羌、胡渠帥數千人叩頭請淮表留妻,淮不從。妻上道,莫不流涕,人人扼腕,欲劫留之。淮五子叩頭流血請淮,淮不忍視,乃命左右追妻。於是追者數千騎,數日而還。淮以書白司馬宣王曰:「五子哀母,不惜其身;若無其母,是無五子;無五子,亦無淮也。今輒追還,若於法未通,當受罪於主者,覲展在近。」書至,宣王亦宥之。正元二年薨,追贈大將軍,謚曰貞侯。子統嗣。統官至荊州刺史,薨。子正嗣。咸熈中,開建五等,以淮著勳前朝,改封汾陽子。晉諸公贊曰:淮弟配,字仲南,有重名,位至城陽太守。裴秀、賈充皆配女壻。子展,字泰舒。有器度幹用,歷職著績,終於太僕。次弟豫,字泰寧,相國參軍,知名,早卒。女適王衍。配弟鎮,字季南,謁者僕射。鎮子弈,字泰業。山濤啟事稱弈高簡有雅量,歷位雍州刺史、尚書。

評曰:滿寵立志剛毅,勇而有謀。田豫居身清白,規畧明練。牽招秉義壯烈,威績顯著。郭淮方策精詳,垂問秦、雍。而豫位止小州,招終於郡守,未盡其用也。


\end{pinyinscope}