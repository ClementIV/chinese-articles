\article{bing-yuan-chuan}

\begin{pinyinscope}
邴原字根矩,北海朱虛人也。少與管寧俱以操尚稱,州府辟命皆不就。黃巾起,原將家屬入海,住鬱洲山中。時孔融為北海相,舉原有道。原以黃巾方盛,遂至遼東,與同郡劉政俱有勇略雄氣。遼東太守公孫度畏惡欲殺之,盡收捕其家,政得脫。度告諸縣:「敢有藏政者與同罪。」政窘急,往投原,

魏氏春秋曰:政投原曰:「窮鳥入懷。」原曰:「安知此懷之可入邪?」原匿之月餘,時東萊太史慈當歸,原因以政付之。旣而謂度曰:「將軍前日欲殺劉政,以其為己害。今政已去,君之害豈不除哉!」度曰:「然。」原曰:「君之畏政者,以其有智也。今政已免,智將用矣,尚奚拘政之家?不若赦之,無重怨。」度乃出之。原又資送政家,皆得歸故郡。原在遼東,一年中往歸原居者數百家,游學之士,教授之聲,不絕。

後得歸,太祖辟為司空掾。原女早亡,時太祖愛子倉舒亦沒,太祖欲求合葬,原辭曰:「合葬,非禮也。原之所以自容於明公,公之所以待原者,以能守訓典而不易也。若聽明公之命,則是凡庸也,明公焉以為哉?」太祖乃止,徙署丞相徵事。獻帝起居注曰:建安十五年,初置徵事二人,原與平原王烈俱以選補。崔琰為東曹掾,記讓曰:「徵事邴原、議郎張範,皆秉德純懿,志行忠方,清靜足以厲俗,貞固足以幹事,所謂龍翰鳳翼,國之重寶。舉而用之,不仁者遠。」代涼茂為五官將長史,閉門自守,非公事不出。太祖征吳,原從行,卒。原別傳曰:原十一而喪父,家貧,早孤。鄰有書舍,原過其傍而泣。師問曰:「童子何悲?」原曰:「孤者易傷,貧者易感。夫書者,必皆具有父兄者,一則羨其不孤,二則羨其得學,心中惻然而為涕零也。」師亦哀原之言而為之泣曰:「欲書可耳!」荅曰:「無錢資。」師曰:「童子苟有志,我徒相教,不求資也。」於是遂就書。一冬之間,誦孝經、論語。自在童齓之中,嶷然有異。及長,金玉其行。欲遠游學,詣安丘孫崧。崧辭曰:「君鄉里鄭君,君知之乎?」原荅曰:「然。」崧曰:「鄭君學覽古今,博聞彊識,鉤深致遠,誠學者之師模也。君乃舍之,躡屣千里,所謂以鄭為東家丘者也。君似不知而曰然者,何?」原曰:「先生之說,誠可謂苦藥良鍼矣;然猶未達僕之微趣也。人各有志,所規不同,故乃有登山而採玉者,有入海而採珠者,豈可謂登山者不知海之深,入海者不知山之高哉!君謂僕以鄭為東家丘,君以僕為西家愚夫邪?」崧辭謝焉。又曰:「兖、豫之士,吾多所識,未有若君者;當以書相分。」原重其意,難辭之,持書而別。原心以為求師啟學,志高者通,非若交游待分而成也。書何為哉?乃藏書於家而行。原舊能飲酒,自行之後,八九年間,酒不向口。單步負笈,苦身持力,至陳留則師韓子助,潁川則宗陳仲弓,汝南則交范孟博,涿郡則親盧子幹。臨別,師友以原不飲酒,會米肉送原。原曰:「本能飲酒,但以荒思廢業,故斷之耳。今當遠別,因見貺餞,可一飲燕。」於是共坐飲酒,終日不醉。歸以書還孫崧,解不致書之意。後為郡所召,署功曹主簿。時魯國孔融在郡,教選計當任公卿之才,乃以鄭玄為計掾,彭璆為計吏,原為計佐。融有所愛一人,常盛嗟嘆之。後恚望,欲殺之,朝吏皆請。時其人亦在坐,叩頭流血,而融意不解。原獨不為請。融謂原曰:「衆皆請而君何獨不?」原對曰:「明府於某,本不薄也,常言歲終當舉之,此所謂『吾一子』也。如是,朝吏受恩未有在某前者矣,而今乃欲殺之。明府愛之,則引而方之於子,憎之,則推之欲危其身。原愚,不知明府以何愛之?以何惡之?」融曰:「某生於微門,吾成就其兄弟,拔擢而用之;某今孤負恩施。夫善則進之,惡則誅之,固君道也。往者應仲遠為泰山太守,舉一孝廉,旬月之間而殺之。夫君人者,厚薄何常之有!」原對曰:「仲遠舉孝廉,殺之,其義焉在?夫孝廉,國之俊選也。舉之若是,則殺之非也;若殺之是,則舉之非也。詩云:『彼己之子,不遂其媾。』蓋譏之也。語云:『愛之欲其生,惡之欲其死。旣欲其生,又欲其死,是惑也。』仲遠之惑甚矣。明府奚取焉?」融乃大笑曰:「吾但戲耳!」原又曰:「君子於其言,出乎身,加乎民;言行,君子之樞機也。安有欲殺人而可以為戲者哉?」融無以荅。是時漢朝陵遲,政以賄成,原乃將家人入鬱洲山中。郡舉有道,融書喻原曰:「脩性保貞,清虛守高,危邦不入,乆潛樂土。王室多難,西遷鎬京。聖朝勞謙,疇咨儁乂。我徂求定,策命懇惻。國之將隕,釐不恤緯,家之將亡,緹縈跋涉,彼匹婦也,猶執此義。實望根矩,仁為己任,授手援溺,振民於難。乃或晏晏居息,莫我肯顧,謂之君子,固如此乎!根矩,根矩,可以來矣!」原遂到遼東。遼東多虎,原之邑落獨無虎患。原甞行而得遺錢,拾以繫樹枝,此錢旣不見取,而繫錢者愈多。問其故,荅者謂之神樹。原惡其由己而成淫祀,乃辨之,於是里中遂斂其錢以為社供。後原欲歸鄉里,止於三山。孔融書曰:「隨會在秦,賈季在翟,諮仰靡所,歎息增懷。頃知來至,近在三山。詩不云乎,『來歸自鎬,我行永久』。今遣五官掾,奉問榜人舟楫之勞,禍福動靜告慰。亂階未已,阻兵之雄,若棊奕爭梟。」原於是遂復反還。積十餘年,後乃遁還。南行已數日,而度甫覺。度知原之不可復追也,因曰:「邴君所謂雲中白鶴,非鶉鷃之網所能羅矣。又吾自遣之,勿復求也。」遂免危難。自反國土,原於是講述禮樂,吟詠詩書,門徒數百,服道數十。時鄭玄博學洽聞,注解典籍,故儒雅之士集焉。原亦自以高遠清白,頤志澹泊,口無擇言,身無擇行,故英偉之士向焉。是時海內清議,云青州有邴、鄭之學。魏太祖為司空,辟原署東閤祭酒。太祖北伐三郡單于,還住昌國,燕士大夫。酒酣,太祖曰:「孤反,鄴守諸君必將來迎,今日明旦,度皆至矣。其不來者,獨有邴祭酒耳!」言訖未久,而原先至。門下通謁,太祖大驚喜,擥履而起,遠出迎原曰:「賢者誠難測度!孤謂君將不能來,而遠自屈,誠副饑虛之心。」謁訖而出,軍中士大夫詣原者數百人。太祖怪而問之,時荀文若在坐,對曰:「獨可省問邴原耳!」太祖曰:「此君名重,乃亦傾士大夫心?」文若曰:「此一世異人,士之精藻,公宜盡禮以待之。」太祖曰:「固孤之宿心也。」自是之後,見敬益重。原雖在軍歷署,常以病疾,高枕里巷,終不當事,又希會見。河內張範,名公之子也,其志行有與原符,甚相親敬。令曰:「邴原名高德大,清規邈世,魁然而峙,不為孤用。聞張子頗欲學之,吾恐造之者富,隨之者貧也。」魏太子為五官中郎將,天下向慕,賔客如雲,而原獨守道持常,自非公事不妄舉動。太祖微使人從容問之,原曰:「吾聞國危不事冢宰,君去不奉世子,此典制也。」於是乃轉五官長史,令曰:「子弱不才,懼其難正,貪欲相屈,以匡勵之。雖云利賢,能不恧恧!」太子燕會,衆賔百數十人,太子建議曰:「君父各有篤疾,有藥一丸,可救一人,當救君邪,父邪?」衆人紛紜,或父或君。時原在坐,不與此論。太子諮之於原,原悖然對曰:「父也。」太子亦不復難之。

是後大鴻臚鉅鹿張泰、河南尹扶風龐辿以清賢稱,荀綽兾州記曰:鉅鹿張貔,字邵虎。祖父泰,字伯陽,有名於魏。父邈,字叔遼,遼東太守。著名自然好學論,在嵇康集。為人弘深有遠識,恢恢然,使求之者莫之能測也。宦歷二宮,元康初為城陽太守,未行而卒。永寧太僕東郡張閣以簡質聞。杜恕著家戒稱閣曰:「張子臺,視之似鄙樸人,然其心中不知天地間何者為美,何者為好,敦然似如與陰陽合德者。作人如此,自可不富貴,然而患禍當何從而來?世有高亮如子臺者,皆多力慕,體之不如也。」


\end{pinyinscope}