\article{yuan-shu-chuan}

\begin{pinyinscope}
袁術字公路,司空逢子,紹之從弟也。以俠氣聞。舉孝廉,除郎中,歷職內外,後為折衝校尉、虎賁中郎將。董卓之將廢帝,以術為後將軍;術亦畏卓之禍,出奔南陽。會長沙太守孫堅殺南陽太守張咨,術得據其郡。南陽戶口數百萬,而術奢淫肆欲,徵斂無度,百姓苦之。旣與紹有隙,又與劉表不平而北連公孫瓚;紹與瓚不和而南連劉表。其兄弟攜貳,舍近交遠如此。

吳書曰:時議者以靈帝失道,使天下叛亂,少帝幼弱,為賊臣所立,又不識母氏所出。幽州牧劉虞宿有德望,紹等欲立之以安當時,使人報術。術觀漢室衰陵,陰懷異志,故外託公義以拒紹。紹復與術書曰:「前與韓文節共建永世之道,欲海內見再興之主。今西名有幼君,無血脉之屬,公卿以下皆媚事卓,安可復信!但當使兵往屯關要,皆自蹙死於西。東立聖君,太平可兾,如何有疑!又室家見戮,不念子胥,可復北靣乎?違天不祥,願詳思之。」術荅曰:「聖主聦叡,有周成之質。賊卓因危亂之際,威服百寮,此乃漢家小厄之會。亂尚未厭,復欲興之。乃云今主『無血脉之屬』,豈不誣乎!先人以來,弈世相承,忠義為先。太傅公仁慈惻隱,雖知賊卓必為禍害,以信徇義,不忍去也。門戶滅絕,死亡流漫,幸蒙遠近來相赴助,不因此時上討國賊,下刷家恥,而圖於此,非所聞也。又曰『室家見戮,可復北靣』,此卓所為,豈國家哉?君命,天也,天不可讎,況非君命乎!慺慺赤心,志在滅卓,不識其他。」引軍入陳留。太祖與紹合擊,大破術軍。術以餘衆奔九江,殺楊州刺史陳溫,領其州。臣松之案英雄記:「陳溫字元悌,汝南人。先為楊州刺史,自病死。袁紹遣袁遺領州,敗散,奔沛國,為兵所殺。袁術更用陳瑀為楊州。瑀字公瑋,下邳人。瑀旣領州,而術敗於封丘,南向壽春,瑀拒術不納。術退保陰陵,更合軍攻瑀,瑀懼走歸下邳。」如此,則溫不為術所殺,與本傳不同。以張勳、橋蕤等為大將軍。李傕入長安,欲結術為援,以術為左將軍,封陽翟侯,假節,遣太傅馬日磾因循行拜授。術奪日磾節,拘留不遣。三輔決錄注曰:日磾字翁叔,馬融之族子。少傳融業,以才學進。與楊彪、盧植、蔡邕等典校中書,歷位九卿,遂登台輔。獻帝春秋曰:術從日磾借節觀之,因奪不還,備軍中千餘人,使促辟之。日磾謂術曰:「卿家先世諸公,辟士云何,而言促之,謂公府掾可劫得乎!」從術求去,而術留之不遣;旣以失節屈辱,憂恚而死。

時沛相下邳陳珪,故太尉球弟子也。術與珪俱公族子孫,少共交游,書與珪曰:「昔秦失其政,天下羣雄爭而取之,兼智勇者卒受其歸。今世事紛擾,復有瓦解之勢矣,誠英乂有為之時也。與足下舊交,豈肯左右之乎?若集大事,子實為吾心膂。」珪中子應時在下邳,術並脅質應,圖必致珪。珪荅書曰:「昔秦末世,肆暴恣情,虐流天下,毒被生民,下不堪命,故遂土崩。今雖季世,未有亡秦苛暴之亂也。曹將軍神武應期,興復典刑,將撥平凶慝,清定海內,信有徵矣。以為足下當戮力同心,匡翼漢室,而陰謀不軌,以身試禍,豈不痛哉!若迷而知反,尚可以免。吾備舊知,故陳至情,雖逆於耳,肉骨之惠也。欲吾營私阿附,有犯死不能也。」

興平二年冬,天子敗於曹陽。術會羣下謂曰:「今劉氏微弱,海內鼎沸。吾家四世公輔,百姓所歸,欲應天順民,於諸君意如何?」衆莫敢對。主簿閻象進曰:「昔周自后稷至于文王,積德累功,參分天下有其二,猶服事殷。明公雖弈世克昌,未若有周之盛,漢室雖微,未若殷紂之暴也。」術嘿然不恱。用河內張烱之符命,遂僭號,典略曰:術以袁姓出陳,陳,舜之後,以土承火,得應運之次。又見讖文云:「代漢者,當塗高也。」自以名字當之,乃建號稱仲氏。以九江太守為淮南尹。置公卿,祠南北郊。荒侈滋甚,後宮數百皆服綺縠,餘粱肉,九州春秋曰:司隷馮方女,國色也,避亂楊州,術登城見而恱之,遂納焉,甚愛幸。諸婦害其寵,語之曰:「將軍貴人有志節,當時時涕泣憂愁,必長見敬重。」馮氏以為然,後見術輙垂涕,術果以有心志,益哀之。諸婦人因共絞殺,懸之厠梁,術誠以為不得志而死,乃厚加殯斂。而士卒凍餒,江淮間空盡,人民相食。術前為呂布所破,後為太祖所敗,奔其部曲雷薄、陳蘭於灊山,復為所拒,憂懼不知所出。將歸帝號於紹,欲至青州從袁譚,發病道死。魏書曰:術歸帝號於紹曰:「漢之失天下乆矣,天子提挈,政在家門,豪雄角逐,分裂疆宇,此與周之末年七國分勢無異,卒彊者兼之耳。加袁氏受命當王,符瑞炳然。今君擁有四州,民戶百萬,以彊則無與比大,論德則無與比高。曹操欲扶衰拯弱,安能續絕命救已滅乎?」紹陰然之。吳書曰:術旣為雷薄等所拒,留住三日,士衆絕糧,乃還至江亭,去壽春八十里。問廚下,尚有麥屑三十斛。時盛暑,欲得蜜漿,又無蜜。坐櫺牀上,歎息良乆,乃大咤曰:「袁術至於此乎!」因頓伏牀下,嘔血斗餘,遂死。妻子依術故吏廬江太守劉勳,孫策破勳,復見收視。術女入孫權宮,子燿拜郎中,燿女又配於權子奮。


\end{pinyinscope}