\article{朱然傳}

\begin{pinyinscope}
子績附

朱然字義封,治姊子也,本姓施氏。初治未有子,然年十三,乃啟策乞以為嗣。策命丹楊郡以羊酒召然,然到吳,策優以禮賀。

然嘗與權同學書,結恩愛。至權統事,以然為餘姚長,時年十九。後遷山陰令,加折衝校尉,督五縣。權奇其能,分丹楊為臨川郡,然為太守,臣松之案:此郡尋罷,非今臨川郡。授兵二千人。會山賊盛起,然平討,旬月而定。曹公出濡須,然備大塢及三關屯,拜偏將軍。建安二十四年,從討關羽,別與潘璋到臨沮禽羽,遷昭武將軍,封西安鄉侯。

虎威將軍呂蒙病篤,權問曰:「卿如不起,誰可代者?」蒙對曰:「朱然膽守有餘,愚以為可任。」蒙卒,權假然節,鎮江陵。黃武元年,劉備舉兵攻宜都,然督五千人與陸遜并力拒備。然別攻破備前鋒,斷其後道,備遂破走。拜征北將軍,封永安侯。

魏遣曹真、夏侯尚、張郃等攻江陵,魏文帝自住宛,為其勢援,連屯圍城。權遣將軍孫盛督萬人備州上,立圍塢,為然外救。郃渡兵攻盛,盛不能拒,即時却退,郃據州上圍守,然中外斷絕。權遣潘璋、楊粲等解,而圍不解。時然城中兵多腫病,堪戰者裁五千人。真等起土山,鑿地道,立樓櫓,臨城弓矢雨注,將士皆失色,然晏如而無恐意,方厲吏士,伺間隙攻破兩屯。魏攻圍然凡六月日,未退。江陵令姚泰領兵備城北門,見外兵盛,城中人少,穀食欲盡,因與敵交通,謀為內應。垂發,事覺,然治戮泰。尚等不能克,乃徹攻退還。由是然名震於敵國,改封當陽侯。

六年,權自率衆攻石陽,及至旋師,潘璋斷後。夜出錯亂,敵追擊璋,璋不能禁。然即還住拒敵,使前船得引極遠,徐乃後發。黃龍元年,拜車騎將軍、右護軍,領兖州牧。頃之,以兖州在蜀分,解牧職。

嘉禾三年,權與蜀克期大舉,權自向新城,然與全琮各受斧鉞,為左右督。會吏士疾病,故未攻而退。

赤烏五年,征柤中,襄陽記曰:柤音如租稅之租。柤中在上黃界,去襄陽一百五十里。魏時夷王梅敷兄弟三人,部曲萬餘家屯此,分布在中廬宜城西山鄢、沔二谷中,土地平敞,宜桑麻,有水陸良田,沔南之膏腴沃壤,謂之柤中。魏將蒲忠、胡質各將數千人,忠要遮險隘,圖斷然後,質為忠繼援。時然所督兵將先四出,聞問不暇收合,便將帳下見兵八百人逆掩。忠戰不利,質等皆退。孫氏異同評曰:魏志及江表傳云然以景初元年、正始二年再出為寇,所破胡質、蒲忠在景初元年。魏志承魏書,依違不說質等為然所破,而直云然退耳。吳志說赤烏五年,於魏為正始三年,魏將蒲忠與朱然戰,忠不利,質等皆退。按魏少帝紀及孫權傳,是歲並無事,當是陳壽誤以吳嘉禾六年為赤烏五年耳。九年,復征柤中,魏將李興等聞然深入,率步騎六千斷然後道,然夜出逆之,軍以勝反。先是,歸義馬茂懷姦,覺誅,權深忿之。然臨行上疏曰:「馬茂小子,敢負恩養。臣今奉天威,事蒙克捷,欲令所獲,震耀遠近,方舟塞江,使足可觀,以解上下之忿。惟陛下識臣先言,責臣後效。」權時抑表不出。然旣獻捷,羣臣上賀,權乃舉酒作樂,而出然表曰:「此家前初有表,孤以為難必,今果如其言,可謂明於見事也。」遣使拜然為左大司馬、右軍師。

然長不盈七尺,氣候分明,內行脩絜,其所文采,惟施軍器,餘皆質素。終日欽欽,常在戰場,臨急膽定,尤過絕人,雖世無事,每朝夕嚴鼓,兵在營者,咸行裝就隊,以此玩敵,使不知所備,故出輒有功。諸葛瑾子融、步隲子恊,雖各襲任,權特復使然總為大督。又陸遜亦卒,功臣名將存者惟然,莫與比隆。寢疾二年,後漸增篤,權晝為減膳,夜為不寐,中使醫藥口食之物,相望於道。然每遣使表疾病消息,權輒召見,口自問訊,入賜酒食,出送布帛。自創業功臣疾病,權意之所鍾,呂蒙、凌統最重,然其次矣。年六十八,赤烏十二年卒,權素服舉哀,為之感慟。子績嗣。

績字公緒,以父任為郎,後拜建忠都尉。叔父才卒,績領其兵,隨太常潘濬討五溪,以膽力稱。遷偏將軍營下督,領盜賊事,持法不傾。魯王霸注意交績,嘗至其廨,就之坐,欲與結好,績下地住立,辭而不當。然卒,績襲業,拜平魏將軍,樂鄉督。明年,魏征南將軍王昶率衆攻江陵城,不克而退。績與奮威將軍諸葛融書曰:「昶遠來疲困,馬無所食,力屈而走,此天助也。今追之力少,可引兵相繼,吾欲破之於前,足下乘之於後,豈一人之功哉,宜同斷金之義。」融荅許績。績便引兵及昶於紀南,紀南去城三十里,績先戰勝而融不進,績後失利。權深嘉績,盛責怒融,融兄大將軍恪貴重,故融得不廢。初績與恪、融不平,及此事變,為隙益甚。建興元年,遷鎮東將軍。二年春,恪向新城,要績并力,而留置半州,使融兼其任。冬,恪、融被害,績復還樂鄉,假節。太平二年,拜驃騎將軍。孫綝秉政,大臣疑貳,績恐吳必擾亂,而中國乘釁,乃密書結蜀,使為并兼之慮。蜀遣右將軍閻宇將兵五千,增白帝守,以須績之後命。永安初,遷上大將軍、都護督,自巴丘上迄西陵。元興元年,就拜左大司馬。初,然為治行喪竟,乞復本姓,權不許,績以五鳳中表還為施氏,建衡二年卒。


\end{pinyinscope}