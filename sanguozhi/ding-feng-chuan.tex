\article{丁奉傳}

\begin{pinyinscope}
丁奉字承淵,廬江安豐人也。少以驍勇為小將,屬甘寧、陸遜、潘璋等。數隨征伐,戰鬬常冠軍。每斬將搴旗,身被創夷。稍遷偏將軍。孫亮即位,為冠軍將軍,封都亭侯。

魏遣諸葛誕、胡遵等攻東興,諸葛恪率軍拒之。諸將皆曰:「敵聞太傅自來,上岸必遁走。」奉獨曰:「不然。彼動其境內,悉許、洛兵大舉而來,必有成規,豈虛還哉?無恃敵之不至,恃吾有以勝之。」及恪上岸,奉與將軍唐咨、呂據、留贊等,俱從山西上。奉曰:「今諸軍行遲,若敵據便地,則難與爭鋒矣。」乃辟諸軍使下道,帥麾下三千人徑進。時北風,奉舉帆二日至,遂據徐塘。天寒雪,敵諸將置酒高會,奉見其前部兵少,相謂曰:「取封侯爵賞,正在今日!」乃使兵解鎧著冑,持短兵。敵人從而笑焉,不為設備。奉縱兵斫之,大破敵前屯。會據等至,魏軍遂潰。遷滅寇將軍,進封都鄉侯。

魏將文欽來降,以奉為虎威將軍,從孫峻至壽春迎之,與敵追軍戰於高亭。奉跨馬持矛,突入其陣中,斬首數百,獲其軍器。進封安豐侯。

太平二年,魏大將軍諸葛誕據壽春來降,魏人圍之。遣朱異、唐咨等往救,復使奉與黎斐解圍。奉為先登,屯於黎漿,力戰有功,拜左將軍。

孫休即位,與張布謀,欲誅孫綝,布曰:「丁奉雖不能吏書,而計略過人,能斷大事。」休召奉告曰:「綝秉國威,將行不軌,欲與將軍誅之。」奉曰:「丞相兄弟友黨甚盛,恐人心不同,不可卒制,可因臘會,有陛下兵以誅之也。」休納其計,因會請綝,奉與張布目左右斬之。遷大將軍,加左右都護。永安二年,假節領徐州牧。六年,魏伐蜀,奉率諸軍向壽春,為救蜀之勢。蜀亡,軍還。

休薨,奉與丞相濮陽興等從萬彧之言,共迎立孫皓,遷右大司馬左軍師。寶鼎三年,皓命奉與諸葛靚攻合肥。奉與晉大將石苞書,搆而間之,苞以徵還。建衡元年,奉復帥衆治徐塘,因攻晉穀陽。穀陽民知之,引去,奉無所獲。皓怒,斬奉導軍。三年,卒。奉貴而有功,漸以驕矜,或有毀之者,皓追以前出軍事,徙奉家於臨川。奉弟封,官至後將軍,先奉死。

評曰:凡此諸將,皆江表之虎臣,孫氏之所厚待也。以潘璋之不脩,權能忘過記功,其保據東南,宜哉!陳表將家支庶,而與冑子名人比翼齊衡,拔萃出類,不亦美乎!


\end{pinyinscope}