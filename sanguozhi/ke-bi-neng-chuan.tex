\article{軻比能傳}

\begin{pinyinscope}
軻比能本小種鮮卑,以勇健,斷法平端,不貪財物,衆推以為大人。部落近塞,自袁紹據河北,中國人多亡叛歸之,教作兵器鎧楯,頗學文字。故其勒御部衆,擬則中國,出入弋獵,建立旌麾,以鼓節為進退。建安中,因閻柔上貢獻。太祖西征關中,田銀反河間,比能將三千餘騎隨柔擊破銀。後代郡烏丸反,比能復助為寇害,太祖以鄢陵侯彰為驍騎將軍,北征,大破之。比能走出塞,後復通貢獻。延康初,比能遣使獻馬,文帝亦立比能為附義王。黃初二年,比能出諸魏人在鮮卑者五百餘家,還居代郡。明年,比能帥部落大人小子代郡烏丸脩武盧等三千餘騎,驅牛馬七萬餘口交市,遣魏人千餘家居上谷。後與東部鮮卑大人素利及步度根三部爭鬪,更相攻擊。田豫和合,使不得相侵。五年,比能復擊素利,豫帥輕騎徑進掎其後。比能使別小帥瑣奴拒豫,豫進討,破走之,由是懷貳。乃與輔國將軍鮮于輔書曰:「夷狄不識文字,故校尉閻柔保我於天子。我與素利為讎,往年攻擊之,而田校尉助素利。我臨陣使瑣奴往,聞使君來,即便引軍退。步度根數數鈔盜,又殺我弟,而誣我以鈔盜。我夷狄雖不知禮義,兄弟子孫受天子印綬,牛馬尚知美水草,況我有人心邪!將軍當保明我於天子。」輔得書以聞帝,帝復使豫招納安慰。比能衆遂彊盛,控弦十餘萬騎。每鈔畧得財物,均平分付,一決目前,終無所私,故得衆死力,餘部大人皆敬憚之,然猶未能及檀石槐也。

太和二年,豫遣譯夏舍詣比能女壻鬱築鞬部,舍為鞬所殺。其秋,豫將西部鮮卑蒲頭、泄歸泥出塞討鬱築鞬,大破之。還至馬城,比能自將三萬騎圍豫七日。上谷太守閻志,柔之弟也,素為鮮卑所信。志往解喻,即解圍去。後幽州刺史王雄并領校尉,撫以恩信。比能數款塞,詣州奉貢獻。至青龍元年,比能誘納步度根,使叛并州,與結和親,自勒萬騎迎其累重於陘北。并州刺史畢軌遣將軍蘇尚、董弼等擊之,比能遣子將騎與尚等會戰於樓煩,臨陣害尚、弼。至三年中,雄遣勇士韓龍刺殺比能,更立其弟。

素利、彌加、厥機皆為大人,在遼西、右北平、漁陽塞外,道遠初不為邊患,然其種衆多於比能。建安中,因閻柔上貢獻,通市,太祖皆表寵以為王。厥機死,又立其子沙末汗為親漢王。延康初,又各遣使獻馬。文帝立素利、彌加為歸義王。素利與比能更相攻擊。太和二年,素利死。子小,以弟成律歸為王,代攝其衆。


\end{pinyinscope}