\article{呂凱傳}

\begin{pinyinscope}
呂凱字季平、永昌不韋人也。

孫盛蜀世譜曰:初,秦徙呂不韋子弟宗族於蜀漢。漢武帝時,開西南夷,置郡縣,徙呂氏以充之,因曰不韋縣。仕郡五官掾功曹。時雍闓等聞先主薨於永安,驕黠滋甚。都護李嚴與闓書六紙,解喻利害,闓但荅一紙曰:「蓋聞天無二日,土無二王,今天下鼎立,正朔有三,是以遠人惶惑,不知所歸也。」其桀慢如此。闓又降於吳,吳遙署闓為永昌太守。永昌旣在益州郡之西,道路壅塞,與蜀隔絕,而郡太守改易,凱與府丞蜀郡王伉帥厲吏民,閉境拒闓。闓數移檄永昌,稱說云云。凱荅檄曰:「天降喪亂,姧雄乘釁,天下切齒,萬國悲悼,臣妾大小,莫不思竭筋力,肝腦塗地,以除國難。伏惟將軍世受漢恩,以為當躬聚黨衆,率先啟行,上以報國家,下不負先人,書功竹帛,遺名千載。何期臣僕吳越,背本就末乎?昔舜勤民事,隕于蒼梧,書籍嘉之,流聲無窮。崩于江浦,何足可悲!文、武受命,成王乃平。先帝龍興,海內望風,宰臣聦睿,自天降康。而將軍不覩盛衰之紀,成敗之符,譬如野火在原,蹈履河冰,火滅冰泮,將何所依附?曩者將軍先君雍侯,造怨而封,竇融知興,歸志世祖,皆流名後葉,世歌其美。今諸葛丞相英才挺出,深覩未萌,受遺託孤,翊贊季興,與衆無忌,錄功忘瑕。將軍若能翻然改圖,易跡更步,古人不難追,鄙土何足宰哉!蓋聞楚國不恭,齊桓是責,夫差僭號,晉人不長,況臣於非主,誰肯歸之邪?竊惟古義,臣無越境之交,是以前後有來無往。重承告示,發憤忘食,故略陳所懷,惟將軍察焉。」凱威恩內著,為郡中所信,故能全其節。

及丞相亮南征討闓,旣發在道,而闓已為高定部曲所殺。亮至南,上表曰:「永昌郡吏呂凱、府丞王伉等,執忠絕域,十有餘年,雍闓、高定偪其東北,而凱等守義不與交通。臣不意永昌風俗敦直乃爾!」以凱為雲南太守,封陽遷亭侯。會為叛夷所害,子祥嗣。而王伉亦封亭侯,為永昌太守。蜀世譜曰:呂祥後為晉南夷校尉,祥子及孫世為永昌太守。李雄破寧州,諸呂不肯附,舉郡固守。王伉等亦守正節。


\end{pinyinscope}