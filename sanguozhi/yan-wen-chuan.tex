\article{yan-wen-chuan}

\begin{pinyinscope}
閻溫字伯儉,天水西城人也。以涼州別駕守上邽令。馬超走奔上邽,郡人任養等舉衆迎之。溫止之,不能禁,乃馳還州。超復圍州所治兾城甚急,州乃遣溫密出,告急於夏侯淵。賊圍數重,溫夜從水中潛出。明日,賊見其迹,遣人追遮之,於顯親界得溫,執還詣超。超解其縛,謂曰:「今成敗可見,足下為孤城請救而執於人手,義何所施?若從吾言,反謂城中,東方無救,此轉禍為福之計也。不然,今為戮矣。」溫偽許之,超乃載溫詣城下。溫向城大呼曰:「大軍不過三日至,勉之!」城中皆泣,稱萬歲。超怒數之曰:「足下不為命計邪?」溫不應。時超攻城乆不下,故徐誘溫,兾其改意。復謂溫曰:「城中故人,有欲與吾同者不?」溫又不應。遂切責之,溫曰:「夫事君有死無貳,而卿乃欲令長者出不義之言,吾豈苟生者乎?」超遂殺之。

先是,河右擾亂,隔絕不通,燉煌太守馬艾卒官,府又無丞。功曹張恭素有學行,郡人推行長史事,恩信甚著,乃遣子就東詣太祖,請太守。時酒泉黃華、張掖張進各據其郡,欲與恭并勢。就至酒泉,為華所拘執,劫以白刃。就終不回,私與恭疏曰:「大人率厲燉煌,忠義顯然,豈以就在困厄之中而替之哉?昔樂羊食子,李通覆家,經國之臣,寧懷妻孥邪?今大軍垂至,但當促兵以掎之耳;願不以下流之愛,使就有恨於黃壤也。」恭即遣從弟華攻酒泉沙頭、乾齊二縣。恭又連兵尋繼華後,以為首尾之援。別遣鐵騎二百,迎吏官屬,東緣酒泉北塞,徑出張掖北河,逢迎太守尹奉。於是張進須黃華之助;華欲救進,西顧恭兵,恐急擊其後,遂詣金城太守蘇則降。就竟平安。奉得之官。黃初二年,下詔襃揚,賜恭爵關內侯,拜西域戊己校尉。數歲徵還,將授以侍臣之位,而以子就代焉。恭至燉煌,固辭疾篤。太和中卒,贈執金吾。就後為金城太守,父子著稱於西州。

世語曰:就子斆,字祖文,弘毅有幹正,晉武帝世為廣漢太守。王濬在益州,受中制募兵討吳,無虎符,斆收濬從事列上,由此召斆還。帝責斆:「何不密啟而便收從事?」斆曰:「蜀漢絕遠,劉備嘗用之。輒收,臣猶以為輕。」帝善之。官至匈奴中郎將。斆子固,字元安,有斆風,為黃門郎,早卒。斆,一本作勃。魏略勇俠傳載孫賔碩、祝公道、楊阿若、鮑出等四人,賔碩雖漢人,而魚豢編之魏書,蓋以其人接魏,事義相類故也。論其行節,皆龐、閻之流。其祝公道一人,已見賈逵傳。今列賔碩等三人于後。孫賔碩者,北海人也,家素貧。當漢桓帝時,常侍左悺、唐衡等權侔人主。延熹中,衡弟為京兆虎牙都尉,秩比二千石,而統屬郡。衡弟初之官,不脩敬於京兆尹,入門不持版,郡功曹趙息呵廊下曰:「虎牙儀如屬城,何得放臂入府門?」促收其主簿。衡弟顧促取版,旣入見尹,尹欲脩主人,勑外為市買。息又啟云:「左悺子弟,來為虎牙,非德選,不足為特酤買,宜隨中舍菜食而已。」及其到官,遣吏奉牋謝尹,息又勑門,言「無常見此無陰兒輩子弟邪,用其箋記為通乎?」晚乃通之,又不得即令報。衡弟皆知之,甚恚,欲滅諸趙。因書與衡,求為京兆尹,旬月之間,得為之。息自知前過,乃逃走。時息從父仲臺,見為涼州刺史,於是衡為詔徵仲臺,遣歸。遂詔中都官及郡部督郵,捕諸趙尺兒以上,及仲臺皆殺之,有藏者與同罪。時息從父岐為皮氏長,聞有家禍,因從官舍逃,走之河間,變姓字,又轉詣北海,著絮巾布袴,常於市中販胡餅。賔碩時年二十餘,乘犢車,將騎入市。觀見岐,疑其非常人也。因問之曰:「自有餅邪,販之邪?」岐曰:「販之。」賔碩曰:「買幾錢?賣幾錢?」岐曰:「買三十,賣亦三十。」賔碩曰:「視處士之望,非似賣餅者,殆有故!」乃開車後戶,顧所將兩騎,令下馬扶上之。時岐以為是唐氏耳目也,甚怖,面失色。賔碩閉車後戶,下前襜,謂之曰:「視處士狀貌,旣非販餅者,加今面色變動,即不有重怨,則當亡命。我北海孫賔碩也,闔門百口,又有百歲老母在堂,勢能相度者也,終不相負,必語我以實。」岐乃具告之。賔碩遂載岐驅歸。住車門外,先入,白母言:「今日出得死友在外,當來入拜。」乃出,延岐入,椎牛鍾酒,快相娛樂。一二日,因載著別田舍,藏置複壁中。後數歲,唐衡及弟皆死。岐乃得出,還本郡。三府並辟,展轉仕進,至郡守、刺史、太僕,而賔碩亦從此顯名於東國,仕至豫州刺史。初平末,賔碩以東方飢荒,南客荊州。至興平中,趙岐以太僕持節使安慰天下,南詣荊州,乃復與賔碩相遇,相對流涕。岐為劉表陳其本末,由是益禮賔碩。頃之,賔碩病亡,岐在南,為行喪也。楊阿若後名豐,字伯陽,酒泉人。少游俠,常以報仇解怨為事,故時人為之號曰:「東市相斫楊阿若,西市相斫楊阿若。」至建安年中,太守徐揖誅郡中彊族黃氏。時黃昂得脫在外,乃以其家粟金數斛,募衆得千餘人以攻揖。揖城守。豐時在外,以昂為不義,乃告揖,捐妻子走詣張掖求救。會張掖又反,殺太守,而昂亦陷城殺揖,二郡合勢。昂恚豐不與己同,乃重募取豐,欲令張掖以麻繫其頭,生致之。豐遂逃走。武威太守張猛假豐為都尉,使齎檄告酒泉,聽豐為揖報仇。豐遂單騎入南羌中,合衆得千餘騎,從樂浪南山中出,指趨郡城。未到三十里,皆令騎下馬,曳柴揚塵。酒泉郡人望見塵起,以為東大兵到,遂破散。昂獨走出,羌捕得昂,豐謂昂曰:「卿前欲生繫我頸,今反為我所繫,云何?」昂慙謝,豐遂殺之。時黃華在東,又還領郡。豐畏華,復走依燉煌。至黃初中,河西興復,黃華降,豐乃還郡。郡舉孝廉,州表其義勇,詔即拜駙馬都尉。後二十餘年,病亡。鮑出字文才,京兆新豐人也。少游俠。興平中,三輔亂,出與老母兄弟五人家居本縣,以饑餓,留其母守舍,相將行採蓬實,合得數升,使其二兄初、雅及其弟成持歸,為母作食,獨與小弟在後採蓬。初等到家,而噉人賊數十人已略其母,以繩貫其手掌,驅去。初等怖恐,不敢追逐。須臾,出從後到,知母為賊所略,欲追賊。兄弟皆云:「賊衆,當如何?」出怒曰:「有母而使賊貫其手,將去煑噉之,用活何為?」乃攘臂結衽獨追之,行數里及賊。賊望見出,乃共布列待之。出到,回從一頭斫賊四五人。賊走,復合聚圍出,出跳越圍斫之,又殺十餘人。時賊分布,驅出母前去。賊連擊出,不勝,乃走與前輩合。出復追擊之,還見其母與比舍嫗同貫相連,出遂復奮擊賊。賊問出曰:「卿欲何得?」出責數賊,指其母以示之,賊乃解還出母。比舍嫗獨不解,遙望出求哀。出復斫賊,賊謂出曰:「已還卿母,何為不止?」出又指求哀嫗:「此我嫂也。」賊復解還之。出得母還,遂相扶將,客南陽。建安五年,關中始開,出來北歸,而其母不能步行,兄弟欲共輿之。出以輿車歷山險危,不如負之安穩,乃以籠盛其母,獨自負之,到鄉里。鄉里士大夫嘉其孝烈,欲薦州郡,郡辟召出,出曰:「田民不堪冠帶。」至青龍中,母年百餘歲乃終,出時年七十餘,行喪如禮,於今年八九十,才若五六十者。魚豢曰:昔孔子歎顏回,以為三月不違仁者,蓋觀其心耳,孰如孫、祝菜色於市里,顛倒於牢獄,據有實事哉?且夫濮陽周氏不敢匿迹,魯之朱家不問情實,是何也?懼禍之及,且心不安也。而太史公猶貴其竟脫季布,豈若二賢,厥義多乎?今故遠收孫、祝,而近錄楊、鮑,旣不欲其泯滅,且敦薄俗。至於鮑出,不染禮教,心痛意發,起於自然,迹雖在編戶,與篤烈君子何以異乎?若夫楊阿若,少稱任俠,長遂蹈義,自西徂東,摧討逆節,可謂勇而有仁者也。

評曰:李典貴尚儒雅,義忘私隙,美矣。李通、臧霸、文聘、呂虔鎮衞州郡,並著威惠。許褚、典韋折衝左右,抑亦漢之樊噲也。龐悳授命叱敵,有周苛之節。龐淯不憚伏劒,而誠感鄰國。閻溫向城大呼,齊解、路之烈焉。


\end{pinyinscope}