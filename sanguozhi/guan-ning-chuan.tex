\article{guan-ning-chuan}

\begin{pinyinscope}
管寧字幼安,北海朱虛人也。

傅子曰:齊相管仲之後也。昔田氏有齊而管氏去之,或適魯,或適楚。漢興有管少卿為燕令,始家朱虛,世有名節,九世而生寧。年十六喪父,中表愍其孤貧,咸共贈賵,悉辭不受,稱財以送終。長八尺,美須眉。與平原華歆、同縣邴原相友,俱游學於異國,並敬善陳仲弓。天下大亂,聞公孫度令行於海外,遂與原及平原王烈等至于遼東。度虛館以候之。旣往見度,乃廬於山谷。時避難者多居郡南,而寧居北,示無遷志,後漸來從之。太祖為司空,辟寧,度子康絕命不宣。傅子曰:寧往見度,語唯經典,不及世事。還乃因山為廬,鑿坏為室。越海避難者,皆來就之而居,旬月而成邑。遂講詩、書,陳俎豆,飾威儀,明禮讓,非學者無見也。由是度安其賢,民化其德。邴原性剛直,清議以格物,度已下心不安之。寧謂原曰:「潛龍以不見成德,言非其時,皆招禍之道也。」密遣令西還。度庶子康代居郡,外以將軍太守為號,而內實有王心,卑己崇禮,欲官寧以自鎮輔,而終莫敢發言,其敬憚如此。皇甫謐高士傳曰:寧所居屯落,會井汲者,或男女雜錯,或爭井鬬鬩。寧患之,乃多買器,分置井傍,汲以待之,又不使知。來者得而怪之,問知寧所為,乃各相責,不復鬬訟。鄰有牛暴寧田者,寧為牽牛著涼處,自為飲食,過於牛主。牛主得牛,大慙,若犯嚴刑。是以左右無鬬訟之聲,禮讓移於海表。

王烈者,字彥方,於時名聞在原、寧之右。辭公孫度長史,商賈自穢。太祖命為丞相掾,徵事,未至,卒於海表。先賢行狀曰:烈通識達道,秉義不回。以潁川陳太丘為師,二子為友。時潁川荀慈明、賈偉節、李元禮、韓元長皆就陳君學,見烈器業過人,歎服所履,亦與相親。由是英名著於海內。道成德立,還歸舊廬,遂遭父喪,泣淚三年。遇歲饑饉,路有餓殍,烈乃分釜庚之儲,以救邑里之命。是以宗族稱孝,鄉黨歸仁。以典籍娛心,育人為務,遂建學校,敦崇庠序。其誘人也,皆不因其性氣,誨之以道,使之從善遠惡。益者不自覺,而大化隆行,皆成寶器。門人出入,容止可觀,時在市井,行步有異,人皆別之。州閭成風,咸競為善。時國中有盜牛者,牛主得之。盜者曰:「我邂逅迷惑,從今已後將為改過。子旣已赦宥,幸無使王烈聞之。」人有以告烈者,烈以布一端遺之。或問:「此人旣為盜,畏君聞之,反與之布,何也?」烈曰:「昔秦穆公,人盜其駿馬食之,乃賜之酒。盜者不愛其死,以救穆公之難。今此盜人能悔其過,懼吾聞之,是知恥惡。知恥惡,則善心將生,故與布勸為善也。」間年之中,行路老父擔重,人代擔行數十里,欲至家,置而去,問姓名,不以告。頃之,老父復行,失劒於路。有人行而遇之,欲置而去,懼後人得之,劒主於是永失,欲取而購募,或恐差錯,遂守之。至暮,劒主還見之,前者代擔人也。老父擥其袂,問曰:「子前者代吾擔,不得姓名,今子復守吾劒于路,未有若子之仁,請子告吾姓名,吾將以告王烈。」乃語之而去。老父以告烈,烈曰:「世有仁人,吾未之見。」遂使人推之,乃昔時盜牛人也。烈歎曰:「韶樂九成,虞賔以和:人能有感,乃至於斯也!」遂使國人表其閭而異之。時人或訟曲直,將質於烈,或至塗而反,或望廬而還,皆相推以直,不敢使烈聞之。時國主皆親驂乘適烈私館,疇諮政令。察孝廉,三府並辟,皆不就。會董卓作亂,避地遼東,躬秉農器,編於四民,布衣蔬食,不改其樂。東域之人,奉之若君。時衰世弊,識真者少,朋黨之人,互相讒謗。自避世在東國者,多為人所害,烈居之歷年,未甞有患。使遼東彊不淩弱,衆不暴寡,商賈之人,市不二價。太祖累徵召,遼東為解而不遣。以建安二十三年寢疾,年七十八而終。

中國少安,客人皆還,唯寧晏然若將終焉。黃初四年,詔公卿舉獨行君子,司徒華歆薦寧。文帝即位,徵寧,遂將家屬浮海還郡,公孫恭送之南郊,加贈服物。自寧之東也,度、康、恭前後所資遺,皆受而藏諸。旣已西渡,盡封還之。傅子曰:是時康又已死,嫡子不立而立弟恭,恭懦弱,而康孽子淵有儁才。寧曰:「廢嫡立庶,下有異心,亂之所由起也。」乃將家屬乘海即受徵。寧在遼東,積三十七年乃歸,其後淵果襲奪恭位,叛國家而南連吳,僭號稱王,明帝使相國宣文侯征滅之。遼東之死者以萬計,如寧所籌。寧之歸也,海中遇暴風,船皆沒,唯寧乘船自若。時夜風晦冥,船人盡惑,莫知所泊。望見有火光,輒趣之,得島。島無居人,又無火燼,行人咸異焉,以為神光之祐也。皇甫謐曰:「積善之應也。」詔以寧為太中大夫,固辭不受。傅子曰:寧上書天子,且以疾辭,曰:「臣聞傅說發夢,以感殷宗,呂尚啟兆,以動周文,以通神之才悟於聖主,用能匡佐帝業,克成大勳。臣之器朽,實非其人。雖貪清時,釋體蟬蛻。內省頑病,日薄西山。唯陛下聽野人山藪之願,使一老者得盡微命。」書奏,帝親覽焉。明帝即位,太尉華歆遜位讓寧,傅子曰:司空陳羣又薦寧曰:「臣聞王者顯善以消惡,故湯舉伊尹,不仁者遠。伏見徵士北海管寧,行為世表,學任人師,清儉足以激濁,貞正足以矯時。前雖徵命,禮未優備。昔司空荀爽,家拜光祿,先儒鄭玄,即授司農,若加備禮,庶必可致。至延西序,坐而論道,必能昭明古今,有益大化。」遂下詔曰:「太中大夫管寧,耽懷道德,服膺六藝,清虛足以侔古,廉白可以當世。曩遭王道衰缺,浮海遁居,大魏受命,則襁負而至,斯蓋應龍潛升之道,聖賢用舍之義。而黃初以來,徵命屢下,每輙辭疾,拒違不至。豈朝廷之政,與生殊趣,將安樂山林,往而不能反乎!夫以姬公之聖,而耇德不降,則鳴鳥弗聞。尚書君奭曰:「耇造德不降,我則鳴鳥不聞,矧曰其有能格。」鄭玄曰:「耇,老也。造,成也。詩云:『小子有造。』老成德之人,不降志與我並在位,則鳴鳥之聲不得聞,況乃曰有能德格於天者乎!言必無也。鳴鳥謂鳳也。」以秦穆之賢,猶思詢乎黃髮。況朕寡德,曷能不願聞道于子大夫哉!今以寧為光祿勳。禮有大倫,君臣之道,不可廢也。望必速至,稱朕意焉。」又詔青州刺史曰:「寧抱道懷貞,潛翳海隅,比下徵書,違命不至,盤桓利居,高尚其事。雖有素履幽人之貞,而失考父茲恭之義,使朕虛心引領歷年,其何謂邪?徒欲懷安,必肆其志,不惟古人亦有翻然改節以隆斯民乎!日逝月除,時方已過,澡身浴德,將以曷為?仲尼有言:『吾非斯人之徒與而誰與哉!』其命別駕從事郡丞掾,奉詔以禮發遣寧詣行在所,給安車、吏從、茵蓐、道上廚食,上道先奏。」寧稱草莽臣上疏曰:「臣海濵孤微,罷農無伍,祿運幸厚。橫蒙陛下纂承洪緒,德侔三皇。化溢有唐。乆荷渥澤,積祀一紀,不能仰荅陛下恩養之福。沈委篤痾,寢疾彌留,逋違臣隷顛倒之節,夙宵戰怖,無地自厝。臣元年十一月被公車司馬令所下州郡,八月甲申詔書徵臣,更賜安車、衣被、茵蓐,以禮發遣,光寵並臻,優命屢至,怔營竦息,悼心失圖。思自陳聞,申展愚情,而明詔抑割,不令稍脩章表,是以鬱滯,訖于今日。誠謂乾覆,恩有紀極,不意靈潤,彌以隆赫。奉今年二月被州郡所下三年十二月辛酉詔書,重賜安車、衣服,別駕從事與郡功曹以禮發遣,又特被璽書,以臣為光祿勳,躬秉勞謙,引喻周、秦,損上益下。受詔之日,精魄飛散,靡所投死。臣重自省揆,德非園、綺而蒙安車之榮,功無竇融而蒙璽封之寵,楶梲駑下,荷棟梁之任,垂沒之命,獲九棘之位,懼有朱博鼓妖之眚。又年疾日侵,有加無損,不任扶輿進路以塞元責。望慕閶闔,徘徊闕庭,謹拜章陳情,乞蒙哀省,抑恩聽放,無令骸骨填於衢路。」

自黃初至于青龍,徵命相仍,常以八月賜牛酒。詔書問青州刺史程喜:「寧為守節高乎,審老疾尪頓邪?」喜上言:「寧有族人管貢為州吏,與寧鄰比,臣常使經營消息。貢說:『寧常著皁帽、布襦袴、布裠,隨時單複,出入閨庭,能自任杖,不須扶持。四時祠祭,輙自力彊,改加衣服,著絮巾,故在遼東所有白布單衣,親薦饌饋,跪拜成禮。寧少而喪母,不識形象,常特加觴,泫然流涕。又居宅離水七八十步,夏時詣水中澡灑手足,闚於園圃。』臣揆寧前後辭讓之意,獨自以生長潛逸,耆艾智衰,是以栖遲,每執謙退。此寧志行所欲必全,不為守高。」高士傳曰:管寧自越海及歸,常坐一木榻,積五十餘年,未甞箕股,其榻上當膝處皆穿。

正始二年,太僕陶丘一、永寧衞尉孟觀、侍中孫邕、中書侍郎王基薦寧曰:

臣聞龍鳳隱耀,應德而臻,明哲潛遁,俟時而動。是以鸞鷟鳴岐,周道隆興,四皓為佐,漢帝用康。伏見太中大夫管寧,應二儀之中和,總九德之純懿,含章素質,冰絜淵清,玄虛澹泊,與道逍遙;娛心黃老,游志六藝,升堂入室,究其閫奧,韜古今於胷懷,包道德之機要。中平之際,黃巾陸梁,華夏傾蕩,王綱弛頓。遂避時難,乘桴越海,羈旅遼東三十餘年。在乾之姤,匿景藏光,嘉遁養浩,韜韞儒墨,潛化傍流,暢於殊俗。

黃初四年,高祖文皇帝疇咨羣公,思求儁乂,故司徒華歆舉寧應選,公車特徵,振翼遐裔,翻然來翔。行遇屯厄,遭罹疾病,即拜太中大夫。烈祖明皇帝嘉美其德,登為光祿勳。寧疾彌留,未能進道。今寧舊疾已瘳,行年八十,志無衰倦。環堵篳門,偃息窮巷,飯鬻餬口,并日而食,吟詠詩書,不改其樂。困而能通,遭難必濟,經危蹈險,不易其節,金聲玉色,乆而彌彰。揆其終始,殆天所祚,當贊大魏,輔亮雍熙。袞職有闕,羣下屬望。昔高宗刻象,營求賢哲,周文啟龜,以卜良佐。況寧前朝所表,名德已著,而乆栖遲,未時引致,非所以奉遵明訓,繼成前志也。陛下踐阼,纂承洪緒。聖敬日躋,超越周成。每發德音,動諮師傅。若繼二祖招賢故典,賔禮儁邁,以廣緝熙,濟濟之化,侔於前代。

寧清高恬泊,擬跡前軌,德行卓絕,海內無偶。歷觀前世玉帛所命,申公、枚乘、周黨、樊英之儔,測其淵源,覽其清濁,未有厲俗獨行若寧者也。誠宜束帛加璧,備禮徵聘,仍授几杖,延登東序,敷陳墳素,坐而論道,上正璇璣,恊和皇極,下阜羣生,彝倫攸叙,必有可觀,光益大化。若寧固執匪石,守志箕山,追迹洪崖,參蹤巢、許。斯亦聖朝同符唐、虞,優賢揚歷,垂聲千載。今文尚書曰「優賢揚歷」,謂揚其所歷試。左思魏都賦曰:「優賢著於揚歷」也。雖出處殊塗,俯仰異體,至於興治美俗,其揆一也。

於是特具安車蒲輪,束帛加璧聘焉。會寧卒,時年八十四。拜子邈郎中,後為博士。初,寧妻先卒,知故勸更娶,寧曰:「每省曾子、王駿之言,意常嘉之,豈自遭之而違本心哉?」傅子曰:寧以衰亂之時,世多妄變氏族者,違聖人之制,非禮命姓之意,故著氏姓論以原本世系,文多不載。每所居姻親、知舊、鄰里有困窮者,家儲雖不盈擔石,必分以贍救之。與人子言,教以孝;與人弟言,訓以悌;言及人臣,誨以忠。貌甚恭,言甚順,觀其行,邈然若不可及,即之熈熈然,甚柔而溫,因其事而導之於善,是以漸之者無不化焉。寧之亡,天下知與不知,聞之無不嗟歎。醇德之所感若此,不亦至乎!

時鉅鹿張臶,字子明,頴川胡昭,字孔明,亦養志不仕。臶少游太學,學兼內外,後歸鄉里。袁紹前後辟命,不應,移居上黨。并州牧高幹表除樂平令,不就,徙遁常山,門徒且數百人,遷居任縣。太祖為丞相,辟,不詣。太和中,詔求隱學之士能消災復異者,郡累上臶,發遣,老病不行。廣平太守盧毓到官三日,綱紀白承前致版謁臶。毓教曰:「張先生所謂上不事天子,下不友諸侯者也。此豈版謁所可光飾哉!」但遣主簿奉書致羊酒之禮。青龍四年辛亥詔書:「張掖郡玄川溢涌,激波奮蕩,寶石負圖,狀像靈龜,宅于川西,嶷然磐峙,倉質素章,麟鳳龍馬,煥炳成形,文字告命,粲然著明。太史令高堂隆上言:古皇聖帝所未甞蒙,實有魏之禎命,東序之世寶。」尚書顧命篇曰:「大玉、夷玉、天球、河圖在東序。」注曰:「河圖,圖出於河,帝王聖者之所受。」事班天下。任令于綽連齎以問臶,臶密謂綽曰:「夫神以知來,不追已往,禎祥先見而後廢興從之。漢已乆亡,魏已得之,何所追興徵祥乎!此石,當今之變異而將來之禎瑞也。」正始元年,戴鵀之鳥,巢臶門陰。臶告門人曰:「夫戴鵀陽鳥,而巢門陰,此凶祥也。」乃援琴歌詠,作詩二篇,旬日而卒,時年一百五歲。是歲,廣平太守王肅至官,教下縣曰:「前在京都,聞張子明,來至問之,會其已亡,致痛惜之。此君篤學隱居,不與時競,以道樂身。昔絳縣老人屈在泥塗,趙孟升之,諸侯用睦。愍其耄勤好道,而不蒙榮寵,書到,遣吏勞問其家,顯題門戶,務加殊異,以慰旣往,以勸將來。」

胡昭始避地兾州,亦辭袁紹之命,遁還鄉里。太祖為司空丞相,頻加禮辟。昭往應命,旣至,自陳一介野生,無軍國之用,歸誠求去。太祖曰:「人各有志,出處異趣,勉卒雅尚,義不相屈。」昭乃轉居陸渾山中,躬耕樂道,以經籍自娛。閭里敬而愛之。高士傳曰:初,晉宣帝為布衣時,與昭有舊。同郡周生等謀害帝,昭聞而步陟險,邀生於崤、澠之間,止生,生不肯。昭泣與結誠,生感其義,乃止。昭因與斫棗樹共盟而別。昭雖有陰德於帝,口終不言,人莫知之。信行著於鄉黨。建安十六年,百姓聞馬超叛,避兵入山者千餘家,飢乏,漸相劫略,昭常遜辭以解之,是以寇難消息,衆咸宗焉。故其所居部落中,三百里無相侵暴者。建安二十三年,陸渾長張固被書調丁夫,當給漢中。百姓惡憚遠役,並懷擾擾。民孫狼等因興兵殺縣主簿,作為叛亂,縣邑殘破。固率將十餘吏卒,依昭住止,招集遺民,安復社稷。狼等遂南附關羽。羽授印給兵,還為寇賊,到陸渾南長樂亭,自相約誓,言:「胡居士賢者也,一不得犯其部落。」一川賴昭,咸無怵惕。天下安輯,徙宅宜陽。高士傳曰:幽州刺史杜恕甞過昭所居草廬之中,言事論理,辭義謙敬,恕甚重焉。太尉蔣濟辟,不就。正始中,驃騎將軍趙儼、尚書黃休、郭彝、散騎常侍荀顗、鍾毓、太僕庾嶷、案庾氏譜:嶷字劭然,頴川人。子字玄默,晉尚書、陽翟子。嶷弟遁,字德先,太中大夫。遁胤嗣克昌,為世盛門。侍中峻、河南尹純,皆遁之子,豫州牧長史顗,遁之孫,太尉文康公亮、司空冰皆遁之曾孫,貴達至今。弘農太守何楨等文士傳曰:楨字元幹,廬江人,有文學器幹,容貌甚偉。歷幽州刺史、廷尉,入晉為尚書光祿大夫。楨子龕,後將軍;勗,車騎將軍;惲,豫州刺史;其餘多至大官。自後累世昌阜,司空文穆公充,惲之孫也,貴達至今。遞薦昭曰:「天真高絜,老而彌篤。玄虛靜素,有夷、皓之節。宜蒙徵命,以勵風俗。」高士傳曰:朝廷以戎車未息,徵命之事,且須後之,昭以故不即徵。後顗、休復與庾嶷薦昭,有詔訪於本州評議。侍中韋誕駮曰:「禮賢徵士,王政之所重也,古者考行於鄉。今顗等位皆常伯納言,嶷為卿佐,足以取信。附下罔上,忠臣之所不行也。昭宿德耆艾,遺逸山林,誠宜加異。」乃從誕議也。至嘉平二年,公車特徵,會卒,年八十九。拜子纂郎中。初,昭善史書,與鍾繇、邯鄲淳、衞覬、韋誕並有名,尺牘之迹,動見模楷焉。傅子曰:胡徵君怡怡無不愛也,雖僕隷,必加禮焉。外同乎俗,內秉純絜,心非其好,王公不能屈,年八十而不倦於書籍者,吾於胡徵君見之矣。時有隱者焦先,河東人也。魏略曰:先字孝然。中平末,白波賊起。時先年二十餘,與同郡侯武陽相隨。武陽年小,有母,先與相扶接,避白波,東客揚州取婦。建安初來西還,武陽詣大陽占戶,先留陝界。至十六年,關中亂。先失家屬,獨竄於河渚間,食草飲水,無衣履。時大陽長朱南望見之,謂為亡士,欲遣船捕取。武陽語縣:「此狂癡人耳!」遂注其籍。給廩,日五升。後有疫病,人多死者,縣常使埋藏,童兒豎子皆輕易之。然其行不踐邪徑,必循阡陌;及其捃拾,不取大穗;饑不苟食,寒不苟衣,結草以為裳,科頭徒跣。每出,見婦人則隱翳,須去乃出。自作一瓜牛廬,淨埽其中。營木為牀,布草蓐其上。至天寒時,搆火以自炙,呻吟獨語。饑則出為人客作,飽食而已,不取其直。又出於道中,邂逅與人相遇,輙下道藏匿。或問其故,常言「草茅之人,與狐兔同羣」。不肯妄語。太和、青龍中,甞持一杖南渡淺河水,輙獨云未可也,由是人頗疑其不狂。至嘉平中,太守賈穆初之官,故過其廬。先見穆再拜。穆與語,不應;與食,不食。穆謂之曰:「國家使我來為卿作君,我食卿,卿不肯食,我與卿語,卿不應我,如是,我不中為卿作君,當去耳!」先乃曰:「寧有是邪?」遂不復語。其明年,大發卒將伐吳。有竊問先:「今討吳何如?」先不肯應,而謬歌曰:「祝衂祝衂,非魚非肉,更相追逐,本心為當殺牂羊,更殺其羖䍽邪!」郡人不知其謂。會諸軍敗,好事者乃推其意,疑牂羊謂吳,羖䍽謂魏,於是後人僉謂之隱者也。議郎河東董經特嘉異節,與先非故人,密往觀之。經到,乃奮其白鬚,為如與之有舊者,謂曰:「阿先闊乎!念共避白波時不?」先熟視而不言。經素知其昔受武陽恩,因復曰:「念武陽不邪?」先乃曰:「已報之矣。」經又復挑欲與語,遂不肯復應。後歲餘病亡,時年八十九矣。高士傳曰:世莫知先所出。或言生乎漢末,自陝居大陽,無父母兄弟妻子。見漢室衰,乃自絕不言。及魏受禪,常結草為廬於河之湄,獨止其中。冬夏恒不著衣,卧不設席,又無草蓐,以身親土,其體垢汙皆如泥漆,五形盡露,不行人間。或數日一食,欲食則為人賃作,人以衣衣之,乃使限功受直,足得一食輙去,人欲多與,終不肯取,亦有數日不食時。行不由邪徑,目不與女子逆視。口未甞言,雖有驚急,不與人語。遺以食物皆不受。河東太守杜恕甞以衣服迎見,而不與語。司馬景王聞而使安定太守董經因事過視,又不肯語,經以為大賢。其後野火燒其廬,先因露寢。遭冬雪大至,先袒卧不移,人以為死,就視如故,不以為病,人莫能審其意。度年可百歲餘乃卒。或問皇甫謐曰:「焦先何人?」曰:「吾不足以知之也。考之於表,可略而言矣。夫世之所常趣者榮味也,形之所不可釋者衣裳也,身之所不可離者室宅也,口之所不能已者言語也,心之不可絕者親戚也。今焦先棄榮味,釋衣服,離室宅,絕親戚,閉口不言,曠然以天地為棟宇,闇然合至道之前,出羣形之表,入玄寂之幽,一世之人不足以挂其意,四海之廣不能以回其顧,妙乎與夫三皇之先者同矣。結繩已來,未及其至也,豈羣言之所能髣髴,常心之所得測量哉!彼行人所不能行,堪人所不能堪,犯寒暑不以傷其性,居曠野不以恐其形,遭驚急不以迫其慮,離榮愛不以累其心,損視聽不以汙其耳目,舍足於不損之地,居身於獨立之處,延年歷百,壽越期頤,雖上識不能尚也。自羲皇已來,一人而已矣!」魏氏春秋曰:故梁州刺史耿黼以先為「仙人」也,北海傅玄謂之「性同禽獸」,並為之傳,而莫能測之。魏略又載扈累及寒貧者。累字伯重,京兆人也。初平中,山東人有青牛先生者,字正方,客三輔。曉知星曆、風角、鳥情。常食青葙芫華。年似如五六十者,人或親識之,謂其已百餘歲矣。初,累年四十餘,隨正方游學,人謂之得其術。有婦,無子。建安十六年,三輔亂,又隨正方南入漢中。漢中壞,正方入蜀,累與相失,隨徙民詣鄴,遭疾疫喪其婦。至黃初元年,又徙詣洛陽,遂不復娶婦。獨居道側,以㼾甎為障,施一廚牀,食宿其中。晝日潛思,夜則仰視星宿,吟詠內書。人或問之,閉口不肯言。至嘉平中,年八九十,裁若四五十者。縣官以其孤老,給廩日五升。五升不足食,頗行傭作以裨糧,糧盡復出,人與不取。食不求美,衣弊縕故,後一二年病亡。寒貧者,本姓石,字德林,安定人也。建安初,客三輔。是時長安有宿儒欒文博者,門徒數千,德林亦就學,始精詩、書。後好內事,於衆輩中最玄默。至十六年,關中亂,南入漢中。初不治產業,不畜妻孥,常讀老子五千文及諸內書,晝夜吟詠。到二十五年,漢中破,隨衆還長安,遂癡愚不復識人。食不求味,冬夏常衣弊布連結衣。體如無所勝,目如無所見。獨居窮巷小屋,無親里。人與之衣食,不肯取。郡縣以其鰥窮,給廩日五升,食不足,頗行乞,乞不取多。人問其姓字,口不肯言,故因號之曰寒貧也。或素有與相知者,往存恤之,輙拜跪,由是人謂其不癡。車騎將軍郭淮以意氣呼之,問其所欲,亦不肯言。淮因與脯糒及衣,不取其衣,取其脯一朐、糒一升而止。臣松之案魏略云:焦先及楊沛,並作瓜牛廬,止其中。以為瓜當作蝸;蝸牛,螺蟲之有角者也,俗或呼為黃犢。先等作圜舍,形如蝸牛蔽,故謂之蝸牛廬。莊子曰:「有國於蝸之左角者曰觸氏,有國於右角者曰蠻氏,時相與爭地而戰,伏尸數萬,逐北旬有五日而後反。」謂此物也。

評曰:袁渙、邴原、張範躬履清蹈,進退以道,臣松之以為蹈猶履也,「躬履清蹈」,近非言乎!蓋是貢禹、兩龔之匹。涼茂、國淵亦其次也。張承名行亞範,可謂能弟矣。田疇抗節,王脩忠貞,足以矯俗;管寧淵雅高尚,確然不拔;張臶、胡昭闔門守靜,不營當世:故并錄焉。


\end{pinyinscope}