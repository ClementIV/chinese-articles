\article{衞覬傳}

\begin{pinyinscope}
衞覬字伯儒,河東安邑人也。少夙成,以才學稱。太祖辟為司空掾屬,除茂陵令、尚書郎。太祖征袁紹,而劉表為紹援,關中諸將又中立。益州牧劉璋與表有隙,覬以治書侍御史使益州,令璋下兵以綴表軍。至長安,道路不通,覬不得進,遂留鎮關中。時四方大有還民,關中諸將多引為部曲,覬書與荀彧曰:「關中膏腴之地,頃遭荒亂,人民流入荊州者十萬餘家,聞本土安寧,皆企望思歸。而歸者無以自業,諸將各競招懷,以為部曲。郡縣貧弱,不能與爭,兵家遂彊。一旦變動,必有後憂。夫鹽,國之大寶也,自亂來散放,宜如舊置使者監賣,以其直益巿犂牛。若有歸民,以供給之。勤耕積粟,以豐殖關中。遠民聞之,必日夜競還。又使司隷校尉留治關中以為之主,則諸將日削,官民日盛,此彊本弱敵之利也。」彧以白太祖。太祖從之,始遣謁者僕射監鹽官,司隷校尉治弘農。關中服從,乃白召覬還,稍遷尚書。

魏書曰:初,漢朝遷移,臺閣舊事散亂。自都許之後,漸有綱紀,覬以古義多所正定。是時關西諸將,外雖懷附,內未可信。司隷校尉鍾繇求以三千兵入關,外託討張魯,內以脅取質任。太祖使荀彧問覬,覬以為「西方諸將,皆豎夫屈起,無雄天下意,苟安樂目前而已。今國家厚加爵號,得其所志,非有大故,不憂為變也。宜為後圖。若以兵入關中,當討張魯,魯在深山,道徑不通,彼必疑之;一相驚動,地險衆彊,殆難為慮!」彧以覬議呈太祖。太祖初善之,而以繇自典其任,遂從繇議。兵始進而關右大叛,太祖自親征,僅乃平之,死者萬計。太祖悔不從覬議,由是益重覬。魏國旣建,拜侍中,與王粲並典制度。文帝即位,徙為尚書。頃之,還漢朝為侍郎,勸贊禪代之義,為文誥之詔。文帝踐阼,復為尚書,封陽吉亭侯。

明帝即位,進封閺鄉侯,三百戶。閺音聞。覬奏曰:「九章之律,自古所傳,斷定刑罪,其意微妙。百里長吏,皆宜知律。刑法者,國家之所貴重,而私議之所輕賤;獄吏者,百姓之所縣命,而選用者之所卑下。王政之弊,未必不由此也。請置律博士,轉相教授。」事遂施行。時百姓凋匱而役務方殷,覬上疏曰:「夫變情厲性,彊所不能,人臣言之旣不易,人主受之又艱難。且人之所樂者富貴顯榮也,所惡者貧賤死亡也,然此四者,君上之所制也,君愛之則富貴顯榮,君惡之則貧賤死亡;順指者愛所由來,逆意者惡所從至也。故人臣皆爭順指而避逆意,非破家為國,殺身成君者,誰能犯顏色,觸忌諱,建一言,開一說哉?陛下留意察之,則臣下之情可見矣。今議者多好恱耳,其言政治則比陛下於堯舜,其言征伐則比二虜於貍鼠。臣以為不然。昔漢文之時,諸侯彊大,賈誼累息以為至危。況今四海之內,分而為三,羣士陳力,各為其主。其來降者,未肯言舍邪就正,咸稱迫於困急,是與六國分治,無以為異也。當今千里無煙,遺民困苦,陛下不善留意,將遂凋弊難可復振。禮,天子之器必有金玉之飾,飲食之肴必有八珎之味,至於凶荒,則徹膳降服。然則奢儉之節,必視世之豐約也。武皇帝之時,後宮食不過一肉,衣不用錦繡,茵蓐不緣飾,器物無丹漆,用能平定天下,遺福子孫。此皆陛下之所親覽也。當今之務,宜君臣上下,並用籌策,計校府庫,量入為出。深思句踐滋民之術,由恐不及,而尚方所造金銀之物,漸更增廣,工役不輟,侈靡日崇,帑藏日竭。昔漢武信求神仙之道,謂當得雲表之露以餐玉屑,故立僊掌以承高露。陛下通明,每所非笑。漢武有求於露,而由尚見非,陛下無求於露而空設之;不益於好而糜費功夫,誠皆聖慮所宜裁制也。」覬歷漢、魏,時獻忠言,率如此。

受詔典著作,又為魏官儀,凡所撰述數十篇。好古文、鳥篆、隷草,無所不善。建安末,尚書右丞河南潘勗,文章志曰:勗字元茂,初名芝,改名勗,後避諱。或曰勗獻帝時為尚書郎,遷右丞。詔以勗前在二千石曹,才敏兼通,明習舊事,勑并領本職,數加特賜。二十年,遷東海相。未發,留拜尚書左丞。其年病卒,時年五十餘。魏公九錫策命,勗所作也。勗子滿,平原太守,亦以學行稱。滿子尼,字正叔。尼別傳曰:尼少有清才,文辭溫雅。初應州辟,後以父老歸供養。居家十餘年,父終,晚乃出仕。尼嘗贈陸機詩,機荅之,其四句曰:「猗歟潘生,世篤其藻,仰儀前文,丕隆祖考。」位終太常。尼從父岳,字安仁。岳別傳曰:岳美姿容,夙以才穎發名。其所著述,清綺絕倫。為黃門侍郎,為孫秀所殺。尼、岳文翰,並見重於世。尼從子滔,字湯仲。晉諸公贊:滔以博學才量為名。永嘉末,為河南尹,遇害。黃初時,散騎常侍河內王象,亦與覬並以文章顯。王象事別見楊俊傳。覬薨,謚曰敬侯。子瓘嗣。瓘咸熈中為鎮西將軍。晉陽秋曰:瓘字伯玉。清貞有名理,少為傅嘏所知。弱冠為尚書郎,遂歷位內外,為晉尚書令、司空、太保。惠帝初輔政,為楚王瑋所害。世語曰:瓘與扶風內史燉煌索靖,並善草書。瓘子恒,字巨山,黃門侍郎。恒子玠,字叔寶,有盛名,為太子洗馬,早卒。


\end{pinyinscope}