\article{陳震傳}

\begin{pinyinscope}
陳震字孝起,南陽人也。先主領荊州牧,辟為從事,部諸郡,隨先主入蜀。蜀旣定,為蜀郡北部都尉,因易郡名,為汶山太守,轉在犍為。建興三年,入拜尚書,遷尚書令,奉命使吳。七年,孫權稱尊號,以震為衞尉,賀權踐阼,諸葛亮與兄瑾書曰:「孝起忠純之性,老而益篤,及其贊述東西,歡樂和合,有可貴者。」震入吳界,移關候曰:「東之與西,驛使往來,冠蓋相望,申盟初好,日新其事。東尊應保聖祚,告燎受符,剖判土宇,天下響應,各有所歸。於此時也,以同心討賊,則何寇不滅哉!西朝君臣,引領欣賴。震以不才,得充下使,奉聘叙好,踐界踊躍,入則如歸。獻子適魯,犯其山諱,孔子譏之。望必啟告,使行人睦焉。即日張旍誥衆,各自約誓。順流漂疾,國典異制,懼或有違,幸必斟誨,示其所宜。」震到武昌,孫權與震升壇歃盟,交分天下:以徐、豫、幽、青屬吳,并、涼、兾、兖屬蜀,其司州之土,以函谷關為界。震還,封城陽亭侯。九年,都護李平坐誣罔廢;諸葛亮與長史蔣琬、侍中董允書曰:「孝起前臨至吳,為吾說正方腹中有鱗甲,鄉黨以為不可近。吾以為鱗甲者但不當犯之耳,不圖復有蘇、張之事出於不意。可使孝起知之。」十三年,震卒。子濟嗣。


\end{pinyinscope}