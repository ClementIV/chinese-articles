\article{劉理傳}

\begin{pinyinscope}
劉理字奉孝,亦後主庶弟也,與永異母。章武元年六月,使司徒靖立理為梁王,策曰:「小子理,朕統承漢序,祗順天命,遵脩典秩,建爾于東,為漢藩輔。惟彼梁土,畿甸之邦,民狎教化,易導以禮。往悉乃心,懷保黎庶,以永爾國,王其敬之哉!」建興八年,改封理為安平王。延熈七年卒,謚曰悼王。子哀王胤嗣,十九年卒。子殤王承嗣,二十年卒。景耀四年詔曰:「安平王,先帝所命。三世早夭,國嗣頹絕,朕用傷悼。其以武邑侯輯襲王位。」輯,理子也,咸熈元年,東遷洛陽,拜奉車都尉,封鄉侯。


\end{pinyinscope}