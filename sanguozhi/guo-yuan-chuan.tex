\article{guo-yuan-chuan}

\begin{pinyinscope}
國淵字子尼,樂安蓋人也。師事鄭玄。

玄別傳曰:淵始未知名,玄稱之曰:「國子尼,美才也,吾觀其人,必為國器。」後與邴原、管寧等避亂遼東。魏書曰:淵篤學好古,在遼東,常講學於山巖,士人多推慕之,由此知名。旣還舊土,太祖辟為司空掾屬,每於公朝論議,常直言正色,退無私焉。太祖欲廣置屯田,使淵典其事。淵屢陳損益,相土處民,計民置吏,明功課之法,五年中倉廩豐實,百姓競勸樂業。太祖征關中,以淵為居府長史,統留事。田銀、蘇伯反河閒,銀等旣破,後有餘黨,皆應伏法。淵以為非首惡,請不行刑。太祖從之,賴淵得生者千餘人。破賊文書,舊以一為十,及淵上首級,如其實數。太祖問其故,淵曰:「夫征討外寇,多其斬獲之數者,欲以大武功,且示民聽也。河閒在封域之內,銀等叛逆,雖克捷有功,淵竊恥之。」太祖大恱,遷魏郡太守。

時有投書誹謗者,太祖疾之,欲必知其主。淵請留其本書,而不宣露。其書多引二京賦,淵勑功曹曰:「此郡旣大,今在都輦,而少學問者。其簡開解年少,欲遣就師。」功曹差三人,臨遣引見,訓以「所學未及,二京賦,博物之書也,世人忽略,少有其師,可求能讀者從受之。」又密喻旨。旬日得能讀者,遂往受業。吏因請使作箋,比方其書,與投書人同手。收攝案問,具得情理。遷太僕。居列卿位,布衣蔬食,祿賜散之舊故宗族,以恭儉自守,卒官。魏書曰:太祖以其子泰為郎。


\end{pinyinscope}