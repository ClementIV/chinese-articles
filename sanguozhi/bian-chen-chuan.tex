\article{弁辰傳}

\begin{pinyinscope}
弁辰亦十二國,又有諸小別邑,各有渠帥,大者名臣智,其次有險側,次有樊濊,次有殺奚,次有邑借。有已柢國、不斯國、弁辰彌離彌凍國、弁辰接塗國、勤耆國、難彌離彌凍國、弁辰古資彌凍國、弁辰古淳是國、冉奚國、弁辰半路國、弁樂奴國、軍彌國、弁軍彌國、弁辰彌烏邪馬國、如湛國、弁辰甘路國、戶路國、州鮮國、馬延國、弁辰狗邪國、弁辰走漕馬國、弁辰安邪國、弁辰瀆盧國、斯盧國、優中國。弁、辰韓合二十四國,大國四五千家,小國六七百家,總四五萬戶。其十二國屬辰王。辰王常用馬韓人作之,世世相繼。辰王不得自立為王。

魏略曰:明其為流移之人,故為馬韓所制。土地肥美,宜種五糓及稻,曉蠶桑,作縑布,乘駕牛馬。嫁娶禮俗,男女有別。以大鳥羽送死,其意欲使死者飛揚。魏略曰:其國作屋,橫累木為之,有似牢獄也。國出鐵,韓、濊、倭皆從取之。諸巿買皆用鐵,如中國用錢,又以供給二郡。俗喜歌舞飲酒。有瑟,其形似筑,彈之亦有音曲。兒生,便以石厭其頭,欲其褊。今辰韓人皆褊頭。男女近倭,亦文身。便步戰,兵仗與馬韓同。其俗,行者相逢,皆住讓路。

弁辰與辰韓雜居,亦有城郭。衣服居處與辰韓同。言語法俗相似,祠祭鬼神有異,施竈皆在戶西。其瀆盧國與倭接界。十二國亦有王,其人形皆大。衣服絜清,長髮。亦作廣幅細布。法俗特嚴峻。


\end{pinyinscope}