\article{華佗傳}

\begin{pinyinscope}
華佗字元化,沛國譙人也,一名旉。

臣松之案:古「敷」字與「專」相似,寫書者多不能別。尋佗字元化,其名宜為旉也。游學徐土,兼通數經。沛相陳珪舉孝廉,太尉黃琬辟,皆不就。曉養性之術,時人以為年且百歲而皃有壯容。又精方藥,其療疾,合湯不過數種,心解分劑,不復稱量,煑熟便飲,語其節度,舍去輙愈。若當灸,不過一兩處,每處不過七八壯,病亦應除。若當針,亦不過一兩處,下針言「當引某許,若至,語人」。病者言「已到」,應便拔針,病亦行差。若病結積在內,針藥所不能及,當須刳割者,便飲其麻沸散,須臾便如醉死無所知,因破取。病若在腸中,便斷腸湔洗,縫腹膏摩,四五日差,不痛,人亦不自寤,一月之間,即平復矣。

故甘陵相夫人有娠六月,腹痛不安,佗視脉,日:「胎巳死矣。」使人手摸知所在,在左則男,在右則女。人云「在左」,於是為湯下之,果下男形,即愈。

縣吏尹世苦四支煩,口中乾,不欲聞人聲,小便不利。佗曰:「試作熱食,得汗則愈;不汗,後三日死。」即作熱食而不汗出,佗曰:「藏氣已絕於內,當啼泣而絕。」果如佗言。

府吏兒尋、李延共止,俱頭痛身熱,所苦正同。佗曰:「尋當下之,延當發汗。」或難其異,佗曰:「尋外實,延內實,故治之宜殊。」即各與藥,明旦並起。

鹽瀆嚴昕與數人共候佗,適至,佗謂昕曰:「君身中佳否?」昕曰:「自如常。」佗曰:「君有急病見於靣,莫多飲酒。」坐畢歸,行數里,昕卒頭眩墯車,人扶將還,載歸家,中宿死。

故督郵頓子獻得病已差,詣佗視脉,曰:「尚虛,未得復,勿為勞事,御內即死。臨死,當吐舌數寸。」其妻聞其病除,從百餘里來省之,止宿交接,中間三日發病,一如佗言。

督郵徐毅得病,佗往省之。毅謂佗曰:「昨使醫曹吏劉租針胃管訖,便苦欬嗽,欲卧不安。」佗曰:「刺不得胃管,誤中肝也,食當日減,五日不救。」遂如佗言。

東陽陳叔山小男二歲得疾,下利常先啼,日以羸困。問佗,佗曰:「其母懷軀,陽氣內養,乳中虛冷,兒得母寒,故令不時愈。」佗與四物女宛丸,十日即除。

彭城夫人夜之厠,蠆螫其手,呻呼無賴。佗令溫湯近熱,漬手其中,卒可得寐,但旁人數為易湯,湯令煖之,其旦即愈。

軍吏梅平得病,除名還家,家居廣陵,未至二百里,止親人舍。有頃,佗偶至主人許,主人令佗視平,佗謂平曰:「君早見我,可不至此。今疾已結,促去可得與家相見,五日卒。」應時歸,如佗所刻。

佗行道,見一人病咽塞,嗜食而不得下,家人車載欲往就醫。佗聞其呻吟,駐車往視,語之曰:「向來道邊有賣餅家蒜韲大酢,從取三升飲之,病自當去。」即如佗言,立吐虵一枚,縣車邊,欲造佗。佗尚未還,小兒戲門前,逆見,自相謂曰:「似逢我公,車邊病是也。」疾者前入坐,見佗北壁縣此蛇輩約以十數。

又有一郡守病,佗以為其人盛怒則差,乃多受其貨而不加治,無何棄去,留書罵之。郡守果大怒,令人追捉殺佗。郡守子知之,屬使勿逐。守瞋恚旣甚,吐黑血數升而愈。

又有一士大夫不快,佗云:「君病深,當破腹取。然君壽亦不過十年,病不能殺君,忍病十歲,壽俱當盡,不足故自刳裂。」士大夫不耐痛癢,必欲除之。佗遂下手,所患尋差,十年竟死。

廣陵太守陳登得病,胷中煩懣,靣赤不食。佗脉之曰:「府君胃中有蟲數升,欲成內疽,食腥物所為也。」即作湯二升,先服一升,斯須盡服之。食頃,吐出三升許蟲,赤頭皆動,半身是生魚膾也,所苦便愈。佗曰:「此病後三期當發,遇良醫乃可濟救。」依期果發動,時佗不在,如言而死。

太祖聞而召佗,佗常在左右。太祖苦頭風,每發,心亂目眩,佗針鬲,隨手而差。佗別傳曰:有人病兩脚躄不能行,轝詣佗,佗望見云:「己飽針灸服藥矣,不復須看脉。」便使解衣,點背數十處,相去或一寸,或五寸,縱邪不相當。言灸此各十壯,灸創愈即行。後灸處夾脊一寸,上下行端直均調,如引繩也。

李將軍妻病甚,呼佗視脉,曰:「傷娠而胎不去。」將軍言:「聞實傷娠,胎已去矣。」佗曰:「案脉,胎未去也。」將軍以為不然。佗舍去,婦稍小差。百餘日復動,更呼佗,佗曰:「此脉故事有胎。前當生兩兒,一兒先出,血出甚多,後兒不及生。母不自覺,旁人亦不寤,不復迎,遂不得生。胎死,血脉不復歸,必燥著母脊,故使多脊痛。今當與湯,并針一處,此死胎必出。」湯針旣加,婦痛急如欲生者。佗曰:「此死胎乆枯,不能自出,宜使人探之。」果得一死男,手足完具,色黑,長可尺所。

佗之絕技,凡此類也。然本作士人,以醫見業,意常自悔,後太祖親理,得病篤重,使佗專視。佗曰:「此近難濟,恒事攻治,可延歲月。」佗乆遠家思歸,因曰:「當得家書,方欲暫還耳。」到家,辭以妻病,數乞期不反。太祖累書呼,又勑郡縣發遣。佗恃能厭食事,猶不上道。太祖大怒,使人往檢。若妻信病,賜小豆四十斛,寬假限日;若其虛詐,便收送之。於是傳付許獄,考驗首服。荀彧請曰:「佗術實工,人命所縣,宜含宥之。」太祖曰:「不憂,天下當無此鼠輩耶?」遂考竟佗。佗臨死,出一卷書與獄吏,曰:「此可以活人。」吏畏法不受,佗亦不彊,索火燒之。佗死後,太祖頭風未除。太祖曰:「佗能愈此。小人養吾病,欲以自重,然吾不殺此子,亦終當不為我斷此根原耳。」及後愛子倉舒病困,太祖歎曰:「吾悔殺華佗,令此兒彊死也。」

初,軍吏李成苦欬嗽,晝夜不寤,時吐膿血,以問佗。佗言:「君病腸臃,欬之所吐,非從肺來也。與君散兩錢,當吐二升餘膿血訖,快自養,一月可小起,好自將愛,一年便健。十八歲當一小發,服此散,亦行復差。若不得此藥,故當死。」復與兩錢散,成得藥去。五六歲,親中人有病如成者,謂成曰:「卿今彊健,我欲死,何忍無急去藥,臣松之案:古語以藏為去。以待不祥?先持貸我,我差,為卿從華佗更索。」成與之。已故到譙,適值佗見收,怱怱不忍從求。後十八歲,成病竟發,無藥可服,以至於死。佗別傳曰:人有在青龍中見山陽太守廣陵劉景宗,景宗說中平曰「數見華佗,其治病手脉之候,其驗若神。」琅琊劉勳為河內太守,有女年幾二十,左脚膝裏上有瘡,癢而不痛。瘡愈數十日復發,如此七八年,迎佗使視,佗曰:「是易治之。當得稻糠黃色犬一頭,好馬二匹。」以繩繫犬頸,使走馬牽犬,馬極輒易,計馬走三十餘里,犬不能行,復令步人拖拽,計向五十里。乃以藥飲女,女即安卧不知人。因取大刀斷犬腹近後脚之前,以所斷之處向瘡口,令去二三寸。停之須臾,有若蛇者從瘡中而出,便以鐵椎橫貫蛇頭。蛇在皮中動搖良久,須臾不動,乃牽出,長三尺所,純是蛇,但有眼處而無童子,又逆鱗耳。以膏散著瘡中,七日愈。又有人苦頭眩,頭不得舉,目不得視,積年。佗使悉解衣倒懸,令頭去地一二寸,濡布拭身體,令周帀,候視諸脉,盡出五色。佗令子弟數人以鈹刀決脉,五色血盡,視赤血,乃下,以膏摩被覆,汗自出周帀,飲以亭歷犬血散,立愈。又有婦人長病經年,世謂寒熱注病者。冬十一月中,佗令坐石槽中,平旦用寒水汲灌,云當滿百。始七八灌,會戰欲死,灌者懼,欲止。佗令滿數。將至八十灌,熱氣乃蒸出,嚻嚻高二三尺。滿百灌,佗乃使然火溫牀,厚覆,良久汗洽出,著粉,汗燥便愈。又有人病腹中半切痛,十餘日中,鬢眉墮落。佗曰:「是脾半腐,可刳腹養治也。」使飲藥令卧,破腹就視,脾果半腐壞。以刀斷之,刮去惡肉,以膏傅瘡,飲之以藥,百日平復。

廣陵吳普、彭城樊阿皆從佗學。普依準佗治,多所全濟。佗語普曰:「人體欲得勞動,但不當使極爾。動搖則糓氣得消,血脉流通,病不得生,譬猶戶樞不朽是也。是以古之仙者為導引之事,熊頸鴟顧,引輓腰體,動諸關節,以求難老。吾有一術,名五禽之戲,一曰虎,二曰鹿,三曰熊,四曰猨,五曰鳥,亦以除疾,並利蹄足,以當導引。體中不快,起作一禽之戲,沾濡汗出,因上著粉,身體輕便,腹中欲食。」普施行之,年九十餘,耳目聦明,齒牙完堅。阿善針術。凡醫咸言背及胷藏之間不可妄針,針之不過四分,而阿針背入一二寸,巨闕胷藏針下五六寸,而病輙皆瘳。阿從佗求可服食益於人者,佗授以漆葉青黏散。漆葉屑一升,青黏屑十四兩,以是為率,言乆服去三蟲,利五藏,輕體,使人頭不白。阿從其言,壽百餘歲。漆葉處所而有,青黏生於豐、沛、彭城及朝歌云。佗別傳曰:青黏者,一名地節,一名黃芝,主理五臟,益精氣。本出於迷入山者,見仙人服之,以告佗。佗以為佳,輒語阿,阿又祕之。近者人見阿之壽而氣力彊盛,怪之,遂責阿所服,因醉亂誤道之。法一施,人多服者,皆有大驗。文帝典論論郤儉等事曰:「潁川郤儉能辟穀,餌伏苓。甘陵甘始亦善行氣,老有少容。廬江左慈知補導之術。並為軍吏。初,儉之至,巿伏苓價暴數倍。議郎安平李覃學其辟穀,餐伏苓,飲寒水,中泄利,殆至隕命。後始來,衆人無不鳥視狼顧,呼吸吐納。軍謀祭酒弘農董芬為之過差,氣閉不通,良久乃蘇。左慈到,又競受其補導之術,至寺人嚴峻,往從問受。閹豎真無事於斯術也,人之逐聲,乃至於是。光和中,北海王和平亦好道術,自以當仙。濟南孫邕少事之,從至京師。會和平病死,邕因葬之東陶,有書百餘卷,藥數囊,悉以送之。後弟子夏榮言其尸解。邕至今恨不取其寶書仙藥。劉向惑於鴻寶之說,君游眩於子政之言,古今愚謬,豈惟一人哉!」東阿王作辯道論曰:「世有方士,吾王悉所招致,甘陵有甘始,廬江有左慈,陽城有郤儉。始能行氣導引,慈曉房中之術,儉善辟穀,悉號三百歲。卒所以集之於魏國者,誠恐斯人之徒,接姦宄以欺衆,行妖慝以惑民,豈復欲觀神仙於瀛洲,求安期於海島,釋金輅而履雲輿,棄六驥而美飛龍哉?自家王與太子及余兄弟咸以為調笑,不信之矣。然始等知上遇之有恒,奉不過於員吏,賞不加於無功,海島難得而游,六黻難得而佩,終不敢進虛誕之言,出非常之語。余甞試郤儉絕穀百日,躬與之寢處,行步起居自若也。夫人不食七日則死,而儉乃如是。然不必益壽,可以療疾而不憚飢饉焉。左慈善脩房內之術,差可終命,然自非有志至精,莫能行也。甘始者,老而有少容,自諸術士咸共歸之。然始辭繁寡實,頗有怪言。余嘗辟左右,獨與之談,問其所行,溫顏以誘之,美辭以導之,始語余:『吾本師姓韓字世雄,嘗與師於南海作金,前後數四,投數萬斤金於海。』又言:『諸梁時,西域胡來獻香罽、腰帶、割玉刀,時悔不取也。』又言:『車師之西國。兒生,擘背出脾,欲其食少而弩行也。』又言:『取鯉魚五寸一雙,合其一煑藥,俱投沸膏中,有藥者奮尾鼓鰓,游行沈浮,有若處淵,其一者已熟而可噉。』余時問:『言率可試不?』言:『是藥去此逾萬里,當出塞;始不自行不能得也。』言不盡於此,頗難悉載,故粗舉其巨怪者。始若遭秦始皇、漢武帝,則復為徐巿、欒大之徒也。」


\end{pinyinscope}