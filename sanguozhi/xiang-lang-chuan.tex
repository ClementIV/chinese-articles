\article{向朗傳}

\begin{pinyinscope}
向朗字巨達,襄陽宜城人也。

襄陽記曰:朗少師事司馬德操,與徐元直、韓德高、龐士元皆親善。荊州牧劉表以為臨沮長。表卒,歸先主。先主定江南,使朗督秭歸、夷道、巫山、夷陵四縣軍民事。蜀旣平,以朗為巴西太守,頃之轉任䍧牱,又徙房陵。後主踐阼,為步兵校尉,代王連領丞相長史。丞相亮南征,朗留統後事。五年,隨亮漢中。朗素與馬謖善,謖逃亡,朗知情不舉,亮恨之,免官還成都。數年,為光祿勳,亮卒後徒左將軍,追論舊功,封顯明亭侯,位特進。初,朗少時雖涉獵文學,然不治素檢,以吏能見稱。自去長史,優游無事垂三十年,臣松之案:朗坐馬謖免長史,則建興六年中也。朗至延熈十年卒,整二十年耳,此云「三十」,字之誤也。乃更潛心典籍,孜孜不倦。年踰八十,猶手自校書,刊定謬誤,積聚篇卷,於時最多。開門接賔,誘納後進,但講論古義,不干時事,以是見稱。上自執政,下及童冠,皆敬重焉。延熈十年卒。襄陽記曰:朗遺言戒子曰:「傳稱師克在和不在衆,此言天地和則萬物生,君臣和則國家平,九族和則動得所求,靜得所安,是以聖人守和,以存以亡也。吾,楚國之小子耳,而早喪所天,為二兄所誘養,使其性行不隨祿利以墯。今但貧耳;貧非人患,惟和為貴,汝其勉之!」子條嗣,景耀中為御史中丞。襄陽記曰:條字文豹,亦博學多識,入晉為江陽太守、南中軍司馬。

朗兄子寵,先主時為牙門將。秭歸之敗,寵營特完。建興元年封都亭侯,後為中部督,典宿衞兵。諸葛亮當北行,表與後主曰:「將軍向寵,性行淑均,曉暢軍事,試用於昔,先帝稱之曰能,是以衆論舉寵為督。愚以為營中之事,悉以咨之,必能使行陣和睦,優劣得所也。」遷中領軍。延熈三年,征漢嘉蠻夷,遇害。寵弟充,歷射聲校尉尚書。襄陽記曰:魏咸熈元年六月,鎮西將軍衞瓘至於成都,得璧玉印各一枚,文似「成信」字,魏人宣示百官,藏于相國府。充聞之曰:「吾聞譙周之言,先帝諱備,其訓具也,後主諱禪,其訓授也,如言劉已具矣,當授與人也。今中撫軍名炎,而漢年極於炎興,瑞出成都,而藏之於相國府,此殆天意也。」是歲,拜充為梓潼太守,明年十二月而晉武帝即尊位,炎興於是乎徵焉。孫盛曰:昔公孫述自以起成都,號曰成氏,二玉之文,殆述所作乎!


\end{pinyinscope}