\article{李嚴傳}

\begin{pinyinscope}
李嚴字正方,南陽人也。少為郡職吏,以才幹稱。荊州牧劉表使歷諸郡縣。曹公入荊州時,嚴宰秭歸,遂西詣蜀,劉璋以為成都令,復有能名。建安十八年,署嚴為護軍,拒先主於緜竹。嚴率衆降先主,先主拜嚴裨將軍。成都旣定,為犍為太守、興業將軍。二十三年,盜賊馬秦、高勝等起事於郪,

音淒。合聚部伍數萬人,到資中縣。時先主在漢中,嚴不更發兵,但率將郡士五千人討之,斬秦、勝等首。枝黨星散,悉復民籍。又越嶲夷率高定遣軍圍新道縣,嚴馳往赴救,賊皆破走。加輔漢將軍,領郡如故。

章武二年,先主徵嚴詣永安宮,拜尚書令。三年,先主疾病,嚴與諸葛亮並受遺詔輔少主;以嚴為中都護,統內外軍事,留鎮永安。建興元年,封都鄉侯,假節,加光祿勳。四年,轉為前將軍。以諸葛亮欲出軍漢中,嚴當知後事,移屯江州,留護軍陳到駐永安,皆統屬嚴。嚴與孟達書曰:「吾與孔明俱受寄託,憂深責重,思得良伴。」亮亦與達書曰:「部分如流,趨捨罔滯,正方性也。」其見貴重如此。諸葛亮集有嚴與亮書,勸亮宜受九錫,進爵稱王。亮荅書曰:「吾與足下相知乆矣,可不復相解!足下方誨以光國,戒之以勿拘之道,是以未得默已。吾本東方下士,誤用於先帝,位極人臣,祿賜百億,今討賊未效,知己未荅,而方寵齊、晉,坐自貴大,非其義也。若滅魏斬叡,帝還故居,與諸子並升,雖十命可受,況於九邪!」八年,遷驃騎將軍。以曹真欲三道向漢川,亮命嚴將二萬人赴漢中。亮表嚴子豐為江州都督督軍,典嚴後事。亮以明年當出軍,命嚴以中都護署府事。嚴改名為平。

九年春,亮軍祁山,平催督運事。秋夏之際,值天霖雨,運糧不繼,平遣參軍狐忠、督軍成藩喻指,呼亮來還;亮承以退軍。平聞軍退,乃更陽驚,說「軍糧饒足,何以便歸」!欲以解己不辦之責,顯亮不進之愆也。又表後主,說「軍偽退,欲以誘賊與戰」。亮具出其前後手筆書疏本末,平違錯章灼。平辭窮情竭,首謝罪負。於是亮表平曰:「自先帝崩後,平所在治家,尚為小惠,安身求名,無憂國之事。臣當北出,欲得平兵以鎮漢中,平窮難縱橫,無有來意,而求以五郡為巴州刺史。去年臣欲西征,欲令平主督漢中,平說司馬懿等開府辟召。臣知平鄙情,欲因行之際偪臣取利也,是以表平子豐督主江州,隆崇其遇,以取一時之務。平至之日,都委諸事,羣臣上下皆怪臣待平之厚也。正以大事未定,漢室傾危,伐平之短,莫若襃之。然謂平情在於榮利而已,不意平心顛倒乃爾。若事稽留,將致禍敗,是臣不敏,言多增咎。」亮公文上尚書曰:「平為大臣,受恩過量,不思忠報,橫造無端,危恥不辦,迷罔上下,論獄棄科,導人為姧,狹情志狂,若無天地。自度姧露,嫌心遂生,聞軍臨至,西嚮託疾還沮、漳,軍臨至沮,復還江陽,平參軍狐忠勤諫乃止。今篡賊未滅,社稷多難,國事惟和,可以克捷,不可苞含,以危大業。輙與行中軍師車騎將軍都鄉侯臣劉琰,使持節前軍師征西大將軍領涼州刺史南鄭侯臣魏延、前將軍都亭侯臣袁綝、左將軍領荊州刺史高陽鄉侯臣吳壹、督前部右將軍玄鄉侯臣高翔、督後部後將軍安樂亭侯臣吳班、領長史綏軍將軍臣楊儀、督左部行中監軍揚武將軍臣鄧芝、行前監軍征南將軍臣劉巴、行中護軍偏將軍臣費禕、行前護軍偏將軍漢成亭侯臣許允、行左護軍篤信中郎將臣丁咸、行右護軍偏將軍臣劉敏、行護軍征南將軍當陽亭侯臣姜維、行中典軍討虜將軍臣上官雝、行中參軍昭武中郎將臣胡濟、行參軍建議將軍臣閻晏、行參軍偏將軍臣爨習、行參軍裨將軍臣杜義、行參軍武略中郎將臣杜祺、行參軍綏戎都尉臣盛勃、領從事中郎武略中郎將臣樊岐等議,輙解平任,免官祿、節傳、印綬、符策,削其爵土。」乃廢平為民,徙梓潼郡。諸葛亮又與平子豐教曰:「吾與君父子戮力以獎漢室,此神明所聞,非但人知之也。表都護典漢中,委君於東關者,不與人議也。謂至心感動,終始可保,何圖中乖乎!昔楚卿屢絀,亦乃克復,思道則福,應自然之數也。願寬慰都護,勤追前闕。今雖解任,形業失故,奴婢賔客百數十人,君以中郎參軍居府,方之氣類,猶為上家。若都護思負一意,君與公琰推心從事者,否可復通,逝可復還也。詳思斯戒,明吾用心,臨書長歎,涕泣而已。」十二年,平聞亮卒,發病死。平常兾亮當自補復,策後人不能,故以激憤也。習鑿齒曰:昔管仲奪伯氏駢邑三百,沒齒而無怨言,聖人以為難。諸葛亮之使廖立垂泣,李平致死,豈徒無怨言而已哉!夫水至平而邪者取法,鏡至明而醜者無怒,水鏡之所以能窮物而無怨者,以其無私也。水鏡無私,猶以免謗,況大人君子懷樂生之心,流矜恕之德,法行於不可不用,刑加乎自犯之罪,爵之而非私,誅之而不怒,天下有不服者乎!諸葛亮於是可謂能用刑矣,自秦、漢已來未之有也。豐官至朱提太守。蘇林漢書音義曰:朱音銖;提音如北方人名匕曰提也。


\end{pinyinscope}