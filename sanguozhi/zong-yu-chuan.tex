\article{宗預傳}

\begin{pinyinscope}
宗預字德豔,南陽安衆人也。建安中,隨張飛入蜀。建興初,丞相亮以為主簿,遷參軍右中郎將。及亮卒,吳慮魏或承衰取蜀,增巴丘守兵萬人,一欲以為救援,二欲以事分割也。蜀聞之,亦益永安之守,以防非常。預將命使吳,孫權問預曰:「東之與西,譬猶一家,而聞西更增白帝之守,何也?」預對曰:「臣以為東益巴丘之戍,西增白帝之守,皆事勢宜然,俱不足以相問也。」權大笑,嘉其抗直,甚愛待之,見敬亞於鄧芝、費禕。遷為侍中,徙尚書。延熈十年,為屯騎校尉。時車騎將軍鄧芝自江州還,來朝,謂預曰:「禮,六十不服戎,而卿甫受兵,何也?」預荅曰:「卿七十不還兵,我六十何為不受邪?」

臣松之以為芝以年啁預,是不自顧。然預之此荅,觸人所忌。載之記牒,近為煩文。芝性驕慠,自大將軍費禕等皆避下之,而預獨不為屈。預復東聘吳,孫權捉預手,涕泣而別曰:「君每銜命結二國之好。今君年長,孤亦衰老,恐不復相見!」遺預大珠一斛,吳歷曰:預臨別,謂孫權曰:「蜀土僻小,雖云鄰國,東西相賴,吳不可無蜀,蜀不可無吳,君臣憑恃,唯陛下重垂神慮。」又自說「年老多病,恐不復得奉聖顏」。孫盛曰:夫帝王之保,唯道與義,道義旣建,雖小可大,殷、周是也。苟任詐力,雖彊必敗,秦、項是也。況乎居偏鄙之城,恃山水之固,而欲連橫萬里,永相資賴哉?昔九國建合從之計,而秦人卒併六合;囂、述營輔車之謀,而光武終兼隴、蜀。夫以九國之彊,隴、漢之大,莫能相救,坐觀屠覆。何者?道德之基不固,而彊弱之心難一故也。而云「吳不可無蜀,蜀不可無吳」,豈不諂哉!乃還。遷後將軍,督永安,就拜征西大將軍,賜爵關內侯。景耀元年,以疾徵還成都。後為鎮軍大將軍,領兖州刺史。時都護諸葛瞻初統朝事,廖化過預,欲與預共詣瞻許。預曰:「吾等年踰七十,所竊已過,但少一死耳,何求於年少輩而屑屑造門邪?」遂不往。

廖化字元儉,本名淳,襄陽人也。為前將軍關羽主簿,羽敗,屬吳。思歸先主,乃詐死,時人謂為信然,因携持老母晝夜西行。會先主東征,遇於秭歸。先主大恱,以化為宜都太守。先主薨,為丞相參軍,後為督廣武,稍遷至右車騎將軍,假節,領并州刺史,封中鄉侯,以果烈稱。官位與張翼齊,而在宗預之右。漢晉春秋曰:景耀五年,姜維率衆出狄道,廖化曰:「『兵不戢,必自焚』,伯約之謂也。智不出敵,而力少於寇,用之無厭,何以能立?詩云『不自我先,不自我後』,今日之事也。」

咸熈元年春,化、預俱內徙洛陽,道病卒。


\end{pinyinscope}