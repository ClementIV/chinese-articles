\article{廖立傳}

\begin{pinyinscope}
廖立

廖音理救反。字公淵,武陵臨沅人。先主領荊州牧,辟為從事,年未三十,擢為長沙太守。先主入蜀,諸葛亮鎮荊土,孫權遣使通好於亮,因問士人皆誰相經緯者,亮荅曰:「龐統、廖立,楚之良才,當贊興世業者也。」建安二十年,權遣呂蒙奄襲南三郡,立脫身走,自歸先主。先主素識待之,不深責也,以為巴郡太守。二十四年,先主為漢中王,徵立為侍中。後主襲位,徙長水校尉。

立本意,自謂才名宜為諸葛亮之貳,而更游散在李嚴等下,常懷怏怏。後丞相掾李邵、蔣琬至,立計曰:「軍當遠出,卿諸人好諦其事。昔先帝不取漢中,走與吳人爭南三郡,卒以三郡與吳人,徒勞役吏士,無益而還。旣亡漢中,使夏侯淵、張郃深入于巴,幾喪一州。後至漢中,使關侯身死無孑遺,上庸覆敗,徒失一方。是羽怙恃勇名,作軍無法,直以意突耳,故前後數喪師衆也。如向朗、文恭,凡俗之人耳。恭作治中無綱紀;朗昔奉馬良兄弟,謂為聖人,今作長史,素能合道。中郎郭演長,從人者耳,不足與經大事,而作侍中。今弱世也,欲任此三人,為不然也。王連流俗,苟作掊克,使百姓疲弊,以致今日。」邵、琬具白其言於諸葛亮。亮表立曰:「長水校尉廖立,坐自貴大,臧否羣士,公言國家不任賢達而任俗吏,又言萬人率者皆小子也;誹謗先帝,疵毀衆臣。人有言國家兵衆簡練,部伍分明者,立舉頭視屋,憤咤作色曰:『何足言!』凡如是者不可勝數。羊之亂羣,猶能為害,況立託在大位,中人以下識真偽邪?」亮集有亮表曰:「立奉先帝無忠孝之心,守長沙則開門就敵,領巴郡則有闇昧闟茸其事,隨大將軍則誹謗譏訶,侍梓宮則挾刃斷人頭於梓宮之側。陛下即位之後,普增職號,立隨比為將軍,面語臣曰:『我何宜在諸將軍中!不表我為卿,上當在五校!』臣荅:『將軍者,隨大比耳。至於卿者,正方亦未為卿也。且宜處五校。』自是之後,怏怏懷恨。」詔曰:「三苗亂政,有虞流宥,廖立狂惑,朕不忍刑,亟徙不毛之地。」於是廢立為民,徙汶山郡。立躬率妻子耕殖自守,聞諸葛亮卒,垂泣歎曰:「吾終為左袵矣!」後監軍姜維率偏軍經汶山,往詣立,稱立意氣不衰,言論自若。立遂終於徙所。妻子還蜀。


\end{pinyinscope}