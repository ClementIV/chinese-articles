\article{si-ma-zhi-chuan}

\begin{pinyinscope}
司馬芝字子華,河內溫人也。少為書生,避亂荊州,於魯陽山遇賊,同行者皆棄老弱走,芝獨坐守老母。賊至,以刃臨芝,芝叩頭曰:「母老,唯在諸君!」賊曰:「此孝子也,殺之不義。」遂得免害,以鹿車推載母。居南方十餘年,躬耕守節。

太祖平荊州,以芝為菅長。時天下草創,多不奉法。郡主簿劉節,舊族豪俠,賔客千餘家,出為盜賊,入亂吏治。頃之,芝差節客王同等為兵,掾史據白:「節家前後未甞給繇,若至時藏匿,必為留負。」芝不聽,與節書曰:「君為大宗,加股肱郡,而賔客每不與役,旣衆庶怨望,或流聲上聞。今條同等為兵,幸時發遣。」兵已集郡,而節藏同等,因令督郵以軍興詭責縣,縣掾史窮困,乞代同行。芝乃馳檄濟南,具陳節罪。太守郝光素敬信芝,即以節代同行,青州號芝「以郡主簿為兵」。遷廣平令。征虜將軍劉勳,貴寵驕豪,又芝故郡將,賔客子弟在界數犯法。勳與芝書,不著姓名,而多所屬託,芝不報其書,一皆如法。後勳以不軌誅,交關者皆獲罪,而芝以見稱。

魏略曰:勳字子臺,琅邪人。中平末,為沛國建平長,與太祖有舊。後為廬江太守,為孫策所破,自歸太祖,封列侯,遂從在散伍議中。勳兄為豫州刺史,病亡。兄子威,又代從政。勳自恃與太祖有宿,日驕慢,數犯法,又誹謗。為李申成所白,收治,并免威官。

遷大理正。有盜官練置都厠上者,吏疑女工,收以付獄。芝曰:「夫刑罪之失,失在苛暴。今贓物先得而後訊其辭,若不勝掠,或至誣服。誣服之情,不可以折獄。且簡而易從,大人之化也。不失有罪,庸世之治耳。今宥所疑,以隆易從之義,不亦可乎!」太祖從其議。歷甘陵、沛、陽平太守,所在有績。黃初中,入為河南尹,抑彊扶弱,私請不行。會內官欲以事託芝,不敢發言,因芝妻伯父董昭。昭猶憚芝,不為通。芝為教與羣下曰:「蓋君能設教,不能使吏必不犯也。吏能犯教,而不能使君必不聞也。夫設教而犯,君之劣也;犯教而聞,吏之禍也。君劣於上,吏禍於下,此政事所以不理也。可不各勉之哉!」於是下吏莫不自勵。門下循行甞疑門幹盜簪,幹辭不符,曹執為獄。芝教曰:「凡物有相似而難分者,自非離婁,鮮能不惑。就其實然,循行何忍重惜一簪,輕傷同類乎!其寢勿問。」

明帝即位,賜爵關內侯。頃之,特進曹洪乳母當,與臨汾公主侍者共事無澗神繫獄。臣松之案:無澗,山名,在洛陽東北。卞太后遣黃門詣府傳令,芝不通,輙勑洛陽獄考竟,而上疏曰:「諸應死罪者,皆當先表須報。前制書禁絕淫祀以正風俗,今當等所犯妖刑,辭語始定,黃門吳達詣臣,傳太皇太后令。臣不敢通,懼有救護,速聞聖聽,若不得已,以垂宿留。由事不早竟,是臣之罪,是以冒犯常科,輙勑縣考竟,擅行刑戮,伏須誅罰。」帝手報曰:「省表,明卿至心,欲奉詔書,以權行事,是也。此乃卿奉詔之意,何謝之有?後黃門復往,慎勿通也。」芝居官十一年,數議科條所不便者。其在公卿間,直道而行。會諸王來朝,與京都人交通,坐免。

後為大司農。先是諸典農各部吏民,末作治生,以要利入。芝奏曰:「王者之治,崇本抑末,務農重穀。王制:『無三年之儲,國非其國也。』管子區言以積穀為急。方今二虜未滅,師旅不息,國家之事,唯在穀帛。武皇帝特開屯田之官,專以農桑為業。建安中,天下倉廩充實,百姓殷足。自黃初以來,聽諸典農治生,各為部下之計,誠非國家大體所宜也。夫王者以海內為家,故傳曰:『百姓不足,君誰與足!』富足之田,在於不失天時而盡地力。今商旅所求,雖有加倍之顯利,然於一統之計,已有不貲之損,不如墾田益一畒之收也。夫農民之事田,自正月耕種,芸鋤條桑,耕熯種麥,穫刈築場,十月乃畢。治廩繫橋,運輸租賦,除道理梁,墐塗室屋,以是終歲,無日不為農事也。今諸典農,各言『留者為行者宗田計,課其力,勢不得不爾。不有所廢,則當素有餘力。』臣愚以為不宜復以商事雜亂,專以農桑為務,於國計為便。」明帝從之。

每上官有所召問,常先見掾史,為斷其意故,教其所以荅塞之狀,皆如所度。芝性亮直,不矜廉隅。與賔客談論,有不可意,便靣折其短,退無異言。卒於官,家無餘財,自魏迄今為河南尹者莫及芝。

芝亡,子岐嗣,從河南丞轉廷尉正,遷陳留相。梁郡有繫囚,多所連及,數歲不決。詔書徙獄於岐屬縣,縣請豫治牢具。岐曰:「今囚有數十,旣巧詐難符,且已倦楚毒,其情易見。豈當復乆處囹圄邪!」及囚室,詰之,皆莫敢匿詐,一朝決竟,遂超為廷尉。是時大將軍爽專權,尚書何晏、鄧颺等為之輔翼。南陽圭泰甞以言迕指,考繫廷尉。颺訊獄,將致泰重刑。岐數颺曰:「夫樞機大臣,王室之佐,旣不能輔化成德,齊美古人,而乃肆其私忿,枉論無辜。使百姓危心,非此焉在?」颺於是慙怒而退。岐終恐乆獲罪,以疾去官。居家未朞而卒,年三十五。子肇嗣。肇,晉太康中為兾州刺史、尚書,見百官志。

評曰:徐弈、何夔、邢顒貴尚峻厲,為世名人。毛玠清公素履,司馬芝忠亮不傾,庶乎不吐剛茹柔。崔琰高格最優,鮑勛秉正無虧,而皆不免其身,惜哉!大雅貴「旣明且哲」,虞書尚「直而能溫」,自非兼才,疇克備諸!


\end{pinyinscope}