\article{周宣傳}

\begin{pinyinscope}
周宣字孔和,樂安人也。為郡吏。太守楊沛夢人曰:「八月一日曹公當至,必與君杖,飲以藥酒。」使宣占之。是時黃巾賊起,宣對曰:「夫杖起弱者,藥治人病,八月一日,賊必除滅。」至期,賊果破。

後東平劉楨夢蛇生四足,穴居門中,使宣占之,宣曰:「此為國夢,非君家之事也。當殺女子而作賊者。」頃之,女賊鄭、姜遂俱夷討,以虵女子之祥,足非虵之所宜故也。

文帝問宣曰:「吾夢殿屋兩瓦墮地,化為雙鴛鴦,此何謂也?」宣對曰:「後宮當有暴死者。」帝曰:「吾詐卿耳!」宣對曰:「夫夢者意耳,苟以形言,便占吉凶。」言未畢,而黃門令奏宮人相殺。無幾,帝復問曰:「我昨夜夢青氣自地屬天。」宣對曰:「天下當有貴女子寃死。」是時,帝已遣使賜甄后璽書,聞宣言而悔之,遣人追使者不及。帝復問曰:「吾夢摩錢文,欲令滅而更愈明,此何謂邪?」宣悵然不對。帝重問之,宣對曰:「此自陛下家事,雖意欲爾而太后不聽,是以文欲滅而明耳。」時帝欲治弟植之罪,偪於太后,但加貶爵。以宣為中郎,屬太史。

嘗有問宣曰:「吾昨夜夢見芻狗,其占何也?」宣荅曰:「君欲得美食耳!」有頃,出行,果遇豐膳。後又問宣曰:「昨夜復夢見芻狗,何也?」宣曰:「君欲墮車折脚,宜戒慎之。」頃之,果如宣言。後又問宣:「昨夜復夢見芻狗,何也?」宣曰:「君家欲失火,當善護之。」俄遂火起。語宣曰:「前後三時,皆不夢也。聊試君耳,何以皆驗邪?」宣對曰:「此神靈動君使言,故與真夢無異也。」又問宣曰:「三夢芻狗而其占不同,何也?」宣曰:「芻狗者,祭神之物。故君始夢,當得餘食也。祭祀旣訖,則芻狗為車所轢,故中夢當墮車折脚也。芻狗旣車轢之後,必載以為樵,故後夢憂失火也。」宣之叙夢,凡此類也。十中八九,世以比建平之相矣。其餘效故不次列。明帝末卒。


\end{pinyinscope}