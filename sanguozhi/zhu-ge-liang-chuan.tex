\article{諸葛亮傳}

\begin{pinyinscope}
諸葛亮字孔明,琅邪陽都人也。漢司隷校尉諸葛豐後也。父珪,字君貢,漢末為太山郡丞。亮早孤,從父玄為袁術所署豫章太守,玄將亮及亮弟均之官。會漢朝更選朱皓代玄。玄素與荊州牧劉表有舊,往依之。

獻帝春秋曰:初,豫章太守周術病卒,劉表上諸葛玄為豫章太守,治南昌。漢朝聞周術死,遣朱皓代玄。皓從揚州刺史劉繇求兵擊玄,玄退屯西城,皓入南昌。建安二年正月,西城民反,殺玄,送首詣繇。此書所云,與本傳不同。玄卒,亮躬耕隴畒,好為梁父吟。漢晉春秋曰:亮家于南陽之鄧縣,在襄陽城西二十里,號曰隆中。身長八尺,每自比於管仲、樂毅,時人莫之許也。惟博陵崔州平、潁川徐庶元直與亮友善,謂為信然。案崔氏譜:州平,太尉烈子,均之弟也。魏略曰:亮在荊州,以建安初與潁川石廣元、徐元直、汝南孟公威等俱游學,三人務於精熟,而亮獨觀其大略。每晨夜從容,常抱膝長嘯,而謂三人曰:「卿諸人仕進可至郡守刺史也。」三人問其所志,亮但笑而不言。後公威思鄉里,欲北歸,亮謂之曰:「中國饒士大夫,遨游何必故鄉邪!」臣松之以為魏略此言,謂諸葛亮為公威計者可也,若謂兼為己言,可謂未達其心矣。老氏稱知人者智,自知者明,凡在賢達之流,固必兼而有焉。以諸葛之鑒識,豈不能自審其分乎?夫其高吟俟時,情見乎言,志氣所存,旣已定於其始矣。若使游步中華,騁其龍光,豈夫多士所能沈翳哉!委質魏氏,展其器能,誠非陳長文、司馬仲達所能頡頏,而況於餘哉!苟不患功業不就,道之不行,雖志恢宇宙而終不北向者,蓋以權御已移,漢祚將傾,方將翊贊宗傑,以興微繼絕克復為己任故也。豈其區區利在邊鄙而已乎!此相如所謂「鵾鵬已翔於遼廓,而羅者猶視於藪澤」者矣。公威名建,在魏亦貴達。

時先主屯新野。徐庶見先主,先主器之,謂先主曰:「諸葛孔明者,卧龍也,將軍豈願見之乎?」襄陽記曰:劉備訪世事於司馬德操。德操曰:「儒生俗士,豈識時務?識時務者在乎俊傑。此間自有伏龍、鳳雛。」備問為誰,曰:「諸葛孔明、龐士元也。」先主曰:「君與俱來。」庶曰:「此人可就見,不可屈致也。將軍宜枉駕顧之。」由是先主遂詣亮,凡三往,乃見。因屏人曰:「漢室傾頹,姧臣竊命,主上蒙塵。孤不度德量力,欲信大義於天下,而智術淺短,遂用猖獗,至于今日。然志猶未已,君謂計將安出?」亮荅曰:「自董卓已來,豪傑並起,跨州連郡者不可勝數。曹操比於袁紹,則名微而衆寡,然操遂能克紹,以弱為彊者,非惟天時,抑亦人謀也。今操已擁百萬之衆,挾天子而令諸侯,此誠不可與爭鋒。孫權據有江東,已歷三世,國險而民附,賢能為之用,此可與為援而不可圖也。荊州北據漢、沔,利盡南海,東連吳會,西通巴、蜀,此用武之國,而其主不能守,此殆天所以資將軍,將軍豈有意乎?益州險塞,沃野千里,天府之土,高祖因之以成帝業。劉璋闇弱,張魯在北,民殷國富而不知存卹,智能之士思得明君。將軍旣帝室之冑,信義著於四海,總攬英雄,思賢如渴,若跨有荊、益,保其巖阻,西和諸戎,南撫夷越,外結好孫權,內脩政理;天下有變,則命一上將將荊州之軍以向宛、洛,將軍身率益州之衆出於秦川,百姓孰敢不簞食壺漿以迎將軍者乎?誠如是,則霸業可成,漢室可興矣。」先主曰:「善!」於是與亮情好日密。關羽、張飛等不恱,先主解之曰:「孤之有孔明,猶魚之有水也。願諸君勿復言。」羽、飛乃止。魏略曰:劉備屯於樊城。是時曹公方定河北,亮知荊州次當受敵,而劉表性緩,不曉軍事。亮乃北行見備,備與亮非舊,又以其年少,以諸生意待之。坐集旣畢,衆賔皆去,而亮獨留,備亦不問其所欲言。備性好結毦,時適有人以髦牛尾與備者,備因手自結之。亮乃進曰:「明將軍當復有遠志,但結毦而已邪!」備知亮非常人也,乃投毦而荅曰:「是何言與!我聊以忘憂爾。」亮遂言曰:「將軍度劉鎮南孰與曹公邪?」備曰:「不及。」亮又曰:「將軍自度何如也?」備曰:「亦不如。」曰:「今皆不及,而將軍之衆不過數千人,以此待敵,得無非計乎!」備曰:「我亦愁之,當若之何?」亮曰:「今荊州非少人也,而著籍者寡,平居發調,則人心不恱;可語鎮南,令國中凡有游戶,皆使自實,因錄以益衆可也。」備從其計,故衆遂彊。備由此知亮有英略,乃以上客禮之。九州春秋所言亦如之。臣松之以為亮表云「先帝不以臣卑鄙,猥自枉屈,三顧臣於草廬之中,諮臣以當世之事」,則非亮先詣備,明矣。雖聞見異辭,各生彼此,然乖背至是,亦良為可怪。

劉表長子琦,亦深器亮。表受後妻之言,愛少子琮,不恱於琦。琦每欲與亮謀自安之術,亮輒拒塞,未與處畫。琦乃將亮游觀後園,共上高樓,飲宴之間,令人去梯,因謂亮曰:「今日上不至天,下不至地,言出子口,入於吾耳,可以言未?」亮荅曰:「君不見申生在內而危,重耳在外而安乎?」琦意感寤,陰規出計。會黃祖死,得出,遂為江夏太守。俄而表卒,琮聞曹公來征,遣使請降。先主在樊聞之,率其衆南行,亮與徐庶並從,為曹公所追破,獲庶母。庶辭先主而指其心曰:「本欲與將軍共圖王霸之業者,以此方寸之地也。今已失老母,方寸亂矣,無益於事,請從此別。」遂詣曹公。魏略曰:庶先名福,本單家子,少好任俠擊劒。中平末,甞為人報讎,白堊突靣,被髮而走,為吏所得,問其姓字,閉口不言。吏乃於車上立柱維磔之,擊鼓以令於市鄽,莫敢識者,而其黨伍共篡解之,得脫。於是感激,棄其刀戟,更踈巾單衣,折節學問。始詣精舍,諸生聞其前作賊,不肯與共止。福乃卑躬早起,常獨掃除,動靜先意,聽習經業,義理精孰。遂與同郡石韜相親愛。初平中,中州兵起,乃與韜南客荊州,到,又與諸葛亮特相善。及荊州內附,孔明與劉備相隨去,福與韜俱來北。至黃初中,韜仕歷郡守、典農校尉,福至右中郎將、御史中丞。逮大和中,諸葛亮出隴右,聞元直、廣元仕財如此,嘆曰:「魏殊多士邪!何彼二人不見用乎?」庶後數年病卒,有碑在彭城,今猶存焉。

先主至于夏口,亮曰:「事急矣,請奉命求救於孫將軍。」時權擁軍在柴桑,觀望成敗,亮說權曰:「海內大亂,將軍起兵據有江東,劉豫州亦收衆漢南,與曹操並爭天下。今操芟夷大難,略已平矣,遂破荊州,威震四海。英雄無所用武,故豫州遁逃至此。將軍量力而處之:若能以吳、越之衆與中國抗衡,不如早與之絕;若不能當,何不案兵束甲,北面而事之!今將軍外託服從之名,而內懷猶豫之計,事急而不斷,禍至無日矣!」權曰:「苟如君言,劉豫州何不遂事之乎?」亮曰:「田橫,齊之壯士耳,猶守義不辱,况劉豫州王室之冑,英才蓋世,衆士慕仰,若水之歸海,若事之不濟,此乃天也,安能復為之下乎!」權勃然曰:「吾不能舉全吳之地,十萬之衆,受制於人。吾計決矣!非劉豫州莫可以當曹操者,然豫州新敗之後,安能抗此難乎?」亮曰:「豫州軍雖敗於長阪,今戰士還者及關羽水軍精甲萬人,劉琦合江夏戰士亦不下萬人。曹操之衆遠來疲弊,聞追豫州,輕騎一日一夜行三百餘里,此所謂『彊弩之末,勢不能穿魯縞』者也。故兵法忌之,曰『必蹶上將軍』。且北方之人,不習水戰;又荊州之民附操者,偪兵勢耳,非心服也。今將軍誠能命猛將統兵數萬,與豫州協規同力,破操軍必矣。操軍破,必北還,如此則荊、吳之勢彊,鼎足之形成矣。成敗之機,在於今日。」權大恱,即遣周瑜、程普、魯肅等水軍三萬,隨亮詣先主,并力拒曹公。袁子曰:張子布薦亮於孫權,亮不肯留。人問其故,曰:「孫將軍可謂人主,然觀其度,能賢亮而不能盡亮,吾是以不留。」臣松之以為袁孝尼著文立論,甚重諸葛之為人,至如此言則失之殊遠。觀亮君臣相遇,可謂希世一時,終始以分,誰能間之?寧有中違斷金,甫懷擇主,設使權盡其量,便當翻然去就乎?葛生行己,豈其然哉!關羽為曹公所獲,遇之甚厚,可謂能盡其用矣,猶義不背本,曾謂孔明之不若雲長乎!曹公敗于赤壁,引軍歸鄴。先主遂收江南,以亮為軍師中郎將,使督零陵、桂陽、長沙三郡,調其賦稅,以充軍實。零陵先賢傳云:亮時住臨烝。

建安十六年,益州牧劉璋遣法正迎先主,使擊張魯。亮與關羽鎮荊州。先主自葭萌還攻璋,亮與張飛、趙雲等率衆泝江,分定郡縣,與先主共圍成都。成都平,以亮為軍師將軍,署左將軍府事。先主外出,亮常鎮守成都,足食足兵。二十六年,羣下勸先主稱尊號,先主未許,亮說曰:「昔吳漢、耿弇等初勸世祖即帝位,世祖辭讓,前後數四,耿純進言曰:『天下英雄喁喁,兾有所望。如不從議者,士大夫各歸求主,無為從公也。』世祖感純言深至,遂然諾之。今曹氏篡漢,天下無主,大王劉氏苗族,紹世而起,今即帝位,乃其宜也。士大夫隨大王乆勤苦者,亦欲望尺寸之功如純言耳。」先主於是即帝位,策亮為丞相曰:「朕遭家不造,奉承大統,兢兢業業,不敢康寧,思盡百姓,懼未能綏。於戲!丞相亮其悉朕意,無怠輔朕之闕,助宣重光,以照明天下,君其勗哉!」亮以丞相錄尚書事,假節。張飛卒後,領司隷校尉。蜀記曰:晉初,扶風王駿鎮關中,司馬高平劉寶、長史熒陽桓隰諸官屬士大夫共論諸葛亮,于時譚者多譏亮託身非所,勞困蜀民,力小謀大,不能度德量力。金城郭冲以為亮權智英略,有踰管、晏,功業未濟,論者惑焉,條亮五事隱沒不聞於世者,寶等亦不能復難。扶風王慨然善冲之言。臣松之以為亮之異美,誠所願聞,然冲之所說,實皆可疑,謹隨事難之如左:其一事曰:亮刑法峻急,刻剥百姓,自君子小人咸懷怨歎,法正諫曰:「昔高祖入關,約法三章,秦民知德,今君假借威力,跨據一州,初有其國,未垂惠撫;且客主之義,宜相降下,願緩刑弛禁,以慰其望。」亮荅曰;「君知其一,未知其二。秦以無道,政苛民怨,匹夫大呼,天下土崩,高祖因之,可以弘濟。劉璋闇弱,自焉已來有累世之恩,文法羈縻,互相承奉,德政不舉,威刑不肅。蜀土人士,專權自恣,君臣之道,漸以陵替;寵之以位,位極則賤,順之以恩,恩竭則慢。所以致弊,實由於此。吾今威之以法,法行則知恩,限之以爵,爵加則知榮;榮恩並濟,上下有節。為治之要,於斯而著。」難曰:案法正在劉主前死,今稱法正諫,則劉主在也。諸葛職為股肱,事歸元首,劉主之世,亮又未領益州,慶賞刑政不出於己。尋沖所述亮荅,專自有其能,有違人臣自處之宜。以亮謙順之體,殆必不然。又云亮刑法峻急,刻剥百姓,未聞善政以刻剥為稱。其二事曰:曹公遣刺客見劉備,方得交接,開論伐魏形勢,甚合備計。稍欲親近,刺者尚未得便會,旣而亮入,魏客神色失措。亮因而察之,亦知非常人。須臾,客如厠,備謂亮曰;「向得奇士,足以助君補益。」亮問所在,備曰:「起者其人也。」亮徐歎曰:「觀客色動而神懼,視低而忤數,姦形外漏,邪心內藏,必曹氏刺客也。」追之,已越墻而走。難曰:凡為刺客,皆暴虎馮河,死而無悔者也。劉主有知人之鑒,而惑於此客,則此客必一時之奇士也。又語諸葛云「足以助君補益」,則亦諸葛之流亞也。凡如諸葛之儔,鮮有為人作刺客者矣,時主亦當惜其器用,必不投之死地也。且此人不死,要應顯達為魏,竟是誰乎?何其寂蔑而無聞!

章武三年春,先主於永安病篤,召亮於成都,屬以後事,謂亮曰:「君才十倍曹丕,必能安國,終定大事。若嗣子可輔,輔之;如其不才,君可自取。」亮涕泣曰:「臣敢竭股肱之力,效忠貞之節,繼之以死!」先主又為詔勑後主曰:「汝與丞相從事,事之如父。」孫盛曰:夫杖道扶義,體存信順,然後能匡主濟功,終定大業。語曰弈者舉棊不定猶不勝其偶,況量君之才否而二三其節,可以摧服彊鄰囊括四海者乎?備之命亮,亂孰甚焉!世或有謂備欲以固委付之誠,且以一蜀人之志。君子曰,不然;苟所寄忠賢,則不須若斯之誨,如非其人,不宜啟篡逆之塗。是以古之顧命,必貽話言;詭偽之辭,非託孤之謂。幸值劉禪闇弱,無猜險之性,諸葛威略,足以檢衞異端,故使異同之心無由自起耳。不然,殆生疑隙不逞之釁。謂之為權,不亦惑哉!建興元年,封亮武鄉侯,開府治事。頃之,又領益州牧。政事無巨細,咸決於亮。南中諸郡,並皆叛亂,亮以新遭大喪,故未便加兵,且遣使聘吳,因結和親,遂為與國。亮集曰:是歲,魏司徒華歆、司空王朗、尚書令陳羣、太史令許芝、謁者僕射諸葛璋各有書與亮,陳天命人事,欲使舉國稱藩。亮遂不報書,作正議曰:「昔在項羽,起不由德,雖處華夏,秉帝者之勢,卒就湯鑊,為後永戒。魏不審鑒,今次之矣;免身為幸,戒在子孫。而二三子各以耆艾之齒,承偽指而進書,有若崇、竦稱莽之功,亦將偪于元禍苟免者邪!昔世祖之創迹舊基,奮羸卒數千,摧莽彊旅四十餘萬於昆陽之郊。夫據道討淫,不在衆寡。及至孟德,以其譎勝之力,舉數十萬之師,救張郃於陽平,勢窮慮悔,僅能自脫,辱其鋒銳之衆,遂喪漢中之地,深知神器不可妄獲,旋還未至,感毒而死。子桓淫逸,繼之以篡。縱使二三子多逞蘇、張詭靡之說,奉進驩兜滔天之辭,欲以誣毀唐帝,諷解禹、稷,所謂徒喪文藻煩勞翰墨者矣。夫大人君子之所不為也。又軍誡曰:『萬人必死,橫行天下。』昔軒轅氏整卒數萬,制四方,定海內,況以數十萬之衆,據正道而臨有罪,可得干擬者哉!」

三年春,亮率衆南征,詔賜亮金鈇鉞一具,曲蓋一,前後羽葆鼓吹各一部,虎賁六十人。事在亮集。其秋悉平。軍資所出,國以富饒,漢晉春秋曰:亮至南中,所在戰捷。聞孟獲者,為夷、漢所服,募生致之。旣得,使觀於營陣之間,曰:「此軍何如?」獲對曰:「向者不知虛實,故敗。今蒙賜觀看營陣,若祇如此,即定易勝耳。」亮笑,縱使更戰,七縱七禽,而亮猶遣獲。獲止不去,曰:「公,天威也,南人不復反矣。」遂至滇池。南中平,皆即其渠率而用之。或以諫亮,亮曰:「若留外人,則當留兵,兵留則無所食,一不易也;加夷新傷破,父兄死喪,留外人而無兵者,必成禍患,二不易也;又夷累有廢殺之罪,自嫌釁重,若留外人,終不相信,三不易也;今吾欲使不留兵,不運糧,而綱紀粗定,夷、漢粗安故耳。」乃治戎講武,以俟大舉。五年,率諸軍北駐漢中,臨發,上疏曰:

先帝創業未半,而中道崩殂,今天下三分,益州疲弊,此誠危急存亡之秋也。然侍衞之臣不懈於內,忠志之士忘身於外者,蓋追先帝之殊遇,欲報之於陛下也。誠宜開張聖聽,以光先帝遺德,恢弘志士之氣,不宜妄自菲薄,引喻失義,以塞忠諫之路也。宮中府中俱為一體,陟罰臧否,不宜異同。若有作姧犯科及為忠善者,宜付有司論其刑賞,以昭陛下平明之理,不宜偏私,使內外異法也。侍中、侍郎郭攸之、費禕、董允等,此皆良實,志慮忠純,是以先帝簡拔以遺陛下。愚以為宮中之事,事無大小,悉以咨之,然後施行,必能裨補闕漏,有所廣益。將軍向寵,性行淑均,曉暢軍事,試用於昔日,先帝稱之曰能,是以衆議舉寵為督。愚以為營中之事,悉以咨之,必能使行陣和睦,優劣得所。親賢臣,遠小人,此先漢所以興隆也;親小人,遠賢臣,此後漢所以傾頹也。先帝在時,每與臣論此事,未甞不歎息痛恨於桓、靈也。侍中、尚書、長史、參軍,此悉貞良死節之臣,願陛下親之信之,則漢室之隆,可計日而待也。

臣本布衣,躬耕於南陽,苟全性命於亂世,不求聞達於諸侯。先帝不以臣卑鄙,猥自枉屈,三顧臣於草廬之中,諮臣以當世之事,由是感激,遂許先帝以驅馳。後值傾覆,受任於敗軍之際,奉命於危難之間,爾來二十有一年矣。臣松之案:劉備以建安十三年敗,遣亮使吳,亮以建興五年抗表北伐,自傾覆至此整二十年。然則備始與亮相遇,在敗軍之前一年時也。先帝知臣謹慎,故臨崩寄臣以大事也。

受命以來,夙夜憂歎,恐託付不效,以傷先帝之明,故五月渡瀘,深入不毛。漢書地理志曰:瀘惟水出牂牁郡句町縣。今南方已定,兵甲已足,當獎率三軍,北定中原,庶竭駑鈍,攘除姧凶,興復漢室,還于舊都。此臣所以報先帝,而忠陛下之職分也。至於斟酌損益,進盡忠言,則攸之、禕、允之任也。願陛下託臣以討賊興復之效;不效,則治臣之罪,以告先帝之靈;責攸之、禕、允等之慢,以彰其咎。陛下亦宜自謀,以諮諏善道,察納雅言,深追先帝遺詔。臣不勝受恩感激,今當遠離,臨表涕零,不知所言。

遂行,屯于沔陽。郭沖三事曰:亮屯于陽平,遣魏延諸軍并兵東下,亮惟留萬人守城。晉宣帝率二十萬衆拒亮,而與延軍錯道,徑至前,當亮六十里所,偵候白宣帝說亮在城中兵少力弱。亮亦知宣帝垂至,已與相偪,欲前赴延軍,相去又遠,回迹反追,勢不相及,將士失色,莫知其計。亮意氣自若,勑軍中皆卧旗息鼓,不得妄出菴幔,又令大開四城門,埽地却洒。宣帝常謂亮持重,而猥見勢弱,疑其有伏兵,於是引軍北趣山。明日食時,亮謂參佐拊手大笑曰:「司馬懿必謂吾怯,將有彊伏,循山走矣。」候邏還白,如亮所言。宣帝後知,深以為恨。難曰:案陽平在漢中。亮初屯陽平,宣帝尚為荊州都督,鎮宛城,至曹真死後,始與亮於關中相抗禦耳。魏甞遣宣帝自宛由西城伐蜀,值霖雨,不果。此之前後,無復有於陽平交兵事。就如沖言,宣帝旣舉二十萬衆,已知亮兵少力弱,若疑其有伏兵,正可設防持重,何至便走乎?案魏延傳云:「延每隨亮出,輒欲請精兵萬人,與亮異道會于潼關,亮制而不許;延常謂亮為怯,歎己才用之不盡也。」亮尚不以延為萬人別統,豈得如沖言,頓使將重兵在前,而以輕弱自守乎?且沖與扶風王言,顯彰宣帝之短,對子毀父,理所不容,而云「扶風王慨然善沖之言」,故知此書舉引皆虛。

六年春,揚聲由斜谷道取郿,使趙雲、鄧芝為疑軍,據箕谷,魏大將軍曹真舉衆拒之。亮身率諸軍攻祁山,戎陣整齊,賞罰肅而號令明,南安、天水、安定三郡叛魏應亮,關中響震。魏略曰:始,國家以蜀中惟有劉備。備旣死,數歲寂然無聞,是以略無備預;而卒聞亮出,朝野恐懼,隴右、祁山尤甚,故三郡同時應亮。魏明帝西鎮長安,命張郃拒亮,亮使馬謖督諸軍在前,與郃戰于街亭。謖違亮節度,舉動失宜,大為郃所破。亮拔西縣千餘家,還于漢中,郭沖四事曰:亮出祁山,隴西、南安二郡應時降,圍天水,拔兾城,虜姜維,驅略士女數千人還蜀。人皆賀亮,亮顏色愀然有戚容,謝曰:「普天之下,莫非漢民,國家威力未舉,使百姓困於犲狼之吻。一夫有死,皆亮之罪,以此相賀,能不為愧。」於是蜀人咸知亮有吞魏之志,非惟拓境而已。難曰:亮有吞魏之志乆矣,不始於此衆人方知也,且于時師出無成,傷缺而反者衆,三郡歸降而不能有。姜維,天水之匹夫耳,獲之則於魏何損?拔西縣千家,不補街亭所喪,以何為功,而蜀人相賀乎?戮謖以謝衆。上疏曰:「臣以弱才,叨竊非據,親秉旄鉞以厲三軍,不能訓章明法,臨事而懼,至有街亭違命之闕,箕谷不戒之失,咎皆在臣授任無方。臣明不知人,恤事多闇,春秋責帥,臣職是當。請自貶三等,以督厥咎。」於是以亮為右將軍,行丞相事,所總統如前。漢晉春秋曰:或勸亮更發兵者,亮曰:「大軍在祁山、箕谷,皆多於賊,而不能破賊為賊所破者,則此病不在兵少也,在一人耳。今欲減兵省將,明罰思過,校變通之道於將來;若不能然者,雖兵多何益!自今已後,諸有忠慮於國,但勤攻吾之闕,則事可定,賊可死,功可蹻足而待矣。」於是考微勞,甄烈壯,引咎責躬,布所失於天下,厲兵講武,以為後圖,戎士簡練,民忘其敗矣。亮聞孫權破曹休,魏兵東下,關中虛弱。十一月,上言曰:「先帝慮漢賊不兩立,王業不偏安,故託臣以討賊也。以先帝之明,量臣之才,故知臣伐賊才弱敵彊也;然不伐賊,王業亦亡,惟坐待亡,孰與伐之?是故託臣而弗疑也。臣受命之日,寢不安席,食不甘味,思惟北征,宜先入南,故五月渡瀘,深入不毛,并日而食。臣非不自惜也,顧王業不得偏全於蜀都,故冒危難以奉先帝之遺意也,而議者謂為非計。今賊適疲於西,又務於東,兵法乘勞,此進趨之時也。謹陳其事如左:高帝明並日月,謀臣淵深,然涉險被創,危然後安。今陛下未及高帝,謀臣不如良、平,而欲以長計取勝,坐定天下,此臣之未解一也。劉繇、王朗各據州郡,論安言計,動引聖人,羣疑滿腹,衆難塞胷,今歲不戰,明年不征,使孫策坐大,遂并江東,此臣之未解二也。曹操智計殊絕於人,其用兵也,髣髴孫、吳,然困於南陽,險於烏巢,危於祁連,偪於黎陽,幾敗伯山,殆死潼關,然後偽定一時耳,況臣才弱,而欲以不危而定之,此臣之未解三也。曹操五攻昌霸不下,四越巢湖不成,任用李服而李服圖之,委夏侯而夏侯敗亡,先帝每稱操為能,猶有此失,況臣駑下,何能必勝?此臣之未解四也。自臣到漢中,中間朞年耳,然喪趙雲、陽羣、馬玉、閻芝、丁立、白壽、劉郃、鄧銅等及曲長屯將七十餘人,突將無前。賨、叟、青羌散騎、武騎一千餘人,此皆數十年之內所糾合四方之精銳,非一州之所有,若復數年,則損三分之二也,當何以圖敵?此臣之未解五也。今民窮兵疲,而事不可息,事不可息,則住與行勞費正等,而不及虛圖之,欲以一州之地與賊持乆,此臣之未解六也。夫難平者,事也。昔先帝敗軍於楚,當此時,曹操拊手,謂天下以定。然後先帝東連吳、越,西取巴、蜀,舉兵北征,夏侯授首,此操之失計而漢事將成也。然後吳更違盟,關羽毀敗,秭歸蹉跌,曹丕稱帝。凡事如是,難可逆見。臣鞠躬盡力,死而後已,至於成敗利鈍,非臣之明所能逆覩也。」於是有散關之役。此表,亮集所無,出張儼默記。

冬,亮復出散關,圍陳倉,曹真拒之,亮糧盡而還。魏將王雙率騎追亮,亮與戰,破之,斬雙。七年,亮遣陳戒攻武都、陰平。魏雍州刺史郭淮率衆欲擊戒,亮自出至建威,淮退還,遂平二郡。詔策亮曰:「街亭之役,咎由馬謖,而君引愆,深自貶抑,重違君意,聽順所守。前年耀師,馘斬王雙;今歲爰征,郭淮遁走;降集氐、羌,興復二郡,威鎮凶暴,功勳顯然。方今天下騷擾,元惡未梟,君受大任,幹國之重,而乆自挹損,非所以光揚洪烈矣。今復君丞相,君其勿辭。」漢晉春秋曰:是歲,孫權稱尊號,其羣臣以並尊二帝來告。議者咸以為交之无益,而名體弗順,宜顯明正義,絕其盟好。亮曰:「權有僭逆之心乆矣,國家所以略其釁情者,求掎角之援也。今若加顯絕,讎我必深,便當移兵東戍,與之角力,須并其土,乃議中原。彼賢才尚多,將相緝穆,未可一朝定也。頓兵相持,坐而須老,使北賊得計,非筭之上者。昔孝文卑辭匈奴,先帝優與吳盟,皆應權通變,弘思遠益,非匹夫之為分者也。今議者咸以權利在鼎足,不能并力,且志望以滿,无上岸之情,推此,皆似是而非也。何者?其智力不侔,故限江自保;權之不能越江,猶魏賊之不能渡漢,非力有餘而利不取也。若大軍致討,彼高當分裂其地以為後規,下當略民廣境,示武於內,非端坐者也。若就其不動而睦於我,我之北伐,无東顧之憂,河南之衆不得盡西,此之為利,亦已深矣。權僭之罪,未宜明也。」乃遣衞尉陳震慶權正號。

九年,亮復出祁山,以木牛運,漢晉春秋曰:亮圍祁山,招鮮卑軻比能,比能等至故北地石城以應亮。於是魏大司馬曹真有疾,司馬宣王自荊州入朝,魏明帝曰:「西方事重,非君莫可付者。」乃使西屯長安,督張郃、費曜、戴陵、郭淮等。宣王使曜、陵留精兵四千守上邽,餘衆悉出,西救祁山。郃欲分兵駐雍、郿,宣王曰:「料前軍能獨當之者,將軍言是也;若不能當而分為前後,此楚之三軍所以為黥布禽也。」遂進。亮分兵留攻,自逆宣王于上邽。郭淮、費曜等徼亮,亮破之,因大芟刈其麥,與宣王遇于上邽之東,斂兵依險,軍不得交,亮引而還。宣王尋亮至于鹵城。張郃曰:「彼遠來逆我,請戰不得,謂我利在不戰,欲以長計制之也。且祁山知大軍以在近,人情自固,可止屯於此,分為奇兵,示出其後,不宜進前而不敢偪,坐失民望也。今亮縣軍食少,亦行去矣。」宣王不從,故尋亮。旣至,又登山掘營,不肯戰。賈栩、魏平數請戰,因曰:「公畏蜀如虎,柰天下笑何!」宣王病之。諸將咸請戰。五月辛巳,乃使張郃攻无當監何平於南圍,自案中道向亮。亮使魏延、高翔、吳班赴拒,大破之,獲甲首三千級、玄鎧五千領、角弩三千一百張,宣王還保營。糧盡退軍,與魏將張郃交戰,射殺郃。郭沖五事曰:魏明帝自征蜀,幸長安,遣宣王督張郃諸軍,雍、涼勁卒三十餘萬,潛軍密進,規向劒閣。亮時在祁山,旌旗利器,守在險要,十二更下,在者八万。時魏軍始陳,幡兵適交,參佐咸以賊衆彊盛,非力不制,宜權停下兵一月,以并聲勢。亮曰:「吾統武行師,以大信為本,得原失信,古人所惜;去者束裝以待期,妻子鶴望而計日,雖臨征難,義所不廢。」皆催遣令去。於是去者感恱,願留一戰,住者憤踊,思致死命。相謂曰:「諸葛公之恩,死猶不報也。」臨戰之日,莫不拔刃爭先,以一當十,殺張郃,却宣王,一戰大剋,此信之由也。難曰:臣松之案:亮前出祁山,魏明帝身至長安耳,此年不復自來。且亮大軍在關、隴,魏人何由得越亮徑向劒閣?亮旣在戰場,本無乆駐之規,而方休兵還蜀,皆非經通之言。孫盛、習鑿齒搜求異同,罔有所遺,而並不載沖言,知其乖剌多矣。

十二年春,亮悉大衆由斜谷出,以流馬運,據武功五丈原,與司馬宣王對於渭南。亮每患糧不繼,使己志不申,是以分兵屯田,為乆駐之基。耕者雜於渭濵居民之間,而百姓安堵,軍無私焉。漢晉春秋曰:亮自至,數挑戰。宣王亦表固請戰。使衞尉辛毗持節以制之。姜維謂亮曰:「辛佐治仗節而到,賊不復出矣。」亮曰:「彼本無戰情,所以固請戰者,以示武於其衆耳。將在軍,君命有所不受,苟能制吾,豈千里而請戰邪!」魏氏春秋曰:亮使至,問其寢食及其事之煩簡,不問戎事。使對曰:「諸葛公夙興夜寐,罰二十以上,皆親擥焉;所噉食不至數升。」宣王曰:「亮將死矣。」相持百餘日。其年八月,亮疾病,卒于軍,時年五十四。魏書曰:亮糧盡勢窮,憂恚歐血,一夕燒營遁走,入谷,道發病卒。漢晉春秋曰:亮卒于郭氏塢。晉陽秋曰:有星赤而芒角,自東北西南流,投于亮營,三投再還,往大還小。俄而亮卒。臣松之以為亮在渭濵,魏人躡迹,勝負之形,未可測量,而云歐血,蓋因亮自亡而自誇大也。夫以孔明之略,豈為仲達歐血乎?及至劉琨喪師,與晉元帝箋亦云「亮軍敗歐血」,此則引虛記以為言也。其云入谷而卒,緣蜀人入谷發喪故也。及軍退,宣王案行其營壘處所,曰:「天下奇才也!」漢晉春秋曰:楊儀等整軍而出,百姓奔告宣王,宣王追焉。姜維令儀反旗鳴鼓,若將向宣王者,宣王乃退,不敢偪。於是儀結陣而去,入谷然後發喪。宣王之退也,百姓為之諺曰:「死諸葛走生仲達。」或以告宣王,宣王曰:「吾能料生,不便料死也。」

亮遺命葬漢中定軍山,因山為墳,冢足容棺,歛以時服,不須器物。詔策曰:「惟君體資文武,明叡篤誠,受遺託孤,匡輔朕躬,繼絕興微,志存靖亂;爰整六師,無歲不征,神武赫然,威鎮八荒,將建殊功於季漢,參伊、周之巨勳。如何不弔,事臨垂克,遘疾隕喪!朕用傷悼,肝心若裂。夫崇德序功,紀行命謚,所以光昭將來,刊載不朽。今使使持節左中郎將杜瓊,贈君丞相武鄉侯印綬,謚君為忠武侯。魂而有靈,嘉茲寵榮。嗚呼哀哉!嗚呼哀哉!」

初,亮自表後主曰:「成都有桑八百株,薄田十五頃,子弟衣食自有餘饒。至於臣在外任,無別調度,隨身衣食,悉仰於官,不別治生,以長尺寸。若臣死之日,不使內有餘帛,外有贏財,以負陛下。」及卒,如其所言。

亮性長於巧思,損益連弩,木牛流馬,皆出其意;推演兵法,作八陣圖,咸得其要云。魏氏春秋曰:亮作八務、七戒、六恐、五懼,皆有條章,以訓厲臣子。又損益連弩,謂之元戎,以鐵為矢,矢長八寸,一弩十矢俱發。亮集載作木牛流馬法曰:「木牛者,方腹曲頭,一脚四足,頭入領中,舌著於腹。載多而行少,宜可大用,不可小使;特行者數十里,羣行者二十里也。曲者為牛頭,雙者為牛脚,橫者為牛領,轉者為牛足,覆者為牛背,方者為牛腹,垂者為牛舌,曲者為牛肋,刻者為牛齒,立者為牛角,細者為牛鞅,攝者為牛鞦䩜。牛仰雙轅,人行六尺,牛行四步。載一歲糧,日行二十里,而人不大勞。流馬尺寸之數,肋長三尺五寸,廣三寸,厚二寸二分,左右同。前軸孔分墨去頭四寸,徑中二寸。前脚孔分墨二寸,去前軸孔四寸五分,廣一寸。前杠孔去前脚孔分墨二寸七分,孔長二寸,廣一寸。後軸孔去前杠分墨一尺五分,大小與前同。後脚孔分墨去後軸孔三寸五分,大小與前同。後杠孔去後脚孔分墨二寸七分,後載剋去後杠孔分墨四寸五分。前杠長一尺八寸,廣二寸,厚一寸五分。後杠與等版方囊二枚,厚八分,長二尺七寸,高一尺六寸五分,廣一尺六寸,每枚受米二斛三斗。從上杠孔去肋下七寸,前後同。上杠孔去下杠孔分墨一尺三寸,孔長一寸五分,廣七分,八孔同。前後四脚,廣二寸,厚一寸五分。形制如象,靬長四寸,徑面四寸三分。孔徑中三脚杠,長二尺一寸,廣一寸五分,厚一寸四分,同杠耳。」亮言教書奏多可觀,別為一集。

景耀六年春,詔為亮立廟於沔陽。襄陽記曰:亮初亡,所在各求為立廟,朝議以禮秩不聽,百姓遂因時節私祭之於道陌上。言事者或以為可聽立廟於成都者,後主不從。步兵校尉習隆、中書郎向充等共上表曰:「臣聞周人懷召伯之德,甘棠為之不伐;越王思范蠡之功,鑄金以存其像。自漢興以來,小善小德而圖形立廟者多矣。況亮德範遐邇,勳蓋季世,王室之不壞,實斯人是賴,而蒸甞止於私門,廟像闕而莫立,使百姓巷祭,戎夷野祀,非所以存德念功,述追在昔者也。今若盡順民心,則瀆而無典,建之京師,又偪宗廟,此聖懷所以惟疑也。臣愚以為宜因近其墓,立之於沔陽,使所親屬以時賜祭,凡其臣故吏欲奉祠者,皆限至廟。斷其私祀,以崇正禮。」於是始從之。秋,魏鎮西將軍鍾會征蜀,至漢川,祭亮之廟,令軍士不得於亮墓所左右芻牧樵採。亮弟均,官至長水校尉。亮子瞻,嗣爵。襄陽記曰:黃承彥者,高爽開列,為沔南名士,謂諸葛孔明曰:「聞君擇婦;身有醜女,黃頭黑色,而才堪相配。」孔明許,即載送之。時人以為笑樂,鄉里為之諺曰:「莫作孔明擇婦,止得阿承醜女。」

諸葛氏集目錄開府作牧第一權制第二南征第三北出第四計筭第五訓厲第六綜覈上第七綜覈下第八雜言上第九雜言下第十貴和第十一兵要第十二傳運第十三與孫權書第十四與諸葛瑾書第十五與孟達書第十六廢李平第十七法檢上第十八法檢下第十九科令上第二十科令下第二十一軍令上第二十二軍令中第二十三軍令下第二十四右二十四篇,凡十萬四千一百一十二字。

臣壽等言:臣前在著作郎,侍中領中書監濟北侯臣荀勗、中書令關內侯臣和嶠奏,使臣定故蜀丞相諸葛亮故事。亮毗佐危國,負阻不賔,然猶存錄其言,恥善有遺,誠是大晉光明至德,澤被無疆,自古以來,未之有倫也。輒刪除複重,隨類相從,凡為二十四篇,篇名如右。

亮少有逸羣之才,英霸之器,身長八尺,容貌甚偉,時人異焉。遭漢末擾亂,隨叔父玄避難荊州,躬耕于野,不求聞達。時左將軍劉備以亮有殊量,乃三顧亮於草廬之中;亮深謂備雄姿傑出,遂解帶寫誠,厚相結納。及魏武帝南征荊州,劉琮舉州委質,而備失勢衆寡,無立錐之地。亮時年二十七,乃建奇策,身使孫權,求援吳會。權旣宿服仰備,又覩亮奇雅,甚敬重之,即遣兵三萬人以助備。備得用與武帝交戰,大破其軍,乘勝克捷,江南悉平。後備又西取益州。益州旣定,以亮為軍師將軍。備稱尊號,拜亮為丞相,錄尚書事。及備殂沒,嗣子幼弱,事無巨細,亮皆專之。於是外連東吳,內平南越,立法施度,整理戎旅,工械技巧,物究其極,科教嚴明,賞罰必信,無惡不懲,無善不顯,至於吏不容姧,人懷自厲,道不拾遺,彊不侵弱,風化肅然也。

當此之時,亮之素志,進欲龍驤虎視,苞括四海,退欲跨陵邊疆,震蕩宇內。又自以為無身之日,則未有能蹈涉中原、抗衡上國者,是以用兵不戢,屢耀其武。然亮才,於治戎為長,奇謀為短,理民之幹,優於將略。而所與對敵,或值人傑,加衆寡不侔,攻守異體,故雖連年動衆,未能有克。昔蕭何薦韓信,管仲舉王子城父,皆忖己之長,未能兼有故也。亮之器能政理,抑亦管、蕭之亞匹也,而時之名將無城父、韓信,故使功業陵遲,大義不及邪?蓋天命有歸,不可以智力爭也。

青龍二年春,亮帥衆出武功,分兵屯田,為乆駐之基。其秋病卒,黎庶追思,以為口實。至今梁、益之民咨述亮者,言猶在耳,雖甘棠之詠召公,鄭人之歌子產,無以遠譬也。孟軻有云:「以逸道使民,雖勞不怨;以生道殺人,雖死不忿。」信矣!論者或怪亮文彩不豔,而過於丁寧周至。臣愚以為咎繇大賢也,周公聖人也,考之尚書,咎繇之謩略而雅,周公之誥煩而悉。何則?咎繇與舜、禹共譚,周公與羣下矢誓故也。亮所與言,盡衆人凡士,故其文指不及得遠也。然其聲教遺言,皆經事綜物,公誠之心,形于文墨,足以知其人之意理,而有補於當世。

伏惟陛下邁蹤古聖,蕩然無忌,故雖敵國誹謗之言,咸肆其辭而無所革諱,所以明大通之道也。謹錄寫上詣著作。臣壽誠惶誠恐,頓首頓首,死罪死罪。

泰始十年二月一日癸巳,平陽侯相臣陳壽上。

喬字伯松,亮兄瑾之第二子也,本字仲慎。與兄元遜俱有名於時,論者以為喬才不及兄,而性業過之。初,亮未有子,求喬為嗣,瑾啟孫權遣喬來西,亮以喬為己適子,故易其字焉。拜為駙馬都尉,隨亮至漢中。亮與兄瑾書曰:「喬本當還成都,今諸將子弟皆得傳運,思惟宜同榮辱。今使喬督五六百兵,與諸子弟傳於谷中。」書在亮集。年二十五,建興元年卒。子攀,官至行護軍翊武將軍,亦早卒。諸葛恪見誅於吳,子孫皆盡,而亮自有冑裔,故攀還復為瑾後。

瞻字思遠。建興十二年,亮出武功,與兄瑾書曰:「瞻今已八歲而聦慧可愛,嫌其早成,恐不為重器耳。」年十七,尚公主,拜騎都尉。其明年為羽林中郎將,屢遷射聲校尉、侍中、尚書、尚書僕射,加軍師將軍。瞻工書畫,彊識念,蜀人追思亮,咸愛其才敏。每朝廷有一善政佳事,雖非瞻所建倡,百姓皆傳相告曰:「葛侯之所為也。」是以美聲溢譽,有過其實。景耀四年,為行都護衞將軍,與輔國大將軍南鄉侯董厥並平尚書事。六年冬,魏征西將軍鄧艾伐蜀,自陰平由景谷道旁入。瞻督諸軍至涪停住,前鋒破,退還,住緜竹。艾遣書誘瞻曰:「若降者必表為琅邪王。」瞻怒,斬艾使。遂戰,大敗,臨陣死,時年三十七。衆皆離散,艾長驅至成都。瞻長子尚,與瞻俱沒。干寶曰:瞻雖智不足以扶危,勇不足以拒敵,而能外不負國,內不改父之志,忠孝存焉。華陽國志曰:尚歎曰:「父子荷國重恩,不早斬黃皓,以致傾敗,用生何為!」乃馳赴魏軍而死。次子京及攀子顯等,咸熈元年內移河東。案諸葛氏譜云:京字行宗。晉泰始起居注載詔曰:「諸葛亮在蜀,盡其心力,其子瞻臨難而死義,天下之善一也。」其孫京,隨才署吏,後為郿令。尚書僕射山濤啟事曰:「郿令諸葛京,祖父亮,遇漢亂分隔,父子在蜀,雖不達天命,要為盡心所事。京治郿自復有稱,臣以為宜以補東宮舍人,以明事人之理,副梁、益之論。」京位至江州刺史。

董厥者,丞相亮時為府令史,亮稱之曰:「董令史,良士也。吾每與之言,思慎宜適。」徙為主簿。亮卒後,稍遷至尚書僕射,代陳祗為尚書令,遷大將軍,平臺事,而義陽樊建代焉。案晉百官表:董厥字龔襲,亦義陽人。建字長元。延熈十四年,以校尉使吳,值孫權病篤,不自見建。權問諸葛恪曰:「樊建何如宗預也?」恪對曰:「才識不及預,而雅性過之。」後為侍中,守中書令。自瞻、厥、建統事,姜維常征伐在外,宦人黃皓竊弄機柄,咸共將護,無能匡矯,孫盛異同記曰:瞻、厥等以維好戰無功,國內疲弊,宜表後主,召還為益州刺史,奪其兵權;蜀長老猶有瞻表以閻宇代維故事。晉永和三年,蜀史常璩說蜀長老云:「陳壽甞為瞻吏,為瞻所辱,故因此事歸惡黃皓,而云瞻不能匡矯也。」然建特不與皓和好往來。蜀破之明年春,厥、建俱詣京都,同為相國參軍,其秋並兼散騎常侍,使蜀慰勞。漢晉春秋曰:樊建為給事中,晉武帝問諸葛亮之治國,建對曰:「聞惡必改,而不矜過,賞罰之信,足感神明。」帝曰:「善哉!使我得此人以自輔,豈有今日之勞乎!」建稽首曰:「臣竊聞天下之論,皆謂鄧艾見枉,陛下知而不理,此豈馮唐之所謂『雖得頗、牧而不能用』者乎!」帝笑曰:「吾方欲明之,卿言起我意。」於是發詔治艾焉。

評曰:諸葛亮之為相國也,撫百姓,示儀軌,約官職,從權制,開誠心,布公道;盡忠益時者雖讎必賞,犯法怠慢者雖親必罰,服罪輸情者雖重必釋,游辭巧飾者雖輕必戮;善無微而不賞,惡無纖而不貶;庶事精練,物理其本,循名責實,虛偽不齒;終於邦域之內,咸畏而愛之,刑政雖峻而無怨者,以其用心平而勸戒明也。可謂識治之良才,管、蕭之亞匹矣。然連年動衆,未能成功,蓋應變將略,非其所長歟!

袁子曰:或問諸葛亮何如人也,袁子曰:張飛、關羽與劉備俱起,爪牙腹心之臣,而武人也。晚得諸葛亮,因以為佐相,而羣臣恱服,劉備足信、亮足重故也。及其受六尺之孤,攝一國之政,事凡庸之君,專權而不失禮,行君事而國人不疑,如此即以為君臣百姓之心欣戴之矣。行法嚴而國人恱服,用民盡其力而下不怨。及其兵出入如賔,行不寇,芻蕘者不獵,如在國中。其用兵也,止如山,進退如風,兵出之日,天下震動,而人心不憂。亮死至今數十年,國人歌思,如周人之思召公也,孔子曰「雍也可使南靣」,諸葛亮有焉。又問諸葛亮始出隴右,南安、天水、安定三郡人反應之,若亮速進,則三郡非中國之有也,而亮徐行不進;旣而官兵上隴,三郡復,亮無尺寸之功,失此機,何也?袁子曰:蜀兵輕銳,良將少,亮始出,未知中國彊弱,是以疑而甞之;且大會者不求近功,所以不進也。曰:何以知其疑也?袁子曰:初出遲重,屯營重複,後轉降未進兵欲戰,亮勇而能鬬,三郡反而不速應,此其疑徵也。曰:何以知其勇而能鬬也?袁子曰:亮之在街亭也,前軍大破,亮屯去數里,不救;官兵相接,又徐行,此其勇也。亮之行軍,安靜而堅重;安靜則易動,堅重則可以進退。亮法令明,賞罰信,士卒用命,赴險而不顧,此所以能鬬也。曰:亮帥數萬之衆,其所興造,若數十萬之功,是其奇者也。所至營壘、井竈、圊溷、藩籬、障塞皆應繩墨,一月之行,去之如始至,勞費而徒為飾好,何也?袁子曰:蜀人輕脫,亮故堅用之。曰:何以知其然也?袁子曰:亮治實而不治名,志大而所欲遠,非求近速者也。曰:亮好治官府、次舍、橋梁、道路,此非急務,何也?袁子曰:小國賢才少,故欲其尊嚴也。亮之治蜀,田疇辟,倉廩實,器械利,蓄積饒,朝會不華,路無醉人。夫本立故末治,有餘力而後及小事,此所以勸其功也。曰:子之論諸葛亮,則有證也。以亮之才而少其功,何也?袁子曰:亮,持本者也,其於應變,則非所長也,故不敢用其短。曰:然則吾子美之,何也?袁子曰:此固賢者之遠矣,安可以備體責也。夫能知所短而不用,此賢者之大也;知所短則知所長矣。夫前識與言而不中,亮之所不用也,此吾之所謂可也。

吳大鴻臚張儼作默記,其述佐篇論亮與司馬宣王書曰:漢朝傾覆,天下崩壞,豪傑之士,競希神器。魏氏跨中土,劉氏據益州,並稱兵海內,為世霸王。諸葛、司馬二相遭值際會,託身明主,或收功於蜀漢,或冊名於伊、洛。丕、備旣沒,後嗣旣統,各受保阿之任,輔翼幼主,不負然諾之誠,亦一國之宗臣,霸王之賢佐也。歷前世以觀近事,二相優劣,可得而詳也。孔明起巴、蜀之地,蹈一州之土,方之大國,其戰士人民,蓋有九分之一也,而以貢贄大吳,抗對北敵,至使耕戰有伍,刑法整齊,提步卒數萬,長驅祁山,慨然有飲馬河、洛之志。仲達據天下十倍之地,仗兼并之衆,據牢城,擁精銳,無禽敵之意,務自保全而已,使彼孔明自來自去。若此人不亡,終其志意,連年運思,刻日興謀,則涼、雍不解甲,中國不釋鞌,勝負之勢,亦已決矣。昔子產治鄭,諸侯不敢加兵,蜀相其近之矣。方之司馬,不亦優乎!或曰,兵者凶器,戰者危事也,有國者不務保安境內,綏靜百姓,而好開闢土地,征伐天下,未為得計也。諸葛丞相誠有匡佐之才,然處孤絕之地,戰士不滿五萬,自可閉關守險,君臣無事。空勞師旅,無歲不征,未能進咫尺之地,開帝王之基,而使國內受其荒殘,西土苦其役調。魏司馬懿才用兵衆,未易可輕,量敵而進,兵家所慎;若丞相必有以筭之,則未見坦然之勳,若無筭以裁之,則非明哲之謂,海內歸向之意也,余竊疑焉,請聞其說。荅曰:蓋聞湯以七十里、文王以百里之地而有天下,皆用征伐而定之。揖讓而登王位者,惟舜、禹而已。今蜀、魏為敵戰之國,勢不俱王,自操、備時,彊弱縣殊,而備猶出兵陽平,禽夏侯淵。羽圍襄陽,將降曹仁,生獲于禁,當時北邊大小憂懼,孟德身出南陽,樂進、徐晃等為救,圍不即解,故蔣子通言彼時有徙許渡河之計,會國家襲取南郡,羽乃解軍。玄德與操,智力多少,士衆衆寡,用兵行軍之道,不可同年而語,猶能暫以取勝,是時又無大吳掎角之勢也。今仲達之才,減於孔明,當時之勢,異於曩日,玄德尚與抗衡,孔明何以不可出軍而圖敵邪?昔樂毅以弱燕之衆,兼從五國之兵,長驅彊齊,下七十餘城。今蜀漢之卒,不少燕軍,君臣之接,信於樂毅,加以國家為脣齒之援,東西相應,首尾如蛇,形勢重大,不比於五國之兵也,何憚於彼而不可哉?夫兵以奇勝,制敵以智,土地廣狹,人馬多少,未可偏恃也。余觀彼治國之體,當時旣肅整,遺教在後,及其辭意懇切,陳進取之圖,忠謀謇謇,義形於主,雖古之管、晏,何以加之乎?

蜀記曰:晉永興中,鎮南將軍劉弘至隆中,觀亮故宅,立碣表閭,命太傅掾犍為李興為文曰:「天子命我于沔之陽,聽鼓鞞而永思,庶先哲之遺光,登隆山以遠望,軾諸葛之故鄉。蓋神物應機,大器無方,通人靡滯,大德不常。故谷風發而騶虞嘯,雲雷升而潛鱗驤;摯解褐於三聘,尼得招而褰裳,管豹變於受命,貢感激以回莊,異徐生之摘寶,釋卧龍於深藏,偉劉氏之傾蓋,嘉吾子之周行。夫有知己之主,則有竭命之良,固所以三分我漢鼎,跨帶我邊荒,抗衡我北面,馳騁我魏疆者也。英哉吾子,獨含天靈。豈神之祇,豈人之精?何思之深,何德之清!異世通夢,恨不同生。推子八陣,不在孫、吳,木牛之奇,則非般模,神弩之功,一何微妙!千井齊甃,又何祕要!昔在顛、夭,有名無迹,孰若吾儕,良籌妙畫?臧文旣沒,以言見稱,又未若子,言行並徵。夷吾反坫,樂毅不終,奚比於爾,明哲守沖。臨終受寄,讓過許由,負扆莅事,民言不流。刑中於鄭,教美于魯,蜀民知恥,河、渭安堵。匪皐則伊,寧彼管、晏,豈徒聖宣,慷慨屢歎!昔爾之隱,卜惟此宅,仁智所處,能無規廓。日居月諸,時殞其夕,誰能不歿,貴有遺格。惟子之勳,移風來世,詠歌餘典,懦夫將厲。遐哉邈矣,厥規卓矣,凡若吾子,難可究已。疇昔之乖,萬里殊塗;今我來思,覿爾故墟。漢高歸魂於豐、沛,太公五世而反周,想魍魎以髣髴,兾影響之有餘。魂而有靈,豈其識諸!」王隱晉書云:李興,密之子;一名安。


\end{pinyinscope}