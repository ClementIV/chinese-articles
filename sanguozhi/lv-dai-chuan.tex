\article{呂岱傳}

\begin{pinyinscope}
呂岱字定公,廣陵海陵人也,為郡縣吏,避亂南渡。孫權統事,岱詣幕府,出守吳丞。權親斷諸縣倉庫及囚繫,長丞皆見,岱處法應問,甚稱權意,召署錄事,出補餘姚長,召募精健,得千餘人。會稽東冶五縣賊呂合、秦狼等為亂,權以岱為督軍校尉,與將軍蔣欽等將兵討之,遂禽合、狼,五縣平定,拜昭信中郎將。

吳書曰:建安十六年,岱督郎將尹異等,以兵二千人西誘漢中賊帥張魯到漢興寋城,魯嫌疑斷道,事計不立,權遂召岱還。

建安二十年,督孫茂等十將從取長沙三郡。又安成、攸、永新、茶陵四縣吏共入陰山城,合衆拒岱,岱攻圍,即降,三郡克定。權留岱鎮長沙。安成長吳碭及中郎將袁龍等首尾關羽,復為反亂。碭據攸縣,龍在醴陵。權遣橫江將軍魯肅攻攸,碭得突走。岱攻醴陵,遂禽斬龍,遷廬陵太守。

延康元年,代步隲為交州刺史。到州,高涼賊帥錢愽乞降,岱因承制,以愽為高涼西部都尉。又鬱林夷賊攻圍郡縣,岱討破之。是時桂陽湞陽賊王金合衆於南海界上,首亂為害,權又詔岱討之,生縛金,傳送詣都,斬首獲生凡萬餘人。遷安南將軍,假節,封都鄉侯。

交阯太守士燮卒,權以燮子徽為安遠將軍,領九真太守,以校尉陳時代燮。岱表分海南三郡為交州,以將軍戴良為刺史,海東四郡為廣州,岱自為刺史。遣良與時南入,而徽不承命,舉兵戍海口以拒良等。岱於是上疏請討徽罪,督兵三千人晨夜浮海。或謂岱曰:「徽藉累世之恩,為一州所附,未易輕也。」岱曰:「今徽雖懷逆計,未虞吾之卒至,若我潛軍輕舉,掩其無備,破之必也。稽留不速,使得生心,嬰城固守,七郡百蠻,雲合響應,雖有智者,誰能圖之?」遂行,過合浦,與良俱進。徽聞岱至,果大震怖,不知所出,即率兄弟六人肉袒迎岱。岱皆斬送其首。徽大將甘醴、桓治等率吏民攻岱,岱奮擊,大破之,進封番禺侯。於是除廣州,復為交州如故。岱旣定交州,復進討九真,斬獲以萬數。又遣從事南宣國化,曁徼外扶南、林邑、堂明諸王,各遣使奉貢。權嘉其功,進拜鎮南將軍。

黃龍三年,以南土清定,召岱還屯長沙漚口。王隱交廣記曰:吳後復置廣州,以南陽滕脩為刺史。或語脩蝦鬚長一丈,脩不信,其人後故至東海,取蝦鬚長四丈四尺,封以示脩,脩乃服之。會武陵蠻夷蠢動,岱與太常潘濬共討定之。嘉禾二年,權令岱領潘璋士衆,屯陸口,後徙蒲圻。四年,廬陵賊李桓、路合、會稽東冶賊隨春、南海賊羅厲等一時並起。權復詔岱督劉纂、唐咨等分部討擊,春即時首降,岱拜春偏將軍,使領其衆,遂為列將,桓、厲等皆見斬獲,傳首詣都。權詔岱曰:「厲負險作亂,自致梟首;桓凶狡反覆,已降復叛。前後討伐,歷年不禽,非君規略,誰能梟之?忠武之節,於是益著。元惡旣除,大小震懾,其餘細類,埽地族矣。自今已去,國家永無南顧之虞,三郡晏然,無怵惕之驚,又得惡民以供賦役,重自歎息。賞不踰月,國之常典,制度所宜,君其裁之。」

潘濬卒,岱代濬領荊州文書,與陸遜並在武昌,故督蒲圻。頃之,廖式作亂,攻圍城邑,零陵、蒼梧、鬱林諸郡搔擾,岱自表輒行,星夜兼路。權遣使追拜岱交州牧,及遣諸將唐咨等駱驛相繼,攻討一年破之,斬式及遣諸所偽署臨賀太守費楊等,并其支黨,郡縣悉平,復還武昌。時年已八十,然體素精勤,躬親王事。奮威將軍張承與岱書曰:「昔旦奭翼周,二南作歌,今則足下與陸子也。忠勤相先,勞謙相讓,功以權成,化與道合,君子歎其德,小人悅其美。加以文書鞅掌,賔客終日,罷不舍事,勞不言倦,又知上馬輒自超乘,不由跨躡,如此足下過廉頗也,何其事事快也。周易有之,禮言恭,德言盛,足下何有盡此美耶!」及陸遜卒,諸葛恪代遜,權乃分武昌為兩部,岱督右部,自武昌上至蒲圻。遷上大將軍,拜子凱副軍校尉,監兵蒲圻。孫亮即位,拜大司馬。

岱清身奉公,所在可述。初在交州,歷年不餉家,妻子飢乏。權聞之歎息,以讓羣臣曰:「呂岱出身萬里,為國勤事,家門內困,而孤不早知。股肱耳目,其責安在?」於是加賜錢米布絹,歲有常限。

始,岱親近吳郡徐原,慷慨有才志,岱知其可成,賜巾褠,與共言論,後遂薦拔,官至侍御史。原性忠壯,好直言,岱時有得失,原輒諫諍,又公論之,人或以告岱,岱歎曰:「是我所以貴德淵者也。」及原死,岱哭之甚哀,曰:「德淵,呂岱之益友,今不幸,岱復於何聞過?」談者美之。

太平元年,年九十六卒,子凱嗣。遺令殯以素棺,疏巾布褠,葬送之制,務從約儉,凱皆奉行之。


\end{pinyinscope}