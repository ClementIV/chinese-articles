\article{dong-zhao-chuan}

\begin{pinyinscope}
董昭字公仁,濟陰定陶人也。舉孝廉,除廮陶長、栢人令,袁紹以為參軍事。紹逆公孫瓚於界橋,鉅鹿太守李邵及郡冠蓋,以瓚兵彊,皆欲屬瓚。紹聞之,使昭領鉅鹿。問:「禦以何術?」對曰:「一人之微,不能消衆謀,欲誘致其心,唱與同議,及得其情,乃當權以制之耳。計在臨時,未可得言。」時郡右姓孫伉等數十人專為謀主,驚動吏民。昭至郡,偽作紹檄告郡云:「得賊羅候安平張吉辭,當攻鉅鹿,賊故孝廉孫伉等為應,檄到收行軍法,惡止其身,妻子勿坐。」昭案檄告令,皆即斬之。一郡惶恐,乃以次安慰,遂皆平集。事訖白紹,紹稱善。會魏郡太守栗攀為兵所害,紹以昭領魏郡太守。時郡界大亂,賊以萬數,遣使往來,交易市買。昭厚待之,因用為間,乘虛掩討,輒大克破。二日之中,羽檄三至。

昭弟訪,在張邈軍中。邈與紹有隙,紹受讒將致罪於昭。昭欲詣漢獻帝,至河內,為張楊所留。因楊上還印綬,拜騎都尉。時太祖領兖州,遣使詣楊,欲令假塗西至長安,楊不聽。昭說楊曰:「袁、曹雖為一家,勢不乆羣。曹今雖弱,然實天下之英雄也,當故結之。況今有緣,宜通其上事,并表薦之;若事有成,永為深分。」楊於是通太祖上事,表薦太祖。昭為太祖作書與長安諸將李傕、郭汜等,各隨輕重致殷勤。楊亦遣使詣太祖。太祖遺楊犬馬金帛,遂與西方往來。天子在安邑,昭從河內往,詔拜議郎。

建安元年,太祖定黃巾于許,遣使詣河東。會天子還洛陽,韓暹、楊奉、董承及楊各違戾不和。昭以奉兵馬最彊而少黨援,作太祖書與奉曰:「吾與將軍聞名慕義,便推赤心。今將軍拔萬乘之艱難,反之舊都,翼佐之功,超世無疇,何其休哉!方今羣凶猾夏,四海未寧,神器至重,事在維輔;必須衆賢以清王軌,誠非一人所能獨建。心腹四支,實相恃賴,一物不備,則有闕焉。將軍當為內主,吾為外援。今吾有糧,將軍有兵,有無相通,足以相濟,死生契闊,相與共之。」奉得書喜恱,語諸將軍曰:「兖州諸軍近在許耳,有兵有糧,國家所當依仰也。」遂共表太祖為鎮東將軍,襲父爵費亭侯;昭遷符節令。

太祖朝天子於洛陽,引昭並坐,問曰:「今孤來此,當施何計?」昭曰:「將軍興義兵以誅暴亂,入朝天子,輔翼王室,此五伯之功也。此下諸將,人殊意異,未必服從,今留匡弼,事勢不便,惟有移駕幸許耳。然朝廷播越,新還舊京,遠近跂望,兾一朝獲安。今復徙駕,不厭衆心。夫行非常之事,乃有非常之功,願將軍筭其多者。」太祖曰:「此孤本志也。楊奉近在梁耳,聞其兵精,得無為孤累乎?」昭曰:「奉少黨援,將獨委質。鎮東、費亭之事,皆奉所定,又聞書命申束,足以見信。宜時遣使厚遺荅謝,以安其意。說『京都無糧,欲車駕暫幸魯陽,魯陽近許,轉運稍易,可無縣乏之憂』。奉為人勇而寡慮,必不見疑,比使往來,足以定計。奉何能為累!」太祖曰:「善。」即遣使詣奉。徙大駕至許。奉由是失望,與韓暹等到定陵鈔暴。太祖不應,密往攻其梁營,降誅即定。奉、暹失衆,東降袁術。三年,昭遷河南尹。時張楊為其將楊醜所殺,楊長史薛洪、河內太守繆尚城守待紹救。太祖令昭單身入城,告喻洪、尚等,即日舉衆降。以昭為兾州牧。

太祖令劉備拒袁術,昭曰:「備勇而志大,關羽、張飛為之羽翼,恐備之心未可得論也!」太祖曰:「吾已許之矣。」備到下邳,殺徐州刺史車冑,反。太祖自征備,徙昭為徐州牧。袁紹遣將顏良攻東郡,又徙昭為魏郡太守,從討良。良死後,進圍鄴城。袁紹同族春卿為魏郡太守,在城中,其父元長在楊州,太祖遣人迎之。昭書與春卿曰:「蓋聞孝者不背親以要利,仁者不忘君以徇私,志士不探亂以徼幸,智者不詭道以自危。足下大君,昔避內難,南游百越,非疏骨肉,樂彼吴會,智者深識,獨或宜然。曹公愍其守志清恪,離羣寡儔,故特遣使江東,或迎或送,今將至矣。就令足下處偏平之地,依德義之主,居有泰山之固,身為喬松之偶,以義言之,猶宜背彼向此,舍民趣父也。且邾儀父始與隱公盟,魯人嘉之,而不書爵,然則王所未命,爵尊不成,春秋之義也。況足下今日之所託者乃危亂之國,所受者乃矯誣之命乎?苟不逞之與羣,而厥父之不恤,不可以言孝。忘祖宗所居之本朝,安非正之姧職,難可以言忠。忠孝並替,難以言智。又足下昔日為曹公所禮辟,夫戚族人而疏所生,內所寓而外王室,懷邪祿而叛知己,遠福祚而近危亡,棄明義而收大恥,不亦可惜邪!若能翻然易節,奉帝養父,委身曹公,忠孝不墜,榮名彰矣。宜深留計,早決良圖。」鄴旣定,以昭為諫議大夫。後袁尚依烏丸蹋頓,太祖將征之。患軍糧難致,鑿平虜、泉州二渠入海通運,昭所建也。太祖表封千秋亭侯,轉拜司空軍祭酒。

後昭建議:「宜脩古建封五等。」太祖曰:「建設五等者,聖人也,又非人臣所制,吾何以堪之?」昭曰:「自古已來,人臣匡世,未有今日之功。有今日之功,未有乆處人臣之勢者也。今明公恥有慙德而未盡善,樂保名節而無大責,德美過於伊、周,此至德之所極也。然太甲、成王未必可遭,今民難化,甚於殷、周,處大臣之勢,使人以大事疑己,誠不可不重慮也。明公雖邁威德,明法術,而不定其基,為萬世計,猶未至也。定基之本,在地與人,宜稍建立,以自藩衞。明公忠節頴露,天威在顏,耿弇牀下之言,朱英無妄之論,不得過耳。昭受恩非凡,不敢不陳。」

獻帝春秋曰:昭與列侯諸將議,以丞相宜進爵國公,九錫備物,以彰殊勳;書與荀彧曰:「昔周旦、呂望,當姬氏之盛,因二聖之業,輔翼成王之幼,功勳若彼,猶受上爵,錫土開宇。末世田單,驅彊齊之衆,報弱燕之怨,收城七十,迎復襄王;襄王加賞於單,使東有掖邑之封,西有菑上之虞。前世錄功,濃厚如此。今曹公遭海內傾覆,宗廟焚滅,躬擐甲冑,周旋征伐,櫛風沐雨,且三十年,芟夷羣凶,為百姓除害,使漢室復存,劉氏奉祀。方之曩者數公,若太山之與丘垤,豈同日而論乎?今徒與列將功臣,並侯一縣,此豈天下所望哉!」後太祖遂受魏公、魏王之號,皆昭所創。

及關羽圍曹仁於樊,孫權遣使辭以「遣兵西上,欲掩取羽。江陵、公安累重,羽失二城,必自奔走,樊軍之圍,不救自解。乞密不漏,令羽有備。」太祖詰羣臣,羣臣咸言宜當密之。昭曰:「軍事尚權,期於合宜。宜應權以密,而內露之。羽聞權上,若還自護,圍則速解,便獲其利。可使兩賊相對銜持,坐待其弊。祕而不露,使權得志,非計之上。又,圍中將吏不知有救,計糧怖懼,儻有他意,為難不小。露之為便。且羽為人彊梁,自恃二城守固,必不速退。」太祖曰:「善。」即勑救將徐晃以權書射著圍裏及羽屯中,圍裏聞之,志氣百倍。羽果猶豫。權軍至,得其二城,羽乃破敗。

文帝即王位,拜昭將作大匠。及踐阼,遷大鴻臚,進封右鄉侯。二年,分邑百戶,賜昭弟訪爵關內侯,徙昭為侍中。三年,征東大將軍曹休臨江在洞浦口,自表:「願將銳卒虎步江南,因敵取資,事必克捷;若其無臣,不須為念。」帝恐休便渡江,驛馬詔止。時昭侍側,因曰:「竊見陛下有憂色,獨以休濟江故乎?今者渡江,人情所難,就休有此志,勢不獨行,當須諸將。臧霸等旣富且貴,無復他望,但欲終其天年,保守祿祚而已,何肯乘危自投死地,以求徼倖?苟霸等不進,休意自沮。臣恐陛下雖有勑渡之詔,猶必沈吟,未便從命也。」是後無幾,暴風吹賊船,悉詣休等營下,斬首獲生,賊遂迸散。詔勑諸軍促渡。軍未時進,賊救船遂至。

大駕幸宛,征南大將軍夏侯尚等攻江陵,未拔。時江水淺狹,尚欲乘船將步騎入渚中安屯,作浮橋,南北往來,議者多以為城必可拔。昭上疏曰:「武皇帝智勇過人,而用兵畏敵,不敢輕之若此也。夫兵好進惡退,常然之數。平地無險,猶尚艱難,就當深入,還道宜利,兵有進退,不可如意。今屯渚中,至深也;浮橋而濟,至危也;一道而行,至狹也:三者兵家所忌,而今行之。賊頻攻橋,誤有漏失,渚中精銳,非魏之有,將轉化為吳矣。臣私慼之,忘寢與食,而議者怡然不以為憂,豈不惑哉!加江水向長,一旦暴增,何以防禦?就不破賊,尚當自完。柰何乘危,不以為懼?事將危也,惟陛下察之!」帝悟昭言,即詔尚等促出。賊兩頭並前,官兵一道引去,不時得泄,將軍石建、高遷僅得自免。軍出旬日,江水暴長。帝曰:「君論此事,何其審也!正使張、陳當之,何以復加。」五年,徙封成都鄉侯,拜太常。其年,徙光祿大夫、給事中。從大駕東征,七年還,拜太僕。明帝即位,進爵樂平侯,邑千戶,轉衞尉。分邑百戶,賜一子爵關內侯。

太和四年,行司徒事,六年,拜真。昭上疏陳末流之弊曰:「凡有天下者,莫不貴尚敦樸忠信之士,深疾虛偽不真之人者,以其毀教亂治,敗俗傷化也。近魏諷則伏誅建安之末,曹偉則斬戮黃初之始。伏惟前後聖詔,深疾浮偽,欲以破散邪黨,常用切齒;而執法之吏皆畏其權勢,莫能糾擿,毀壞風俗,侵欲滋甚。竊見當今年少,不復以學問為本,專更以交游為業;國士不以孝悌清脩為首,乃以趨勢游利為先。合黨連羣,互相襃歎,以毀訾為罰戮,用黨譽為爵賞,附己者則歎之盈言,不附者則為作瑕釁。至乃相謂『今世何憂不度邪,但求人道不勤,羅之不博耳;又何患其不知己矣,但當吞之以藥而柔調耳。』又聞或有使奴客名作在職家人,冒之出入,往來禁奧,交通書疏,有所探問。凡此諸事,皆法之所不取,刑之所不赦,雖諷、偉之罪,無以加也。」帝於是發切詔,斥免諸葛誕、鄧颺等。昭年八十一薨,謚曰定侯。子冑嗣。冑歷位郡守、九卿。


\end{pinyinscope}