\article{蔣欽傳}

\begin{pinyinscope}
蔣欽字公弈,九江壽春人也。孫策之襲袁術,欽隨從給事。及策東渡,拜別部司馬,授兵。與策周旋,平定三郡,又從定豫章。調授葛陽尉,歷三縣長,討平盜賊,遷西部都尉。會稽冶賊呂合、秦狼等為亂,欽將兵討擊,遂禽合、狼,五縣平定,徙討越中郎將,以涇拘、昭陽為奉邑。賀齊討黝賊,欽督萬兵,與齊并力,黝賊平定。從征合肥,魏將張遼襲權於津北,欽力戰有功,遷盪寇將軍,領濡須督。後召還都,拜津右護軍,典領辭訟。

權甞入其堂內,母踈帳縹被,妻妾布裙。權歎其在貴守約,即勑御府為母作錦被,改易帷帳,妻妾衣服悉皆錦繡。

初,欽屯宣城,甞討豫章賊。蕪湖令徐盛收欽屯吏,表斬之,權以欽在遠不許,盛由是自嫌於欽。曹公出濡須,欽與呂蒙持諸軍節度。盛常畏欽因事害己,而欽每稱其善。盛旣服德,論者美焉。

江表傳曰:權謂欽曰:「盛前白卿,卿今舉盛,欲慕祁奚邪?」欽對曰:「臣聞公舉不挾私怨,盛忠而勤彊,有膽略器用,好萬人督也。今大事未定,臣當助國求才,豈敢挾私恨以蔽賢乎!」權嘉之。

權討關羽,欽督水軍入沔,還,道病卒。權素服舉哀,以蕪湖民二百戶、田二百頃,給欽妻子。子壹封宣城侯,領兵拒劉備有功,還赴南郡,與魏交戰,臨陣卒。壹無子,弟休領兵,後有罪失業。


\end{pinyinscope}