\article{劉劭傳}

\begin{pinyinscope}
劉劭字孔才,廣平邯鄲人也。建安中,為計吏,詣許。太史上言:「正旦當日蝕。」劭時在尚書令荀彧所,坐者數十人,或云當廢朝,或云宜却會。劭曰:「梓慎、裨竈,古之良史,猶占水火,錯失天時。禮記曰諸侯旅見天子,及門不得終禮者四,日蝕在一。然則聖人垂制,不為變異豫廢朝禮者,或災消異伏,或推術謬誤也。」彧善其言。勑朝會如舊,日亦不蝕。

晉永和中,廷尉王彪之與揚州刺史殷浩書曰:「太史上元日合朔,談者或有疑,應却會與不?昔建元元年,亦元日合朔,庾車騎寫劉孔才所論以示八座。于時朝議有謂孔才所論為不得禮議,荀令從之,是勝人之一失也。何者?禮云,諸侯旅見天子,入門不得終禮而廢者四:太廟火,日蝕,后之喪,雨霑服失容。尋此四事之指,自謂諸侯雖已入門而卒暴有之,則不得終禮。非為先存其事,而徼倖史官推術錯謬,故不豫廢朝禮也。夫三辰有災,莫大日蝕,史官告譴,而無懼容,不脩豫防之禮,而廢消救之術,方大饗華夷,君臣相慶,豈是將處天災罪己之謂?且檢之事實,合朔之儀,至尊靜躬殿堂,不聽政事,冕服御坐門闥之制,與元會禮異。自不得兼行,則當權其事宜。合朔之禮,不輕於元會。元會有可却之準,合朔無可廢之義。謂應依建元故事,却元會。」浩從之,竟却會。

御史大夫郗慮辟劭,會慮免,拜太子舍人,遷祕書郎。黃初中,為尚書郎、散騎侍郎。受詔集五經羣書,以類相從,作皇覽。明帝即位,出為陳留太守,敦崇教化,百姓稱之。徵拜騎都尉,與議郎庾嶷、荀詵等定科令,作新律十八篇,著律略論。遷散騎常侍。時聞公孫淵受孫權燕王之號,議者欲留淵計吏,遣兵討之,劭以為「昔袁尚兄弟歸淵父康,康斬送其首,是淵先世之効忠也。又所聞虛實,未可審知。古者要荒未服,脩德而不征,重勞民也。宜加寬貸,使有以自新。」後淵果斬送權使張彌等首。劭嘗作趙都賦,明帝美之,詔劭作許都、洛都賦。時外興軍旅,內營宮室,劭作二賦,皆諷諫焉。

青龍中,吳圍合肥,時東方吏士皆分休,征東將軍滿寵表請中軍兵,并召休將士,須集擊之。劭議以為「賊衆新至,心專氣銳。寵以少人自戰其地,若便進擊,不必能制。寵求待兵,未有所失也。以為可先遣步兵五千,精騎三千,軍前發,揚聲進道,震曜形勢。騎到合肥,疏其行隊,多其旌鼔,曜兵城下,引出賊後,擬其歸路,要其糧道。賊聞大軍來,騎斷其後,必震怖遁走,不戰自破賊矣。」帝從之。兵比至合肥,賊果退還。

時詔書博求衆賢。散騎侍郎夏侯惠薦劭曰:「伏見常侍劉劭,深忠篤思,體周於數,凡所錯綜,源流弘遠,是以羣才大小,咸取所同而斟酌焉。故性實之士服其平和良正,清靜之人慕其玄虛退讓,文學之士嘉其推步詳密,法理之士明其分數精比,意思之士知其沈深篤固,文章之士愛其著論屬辭,制度之士貴其化略較要,策謀之士贊其明思通微,凡此諸論,皆取適己所長而舉其支流者也。臣數聽其清談,覽其篤論,漸漬歷年,服膺彌乆,實為朝廷奇其器量。以為若此人者,宜輔翼機事,納謀幃幄,當與國道俱隆,非世俗所常有也。惟陛下垂優游之聽,使劭承清閑之歡,得自盡於前,則德音上通,煇燿日新矣。」臣松之以為凡相稱薦,率多溢美之辭,能不違中者或寡矣。惠之稱劭云「玄虛退讓」及「明思通微」,近於過也。

景初中,受詔作都官考課。劭上疏曰:「百官考課,王政之大較,然而歷代弗務,是以治典闕而未補,能否混而相蒙。陛下以上聖之宏略,愍王綱之弛頹,神慮內鑒,明詔外發。臣奉恩曠然,得以啟矇,輒作都官考課七十二條,又作說略一篇。臣學寡識淺,誠不足以宣暢聖旨,著定典制。」又以為宜制禮作樂,以移風俗,著樂論十四篇,事成未上。會明帝崩,不施行。正始中。執經講學,賜爵關內侯。凡所選述,法論、人物志之類百餘篇。卒,追贈光祿勳。子琳嗣。

劭同時東海繆襲亦有才學,多所述叙,官至尚書、光祿勳。先賢行狀曰:繆斐字文雅。該覽經傳,事親色養。徵博士,六辟公府。漢帝在長安,公卿博舉名儒。時舉斐任侍中,並無所就。即襲父也。文章志曰:襲字熈伯。辟御史大夫府,歷事魏四世。正始六年,年六十卒。子恱字孔懌,晉光祿大夫。襲孫紹、播、徵、胤等,並皆顯達。

襲友人山陽仲長統,漢末為尚書郎,早卒。著昌言,詞佳可觀省。襲撰統昌言表,稱統字公理,少好學,博涉書記,贍於文辭。年二十餘,游學青、徐、并、兾之閒,與交者多異之。并州刺史高幹素貴有名,招致四方游士,多歸焉。統過幹,幹善待遇之,訪以世事。統謂幹曰:「君有雄志而無雄才,好士而不能擇人,所以為君深戒也。」幹雅自多,不納統言。統去之,無幾而幹敗。并、兾之士以是識統。大司農常林與統共在上黨,為臣道統性倜儻,敢直言,不矜小節,每列郡命召,輒稱疾不就。默語無常,時人或謂之狂。漢帝在許,尚書令荀彧領典樞機,好士愛奇,聞統名,啟召以為尚書郎。後參太祖軍事,復還為郎。延康元年卒,時年四十餘。統每論說古今世俗行事,發憤歎息,輒以為論,名曰昌言,凡二十四篇。

散騎常侍陳留蘇林、魏略曰:林字孝友,博學,多通古今寄指,凡諸書傳文間危疑,林皆釋之。建安中,為五官將文學,甚見禮待。黃初中,為博士給事中。文帝作典論所稱蘇林者是也。以老歸第,國家每遣人就問之,數加賜遺。年八十餘卒。光祿大夫京兆韋誕、文章叙錄曰:誕字仲將,太僕端之子。有文才,善屬辭章。建安中,為郡上計吏,特拜郎中,稍遷侍中中書監,以光祿大夫遜位,年七十五卒於家。初,邯鄲淳、衞覬及誕並善書,有名。覬孫恒撰四體書勢,其序古文曰:「自秦用篆書,焚燒先典,而古文絕矣。漢武帝時,魯恭王壞孔子宅,得尚書、春秋、論語、孝經,時人已不復知有古文,謂之科斗書,漢世祕藏,希得見之。魏初傳古文者,出於邯鄲淳。敬侯寫淳尚書,後以示淳,而淳不別。至正始中,立三字石經,轉失淳法。因科斗之名,遂效其法。太康元年,汲縣民盜發魏襄王冢,得策書十餘萬言。案敬侯所書,猶有髣髴。」敬侯謂覬也。其序篆書曰:「秦時李斯號為工篆,諸山及銅人銘皆斯書也。漢建初中,扶風曹喜少異於斯而亦稱善。邯鄲淳師焉,略究其妙。韋誕師淳而不及也。太和中,誕為武都太守,以能書留補侍中,魏氏寶器銘題皆誕書云。漢末又有蔡邕采斯、喜之法,為古今雜形,然精密簡理不如淳也。」其序錄隷書,已略見武紀。又曰:「師宜官為大字,邯鄲淳為小字。梁鵠謂淳得次仲法,然鵠之用筆盡其勢矣。」其序草書曰:「漢興而有草書,不知作者姓名。至章帝時,齊相杜度號善作篇,後有崔瑗、崔寔亦皆稱工。杜氏然字甚安而書體微瘦,崔氏甚得筆勢而結字小疏。弘農張伯英者因而而轉精其巧。凡家之衣帛,必書而後練之,臨池學書,池水盡黑。下筆必為楷則,號『怱怱不暇草』,寸紙不見遺,至今世人尤寶之,韋仲將謂之草聖。伯英弟文舒者,次伯英。又有姜孟潁、梁孔達、田彥和及韋仲將之徒,皆伯英弟子,有名於世,然殊不及文舒也。」樂安太守譙國夏侯惠、惠,淵子。事在淵傳。陳郡太守任城孫該、文章叙錄曰:該字公達。彊志好學。年二十,上計掾,召為郎中。著魏書。遷博士司徒右長史,復還入著作。景元二年卒官。郎中令河東杜摯等亦著文賦,頗傳於世。文章叙錄曰:摯字德魯。初上笳賦,署司徒軍謀吏。後舉孝廉,除郎中,轉補校書。摯與毌丘儉鄉里相親,故為詩與儉,求仙人藥一丸,欲以感切儉求助也。其詩曰:「騏驥馬不試,婆娑槽櫪間。壯士志未伸,坎軻多辛酸。伊摯為媵臣,呂望身操竿;夷吾困商販,寗戚對牛歎;食其處監門,淮陰飢不餐;買臣老負薪,妻畔呼不還,釋之宦十年,位不增故官。才非八子倫,而與齊其患。無知不在此,袁盎未有言。被此篤病乆,榮衞動不安,聞有韓衆藥,信來給一丸。」儉荅曰:「鳳鳥翔京邑,哀鳴有所思。才為聖世出,德音何不怡!八子未遭遇,今者遘明時。胡康出壟畒,楊偉無根基,飛騰沖雲天,奮迅協光熙。駿驥骨法異,伯樂觀知之,但當養羽翮,鴻舉必有期。體無纖微疾,安用問良醫?聯翩輕栖集,還為燕雀嗤。韓衆藥雖良,或更不能治。悠悠千里情,薄言荅嘉詩。信心感諸中,中實不在辭。」摯竟不得遷,卒于祕書。廬江何氏家傳曰:明帝時,有譙人胡康,年十五,以異才見送,又陳損益,求試劇縣。詔特引見。衆論翕然,號為神童。詔付祕書,使博覽典籍。帝以問祕書丞何禎:「康才何如?」禎荅曰:「康雖有才,性質不端,必有負敗。」後果以過見譴。臣松之案:魏朝自微而顯者,不聞胡康;疑是孟康。康事見杜恕傳。楊偉見曹爽傳。


\end{pinyinscope}