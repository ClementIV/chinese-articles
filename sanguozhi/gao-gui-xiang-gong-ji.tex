\article{gao-gui-xiang-gong-ji}

\begin{pinyinscope}
高貴鄉公諱髦,字彥士,文帝孫,東海定王霖子也。正始五年,封歘縣高貴鄉公。少好學,夙成。齊王廢,公卿議迎立公。十月己丑,公至于玄武館,羣臣奏請舍前殿,公以先帝舊處,避止西廂;羣臣又請以法駕迎,公不聽。庚寅,公入于洛陽,羣臣迎拜西掖門南,公下輿將荅拜,儐者請曰:「儀不拜。」公曰:「吾人臣也。」遂荅拜。至止車門下輿。左右曰:「舊乘輿入。」公曰:「吾被皇太后徵,未知所為!」遂步至太極東堂,見于太后。其日即皇帝位於太極前殿,百寮陪位者欣欣焉。

魏氏春秋曰:公神明爽儁,德音宣朗。罷朝,景王私曰:「上何如主也?」鍾會對曰:「才同陳思,武類太祖。」景王曰:「若如卿言,社稷之福也。」詔曰:「昔三祖神武聖德,應天受祚。齊王嗣位,肆行非度,顛覆厥德。皇太后深惟社稷之重,延納宰輔之謀,用替厥位,集大命于余一人。以眇眇之身,託于王公之上,夙夜祗畏,懼不能嗣守祖宗之大訓,恢中興之弘業,戰戰兢兢,如臨于谷。今羣公卿士股肱之輔,四方征鎮宣力之佐,皆積德累功,忠勤帝室;庶憑先祖先父有德之臣,左右小子,用保乂皇家,俾朕蒙闇,垂拱而治。蓋聞人君之道,德厚侔天地,潤澤施四海,先之以慈愛,示之以好惡,然後教化行於上,兆民聽於下。朕雖不德,昧於大道,思與宇內共臻茲路。書不云乎:『安民則惠,黎民懷之。』」大赦,改元。減乘輿服御後宮用度,及罷尚方御府百工技巧靡麗無益之物。

正元元年冬十月壬辰,遣侍中持節分適四方,觀風俗,勞士民,察冤枉失職者。癸巳,假大將軍司馬景王黃鉞,入朝不趨,奏事不名,劒履上殿。戊戌,黃龍見于鄴井中。甲辰,命有司論廢立定策之功,封爵、增邑、進位、班賜各有差。

二年春正月乙丑,鎮東將軍毌丘儉、楊州刺史文欽反。戊寅,大將軍司馬景王征之。癸未,車騎將軍郭淮薨。閏月己亥,破欽於樂嘉。欽遁走,遂奔吳。甲辰,安風津都尉斬儉,傳首京都。世語曰:大將軍奉天子征儉,至項;儉旣破,天子先還。臣松之檢諸書都無此事,至諸葛誕反,司馬文王始挾太后及帝與俱行耳。故發詔引漢二祖及明帝親征以為前比,知明帝已後始有此行也。案張璠、虞溥、郭頒皆晉之令史,璠、頒出為官長,溥,鄱陽內史。璠撰後漢紀,雖似未成,辭藻可觀。溥著江表傳,亦粗有條貫。惟頒撰魏晉世語,蹇乏全無宮商,最為鄙劣,以時有異事,故頗行於世。干寶、孫盛等多采其言以為晉書,其中虛錯如此者,往往而有之。壬子,復特赦淮南士民諸為儉、欽所詿誤者。以鎮南將軍諸葛誕為鎮東大將軍。司馬景王薨于許昌。二月丁巳,以衞將軍司馬文王為大將軍,錄尚書事。

甲子,吳大將孫峻等衆號十萬至壽春,諸葛誕拒擊破之,斬吳左將軍留贊,獻捷于京都。三月,立皇后卞氏,大赦。夏四月甲寅,封后父卞隆為列侯。甲戌,以征南大將軍王昶為驃騎將軍。秋七月,以征東大將軍胡遵為衞將軍,鎮東大將軍諸葛誕為征東大將軍。

八月辛亥,蜀大將軍姜維寇狄道,雍州刺史王經與戰洮西,經大敗,還保狄道城。辛未,以長水校尉鄧艾行安西將軍,與征西將軍陳泰并力拒維。戊辰,復遣太尉司馬孚為後繼。九月庚子,講尚書業終,賜執經親授者司空鄭冲、侍中鄭小同等各有差。甲辰,姜維退還。冬十月,詔曰:「朕以寡德,不能式遏寇虐,乃令蜀賊陸梁邊陲。洮西之戰,至取負敗,將士死亡,計以千數,或沒命戰場,冤魂不反,或牽掣虜手,流離異域,吾深痛愍,為之悼心。其令所在郡典農及安撫夷二護軍各部大吏慰卹其門戶,無差賦役一年;其力戰死事者,皆如舊科,勿有所漏。」

十一月甲午,以隴右四郡及金城連年受敵,或亡叛投賊,其親戚留在本土者不安,皆特赦之。癸丑,詔曰:「往者洮西之戰,將吏士民或臨陣戰亡,或沉溺洮水,骸骨不収,棄於原野,吾常痛之。其告征西、安西將軍,各令部人於戰處及水次鈎求屍喪,収斂藏埋,以慰存亡。」

甘露元年春正月辛丑,青龍見軹縣井中。乙巳,沛王林薨。魏氏春秋曰:二月丙辰,帝宴羣臣於太極東堂,與侍中荀顗、尚書崔贊、袁亮、鍾毓、給事中中書令虞松等並講述禮典,遂言帝王優劣之差。帝慕夏少康,因問顗等曰:「有夏旣衰,后相殆滅,少康收集夏衆,復禹之績,高祖拔起隴畒,驅帥豪儁,芟夷秦、項,包舉宇內,斯二主可謂殊才異略,命世大賢者也。考其功德,誰宜為先?」顗等對曰:「夫天下重器,王者天授,聖德應期,然後能受命創業。至於階緣前緒,興復舊績,造之與因,難易不同。少康功德雖美,猶為中興之君,與世祖同流可也。至如高祖,臣等以為優。」帝曰:「自古帝王,功德言行互有高下,未必創業者皆優,紹繼者咸劣也。湯、武、高祖雖俱受命,賢聖之分,所覺縣殊。少康、殷宗中興之美,夏啟、周成守文之盛,論德較實,方諸漢祖,吾見其優,未聞其劣;顧所遇之時殊,故所名之功異耳。少康生於滅亡之後,降為諸侯之隷,崎嶇逃難,僅以身免,能布其德而兆其謀,卒滅過、戈,克復禹績,祀夏配天,不失舊物,非至德弘仁,豈濟斯勳?漢祖因土崩之勢,仗一時之權,專任智力以成功業,行事動靜多違聖檢;為人子則數危其親,為人君則囚繫賢相,為人父則不能衞子;身沒之後,社稷幾傾,若與少康易時而處,或未能復大禹之績也。推此言之,宜高夏康而下漢祖矣。諸卿具論詳之。」翌日丁巳,講業旣畢,顗、亮等議曰:「三代建國,列土而治,當其衰弊,無土崩之勢,可懷以德,難屈以力。逮至戰國,彊弱相兼,去道德而任智力。故秦之弊可以力爭。少康布德,仁者之英也;高祖任力,智者之儁也。仁智不同,二帝殊矣。詩、書述殷中宗、高宗,皆列大雅,少康功美過於二宗,其為大雅明矣。少康為優,宜如詔旨。」贊、毓、松等議曰:「少康雖積德累仁,然上承大禹遺澤餘慶,內有虞、仍之援,外有靡、艾之助,寒浞讒慝,不德于民,澆、豷無親,外內棄之,以此有國,蓋有所因。至於漢祖,起自布衣,率烏合之士,以成帝者之業。論德則少康優,課功則高祖多,語資則少康易,校時則高祖難。」帝曰:「諸卿論少康因資,高祖創造,誠有之矣,然未知三代之世,任德濟勳如彼之難,秦、項之際,任力成功如此之易。且太上立德,其次立功,漢祖功高,未若少康盛德之茂也。且夫仁者必有勇,誅暴必用武,少康武烈之威,豈必降於高祖哉?但夏書淪亡,舊文殘缺,故勳美闕而罔載,唯有伍員粗述大略,其言復禹之績,不失舊物,祖述聖業,舊章不行,自非大雅兼才,孰能與於此,向令墳、典具存,行事詳備,亦豈有異同之論哉?」於是羣臣咸恱服。中書令松進曰:「少康之事,去世乆遠,其文昧如,是以自古及今,議論之士莫有言者,德美隱而不宣。陛下旣垂心遠鑒,考詳古昔,又發德音,贊明少康之美,使顯於千載之上,宜錄以成篇,永垂于後。」帝曰:「吾學不博,所聞淺狹,懼於所論,未獲其宜;縱有可采,億則屢中,又不足貴,無乃致笑後賢,彰吾闇昧乎!」於是侍郎鍾會退論次焉。

夏四月庚戌,賜大將軍司馬文王衮冕之服,赤舄副焉。

丙辰,帝幸太學,問諸儒曰:「聖人幽贊神明,仰觀俯察,始作八卦,後聖重之為六十四,立爻以極數,凡斯大義,罔有不備,而夏有連山,殷有歸藏,周曰周易,易之書,其故何也?」易博士淳于俊對曰:「包羲因燧皇之圖而制八卦,神農演之為六十四,黃帝、堯、舜通其變,三代隨時,質文各繇其事。故易者,變易也,名曰連山,似山出內雲氣,連天地也;歸藏者,萬事莫不歸藏於其中也。」帝又曰:「若使包羲因燧皇而作易,孔子何以不云燧人氏沒包羲氏作乎?」俊不能荅。帝又問曰:「孔子作彖、象,鄭玄作注,雖聖賢不同,其所釋經義一也。今彖、象不與經文相連,而注連之,何也?」俊對曰;「鄭玄合彖、象於經者,欲使學者尋省易了也。」帝曰:「若鄭玄合之,於學誠便,則孔子曷為不合以了學者乎?」俊對曰:「孔子恐其與文王相亂,是以不合,此聖人以不合為謙。」帝曰:「若聖人以不合為謙,則鄭玄何獨不謙邪?」俊對曰:「古義弘深,聖問奧遠,非臣所能詳盡。」帝又問曰:「繫辭云『黃帝、堯、舜垂衣裳而天下治』,此包羲、神農之世為無衣裳。但聖人化天下,何殊異爾邪?」俊對曰:「三皇之時,人寡而禽獸衆,故取其羽皮而天下用足,及至黃帝,人衆而禽獸寡,是以作為衣裳以濟時變也。」帝又問:「乾為天,而復為金,為玉,為老馬,與細物並邪?」俊對曰:「聖人取象,或遠或近,近取諸物,遠則天地。」

講易畢,復命講尚書。帝問曰:「鄭玄云『稽古同天,言堯同於天也』。王肅云『堯順考古道而行之』。二義不同,何者為是?」博士庾峻對曰:「先儒所執,各有乖異,臣不足以定之。然洪範稱『三人占,從二人之言』。賈、馬及肅皆以為『順考古道』。以洪範言之,肅義為長。」帝曰:「仲尼言『唯天為大,唯堯則之』。堯之大美,在乎則天,順考古道,非其至也。今發篇開義以明聖德,而舍其大,更稱其細,豈作者之意邪?」峻對曰:「臣奉遵師說,未喻大義,至於折中,裁之聖思。」次及四嶽舉鯀,帝又問曰:「夫大人者,與天地合其德,與日月合其明,思無不周,明無不照,今王肅云『堯意不能明鯀,是以試用』。如此,聖人之明有所未盡邪?」峻對曰:「雖聖人之弘,猶有所未盡,故禹曰『知人則哲,惟帝難之』,然卒能改授聖賢,緝熈庶績,亦所以成聖也。」帝曰:「夫有始有卒,其唯聖人。若不能始,何以為聖?其言『惟帝難之』,然卒能改授,蓋謂知人,聖人所難,非不盡之言也。經云:『知人則哲,能官人。』若堯疑鯀,試之九年,官人失叙,何得謂之聖哲?」峻對曰:「臣竊觀經傳,聖人行事不能無失,是以堯失之四凶,周公失之二叔,仲尼失之宰予。」帝曰:「堯之任鯀,九載無成,汨陳五行,民用昏墊。至於仲尼失之宰予,言行之間,輕重不同也。至於周公、管、蔡之事,亦尚書所載,皆博士所當通也。」峻對曰:「此皆先賢所疑,非臣寡見所能究論。」次及「有鰥在下曰虞舜」,帝問曰:「當堯之時,洪水為害,四凶在朝,宜速登賢聖濟斯民之時也。舜年在旣立,聖德光明,而乆不進用,何也?」峻對曰:「堯咨嗟求賢,欲遜己位,嶽曰『否德忝帝位』。堯復使嶽揚舉仄陋,然後薦舜。薦舜之本,實由於堯,此蓋聖人欲盡衆心也。」帝曰:「堯旣聞舜而不登用,又時忠臣亦不進達,乃使獄揚仄陋而後薦舉,非急於用聖恤民之謂也。」峻對曰:「非臣愚見所能逮及。」

於是復命講禮記。帝問曰:「『太上立德,其次務施報』。為治何由而教化各異;皆脩何政而能致於立德,施而不報乎?」博士馬照對曰:「太上立德,謂三皇五帝之世以德化民,其次報施,謂三王之世以禮為治也。」帝曰:「二者致化薄厚不同,將主有優劣邪?時使之然乎?」照對曰:「誠由時有樸文,故化有薄厚也。」帝集載帝自敘始生禎祥曰:「昔帝王之生,或有禎祥,蓋所以彰顯神異也。惟予小子,支胤末流,謬為靈祇之所相祐也,豈敢自比於前喆,聊記錄以示後世焉。其辭曰:惟正始三年九月辛未朔,二十五日乙未直成,予生。于時也,天氣清明,日月暉光,爰有黃氣,烟熅於堂,照曜室宅,其色煌煌。相而論之曰:未者為土,魏之行也;厥日直成,應嘉名也;烟熅之氣,神之精也;無災無害,蒙神靈也。齊王不弔,顛覆厥度,羣公受予,紹繼皇祚。以眇眇之身,質性頑固,未能涉道,而遵大路,臨深履冰,涕泗憂懼。古人有云,懼則不亡。伊予小子,曷敢怠荒?庶不忝辱,永奉烝甞。」傅暢晉諸公贊曰:帝常與中護軍司馬望、侍中王沈、散騎常侍裴秀、黃門侍郎鍾會等講宴於東堂,并屬文論。名秀為儒林丈人,沈為文籍先生,望、會亦各有名號。帝性急,請召欲速。秀等在內職,到得及時,以望在外,特給追鋒車,虎賁卒五人,每有集會,望輒奔馳而至。

五月,鄴及上洛並言甘露降。夏六月丙午,改元為甘露。乙丑,青龍見元城縣界井中。秋七月己卯,衞將軍胡遵薨。

癸未,安西將軍鄧艾大破蜀大將姜維於上邽,詔曰:「兵未極武,醜虜摧破,斬首獲生,動以萬計,自頃戰克,無如此者。今遣使者犒賜將士,大會臨饗,飲宴終日,稱朕意焉。」

八月庚午,命大將軍司馬文王加號大都督,奏事不名,假黃鉞。癸酉,以太尉司馬孚為太傅。九月,以司徒高柔為太尉。冬十月,以司空鄭沖為司徒,尚書左僕射盧毓為司空。

二年春二月,青龍見溫縣井中。三月,司空盧毓薨。

夏四月癸卯,詔曰:「玄菟郡高顯縣吏民反叛,長鄭熙為賊所殺。民王簡負擔熙喪,晨夜星行,遠致本州,忠節可嘉。其特拜簡為忠義都尉,以旌殊行。」

甲子,以征東大將軍諸葛誕為司空。

五月辛未,帝幸辟雍,會命羣臣賦詩。侍中和逌、尚書陳騫等作詩稽留,有司奏免官,詔曰:「吾以暗昧,愛好文雅,廣延詩賦,以知得失,而乃爾紛紜,良用反仄。其原逌等。主者宜勑自今以後,羣臣皆當玩習古義,脩明經典,稱朕意焉。」

乙亥,諸葛誕不就徵,發兵反,殺揚州刺史樂綝。丙子,赦淮南將吏士民為誕所詿誤者。丁丑,詔曰:「諸葛誕造為凶亂,盪覆揚州。昔黥布逆叛,漢祖親戎,隗嚻違戾,光武西伐,及烈祖明皇帝躬征吳、蜀,皆所以奮揚赫斯,震耀威武也。今宜皇太后與朕暫共臨戎,速定醜虜,時寧東夏。」己卯,詔曰:「諸葛誕造構逆亂,迫脅忠義,平寇將軍臨渭亭侯龐會、騎督偏將軍路蕃,各將左右,斬門突出,忠壯勇烈,所宜嘉異。其進會爵鄉侯,蕃封亭侯。」

六月乙巳,詔:「吳使持節都督夏口諸軍事鎮軍將軍沙羡侯孫壹,賊之枝屬,位為上將,畏天知命,深鑒禍福,翻然舉衆,遠歸大國,雖微子去殷,樂毅遁燕,無以加之。其以壹為侍中車騎將軍、假節、交州牧、吳侯,開府辟召儀同三司,依古侯伯八命之禮,衮冕赤舄,事從豐厚。」臣松之以為壹畏逼歸命,事無可嘉,格以古義,欲蓋而名彰者也。當時之宜,未得遠遵式典,固應量才受賞,足以疇其來情而已。至乃光錫八命,禮同台鼎,不亦過乎!於招攜致遠,又無取焉。何者?若使彼之將守,與時無嫌,終不恱於殊寵,坐生叛心,以叛而愧,辱孰甚焉?如其憂危將及,非奔不免,則必逃死苟存,無希榮利矣,然則高位厚祿何為者哉?魏初有孟達、黃權,在晉有孫秀、孫楷;達、權爵賞比壹為輕,秀、楷禮秩優異尤甚。及至吳平,而降黜數等,不承權輿,豈不緣在始失中乎?

甲子,詔曰:「今車駕駐項,大將軍恭行天罰,前臨淮浦。昔相國大司馬征討,皆與尚書俱行,今宜如舊。」乃令散騎常侍裴秀、給事黃門侍郎鍾會咸與大將軍俱行。秋八月,詔曰:「昔燕刺王謀反,韓誼等諫而死,漢朝顯登其子。諸葛誕創造凶亂,主簿宣隆、部曲督秦絜秉節守義,臨事固爭,為誕所殺,所謂無比干之親而受其戮者。其以隆、絜子為騎都尉,加以贈賜,光示遠近,以殊忠義。」

九月,大赦。冬十二月,吳大將全端、全懌等率衆降。

三年春二月,大將軍司馬文王陷壽春城,斬諸葛誕。三月,詔曰:「古者克敵,收其屍以為京觀,所以懲昏逆而章武功也。漢孝武元鼎中,改桐鄉為聞喜,新鄉為獲嘉,以著南越之亡。大將軍親總六戎,營據丘頭,內夷羣凶,外殄寇虜,功濟兆民,聲振四海。克敵之地,宜有令名,其改丘頭為武丘,明以武平亂,後世不忘,亦京觀二邑之義也。」

夏五月,命大將軍司馬文王為相國,封晉公,食邑八郡,加之九錫,文王前後九讓乃止。

六月丙子,詔曰:「昔南陽郡山賊擾攘,欲劫質故太守東里衮,功曹應余獨身捍衮,遂免於難。余顛沛殞斃,殺身濟君。其下司徒,署余孫倫吏,使蒙伏節之報。」楚國先賢傳曰:余字子正,天姿方毅,志尚仁義,建安二十三年為郡功曹。是時吳、蜀不賔,疆埸多虞。宛將侯音扇動山民,保城以叛。余與太守東里衮當擾攘之際、迸竄得出。音即遣騎追逐,去城十里相及,賊便射衮,飛矢交流。余前以身當箭,被七創,因謂追賊曰:「侯音狂狡,造為凶逆,大軍尋至,誅夷在近。謂卿曹本是善人,素無惡心,當思反善,何為受其指揮?我以身代君,已被重創,若身死君全,隕沒無恨。」因仰天號哭泣涕,血淚俱下。賊見其義烈,釋衮不害。賊去之後,余亦命絕。征南將軍曹仁討平音,表余行狀,并脩祭醊。太祖聞之,嗟嘆良乆,下荊州復表門閭,賜穀千斛。衮後為于禁司馬,見魏略游說傳。

辛卯,大論淮南之功,封爵行賞各有差。

秋八月甲戌,以驃騎將軍王昶為司空。丙寅,詔曰:「夫養老興教,三代所以樹風化垂不朽也,必有三老、五更以崇至敬,乞言納誨,著在惇史,然後六合承流,下觀而化。宜妙簡德行,以充其選。關內侯王祥,履仁秉義,雅志淳固。關內侯鄭小同,溫恭孝友,帥禮不忒。其以祥為三老,小同為五更。」車駕親率羣司,躬行古禮焉。漢晉春秋曰:帝乞言於祥,祥對曰:「昔者明王禮樂旣備,加之以忠誠,忠誠之發,形于言行。夫大人者,行動乎天地;天且弗違,況於人乎?」祥事別見呂虔傳。小同,鄭玄孫也。玄別傳曰:「玄有子,為孔融吏,舉孝廉。融之被圍,往赴,為賊所害。有遺腹子,以丁卯日生;而玄以丁卯歲生,故名曰小同。」魏名臣奏載太尉華歆表曰:「臣聞勵俗宣化,莫先於表善,班祿敘爵,莫美於顯能,是以楚人思子文之治,復命其胤,漢室嘉江公之德,用顯其世。伏見故漢大司農北海鄭玄,當時之學,名冠華夏,為世儒宗。文皇帝旌錄先賢,拜玄適孫小同以為郎中,長假在家。小同年踰三十,少有令質,學綜六經,行著鄉邑。海、岱之人莫不嘉其自然,美其氣量。迹其所履,有質直不渝之性,然而恪恭靜默,色養其親,不治可見之美,不競人間之名,斯誠清時所宜式敘,前後明詔所斟酌而求也。臣老病委頓,無益視聽,謹具以聞。」魏氏春秋曰:小同詣司馬文王,文王有密疏,未之屏也。如廁還,謂之曰:「卿見吾疏乎?」對曰:「否。」文王猶疑而鴆之,卒。鄭玄注文王世子曰「三老、五更各一人,皆年老更事致仕者也」。注樂記曰「皆老人更知三德五事者也」。蔡邕明堂論云:「更」應作「叟」。叟,長老之稱,字與「更」相似,書者遂誤以為「更」。「嫂」字「女」傍「叟」,今亦以為「更」,以此驗知應為「叟」也。臣松之以為邕謂「更」為「叟」,誠為有似,而諸儒莫之從,未知孰是。

是歲,青龍、黃龍仍見頓丘、冠軍、陽夏縣界井中。

四年春正月,黃龍二,見寧陵縣界井中。漢晉春秋曰:是時龍仍見,咸以為吉祥。帝曰:「龍者,君德也。上不在天,下不在田,而數屈於井,非嘉兆也。」仍作潛龍之詩以自諷,司馬文王見而惡之。夏六月,司空王昶薨。秋七月,陳留王峻薨。冬十月丙寅,分新城郡,復置上庸郡。十一月癸卯,車騎將軍孫壹為婢所殺。

五年春正月朔,日有蝕之。夏四月,詔有司率遵前命,復進大將軍司馬文王位為相國,封晉公,加九錫。

五月己丑,高貴鄉公卒,年二十。漢晉春秋曰:帝見威權日去,不勝其忿。乃召侍中王沈、尚書王經、散騎常侍王業,謂曰:「司馬昭之心,路人所知也。吾不能坐受廢辱,今日當與卿等自出討之。」王經曰:「昔魯昭公不忍季氏,敗走失國,為天下笑。今權在其門,為日乆矣,朝廷四方皆為之致死,不顧逆順之理,非一日也。且宿衞空闕,兵甲寡弱,陛下何所資用,而一旦如此,無乃欲除疾而更深之邪!禍殆不測,宜見重詳。」帝乃出懷中版令投地,曰:「行之決矣。正使死,何所懼?況不必死邪!」於是入白太后,沈、業奔走告文王,文王為之備。帝遂帥僮僕數百,鼓譟而出。文王弟屯騎校尉伷入,遇帝於東止車門,左右呵之,伷衆奔走。中護軍賈充又逆帝戰於南闕下,帝自用劒。衆欲退,太子舍人成濟問充曰:「事急矣。當云何?」充曰:「畜養汝等,正謂今日。今日之事,無所問也。」濟即前刺帝,刃出於背。文王聞,大驚,自投於地曰:「天下其謂我何!」太傅孚奔往,枕帝股而哭,哀甚,曰:「殺陛下者,臣之罪也。」臣松之以為習鑿齒書,雖最後出,然述此事差有次第。故先載習語,以其餘所言微異者次其後。世語曰:王沈、王業馳告文王,尚書王經以正直不出,因沈、業申意。晉諸公贊曰:沈、業將出,呼王經。經不從,曰:「吾子行矣!」干寶晉紀曰:成濟問賈充曰:「事急矣。若之何?」充曰:「公畜養汝等,為今日之事也。夫何疑!」濟曰:「然。」乃抽戈犯蹕。魏氏春秋曰:戊子夜,帝自將宂從僕射李昭、黃門從官焦伯等下陵雲臺,鎧仗授兵,欲因際會,自出討文王。會雨,有司奏却日,遂見王經等出黃素詔於懷曰:「是可忍也,孰不可忍也!今日便當決行此事。」入白太后,遂拔劒升輦,帥殿中宿衞蒼頭官僮擊戰鼓,出雲龍門。賈充自外而入,帝師潰散,猶稱天子,手劒奮擊,衆莫敢逼。充帥厲將士,騎督成倅弟成濟以矛進,帝崩于師。時暴雨雷霆,晦冥。魏末傳曰:賈充呼帳下督成濟謂曰:「司馬家事若敗,汝等豈復有種乎?何不出擊!」倅兄弟二人乃帥帳下人出,顧曰:「當殺邪?執邪?」充曰:「殺之。」兵交,帝曰:「放仗!」大將軍士皆放仗。濟兄弟因前刺帝,帝倒車下。皇太后令曰:「吾以不德,遭家不造,昔援立東海王子髦,以為明帝嗣,見其好書疏文章,兾可成濟,而情性暴戾,日月滋甚。吾數呵責,遂更忿恚,造作醜逆不道之言以誣謗吾,遂隔絕兩宮。其所言道,不可忍聽,非天地所覆載。吾即密有令語大將軍,不可以奉宗廟,恐顛覆社稷,死無面目以見先帝。大將軍以其尚幼,謂當改心為善,殷勤執據。而此兒忿戾,所行益甚,舉弩遙射吾宮,祝當令中吾項,箭親墮吾前。吾語大將軍,不可不廢之,前後數十。此兒具聞,自知罪重,便圖為弒逆,賂遺吾左右人,令因吾服藥,密行酖毒,重相設計。事已覺露,直欲因際會舉兵入西宮殺吾,出取大將軍,呼侍中王沈、散騎常侍王業、世語曰:業,武陵人,後為晉中護軍。尚書王經,出懷中黃素詔示之,言今日便當施行。吾之危殆,過於累卵。吾老寡,豈復多惜餘命邪?但傷先帝遺意不遂,社稷顛覆為痛耳。賴宗廟之靈,沈、業即馳語大將軍,得先嚴警,而此兒便將左右出雲龍門,雷戰鼓,躬自拔刃,與左右雜衞共入兵陣間,為前鋒所害。此兒旣行悖逆不道,而又自陷大禍,重令吾悼心不可言。昔漢昌邑王以罪廢為庶人,此兒亦宜以民禮葬之,當令內外咸知此兒所行。又尚書王經,凶逆無狀,其收經及家屬皆詣廷尉。」

庚寅,太傅孚、大將軍文王、太尉柔、司徒沖稽首言:「伏見中令,故高貴鄉公悖逆不道,自陷大禍,依漢昌邑王罪廢故事,以民禮葬。臣等備位,不能匡救禍亂,式遏姦逆,奉令震悚,肝心悼慄。春秋之義,王者無外,而書『襄王出居于鄭』,不能事母,故絕之於位也。今高貴鄉公肆行不軌,幾危社稷,自取傾覆,人神所絕,葬以民禮,誠當舊典。然臣等伏惟殿下仁慈過隆,雖存大義,猶垂哀矜,臣等之心實有不忍,以為可加恩以王禮葬之。」太后從之。漢晉春秋曰:丁卯,葬高貴鄉公于洛陽西北三十里瀍澗之濵。下車數乘,不設旌旐,百姓相聚而觀之,曰:「是前日所殺天子也。」或掩靣而泣,悲不自勝。臣松之以為若但下車數乘,不設旌旐,何以為王禮葬乎?斯蓋惡之過言,所謂不如是之甚者。

使使持節行中護軍中壘將軍司馬炎北迎常道鄉公璜嗣明帝後。辛卯,羣公奏太后曰:「殿下聖德光隆,寧濟六合,而猶稱令,與藩國同。請自今殿下令書,皆稱詔制,如先代故事。」

癸卯,大將車固讓相國、晉公、九錫之寵。太后詔曰:「夫有功不隱,周易大義,成人之美,古賢所尚,今聽所執,出表示外,以章公之謙光焉。」

戊申,大將軍文王上言:「高貴鄉公率將從駕人兵,拔刃鳴金鼓向臣所止;懼兵刃相接,即勑將士不得有所傷害,違令以軍法從事。騎督成倅弟太子舍人濟,橫入兵陣傷公,遂至隕命;輙收濟行軍法。臣聞人臣之節,有死無二,事上之義,不敢逃難。前者變故卒至,禍同發機,誠欲委身守死,唯命所裁。然惟本謀乃欲上危皇太后,傾覆宗廟。臣忝當大任,義在安國,懼雖身死,罪責彌重。欲遵伊、周之權,以安社稷之難,即駱驛申勑,不得迫近輦輿,而濟遽入陣間,以致大變。哀怛痛恨,五內摧裂,不知何地可以隕墜?科律大逆無道,父母妻子同產皆斬。濟凶戾悖逆,干國亂紀,罪不容誅。輒勑侍御史收濟家屬,付廷尉,結正其罪。」魏氏春秋曰:成濟兄弟不即伏罪,袒而升屋,醜言悖慢;自下射之,乃殪。太后詔曰:「夫五刑之罪,莫大於不孝。夫人有子不孝,尚告治之,此兒豈復成人主邪?吾婦人不達大義,以謂濟不得便為大逆也。然大將軍志意懇切,發言惻愴,故聽如所奏。當班下遠近,使知本末也。」世語曰:初,青龍中,石苞鬻鐵於長安,得見司馬宣王,宣王知焉。後擢為尚書郎,歷青州刺史、鎮東將軍。甘露中入朝,當還,辭高貴鄉公,留中盡日。文王遣人要令過。文王問苞:「何淹留也?」苞曰:「非常人也。」明日發至滎陽,數日而難作。

六月癸丑,詔曰:「古者人君之為名字,難犯而易諱。今常道鄉公諱字甚難避,其朝臣博議改易,列奏。」


\end{pinyinscope}