\article{杜微傳}

\begin{pinyinscope}
杜微字國輔,梓潼涪人也。少受學於廣漢任安。劉璋辟為從事,以疾去官。及先主定蜀,微常稱聾,閉門不出。建興二年,丞相亮領益州牧,選迎皆妙簡舊德,以秦宓為別駕,五梁為功曹,微為主簿。微固辭,轝而致之。旣至,亮引見微,微自陳謝。高以微不聞人語,於坐上與書曰:「服聞德行,饑渴歷時,清濁異流,無緣咨覯。王元泰、李伯仁、王文儀、楊季休、丁君幹、李永南兄弟、文仲寶等,每歎高志,未見如舊。猥以空虛,統領貴州,德薄任重,慘慘憂慮。朝廷主公今年始十八,天姿仁敏,愛德下士。天下之人思慕漢室,欲與君因天順民,輔此明主,以隆季興之功,著勳於竹帛也。以謂賢愚不相為謀,故自割絕,守勞而已,不圖自屈也。」微自乞老病求歸,亮又與書荅曰:「曹丕篡弒,自立為帝,是猶土龍芻狗之有名也。欲與羣賢因其邪偽,以正道滅之。怪君未有相誨,便欲求還於山野。丕又大興勞役,以向吳、楚。今因丕多務,且以境勤農育養民物,並治甲兵,以待其挫,然後伐之,可使兵不戰民不勞而天下定也。君但當以德輔時耳,不責君軍事,何為汲汲欲求去乎!」其敬微如此。拜為諫議大夫,以從其志。

五梁者,字德山,犍為南安人也,以儒學節操稱。從議郎遷諫議大夫、五官中郎將。


\end{pinyinscope}