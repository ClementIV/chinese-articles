\article{孫休傳}

\begin{pinyinscope}
孫休字子烈,權第六子。年十三,從中書郎射慈、郎中盛冲受學。太元二年正月,封琅邪王,居虎林。四月,權薨,休弟亮承統,諸葛恪秉政,不欲諸王在濵江兵馬之地,徙休於丹楊郡。太守李衡數以事侵休,休上書乞徙他郡,詔徙會稽。居數歲,夢乘龍上天,顧不見尾,覺而異之。孫亮廢,己未,孫綝使宗正孫楷與中書郎董朝迎休。休初聞問,意疑,楷、朝具述綝等所以奉迎本意,留一日二夜,遂發。十月戊寅,行至曲阿,有老公干休叩頭曰:「事乆變生,天下喁喁,願陛下速行。」休善之,是日進及布塞亭。武衞將軍恩行丞相事,率百僚以乘輿法駕迎於永昌亭,築宮,以武帳為便殿,設御座。己卯,休至,望便殿止住,使孫楷先見恩。楷還,休乘輦進,羣臣再拜稱臣。休升便殿,謙不即御坐,止東廂。戶曹尚書前即階下讚奏,丞相奉璽符。休三讓,羣臣三請。休曰:「將相諸侯咸推寡人,寡人敢不承受璽符。」羣臣以次奉引,休就乘輿,百官陪位,綝以兵千人迎於半野,拜于道側,休下車荅拜。即日,御正殿,大赦,改元。是歲,於魏甘露三年也。

永安元年冬十月壬午,詔曰:「夫襃德賞功,古今通義。其以大將軍綝為丞相、荊州牧,增食五縣。武衞將軍恩為御史大夫、衞將軍、中軍督,封縣侯。威遠將軍授為右將軍、縣侯。偏將軍幹雜號將軍、亭侯。長水校尉張布輔導勤勞,以布為輔義將軍,封永康侯。董朝親迎,封為鄉侯。」又詔曰:「丹陽太守李衡,以往事之嫌,自拘有司。夫射鉤斬袪,在君為君,遣衡還郡,勿令自疑。」

襄陽記曰:衡字叔平,本襄陽卒家子也,漢末入吳為武昌庶民。聞羊衜有人物之鑒,往干之,衜曰:「多事之世,尚書劇曹郎才也。」是時校事呂壹操弄權柄,大臣畏偪,莫有敢言,衜曰:「非李衡無能困之者。」遂共薦為郎。權引見,衡口陳壹姦短數千言,權有愧色。數月,壹被誅,而衡大見顯擢。後常為諸葛恪司馬,幹恪府事。恪被誅,求為丹楊太守。時孫休在郡治,衡數以法繩之。妻習氏每諫衡,衡不從。會休立,衡憂懼,謂妻曰:「不用卿言,以至於此。」遂欲奔魏。妻曰:「不可。君本庶民耳,先帝相拔過重,旣數作無禮,而復逆自猜嫌,逃叛求活,以此北歸,何面見中國人乎?」衡曰:「計何所出?」妻曰:「琅邪王素好善慕名,方欲自顯於天下,終不以私嫌殺君明矣。可自囚詣獄,表列前失,顯求受罪。如此,乃當逆見優饒,非但直活而已。」衡從之,果得無患,又加威遠將軍,援以棨戟。衡每欲治家,妻輙不聽,後密遣客十人於武陵龍陽汎洲上作宅,種甘橘千株。臨死,勑兒曰:「汝母惡我治家,故窮如是。然吾州里有千頭木奴,不責汝衣食,歲上一匹絹,亦可足用耳。」衡亡後二十餘日,兒以白母,母曰:「此當是種甘橘也,汝家失十戶客來七八年,必汝父遣為宅。汝父恒稱太史公言,『江陵千樹橘,當封君家』。吾荅曰:『且人患無德義,不患不富,若貴而能貧,方好耳,用此何為!』」吳末,衡甘橘成,歲得絹數千匹,家道殷足。晉咸康中,其宅止枯樹猶在。己丑,封孫皓為烏程侯,皓弟德錢唐侯,謙永安侯。江表傳曰:羣臣奏立皇后、太子,詔曰:「朕以寡德,奉承洪業,莅事日淺,恩澤未敷,加后妃之號,嗣子之位,非所急也。」有司又固請,休謙虛不許。

十一月甲午,風四轉五復,蒙霧連日。綝一門五侯皆典禁兵,權傾人主,有所陳述,敬而不違,於是益恣。休恐其有變,數加賞賜。丙申,詔曰:「大將軍忠款內發,首建大計以安社稷,卿士內外,咸贊其議,並有勳勞。昔霍光定計,百僚同心,無復是過。亟案前日與議定策告廟人名,依故事應加爵位者,促施行之。」戊戌,詔曰:「大將軍掌中外諸軍事,事統煩多,其加衞將軍御史大夫恩侍中,與大將軍分省諸事。」壬子,詔曰:「諸吏家有五人三人兼重為役,父兄在都,子弟給郡縣吏,旣出限米,軍出又從,至於家事無經護者,朕甚愍之。其有五人三人為役,聽其父兄所欲留,為留一人,除其米限,軍出不從。」又曰:「諸將吏奉迎陪位在永昌亭者,皆加位一級。」頃之,休聞綝逆謀,陰與張布圖計。十二月戊辰臘,百僚朝賀,公卿升殿,詔武士縛綝,即日伏誅。己巳,詔以左將軍張布討姦臣,加布為中軍督,封布弟惇為都亭侯,給兵三百人,惇弟恂為校尉。

詔曰:「古者建國,教學為先,所以道世治性,為時養器也。自建興以來,時事多故,吏民頗以目前趨務,去本就末,不循古道。夫所尚不惇,則傷化敗俗。其案古置學官,立五經博士,核取應選,加其寵祿,科見吏之中及將吏子弟有志好者,各令就業。一歲課試,差其品第,加以位賞。使見之者樂其榮,聞之者羨其譽。以敦王化,以隆風俗。」

二年春正月,震電。三月,備九卿官,詔曰:「朕以不德,託于王公之上,夙夜戰戰,忘寢與食。今欲偃武脩文,以崇大化。推此之道,當由士民之贍,必須農桑。管子有言:『倉廩實,知禮節;衣食足,知榮辱。』夫一夫不耕,有受其饑,一婦不織,有受其寒;饑寒並至而民不為非者,未之有也。自頃年已來,州郡吏民及諸營兵,多違此業,皆浮船長江,賈作上下,良田漸廢,見穀日少,欲求大定,豈可得哉?亦由租入過重,農人利薄,使之然乎!今欲廣開田業,輕其賦稅,差科彊羸,課其田畝,務令優均,官私得所,使家給戶贍,足相供養,則愛身重命,不犯科法,然後刑罰不用,風俗可整。以羣僚之忠賢,若盡心於時,雖太古盛化,未可卒致,漢文升平,庶幾可及。及之則臣主俱榮,不及則損削侵辱,何可從容俯仰而已?諸卿尚書,可共咨度,務取便佳。田桑已至,不可後時。事定施行,稱朕意焉。」

三年春三月,西陵言赤烏見。秋,用都尉嚴密議,作浦里塘。會稽郡謠言王亮當還為天子,而亮宮人告亮使巫禱祠,有惡言。有司以聞,黜為候官侯,遣之國。道自殺,衞送者伏罪。吳錄曰:或云休鴆殺之。至晉太康中,吳故少府丹楊戴顒迎亮喪,葬之賴鄉。以會稽南部為建安郡,分宜都置建平郡。吳歷曰:是歲得大鼎於建德縣。

四年夏五月,大雨,水泉涌溢。秋八月,遣光祿大夫周弈、石偉巡行風俗,察將吏清濁,民所疾苦,為黜陟之詔。楚國先賢傳曰:石偉字公操,南郡人。少好學,脩節不怠,介然獨立,有不可奪之志。舉茂才、賢良方正,皆不就。孫休即位,特徵偉,累遷至光祿勳。及皓即位,朝政昏亂,偉乃辭老耄痼疾乞身,就拜光祿大夫。吳平,建威將軍王戎親詣偉。太康二年,詔曰:「吳故光祿大夫石偉,秉志清白,皓首不渝,難處危亂,廉節可紀。年已過邁,不堪遠涉,其以偉為議郎,加二千石秩,以終厥世。」偉遂陽狂及盲,不受晉爵。年八十三,太熈元年卒。九月,布山言白龍見。是歲,安吳民陳焦死,埋之,六日更生,穿土中出。

五年春二月,白虎門北樓災。秋七月,始新言黃龍見。八月壬午,大雨震電,水泉涌溢。乙酉,立皇后朱氏。戊子,立子𩅦為太子,大赦。吳錄載休詔曰:「人之有名,以相紀別,長為作字,憚其名耳。禮,名子欲令難犯易避,五十稱伯仲,古或一字。今人競作好名好字,又令相配,所行不副,此瞽字伯明者也,孤常哂之。或師友父兄所作,或自己為;師友尚可,父兄猶非,自為最不謙。孤今為四男作名字:太子名𩅦,𩅦音如湖水灣澳之灣,字莔,莔音如迄今之迄;次子名𩃙,𩃙音如兕觥之觥,字𧟨,𧟨音如玄礥首之礥;次子名壾,壾音如草莽之莽,字昷,昷音如舉物之舉;次子名𠅬,𠅬音如襃衣下寬大之襃,字㷏,㷏音如有所擁持之擁。此都不與世所用者同,故鈔舊文會合作之。夫書八體損益,因事而生,今造此名字,旣不相配,又字但一,庶易棄避,其普告天下,使咸聞知。」臣松之以為傳稱「名以制義,義以出禮,禮以體政,政以正民。是以政成而民聽,易則生亂」。斯言之作,豈虛也哉!休欲令難犯,何患無名,而乃造無況之字,制不典之音,違明誥於前脩,垂嗤騃於後代,不亦異乎!是以墳土未乾而妻子夷滅。師服之言,於是乎徵矣。冬十月,以衞將軍濮陽興為丞相,廷尉丁密、光祿勳孟宗為左右御史大夫。休以丞相興及左將軍張布有舊恩,委之以事,布典宮省,興關軍國。休銳意於典籍,欲畢覽百家之言,尤好射雉,春夏之間常晨出夜還,唯此時舍書。休欲與愽士祭酒韋曜、博士盛沖講論道藝,曜、沖素皆切直,布恐入侍,發其陰失,令己不得專,因妄飾說以拒遏之。休荅曰:「孤之涉學,羣書略徧,所見不少也;其明君闇王,姦臣賊子,古今賢愚成敗之事,無不覽也。今曜等入,但欲與論講書耳,不為從曜等始更受學也。縱復如此,亦何所損?君特當以曜等恐道臣下姦變之事,以此不欲令入耳。如此之事,孤已自備之,不須曜等然後乃解也。此都無所損,君意特有所忌故耳。」布得詔陳謝,重自序述,又言懼妨政事。休荅曰:「書籍之事,患人不好,好之無傷也。此無所為非,而君以為不宜,是以孤有所及耳。王務學業,其流各異,不相妨也。不圖君今日在事,更行此於孤也,良所不取。」布拜表叩頭,休荅曰:「聊相開悟耳,何至叩頭乎!如君之忠誠,遠近所知。往者所以相感,今日之巍巍也。詩云:『靡不有初,鮮克有終。』終之實難,君其終之。」初休為王時,布為左右將督,素見信愛,及至踐阼,厚加寵待,專擅國勢,多行無禮,自嫌瑕短,懼曜、沖言之,故尤患忌。休雖解此旨,心不能恱,更恐其疑懼,竟如布意,廢其講業,不復使沖等入。是歲使察戰到交阯調孔爵、大豬。臣松之案:察戰吳官號,今揚都有察戰巷。

六年夏四月,泉陵言黃龍見。五月,交阯郡吏呂興等反,殺太守孫諝。諝先是科郡上手工千餘人送建業,而察戰至,恐復見取,故興等因此扇動兵民,招誘諸夷也。冬十月,蜀以魏見伐來告。癸未,建業石頭小城火,燒西南百八十丈。甲申,使大將軍丁奉督諸軍向魏壽春,將軍留平別詣施績於南郡,議兵所向,將軍丁封、孫異如沔中,皆救蜀。蜀主劉禪降魏問至,然後罷。呂興旣殺孫諝,使使如魏,請太守及兵。丞相興建取屯田萬人以為兵。分武陵為天門郡。吳歷曰:是歲青龍見於長沙,白燕見於慈胡,赤雀見於豫章。

七年春正月,大赦。二月,鎮軍陸抗、撫軍步恊、征西將軍留平、建平太守盛曼,率衆圍蜀巴東守將羅憲。夏四月,魏將新附督王稚浮海入句章,略長吏賞林及男女二百餘口。將軍孫越徼得一船,獲三十人。秋七月,海賊破海鹽,殺司鹽校尉駱秀。使中書郎劉川發兵廬陵。豫章民張節等為亂,衆萬餘人。魏使將軍胡烈步騎二萬侵西陵,以救羅憲,陸抗等引軍退。復分交州置廣州。壬午,大赦。癸未,休薨,江表傳曰:休寢疾,口不能言,乃手書呼丞相濮陽興入,令子𩅦出拜之。休把興臂,而指𩅦以託之。時年三十,謚曰景皇帝。葛洪抱朴子曰:吳景帝時,戍將於廣陵掘諸冢,取版以治城,所壞甚多。後發一大冢,內有重閤,戶扇皆樞轉可開閉,四周為徼道通車,其高可以乘馬。又鑄銅為人數十枚,長五尺,皆大冠朱衣,執劒列侍靈座,皆刻銅人背後石壁,言殿中將軍,或言侍郎、常侍。似公王之冢。破其棺,棺中有人,髮已班白,衣冠鮮明,面體如生人。棺中雲母厚尺許,以白玉璧三十枚藉尸。兵人輩共舉出死人,以倚冢壁。有一玉長一尺許,形似冬瓜,從死人懷中透出墮地。兩耳及鼻孔中,皆有黃金如棗許大,此則骸骨有假物而不朽之効也。


\end{pinyinscope}