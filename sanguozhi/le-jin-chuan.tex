\article{le-jin-chuan}

\begin{pinyinscope}
樂進字文謙,陽平衞國人也。容貌短小,以膽烈從太祖,為帳下吏。遣還本郡募兵,得千餘人,還為軍假司馬、陷陣都尉。從擊呂布於濮陽,張超於雍丘,橋蕤於苦,皆先登有功,封廣昌亭侯。從征張繡於安衆,圍呂布於下邳,破別將,擊眭固於射犬,攻劉備於沛,皆破之,拜討寇校尉。渡河攻獲嘉,還,從擊袁紹於官渡,力戰,斬紹將淳于瓊。從擊譚、尚於黎陽,斬其大將嚴敬,行游擊將軍。別擊黃巾,破之,定樂安郡。從圍鄴,鄴定,從擊袁譚於南皮,先登,入譚東門。譚敗,別攻雍奴,破之。建安十一年,太祖表漢帝,稱進及于禁、張遼曰:「武力旣弘,計略周備,質忠信一,守執節義,每臨戰攻,常為督率,奮彊突固,無堅不陷,自援枹鼓,手不知倦。又遣別征,統御師旅,撫衆則和,奉令無犯,當敵制決,靡有遺失。論功紀用,宜各顯寵。」於是禁為虎威;進,折衝;遼,盪寇將軍。

進別征高幹,從北道入上黨,回出其後。幹等還守壺關,連戰斬首。幹堅守未下,會太祖自征之,乃拔。太祖征管承,軍淳于,遣進與李典擊之。承破走,逃入海島,海濵平,荊州未服,遣屯陽翟。後從平荊州,留屯襄陽,擊關羽、蘇非等,皆走之,南郡諸縣山谷蠻夷詣進降。又討劉備臨沮長杜普、旌陽長梁大,皆大破之。後從征孫權,假進節。太祖還,留進與張遼、李典屯合肥,增邑五百,并前凡千二百戶。以進數有功,分五百戶,封一子列侯;進遷右將軍。建安二十三年薨,謚曰威侯。子綝嗣。綝果毅有父風,官至揚州刺史。諸葛誕反,掩襲殺綝,詔悼惜之,追贈衞尉,謚曰愍侯。子肇嗣。


\end{pinyinscope}