\article{朱桓傳}

\begin{pinyinscope}
子異附

朱桓字休穆,吳郡吳人也。孫權為將軍,桓給事幕府,除餘姚長。往遇疫癘,穀食荒貴,桓分部良吏,隱親醫藥,飱粥相繼,士民感戴之。遷盪寇校尉,授兵二千人,使部伍吳、會二郡,鳩合遺散,期年之間,得萬餘人。後丹楊、鄱陽山賊蜂起,攻沒城郭,殺略長吏,處處屯聚。桓督領諸將,周旋赴討,應皆平定。稍遷裨將軍,封新城亭侯。

後代周泰為濡須督。黃武元年,魏使大司馬曹仁步騎數萬向濡須,仁欲以兵襲取州上,偽先揚聲,欲東攻羨溪。桓分兵將赴羨溪,旣發,卒得仁進軍拒濡須七十里間。桓遣使追還羨溪兵,兵未到而仁奄至。時桓手下及所部兵,在者五千人,諸將業業,各有懼心,桓喻之曰:「凡兩軍交對,勝負在將,不在衆寡。諸君聞曹仁用兵行師,孰與桓邪?兵法所以稱客倍而主人半者,謂俱在平原,無城池之守,又謂士衆勇怯齊等故耳。今人旣非智勇,加其士卒甚怯,又千里步涉,人馬罷困,桓與諸軍,共據高城,南臨大江,北背山陵,以逸待勞,為主制客,此百戰百勝之勢也。雖曹丕自來,尚不足憂,況仁等邪!」桓因偃旗鼓,外示虛弱,以誘致仁。仁果遣其子泰攻濡須城,分遣將軍常雕督諸葛虔、王雙等,乘油船別襲中洲。中洲者,部曲妻子所在也。仁自將萬人留橐臯,復為泰等後拒。桓部兵將攻取油船,或別擊雕等,桓等身自拒泰,燒營而退,遂梟雕,生虜雙,送武昌,臨陣斬溺,死者千餘。權嘉桓功,封嘉興侯,遷奮武將軍,領彭城相。

黃武七年,鄱陽太守周魴譎誘魏大司馬曹休,休將步騎十萬至皖城以迎魴。時陸遜為元帥,全琮與桓為左右督,各督三萬人擊休。休知見欺,當引軍還,自負衆盛,邀於一戰。桓進計曰:「休本以親戚見任,非智勇名將也。今戰必敗,敗必走,走當由夾石、挂車,此兩道皆險阨,若以萬兵柴路,則彼衆可盡,而休可生虜,臣請將所部以斷之。若蒙天威,得以休自效,便可乘勝長驅,進取壽春,割有淮南,以規許、洛,此萬世一時,不可失也。」權先與陸遜議,遜以為不可,故計不施行。

黃龍元年,拜桓前將軍,領青州牧,假節。嘉禾六年,魏廬江主簿呂習請大兵自迎,欲開門為應。桓與衞將軍全琮俱以師迎。旣至,事露,軍當引還。城外有溪水,去城一里所,廣三十餘丈,深者八九尺,淺者半之,諸軍勒兵渡去,桓自斷後。時廬江太守李膺整嚴兵騎,欲須諸軍半渡,因迫擊之。及見桓節蓋在後,卒不敢出,其見憚如此。

是時全琮為督,權又令偏將軍胡綜宣傳詔命,參與軍事。琮以軍出無獲,議欲部分諸將,有所掩襲。桓素氣高,恥見部伍,乃往見琮,問行意,感激發怒,與琮校計。琮欲自解,因曰:「上自令胡綜為督,綜意以為宜爾。」桓愈恚恨,還乃使人呼綜。綜至軍門,桓出迎之,顧謂左右曰:「我縱手,汝等各自去。」有一人旁出,語綜使還。桓出,不見綜,知左右所為,因斫殺之。桓佐軍進諫,刺殺佐軍,遂託狂發,詣建業治病。權惜其功能,故不罪。孫盛曰:書云臣無作威作福,作威作福,則凶于而家,害于而國。桓之賊忍,殆虎狼也,人君且猶不可,況將相乎?語曰,得一夫而失一國,縱罪虧刑,失孰大焉!使子異攝領部曲,令醫視護,數月復遣還中洲。權自出祖送,謂曰:「今寇虜尚存,王塗未一,孤當與君共定天下,欲令君督五萬人專當一靣,以圖進取,想君疾未復發也。」桓曰:「天授陛下聖姿,當君臨四海,猥重任臣,以除姧逆,臣疾當自愈。」吳錄曰:桓奉觴曰:「臣當遠去,願一捋陛下鬚,無所復恨。」權馮几前席,桓進前捋鬚曰:「臣今日真可謂捋虎鬚也。」權大笑。

桓性護前,恥為人下,每臨敵交戰,節度不得自由,輒嗔恚憤激。然輕財貴義,兼以彊識,與人一靣,數十年不忘,部曲萬口,妻子盡識之。愛養吏士,贍護六親,俸祿產業,皆與共分。及桓疾困,舉營憂戚。年六十二,赤烏元年卒。吏士男女無不號慕。又家無餘財,權賜鹽五十斛以周喪事。子異嗣。

異字季文,以父任除郎,文士傳曰:張惇子純與張儼及異俱童少,往見驃騎將軍朱據。據聞三人才名,欲試之,告曰:「老鄙相聞,饑渴甚矣。夫騕䮍以迅驟為功,鷹隼以輕疾為妙,其為吾各賦一物,然後乃坐。」儼乃賦犬曰:「守則有威,出則有獲,韓盧、宋鵲,書名竹帛。」純賦席曰:「席以冬設,簟為夏施,揖讓而坐,君子攸宜。」異賦弩曰:「南嶽之幹,鍾山之銅,應機命中,獲隼高墉。」三人各隨其目所見而賦之,皆成而後坐,據大歡恱。後拜騎都尉,代桓領兵。赤烏四年,隨朱然攻魏樊城,建計破其外圍,還拜偏將軍。魏廬江太守文欽營住六安,多設屯砦,置諸道要,以招誘亡叛,為邊寇害。異乃身率其手下二千人,掩破欽七屯,斬首數百,遷揚武將軍。權與論攻戰,辭對稱意。權謂異從父驃騎將軍據曰:「本知季文懀定,見之復過所聞。」十三年,文欽詐降,密書與異,欲令自迎。異表呈欽書,因陳其偽,不可便迎。權詔曰:「方今北土未一,欽云欲歸命,宜且迎之。若嫌其有譎者,但當設計網以羅之,盛重兵以防之耳。」乃遣呂據督二萬人,與異并力,至北界,欽果不降。建興元年,遷鎮南將軍。是歲魏遣胡遵、諸葛誕等出東興,異督水軍攻浮梁,壞之,魏軍大破。吳書曰:異又隨諸葛恪圍新城,城旣不拔,異等皆言宜速還豫章,襲石頭城,不過數日可拔。恪以書曉異,異投書於地曰:「不用我計,而用傒子言!」恪大怒,立奪其兵,遂廢還建業。太平二年,假節,為大都督,救壽春圍,不解。還軍,為孫綝所枉害。吳書曰:綝要異相見,將往,恐陸抗止之,異曰:「子通,家人耳,當何所疑乎!」遂往。綝使力人於坐上取之。異曰:「我吳國忠臣,有何罪乎?」乃拉殺之。

評曰:朱治、呂範以舊臣任用,朱然、朱桓以勇烈著聞,呂據、朱異、施績咸有將領之才,克紹堂構。若範、桓之越隘,得以吉終,至於據、異無此之尤而反罹殃者,所遇之時殊也。


\end{pinyinscope}