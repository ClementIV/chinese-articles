\article{cao-zhen-chuan}

\begin{pinyinscope}
曹真字子丹,太祖族子也。太祖起兵,真父邵募徒衆,為州郡所殺。

魏略曰:真本姓秦,養曹氏。或云其父伯南夙與太祖善。興平末,袁術部黨與太祖攻劫,太祖出,為寇所追,走入秦氏,伯南開門受之。寇問太祖所在,荅云:「我是也。」遂害之。由此太祖思其功,故變其姓。魏書曰:邵以忠篤有才智,為太祖所親信。初平中,太祖興義兵,邵募徒衆,從太祖周旋。時豫州刺史黃琬欲害太祖,太祖避之而邵獨遇害。太祖哀真少孤,收養與諸子同,使與文帝共止。常獵,為虎所逐,顧射虎,應聲而倒。太祖壯其鷙勇,使將虎豹騎。討靈丘賊,拔之,封靈壽亭侯。以偏將軍將兵擊劉備別將於下辯,破之,拜中堅將軍。從至長安,領中領軍。是時,夏侯淵沒於陽平,太祖憂之。以真為征蜀護軍,督徐晃等破劉備別將高詳於陽平。太祖自至漢中,拔出諸軍,使真至武都迎曹洪等還屯陳倉。文帝即王位,以真為鎮西將軍,假節都督雍、涼州諸軍事。錄前後功,進封東鄉侯。張進等反於酒泉,真遣費耀討破之,斬進等。黃初三年還京都,以真為上軍大將軍,都督中外諸軍事,假節鉞。與夏侯尚等征孫權,擊牛渚屯,破之。轉拜中軍大將軍,加給事中。七年,文帝寢疾,真與陳羣、司馬宣王等受遺詔輔政。明帝即位,進封邵陵侯,臣松之案:真父名邵。封邵陵侯,若非書誤,則事不可論。遷大將軍。

諸葛亮圍祁山,南安、天水、安定三郡反應亮。帝遣真督諸軍軍郿,遣張郃擊亮將馬謖,大破之。安定民楊條等略吏民保月支城,真進軍圍之。條謂其衆曰:「大將軍自來,吾願早降耳。」遂自縛出。三郡皆平。真以亮懲於祁山,後出必從陳倉,乃使將軍郝昭、王生守陳倉,治其城。明年春,亮果圍陳倉,已有備而不能克。增邑,并前二千九百戶。四年,朝洛陽,遷大司馬,賜劒履上殿,入朝不趨。真以「蜀連出侵邊境,宜遂伐之。數道並入,可大克也」。帝從其計。真當發西討,帝親臨送。真以八月發長安,從子午道南入。司馬宣王泝漢水,當會南鄭。諸軍或從斜谷道,或從武威入。會大霖雨三十餘日,或棧道斷絕,詔真還軍。

真少與宗人曹遵、鄉人朱讚並事太祖。遵、讚早亡,真愍之,乞分所食邑封遵、讚子。詔曰:「大司馬有叔向撫孤之仁,篤晏平乆要之分。君子成人之羙,聽分真邑賜遵、讚子爵關內侯,各百戶。」真每征行,與將士同勞苦,軍賞不足,輒以家財班賜,士卒皆願為用。真病還洛陽,帝自幸其第省疾。真薨,謚曰元侯。子爽嗣。帝追思真功,詔曰:「大司馬蹈履忠節,佐命二祖,內不恃親戚之寵,外不驕白屋之士,可謂能持盈守位,勞謙其德者也。其悉封真五子羲、訓、則、彥、皚皆為列侯。」初,文帝分真邑二百戶,封真弟彬為列侯。


\end{pinyinscope}