\article{王淩}

\begin{pinyinscope}
王淩字彥雲,太原祁人也。叔父允,為漢司徒,誅董卓。卓將李傕、郭汜等為卓報仇,入長安,殺允,盡害其家。淩及兄晨,時年皆少,踰城得脫,亡命歸鄉里。淩舉孝廉,為發干長,

魏略曰:淩為長,遇事,髠刑五歲,當道埽除。時太祖車過,問此何徒,左右以狀對。太祖曰;「此子師兄子也,所坐亦公耳。」於是主者選為驍騎主簿。稍遷至中山太守,所在有治,太祖辟為丞相掾屬。

文帝踐阼,拜散騎常侍,出為兖州刺史,與張遼等至廣陵討孫權。臨江,夜大風,吳將呂範等船漂至北岸。淩與諸將逆擊,捕斬首虜,獲舟船,有功,封宜城亭侯,加建武將軍,轉在青州。是時海濵乘喪亂之後,法度未整。淩布政施教,賞善罰惡,甚有綱紀,百姓稱之,不容於口。後從曹休征吳,與賊遇於夾石,休軍失利,淩力戰決圍,休得免難。仍徙為揚、豫州刺史,咸得軍民之歡心。始至豫州,旌先賢之後,求未顯之士,各有條教,意義甚美。初,淩與司馬朗、賈逵友善,及臨兖、豫,繼其名跡。正始初,為征東將軍,假節都督揚州諸軍事。二年,吳大將全琮數萬衆寇芍陂,淩率諸軍逆討,與賊爭塘,力戰連日,賊退走。進封南鄉侯,邑千三百五十戶,遷車騎將軍、儀同三司。

是時,淩򠐁甥令狐愚以才能為兖州刺史,屯平阿。舅甥並典兵,專淮南之重。淩就遷為司空。司馬宣王旣誅曹爽,進淩為太尉,假節鉞。淩、愚密協計,謂齊王不任天位,楚王彪長而才,欲迎立彪都許昌。嘉平元年九月,愚遣將張式至白馬,與彪相問往來。淩又遣舍人勞精詣洛陽,語子廣。廣言:「廢立大事,勿為禍先。」漢晉春秋曰:淩、愚謀,以帝幼制於彊臣,不堪為主,楚王彪長而才,欲迎立之,以興曹氏。淩使人告廣,廣曰:「凡舉大事,應本人情。今曹爽以驕奢失民,何平叔虛而不治,丁、畢、桓、鄧雖並有宿望,皆專競於世。加變易朝典,政令數改,所存雖高而事不下接,民習於舊,衆莫之從。故雖勢傾四海,聲震天下,同日斬戮,名士減半,而百姓安之,莫或之哀,失民故也。今懿情雖難量,事名有逆,而擢用賢能,廣樹勝己,脩先朝之政令,副衆心之所求。爽之所以為惡者,彼莫不必改,夙夜匪解,以恤民為先。父子兄弟並握兵要,未易亡也。」淩不從。臣松之以為如此言之類,皆前史所不載,而猶出習氏。且制言法體不似於昔,疑悉鑿齒所自造者也。其十一月,愚復遣式詣彪,未還,會愚病死。魏書曰:愚字公浩,本名浚,黃初中,為和戎護軍。烏丸校尉田豫討胡有功,小違節度,愚以法繩之。帝怒,械繫愚,免官治罪,詔曰「浚何愚」!遂以名之。正始中,為曹爽長史,後出為兖州刺史。魏畧曰:愚聞楚王彪有智勇。初東郡有譌言云:「白馬河出妖馬,夜過官牧邊鳴呼,衆馬皆應,明日見其迹,大如斛,行數里,還入河中。」又有謠言:「白馬素羈西南馳,其誰乘者朱虎騎。」楚王小字朱虎,故愚與王淩陰謀立楚王。乃先使人通意於王,言「使君謝王,天下事不可知,願王自愛」!彪亦陰知其意,荅言「謝使君,知厚意也。」二年,熒惑守南斗,淩謂:「斗中有星,當有暴貴者。」魏畧曰:淩聞東平民浩詳知星,呼問詳。詳疑淩有所挾,欲恱其意,不言吳當有死喪,而言淮南楚分也,今吳、楚同占,當有王者興。故淩計遂定。三年春,吳賊塞涂水。淩欲因此發,大嚴諸軍,表求討賊;詔報不聽。淩陰謀滋甚,遣將軍楊弘以廢立事告兖州刺史黃華,華、弘連名以白太傅司馬宣王。宣王將中軍乘水道討淩,先下赦赦淩罪,又將尚書廣東,使為書喻淩,大軍掩至百尺逼淩。淩自知勢窮,乃乘船單出迎宣王,遣掾王彧謝罪,送印綬、節鉞。軍到丘頭,淩靣縛水次。宣王承詔遣主簿解縛反服,見淩,慰勞之,還印綬、節鉞,遣步騎六百人送還京都。淩至項,飲藥死。魏畧載淩與太傅書曰:「卒聞神軍密發,巳在百尺,雖知命窮盡,遲於相見,身首分離,不以為恨,前後遣使,有書未得還報,企踵西望,無物以譬。昨遣書之後,便乘船來相迎宿丘頭,旦發於浦口,奉被露布赦書,又得二十三日況,累紙誨示,聞命驚愕,五內失守,不知何地可以自處?僕乆忝朝恩,歷試無効,統御戎馬,董齊東夏,事有闕廢,中心犯義,罪在三百,妻子同縣,無所禱矣。不圖聖恩天覆地載,橫蒙視息,復覩日月。亡甥令狐愚攜惑群小之言,僕即時呵抑,使不得竟其語。旣人已知,神明所鑒,夫非事無陰,卒至發露,知此梟夷之罪也。生我者父母,活我者子也。」又重曰:「身陷刑罪,謬蒙赦宥。今遣掾送印綬,頃至,當如詔書自縛歸命。雖足下私之,官法有分。」及到,如書。太傅使人解其縛。淩旣蒙赦,加怙舊好,不復自疑,徑乘小船自趣太傅。太傅使人逆止之,住船淮中,相去十餘丈。淩知見外,乃遙謂太傅曰:「卿直以折簡召我,我當敢不至邪?而乃引軍來乎!」太傅曰:「以卿非肯逐折簡者故也。」淩曰:「卿負我!」太傅曰:「我寧負卿,不負國家。」遂使人送來西。淩自知罪重,試索棺釘,以觀太傅意,太傅給之。淩行到項,夜呼掾屬與決曰:「行年八十,身名並滅邪!」遂自殺。干寶晉紀曰:淩到項,見賈逵祠在水側,淩呼曰:「賈梁道,王淩固忠於魏之社稷者,唯尔有神,知之。」其年八月,太傅有疾,夢淩、逵為癘,甚惡之,遂薨。宣王遂至壽春。張式等皆自首,乃窮治其事。彪賜死,諸相連者悉夷三族。魏畧載:山陽單固,字恭夏,為人有器實。正始中,兖州刺史令狐愚與固父伯龍善,辟固,欲以為別駕。固不樂為州吏,辭以疾。愚禮意愈厚,固不欲應。固母夏侯氏謂固曰:「使君與汝父乆善,故命汝不止,汝亦故當仕進,自可往耳。」固不獲已,遂往,與兼治中從事楊康並為愚腹心。後愚與王淩通謀,康、固皆知其計。會愚病,康應司徒召詣洛陽,固亦以疾解祿。康在京師露其事,太傅乃東取王淩。到壽春,固見太傅,太傅問曰:「卿知其事為邪?」固對不知。太傅曰:「且置近事。問卿,令狐反乎?」固又曰無。而楊康白,事事與固連。遂收捕固及家屬,皆繫廷尉,考實數十,固故云無有。太傅錄楊康,與固對相詰。固辭窮,乃罵康曰:「老庸旣負使君,又滅我族,顧汝當活邪!」辭定,事上,須報廷尉,以舊皆聽得與其母妻子相見。固見其母,不仰視,其母知其慙也,字謂之曰:「恭夏,汝本自不欲應州郡也,我強故耳。汝為人吏,自當尔耳。此自門戶衰,我無恨也。汝本意與我語。」固終不仰,又不語,以至於死。初,楊康自以白其事,兾得封拜,後以辭頗參錯,亦并斬。臨刑,俱出獄,固又罵康曰:「老奴,汝死自分耳。若令死者有知,汝何靣目以行地下也。」朝議咸以為春秋之義,齊崔杼、鄭歸生皆加追戮,陳尸斲棺,載在方策。淩、愚罪宜如舊典。乃發淩、愚冢,剖棺,暴尸於所近市三日,燒其印綬、朝服,親土埋之。干寶晉紀曰:兖州武吏東平馬隆,託為愚家客,以私財更殯葬,行服三年,種植松栢。一州之士媿之。進弘、華爵為鄉侯。廣有志尚學行,死時年四十餘。魏氏春秋曰:廣字公淵。弟飛梟、金虎,並才武過人。太傅嘗從容問蔣濟,濟曰:「淩文武俱贍,當今無雙。廣等志力,有美於父耳。」退而悔之,告所親曰:「吾此言,滅人門宗矣。」魏末傳曰:淩少子字明山,最知名,善書,多技蓺,人得其書,皆以為法。走向太原,追軍及之,時有飛鳥集桑樹,隨枝低卬,舉弓射之即倒,追人乃止不復進。明山投親家食,親家告吏,乃就執之。


\end{pinyinscope}