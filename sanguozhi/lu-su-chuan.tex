\article{魯肅傳}

\begin{pinyinscope}
魯肅字子敬,臨淮東城人也。生而失父,與祖母居。家富於財,性好施與。爾時天下已亂,肅不治家事,大散財貨,摽賣田地,以賑窮弊結士為務,甚得鄉邑歡心。

周瑜為居巢長,將數百人故過候肅,并求資糧。肅家有兩囷米,各三千斛,肅乃指一囷與周瑜,瑜益知其奇也,遂相親結,定僑、札之分。袁術聞其名,就署東城長。肅見術無綱紀,不足與立事,乃攜老弱將輕俠少年百餘人,南到居巢就瑜。瑜之東渡,因與同行,

吳書曰:肅體貌魁奇,少有壯節,好為奇計。天下將亂,乃學擊劒騎射,招聚少年,給其衣食,往來南山中射獵,陰相部勒,講武習兵。父老咸曰:「魯氏世衰,乃生此狂兒!」後雄傑並起,中州擾亂,肅乃命其屬曰:「中國失綱,寇賊橫暴,淮、泗間非遺種之地,吾聞江東沃野萬里,民富兵彊,可以避害,寧肯相隨俱至樂土,以觀時變乎?」其屬皆從命。乃使細弱在前,彊壯在後,男女三百餘人行。州追騎至,肅等徐行,勒兵持滿,謂之曰:「卿等丈夫,當解大數。今日天下兵亂,有功弗賞,不追無罰,何為相偪乎?」又自植盾,引弓射之,矢皆洞貫。騎旣嘉肅言,且度不能制,乃相率還。肅渡江往見策,策亦雅奇之。留家曲阿。會祖母亡,還葬東城。

劉子揚與肅友善,遺肅書曰:「方今天下豪傑並起,吾子姿才,尤宜今日。急還迎老母,無事滯於東城。近鄭寶者,今在巢湖,擁衆萬餘,處地肥饒,廬江閒人多依就之,況吾徒乎?觀其形勢,又可博集,時不可失,足下速之。」肅荅然其計。葬畢還曲阿,欲北行。會瑜已徙肅母到吳,肅具以狀語瑜。時孫策已薨,權尚住吳,瑜謂肅曰:「昔馬援荅光武云『當今之世,非但君擇臣,臣亦擇君』。今主人親賢貴士,納奇錄異,且吾聞先哲祕論,承運代劉氏者,必興於東南,推步事勢,當其歷數。終搆帝基,以協天符,是烈士攀龍附鳳馳騖之秋。吾方達此,足下不須以子揚之言介意也。」肅從其言。瑜因薦肅才宜佐時,當廣求其比,以成功業,不可令去也。

權即見肅,與語甚恱之。衆賔罷退,肅亦辭出,乃獨引肅還,合榻對飲。因密議曰:「今漢室傾危,四方雲擾,孤承父兄遺業,思有桓文之功。君旣惠顧,何以佐之?」肅對曰:「昔高帝區區欲尊事義帝而不獲者,以項羽為害也。今之曹操,猶昔項羽,將軍何由得為桓文乎?肅竊料之,漢室不可復興,曹操不可卒除。為將軍計,惟有鼎足江東,以觀天下之釁。規模如此,亦自無嫌。何者?北方誠多務也。因其多務,勦除黃祖,進伐劉表,竟長江所極,據而有之,然後建號帝王以圖天下,此高帝之業也。」權曰:「今盡力一方,兾以輔漢耳,此言非所及也。」張昭非肅謙下不足,頗訾毀之,云肅年少麤踈,未可用。權不以介意,益貴重之,賜肅母衣服幃帳,居處雜物,富擬其舊。

劉表死。肅進說曰:「夫荊楚與國鄰接,水流順北,外帶江漢,內阻山陵,有金城之固,沃野萬里,士民殷富,若據而有之,此帝王之資也。今表新亡,二子素不輯睦,軍中諸將,各有彼此。加劉備天下梟雄,與操有隙,寄寓於表,表惡其能而不能用也。若備與彼協心,上下齊同,則宜撫安,與結盟好;如有離違,宜別圖之,以濟大事。肅請得奉命弔表二子,并慰勞其軍中用事者,及說備使撫表衆,同心一意,共治曹操,備必喜而從命。如其克諧,天下可定也。今不速往,恐為操所先。」權即遣肅行。到夏口,聞曹公已向荊州,晨夜兼道。比至南郡,而表子琮已降曹公,備惶遽奔走,欲南渡江。肅徑迎之,到當陽長阪,與備會,宣騰權旨,及陳江東彊固,勸備與權併力。備甚歡恱。時諸葛亮與備相隨,肅謂亮曰「我子瑜友也」,即共定交。備遂到夏口,遣亮使權,肅亦反命。臣松之案:劉備與權併力,共拒中國,皆肅之本謀。又語諸葛亮曰「我子瑜友也」,則亮已亟聞肅言矣。而蜀書亮傳云:「亮以連橫之略說權,權乃大喜。」如似此計始出於亮。若二國史官,各記所聞,競欲稱揚本國容美,各取其功。今此二書,同出一人,而舛互若此,非載述之體也。

會權得曹公欲東之問,與諸將議,皆勸權迎之,而肅獨不言。權起更衣,肅追於宇下,權知其意,執肅手曰:「卿欲何言?」肅對曰:「向察衆人之議,專欲誤將軍,不足與圖大事。今肅可迎操耳,如將軍,不可也。何以言之?今肅迎操,操當以肅還付鄉黨,品其名位,猶不失下曹從事,乘犢車,從吏卒,交游士林,累官故不失州郡也。將軍迎操,將安所歸?願早定大計,莫用衆人之議也。」權歎息曰:「此諸人持議,甚失孤望;今卿廓開大計,正與孤同,此天以卿賜我也。」魏書及九州春秋曰:曹公征荊州,孫權大懼,魯肅實欲勸權拒曹公,乃激說權曰:「彼曹公者,實嚴敵也,新并袁紹,兵馬甚精,乘戰勝之威,伐喪亂之國,克可必也。不如遣兵助之,且送將軍家詣鄴;不然,將危。」權大怒,欲斬肅,肅因曰:「今事已急,即有他圖,何不遣兵助劉備,而欲斬我乎?」權然之,乃遣周瑜助備。孫盛曰:吳書及江表傳,魯肅一見孫權便說拒曹公而論帝王之略,劉表之死也,又請使觀變,無緣方復激說勸迎曹公也。又是時勸迎者衆,而云獨欲斬肅,非其論也。

時周瑜受使至鄱陽,肅勸追召瑜還。遂任瑜以行事,以肅為贊軍校尉,助畫方略。曹公破走,肅即先還,權大請諸將迎肅。肅將入閤拜,權起禮之,因謂曰:「子敬,孤持鞌下馬相迎,足以顯卿未?」肅趨進曰:「未也。」衆人聞之,無不愕然。就坐,徐舉鞭言曰:「願至尊威德加乎四海,總括九州,克成帝業,更以安車軟輪徵肅,始當顯耳。」權撫掌歡笑。

後備詣京見權,求都督荊州,惟肅勸權借之,共拒曹公。漢晉春秋曰:呂範勸留備,肅曰:「不可。將軍雖神武命世,然曹公威力實重,初臨荊州,恩信未洽,宜以借備,使撫安之。多操之敵,而自為樹黨,計之上也。」權即從之。曹公聞權以土地業備,方作書,落筆於地。

周瑜病困,上疏曰:「當今天下,方有事役,是瑜乃心夙夜所憂,願至尊先慮未然,然後康樂。今旣與曹操為敵,劉備近在公安,邊境密邇,百姓未附,宜得良將以鎮撫之。魯肅智略足任,乞以代瑜。瑜隕踣之日,所懷盡矣。」江表傳載:初瑜疾困,與權牋曰:「瑜以凡才,昔受討逆殊特之遇,委以腹心,遂荷榮任,統御兵馬,志執鞭弭,自效戎行。規定巴蜀,次取襄陽,憑賴威靈,謂若在握。至以不謹,道遇暴疾,昨自醫療,日加無損。人生有死,脩短命矣,誠不足惜,但恨微志未展,不復奉教命耳。方今曹公在北,疆埸未靜,劉備寄寓,有似養虎,天下之事,未知終始,此朝士旰食之秋,至尊垂慮之日也。魯肅忠烈,臨事不苟,可以代瑜。人之將死,其言也善,儻或可採,瑜死不朽矣。」案此牋與本傳所載,意旨雖同,其辭乖異耳。即拜肅奮武校尉,代瑜領兵。瑜士衆四千餘人,奉邑四縣,皆屬焉。令程普領南郡太守。肅初住江陵,後下屯陸口,威恩大行,衆增萬餘人,拜漢昌太守、偏將軍。十九年,從權破皖城,轉橫江將軍。

先是,益州牧劉璋綱維頹弛,周瑜、甘寧並勸權取蜀,權以咨備,備內欲自規,乃偽報曰:「備與璋託為宗室,兾憑英靈,以匡漢朝。今璋得罪左右,備獨竦懼,非所敢聞,願加寬貸。若不獲請,備當放髮歸於山林。」後備西圖璋,留關羽守,權曰:「猾虜乃敢挾詐!」及羽與肅鄰界,數生狐疑,疆埸紛錯,肅常以歡好撫之。備旣定益州,權求長沙、零、桂,備不承旨,權遣呂蒙率衆進取。備聞,自還公安,遣羽爭三郡。肅住益陽,與羽相拒。肅邀羽相見,各駐兵馬百步上,但請將軍單刀俱會。肅因責數羽曰:「國家區區本以土地借卿家者,卿家軍敗遠來,無以為資故也。今已得益州,旣無奉還之意,但求三郡,又不從命。」語未究竟,坐有一人曰:「夫土地者,惟德所在耳,何常之有!」肅厲聲呵之,辭色甚切。羽操刀起謂曰:「此自國家事,是人何知!」目使之去。吳書曰:肅欲與羽會語,諸將疑恐有變,議不可往。肅曰:「今日之事,宜相開譬。劉備負國,是非未決,羽亦何敢重欲干命!」乃自就羽。羽曰:「烏林之役,左將軍身在行間,寢不脫介,自力破魏,豈得徒勞,無一塊壤,而足下來欲收地邪?」肅曰:「不然。始與豫州觀於長阪,豫州之衆不當一校,計窮慮極,志勢摧弱,圖欲遠竄,望不及此。主上矜愍豫州之身無有處所,不愛土地士人之力,使有所庇廕以濟其患,而豫州私獨飾情,愆德隳好。今已藉手於西州矣,又欲翦并荊州之土,斯蓋凡夫所不忍行,而況整領人物之主乎!肅聞貪而棄義,必為禍階。吾子屬當重任,曾不能明道處分,以義輔時,而負恃弱衆以圖力爭,師曲為老,將何獲濟?」羽無以荅。備遂割湘水為界,於是罷軍。

肅年四十六,建安二十二年卒。權為舉哀,又臨其葬。諸葛亮亦為發哀。吳書曰:肅為人方嚴,寡於玩飾,內外節儉,不務俗好。治軍整頓,禁令必行,雖在軍陣,手不釋卷。又善談論,能屬文辭,思度弘遠,有過人之明。周瑜之後,肅為之冠。權稱尊號,臨壇,顧謂公卿曰:「昔魯子敬甞道此,可謂明於事勢矣。

肅遺腹子淑旣壯,濡須督張承謂終當到至。永安中,為昭武將軍、都亭侯、武昌督。建衡中,假節,遷夏口督。所在嚴整,有方幹。鳳皇三年卒。子睦襲爵,領兵馬。


\end{pinyinscope}