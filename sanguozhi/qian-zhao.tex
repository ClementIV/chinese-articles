\article{牽招}

\begin{pinyinscope}
牽招字子經,安平觀津人也。年十餘歲,詣同縣樂隱受學。後隱為車騎將軍何苗長史,招隨卒業。值京都亂,苗、隱見害,招俱與隱門生史路等觸蹈鋒刃,共殯斂隱屍,送喪還歸。道遇寇鈔,路等皆悉散走。賊欲斫棺取釘,招垂淚請赦。賊義之,乃釋而去。由此顯名。

兾州牧袁紹辟為督軍從事,兼領烏丸突騎。紹舍人犯令,招先斬乃白,紹奇其意而不見罪也。紹卒,又事紹子尚。建安九年,太祖圍鄴。尚遣招至上黨,督致軍糧。未還,尚破走,到中山。時尚外兄高幹為并州刺史,招以并州左有恒山之險,右有大河之固,帶甲五萬,北阻彊胡,勸幹迎尚,并力觀變。幹旣不能,而陰欲害招。招聞之,間行而去,道隔不得追尚,遂東詣太祖。太祖領兾州,辟為從事。

太祖將討袁譚,而柳城烏丸欲出騎助譚。太祖以招嘗領烏丸,遣詣柳城。到,值峭王嚴,以五千騎當遣詣譚。又遼東太守公孫康自稱平州牧,遣使韓忠齎單于印綬往假峭王。峭王大會群長,忠亦在坐。峭王問招:「昔袁公言受天子之命,假我為單于;今曹公復言當更白天子,假我真單于;遼東復持印綬來。如此,誰當為正?」招荅曰:「昔袁公承制,得有所拜假;中間違錯,天子命曹公代之,言當白天子,更假真單于,是也。遼東下郡,何得擅稱拜假也?」忠曰:「我遼東在滄海之東,擁兵百萬,又有扶餘、濊貊之用;當今之勢,彊者為右,曹操獨何得為是也?」招呵忠曰:「曹公允恭明哲,翼戴天子,伐叛柔服,寧靜四海,汝君臣頑嚚,今恃險遠,背違王命,欲擅拜假,侮弄神器,方當屠戮,何敢慢易咎毀大人?」便捉忠頭頓築,拔刀欲斬之。峭王驚怖,徒跣抱招,以救請忠,左右失色。招乃還坐,為峭王等說成敗之效,禍福所歸,皆下席跪伏,敬受勑教,便辭遼東之使,罷所嚴騎。

太祖滅譚於南皮,署招軍謀掾,從討烏丸。至柳城,拜護烏丸校尉。還鄴,遼東送袁尚首,縣在馬市,招覩之悲感,設祭頭下。太祖義之,舉為茂才。從平漢中,太祖還,留招為中護軍。事罷,還鄴,拜平虜校尉,將兵督青、徐州郡諸軍事,擊東萊賊,斬其渠率,東土寧靜。

文帝踐阼,拜招使持節護鮮卑校尉,屯昌平。是時,邊民流散山澤,又亡叛在鮮卑中者,處有千數。招廣布恩信,招誘降附。建義中郎將公孫集等率將部曲,咸各歸命;使還本郡。又懷來鮮卑素利、彌加等十餘萬落,皆令款塞。

大軍欲征吳,召招還,至,值軍罷,拜右中郎將,出為鴈門太守。郡在邊陲,雖有候望之備,而寇鈔不斷。招旣教民戰陣,又表復烏丸五百餘家租調,使備鞌馬,遠遣偵候。虜每犯塞,勒兵逆擊,來輙摧破,於是吏民膽氣日銳,荒野無虞。又構閒離散,使虜更相猜疑。鮮卑大人步度根、泄歸泥等與軻比能為隙,將部落三萬餘家詣郡附塞。勑令還擊比能,殺比能弟苴羅侯,及叛烏丸歸義侯王同、王寄等,大結怨讎。是以招自出,率將歸泥等討比能於雲中故郡,大破之。招通河西鮮卑附頭等十餘萬家,繕治陘北故上館城,置屯戍以鎮內外,夷虜大小莫不歸心,諸亡叛雖親戚不敢藏匿,咸悉收送。於是野居晏閉,寇賊靜息。招乃簡選有才識者,詣太學受業,還相授教,數年中庠序大興。郡所治廣武,井水鹹苦,民皆擔輦遠汲流水,往返七里。招準望地勢,因山陵之宜,鑿原開渠,注水城內,民賴其益。

明帝即位,賜爵關內侯。太和二年,護烏丸校尉田豫出塞,為軻比能所圍於故馬邑城,移招求救。招即整勒兵馬,欲赴救豫。并州以常憲禁招,招以為節將見圍,不可拘於吏議,自表輙行。又並馳布羽檄,稱陳形勢,云當西北掩取虜家,然後東行,會誅虜身。檄到,豫軍踴躍。又遺一通於虜蹊要,虜即恐怖,種類離散。軍到故平城,便皆潰走。比能復大合騎來,到故平州塞北。招潛行撲討,大斬首級。招以蜀虜諸葛亮數出,而比能狡猾,能相交通,表為防備,議者以為縣遠,未之信也。會亮時在祁山,果遣使連結比能。比能至故北地石城,與相首尾。帝乃詔招,使從便宜討之。時比能已還漠南,招與刺史畢軌議曰:「胡虜遷徙無常。若勞師遠追,則遲速不相及。若欲潛襲,則山溪艱險,資糧轉運,難以密辦。可使守新興、鴈門二牙門,出屯陘北,外以鎮撫,內令兵田,儲畜資糧,秋冬馬肥,州郡兵合,乘釁征討,計必全克。未及施行,會病卒。招在郡十二年,威風遠振。其治邊之稱,次於田豫,百姓追思之。而漁陽傅容在鴈門有名績,繼招後,在遼東又有事功云。

招子嘉嗣。次子弘,亦猛毅有招風,以隴西太守隨鄧艾伐蜀有功,咸熈中為振威護軍。嘉與晉司徒李胤同母,早卒。

案晉書:弘後為揚州、涼州刺史,以果烈死事於邊。嘉子秀,字成叔。荀綽兾州記曰:秀有儁才,性豪俠有氣,弱冠得美名。於太康中為衞瓘、崔洪、石崇等所提攜,以新安令博士為司空從事中郎。與帝舅黃門侍郎王愷素相輕侮。愷諷司隷荀愷,令都官誣奏秀夜在道中載高平國守士田興妻。秀即表訴被誣陷之由,論愷穢行,文辭尤厲。于時朝臣雖多證明,秀名譽由是而損。後張華請為長史,稍遷至尚書。河間王以秀為平北將軍,假節,在馮翊遇害。世人玩其辭賦,惜其材幹。


\end{pinyinscope}