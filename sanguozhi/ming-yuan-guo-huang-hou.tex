\article{ming-yuan-guo-huang-hou}

\begin{pinyinscope}
明元郭皇后,西平人也,世河右大族。黃初中,本郡反叛,遂沒入宮。明帝即位,甚見愛幸,拜為夫人。叔父立為騎都尉,從父芝為虎賁中郎將。帝疾困,遂立為皇后。齊王即位,尊后為皇太后,稱永寧宮,追封謚太后父滿為西都定侯,以立子建紹其爵。封太后母杜為郃陽君。芝遷散騎常侍、長水校尉,

魏略曰:諸郭之中,芝最壯直。先時自以他功封侯。立,宣德將軍,皆封列侯。建兄德,出養甄氏。德及建俱為鎮護將軍,皆封列侯,並掌宿衞。值三主幼弱,宰輔統政,與奪大事,皆先咨啟於太后而後施行。毌丘儉、鍾會等作亂,咸假其命而以為辭焉。景元四年十二月崩,五年二月,葬高平陵西。晉諸公贊曰:建安叔始,有器局而強問,泰始中疾薨。子嘏嗣,為給事中。

評曰:魏后妃之家,雖云富貴,未有若衰漢乘非其據,宰割朝政者也。鑒往易軌,於斯為美。追觀陳羣之議,棧潛之論,適足以為百王之規典,垂憲範乎後葉矣。


\end{pinyinscope}