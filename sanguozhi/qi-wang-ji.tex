\article{qi-wang-ji}

\begin{pinyinscope}
齊王諱芳,字蘭卿。明帝無子,養王及秦王詢;宮省事祕,莫有知其所由來者。

魏氏春秋曰:或云任城王楷子。青龍三年,立為齊王。景初三年正月丁亥朔,帝甚病,乃立為皇太子。是日,即皇帝位,大赦。尊皇后曰皇太后。大將軍曹爽、太尉司馬宣王輔政。詔曰:「朕以眇身,繼承鴻業,煢煢在疚,靡所控告。大將軍、太尉奉受末命,夾輔朕躬,司徒、司空、冢宰、元輔總率百僚,以寧社稷,其與羣卿大夫勉勗乃心,稱朕意焉。諸所興作宮室之役,皆以遺詔罷之。官奴婢六十已上,免為良人。」二月,西域重譯獻火浣布,詔大將軍、太尉臨試以示百寮。異物志曰:斯調國有火州,在南海中。其上有野火,春夏自生,秋冬自死。有木生於其中而不消也,枝皮更活,秋冬火死則皆枯瘁。其俗常冬采其皮以為布,色小青黑;若塵垢洿之,便投火中,則更鮮明也。傅子曰:漢桓帝時,大將軍梁兾以火浣布為單衣,常大會賔客,兾陽爭酒,失杯而汙之,偽怒,解衣曰:「燒之。」布得火,煒燁赫然,如燒凡布,垢盡火滅,粲然潔白,若用灰水焉。搜神記曰:崑崙之墟有炎火之山,山上有鳥獸草木,皆生於炎火之中,故有火浣布,非此山草木之皮枲,則其鳥獸之毛也。漢世西域舊獻此布,中間乆絕;至魏初,時人疑其無有。文帝以為火性酷烈,無含生之氣,著之典論,明其不然之事,絕智者之聽。及明帝立,詔三公曰:「先帝昔著典論,不朽之格言,其刊石于廟門之外及太學,與石經並,以永示來世。」至是西域使至而獻火浣布焉,於是刊滅此論,而天下笑之。臣松之昔從征西至洛陽,歷觀舊物,見典論石在太學者尚存,而廟門外無之,問諸長老,云晉初受禪,即用魏廟,移此石於太學,非兩處立也。竊謂此言為不然。又東方朔神異經曰:南荒之外有火山,長三十里,廣五十里,其中皆生不燼之木,晝夜火燒,得暴風不猛,猛雨不滅。火中有鼠,重百斤,毛長二尺餘,細如絲,可以作布。常居火中,色洞赤,時時出外而色白,以水逐而沃之即死,續其毛,織以為布。

丁丑詔曰:「太尉體道正直,盡忠三世,南擒孟達,西破蜀虜,東滅公孫淵,功蓋海內。昔周成建保傅之官,近漢顯宗崇寵鄧禹,所以優隆儁乂,必有尊也。其以太尉為太傅,持節統兵都督諸軍事如故。」三月,以征東將軍滿寵為太尉。夏六月,以遼東東沓縣吏民渡海居齊郡界,以故縱城為新沓縣以居徙民。秋七月,上始親臨朝,聽公卿奏事。八月,大赦。冬十月,以鎮南將軍黃權為車騎將軍。

十二月,詔曰:「烈祖明皇帝以正月棄背天下,臣子永惟忌日之哀,其復用夏正;雖違先帝通三統之義,斯亦禮制所由變改也。又夏正於數為得天正,其以建寅之月為正始元年正月,以建丑月為後十二月。」

正始元年春二月乙丑,加侍中中書監劉放、侍中中書令孫資為左右光祿大夫。丙戌,以遼東汶、北豐縣民流徙渡海,規齊郡之西安、臨菑、昌國縣界為新汶、南豐縣,以居流民。

自去冬十二月至此月不雨。丙寅,詔令獄官亟平冤枉,理出輕微;羣公卿士讜言嘉謀,各悉乃心。夏四月,車騎將軍黃權薨。秋七月,詔曰:「易稱損上益下,節以制度,不傷財,不害民。方今百姓不足而御府多作金銀雜物,將奚以為?今出黃金銀物百五十種,千八百餘斤,銷冶以供軍用。」八月,車駕巡省洛陽界秋稼,賜高年力田各有差。

二年春二月,帝初通論語,使太常以太牢祭孔子於辟雍,以顏淵配。

夏五月,吳將朱然等圍襄陽之樊城,太傅司馬宣王率衆拒之。干寶晉紀曰:吳將全琮寇芍陂,朱然、孫倫五萬人圍樊城,諸葛瑾、步隲寇柤中;琮已破走而樊圍急。宣王曰:「柤中民夷十萬,隔在水南,流離無主,樊城被攻,歷月不解,此危事也,請自討之。」議者咸言:「賊遠圍樊城不可拔,挫於堅城之下,有自破之勢,宜長策以御之。」宣王曰:「軍志有之:將能而御之,此為縻軍;不能而任之,此為覆軍。今疆埸騷動,民心疑惑,是社稷之大憂也。」六月,督諸軍南征,車駕送津陽城門外。宣王以南方暑溼,不宜持乆,使輕騎挑之,然不敢動。於是乃令諸軍休息洗沐,簡精銳,募先登,申號令,示必攻之勢。然等聞之,乃夜遁。追至三州口,大殺獲。六月辛丑,退。己卯,以征東將軍王陵為車騎將軍。冬十二月,南安郡地震。

三年春正月,東平王徽薨。三月,太尉滿寵薨。秋七月甲申,南安郡地震。乙酉,以領軍將軍蔣濟為太尉。冬十二月,魏郡地震。

四年春正月,帝加元服,賜羣臣各有差。夏四月乙卯,立皇后甄氏,大赦。五月朔,日有蝕之,旣。秋七月,詔祀故大司馬曹真、曹休、征南大將軍夏侯尚、太常桓階、司空陳羣、太傅鍾繇、車騎將軍張郃、左將軍徐晃、前將軍張遼、右將軍樂進、太尉華歆、司徒王朗、驃騎將軍曹洪、征西將軍夏侯淵、後將軍朱靈、文聘、執金吾臧霸、破虜將軍李典、立義將軍龐德、武猛校尉典韋於太祖廟庭。冬十二月,倭國女王俾彌呼遣使奉獻。

五年春二月,詔大將軍曹爽率衆征蜀。夏四月朔,日有蝕之。五月癸巳,講尚書經通,使太常以太牢祠孔子於辟雍,以顏淵配;賜太傳、大將軍及侍講者各有差。丙午,大將軍曹爽引軍還。秋八月,秦王詢薨。九月,鮮卑內附,置遼東屬國,立昌黎縣以居之。冬十一月癸卯,詔祀故尚書令荀攸於太祖廟庭。臣松之以為故魏氏配饗不及荀彧,蓋以其末年異議,又位非魏臣故也。至於升程昱而遺郭嘉,先鍾繇而後荀攸,則未詳厥趣也。徐佗謀逆而許褚心動,忠誠之至遠同於日磾,且潼關之危,非褚不濟,褚之功烈有過典韋,今祀韋而不及褚,文所未達也。己酉,復秦國為京兆郡。十二月,司空崔林薨。

六年春二月丁卯,南安郡地震。丙子,以驃騎將軍趙儼為司空;夏六月,儼薨。八月丁卯,以太常高柔為司空。癸巳,以左光祿大夫劉放為驃騎將軍,右光祿大夫孫資為衞將軍。冬十一月,祫祭太祖廟,始祀前所論佐命臣二十一人。十二月辛亥,詔故司徒王朗所作易傳,令學者得以課試。乙亥,詔曰:「明日大會羣臣,其令太傅乘輿上殿。」

七年春二月,幽州刺史毌丘儉討高句驪,夏五月,討濊貊,皆破之。韓那奚等數十國各率種落降。秋八月戊申,詔曰:「屬到巿觀見所斥賣官奴婢,年皆七十,或𤸇疾殘病,所謂天民之窮者也。且官以其力竭而復鬻之,進退無謂,其悉遣為良民。若有不能自存者,郡縣振給之。」臣松之案:帝初即位,有詔「官奴婢六十以上免為良人」。旣有此詔,則宜遂為永制。七八年間,而復貨年七十者,且七十奴婢及𤸇疾殘病,並非可售之物,而鬻之於巿,此皆事之難解。

己酉,詔曰:「吾乃當以十九日親祠,而昨出已見治道,得雨當復更治,徒棄功夫。每念百姓力少役多,夙夜存心。道路但當期於通利,聞乃檛捶老小,務崇脩飾,疲困流離,以至哀歎,吾豈安乘此而行,致馨德於宗廟邪?自今已後,明申勑之。」冬十二月,講禮記通,使太常以太牢祀孔子於辟雍,以顏淵配。習鑿齒漢晉春秋曰:是年,吳將朱然入柤中,斬獲數千;柤中民吏萬餘家渡沔。司馬宣王謂曹爽曰:「若便令還,必復致寇,宜權留之。」爽曰:「今不脩守沔南,留民沔北,非長策也。」宣王曰:「不然。凡物置之安地則安,危地則危,故兵書曰,成敗形也,安危勢也,形勢御衆之要,不可不審。設令賊二萬人斷沔水,三萬人與沔南諸軍相持,萬人陸鈔柤中,君將何以救之?」爽不聽,卒令還。然後襲殺之。袁淮言於爽曰:「吳楚之民脆弱寡能,英才大賢不出其土,比技量力,不足與中國相抗,然自上世以來常為中國患者,蓋以江漢為池,舟楫為用,利則陸鈔,不利則入水,攻之道遠,中國之長技無所用之也。孫權自十數年以來,大畋江北,繕治甲兵,精其守禦,數出盜竊,敢遠其水,陸次平土,此中國所願聞也。夫用兵者,貴以飽待飢,以逸擊勞,師不欲乆,行不欲遠,守少則固,力專則彊。當今宜捐淮、漢已南,退却避之。若賊能入居中央,來侵邊境,則隨其所短,中國之長技得用矣。若不敢來,則邊境得安,無鈔盜之憂矣。使我國富兵彊,政脩民一,陵其國不足為遠矣。今襄陽孤在漢南,賊循漢而上,則斷而不通,一戰而勝,則不攻而自服,故置之無益於國,亡之不足為辱。自江夏已東,淮南諸郡,三后已來,其所亡幾何,以近賊疆界易鈔掠之故哉!若徙之淮北,遠絕其間,則民人安樂,何鳴吠之驚乎?」遂不徙。

八年春二月朔,日有蝕之。夏五月,分河東之汾北十縣為平陽郡。

秋七月,尚書何晏奏曰:「善為國者必先治其身,治其身者慎其所習。所習正則其身正,其身正則不令而行;所習不正則其身不正,其身不正則雖令不從。是故為人君者,所與游必擇正人,所觀覽必察正象,放鄭聲而弗聽,遠佞人而弗近,然後邪心不生而正道可弘也。季末闇主不知損益,斥遠君子,引近小人,忠良疏遠,便辟褻狎,亂生近暱,譬之社鼠;考其昏明,所積以然,故聖賢諄諄以為至慮。舜戒禹曰『鄰哉鄰哉』,言慎所近也,周公戒成王曰『其朋其朋』,言慎所與也。書云:『一人有慶,兆民賴之。』可自今以後,御幸式乾殿及游豫後園,皆大臣侍從,因從容戲宴,兼省文書,詢謀政事,講論經義,為萬世法。」冬十二月,散騎常侍諫議大夫孔晏乂奏曰:「禮,天子之宮,有斲礱之制,無朱丹之飾,宜循禮復古。今天下已平,君臣之分明,陛下但當不懈于位,平公正之心,審賞罰以使之。可絕後園習騎乘馬,出必御輦乘車,天下之福,臣子之願也。」晏乂咸因闕以進規諫。

九年春二月,衞將軍中書令孫資,癸巳,驃騎將軍中書監劉放,三月甲午,司徒衞臻,各遜位,以侯就第,位特進。四月,以司空高柔為司徒;光祿大夫徐邈為司空,固辭不受。秋九月,以車騎將軍王淩為司空。冬十月,大風發屋折樹。

嘉平元年春正月甲午,車駕謁高平陵。孫盛魏世籍曰:高平陵在洛水南大石山,去洛城九十里。太傅司馬宣王奏免大將軍曹爽、爽弟中領軍羲、武衞將軍訓、散騎常侍彥官,以侯就第。戊戌,有司奏収黃門張當付廷尉,考實其辭,爽與謀不軌。又尚書丁謐、鄧颺、何晏、司隷校尉畢軌、荊州刺史李勝、大司農桓範皆與爽通姦謀,夷三族。語在爽傳。丙午,大赦。丁未,以太傅司馬宣王為丞相,固讓乃止。孔衍漢魏春秋曰:詔使太常王肅冊命太傅為丞相,增邑萬戶,羣臣奏事不得稱名,如漢霍光故事。太傅上書辭讓曰:「臣親受顧命,憂深責重,憑賴天威,摧弊姦凶,贖罪為幸,功不足論。又三公之官,聖王所制,著之典禮。至於丞相,始自秦政。漢氏因之,無復變改。今三公之官皆備,橫復寵臣,違越先典,革聖明之經,襲秦漢之路,雖在異人,臣所宜正,況當臣身而不固爭,四方議者將謂臣何!」書十餘上,詔乃許之,復加九錫之禮。太傅又言:「太祖有大功大德,漢氏崇重,故加九錫,此乃歷代異事,非後代之君臣所得議也。」又辭不受。

夏四月乙丑,改年。丙子,太尉蔣濟薨。冬十二月辛卯,以司空王淩為太尉。庚子,以司隷校尉孫禮為司空。

二年夏五月,以征西將軍郭淮為車騎將軍。冬十月,以特進孫資為驃騎將軍。十一月,司空孫禮薨。十二月甲辰,東海王霖薨。乙未,征南將軍王昶渡江,掩攻吳,破之。

三年春正月,荊州刺史王基、新城太守州泰攻吳,破之,降者數千口。二月,置南郡之夷陵縣以居降附。三月,以尚書令司馬孚為司空。四月甲申,以征南將軍王昶為征南大將軍。壬辰,大赦。丙午,聞太尉王淩謀廢帝,立楚王彪,太傅司馬宣王東征淩。五月甲寅,淩自殺。六月,彪賜死。秋七月壬戌,皇后甄氏崩。辛未,以司空司馬孚為太尉。戊寅,太傅司馬宣王薨,以衞將軍司馬景王為撫軍大將軍,錄尚書事。乙未,葬懷甄后于太清陵。庚子,驃騎將軍孫資薨。十一月,有司奏諸功臣應饗食於太祖廟者,更以官為次,太傅司馬宣王功高爵尊,最在上。十二月,以光祿勳鄭冲為司空。

四年春正月癸卯,以撫軍大將軍司馬景王為大將軍。二月,立皇后張氏,大赦。夏五月,魚二,見于武庫屋上。漢晉春秋曰:初,孫權築東興隄以遏巢湖。後征淮南,壞不復脩。是歲諸葛恪帥軍更於隄左右結山,挾築兩城,使全端、留略守之,引軍而還。諸葛誕言於司馬景王曰:「致人而不至於人者,此之謂也。今因其內侵,使文舒逼江陵,仲恭向武昌,以羈吳之上流,然後簡精卒攻兩城,比救至,可大獲也。」景王從之。冬十一月,詔征南大將軍王昶、征東將軍胡遵、鎮南將軍毌丘儉等征吳。十二月,吳大將軍諸葛恪拒戰,大破衆軍於東關。不利而還。漢晉春秋曰:毌丘儉、王昶聞東軍敗,各燒屯走。朝議欲貶黜諸將,景王曰:「我不聽公休,以至於此。此我過也,諸將何罪?」悉原之。時司馬文王為監軍,統諸軍,唯削文王爵而已。是歲,雍州刺史陳泰求勑并州并力討胡,景王從之。未集,而鴈門、新興二郡以為將遠役,遂驚反。景王又謝朝士曰:「此我過也,非玄伯之責!」於是魏人愧恱,人思其報。習鑿齒曰:司馬大將軍引二敗以為己過,過消而業隆,可謂智矣。夫民忘其敗,而下思其報,雖欲不康,其可得邪?若乃諱敗推過,歸咎萬物,常執其功而隱其喪,上下離心,賢愚解體,是楚再敗而晉再克也,謬之甚矣!君人者,苟統斯理而以御國,則朝無秕政,身靡留愆,行失而名揚,兵挫而戰勝,雖百敗可也,況於再乎!

五年夏四月,大赦。五月,吳太傅諸葛恪圍合肥新城,詔太尉司馬孚拒之。漢晉春秋曰:是時姜維亦出圍狄道。司馬景王問虞松曰:「今東西有事,二方皆急,而諸將意沮,若之何?」松曰:「昔周亞夫堅壁昌邑而吳楚自敗,事有似弱而彊,或似彊而弱,不可不察也。今恪悉其銳衆,足以肆暴,而坐守新城,欲以致一戰耳。若攻城不拔,請戰不得,師老衆疲,勢將自走,諸將之不徑進,乃公之利也。姜維有重兵而縣軍應恪,投食我麥,非深根之寇也。且謂我并力於東,西方必虛,是以徑進。今若使關中諸軍倍道急赴,出其不意,殆將走矣。」景王曰:「善!」乃使郭淮、陳泰悉關中之衆,解狄道之圍;勑毌丘儉等案兵自守,以新城委吳。姜維聞淮進兵,軍食少,乃退屯隴西界。秋七月,恪退還。是時,張特守新城。魏略曰:特字子產,涿郡人。先時領牙門,給事鎮東諸葛誕,誕不以為能也,欲遣還護軍。會毌丘儉代誕,遂使特屯守合肥新城。及諸葛恪圍城,特與將軍樂方等三軍衆合有三千人,吏兵疾病及戰死者過半,而恪起土山急攻,城將陷,不可護。特乃謂吳人曰:「今我無心復戰也。然魏法,被攻過百日而救不至者,雖降,家不坐也。自受敵以來,已九十餘日矣。此城中本有四千餘人,而戰死者已過半,城雖陷,尚有半人不欲降,我當還為相語之,條名別善惡,明日早送名,且持我印綬去以為信。」乃投其印綬以與之。吳人聽其辭而不取印綬。不攻。頃之,特還,乃夜徹諸屋材柵,補其缺為二重。明日,謂吳人曰:「我但有鬬死耳!」吳人大怒,進攻之,不能拔,遂引去。朝廷嘉之,加雜號將軍,封列侯,又遷安豐太守。

八月,詔曰:「故中郎西平郭脩,砥節厲行,秉心不回。乃者蜀將姜維寇鈔脩郡,為所執略。往歲偽大將軍費禕驅率羣衆,陰圖闚𨵦,道經漢壽,請會衆賔,脩於廣坐之中手刃擊禕,勇過聶政,功逾介子,可謂殺身成仁,釋生取義者矣。夫追加襃寵,所以表揚忠義;祚及後胤,所以獎勸將來。其追封脩為長樂鄉侯,食邑千戶,謚曰威侯;子襲爵,加拜奉車都尉;賜銀千鉼,絹千匹,以光寵存亡,永垂來世焉。」魏氏春秋曰:脩字孝先,素有業行,著名西州。姜維劫之,脩不為屈。劉禪以為左將軍,脩欲刺禪而不得親近,每因慶賀,且拜且前,為禪左右所遏,事輙不克,故殺禕焉。臣松之以為古之舍生取義者,必有理存焉,或感恩懷德,投命無悔,或利害有機,奮發以應會,詔所稱聶政、介子是也。事非斯類,則陷乎妄作矣。魏之與蜀,雖為敵國,非有趙襄滅智之仇,燕丹危亡之急;且劉禪凡下之主,費禕中才之相,二人存亡,固無關於興喪。郭脩在魏,西州之男子耳,始獲於蜀,旣不能抗節不辱,於魏又無食祿之責,不為時主所使,而無故規規然糜身於非所,義無所加,功無所立,可謂「折柳樊圃」,其狂也且,此之謂也。

自帝即位至于是歲,郡國縣道多所置省,俄或還復,不可勝紀。

六年春二月己丑,鎮東將軍毌丘儉上言:「昔諸葛恪圍合肥新城,城中遣士劉整出圍傳消息,為賊所得,考問所傳,語整曰:『諸葛公欲活汝,汝可具服。』整罵曰:『死狗,此何言也!我當必死為魏國鬼,不苟求活,逐汝去也。欲殺我者,便速殺之。』終無他辭。又遣士鄭像出城傳消息,或以語恪,恪遣馬騎尋圍跡索,得像還。四五人靮頭靣縛,將繞城表,勑語像,使大呼,言『大軍已還洛,不如早降。』像不從其言,更大呼城中曰:『大軍近在圍外,壯士努力!』賊以刀築其口,使不得言,像遂大呼,令城中聞知。整、像為兵,能守義執節,子弟宜有差異。」詔曰:「夫顯爵所以襃元功,重賞所以寵烈士。整、像召募通使,越蹈重圍,冒突白刃,輕身守信,不幸見獲,抗節彌厲,揚六軍之大勢,安城守之懼心,臨難不顧,畢志傳命。昔解楊執楚,有隕無貳,齊路中大夫以死成命,方之整、像,所不能加。今追賜整、像爵關中侯,各除士名,使子襲爵,如部曲將死事科。」

庚戌,中書令李豐與皇后父光祿大夫張緝等謀廢易大臣,以太常夏侯玄為大將軍。事覺,諸所連及者皆伏誅。辛亥,大赦。三月,廢皇后張氏。夏四月,立皇后王氏,大赦。五月,封后父奉車都尉王夔為廣明鄉侯、光祿大夫,位特進,妻田氏為宣陽鄉君。秋九月,大將軍司馬景王將謀廢帝,以聞皇太后。世語及魏氏春秋並云:此秋,姜維寇隴右。時安東將軍司馬文王鎮許昌,徵還擊維,至京師,帝於平樂觀以臨軍過。中領軍許允與左右小臣謀,因文王辭,殺之,勒其衆以退大將軍。已書詔於前。文王入,帝方食栗,優人雲午等唱曰:「青頭雞,青頭雞。」青頭雞者,鴨也。帝懼不敢發。文王引兵入城,景王因是謀廢帝。臣松之案夏侯玄傳及魏略,許允此年春與李豐事相連。豐旣誅,即出允為鎮北將軍,未發,以放散官物收付廷尉,徙樂浪,追殺之。允此秋不得故為領軍而建此謀。甲戌,太后令曰:「皇帝芳春秋已長,不親萬機,耽淫內寵,沈漫女德,日延倡優,縱其醜謔;迎六宮家人留止內房,毀人倫之叙,亂男女之節;恭孝日虧,悖慠滋甚,不可以承天緒,奉宗廟。使兼太尉高柔奉策,用一元大武告于宗廟,遣芳歸藩于齊,以避皇位。」魏書曰:是日,景王承皇太后令,詔公卿中朝大臣會議,羣臣失色。景王流涕曰:「皇太后令如是,諸君其若王室何!」咸曰:「昔伊尹放太甲以寧殷,霍光廢昌邑以安漢,夫權定社稷以濟四海,二代行之於古,明公當之於今,今日之事,亦唯公命。」景王曰:「諸君所以望師者重,師安所避之?」於是乃與羣臣共為奏永寧宮曰:「守尚書令太尉長社侯臣孚、大將軍武陽侯臣師、司徒萬歲亭侯臣柔、司空文陽亭侯臣冲、行征西安東將軍新城侯臣昭、光祿大夫關內侯臣邕、太常臣晏、衞尉昌邑侯臣偉、太僕臣嶷、廷尉定陵侯臣毓、大鴻臚臣芝、大司農臣祥、少府臣袤、永寧衞尉臣楨、永寧太僕臣閣、大長秋臣模、司隷校尉潁昌侯臣曾、河南尹蘭陵侯臣肅、城門校尉臣慮、中護軍永安亭侯臣望、武衞將軍安壽亭侯臣演、中堅將軍平原侯臣德、中壘將軍昌武亭侯臣廙、屯騎校尉關內侯臣陔、步兵校尉臨晉侯臣建、射聲校尉安陽鄉侯臣溫、越騎校尉睢陽侯臣初、長水校尉關內侯臣超、侍中臣小同、臣顗、臣酆、博平侯臣表、侍中中書監安陽亭侯臣誕、散騎常侍臣瓌、臣儀、關內侯臣芝、尚書僕射光祿大夫高樂亭侯臣毓、尚書關內侯臣觀、臣嘏、長合鄉侯臣亮、臣贊、臣騫、中書令臣康、御史中丞臣鈐、博士臣範、臣峻等稽首言:臣等聞天子者,所以濟育羣生,永安萬國,三祖勳烈,光被六合。皇帝即位,纂繼洪業,春秋已長,未親萬機,耽淫內寵,沈漫女色,廢捐講學,棄辱儒士,日延小優郭懷、袁信等於建始芙蓉殿前裸袒游戲,使與保林女尚等為亂,親將後宮瞻觀。又於廣望觀上,使懷、信等於觀下作遼東妖婦,嬉褻過度,道路行人掩目,帝於觀上以為讌笑。於陵雲臺曲中施帷,見九親婦女,帝臨宣曲觀,呼懷、信使入帷共飲酒。懷、信等更行酒,婦女皆醉,戲侮無別。使保林李華、劉勳等與懷、信等戲,清商令令狐景呵華、勳曰:『諸女,上左右人,各有官職,何以得爾?』華、勳數讒毀景。帝常喜以彈彈人,以此恚景,彈景不避首目。景語帝曰:『先帝持門戶急,今陛下日將妃后游戲無度,至乃共觀倡優,裸袒為亂,不可令皇太后聞。景不愛死,為陛下計耳。』帝言:『我作天子,不得自在邪?太后何與我事!』使人燒鐵灼景,身體皆爛。甄后崩後,帝欲立王貴人為皇后。太后更欲外求,帝恚語景等:『魏家前後立皇后,皆從所愛耳,太后必違我意,知我當往不也?』後卒待張皇后疏薄。太后遭郃陽君喪,帝日在後園,倡優音樂自若,不數往定省。清商丞龐熈諫帝:『皇太后至孝,今遭重憂,水漿不入口,陛下當數往寬慰,不可但在此作樂。』帝言:『我自爾,誰能柰我何?』皇太后還北宮,殺張美人及禺婉,帝恚望,語景等:『太后橫殺我所寵愛,此無復母子恩。』數往至故處啼哭,私使暴室厚殯棺,不令太后知也。每見九親婦女有美色,或留以付清商。帝至後園竹間戲,或與從官攜手共行。熈白:『從官不宜與至尊相提挈。』帝怒,復以彈彈熈。日游後園,每有外文書入,帝不省,左右曰『出』,帝亦不索視。太后令帝常在式乾殿上講學,不欲,使行來,帝徑去;太后來問,輙詐令黃門荅言『在』耳。景、熈等畏恐,不敢復止,更共讇媚。帝肆行昏淫,敗人倫之叙,亂男女之節,恭孝彌頹,凶德浸盛。臣等憂懼傾覆天下,危墜社稷,雖殺身斃命不足以塞責。今帝不可以承天緒,臣請依漢霍光故事,收帝璽綬。帝本以齊王踐祚,宜歸藩于齊。使司徒臣柔持節,與有司以太牢告祀宗廟。臣謹昧死以聞。」奏可。是日遷居別宮,年二十三。使者持節送衞,營齊王宮於河內重門,制度皆如藩國之禮。魏略曰:景王將廢帝,遣郭芝入白太后,太后與帝對坐。芝謂帝曰:「大將軍欲廢陛下,立彭城王據。」帝乃起去。太后不恱。芝曰:「太后有子不能教,今大將軍意已成,又勒兵于外以備非常,但當順旨,將復何言!」太后曰:「我欲見大將軍,口有所說。」芝曰:「何可見邪?但當速取璽綬。」太后意折,乃遣傍侍御取璽綬著坐側。芝出報景王,景王甚歡。又遣使者授齊王印綬,當出就西宮。帝受命,遂載王車,與太后別,垂涕,始從太極殿南出,羣臣送者數十人,太尉司馬孚悲不自勝,餘多流涕。王出後,景王又使使者請璽綬。太后曰:「彭城王,我之季叔也,今來立,我當何之!且明皇帝當絕嗣乎?吾以為高貴鄉公者,文皇帝之長孫,明皇帝之弟子,於禮,小宗有後大宗之義,其詳議之。」景王乃更召群臣,以皇太后令示之,乃定迎高貴鄉公。是時太常已發二日,待璽綬於溫。事定,又請璽綬。太后令曰:「我見高貴鄉公,小時識之,明日我自欲以璽綬手授之也。」

丁丑,令曰:「東海王霖,高祖文皇帝之子。霖之諸子,與國至親,高貴鄉公髦有大成之量,其以為明皇帝嗣。」魏書曰:景王復與羣臣共奏永寧宮曰:「臣等聞人道親親故尊祖,尊祖故敬宗。禮,大宗無嗣,則擇支子之賢者;為人後者,為之子也。東海定王子高貴鄉公,文皇帝之孫,宜承正統,以嗣烈祖明皇帝後。率土有賴,萬邦幸甚,臣請徵公詣洛陽宮。」奏可。使中護軍望、兼太常河南尹肅持節,與少府袤、尚書亮、侍中表等奉法駕,迎公于元城。魏世譜曰:晉受禪,封齊王為邵陵縣公。年四十三,泰始十年薨,謚曰厲公。


\end{pinyinscope}