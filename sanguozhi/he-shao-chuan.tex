\article{賀邵傳}

\begin{pinyinscope}
賀邵字興伯,會稽山陰人也。

吳書曰:邵,賀齊之孫,景之子。孫休即位,從中郎為散騎中常侍,出為吳郡太守。孫皓時,入為左典軍,遷中書令,領太子太傅。皓兇暴驕矜,政事日弊。邵上疏諫曰:

古之聖王,所以潛處重闈之內而知萬里之情,垂拱袵席之上,明照八極之際者,任賢之功也。陛下以至德淑姿,統承皇業,宜率身履道,恭奉神器,旌賢表善,以康庶政。自頃年以來,朝列紛錯,真偽相貿,上下空任,文武曠位,外無山嶽之鎮,內無拾遺之臣;佞諛之徒拊翼天飛,干弄朝威,盜竊榮利,而忠良排墜,信臣被害。是以正士摧方,而庸臣苟媚,先意承旨,各希時趣,人執反禮之評,士吐詭道之論,遂使清流變濁,忠臣結舌。陛下處九天之上,隱百重之室,言出風靡,令行景從,親洽寵媚之臣,日聞順意之辭,將謂此輩實賢,而天下已平也。臣心所不安,敢不以聞。

臣聞興國之君樂聞其過,荒亂之主樂聞其譽;聞其過者過日消而福臻,聞其譽者譽日損而禍至。是以古之人君,揖讓以進賢,虛己以求過,譬天位於乘犇,以虎尾為警戒。至於陛下,嚴刑法以禁直辭,黜善士以逆諫臣,眩燿毀譽之實,沈淪近習之言。昔高宗思佐,夢寐得賢,而陛下求之如忘,忽之如遺。故常侍王蕃忠恪在公,才任輔弼,以醉酒之間加之大戮。近鴻臚葛奚,先帝舊臣,偶有逆迕,昏醉之言耳,三爵之後,禮所不諱,陛下猥發雷霆,謂之輕慢,飲之醇酒,中毒隕命。自是之後,海內悼心,朝臣失圖,仕者以退為幸,居者以出為福,誠非所以保光洪緒,熈隆道化也。

又何定本趨走小人,僕隷之下,身無錙銖之行,能無鷹犬之用,而陛下愛其佞媚,假其威柄,使定恃寵放恣,自擅威福,口正國議,手弄天機,上虧日月之明,下塞君子之路。夫小人求入,必進姦利,定間妄興事役,發江邊戍兵以驅麋鹿,結罝山陵,芟夷林莽,殫其九野之獸,聚其重圍之內,上無益時之分,下有損耗之費。而兵士罷於運送,人力竭於驅逐,老弱饑凍,大小怨歎。臣竊觀天變,自比年以來陰陽錯謬,四時逆節,日食地震,中夏隕霜,參之典籍,皆陰氣陵陽,小人弄勢之所致也。臣甞覽書傳,驗諸行事,災祥之應,所為寒慄。昔高宗脩己以消鼎雉之異,宋景崇德以退熒惑之變,願陛下上懼皇天譴告之誚,下追二君攘災之道,遠覽前代任賢之功,近寤今日謬授之失,清澄朝位,旌叙俊乂,放退佞邪,抑奪姦勢,如是之輩,一勿復用,廣延淹滯,容受直辭,祗承乾指,敬奉先業,則大化光敷,天人望塞也。

傳曰:「國之興也,視民如赤子;其亡也,以民為草芥。」陛下昔韜神光,潛德東夏,以聖哲茂姿,龍飛應天,四海延頸,八方拭目,以成康之化必隆於旦夕也。自登位以來,法禁轉苛,賦調益繁;中宮內豎,分布州郡,橫興事役,競造姦利;百姓罹杼軸之困,黎民罷無已之求,老幼饑寒,家戶菜色,而所在長吏,迫畏罪負,嚴法峻刑,苦民求辦。是以人力不堪,家戶離散,呼嗟之聲,感傷和氣。又江邊戍兵,遠當以拓土廣境,近當以守界備難,宜特優育,以待有事,而徵發賦調,煙至雲集,衣不全短褐,食不贍朝夕,出當鋒鏑之難,入抱無聊之慼。是以父子相棄,叛者成行。願陛下寬賦除煩,振恤窮乏,省諸不急,盪禁約法,則海內樂業,大化普洽。夫民者國之本,食者民之命也,今國無一年之儲,家無經月之畜,而後宮之中坐食者萬有餘人。內有離曠之怨,外有損耗之費,使庫廩空於無用,士民饑於糟糠。

又北敵注目,伺國盛衰,陛下不恃己之威德,而怙敵之不來,忽四海之困窮,而輕虜之不為難,誠非長策廟勝之要也。昔大皇帝勤身苦體,創基南夏,割據江山,拓土萬里,雖承天贊,實由人力也。餘慶遺祚,至於陛下,陛下宜勉崇德器,以光前烈,愛民養士,保全先軌,何可忽顯祖之功勤,輕難得之大業,忘天下之不振,替興衰之巨變哉?臣聞否泰無常,吉凶由人,長江之限不可乆恃,苟我不守,一葦可航也。昔秦建皇帝之號,據殽函之阻,德化不脩,法政苛酷,毒流生民,忠臣杜口,是以一夫大呼,社稷傾覆。近劉氏據三關之險,守重山之固,可謂金城石室,萬世之業,任授失賢,一朝喪沒,君臣係頸,共為羈僕。此當世之明鑒,目前之烱戒也。願陛下遠考前事,近鑒世變,豐基彊本,割情從道,則成康之治興,聖祖之祚隆矣。

書奏,皓深恨之。邵奉公貞正,親近所憚。乃共譖邵與樓玄謗毀國事,俱被詰責,玄見送南州,邵原復職。後邵中惡風,口不能言,去職數月,皓疑其託疾,收付酒藏,掠考千所,邵卒無一語,竟見殺害,家屬徙臨海。并下詔誅玄子孫,是歲天冊元年也,邵年四十九。邵子循,字彥先。虞預晉書曰:循丁家禍,流放海濵,吳平,還鄉里。節操高厲,童齔不羣,言行舉動必以禮讓。好學博聞,尤善三禮。舉秀才,除陽羨、武康令。顧榮、陸機、陸雲表薦循曰:「伏見吳興武康令賀循德量邃茂,才鑒清遠,服膺道素,風操凝峻,歷踐三城,刑政肅穆,守職下縣,編名凡萃,出自新邦,朝無知己,恪居遐外,志不自營,年時倏忽,而邈無階緒,實州黨愚智所為悵然。臣等並以凡才,累授飾進,被服恩澤,忝豫朝末,知良士後時,而守局無言,懼有蔽賢之咎,是以不勝愚管,謹冒死表聞。」久之,召為太子舍人。石冰破揚州,循亦合衆,事平,杜門不出。陳敏作亂,以循為丹陽內史,循稱疾固辭,敏不敢逼。于時江東豪右無不受敏爵位,惟循與同郡朱誕不挂賊網。後除吳國內史,不就。元皇帝為鎮東將軍,請循為軍司馬,帝為晉王,以循為中書令,固讓不受,轉太常,領太子太傅。時朝廷初建,動有疑議,宗廟制度皆循所定,朝野諮詢,為一時儒宗。年六十,太興二年卒。追贈司空,謚曰穆。循諸所著論,並傳於世。子隰,臨海太守。


\end{pinyinscope}