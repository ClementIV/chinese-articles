\article{滕胤傳}

\begin{pinyinscope}
滕胤字承嗣,北海劇人也。伯父耽,父冑,與劉繇州里通家,以世擾亂,渡江依繇。孫權為車騎將軍,拜耽右司馬,以寬厚稱,早卒,無嗣。冑善屬文,權待以賔禮,軍國書疏,常令損益潤色之,亦不幸短命。權為吳王,追錄舊恩,封胤都亭侯。少有節操,美容儀。

吳書曰:胤年十二,而孤單煢立,能治身厲行。為人白晢,威儀可觀。每正朔朝賀脩勤,在位大臣見者,無不歎賞。弱冠尚公主。年三十,起家為丹楊太守,徙吳郡、會稽,所在見稱。吳書曰:胤上表陳及時宜,及民間優劣,多所匡弼。權以胤故,增重公主之賜,屢加存問。胤每聽辭訟,斷罪法,察言觀色,務盡情理。人有窮冤悲苦之言,對之流涕。

太元元年,權寢疾,詣都,留為太常,與諸葛恪等俱受遺詔輔政。孫亮即位,加衞將軍。

恪將悉衆伐魏,胤諫恪曰:「君以喪代之際,受伊、霍之託,入安本朝,出摧彊敵,名聲振於海內,天下莫不震動,萬姓之心,兾得蒙君而息。今猥以勞役之後,興師出征,民疲力屈,遠主有備。若攻城不克,野略無獲,是喪前勞而招後責也。不如案甲息師,觀隙而動。且兵者大事,事以衆濟,衆苟不恱,君獨安之?」恪曰:「諸云不可者,皆不見計筭,懷居苟安者也,而子復以為然,吾何望焉?夫以曹芳闇劣,而政在私門,彼之臣民,固有離心。今吾因國家之資,藉戰勝之威,則何往而不克哉!」以胤為都下督,掌統留事。胤白日接賔客,夜省文書,或通曉不寐。吳書曰:胤寵任彌高,接士愈勤,表奏書疏,皆自經意,不以委下。


\end{pinyinscope}