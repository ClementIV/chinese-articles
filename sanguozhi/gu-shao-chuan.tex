\article{顧邵傳}

\begin{pinyinscope}
邵字孝則,博覽書傳,好樂人倫。少與舅陸績齊名,而陸遜、張敦、卜靜等皆亞焉。

吳錄曰:敦字叔方,靜字玄風,並吳郡人。敦德量淵懿,清虛淡泊,又善文辭。孫權為車騎將軍,辟西曹掾,轉主簿,出補海昏令,甚有惠化,年三十二卒。卜靜終於剡令。自州郡庶幾及四方人士,往來相見,或言議而去,或結厚而別,風聲流聞,遠近稱之。權妻以策女。年二十七,起家為豫章太守。下車祀先賢徐孺子之墓,優待其後;禁其淫祀非禮之祭者。小吏資質佳者,輒令就學,擇其先進,擢置右職,舉善以教,風化大行。初,錢唐丁諝出於役伍,陽羨張秉生於庶民,烏程吳粲、雲陽殷禮起乎微賤,邵皆拔而友之,為立聲譽。秉遭大喪,親為制服結絰。邵當之豫章,發在近路,值秉疾病,時送者百數,邵辭賔客曰:「張仲節有疾,苦不能來別,恨不見之,暫還與訣,諸君少時相待。」其留心下士,惟善所在,皆此類也。諝至典軍中郎,秉雲陽太守,禮零陵太守,禮子基作通語曰:禮字德嗣,弱不好弄,潛識過人。少為郡吏,年十九,守吳縣丞。孫權為王,召除郎中。後與張溫俱使蜀,諸葛亮甚稱歎之。稍遷至零陵太守,卒官。文士傳曰:禮子基,無難督,以才學知名,著通語數十篇。有三子。巨字元大,有才器,初為吳偏將軍,統家部曲,城夏口,吳平後,為蒼梧太守。少子祐,字慶元,吳郡太守。粲太子少傅。世以邵為知人。在郡五年,卒官,子譚、承云。


\end{pinyinscope}