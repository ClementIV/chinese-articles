\article{jia-xu-chuan}

\begin{pinyinscope}
賈詡字文和,武威姑臧人也。少時人莫知,唯漢陽閻忠異之,謂詡有良、平之奇。

九州春秋曰:中平元年,車騎將軍皇甫嵩旣破黃巾,威震天下。閻忠時罷信都令,說嵩曰:「夫難得而易失者時也,時至而不旋踵者機也,故聖人常順時而動,智者必因機以發。今將軍遭難得之運,蹈易解之機,而踐運不撫,臨機不發,將何以享大名乎?」嵩曰:「何謂也?」忠曰:「天道無親,百姓與能,故有高人之功者,不受庸主之賞。今將軍授鉞於初春,收功於末冬,兵動若神,謀不再計,旬月之間,神兵電掃,攻堅易於折枯,摧敵甚於湯雪,七州席卷,屠三十六萬方,夷黃巾之師,除邪害之患,或封戶刻石,南向以報德,威震本朝,風馳海外。是以羣雄迴首,百姓企踵,雖湯武之舉,未有高於將軍者。身建高人之功,北面以事庸主,將何以圖安?」嵩曰:「心不忘忠,何為不安?」忠曰:「不然。昔韓信不忍一飱之遇,而棄三分之利,拒蒯通之忠,忽鼎跱之勢,利劒已揣其喉,乃嘆息而悔,所以見烹於兒女也。今主勢弱於劉、項,將軍權重於淮陰,指麾可以振風雲,叱咤足以興雷電;赫然奮發,因危抵頹,崇恩以綏前附,振武以臨後服;徵兾方之士,動七州之衆,羽檄先馳於前,大軍震響於後,蹈蹟漳河,飲馬孟津,舉天網以網羅京都,誅閹宦之罪,除羣怨之積忿,解乆危之倒懸。如此則攻守無堅城,不招必影從,雖兒童可使奮空拳以致力,女子可使其褰裳以用命,況厲智能之士,因迅風之勢,則大功不足合,八方不足同也。功業已就,天下已順,乃燎于上帝,告以天命,混齊六合,南面以制,移神器於己家,推亡漢以定祚,實神機之至決,風發之良時也。夫木朽不彫,世衰難佐,將軍雖欲委忠難佐之朝,彫畫朽敗之木,猶逆坂而走丸,必不可也。方今權宦羣居,同惡如市,主上不自由,詔命出左右。如有至聦不察,機事不先,必嬰後悔,亦無及矣。」嵩不從,忠乃亡去。英雄記曰:涼州賊王國等起兵,共劫忠為主,統三十六部,號車騎將軍。忠感慨發病而死。察孝廉為郎,疾病去官,西還至汧,道遇叛氐,同行數十人皆為所執。詡曰:「我段公外孫也,汝別埋我,我家必厚贖之。」時太尉段熲,昔乆為邊將,威震西土,故詡假以懼氐。氐果不敢害,與盟而送之,其餘悉死。詡實非段甥,權以濟事,咸此類也。

董卓之入洛陽,詡以太尉掾為平津都尉,遷討虜校尉。卓壻中郎將牛輔屯陝,詡在輔軍。卓敗,輔又死,衆恐懼,校尉李傕、郭汜、張濟等欲解散,間行歸鄉里。詡曰:「聞長安中議欲盡誅涼州人,而諸君棄衆單行,即一亭長能束君矣。不如率衆而西,所在收兵,以攻長安,為董公報仇,幸而事濟,奉國家以征天下,若不濟,走未後也。」衆以為然。傕乃西攻長安。語在卓傳。臣松之以為傳稱「仁人之言,其利愽哉」!然則不仁之言,理必反是。夫仁功難著,而亂源易成,是故有禍機一發而殃流百世者矣。當是時,元惡旣梟,天地始開,致使厲階重結,大梗殷流,邦國遘殄悴之哀,黎民嬰周餘之酷,豈不由賈詡片言乎?詡之罪也,一何大哉!自古兆亂,未有如此之甚。後詡為左馮翊,傕等欲以功侯之,詡曰:「此救命之計,何功之有!」固辭不受。又以為尚書僕射,詡曰:「尚書僕射,官之師長,天下所望,詡名不素重,非所以服人也。縱詡昧於榮利,柰國朝何!」乃更拜詡尚書,典選舉,多所匡濟,傕等親而憚之。獻帝紀曰:郭汜、樊稠與傕互相違戾,欲鬬者數矣。詡輒以道理責之,頗受詡言。魏書曰:詡典選舉,多選舊名以為令僕,論者以此多詡。會母喪去官,拜光祿大夫。傕、汜等鬬長安中,獻帝紀曰:傕等與詡議,迎天子置其營中。詡曰:「不可。脅天子,非義也。」傕不聽。張繡謂詡曰:「此中不可乆處,君胡不去?」詡曰:「吾受國恩,義不可背。卿自行,我不能也。」傕復請詡為宣義將軍。獻帝紀曰:傕時召羌、胡數千人,先以御物繒綵與之,又許以宮人婦女,欲令攻郭汜。羌、胡數來闚省門,曰:「天子在中邪!李將軍許我宮人美女,今皆安在?」帝患之,使詡為之方計。詡乃密呼羌、胡大帥飲食之,許以封爵重寶,於是皆引去。傕由此衰弱。傕等和,出天子,祐護大臣,詡有力焉。獻帝紀曰:天子旣東,而李傕來追,王師敗績。司徒趙溫、太常王偉、衞尉周忠、司隷榮邵皆為傕所嫌,欲殺之。詡謂傕曰:「此皆天子大臣,卿柰何害之?」傕乃止。天子旣出,詡上還印綬。是時將軍段煨屯華陰,典略稱煨在華陰時,脩農事,不虜略。天子東還,煨迎道貢遺周急。獻帝紀曰:後以煨為大鴻臚光祿大夫,建安十四年,以壽終。與詡同郡,遂去傕託煨。詡素知名,為煨軍所望。煨內恐其見奪,而外奉詡禮甚備,詡愈不自安。

張繡在南陽,詡陰結繡,繡遣人迎詡。詡將行,或謂詡曰:「煨待君厚矣,君安去之?」詡曰:「煨性多疑,有忌詡意,禮雖厚,不可恃,乆將為所圖。我去必喜,又望吾結大援於外,必厚吾妻子。繡無謀主,亦願得詡,則家與身必俱全矣。」詡遂往,繡執子孫禮,煨果善視其家。詡說繡與劉表連和。傅子曰:詡南見劉表,表以客禮待之。詡曰:「表,平世三公才也;不見事變,多疑無決,無能為也。」太祖比征之,一朝引軍退,繡自追之。詡謂繡曰:「不可追也,追必敗。」繡不從,進兵交戰,大敗而還。詡謂繡曰:「促更追之,更戰必勝。」繡謝曰:「不用公言,以至於此。今已敗,柰何復追?」詡曰:「兵勢有變,亟往必利。」繡信之,遂收散卒赴追,大戰,果以勝還。問詡曰:「繡以精兵追退軍,而公曰必敗;退以敗卒擊勝兵,而公曰必剋。悉如公言,何其反而皆驗也?」詡曰:「此易知耳。將軍雖善用兵,非曹公敵也。軍雖新退,曹公必自斷後;追兵雖精,將旣不敵,彼士亦銳,故知必敗。曹公攻將軍無失策,力未盡而退,必國內有故;已破將軍,必輕軍速進,縱留諸將斷後,諸將雖勇,亦非將軍敵,故雖用敗兵而戰必勝也。」繡乃服。是後,太祖拒袁紹於官渡,紹遣人招繡,并與詡書結援。繡欲許之,詡顯於繡坐上謂紹使曰:「歸謝袁本初,兄弟不能相容,而能容天下國士乎?」繡驚懼曰:「何至於此!」竊謂詡曰:「若此,當何歸?」詡曰:「不如從曹公。」繡曰:「袁彊曹弱,又與曹為讎,從之如何?」詡曰:「此乃所以宜從也。夫曹公奉天子以令天下,其宜從一也。紹彊盛,我以少衆從之,必不以我為重。曹公衆弱,其得我必喜,其宜從二也。夫有霸王之志者,固將釋私怨,以明德於四海,其宜從三也。願將軍無疑!」繡從之,率衆歸太祖。太祖見之,喜,執詡手曰:「使我信重於天下者,子也。」表詡為執金吾,封都亭侯,遷兾州牧。兾州未平,留參司空軍事。袁紹圍太祖於官渡,太祖糧方盡,問詡計焉出,詡曰:「公明勝紹,勇勝紹,用人勝紹,決機勝紹,有此四勝而半年不定者,但顧萬全故也。必決其機,須臾可定也。」太祖曰:「善。」乃并兵出,圍擊紹三十餘里營,破之。紹軍大潰,河北平。

太祖領兾州牧,徙詡為太中大夫。建安十三年,太祖破荊州,欲順江東下。詡諫曰:「明公昔破袁氏,今收漢南,威名遠著,軍勢旣大;若乘舊楚之饒,以饗吏士,撫安百姓,使安土樂業,則可不勞衆而江東稽服矣。」太祖不從,軍遂無利。臣松之以為詡之此謀,未合當時之宜。于時韓、馬之徒尚狼顧關右,魏武不得安坐郢都以威懷吳會,亦已明矣。彼荊州者,孫、劉之所必爭也。荊人服劉主之雄姿,憚孫權之武略,為日旣乆,誠非曹氏諸將所能抗禦。故曹仁守江陵,敗不旋踵,何撫安之得行,稽服之可期?將此旣新平江、漢,威懾揚、越,資劉表水戰之具,藉荊楚檝櫂之手,實震蕩之良會,廓定之大機。不乘此取吳,將安俟哉?至於赤壁之敗,蓋有運數。實由疾疫大興,以損淩厲之鋒,凱風自南,用成焚如之勢。天實為之,豈人事哉?然則魏武之東下,非失筭也。詡之此規,為無當矣。魏武後克平張魯,蜀中一日數十驚,劉備雖斬之而不能止,由不用劉曄之計,以失席卷之會,斤石旣差,悔無所及,即亦此事之類也。世咸謂劉計為是,即愈見賈言之非也。太祖後與韓遂、馬超戰於渭南,超等索割地以和,并求任子。詡以為可偽許之。又問詡計策,詡曰:「離之而已。」太祖曰:「解。」一承用詡謀。語在武紀。卒破遂、超,詡本謀也。

是時,文帝為五官將,而臨菑侯植才名方盛,各有黨與,有奪宗之議。文帝使人問詡自固之術,詡曰:「願將軍恢崇德度,躬素士之業,朝夕孜孜,不違子道。如此而已。」文帝從之,深自砥礪。太祖又嘗屏除左右問詡,詡嘿然不對。太祖曰:「與卿言而不荅,何也?」詡曰:「屬適有所思,故不即對耳。」太祖曰:「何思?」詡曰:「思袁本初、劉景升父子也。」太祖大笑,於是太子遂定。詡自以非太祖舊臣,而策謀深長,懼見猜疑,闔門自守,退無私交,男女嫁娶,不結高門,天下之論智計者歸之。

文帝即位,以詡為太尉,魏略曰:文帝得詡之對太祖,故即位首登上司。荀勗別傳曰:晉司徒闕,武帝問其人於勗。荅曰:「三公具瞻所歸,不可用非其人。昔魏文帝用賈詡為三公,孫權笑之。」進爵魏壽鄉侯,增邑三百,并前八百戶。又分邑二百,封小子訪為列侯。以長子穆為駙馬都尉。帝問詡曰:「吾欲伐不從命以一天下,吳、蜀何先?」對曰:「攻取者先兵權,建本者尚德化。陛下應期受禪,撫臨率土,若綏之以文德而俟其變,則平之不難矣。吳、蜀雖蕞爾小國,依岨山水,劉備有雄才,諸葛亮善治國,孫權識虛實,陸議見兵勢,據險守要,汎舟江湖,皆難卒謀也。用兵之道,先勝後戰,量敵論將,故舉無遺策。臣竊料羣臣,無備、權對,雖以天威臨之,未見萬全之勢也。昔舜舞干戚而有苗服,臣以為當今宜先文後武。」文帝不納。後興江陵之役,士卒多死。詡年七十七,薨,謚曰肅侯。子穆嗣,歷位郡守。穆薨,子模嗣。世語曰:模,晉惠帝時為散騎常侍、護軍將軍,模子胤,胤弟龕,從弟疋,皆至大官,並顯於晉也。

評曰:荀彧清秀通雅,有王佐之風,然機鑒先識,未能充其志也。世之論者,多譏彧協規魏氏,以傾漢祚;君臣易位,實彧之由。雖晚節立異,無救運移;功旣違義,識亦疚焉。陳氏此評,蓋亦同乎世識。臣松之以為斯言之作,誠未得其遠大者也。彧豈不知魏武之志氣,非衰漢之貞臣哉?良以于時王道旣微,橫流已及,雄豪虎視,人懷異心,不有撥亂之資,仗順之略,則漢室之亡忽諸,黔首之類殄矣。夫欲翼讚時英,一匡屯運,非斯人之與而誰與哉?是故經綸急病,若救身首,用能動于嶮中,至于大亨,蒼生蒙舟航之接,劉宗延二紀之祚,豈非荀生之本圖,仁恕之遠致乎?及至霸業旣隆,翦漢迹著,然後亡身殉節,以申素情,全大正於當年,布誠心於百代,可謂任重道遠,志行義立。謂之未充,其殆誣歟!荀攸、賈詡,庶乎筭無遺策,經達權變,其良、平之亞與!臣松之以為列傳之體,以事類相從。張子房青雲之士,誠非陳平之倫。然漢之謀臣,良、平而已。若不共列,則餘無所附,故前史合之,蓋其宜也。魏氏如詡之儔,其比幸多,詡不編程、郭之篇,而與二荀並列;失其類矣。且攸、詡之為人,其猶夜光之與蒸燭乎!其照雖均,質則異焉。今荀、賈之評,共同一稱,尤失區別之宜也。


\end{pinyinscope}