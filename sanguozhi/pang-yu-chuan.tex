\article{pang-yu-chuan}

\begin{pinyinscope}
龐淯字子異,酒泉表氏人也。初以涼州從事守破羌長,會武威太守張猛反,殺刺史邯鄲商,猛令曰:「敢有臨商喪,死不赦。」淯聞之,棄官,晝夜奔走,號哭喪所訖,詣猛門,衷匕首,欲因見以殺猛。猛知其義士,勑遣不殺,由是以忠烈聞。

魏略曰:猛兵欲來縛淯,猛聞之,歎曰:「猛以殺刺史為罪。此人以至忠為名,如又殺之,何以勸一州履義之士邪!」遂使行服。典略曰:張猛字叔威,本燉煌人也。猛父奐,桓帝時仕歷郡守、中郎將、太常,遂居華陰,終因葬焉。建安初,猛仕郡為功曹,是時河西四郡以去涼州治遠,隔以河寇,上書求別置州。詔以陳留人邯鄲商為雍州刺史,別典四郡。時武威太守缺,詔又以猛父昔在河西有威名,乃以猛補之。商、猛俱西。初,猛與商同歲,每相戲侮,及共之官,行道更相責望。曁到,商欲誅猛。猛覺之,遂勒兵攻商。商治舍與猛側近,商聞兵至,恐怖登屋,呼猛字曰:「叔威,汝欲殺我耶?然我死者有知,汝亦族矣。請和解,尚可乎?」猛因呼曰;「來。」商踰屋就猛,猛因責數之,語畢,以商屬督郵。督郵錄商,閉置傳舍。後商欲逃,事覺,遂殺之。是歲建安十四年也。至十五年,將軍韓遂自上討猛,猛發兵遣軍東拒。其吏民畏遂,乃反共攻猛。初奐為武威太守時,猛方在孕。母夢帶奐印綬,登樓而歌,旦以告奐。奐訊占夢者,曰:「夫人方生男,後當復臨此郡,其必死官乎!」及猛被攻,自知必死,曰:「使死者無知則已矣,若有知,豈使吾頭東過華陰歷先君之墓乎?」乃登樓自燒而死。太守徐揖請為主簿。後郡人黃昂反,圍城。淯棄妻子,夜踰城出圍,告急於張掖、燉煌二郡。初疑未肯發兵,淯欲伏劒,二郡感其義,遂為興兵。軍未至而郡城邑已陷,揖死。淯乃收歛揖喪,送還本郡,行服三年乃還。太祖聞之,辟為掾屬。文帝踐阼,拜駙馬都尉,遷西海太守,賜爵關內侯。後徵拜中散大夫,薨。子曾嗣。

初,淯外祖父趙安為同縣李壽所殺,淯舅兄弟三人同時病死,壽家喜。淯母娥自傷父讎不報,乃帷車袖劒,白日刺壽於都亭前,訖,徐詣縣,顏色不變,曰:「父讎已報,請受戮。」祿福長尹嘉解印綬縱娥,娥不肯去,遂彊載還家。會赦得免,州郡歎貴,刊石表閭。皇甫謐列女傳曰:酒泉烈女龐娥親者,表氏龐子夏之妻,祿福趙君安之女也。君安為同縣李壽所殺,娥親有男弟三人,皆欲報讎,壽深以為備。會遭災疫,三人皆死。壽聞大喜,請會宗族,共相慶賀,云:「趙氏強壯已盡,唯有女弱,何足復憂!」防備懈弛。娥親子淯出行,聞壽此言,還以啟娥親。娥親旣素有報讎之心,及聞壽言,感激愈深,愴然隕涕曰:「李壽,汝莫喜也,終不活汝!戴履天地,為吾門戶,吾三子之羞也。焉知娥親不手刃殺汝,而自儌倖邪?」陰巿名刀,挾長持短,晝夜哀酸,志在殺壽。壽為人凶豪,聞娥親之言,更乘馬帶刀,鄉人皆畏憚之。比鄰有徐氏婦,憂娥親不能制,恐逆見中害,每諫止之,曰:「李壽,男子也,凶惡有素,加今備衞在身。趙雖有猛烈之志,而彊弱不敵。邂逅不制,則為重受禍於壽,絕滅門戶,痛辱不輕也。願詳舉動,為門戶之計。」娥親曰:「父母之讎,不同天地共日月者也。李壽不死,娥親視息世間,活復何求!今雖三弟早死,門戶泯絕,而娥親猶在,豈可假手於人哉!若以卿心況我,則李壽不可得殺;論我之心,壽必為我所殺明矣。」夜數磨礪所持刀訖,扼腕切齒,悲涕長歎,家人及鄰里咸共笑之。娥親謂左右曰:「卿等笑我,直以我女弱不能殺壽故也。要當以壽頸血汙此刀刃,令汝輩見之。」遂棄家事,乘鹿車伺壽。至光和二年二月上旬,以白日清時,於都亭之前,與壽相遇,便下車扣壽馬,叱之。壽驚愕,迴馬欲走。娥親奮刀斫之,并傷其馬。馬驚,壽擠道邊溝中。娥親尋復就地斫之,探中樹蘭,折所持刀。壽被創未死,娥親因前欲取壽所佩刀殺壽,壽護刀瞋目大呼,跳梁而起。娥親迺挺身奮手,左抵其額,右樁其喉,反覆盤旋,應手而倒。遂拔其刀以截壽頤,持詣都亭,歸罪有司,徐步詣獄,辭顏不變。時祿福長漢陽尹嘉不忍論娥親,即解印綬去官,弛法縱之。娥親曰:「讎塞身死,妾之明分也。治獄制刑,君之常典也。何敢貪生以枉官法?」鄉人聞之,傾城奔往,觀者如堵焉,莫不為之悲喜慷慨嗟嘆也。守尉不敢公縱,陰語使去,以便宜自匿。娥親抗聲大言曰:「枉法逃死,非妾本心。今讎人已雪,死則妾分,乞得歸法以全國體。雖復萬死,於娥親畢足,不敢貪生為明廷負也。」尉故不聽所執,娥親復言曰:「匹婦雖微,猶知憲制。殺人之罪,法所不縱。今旣犯之,義無可逃。乞就刑戮,隕身朝巿,肅明王法,娥親之願也。」辭氣愈厲,面無懼色。尉知其難奪,彊載還家。涼州刺史周洪、酒泉太守劉班等並共表上,稱其烈義,刊石立碑,顯其門閭。太常弘農張奐貴尚所履,以束帛二十端禮之。海內聞之者,莫不改容贊善,高大其義。故黃門侍郎安定梁寬追述娥親,為其作傳。玄晏先生以為父母之讎,不與共天地,蓋男子之所為也。而娥親以女弱之微,念父辱之酷痛,感讎黨之凶言,奮劒仇頸,人馬俱摧,塞亡父之怨魂,雪三弟之永恨,近古以來,未之有也。詩云「脩我戈矛,與子同仇」,娥親之謂也。


\end{pinyinscope}