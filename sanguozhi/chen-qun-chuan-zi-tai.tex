\article{陳羣傳子泰}

\begin{pinyinscope}
陳羣字長文,潁川許昌人也。祖父寔,父紀,叔父諶,皆有盛名。

寔字仲弓,紀字元方,諶字季方。魏書曰:寔德冠當時,紀、諶並名重於世。寔為太丘長,遭黨錮,隱居荊山,遠近宗師之。靈帝崩,何進輔政,引用天下名士,徵寔,欲以為參軍,以老病,遂不屈節,諶為司空掾,早卒。紀歷位平原相、侍中、大鴻臚,著書數十篇,世謂之陳子。寔之亡也,司空荀爽、太僕令韓融並制緦麻,執子孫禮。四方至者車數千乘,自太原郭泰等無不造門。傅子曰:寔亡,天下致弔,會其葬者三萬人,制縗麻者以百數。先賢行狀曰:大將軍何進遣屬弔祠,謚曰文範先生。于時,寔、紀高名並著,而諶又配之,世號曰三君。每宰府辟命,率皆同時,羔鴈成羣,丞掾交至。豫州百姓皆圖畫寔、紀、諶之形象。羣為兒時,寔常奇異之,謂宗人父老曰:「此兒必興吾宗。」魯國孔融高才倨傲,年在紀、羣之間,先與紀友,後與羣交,更為紀拜,由是顯名。劉備臨豫州,辟羣為別駕。時陶謙病死,徐州迎備,備欲往,羣說備曰:「袁術尚彊,今東,必與之爭。呂布若襲將軍之後,將軍雖得徐州,事必無成。」備遂東,與袁術戰。布果襲下邳,遣兵助術,大破備軍,備恨不用羣言。舉茂才,除柘令,不行,隨紀避難徐州。屬呂布破,太祖辟羣為司空西曹掾屬。時有薦樂安王模、下邳周逵者,太祖辟之。羣封還教,以為模、逵穢德,終必敗,太祖不聽。後模、逵皆坐姦宄誅,太祖以謝羣。羣薦廣陵陳矯、丹楊戴乾,太祖皆用之。後吳人叛,乾忠義死難,矯遂為名臣,世以羣為知人。除蕭、贊、長平令,父卒去官。後以司徒掾舉高第,為治書侍御史,轉參丞相軍事。魏國旣建,遷為御史中丞。

時太祖議復肉刑,令曰:「安得通理君子達於古今者,使平斯事乎!昔陳鴻臚以為死刑有可加於仁恩者,正謂此也。御史中丞能申其父之論乎?」羣對曰:「臣父紀以為漢除肉刑而增加笞,本興仁惻而死者更衆,所謂名輕而實重者也。名輕則易犯,實重則傷民。書曰:『惟敬五刑,以成三德。』易著劓、刖、滅趾之法,所以輔政助教,懲惡息殺也。且殺人償死,合於古制;至於傷人,或殘毀其體而裁翦毛髮,非其理也。若用古刑,使淫者下蠶室,盜者刖其足,則永無淫放穿窬之姦矣。夫三千之屬,雖未可悉復,若斯數者,時之所患,宜先施用。漢律所殺殊死之罪,仁所不及也,其餘逮死者,可以刑殺。如此,則所刑之與所生足以相貿矣。今以笞死之法易不殺之刑,是重人支體而輕人軀命也。」時鍾繇與羣議同,王朗及議者多以為未可行。太祖深善繇、羣言,以軍事未罷,顧衆議,故且寢。

羣轉為侍中,領丞相東西曹掾。在朝無適無莫,雅杖名義,不以非道假人。文帝在東宮,深敬器焉,待以交友之禮,常歎曰:「自吾有回,門人日以親。」及即王位,封羣昌武亭侯,徙為尚書。制九品官人之法,羣所建也。及踐阼,遷尚書僕射,加侍中,徙尚書令,進爵潁鄉侯。帝征孫權,至廣陵,使羣領中領軍。帝還,假節,都督水軍。還許昌,以羣為鎮軍大將軍,領中護軍,錄尚書事。帝寢疾,羣與曹真、司馬宣王等並受遺詔輔政。明帝即位,進封潁陰侯,增邑五百,并前千三百戶,與征東大將軍曹休、中軍大將軍曹真、撫軍大將軍司馬宣王並開府。頃之,為司空,故錄尚書事。

是時,帝初莅政,羣上疏曰:「詩稱『儀刑文王,萬邦作孚』;又曰『刑于寡妻,至于兄弟,以御于家邦』。道自近始,而化洽於天下。自喪亂已來,干戈未戢,百姓不識王教之本,懼其陵遲巳甚。陛下當盛魏之隆,荷二祖之業,天下想望至治,唯有以崇德布化,惠恤黎庶,則兆民幸甚。夫臣下雷同,是非相蔽,國之大患也。若不和睦則有讎黨,有讎黨則毀譽無端,毀譽無端則真偽失實,不可不深防備,有以絕其源流。」太和中,曹真表欲數道伐蜀,從斜谷入。羣以為「太祖昔到陽平攻張魯,多收豆麥以益軍糧,魯未下而食猶乏。今旣無所因,且斜谷阻險,難以進退,轉運必見鈔截,多留兵守要,則損戰士,不可不熟慮也」。帝從羣議。真復表從子午道。羣又陳其不便,并言軍事用度之計。詔以羣議下真,真據之遂行。會霖雨積日,羣又以為宜詔真還,帝從之。

後皇女淑薨,追封謚平原懿公主。羣上疏曰:「長短有命,存亡有分。故聖人制禮,或抑或致,以求厥中。防墓有不脩之儉,嬴、博有不歸之䰟。夫大人動合天地,垂之無窮,又大德不踰閑,動為師表故也。八歲下殤,禮所不備,況未朞月,而以成人禮送之,加為制服,舉朝素衣,朝夕哭臨,自古已來,未有此比。而乃復自往視陵,親臨祖載。願陛下抑割無益有損之事,但悉聽羣臣送葬,乞車駕不行,此萬國之至望也。聞車駕欲幸摩陂,實到許昌,二宮上下,皆悉俱東,舉朝大小,莫不驚怪。或言欲以避衰,或言欲於便處移殿舍,或不知何故。臣以為吉凶有命,禍福由人,移徙求安,則亦無益。若必當移避,繕治金墉城西宮,及孟津別宮,皆可權時分止。可無舉宮暴露野次,廢損盛節蠶農之要。又賊地聞之,以為大衰。加所煩費,不可計量。且由吉士賢人,當盛衰,處安危,秉道信命,非徙其家以寧,鄉邑從其風化,無恐懼之心。況乃帝王萬國之主,靜則天下安,動則天下擾;行止動靜,豈可輕脫哉?」帝不聽。

青龍中,營治宮室,百姓失農時。羣上疏曰:「禹承唐、虞之盛,猶卑宮室而惡衣服,況今喪亂之後,人民至少,比漢文、景之時,不過一大郡。臣松之案:漢書地理志云:元始二年,天下戶口最盛,汝南郡為大郡,有三十餘萬戶。則文、景之時不能如是多也。案晉太康三年地記,晉戶有三百七十七萬,吳、蜀戶不能居半。以此言之,魏雖始承喪亂,方晉亦當無乃大殊。長文之言,於是為過。加邊境有事,將士勞苦,若有水旱之患,國家之深憂也。且吳、蜀未滅,社稷不安。宜及其未動,講武勸農,有以待之。今舍此急而先宮室,臣懼百姓遂困,將何以應敵?昔劉備自成都至白水,多作傳舍,興費人役,太祖知其疲民也。今中國勞力,亦吳、蜀之所願。此安危之機也,惟陛下慮之。」帝荅曰:「王者宮室,亦宜並立。滅賊之後,但當罷守耳,豈可復興役邪?是故君之職,蕭何之大略也。」羣又曰:「昔漢祖唯與項羽爭天下,羽已滅,宮室燒焚,是以蕭何建武庫、太倉,皆是要急,然猶非其壯麗。今二虜未平,誠不宜與古同也。孫盛曰:周禮,天子之宮,有斲礲之制。然質文之飾,與時推移。漢承周、秦之弊,宜敦簡約之化,而何崇飾宮室,示侈後嗣。此乃武帝千門萬戶所以大興,豈無所復增之謂邪?況乃魏氏方有吳、蜀之難,四海罹塗炭之艱,而述蕭何之過議,以為令軌,豈不惑於大道而昧得失之辨哉?使百代之君眩於奢儉之中,何之由矣。詩云:「斯言之玷,不可為也。」其斯之謂乎!夫人之所欲,莫不有辭,況乃天王,莫之敢違。前欲壞武庫,謂不可不壞也;後欲置之,謂不可不置也。若必作之,固非臣下辭言所屈;若少留神,卓然回意,亦非臣下之所及也。漢明帝欲起德陽殿,鍾離意諫,即用其言,後乃復作之;殿成,謂羣臣曰:『鍾離尚書在,不得成此殿也。』夫王者豈憚一臣,蓋為百姓也。今臣曾不能少凝聖聽,不及意遠矣。」帝於是有所減省。

初,太祖時,劉廙坐弟與魏諷謀反,當誅。羣言之太祖,太祖曰:「廙,名臣也,吾亦欲赦之。」乃復位。廙深德羣,羣曰:「夫議刑為國,非為私也;且自明主之意,吾何知焉?」其弘博不伐,皆此類也。青龍四年薨,謚曰靖侯。子泰嗣。帝追思羣功德,分羣戶邑,封一子列侯。魏書曰:羣前後數密陳得失,每上封事,輒削其草,時人及其子弟莫能知也。論者或譏羣居位拱默,正始中詔撰羣臣上書,以為名臣奏議,朝士乃見羣諫事,皆歎息焉。袁子曰:或云「故少府楊阜豈非忠臣哉?見人主之非,則勃然怒而觸之,與人言未嘗不道也,豈非所謂『王臣謇謇,匪躬之故』者歟!」荅曰:「然可謂直士,忠則吾不知也。夫仁者愛人。施於君謂之忠,施於親謂之孝。忠孝者,其本一也。故仁愛之至者,君親有過,諫而不入,求之反覆,不得已而言,不忍宣也。今為人臣,見人主失道,直詆其非而播揚其惡,可謂直士,未為忠臣也。故司空陳羣則不然,其談論終日,未嘗言人主之非;書數十上而外人不知。君子謂羣於是乎長者矣。」

泰字玄伯。青龍中,除散騎侍郎。正始中,徙游擊將軍,為并州刺史,加振威將軍,使持節,護匈奴中郎將,懷柔夷民,甚有威惠。京邑貴人多寄寶貨,因泰市奴婢,泰皆挂之於壁,不發其封,及徵為尚書,悉以還之。嘉平初,代郭淮為雍州刺史,加奮威將軍。蜀大將軍姜維率衆依麴山築二城,使牙門將句安、李歆等守之,聚羌胡質任等寇偪諸郡。征西將軍郭淮與泰謀所以禦之,泰曰:「麴城雖固,去蜀險遠,當須運糧。羌夷患維勞役,必未肯附。今圍而取之,可不血刃而拔其城;雖其有救,山道阻險,非行兵之地也。」淮從泰計,使泰率討蜀護軍徐質、南安太守鄧艾等進兵圍之,斷其運道及城外流水。安等挑戰,不許,將士困窘,分糧聚雪以稽日月。維果來救,出自牛頭山,與泰相對。泰曰:「兵法貴在不戰而屈人。今絕牛頭,維無反道,則我之禽也。」勑諸軍各堅壘勿與戰,遣使白淮,欲自南渡白水,循水而東,使淮趣牛頭,截其還路,可并取維,不惟安等而已。淮善其策,進率諸軍軍洮水。維懼,遁走,安等孤縣,遂皆降。

淮薨,泰代為征西將軍,假節都督雍、涼諸軍事。後年,雍州刺史王經白泰,云姜維、夏侯霸欲三道向祁山、石營、金城,求進兵為翅,使涼州軍至枹䍐,討蜀護軍向祁山。泰量賊勢終不能三道,且兵勢惡分,涼州未宜越境,報經:「審其定問,知所趣向,須東西勢合乃進。」時維等將數萬人至枹䍐,趣狄道。泰勑經進屯狄道,須軍到,乃規取之。泰進軍陳倉。會經所統諸軍於故關與賊戰不利,經輒渡洮。泰以經不堅據狄道,必有它變。並遣五營在前,泰率諸軍繼之。經巳與維戰,大敗,以萬餘人還保狄道城,餘皆奔散。維乘勝圍狄道。泰軍上邽,分兵守要,晨夜進前。鄧艾、胡奮、王秘亦到,即與艾、祕等分為三軍,進到隴西。艾等以為「王經精卒破衂於西,賊衆大盛,乘勝之兵旣不可當,而將軍以烏合之卒,繼敗軍之後,將士失氣,隴右傾蕩。古人有言:『蝮蛇螫手,壯士解其腕。』孫子曰:『兵有所不擊,地有所不守。』蓋小有所失而大有所全故也。今隴右之害,過於蝮蛇,狄道之地,非徒不守之謂。姜維之兵,是所辟之鋒。不如割險自保,觀釁待弊,然後進救,此計之得者也。」泰曰:「姜維提輕兵深入,正欲與我爭鋒原野,求一戰之利。王經當高壁深壘,挫其銳氣。今乃與戰,使賊得計,走破王經,封之狄道。若維以戰克之威,進兵東向,據櫟陽積穀之實,放兵收降,招納羌、胡,東爭關、隴,傳檄四郡,此我之所惡也。而維以乘勝之兵,挫峻城之下,銳氣之卒,屈力致命,攻守勢殊,客主不同。兵書云『脩櫓橨榅,三月乃成,拒堙三月而後已』。誠非輕軍遠入,維之詭謀倉卒所辦。縣軍遠僑,糧穀不繼,是我速進破賊之時也,所謂疾雷不及掩耳,自然之勢也。洮水帶其表,維等在其內,今乘高據勢,臨其項領,不戰必走。寇不可縱,圍不可乆,君等何言如此?」遂進軍度高城嶺,潛行,夜至狄道東南高山上,多舉烽火,鳴鼓角。狄道城中將士見救者至,皆憤踊。維始謂官救兵當須衆集乃發,而卒聞已至,謂有奇變宿謀,上下震懼。自軍之發隴西也,以山道深險,賊必設伏。泰詭從南道,維果三日施伏。臣松之案:此傳云「謂救兵當須衆集,而卒聞已至,謂有奇變,上下震懼」,此則救至出於不意。若不知救至,何故伏兵深險乃經三日乎?設伏相伺,非不知之謂。此皆語之不通也。定軍潛行,卒出其南。維乃緣山突至,泰與交戰,維退還。涼州軍從金城南至沃于阪。泰與經共密期,當共向其還路,維等聞之,遂遁,城中將士得出。經歎曰:「糧不至旬,向不應機,舉城屠裂,覆喪一州矣。」泰慰勞將士,前後遣還,更差軍守,並治城壘,還屯上邽。

初,泰聞經見圍,以州軍將士素皆一心,加得保城,非維所能卒傾。表上進軍晨夜速到還。衆議以經奔北,城不足自固,維若斷涼州之道,兼四郡民夷,據關、隴之險,敢能沒經軍而屠隴右。宜須大兵四集,乃致攻討。大將軍司馬文王曰:「昔諸葛亮常有此志,卒亦不能。事大謀遠,非維所任也。且城非倉卒所拔,而糧少為急,征西速救,得上策矣。」泰每以一方有事,輒以虛聲擾動天下,故希簡白上事,驛書不過六百里。司馬文王語荀顗曰:「玄伯沈勇能斷,荷方伯之重,救將陷之城,而不求益兵,又希簡上事,必能辦賊故也。都督大將,不當爾邪!」

後徵泰為尚書右僕射,典選舉,加侍中光祿大夫。吳大將孫峻出淮、泗。以泰為鎮軍將軍,假節都督淮北諸軍事,詔徐州監軍已下受泰節度。峻退,軍還,轉為左僕射。諸葛誕作亂壽春,司馬文王率六軍軍丘頭,泰總署行臺。司馬景王、文王皆與泰親友,及沛國武陔亦與泰善。文王問陔曰:「玄伯何如其父司空也?」陔曰:「通雅博暢,能以天下聲教為己任者,不如也;明統簡至,立功立事,過之。」泰前後以功增邑二千六百戶,賜子弟一人亭侯,二人關內侯。景元元年薨,追贈司空。謚曰穆侯。干寶晉紀曰:高貴鄉公之殺,司馬文王會朝臣謀其故。太常陳泰不至,使其舅荀顗召之。顗至,告以可否。泰曰:「世之論者,以泰方於舅,今舅不如泰也。」子弟內外咸共逼之,垂涕而入。王待之曲室,謂曰:「玄伯,卿何以處我?」對曰:「誅賈充以謝天下。」文王曰:「為吾更思其次。」泰曰:「泰言惟有進於此,不知其次。」文王乃不更言。魏氏春秋曰:帝之崩也,太傅司馬孚、尚書右僕射陳泰枕帝尸於股,號哭盡哀。時大將軍入于禁中,泰見之悲慟,大將軍亦對之泣,謂曰:「玄伯,其如我何?」泰曰:「獨有斬賈充,少可以謝天下耳。」大將軍乆之曰:「卿更思其他。」泰曰:「豈可使泰復發後言。」遂歐血薨。臣松之案本傳,泰不為太常,未詳干寶所由知之。孫盛改易泰言,雖為小勝。然檢盛言諸所改易,皆非別有異聞,率更自以意制,多不如舊。凡記言之體,當使若出其口。辭勝而違實,固君子所不取,況復不勝而徒長虛妄哉?案博物記曰:太丘長陳寔、寔子鴻臚紀、紀子司空羣、羣子泰四世,於漢、魏二朝並有重名,而其德漸漸小減。時人為其語曰:「公慙卿,卿慙長。」子恂嗣。恂薨,無嗣。弟溫紹封。咸熈中開建五等,以泰著勳前朝,改封溫為慎子。案陳氏譜:羣之後,名位遂微。諶孫佐,官至青州刺史。佐弟坦,廷尉。佐子準,太尉,封廣陵郡公。準弟戴、徵及從弟堪,並至大位。準孫逵,字林道,有譽江左,為西中郎將,追贈衞將軍。


\end{pinyinscope}