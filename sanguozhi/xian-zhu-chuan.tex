\article{先主傳}

\begin{pinyinscope}
先主姓劉,諱備,字玄德,涿郡涿縣人,漢景帝子中山靖王勝之後也。勝子貞,元狩六年封涿縣陸城亭侯。坐酎金失侯,因家焉。

典略曰:備本臨邑侯枝屬也。先主祖雄,父弘,世仕州郡。雄舉孝廉,官至東郡范令。

先主少孤,與母販履織席為業。舍東南角籬上有桑樹生高五丈餘,遙望見童童如小車蓋,往來者皆怪此樹非凡,或謂當出貴人。漢晉春秋曰:涿人李定云:「此家必出貴人。」先主少時與宗中諸小兒於樹下戲,言:「吾必當乘此羽葆蓋車。」叔父子敬謂曰:「汝勿妄語,滅吾門也!」年十五,母使行學,與同宗劉德然、遼西公孫瓚俱事故九江太守同郡盧植。德然父元起常資給先主,與德然等。元起妻曰:「各自一家,何能常爾邪!」起曰:「吾宗中有此兒,非常人也。」而瓚深與先主相友。瓚年長,先主以兄事之。先主不甚樂讀書,喜狗馬、音樂、美衣服。身長七尺五寸,垂手下膝,顧自見其耳。少語言,善下人,喜怒不形於色。好交結豪俠,年少爭附之。中山大商張世平、蘇雙等貲累千金,販馬周旋於涿郡,見而異之,乃多與之金財。先主由是得用合徒衆。

靈帝末,黃巾起,州郡各舉義兵,先主率其屬從校尉鄒靖討黃巾賊有功,除安喜尉。典略曰:平原劉子平知備有武勇,時張純反叛,青州被詔,遣從事將兵討純,過平原,子平薦備於從事,遂與相隨,遇賊於野,備中創陽死,賊去後,故人以車載之,得免。後以軍功,為中山安喜尉。督郵以公事到縣,先主求謁,不通,直入縛督郵,杖二百,解綬繫其頸著馬枊,五葬反。棄官亡命。典略曰:其後州郡被詔書,其有軍功為長吏者,當沙汰之,備疑在遣中。督郵至縣,當遣備,備素知之。聞督郵在傳舍,備欲求見督郵,督郵稱疾不肯見備,備恨之,因還治,將吏卒更詣傳舍,突入門,言「我被府君密教收督郵」。遂就牀縛之,將出到界,自解其綬以繫督郵頸,縛之著樹,鞭杖百餘下,欲殺之。督郵求哀,乃釋去之。頃之,大將軍何進遣都尉毌丘毅詣丹楊募兵,先主與俱行,至下邳遇賊,力戰有功,除為下密丞。復去官。後為高唐尉,遷為令。英雄記云:靈帝末年,備甞在京師,後與曹公俱還沛國,募召合衆。會靈帝崩,天下大亂,備亦起軍從討董卓。為賊所破,往奔中郎將公孫瓚,瓚表為別部司馬,使為青州刺史田楷以拒兾州牧袁紹。數有戰功,試守平原令,後領平原相。郡民劉平素輕先主,恥為之下,使客刺之。客不忍刺,語之而去。其得人心如此。魏書曰:劉平結客刺備,備不知而待客甚厚,客以狀語之而去。是時人民飢饉,屯聚鈔暴。備外禦寇難,內豐財施,士之下者,必與同席而坐,同簋而食,無所簡擇。衆多歸焉。

袁紹攻公孫瓚,先主與田楷東屯齊。曹公征徐州,徐州牧陶謙遣使告急於田楷,楷與先主俱救之。時先主自有兵千餘人,及幽州烏丸雜胡騎,又略得飢民數千人。旣到,謙以丹楊兵四千益先主,先主遂去楷歸謙。謙表先主為豫州刺史,屯小沛。謙病篤,謂別駕麋笁曰:「非劉備不能安此州也。」謙死,笁率州人迎先主,先主未敢當。下邳陳登謂先主曰:「今漢室陵遲,海內傾覆,立功立事,在於今日。彼州殷富,戶口百萬,欲屈使君撫臨州事。」先主曰:「袁公路近在壽春,此君四世五公,海內所歸,君可以州與之。」登曰:「公路驕豪,非治亂之主。今欲為使君合步騎十萬,上可以匡主濟民,成五霸之業,下可以割地守境,書功於竹帛。若使君不見聽許,登亦未敢聽使君也。」北海相孔融謂先主曰:「袁公路豈憂國忘家者邪?冢中枯骨,何足介意。今日之事,百姓與能,天與不取,悔不可追。」先主遂領徐州。獻帝春秋曰:陳登等遣使詣袁紹曰:「天降灾沴,禍臻鄙州,州將殂殞,生民無主,恐懼姦雄一旦承隙,以貽盟主日昃之憂,輒共奉故平原相劉備府君以為宗主,永使百姓知有依歸。方今寇難縱橫,不遑釋甲,謹遣下吏奔告于執事。」紹荅曰:「劉玄德弘雅有信義,今徐州樂戴之,誠副所望也。」袁術來攻先主,先主拒之於盱眙、淮陰。曹公表先主為鎮東將軍,封宜城亭侯,是歲建安元年也。先主與術相持經月,呂布乘虛襲下邳。下邳守將曹豹反,間迎布。布虜先主妻子,先主轉軍海西。英雄記曰:備留張飛守下邳,引兵與袁術戰於淮陰石亭,更有勝負。陶謙故將曹豹在下邳,張飛欲殺之。豹衆堅營自守,使人招呂布。布取下邳,張飛敗走。備聞之,引兵還,北至下邳,兵潰。收散卒,東取廣陵,與袁術戰,又敗。楊奉、韓暹寇徐、揚閒,先主邀擊,盡斬之。先主求和於呂布,布還其妻子。先主遣關羽守下邳。

先主還小沛,英雄記曰:備軍在廣陵,飢餓困踧,吏士大小自相啖食,窮餓侵逼,欲還小沛,遂使吏請降布。布令備還州,并勢擊術。具刺史車馬童僕,發遣備妻子部曲家屬於泗水上,祖道相樂。魏書曰:諸將謂布曰:「備數反覆難養,宜早圖之。」布不聽,以狀語備。備心不安而求自託,使人說布,求屯小沛,布乃遣之。復合兵得萬餘人。呂布惡之,自出兵攻先主,先主敗走歸曹公。曹公厚遇之,以為豫州牧。將至沛收散卒,給其軍糧,益與兵使東擊布。布遣高順攻之,曹公遣夏侯惇往,不能救,為順所敗,復虜先主妻子送布。曹公自出東征,英雄記曰:建安三年春,布使人齎金欲詣河內買馬,為備兵所鈔。布由是遣中郎將高順、北地太守張遼等攻備。九月,遂破沛城,備單身走,獲將士妻息。十月,曹公自征布,備於梁國界中與曹公相遇,遂隨公俱東征。助先主圍布於下邳,生禽布。先主復得妻子,從曹公還許。表先主為左將軍,禮之愈重,出則同輿,坐則同席。袁術欲經徐州北就袁紹,曹公遣先主督朱靈、路招要擊術。未至,術病死。

先主未出時,獻帝舅車騎將軍董承臣松之按:董承,漢靈帝母董太后之姪,於獻帝為丈人。蓋古無丈人之名,故謂之舅也。辭受帝衣帶中密詔,當誅曹公。先主未發。是時曹公從容謂先主曰:「今天下英雄,惟使君與操耳。本初之徒,不足數也。」先主方食,失匕箸。華陽國志云:于時正當雷震,備因謂操曰:「聖人云『迅雷風烈必變』,良有以也。一震之威,乃可至於此也!」遂與承及長水校尉种輯、將軍吳子蘭、王子服等同謀。會見使,未發。事覺,承等皆伏誅。獻帝起居注曰:承等與備謀未發,而備出。承謂服曰:「郭多有數百兵,壞李傕數萬人,但足下與吾同不耳!昔呂不韋之門,須子楚而後高,今吾與子由是也。」服曰:「惶懼不敢當,且兵又少。」承曰:「舉事訖,得曹公成兵,顧不足邪?」服曰:「今京師豈有所任乎?」承曰:「長水校尉种輯、議郎吳碩是吾腹心辦事者。」遂定計。

先主據下邳。靈等還,先主乃殺徐州刺史車冑,留關羽守下邳,而身還小沛。胡沖吳歷曰:曹公數遣親近密覘諸將有賔客酒食者,輒因事害之。備時閉門,將人種蕪菁,曹公使人闚門。旣去,備謂張飛、關羽曰:「吾豈種菜者乎?曹公必有疑意,不可復留。」其夜開後柵,與飛等輕騎俱去,所得賜遺衣服,悉封留之,乃往小沛收合兵衆。臣松之案:魏武帝遣先主統諸將要擊袁術,郭嘉等並諫,魏武不從,其事顯然,非因種菜遁逃而去。如胡沖所云,何乖僻之甚乎!東海昌霸反,郡縣多叛曹公為先主,衆數萬人,遣孫乾與袁紹連和,曹公遣劉岱、王忠擊之,不克。五年,曹公東征先主,先主敗績。魏書曰:是時公方有急於官渡,乃分留諸將屯官渡,自勒精兵征備。備初謂公與大敵連,不得東,而候騎卒至,言曹公自來。備大驚,然猶未信。自將數十騎出望公軍,見麾旌,便棄衆而走。曹公盡收其衆,虜先主妻子,并禽關羽以歸。

先主走青州。青州刺史袁譚,先主故茂才也,將步騎迎先主。先主隨譚到平原,譚馳使白紹。紹遣將道路奉迎,身去鄴二百里,與先主相見。魏書曰:備歸紹,紹父子傾心敬重。駐月餘日,所失亡士卒稍稍來集。曹公與袁紹相拒於官渡,汝南黃巾劉辟等叛曹公應紹。紹遣先主將兵與辟等略許下。關羽亡歸先主。曹公遣曹仁將兵擊先主,先主還紹軍,陰欲離紹,乃說紹南連荊州牧劉表。紹遣先主將本兵復至汝南,與賊龔都等合,衆數千人。曹公遣蔡楊擊之,為先主所殺。

曹公旣破紹,自南擊先主。先主遣麋笁、孫乾與劉表相聞,表自郊迎,以上賔禮待之,益其兵,使屯新野。荊州豪傑歸先主者日益多,表疑其心,陰禦之。九州春秋曰:備住荊州數年,甞於表坐起至厠,見髀裏肉生,慨然流涕。還坐,表怪問備,備曰:「吾常身不離鞍,髀肉皆消。今不復騎,髀裏肉生。日月若馳,老將至矣,而功業不建,是以悲耳。」世語曰:備屯樊城,劉表禮焉,憚其為人,不甚信用。曾請備宴會,蒯越、蔡瑁欲因會取備,備覺之,偽如厠,潛遁出。所乘馬名的盧,騎的盧走,墮襄陽城西檀溪水中,溺不得出。備急曰:「的盧:今日厄矣,可努力!」的盧乃一踊三丈,遂得過,乘桴渡河,中流而追者至,以表意謝之,曰:「何去之速乎!」孫盛曰:此不然之言。備時羈旅,客主勢殊,若有此變,豈敢晏然終表之世而無釁故乎?此皆世俗妄說,非事實也。使拒夏侯惇、于禁等於博望。乆之,先主設伏兵,一旦自燒屯偽遁,惇等追之,為伏兵所破。

十二年,曹公北征烏丸,先主說表襲許,表不能用。漢晉春秋曰:曹公自柳城還,表謂備曰:「不用君言,故為失此大會。」備曰:「今天下分裂,日尋干戈,事會之來,豈有終極乎?若能應之於後者,則此未足為恨也。」曹公南征表,會表卒,英雄記曰:表病,上備領荊州刺史。魏書曰:表病篤,託國於備,顧謂曰:「我兒不才,而諸將並零落,我死之後,卿便攝荊州。」備曰:「諸子自賢,君其憂病。」或勸備宜從表言,備曰:「此人待我厚,今從其言,人必以我為薄,所不忍也。」臣松之以為表夫妻素愛琮,捨適立庶,情計乆定,無緣臨終舉荊州以授備,此亦不然之言。子琮代立,遣使請降。先主屯樊,不知曹公卒至,至宛乃聞之,遂將其衆去。過襄陽,諸葛亮說先主攻琮,荊州可有。先主曰:「吾不忍也。」孔衍漢魏春秋曰:劉琮乞降,不敢告備。備亦不知,乆之乃覺,遣所親問琮。琮令宋忠詣備宣旨。是時曹公在宛,備乃大驚駭,謂忠曰:「卿諸人作事如此,不早相語,今禍至方告我,不亦太劇乎!」引刀向忠曰:「今斷卿頭,不足以解忿,亦恥大丈夫臨別復殺卿輩!」遣忠去,乃呼部曲議。或勸備劫將琮及荊州吏士徑南到江陵,備荅曰:「劉荊州臨亡託我以孤遺,背信自濟,吾所不為,死何面目以見劉荊州乎!」乃駐馬呼琮,琮懼不能起。琮左右及荊州人多歸先主。典略曰:備過辭表墓,遂涕泣而去。比到當陽,衆十餘萬,輜重數千兩,日行十餘里,別遣關羽乘船數百艘,使會江陵。或謂先主曰:「宜速行保江陵,今雖擁大衆,被甲者少,若曹公兵至,何以拒之?」先主曰:「夫濟大事必以人為本,今人歸吾,吾何忍棄去!」習鑿齒曰:先主雖顛沛險難而信義愈明,勢偪事危而言不失道。追景升之顧,則情感三軍;戀赴義之士,則甘與同敗。觀其所以結物情者,豈徒投醪撫寒含蓼問疾而已哉!其終濟大業,不亦宜乎!

曹公以江陵有軍實,恐先主據之,乃釋輜重,輕軍到襄陽。聞先主已過,曹公將精騎五千急追之,一日一夜行三百餘里,及於當陽之長坂。先主棄妻子,與諸葛亮、張飛、趙雲等數十騎走,曹公大獲其人衆輜重。先主斜趣漢津,適與羽舩會,得濟沔,遇表長子江夏太守琦衆萬餘人,與俱到夏口。先主遣諸葛亮自結於孫權,江表傳曰:孫權遣魯肅弔劉表二子,并令與備相結。肅未至而曹公已濟漢津。肅故進前,與備相遇於當陽。因宣權旨,論天下事勢,致殷勤之意。且問備曰:「豫州今欲何至?」備曰:「與蒼梧太守吳巨有舊,欲往投之。」肅曰:「孫討虜聦明仁惠,敬賢禮士,江表英豪咸歸附之,已據有六郡,兵精糧多,足以立事。今為君計,莫若遣腹心使自結於東,崇連和之好,共濟世業,而云欲投吳巨,巨是凡人,偏在遠郡,行將為人所併,豈足託乎?」備大喜,進住鄂縣,即遣諸葛亮隨肅詣孫權,結同盟誓。權遣周瑜、程普等水軍數萬與先主并力,江表傳曰:備從魯肅計,進住鄂縣之樊口。諸葛亮詣吳未還,備聞曹公軍下,恐懼,日遣邏吏於水次候望權軍。吏望見瑜船,馳往白備,備曰:「何以知之非青徐軍邪?」吏對曰:「以船知之。」備遣人慰勞之。瑜曰:「有軍任,不可得委署,儻能屈威,誠副其所望。」備謂關羽、張飛曰:「彼欲致我,我今自結託於東而不往,非同盟之意也。」乃乘單舸往見瑜,問曰:「今拒曹公,深為得計。戰卒有幾?」瑜曰:「三萬人。」備曰:「恨少。」瑜曰:「此自足用,豫州但觀瑜破之。」備欲呼魯肅等共會語,瑜曰:「受命不得妄委署,若欲見子敬,可別過之。又孔明已俱來,不過三兩日到也。」備雖深愧異瑜,而心未許之能必破北軍也,故差池在後,將二千人與羽、飛俱,未肯係瑜,蓋為進退之計也。孫盛曰:劉備雄才,處必亡之地,告急於吳,而獲奔助,無緣復顧望江渚而懷後計。江表傳之言,當是吳人欲專美之辭。與曹公戰于赤壁,大破之,焚其舟船。先主與吳軍水陸并進,追到南郡,時又疾疫,北軍多死,曹公引歸。江表傳曰:周瑜為南郡太守,分南岸地以給備。備別立營於油江口,改名為公安。劉表吏士見從北軍,多叛來投備。備以瑜所給地少,不足以安民,後從權借荊州數郡。

先主表琦為荊州刺史,又南征四郡。武陵太守金旋、長沙太守韓玄、桂陽太守趙範、零陵太守劉度皆降。三輔決錄注曰:金旋字元機,京兆人,歷位黃門郎、漢陽太守,徵拜議郎,遷中郎將,領武陵太守,為備所攻劫死。子禕,事見魏武本紀。廬江雷緒率部曲數萬口稽顙。琦病死,羣下推先主為荊州牧,治公安。江表傳曰:備別立營於油江口,改名為公安。權稍畏之,進妹固好。先主至京見權,綢繆恩紀。山陽公載記曰:備還,謂左右曰:「孫車騎長上短下,其難為下,吾不可以再見之。」乃晝夜兼行。臣松之案:魏書載劉備與孫權語,與蜀志述諸葛亮與權語正同。劉備未破魏軍之前,尚未與孫權相見,不得有此說。故知蜀志為是。權遣使云欲共取蜀,或以為宜報聽許,吳終不能越荊有蜀,蜀地可為己有。荊州主簿殷觀進曰:「若為吳先驅,進未能克蜀,退為吳所乘,即事去矣。今但可然贊其伐蜀,而自說新據諸郡,未可與動,吳必不敢越我而獨取蜀。如此進退之計,可以收吳、蜀之利。」先主從之,權果輟計。遷觀為別駕從事。獻帝春秋曰:孫權欲與備共取蜀,遣使報備曰:「米賊張魯居王巴、漢,為曹操耳目,規圖益州。劉璋不武,不能自守。若操得蜀,則荊州危矣。今欲先攻取璋,進討張魯,首尾相連,一統吳、楚,雖有十操,無所憂也。」備欲自圖蜀,拒荅不聽,曰:「益州民富彊,土地險阻,劉璋雖弱,足以自守。張魯虛偽,未必盡忠於操。今暴師於蜀、漢,轉運於萬里,欲使戰克攻取,舉不失利,此吳起不能定其規,孫武不能善其事也。曹操雖有無君之心,而有奉主之名,議者見操失利於赤壁,謂其力屈,無復遠志也。今操三分天下已有其二,將欲飲馬於滄海,觀兵於吳會,何肯守此坐須老乎?今同盟無故自相攻伐,借樞於操,使敵承其隙,非長計也。」權不聽,遣孫瑜率水軍住夏口。備不聽軍過,謂瑜曰:「汝欲取蜀,吾當被髮入山,不失信於天下也。」使關羽屯江陵,張飛屯秭歸,諸葛亮據南郡,備自住潺陵。權知備意,因召瑜還。

十六年,益州牧劉璋遙聞曹公將遣鍾繇等向漢中討張魯,內懷恐懼。別駕從事蜀郡張松說璋曰:「曹公兵彊無敵於天下,若因張魯之資以取蜀土,誰能禦之者乎?」璋曰:「吾固憂之而未有計。」松曰:「劉豫州,使君之宗室而曹公之深讎也,善用兵,若使之討魯,魯必破。魯破,則益州彊,曹公雖來,無能為也。」璋然之,遣法正將四千人迎先主,前後賂遺以巨億計。正因陳益州可取之策。吳書曰:備前見張松,後得法正,皆厚以恩意接納,盡其殷勤之歡。因問蜀中闊狹,兵器府庫人馬衆寡,及諸要害道里遠近,松等具言之,又畫地圖山川處所,由是盡知益州虛實也。先主留諸葛亮、關羽等據荊州,將步卒數萬人入益州。至涪,璋自出迎,相見甚歡。張松令法正白先主,及謀臣龐統進說,便可於會所襲璋。先主曰:「此大事也,不可倉卒。」璋推先主行大司馬,領司隷校尉;先主亦推璋行鎮西大將軍,領益州牧。璋增先主兵,使擊張魯,又令督白水軍。先主并軍三萬餘人,車甲器械資貨甚盛。是歲,璋還成都。先主北到葭萌,未即討魯,厚樹恩德,以收衆心。

明年,曹公征孫權,權呼先主自救。先主遣使告璋曰:「曹公征吳,吳憂危急。孫氏與孤本為脣齒,又樂進在青泥與關羽相拒,今不往救羽,進必大克,轉侵州界,其憂有甚於魯。魯自守之賊,不足慮也。」乃從璋求萬兵及資實,欲以東行。璋但許兵四千,其餘皆給半。魏書曰:備因激怒其衆曰:「吾為益州征彊敵,師徒勤瘁,不遑寧居;今積帑藏之財而恡於賞功,望士大夫為出死力戰,其可得乎!」張松書與先主及法正曰:「今大事垂可立,如何釋此去乎!」松兄廣漢太守肅,懼禍逮己,白璋發其謀。於是璋收斬松,嫌隙始構矣。益部耆舊雜記曰:張肅有威儀,容貌甚偉。松為人短小,放蕩不治節操,然識達精果,有才幹。劉璋遣詣曹公,曹公不甚禮松;主簿楊脩深器之,白公辟松,公不納。脩以公所撰兵書示松,松飲宴之間一看便闇誦。脩以此益異之。璋勑關戍諸將文書勿復關通先主。先主大怒,召璋白水軍督楊懷,責以無禮,斬之。乃使黃忠、卓膺勒兵向璋。先主徑至關中,質諸將并士卒妻子,引兵與忠、膺等進到涪,據其城。璋遣劉璝、冷苞、張任、鄧賢等拒先主於涪,益部耆舊雜記曰:張任,蜀郡人,家世寒門。少有膽勇,有志節,仕州為從事。皆破敗,退保緜竹。璋復遣李嚴督緜竹諸軍,嚴率衆降先主。先主軍益彊,分遣諸將平下屬縣,諸葛亮、張飛、趙雲等將兵泝流定白帝、江州、江陽,惟關羽留鎮荊州。先主進軍圍雒;時璋子循守城,被攻且一年。

十九年夏,雒城破,益部耆舊雜記曰:劉璋遣張任與劉璝率精兵拒捍先主於涪,為先主所破,退與璋子循守雒城。任勒兵出於鴈橋,戰復敗。禽任。先主聞任之忠勇,令軍降之,任厲聲曰:「老臣終不復事二主矣。」乃殺之。先主歎惜焉。進圍成都數十日,璋出降。傅子曰:初,劉備襲蜀,丞相掾趙戩曰:「劉備其不濟乎?拙於用兵,每戰則敗,奔亡不暇,何以圖人?蜀雖小區,險固四塞,獨守之國,難卒并也。」徵士傅幹曰:「劉備寬仁有度,能得人死力。諸葛亮達治知變,正而有謀,而為之相;張飛、關羽勇而有義,皆萬人之敵,而為之將:此三人者,皆人傑也。以備之略,三傑佐之,何為不濟?」典略曰:趙戩,字叔茂,京兆長陵人也。質而好學,言稱詩書,愛恤於人不論踈密。辟公府,入為尚書選部郎。董卓欲以所私並充臺閣,戩拒不聽。卓怒,召戩欲殺之,觀者皆為戩懼,而戩自若。及見卓,引辭正色,陳說是非,卓雖凶戾,屈而謝之。遷平陵令。故將王允被害,莫敢近者,戩棄官收歛之。三輔亂,戩客荊州,劉表以為賔客。曹公平荊州,執戩手曰:「何相見之晚也!」遂辟為掾。後為五官將司馬,相國鍾繇長史,年六十餘卒。蜀中殷盛豐樂,先主置酒大饗士卒,取蜀城中金銀分賜將士,還其穀帛。先主復領益州牧,諸葛亮為股肱,法正為謀主,關羽、張飛、馬超為爪牙,許靖、麋笁、簡雍為賔友。及董和、黃權、李嚴等本璋之所授用也,吳壹、費觀等又璋之婚親也,彭羡又璋之所排擯也,劉巴者宿昔之所忌恨也,皆處之顯任,盡其器能。有志之士無不競勸。

二十年,孫權以先主已得益州,使使報欲得荊州。先主言:「須得涼州,當以荊州相與。」權忿之,乃遣呂蒙襲奪長沙、零陵、桂陽三郡。先主引兵五萬下公安,令關羽入益陽。是歲,曹公定漢中,張魯遁走巴西。先主聞之,與權連和,分荊州、江夏、長沙、桂陽東屬,南郡、零陵、武陵西屬,引軍還江州。遣黃權將兵迎張魯,張魯已降曹公。曹公使夏侯淵、張郃屯漢中,數數犯暴巴界。先主令張飛進兵宕渠,與郃等戰於瓦口,破郃等,郃收兵還南鄭。先主亦還成都。

二十三年,先主率諸將進兵漢中。分遣將軍吳蘭、雷銅等入武都,皆為曹公軍所沒。先主次于陽平關,與淵、郃等相拒。

二十四年春,自陽平南渡沔水,緣山稍前,於定軍山勢作營。淵將兵來爭其地。先主命黃忠乘高鼓譟攻之,大破淵軍,斬淵及曹公所署益州刺史趙顒等。曹公自長安舉衆南征。先主遙策之曰:「曹公雖來,無能為也,我必有漢川矣。」及曹公至,先主歛衆拒險,終不交鋒,積月不拔,亡者日多。夏,曹公果引軍還,先主遂有漢中。遣劉封、孟達、李平等攻申耽於上庸。

秋,羣下上先主為漢中王,表於漢帝曰:「平西將軍都亭侯臣馬超、左將軍長史領鎮軍將軍臣許靖、營司馬臣龐羲、議曹從事中郎軍議中郎將臣射援、三輔決錄注曰:援字文雄,扶風人也。其先本姓謝,與北地諸謝同族。始祖謝服為將軍出征,天子以謝服非令名,改為射,子孫氏焉。兄堅,字文固,少有美名,辟公府為黃門侍郎。獻帝之初,三輔飢亂,堅去官,與弟援南入蜀依劉璋,璋以堅為長史。劉備代璋,以堅為廣漢、蜀郡太守。援亦少有名行,太尉皇甫嵩賢其才而以女妻之,丞相諸葛亮以援為祭酒,遷從事中郎,卒官。軍師將軍臣諸葛亮、盪寇將軍漢壽亭侯臣關羽、征虜將軍新亭侯臣張飛、征西將軍臣黃忠、鎮遠將軍臣賴恭、揚武將軍臣法正、興業將軍臣李嚴等一百二十人上言曰:昔唐堯至聖而四凶在朝,周成仁賢而四國作難,高后稱制而諸呂竊命,孝昭幼冲而上官逆謀,皆馮世寵,藉履國權,窮凶極亂,社稷幾危。非大舜、周公、朱虛、博陸,則不能流放禽討,安危定傾。伏惟陛下誕姿聖德,統理萬邦,而遭厄運不造之艱。董卓首難,蕩覆京畿,曹操階禍,竊執天衡;皇后太子鴆殺見害,剥亂天下,殘毀民物。乆令陛下蒙塵憂厄,幽處虛邑。人神無主,遏絕王命,厭昧皇極,欲盜神器。左將軍領司隷校尉豫、荊、益三州牧宜城亭侯備,受朝爵秩,念在輸力,以殉國難。覩其機兆,赫然憤發,與車騎將軍董承同謀誅操,將安國家,克寧舊都。會承機事不密,令操游魂得遂長惡,殘泯海內。臣等每懼王室大有閻樂之禍,小有定安之變,趙高使閻樂殺二世。王莽廢孺子以為定安公。夙夜惴惴,戰慄累息。昔在虞書,敦序九族,周監二代,封建同姓,詩著其義,歷載長乆。漢興之初,割裂疆土,尊王子弟,是以卒折諸呂之難,而成太宗之基。臣等以備肺腑枝葉,宗子藩翰,心存國家,念在弭亂。自操破於漢中,海內英雄望風蟻附,而爵號不顯,九錫未加,非所以鎮衞社稷,光昭萬世也。奉辭在外,禮命斷絕。昔河西太守梁統等值漢中興,限於山河,位同權均,不能相率,咸推竇融以為元帥,卒立效績,摧破隗嚻。今社稷之難,急於隴、蜀。操外吞天下,內殘羣寮,朝廷有蕭墻之危,而禦侮未建,可為寒心。臣等輒依舊典,封備漢中王,拜大司馬,董齊六軍,糾合同盟,埽滅凶逆。以漢中、巴、蜀、廣漢、犍為為國,所署置依漢初諸侯王故典。夫權宜之制,苟利社稷,專之可也。然後功成事立,臣等退伏矯罪,雖死無恨。」遂於沔陽設壇場,陳兵列衆,羣臣陪位,讀奏訖,御王冠于先主。

先主上言漢帝曰:「臣以具臣之才,荷上將之任,董督三軍,奉辭于外,不得埽除寇難,靖匡王室,乆使陛下聖教陵遲,六合之內否而未泰,惟憂反側,疢如疾首。曩者董卓造為亂階,自是之後,羣兇縱橫,殘剥海內。賴陛下聖德威靈,人臣同應,或忠義奮討,或上天降罰,暴逆並殪,以漸冰消。惟獨曹操乆未梟除,侵擅國權,恣心極亂。臣昔與車騎將軍董承圖謀討操,機事不密,承見陷害,臣播越失據,忠義不果。遂得使操窮凶極逆,主后戮殺,皇子鴆害。雖糾合同盟,念在奮力,懦弱不武,歷年未效。常恐殞沒,孤負國恩,寤寐永歎,夕惕若厲。今臣羣寮以為在昔虞書敦叙九族,庶明勵翼,鄭玄注曰:庶,衆也;勵,作也;叙,次序也。序九族而親之,以衆明作羽翼之臣也。五帝損益,此道不廢。周監二代,並建諸姬,實賴晉、鄭夾輔之福。高祖龍興,尊王子弟,大啟九國,卒斬諸呂,以安大宗。今操惡直醜正,寔繁有徒,包藏禍心,篡盜已顯。旣宗室微弱,帝族無位,斟酌古式,依假權宜,上臣大司馬漢中王。臣伏自三省,受國厚恩,荷任一方,陳力未效,所獲已過,不宜復忝高位以重罪謗。羣寮見逼,迫臣以義。臣退惟寇賊不梟,國難未已,宗廟傾危,社稷將墜,成臣憂責碎首之負。若應權通變,以寧靖聖朝,雖赴水火,所不得辭,敢慮常宜,以防後悔。輒順衆議,拜受印璽,以崇國威。仰惟爵號,位高寵厚,俯思報效,憂深責重,驚怖累息,如臨于谷。盡力輸誠,獎厲六師,率齊羣義,應天順時,撲討凶逆,以寧社稷,以報萬分,謹拜章因驛上還所假左將軍、宜城亭侯印綬。」於是還治成都。拔魏延為都督,鎮漢中。典略曰:備於是起館舍,築亭障,從成都至白水關,四百餘區。時關羽攻曹公將曹仁,禽于禁於樊。俄而孫權襲殺羽,取荊州。

二十五年,魏文帝稱尊號,改年曰黃初。或傳聞漢帝見害,先主乃發喪制服,追謚曰孝愍皇帝。是後在所並言衆瑞,日月相屬,故議郎陽泉侯劉豹、青衣侯向舉、偏將軍張裔·黃權、大司馬屬殷純、益州別駕從事趙莋、治中從事楊洪、從事祭酒何宗、議曹從事杜瓊、勸學從事張爽、尹默、譙周等上言:「臣聞河圖、洛書,五經讖、緯,孔子所甄,驗應自遠。謹按洛書甄曜度曰:『赤三日德昌,九世會備,合為帝際。』洛書寶號命曰:『天度帝道備稱皇,以統握契,百成不敗。』洛書錄運期曰:『九侯七傑爭命民炊骸,道路籍籍履人頭,誰使主者玄且來。』孝經鉤命決錄曰:『帝三建九會備。』臣父羣未亡時,言西南數有黃氣,直立數丈,見來積年,時時有景雲祥風,從璿璣下來應之,此為異瑞。又二十二年中,數有氣如旗,從西竟東,中天而行,圖、書曰『必有天子出其方』。加是年太白、熒惑、填星,常從歲星相追。近漢初興,五星從歲星謀;歲星主義,漢位在西,義之上方,故漢法常以歲星候人主。當有聖主起於此州,以致中興。時許帝尚存,故羣下不敢漏言。頃者熒惑復追歲星,見在胃昴畢;昴畢為天綱,經曰『帝星處之,衆邪消亡』。聖諱豫覩,推癸期驗,符合數至,若此非一。臣聞聖王先天而天不違,後天而奉天時,故應際而生,與神合契。願大王應天順民,速即洪業,以寧海內。」

太傅許靖、安漢將軍糜笁、軍師將軍諸葛亮、太常賴恭、光祿勳黃柱、少府王謀等上言:「曹丕篡弒,湮滅漢室,竊據神器,劫迫忠良,酷烈無道。人鬼忿毒,咸思劉氏。今上無天子,海內惶惶,靡所式仰。羣下前後上書者八百餘人,咸稱述符瑞,圖、讖明徵。間黃龍見武陽赤水,九日乃去。孝經援神契曰『德至淵泉則黃龍見』,龍者,君之象也。易乾九五『飛龍在天』,大王當龍升,登帝位也。又前關羽圍樊、襄陽,襄陽男子張嘉、王休獻玉璽,璽潛漢水,伏於淵泉,暉景燭燿,靈光徹天。夫漢者,高祖本所起定天下之國號也,大王襲先帝軌迹,亦興於漢中也。今天子玉璽神光先見,璽出襄陽,漢水之末,明大王承其下流,授與大王以天子之位,瑞命符應,非人力所致。昔周有烏魚之瑞,咸曰休哉。二祖受命,圖、書先著,以為徵驗。今上天告祥,羣儒英俊並進河、洛,孔子讖、記咸悉具至。伏惟大王出自孝景皇帝中山靖王之冑,本支百世,乾祇降祚,聖姿碩茂,神武在躬,仁覆積德,愛人好士,是以四方歸心焉。考省靈圖,啟發讖、緯,神明之表,名諱昭著。宜即帝位,以纂二祖,紹嗣昭穆,天下幸甚。臣等謹與博士許慈、議郎孟光建立禮儀,擇令辰,上尊號。」即皇帝位於成都武擔之南。蜀本紀曰:武都有丈夫化為女子,顏色美好,蓋山精也。蜀王娶以為妻,不習水土,疾病欲歸國,蜀王留之,無幾物故。蜀王發卒之武都擔土,於成都郭中葬,蓋地數畝,高十丈,號曰武擔也。臣松之案:武擔,山名,在成都西北,蓋以乾位在西北,故就之以即阼。為文曰:「惟建安二十六年四月丙午,皇帝備敢用玄牡,昭告皇天上帝后土神祇:漢有天下,歷數無疆。曩者王莽篡盜,光武皇帝震怒致誅,社稷復存。今曹操阻兵安忍,戮殺主后,滔天泯夏,罔顧天顯。操子丕,載其凶逆,竊居神器。羣臣將士以為社稷隳廢,備宜脩之,嗣武二祖,龔行天罰。備雖否德,懼忝帝位。詢于庶民,外及蠻夷君長,僉曰『天命不可以不荅,祖業不可以乆替,四海不可以無主』。率土式望,在備一人。備畏天明命,又懼漢邦將湮于地,謹擇元日,與百寮登壇,受皇帝璽綬。脩燔瘞,告類于天神,惟神饗祚于漢家,永綏四海!」魏書曰:備聞曹公薨,遣掾韓冉奉書弔,并致賻贈之禮。文帝惡其因喪求好,勑荊州刺史斬冉,絕使命。典略曰:備遣軍謀掾韓冉齎書弔,并貢錦布。冉稱疾,住上庸。上庸致其書,適會受終,有詔報荅以引致之。備得報書,遂稱制。

章武元年夏四月,大赦,改年。以諸葛亮為丞相,許靖為司徒。置百官,立宗廟,祫祭高皇帝以下。臣松之以為先主雖云出自孝景,而世數悠遠,昭穆難明,旣紹漢祚,不知以何帝為元祖以立親廟。于時英賢作輔,儒生在官,宗廟制度必有憲章,而載記闕略,良可恨哉!五月,立皇后吳氏,子禪為皇太子。六月,以子永為魯王,理為梁王。車騎將軍張飛為其左右所害。初,先主忿孫權之襲關羽,將東征,秋七月,遂帥諸軍伐吳。孫權遣書請和,先主盛怒不許,吳將陸議、李異、劉阿等屯巫、秭歸;將軍吳班、馮習自巫攻破異等,軍次秭歸,武陵五谿蠻夷遣使請兵。

二年春正月,先主軍還秭歸,將軍吳班、陳式水軍屯夷陵,夾江東西岸。二月,先主自秭歸率諸將進軍,緣山截嶺,於夷道猇亭猇,許交反。駐營,自佷山佷,音恒。通武陵,遣侍中馬良安慰五谿蠻夷,咸相率響應。鎮北將軍黃權督江北諸軍,與吳軍相拒於夷陵道。夏六月,黃氣見自秭歸十餘里中,廣數十丈。後十餘日,陸議大破先主軍於猇亭,將軍馮習、張南等皆沒。先主自猇亭還秭歸,收合離散兵,遂棄船舫,由步道還魚復,改魚復縣曰永安。吳遣將軍李異、劉阿等踵躡先主軍,屯駐南山。秋八月,收兵還巫。司徒許靖卒。冬十月,詔丞相亮營南北郊於成都。孫權聞先主住白帝,甚懼,遣使請和。先主許之,遣太中大夫宗瑋報命。冬十二月,漢嘉太守黃元聞先主疾不豫,舉兵拒守。

三年春二月,丞相亮自成都到永安。三月,黃元進兵攻臨卭縣。遣將軍陳曶音笏。討元,元軍敗,順流下江,為其親兵所縛,生致成都,斬之。先主病篤,託孤於丞相亮,尚書令李嚴為副。夏四月癸巳,先主殂于永安宮,時年六十三。諸葛亮集載先主遺詔勑後主曰:「朕初疾但下痢耳,後轉雜他病,殆不自濟。人五十不稱夭,年已六十有餘,何所復恨,不復自傷,但以卿兄弟為念。射君到,說丞相歎卿智量,甚大增脩,過於所望,審能如此,吾復何憂!勉之,勉之!勿以惡小而為之,勿以善小而不為。惟賢惟德,能服於人。汝父德薄,勿效之。可讀漢書、禮記,閑暇歷觀諸子及六韜、商君書,益人意智。聞丞相為寫申、韓、管子、六韜一通已畢,未送,道亡,可自更求聞達。」臨終時,呼魯王與語:「吾亡之後,汝兄弟父事丞相,令卿與丞相共事而已。」

亮上言於後主曰:「伏惟大行皇帝邁仁樹德,覆燾無疆,昊天不弔,寢疾彌留,今月二十四日奄忽升遐,臣妾號咷,若喪考妣。乃顧遺詔,事惟太宗,動容損益;百寮發哀,滿三日除服,到葬期復如禮;其郡國太守、相、都尉、縣令長,三日便除服。臣亮親受勑戒,震威神靈,不敢有違。臣請宣下奉行。」五月,梓宮自永安還成都,謚曰昭烈皇帝。秋,八月,葬惠陵。葛洪神仙傳曰:仙人李意其,蜀人也。傳世見之,云是漢文帝時人。先主欲伐吳,遣人迎意其。意其到,先主禮敬之,問以吉凶。意其不荅而求紙筆,畫作兵馬器仗數十紙已,便一一以手裂壞之,又畫作一大人,掘地埋之,便徑去。先主大不喜。而由出軍征吳大敗還,忿恥發病死,衆人乃知意其畫作大人而埋之者,即是言先主死意。

評曰:先主之弘毅寬厚,知人待士,蓋有高祖之風,英雄之器焉。及其舉國託孤於諸葛亮,而心神無貳,誠君臣之至公,古今之盛軌也。機權幹略,不逮魏武,是以基宇亦狹。然折而不撓,終不為下者,抑揆彼之量必不容己,非唯競利,且以避害云爾。


\end{pinyinscope}