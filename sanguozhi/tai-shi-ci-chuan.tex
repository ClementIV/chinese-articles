\article{太史慈傳}

\begin{pinyinscope}
太史慈字子義,東萊黃人也。少好學,仕郡奏曹史。會郡與州有隙,曲直未分,以先聞者為善。時州章已去,郡守恐後之,求可使者。慈年二十一,以選行,晨夜取道,到洛陽,詣公車門,見州吏始欲求通。慈問曰:「君欲通章邪?」吏曰:「然。」問:「章安在?」曰:「車上。」慈曰:「章題署得無誤邪?取來視之。」吏殊不知其東萊人也,因為取章。慈已先懷刀,便截敗之。吏踴躍大呼,言「人壞我章」!慈將至車間,與語曰:「向使君不以章相與,吾亦無因得敗之,是為吉凶禍福等耳,吾不獨受此罪。豈若默然俱出去,可以存易亡,無事俱就刑辟。」吏言:「君為郡敗吾章,已得如意,欲復亡為?」慈荅曰:「初受郡遣,但來視章通與未耳。吾用意太過,乃相敗章。今還,亦恐以此見譴怒,故俱欲去爾。」吏然慈言,即日俱去。慈旣與出城,因遁還通郡章。州家聞之,更遣吏通章,有司以格章之故不復見理,州受其短。由是知名,而為州家所疾,恐受其禍,乃避之遼東。

北海相孔融聞而奇之,數遣人訊問其母,并致餉遺。時融以黃巾寇暴,出屯都昌,為賊管亥所圍。慈從遼東還,母謂慈曰:「汝與孔北海未嘗相見,至汝行後,贍恤殷勤,過於故舊,今為賊所圍,汝宜赴之。」慈留三日,單步徑至都昌。時圍尚未密,夜伺間隙,得入見融,因求兵出斫賊。融不聽,欲待外救。外救未有至者,而圍日偪。融欲告急平原相劉備,城中人無由得出,慈自請求行。融曰:「今賊圍甚密,衆人皆言不可,卿意雖壯,無乃實難乎?」慈對曰:「昔府君傾意於老母,老母感遇,遣慈赴府君之急,固以慈有可取,而來必有益也。今衆人言不可,慈亦言不可,豈府君愛顧之義,老母遣慈之意邪?事已急矣,願府君無疑。」融乃然之。於是嚴行蓐食,須明,便帶鞬攝弓上馬,將兩騎自隨,各作一的持之,開門直出。外圍下左右人並驚駭,兵馬互出。慈引馬至城下塹內,植所持的各一,出射之,射之畢,徑入門。明晨復如此,圍下人或起或卧,慈復植的,射之畢,復入門。明晨復出如此,無復起者,於是下鞭馬直突圍中馳去。比賊覺知,慈行已過,又射殺數人,皆應弦而倒,故無敢追者。遂到平原,說備曰:「慈,東萊之鄙人也,與孔北海親非骨肉,比非鄉黨,特以名志相好,有分災共患之義。今管亥暴亂,北海被圍,孤窮無援,危在旦夕。以君有仁義之名,能救人之急,故北海區區,延頸恃仰,使慈冒白刃,突重圍,從萬死之中自託於君,惟君所以存之。」備斂容荅曰:「孔北海知世間有劉備邪!」即遣精兵三千人隨慈。賊聞兵至,解圍散走。融旣得濟,益奇貴慈,曰:「卿吾之少友也。」事畢,還啟其母,母曰:「我喜汝有以報孔北海也。」

楊州刺史劉繇與慈同郡,慈自遼東還,未與相見,暫渡江到曲阿見繇,未去,會孫策至。或勸繇可以慈為大將軍,繇曰:「我若用子義,許子將不當笑我邪?」但使慈偵視輕重。時獨與一騎卒遇策。策從騎十三,皆韓當、宋謙、黃蓋輩也。慈便前鬬,正與策對。策刺慈馬,而擥得慈項上手戟,慈亦得策兜鍪。會兩家兵騎並各來赴,於是解散。

慈當與繇俱奔豫章,而遁於蕪湖,亡入山中,稱丹楊太守。是時,策已平定宣城以東,惟涇以西六縣未服。慈因進住涇縣,立屯府,大為山越所附。策躬自攻討,遂見囚執。策即解縛,捉其手曰:「寧識神亭時邪?若卿爾時得我云何?」慈曰:「未可量也。」策大笑曰:「今日之事,當與卿共之。」

吳歷云:慈於神亭戰敗,為策所執。策素聞其名,即解縛請見,咨問進取之術。慈荅曰:「破軍之將,不足與論事。」策曰:「昔韓信定計於廣武,今策決疑於仁者,君何辭焉?」慈曰:「州軍新破,士卒離心,若儻分散,難復合聚;欲出宣恩安集,恐不合尊意。」策長跪荅曰:「誠本心所望也。明日中,望君來還。」諸將皆疑,策曰:「太史子義,青州名士,以信義為先,終不欺策。」明日,大請諸將,豫設酒食,立竿視影。日中而慈至,策大恱,常與參論諸軍事。臣松之案;吳歷云慈於神亭戰敗,為策所得,與本傳大異,疑為謬誤。江表傳曰:策問慈曰:「聞卿昔為太守劫州章,赴文舉,詣玄德,皆有烈義,天下智士也,但所託未得其人耳。射鈎斬袪,古人不嫌。孤是卿知己,勿憂不如意也。」出教曰:「龍欲騰翥,先階尺木者也。」即署門下督,還吳授兵,拜折衝中郎將。後劉繇亡於豫章,士衆萬餘人未有所附,策命慈往撫安焉。江表傳曰:策謂慈曰:「劉牧往責吾為袁氏攻廬江,其意頗猥,理恕不足。何者?先君手下兵數千餘人,盡在公路許。孤志在立事,不得不屈意於公路,求索故兵,再往纔得千餘人耳。仍令孤攻廬江,爾時事勢,不得不為行。但其後不遵臣節,自棄作邪僭事,諫之不從。丈夫義交,苟有大故,不得不離,孤交求公路及絕之本末如此。今劉繇喪亡,恨不及其生時與共論辯。今兒子在豫章,不知華子魚待遇何如,其故部曲復依隨之否?卿則州人,昔又從事,寧能往視其兒子,並宣孤意於其部曲?部曲樂來者便與俱來,不樂來者且安慰之。并觀察子魚所以牧禦方規何似,視廬陵、鄱陽人民親附之否?卿手下兵,宜將多少,自由意。」慈對曰:「慈有不赦之罪,將軍量同桓、文,待遇過望。古人報生以死,期於盡節,沒而後已。今並息兵,兵不宜多,將數十人,自足以往還也。」左右皆曰:「慈必北去不還。」策曰:「子義捨我,當復與誰?」餞送昌門,把腕別曰:「何時能還?」荅曰:「不過六十日。」果如期而反。江表傳曰:策初遣慈,議者紛紜,謂慈未可信,或云與華子魚州里,恐留彼為籌策,或疑慈西託黃祖,假路還北,多言遣之非計。策曰:「諸君語皆非也,孤斷之詳矣。太史子義雖氣勇有膽烈,然非縱橫之人。其心有士謨,志經道義,貴重然諾,一以意許知己,死亡不相負,諸君勿復憂也。」慈從豫章還,議者乃始服。慈見策曰:「華子魚良德也,然非籌略才,無他方規,自守而已。又丹楊僮芝自擅廬陵,詐言被詔書為太守。鄱陽民帥別立宗部,阻兵守界,不受子魚所遣長吏,言『我以別立郡,須漢遣真太守來,當迎之耳』。子魚不但不能諧廬陵、鄱陽,近自海昏有上繚壁,有五六千家相結聚作宗伍,惟輸租布於郡耳,發召一人遂不可得,子魚亦覩視之而已。」策拊掌大笑,乃有兼并之志矣。頃之,遂定豫章。

劉表從子磐,驍勇,數為寇於艾、西安諸縣。策於是分海昏、建昌左右六縣,以慈為建昌都尉,治海昏,并督諸將拒磐。磐絕跡不復為寇。

慈長七尺七寸,美鬚髯,猨臂善射,弦不虛發。嘗從策討麻保賊,賊於屯裏緣樓上行詈,以手持樓棼,慈引弓射之,矢貫手着棼,圍外萬人莫不稱善。其妙如此。曹公聞其名,遺慈書,以篋封之,發省無所道,而但貯當歸。孫權統事,以慈能制磐,遂委南方之事。年四十一,建安十一年卒。吳書曰:慈臨亡,歎息曰:「丈夫生世,當帶七尺之劒,以升天子之階。今所志未從,柰何而死乎!」權甚悼惜之。子享,官至越騎校尉。吳書曰:享字元復,歷尚書、吳郡太守。


\end{pinyinscope}