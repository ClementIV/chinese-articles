\article{桓階傳}

\begin{pinyinscope}
桓階字伯緒,長沙臨湘人也。

魏書曰:階祖父超,父勝,皆歷典州郡。勝為尚書,著名南方。仕郡功曹。太守孫堅舉階孝廉,除尚書郎。父喪還鄉里。會堅擊劉表戰死,階冒難詣表乞堅喪,表義而與之。後太祖與袁紹相拒於官渡,表舉州以應紹。階說其太守張羨曰:「夫舉事而不本於義,未有不敗者也。故齊桓率諸侯以尊周,晉文逐叔帶以納王。今袁氏反此,而劉牧應之,取禍之道也。明府必欲立功明義,全福遠禍,不宜與之同也。」羨曰:「然則何向而可?」階曰:「曹公雖弱,杖義而起,救朝廷之危,奉王命而討有罪,孰敢不服?今若舉四郡保三江以待其來,而為之內應,不亦可乎!」羨曰:「善。」乃舉長沙及旁三郡以拒表,遣使詣太祖。太祖大恱。會紹與太祖連戰,軍未得南。而表急攻羨,羨病死。城陷,階遂自匿。乆之,劉表辟為從事祭酒,欲妻以妻妹蔡氏。階自陳已結婚,拒而不受,因辭疾告退。

太祖定荊州,聞其為張羨謀也,異之,辟為丞相掾主簿,遷趙郡太守。魏國初建,為虎賁中郎將侍中。時太子未定,而臨菑侯植有寵。階數陳文帝德優齒長,宜為儲副,公規密諫,前後懇至。魏書稱階諫曰:「今太子位冠羣子,名昭海內,仁聖達節,天下莫不聞;而大王甫以植而問臣,臣誠惑之。」於是太祖知階篤於守正,深益重焉。又毛玠、徐弈以剛蹇少黨,而為西曹掾丁儀所不善,儀屢言其短,賴階左右以自全保。其將順匡救,多此類也。遷尚書,典選舉。曹仁為關羽所圍,太祖遣徐晃救之,不解。太祖欲自南征,以問羣下。羣下皆謂:「王不亟行,今敗矣。」階獨曰:「大王以仁等為足以料事勢不也?」曰:「能。」「大王恐二人遺力邪?」曰:「不。」「然則何為自往?」曰:「吾恐虜衆多,而晃等勢不便耳。」階曰:「今仁等處重圍之中而守死無貳者,誠以大王遠為之勢也。夫居萬死之地,必有死爭之心;內懷死爭,外有彊救,大王案六軍以示餘力,何憂於敗而欲自往?」太祖善其言,駐軍於摩陂。賊遂退。

文帝踐阼,遷尚書令,封高鄉亭侯,加侍中。階疾病,帝自臨省,謂曰:「吾方託六尺之孤,寄天下之命於卿。勉之!」徙封安樂鄉侯,邑六百戶,又賜階三子爵關內侯,祐以嗣子不封,病卒,又追贈關內侯。後階疾篤,遣使者即拜太常,薨,帝為之流涕,謚曰貞侯。子嘉嗣。以階弟纂為散騎侍郎,賜爵關內侯。嘉尚升遷亭公主,會嘉平中,以樂安太守與吳戰於東關,軍敗,沒,謚曰壯侯。子翊嗣。世說曰:階孫陵,字元徽,有名於晉武帝世,至熒陽太守,卒。


\end{pinyinscope}