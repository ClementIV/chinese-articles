\article{張翼傳}

\begin{pinyinscope}
張翼字伯恭,犍為武陽人也。高祖父司空浩,曾祖父廣陵太守綱,皆有名迹。

益部耆舊傳曰:浩字叔明,治律、春秋,游學京師,與廣漢鐔粲、漢中李郃、蜀郡張霸共結為友善。大將軍鄧隲辟浩,稍遷尚書僕射,出為彭城相,薦隱士閭丘邈等,徵拜廷尉。延光三年,安帝議廢太子,唯浩與太常桓焉、太僕來歷議以為不可。順帝初立,拜浩司空,年八十三卒。續漢書曰:綱字文紀,少以三公子經明行脩舉孝廉,不就司徒辟,以高第為侍御史。漢安元年,拜光祿大夫,與侍中杜喬等八人同日受詔,持節分出,案行天下貪廉,墨綬有罪便收,刺史二千石以驛表聞,威惠清忠,名振郡國,號曰八儁。是時,大將軍梁兾侵擾百姓,喬等七人皆奉命四出,唯綱獨埋車輪於洛陽都亭不去,曰:「豺狼當路,安問狐狸?」遂上書曰:「大將軍梁兾、河南尹不疑,蒙外戚之援,荷國厚恩,以芻蕘之姿,安居阿保,不能敷揚五教,翼贊日月,而專為封豕長虵,肆其貪饕,甘心好貨,縱恣無猒,多樹諂諛以害忠良,誠天威所不赦,大辟所宜加也。謹條其無君之心十五事於左,皆忠臣之所切齒也。」書奏御,京師震悚。時兾妹為皇后,內寵方盛,兾兄弟權重於人主,順帝雖知綱言不誣,然無心治兾。兾深恨綱。會廣陵賊張嬰等衆數萬人殺刺史二千石,兾欲陷綱,乃諷尚書以綱為廣陵太守;若不為嬰所殺,則欲以法中之。前太守往,輒多請兵,及綱受拜,詔問當得兵馬幾何,綱對曰無用兵馬,遂單車之官,徑詣嬰壘門,示以禍福。嬰大驚懼,走欲閉門。綱又於門外罷遣吏兵,留所親者十餘人,以書語其長老素為嬰所信者,請與相見,問以本變,因示以詔恩,使還請嬰。嬰見綱意誠,即出見綱。綱延置上坐,問其疾苦,禮畢,乃謂之曰:「前後二千石,多非其人,杜塞國恩,肆其私求。鄉郡遠,天子不能朝夕聞也,故民人相聚以避害。二千石信有罪矣;為之者乃非義也。忠臣不欺君以自榮,孝子不捐父以求福,天子聖人,欲文德以來之,故使太守來,思以爵祿相榮,不願以刑也。今誠轉禍為福之時也;若聞義不服,天子赫然發怒,大兵雲合,豈不危乎!宜深計其利害。」嬰聞,泣曰:「荒裔愚人,數為二千石所侵枉,不堪其困,故遂相聚偷生。明府仁及草木,乃嬰等更生之澤,但恐投兵之日,不免孥戮耳。」綱曰:「豈其然乎!要之以天地,誓之以日月,方當相顯以爵位,何禍之有乎?」嬰曰:「苟赦其罪,得全首領以就農畝,則抱戴沒齒,爵祿非所望也。」嬰雖為大賊,起於狂暴,自以為必死,及得綱言,曠然開明,乃辭還營。明日,遂將所部萬餘人,與妻子面縛詣綱降。綱悉釋縛慰納,謂嬰曰:「卿諸人一旦解散,方垂盪然,當條名上之,必受封賞。」嬰曰:「乞歸故業,不願以穢名汙明時也。」綱以其至誠,乃各從其意,親為安處居宅。子弟欲為吏者,隨才任職,欲為民者,勸以農桑,田業並豐,南州晏然。論功,綱當封,為兾所遏絕,故不得侯。天子美其功,徵欲用之。嬰等上書,乞留在郡二歲。建康元年,病卒官,時年三十六。嬰等三百餘人,皆衰杖送綱喪至洛陽,葬訖,為起冢立祠,四時奉祭,思慕如喪考妣。天子追念不已,下詔襃揚,除一子為郎。先主定益州,領牧,翼為書佐。建安末,舉孝廉,為江陽長,徙涪陵令,遷梓潼太守,累遷至廣漢、蜀郡太守。建興九年,為庲降都督、綏南中郎將。翼性持法嚴,不得殊俗之歡心。耆率劉冑背叛作亂,翼舉兵討冑。冑未破,會被徵當還,羣下咸以為宜便馳騎即罪,翼曰:「不然。吾以蠻夷蠢動,不稱職故還耳,然代人未至,吾方臨戰場,當運糧積穀,為滅賊之資,豈可以黜退之故而廢公家之務乎?」於是統攝不懈,代到乃發。馬忠因其成基以破殄冑,丞相亮聞而善之。亮出武功,以翼為前軍都督,領扶風太守。亮卒,拜前領軍,追論討劉冑功,賜爵關內侯。延熈元年,入為尚書,稍遷督建威,假節,進封都亭侯,征西大將軍。

十八年,與衞將軍姜維俱還成都。維議復出軍,唯翼庭爭,以為國小民勞,不宜黷武。維不聽,將翼等行,進翼位鎮南大將軍。維至狄道,大破魏雍州刺史王經,經衆死於洮水者以萬計。翼曰:「可止矣,不宜復進,進或毀此大功。」維大怒,曰:「為蛇畫足。」維竟圍經於狄道,城不能克。自翼建異論,維心與翼不善,然常牽率同行,翼亦不得已而往。景耀二年,遷左車騎將軍,領兾州刺史。六年,與維咸在劒閣,共詣降鍾會于涪。明年正月,隨會至成都,為亂兵所殺。華陽國志曰:翼子微,篤志好學,官至廣漢太守。


\end{pinyinscope}