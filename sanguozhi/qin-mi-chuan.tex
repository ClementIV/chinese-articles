\article{秦宓傳}

\begin{pinyinscope}
秦宓字子勑,廣漢緜竹人也。少有才學,州郡辟命,輒稱疾不往。奏記州牧劉焉,薦儒士任定祖曰:「昔百里、蹇叔以耆艾而定策,甘羅、子奇以童冠而立功,故書美黃髮,而易稱顏淵,固知選士用能,不拘長幼,明矣。乃者以來,海內察舉率多英儁而遺舊齒,衆論不齊,異同相半,此乃承平之翔步,非亂世之急務也。夫欲救危撫亂,脩己以安人,則宜卓犖超倫,與時殊趣,震驚鄰國,駭動四方,上當天心,下合人意;天人旣和,內省不疚,雖遭凶亂,何憂何懼!昔楚葉公好龍,神龍下之,好偽徹天,何況於真?今處士任安,仁義直道,流名四遠,如今見察,則一州斯服。昔湯舉伊尹,不仁者遠,何武貢二龔,雙名竹帛,故貪尋常之高而忽萬仞之嵩,樂面前之飾而忘天下之譽,斯誠往古之所重慎也。甫欲鑿石索玉,剖蚌求珠,今乃隨、和炳然,有如皎日,復何疑哉!誠知晝不操燭,日有餘光,但愚情區區,貪陳所見。」

益部耆舊傳曰:安,廣漢人。少事聘士楊厚,究極圖籍,游覽京師,還家講授,與董扶俱以學行齊聲。郡請功曹,州辟治中別駕,終不乆居。舉孝廉茂才,太尉載辟,除博士,公車徵,皆稱疾不就。州牧劉焉表薦安味精道度,厲節高邈,揆其器量,國之元寶,宜處弼疑之輔,以消非常之咎。玄纁之禮,所宜招命。王塗隔塞,遂無聘命。年七十九,建安七年卒,門人慕仰,為立碑銘。後丞相亮問秦宓以安所長,宓曰:「記人之善,忘人之過。」

劉璋時,宓同郡王商為治中從事,與宓書曰:「貧賤困苦,亦何時可以終身!卞和衒玉以燿世,宜一來,與州尊相見。」宓荅書曰:「昔堯優許由,非不弘也,洗其兩耳;楚聘莊周,非不廣也,執竿不顧。易曰『確乎其不可拔』,夫何衒之有?且以國君之賢,子為良輔,不以是時建蕭、張之策,未足為智也。僕得曝背乎隴畒之中,誦顏氏之簞瓢,詠原憲之蓬戶,時翱翔於林澤,與沮、溺之等儔,聽玄猿之悲吟,察鶴鳴於九皐,安身為樂,無憂為福,處空虛之名,居不靈之龜,知我者希,則我貴矣。斯乃僕得志之秋也,何困苦之戚焉!」後商為嚴君平、李弘立祠,宓與書曰:「疾病伏匿,甫知足下為嚴、李立祠,可謂厚黨勤類者也。觀嚴文章,冠冒天下,由、夷逸操,山嶽不移,使楊子不歎,固自昭明。如李仲元不遭法言,令名必淪,其無虎豹之文故也,可謂攀龍附鳳者矣。如楊子雲潛心著述,有補於世,泥蟠不滓,行參聖師,于今海內談詠厥辭。邦有斯人,以耀四遠,怪子替茲,不立祠堂。蜀本無學士,文翁遣相如東受七經,還教吏民,於是蜀學比於齊、魯。故地里志曰:『文翁倡其教,相如為之師。』漢家得士,盛於其世;仲舒之徒,不達封禪,相如制其禮。夫能制禮造樂,移風易俗,非禮所秩有益於世者乎!雖有王孫之累,猶孔子大齊桓之霸,公羊賢叔術之讓。僕亦善長卿之化,宜立祠堂,速定其銘。」

先是,李權從宓借戰國策,宓曰:「戰國從橫,用之何為?」權曰:「仲尼、嚴平,會聚衆書,以成春秋、指歸之文,故海以合流為大,君子以博識為弘。」宓報曰:「書非史記周圖,仲尼不采;道非虛無自然,嚴平不演。海以受淤,歲一蕩清;君子博識,非禮不視。今戰國反覆儀、秦之術,殺人自生,亡人自存,經之所疾。故孔子發憤作春秋,大乎居正,復制孝經,廣陳德行。杜漸防萌,預有所抑,是以老氏絕禍於未萌,豈不信邪!成湯大聖,覩野魚而有獵逐之失,定公賢者,見女樂而弃朝事,臣松之案:書傳魯定公無善可稱。宓謂之賢者,淺學所未達也。若此輩類,焉可勝陳。道家法曰:『不見所欲,使心不亂。』是故天地貞觀,日月貞明;其直如矢,君子所履。洪範記灾,發於言貌,何戰國之譎權乎哉!」

或謂宓曰:「足下欲自比於巢、許、四皓,何故揚文藻見瓌穎乎?」宓荅曰:「僕文不能盡言,言不能盡意,何文藻之有揚乎!昔孔子三見哀公,言成七卷,事蓋有不可嘿嘿也。劉向七略曰:孔子三見哀公,作三朝記七篇,今在大戴禮。臣松之案:中經部有孔子三朝八卷,一卷目錄,餘者所謂七篇。接輿行且歌,論家以光篇;漁父詠滄浪,賢者以燿章。此二人者,非有欲於時者也。夫虎生而文炳,鳳生而五色,豈以五采自飾畫哉?天性自然也。蓋河、洛由文興,六經由文起,君子懿文德,采藻其何傷!以僕之愚,猶恥革子成之誤,況賢於己者乎!」臣松之案:今論語作棘子成。子成曰:「君子質而已矣,何以文為!」屈於子貢之言,故謂之誤也。

先主旣定益州,廣漢太守夏侯纂請宓為師友祭酒,領五官掾,稱曰仲父。宓稱疾,卧在第舍,纂將功曹古朴、主簿王普,厨膳即宓第宴談,宓卧如故。纂問朴曰:「至於貴州養生之具,實絕餘州矣,不知士人何如餘州也?」朴對曰:「乃自先漢已來,其爵位者或不如餘州耳,至於著作為世師式,不負於餘州也。嚴君平見黃、老作指歸,揚雄見易作太玄,見論語作法言,司馬相如為武帝制封禪之文,于今天下所共聞也。」纂曰:「仲父何如?」宓以簿擊頰,簿,手版也。曰:「願明府勿以仲父之言假於小草,民請為明府陳其本紀。蜀有汶阜之山,江出其腹,帝以會昌,神以建福,故能沃野千里。河圖括地象曰:岷山之地,上為東井絡,帝以會昌,神以建福,上為天井。左思蜀都賦曰:遠則岷山之精,上為井絡,天帝運期而會昌,景福肹蠁而興作。淮、濟四瀆,江為其首,此其一也。禹生石紐,今之汶山郡是也。帝王世紀曰:鯀納有莘氏女曰志,是為脩己。上山行,見流星貫昴,夢接意感,又吞神珠,臆圮胷折,而生禹於石紐。譙周蜀本紀曰:禹本汶山廣柔縣人也,生於石紐,其地名刳兒坪,見世帝紀。昔堯遭洪水,鯀所不治,禹疏江決河,東注于海,為民除害,生民已來功莫先者,此其二也。天帝布治房心,決政參伐,參伐則益州分野,三皇乘祇車出谷口,今之斜谷是也。蜀記曰:三皇乘祇車出谷口。未詳宓所由知為斜谷也。此便鄙州之阡陌,明府以雅意論之,何若於天下乎?」於是纂逡巡無以復荅。

益州辟宓為從事祭酒。先主旣稱尊號,將東征吳,宓陳天時必無其利,坐下獄幽閉,然後貸出。建興二年,丞相亮領益州牧,選宓迎為別駕,尋拜左中郎將、長水校尉。吳遣使張溫來聘,百官皆往餞焉。衆人皆集而宓未往,亮累遣使促之,溫曰:「彼何人也?」亮曰:「益州學士也。」及至,溫問曰:「君學乎?」宓曰:「五尺童子皆學,何必小人!」溫復問曰:「天有頭乎?」宓曰:「有之。」溫曰:「在何方也?」宓曰:「在西方。詩曰:『乃眷西顧。』以此推之,頭在西方。」溫曰:「天有耳乎?」宓曰:「天處高而聽卑,詩云:『鶴鳴于九臯,聲聞于天。』若其無耳,何以聽之?」溫曰:「天有足乎?」宓曰:「有。詩云:『天步艱難,之子不猶。』若其無足,何以步之?」溫曰:「天有姓乎?」宓曰:「有。」溫曰:「何姓?」宓曰:「姓劉。」溫曰:「何以知之?」荅曰:「天子姓劉,故以此知之。」溫曰:「日生於東乎?」宓曰:「雖生於東而沒於西。」荅問如響,應聲而出,於是溫大敬服。宓之文辯,皆此類也。遷大司農,四年卒。初宓見帝系之文,五帝皆同一族,宓辨其不然之本。又論皇帝王霸豢龍之說,甚有通理。譙允南少時數往諮訪,紀錄其言於春秋然否論,文多故不載。

評曰:許靖夙有名譽,旣以篤厚為稱,又以人物為意,雖行事舉動,未悉允當,蔣濟以為「大較廊廟器」也。萬機論論許子將曰:許文休者,大較廊廟器也,而子將貶之。若實不貴之,是不明也;誠令知之,蓋善人也。麋笁、孫乾、簡雍、伊籍,皆雍容風議,見禮於世。秦宓始慕肥遯之高,而無若愚之實。然專對有餘,文藻壯美,可謂一時之才士矣。


\end{pinyinscope}