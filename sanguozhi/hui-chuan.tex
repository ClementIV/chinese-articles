\article{濊傳}

\begin{pinyinscope}
濊南與辰韓,北與高句麗、沃沮接,東窮大海,今朝鮮之東皆其地也。戶二萬。昔箕子旣適朝鮮,作八條之教以教之,無門戶之閉而民不為盜。其後四十餘世,朝鮮侯準僭號稱王。陳勝等起,天下叛秦,燕、齊、趙民避地朝鮮數萬口。燕人衞滿,魋結夷服,復來王之。漢武帝伐滅朝鮮,分其地為四郡。自是之後,胡漢稍別。無大君長,自漢已來,其官有侯邑君、三老,統主下戶。其耆老舊自謂與句麗同種。其人性愿愨,少嗜慾,有廉恥,不請句麗。言語法俗大抵與句麗同,衣服有異。男女衣皆著曲領,男子繫銀花廣數寸以為飾。自單單大山領以西屬樂浪,自領以東七縣,都尉主之,皆以濊為民。後省都尉,封其渠帥為侯,今不耐濊皆其種也。漢末更屬句麗。其俗重山川,山川各有部分,不得妄相涉入。同姓不婚。多忌諱,疾病死亡輙捐棄舊宅,更作新居。有麻布,蠶桑作緜。曉候星宿,豫知年歲豐約。不以誅玉為寶。常用十月節祭天,晝夜飲酒歌舞,名之為舞天,又祭虎以為神。其邑落相侵犯,輙相罰責生口牛馬,名之為責禍。殺人者償死。少寇盜。作矛長三丈,或數人共持之,能步戰。樂浪檀弓出其地。其海出班魚皮,土地饒文豹,又出果下馬,漢桓時獻之。

臣松之案:果下馬高三尺,乘之可於果樹下行,故謂之果下。見博物志、魏都賦。

正始六年,樂浪太守劉茂、帶方太守弓遵以領東濊屬句麗,興師伐之,不耐侯等舉邑降。其八年,詣闕朝貢,詔更拜不耐濊王。居處雜在民間,四時詣郡朝謁。二郡有軍征賦調,供給役使,遇之如民。


\end{pinyinscope}