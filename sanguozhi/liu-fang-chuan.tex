\article{liu-fang-chuan}

\begin{pinyinscope}
劉放字子棄,涿郡人,漢廣陽順王子西鄉侯宏後也。歷郡綱紀,舉孝廉。遭世大亂,時漁陽王松據其土,放往依之。太祖克兾州,放說松曰:「往者董卓作逆,英雄並起,阻兵擅命,人自封殖,惟曹公能拔拯危亂,翼戴天子,奉辭伐罪,所向必克。以二袁之彊,守則淮南冰消,戰則官渡大敗;乘勝席卷,將清河朔,威刑旣合,大勢以見。速至者漸福,後服者先亡,此乃不俟終日馳騖之時也。昔黥布棄南面之尊,仗劔歸漢,誠識廢興之理,審去就之分也。將軍宜投身委命,厚自結納。」松然之。會太祖討袁譚於南皮,以書招松,松舉雍奴、泉州、安次以附之。放為松荅太祖書,其文甚麗。太祖旣善之,又聞其說,由是遂辟放。建安十年,與松俱至。太祖大恱,謂放曰:「昔班彪依竇融而有河西之功,今一何相似也!」乃以放參司空軍事,歷主簿記室,出為郃陽、祋祤、

祋音都活反。祤音詡。贊令。

魏國旣建,與太原孫資俱為祕書郎。先是,資亦歷縣令,參丞相軍事。資別傳曰:資字彥龍。幼而岐嶷,三歲喪二親,長於兄嫂。講業太學,博覽傳記,同郡王允一見而奇之。太祖為司空,又辟資。會兄為鄉人所害,資手刃報讎,乃將家屬避地河東,故遂不應命。尋復為本郡所命,以疾辭。友人河東賈逵謂資曰:「足下抱逸群之才,值舊邦傾覆,主將殷勤,千里延頸,宜崇古賢桑梓之義。而乆盤桓,拒違君命,斯猶曜和璧於秦王之庭,而塞以連城之價耳。竊為足下不取也!」資感其言,遂往應之。到署功曹,舉計吏。尚書令荀彧見資,歎曰:「北州承喪亂已乆,謂其賢智零落,今日乃復見孫計君乎!」表留以為尚書郎。辭以家難,得還河東。文帝即位,放、資轉為左右丞。數月,放徙為令。黃初初,改祕書為中書,以放為監,資為令,各加給事中;放賜爵關內侯,資為關中侯,遂掌機密。三年,放進爵魏壽亭侯,資關內侯。明帝即位,尤見寵任,同加散騎常侍;進放爵西鄉侯,資樂陽亭侯。資別傳曰:諸葛亮出在南鄭,時議者以為可因大發兵,就討之,帝意亦然,以問資。資曰:「昔武皇帝征南鄭,取張魯,陽平之役,危而後濟。又自往拔出夏侯淵軍,數言『南鄭直為天獄,中斜谷道為五百里石穴耳』,言其深險,喜出淵軍之辭也。又武皇帝聖於用兵,察蜀賊栖於山巖,視吳虜竄於江湖,皆撓而避之,不責將士之力,不爭一朝之忿,誠所謂見勝而戰,知難而退也。今若進軍就南鄭討亮,道旣險阻,計用精兵又轉運鎮守南方四州遏禦水賊,凡用十五六萬人,必當復更有所發興。天下騷動,費力廣大,此誠陛下所宜深慮。夫守戰之力,力役三倍。但以今日見兵,分命大將據諸要險,威足以震攝彊寇,鎮靜疆埸,將士虎睡,百姓無事。數年之間,中國日盛,吳蜀二虜必自弊。」帝由是止。時吳人彭綺又舉義江南,議者以為因此伐之,必有所克。帝問資,資曰:「鄱陽宗人前後數有舉義者,衆弱謀淺,旋輒乖散。昔文皇帝嘗密問賊形勢,言洞浦殺萬人,得船千萬,數日間船人復會;江陵被圍歷月,權裁以千數百兵住東門,而其土地無崩解者。是有法禁,上下相奉持之明驗也。以此推綺,懼未能為權腹心大疾也。」綺果尋敗亡。

太和末,吳遣將周賀浮海詣遼東,招誘公孫淵。帝欲邀討之,朝議多以為不可。惟資決行策,果大破之,進爵左鄉侯。魏氏春秋曰:烏丸校尉田豫帥西部鮮卑泄歸尼等出塞,討軻比能、智鬱築鞬,破之,還至馬邑故城,比能帥三萬騎圍豫。帝聞之,計未有所出,如中書省以問監、令。令孫資對曰:「上谷太守閻志,柔弟也,為比能素所歸信。令馳詔使說比能,可不勞師而自解矣。」帝從之,比能果釋豫而還。放善為書檄,三祖詔命有所招喻,多放所為。青龍初,孫權與諸葛亮連和,欲俱出為寇。邊候得權書,放乃改易其辭,往往換其本文而傅合之,與征東將軍滿寵,若欲歸化,封以示亮。亮騰與吳大將步隲等,以見權。權懼亮自疑,深自解說。是歲,俱加侍中、光祿大夫。資別傳曰:是時,孫權、諸葛亮號稱劇賊,無歲不有軍征。而帝總攝群下,內圖禦寇之計,外規廟勝之畫,資皆管之。然自以受腹心,常讓事於帝曰:「動大衆,舉大事,宜與群下共之;旣以示明,且於探求為廣。」旣朝臣會議,資奏當其是非,擇其善者推成之,終不顯己之德也。若衆人有譴過及愛憎之說,輒復為請解,以塞譖潤之端。如征東將軍滿寵、涼州刺史徐邈,並有譖毀之者,資皆盛陳其素行,使卒無纖介。寵、邈得保其功名者,資之力也。初,資在邦邑,名出同類之右。鄉人司空掾田豫、梁相宗豔皆妬害之,而楊豐黨附豫等,專為資構造謗端,怨隙甚重。資旣不以為言,而終無恨意。豫等慙服,求釋宿憾,結為婚姻。資謂之曰:「吾無憾心,不知所釋。此為卿自薄之,卿自厚之耳!」乃為長子宏取其女。及當顯位,而田豫老疾在家。資遇之甚厚,又致其子於本郡,以為孝廉。而楊豐子後為尚方吏,帝以職事譴怒,欲致之法,資請活之。其不念舊惡如此。景初二年,遼東平定,以參謀之功,各進爵,封本縣,放方城侯,資中都侯。

其年,帝寢疾,欲以燕王宇為大將軍,及領軍將軍夏侯獻、武衞將軍曹爽、屯騎校尉曹肇、驍騎將軍秦朗共輔政。宇性恭良,陳誠固辭。帝引見放、資,入卧內,問曰:「燕王正爾為?」放、資對曰:「燕王實自知不堪大任故耳。」帝曰:「曹爽可代宇不?」放、資因贊成之。又深陳宜速召太尉司馬宣王,以綱維皇室。帝納其言,即以黃紙授放作詔。放、資旣出,帝意復變,詔止宣王勿使來。尋更見放、資曰:「我自召太尉,而曹肇等反使吾止之,幾敗吾事!」命更為詔,帝獨召爽與放、資俱受詔命,遂免宇、獻、肇、朗官。太尉亦至,登牀受詔,然後帝崩。世語曰:放、資乆典機任,獻、肇心內不平。殿中有雞棲樹,二人相謂:「此亦乆矣,其能復幾?」指謂放、資。放、資懼,乃勸帝召宣王。帝作手詔,令給使辟邪至,以授宣王。宣王在汲,獻等先詔令於軹關西還長安,辟邪又至,宣王疑有變,呼辟邪具問,乃乘追鋒車馳至京師。帝問放、資:「誰可與太尉對者?」放曰:「曹爽。」帝曰:「堪其事不?」爽在左右,汗流不能對。放躡其足,耳之曰:「臣以死奉社稷。」曹肇弟纂為大將軍司馬,燕王頗失指。肇出,纂見,驚曰:「上不安,云何悉共出?宜還。」已暮,放、資宣詔宮門,不得復內肇等,罷燕王。肇明日至門,不得入,懼,詣延尉,以處事失宜免。帝謂獻曰:「吾已差,便出。」獻流涕而出,亦免。案世語所云樹置先後,與本傳不同。資別傳曰:帝詔資曰:「吾年稍長,又歷觀書傳中,皆歎息無所不念。圖萬年後計,莫過使親人廣據職勢,兵任又重。今射聲校尉缺,乆欲得親人,誰可用者?」資曰:「陛下思深慮遠,誠非愚臣所及。書傳所載,皆聖聽所究,向使漢高不知平、勃能安劉氏,孝武不識金、霍付屬以事,殆不可言!文皇帝始召曹真還時,親詔臣以重慮,及至晏駕,陛下即阼,猶有曹休外內之望,賴遭日月,御勒不傾,使各守分職,纖介不間。以此推之,親臣貴戚,雖當據勢握兵,宜使輕重素定。若諸侯典兵,力均衡平,寵齊愛等,則不相為服;不相為服,則意有異同。今五營所領見兵,常不過數百,選授校尉,如其輩類,為有疇匹。至於重大之任,能有所維綱者,宜以聖意簡擇,如平、勃、金、霍、劉章等一二人,漸殊其威重,使相鎮固,於事為善。」帝曰:「然。如卿言,當為吾遠慮所圖。今日可參平、勃,侔金、霍,雙劉章者,其誰哉?」資曰:「臣聞知人則哲,惟帝難之。唐虞之聖,凡所進用,明試以功。陳平初事漢祖,絳、灌等謗平有受金盜嫂之罪。周勃以吹簫引彊,始事高祖,亦未知名也;高祖察其行跡,然後知可付以大事。霍光給事中二十餘年,小心謹慎,乃見親信。日磾夷狄,以至孝質直,特見擢用,左右尚曰『妄得一胡兒而重貴之』。平、勃雖安漢嗣,其終,勃被反名,平劣自免於呂須之讒。上官桀、桑弘羊與霍光爭權,幾成禍亂。此誠知人之不易,為臣之難也。又所簡擇,當得陛下所親,當得陛下所信,誠非愚臣之所能識別。」臣松之以為孫、劉于時號為專任,制斷機密,政事無不綜。資、放被託付之問,當安危所斷,而更依違其對,無有適莫。受人親任,理豈得然?案本傳及諸書並云放、資稱贊曹爽,勸召宣王,魏室之亡,禍基於此。資之別傳,出自其家,欲以是言掩其大失,然恐負國之玷,終莫能磨也。齊王即位,以放、資決定大謀,增邑三百,放并前千一百,資千戶;封愛子一人亭侯,次子騎都尉,餘子皆郎中。正始元年,更加放左光祿大夫,資右光祿大夫,金印紫綬,儀同三司。六年,放轉驃騎,資衞將軍,領監、令如故。七年,復封子一人亭侯,各年老遜位,以列侯朝朔望,位特進。資別傳曰:大將軍爽專事,多變易舊章。資歎曰:「吾累世蒙寵,加以豫聞屬託,今縱不能匡弼時事,可以坐受素飡之祿邪?」遂固稱疾。九年二月,乃賜詔曰:「君掌機密三十餘年,經營庶事,勳著前朝。曁朕統位,動賴良謀。是以曩者增崇寵章,同之三事,外帥群官,內望讜言。屬以年耆疾篤,上還印綬,前後鄭重,辭旨懇切。天地以大順成德,君子以善恕成仁,重以職事,違奪君志;今聽所執,賜錢百萬,使兼光祿勳少府親策詔君養疾于第。君其勉進醫藥,頤神和氣,以永無疆之祚。置舍人官騎,加以日秩肴酒之膳焉。」曹爽誅後,復以資為侍中,領中書令。嘉平二年,放薨,謚曰敬侯。子正嗣。臣松之案頭責子羽曰:士卿劉許字文生,正之弟也。與張華六人,並稱文辭可觀,意思詳序。晉惠帝世,許為越騎校尉。資復遜位歸第,就拜驃騎將軍,轉侍中,特進如故。三年薨,謚曰貞侯。子宏嗣。

放才計優資,而自脩不如也。放、資旣善承順主上,又未嘗顯言得失,抑辛毗而助王思,以是獲譏於世。然時因群臣諫諍,扶贊其義,并時密陳損益,不專導諛言云。及咸熈中,開建五等,以放、資著勳前朝,改封正方城子,宏離石子。案孫氏譜:宏為南陽太守。宏子楚,字子荊。晉陽秋曰:楚鄉人王濟,豪俊公子也,為本州大中正。訪問關求楚品狀,濟曰:「此人非卿所能名。」自狀之曰:「天才英博,亮拔不羣。」楚位至討虜護軍、馮翊太守。楚子洵,潁川太守。洵子盛,字安國,給事中,祕書監。盛從父弟綽,字興公,廷尉正。楚及盛、綽,並有文藻,盛又善言名理,諸所論著,並傳於世。

評曰:程昱、郭嘉、董昭、劉曄、蔣濟才策謀略,世之奇士,雖清治德業殊於荀攸,而籌畫所料是其倫也。劉放文翰,孫資勤慎,並管喉舌,權聞當時,雅亮非體,是故譏諛之聲,每過其實矣。


\end{pinyinscope}