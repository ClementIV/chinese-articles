\article{馬超傳}

\begin{pinyinscope}
馬超字孟起,右扶風茂陵人也。父騰,靈帝末與邊章、韓遂等俱起事於西州。初平三年,遂、騰率衆詣長安。漢朝以遂為鎮西將軍,遣還金城,騰為征西將軍,遣屯郿。後騰襲長安,敗走,退還涼州。司隷校尉鍾繇鎮關中,移書遂、騰,為陳禍福。騰遣超隨繇討郭援、高幹於平陽,超將龐德親斬援首。後騰與韓遂不和,求還京畿。於是徵為衞尉,以超為偏將軍,封都亭侯,領騰部曲。

典略曰:騰字壽成,馬援後也。桓帝時,其父字子碩,甞為天水蘭干尉。後失官,因留隴西,與羌錯居。家貧無妻,遂娶羌女,生騰。騰少貧無產業,常從鄣山中斫材木,負販詣城市,以自供給。騰為人長八尺餘,身體洪大,面鼻雄異,而性賢厚,人多敬之。靈帝末,涼州刺史耿鄙任信姦吏,民王國等及氐、羌反叛。州郡募發民中有勇力者,欲討之,騰在募中。州郡異之,署為軍從事,典領部衆。討賊有功,拜軍司馬,後以功遷偏將軍,又遷征西將軍,常屯汧、隴之間。初平中,拜征東將軍。是時,西州少穀,騰自表軍人多乏,求就穀於池陽,遂移屯長平岸頭。而將王承等恐騰為己害,乃攻騰營。時騰近出無備,遂破走,西上。會三輔亂,不復來東,而與鎮西將軍韓遂結為異姓兄弟,始甚相親,後轉以部曲相侵入,更為讎敵。騰攻遂,遂走,合衆還攻騰,殺騰妻子,連兵不解。建安之初,國家綱紀始弛,乃使司隷校尉鍾繇、涼州牧韋端和解之。徵騰還屯槐里,轉拜為前將軍,假節,封槐里侯。北備胡寇,東備白騎,待士進賢,矜救民命,三輔甚安愛之。十三年,徵為衞尉,騰自見年老,遂入宿衞。初,曹公為丞相,辟騰長子超,不就。超後為司隷校尉督軍從事,討郭援,為飛矢所中,乃以囊囊其足而戰,破斬援首。詔拜徐州刺史,後拜諫議大夫。及騰之入,因詔拜為偏將軍,使領騰營。又拜超弟休奉車都尉,休弟鐵騎都尉,徙其家屬皆詣鄴,惟超獨留。

超旣統衆,遂與韓遂合從,及楊秋、李堪、成宜等相結,進軍至潼關。曹公與遂、超單馬會語,超負其多力,陰欲突前捉曹公,曹公左右將許褚瞋目眄之,超乃不敢動。曹公用賈詡謀,離間超、遂,更相猜疑,軍以大敗。山陽公載記曰:初,曹公軍在蒲阪,欲西渡,超謂韓遂曰:「宜於渭北拒之,不過二十日,河東穀盡,彼必走矣。」遂曰:「可聽令渡,蹙於河中,顧不快耶!」超計不得施。曹公聞之曰:「馬兒不死,吾無葬地也。」超走保諸戎,曹公追至安定,會北方有事,引軍東還。楊阜說曹公曰:「超有信、布之勇,甚得羌、胡心。若大軍還,不嚴為其備,隴上諸郡非國家之有也。」超果率諸戎以擊隴上郡縣,隴上郡縣皆應之,殺涼州刺史韋康,據兾城,有其衆。超自稱征西將軍,領并州牧,督涼州軍事。康故吏民楊阜、姜叙、梁寬、趙衢等合謀擊超。阜、叙起於鹵城,超出攻之,不能下;寬、衢閉兾城門,超不得入。進退狼狽,乃奔漢中依張魯。魯不足與計事,內懷於邑,聞先主圍劉璋於成都,密書請降。典略曰:建安十六年,超與關中諸將侯選、程銀、李堪、張橫、梁興、成宜、馬玩、楊秋、韓遂等,凡十部,俱反,其衆十萬,同據河、潼,建列營陣。是歲,曹公西征,與超等戰於河、渭之交,超等敗走。超至安定,遂奔涼州。詔收滅超家屬。超復敗於隴上。後奔漢中,張魯以為都講祭酒,欲妻之以女,或諫魯曰:「有人若此不愛其親,焉能愛人?」魯乃止。初,超未反時,其小婦弟种留三輔,及超敗,种先入漢中。正旦,种上壽於超,超搥胷吐血曰:「闔門百口,一旦同命,今二人相賀邪?」後數從魯求兵,欲北取涼州,魯遣往,無利。又魯將楊白等欲害其能,超遂從武都逃入氐中,轉奔往蜀。是歲建安十九年也。

先主遣人迎超,超將兵徑到城下。城中震怖,璋即稽首,典略曰:備聞超至,喜曰:「我得益州矣。」乃使人止超,而潛以兵資之。超到,令引軍屯城北,超至未一旬而成都潰。以超為平西將軍,督臨沮,因為前都亭侯。山陽公載記曰:超因見備待之厚,與備言,常呼備字,關羽怒,請殺之。備曰:「人窮來歸我,卿等怒,以呼我字故而殺之,何以示於天下也!」張飛曰:「如是,當示之以禮。」明日大會,請超入,羽、飛並杖刀立直,超顧坐席,不見羽、飛,見其直也,乃大驚,遂止不復呼備字。明日歎曰:「我今乃知其所以敗。為呼人主字,幾為關羽、張飛所殺。」自後乃尊事備。臣松之按,以為超以窮歸備,受其爵位,何容傲慢而呼備字?且備之入蜀,留關羽鎮荊州,羽未甞在益土也。故羽聞馬超歸降,以書問諸葛亮「超人才可誰比類」,不得如書所云。羽焉得與張飛立直乎?凡人行事,皆謂其可也,知其不可,則不行之矣。超若果呼備字,亦謂於理宜爾也。就令羽請殺超,超不應聞,但見二子立直,何由便知以呼字之故,云幾為關、張所殺乎?言不經理,深可忿疾也。袁暐、樂資等諸所記載,穢雜虛謬,若此之類,殆不可勝言也。

先主為漢中王,拜超為左將軍,假節。章武元年,遷驃騎將軍,領涼州牧,進封斄鄉侯,策曰:「朕以不德,獲繼至尊,奉承宗廟。曹操父子,世載其罪,朕用慘怛,疢如疾首。海內怨憤,歸正反本,曁于氐、羌率服,獯粥慕義。以君信著北土,威武並昭,是以委任授君,抗颺虓虎,兼董萬里,求民之瘼。其明宣朝化,懷保遠邇,肅慎賞罰,以篤漢祐,以對于天下。」二年卒,時年四十七。臨沒上疏曰:「臣門宗二百餘口,為孟德所誅略盡,惟有從弟岱,當為微宗血食之繼,深託陛下,餘無復言。」追謚超曰威侯,子承嗣。岱位至平北將軍,進爵陳倉侯。超女配安平王理。典略曰:初超之入蜀,其庶妻董及子秋,留依張魯。魯敗,曹公得之,以董賜閻圃,以秋付魯,魯自手殺之。


\end{pinyinscope}