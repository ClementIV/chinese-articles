\article{魏延傳}

\begin{pinyinscope}
魏延字文長,義陽人也。以部曲隨先主入蜀,數有戰功,遷牙門將軍。先主為漢中王,遷治成都,當得重將以鎮漢川,衆論以為必在張飛,飛亦以心自許。先主乃拔延為督漢中鎮遠將軍,領漢中太守,一軍盡驚。先主大會羣臣,問延曰:「今委卿以重任,卿居之欲云何?」延對曰:「若曹操舉天下而來,請為大王拒之;偏將十萬之衆至,請為大王吞之。」先主稱善,衆咸壯其言。先主踐尊號,進拜鎮北將軍。建興元年,封都亭侯。五年,諸葛亮駐漢中,更以延為督前部,領丞相司馬、涼州刺史,八年,使延西入羌中,魏後將軍費瑤、雍州刺史郭淮與延戰于陽谿,延大破淮等,遷為前軍師征西大將軍,假節,進封南鄭侯。

延每隨亮出,輙欲請兵萬人,與亮異道會于潼關,如韓信故事,亮制而不許。延常謂亮為怯,歎恨己才用之不盡。

魏略曰:夏侯楙為安西將軍,鎮長安,亮於南鄭與羣下計議,延曰:「聞夏侯楙少,主壻也,怯而無謀。今假延精兵五千,負糧五千,直從褒中出,循秦嶺而東,當子午而北,不過十日可到長安。楙聞延奄至,必乘船逃走。長安中惟有御史、京兆太守耳,黃門邸閣與散民之穀足周食也。比東方相合聚,尚二十許日,而公從斜谷來,必足以達。如此,則一舉而咸陽以西可定矣。」亮以為此縣危,不如安從坦道,可以平取隴右,十全必克而無虞,故不用延計。延旣善養士卒,勇猛過人,又性矜高,當時皆避下之。唯楊儀不假借延,延以為至忿,有如水火。十二年,亮出北谷口,延為前鋒。出亮營十里,延夢頭上生角,以問占夢趙直,直詐延曰:「夫麒麟有角而不用,此不戰而賊欲自破之象也。」退而告人曰:「角之為字,刀下用也;頭上用刀,其凶甚矣。」

秋,亮病困,密與長史楊儀、司馬費禕、護軍姜維等作身歿之後退軍節度,令延斷後,姜維次之;若延或不從命,軍便自發。亮適卒,祕不發喪,儀令禕往揣延意指。延曰:「丞相雖亡,吾自見在。府親官屬便可將喪還葬,吾自當率諸軍擊賊,云何以一人死廢天下之事邪?且魏延何人,當為楊儀所部勒,作斷後將乎!」因與禕共作行留部分,令禕手書與己連名,告下諸將。禕紿延曰:「當為君還解楊長史,長史文吏,稀更軍事,必不違命也。」禕出門馳馬而去,延尋悔,追之已不及矣。延遣人覘儀等,遂使欲案亮成規,諸營相次引軍還。延大怒,纔儀未發,率所領徑先南歸,所過燒絕閣道。延、儀各相表叛逆,一日之中,羽檄交至。後主以問侍中董允、留府長史蔣琬,琬、允咸保儀疑延。儀等槎山通道,晝夜兼行,亦繼延後。延先至,據南谷口,遣兵逆擊儀等,儀等令何平在前禦延。平叱延先登曰:「公亡,身尚未寒,汝輩何敢乃爾!」延士衆知曲在延,莫為用命,軍皆散。延獨與其子數人逃亡,奔漢中。儀遣馬岱追斬之,致首於儀,儀起自踏之,曰:「庸奴!復能作惡不?」遂夷延三族。初,蔣琬率宿衞諸營赴難北行,行數十里,延死問至,乃旋。原延意不北降魏而南還者,但欲除殺儀等。平日諸將素不同,兾時論必當以代亮。本指如此。不便背叛。魏略曰:諸葛亮病,謂延等云:「我之死後,但謹自守,慎勿復來也。」令延攝行己事,密持喪去。延遂匿之,行至襃口,乃發喪。亮長史楊儀宿與延不和,見延攝行軍事,懼為所害,乃張言延欲舉衆北附,遂率其衆攻延。延本無此心,不戰軍走,追而殺之。臣松之以為此蓋敵國傳聞之言,不得與本傳爭審。


\end{pinyinscope}