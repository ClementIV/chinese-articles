\article{dong-zhuo-chuan}

\begin{pinyinscope}
董卓字仲穎,隴西臨洮人也。

英雄記曰:卓父君雅,由微官為潁川綸氏尉。有三子:長子擢,字孟高,早卒;次即卓;卓弟旻字叔穎。少好俠,甞游羌中,盡與諸豪帥相結。後歸耕於野,而豪帥有來從之者,卓與俱還,殺耕牛與相宴樂。諸豪帥感其意,歸相斂,得雜畜千餘頭以贈卓。吳書曰:郡召卓為吏,使監領盜賊。胡甞出鈔,多虜民人,涼州刺史成就辟卓為從事,使領兵騎討捕,大破之,斬獲千計。并州刺史段熲薦卓公府,司徒袁隗辟為掾。漢桓帝末,以六郡良家子為羽林郎。卓有才武,旅力少比,雙帶兩鞬,左右馳射。為軍司馬,從中郎將張奐征并州有功,拜郎中,賜縑九千匹,卓悉以分與吏士。遷廣武令,蜀郡北部都尉,西域戊己校尉,免。徵拜并州刺史、河東太守,英雄記曰:卓數討羌、胡,前後百餘戰。遷中郎將,討黃巾,軍敗抵罪。韓遂等起涼州,復為中郎將,西拒遂。於望垣硤北為羌、胡數萬人所圍,糧食乏絕。卓偽欲捕魚,堰其還道當所渡水為池,使水渟滿數十里,默從堰下過其軍而決堰。比羌、胡聞知追逐,水已深,不得渡。時六軍上隴西,五軍敗績,卓獨全衆而還,屯住扶風。拜前將軍,封斄鄉侯,徵為并州牧。靈帝紀曰:中平五年,徵卓為少府,勑以營吏士屬左將軍皇甫嵩,詣行在所。卓上言:「涼州擾亂,鯨鯢未滅,此臣奮發效命之秋。吏士踊躍,戀恩念報,各遮臣車,辭聲懇惻,未得即路也。輙且行前將軍事,盡心慰卹,効力行陣。」六年,以卓為并州牧,又勑以吏兵屬皇甫嵩。卓復上言:「臣掌戎十年,士卒大小相狎彌乆,戀臣畜養之恩,樂為國家奮一旦之命,乞將之州,效力邊陲。」卓再違詔勑,會為何進所召。

靈帝崩,少帝即位。大將軍何進與司隷校尉袁紹謀誅諸閹官,太后不從。進乃召卓使將兵詣京師,并密令上書曰:「中常侍張讓等竊幸乘寵,濁亂海內。昔趙鞅興晉陽之甲,以逐君側之惡。臣輙鳴鍾鼓如洛陽,即討讓等。」欲以脅迫太后。卓未至,進敗。續漢書曰:進字遂高,南陽人,太后異母兄也。進本屠家子,父曰真。真死後,進以妹倚黃門得入掖庭,有寵,光和三年立為皇后,進由是貴幸。中平元年,黃巾起,拜進大將軍。典略載卓表曰:「臣伏惟天下所以有逆不止者,各由黃門常侍張讓等侮慢天常,操擅王命,父子兄弟並據州郡,一書出門,便獲千金,京畿諸郡數百萬膏腴美田皆屬讓等,至使怨氣上蒸,妖賊蠭起。臣前奉詔討於扶羅,將士饑乏,不肯渡河,皆言欲詣京師先誅閹豎以除民害,從臺閣求乞資直。臣隨慰撫,以至新安。臣聞揚湯止沸,不如滅火去薪,潰癕雖痛,勝於養肉,及溺呼船,悔之無及。」中常侍段珪等劫帝走小平津,卓遂將其衆迎帝于北芒,還宮。張璠漢紀曰:帝以八月庚午為諸黃門所劫,步出穀門,走至河上。諸黃門旣投河死。時帝年十四,陳留王年九歲,兄弟獨夜步行欲還宮,闇暝,逐螢火而行,數里,得民家以露車載送。辛未,公卿以下與卓共迎帝於北芒阪下。獻帝春秋曰:先是童謠曰:「侯非侯,王非王,千乘萬騎走北芒。」卓時適至,屯顯陽苑。聞帝當還,率衆迎帝。典略曰:帝望見卓兵涕泣。羣公謂卓曰:「有詔郤兵。」卓曰:「公諸人為國大臣,不能匡正王室,至使國家播蕩,何郤兵之有!」遂俱入城。獻帝紀曰:卓與帝語,語不可了。乃更與陳留王語,問禍亂由起;王荅,自初至終,無所遺失。卓大喜,乃有廢立意。英雄記曰:河南中部掾閔貢扶帝及陳留王上至雒舍止。帝獨乘一馬,陳留王與貢共乘一馬,從雒舍南行。公卿百官奉迎於北芒阪下,故太尉崔烈在前導。卓將步騎數千來迎,烈呵使避,卓罵烈曰:「晝夜三百里來,何云避,我不能斷卿頭邪?」前見帝曰:「陛下令常侍小黃門作亂乃爾,以取禍敗,為負不小邪?」又趨陳留王,曰:「我董卓也,從我抱來。」乃於貢抱中取王。英雄記曰:一本云王不就卓抱,卓與王併馬而行也。時進弟車騎將軍苗為進衆所殺,英雄記云:苗,太后之同母兄,先嫁朱氏之子。進部曲將吳匡,素怨苗不與進同心,又疑其與宦官通謀,乃令軍中曰:「殺大將軍者,車騎也。」遂引兵與卓弟旻共攻殺苗於朱爵闕下。進、苗部曲無所屬,皆詣卓。卓又使呂布殺執金吾丁原,并其衆,故京都兵權唯在卓。九州春秋曰:卓初入洛陽,步騎不過三千,自嫌兵少,不為遠近所服;率四五日,輙夜遣兵出四城門,明日陳旌鼓而入,宣言云「西兵復入至洛中」。人不覺,謂卓兵不可勝數。

先是,進遣騎都尉太山鮑信所在募兵,適至,信謂紹曰:「卓擁彊兵,有異志,今不早圖,將為所制;及其初至疲勞,襲之可禽也。」紹畏卓,不敢發,信遂還鄉里。

於是以乆不雨,策免司空劉弘而卓代之,俄遷太尉,假節鉞虎賁。遂廢帝為弘農王。尋又殺王及何太后。立靈帝少子陳留王,是為獻帝。獻帝紀曰:卓謀廢帝,會羣臣於朝堂,議曰:「大者天地,其次君臣,所以為治。今皇帝闇弱,不可以奉宗廟,為天下主。欲以依伊尹、霍光故事,立陳留王,何如?」尚書盧植曰:「案尚書太甲旣立不明,伊尹放之桐宮。昌邑王立二十七日,罪過千餘,故霍光廢之。今上富於春秋,行未有失,非前事之比也。」卓怒,罷坐,欲誅植,侍中蔡邕勸之,得免。九月甲戌,卓復大會羣臣曰:「太后逼迫永樂太后,令以憂死,逆婦姑之禮,無孝順之節。天子幼質,軟弱不君。昔伊尹放太甲,霍光廢昌邑,著在典籍,僉以為善。今太后宜如太甲,皇帝宜如昌邑。陳留王仁孝,宜即尊皇祚。」獻帝起居注載策曰:「孝靈皇帝不究高宗眉壽之祚,早棄臣子。皇帝承紹,海內側望,而帝天姿輕佻,威儀不恪,在喪慢惰,衰如故焉;凶德旣彰,淫穢發聞,損辱神器,忝汙宗廟。皇太后教無母儀,統政荒亂。永樂太后暴崩,衆論惑焉。三綱之道,天地之紀,而乃有闕,罪之大者。陳留王恊,聖德偉茂,規矩邈然,豐下兊上,有堯圖之表;居喪哀戚,言不及邪,岐嶷之性,有周成之懿。休聲美稱,天下所聞,宜承洪業,為萬世統,可以承宗廟。廢皇帝為弘農王。皇太后還政。」尚書讀冊畢,羣臣莫有言,尚書丁宮曰:「天禍漢室,喪亂弘多。昔祭仲廢忽立突,春秋大其權。今大臣量宜為社稷計,誠合天人,請稱萬歲。」卓以太后見廢,故公卿以下不布服,會葬,素衣而已。卓遷相國,封郿侯,贊拜不名,劒履上殿,又封卓母為池陽君,置家令、丞。卓旣率精兵來,適值帝室大亂,得專廢立,據有武庫甲兵、國家珍寶,威震天下。卓性殘忍不仁,遂以嚴刑脅衆,睚眦之隙必報,人不自保。魏書曰:卓所願無極,語賔客曰:「我相,貴無上也。」英雄記曰:卓欲震威,侍御史擾龍宗詣卓白事,不解劒,立撾殺之,京師震動。發何苗棺,出其尸,枝解節棄於道邊。又收苗母舞陽君殺之,棄尸於苑枳落中,不復收斂。甞遣軍到陽城。時適二月社,民各在其社下,悉就斷其男子頭,駕其車牛,載其婦女財物,以所斷頭繫車轅軸,連軫而還洛,云攻賊大獲,稱萬歲。入開陽城門,焚燒其頭,以婦女與甲兵為婢妾。至於姦亂宮人公主。其凶逆如此。

初,卓信任尚書周毖、城門校尉伍瓊等,用其所舉韓馥、劉岱、孔伷、張咨、張邈等出宰州郡。而馥等至官,皆合兵將以討卓。卓聞之,以為毖、瓊等通情賣己,皆斬之。英雄記曰:毖字仲遠,武威人。瓊字德瑜,汝南人。謝承後漢書曰:伍孚字德瑜,少有大節,為郡門下書佐。其本邑長有罪,太守使孚出教,就曹下督郵收之。孚不肯受教,伏地仰諫曰:「君雖不君,臣不可不臣,明府柰何令孚受教,勑外収本邑長乎?更乞授他吏。」太守奇而聽之。後大將軍何進辟為東曹屬,稍遷侍中、河南尹、越騎校尉。董卓作亂,百僚震慄。孚著小鎧,於朝服裏挾佩刀見卓,欲伺便刺殺之。語闋辭去,卓送至閤中,孚因出刀刺之。卓多力,退郤不中,即収孚。卓曰:「卿欲反邪?」孚大言曰:「汝非吾君,吾非汝臣,何反之有?汝亂國篡主,罪盈惡大,今是吾死日,故來誅姦賊耳,恨不車裂汝於市朝以謝天下。」遂殺孚。謝承記孚字及本郡,則與瓊同,而致死事乃與孚異也,不知孚為瓊之別名,為別有伍孚也?蓋未詳之。

河內太守王匡遣泰山兵屯河陽津,將以圖卓。卓遣疑兵若將於平陰渡者,潛遣銳衆從小平北渡,繞擊其後,大破之津北,死者略盡。卓以山東豪傑並起,恐懼不寧。初平元年二月,乃徙天子都長安。焚燒洛陽宮室,悉發掘陵墓,取寶物。華嶠漢書曰:卓欲遷都長安,召公卿以下大議。司徒楊彪曰:「昔盤庚五遷,殷民胥怨,故作三篇以曉天下之民。今海內安穩,無故移都,恐百姓驚動,麋沸蟻聚為亂。」卓曰:「關中肥饒,故秦得并吞六國。今徙西京,設令關東豪彊敢有動者,以我彊兵踧之,可使詣滄海。」彪曰:「海內動之甚易,安之甚難。又長安宮室壞敗,不可卒復。」卓曰:「武帝時居杜陵南山下,有成瓦窑數千處,引涼州材木東下以作宮室,為功不難。」卓意不得,便作色曰:「公欲沮我計邪?邊章、韓約有書來,欲令朝廷必徙都。若大兵來下,我不能復相救,公便可與袁氏西行。」彪曰:「西方自彪道徑也,顧未知天下何如耳!」議罷。卓勑司隷校尉宣璠以災異劾奏,因策免彪。續漢書曰:太尉黃琬、司徒楊彪、司空荀爽俱詣卓,卓言:「昔高祖都關中,十一世後中興,更都洛陽。從光武至今復十一世,案石苞室讖,宜復還都長安。」坐中皆驚愕,無敢應者。彪曰:「遷都改制,天下大事,皆當因民之心,隨時之宜。昔盤庚五遷,殷民胥怨,故作三篇以曉之。往者王莽篡逆,變亂五常,更始赤眉之時,焚燒長安,殘害百姓,民人流亡,百無一在。光武受命,更都洛邑,此其宜也。今方建立聖主,光隆漢祚,而無故捐宮廟,棄園陵,恐百姓驚愕,不解此意,必麋沸蟻聚以致擾亂。石苞室讖,妖邪之書,豈可信用?」卓作色曰:「楊公欲沮國家計邪?關東方亂,所在賊起。崤函險固,國之重防。又隴右取材,功夫不難。杜陵南山下有孝武故陶處,作塼瓦,一朝可辦。宮室官府,蓋何足言!百姓小民,何足與議。若有前郤,我以大兵驅之,豈得自在。」百寮恐怖失色。琬謂卓曰:「此大事。楊公之語,得無重思!」卓罷坐,即日令司隷奏彪及琬,皆免官。大駕即西。卓部兵燒洛陽城外面百里。又自將兵燒南北宮及宗廟、府庫、民家,城內埽地殄盡。又收諸富室,以罪惡沒入其財物;無辜而死者,不可勝計。獻帝紀曰:卓獲山東兵,以豬膏塗布十餘匹,用纏其身,然後燒之,先從足起。獲袁紹豫州從事李延,煑殺之。卓所愛胡,恃寵放縱,為司隷校尉趙謙所殺。卓大怒曰:「我愛狗,尚不欲令人呵之,而況人乎!」乃召司隷都官撾殺之。卓至西京,為太師,號曰尚父。乘青蓋金華車,爪畫兩轓,時人號曰竿摩車。魏書曰:言其逼天子也。獻帝紀曰:卓旣為太師,復欲稱尚父,以問蔡邕。邕曰:「昔武王受命,太公為師,輔佐周室,以伐無道,是以天下尊之,稱為尚父。今公之功德誠為巍巍,宜須關東悉定,車駕東還,然後議之。」乃止。京師地震,卓又問邕。邕對曰:「地動陰盛,大臣踰制之所致也。公乘青蓋車,遠近以為非宜。」卓從之,更乘金華皂蓋車也。卓弟旻為左將軍,封鄠侯;兄子璜為侍中中軍校尉典兵;宗族內外並列朝廷。英雄記曰:卓侍妾懷抱中子,皆封侯,弄以金紫。孫女名白,時尚未笄,封為渭陽君。於郿城東起壇,從廣二丈餘,高五六尺,使白乘軒金華青蓋車,都尉、中郎將、刺史二千石在郿者,各令乘軒簪筆,為白導從,之壇上,使兄子璜為使者授印綬。公卿見卓,謁拜車下,卓不為禮。召呼三臺尚書以下自詣卓府啟事。山陽公載記曰:初卓為前將軍,皇甫嵩為左將軍,俱征韓遂,各不相下。後卓徵為少府并州牧,兵當屬嵩,卓大怒。及為太師,嵩為御史中丞,拜於車下。卓問嵩:「義真服未乎?」嵩曰:「安知明公乃至於是!」卓曰:「鴻鵠固有遠志,但燕雀自不知耳。」嵩曰:「昔與明公俱為鴻鵠,不意今日變為鳳皇耳。」卓笑曰:「卿早服,今日可不拜也。」張璠漢紀曰:卓抵其手謂皇甫嵩曰:「義真怖未乎?」嵩對曰:「明公以德輔朝廷,大慶方至,何怖之有?若淫刑以逞,將天下皆懼,豈獨嵩乎?」卓默然,遂與嵩和解。築郿塢,高與長安城埒,積穀為三十年儲,英雄記曰:郿去長安二百六十里。云事成,雄據天下,不成,守此足以畢老。甞至郿行塢,公卿已下祖道於橫門外。橫音光。卓豫施帳幔飲,誘降北地反者數百人,於坐中先斷其舌,或斬手足,或鑿眼,或鑊煑之,未死,偃轉杯案閒,會者皆戰慄亡失匕箸,而卓飲食自若。太史望氣,言當有大臣戮死者。故太尉張溫時為衞尉,素不善卓,卓心怨之,因天有變,欲以塞咎,使人言溫與袁術交關,遂笞殺之。傅子曰:靈帝時牓門賣官,於是太尉段熲、司徒崔烈、太尉樊陵、司空張溫之徒,皆入錢上千萬下五百萬以買三公。熲數征伐有大功,烈有北州重名,溫有傑才,陵能偶時,皆一時顯士,猶以貨取位,而況於劉嚻、唐珍、張顥之黨乎!風俗通曰:司隷劉嚻以黨諸常侍,致位公輔。續漢書曰:唐珍,中常侍唐衡弟。張顥,中常侍張奉弟。法令苛酷,愛憎淫刑,更相被誣,冤死者千數。百姓嗷嗷,道路以目。魏書曰:卓使司隷校尉劉嚻籍吏民有為子不孝,為臣不忠,為吏不清,為弟不順,有應此者皆身誅,財物沒官。於是愛憎互起,民多冤死。悉椎破銅人、鍾虡,及壞五銖錢。更鑄為小錢,大五分,無文章,肉好無輪郭,不磨鑢。於是貨輕而物貴,穀一斛至數十萬。自是後錢貨不行。

三年四月,司徒王允、尚書僕射士孫瑞、卓將呂布共謀誅卓。是時,天子有疾新愈,大會未央殿。布使同郡騎都尉李肅等將親兵十餘人,偽著衞士服守掖門。布懷詔書。卓至,肅等格卓。卓驚呼布所在。布曰「有詔」,遂殺卓,夷三族。主簿田景前趨卓尸,布又殺之;凡所殺三人,餘莫敢動。英雄記曰:時有謠言曰:「千里草,何青青,十日卜,猶不生。」又作董逃之歌。又有道士書布為「呂」字以示卓,卓不知其為呂布也。卓當入會,陳列步騎,自營至宮,朝服導引行其中。馬躓不前,卓心怪欲止,布勸使行,乃衷甲而入。卓旣死,當時日月清淨,微風不起。旻、璜等及宗族老弱悉在郿,皆還,為其羣下所斫射。卓母年九十,走至塢門曰「乞脫我死」,即斬首。袁氏門生故吏改殯諸袁死於郿者,斂聚董氏尸於其側而焚之。暴卓尸於市。卓素肥,膏流浸地,草為之丹。守尸吏暝以為大炷,置卓臍中以為燈,光明達旦,如是積日。後卓故部曲收所燒者灰,并以一棺棺之,葬於郿。卓塢中金有二三萬斤,銀八九萬斤,珠玉錦綺奇玩雜物皆山崇阜積,不可知數。長安士庶咸相慶賀,諸阿附卓者皆下獄死。謝承後漢書曰:蔡邕在王允坐,聞卓死,有歎惜之音。允責邕曰:「卓,國之大賊,殺主殘臣,天地所不祐,人神所同疾。君為王臣,世受漢恩,國主危難,曾不倒戈,卓受天誅,而更嗟痛乎?」便使收付廷尉。邕謝允曰:「雖以不忠,猶識大義,古今安危,耳所厭聞,口所常玩,豈當背國而向卓也?狂瞽之詞,謬出患入,願黥首為刑以繼漢史。」公卿惜邕才,咸共諫允。允曰:「昔武帝不殺司馬遷,使作謗書,流於後世。方今國祚中衰,戎馬在郊,不可令佞臣執筆在幼主左右,後令吾徒並受謗議。」遂殺邕。臣松之以為蔡邕雖為卓所親任,情必不黨。寧不知卓之姦凶,為天下所毒,聞其死亡,理無歎惜。縱復令然,不應反言於王允之坐。斯殆謝承之妄記也。史遷紀傳,博有奇功於世,而云王允謂孝武應早殺遷,此非識者之言。但遷為不隱孝武之失,直書其事耳,何謗之有乎?王允之忠正,可謂內省不疚者矣,旣無懼於謗,且欲殺邕,當論邕應死與不,豈可慮其謗己而枉戮善人哉!此皆誣罔不通之甚者。張璠漢紀曰:初,蔡邕以言事見徙,名聞天下,義動志士。及還,內寵惡之。邕恐,乃亡命海濵,往來依太山羊氏,積十年。卓為太尉,辟為掾,以高第為侍御史治書,三日中遂至尚書。後遷巴東太守,卓上留拜侍中,至長安為左中郎將。卓重其才,厚遇之。每有朝廷事,常令邕具草。及允將殺邕,時名士多為之言,允悔欲止,而邕已死。

初,卓女壻中郎將牛輔典兵別屯陝,分遣校尉李傕、郭汜、張濟略陳留、潁川諸縣。卓死,呂布使李肅至陝,欲以詔命誅輔。輔等逆與肅戰,肅敗走弘農,布誅肅。魏書曰:輔恇怯失守,不能自安。常把辟兵符,以鈇鑕致其旁,欲以自彊。見客,先使相者相之,知有反氣與不,又筮知吉凶,然後乃見之。中郎將董越來就輔,輔使筮之,得兊下離上,筮者曰:「火勝金,外謀內之卦也。」即時殺越。獻帝紀云:筮人常為越所鞭,故因此以報之。其後輔營兵有夜叛出者,營中驚,輔以為皆叛,乃取金寶,獨與素所厚友胡赤兒等五六人相隨,踰城北渡河,赤兒等利其金寶,斬首送長安。

比傕等還,輔已敗,衆無所依,欲各散歸。旣無赦書,而聞長安中欲盡誅涼州人,憂恐不知所為。用賈詡策,遂將其衆而西,所在收兵,比至長安,衆十餘萬,九州春秋曰:傕等在陝,皆恐怖,急擁兵自守。胡文才、楊整脩皆涼州大人,而司徒王允素所不善也。及李傕之叛,允乃呼文才、整脩使東解釋之,不假借以溫顏,謂曰:「關東鼠子欲何為邪?卿往呼之。」於是二人往,實召兵而還。與卓故部曲樊稠、李蒙、王方等合圍長安城。十日城陷,與布戰城中,布敗走。傕等放兵略長安老少,殺之悉盡,死者狼藉。誅殺卓者,尸王允於市。張璠漢紀曰:布兵敗,住馬青瑣門外,謂允曰:「公可以去。」允曰:「安國家,吾之上願也,若不獲,則奉身以死。朝廷幼主恃我而已,臨難苟免,吾不為也。努力謝關東諸公,以國家為念。」傕、汜入長安城,屯南宮掖門,殺太僕魯馗、大鴻臚周奐、城門校尉崔烈、越騎校尉王頎。吏民死者不可勝數。司徒王允扶天子上宣平城門避兵,傕等於城門下拜,伏地叩頭。帝謂傕等曰:「卿無作威福,而乃放兵縱橫,欲何為乎?」傕等曰:「董卓忠於陛下,而無故為呂布所殺。臣等為卓報讎,弗敢為逆也。請事竟,詣廷尉受罪。」允窮逼出見傕,傕誅允及妻子宗族十餘人。長安城中男女大小莫不流涕。允字子師,太原祁人也。少有大節,郭泰見而奇之,曰:「王生一日千里,王佐之才也。」泰雖先達,遂與定交。三公並辟,歷豫州刺史,辟荀爽、孔融為從事,遷河南尹、尚書令。及為司徒,其所以扶持王室,甚得大臣之節,自天子以下,皆倚賴焉。卓亦推信之,委以朝廷。華嶠曰:夫士以正立,以謀濟,以義成,若王允之推董卓而分其權,伺其間而弊其罪。當此之時,天下之難解矣,本之皆主於忠義也,故推卓不為失正,分權不為不義,伺閒不為狙詐,是以謀濟義成,而歸於正也。葬卓於郿,大風暴雨震卓墓,水流入藏,漂其棺槨。傕為車騎將軍、池陽侯,領司隷校尉、假節。汜為後將軍、美陽侯。稠為右將軍、萬年侯。傕、汜、稠擅朝政。英雄記曰:傕,北地人。汜,張掖人,一名多。濟為驃騎將軍、平陽侯,屯弘農。

是歲,韓遂、馬騰等降,率衆詣長安。以遂為鎮西將軍,遣還涼州,騰征西將軍,屯郿。侍中馬宇與諫議大夫种邵、左中郎將劉範等謀,欲使騰襲長安,己為內應,以誅傕等。騰引兵至長平觀,宇等謀泄,出奔槐里。稠擊騰,騰敗走,還涼州;又攻槐里,宇等皆死。時三輔民尚數十萬戶,傕等放兵劫略,攻剽城邑,人民饑困,二年間相啖食略盡。獻帝紀曰:是時新遷都,宮人多亡衣服,帝欲發御府繒以與之,李傕弗欲,曰:「宮中有衣,胡為復作邪?」詔賣廄馬百餘匹,御府大司農出雜繒二萬匹,與所賣廄馬直,賜公卿以下及貧民不能自存者。李傕曰「我邸閣儲偫少」,乃悉載置其營。賈詡曰「此上意,不可拒」,傕不從之。

諸將爭權,遂殺稠,并其衆。九州春秋曰:馬騰、韓遂之敗,樊稠追至陳倉。遂語稠曰:「天地反覆,未可知也。本所爭者非私怨,王家事耳。與足下州里人,今雖小違,要當大同,欲相與善語以別。邂逅萬一不如意,後可復相見乎!」俱郤騎前接馬,交臂相加,共語良乆而別。傕兄子利隨稠,利還告傕「韓、樊交馬語」,不知所道,意愛甚密。傕以是疑稠與韓遂私和而有異意。稠欲將兵東出關,從傕索益兵。因請稠會議,便於坐殺稠。汜與傕轉相疑,戰鬬長安中。典略曰:傕數設酒請汜,或留汜止宿。汜妻懼傕與汜婢妾而奪己愛,思有以離間之。會傕送饋,妻乃以豉為藥,汜將食,妻曰:「食從外來,儻或有故!」遂摘藥示之,曰:「一栖不二雄,我固疑將軍之信李公也。」他日傕復請汜,大醉。汜疑傕藥之,絞糞汁飲之乃解。於是遂生嫌隙,而治兵相攻。傕質天子於營,燒宮殿城門,略官寺,盡收乘輿服御物置其家。獻帝起居注曰:初,汜謀迎天子幸其營,夜有亡告傕者,傕使兄子暹將數千兵圍宮,以車三乘迎天子。楊彪曰:「自古帝王無在人臣家者。舉事當合天下心,諸君作此,非是也。」暹曰:「將軍計定矣。」於是天子一乘,貴人伏氏一乘,賈詡、左靈一乘,其餘皆步從。是日,傕復移乘輿幸北塢,使校尉監塢門,內外隔絕。諸侍臣皆有饑色,時盛暑熱,人盡寒心。帝求米五斛、牛骨五具以賜左右,傕曰:「朝餔上飯,何用米為?」乃與腐牛骨,皆臭不可食。帝大怒,欲詰責之。侍中楊琦上封事曰:「傕,邊鄙之人,習於夷風,今又自知所犯悖逆,常有怏怏之色,欲輔車駕幸黃白城以紓其憤。臣願陛下忍之,未可顯其罪也。」帝納之。初,傕屯黃白城,故謀欲徙之。傕以司徒趙溫不與己同,乃內溫塢中。溫聞傕欲移乘輿,與傕書曰:「公前託為董公報讎,然實屠陷王城,殺戮大臣,天下不可家見而戶釋也。今爭睚眥之隙,以成千鈞之讎,民在塗炭,各不聊生,曾不改寤,遂成禍亂。朝廷仍下明詔,欲令和解,詔命不行,恩澤日損,而復欲輔乘輿於黃白城,此誠老夫所不解也。於易,一過為過,再為涉,三而弗改,滅其頂,凶。不如早共和解,引兵還屯,上安萬乘,下全生民,豈不幸甚!」傕大怒,欲遣人害溫。其從弟應,溫故掾也,諫之數日乃止。帝聞溫與傕書,問侍中常洽曰:「傕弗知臧否,溫言太切,可為寒心。」對曰:「李應已解之矣。」帝乃恱之。傕使公卿詣汜請和,汜皆執之。華嶠漢書曰:汜饗公卿,議欲攻傕。楊彪曰:「羣臣共鬬,一人劫天子,一人質公卿,此可行乎?」汜怒,欲手刃之,中郎將楊密及左右多諫,汜乃歸之。相攻擊連月,死者萬數。獻帝起居注曰:傕性喜鬼恠左道之術,常有道人及女巫歌謳擊鼓下神,祠祭六丁,符劾厭勝之具,無所不為。又於朝廷省門外,為董卓作神坐,數以牛羊祠之,訖,過省閤問起居,求入見。傕帶三刀,手復與鞭合持一刃。侍中、侍郎見傕帶仗,皆惶恐,亦帶劒持刀,先入在帝側。傕對帝,或言「明陛下」,或言「明帝」,為帝說郭汜無狀,帝亦隨其意荅應之。傕喜,出言「明陛下真賢聖主」,意遂自信,自謂良得天子歡心也。雖然,猶不欲令近臣帶劒在帝邊,謂人言「此曹子將欲圖我邪?而皆持刀也」。侍中李禎,傕州里,素與傕通,語傕「所以持刀者,軍中不可不爾,此國家故事」。傕意乃解。天子以謁者僕射皇甫酈涼州舊姓,有專對之才,遣令和傕、汜。酈先詣汜,汜受詔命。詣傕,傕不肯,曰:「我有討呂布之功,輔政四年,三輔清靜,天下所知也。郭多,盜馬虜耳,何敢乃欲與吾等邪?必欲誅之。君為涼州人,觀吾方略士衆,足辦多不?多又劫質公卿,所為如是,而君苟欲利郭多,李傕有膽自知之。」酈荅曰:「昔有窮后羿恃其善射,不思患難,以至於斃。近董公之彊,明將軍目所見,內有王公以為內主,外有董旻、承、璜以為鯁毒,呂布受恩而反圖之,斯須之間,頭縣竿端,此有勇而無謀也。今將軍身為上將,把鉞仗節,子孫握權,宗族荷寵,國家好爵而皆據之。今郭多劫質公卿,將軍脅至尊,誰為輕重邪?張濟與郭多、楊定有謀,又為冠帶所附。楊奉,白波帥耳,猶知將軍所為非是,將軍雖拜寵之,猶不肯盡力也。」傕不納酈言,而呵之令出。酈出,詣省門,白傕不肯從詔,辭語不順。侍中胡邈為傕所幸,呼傳詔者令飾其辭。又謂酈曰:「李將軍於卿不薄,又皇甫公為太尉,李將軍力也。」酈荅曰:「胡敬才,卿為國家常伯,輔弼之臣也,語言如此,寧可用邪?」邈曰:「念卿失李將軍意,恐不易耳!我與卿何事者?」酈言:「我累世受恩,身又常在幃幄,君辱臣死,當坐國家,為李傕所殺,則天命也。」天子聞酈荅語切,恐傕聞之,便勑遣酈。酈裁出營門,傕遣虎賁王昌呼之。昌知酈忠直,縱令去,還荅傕,言追之不及。天子使左中郎將李固持節拜傕為大司馬,在三公之右。傕自以為得鬼神之力,乃厚賜諸巫。

傕將楊奉與傕軍吏宋果等謀殺傕,事泄,遂將兵叛傕。傕衆叛,稍衰弱。張濟自陝和解之,天子乃得出,至新豐、霸陵間。獻帝起居注曰:初,天子出到宣平門,當渡橋,汜兵數百人遮橋問「是天子邪」?車不得前。傕兵數百人皆持大戟在乘輿車左右,侍中劉艾大呼云:「是天子也。」使侍中楊琦高舉車帷。帝言諸兵:「汝不郤,何敢迫近至尊邪?」汜等兵乃郤。旣度橋,士衆咸呼萬歲。郭汜復欲脅天子還都郿。天子奔奉營,奉擊汜,破之。汜走南山,奉及將軍董承以天子還洛陽。傕、汜悔遣天子,復相與和,追及天子於弘農之曹陽。奉急招河東故白波帥韓暹、胡才、李樂等合,與傕、汜大戰。奉兵敗,傕等縱兵殺公卿百官,略宮人入弘農。獻帝紀曰:時尚書令士孫瑞為亂兵所害。三輔決錄注曰:瑞字君榮,扶風人,世為學門。瑞少傳家業,博達無所不通,仕歷顯位。卓旣誅,遷大司農,為三老。每三公缺,瑞常在選中。太尉周忠、皇甫嵩,司徒淳于嘉、趙溫,司空楊彪、張喜等為公,皆辭拜讓瑞。天子都許,追論瑞功,封子萌澹津亭侯。萌字文始,亦有才學,與王粲善。臨當就國,粲作詩以贈萌,萌有荅,在粲集中。天子走陝,北渡河,失輜重,步行,唯皇后貴人從,至大陽,止人家屋中。獻帝紀曰:初,議者欲令天子浮河東下,太尉楊彪曰:「臣弘農人,從此已東有三十六灘,非萬乘所當從也。」劉艾曰:「臣前為陝令,知其危險,有師猶有傾覆,況今無師,太尉謀是也。」乃止。及當北渡,使李樂具船。天子步行趨河岸,岸高不得下,董承等謀欲以馬羈相續以繫帝腰。時中宮僕伏德扶中宮,一手持十匹絹,乃取德絹連續為輦。行軍校尉尚弘多力,令弘居前負帝,乃得下登船。其餘不得渡者甚衆,復遣船收諸不得渡者,皆爭攀船,船上人以刃櫟斷其指,舟中之指可掬。奉、暹等遂以天子都安邑,御乘牛車。太尉楊彪、太僕韓融近臣從者十餘人。以暹為征東、才為征西、樂征北將軍,並與奉、承持政。遣融至弘農與傕、汜等連和,還所略宮人公卿百官,及乘輿車馬數乘。是時蝗蟲起,歲旱無穀,從官食棗菜。魏書曰:乘輿時居棘籬中,門戶無關閉。天子與羣臣會,兵士伏籬上觀,互相鎮壓以為笑。諸將專權,或擅笞殺尚書。司隷校尉出入,民兵抵擲之。諸將或遣婢詣省閤,或自齎酒啖,過天子飲,侍中不通,喧呼罵詈,遂不能止。又競表拜諸營壁民為部曲,求其禮遺。醫師、走卒皆為校尉,御史刻印不供,乃以錐畫,示有文字,或不時得也。諸將不能相率,上下亂,糧食盡。奉、暹、承乃以天子還洛陽。出箕關,下軹道,張楊以食迎道路,拜大司馬。語在楊傳。天子入洛陽,宮室燒盡,街陌荒蕪,百官披荊棘,依丘牆間。州郡各擁兵自為,莫有至者。饑窮稍甚,尚書郎以下自出樵采,或饑死牆壁閒。

太祖乃迎天子都許。暹、奉不能奉王法,各出奔,寇徐、揚閒,為劉備所殺。英雄記曰:備誘奉與相見,因於坐上執之。暹失奉勢,孤,時欲走還并州,為杼秋屯帥張宣所邀殺。董承從太祖歲餘,誅。建安二年,遣謁者僕射裴茂率關西諸將誅傕,夷三族。典略曰:傕頭至,有詔高縣。汜為其將五習所襲,死於郿。濟饑餓,至南陽寇略,為穰人所殺,從子繡攝其衆。才、樂留河東,才為怨家所殺,樂病死。遂、騰自還涼州,更相寇,後騰入為衞尉,子超領其部曲。十六年,超與關中諸將及遂等反,太祖征破之。語在武紀。遂奔金城,為其將所殺。超據漢陽,騰坐夷三族。趙衢等舉義兵討超,超走漢中從張魯,後奔劉備,死於蜀。


\end{pinyinscope}