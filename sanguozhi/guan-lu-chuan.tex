\article{管輅傳}

\begin{pinyinscope}
管輅字公明,平原人也。容皃粗醜,無威儀而嗜酒,飲食言戲,不擇非類,故人多愛之而不敬也。

輅別傳曰:輅年八九歲,便喜仰視星辰,得人輒問其名,夜不肯寐。父母常禁之,猶不可止。自言「我年雖小,然眼中喜視天文。」常云:「家雞野鵠,猶尚知時,況於人乎?」與鄰比兒共戲土壤中,輒畫地作天文及日月星辰。每荅言說事,語皆不常,宿學耆人不能折之,皆知其當有大異之才。及成人,果明周易,仰觀、風角、占、相之道,無不精微。體性寬大,多所含受;憎己不讎,愛己不襃,每欲以德報怨。常謂:「忠孝信義,人之根本,不可不厚;廉介細直,士之浮飾,不足為務也。」自言:「知我者稀,則我貴矣,安能斷江、漢之流,為激石之清?樂與季主論道,不欲與漁父同舟,此吾志也。」其事父母孝,篤兄弟,順愛士友,皆仁和發中,終無所闕。臧否之士,晚亦服焉。父為琅邪即丘長,時年十五,來至官舍讀書。始讀詩、論語及易本,便開淵布筆,辭義斐然。于時黌上有遠方及國內諸生四百餘人,皆服其才也。琅琊太守單子春雅有材度,聞輅一黌之儁,欲得見,輅父即遣輅造之。大會賔客百餘人,坐上有能言之士,輅問子春:「府君名士,加有雄貴之姿,輅旣年少,膽未堅剛,若欲相觀,懼失精神,請先飲三升清酒,然後而言之。」子春大喜,便酌三升清酒,獨使飲之。酒盡之後,問子春:「今欲與輅為對者,若府君四坐之士邪?」子春曰:「吾欲自與卿旗鼓相當。」輅言:「始讀詩、論、易本,學問微淺,未能上引聖人之道,陳秦、漢之事,但欲論金木水火土鬼神之情耳。」子春言;「此最難者,而卿以為易邪?」於是唱大論之端,遂經於陰陽,文采葩流,枝葉橫生,少引聖籍,多發天然。子春及衆士互共攻劫,論難鋒起,而輅人人荅對,言皆有餘。至日向暮,酒食不行。子春語衆人曰:「此年少盛有才器,聽其言論,正似司馬犬子游獵之賦,何其磊落雄壯,英神以茂,必能明天文地理變化之數,不徒有言也。」於是發聲徐州,號之神童。

父為利漕,利漕民郭恩兄弟三人,皆得躄疾,使輅筮其所由。輅曰:「卦中有君本墓,墓中有女鬼,非君伯母,當叔母也。昔饑荒之世,當有利其數升米者,排著井中,嘖嘖有聲,推一大石,下破其頭,孤魂寃痛,自訴於天。」於是恩涕泣服罪。輅別傳曰:利漕民郭恩,字義博,有才學,善周易、春秋,又能仰觀。輅就義博讀易,數十日中,意便開發,言難踰師。於此分蓍下卦,用思精妙,占黌上諸生疾病死亡貧富喪衰,初無差錯,莫不驚怪,謂之神人也。又從義博學仰觀,三十日中通夜不卧,語義博:「君但相語墟落處所耳,至於推運會,論災異,自當出吾天分。」學未一年,義博反從輅問易及天文事要。義博每聽輅語,未嘗不推机慷愾。自言「登聞君至論之時,忘我篤疾,明闇之不相逮,何其遠也」!義博設主人,獨請輅,具告辛苦,自說:「兄弟三人俱得躄疾,不知何故?試相為作卦,知其所由。若有咎殃者,天道赦人,當為吾祈福於神明,勿有所愛。兄弟俱行,此為更生。」輅便作卦,思之未詳。會日夕,因留宿,至中夜,語義博曰:「吾以此得之。」旣言其事,義博悲涕沾衣,曰:「皇漢之末,實有斯事。君不名主,諱也。我不得言,禮也。兄弟躄來三十餘載,脚如棘子,不可復治,但願不及子孫耳。」輅言火形不絕,水形無餘,不及後也。

廣平劉奉林婦病困,已買棺器。時正月也,使輅占,曰:「命在八月辛卯日日中之時。」林謂必不然,而婦漸差,至秋發動,一如輅言。輅別傳曰:鮑子春為列人令,有明思才理,與輅相見,曰:「聞君為劉奉林卜婦死亡日,何其詳妙,試為論其意義。」輅論爻象之旨,說變化之義,若規員矩方,無不合也。子春自言:「吾少好譚易,又喜分蓍,可謂盲者欲視白黑,聾者欲聽清濁,苦而無功也。聽君語後,自視體中,真為憒憒者也。」

輅往見安平太守王基,基令作卦,輅曰:「當有賤婦人生一男兒,墯地便走入竈中死。又牀上當有一大虵銜筆,小大共視,須臾去之也。又烏來入室中,與鷰共鬪,鷰死,烏去。有此三怪。」基大驚,問其吉凶。輅曰:「直客舍乆遠,魑魅魍魎為怪耳。兒生便走,非能自走,直宋無忌之妖將其入竈也。大虵銜筆,直老書佐耳。烏與鷰鬪,直老鈴下耳。今卦中見象而不見其凶,知非妖咎之徵,自無所憂也。」後卒無患。輅別傳曰:基與輅共論易,數日中,大以為喜樂,語輅言:「俱相聞善卜,定共清論。君一時異才,當上竹帛也。」輅為基出卦,知其無咎,因謂基曰:「昔高宗之鼎,非雉所鳴,殷之階庭,非木所生,而野鳥一鴝,武丁為高宗,桑穀暫生,太戊以興焉。知三事不為吉祥,願府君安身養德,從容光大,勿以知神姧汙累天真。」

時信都令家婦女驚恐,更互疾病,使輅筮之。輅曰:「君北堂西頭,有兩死男子,一男持矛,一男持弓箭,頭在壁內,脚在壁外。持矛者主刺頭,故頭重痛不得舉也。持弓箭者主射胷腹,故心中縣痛不得飲食也。晝則浮游,夜來病人,故使驚恐也。」於是掘徙骸骨,家中皆愈。輅別傳曰:王基即遣信都令遷掘其室中,入地八尺,果得二棺,一棺中有矛,一棺中有角弓及箭,箭乆遠,木皆消爛,但有鐵及角完耳。及徙骸骨,去城一十里埋之,無復疾病。基曰:「吾少好讀易,玩之以乆,不謂神明之數,其妙如此。」便從輅學易,推論天文。輅每開變化之象,演吉凶之兆,未嘗不纖微委曲,盡其精神。基曰:「始聞君言,如何可得,終以皆亂,此自天授,非人力也。」於是藏周易,絕思慮,不復學卜筮之事。輅鄉里乃太原問輅:「君往者為王府君論怪,云老書佐為虵,老鈴下為烏,此本皆人,何化之微賤乎?為見於爻象,出君意乎?」輅言:「苟非性與天道,何由背爻象而任胷心者乎?夫萬物之化,無有常形,人之變異,無有常體,或大為小,或小為大,固無優劣。夫萬物之化,一例之道也。是以夏鯀,天子之父,趙王如意,漢祖之子,而鯀為黃熊,如意為蒼狗,斯亦至尊之位而為黔喙之類也。況虵者協辰巳之位,烏者棲太陽之精,此乃騰黑之明象,白日之流景,如書佐、鈴下,各以微軀化為虵、烏,不亦過乎!」

清河王經去官還家,輅與相見。經曰:「近有一怪,大不喜之,欲煩作卦。」卦成,輅曰:「爻吉,不為怪也。君夜在堂戶前,有一流光如燕爵者,入君懷中,殷殷有聲,內神不安,解衣彷徉,招呼婦人,覔索餘光。」經大笑曰:「實如君言。」輅曰:「吉,遷官之徵也,其應行至。」頃之,經為江夏太守。輅別傳曰:經欲使輅卜,而有疑難之言,輅笑而荅之曰:「君侯州里達人,何言之鄙!昔司馬季主有言,夫卜者必法天地,象四時,順仁義。伏羲作八卦,周文王三百六十四爻,而天下治。病者或以愈,且死或以生,患或以免,事或以成,嫁女娶妻或以生長,豈直數千錢哉?以此推之,急務也。苟道之明,聖賢不讓,況吾小人,敢以為難!」彥緯斂手謝輅:「前言戲之耳。」於是輅為作卦,其言皆驗。經每論輅,以為得龍雲之精,能養和通幽者,非徒合會之才也。

輅又至郭恩家,有飛鳩來在梁頭,鳴甚悲。輅曰:「當有老公從東方來,攜豚一頭,酒一壺。主人雖喜,當有小故。」明日果有客,如所占。恩使客節酒、戒肉、慎火,而射鷄作食,箭從樹間激中數歲女子手,流血驚怖。輅別傅曰:義博從輅學鳥鳴之候,輅言君雖好道,天才旣少,又不解音律,恐難為師也。輅為說八風之變,五音之數,以律呂為衆鳥之商,六甲為時日之端,反覆譴曲,出入無窮。義博靜然沈思,馳精數日,卒無所得。義博言:「才不出位,難以追徵於此。」遂止。

輅至安德令劉長仁家,有鳴鵲來在閣屋上,其聲甚急。輅曰:「鵲言東北有婦昨殺夫,牽引西家人夫離婁,候不過日在虞淵之際,告者至矣。」到時,果有東北同伍民來告,鄰婦手殺其夫,詐言西家人與夫有嫌,來殺我壻。輅別傳曰:勃海劉長仁有辯才,初雖聞輅能曉鳥鳴,後每見難輅曰:「夫生民之音曰言,鳥獸之聲曰鳴,故言者則有知之貴靈,鳴者則無知之賤名,何由以鳥鳴為語,亂神明之所異也?孔子曰『吾不與鳥獸同群』,明其賤也。」輅荅曰:「夫天雖有大象而不能言,故運星精於上,流神明於下,驗風雲以表異,役鳥獸以通靈。表異者必有浮沉之候,通靈者必有宮商之應,是以宋襄失德,六鶂並退,伯姬將焚,鳥唱其災,四國未火,融風以發,赤鳥夾日,殃在荊楚。此乃上天之所使,自然之明符。考之律呂則音聲有本,求之人事則吉凶不失。昔在秦祖,以功受封,葛盧聽音,著在春秋,斯皆典謨之實,非聖賢之虛名也。商之將興,由一燕卵也。文王受命,丹鳥銜書,此乃聖人之靈祥,周室之休祚,何賤之有乎?夫鳴鳥之聽,精在鶉火,妙在八神,自非斯倫,猶子路之於死生也。」長仁言:「君辭雖茂,華而不實,未敢之信。」須臾有鳴鵲之驗,長仁乃服。

輅至列人典農王弘直許,有飄風高三尺餘,從申上來,在庭中幢幢回轉,息以復起,良乆乃止。直以問輅,輅曰:「東方當有馬吏至,恐父哭子,如何!」明日膠東吏到,直子果亡。直問其故,輅曰:「其日乙卯,則長子之候也。木落於申,斗建申,申破寅,死喪之候也。日加午而風發,則馬之候也。離為文章,則吏之候也。申未為虎,虎為大人,則父之候也。」有雄雉飛來,登直內鈴柱頭,直大以不安,令輅作卦,輅曰:「到五月必遷。」時三月也,至期,直果為勃海太守。輅別傳曰:輅又曰:「夫風以時動,爻以象應,時者神之駈使,象者時之形表,一時其道,不足為難。」王弘直亦大學問,有道術,皆不能精。問輅:「風之推變,乃可爾乎?」輅言:「此但風之毛髮,何足為異?若夫列宿不守,衆神亂行,八風橫起,怒氣電飛,山崩石飛,樹木摧傾,揚塵萬里,仰不見天,鳥獸藏竄,兆民駭驚,於是使梓慎之徒,登高臺,望風氣,分災異,刻期日,然後知神思遐幽,靈風可懼。」

館陶令諸葛原遷新興太守,輅往祖餞之,賔客並會。原自起取燕卵、蠭窠、䵹鼄著器中,使射覆。卦成,輅曰:「第一物,含氣須變,依乎宇堂,雄雌以形,翅翼舒張,此燕卵也。第二物,家室倒縣,門戶衆多,藏精育毒,得秋乃化,此蠭窠也。第三物,觳觫長足,吐絲成羅,尋網求食,利在昏夜,此䵹鼄也。」舉坐驚喜。輅別傳曰:諸葛原字景春,亦學士。好卜筮,數與輅共射覆,不能窮之。景春與輅有榮辱之分,因輅餞之,大有高談之客。諸人多聞其善卜、仰觀,不知其有大異之才,於是先與輅共論聖人著作之原,又叙五帝、三王受命之符。輅解景春微旨,遂開張戰地,示以不固,藏匿孤虛,以待來攻。景春奔北,軍師摧衂,自言吾覩卿旌旗,城池已壞也。其欲戰之士,於此鳴鼓角,舉雲梯,弓弩大起,牙旗雨集。然後登城曜威,開門受敵,上論五帝,如江如漢,下論三王,如翮如翰;其英者若春華之俱發,其攻者若秋風之落葉。聽者眩惑,不達其義,言者收聲,莫不心服,雖白起之坑趙卒,項羽之塞濉水,无以尚之。于時客皆欲靣縛銜璧,求束手於軍鼓之下。輅猶總干山立,未便許之。至明日,離別之際,然後有腹心始終。一時海內俊士,八九人矣。蔡元才在朋友中最有清才,在衆人中言:「本聞卿作狗,何意為龍?」輅言:「潛陽未變,非卿所知,焉有狗耳,得聞龍聲乎!」景春言:「今當遠別,後會何期?且復共一射覆。」輅占旣皆中。景春大笑:「卿為我論此卦意,紓我心懷」。輅為開爻散理,分賦形象,言徵辭合,妙不可述。景春及衆客莫不言聽後論之美,勝於射覆之樂。景春與輅別,戒以二事,言:「卿性樂酒,量雖溫克,然不可保,寧當節之。卿有水鏡之才,所見者妙,仰觀雖神,禍如膏火,不可不慎。持卿叡才,游於雲漢之間,不憂不富貴也。」輅言:「酒不可極,才不可盡,吾欲持酒以禮,持才以愚,何患之有也?」

輅族兄孝國,居在厈丘,輅往從之,與二客會。客去後,輅謂孝國曰:「此二人天庭及口耳之間同有凶氣,異變俱起,雙魂無宅,輅別傳曰:輅又曰:「厚味腊毒,夭精幽夕,坎為棺槨,兊為喪車。」流魂于海,骨歸于家,少許時當並死也。」復數十日,二人飲酒醉,夜共載車,牛驚下道入漳河中,皆即溺死也。

當此之時,輅之鄰里,外戶不閉,無相偷竊者。清河太守華表,召輅為文學掾。安平趙孔曜薦輅於兾州刺史裴徽曰:「輅雅性寬大,與世無忌,仰觀天文則同妙甘公、石申,俯覽周易則齊思季主。今明使君方垂神幽藪,留精九臯,輅宜蒙陰和之應,得及羽儀之時。」徽於是辟為文學從事,引與相見,大善友之。徙部鉅鹿,遷治中別駕。

初應州召,與弟季儒共載,至武城西,自卦吉凶,語儒云:「當在故城中見三貍,爾者乃顯。」前到河西故城角,正見三貍共踞城側,兄弟並喜。正始九年舉秀才。輅別傳曰:輅為華清河所召,為北黌文學,一時士友無不歎慕。安平趙孔曜,明敏有思識,與輅有管、鮑之分,故從發干來,就郡黌上與輅相見,言:「卿腹中汪汪,故時死人半,今生人無雙,當去俗騰飛,翱翔昊蒼,云何在此?聞卿消息,使吾食不甘味也。兾州裴使君才理清明,能釋玄虛,每論易及老、莊之道,未嘗不注精於嚴、瞿之徒也。又眷吾意重,能相明信者。今當故往,為卿陳感虎開石之誠。」輅言:「吾非四淵之龍,安能使白日晝陰?卿若能動東風,興朝雲,吾志所不讓也。」於是遂至兾州見裴使君。使君言:「君顏色何以消減於故邪?」孔曜言:「體中無藥石之疾,然見清河郡內有一騏驥,拘縶後廄歷年,去王良、伯樂百八十里,不得騁天骨,起風塵,以此憔悴耳。」使君言:「騏驥今何在也?」孔曜言:「平原管輅字公明,年三十六,雅性寬大,與世無忌,可謂士雄。仰觀天文則能同妙甘公、石申,俯覽周易則能思齊季主,游步道術,開神無窮,可謂士英。抱荊山之璞,懷夜光之寶,而為清河郡所錄北黌文學,可為痛心疾首也。使君方欲流精九臯,垂神幽藪,欲令明主不獨治,逸才不乆滯,高風遐被,莫不草靡,宜使輅特蒙陰和之應,得及羽儀之時,必能翼宣隆化,揚聲九圍也。」裴使君聞言,則忼慨曰:「何乃爾邪!雖在大州,未見異才可用釋人鬱悶者,思還京師,得共論道耳,況草間自有清妙之才乎?如此便相為取之,莫使騏驥更為凡馬,荊山反成凡石。」即檄召輅為文學從事。一相見,清論終日,不覺罷倦。天時大熱,移牀在庭前樹下,乃至雞向晨,然後出。再相見,便轉為鉅鹿從事。三見,轉治中。四見,轉為別駕。至十月,舉為秀才。輅辭裴使君,使君言:「丁、鄧二尚書,有經國才畧,於物理無不精也。何尚書神明精微,言皆巧妙,巧妙之志,殆破秋豪,君當慎之!自言不解易九事,必當以相問。比至洛,宜善精其理也。」輅言:「何若巧妙,以攻難之才,游形之表,未入於神。夫入神者,當步天元,推陰陽,探玄虛,極幽明,然後覽道無窮,未暇細言。若欲差次老、莊而參爻、象,愛微辯而興浮藻,可謂射侯之巧,非能破秋豪之妙也。若九事皆至義者,不足勞思也。若陰陽者,精之以久。輅去之後,歲朝當有時刑大風,風必摧破樹木。若發於乾者,必有天威,不足共清譚者。」

十二月二十八日,吏部尚書何晏請之,鄧颺在晏許。晏謂輅曰:「聞君著爻神妙,試為作一卦,知位當至三公不?」又問:「連夢見青蠅數十頭,來在鼻上,驅之不肯去,有何意故?」輅曰:「夫飛鴞,天下賤鳥,及其在林食椹,則懷我好音,況輅心非草木,敢不盡忠?昔元、凱之弼重華,宣慈惠和,周公之翼成王,坐而待旦,故能流光六合,萬國咸寧。此乃履道休應。非卜筮之所明也。今君侯位重山嶽,勢若雷電,而懷德者鮮,畏威者衆,殆非小心翼翼多福之仁。又鼻者艮,此天中之山,臣松之案:相書謂鼻之所在為天中。鼻有山象,故曰:「天中之山」也。高而不危,所以長守貴。今青蠅臭惡,而集之焉。位峻者顛,輕豪者亡,不可不思害盈之數,盛衰之期。是故山在地中曰謙,雷在天上曰壯;謙則襃多益寡,壯則非禮不履。未有損己而不光大,行非而不傷敗。願君侯上追文王六爻之旨,下思尼父彖象之義,然後三公可決,青蠅可驅也。」颺曰:「此老生之常譚。」輅荅曰:「夫老生者見不生,常譚者見不譚。」晏曰:「過歲更當相見。」輅別傳曰:輅為何晏所請,果共論易九事,九事皆明。晏曰:「君論陰陽,此世無雙。」時鄧颺與晏共坐,颺言:「君見謂善易,而語初不及易中辭義,何故也?」輅尋聲荅之曰:「夫善易者不論易也。」晏含笑而讚之「可謂要言不煩也」。因請輅為卦。輅旣稱引鑒誡,晏謝之曰:「知機其神乎,古人以為難;交疏而吐其誠,今人以為難。君今一靣而盡二難之道,可謂明德惟馨。詩不云乎,『中心藏之,何日忘之』!」輅還邑舍,具以此言語舅氏,舅氏責輅言太切至。輅曰;「與死人語,何所畏邪?」舅大怒,謂輅狂悖。歲朝,西北大風,塵埃蔽天,十餘日,聞晏、颺皆誅,然後舅氏乃服。輅別傳曰:舅夏大夫問輅:「前見何、鄧之日,為已有凶氣未也?」輅言:「與禍人共會,然後知神明交錯;與吉人相近,又知聖賢求精之妙。夫鄧之行步,則筋不束骨,脉不制肉,起立傾倚,若無手足,謂之鬼躁。何之視候,則魂不守宅,血不華色,精爽煙浮,容若槁木,謂之鬼幽。故鬼躁者為風所收,鬼幽者為火所燒,自然之符,不可以蔽也。」輅後因得休,裴使君問:「何平叔一代才名,其實何如?」輅曰:「其才若盆盎之水,所見者清,所不見者濁。神在廣博,志不務學,弗能成才。欲以盆盎之水,求一山之形,形不可得,則智由此惑。故說老、莊則巧而多華,說易生義則美而多偽;華則道浮,偽則神虛;得上才則淺而流絕,得中才則游精而獨出,輅以為少功之才也。」裴使君曰:「誠如來論。吾數與平叔共說老、莊及易,常覺其辭妙於理,不能折之。又時人吸習,皆歸服之焉,益令不了。相見得清言,然後灼灼耳。」

始輅過魏郡太守鍾毓,共論易義,輅因言「卜可知君生死之日。」毓使筮其生日月,如言無蹉跌。毓大愕然,曰:「君可畏也。死以付天,不以付君。」遂不復筮。毓問輅:「天下當太平否?」輅曰:「方今四九天飛,利見大人,神武升建,王道文明,何憂不平?」毓未解輅言,無幾,曹爽等誅,乃覺寤云。輅別傳曰:魏郡太守鍾毓,清逸有才,難輅易二十餘事,自以為難之至精也。輅尋聲投響,言無留滯,分張爻象,義皆殊妙。毓即謝輅。輅卜知毓生日月,毓愕然曰:「聖人運神通化,連屬事物,何聦明乃爾!」輅言:「幽明同化,死生一道,悠悠太極,終而復始。文王損命,不以為憂,仲尼曳杖,不以為懼,緒煩蓍筮,宜盡其意。」毓曰:「生者好事,死者惡事,哀樂之分,吾所不能齊,且以付天,不以付君也。」石苞為鄴典農,與輅相見,問曰:「聞君鄉里翟文耀能隱形,其事可信乎?」輅言:「此但陰陽蔽匿之數,苟得其數,則四嶽可藏,河海可逃。況以七尺之形,游變化之內,散雲霧以幽身,布金水以滅迹,術足數成,不足為難。」苞曰:「欲聞其妙,君且善論其數也。」輅言:「夫物不精不為神,數不妙不為術,故精者神之所合,妙者智之所遇,合之幾微,可以性通,難以言論。是故魯班不能說其手,離朱不能說其目。非言之難,孔子曰『書不盡言』,言之細也,『言不盡意』,意之微也,斯皆神妙之謂也。請舉其大體以驗之。夫白日登天,運景萬里,無物不照,及其入地,一炭之光,不可得見。三五盈月,清耀燭夜,可以遠望,及其在晝,明不如鏡。今逃日月者必陰陽之數,陰陽之數通於萬類,鳥獸猶化,況於人乎!夫得數者妙,得神者靈,非徒生者有驗,死亦有徵。是以杜伯乘火氣以流精,彭生託水變以立形。是故生者能出亦能入,死者能顯亦能幽,此物之精氣,化之游魂,人鬼相感,數使之然也。」苞曰:「目見陰陽之理,不過於君,君何以不隱?」輅曰:「夫陵虛之鳥,愛其清高,不願江、漢之魚;淵沼之魚,樂其濡溼,不易騰風之鳥:由性異而分不同也。僕自欲正身以明道,直己以親義,見數不以為異,知術不以為奇,夙夜研機,孳孳溫故,而素隱行怪,未暇斯務也。」

平原太守劉邠取印囊及山鷄毛著器中,使筮。輅曰:「內方外圓,五色成文,含寶守信,出則有章,此印囊也。高岳巖巖,有鳥朱身,羽翼玄黃,鳴不失晨,此山鷄毛也。」邠曰:「此郡官舍,連有變怪,使人恐怖,其理何由?」輅曰:「或因漢末之亂,兵馬擾攘,軍尸流血,汙染丘山,故因昏夕,多有怪形也。明府道德高妙,自天祐之,願安百祿,以光休寵。」輅別傳曰:故郡將劉邠字令元,清和有思理,好易而不能精。與輅相見,意甚喜歡,自說注易向訖也。輅言:「今明府欲勞不世之神,經緯大道,誠富美之秋。然輅以為注易之急,急於水火;水火之難,登時之驗,易之清濁,延于萬代,不可不先定其神而後垂明思也。自旦至今,聽採聖論,未有易之一分,易安可注也!輅不解古之聖人,何以處乾位於西北,坤位於西南。夫乾坤者天地之象,然天地至大,為神明君父,覆載萬物,生長無首,何以安處二位與六卦同列?乾之象彖曰:『大哉乾元,萬物資始,乃統天。』夫統者,屬也,尊莫大焉,何由有別位也?」邠依易繫辭,諸為之理以為注,不得其要。輅尋聲下難,事皆窮析。曰:「夫乾坤者,易之祖宗,變化之根源,今明府論清濁者有疑,疑則無神,恐非注易之符也。」輅於此為論八卦,八卦之道及爻象之精,大論開廓,衆化相連。邠所解者,皆以為妙,所不解者,皆以為神。自說:「欲注易八年,用思勤苦,歷載靡寧,定相得至論,此才不及易,不愛乆勞,喜承雅言,如此相為高枕偃息矣。」欲從輅學射覆,輅言:「今明府以虛神於注易,亦宜絕思於靈蓍。靈蓍者,二儀之明數,陰陽之幽契,施之於道則定天下吉凶,用之於術則收天下豪纖。纖微,未可以為易也。」邠曰:「以為術者易之近數,欲求其端耳。若如來論,何事於斯?」留輅五日,不遑恤官,但共清譚。邠自言:「數與何平叔論易及老、莊之道,至於精神遐流,與化周旋,清若金水,鬱若山林,非君侶也。」邠又曰:「此郡官舍,連有變怪,變怪多形,使人怖恐,君似當達此數者,其理何由也。」輅言:「此郡所以名平原者,本有原,山無木石,與地自然;含陰不能吐雲,含陽不能激風,陰陽雖弱,猶有微神;微神不真,多聚凶姧,以類相求,魍魎成群。或因漢末兵馬擾攘,軍尸流血,汙染丘岳,彊魂相感,變化無常,故因昏夕之時,多有怪形也。昔夏禹文明,不怪於黃龍,周武信時,不惑於暴風,今明府道德高妙,神不懼妖,自天祐之,吉無不利,願安百祿以光休寵也。」邠曰:「聽雅論為近其理,每有變怪,輒聞鼓角聲音,或見弓劒形象。夫以土山之精,伯有之魂,實能合會,干犯明靈也。」邠問輅:「易言剛健篤實,輝光日新,斯為同不也?」輅曰:「不同之名,朝旦為輝,日中為光。」晉諸公贊曰:邠本名炎,犯於太子諱,改為邠。位至太子僕。子粹,字純嘏,侍中。次宏,字終嘏,太常。次漢,字仲嘏,光祿大夫。漢清沖有貴識,名亞樂廣。宏子咸,徐州刺吏。次耽,晉陵內史。耽子恢,字真長,尹丹楊,為中興名士也。

清河令徐季龍使人行獵,令輅筮其所得。輅曰:「當獲小獸,復非食禽,雖有爪牙,微而不彊,雖有文章,蔚而不明,非虎非雉,其名曰狸。」獵人暮歸,果如輅言。季龍取十三種物,著大篋中,使輅射。云:「器中藉藉有十三種物。」先說鷄子,後道蠶蛹,遂一一名之,惟以梳為枇耳。輅別傳曰:清河令徐季龍,字開明,有才機。與輅相見,共論龍動則景雲起,虎嘯則谷風至,以為火星者龍,參星者虎,火出則雲應,參出則風到,此乃陰陽之感化,非龍虎之所致也。輅言:「夫論難當先審其本,然後求其理,理失則機謬,機謬則榮辱之主。若以參星為虎,則谷風更為寒霜之風,寒霜之風非東風之名。是以龍者陽精,以潛為陰,幽靈上通,和氣感神,二物相扶,故能興雲。夫虎者,陰精而居于陽,依木長嘯,動於巽林,二氣相感,故能運風。若磁石之取鐵,不見其神而金自來,有徵應以相感也。況龍有潛飛之化,虎有文明之變,招雲召風,何足為疑?」季龍言:「夫龍之在淵,不過一井之底,虎之悲嘯,不過百步之中,形氣淺弱,所通者近,何能剽景雲而馳東風?」輅言:「君不見陰陽燧在掌握之中,形不出手,乃上引太陽之火,下引太陰之水,噓吸之間,煙景以集。苟精氣相感,縣象應乎二燧;苟不相感,則二女同居,志不相得。自然之道,無有遠近。」季龍言:「世有軍事,則感雞雉先鳴,其道何由?復有他占,惟在雞雉而巳?」輅言:「貴人有事,其應在天,在天則日月星辰也。兵動民憂,其應在物,在物則山林鳥獸也。夫雞者兊之畜,金者兵之精,雉者離之鳥,獸者武之神,故太白揚輝則鷄鳴,熒惑流行則雉驚,各感數而動。又兵之神道,布在六甲,六甲推移,其占無常。是以晉柩牛呴,果有西軍,鴻嘉石鼓,鳴則有兵,不專近在於雞雉也。」季龍言:「魯昭公八年,有石言於晉,師曠以為作事不時,怨讟動於民,則有非言之物而言,於理為合不?」輅言:「晉平奢泰,崇飾宮室,斬伐林木,殘破金石,民力旣盡,怨及山澤,神痛人感,二精並作,金石同氣,則兊為口舌,口舌之妖,動于靈石。傳曰輕百姓,飾城郭,則金不從革,此之謂也。」季龍欽嘉,留輅經數日。輅占獵旣驗,季龍曰:「君雖神妙,但不多藏物耳,何能皆得之?」輅言:「吾與天地參神,蓍龜通靈,抱日月而遊杳冥,極變化而覽未然,況茲近物,能蔽聦明?」季龍大笑,「君旣不謙,又念窮在近矣。」輅言:「君尚未識謙言,焉能論道?夫天地者則乾坤之卦,蓍龜者則卜筮之數,日月者離坎之象,變化者陰陽之爻,杳冥者神化之源,未然者則幽冥之先,此皆周易之紀綱,何僕之不謙?」季龍於是取十三種物,欲以窮之,輅射之皆中。季龍乃歎曰:「作者之謂聖,述者之謂明,豈此之謂乎!」

輅隨軍西行,過毌丘儉墓下,倚樹哀吟,精神不樂。人問其故,輅曰:「林木雖茂,無形可乆;碑誄雖美,無後可守。玄武藏頭,蒼龍無足,白虎銜尸,朱雀悲哭,四危以備,法當滅族。不過二載,其應至矣。」卒如其言。後得休,過清河倪太守。時天旱,倪問輅雨期,輅曰:「今夕當雨。」是日暘燥,晝無形似,府丞及令在坐,咸謂不然。到鼓一中,星月皆沒,風雲並起,竟成快雨。於是倪盛脩主人禮,共為懽樂。輅別傳曰:輅與倪清河相見,旣刻雨期,倪猶未信。輅曰:「夫造化之所以為神,不疾而速,不行而至。十六日壬子,直滿,畢星中已有水氣,水氣之發,動於卯辰,此必至之應也。又天昨檄召五星,宣布星符,刺下東井,告命南箕,使召雷公、電父、風伯、雨師,羣岳吐陰,衆川激精,雲漢垂澤,蛟龍含靈,㷸㷸朱電,吐咀杳冥,殷殷雷聲,噓吸雨靈,習習谷風,六合皆同,欬唾之間,品物流形。天有常期,道有自然,不足為難也。」倪曰:「譚高信寡,相為憂之。」於是便留輅,往請府丞及清河令。若夜雨者當為啖二百斤犢肉,若不雨當住十日。輅曰:「言念費損!」至日向暮,了無雲氣,衆人並嗤輅。輅言:「樹上已有少女微風,樹間又有陰鳥和鳴。又少男風起,衆鳥和翔,其應至矣。」須臾,果有艮風鳴鳥。日未入,東南有山雲樓起。黃昏之後,雷聲動天。到鼓一中,星月皆沒,風雲並興,玄氣四合,大雨河傾。倪調輅言:「誤中耳,不為神也。」輅曰:「誤中與天期,不亦工乎!」

正元二年,弟辰謂輅曰:「大將軍待君意厚,兾當富貴乎?」輅長歎曰:「吾自知有分直耳,然天與我才明,不與我年壽,恐四十七八間,不見女嫁兒娶婦也。若得免此,欲作洛陽令,可使路不拾遣,枹鼓不鳴。但恐至太山治鬼,不得治生人,如何!」辰問其故,輅曰:「吾額上無生骨,眼中無守精,鼻無梁柱,脚無天根,背無三甲,腹無三壬,此皆不壽之驗。又吾本命在寅,加月食夜生。天有常數,不可得諱,但人不知耳。吾前後相當死者過百人,略無錯也。」是歲八月,為少府丞。明年二月卒,年四十八。輅別傳曰:旣有明才,遭朱陽之運,于時名勢赫弈,若火猛風疾。當塗之士,莫不枝附葉連。賔客如雲,無多少皆為設食。賔無貴賤,候之以禮。京城紛紛,非徒歸其名勢而已,然亦懷其德焉。向不夭命,輅之榮華,非世所測也。弟辰嘗欲從輅學卜及仰觀事,輅言:「卿不可教耳。夫卜非至精不能見其數,非至妙不能覩其道,孝經、詩、論,足為三公,無用知之也。」於是遂止。子弟無能傳其術者。辰叙曰:「夫晉、魏之士,見輅道術神妙,占候無錯,以為有隱書及象甲之數。辰每觀輅書傳,惟有易林、風角及鳥鳴、仰觀星書三十餘卷,世所共有。然輅獨在少府官舍,無家人子弟隨之,其亡沒之際,好奇不哀喪者,盜輅書,惟餘易林、風角及鳥鳴書還耳。夫術數有百數十家,其書有數千卷,書不少也。然而世鮮名人,皆由無才,不由無書也。裴兾州、何、鄧二尚書及鄉里劉太常、潁川兄弟,以輅稟受天才,明陰陽之道,吉凶之情,一得其源,遂涉其流,亦不為難,常歸服之。輅自言與此五君共語使人精神清發,昏不暇寐。自此以下,殆白日欲寢矣。又自言當世無所願,欲得與魯梓慎、鄭裨竈、晉卜偃、宋子韋、楚甘公、魏石申共登靈臺,披神圖,步三光,明災異,運蓍龜,決狐疑,無所復恨也。辰不以闇淺,得因孔懷之親,數與輅有所諮論。至於辨人物,析臧否,說近義,彈曲直,拙而不工也。若敷皇、羲之典,揚文、孔之辭,周流五曜,經緯三度,口滿聲溢,微言風集,若仰眺飛鴻,漂漂兮景沒,若俯臨深溪,杳杳兮精絕;偪以攻難,而失其端,欲受學求道,尋以迷昏,無不扼腕椎指,追音響長歎也。昔京房雖善卜及風律之占,卒不免禍,而輅自知四十八當亡,可謂明哲相殊。又京房目見遘讒之黨,耳聽青蠅之聲,靣諫不從,而猶道路紛紜。輅處魏、晉之際,藏智以朴,卷舒有時,妙不見求,愚不見遺,可謂知幾相邈也。京房上不量萬乘之主,下不避佞諂之徒,欲以天文、洪範,利國利身,困不能用,卒陷大刑,可謂枯龜之餘智,膏燭之末景,豈不哀哉!世人多以輅疇之京房,辰不敢許也。至於仰察星辰,俯定吉凶,遠期不失年歲,近期不失日月,辰以甘、石之妙不先也。射覆名物,見術流速,東方朔不過也。觀骨形而審貴賤,覽形色而知生死,許負、唐舉不超也。若夫疏風氣而探徵候,聽鳥鳴而識神機,亦一代之奇也。向使輅官達,為宰相大臣,膏腴流於明世,華曜列乎竹帛,使幽驗皆舉,祕言不遺,千載之後,有道者必信而貴之,無道者必疑而怪之;信者以妙過真,夫妙與神合者,得神無所惑也。恨輅才長命短,道貴時賤,親賢遐潛,不宣於良史,而為鄙弟所見追述,旣自闇濁,又從來乆遠,所載卜占事,雖不識本卦,捃拾殘餘,十得二焉。至於仰觀靈曜,說魏、晉興衰,及五運浮沉,兵革災異,十不收一。無源何以成河?無根何以垂榮?雖秋菊可採,不及春英,臨文忼慨,伏用哀慙。將來君子,幸以高明求其義焉。往孟荊州為列人典農,常問亡兄,昔東方朔射覆得何卦,正知守宮、蜥蜴二物者。亡兄於此為安卦生象,辭喻交錯,微義豪起,變化相推,會於辰巳,分別龍虵,各使有理。言絕之後,孟荊州長歎息曰:『吾聞君論,精神騰躍,殆欲飛散,何其汪汪乃至於斯邪!』」臣松之案:辰所稱鄉里劉太常者,謂劉寔也。辰撰輅傳,寔時為太常,潁川則寔弟智也。寔、智並以儒學為名,無能言之。世語稱寔博辯,猶不足以並裴、何之流也。又案輅自說,云「本命在寅」,則建安十五年生也。至正始九年,應三十九,而傳云三十六,以正元三年卒,應四十七,傳云四十八,皆為不相應也。近有閻續伯者,名纘,該微通物,有良史風。為天下補綴遺脫,敢以所聞列于篇左。皆從受之於大人先哲,足以取信者,兾免虛誣之譏云爾。嘗受辰傳所謂劉太常者曰:「輅始見聞,由於為鄰婦卜亡牛,云當在西靣窮牆中,縣頭上向。教婦人令視諸丘冢中,果得牛。婦人因以為藏己牛,告官案驗,乃知以術知,故裴兾州遂聞焉。」又云:「路中小人失妻者,輅為卜,教使明旦於東陽城門中伺擔豚人牽與共鬬。具如其言,豚逸走,即共追之。豚入人舍,突破主人甕,婦從甕中出。」劉侯云甚多此類,辰所載纔十一二耳。劉侯云:「辰,孝廉才也。」中書令史紀玄龍,輅鄉里人,云:「輅在田舍,嘗候遠鄰,主人患數失火。輅卜,教使明日於南陌上伺,當有一角巾諸生,駕黑牛故車,必引留,為設賔主,此能消之。即從輅戒。諸生有急求去,不聽,遂留當宿,意大不安,以為圖己。主人罷入,生乃把刀出門,倚兩薪積閒,側立假寐。欻有一小物直來過前,如獸,手中持火,以口吹之。生驚,舉刀斫,正斷要,視之則狐。自此主人不復有災。」前長廣太守陳承祐口受城門校尉華長駿語云:「昔其父為清河太守時,召輅作吏,駿與少小,後以鄉里,遂加恩意,常與同載周旋,具知其事。云諸要驗,三倍於傳。辰旣短才,又年縣小,又多在田舍,故益不詳。辰仕宦至州主簿、部從事,太康之初物故。」駿又云:「輅卜亦不悉中,十得七八,駿問其故,輅云:『理無差錯,來卜者或言不足以宣事寔,故使爾。』華城門夫人者,魏故司空涿郡盧公女也,得疾,連年不差。華家時居西城下南纏里中,三廄在其東南。輅卜當有師從東方來,自言能治,便聽使之,必得其力。後無何,有南征廄騶,當充甲卒,來詣盧公,占能治女郎。公即表請留之,專使其子將詣華氏療疾,初用散藥,後復用丸治,尋有效,即奏除騶名,以補太醫。」又云:「隨輅父在利漕時,有治下屯民捕鹿者,其晨行還,見毛血,人取鹿處來詣廄告輅,輅為卦語云:『此有盜者,是汝東巷中第三家也。汝徑往門前,伺無人時,取一瓦子,密發其碓屋東頭第七椽,以瓦著下,不過明日食時,自送還汝。』其夜,盜者父病頭痛,壯熱煩疼,然亦來詣輅卜。輅為發祟,盜者具服。輅令擔皮肉藏還者故處,病當自愈。乃密教鹿主往取。又語使復往如前,舉椽棄瓦。盜父亦差。又都尉治內史有失物者,輅使明晨於寺門外看,當逢一人,使指天畫地,舉手四向,自當得之。暮果獲於故處矣。」

評曰:華佗之醫診,杜夔之聲樂,朱建平之相術,周宣之相夢,管輅之術筮,誠皆玄妙之殊巧,非常之絕技矣。昔史遷著扁鵲、倉公、日者之傳,所以廣異聞而表奇事也。故存錄云爾。


\end{pinyinscope}