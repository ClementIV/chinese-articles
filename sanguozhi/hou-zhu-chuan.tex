\article{後主傳}

\begin{pinyinscope}
後主諱禪,字公嗣,先主子也。建安二十四年,先主為漢中王,立為王太子。及即尊號,冊曰:「惟章武元年五月辛巳,皇帝若曰:太子禪,朕遭漢運艱難,賊臣篡盜,社稷無主,格人羣正,以天明命,朕繼大統。今以禪為皇太子,以承宗廟,祗肅社稷。使使持節丞相亮授印緩,敬聽師傅,行一物而三善皆得焉,可不勉與!」

禮記曰:行一物而三善者,惟世子而已,其齒於學之謂也。鄭玄曰:物猶事也。

三年夏四月,先主殂于永安宮。五月,後主襲位於成都,時年十七。尊皇后曰皇太后。大赦,改元。是歲魏黃初四年也。魏略曰:初備在小沛,不意曹公卒至,遑遽棄家屬,後奔荊州。禪時年數歲,竄匿,隨人西入漢中,為人所賣。及建安十六年,關中破亂,扶風人劉括避亂入漢中,買得禪,問知其良家子,遂養為子,與娶婦,生一子。初禪與備相失時,識其父字玄德。比舍人有姓簡者,及備得益州而簡為將軍,備遣簡到漢中,舍都邸。禪乃詣簡,簡相檢訊,事皆符驗。簡喜,以語張魯,魯為洗沐送詣益州,備乃立以為太子。初備以諸葛亮為太子太傅,及禪立,以亮為丞相,委以諸事,謂亮曰:「政由葛氏,祭則寡人。」亮亦以禪未閑於政,遂總內外。臣松之案:二主妃子傳曰「後主生於荊州」,後主傳云「初即帝位,年十七」,則建安十二年生也。十三年敗於長阪,備棄妻子走,趙雲傳曰「雲身抱弱子以免」,即後主也。如此,備與禪未甞相失也。又諸葛亮以禪立之明年領益州牧,其年與主簿杜微書曰「朝廷今年十八」,與禪傳相應,理當非虛。而魚豢云備敗於小沛,禪時年始生,及奔荊州,能識其父字玄德,計當五六歲。備敗於小沛時,建安五年也,至禪初立,首尾二十四年,禪應過二十矣。以事相驗,理不得然。此則魏略之妄說,乃至二百餘言,異也!又案諸書記及諸葛亮集,亮亦不為太子太傅。

建興元年夏,牂牁太守朱襃擁郡反。魏氏春秋曰:初,益州從事常房行部,聞襃將有異志,收其主簿案問,殺之。襃怒,攻殺房,誣以謀反。諸葛亮誅房諸子,徙其四弟於越嶲,欲以安之。襃猶不悛改,遂以郡叛應雍闓。臣松之案:以為房為襃所誣,執政所宜澄察,安有妄殺不辜以恱姦慝?斯殆妄矣!先是,益州郡有大姓雍闓反,流太守張裔於吳,據郡不賔,越嶲夷王高定亦背叛。是歲,主皇后張氏。遣尚書郎鄧芝固好於吳,吳王孫權與蜀和親使聘,是歲通好。

二年春,務農殖穀,閉關息民。

三年春三月,丞相亮南征四郡,四郡皆平。改益州郡為建寧郡,分建寧、永昌郡為雲南郡,又分建寧、牂牁為興古郡。十二月,亮還成都。

四年春,都護李嚴自永安還住江州,築大城。今巴郡故城是。

五年春,丞相亮出屯漢中,營沔北陽平石馬。諸葛亮集載禪三月下詔曰:「朕聞天地之道,福仁而禍淫;善積者昌,惡積者喪,古今常數也。是以湯、武脩德而王,桀、紂極暴而亡。曩者漢祚中微,網漏凶慝,董卓造難,震蕩京畿。曹操階禍,竊執天衡,殘剥海內,懷無君之心。子丕孤豎,敢尋亂階,盜據神器,更姓改物,世濟其凶。當此之時,皇極幽昧,天下無主,則我帝命隕越于下。昭烈皇帝體明叡之德,光演文武,應乾坤之運,出身平難,經營四方,人鬼同謀,百姓與能。兆民欣戴。奉順符讖,建位易號,丕承天序,補弊興衰,存復祖業,膺誕皇綱,不墜于地。萬國未定,早世遐殂。朕以幼冲,繼統鴻基,未習保傅之訓,而嬰祖宗之重。六合壅否,社稷不建,永惟所以,念在匡救,光載前緒,未有攸濟,朕甚懼焉。是以夙興夜寐,不敢自逸,每從菲薄以益國用,勸分務穡以阜民財,授方任能以參其聽,斷私降意以養將士。欲奮劒長驅,指討凶逆,朱旗未舉,而丕復隕喪,斯所謂不燃我薪而自焚也。殘類餘醜,又支天禍,恣睢河、洛,阻兵未弭。諸葛丞相弘毅忠壯,忘身憂國,先帝託以天下,以勗朕躬。今授之以旄鉞之重,付之以專命之權,統領步騎二十萬衆,董督元戎,龔行天罰,除患寧亂,克復舊都,在此行也。昔項籍總一彊衆,跨州兼土,所務者大,然卒敗垓下,死於東城,宗族焚如,為笑千載,皆不以義,陵上虐下故也。今賊效尤,天人所怨,奉時宜速,庶憑炎精祖宗威靈相助之福,所向必克。吳王孫權同恤灾患,潛軍合謀,掎角其後。涼州諸國王各遣月支、康居胡侯支富、康植等二十餘人詣受節度,大軍北出,便欲率將兵馬,奮戈先驅。天命旣集,人事又至,師貞勢并,必無敵矣。夫王者之兵,有征無戰,尊而且義,莫敢抗也,故鳴條之役,軍不血刃,牧野之師,商人倒戈。今旍麾首路,其所經至,亦不欲窮兵極武。有能棄邪從正,簞食壺漿以迎王師者,國有常典,封寵大小,各有品限。及魏之宗族、支葉、中外,有能規利害、審逆順之數,來詣降者,皆原除之。昔輔果絕親於智氏,而蒙全宗之福,微子去殷,項伯歸漢,皆受茅土之慶。此前世之明驗也。若其迷沈不反,將助亂人,不式王命,戮及妻孥,罔有攸赦。廣宣恩威,貸其元帥,弔其殘民。他如詔書律令,丞相其露布天下,使稱朕意焉。」

六年春,亮出攻祁山,不克。冬,復出散關,圍陳倉,糧盡退。魏將王雙率軍追亮,亮與戰,破之,斬雙,還漢中。

七年春,亮遣陳式攻武都、陰平,遂克定二郡。冬,亮徙府營於南山下原上,築漢、樂二城。是歲,孫權稱帝,與蜀約盟,共交分天下。

八年秋,魏使司馬懿由西城,張郃由子午,曹真由斜谷,斜,余奢反。欲攻漢中。丞相亮待之於城固、赤阪,大雨道絕,真等皆還。是歲,魏延破魏雍州刺史郭淮于陽谿。徙魯王永為甘陵王。梁王理為安平王,皆以魯、梁在吳分界故也。

九年春二月,亮復出軍圍祁山,始以木牛運。魏司馬懿、張郃救祁山。夏六月,亮糧盡退軍,郃追至青封,與亮交戰,被箭死。秋八月,都護李平廢徙梓潼郡。漢晉春秋曰:冬十月,江陽至江州有鳥從江南飛渡江北,不能達,墮水死者以千數。

十年,亮休士勸農於黃沙,作流馬木牛畢,教兵講武。

十一年冬,亮使諸軍運米,集於斜谷口,治斜谷邸閣。是歲,南夷劉冑反,將軍馬忠破平之。

十二年春二月,亮由斜谷出,始以流馬運。秋八月,亮卒于渭濵。征西大將軍魏延與丞相長史楊儀爭權不和,舉兵相攻,延敗走;斬延首,儀率諸軍還成都。大赦。以左將軍吳壹為車騎將軍,假節督漢中。以丞相留府長史蔣琬為尚書令,總統國事。

十三年春正月,中軍師楊儀廢徙漢嘉郡。夏四月,進蔣琬位為大將軍。

十四年夏四月,後主至湔,臣松之案:湔,縣名也,屬蜀郡,音翦。登觀阪,看汶水之流,旬日還成都。徙武都氐王苻健及氐民四百餘戶於廣都。

十五年夏六月,皇后張氏薨。

延熈元年春正月,立皇后張氏。大赦,改元。立子璿為太子,子瑤為安定王。冬十一月,大將軍蔣琬出屯漢中。

二年春三月,進蔣琬位為大司馬。

三年春,使越嶲太守張嶷平定越嶲郡。

四年冬十月,尚書令費禕至漢中,與蔣琬諮論事計,歲盡還。

五年春正月,監軍姜維督偏軍,自漢中還屯涪縣。

六年冬十月,大司馬蔣琬自漢中還,住涪。十一月,大赦。以尚書令費禕為大將軍。

七年閏月,魏大將軍曹爽、夏侯玄等向漢中,鎮北大將軍王平拒興勢圍,大將軍費禕督諸軍往赴救,魏軍退。夏四月,安平王理卒。秋九月,禕還成都。

八年秋八月,皇太后薨。十二月,大將軍費禕至漢中,行圍守。

九年夏六月,費禕還成都。秋,大赦。冬十一月,大司馬蔣琬卒。魏略曰:琬卒,禪乃自攝國事。

十年,涼州胡王白虎文、治無戴等率衆降,衞將軍姜維迎逆安撫,居之于繁縣。是歲,汶山平康夷反,維往討,破平之。

十一年夏五月,大將軍費禕出屯漢中。秋,涪陵屬國民夷反,車騎將軍鄧芝往討,皆破平之。

十二年春正月,魏誅大將軍曹爽等,右將軍夏侯霸來降。夏四月,大赦。秋,衞將軍姜維出攻雍州,不克而還。將軍句安、李韶降魏。

十三年,姜維復出西平,不克而還。

十四年夏,大將軍費禕還成都。冬,復北駐漢壽。大赦。

十五年,吳王孫權薨。立子琮為西河王。

十六年春正月,大將軍費禕為魏降人郭循所殺于漢壽。夏四月,衞將軍姜維復率衆圍南安,不克而還。

十七年春正月,姜維還成都。大赦。夏六月,維復率衆出隴西。冬,拔狄道、河間、臨洮三縣民,居于緜竹、繁縣。

十八年春,姜維還成都。夏,復率諸軍出狄道,與魏雍州刺史王經戰于洮西,大破之。經退保狄道城,維却住鍾題。

十九年春,進姜維位為大將軍,督戎馬,與鎮西將軍胡濟期會上邽,濟失誓不至。秋八月,維為魏大將軍鄧艾所破于上邽。維退軍還成都。是歲,立子瓚為新平王。大赦。

二十年,聞魏大將軍諸葛誕據壽春以叛,姜維復率衆出駱谷,至芒水。是歲大赦。

景耀元年,姜維還成都。史官言景星見,於是大赦,改年。宦人黃皓始專政。吳大將軍孫綝廢其主亮,立琅邪王休。

二年夏六月,立子諶為北地王,恂為新興王,虔為上黨王。

三年秋九月,追謚故將軍關羽、張飛、馬超、龐統、黃忠。

四年春三月,追謚故將軍趙雲。冬十月,大赦。

五年春正月,西河王琮卒。是歲,姜維復率衆出侯和,為鄧艾所破,還住沓中。

六年夏,魏大興徒衆,命征西將軍鄧艾、鎮西將軍鍾會、雍州刺史諸葛緒數道並攻。於是遣左右車騎將軍張翼、廖化、輔國大將軍董厥等拒之。大赦。改元為炎興。冬,鄧艾破衞將軍諸葛瞻於緜竹。用光祿大夫譙周策,降於艾,奉書曰:「限分江、漢,遇值深遠,階緣蜀土,斗絕一隅,干運犯冒,漸苒歷載,遂與京畿攸隔萬里。每惟黃初中,文皇帝命虎牙將軍鮮于輔,宣溫密之詔,申三好之恩,開示門戶,大義炳然,而否德暗弱,竊貪遺緒,俛仰累紀,未率大教。天威旣震,人鬼歸能之數,怖駭王師,神武所次,敢不革面,順以從命!輒勑羣帥投戈釋甲,官府帑藏一無所毀。百姓布野,餘糧棲畝,以俟后來之惠,全元元之命。伏惟大魏布德施化,宰輔伊、周,含覆藏疾。謹遣私署侍中張紹、光祿大夫譙周、駙馬都尉鄧良奉齎印緩,請命告誠,敬輸忠款,存亡勑賜,惟所裁之。輿櫬在近,不復縷陳。」是日,北地王諶傷國之亡,先殺妻子,次以自殺。漢晉春秋曰:後主將從譙周之策,北地王諶怒曰:「若理窮力屈,禍敗必及,便當父子君臣背城一戰,同死社稷,以見先帝可也。」後主不納,遂送璽緩。是日,諶哭於昭烈之廟,先殺妻子,而後自殺,左右無不為涕泣者。紹、良與艾相遇於雒縣。艾得書,大喜,即報書,王隱蜀記曰:艾報書云:「王綱失道,羣英並起,龍戰虎爭,終歸真主,此蓋天命去就之道也。自古聖帝,爰逮漢、魏,受命而王者,莫不在乎中土。河出圖,洛出書,聖人則之,以興洪業,其不由此,未有不顛覆者也。隗嚻憑隴而亡,公孫述據蜀而滅,此皆前世覆車之鑒也。聖上明哲,宰相忠賢,將比隆黃軒,侔功往代。銜命來征,思聞嘉響,果煩來使,告以德音,此非人事,豈天啟哉!昔微子歸周,實為上賔,君子豹變,義存大易,來辭謙冲,以禮輿櫬,皆前哲歸命之典也。全國為上,破國次之,自非通明智達,何以見王者之義乎!」禪又遣太常張峻、益州別駕汝超受節度,遣太僕蔣顯有命勑姜維。又遣尚書郎李虎送士民簿,領戶二十八萬,男女口九十四萬,帶甲將士十萬二千,吏四萬人,米四十餘萬斛,金銀各二千斤,錦綺綵絹各二十萬匹,餘物稱此。遣紹、良先還。艾至城北,後主輿櫬自縛,詣軍壘門。艾解縛焚櫬,延請相見。晉諸公贊曰:劉禪乘騾車詣艾,不具亡國之禮。因承制拜後主為驃騎將軍。諸圍守悉被後主勑,然後降下。艾使後主止其故宮,身往造焉。資嚴未發,明年春正月,艾見收。鍾會自涪至成都作亂。會旣死,蜀中軍衆鈔略,死喪狼藉,數日乃安集。

後主舉家東遷,旣至洛陽,策命之曰:「惟景元五年三月丁亥。皇帝臨軒,使太常嘉命劉禪為安樂縣公。於戲,其進聽朕命!蓋統天載物,以咸寧為大,光宅天下,以時雍為盛。故孕育羣生者,君人之道也,乃順承天者,坤元之義也。上下交暢,然後萬物恊和,庶類獲乂。乃者漢氏失統,六合震擾。我太祖承運龍興,弘濟八極,是用應天順民,撫有區夏。于時乃考因羣傑虎爭,九服不靜,乘間阻遠,保據庸蜀,遂使西隅殊封,方外壅隔。自是以來,干戈不戢,元元之民不得保安其性,幾將五紀。朕永惟祖考遺志,思在綏緝四海,率土同軌,故爰整六師,耀威梁、益。公恢崇德度,深秉大正,不憚屈身委質,以愛民全國為貴,降心回慮,應機豹變,履言思順,以享左右無疆之休,豈不遠歟!朕嘉與君公長饗顯祿,用考咨前訓,開國胙土,率遵舊典,鍚茲玄牡,苴以白茅,永為魏藩輔,往欽哉!公其祗服朕命,克廣德心,以終乃顯烈。」食邑萬戶,賜絹萬匹,奴婢百人,他物稱是。子孫為三都尉封侯者五十餘人。尚書令樊建、侍中張紹、光祿大夫譙周、祕書令郤正、殿中督張通並封列侯。漢晉春秋曰:司馬文王與禪宴,為之作故蜀技,旁人皆為之感愴,而禪喜笑自若。王謂賈充曰:「人之無情,乃可至於是乎!雖使諸葛亮在,不能輔之乆全,而況姜維邪?」充曰:「不如是,殿下何由并之。」他日,王問禪曰:「頗思蜀否?」禪曰:「此間樂,不思蜀。」郤正聞之,求見禪曰:「若王後問,宜泣而荅曰『先人墳墓遠在隴、蜀,乃心西悲,無日不思』,因閉其目。」會王復問,對如前,王曰:「何乃似郤正語邪!」禪驚視曰:「誠如尊命。」左右皆笑。公太始七年薨於洛陽。蜀記云:謚曰思,公子恂嗣。

評曰:後主任賢相則為循理之君,惑閹豎則為昏闇之后,傳曰「素絲無常,唯所染之」,信矣哉!禮,國君繼體,踰年改元,而章武之三年,則革稱建興,考之古義,體理為違。又國不置史,注記無官,是以行事多遺,灾異靡書。諸葛亮雖達於為政,凡此之類,猶有未周焉。然經載十二而年名不易,軍旅屢興而赦不妄下,不亦卓乎!自亮沒後,茲制漸虧,優劣著矣。華陽國志曰:丞相亮時,有言公惜赦者,亮荅曰:「治世以大德,不以小惠,故匡衡、吳漢不願為赦。先帝亦言吾周旋陳元方、鄭康成間,每見啟告,治亂之道悉矣,曾不語赦也。若劉景升、季玉父子,歲歲赦宥,何益於治!」臣松之以為「赦不妄下」,誠為可稱,至於「年名不易」,猶所未達。案建武、建安之號,皆乆而不改,未聞前史以為美談。「經載十二」,蓋何足云?豈別有他意,求之未至乎!亮沒後,延熈之號,數盈二十,「茲制漸虧」,事又不然也。


\end{pinyinscope}