\article{杜瓊傳}

\begin{pinyinscope}
杜瓊字伯瑜,蜀郡成都人也。少受學於任安,精究安術。劉璋時辟為從事。先主定益州,領牧,以瓊為議曹從事。後主踐阼,拜諫議大夫,遷左中郎將、大鴻臚、太常。為人靜默少言,闔門自守,不與世事。蔣琬、費禕等皆器重之。雖學業入深,初不視天文有所論說。後進通儒譙周常問其意,瓊荅曰:「欲明此術甚難,須當身視,識其形色,不可信人也。晨夜苦劇,然後知之,復憂漏泄,不如不知,是以不復視也。」周因問曰:「昔周徵君以為當塗高者魏也,其義何也?」瓊荅曰:「魏,闕名也,當塗而高,聖人取類而言耳。」又問周曰:「寧復有所怪邪?」周曰:「未達也。」瓊又曰:「古者名官職不言曹;始自漢已來,名官盡言曹,吏言屬曹,卒言侍曹,此殆天意也。」

瓊年八十餘,延熈十三年卒。著韓詩章句十餘萬言,不教諸子,內學無傳業者。周緣瓊言,乃觸類而長之曰:「春秋傳著晉穆侯名太子曰仇,弟曰成師。師服曰:『異哉君之名子也!嘉耦曰妃,怨偶曰仇,今君名太子曰仇,弟曰成師,始兆亂矣,兄其替乎?』其後果如服言。及漢靈帝名二子曰史侯、董侯,旣立為帝,後皆免為諸侯,與師服言相似也。先主諱備,其訓具也,後主諱禪,其訓授也,如言劉已具矣,當授與人也;意者甚於穆侯、靈帝之名子。」後宦人黃皓弄權於內,景耀五年,宮中大樹無故自折,周深憂之,無所與言,乃書柱曰:「衆而大,期之會,具而授,若何復?」言曹者衆也,魏者大也,衆而大,天下其當會也。具而授,如何復有立者乎?蜀旣亡,咸以周言為驗。周曰:「此雖己所推尋,然有所因,由杜君之辭而廣之耳,殊無神思獨至之異也。」


\end{pinyinscope}