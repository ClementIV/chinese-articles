\article{劉琰傳}

\begin{pinyinscope}
劉琰字威碩,魯國人也。先主在豫州,辟為從事,以其宗姓,有風流,善談論,厚親待之,遂隨從周旋,常為賔客。先主定益州,以琰為固陵太守。後主立,封都鄉侯,班位每亞李嚴,為衞尉中軍師後將軍,遷車騎將軍。然不豫國政,但領兵千餘,隨丞相亮諷議而已。車服飲食,號為侈靡,侍婢數十,皆能為聲樂,又悉教誦讀魯靈光殿賦。建興十年,與前軍師魏延不和,言語虛誕,亮責讓之。琰與亮牋謝曰:「琰稟性空虛,本薄操行,加有酒荒之病,自先帝已來,紛紜之論,殆將傾覆。頗蒙明公本其一心在國,原其身中穢垢,扶持全濟,致其祿位,以至今日。間者迷醉,言有違錯,慈恩含忍,不致之于理,使得全完,保育性命。雖必克己責躬,改過投死,以誓神靈;無所用命,則靡寄顏。」於是亮遣琰還成都,官位如故。

琰失志慌惚。十二年正月,琰妻胡氏入賀太后,太后令特留胡氏,經月乃出。胡氏有美色,琰疑其與後主有私,呼卒,五百撾胡,至於以履搏靣,而後棄遣。胡具以告言琰,琰坐下獄。有司議曰:「卒非撾妻之人,靣非受履之地。」琰竟棄市。自是大臣妻母朝慶遂絕。


\end{pinyinscope}