\article{xu-chu-chuan}

\begin{pinyinscope}
許褚字仲康,譙國譙人也。長八尺餘,腰大十圍,容貌雄毅,勇力絕人。漢末,聚少年及宗族數千家,共堅壁以禦寇。時汝南葛陂賊萬餘人攻褚壁,褚衆少不敵,力戰疲極。兵矢盡,乃令壁中男女,聚治石如杅斗者置四隅。褚飛石擲之,所值皆摧碎。賊不敢進。糧乏,偽與賊和,以牛與賊易食,賊來取牛,牛輒奔還。褚乃出陳前,一手逆曳牛尾,行百餘步。賊衆驚,遂不敢取牛而走。由是淮、汝、陳、梁閒,聞皆畏憚之。

太祖徇淮、汝,褚以衆歸太祖。太祖見而壯之曰:「此吾樊噲也。」即日拜都尉,引入宿衞。諸從褚俠客,皆以為虎士。從征張繡,先登,斬首萬計,遷校尉。從討袁紹於官渡。時常從士徐他等謀為逆,以褚常侍左右,憚之不敢發。伺褚休下日,他等懷刀入。褚至下舍心動,即還侍。他等不知,入帳見褚,大驚愕。他色變,褚覺之,即擊殺他等。太祖益親信之,出入同行,不離左右。從圍鄴,力戰有功,賜爵關內侯。從討韓遂、馬超於潼關。太祖將北渡,臨濟河,先渡兵,獨與褚及虎士百餘人留南岸斷後。超將步騎萬餘人,來奔太祖軍,矢下如雨。褚白太祖,賊來多,今兵渡以盡,宜去,乃扶太祖上船。賊戰急,軍爭濟,船重欲沒。褚斬攀船者,左手舉馬鞍鞌太祖。船工為流矢所中死,褚右手並泝船,僅乃得渡。是日,微褚幾危。其後太祖與遂、超等單馬會語,左右皆不得從,唯將褚。超負其力,陰欲前突太祖,素聞褚勇,疑從騎是褚。乃問太祖曰:「公有虎侯者安在?」太祖顧指褚,褚瞋目盼之。超不敢動,乃各罷。後數日會戰,大破超等,褚身斬首級,遷武衞中郎將。武衞之號,自此始也。軍中以褚力如虎而癡,故號曰虎癡;是以超問虎侯,至今天下稱焉,皆謂其姓名也。

褚性謹慎奉法,質重少言。曹仁自荊州來朝謁,太祖未出,入與褚相見於殿外。仁呼褚入便坐語,褚曰:「王將出。」便還入殿,仁意恨之。或以責褚曰:「征南宗室重臣,降意呼君,君何故辭?」褚曰:「彼雖親重,外藩也。褚備內臣,衆談足矣,入室何私乎?」太祖聞,愈愛待之,遷中堅將軍。太祖崩,褚號泣歐血。文帝踐阼,進封萬歲亭侯,遷武衞將軍,都督中軍宿衞禁兵,甚親近焉。初,褚所將為虎士者從征伐,太祖以為皆壯士也,同日拜為將,其後以功為將軍封侯者數十人,都尉、校尉百餘人,皆劒客也。明帝即位,進牟鄉侯,邑七百戶,賜子爵一人關內侯。褚薨,謚曰壯侯。子儀嗣。褚兄定,亦以軍功封為振威將軍,都督徼道虎賁。太和中,帝思褚忠孝,下詔襃贊,復賜褚子孫二人爵關內侯。儀為鍾會所殺。泰始初,子綜嗣。


\end{pinyinscope}