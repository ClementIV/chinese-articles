\article{zhang-miao-chuan}

\begin{pinyinscope}
張邈字孟卓,東平壽張人也。少以俠聞,振窮救急,傾家無愛,士多歸之。太祖、袁紹皆與邈友。辟公府,以高第拜騎都尉,遷陳留太守。董卓之亂,太祖與邈首舉義兵。汴水之戰,邈遣衞茲將兵隨太祖。袁紹旣為盟主,有驕矜色,邈正議責紹。紹使太祖殺邈,太祖不聽,責紹曰:「孟卓,親友也,是非當容之。今天下未定,不宜自相危也。」邈知之,益德太祖。太祖之征陶謙,勑家曰;「我若不還,往依孟卓。」後還,見邈,垂泣相對。其親如此。

呂布之捨袁紹從張楊也,過邈臨別,把手共誓。紹聞之,大恨。邈畏太祖終為紹擊己也,心不自安。興平元年,太祖復征謙,邈弟超,與太祖將陳宮、從事中郎許汜、王楷共謀叛太祖。宮說邈曰:「今雄傑並起,天下分崩,君以千里之衆,當四戰之地,撫劒顧眄,亦足以為人豪,而反制於人,不以鄙乎!今州軍東征,其處空虛,呂布壯士,善戰無前,若權迎之,共牧兖州,觀天下形勢,俟時事之變通,此亦縱橫之一時也。」邈從之。太祖初使宮將兵留屯東郡,遂以其衆東迎布為兖州牧,據濮陽。郡縣皆應,唯鄄城、東阿、范為太祖守。太祖引軍還,與布戰於濮陽,太祖軍不利,相持百餘日。是時歲旱、蟲蝗、少穀,百姓相食,布東屯山陽。二年間,太祖乃盡復收諸城,擊破布於鉅野。布東奔劉備。

英雄記曰:布見備,甚敬之,謂備曰:「我與卿同邊地人也。布見關東起兵,欲誅董卓。布殺卓東出,關東諸將無安布者,皆欲殺布耳。」請備於帳中坐婦牀上,令婦向拜,酌酒飲食,名備為弟。備見布語言無常,外然之而內不說。邈從布,留超將家屬屯雍丘。太祖攻圍數月,屠之,斬超及其家。邈詣袁術請救未至,自為其兵所殺。獻帝春秋曰:袁術議稱尊號,邈謂術曰:「漢據火德,絕而復揚,德澤豐流,誕生明公。公居軸處中,入則享于上席,出則為衆目之所屬,華、霍不能增其高,淵泉不能同其量,可謂巍巍蕩蕩,無與為貳。何為捨此而欲稱制?恐福不盈眥,禍將溢世。莊周之稱郊祭犧牛,養飼經年,衣以文繡,宰執鸞刀,以入廟門,當此之時,求為孤犢不可得也!」案本傳,邈詣術,未至而死。而此云諫稱尊號,未詳孰是。

備東擊術,布襲取下邳,備還歸布。布遣備屯小沛。布自稱徐州刺史。英雄記曰:布初入徐州,書與袁術。術報書曰:「昔董卓作亂,破壞王室,禍害術門戶,術舉兵關東,未能屠裂卓。將軍誅卓,送其頭首,為術掃滅讐耻,使術明目於當世,死生不愧,其功一也。昔將金元休向兖州,甫詣封部,為曹操逆所拒破,流離迸走,幾至滅亡。將軍破兖州,術復明目於遐邇,其功二也。術生年已來,不聞天下有劉備,備乃舉兵與術對戰;術憑將軍威靈,得以破備,其功三也。將軍有三大功在術,術雖不敏,奉以生死。將軍連年攻戰,軍糧苦少,今送米二十萬斛,迎逢道路,非直此止,當駱驛復致;若兵器戰具,佗所乏少,大小唯命。」布得書大喜,遂造下邳。典略曰:元休名尚,京兆人也。尚與同郡韋休甫、第五文休俱著名,號為三休。尚,獻帝初為兖州刺史,東之郡,而太祖已臨兖州。尚南依袁術。術僭號,欲以尚為太尉,不敢顯言,私使人諷之,尚無屈意,術亦不敢彊也。建安初,尚逃還,為術所害。其後尚喪與太傅馬日磾喪俱至京師,天子嘉尚忠烈,為之咨嗟,詔百官弔祭,拜子瑋郎中,而日磾不與焉。英雄記曰:布水陸東下,軍到下邳西四十里。備中郎將丹楊許耽夜遣司馬章誑來詣布,言「張益德與下邳相曹豹共爭,益德殺豹,城中大亂,不相信。丹楊兵有千人屯西白城門內,聞將軍來東,大小踊躍,如復更生。將軍兵向城西門,丹楊軍便開門內將軍矣」。布遂夜進,晨到城下。天明,丹楊兵悉開門內布兵。布於門上坐,步騎放火,大破益德兵,獲備妻子軍資及部曲將吏士家口。建安元年六月夜半時,布將河內郝萌反,將兵入布所治下邳府,詣廳事閤外,同聲大呼攻閤,閤堅不得入。布不知反者為誰,直牽婦,科頭袒衣,相將從溷上排壁出,詣都督高順營,直排順門入。順問:「將軍有所隱不?」布言「河內兒聲」。順言「此郝萌也」。順即嚴兵入府,弓弩並射萌衆;萌衆亂走,天明還故營。萌將曹性反萌,與對戰,萌刺傷性,性斫萌一臂。順斫萌首,牀輿性,送詣布。布問性,言「萌受袁術謀,謀者悉誰?」性言「陳宮同謀。」時宮在坐上,靣赤,傍人悉覺之。布以宮大將,不問也。性言「萌常以此問,性言呂將軍大將有神,不可擊也,不意萌狂惑不止。」布謂性曰:「卿健兒也!」善養視之。創愈,使安撫萌故營,領其衆。術遣將紀靈等步騎三萬攻備,備求救於布。布諸將謂布曰:「將軍常欲殺備,今可假手於術。」布曰:「不然。術若破備,則北連太山諸將,吾為在術圍中,不得不救也。」便嚴步兵千、騎二百,馳往赴備。靈等聞布至,皆斂兵不敢復攻。布於沛西南一里安屯,遣鈴下請靈等,靈等亦請布共飲食。布謂靈等曰:「玄德,布弟也。弟為諸君所困,故來救之。布性不喜合鬬,但喜解鬬耳。」布令門候於營門中舉一隻戟,布言:「諸君觀布射戟小支,一發中者諸君當解去,不中可留決鬬。」布舉弓射戟,正中小支。諸將皆驚,言「將軍天威也」!明日復歡會,然後各罷。

術欲結布為援,乃為子索布女,布許之。術遣使韓胤以僭號議告布,并求迎婦。沛相陳珪恐術、布成婚,則徐、揚合從,將為國難,於是往說布曰;「曹公奉迎天子,輔讚國政,威靈命世,將征四海,將軍宜與恊同策謀,圖太山之安。今與術結婚,受天下不義之名,必有累卵之危。」布亦怨術初不己受也,女已在塗,追還絕婚,械送韓胤,梟首許市。珪欲使子登詣太祖,布不肯遣。會使者至,拜布左將軍。布大喜,即聽登往,并令奉章謝恩。英雄記曰:初,天子在河東,有手筆版書召布來迎。布軍無畜積,不能自致,遣使上書。朝廷以布為平東將軍,封平陶侯。使人於山陽界亡失文字,太祖又手書厚加慰勞布,說起迎天子,當平定天下意,并詔書購捕公孫瓚、袁術、韓暹、楊奉等。布大喜,復遣使上書於天子曰:「臣本當迎大駕,知曹操忠孝,奉迎都許。臣前與操交兵,今操保傅陛下,臣為外將,欲以兵自隨,恐有嫌疑,是以待罪徐州,進退未敢自寧。」答太祖曰:「布獲罪之人,分為誅首,手命慰勞,厚見褒獎。重見購捕袁術等詔書,布當以命為效。」太祖更遣奉車都尉王則為使者,齎詔書,又封平東將軍印綬來拜布。太祖又手書與布曰:「山陽屯送將軍所失大封,國家無好金,孤自取家好金更相為作印,國家無紫綬,自取所帶紫綬以籍心。將軍所使不良。袁術稱天子,將軍止之,而使不通章。朝廷信將軍,使復重上,以相明忠誠。」布乃遣登奉章謝恩,并以一好綬答太祖。登見太祖,因陳布勇而無計,輕於去就,宜早圖之。太祖曰:「布,狼子野心,誠難乆養,非卿莫能究其情也。」即增珪秩中二千石,拜登廣陵太守。臨別,太祖執登手曰:「東方之事便以相付。」令登陰合部衆以為內應。

始,布因登求徐州牧,登還,布怒,拔戟斫机曰:「卿父勸吾恊同曹公,絕婚公路;今吾所求無一獲,而卿父子並顯重,為卿所賣耳!卿為吾言,其說云何?」登不為動容,徐喻之曰;「登見曹公言:『待將軍譬如養虎,當飽其肉,不飽則將噬人。』公曰:『不如卿言也。譬如養鷹,饑則為用,飽則揚去。』其言如此。」布意乃解。

術怒,與韓暹、楊奉等連勢,遣大將張勳攻布。布謂珪曰:「今致術軍,卿之由也,為之柰何?」珪曰:「暹、奉與術,卒合之軍耳,策謀不素定,不能相維持,子登策之,比之連雞,勢不俱棲,可解離也。」布用珪策,遣人說暹、奉,使與己并力共擊術軍,軍資所有,悉許暹、奉。於是暹、奉從之,勳大破敗。九州春秋載布與暹、奉書曰:「二將軍拔大駕來東,有元功於國,當書勳竹帛,萬世不朽。今袁術造逆,當共誅討,奈何與賊臣還共伐布?布有殺董卓之功,與二將軍俱為功臣,可因今共擊破術,建功於天下,此時不可失也。」暹、奉得書,即迴計從布。布進軍,去勳等營百步,暹、奉兵同時並發,斬十將首,殺傷墮水死者不可勝數。英雄記曰:布後又與暹、奉二軍向壽春,水陸並進,所過虜略。到鍾離,大獲而還。旣渡淮北,留書與術曰:「足下恃軍彊盛,常言猛將武士,欲相吞滅,每抑止之耳!布雖無勇,虎步淮南,一時之間,足下鼠竄壽春,無出頭者。猛將武士,為悉何在?足下喜為大言以誣天下,天下之人安可盡誣?古者兵交,使在其間,造策者非布先唱也。相去不遠,可復相聞。」布渡畢,術自將步騎五千揚兵淮上,布騎皆於水北大咍笑之而還。時有東海蕭建為琅邪相,治莒,保城自守,不與布通。布與建書曰:「天下舉兵,本以誅董卓爾。布殺卓,來詣關東,欲求兵西迎大駕,光復洛京,諸將自還相攻,莫肯念國。布,五原人也,去徐州五千餘里,乃在天西北角,今不來共爭天東南之地。莒與下邳相去不遠,宜當共通。君如自遂以為郡郡作帝,縣縣自王也!昔樂毅攻齊,呼吸下齊七十餘城,唯莒、即墨二城不下,所以然者,中有田單故也。布雖非樂毅,君亦非田單,可取布書與智者詳共議之。」建得書,即遣主簿齎牋上禮,貢良馬五匹。建尋為臧霸所襲破,得建資實。布聞之,自將步騎向莒。高順諫曰:「將軍躬殺董卓,威震夷狄,端坐顧盼,遠近自然畏服,不宜輕自出軍;如或不捷,損名非小。」布不從。霸畏布鈔暴,果登城拒守。布不能拔,引還下邳。霸後復與布和。

建安三年,布復叛為術,遣高順攻劉備於沛,破之。太祖遣夏侯惇救備,為順所敗。太祖自征布,至其城下,遺布書,為陳禍福。布欲降,陳宮等自以負罪深,沮其計。獻帝春秋曰:太祖軍至彭城。陳宮謂布:「宜逆擊之,以逸擊勞,無不克也。」布曰:「不如待其來攻,蹙著泗水中。」及太祖軍攻之急,布於白門樓上謂軍士曰:「卿曹無相困,我當自首明公。」陳宮曰:「逆賊曹操,何等明公!今日降之,若卵投石,豈可得全也!」布遣人求救於術,自將千餘騎出戰,敗走,還保城,不敢出。英雄記曰:布遣許汜、王楷告急於術。術曰:「布不與我女,理自當敗,何為復來相聞邪?」汜、楷曰:「明上今不救布,為自敗耳!布破,明上亦破也。」術時僭號,故呼為明上。術乃嚴兵為布作聲援。布恐術為女不至,故不遣兵救也,以緜纏女身,縛著馬上,夜自送女出與術,與太祖守兵相觸,格射不得過,復還城。布欲令陳宮、高順守城,自將騎斷太祖糧道。布妻謂曰:「將軍自出斷曹公糧道是也。宮、順素不和,將軍一出,宮、順必不同心共城守也,如有蹉跌,將軍當於何自立乎?願將軍諦計之,無為宮等所誤也。妾昔在長安,已為將軍所棄,賴得龐舒私藏妾身耳,今不須顧妾也。」布得妻言,愁悶不能自決。魏氏春秋曰:陳宮謂布曰:「曹公遠來,勢不能乆。若將軍以步騎出屯,為勢於外,宮將餘衆閉守於內,若向將軍,宮引兵而攻其背,若來攻城,將軍為救於外。不過旬日,軍食必盡,擊之可破。」布然之。布妻曰:「昔曹氏待公臺如赤子,猶舍而來。今將軍厚公臺不過於曹公,而欲委全城,捐妻子,孤軍遠出,若一旦有變,妾豈得為將軍妻哉!」布乃止。術亦不能救。布雖驍猛,然無謀而多猜忌,不能制御其黨,但信諸將。諸將各異意自疑,故每戰多敗。太祖塹圍之三月,上下離心,其將侯成、宋憲、魏續縛陳宮,將其衆降。九州春秋曰:初,布騎將侯成遣客牧馬十五匹,客悉驅馬去,向沛城,欲歸劉備。成自將騎逐之,悉得馬還。諸將合禮賀成,成釀五六斛酒,獵得十餘頭豬,未飲食,先持半豬五斗酒自入詣布前,跪言:「間蒙將軍恩,逐得所失馬,諸將來相賀,自釀少酒,獵得豬,未敢飲食,先奉上微意。」布大怒曰:「布禁酒,卿釀酒,諸將共飲食作兄弟,共謀殺布邪?」成大懼而去,棄所釀酒,還諸將禮。由是自疑,會太祖圍下邳,成遂領衆降。布與其麾下登白門樓。兵圍急,乃下降。遂生縛布,布曰:「縛太急,小緩之。」太祖曰:「縛虎不得不急也。」布請曰:「明公所患不過於布,今已服矣,天下不足憂。明公將步,令布將騎,則天下不足定也。」太祖有疑色。劉備進曰:「明公不見布之事丁建陽及董太師乎!」太祖頷之。布因指備曰:「是兒最叵信者。」英雄記曰:布謂太祖曰:「布待諸將厚也,諸將臨急皆叛布耳。」太祖曰:「卿背妻,愛諸將婦,何以為厚?」布默然。獻帝春秋曰:布問太祖:「明公何瘦?」太祖曰:「君何以識孤?」布曰:「昔在洛,會溫氏園。」太祖曰:「然。孤忘之矣。所以瘦,恨不早相得故也。」布曰:「齊桓舍射鉤,使管仲相;今使布竭股肱之力,為公前驅,可乎?」布縛急,謂劉備曰:「玄德,卿為坐客,我為執虜,不能一言以相寬乎?」太祖笑曰:「何不相語,而訴明使君乎?」意欲活之,命使寬縛。主簿王必趨進曰:「布,勍虜也。其衆近在外,不可寬也。」太祖曰:「本欲相緩,主簿復不聽,如之何?」於是縊殺布。布與宮、順等皆梟首送許,然後葬之。英雄記曰:順為人清白有威嚴,不飲酒,不受饋遺。所將七百餘兵,號為千人,鎧甲鬬具皆精練齊整,每所攻擊無不破者,名為陷陣營。順每諫布,言「凡破家亡國,非無忠臣明智者也,但患不見用耳。將軍舉動,不肯詳思,輙喜言誤,誤不可數也」。布知其忠,然不能用。布從郝萌反後,更疏順。以魏續有外內之親,悉奪順所將兵以與續。及當攻戰,故令順將續所領兵,順亦終無恨意。

太祖之禽宮也,問宮欲活老母及女不?宮對曰:「宮聞孝治天下者不絕人之親,仁施四海者不乏人之祀,老母在公,不在宮也。」太祖召養其母終其身,嫁其女。魚氏典略曰:陳宮字公臺,東郡人也。剛直烈壯,少與海內知名之士皆相連結。及天下亂,始隨太祖,後自疑,乃從呂布,為布畫策,布每不從其計。下邳敗,軍士執布及宮,太祖皆見之,與語平生,故布有求活之言。太祖謂宮曰:「公臺,卿平常自謂智計有餘,今竟何如?」宮顧指布曰:「但坐此人不從宮言,以至於此。若其見從,亦未必為禽也。」太祖笑曰:「今日之事當云何?」宮曰:「為臣不忠,為子不孝,死自分也。」太祖曰:「卿如是,柰卿老母何?」宮曰:「宮聞將以孝治天下者不害人之親,老母之存否,在明公也。」太祖曰:「若卿妻子何?」宮曰:「宮聞將施仁政於天下者不絕人之祀,妻子之存否,亦在明公也。」太祖未復言。宮曰:「請出就戮,以明軍法。」遂趨出,不可止。太祖泣而送之,宮不還顧。宮死後,太祖待其家皆厚於初。

陳登者,字元龍,在廣陵有威名。又掎角呂布有功,加伏波將軍,年三十九卒。後許汜與劉備並在荊州牧劉表坐,表與備共論天下人,汜曰:「陳元龍湖海之士,豪氣不除。」備謂表曰:「許君論是非?」表曰:「欲言非,此君為善士,不宜虛言;欲言是,元龍名重天下。」備問汜:「君言豪,寧有事邪?」汜曰:「昔遭亂過下邳,見元龍。元龍無客主之意,乆不相與語,自上大牀卧,使客卧下牀。」備曰:「君有國士之名,今天下大亂,帝主失所,望君憂國忘家,有救世之意,而君求田問舍,言無可采,是元龍所諱也,何緣當與君語?如小人,欲卧百尺樓上,卧君於地,何但上下牀之閒邪?」表大笑。備因言曰:「若元龍文武膽志,當求之於古耳,造次難得比也。」先賢行狀曰:登忠亮高爽,沈深有大略,少有扶世濟民之志。博覽載籍,雅有文藝,舊典文章,莫不貫綜。年二十五,舉孝廉,除東陽長,養耆育孤,視民如傷。是時世荒民饑,州牧陶謙表登為典農校尉,乃巡土田之宜,盡鑿溉之利,秔稻豐積。奉使到許,太祖以登為廣陵太守,令陰合衆以圖呂布。登在廣陵,明審賞罰,威信宣布。海賊薛州之羣萬有餘戶,束手歸命。未及期年,功化以就,百姓畏而愛之。登曰:「此可用矣。」太祖到下邳,登率郡兵為軍先驅。時登諸弟在下邳城中,布乃質執登三弟,欲求和同。登執意不撓,進圍日急。布刺姦張弘,懼於後累,夜將登三弟出就登。布旣伏誅,登以功加拜伏波將軍,甚得江、淮間歡心,於是有吞滅江南之志。孫策遣軍攻登於匡琦城。賊初到,旌甲覆水,羣下咸以今賊衆十倍於郡兵,恐不能抗,可引軍避之,與其空城。水人居陸,不能乆處,必尋引去。登厲聲曰:「吾受國命,來鎮此土。昔馬文淵之在斯位,能南平百越,北滅羣狄,吾旣不能遏除凶慝,何逃寇之謂邪!吾其出命以報國,仗義以整亂,天道與順,克之必矣。」乃閉門自守,示弱不與戰,將士銜聲,寂若無人。登乘城望形勢,知其可擊,乃申令將士,宿整兵器,昧爽,開南門,引軍指賊營,步騎鈔其後。賊周章方結陣,不得還舩。登手執軍鼓,縱兵乘之,賊遂大破,皆棄舩迸走。登乘勝追奔,斬虜以萬數。賊忿喪軍,尋復大興兵向登。登以兵不敵,使功曹陳矯求救於太祖。登密去城十里治軍營處所,令多取柴薪,兩束一聚,相去十步,從橫成行,令夜俱起火,火然其聚。城上稱慶,若大軍到。賊望火驚潰,登勒兵追奔,斬首萬級。遷登為東城太守。廣陵吏民佩其恩德,共拔郡隨登,老弱繈負而追之。登曉語令還,曰:「太守在卿郡,頻致吳寇,幸而克濟。諸卿何患無令君乎?」孫權遂跨有江外。太祖每臨大江而歎,恨不早用陳元龍計,而令封豕養其爪牙。文帝追美登功,拜登息肅為郎中。


\end{pinyinscope}