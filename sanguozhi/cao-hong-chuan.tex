\article{cao-hong-chuan}

\begin{pinyinscope}
曹洪字子廉,太祖從弟也。

魏書曰:洪伯父鼎為尚書令,任洪為蘄春長。太祖起義兵討董卓,至熒陽,為卓將徐榮所敗。太祖失馬,賊追甚急,洪下,以馬授太祖,太祖辭讓,洪曰:「天下可無洪,不可無君。」遂步從到汴水,水深不得渡,洪循水得船,與太祖俱濟,還奔譙。揚州刺史陳溫素與洪善,洪將家兵千餘人,就溫募兵,得廬江上甲二千人,東到丹楊復得數千人,與太祖會龍亢。

太祖征徐州,張邈舉兖州叛迎呂布。時大饑荒,洪將兵在前,先據東平、范,聚糧穀以繼軍。太祖討邈、布於濮陽,布破走,遂據東阿,轉擊濟陰、山陽、中牟、陽武、京、密十餘縣,皆拔之。以前後功拜鷹揚校尉,遷揚武中郎將。

天子都許,拜洪諫議大夫。別征劉表,破表別將於舞陽、陰葉、堵陽、博望,有功,遷厲鋒將軍,封國明亭侯。累從征伐,拜都護將軍。文帝即位,為衞將軍,遷驃騎將軍,進封野王侯,益邑千戶,并前二千一百戶,位特進;後徙封都陽侯。

始,洪家富而性吝嗇,文帝少時假求不稱,常恨之,遂以舍客犯法,下獄當死。羣臣並救莫能得。卞太后謂郭后曰:「令曹洪今日死,吾明日勑帝廢后矣。」於是泣涕屢請,乃得免官削爵土。魏略曰:文帝收洪,時曹真在左右,請之曰:「今誅洪,洪必以真為譖也。」帝曰:「我自治之,卿何豫也?」會卞太后責怒帝,言「梁、沛之間,非子廉無有今日」。詔乃釋之。猶尚沒入其財產。太后又以為言,後乃還之。初,太祖為司空時,以己率下,每歲發調,使本縣平貲。于時譙令平洪貲財與公家等,太祖曰:「我家貲郍得如子廉邪!」文帝在東宮,嘗從洪貸絹百匹,洪不稱意。及洪犯法,自分必死,旣得原,喜,上書謝曰:「臣少不由道,過在人倫,長竊非任,遂蒙含貸。性無檢度知足之分,而有犲狼無厭之質,老惛倍貪,觸突國網,罪迫三千,不在赦宥,當就辜誅,棄諸市朝,猶蒙天恩,骨肉更生。臣仰視天日,愧負靈神,俯惟愆闕,慙愧怖悸,不能雉經以自裁割,謹塗顏闕門,拜章陳情。」洪先帝功臣,時人多為觖望。明帝即位,拜後將軍,更封樂城侯,邑千戶,位特進,復拜驃騎將軍。太和六年薨,謚曰恭侯。子馥,嗣侯。初,太祖分洪戶封子震列侯。洪族父瑜,脩慎篤敬,官至衞將軍,封列侯。


\end{pinyinscope}