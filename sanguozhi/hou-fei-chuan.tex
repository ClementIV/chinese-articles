\article{hou-fei-chuan}

\begin{pinyinscope}
易稱「男正位乎外,女正位乎內;男女正,天地之大義也」。古先哲王,莫不明后妃之制,順天地之德,故二妃嬪媯,虞道克隆,任、姒配姬,周室用熈,廢興存亡,恒此之由。春秋說云天子十二女,諸侯九女,考之情理,不易之典也。而末世奢縱,肆其侈欲,至使男女怨曠,感動和氣,惟色是崇,不本淑懿,故風教陵遲而大綱毀泯,豈不惜哉!嗚呼,有國有家者,其可以永鑒矣!

漢制,帝祖母曰太皇太后,帝母曰皇太后,帝妃曰皇后,其餘內官十有四等。魏因漢法,母后之號皆如舊制,自夫人以下,世有增損。太祖建國,始命王后,其下五等:有夫人,有昭儀,有倢伃,有容華,有美人。文帝增貴嬪、淑媛、脩容、順成、良人。明帝增淑妃、昭華、脩儀;除順成官。太和中始復命夫人,登其位於淑妃之上。自夫人以下爵凡十二等:貴嬪、夫人,位次皇后,爵無所視;淑妃位視相國,爵比諸侯王;淑媛位視御史大夫,爵比縣公;昭儀比縣侯;昭華比鄉侯;脩容比亭侯;脩儀比關內侯;倢伃視中二千石;容華視真二千石;美人視比二千石;良人視千石。


\end{pinyinscope}