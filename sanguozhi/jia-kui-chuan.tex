\article{jia-kui-chuan}

\begin{pinyinscope}
賈逵字梁道,河東襄陵人也。自為兒童,戲弄常設部伍,祖父習異之,曰:「汝大必為將率。」口授兵法數萬言。

魏略曰:逵世為著姓,少孤家貧,冬常無絝,過其妻兄柳孚宿,其明無何,著孚絝去,故時人謂之通健。初為郡吏,守絳邑長。郭援之攻河東,所經城邑皆下,逵堅守,援攻之不拔,乃召單于并軍急攻之。城將潰,絳父老與援要,不害逵。絳人旣潰,援聞逵名,欲使為將,以兵劫之,逵不動。左右引逵使叩頭,逵叱之曰:「安有國家長吏為賊叩頭!」援怒,將斬之。絳吏民聞將殺逵,皆乘城呼曰:「負要殺我賢君,寧俱死耳!」左右義逵,多為請,遂得免。魏略曰:援捕得逵,逵不肯拜,謂援曰:「王府君臨郡積年,不知足下曷為者也?」援怒曰:「促斬之。」諸將覆護,乃囚於壺關,閉著土窖中,以車輪蓋上,使人固守。方將殺之,逵從窖中謂守者曰:「此間无健兒邪,而當使義士死此中乎?」時有祝公道者,與逵非故人,而適聞其言,憐其守正危厄,乃夜盜往引出,折械遣去,不語其名姓。初,逵過皮氏,曰:「爭地先據者勝。」及圍急,知不免,乃使人間行送印綬歸郡,且曰「急據皮氏」。援旣并絳衆,將進兵。逵恐其先得皮氏,乃以他計疑援謀人祝奧,援由是留七日。郡從逵言,故得无敗。孫資別傳曰:資舉河東計吏,到許,薦於相府曰:「逵在絳邑,帥厲吏民,與賊郭援交戰,力盡而敗,為賊所俘,挺然直志,顏辭不屈;忠言聞於大衆,烈節顯于當時,雖古之直髮、據鼎,罔以加也。其才兼文武,誠時之利用。」魏略曰:郭援破後,逵乃知前出己者為祝公道。公道,河南人也。後坐他事,當伏法。逵救之,力不能解,為之改服焉。

後舉茂才,除澠池令。高幹之反,張琰將舉兵以應之。逵不知其謀,往見琰。聞變起,欲還,恐見執,乃為琰畫計,如與同謀者,琰信之。時縣寄治蠡城,城塹不固,逵從琰求兵脩城。諸欲為亂者皆不隱其謀,故逵得盡誅之。遂脩城拒琰。琰敗,逵以喪祖父去官,司徒辟為掾,以議郎參司隷軍事。太祖征馬超,至弘農,曰「此西道之要」,以逵領弘農太守。召見計事,大恱之,謂左右曰:「使天下二千石悉如賈逵,吾何憂?」其後發兵,逵疑屯田都尉藏亡民。都尉自以不屬郡,言語不順。逵怒,收之,數以罪,檛折脚,坐免。然太祖心善逵,以為丞相主簿。魏略曰:太祖欲征吳而大霖雨,三軍多不願行。太祖知其然,恐外有諫者,教曰:「今孤戒嚴,未知所之,有諫者死。」逵受教,謂其同寮三主簿曰:「今實不可出,而教如此,不可不諫也。」乃建諫草以示三人,三人不獲已,皆署名,入白事。太祖怒,收逵等。當送獄,取造意者,逵即言「我造意」,遂走詣獄。獄吏以逵主簿也,不即著械。謂獄吏曰:「促械我。尊者且疑我在近職,求緩於卿,今將遣人來察我。」逵著械適訖,而太祖果遣家中人就獄視逵。旣而教曰:「逵無惡意,原復其職。」始,逵為諸生,略覽大義,取其可用。最好春秋左傳,及為牧守,常自課讀之,月常一遍。逵前在弘農,與典農校尉爭公事,不得理,乃發憤生癭,後所病稍大,自啟願欲令醫割之。太祖惜逵忠,恐其不活,教「謝主簿,吾聞『十人割癭九人死』」。逵猶行其意,而癭愈大。逵本名衢,後改為逵。太祖征劉備,先遣逵至斜谷觀形勢。道逢水衡,載囚人數十車,逵以軍事急,輒竟重者一人,皆放其餘。太祖善之,拜諫議大夫,與夏侯尚並掌軍計。太祖崩洛陽,逵典喪事。魏略曰:時太子在鄴,鄢陵侯未到,士民頗苦勞役,又有疾癘,於是軍中搔動。羣寮恐天下有變,欲不發喪。逵建議為不可祕,乃發哀,令內外皆入臨,臨訖,各安叙不得動。而青州軍擅擊鼓相引去。衆人以為宜禁止之,不從者討之。逵以為「方大喪在殯,嗣王未立,宜因而撫之」。乃為作長檄,告所在給其廩食。時鄢陵侯彰行越騎將軍,從長安來赴,問逵先王璽綬所在。逵正色曰:「太子在鄴,國有儲副。先王璽綬,非君侯所宜問也。」遂奉梓宮還鄴。

文帝即王位,以鄴縣戶數萬在都下,多不法,乃以逵為鄴令。月餘,遷魏郡太守。魏略曰:初,魏郡官屬頗以公事期會有所急切,會聞逵當為郡,舉府皆詣縣門外。及遷書到,逵出門,而郡官屬悉當門,謁逵於車下。逵抵掌曰:「詣治所,何宜如是!」大軍出征,復為丞相主簿祭酒。逵嘗坐人為罪,王曰:「叔嚮猶十世宥之,況逵功德親在其身乎?」從至黎陽,津渡者亂行,逵斬之,乃整。至譙,以逵為豫州刺史。魏略曰:逵為豫州。逵進曰:「臣守天門,出入六年,天門始開,而臣在外。唯殿下為兆民計,無違天人之望。」是時天下初復,州郡多不攝。逵曰:「州本以御史出監諸郡,以六條詔書察長吏二千石已下,故其狀皆言嚴能鷹揚有督察之才,不言安靜寬仁有愷悌之德也。今長吏慢法,盜賊公行,州知而不糾,天下復何取正乎?」兵曹從事受前刺史假,逵到官數月,乃還;考竟其二千石以下阿縱不如法者,皆舉奏免之。帝曰:「逵真刺史矣。」布告天下,當以豫州為法。賜爵關內侯。

州南與吳接,逵明斥候,繕甲兵,為守戰之備,賊不敢犯。外脩軍旅,內治民事,遏鄢、汝,造新陂,又斷山溜長谿水,造小弋陽陂,又通運渠二百餘里,所謂賈侯渠者也。黃初中,與諸將並征吳,破呂範於洞浦,進封陽里亭侯,加建威將軍。明帝即位,增邑二百戶,并前四百戶。時孫權在東關,當豫州南,去江四百餘里。每出兵為寇,輙西從江夏,東從廬江。國家征伐,亦由淮、沔。是時州軍在項,汝南、弋陽諸郡,守境而已。權無北方之虞,東西有急,并軍相救,故常少敗。逵以為宜開直道臨江,若權自守,則二方無救;若二方無救,則東關可取。乃移屯潦口,陳攻取之計,帝善之。

吳將張嬰、王崇率衆降。太和二年,帝使逵督前將軍滿寵、東莞太守胡質等四軍,從西陽直向東關,曹休從皖,司馬宣王從江陵。逵至五將山,休更表賊有請降者,求深入應之。詔宣王駐軍,逵東與休合進。逵度賊無東關之備,必并軍於皖;休深入與賊戰,必敗。乃部署諸將,水陸並進,行二百里,得生賊,言休戰敗,權遣兵斷夾石。諸將不知所出,或欲待後軍。逵曰:「休兵敗於外,路絕於內,進不能戰,退不得還,安危之機,不及終日。賊以軍無後繼,故至此;今疾進,出其不意,此所謂先人以奪其心也,賊見吾兵必走。若待後軍,賊已斷險,兵雖多何益!」乃兼道進軍,多設旗皷為疑兵,賊見逵軍,遂退。逵據夾石,以兵糧給休,休軍乃振。初,逵與休不善。黃初中,文帝欲假逵節,休曰:「逵性剛,素侮易諸將,不可為督。」帝乃止。及夾石之敗,微逵,休軍幾無救也。魏略曰:休怨逵進遲,乃呵責逵,遂使主者敕豫州刺史往拾棄仗。逵恃心直,謂休曰:「本為國家作豫州刺史,不來相為拾棄仗也。」乃引軍還。遂與休更相表奏,朝廷雖知逵直,猶以休為宗室任重,兩無所非也。魏書云:休猶挾前意,欲以後期罪逵,逵終無言,時人益以此多逵。習鑿齒曰:夫賢人者,外身虛己,內以下物,嫌忌之名,何由而生乎?有嫌忌之名者,必與物為對,存勝負於己身者也。若以其私憾敗國殄民,彼雖傾覆,於我何利?我苟無利,乘之曷為?以是稱說,臧獲之心耳。今忍其私忿而急彼之憂,冒難犯危而免之於害,使功顯於明君,惠施於百姓,身登於君子之塗,義愧於敵人之心,雖豺虎猶將不覺所復,而況於曹休乎?然則濟彼之危,所以成我之勝,不計宿憾,所以服彼之心,公義旣成,私利亦弘,可謂善爭矣。在於未能忘勝之流,不由於此而能濟勝者,未之有也。

會病篤,謂左右曰:「受國厚恩,恨不斬孫權以下見先帝。喪事一不得有所脩作。」薨,謚曰肅侯。魏書曰:逵時年五十五。子充嗣。豫州吏民追思之,為刻石立祠。青龍中,帝東征,乘輦入逵祠,詔曰:「昨過項,見賈逵碑像,念之愴然。古人有言,患名之不立,不患年之不長。逵存有忠勳,沒而見思,可謂死而不朽者矣。其布告天下,以勸將來。」魏略云:甘露二年,車駕東征,屯項,復入逵祠下,詔曰:「逵沒有遺愛,歷世見祀。追聞風烈,朕甚嘉之。昔先帝東征,亦幸于此,親發德音,襃揚逵美,徘徊之心,益有慨然!夫禮賢之義,或掃其墳墓,或脩其門閭,所以崇敬也。其掃除祠堂,有穿漏者補治之。」充,咸熈中為中護軍。晉諸公贊曰:充字公閭,甘露中為大將軍長史。高貴鄉公之難,司馬文王賴充以免。為晉室元功之臣,位至太宰,封魯公。謚曰武公。魏略列傳以逵及李孚、楊沛三人為一卷,今列孚、沛二人繼逵後耳。孚字子憲,鉅鹿人也。興平中,本郡人民饑困。孚為諸生,當種薤,欲以成計。有從索者,亦不與一莖,亦不自食,故時人謂能行意。後為吏。建安中,袁尚領冀州,以孚為主簿。後尚與其兄譚爭鬬,尚出軍詣平原,留別駕審配守鄴城,孚隨尚行。會太祖圍鄴,尚還欲救鄴。行未到,尚疑鄴中守備少,復欲令配知外動止,與孚議所遣。孚荅尚言:「今使小人往,恐不足以知外內,且恐不能自達。孚請自往。」尚問孚:「當何所得?」孚曰:「聞鄴圍甚堅,多人則覺,以為直當將三騎足矣。」尚從其計。孚自選溫信者三人,不語所之,皆勑使具脯粮,不得持兵仗,各給快馬。遂辭尚來南,所在止亭傳。及到梁淇,使從者斫問事杖三十枚,繫著馬邊,自著平上幘,將三騎,投暮詣鄴下。是時大將軍雖有禁令,而芻牧者多。故孚因此夜到,以鼓一中,自稱都督,歷北圍,循表而東,從東圍表,又循圍而南,步步呵責守圍將士,隨輕重行其罰。遂歷太祖營前,徑南過,從南圍角西折,當章門,復責怒守圍者,收縛之。因開其圍,馳到城下,呼城上人,城上人以繩引,孚得入。配等見孚,悲喜,鼓譟稱萬歲。守圍者以狀聞,太祖笑曰:「此非徒得入也,方且復得出。」孚事訖欲得還,而顧外圍必急,不可復冒。謂己使命當速反,乃陰心計,請配曰:「今城中穀少,無用老弱為也,不如驅出之以省穀也。」配從其計,乃復夜簡別數千人,皆使持白幡,從三門並出降。又使人人持火,孚乃無何將本所從作降人服,隨輩夜出。時守圍將士聞城中悉降,火光照曜。但共觀火,不復視圍。孚出北門,遂從西北角突圍得去。其明,太祖聞孚已得出,抵掌笑曰:「果如吾言也。」孚北見尚,尚甚歡喜。會尚不能救鄴,破走至中山,而袁譚又追擊尚,尚走。孚與尚相失,遂詣譚,復為譚主簿,東還平原。太祖進攻譚,譚戰死。孚還城,城中雖必降,尚擾亂未安。孚權宜欲得見太祖,乃騎詣牙門,稱冀州主簿李孚欲口白密事。太祖見之,孚叩頭謝。太祖問其所白,孚言「今城中彊弱相陵,心皆不定,以為宜令新降為內所識信者宣傳明教。」公謂孚曰:「卿便還宣之。」孚跪請教,公曰:「便以卿意宣也。」孚還入城,宣教「各安故業,不得相侵陵。」城中以安,乃還報命,公以孚為良足用也。會為所閒,裁署宂散。出守解長,名為嚴能。稍遷至司隷校尉,時年七十餘矣,其於精斷無衰,而術略不損於故。終於陽平太守。孚本姓馮,後改為李。楊沛字孔渠,馮翊萬年人也。初平中,為公府令史,以牒除為新鄭長。興平末,人多饑窮,沛課民益畜乾椹,收䝁豆,閱其有餘以補不足,如此積得千餘斛,藏在小倉。會太祖為兖州刺史,西迎天子,所將千餘人皆無糧。過新鄭,沛謁見,乃皆進乾椹。太祖甚喜。及太祖輔政,遷沛為長社令。時曹洪賔客在縣界,徵調不肯如法,沛先撾折其脚,遂殺之。由此太祖以為能。累遷九江、東平、樂安太守,並有治迹。坐與督軍爭鬬,髠刑五歲。輸作未竟,會太祖出征在譙,聞鄴下頗不奉科禁,乃發教選鄴令,當得嚴能如楊沛比,故沛從徒中起為鄴令。已拜,太祖見之,問曰:「以何治鄴?」沛曰:「竭盡心力,奉宣科法。」太祖曰:「善。」顧謂坐席曰:「諸君,此可畏也。」賜其生口十人,絹百匹,旣欲以勵之,且以報乾椹也。沛辭去,未到鄴,而軍中豪右曹洪、劉勳等畏沛,各遣家馳騎告子弟,使各自檢勑。沛為令數年,以公能轉為護羌都尉。十六年,馬超反,大軍西討,沛隨軍,都督孟津渡事。太祖已南過,其餘未畢,而中黃門前渡,忘持行軒,私北還取之,從吏求小舩,欲獨先渡。吏呵不肯,黃門與吏爭言。沛問黃門:「有疏邪?」黃門云:「無疏。」沛怒曰:「何知汝不欲逃邪?」遂使人捽其頭,與杖欲捶之,而逸得去,衣幘皆裂壞,自訴於太祖。太祖曰:「汝不死為幸矣。」由是聲名益振。及關中破,代張旣領京兆尹。黃初中,儒雅並進,而沛本以事能見用,遂以議郎宂散里巷。沛前後宰歷城守,不以私計介意,又不肯以事貴人,故身退之後,家無餘積。治疾於家,借舍從兒,無他奴婢。後占河南夕陽亭部荒田二頃,起瓜牛廬,居止其中,其妻子凍餓。沛病亡,鄉人親友及故吏民為殯葬也。

評曰:自漢季以來,刺史總統諸郡,賦政于外,非若曩時司察之而已。太祖創基,迄終魏業,此皆其流稱譽有名實者也。咸精達事機,威恩兼著,故能肅齊萬里,見述于後也。


\end{pinyinscope}