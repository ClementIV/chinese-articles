\article{zhang-xiu-chuan}

\begin{pinyinscope}
張繡,武威祖厲人,驃騎將軍濟族子也。邊章、韓遂為亂涼州,金城麴勝襲殺祖厲長劉儁。繡為縣吏,閒伺殺勝,郡內義之。遂招合少年,為邑中豪傑。董卓敗,濟與李傕等擊呂布,為卓報仇。語在卓傳。繡隨濟,以軍功稍遷至建忠將軍,封宣威侯。濟屯弘農,士卒饑餓,南攻穰,為流矢所中死。繡領其衆,屯宛,與劉表合。太祖南征,軍淯水,繡等舉衆降。太祖納濟妻,繡恨之。太祖聞其不恱,密有殺繡之計。計漏,繡掩襲太祖。太祖軍敗,二子沒。繡還保穰,

傅子曰:繡有所親胡車兒,勇冠其軍。太祖愛其驍健,手以金與之。繡聞而疑太祖欲因左右刺之,遂反。吳書曰:繡降,用賈詡計,乞徙軍就高道,道由太祖屯中。繡又曰:「車少而重,乞得使兵各被甲。」太祖信繡,皆聽之。繡乃嚴兵入屯,掩太祖。太祖不備,故敗。太祖比年攻之,不克。太祖拒袁紹於官渡,繡從賈詡計,復以衆降。語在詡傳。繡至,太祖執其手,與歡宴,為子均取繡女,拜揚武將軍。官渡之役,繡力戰有功,遷破羌將軍。從破袁譚於南皮,復增邑凡二千戶。是時天下戶口減耗,十裁一在,諸將封未有滿千戶者,而繡特多。從征烏丸於柳城,未至,薨,諡曰定侯。魏略曰:五官將數因請會,發怒曰:「君殺吾兄,何忍持面視人邪!」繡心不自安,乃自殺。子泉嗣,坐與魏諷謀反,誅,國除。


\end{pinyinscope}