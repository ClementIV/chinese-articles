\article{李恢傳}

\begin{pinyinscope}
李恢字德昂,建寧俞元人也。仕郡督郵,姑夫爨習為建伶令,有違犯之事,恢坐習免官。太守董和以習方土大姓,寢而不許。

華陽國志曰:習後官至領軍。後貢恢於州,涉道未至,聞先主自葭萌還攻劉璋。恢知璋之必敗,先主必成也,乃託名郡使,北詣先主,遇於緜竹。先主嘉之,從至雒城,遣恢至漢中交好馬超,超遂從命。成都旣定,先主領益州牧,以恢為功曹書佐主簿。後為亡虜所誣,引恢謀反,有司執送,先主明其不然,更遷恢為別駕從事。章武元年,庲降都督鄧方卒,先主問恢:「誰可代者?」恢對曰:「人之才能,各有長短,故孔子曰『其使人也器之』。且夫明主在上,則臣下盡情,是以先零之役,趙充國曰『莫若老臣』。臣竊不自量,惟陛下察之。」先主笑曰:「孤之本意,亦已在卿矣。」遂以恢為庲降都督,使持節領交州剌史,住平夷縣。臣松之訊之蜀人,云庲降地名,去蜀二千餘里,時未有寧州,號為南中,立此職以總攝之。晉泰始中,始分為寧州。

先主薨,高定恣睢於越嶲,雍闓跋扈於建寧,朱襃反叛於䍧牱。丞相亮南征,先由越嶲,而恢案道向建寧。諸縣大相糾合,圍恢軍於昆明。時恢衆少敵倍,又未得亮聲息,紿謂南人曰:「官軍糧盡,欲規退還,吾中間乆斥鄉里,乃今得旋,不能復北,欲還與汝等同計謀,故以誠相告。」南人信之,故圍守怠緩。於是恢出擊,大破之,追犇逐北,南至槃江,東接䍧牱,與亮聲勢相連。南土平定,恢軍功居多,封漢興亭侯,加安漢將軍。後軍還,南夷復叛,殺害守將。恢身往撲討,鉏盡惡類,徙其豪帥于成都,賦出叟、濮耕牛戰馬金銀犀革,充繼軍資,于時費用不乏。

建興七年,以交州屬吳,解恢刺史。更領建寧太守,以還居本郡。徙居漢中,九年卒。子遺嗣。恢弟子球,羽林右部督,隨諸葛瞻拒鄧艾,臨陣授命,死于緜竹。


\end{pinyinscope}