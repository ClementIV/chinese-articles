\article{孫禮傳}

\begin{pinyinscope}
孫禮字德達,涿郡容城人也。太祖平幽州,召為司空軍謀掾。初喪亂時,禮與母相失,同郡馬台求得禮母,禮推家財盡以與台。台後坐法當死,禮私導令踰獄自首,旣而曰:「臣無逃亡之義。」徑詣刺姧主簿溫恢。恢嘉之,具白太祖,各減死一等。

後除河間郡丞,稍遷熒陽都尉。魯山中賊數百人,保固險阻,為民作害;乃徙禮為魯相。禮至官,出俸穀,發吏民,募首級,招納降附,使還為閒,應時平泰。歷山陽、平原、平昌、琅邪太守。從大司馬曹休征吳於夾石,禮諫以為不可深入,不從而敗。遷陽平太守,入為尚書。

明帝方脩宮室,而節氣不和,天下少穀。禮固爭,罷役,詔曰:「敬納讜言,促遣民作。」時李惠監作,復奏留一月,有所成訖。禮徑至作所,不復重奏,稱詔罷民,帝奇其意而不責也。

帝獵於大石山,虎趨乘輿,禮便投鞭下馬,欲奮劒斫虎,詔令禮上馬。明帝臨崩之時,以曹爽為大將軍,宜得良佐,於牀下受遺詔,拜禮大將軍長史,加散騎常侍。禮亮直不撓,爽弗便也,以為揚州刺史,加伏波將軍,賜爵關內侯。吳大將全琮帥數萬衆來侵寇,時州兵休使,在者無幾。禮躬勒衞兵禦之,戰於芍陂,自旦及暮,將士死傷過半。禮犯蹈白刃,馬被數創,手秉枹鼓,奮不顧身,賊衆乃退。詔書慰勞,賜絹七百匹。禮為死事者設祀哭臨,哀號發心,皆以絹付亡者家,無以入身。

徵拜少府,出為荊州刺史,遷兾州牧。太傅司馬宣王謂禮曰:「今清河、平原爭界八年,更二刺史,靡能決之;虞、芮待文王而了,宜善令分明。」禮曰:「訟者據墟墓為驗,聽者以先老為正,而老者不可加以榎楚,又墟墓或遷就高敞,或徙避仇讎。如今所聞,雖臯陶猶將為難。若欲使必也無訟,當以烈祖初封平原時圖決之。何必推古問故,以益辭訟?昔成王以桐葉戲叔虞,周公便以封之。今圖藏在天府,便可於坐上斷也,豈待到州乎?」宣王曰:「是也。當別下圖。」禮到,案圖宜屬平原。而曹爽信清河言,下書云:「圖不可用,當參異同。」禮上疏曰:「管仲霸者之佐,其器又小,猶能奪伯氏駢邑,使沒齒無怨言。臣受牧伯之任,奉聖朝明圖,驗地著之界,界實以王翁河為限;而鄃以馬丹候為驗,詐以鳴犢河為界。假虛訟訴,疑誤臺閣。竊聞衆口鑠金,浮石沈木,三人成巿虎,慈母投其杼。今二郡爭界八年,一朝決之者,緣有解書圖畫,可得尋案擿校也。平原在兩河,向東上,其閒有爵隄,爵隄在高唐西南,所爭地在高唐西北,相去二十餘里,可謂長歎息流涕者也。案解與圖奏而鄃不受詔,此臣軟弱不勝其任,臣亦何顏尸祿素餐。」輒束帶著履,駕車待放。爽見禮奏,大怒。劾禮怨望,結刑五歲。在家期年,衆人多以為言,除城門校尉。

時匈奴王劉靖部衆彊盛,而鮮卑數寇邊,乃以禮為并州刺史,加振武將軍,使持節,護匈奴中郎將。往見太傅司馬宣王,有忿色而無言。宣王曰:「卿得并州,少邪?恚理分界失分乎?今當遠別,何不懽也!」禮曰:「何明公言之乖細也!禮雖不德,豈以官位往事為意邪?本謂明公齊蹤伊、呂,匡輔魏室,上報明帝之託,下建萬世之勳。今社稷將危,天下兇兇,此禮之所以不恱也。」因涕泣橫流。宣王曰:「且止,忍不可忍。」爽誅後,入為司隷校尉,凡臨七郡五州,皆有威信。遷司空,封大利亭侯,邑一百戶。禮與盧毓同郡時輩,而情好不睦。為人雖互有長短,然名位略齊云。嘉平二年薨,謚曰景侯。孫元嗣。


\end{pinyinscope}