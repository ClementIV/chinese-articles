\article{劉封傳}

\begin{pinyinscope}
劉封者,本羅侯寇氏之子,長沙劉氏之甥也。先主至荊州,以未有繼嗣,養封為子。及先主入蜀,自葭萌還攻劉璋,時封年二十餘,有武藝,氣力過人,將兵俱與諸葛亮、張飛等泝流西上,所在戰克。益州旣定,以封為副軍中郎將。

初,劉璋遣扶風孟達副法正,各將兵二千人,使迎先主,先主因令達并領其衆,留屯江陵。蜀平後,以達為宜都太守。建安二十四年,命達從秭歸北攻房陵,房陵太守蒯祺為達兵所害。達將進攻上庸,先主陰恐達難獨任,乃遣封自漢中乘沔水下統達軍,與達會上庸。上庸太守申耽舉衆降,遣妻子及宗族詣成都。先主加耽征北將軍,領上庸太守員鄉侯如故,以耽弟儀為建信將軍、西城太守,遷封為副軍將軍。自關羽圍樊城、襄陽,連呼封、達,令發兵自助。封、達辭以山郡初附,未可動搖,不承羽命。會羽覆敗,先主恨之。又封與達忿爭不和,封尋奪達鼔吹。達旣懼罪,又忿恚封,遂表辭先主,率所領降魏。

魏略載達辭先主表曰:「伏惟殿下將建伊、呂之業,追桓、文之功,大事草創,假勢吳、楚,是以有為之士深覩歸趣。臣委質已來,愆戾山積,臣猶自知,況於君乎!今王朝以興,英俊鱗集,臣內無輔佐之器,外無將領之才,列次功臣,誠自愧也。臣聞范蠡識微,浮於五湖;咎犯謝罪,逡巡於河上。夫際會之間,請命乞身。何則?欲絜去就之分也。況臣卑鄙,無元功巨勳,自繫於時,竊慕前賢,早思遠恥。昔申生至孝見疑於親,子胥至忠見誅於君,蒙恬拓境而被大刑,樂毅破齊而遭讒佞,臣每讀其書,未甞不慷慨流涕,而親當其事,益以傷絕。何者?荊州覆敗,大臣失節,百無一還。惟臣尋事,自致房陵、上庸,而復乞身,自放於外。伏想殿下聖恩感悟,愍臣之心,悼臣之舉。臣誠小人,不能始終,知而為之,敢謂非罪!臣每閒交絕無惡聲,去臣無怨辭,臣過奉教於君子,願君王勉之也。」魏文帝善達之姿才容觀,以為散騎常侍、建武將軍,封平陽亭侯。合房陵、上庸、西城三郡為新城郡,以達領新城太守。遣征南將軍夏侯尚、右將軍徐晃與達共襲封。達與封書曰:

古人有言:『疏不間親,新不加舊。』此謂上明下直,讒慝不行也。若乃權君譎主,賢父慈親,猶有忠臣蹈功以罹禍,孝子抱仁以陷難,種、商、白起、孝己、伯奇,皆其類也。其所以然,非骨肉好離,親親樂患也。或有恩移愛易,亦有讒間其間,雖忠臣不能移之於君,孝子不能變之於父者也。勢利所加,改親為讎,況非親親乎!故申生、衞伋、禦寇、楚建禀受形之氣,當嗣立之正,而猶如此。今足下與漢中王,道路之人耳,親非骨血而據勢權,義非君臣而處上位,征則有偏任之威,居則有副軍之號,遠近所聞也。自立阿斗為太子已來,有識之人相為寒心。如使申生從子輿之言,必為太伯;衞伋聽其弟之謀,無彰父之譏也。且小白出奔,入而為霸;重耳踰垣,卒以克復。自古有之,非獨今也。

夫智貴免禍,明尚夙達,僕揆漢中王慮定於內,疑生於外矣;慮定則心固,疑生則心懼,亂禍之興作,未曾不由廢立之間也。私怨人情,不能不見,恐左右必有以間於漢中王矣。然則疑成怨聞,其發若踐機耳。今足下在遠,尚可假息一時;若大軍遂進,足下失據而還,竊相為危之。昔微子去殷,智果別族,違難背禍,猶皆如斯。國語曰:智宣子將以瑤為後,智果曰:「不如霄也。」宣子曰:「霄也佷。」對曰:「霄也佷在面,瑤之賢於人者五,其不逮者一也。美鬚長大則賢,射御足力則賢,技藝異俗則賢,巧文辯惠則賢,彊毅果敢則賢,如是而甚不仁;以五者賢陵人,而不仁行之,其誰能待之!若果立瑤也。智宗必滅。」不聽。智果別族于太史氏為輔氏。及智氏亡,惟輔果在焉。今足下棄父母而為人後,非禮也;知禍將至而留之,非智也;見正不從而疑之,非義也。自號為丈夫,為此三者,何所貴乎?以足下之才,棄身來東,繼嗣羅侯,不為背親也;北面事君,以正綱紀,不為棄舊也;怒不致亂,以免危亡,不為徒行也。加陛下新受禪命,虛心側席,以德懷遠,若足下翻然內向,非但與僕為倫,受三百戶封,繼統羅國而已,當更剖符大邦,為始封之君。陛下大軍,金鼔以震,當轉都宛、鄧;若二敵不平,軍無還期。足下宜因此時早定良計。易有『利見大人』,詩有『自求多福』,行矣。今足下勉之,無使狐突閉門不出。

封不從達言。

申儀叛封,封破走還成都。申耽降魏,魏假耽懷集將軍,徙居南陽,儀魏興太守,封真鄉侯,屯洵口。魏略曰:申儀兄名耽,字義舉。初在西平、上庸間聚衆數千家,後與張魯通,又遣使詣曹公,曹公加其號為將軍,因使領上庸都尉。至建安末,為蜀所攻,以其郡西屬。黃初中,儀復來還,詔即以兄故號加儀,因拜魏興太守,封列侯。太和中,儀與孟達不和,數上言達有貳心於蜀,及達反,儀絕蜀道,使救不到。達死後,儀詣宛見司馬宣王,宣王勸使來朝。儀至京師,詔轉儀拜樓舡將軍,在禮請中。封旣至,先主責封之侵陵達,又不救羽。諸葛亮慮封剛猛,易世之後終難制御,勸先主因此除之。於是賜封死,使自裁。封嘆曰:「恨不用孟子度之言!」先主為之流涕。達本字子敬,避先主叔父敬,改之。封子林為牙門將,咸熙元年內移河東。達子興為議督軍,是歲徙還扶風。


\end{pinyinscope}