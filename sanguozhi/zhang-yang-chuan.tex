\article{zhang-yang-chuan}

\begin{pinyinscope}
張楊字稚叔,雲中人也。以武勇給并州,為武猛從事。靈帝末,天下亂,帝以所寵小黃門蹇碩為西園上軍校尉,軍京都,欲以御四方,徵天下豪傑以為偏裨。太祖及袁紹等皆為校尉,屬之。

靈帝紀曰:以虎賁中郎將袁紹為中軍校尉,屯騎校尉鮑鴻為下軍校尉,議郎曹操為典軍校尉,趙融、馮芳為助軍校尉,夏牟、淳于瓊為左右校尉。并州刺史丁原遣楊將兵詣碩,為假司馬。靈帝崩,碩為何進所殺。楊復為進所遣,歸本州募兵,得千餘人,因留上黨,擊山賊。進敗,董卓作亂。楊遂以所將攻上黨太守於壺關,不下,略諸縣,衆至數千人。山東兵起,欲誅卓。袁紹至河內,楊與紹合,復與匈奴單于於夫羅屯漳水。單于欲叛,紹、楊不從。單于執楊與俱去,紹使將麴義追擊於鄴南,破之。單于執楊至黎陽,攻破渡遼將軍耿祉軍,衆復振。卓以楊為建義將軍、河內太守。天子之在河東,楊將兵控安邑,拜安國將軍,封晉陽侯。楊欲迎天子還洛,諸將不聽;楊還野王。建安元年,楊奉、董承、韓暹挾天子還舊京,糧乏。楊以糧迎道路,遂至洛陽。謂諸將曰:「天子當與天下共之,幸有公卿大臣,楊當捍外難,何事京都?」遂還野王。即拜為大司馬。英雄記曰:楊性仁和,無威刑。下人謀反,發覺,對之涕泣,輒原不問。楊素與呂布善。太祖之圍布,楊欲救之,不能。乃出兵東市,遙為之勢。其將楊醜殺楊以應太祖,楊將眭固殺醜,將其衆,欲北合袁紹。太祖遣史渙邀擊,破之於犬城,斬固,盡收其衆也。典略曰:固字白兎,旣殺楊醜,軍屯射犬。時有巫誡固曰:「將軍字兎而此邑名犬,兎見犬,其勢必驚,宜急移去。」固不從,遂戰死。


\end{pinyinscope}