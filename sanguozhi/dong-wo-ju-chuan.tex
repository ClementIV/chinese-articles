\article{東沃沮傳}

\begin{pinyinscope}
東沃沮在高句麗蓋馬大山之東,濵大海而居。其地形東北狹,西南長,可千里,北與挹婁、夫餘,南與濊貊接。戶五千,無大君王,世世邑落,各有長帥。其言語與句麗大同,時時小異。漢初,燕亡人衞滿王朝鮮,時沃沮皆屬焉。漢武帝元封二年,伐朝鮮,殺滿孫右渠,分其地為四郡,以沃沮城為玄菟郡。後為夷貊所侵,徙郡句麗西北,今所謂玄菟故府是也。沃沮還屬樂浪。漢以土地廣遠,在單單大領之東,分置東部都尉,治不耐城,別主領東七縣,時沃沮亦皆為縣。漢光武六年,省邊郡,都尉由此罷。其後皆以其縣中渠帥為縣侯,不耐、華麗、沃沮諸縣皆為侯國。夷狄更相攻伐,唯不耐濊侯至今猶置功曹、主簿諸曹,皆濊民作之。沃沮諸邑落渠帥,皆自稱三老,則故縣國之制也。國小,迫於大國之間,遂臣屬句麗。句麗復置其中大人為使者,使相主領,又使大加統責其租稅,貊布、魚、鹽、海中食物,千里擔負致之,又送其美女以為婢妾,遇之如奴僕。

其土地肥美,背山向海,宜五糓,善田種。人性質直彊勇,少牛馬,便持矛步戰。食飲居處,衣服禮節,有似句麗。

魏畧曰:其嫁娶之法,女年十歲,已相設許。壻家迎之,長養以為婦。至成人,更還女家。女家責錢,錢畢,乃復還壻。其葬作大木槨,長十餘丈,開一頭作戶。新死者皆假埋之,才使覆形,皮肉盡,乃取骨置槨中。舉家皆共一槨,刻木如生形,隨死者為數。又有瓦䥶,置米其中,編縣之於槨戶邊。

毌丘儉討句麗,句麗王宮奔沃沮,遂進師擊之。沃沮邑落皆破之,斬獲首虜三千餘級,宮奔北沃沮。北沃沮一名置溝婁,去南沃沮八百餘里,其俗南北皆同,與挹婁接。挹婁喜乘船寇鈔,北沃沮畏之,夏月恒在山巖深穴中為守備,冬月氷凍,船道不通,乃下居村落。王頎別遣追討宮,盡其東界。問其耆老「海東復有人不」?耆老言國人嘗乘船捕魚,遭風見吹數十日,東得一島,上有人,言語不相曉,其俗常以七月取童女沈海。又言有一國亦在海中,純女無男。又說得一布衣,從海中浮出,其身如中國人衣,其兩袖長三丈。又得一破船,隨波出在海岸邊,有一人項中復有面,生得之,與語不相通,不食而死。其域皆在沃沮東大海中。


\end{pinyinscope}