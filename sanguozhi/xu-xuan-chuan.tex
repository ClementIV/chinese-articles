\article{徐宣傳}

\begin{pinyinscope}
徐宣字寶堅,廣陵海西人也。避亂江東,又辭孫策之命,還本郡。與陳矯並為綱紀,二人齊名而私好不恊,然俱見器於太守陳登,與登並心於太祖。海西、淮浦二縣民作亂,都尉衞彌、令梁習夜奔宣家,密送免之。太祖遣督軍扈質來討賊,以兵少不進。宣潛見責之,示以形勢,質乃進破賊。太祖辟為司空掾屬,除東緡、發干令,遷齊郡太守,入為門下督,從到壽春。會馬超作亂,大軍西征,太祖見官屬曰:「今當遠征,而此方未定,以為後憂,宜得清公大德以鎮統之。」乃以宣為左護軍,留統諸軍。還,為丞相東曹掾,出為魏郡太守。太祖崩洛陽,羣臣入殿中發哀。或言可易諸城守,用譙、沛人。宣厲聲曰:「今者遠近一統,人懷效節,何必譙、沛,而沮宿衞者心。」文帝聞曰:「所謂社稷之臣也。」帝旣踐阼,為御史中丞,賜爵關內侯,徙城門校尉,旬月遷司隷校尉,轉散騎常侍。從至廣陵,六軍乘舟,風浪暴起,帝船囬倒,宣病在後,陵波而前,羣寮莫先至者。帝壯之,遷尚書。

明帝即位,封津陽亭侯,邑二百戶。中領軍桓範薦宣曰:「臣聞帝王用人,度世授才,爭奪之時,以策略為先,分定之後,以忠義為首。故晉文行舅犯之計而賞雍季之言,

呂氏春秋曰:昔晉文公將與楚人戰於城濮,召咎犯而問曰:「楚衆我寡,柰何而可?」咎犯對曰:「臣聞繁禮之君,不足於文,繁戰之君,不足於詐,君亦詐之而已。」文公以咎犯言告雍季,雍季曰:「竭澤而漁,豈不得魚,而明年無魚。焚藪而田,豈不得獸,而明年無獸。詐偽之道,雖今偷可,後將無復,非長術也。」文公用咎犯之言,而敗楚人於城濮。反而為賞,雍季在上。左右諫曰:「城濮之功,咎犯之謀也。君用其言而後其身,或者不可乎!」文公曰:「雍季之言,百代之利也;咎犯之言,一時之務也。焉有以一時之務,先百代之利乎?」高祖用陳平之智而託後於周勃也。竊見尚書徐宣,體忠厚之行,秉直亮之性;清雅特立,不拘世俗;確然難動,有社稷之節;歷位州郡,所在稱職。今僕射缺,宣行掌後事;腹心任重,莫宜宣者。」帝遂以宣為左僕射,後加侍中光祿大夫。車駕幸許昌,總統留事。帝還,主者奏呈文書。詔曰:「吾省與僕射何異?」竟不視。尚方令坐猥見考竟,宣上疏陳威刑太過,又諫作宮殿窮盡民力,帝皆手詔嘉納。宣曰:「七十有縣車之禮,今已六十八,可以去矣。」乃固辭疾遜位,帝終不許。青龍四年薨,遺令布衣疏巾,歛以時服。詔曰:「宣體履至實,直內方外,歷在三朝,公亮正色,有託孤寄命之節,可謂柱石臣也。常欲倚以台輔,未及登之,惜乎大命不永!其追贈車騎將軍,葬如公禮。」謚曰貞侯。子欽嗣。


\end{pinyinscope}