\article{徐邈傳}

\begin{pinyinscope}
徐邈字景山,燕國薊人也。太祖平河朔,召為丞相軍謀掾,試守奉高令,入為東曹議令史。魏國初建,為尚書郎。時科禁酒,而邈私飲至於沈醉。校事趙達問以曹事,邈曰:「中聖人。」達白之太祖,太祖甚怒。度遼將軍鮮于輔進曰:「平日醉客謂酒清者為聖人,濁者為賢人,邈性脩慎,偶醉言耳。」竟坐得免刑。後領隴西太守,轉為南安。文帝踐阼,歷譙相,平陽、安平太守,潁川典農中郎將,所在著稱,賜爵關內侯。車駕幸許昌,問邈曰:「頗復中聖人不?」邈對曰:「昔子反斃於穀陽,御叔罰於飲酒,臣嗜同二子,不能自懲,時復中之。然宿瘤以醜見傳,而臣以醉見識。」帝大笑,顧左右曰:「名不虛立。」遷撫軍大將軍軍師。

明帝以涼州絕遠,南接蜀寇,以邈為涼州刺史,使持節領護羌校尉。至,值諸葛亮出祁山,隴右三郡反,邈輙遣參軍及金城太守等擊南安賊,破之。河右少雨,常苦乏糓,邈上脩武威、酒泉鹽池以收虜糓,又廣開水田,募貧民佃之,家家豐足,倉庫盈溢。乃支度州界軍用之餘,以市金帛犬馬,通供中國之費。以漸收斂民間私杖,藏之府庫。然後率以仁義,立學明訓,禁厚葬,斷淫祀,進善黜惡,風化大行,百姓歸心焉。西域流通,荒戎入貢,皆邈勳也。討叛羌柯吾有功,封都亭侯,邑三百戶,加建威將車。邈與羌、胡從事,不問小過;若犯大罪,先告部帥,使知,應死者乃斬以徇,是以信服畏威。賞賜皆散與將士,無入家者,妻子衣食不充;天子聞而嘉之,隨時供給其家。彈邪繩枉,州界肅清。

正始元年,還為大司農。遷為司隷校尉,百僚敬憚之。公事去官。後為光祿大夫,數歲即拜司空,邈歎曰:「三公論道之官,無其人則缺,豈可以老病忝之哉?」遂固辭不受。嘉平元年,年七十八,以大夫薨于家,用公禮葬,謚曰穆侯。子武嗣。六年,朝廷追思清節之士,詔曰:「夫顯賢表德,聖王所重;舉善而教,仲尼所美。故司空徐邈、征東將軍胡質、衞尉田豫皆服職前朝,歷事四世,出統戎馬,入贊庶政,忠清在公,憂國忘私,不營產業,身沒之後,家無餘財,朕甚嘉之。其賜邈等家糓二千斛,錢三十萬,布告天下。」

邈同郡韓觀曼遊,有鑒識器幹,與邈齊名,而在孫禮、盧毓先,為豫州刺史,甚有治功,卒官。

魏名臣奏載黃門侍郎杜恕表,稱:「韓觀、王昶,信有兼才,高官重任,不但三州。」盧欽著書,稱邈曰:「徐公志高行絜,才博氣猛。其施之也,高而不狷,絜而不介,博而守約,猛而能寬。聖人以清為難,而徐公之所易也。」或問欽:「徐公當武帝之時,人以為通,自在涼州及還京師,人以為介,何也?」欽荅曰;「往者毛孝先、崔季珪等用事,貴清素之士,于時皆變易車服以求名高,而徐公不改其常,故人以為通。比來天下奢靡,轉相倣效,而徐公雅尚自若,不與俗同,故前日之通,乃今日之介也。是世人之無常,而徐公之有常也。」


\end{pinyinscope}