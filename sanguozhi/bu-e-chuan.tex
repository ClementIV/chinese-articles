\article{步隲傳}

\begin{pinyinscope}
步隲字子山,臨淮淮陰人也。

吳書曰:晉有大夫揚食采於步,後有步叔,與七十子師事仲尼。秦漢之際有為將軍者,以功封淮陰侯,隲其後也。世亂,避難江東,單身窮困,與廣陵衞旌同年相善,俱以種瓜自給,晝勤四體,夜誦經傳。吳書曰:隲博研道藝,靡不貫覽,性寬雅沈深,能降志辱身。

會稽焦征羌,郡之豪族,吳錄曰:征羌名矯,嘗為征羌令。人客放縱。隲與旌求食其地,懼為所侵,乃共脩刺奉瓜,以獻征羌。征羌方在內卧,駐之移時,旌欲委去,隲止之曰:「本所以來,畏其彊也;而今舍去,欲以為高,祗結怨耳。」良乆,征羌開牖見之,身隱几坐帳中,設席致地,坐隲、旌於牖外,旌愈恥之,隲辭色自若。征羌作食,身享大案,殽膳重沓,以小盤飯與隲、旌,惟菜茹而已。旌不能食,隲極飯致飽乃辭出。旌怒隲曰:「何能忍此?」隲曰:「吾等貧賤,是以主人以貧賤遇之,固其宜也,當何所恥?」吳錄曰:衞旌字子旗,官至尚書。

孫權為討虜將軍,召隲為主記,吳書曰:歲餘,隲以疾免,與琅邪諸葛瑾、彭城嚴畯俱游吳中,並著聲名,為當時英俊。除海鹽長,還辟車騎將軍東曹掾。吳書曰:權為徐州牧,以隲為治中從事,舉茂才。建安十五年,出領鄱陽太守。歲中,徙交州刺史、立武中郎將,領武射吏千人,便道南行。明年,追拜使持節、征南中郎將。劉表所置蒼梧太守吳巨陰懷異心,外附內違。隲降意懷誘,請與相見,因斬徇之,威聲大震。士燮兄弟,相率供命,南土之賔,自此始也。益州大姓雍闓等殺蜀所署太守正昂,與燮相聞,求欲內附。隲因承制遣使宣恩撫納,由是加拜平戎將軍,封廣信侯。

延康元年,權遣呂岱代隲,隲將交州義士萬人出長沙。會劉備東下,武陵蠻夷蠢動,權遂命隲上益陽。備旣敗績,而零、桂諸郡猶相驚擾,處處阻兵;隲周旋征討,皆平之。黃武二年,遷右將軍左護軍,改封臨湘侯。五年,假節,徙屯漚口。

權稱尊號,拜驃騎將軍,領兾州牧。是歲,都督西陵,代陸遜撫二境,頃以兾州在蜀分,解牧職。時權太子登駐武昌,愛人好善,與隲書曰:「夫賢人君子,所以興隆大化,佐理時務者也。受性闇蔽,不達道數,雖實驅驅欲盡心於明德,歸分於君子,至於遠近士人,先後之宜,猶或緬焉,未之能詳。傳曰:『愛之能勿勞乎?忠焉能勿誨乎?』斯其義也,豈非所望於君子哉!」隲於是條于時事業在荊州界者,諸葛瑾、陸遜、朱然、程普、潘濬、裴玄、夏侯承、衞旌、李肅、吳書曰:肅字偉恭,南陽人。少以才聞,善論議,臧否得中,甄奇錄異,薦述後進,題目品藻,曲有條貫,衆人以此服之。權擢以為選舉,號為得才。求出補吏,為桂陽太守,吏民恱服。徵為卿。會卒,知與不知,並痛惜焉。周條、石幹十一人,甄別行狀,因上疏獎勸曰:「臣聞人君不親小事,百官有司各任其職。故舜命九賢,則無所用心,彈五絃之琴,詠南風之詩,不下堂廟而天下治也。齊桓用管仲,被髮載車,齊國旣治,又致匡合。近漢高祖擥三傑以興帝業,西楚失雄俊以喪成功。汲黯在朝,淮南寢謀;郅都守邊,匈奴竄迹。故賢人所在,折衝萬里,信國家之利器,崇替之所由也。方今王化未被於漢北,河、洛之濵尚有僭逆之醜,誠擥英雄拔俊任賢之時也。願明太子重以輕意,則天下幸甚。」

後中書呂壹典校文書,多所糾舉,隲上疏曰:「伏聞諸典校擿抉細微,吹毛求瑕,重案深誣,輒欲陷人以成威福;無罪無辜,橫受大刑,是以使民跼天蹐地,誰不戰慄?昔之獄官,惟賢是任,故皐陶作士,呂侯贖刑,張、于廷尉,民無冤枉,休泰之祚,實由此興。今之小臣,動與古異,獄以賄成,輕忽人命,歸咎于上,為國速怨。夫一人吁嗟,王道為虧,甚可仇疾。明德慎罰,哲人惟刑,書傳所美。自今蔽獄,都下則宜諮顧雍,武昌則陸遜、潘濬,平心專意,務在得情,隲黨神明,受罪何恨?」又曰:「天子父天母地,故宮室百官,動法列宿。若施政令,欽順時節,官得其人,則陰陽和平,七曜循度。至於今日,官寮多闕,雖有大臣,復不信任,如此天地焉得無變?故頻年枯旱,亢陽之應也。又嘉禾六年五月十四日,赤烏二年正月一日及二十七日,地皆震動。地陰類,臣之象,陰氣盛故動,臣下專政之故也。夫天地見異,所以警悟人主,可不深思其意哉!」又曰:「丞相顧雍、上大將軍陸遜、太常潘濬,憂深責重,志在謁誠,夙夜兢兢,寢食不寧,念欲安國利民,建乆長之計,可謂心膂股肱,社稷之臣矣。宜各委任,不使他官監其所司,責其成效,課其負殿。此三臣者,思慮不到則已,豈敢專擅威福欺負所天乎?」又曰:「縣賞以顯善,設刑以威姧,任賢而使能,審明於法術,則何功而不成,何事而不辨,何聽而不聞,何視而不覩哉?若今郡守百里,皆各得其人,共相經緯,如是,庶政豈不康哉?竊聞諸縣並有備吏,吏多民煩,俗以之弊。但小人因緣銜命,不務奉公而作威福,無益視聽,更為民害,愚以為可一切罷省。」權亦覺梧,遂誅呂壹。隲前後薦達屈滯,救解患難,書數十上。權雖不能悉納,然時采其言,多蒙濟賴。吳錄云:隲表言曰:「北降人王潛等說,此相部伍,圖以東向,多作布囊,欲以盛沙塞江,以大向荊州。夫備不豫設,難以應卒,宜為之防。」權曰:「此曹衰弱,何能有圖?必不敢來。若不如孤言,當以牛千頭,為君作主人。」後有呂範、諸葛恪為說隲所言,云:「每讀步隲表,輒失笑。此江與開闢俱生,寧有可以沙囊塞理也!」

赤烏九年,代陸遜為丞相,猶誨育門生,手不釋書,被服居處有如儒生。然門內妻妾服飾奢綺,頗以此見譏。在西陵二十年,鄰敵敬其威信。性寬弘得衆,喜怒不形於聲色,而外內肅然。

十一年卒,子恊嗣,統隲所領,加撫軍將軍。恊卒,子璣嗣侯。恊弟闡,繼業為西陵督,加昭武將軍,封西亭侯。鳳皇元年,召為繞帳督。闡累世在西陵,卒被徵命,自以失職,又懼有讒禍,於是據城降晉。遣璣與弟璿詣洛陽為任,晉以闡為都督西陵諸軍事、衞將軍、儀同三司,加侍中,假節領交州牧,封宜都公;璣監江陵諸軍事、左將軍,加散騎常侍,領廬陵太守,改封江陵侯;璿給事中、宣威將軍,封都鄉侯。命車騎將軍羊祜、荊州刺史楊肇往赴救闡。孫皓使陸抗西行,祜等遁退。抗陷城,斬闡等,步氏泯滅,惟璿紹祀。

潁川周昭著書稱步隲及嚴畯等曰:「古今賢士大夫所以失名喪身傾家害國者,其由非一也,然要其大歸,總其常患,四者而已。急而論議一也,爭名勢二也,重朋黨三也,務欲速四也。急論議則傷人,爭名勢則敗友,重朋黨則蔽主,務欲速則失德,此四者不除,未有能全也。當世君子能不然者,亦比有之,豈獨古人乎!然論其絕異,未若顧豫章、諸葛使君、步丞相、嚴衞尉、張奮威之為美也。論語言『夫子恂恂然善誘人』,又曰『成人之美,不成人之惡』,豫章有之矣。『望之儼然,即之也溫,聽其言也厲』,使君體之矣。『恭而安,威而不猛』,丞相履之矣。學不求祿,心無苟得,衞尉、奮威蹈之矣。此五君者,雖德實有差,輕重不同,至於趣舍大檢,不犯四者,俱一揆也。昔丁諝出於孤家,吾粲由於牧豎,豫章揚其善,以並陸、全之列,是以人無幽滯而風俗厚焉。使君、丞相、衞尉三君,昔以布衣俱相友善,諸論者因各叙其優劣。初,先衞尉,次丞相,而後有使君也;其後並事明主,經營世務,出處之才有不同,先後之名須反其初,此世常人所決勤薄也。至於三君分好,卒無虧損,豈非古人交哉!又魯橫江昔杖萬兵,屯據陸口,當世之美業也,能與不能,孰不願焉?而橫江旣亡,衞尉應其選,自以才非將帥,深辭固讓,終於不就。後徙九列,遷典八座,榮不足以自曜,祿不足以自奉。至於二君,皆位為上將,窮富極貴。衞尉旣無求欲,二君又不稱薦,各守所志,保其名好。孔子曰:『君子矜而不爭,羣而不黨。』斯有風矣。又奮威之名,亦三君之次也,當一方之戍,受上將之任,與使君、丞相不異也。然歷國事,論功勞,實有先後,故爵位之榮殊焉。而奮威將處此,決能明其部分,心無失道之欲,事無充詘之求,每升朝堂,循禮而動,辭氣謇謇,罔不惟忠。叔嗣雖親貴,言憂其敗,蔡文至雖疏賤,談稱其賢。女配太子,受禮若弔,慷愾之趨,惟篤人物,成敗得失,皆如所慮,可謂守道見機,好古之士也。若乃經國家,當軍旅,於馳騖之際,立霸王之功,此五者未為過人。至其純粹履道,求不苟得,升降當世,保全名行,邈然絕俗,實有所師。故粗論其事,以示後之君子。」周昭者字恭遠,與韋曜、薛瑩、華覈並述吳書,後為中書郎,坐事下獄,覈表救之,孫休不聽,遂伏法云。

評曰:張昭受遺輔佐,功勳克舉,忠謇方直,動不為己;而以嚴見憚,以高見外,旣不處宰相,又不登師保,從容閭巷,養老而已,以此明權之不及策也。顧雍依杖素業,而將之智局,故能究極榮位。諸葛瑾、步隲並以德度規檢見器當世,張承、顧邵虛心長者,好尚人物,周昭之論,稱之甚美,故辭錄焉。譚獻納在公,有忠貞之節。休、承脩志,咸庶為善。愛惡相攻,流播南裔,哀哉!


\end{pinyinscope}