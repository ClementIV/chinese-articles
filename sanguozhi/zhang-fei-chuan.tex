\article{張飛傳}

\begin{pinyinscope}
張飛字益德,涿郡人也,少與關羽俱事先主。羽年長數歲,飛兄事之。先主從曹公破呂布,隨還許,曹公拜飛為中郎將。先主背曹公依袁紹、劉表。表卒,曹公入荊州,先主奔江南。曹公追之,一日一夜,及於當陽之長阪。先主聞曹公卒至,棄妻子走,使飛將二十騎拒後。飛據水斷橋,瞋目橫矛曰:「身是張益德也,可來共決死!」敵皆無敢近者,故遂得免。

先主旣定江南,以飛為宜都太守、征虜將軍,封新亭侯,後轉在南郡。先主入益州,還攻劉璋,飛與諸葛亮等泝流而上,分定郡縣。至江州,破璋將巴郡太守嚴顏,生獲顏。飛呵顏曰:「大軍至,何以不降,而敢拒戰?」顏荅曰:「卿等無狀,侵奪我州,我州但有斷頭將軍,無有降將軍也。」飛怒,令左右牽去斫頭,顏色不變,曰:「斫頭便斫頭,何為怒邪!」飛壯而釋之,引為賔客。

華陽國志曰:初,先主入蜀,至巴郡,顏拊心歎曰:「此所謂獨坐窮山,放虎自衞也!」飛所過戰克,與先主會于成都。益州旣平,賜諸葛亮、法正、飛及關羽金各五百斤,銀千斤,錢五千萬,錦千匹,其餘頒賜各有差,以飛領巴西太守。

曹公破張魯,留夏侯淵、張郃守漢川。郃別督諸軍下巴西,欲徙其民於漢中,進軍宕渠、蒙頭、盪石,與飛相拒五十餘日。飛率精卒萬餘人,從他道邀郃軍交戰,山道迮狹,前後不得相救,飛遂破郃。郃棄馬緣山,獨與麾下十餘人從間道退,引軍還南鄭,巴土獲安。

先主為漢中王,拜飛為右將軍、假節。章武元年,遷車騎將軍,領司隷校尉,進封西鄉侯,策曰:「朕承天序,嗣奉洪業,除殘靖亂,未燭厥理。今寇虜作害,民被荼毒,思漢之士,延頸鶴望。朕用怛然,坐不安席,食不甘味,整軍誥誓,將行天罰。以君忠毅,侔蹤召、虎,名宣遐邇,故特顯命,高墉進爵,兼司于京。其誕將天威,柔服以德,伐叛以刑,稱朕意焉。詩不云乎,『匪疚匪棘,王國來極。肇敏戎功,用錫爾祉』。可不勉歟!」

初,飛雄壯威猛,亞於關羽,魏謀臣程昱等咸稱羽、飛萬人之敵也。羽善待卒伍而驕於士大夫,飛愛敬君子而不恤小人。先主常戒之曰:「卿刑殺旣過差,又日鞭檛健兒,而令在左右,此取禍之道也。」飛猶不悛。先主伐吳,飛當率兵萬人,自閬中會江州。臨發,其帳下將張達、范彊殺飛,持其首,順流而奔孫權。飛營都督表報先主,先主聞飛都督之有表也,曰:「噫!飛死矣。」追謚飛曰桓侯。長子苞,早夭。次子紹嗣,官至侍中尚書僕射。苞子遵為尚書,隨諸葛瞻於緜竹,與鄧艾戰,死。


\end{pinyinscope}