\article{姜維傳}

\begin{pinyinscope}
姜維字伯約,天水兾人也。少孤,與母居。好鄭氏學。

傅子曰:維為人好立功名,陰養死士,不脩布衣之業。仕郡上計掾,州辟為從事。以父昔為郡功曹,值羌、戎叛亂,身衞郡將,沒於戰場,賜維官中郎,參本郡軍事。建興六年,丞相諸葛亮軍向祁山,時天水太守適出案行,維及功曹梁緒、主簿尹賞、主記梁虔等從行。太守聞蜀軍垂至,而諸縣響應,疑維等皆有異心,於是夜亡保上邽。維等覺太守去,追遲,至城門,城門已閉,不納。維等相率還兾,兾亦不入維。維等乃俱詣諸葛亮。會馬謖敗於街亭,亮拔將西縣千餘家及維等還,故維遂與母相失。魏略曰:天水太守馬遵將維及諸官屬隨雍州刺史郭淮偶自西至洛門案行,會聞亮已到祁山,淮顧遵曰:「是欲不善!」遂驅東還上邽。遵念所治兾縣界在西偏,又恐吏民樂亂,遂亦隨淮去。時維謂遵曰:「明府當還兾。」遵謂維等曰:「卿諸人回,復信皆賊也。」各自行。維亦無如遵何,而家在兾,遂與郡吏上官子脩等還兾。兾中吏民見維等大喜,便推令見亮。二人不獲已,乃共詣亮。亮見,大恱。未及遣迎兾中人,會亮前鋒為張郃、費繇等所破,遂將維等却縮。維不得還,遂入蜀。諸軍攻兾,皆得維母妻子,亦以維本無去意,故不沒其家,但繫保官以延之。此語與本傳不同。亮辟維為倉曹掾,加奉義將軍,封當陽亭侯,時年二十七。亮與留府長史張裔、參軍蔣琬書曰:「姜伯約忠勤時事,思慮精密,考其所有,永南、季常諸人不如也。其人,涼州上士也。」又曰:「須先教中虎步兵五六千人。姜伯約甚敏於軍事,旣有膽義,深解兵意。此人心存漢室,而才兼於人,畢教軍事,當遣詣宮,覲見主上。」孫盛雜記曰:初,姜維詣亮,與母相失,復得母書,令求當歸。維曰:「良田百頃,不在一畒,但有遠志,不在當歸也。」後遷中監軍征西將軍。

十二年,亮卒,維還成都,為右監軍輔漢將軍,統諸軍,進封平襄侯。延熈元年,隨大將軍蔣琬住漢中。琬旣遷大司馬,以維為司馬,數率偏軍西入。六年,遷鎮西大將軍,領涼州刺史。十年,遷衞將軍,與大將軍費禕共錄尚書事。是歲,汶山平康夷反,維率衆討定之。又出隴西、南安、金城界,與魏大將軍郭淮、夏侯霸等戰於洮西。胡王治無戴等舉部落降,維將還安處之。十二年,假維節,復出西平,不克而還。維自以練西方風俗,兼負其才武,欲誘諸羌、胡以為羽翼,謂自隴以西可斷而有也。每欲興軍大舉,費禕常裁制不從,與其兵不過萬人。漢晉春秋曰:費禕謂維曰:「吾等不如丞相亦已遠矣;丞相猶不能定中夏,況吾等乎!且不如保國治民,敬守社稷,如其功業,以俟能者,無以為希兾徼倖而決成敗於一舉。若不如志,悔之無及。」

十六年春,禕卒。夏,維率將數萬人出石營,經董亭,圍南安,魏雍州刺史陳泰解圍至洛門,維糧盡退還。明年,加督中外軍事。復出隴西,守狄道長李簡舉城降。進圍襄武,與魏將徐質交鋒,斬首破敵,魏軍敗退。維乘勝多所降下,拔河間、狄道、臨洮三縣民還,後十八年,復與車騎將軍夏侯霸等俱出狄道,大破魏雍州刺史王經於洮西,經衆死者數萬人。經退保狄道城,維圍之。魏征西將軍陳泰進兵解圍,維却住鍾題。

十九年春,就遷維為大將軍。更整勒戎馬,與鎮西大將軍胡濟期會上邽,濟失誓不至,故維為魏大將鄧艾所破於段谷,星散流離,死者甚衆。衆庶由是怨讟,而隴已西亦騷動不寧,維謝過引負,求自貶削。為後將軍,行大將軍事。

二十年,魏征東大將軍諸葛誕反於淮南,分關中兵東下。維欲乘虛向秦川,復率數萬人出駱谷,徑至沈嶺。時長城積穀甚多而守兵乃少,聞維方到,衆皆惶懼。魏大將軍司馬望拒之,鄧艾亦自隴右,皆軍于長城。維前住芒水,皆倚山為營。望、艾傍渭堅圍,維數下挑戰,望、艾不應。景耀元年,維聞誕破敗,乃還成都。復拜大將軍。

初,先主留魏延鎮漢中,皆實兵諸圍以禦外敵,敵若來攻,使不得入。及興勢之役,王平捍拒曹爽,皆承此制。維建議,以為錯守諸圍,雖合周易「重門」之義,然適可禦敵,不獲大利。不若使聞敵至,諸圍皆歛兵聚穀,退就漢、樂二城,使敵不得入平,且重關鎮守以捍之。有事之日,令游軍並進以伺其虛。敵攻關不克,野無散穀,千里縣糧,自然疲乏。引退之日,然後諸城並出,與游軍并力搏之,此殄敵之術也。於是令督漢中胡濟却住漢壽,監軍王含守樂城,護軍蔣斌守漢城,又於西安、建威、武衞、石門、武城、建昌、臨遠皆立圍守。

五年,維率衆出漢、侯和,為鄧艾所破,還住沓中。維本羈旅託國,累年攻戰,功績不立,而宦臣黃皓等弄權於內,右大將軍閻宇與皓恊比,而皓陰欲廢維樹宇。維亦疑之。故自危懼,不復還成都。華陽國志曰;維惡黃皓恣擅,啟後主欲殺之。後主曰:「皓趨走小臣耳,往董允切齒,吾常恨之,君何足介意!」維見皓枝附葉連,懼於失言,遜辭而出。後主勑皓詣維陳謝。維說皓求沓中種麥,以避內逼爾。六年,維表後主:「聞鍾會治兵關中,欲規進取,宜並遣張翼、廖化督諸軍分護陽安關口、陰平橋頭以防未然。」皓徵信鬼巫,謂敵終不自致,啟後主寢其事,而羣臣不知。及鍾會將向駱谷,鄧艾將入沓中,然後乃遣右車騎廖化詣沓中為維援,左車騎張翼、輔國大將軍董厥等詣陽安關口以為諸圍外助。比至陰平,聞魏將諸葛緒向建威,故住待之。月餘,維為鄧艾所摧,還住陰平。鍾會攻圍漢、樂二城,遣別將進攻關口,蔣舒開城出降,傅僉格鬬而死。漢晉春秋曰:蔣舒將出降,乃詭謂傅僉曰:「今賊至不擊而閉城自守,非良圖也。」僉曰:「受命保城,惟全為功,今違命出戰,若喪師負國,死無益矣。」舒曰:「子以保城獲全為功,我以出戰克敵為功,請各行其志。」遂率衆出。僉謂其戰也,至陰平,以降胡烈。烈乘虛襲城,僉格鬬而死,魏人義之。蜀記曰:蔣舒為武興督,在事無稱。蜀命人代之,因留舒助漢中守。舒恨,故開城出降。會攻樂城,不能克,聞關口已下,長驅而前。翼、厥甫至漢壽,維、化亦舍陰平而退,適與翼、厥合,皆退保劒閣以拒會。會與維書曰:「公侯以文武之德,懷邁世之略,功濟巴、漢,聲暢華夏,遠近莫不歸名。每惟疇昔,甞同大化,吳札、鄭喬,能喻斯好。」維不荅書,列營守險。會不能克,糧運縣遠,將議還歸。

而鄧艾自陰平由景谷道傍入,遂破諸葛瞻於緜竹。後主請降於艾,艾前據成都。維等初聞瞻破,或聞後主欲固守成都,或聞欲南入建寧,於是引軍由廣漢、郪道以審虛實。尋被後主敕令,乃投戈放甲,詣會於涪軍前,將士咸怒,拔刀斫石。干寶晉紀云:會謂維曰;「來何遲也?」維正色流涕曰:「今日見此為速矣!」會甚奇之。

會厚待維等,皆權還其印號節蓋。會與維出則同轝,坐則同席,謂長史杜預曰:「以伯約比中土名士,公休、太初不能勝也。」世語曰:時蜀官屬皆天下英俊,無出維右。會旣構鄧艾,艾檻車徵,因將維等詣成都,自稱益州牧以叛。漢晉春秋曰:會陰懷異圖,維見而知其心,謂可構成擾亂以圖克復也,乃詭說會曰:「聞君自淮南已來,筭無遺策,晉道克昌,皆君之力。今復定蜀,威德振世,民高其功,主畏其謀,欲以此安歸乎!夫韓信不背漢於擾攘,以見疑於旣平,大夫種不從范蠡於五湖,卒伏劒而妄死,彼豈闇主愚臣哉?利害使之然也。今君大功旣立,大德已著,何不法陶朱公泛舟絕迹,全功保身,登峨嵋之嶺,而從赤松游乎?」會曰:「君言遠矣,我不能行,且為今之道,或未盡於此也。」維曰:「其佗則君智力之所能,無煩於老夫矣。」由是情好歡甚。華陽國志曰:維教會誅北來諸將,旣死,徐欲殺會,盡坑魏兵,還復蜀祚,密書與後主曰:「願陛下忍數日之辱,臣欲使社稷危而復安,日月幽而復明。」孫盛晉陽秋曰:盛以永和初從安西將軍平蜀,見諸故老,及姜維旣降之後密與劉禪表疏,說欲偽服事鍾會,因殺之以復蜀土,會事不捷,遂至泯滅,蜀人于今傷之。盛以為古人云,非所困而困焉名必辱,非所據而據焉身必危,旣辱且危,死其將至,其姜維之謂乎!鄧艾之入江由,士衆鮮少,維進不能奮節緜竹之下,退不能總帥五將,擁衞蜀主,思後圖之計,而乃反覆於逆順之間,希違情於難兾之會,以衰弱之國,而屢觀兵於三秦,已滅之邦,兾理外之奇舉,不亦闇哉!臣松之以為盛之譏維,又為不當。于時鍾會大衆旣造劒閣,維與諸將列營守險,會不得進,已議還計,全蜀之功,幾乎立矣。但鄧艾詭道傍入,出於其後,諸葛瞻旣敗,成都自潰。維若回軍救內,則會乘其背。當時之勢,焉得兩濟?而責維不能奮節緜竹,擁衞蜀主,非其理也。會欲盡坑魏將以舉大事,授維重兵,使為前驅。若令魏將皆死,兵事在維手,殺會復蜀,不為難矣。夫功成理外,然後為奇,不可以事有差牙,而抑謂不然。設使田單之計,邂逅不會,復可謂之愚闇哉!欲授維兵五萬人,使為前驅。魏將士憤發,殺會及維,維妻子皆伏誅。世語曰:維死時見剖,膽如斗大。

郤正著論論維曰:「姜伯約據上將之重,處羣臣之右,宅舍弊薄,資財無餘,側室無妾媵之褻,後庭無聲樂之娛,衣服取供,輿馬取備,飲食節制,不奢不約,官給費用,隨手消盡;察其所以然者,非以激貪厲濁,抑情自割也,直謂如是為足,不在多求。凡人之談,常譽成毀敗,扶高抑下,咸以姜維投厝無所,身死宗滅,以是貶削,不復料擿,異乎春秋襃貶之義矣。如姜維之樂學不倦,清素節約,自一時之儀表也。」孫盛曰:異哉郤氏之論也!夫士雖百行,操業萬殊,至於忠孝義節,百行之冠冕也。姜維策名魏室,而外奔蜀朝,違君徇利,不可謂忠;捐親苟免,不可謂孝;害加舊邦,不可謂義;敗不死難,不可謂節;且德政未敷而疲民以逞,居禦侮之任而致敵喪守,於夫智勇,莫可云也:凡斯六者,維無一焉。實有魏之逋臣,亡國之亂相,而云人之儀表,斯亦惑矣。縱維好書而微自藻潔,豈異夫盜者分財之義,而程、鄭降階之善也?臣松之以為郤正此論,取其可稱,不謂維始終行事皆可準則也。所云「一時儀表」,止在好學與儉素耳。本傳及魏略皆云維本無叛心,以急逼歸蜀。盛相譏貶,惟可責其背母。餘旣過苦,又非所以難郤正也。

維昔所俱至蜀,梁緒官至大鴻臚,尹賞執金吾,梁虔大長秋,皆先蜀亡歿。

評曰:蔣琬方整有威重,費禕寬濟而博愛,咸承諸葛之成規,因循而不革,是以邊境無虞,邦家和一,然猶未盡治小之宜,居靜之理也。臣松之以為蔣、費為相,克遵畫一,未甞徇功妄動,有所虧喪,外郤駱谷之師,內保寧緝之實,治小之宜,居靜之理,何以過於此哉!今譏其未盡而不著其事,故使覽者不知所謂也。姜維粗有文武,志立功名,而翫衆黷旅,明斷不周,終致隕斃。老子有云:「治大國者猶烹小鮮。」況於區區蕞爾,而可屢擾乎哉?干寶曰:姜維為蜀相,國亡主辱弗之死,而死於鍾會之亂,惜哉!非死之難,處死之難也。是以古之烈士,見危授命,投節如歸,非不愛死也,固知命之不長而懼不得其所也。


\end{pinyinscope}