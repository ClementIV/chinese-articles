\article{he-kui-chuan}

\begin{pinyinscope}
何夔字叔龍,陳郡陽夏人也。曾祖父熈,漢安帝時官至車騎將軍。

華嶠漢書曰:熈字孟孫,少有大志,不拘小節。身長八尺五寸,體貌魁梧,善為容儀。舉孝廉,為謁者,贊拜殿中,音動左右。和帝佳之,歷位司隷校尉、大司農。永初三年,南單于與烏丸俱反,以熈行車騎將軍征之,累有功。烏丸請降,單于復稱臣如舊。會熈暴疾卒。夔幼喪父,與母兄居,以孝友稱。長八尺三寸,容貌矜嚴。魏書曰:漢末閹宦用事,夔從父衡為尚書,有直言,由是在黨中,諸父兄皆禁錮。夔歎曰:「天地閉,賢人隱。」故不應宰司之命。避亂淮南。後袁術至壽春,辟之,夔不應,然遂為術所留。久之,術與橋蕤俱攻圍蘄陽,蘄陽為太祖固守。術以夔彼郡人,欲脅令說蘄陽。夔謂術謀臣李業曰:「昔栁下惠聞伐國之謀而有憂色,曰『吾聞伐國不問仁人,斯言何為至於我哉』!」遂遁匿灊山。術知夔終不為己用,乃止。術從兄山陽太守遺母,夔從姑也,是以雖恨夔而不加害。

建安二年,夔將還鄉里,度術必急追,乃間行得免,明年到本郡。頃之,太祖辟為司空掾屬。時有傳袁術軍亂者,太祖問夔曰;「君以為信不?」夔對曰:「天之所助者順,人之所助者信。術無信順之實,而望天人之助,此不可以得志於天下。夫失道之主,親戚叛之,而況於左右乎!以夔觀之,其亂必矣。」太祖曰;「為國失賢則亡。君不為術所用;亂,不亦宜乎!」太祖性嚴,掾屬公事,往往加杖;夔常畜毒藥,誓死無辱,是以終不見及。孫盛曰:夫君使臣以禮,臣事君以忠,是以上下休嘉,道光化洽。公府掾屬,古之造士也,必擢時儁,搜揚英逸,得其人則論道之任隆,非其才則覆餗之患至。苟有疵釁,刑黜可也。加其捶扑之罰,肅以小懲之戒,豈「導之以德,齊之以禮」之謂與!然士之出處,宜度德投趾;可不之節,必審於所蹈。故高尚之徒,抗心於青雲之表,豈王侯之所能臣,名器之所羈紲哉!自非此族,委身世塗,否泰榮辱,制之由時,故箕子安於孥戮,柳下夷於三黜,蕭何、周勃亦在縲紲,夫豈不辱,君命故也。夔知時制,而甘其寵,挾藥要君,以避微恥。詩云「唯此褊心」,何夔其有焉。放之,可也;宥之,非也。出為城父令。魏書曰:自劉備叛後,東南多變。太祖以陳羣為鄖令,夔為城父令,諸縣皆用名士以鎮撫之,其後吏民稍定。遷長廣太守。郡濵山海,黃巾未平,豪傑多背叛,袁譚就加以官位。長廣縣人管承,徒衆三千餘家,為寇害。議者欲舉兵攻之。夔曰:「承等非生而樂亂也,習於亂,不能自還,未被德教,故不知反善。今兵迫之急,彼恐夷滅,必并力戰。攻之旣未易拔,雖勝,必傷吏民,不如徐喻以恩德,使容自悔,可不煩兵而定。」乃遣郡丞黃珍往,為陳成敗,承等皆請服。夔遣吏成弘領校尉,長廣縣丞等郊迎奉牛酒,詣郡。牟平賊從錢,衆亦數千,夔率郡兵與張遼共討定之。東牟人王營,衆三千餘家,脅昌陽縣為亂。夔遣吏王欽等,授以計略,使離散之。旬月皆平定。

是時太祖始制新科下州郡,又收租稅緜絹。夔以郡初立,近以師旅之後,不可卒繩以法,乃上言曰:「自喪亂已來,民人失所,今雖小安,然服教日淺。所下新科,皆以明罰勑法,齊一大化也。所領六縣,疆域初定,加以饑饉,若一切齊以科禁,恐或有不從教者。有不從教者不得不誅,則非觀民設教隨時之意也。先王辨九服之賦以殊遠近,制三典之刑以平治亂,愚以為此郡宜依遠域新邦之典,其民間小事,使長吏臨時隨宜,上不背正法,下以順百姓之心。比及三年,民安其業,然後齊之以法,則無所不至矣。」太祖從其言。徵還,參丞相軍事。海賊郭祖寇暴樂安、濟南界,州郡苦之。太祖以夔前在長廣有威信,拜樂安太守。到官數月,諸城悉平。

入為丞相東曹掾。夔言於太祖曰:「自軍興以來,制度草創,用人未詳其本,是以各引其類,時忘道德。夔聞以賢制爵,則民慎德;以庸制祿,則民興功。以為自今所用,必先核之鄉閭,使長幼順敘,無相踰越。顯忠直之賞,明公實之報,則賢不肖之分,居然別矣。又可脩保舉故不以實之令,使有司別受其具。在朝之臣,時受教與曹並選者,各任其責。上以觀朝臣之節,下以塞爭競之源,以督羣下,以率萬民,如是則天下幸甚。」太祖稱善。魏國旣建,拜尚書僕射。魏書曰:時丁儀兄弟方進寵,儀與夔不合。尚書傅巽謂夔曰:「人不相好已甚,子友毛玠,玠等儀已害之矣。子宜少下之!」夔曰:「為不義適足害其身,焉能害人?且懷姧佞之心,立於明朝,其得久乎!」夔終不屈志,儀後果以凶偽敗。文帝為太子,以凉茂為太傅,夔為少傅;特命二傅與尚書東曹並選太子諸侯官屬。茂卒,以夔代茂。每月朔,太傅入見太子,太子正法服而禮焉;他日無會儀。夔遷太僕,太子欲與辭,宿戒供,夔無往意;乃與書請之,夔以國有常制,遂不往。其履正如此。然於節儉之世,最為豪汰。文帝踐阼,封成陽亭侯,邑三百戶。疾病,屢乞遜位。詔報曰:「蓋禮賢親舊,帝王之常務也。以親則君有輔弼之勳焉,以賢則君有醇固之茂焉。夫有陰德者必有陽報,今君疾雖未瘳,神明聽之矣。君其即安,以順朕意。」薨,謚曰靖侯。子曾嗣,咸熈中為司徒。干寶晉紀曰:曾字穎考。正元中為司隷校尉。時毌丘儉孫女適劉氏,以孕繫廷尉。女母荀,為武衞將軍荀顗所表活,旣免,辭詣廷尉,乞為官婢以贖女命。曾使主簿程咸為議,議曰:「大魏承秦、漢之弊,未及革制。所以追戮已出之女,誠欲殄醜類之族也。若已產育,則成他家之母。於防則不足懲姧亂之源,於情則傷孝子之思,男不御罪於他族,而女獨嬰戮於二門,非所以哀矜女弱,均法制之大分也。臣以為在室之女,可從父母之刑,旣醮之婦,使從夫家之戮。」朝廷從之,乃定律令。晉諸公讚曰:曾以高雅稱,加性純孝,位至太宰,封朗陵縣公。年八十餘薨,謚曰元公。子邵嗣。邵字敬祖,才識深博,有經國體儀。位亦至太宰,謚康公。子蕤嗣。邵庶兄遵,字思祖,有幹能。少經清職,終於太僕。遵子綏,字伯蔚,亦以幹事稱。永嘉中為尚書,為司馬越所殺。傅子稱曾及荀顗曰:「以文王之道事其親者,其潁昌何侯乎!其荀侯乎!古稱曾、閔,今曰荀、何。內盡其心以事其親,外崇禮讓以接天下。孝子,百世之宗;仁人,天下之令也。有能行仁孝之道者,君子之儀表矣。」


\end{pinyinscope}