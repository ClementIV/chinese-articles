\article{黃忠傳}

\begin{pinyinscope}
黃忠字漢升,南陽人也。荊州牧劉表以為中郎將,與表從子磐共守長沙攸縣。及曹公克荊州,假行裨將軍,仍就故任,統屬長沙太守韓玄。先主南定諸郡,忠遂委質,隨從入蜀。自葭萌受任,還攻劉璋,忠常先登陷陣,勇毅冠三軍。益州旣定,拜為討虜將軍。建安二十四年,於漢中定軍山擊夏侯淵。淵衆甚精,忠推鋒必進,勸率士卒,金鼓振天,歡聲動谷,一戰斬淵,淵軍大敗。遷征西將軍。是歲,先主為漢中王,欲用忠為後將軍,諸葛亮說先主曰:「忠之名望,素非關、馬之倫也。而今便令同列。馬、張在近,親見其功,尚可喻指;關遙聞之,恐必不恱,得無不可乎!」先主曰:「吾自當解之。」遂與羽等齊位,賜爵關內侯。明年卒,追謚剛侯。子叙,早沒,無後。


\end{pinyinscope}