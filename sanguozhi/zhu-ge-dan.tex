\article{諸葛誕}

\begin{pinyinscope}
諸葛誕字公休,琅邪陽都人,諸葛豐後也。初以尚書郎為滎陽令,

魏氏春秋曰:誕為郎,與僕射杜畿試船陶河,遭風覆沒,誕亦俱溺。虎賁浮河救誕,誕曰:「先救杜侯。」誕飄于岸,絕而後蘇。入為吏部郎。人有所屬託,輙顯其言而承用之,後有當否,則公議其得失以為襃貶,自是群僚莫不慎其所舉。累遷御史中丞尚書,與夏侯玄、鄧颺等相善,收名朝廷,京都翕然。言事者以誕、颺等脩浮華,合虛譽,漸不可長。明帝惡之,免誕官。世語曰:是時,當世俊士散騎常侍夏侯玄、尚書諸葛誕、鄧颺之徒,共相題表,以玄、疇四人為四聡,誕、備八人為八達,中書監劉放子熈、孫資子密、吏部尚書衞臻子烈三人,咸不及比,以父居勢位,容之為三豫,凡十五人。帝以構長浮華,皆免官廢錮。會帝崩,正始初,玄等並在職。復以誕為御史中丞尚書,出為揚州刺史,加昭武將軍。

王淩之陰謀也,太傅司馬宣王潛軍東伐,以誕為鎮東將軍、假節都督揚州諸軍事,封山陽亭侯。諸葛恪興東關,遣誕督諸軍討之,與戰,不利。還,徙為鎮南將軍。

後毌丘儉、文欽反,遣使詣誕,招呼豫州士民。誕斬其使,露布天下,令知儉、欽凶逆。大將軍司馬景王東征,使誕督豫州諸軍,渡安風津向壽春。儉、欽之破也,誕先至壽春。壽春中十餘萬口,聞儉、欽敗,恐誅,悉破城門出,流迸山澤,或散走入吴。以誕乆在淮南,乃復以為鎮東大將軍、儀同三司、都督揚州。吴大將孫峻、呂據、留贊等聞淮南亂,會文欽往,乃帥衆將欽徑至壽春;時誕諸軍已至,城不可攻,乃走。誕遣將軍蔣班追擊之,斬贊,傳首,收其印節。進封高平侯,邑三千五百戶,轉為征東大將軍。

誕旣與玄、颺等至親,又王淩、毌丘儉累見夷滅,懼不自安,傾帑藏振施以結衆心,厚養親附及揚州輕俠者數千人為死士。魏書曰:誕賞賜過度。有犯死罪者,虧制以活之。甘露元年冬,吴賊欲向徐堨,計誕所督兵馬足以待之,而復請十萬衆守壽春,又求臨淮築城以備寇,內欲保有淮南。朝廷微知誕有自疑心,以誕舊臣,欲入度之。二年五月,徵為司空。誕被詔書,愈恐,遂反。召會諸將,自出攻揚州刺史樂綝,殺之。世語曰:司馬文王旣秉朝政,長史賈充以為宜遣參佐慰勞四征,於是遣充至壽春。充還啟文王:「誕再在揚州,有威名,民望所歸。今徵,必不來,禍小事淺;不徵,事遲禍大。」乃以為司空。書至,誕曰:「我作公當在王文舒後,今便為司空!不遣使者,健步齎書,使以兵付樂綝,此必綝所為。」乃將左右數百人至揚州,揚州人欲閉門,誕叱曰:「卿非我故吏邪!」徑入,綝逃上樓,就斬之。魏末傳曰:賈充與誕相見,談說時事,因謂誕曰:「洛中諸賢,皆願禪代,君所知也。君以為云何?」誕厲色曰:「卿非賈豫州子?世受魏恩,如何負國,欲以魏室輸人乎?非吾所忍聞。若洛中有難,吾當死之。」充默然。誕旣被徵,請諸牙門置酒飲宴,呼牙門從兵,皆賜酒令醉,謂衆人曰:「前作千人鎧仗始成,欲以擊賊,今當還洛,不復得用,欲蹔出,將見人游戲,須臾還耳;諸君且止。」乃嚴鼓將士七百人出。樂綝聞之,閉州門。誕歷南門宣言曰:「當還洛邑,暫出游戲,揚州何為閉門見備?」前至東門,東門復閉,乃使兵緣城攻門,州人悉走,因風放火,焚其府庫,遂殺綝。誕表曰:「臣受國重任,統兵在東。揚州刺史樂綝專詐,說臣與吳交通,又言被詔當代臣位,無狀日乆。臣奉國命,以死自立,終無異端。忿綝不忠,輙將步騎七百人,以今月六日討綝,即日斬首,函頭驛馬傳送。若聖朝明臣,臣即魏臣;不明臣,臣即吳臣。不勝發憤有日,謹拜表陳愚,悲感泣血,哽咽斷絕,不知所如,乞朝廷察臣至誠。」臣松之以為魏末傳所言,率皆鄙陋。疑誕表言曲,不至於此也。斂淮南及淮北郡縣屯田口十餘萬官兵,揚州新附勝兵者四五萬人,聚穀足一年食,閉城自守。遣長史吴綱將小子靚至吴請救。世語曰:黃初末,吳人發長沙王吳芮冢,以其塼於臨湘為孫堅立廟。芮容貌如生,衣服不朽。後豫發者見吳綱曰:「君何類長沙王吳芮,但微短耳。」綱瞿然曰;「是先祖也,君何由見之?」見者言所由,綱曰:「更葬否?」荅曰:「即更葬矣。」自芮之卒年至冢發,四百餘年,綱,芮之十六世孫矣。吴人大喜,遣將全懌、全端、唐咨、王祚等,率三萬衆,密與文欽俱來應誕。以誕為左都護、假節、大司徒、驃騎將軍、青州牧、壽春侯。是時鎮南將軍王基始至,督諸軍圍壽春,未合。咨、欽等從城東北,因山乘險,得將其衆突入城。

六月,車駕東征,至項。大將軍司馬文王督中外諸軍二十六萬衆,臨淮討之。大將軍屯丘頭。使基及安東將軍陳騫等四靣合圍,表裏再重,壍壘甚峻。又使監軍石苞、兖州刺史州泰等,簡銳卒為游軍,備外寇。欽等數出犯圍,逆擊走之。吴將朱異再以大衆來迎誕等,渡黎漿水,泰等逆與戰,每摧其鋒。孫綝以異戰不進,怒而殺之。城中食轉少,外救不至,衆無所恃。將軍蔣班、焦彝,皆誕爪牙計事者也,棄誕,踰城自歸大將軍。漢晉春秋曰:蔣班、焦彝言於諸葛誕曰:「朱異等以大衆來而不能進,孫綝殺異而歸江東,外以發兵為名,而內實坐須成敗,其歸可見矣。今宜及衆心尚固,士卒思用,并力決死,攻其一靣,雖不能盡克,猶可有全者。」文欽曰:「江東乘戰勝之威乆矣,未有難北方者也。況公今舉十餘萬之衆內附,而欽與全端等皆同居死地,父兄子弟盡在江表,就孫綝不欲,主上及其親戚豈肯聽乎?且中國無歲無事,軍民並疾,今守我一年,勢力已困,異圖生心,變故將起,以往準今,可計日而望也。」班、彝固勸之,欽怒,而誕欲殺班。二人懼,且知誕之必敗也,十一月,乃相攜而降。大將軍乃使反間,以奇變說全懌等,懌等率衆數千人開門來出。城中震懼,不知所為。

三年正月,誕、欽、咨等大為攻具,晝夜五六日攻南圍,欲決圍而出。漢晉春秋曰:文欽曰:「蔣班、焦彝謂我不能出而走,全端、全懌又率衆逆降,此敵無備之時也,可以戰矣。」誕及唐咨等皆以為然,遂共悉衆出攻。圍上諸軍,臨高以發石車火箭逆燒破其攻具,弩矢及石雨下,死傷者蔽地,血流盈壍。復還入城,城內食轉竭,降出者數萬口。欽欲盡出北方人,省食,與吴人堅守,誕不聽,由是爭恨。欽素與誕有隙,徒以計合,事急愈相疑。欽見誕計事,誕遂殺欽。欽子鴦及虎將兵在小城中,聞欽死,勒兵馳赴之,衆不為用。鴦、虎單走,踰城出,自歸大將軍。軍吏請誅之,大將軍令曰:「欽之罪不容誅,其子固應當戮,然鴦、虎以窮歸命,且城未拔,殺之是堅其心也。」乃赦鴦、虎,使將兵數百騎馳巡城,呼語城內云:「文欽之子猶不見殺,其餘何懼?」表鴦、虎為將軍,各賜爵關內侯。城內喜且擾,又日飢困,誕、咨等智力窮。大將軍乃自臨圍,四靣進兵,同時鼓譟登城,城內無敢動者。誕窘急,單乘馬,將其麾下突小城門出。大將軍司馬胡奮部兵逆擊,斬誕,傳首,夷三族。誕麾下數百人,坐不降見斬,皆曰:「為諸葛公死,不恨。」其得人心如此。干寶晉紀曰:數百人拱手為列,每斬一人,輙降之,竟不變,至盡,時人比之田橫。吳將于詮曰:「大丈夫受命其主,以兵救人,旣不能克,又束手於敵,吾弗取也。」乃免冑冒陣而死。唐咨、王祚及諸裨將皆靣縛降,吴兵萬衆,器仗軍實山積。

初圍壽春,議者多欲急攻之,大將軍以為:「城固而衆多,攻之必力屈,若有外寇,表裏受敵,此危道也。今三叛相聚於孤城之中,天其或者將使同就戮,吾當以全策縻之,可坐而制也。」誕以二年五月反,三年二月破滅。六軍按甲,深溝高壘,而誕自困,竟不煩攻而克。干寶晉紀曰:初,壽春每歲雨潦,淮水溢,常淹城邑。故文王之築圍也,誕笑之曰:「是固不攻而自敗也。」及大軍之攻,亢旱踰年。城旣陷,是日大雨,圍壘皆毀。誕子靚,字仲思,吳平還晉。靚子恢,字道明,位至尚書令,追贈左光祿大夫開府。及破壽春,議者又以為淮南仍為叛逆,吴兵室家在江南,不可縱,宜悉坑之。大將軍以為古之用兵,全國為上,戮其元惡而已。吴兵就得亡還,適可以示中國之弘耳。一無所殺,分布三河近郡以安處之。

唐咨本利城人。黃初中,利城郡反,殺太守徐箕,推咨為主。文帝遣諸軍討破之,咨走入海,遂亡至吴,官至左將軍,封侯、持節。誕、欽屠戮,咨亦生禽,三叛皆獲,天下快焉。傅子曰:宋建椎牛禱賽,終自焚滅。文欽日祠祭事天,斬於人手。諸葛誕夫婦聚會神巫,淫祀求福,伏尸淮南,舉族誅夷。此天下所共見,足為明鑒也。拜咨安遠將軍,其餘裨將咸假號位,吴衆恱服。江東感之,皆不誅其家。其淮南將吏士民諸為誕所脅略者,惟誅其首逆,餘皆赦之。聽鴦、虎收斂欽喪,給其車牛,致葬舊墓。習鑿齒曰:自是天下畏威懷德矣。君子謂司馬大將軍於是役也,可謂能以德攻矣。夫建業者異矣,各有所尚,而不能兼并也。故窮武之雄斃於不仁,存義之國喪於懦退,今一征而禽三叛,大虜吳衆,席卷淮浦,俘馘十萬,可謂壯矣。而未及安坐,喪王基之功,種惠吳人,結異類之情,寵鴦葬欽,忘疇昔之隙,不咎誕衆,使揚士懷愧,功高而人樂其成,業廣而敵懷其德,武昭旣敷,文筭又洽,推此道也,天下其孰能當之哉?喪王基,語在基傳。鴦一名俶。晉諸公贊曰,俶後為將軍,破涼州虜,名聞天下。太康中為東夷校尉、假節。當之職,入辭武帝,帝見而惡之,託以他事免俶官。東安公繇,諸葛誕外孫,欲殺俶,因誅楊駿,誣俶謀逆,遂夷三族。


\end{pinyinscope}