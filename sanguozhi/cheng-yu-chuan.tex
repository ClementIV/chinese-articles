\article{cheng-yu-chuan}

\begin{pinyinscope}
程昱字仲德,東郡東阿人也。長八尺三寸,美鬚髯。黃巾起,縣丞王度反應之,燒倉庫。縣令踰城走,吏民負老幼東奔渠丘山。昱使人偵視度,度等得空城不能守,出城西五六里止屯。昱謂縣中大姓薛房等曰:「今度等得城郭不能居,其勢可知。此不過欲虜掠財物,非有堅甲利兵攻守之志也。今何不相率還城而守之?且城高厚,多穀米,今若還求令,共堅守,度必不能乆,攻可破也。」房等以為然。吏民不肯從,曰:「賊在西,但有東耳。」昱謂房等:「愚民不可計事。」乃密遣數騎舉幡於東山上,令房等望見,大呼言「賊已至」,便下山趣城,吏民奔走隨之,求得縣令,遂共城守。度等來攻城,不能下,欲去。昱率吏民開城門急擊之,度等破走。東阿由此得全。

初平中,兖州刺史劉岱辟昱,昱不應。是時岱與袁紹、公孫瓚和親,紹令妻子居岱所,瓚亦遣從事范方將騎助岱。後紹與瓚有隙。瓚擊破紹軍,乃遣使語岱,令遣紹妻子,使與紹絕。別勑范方:「若岱不遣紹家,將騎還。吾定紹,將加兵於岱。」岱議連日不決,別駕王彧白岱:「程昱有謀,能斷大事。」岱乃見昱,問計,昱曰:「若棄紹近援而求瓚遠助,此假人於越以救溺子之說也。夫公孫瓚,非袁紹之敵也。今雖壞紹軍,然終為紹所禽。夫趣一朝之權而不慮遠計,將軍終敗。」岱從之。范方將其騎歸,未至,瓚大為紹所破。岱表昱為騎都尉,昱辭以疾。

劉岱為黃巾所殺。太祖臨兖州,辟昱。昱將行,其鄉人謂曰:「何前後之相背也!」昱笑而不應。太祖與語,說之,以昱守壽張令。太祖征徐州,使昱與荀彧留守鄄城。張邈等叛迎呂布,郡縣響應,唯鄄城、范、東阿不動。布軍降者,言陳宮欲自將兵取東阿,又使汎嶷取范,吏民皆恐。彧謂昱曰:「今兖州反,唯有此三城。宮等以重兵臨之,非有以深結其心,三城必動。君,民之望也,歸而說之,殆可!」昱乃歸,過范,說其令靳允曰:「聞呂布執君母弟妻子,孝子誠不可為心!今天下大亂,英雄並起,必有命世,能息天下之亂者,此智者所詳擇也。得主者昌,失主者亡。陳宮叛迎呂布而百城皆應,似能有為,然以君觀之,布何如人哉!夫布,麤中少親,剛而無禮,匹夫之雄耳。宮等以勢假合,不能相君也。兵雖衆,終必無成。曹使君智略不世出,殆天所授!君必固范,我守東阿,則田單之功可立也。孰與違忠從惡而母子俱亡乎?唯君詳慮之!」允流涕曰:「不敢有貳心。」時汎嶷已在縣,允乃見嶷,伏兵刺殺之,歸勒兵守。

徐衆評曰:允於曹公,未成君臣。母,至親也,於義應去。昔王陵母為項羽所拘,母以高祖必得天下,因自殺以固陵志。明心無所係,然後可得成事人盡死之節。衞公子開方仕齊,積年不歸,管仲以為不懷其親,安能愛君,不可以為相。是以求忠臣必於孝子之門,允宜先救至親。徐庶母為曹公所得,劉備乃遣庶歸,欲為天下者恕人子之情也。曹公亦宜遣允。昱又遣別騎絕倉亭津,陳宮至,不得渡。昱至東阿,東阿令棗祗已率厲吏民,拒城堅守。又兖州從事薛悌與昱恊謀,卒完三城,以待太祖。太祖還,執昱手曰:「微子之力,吾無所歸矣。」乃表昱為東平相,屯范。魏書曰:昱少時常夢上泰山,兩手捧日。昱私異之,以語荀彧。及兖州反,賴昱得完三城。於是彧以昱夢白太祖。太祖曰:「卿當終為吾腹心。」昱本名立,太祖乃加其上「日」,更名昱也。

太祖與呂布戰於濮陽,數不利。蝗蟲起,乃各引去。於是袁紹使人說太祖連和,欲使太祖遷家居鄴。太祖新失兖州,軍食盡,將許之。時昱使適還,引見,因言曰:「竊聞將軍欲遣家,與袁紹連和,誠有之乎?」太祖曰:「然。」昱曰:「意者將軍殆臨事而懼,不然何慮之不深也!夫袁紹據燕、趙之地,有并天下之心,而智不能濟也。將軍自度能為之下乎?將軍以龍虎之威,可為韓、彭之事邪?今兖州雖殘,尚有三城。能戰之士,不下萬人。以將軍之神武,與文若、昱等,收而用之,霸王之業可成也。願將軍更慮之!」太祖乃止。魏略載昱說太祖曰:「昔田橫,齊之世族,兄弟三人更王,據千里之齊,擁百萬之衆,與諸侯並南面稱孤。旣而高祖得天下,而橫顧為降虜。當此之時,橫豈可為心哉!」太祖曰:「然。此誠丈夫之至辱也。」昱曰:「昱愚,不識大旨,以為將軍之志,不如田橫。田橫,齊一壯士耳,猶羞為高祖臣。今聞將軍欲遣家往鄴,將北面而事袁紹。夫以將軍之聦明神武,而反不羞為袁紹之下,竊為將軍恥之!」其後語與本傳略同。

天子都許,以昱為尚書。兖州尚未安集,復以昱為東中郎將,領濟陰太守,都督兖州事。劉備失徐州,來歸太祖。昱說太祖殺備,太祖不聽。語在武紀。後又遣備至徐州要擊袁術,昱與郭嘉說太祖曰:「公前日不圖備,昱等誠不及也。今借之以兵,必有異心。」太祖悔,追之不及。會術病死,備至徐州,遂殺車冑,舉兵背太祖。頃之,昱遷振威將軍。袁紹在黎陽,將南渡。時昱有七百兵守鄄城,太祖聞之,使人告昱,欲益二千兵。昱不肯,曰:「袁紹擁十萬衆,自以所向無前。今見昱兵少,必輕易不來攻。若益昱兵,過則不可不攻,攻之必克,徒兩損其勢。願公無疑!」太祖從之。紹聞昱兵少,果不往。太祖謂賈詡曰:「程昱之膽,過於賁、育。」昱收山澤亡命,得精兵數千人,乃引軍與太祖會黎陽,討袁譚、袁尚。譚、尚破走,拜昱奮武將軍,封安國亭侯。太祖征荊州,劉備奔吳。論者以為孫權必殺備,昱料之曰:「孫權新在位,未為海內所憚。曹公無敵於天下,初舉荊州,威震江表,權雖有謀,不能獨當也。劉備有英名,關羽、張飛皆萬人敵也,權必資之以禦我。難解勢分,備資以成,又不可得而殺也。」權果多與備兵,以禦太祖。是後中夏漸平,太祖拊昱背曰:「兖州之敗,不用君言,吾何以至此!」宗人奉牛酒大會,昱曰:「知足不辱,吾可以退矣。」乃自表歸兵,闔門不出。魏書曰:太祖征馬超,文帝留守,使昱參軍事。田銀、蘇伯等反河間,遣將軍賈信討之。賊有千餘人請降,議者皆以為宜如舊法,昱曰:「誅降者,謂在擾攘之時,天下雲起,故圍而後降者不赦,以示威天下,開其利路,使不至於圍也。今天下略定,且在邦域之中,此必降之賊,殺之無所威懼,非前日誅降之意。臣以為不可誅也;縱誅之,宜先啟聞。」衆議者曰:「軍事有專,無請。」昱不荅。文帝起入,特引見昱曰:「君有所不盡邪?」昱曰:「凡專命者,謂有臨時之急,呼吸之間者耳。今此賊制在賈信之手,無朝夕之變。故老臣不願將軍行之也。」文帝曰:「君慮之善。」即白太祖,太祖果不誅。太祖還,聞之甚說,謂昱曰:「君非徒明於軍計,又善處人父子之間。」

昱性剛戾,與人多迕。人有告昱謀反,太祖賜待益厚。魏國旣建,為衞尉,與中尉邢貞爭威儀,免。文帝踐阼,復為衞尉,進封安鄉侯,增邑三百戶,并前八百戶。分封少子延及孫曉列侯。方欲以為公,會薨,帝為流涕,追贈車騎將軍,謚曰肅侯。魏書曰:昱時年八十。世語曰:初,太祖乏食,昱略其本縣,供三日糧,頗雜以人脯,由是失朝望,故位不至公。子武嗣。武薨,子克嗣。克薨,子良嗣。

曉,嘉平中為黃門侍郎。世語曰:曉字季明,有通識。時校事放橫,曉上疏曰:「周禮云:『設官分職,以為民極。』春秋傳曰:『天有十日,人有十等。』愚不得臨賢,賤不得臨貴。於是並建聖哲,樹之風聲。明試以功,九載考績。各脩厥業,思不出位。故欒書欲拯晉侯,其子不聽;死人橫於街路,邴吉不問。上不責非職之功,下不務分外之賞,吏無兼統之勢,民無二事之役,斯誠為國要道,治亂所由也。遠覽典志,近觀秦漢,雖官名改易,職司不同,至於崇上抑下,顯分明例,其致一也。初無校事之官干與庶政者也。昔武皇帝大業草創,衆官未備,而軍旅勤苦,民心不安,乃有小罪,不可不察,故置校事,取其一切耳,然檢御有方,不至縱恣也。此霸世之權宜,非帝王之正典。其後漸蒙見任,復為疾病,轉相因仍,莫正其本。遂令上察宮廟,下攝衆司,官無局業,職無分限,隨意任情,唯心所適。法造於筆端,不依科詔;獄成於門下,不顧覆訊。其選官屬,以謹慎為粗疏,以𧩪詷為賢能。其治事,以刻暴為公嚴,以循理為怯弱。外則託天威以為聲勢,內則聚羣姦以為腹心。大臣恥與分勢,含忍而不言,小人畏其鋒芒,鬱結而無告。至使尹摸公於目下肆其姧慝;罪惡之著,行路皆知,纖惡之過,積年不聞。旣非周禮設官之意,又非春秋十等之義也。今外有公卿將校總統諸署,內有侍中尚書綜理萬機,司隷校尉督察京輦,御史中丞董攝宮殿,皆高選賢才以充其職,申明科詔以督其違。若此諸賢猶不足任,校事小吏,益不可信。若此諸賢各思盡忠,校事區區,亦復無益。若更高選國士以為校事,則是中丞司隷重增一官耳。若如舊選,尹摸之姧今復發矣。進退推筭,無所用之。昔桑弘羊為漢求利,卜式以為獨烹弘羊,天乃可雨。若使政治得失必感天地,臣恐水旱之災,未必非校事之由也。曹恭公遠君子,近小人,國風託以為刺。衞獻公舍大臣,與小臣謀,定姜謂之有罪。縱令校事有益於國,以禮義言之,尚傷大臣之心,況姧回暴露,而復不罷,是衮闕不補,迷而不返也。」於是遂罷校事官。曉遷汝南太守,年四十餘薨。曉別傳曰:曉大著文章多亡失,今之存者不能十分之一。


\end{pinyinscope}