\article{樓玄傳}

\begin{pinyinscope}
樓玄字承先,沛郡蘄人也。孫休時為監農御史。孫皓即位,與王蕃、郭逴、萬彧俱為散騎中常侍,出為會稽太守,入為大司農。舊禁中主者自用親近人作之,彧陳親密近識,宜用好人,皓因勑有司,求忠清之士,以應其選,遂用玄為宮下鎮禁中侯,主殿中事。玄從九卿持刀侍衞,正身率衆,奉法而行,應對切直,數迕皓意,漸見責怒。後人誣白玄與賀邵相逢,駐共耳語大笑,謗訕政事,遂被詔詰責,送付廣州。

東觀令華覈上疏曰:「臣竊以治國之體,其猶治家。主田野者,皆宜良信。又宜得一人緫其條目,為作維綱,衆事乃理。論語曰:『無為而治者其舜也與!恭己正南面而已。』言所任得其人,故優遊而自逸也。今海內未定,天下多事,事無大小,皆當關聞,動經御坐,勞損聖慮。陛下旣垂意愽古,綜極藝文,加勤心好道,隨節致氣,宜得閑靜以展神思,呼翕清淳,與天同極。臣夙夜思惟,諸吏之中,任幹之事,足委杖者,無勝於樓玄。玄清忠奉公,冠冕當世,衆服其操,無與爭先。夫清者則心平而意直,忠者惟正道而履之,如玄之性,終始可保,乞陛下赦玄前愆,使得自新,擢之宰司,責其後效,使為官擇人,隨才授任,則舜之恭己,近亦可得。」皓疾玄名聲,復徙玄及子據,付交阯將張弈,使以戰自效,陰別勑弈令殺之。據到交趾,病死。玄一身隨弈討賊,持刀步涉,見弈輒拜,弈未忍殺。會弈暴卒,玄殯斂弈,於器中見勑書,還便自殺。

江表傳曰:皓遣將張弈追賜玄鴆,弈以玄賢者,不忍即宣詔致藥,玄陰知之,謂弈曰:「當早告玄,玄何惜邪?」即服藥死。臣松之以玄之清高,必不以安危易操,無緣驟拜張弈,以虧其節。且禍機旣發,豈百拜所免?江表傳所言,於理為長。


\end{pinyinscope}