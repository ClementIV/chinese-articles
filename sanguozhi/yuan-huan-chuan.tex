\article{yuan-huan-chuan}

\begin{pinyinscope}
袁渙字曜卿,陳郡扶樂人也。父滂,為漢司徒。

袁宏漢紀曰:滂字公熈,純素寡欲,終不言人之短。當權寵之盛,或以同異致禍,滂獨中立於朝,故愛憎不及焉。當時諸公子多越法度,而渙清靜,舉動必以禮。郡命為功曹,郡中姧吏皆自引去。後辟公府,舉高第,遷侍御史。除譙令,不就。劉備之為豫州,舉渙茂才。後避地江、淮間,為袁術所命。術每有所咨訪,渙常正議,術不能抗,然敬之不敢不禮也。頃之,呂布擊術於阜陵,渙往從之,遂復為布所拘留。布初與劉備和親,後離隙。布欲使渙作書詈辱備,渙不可,再三彊之,不許。布大怒,以兵脅渙曰:「為之則生,不為則死。」渙顏色不變,笑而應之曰:「渙聞唯德可以辱人,不聞以罵。使彼固君子邪,且不恥將軍之言,彼誠小人邪,將復將軍之意,則辱在此不在於彼。且渙他日之事劉將軍,猶今日之事將軍也,如一旦去此,復罵將軍,可乎?」布慙而止。

布誅,乃得歸太祖。袁氏世紀曰:布之破也,陳羣父子時亦在布之軍,見太祖皆拜。渙獨高揖不為禮,太祖甚嚴憚之。時太祖又給衆官車各數乘,使取布軍中物,唯其所欲。衆人皆重載,唯渙取書數百卷,資糧而已,衆人聞之,大慙。渙謂所親曰:「脫我以行陳,令軍發足以為行糧而已,不以此為我有。由是厲名也,大悔恨之。」太祖益以此重焉。渙言曰:「夫兵者,凶器也,不得已而用之。鼓之以道德,征之以仁義,兼撫其民而除其害。夫然,故可與之死而可與之生。自大亂以來十數年矣,民之欲安,甚於倒縣,然而暴亂未息者,何也?意者政失其道歟!渙聞明君善於救世,故世亂則齊之以義,時偽則鎮之以樸;世異事變,治國不同,不可不察也。夫制度損益,此古今之不必同者也。若夫兼愛天下而反之於正,雖以武平亂而濟之以德,誠百王不易之道也。公明哲超世,古之所以得其民者,公旣勤之矣,今之所以失其民者,公旣戒之矣,海內賴公,得免於危亡之禍,然而民未知義,其唯公所以訓之,則天下幸甚!」太祖深納焉。拜為沛南部都尉。

是時新募民開屯田,民不樂,多逃亡。渙白太祖曰:「夫民安土重遷,不可卒變,易以順行,難以逆動,宜順其意,樂之者乃取,不欲者勿彊。」太祖從之,百姓大恱。遷為梁相。渙每勑諸縣:「務存鰥寡高年,表異孝子貞婦。常談曰『世治則禮詳,世亂則禮簡』,全在斟酌之間耳。方今雖擾攘,難以禮化,然在吾所以為之。」為政崇教訓,恕思而後行,外溫柔而內能斷。魏書曰:穀熟長呂岐善朱淵、爰津,遣使行學還,召用之,與相見,出署淵師友祭酒,津決疑祭酒。淵等因各歸家,不受署。岐大怒,將吏民收淵等,皆杖殺之,議者多非焉。渙教勿劾,主簿孫徽等以為「淵等罪不足死,長吏無專殺之義,孔子稱『唯器與名,不可以假人』。謂之師友而加大戮,刑名相伐,不可以訓。」渙教曰:「主簿以不請為罪,此則然矣。謂淵等罪不足死,則非也。夫師友之名,古今有之。然有君之師友,有士大夫之師友。夫君置師友之官者,所以敬其臣也;有罪加於刑焉,國之法也。今不論其罪而謂之戮師友,斯失之矣。主簿取弟子戮師之名,而加君誅臣之實,非其類也。夫聖哲之治,觀時而動,故不必循常,將有權也。間者世亂,民陵其上,雖務尊君卑臣,猶或未也,而反長世之過,不亦謬乎!」遂不劾。以病去官,百姓思之。後徵為諫議大夫、丞相軍祭酒。前後得賜甚多,皆散盡之,家無所儲,終不問產業,乏則取之於人,不為皦察之行,然時人服其清。

魏國初建,為郎中令,行御史大夫事。渙言於太祖曰:「今天下大難已除,文武並用,長乆之道也。以為可大收篇籍,明先聖之教,以易民視聽,使海內斐然向風,則遠人不服可以文德來之。」太祖善其言。時有傳劉備死者,羣臣皆賀;渙以甞為備舉吏,獨不賀。居官數年卒,太祖為之流涕,賜穀二千斛,一教「以太倉穀千斛賜郎中令之家」,一教「以垣下穀千斛與曜卿家」,外不解其意。教曰:「以太倉穀者,官法也;以垣下穀者,親舊也。」又帝聞渙昔拒呂布之事,問渙從弟敏:「渙勇怯何如?」敏對曰:「渙貌似和柔,然其臨大節,處危難,雖賁育不過也。」渙子侃,亦清粹閑素,有父風,歷位郡守尚書。袁氏世紀曰:渙有四子,侃、㝢、奧、準。侃字公然,論議清當,柔而不犯,善與人交。在廢興之間,人之所趣務者,常謙退不為也。時人以是稱之。歷位黃門選部郎,號為清平。稍遷至尚書,早卒。㝢字宣厚,精辯有機理,好道家之言,少被病,未官而卒,奧字公榮,行足以厲俗,言約而理當,終於光祿勳。準字孝尼,忠信公正,不恥下問,唯恐人之不勝己。以世事多險,故常治退而不敢求進。著書十餘萬言,論治世之務,為易、周官、詩傳,及論五經滯義,聖人之微言,以傳於世。此準之自序也。荀綽九州記稱準有儁才,泰始中為給事中。袁氏子孫世有名位,貴達至今。

初,渙從弟霸,公恪有功幹,魏初為大司農,及同郡何夔並知名於時。而霸子亮,夔子曾,與侃復齊聲友善。亮貞固有學行,疾何晏、鄧颺等,著論以譏切之,位至河南尹、尚書。晉諸公贊曰:亮子粲,字儀祖,文學博識,累為儒官,至尚書。霸弟徽,以儒素稱。遭天下亂,避難交州。司徒辟,不至。袁宏漢紀曰:初,天下將亂,渙慨然歎曰:「漢室陵遲,亂無日矣。苟天下擾攘,逃將安之?若天未喪道,民以義存,唯彊而有禮,可以庇身乎!」徽曰:「古人有言:『知機其神乎』!見機而作,君子所以元吉也。天理盛衰,漢其亡矣!夫有大功必有大事,此又君子之所深識,退藏於密者也。且兵革旣興,外患必衆,徽將遠迹山海,以求免身。」及亂作,各行其志。徽弟敏,有武藝而好水功,官至河隄謁者。


\end{pinyinscope}