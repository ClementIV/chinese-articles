\article{駱統傳}

\begin{pinyinscope}
駱統字公緒,會稽烏傷人也。父俊,官至陳相,為袁術所害。

謝承後漢書曰:俊字孝遠,有文武才幹,少為郡吏,察孝廉,補尚書郎,擢拜陳相。值袁術僭號,兄弟忿爭,天下鼎沸,羣賊並起,陳與比界,姧慝四布,俊厲威武,保疆境,賊不敢犯。養濟百姓,災害不生,歲獲豐稔。後術軍衆饑困,就俊求糧。俊疾惡術,初不應荅。術怒,密使人殺俊。統母改適,為華歆小妻,統時八歲,遂與親客歸會稽。其母送之,拜辭上車,面而不顧,其母泣涕於後。御者曰:「夫人猶在也。」統曰:「不欲增母思,故不顧耳。」事適母甚謹。時饑荒,鄉里及遠方客多有困乏,統為之飲食衰少。其姊仁愛有行,寡歸無子,見統甚哀之,數問其故。統曰:「士大夫糟糠不足,我何心獨飽!」姊曰:「誠如是,何不告我,而自苦若此?」乃自以私粟與統,又以告母,母亦賢之,遂使分施,由是顯名。

孫權以將軍領會稽太守,統年二十,試為烏程相,民戶過萬,咸歎其惠理。權嘉之,召為功曹,行騎都尉,妻以從兄輔女。統志在補察,苟所聞見,夕不待旦。常勸權以尊賢接士,勤求損益,饗賜之日,可人人別進,問其燥濕,加以密意,誘諭使言,察其志趣,令皆感恩戴義,懷欲報之心。權納用焉。出為建忠中郎將,領武射吏三千人。及凌統死,復領其兵。

是時徵役繁數,重以疫癘,民戶損耗,統上疏曰:「臣聞君國者,以據疆土為彊富,制威福為尊貴,曜德義為榮顯,永世胤為豐祚。然財須民生,彊賴民力,威恃民勢,福由民殖,德俟民茂,義以民行,六者旣備,然後應天受祚,保族宜邦。書曰:『衆非后無能胥以寧,后非衆無以辟四方。』推是言之,則民以君安,君以民濟,不易之道也。今彊敵未殄,海內未乂,三軍有無已之役,江境有不釋之備,徵賦調數,由來積紀,加以殃疫死喪之災,郡縣荒虛,田疇蕪曠,聽聞屬城,民戶浸寡,又多殘老,少有丁夫,聞此之日,心若焚燎。思尋所由,小民無知,旣有安土重遷之性,且又前後出為兵者,生則困苦無有溫飽,死則委棄骸骨不反,是以尤用戀本畏遠,同之於死。每有徵發,羸謹居家重累者先見輸送。小有財貨,傾居行賂,不顧窮盡。輕剽者則迸入險阻,黨就羣惡。百姓虛竭,嗷然愁擾,愁擾則不營業,不營業則致窮困,致窮困則不樂生,故口腹急,則姧心動而攜叛多也。又聞民間非居處小能自供,生產兒子,多不起養;屯田貧兵,亦多棄子。天則生之,而父母殺之,旣懼干逆和氣,感動陰陽。且惟殿下開基建國,乃無窮之業也,彊鄰大敵非造次所滅,疆埸常守非期月之戍,而兵民減耗,後生不育,非所以歷遠年,致成功也。夫國之有民,猶水之有舟,停則以安,擾則以危,愚而不可欺,弱而不可勝,是以聖王重焉,禍福由之,故與民消息,觀時制政。方今長吏親民之職,惟以辨具為能,取過目前之急,少復以恩惠為治,副稱殿下天覆之仁,勤恤之德者。官民政俗,日以彫弊,漸以陵遲,勢不可乆。夫治疾及其未篤,除患貴其未深,願殿下少以萬機餘閑,留神思省,補復荒虛,深圖遠計,育殘餘之民,阜人財之用,參曜三光,等崇天地。臣統之大願,足以死而不朽矣。」權感統言,深加意焉。

以隨陸遜破蜀軍於宜都,遷偏將軍。黃武初,曹仁攻濡須,使別將常雕等襲中洲,統與嚴圭共拒破之,封新陽亭侯,後為濡須督。數陳便宜,前後書數十上,所言皆善,文多故不悉載。尤以占募在民間長惡敗俗,生離叛之心,急宜絕置,權與相反覆,終遂行之。年三十六,黃武七年卒。


\end{pinyinscope}