\article{zhao-wang-gan-chuan}

\begin{pinyinscope}
趙王幹,建安二十年封高平亭侯。二十二年,徙封賴亭侯。其年改封弘農侯。黃初二年,進爵,徙封燕公。

魏略曰:幹一名良。良本陳妾子,良生而陳氏死,太祖令王夫人養之。良年五歲而太祖疾困,遺令語太子曰:「此兒三歲亡母,五歲失父,以累汝也。」太子由是親待,隆於諸弟。良年小,常呼文帝為阿翁,帝謂良曰:「我,汝兄耳。」文帝又愍其如是,每為流涕。臣松之案:此傳以母貴賤為次,不計兄弟之年,故楚王彪年雖大,傳在幹後。尋朱建平傳,知彪大幹二十歲。三年,為河間王。五年,改封樂城縣。七年,徙封鉅鹿。太和六年,改封趙王。幹母有寵於太祖。及文帝為嗣,幹母有力。文帝臨崩,有遺詔,是以明帝常加恩意。青龍二年,私通賔客,為有司所奏,賜幹璽書誡誨之,曰:「易稱『開國承家,小人勿用』,詩著『大車惟塵』之誡。自太祖受命創業,深覩治亂之源,鑒存亡之機,初封諸侯,訓以恭慎之至言,輔以天下之端士,常稱馬援之遺誡,重諸侯賔客交通之禁,乃使與犯妖惡同。夫豈以此薄骨肉哉?徒欲使子弟無過失之愆,士民無傷害之悔耳。高祖踐阼,祗慎萬機,申著諸侯不朝之令。朕感詩人常棣之作,嘉采菽之義,亦緣詔文曰『若有詔得詣京都』,故命諸王以朝聘之禮。而楚、中山並犯交通之禁,趙宗、戴捷咸伏其辜。近東平王復使屬官毆壽張吏,有司舉奏,朕裁削縣。今有司以曹纂、王喬等因九族時節,集會王家,或非其時,皆違禁防。朕惟王幼少有恭順之素,加受先帝顧命,欲崇恩禮,延乎後嗣,況近在王之身乎?且自非聖人,孰能無過?已詔有司宥王之失。古人有言:『戒慎乎其所不覩,恐懼乎其所弗聞,莫見乎隱,莫顯乎微,故君子慎其獨焉。』叔父茲率先聖之典,以纂乃先帝之遺命,戰戰兢兢,靖恭厥位,稱朕意焉。」景初、正元、景元中,累增邑,并前五千戶。


\end{pinyinscope}