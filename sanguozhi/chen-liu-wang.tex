\article{chen-liu-wang}

\begin{pinyinscope}
陳留王諱奐,字景明,武帝孫,燕王宇子也。甘露三年,封安次縣常道鄉公。高貴鄉公卒,公卿議迎立公。六月甲寅,入于洛陽,見皇太后,是日即皇帝位於太極前殿,大赦,改年,賜民爵及穀帛各有差。

景元元年夏六月丙辰,進大將軍司馬文王位為相國,封晉公,增封二郡,并前滿十,加九錫之禮,一如前詔;諸羣從子弟,其未有侯者皆封亭侯,賜錢千萬,帛萬匹,文王固讓乃止。己未,故漢獻帝夫人節薨,帝臨于華林園,使使持節追謚夫人為獻穆皇后。及葬,車服制度皆如漢氏故事。癸亥,以尚書右僕射王觀為司空,冬十月,觀薨。

十一月,燕王上表賀冬至,稱臣。詔曰:「古之王者,或有所不臣,王將宜依此義。表不稱臣乎!又當為報。夫後大宗者,降其私親,況所繼者重邪!若便同之臣妾,亦情所未安。其皆依禮典處,當務盡其宜。」有司奏,以為「禮莫崇於尊祖,制莫大於正典。陛下稽德期運,撫臨萬國,紹大宗之重,隆三祖之基。伏惟燕王體尊戚屬,正位藩服,躬秉虔肅,率蹈恭德以先萬國;其於正典,闡濟大順,所不得制。聖朝誠宜崇以非常之制,奉以不臣之禮。臣等平議以為燕王章表,可聽如舊式。中詔所施,或存好問,準之義類,則『燕覿之敬』也,可少順聖敬,加崇儀稱,示不敢斥,宜曰『皇帝敬問大王侍御』。至於制書,國之正典,朝廷所以辨章公制,宣昭軌儀於天下者也,宜循法,故曰『制詔燕王』。凡詔命、制書、奏事、上書諸稱燕王者,可皆上平。其非宗廟助祭之事,皆不得稱王名,奏事、上書、文書及吏民皆不得觸王諱,以彰殊禮,加于羣后。上遵王典尊祖之制,俯順聖敬烝烝之心,二者不愆,禮實宜之,可普告施行。」

十二月甲申,黃龍見華陰縣井中。甲午,以司隷校尉王祥為司空。

二年夏五月朔,日有蝕之。秋七月,樂浪外夷韓、濊貊各率其屬來朝貢。八月戊寅,趙王幹薨。甲寅,復命大將軍進爵晉公,加位相國,備禮崇錫,一如前詔;又固辭乃止。

三年春二月,青龍見於軹縣井中。夏四月,遼東郡言肅慎國遣使重譯入貢,獻其國弓三十張,長三尺五寸,楛矢長一尺八寸,石砮三百枚,皮骨鐵雜鎧二十領,貂皮四百枚。冬十月,蜀大將姜維寇洮陽,鎮西將軍鄧艾拒之,破維於侯和,維遁走。是歲,詔祀故軍祭酒郭嘉於太祖廟庭。

四年春二月,復命大將軍進位爵賜一如前詔,又固辭乃止。

夏五月,詔曰:「蜀,蕞爾小國,土狹民寡,而姜維虐用其衆,曾無廢志;往歲破敗之後,猶復耕種沓中,刻剥衆羌,勞役無已,民不堪命。夫兼弱攻昧,武之善經,致人而不至於人,兵家之上略。蜀所恃賴,唯維而已,因其遠離巢窟,用力為易。今使征西將軍鄧艾督帥諸軍,趣甘松、沓中以羅取維,雍州刺史諸葛緒督諸軍趣武都、高樓,首尾踧討。若禽維,便當東西並進,掃滅巴蜀也。」又命鎮西將軍鍾會由駱谷伐蜀。

秋九月,太尉高柔薨。冬十月甲寅,復命大將軍進位爵賜一如前詔。癸卯,立皇后卞氏,十一月,大赦。

自鄧艾、鍾會率衆伐蜀,所至輙克。是月,蜀主劉禪詣艾降,巴蜀皆平。十二月庚戌,以司徒鄭沖為太保。壬子,分益州為梁州。癸丑,特赦益州士民,復除租賦之半。

五年乙卯,以征西將軍鄧艾為太尉,鎮西將軍鍾會為司徒。皇太后崩。

咸熈元年春正月壬戌,檻車徵鄧艾。甲子,行幸長安。壬申,使使者以璧幣祀華山。是月,鍾會反於蜀,為衆所討;鄧艾亦見殺。二月辛卯,特赦諸在益土者。庚申,葬明元郭后。三月丁丑,以司空王祥為太尉,征北將軍何曾為司徒,尚書左僕射荀顗為司空。己卯,進晉公爵為王,封十郡,并前二十。

漢晉春秋曰:晉公旣進爵為王,太尉王祥、司徒何曾、司空荀顗並詣王。顗曰:「相王尊重,何侯與一朝之臣皆已盡敬,今日便當相率而拜,無所疑也。」祥曰:「相國位勢,誠為尊貴,然要是魏之宰相,吾等魏之三公;公、王相去,一階而已,班列大同,安有天子三公可輙拜人者!損魏朝之望,虧晉王之德,君子愛人以禮,吾不為也。」及入,顗遂拜,而祥獨長揖。王謂祥曰:「今日然後知君見顧之重!」丁亥,封劉禪為安樂公。夏五月庚申,相國晉王奏復五等爵。甲戌,改年。癸未,追命舞陽宣文侯為晉宣王,舞陽忠武侯為晉景王。六月,鎮西將軍衞瓘上雍州兵於成都縣獲璧玉印各一,印文似「成信」字,依周成王歸禾之義,宣示百官,藏于相國府。孫盛曰:昔公孫述自以起成都,號曰成。二玉之文,殆述所作也。

初,自平蜀之後,吳寇屯逼永安,遣荊、豫諸軍掎角赴救。七月,賊皆遁退。八月庚寅,命中撫軍司馬炎副貳相國事,以同魯公拜後之義。

癸巳,詔曰:「前逆臣鍾會構造反亂,聚集征行將士,劫以兵威,始吐姦謀,發言桀逆,逼脅衆人,皆使下議,倉卒之際,莫不驚懾。相國左司馬夏侯和、騎士曹屬朱撫時使在成都,中領軍司馬賈輔、郎中羊琇各參會軍事;和、琇、撫皆抗節不撓,拒會凶言,臨危不顧,詞指正烈。輔語散將王起,說『會姦逆凶暴,欲盡殺將士』,又云『相國已率三十萬衆西行討會』,欲以稱張形勢,感激衆心。起出,以輔言宣語諸軍,遂使將士益懷奮勵。宜加顯寵,以彰忠義。其進和、輔爵為鄉侯,琇、撫爵關內侯。起宣傳輔言,告令將士,所宜賞異。其以起為部曲將。」

癸卯,以衞將軍司馬望為驃騎將軍。九月戊午,以中撫軍司馬炎為撫軍大將軍。

辛未,詔曰:「吳賊政刑暴虐,賦斂無極。孫休遣使鄧句,勑交阯太守鎖送其民,發以為兵。吳將呂興因民心憤怒,又承王師平定巴蜀,即糾合豪傑,誅除句等,驅逐太守長吏,撫和吏民,以待國命。九真、日南郡聞興去逆即順,亦齊心響應,與興恊同。興移書日南州郡,開示大計,兵臨合浦,告以禍福;遣都尉唐譜等詣進乘縣,因南中都督護軍霍弋上表自陳。又交阯將吏各上表,言『興創造事業,大小承命。郡有山寇,入連諸郡,懼其計異,各有攜貳。權時之宜,以興為督交阯諸軍事、上大將軍、定安縣侯,乞賜褒獎,以慰邊荒』。乃心欵誠,形於辭旨。昔儀父朝魯,春秋所美;竇融歸漢,待以殊禮。今國威遠震,撫懷六合,方包舉殊裔,混一四表。興首向王化,舉衆稽服,萬里馳義,請吏帥職,宜加寵遇,崇其爵位。旣使興等懷忠感恱,遠人聞之,必皆競勸。其以興為使持節、都督交州諸軍事、南中大將軍,封定安縣侯,得以便宜從事,先行後上。」策命未至,興為下人所殺。

冬十月丁亥,詔曰:「昔聖帝明王,靜亂濟世,保大定功,文武殊塗,勳烈同歸。是故或舞干戚以訓不庭,或陳師旅以威暴慢。至於愛民全國,康惠庶類,必先脩文教,示之軌儀,不得已然後用兵,此盛德之所同也。往者季漢分崩,九土顛覆,劉備、孫權乘間作禍。三祖綏寧中夏,日不暇給,遂使遺寇僭逆歷世。幸賴宗廟威靈,宰輔忠武,爰發四方,拓定庸、蜀,役不浹時,一征而克。自頃江表衰弊,政刑荒闇,巴、漢平定,孤危無援,交、荊、揚、越靡然向風。今交阯偽將呂興已帥三郡,萬里歸命;武陵邑侯相嚴等糾合五縣,請為臣妾;豫章廬陵山民舉衆叛吳,以助北將軍為號。又孫休病死,主帥改易,國內乖違,人各有心。偽將施績,賊之名臣,懷疑自猜,深見忌惡。衆叛親離,莫有固志,自古及今,未有亡徵若此之甚。若六軍震曜,南臨江、漢,吳會之域必扶老攜幼以迎王師,必然之理也。然興動大衆,猶有勞費,宜告喻威德,開示仁信,使知順附和同之利。相國參軍事徐紹、水曹掾孫彧,昔在壽春,並見虜獲。紹本偽南陵督,才質開壯;彧,孫權支屬,忠良見事。其遣紹南還,以彧為副,宣揚國命,告喻吳人,諸所示語,皆以事實,若其覺悟,不損征伐之計,蓋廟勝長筭,自古之道也。其以紹兼散騎常侍,加奉車都尉,封都亭侯;彧兼給事黃門侍郎,賜爵關內侯。紹等所賜妾及男女家人在此者,悉聽自隨,以明國恩,不必使還,以開廣大信。」

丙午,命撫軍大將軍新昌鄉侯炎為晉世子。是歲,罷屯田官以均政役,諸典農皆為太守,都尉皆為令長;勸募蜀人能內移者,給廩二年,復除二十歲。安彌、福祿縣各言嘉禾生。

二年春二月甲辰,朐䏰縣獲靈龜以獻,歸之于相國府。庚戌,以虎賁張脩昔於成都馳馬至諸營言鍾會反逆,以至沒身,賜脩弟倚爵關內侯。夏四月,南深澤縣言甘露降。吳遣使紀陟、弘璆請和。

五月,詔曰:「相國晉王誕敷神慮,光被四海;震燿武功,則威蓋殊荒,流風邁化,則旁洽無外。愍卹江表,務存濟育,戢武崇仁,示以威德。文告所加,承風嚮慕,遣使納獻,以明委順,方寶纖珍,歡以效意。而王謙讓之至,一皆簿送,非所以慰副初附,從其款願也。孫皓諸所獻致,其皆還送,歸之于王,以恊古義。」王固辭乃止。又命晉王冕十有二旒,建天子旌旗,出警入蹕,乘金根車、六馬,備五時副車,置旄頭雲䍐,樂舞八佾,設鍾虡宮縣。進王妃為王后,世子為太子,王子、王女、王孫,爵命之號如舊儀。癸未,大赦。秋八月辛卯,相國晉王薨。壬辰,晉太子炎紹封襲位,總攝百揆,備物典冊,一皆如前。是月,襄武縣言有大人見,長三丈餘,迹長三尺二寸,白髮,著黃單衣,黃巾,柱杖,呼民王始語云:「今當太平。」九月乙未,大赦。戊午,司徒何曾為晉丞相。癸亥,以驃騎將軍司馬望為司徒,征東大將軍石苞為驃騎將軍,征南大將軍陳騫為車騎將軍。乙亥,葬晉文王。閏月庚辰,康居、大宛獻名馬,歸于相國府,以顯懷萬國致遠之勳。

十二月壬戌,天祿永終,歷數在晉。詔羣公卿士具儀設壇于南郊,使使者奉皇帝璽綬冊,禪位于晉嗣王,如漢魏故事。甲子,使使者奉策。遂改次于金墉城,而終館于鄴,時年二十。魏世譜曰:封帝為陳留王。年五十八,大安元年崩,謚曰元皇帝。

評曰:古者以天下為公,唯賢是與。後代世位,立子以適;若適嗣不繼,則宜取旁親明德,若漢之文、宣者,斯不易之常準也。明帝旣不能然,情繫私愛,撫養嬰孩,傳以大器,託付不專,必參枝族,終於曹爽誅夷,齊王替位。高貴公才慧夙成,好問尚辭,蓋亦文帝之風流也;然輕躁忿肆,自蹈大禍。陳留王恭己南面,宰輔統政,仰遵前式,揖讓而禪,遂饗封大國,作賔于晉,比之山陽,班寵有加焉。


\end{pinyinscope}