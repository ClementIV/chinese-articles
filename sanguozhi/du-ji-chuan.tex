\article{du-ji-chuan}

\begin{pinyinscope}
杜畿字伯侯,京兆杜陵人也。

傅子曰:畿,漢御史大夫杜延年之後。延年父周,自南陽徙茂陵,延年徙杜陵,子孫世居焉。少孤,繼母苦之,以孝聞。年二十,為郡功曹,守鄭縣令。縣囚繫數百人,畿親臨獄,裁其輕重,盡決遣之,雖未悉當,郡中奇其年少而有大意也。舉孝廉,除漢中府丞。會天下亂,遂棄官客荊州,建安中乃還。荀彧進之太祖,傅子曰:畿自荊州還,後至許,見侍中耿紀,語終夜。尚書令荀彧與紀比屋,夜聞畿言,異之,旦遣人謂紀曰:「有國士而不進,何以居位?」旣見畿,知之如舊相識者,遂進畿於朝。太祖以畿為司空司直,遷護羌校尉,使持節,領西平太守。魏略曰:畿少有大志。在荊州數歲,繼母亡後,以三輔開通,負其母喪北歸。道為賊所劫略,衆人奔走,畿獨不去。賊射之,畿請賊曰:「卿欲得財耳,今我無物,用射我何為邪?」賊乃止。畿到鄉里,京兆尹張時,河東人也,與畿有舊,署為功曹。嘗嫌其闊達,不助留意於諸事,言此家疏誕,不中功曹也。畿竊云:「不中功曹,中河東太守也。」

太祖旣定河北,而高幹舉并州反。時河東太守王邑被徵,河東人衞固、范先外以請邑為名,而內實與幹通謀。太祖謂荀彧曰:「關西諸將,恃險與馬,征必為亂。張晟寇殽、澠閒,南通劉表,固等因之,吾恐其為害深。河東被山帶河,四鄰多變,當今天下之要地也。君為我舉蕭何、寇恂以鎮之。」彧曰:「杜畿其人也。」傅子曰:彧稱畿勇足以當大難,智能應變,其可試之。於是追拜畿為河東太守。固等使兵數千人絕陝津,畿至不得渡。太祖遣夏侯惇討之,未至。彧謂畿曰:「宜須大兵。」畿曰:「河東有三萬戶,非皆欲為亂也。今兵迫之急,欲為善者無主,必懼而聽於固。固等勢專,必以死戰。討之不勝,四鄰應之,天下之變未息也;討之而勝,是殘一郡之民也。且固等未顯絕王命,外以請故君為名,必不害新君。吾單車直往,出其不意。固為人多計而無斷,必偽受吾。吾得居郡一月,以計縻之,足矣。」遂詭道從郖津度。郖音豆。魏略曰:初,畿與衞固少相侮狎,固常輕畿。畿嘗與固博而爭道,畿嘗謂固曰:「仲堅,我今作河東也。」固褰衣罵之。及畿之官,而固為郡功曹。張時故任京兆。畿迎司隷,與時會華陰,時、畿相見,於儀當各持版。時歎曰:「昨日功曹,今為郡將軍也!」范先欲殺畿以威衆。傅子曰:先云:「旣欲為虎而惡食人肉,失所以為虎矣。今不殺,必為後患。」且觀畿去就,於門下斬殺主簿已下三十餘人,畿舉動自若。於是固曰:「殺之無損,徒有惡名;且制之在我。」遂奉之。畿謂衞固、范先曰:「衞、范,河東之望也,吾仰成而已。然君臣有定義,成敗同之,大事當共平議。」以固為都督,行丞事,領功曹;將校吏兵三千餘人,皆范先督之。固等喜,雖陽事畿,不以為意。固欲大發兵,畿患之,說固曰:「夫欲為非常之事,不可動衆心。今大發兵,衆必擾,不如徐以貲募兵。」固以為然,從之,遂為貲調發,數十日乃定,諸將貪多應募而少遣兵。又入喻固等曰:「人情顧家,諸將掾史,可分遣休息,急緩召之不難。」固等惡逆衆心,又從之。於是善人在外,陰為己援;惡人分散,各還其家,則衆離矣。會白騎攻東垣,高幹入濩澤,上黨諸縣殺長吏,弘農執郡守,固等密調兵未至。畿知諸縣附己,因出,單將數十騎,赴張辟拒守,吏民多舉城助畿者,比數十日,得四千餘人。固等與幹、晟共攻畿,不下,略諸縣,無所得。會大兵至,幹、晟敗,固等伏誅,其餘黨與皆赦之,使復其居業。

是時天下郡縣皆殘破,河東最先定,少耗減。畿治之,崇寬惠,與民無為。民嘗辭訟,有相告者,畿親見為陳大義,遣令歸諦思之,若意有所不盡,更來詣府。鄉邑父老自相責怒曰:「有君如此,柰何不從其教?」自是少有辭訟。班下屬縣,舉孝子、貞婦、順孫,復其繇役,隨時慰勉之。漸課民畜牸牛、草馬,下逮雞豚犬豕,皆有章程。百姓勤農,家家豐實。畿乃曰:「民富矣,不可不教也。」於是冬月脩戎講武,又開學宮,親自執經教授,郡中化之。魏略曰:博士樂詳,由畿而升。至今河東特多儒者,則畿之由矣。

韓遂、馬超之叛也,弘農、馮翊多舉縣邑以應之。河東雖與賊接,民無異心。太祖西征至蒲阪,與賊夾渭為軍,軍食一仰河東。及賊破,餘畜二十餘萬斛。太祖下令曰:「河東太守杜畿,孔子所謂『禹,吾無閒然矣』。增秩中二千石。」太祖征漢中,遣五千人運,運者自率勉曰:「人生有一死,不可負我府君。」終無一人逃亡,其得人心如此。杜氏新書曰:平虜將軍劉勳,為太祖所親,貴震朝廷。嘗從畿求大棗,畿拒以他故。後勳伏法,太祖得其書,歎曰:「杜畿可謂『不媚於竈』者也。」稱畿功美,以下州郡,曰:「昔仲尼之於顏子,每言不能不歎,旣情愛發中,又宜率馬以驥。今吾亦冀衆人仰高山,慕景行也。」魏國旣建,以畿為尚書。事平,更有令曰:「昔蕭何定關中,寇恂平河內,卿有功,閒將授卿以納言之職;顧念河東吾股肱郡,充實之所,足以制天下,故且煩卿卧鎮之。」畿在河東十六年,常為天下最。

文帝即王位,賜爵關內侯。徵為尚書。及踐阼,進封豐樂亭侯。邑百戶,魏略曰:初畿在郡,被書錄寡婦。是時他郡或有已自相配嫁,依書皆錄奪,啼哭道路。畿但取寡者,故所送少;及趙儼代畿而所送多。文帝問畿:「前君所送何少,今何多也?」畿對曰:「臣前所錄皆亡者妻,今儼送生人婦也。」帝及左右顧而失色。守司隷校尉。帝征吳,以畿為尚書僕射,統留事。其後帝幸許昌,畿復居守。受詔作御樓船,於陶河試船,遇風沒。帝為之流涕。魏氏春秋曰:初,畿嘗見童子謂之曰:「司命使我召子。」畿固請之,童子曰:「今將為君求相代者。君其慎勿言!」言卒,忽然不見。至此二十年矣,畿乃言之。其日而卒,時年六十二。詔曰:「昔冥勤其官而水死,稷勤百穀而山死。韋昭國語註稱毛詩傳曰:「冥,契六世孫也,為夏水官,勤於其職而死於水。稷,周棄也,勤播百穀,死於黑水之山。」故尚書僕射杜畿於孟津試船,遂至覆沒,忠之至也。朕甚愍焉。」追贈太僕,謚曰戴侯。子恕嗣。傅子曰:畿與太僕李恢、東安太守郭智有好。恢子豐交結英雋,以才智顯於天下。智子沖有內實而無外觀,州里弗稱也。畿為尚書僕射,二人各脩子孫禮見畿。旣退,畿歎曰:「孝懿無子;非徒無子,殆將無家。君謀為不死也,其子足繼其業。」時人皆以畿為誤。恢死後,豐為中書令,父子兄弟皆誅;沖為代郡太守,卒繼父業;世乃服畿知人。魏略曰李豐父名義,與此不同,義蓋恢之別名也。

恕字務伯,太和中為散騎黃門侍郎。杜氏新書曰:恕少與馮翊李豐俱為父任,緫角相善。及各成人,豐砥礪名行以要世譽,而恕誕節直意,與豐殊趣。豐竟馳名一時,京師之士多為之游說。而當路者或以豐名過其實,而恕被褐懷玉也。由此為豐所不善。恕亦任其自然,不力行以合時。豐以顯仕朝廷,恕猶居家自若。明帝以恕大臣子,擢拜散騎侍郎,數月,轉補黃門侍郎。恕推誠以質,不治飾,少無名譽。及在朝,不結交援,專心向公。每政有得失,常引綱維以正言,於是侍中辛毗等器重之。

時公卿以下大議損益,恕以為「古之刺史,奉宣六條,以清靜為名,威風著稱,今可勿令領兵,以專民事。」俄而鎮北將軍呂昭又領兾州,世語曰:昭字子展,東平人。長子巽,字長悌,為相國掾,有寵於司馬文王。次子安,字仲悌,與嵇康善,與康俱被誅。次子粹,字季悌,河南尹。粹子預,字景虞,御史中丞。乃上疏曰:

帝王之道,莫尚乎安民;安民之術,在於豐財。豐財者,務本而節用也。方今二賊未滅。戎車亟駕,此自熊虎之士展力之秋也。然搢紳之儒,橫加榮慕,搤腕抗論,以孫、吳為首,州郡牧守,咸共忽恤民之術,脩將率之事。農桑之民,競干戈之業,不可謂務本。帑藏歲虛而制度歲廣,民力歲衰而賦役歲興,不可謂節用。今大魏奄有十州之地,而承喪亂之弊,計其戶口不如往昔一州之民,然而二方僭逆,北虜未賔,三邊遘難,繞天略帀;所以統一州之民,經營九州之地,其為艱難,譬策羸馬以取道里,豈可不加意愛惜其力哉?以武皇帝之節儉,府藏充實,猶不能十州擁兵;郡且二十也。今荊、揚、青、徐、幽、并、雍、涼緣邊諸州皆有兵矣,其所恃內充府庫外制四夷者,惟兖、豫、司、兾而已。臣前以州郡典兵,則專心軍功,不勤民事,宜別置將守,以盡治理之務;而陛下復以兾州寵秩呂昭。兾州戶口最多,田多墾闢,又有桑棗之饒,國家徵求之府,誠不當復任以兵事也。若以北方當須鎮守,自可專置大將以鎮安之。計所置吏士之費,與兼官無異。然昭於人才尚復易;中朝苟乏人,兼才者勢不獨多。以此推之,知國家以人擇官,不為官擇人也。官得其人,則政平訟理;政平故民富實,訟理故囹圄空虛。陛下踐阼,天下斷獄百數十人,歲歲增多,至五百餘人矣。民不益多,法不益峻。以此推之,非政教陵遲,牧守不稱之明效歟?往年牛死,通率天下十能損二;麥不半收,秋種未下。若二賊游魂於疆埸,飛芻輓粟,千里不及。究此之術,豈在彊兵乎?武士勁卒愈多,愈多愈病耳。夫天下猶人之體,腹心充實,四支雖病,終無大患;今兖、豫、司、兾亦天下之腹心也。是以愚臣慺慺,實願四州之牧守,獨脩務本之業,以堪四支之重。然孤論難持,犯欲難成,衆怨難積,疑似難分,故累載不為明主所察。凡言此者,類皆疏賤;疏賤之言,實未易聽。若使善策必出於親貴,親貴固不犯四難以求忠愛,此古今之所常患也。

時又大議考課之制,以考內外衆官。恕以為用不盡其人,雖才且無益,所存非所務,所務非世要。上疏曰:

書稱「明試以功,三考黜陟」,誠帝王之盛制。使有能者當其官,有功者受其祿,譬猶烏獲之舉千鈞,良、樂之選驥足也。雖歷六代而考績之法不著,關七聖而課試之文不垂,臣誠以為其法可粗依,其詳難備舉故也。語曰:「世有亂人而無亂法。」若使法可專任,則唐、虞可不須稷、契之佐,殷、周無貴伊、呂之輔矣。今奏考功者,陳周、漢之法為,終京房之本旨,可謂明考課之要矣。於以崇揖讓之風,興濟濟之治,臣以為未盡善也。其欲使州郡考士,必由四科,皆有事效,然後察舉,試辟公府,為親民長吏,轉以功次補郡守者,或就增秩賜爵,此最考課之急務也。臣以為便當顯其身,用其言,使具為課州郡之法,法具施行,立必信之賞,施必行之罰。至於公卿及內職大臣,亦當俱以其職考課之也。

古之三公,坐而論道,內職大臣,納言補闕,無善不紀,無過不舉。且天下至大,萬機至衆,誠非一明所能徧照。故君為元首,臣作股肱,明其一體相須而成也。是以古人稱廊廟之材,非一木之枝;帝王之業,非一士之略。由是言之,焉有大臣守職辨課可以致雍熙者哉!且布衣之交,猶有務信誓而蹈水火,感知己而披肝膽,徇聲名而立節義者;況於束帶立朝,致位卿相,所務者非特匹夫之信,所感者非徒知己之惠,所徇者豈聲名而已乎!

諸蒙寵祿受重任者,不徒欲舉明主於唐、虞之上而已;身亦欲厠稷、契之列。是以古人不患於念治之心不盡,患於自任之意不足,此誠人主使之然也。唐、虞之君,委任稷、契、夔、龍而責成功,及其罪也,殛鯀而放四凶。今大臣親奉明詔,給事目下,其有夙夜在公,恪勤特立,當官不撓貴勢,執平不阿所私,危言危行以處朝廷者,自明主所察也。若尸祿以為高,拱嘿以為智,當官苟在於免負,立朝不忘於容身,絜行遜言以處朝廷者,亦明主所察也。誠使容身保位,無放退之辜,而盡節在公,抱見疑之勢,公義不脩而私議成俗,雖仲尼為謀,猶不能盡一才,又況於世俗之人乎!今之學者,師商、韓而上法術,競以儒家為迂闊,不周世用,此最風俗之流弊,創業者之所致慎也。

後考課竟不行。杜氏新書曰:時李豐為常侍,黃門郎袁侃見轉為吏部郎,荀俁出為東郡太守,二人皆恕之同班友善。

樂安廉昭以才能拔擢,頗好言事。恕上疏極諫曰:

伏見尚書郎廉昭奏左丞曹璠以罰當關不依詔,坐判問。又云「諸當坐者別奏」。尚書令陳矯自奏不敢辭罰,亦不敢以處重為恭,意至懇惻。臣竊愍然為朝廷惜之!夫聖人不擇世而興,不易民而治,然而生必有賢智之佐者,蓋進之以道,率之以禮故也。古之帝王之所以能輔世長民者,莫不遠得百姓之懽心,近盡羣臣之智力。誠使今朝任職之臣皆天下之選,而不能盡其力,不可謂能使人;若非天下之選,亦不可謂能官人。陛下憂勞萬機,或親燈火,而庶事不康,刑禁日弛,豈非股肱不稱之明效歟?原其所由,非獨臣有不盡忠,亦主有不能使。百里奚愚於虞而智於秦,豫讓苟容中行而著節智伯,斯則古人之明驗矣。今臣言一朝皆不忠,是誣一朝也;然其事類,可推而得。陛下感帑藏之不充實,而軍事未息,至乃斷四時之賦衣,薄御府之私穀,帥由聖意,舉朝稱明,與聞政事密勿大臣,寧有懇懇憂此者乎?

騎都尉王才、幸樂人孟思所為不法,振動京都,而其罪狀發於小吏,公卿大臣初無一言。自陛下踐阼以來,司隷校尉、御史中丞寧有舉綱維以督姦宄,使朝廷肅然者邪?若陛下以為今世無良才,朝廷乏賢佐,豈可追望稷、契之遐蹤,坐待來世之儁乂乎!今之所謂賢者,盡有大官而享厚祿矣,然而奉上之節未立,向公之心不一者,委任之責不專,而俗多忌諱故也。臣以為忠臣不必親,親臣不必忠。何者?以其居無嫌之地而事得自盡也。今有疏者毀人不實其所毀,而必曰私報所憎,譽人不實其所譽,而必曰私愛所親,左右或因之以進憎愛之說。非獨毀譽有之,政事損益,亦皆有嫌。陛下當思所以闡廣朝臣之心,篤厲有道之節,使之自同古人,望與竹帛耳。反使如廉昭者擾亂其間,臣懼大臣遂將容身保位,坐觀得失,為來世戒也!

昔周公戒魯侯曰「無使大臣怨乎不已」,言賢愚明皆當世用也。堯數舜之功,稱去四凶,不言大小,有罪則去也。今者朝臣不自以為不能,以陛下為不任也;不自以為不知,以陛下為不問也。陛下何不遵周公之所以用,大舜之所以去?使侍中、尚書坐則侍帷幄,行則從華輦,親對詔問,所陳必達,則羣臣之行能否,皆可得而知;忠能者進,闇劣者退,誰敢依違而不自盡?以陛下之聖明,親與羣臣論議政事,使羣臣人得自盡,人自以為親,人思所以報,賢愚能否,在陛下之所用。以此治事,何事不辦?以此建功,何功不成?每有軍事,詔書常曰:「誰當憂此者邪?吾當自憂耳。」近詔又曰:「憂公忘私者必不然,但先公後私即自辦也。」伏讀明詔,乃知聖思究盡下情,然亦怪陛下不治其本而憂其末也。人之能否,實有本性,雖臣亦以為朝臣不盡稱職也。明主之用人也,使能者不敢遺其力,而不能者不得處非其任。選舉非其人,未必為有罪也;舉朝共容非其人,乃為怪耳。陛下知其不盡力也,而代之憂其職,知其不能也,而教之治其事,豈徒主勞而臣逸哉?雖聖賢並世,終不能以此為治也。

陛下又患臺閣禁令之不密,人事請屬之不絕,聽伊尹作迎客出入之制,選司徒更惡吏以守寺門;威禁由之,實未得為禁之本也。昔漢安帝時,少府竇嘉辟廷尉郭躬無罪之兄子,猶見舉奏,章劾紛紛。近司隷校尉孔羨辟大將軍狂悖之弟,而有司嘿爾,望風希指,甚於受屬。選舉不以實,人事之大者也。臣松之案大將軍,司馬宣王也。晉書云:「宣王第五弟,名通,為司隷從事。」疑恕所云狂悖者。通子順,封龍陽亭侯。晉初受禪,以不達天命,守節不移,削爵土,徙武威。嘉有親戚之寵,躬非社稷重臣,猶尚如此;以今況古,陛下自不督必行之罰以絕阿黨之原耳。伊尹之制,與惡吏守門,非治世之具也。使臣之言少蒙察納,何患於姦不削滅,而養若昭等乎!

夫糾擿姦宄,忠事也,然而世憎小人行之者,以其不顧道理而苟求容進也。若陛下不復考其終始,必以違衆忤世為奉公,密行白人為盡節,焉有通人大才而更不能為此邪?誠顧道理而弗為耳。使天下皆背道而趨利,則人主之所最病者,陛下將何樂焉,胡不絕其萌乎!夫先意承旨以求容美,率皆天下淺薄無行義者,其意務在於適人主之心而已,非欲治天下安百姓也。陛下何不試變業而示之,彼豈執其所守以違聖意哉?夫人臣得人主之心,安業也;處尊顯之官,榮事也;食千鍾之祿,厚實也。人臣雖愚,未有不樂此而喜干迕者也,迫於道,自彊耳。誠以為陛下當憐而佑之,少委任焉,如何反錄昭等傾側之意,而忽若人者乎?今者外有伺隙之寇,內有貧曠之民,陛下當大計天下之損益,政事之得失,誠不可以怠也。

恕在朝八年,其論議亢直,皆此類也。

出為弘農太守,數歲轉趙相,魏略曰:恕在弘農,寬和有惠愛。及遷,以孟康代恕為弘農。康字公休,安平人。黃初中,以於郭后有外屬,并受九親賜拜,遂轉為散騎侍郎。是時,散騎皆以高才英儒充其選,而康獨緣妃嬙雜在其間,故于時皆共輕之,號為阿九。康旣無才敏,因在宂官,博讀書傳,後遂有所彈駮,其文義雅而切要,衆人乃更加意。正始中,出為弘農,領典農校尉。康到官,清己奉職,嘉善而矜不能,省息獄訟,緣民所欲,因而利之。郡領吏二百餘人,涉春遣休,常四分遣一。事無宿諾,時出案行,皆豫勑督郵平水,不得令屬官遣人探候,脩設曲敬。又不欲煩損吏民,常豫勑吏卒,行各持鐮,所在自刈馬草,不止亭傳,露宿樹下,又所從常不過十餘人。郡帶道路,其諸過賔客,自非公法無所出給;若知舊造之,自出於家。康之始拜,衆人雖知其有志量,以其未嘗宰牧,不保其能也;而康恩澤治能乃爾,吏民稱歌焉。嘉平末,徙渤海太守,徵入為中書令,後轉為監。以疾去官。杜氏新書曰:恕遂去京師,營宜陽一泉塢,因其壘壍之固,小大家焉。明帝崩時,人多為恕言者。起家為河東太守,歲餘,遷淮北都督護軍,復以疾去。恕所在,務存大體而已,其樹惠愛,益得百姓歡心,不及於畿。頃之,拜御史中丞。恕在朝廷,以不得當世之和,故屢在外任。復出為幽州刺史,加建威將軍,使持節,護烏丸校尉。時征北將軍程喜屯薊,尚書袁侃等戒恕曰:「程申伯處先帝之世,傾田園讓於青州。足下今俱杖節,使共屯一城,宜深有以待之。」而恕不以為意。至官未期,有鮮卑大人兒,不由關塞,徑將數十騎詣州,州斬所從來小子一人,無表言上。喜於是劾奏恕,下廷尉,當死。以父畿勤事水死,免為庶人,徙章武郡,是歲嘉平元年。杜氏新書曰:喜欲恕折節謝己,諷司馬宋權示之以微意。恕荅權書曰:「況示委曲。夫法天下事,以善意相待,無不致快也;以不善意相待,無不致嫌隙也。而議者言,凡人天性皆不善,不當待以善意,更墮其調中。僕得此輩,隨欲歸蹈滄海乘桴耳,不能自諧在其間也。然以年五十二,不見廢棄,頗遭明達君子亮其本心;若不見亮,使人刳心著地,正與數斤肉相似,何足有所明,故終不自解說。程征北功名宿著,在僕前甚多,有人出征北乎!若令下官事無大小,咨而後行,則非上司彈繩之意;若咨而不從,又非上下相順之宜。故推一心,任一意,直而行之耳。殺胡之事,天下謂之是邪,是僕諧也;呼為非邪,僕自受之,無所怨咎。程征北明之亦善,不明之亦善,諸君子自共為其心耳,不在僕言也。」喜於是遂深文劾恕。恕倜儻任意,而思不防患,終致此敗。

初,恕從趙郡還,陳留阮武亦從清河太守徵,俱自薄廷尉。謂恕曰:「相觀才性可以由公道而持之不厲,器能可以處大官而求之不順,才學可以述古今而志之不一,此所謂有其才而無其用。今向閑暇,可試潛思,成一家言。」在章武,遂著體論八節。杜氏新書曰:以為人倫之大綱,莫重於君臣;立身之基本,莫大於言行;安上理民,莫精於政法;勝殘去殺,莫善於用兵。夫禮也者,萬物之體也,萬物皆得其體,無有不善,故謂之體論。又著興性論一篇,蓋興於為己也。四年,卒於徙所。

甘露二年,河東樂詳年九十餘,上書訟畿之遺績,朝廷感焉。詔封恕子預為豐樂亭侯,邑百戶。魏略曰:樂詳字文載。少好學,建安初,詳聞公車司馬令南郡謝該善左氏傳,乃從南陽步詣該,問疑難諸要,今左氏樂氏問七十二事,詳所撰也。所問旣了而歸鄉里,時杜畿為太守,亦甚好學,署詳文學祭酒,使教後進,於是河東學業大興。至黃初中,徵拜博士。于時太學初立,有博士十餘人,學多褊狹,又不熟悉,略不親教,備員而已。惟詳五業並授,其或難教,質而不解,詳無慍色,以杖畫地,牽譬引類,至忘寢食,以是獨擅名於遠近。詳學旣精悉,又善推步三五,別受詔與太史典定律歷。太和中,轉拜騎都尉。詳學優能少,故歷三世,竟不出為宰守。至正始中,以年老罷歸於舍,本國宗族歸之,門徒數千人。

恕奏議論駮皆可觀,掇其切世大事著于篇。杜氏新書曰:恕弟理,字務仲。少而機察精要,畿奇之,故名之曰理。年二十一而卒。弟寬,字務叔。清虛玄靜,敏而好古。以名臣門戶,少長京師,而篤志博學,絕於世務,其意欲探賾索隱,由此顯名,當塗之士多交焉。舉孝廉,除郎中。年四十二而卒。經傳之義,多所論駮,皆草創未就,惟刪集禮記及春秋左氏傳解,今存于世。預字元凱,司馬宣王女壻。王隱晉書稱預智謀淵博,明於理亂,常稱「德者非所以企及,立功立言,所庶幾也」。大觀羣典,謂公羊、穀梁,詭辨之言。又非先儒說左氏未究丘明意,而橫以二傳亂之。乃錯綜微言,著春秋左氏經傳集解,又參考衆家,謂之釋例,又作盟會圖、春秋長歷,備成一家之學,至老乃成。尚書郎摯虞甚重之,曰:「左丘明本為春秋作傳,而左傳遂自孤行;釋例本為傳設,而所發明何但左傳,故亦孤行。」預有大功名於晉室,位至征南大將軍,開府,封當陽侯,食邑八千戶。子錫,字世嘏,尚書左丞。晉諸公贊曰:嘏有器局。預從兄武,字世將,亦有才望,為黃門郎,為趙王倫所枉殺。嘏子乂,字弘治。少有令名,為丹陽丞,早卒。阮武者,亦拓落大才也。案阮氏譜:武父諶,字士信,徵辟無所就,造三禮圖傳於世。杜氏新書曰:武字文業,闊達博通,淵雅之士。位止清河太守。武弟炳,字叔文,河南尹。精意醫術,撰藥方一部。炳子坦,字弘舒,晉太子少傅,平東將軍。坦弟柯,字士度。荀綽兖州記曰:坦出紹伯父,亡,次兄當襲爵,父愛柯,言名傳之,遂承封。時幼小,不能讓,及長悔恨,遂幅巾而居,後雖出身,未嘗釋也。性純篤閑雅,好禮無違,存心經誥,博學洽聞。選為濮陽王文學,遷領軍長史,喪官。王衍時為領軍,哭之甚慟。


\end{pinyinscope}