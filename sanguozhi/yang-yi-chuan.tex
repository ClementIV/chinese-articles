\article{楊儀傳}

\begin{pinyinscope}
楊儀字威公,襄陽人也。建安中,為荊州刺史傅羣主簿,背羣而詣襄陽太守關羽。羽命為功曹,遣奉使西詣先主。先主與語論軍國計策,政治得失,大恱之,因辟為左將軍兵曹掾。及先主為漢中王,拔儀為尚書。先主稱尊號,東征吳,儀與尚書令劉巴不睦,左遷遙署弘農太守。建興三年,丞相亮以為參軍,署府事,將南行。五年,隨亮漢中。八年,遷長史,加綏軍將軍。亮數出軍,儀常規畫分部,籌度糧穀,不稽思慮,斯須便了。軍戎節度,取辦於儀。亮深惜儀之才幹,憑魏延之驍勇,常恨二人之不平,不忍有所偏廢也。十二年,隨亮出屯谷口。亮卒于敵場。儀旣領軍還,又誅討延,自以為功勳至大,宜當代亮秉政,呼都尉趙正以周易筮之,卦得家人,默然不恱。而亮平生宓指,以儀性狷狹,意在蔣琬,琬遂為尚書令、益州刺史。儀至,拜為中軍師,無所統領,從容而已。

初,儀為先主尚書,琬為尚書郎,後雖俱為丞相參軍長史,儀每從行,當其勞劇,自為年宦先琬,才能踰之,於是怨憤形于聲色,歎咤之音發於五內。時人畏其言語不節,莫敢從也,惟後軍師費禕往慰省之。儀對禕恨望,前後云云,又語禕曰:「往者丞相亡沒之際,吾若舉軍以就魏氏,處世寧當落度如此邪!令人追悔不可復及。」禕密表其言。十三年,廢儀為民,徙漢嘉郡。儀至徙所,復上書誹謗,辭指激切,遂下郡収儀。儀自殺,其妻子還蜀。

楚國先賢傳云:儀兄慮,字威方。少有德行,為江南冠冕。州郡禮召,諸公辟請,皆不能屈。年十七,夭,鄉人宗貴號曰德行楊君。

評曰:劉封處嫌疑之地,而思防不足以自衞。彭羕、廖立以才拔進,李嚴以幹局達,魏延以勇略任,楊儀以當官顯,劉琰舊仕,並咸貴重。覽其舉措,迹其規矩,招禍取咎,無不自己也。


\end{pinyinscope}