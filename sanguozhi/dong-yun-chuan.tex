\article{董允傳}

\begin{pinyinscope}
董允字休昭,掌軍中郎將和之子也。先主立太子,允以選為舍人,徙洗馬。後主襲位,遷黃門侍郎。丞相亮將北征,住漢中,慮後主富於春秋,朱紫難別,以允秉心公亮,欲任以宮省之事。上疏曰:「侍中郭攸之·費禕、侍郎董允等,先帝簡拔以遺陛下,至於斟酌規益,進盡忠言,則其任也。愚以為宮中之事,事無大小,悉以咨之,必能裨補闕漏,有所廣益。若無興德之言,則戮允等以彰其慢。」亮尋請禕為參軍,允遷為侍中,領虎賁中郎將,統宿衞親兵。攸之性素和順,備員而已。

楚國先賢傳曰:攸之,南陽人,以器業知名於時。獻納之任,允皆專之矣。允處事為防制,甚盡匡救之理。後主常欲采擇以充後宮,允以為古者天子后妃之數不過十二,今嬪嬙已具,不宜增益,終執不聽。後主益嚴憚之。尚書令蔣琬領益州刺史,上疏以讓費禕及允,又表「允內侍歷年,翼贊王室,宜賜爵土以襃勳勞。」允固辭不受。後主漸長大,愛宦人黃皓。皓便僻佞慧,欲自容入。允常上則正色匡主,下則數責於皓。皓畏允,不敢為非。終允之世,皓位不過黃門丞。

允甞與尚書令費禕、中典軍胡濟等共期游宴,嚴駕已辦,而郎中襄陽董恢詣允脩敬。恢年少官微,見允停出,逡巡求去,允不許,曰:「本所以出者,欲與同好游談也,今君已自屈,方展闊積,捨此之談,就彼之宴,非所謂也。」乃命解驂,禕等罷駕不行。其守正下士,凡此類也。襄陽記曰:董恢字休緒,襄陽人。入蜀,以宣信中郎副費禕使吳。孫權甞大醉問禕曰:「楊儀、魏延,牧豎小人也。雖甞有鳴吠之益於時務,然旣已任之,勢不得輕,若一朝無諸葛亮,必為禍亂矣。諸君憒憒,曾不知防慮於此,豈所謂貽厥孫謀乎?」禕愕然四顧視,不能即荅。恢目禕曰:「可速言儀、延之不協起於私忿耳,而無黥、韓難御之心也。今方歸除彊賊,混一函夏,功以才成,業由才廣,若捨此不任,防其後患,是猶備有風波而逆廢舟檝,非長計也。」權大笑樂。諸葛亮聞之,以為知言。還未滿三日,辟為丞相府屬,遷巴郡太守。臣松之案:漢晉春秋亦載此語,不云董恢所教,辭亦小異,此二書俱出習氏而不同若此。本傳云「恢年少官微」,若已為丞相府屬,出作巴郡,則官不微矣。以此疑習氏之言為不審的也。延熈六年,加輔國將軍。七年,以侍中守尚書令,為大將軍費禕副貳。九年,卒。華陽國志曰:時蜀人以諸葛亮、蔣琬、費禕及允為四相,一號四英也。

陳祗代允為侍中,與黃皓互相表裏,皓始預政事。祗死後,皓從黃門令為中常侍、奉車騎都尉,操弄威柄,終至覆國。蜀人無不追思允。及鄧艾至蜀,聞皓姦險,收閉,將殺之,而皓厚賂艾左右,得免。

祗字奉宗,汝南人,許靖兄之外孫也。少孤,長於靖家。弱冠知名,稍遷至選曹郎,矜厲有威容。多技藝,挾數術,費禕甚異之,故超繼允內侍。呂乂卒,祗又以侍中守尚書令,加鎮軍將軍,大將軍姜維雖班在祗上,常率衆在外,希親朝政。祗上承主指,下接閹豎,深見信愛,權重於維。景耀元年卒,後主痛惜,發言流涕,乃下詔曰:「祗統職一紀,柔嘉惟則,幹肅有章,和義利物,庶績允明。命不融遠,朕用悼焉。夫存有令問,則亡加美謚,謚曰忠侯。」賜子粲爵關內侯,拔次子裕為黃門侍郎。自祗之有寵,後主追怨允日深,謂為自輕,由祗媚茲一人,皓搆閒浸潤故耳。允孫宏,晉巴西太守。臣松之以為陳羣子泰,陸遜子抗,傳皆以子繫父,不別載姓,及王肅、杜恕、張承、顧劭之流,莫不皆然,惟董允獨否,未詳其意,當以允名位優重,事跡踰父故邪?夏侯玄、陳表並有騂角之美,而亦如泰者,魏書總名此卷云諸夏侯曹傳,故不復稍加品藻。陳武與表俱至偏將軍,以位不相過故也。


\end{pinyinscope}