\article{周魴傳}

\begin{pinyinscope}
周魴字子魚,吳郡陽羨人也。少好學,舉孝廉,為寧國長,轉在懷安。錢唐大帥彭式等蟻聚為寇,以魴為錢唐侯相,旬月之間,斬式首及其支黨,遷丹楊西部都尉。黃武中,鄱陽大帥彭綺作亂,攻沒屬城,乃以魴為鄱陽太守,與胡綜戮力攻討,遂生禽綺,送詣武昌,加昭義校尉。被命密求山中舊族名帥為北敵所聞知者,令譎挑魏大司馬揚州牧曹休。魴荅,恐民帥小醜不足仗任,事或漏泄,不能致休,乞遣親人齎牋七條以誘休:

其一曰:「魴以千載徼幸,得備州民,遠隔江川,敬恪未顯,瞻望雲景,天寔為之。精誠微薄,名位不昭,雖懷焦渴,曷緣見明?狐死首丘,人情戀本,而逼所制,奉覿禮違。每獨矯首西顧,未甞不寤寐勞歎,展轉反側也。今因隙穴之際,得陳宿昔之志,非神啟之,豈能致此!不勝翹企,萬里託命。謹遣親人董岑、邵南等託叛奉牋。時事變故,列於別紙,惟明公君侯垂日月之光,照遠民之趣,永令歸命者有所戴賴。」

其二曰:「魴遠在邊隅,江汜分絕,恩澤教化,未蒙撫及,而於山谷之間,遙陳所懷,懼以大義,未見信納。夫物有感激,計因變生,古今同揆。魴仕東典郡,始願已獲,銘心立報,永矣無貳。豈圖頃者中被橫譴,禍在漏刻,危於投卵,進有離合去就之宜,退有誣罔枉死之咎,雖志行輕微,存沒一節,顧非其所,能不悵然!敢緣古人,因知所歸,拳拳輸情,陳露肝膈。乞降春天之潤,哀拯其急,不復猜疑,絕其委命。事之宣泄,受罪不測,一則傷慈損計,二則杜絕向化者心,惟明使君遠覽前世,矜而愍之,留神所質,速賜祕報。魴當候望舉動,俟須嚮應。」

其三曰:「魴所代故太守廣陵王靖,往者亦以郡民為變,以見譴責,靖勤自陳釋,而終不解,因立密計,欲北歸命,不幸事露,誅及嬰孩。魴旣目見靖事,且觀東主一所非薄,嫿不復厚,雖或蹔舍,終見翦除。今又令魴領郡者,是欲責後效。必殺魴之趣也。雖尚視息,憂惕焦灼,未知軀命,竟在何時。人居世間,猶白駒過隙,而常抱危怖,其可言乎!惟當陳愚,重自披盡,懼以卑賤,未能采納。願明使君少垂詳察,忖度其言。今此郡民,雖外名降首,而故在山草,看伺空隙,欲復為亂,為亂之日,魴命訖矣。東主頃者潛部分諸將,圖欲北進。呂範、孫韶等入淮,全琮、朱桓趨合肥,諸葛瑾、步隲、朱然到襄陽,陸議、潘璋等討梅敷。東主中營自掩石陽,別遣從弟孫奐治安陸城,脩立邸閣,輦貲運糧,以為軍儲,又命諸葛亮進指關西,江邊諸將無復在者,才留三千所兵守武昌耳。若明使君以萬兵從皖南首江渚,魴便從此率厲吏民,以為內應。此方諸郡,前後舉事,垂成而敗者,由無外援使其然耳;若北軍臨境,傳檄屬城,思詠之民,誰不企踵?願明使君上觀天時,下察人事,中參蓍龜,則足昭往言之不虛也。」

其四曰:「所遣董岑、邵南少長家門,親之信之,有如兒子,是以特令齎牋,託叛為辭,目語心計,不宣脣齒,骨肉至親,無有知者。又已勑之,到州當言往降,欲北叛來者得傳之也。魴建此計,任之於天,若其濟也,則有生全之福;邂逅泄漏,則受夷滅之禍。常中夜仰天,告誓星辰。精誠之微,豈能上感,然事急孤窮,惟天是訴耳。遣使之日,載生載死,形存氣亡,魄爽怳惚。私恐使君未深保明,岑、南二人可留其一,以為後信。一齎教還,教還故當言悔叛還首。東主有常科,悔叛還者,皆自原罪。如是彼此俱塞,永無端原。縣命西望,涕筆俱下。」

其五曰:「鄱陽之民,實多愚勁,帥之赴役,未即應人,倡之為變,聞聲響抃。今雖降首,盤節未解,山栖草藏,亂心猶存,而今東主圖興大衆,舉國悉出,江邊空曠,屯塢虛損,惟有諸刺姦耳。若因是際而搔動此民,一旦可得便會,然要恃外援,表裏機互,不爾以往,無所成也。今使君若從皖道進住江上,魴當從南對岸歷口為應。若未徑到江岸,可住百里上,令此閒民知北軍在彼,即自善也。此閒民非苦飢寒而甘兵寇,苦於征討,樂得北屬,但窮困舉事,不時見應,尋受其禍耳。如使石陽及青、徐諸軍首尾相銜,牽綴往兵,使不得速退者,則善之善也。魴生在江、淮,長於時事,見其便利,百舉百捷,時不再來,敢布腹心。」

其六曰:「東主致恨前者不拔石陽,今此後舉,大合新兵,并使潘濬發夷民,人數甚多,聞豫設科條,當以新羸兵置前,好兵在後,攻城之日,云欲以羸兵填壍,使即時破,雖未能然,是事大趣也。私恐石陽城小,不能乆留往兵,明使君速垂救濟,誠宜疾密。王靖之變,其鑒不遠。今魴歸命,非復在天,正在明使君耳。若見救以往,則功可必成,如見救不時,則與靖等同禍。前彭綺時,聞旌麾在逢龍,此郡民大小歡喜,並思立效。若留一月日間,事當大成,恨去電速,東得增衆專力討綺,綺始敗耳。願使君深察此言。」

其七曰:「今舉大事,自非爵號無以勸之,乞請將軍、侯印各五十紐,郎將印百紐,校尉、都尉印各二百紐,得以假授諸魁帥,獎厲其志,并乞請幢麾數十,以為表幟,使山兵吏民,目瞻見之,知去就之分已決,承引所救畫定。又彼此降叛,日月有人,闊狹之間,輒得聞知。今之大事,事宜神密,若省魴牋,乞加隱祕。伏知智度有常,防慮必深,魴懷憂震灼,啟事蒸仍,乞未罪怪。」

魴因別為密表曰:「方北有逋寇,固阻河洛,乆稽王誅,自擅朔土,臣曾不能吐奇舉善,上以光贊洪化,下以輸展萬一,憂心如擣,假寐忘寢。聖朝天覆,含臣無效,猥發優命,勑臣以前誘致賊休,恨不如計。令於郡界求山谷魁帥為北賊所聞知者,令與北通。臣伏思惟,喜怖交集,竊恐此人不可卒得,假使得之,懼不可信,不如令臣譎休,於計為便。此臣得以經年之兾願,逢值千載之一會,輒自督竭,竭盡頑蔽,撰立牋草以誑誘休者,如別紙。臣知無古人單複之術,加卒奉大略,伀矇狼狽,懼以輕愚,忝負特施,豫懷憂灼。臣聞唐堯先天而天弗違,博詢芻蕘,以成盛勳。朝廷神謨,欲必致休於步度之中,靈贊聖規,休必自送,使六軍囊括,虜無孑遺,威風電邁,天下幸甚。謹拜表以聞,并呈牋草,懼於淺局,追用悚息。」被報施行。休果信魴,帥步騎十萬,輜重滿道,徑來入皖。魴亦合衆,隨陸遜橫截休,休幅裂瓦解,斬獲萬計。

魴初建密計時,頻有郎官奉詔詰問諸事,魴乃詣部郡門下,因下髮謝,故休聞之,不復疑慮。事捷軍旋,權大會諸將歡宴,酒酣,謂魴曰:「君下髮載義,成孤大事,君之功名,當書之竹帛。」加裨將軍,賜爵關內侯。

徐衆評曰:夫人臣立功效節,雖非一塗,然各有分也。為將執桴鼓,則有必死之義,志守則有不假器之義,死必得所,義在不苟。魴為郡守,職在治民,非君所命,自占誘敵,髠剔髮膚,以徇功名,雖事濟受爵,非君子所美。

賊帥董嗣負阻劫鈔,豫章、臨川並受其害。臣松之案:孫亮太平二年始立臨川郡,是時未有臨川。吾粲、唐咨甞以三千兵攻守,連月不能拔。魴表乞罷兵,得以便宜從事。魴遣間諜,授以方策,誘狙殺嗣。嗣弟怖懼,詣武昌降於陸遜,乞出平地,自改為善,由是數郡無復憂惕。

魴在郡十三年卒,賞善罰惡,威恩並行。子處,亦有文武材幹,天紀中為東觀令、無難督。虞預晉書曰:處入晉,為御史中丞,多所彈糾,不避彊禦。齊萬年反,以處為建威將軍,西征,衆寡不敵,處臨陣慷慨,奮不顧身,遂死於戰場,追贈平西將軍。處子玘、札,皆有才力,中興之初,並見寵任。其諸子姪悉處列位,為揚土豪右,而札凶淫放恣,為百姓所苦。泰寧中,王敦誅之,滅其族。


\end{pinyinscope}