\article{hua-xin-chuan}

\begin{pinyinscope}
華歆字子魚,平原高唐人也。高唐為齊名都,衣冠無不游行市里。歆為吏,休沐出府,則歸家闔門。議論持平,終不毀傷人。

魏略曰:歆與北海邴原、管寧俱游學,三人相善,時人號三人為「一龍」,歆為龍頭,原為龍腹,寧為龍尾。臣松之以為邴根矩之徽猷懿望,不必有愧華公,管幼安含德高蹈,又恐弗當為尾。魏略此言,未可以定其先後也。同郡陶丘洪亦知名,自以明見過歆。時王芬與豪傑謀廢靈帝。語在武紀。魏書稱芬有大名於天下。芬陰呼歆、洪共定計,洪欲行,歆止之曰:「夫廢立大事,伊、霍之所難。芬性踈而不武,此必無成,而禍將及族。子其無往!」洪從歆言而止。後芬果敗,洪乃服。舉孝廉,除郎中,病,去官。靈帝崩,何進輔政,徵河南鄭泰、潁川荀攸及歆等。歆到,為尚書郎。董卓遷天子長安,歆求出為下邽令,病不行,遂從藍田至南陽。華嶠譜叙曰:歆少以高行顯名。避西京之亂,與同志鄭泰等六七人,間步出武關。道遇一丈夫獨行,願得俱,皆哀欲許之。歆獨曰:「不可。今已在危險之中,禍福患害,義猶一也。無故受人,不知其義。旣以受之,若有進退,可中棄乎!」衆不忍,卒與俱行。此丈夫中道墮井,皆欲棄之。歆曰:「已與俱矣,棄之不義。」相率共還出之,而後別去。衆乃大義之。時袁術在穰,留歆。歆說術使進軍討卓,術不能用。歆欲棄去,會天子使太傅馬日磾安集關東,日磾辟歆為掾。東至徐州,詔即拜歆豫章太守,以為政清靜不煩,吏民感而愛之。魏略曰:揚州刺史劉繇死,其衆願奉歆為主。歆以為因時擅命,非人臣之宜。衆守之連月,卒謝遣之,不從。孫策略地江東,歆知策善用兵,乃幅巾奉迎。策以其長者,待以上賔之禮。胡冲吳歷曰:孫策擊豫章,先遣虞翻說歆。歆荅曰:「乆在江表,常欲北歸;孫會稽來,吾便去也。」翻還報策,策乃進軍。歆葛巾迎策,策謂歆曰:「府君年德名望,遠近所歸;策年幼稚,宜脩子弟之禮。」便向歆拜。華嶠譜叙曰:孫策略有揚州,盛兵徇豫章,一郡大恐。官屬請出郊迎,教曰:「無然。」策稍進,復白發兵,又不聽。及策至,一府皆造閤,請出避之。乃笑曰:「今將自來,何遽避之?」有頃,門下白曰:「孫將軍至。」請見,乃前與歆共坐,談議良乆,夜乃別去。義士聞之,皆長歎息而心自服也。策遂親執子孫之禮,禮為上賔。是時四方賢士大夫避地江南者甚衆,皆出其下,人人望風。每策大會,坐上莫敢先發言,歆時起更衣,則論議讙譁。歆能劇飲,至石餘不亂,衆人微察,常以其整衣冠為異,江南號之曰「華獨坐」。虞溥江表傳曰:孫策在椒丘,遣虞翻說歆。翻旣去,歆請功曹劉壹入議。壹勸歆住城,遣檄迎軍。歆曰:「吾雖劉刺史所置,上用,猶是剖符吏也。今從卿計,恐死有餘責矣。」壹曰:「王景興旣漢朝所用,且爾時會稽人衆盛彊,猶見原恕,明府何慮?」於是夜逆作檄,明旦出城,遣吏齎迎。策便進軍,與歆相見,待以上賔,接以朋友之禮。孫盛曰:夫大雅之處世也,必先審隱顯之期,以定出處之分,否則括囊以保其身,泰則行義以達其道。歆旣無夷、皓韜邈之風,又失王臣匪躬之操,故撓心於邪儒之說,交臂於陵肆之徒,位奪於一豎,節墯於當時。昔許、蔡失位,不得列於諸侯;州公寔來,魯人以為賤恥。方之於歆,咎孰大焉!後策死。太祖在官渡,表天子徵歆。孫權欲不遣,歆謂權曰:「將軍奉王命,始交好曹公,分義未固,使僕得為將軍效心,豈不有益乎?今空留僕,是為養無用之物,非將軍之良計也。」權恱,乃遣歆。賔客舊人送之者千餘人,贈遺數百金。歆皆無所拒,密各題識,至臨去,悉聚諸物,謂諸賔客曰:「本無拒諸君之心,而所受遂多。念單車遠行,將以懷璧為罪,願賔客為之計。」衆乃各留所贈,而服其德。

歆至,拜議郎,參司空軍事,入為尚書,轉侍中,代荀彧為尚書令。太祖征孫權,表歆為軍師。魏國旣建,為御史大夫。文帝即王位,拜相國,封安樂鄉侯。及踐阼,改為司徒。魏書曰:文帝受禪,歆登壇相儀,奉皇帝璽綬,以成受命之禮。華嶠譜叙曰:文帝受禪,朝臣三公已下並受爵位;歆以形色忤時,徙為司徒,而不進爵。魏文帝乆不懌,以問尚書令陳羣曰:「我應天受禪,百辟群后,莫不人人恱喜,形于聲色,而相國及公獨有不怡者,何也?」羣起離席長跪曰:「臣與相國曾臣漢朝,心雖恱喜,義形其色,亦懼陛下實應且憎。」帝大恱,遂重異之。歆素清貧,祿賜以振施親戚故人,家無擔石之儲。公卿嘗並賜沒入生口,唯歆出而嫁之。帝歎息,孫盛曰:盛聞慶賞威刑,必宗於主,權宜宥怒,出自人君。子路私饋,仲尼毀其食器;田氏盜施,春秋著以為譏。斯襃貶之成言,已然之顯義也。孥戮之家,國刑所肅,受賜之室,乾施所加,若在哀矜,理無偏宥。歆居股肱之任。同元首之重,則當公言皇朝,以彰天澤,而默受嘉賜,獨為君子,旣犯作福之嫌,又違必去之義,可謂匹夫之仁,蹈道則未也。魏書曰:歆性周密,舉動詳慎。常以為人臣陳事,務以諷諫合道為貴,就有所言,不敢顯露,故其事多不見載。華嶠譜叙曰:歆淡於財欲,前後寵賜,諸公莫及,然終不殖產業。陳羣常歎曰:「若華公,可謂通而不泰,清而不介者矣。」傅子曰:敢問今之君子?曰:「袁郎中積德行儉,華太尉積德居順,其智可及也,其清不可及也。事上以忠,濟下以仁,晏嬰、行父何以加諸?」下詔曰:「司徒,國之儁老,所與和陰陽理庶事也。今大官重膳,而司徒蔬食,甚無謂也。」特賜御衣,及為其妻子男女皆作衣服。魏書曰:又賜奴婢五十人。

三府議:「舉孝廉,本以德行,不復限以試經。」歆以為「喪亂以來,六籍墮廢,當務存立,以崇王道。夫制法者,所以經盛衰。今聽孝廉不以經試,恐學業遂從此而廢。若有秀異,可特徵用。患於無其人,何患不得哉?」帝從其言。

黃初中,詔公卿舉獨行君子,歆舉管寧,帝以安車徵之。明帝即位,進封博平侯,增邑五百戶,并前千三百戶,轉拜太尉。列異傳曰:歆為諸生時,嘗宿人門外。主人婦夜產。有頃,兩吏詣門,便辟易却,相謂曰:「公在此。」躊躇良乆,一吏曰:「籍當定,柰何得住?」乃前向歆拜,相將入。出並行,共語曰:「當與幾歲?」一人曰:「當三歲。」天明,歆去。後欲驗其事,至三歲,故往問兒消息,果已死。歆乃自知當為公。臣松之按晉陽秋說魏舒少時寄宿事,亦如之。以為理無二人俱有此事,將由傳者不同。今寧信列異。歆稱病乞退,讓位於寧。帝不許。臨當大會,乃遣散騎常侍繆襲奉詔喻指曰:「朕新莅庶事,一日萬機,懼聽斷之不明。賴有德之臣,左右朕躬,而君屢以疾辭位。夫量主擇君,不居其朝,委榮棄祿,不究其位,古人固有之矣,顧以為周公、伊尹則不然。絜身徇節,常人為之,不望之於君。君其力疾就會,以惠予一人。將立席机莚,命百官緫己,以須君到,朕然後御坐。」又詔襲:「須歆必起,乃還。」歆不得已,乃起。

太和中,遣曹真從子午道伐蜀,車駕東幸許昌。歆上疏曰:「兵亂以來,過踰二紀。大魏承天受命,陛下以聖德當成康之隆,宜弘一代之治,紹三王之迹。雖有二賊負險延命,苟聖化日躋,遠人懷德,將襁負而至。夫兵不得已而用之,故戢而時動。臣誠願陛下先留心於治道,以征伐為後事。且千里運糧,非用兵之利;越險深入,無獨克之功。如聞今年徵役,頗失農桑之業。為國者以民為基,民以衣食為本。使中國無饑寒之患,百姓無離土之心,則天下幸甚,二賊之釁,可坐而待也。臣備位宰相,老病日篤,犬馬之命將盡,恐不復奉望鑾蓋,不敢不竭臣子之懷,唯陛下裁察!」帝報曰:「君深慮國計,朕甚嘉之。賊憑恃山川,二祖勞於前世,猶不克平,朕豈敢自多,謂必滅之哉!諸將以為不一探取,無由自弊,是以觀兵以闚其釁。若天時未至,周武還師,乃前事之鑒,朕敬不忘所戒。」時秋大雨,詔真引軍還。

太和五年,歆薨,謚曰敬侯。魏書云:歆時年七十五。子表嗣。初,文帝分歆戶邑,封歆弟緝列侯。表,咸熈中為尚書。華嶠譜叙曰:歆有三子。表字偉容,年二十餘為散騎侍郎。時同寮諸郎共平尚書事,年少,並兼厲鋒氣,要召名譽。尚書事至,或有不便,故遺漏不視,及傳書者去,即入深文論駮。惟表不然,事來有不便,輒與尚書共論盡其意,主者固執,不得已,然後共奏議。司空陳羣等以此稱之。仕晉,歷太子少傅、太常。稱疾致仕,拜光祿大夫。性清淡,常慮天下退理。司徒李胤、司隷王弘等常稱曰:「若此人者,不可得而貴,不可得而賤,不可得而親,不可得而踈。」中子博,歷三縣內史,治有名跡。少子周,黃門侍郎、常山太守,博學有文思。中年遇疾,終于家。表有三子。長子廙,字長駿。晉諸公贊曰:廙有文翰,歷位尚書令、太子少傅,追贈光祿大夫開府。嶠字叔駿,有才學,撰後漢書,世稱為良史。為祕書監、尚書。澹字玄駿,最知名,為河南尹。廙三子。昆字敬倫,清粹有檢,為尚書。薈字敬叔。世語稱薈貴正。恒字敬則,以通理稱。昆,尚書;薈,河南尹;恒,左光祿大夫開府。澹子軼,字彥夏。有當世才志,為江州刺史。


\end{pinyinscope}