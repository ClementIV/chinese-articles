\article{太子璿傳}

\begin{pinyinscope}
後主太子璿,字文衡。母王貴人,本敬哀張皇后侍人也。延熈元年正月策曰:「在昔帝王,繼體立嗣,副貳國統,古今常道。今以璿為皇太子,昭顯祖宗之威,命使行丞相事左將軍朗持節授印緩。其勉脩茂質,祗恪道義,諮詢典禮,敬友師傅,斟酌衆善,翼成爾德,可不務脩以自勗哉!」時年十五。景耀六年冬,蜀亡。咸熈元年正月,鍾會作亂於成都,璿為亂兵所害。

孫盛蜀世譜曰:璿弟,瑤、琮、瓚、諶、恂、璩六人。蜀敗,諶自殺,餘皆內徙。值永嘉大亂,子孫絕滅。唯永孫玄奔蜀,李雄偽署安樂公以嗣禪後。永和三年討李勢,盛參戎行,見玄於成都也。

評曰:易稱有夫婦然後有父子,夫人倫之始,恩紀之隆,莫尚於此矣。是故紀錄,以究一國之體焉。


\end{pinyinscope}