\article{定公}

\begin{pinyinscope}
元年,春,王三月,晉人執宋仲幾于京師。

夏,六月,癸亥,公之喪至自乾侯。

戊辰,公即位。

秋,七月,癸巳,葬我君昭公。

九月,大雩。

立煬宮。

冬,十月,隕霜殺菽。

元年,春,王正月,辛巳,晉魏舒合諸侯之大夫于狄泉,將以城成周,魏子蒞政,衛彪傒曰,將建天子,而易位以令,非義也,大事奸義,必有大咎,晉不失諸侯,魏子其不免乎,是行也,魏獻子屬役於韓簡子,及原壽過,而田於大陸,焚焉,還,卒於甯,范獻子去其柏槨,以其未復命而田也,孟懿子會城成周,庚寅,栽,宋仲幾不受功曰,滕,薛,郳,吾役也,薛宰曰,宋為無道,絕我小國於周,以我適楚,故我常從宋,晉文公為踐土之盟曰,凡我同盟,各復舊職,若從踐土,若從宋亦唯命,仲幾曰,踐土固然,薛宰曰,薛之皇祖奚仲居薛,以為夏車正,奚仲遷于邳,仲虺居薛,以為湯左相,若復舊職,將承王官,何故以役諸侯,仲幾曰,三代各異物,薛焉得有舊,為宋役,亦其職也,士彌牟曰,晉之從政者新,子姑受功歸,吾視諸故府,仲幾曰,縱子忘之,山川鬼神,其忘諸乎,士伯怒謂韓簡子曰,薛徵於人,宋徵於鬼,宋罪大矣,且己無辭而抑我,以神誣我也,啟寵納侮,其此之謂矣,必以仲幾為戮,乃執仲幾以歸,三月,歸諸京師,城三旬而畢,乃歸,諸侯之戌齊高張後,不從諸侯,晉女叔寬曰,周萇弘,齊高張,皆將不免,萇叔違天,高子違人,天之所壞,不可支也,眾之所為,不可奸也。

夏,叔孫成子逆公之喪于乾侯,季孫曰,子家子亟言於我,未嘗不中吾志也,吾欲與之從政,子必止之,且聽命焉,子家子不見叔孫,易幾而哭,叔孫請見子家子,子家子辭曰,羈未得見,而從君以出,君不命而薨,羈不敢見,叔孫使告之曰,公衍,公為,實使群臣不得事君,若公子宋主社稷,則群臣之願也,凡從君出而可以入者,將唯子是聽,子家氏未有後,季孫願與子從政,此皆季孫之願也,使不敢以告,對曰,若立君,則有卿大夫與守龜在,羈弗敢知,若從君者,則貌而出者,入可也,寇而出者,行可也,若羈也,則君知其出也,而未知其入也,羈將逃也,喪及壞隤,公子宋先入,從公者皆自壞隤反,六月,癸亥,公之喪至自乾侯,戊辰公即位,季孫使役如闞,公氏將溝焉,榮鴐鵝曰,生不能事,死又離之,以自旌也,縱子忍之,後必或恥之,乃止,季孫問於榮鴐鵝曰,吾欲為君謚,使子孫知之,對曰,生弗能事,死又惡之,以自信也,將焉用之,乃止,秋,七月,癸巳,葬昭公於墓道南,孔子之為司寇也,溝而合諸墓,昭公出故,季平子禱于煬公,九月,立煬宮。

周鞏簡公棄其子弟而好用遠人。

二年,春,王正月。

夏,五月,壬辰,雉門及兩觀災。

秋,楚人伐吳。

冬,十月,新作雉門及兩觀。

二年,夏,四月,辛酉,鞏氏之群子弟賊簡公。

桐叛楚,吳子使舒鳩氏誘楚人曰,以師臨我,我伐桐,為我使之無忌。

秋,楚囊瓦伐吳師于豫章,吳人見舟于豫章,而潛師于巢,冬,十月,吳軍楚師于豫章,敗之,遂圍巢,克之,獲楚公子繁。

邾莊公與夷射姑飲酒,私出,閽乞肉焉,奪之杖以敲之。

三年,春,王正月,公如晉,至河乃復。

二月,辛卯,邾子穿卒。

夏,四月。

秋,葬邾莊公。

冬,仲孫何忌及邾子盟于拔。

三年,春,二月,辛卯,邾子在門臺,臨廷,閽以缾水沃廷,邾子望見之,怒,閽曰,夷射姑旋焉,命執之,弗得,滋怒,自投于床,廢于鑪炭,爛遂卒,先葬以車五乘,殉五人,莊公卞急而好潔,故及是,秋,九月,鮮虞人敗晉師于平中,獲晉觀虎,恃其勇也。

冬,盟于郯,脩邾好也。

蔡昭侯為兩佩與兩裘以如楚,獻一佩一裘於昭王,昭王服之,以享蔡侯,蔡侯亦服其一,子常欲之,弗與,三年止之,唐成公如楚,有兩肅爽馬,子常欲之,弗與,亦三年止之,唐人或相與謀,請代先從者,許之,飲先從者酒,醉之,竊馬而獻之子常,子常歸唐侯,自拘於司敗,曰,君以弄馬之故,隱君身,棄國家,群臣請相,夫人以償馬,必如之,唐侯曰,寡人之過也,二三子無辱,皆賞之,蔡人聞之,固請而獻佩于子常,子常朝見蔡侯之徒,命有司曰,蔡君之久也,官不共也,明日禮不畢,將死,蔡侯歸及漢,執玉而沈曰,余所有濟漢而南者,有若大川,蔡侯如晉,以其子元,與其大夫之子為質焉,而請伐楚。

四年,春,王二月,癸巳,陳侯吳卒。

三月,公會劉子,晉侯,宋公,蔡侯,衛侯,陳子,鄭伯,許男,曹伯,莒子,邾子,頓子,胡子,滕子,薛伯,杞伯,小邾子,齊國夏,于召陵,侵楚。

夏,四月,庚辰,蔡公孫姓帥師滅沈,以沈子嘉歸,殺之,五月,公及諸侯盟于皋鼬。

杞伯成卒于會。

六月,葬陳惠公。

許遷于容城。

秋,七月,公至自會。

劉卷卒。

葬杞悼公。

楚人圍蔡。

晉士鞅衛孔圉帥師伐鮮虞。

葬劉文公。

冬,十有一月,庚午,蔡侯以吳子及楚人戰于柏舉,楚師敗績,楚囊瓦出奔鄭,庚辰,吳入郢。

四年,春,三月,劉文公合諸侯于召陵,謀伐楚也,晉荀寅求貨於蔡侯,弗得,言於范獻子曰,國家方危,諸侯方貳,將以襲敵,不亦難乎,水潦方降,疾瘧方起,中山不服,棄盟取怨,無損於楚,而失中山,不如辭蔡侯,吾自方城以來,楚未可以得志,祗取勤焉,乃辭蔡侯,晉人假羽旄於鄭,鄭人與之,明日或旆以會,晉於是乎失諸侯,將會,衛子行敬子言於靈公曰,會同難,嘖有煩言,莫之治也,其使祝佗從,公曰善,乃使子魚,子魚辭曰,臣展四體,以率舊職,猶懼不給,而煩刑書,若又共二,徼大罪也,且夫祝,社稷之常隸也,社稷不動,祝不出竟,官之制也,君以軍行,祓社釁鼓,祝奉以從,於是乎出竟,若嘉好之事,君行師從,卿行旅從,臣無事焉,公曰行也,及皋鼬,將長蔡於衛,衛侯使祝佗私於萇弘曰,聞諸道路,不知信否,若聞蔡將先衛,信乎,萇弘曰,信,蔡叔,康叔之兄也,先衛,不亦可乎,子魚曰,以先王觀之,則尚德也,昔武王克商,成王定之,選建明德,以藩屏周,故周公相王室以尹天下,於周為睦,分魯公以大路大旂,夏后氏之璜,封父之繁弱,殷民六族,條氏,徐氏,蕭氏,索氏,長勺氏,尾勺氏,使帥其宗氏,輯其分族,將其類醜,以法則周公,用即命于周,是使之職事于魯,以昭周公之明德,分之土田倍敦,祝宗卜史,備物典策,官司彝器,因商奄之民,命以伯禽,而封於少皞之虛,分康叔以大路,少帛,綪茷,旃旌,大呂,殷民七族,陶氏,施氏,繁氏,錡氏,樊氏,饑氏,終葵氏,封畛土略,自武父以南,及圃田之北竟,取於有閻之土,以共王職,取於相土之東都,以會王之東蒐,聃季授土,陶叔授民,命以康誥,而封於殷虛,皆啟以商政,疆以周索,分唐叔以大路密須之鼓,闕鞏沽洗,懷姓九宗,職官五正,命以唐誥,而封於夏虛,啟以夏政,疆以戎索,三者皆叔也,而有令德,故昭之以分物,不然,文武成康之伯猶多,而不獲是分也,唯不尚年也,管蔡啟商,惎間王室,王於是乎殺管叔而蔡蔡叔,以車七乘,徒七十人,其子蔡仲,改行帥德,周公舉之,以為己卿士,見諸王,而命之以蔡,其命書云,王曰,胡,無若爾考之違王命也,若之何其使蔡先衛也,武王之母弟八人,周公為太宰,康叔為司寇,聃季為司空,五叔無官,豈尚年哉,曹,文之昭也,晉,武之穆也,曹為伯甸,非尚年也,今將尚之,是反先王也,晉文公為踐土之盟,衛成公不在,夷叔,其母弟也,猶先蔡,其載書云,王若曰,晉重,魯申,衛武,蔡甲午,鄭捷,齊潘,宋王臣,莒期,藏在周府,可覆視也,吾子欲復文武之略,而不正其德,將如之何,萇弘說,告劉子,與范獻子謀之,乃長衛侯於盟,反自召陵,鄭子大叔未至而卒,晉趙簡子為之臨甚哀,曰,黃父之會,夫子語我九言曰,無始亂,無怙富,無恃寵,無違同,無敖禮,無驕能,無復怒,無謀非德,無犯非義。

沈人不會于召陵,晉人使蔡伐之,夏,蔡滅沈,秋楚為沈故圍蔡,伍員為吳行人以謀楚,楚之殺郤宛也,伯氏之族出,伯州犁之孫嚭,為吳大宰以謀楚,楚自昭王即位,無歲不有吳師,蔡侯因之,以其子乾與其大夫之子為質於吳,冬,蔡侯,吳子,唐侯,伐楚,舍舟于淮汭,自豫章與楚夾漢,左司馬戌謂子常曰,子沿漢而與之上下,我悉方城外以毀其舟,還塞大隧,直轅,冥阨,子濟漢而伐之,我自後擊之,必大敗之,既謀而行,武城黑謂子常曰,吳用木也,我用革也,不可久也,不如速戰,史皇謂子常,楚人惡子而好司馬,若司馬毀吳舟于淮,塞城口而入,是獨克吳也,子必速戰,不然不免,乃濟漢而陳,自小別至于大別,三戰,子常知不可,欲奔,史皇曰,安求其事,難而逃之,將何所入,子必死之,初罪必盡說,十一月,庚午,二師陳于柏舉,闔廬之弟夫概王,晨請於闔廬曰,楚瓦不仁,其臣莫有死志,先伐之,其卒必奔,而後大師繼之,必克,弗許,夫概王曰,所謂臣義而行,不待命者,其此之謂也,今日我死,楚可入也,以其屬五千,先擊子常之卒,子常之卒奔,楚師亂,吳師大敗之,子常奔鄭,史皇以其乘廣死,吳從楚師,及清發,將擊之,夫概王曰,困獸猶鬥,況人乎,若知不免,而致死,必敗我,若使先濟者知免,後者慕之,蔑有鬥心矣,半濟而後可擊也,從之,又敗之,楚人為食,吳人及之,奔食而從之,敗諸雍澨,五戰及郢,己卯,楚子取其妹季芊,畀我,以出,涉雎,鍼尹固與王同舟,王使執燧象以奔吳師,庚辰,吳入郢,以班處宮,子山處令尹之宮,夫概王欲攻之,懼而去之,夫概王入之,左司馬戌及息而還,敗吳師于雍澨,傷,初,司馬臣闔廬,故恥為禽焉,謂其臣曰,誰能免吾首,吳句卑曰臣賤可乎,司馬曰,我實失子,可哉,三戰皆傷,曰,吾不用也已,句卑布裳,剄而裹之,藏其身而以其首免,楚子涉睢濟江,入于雲中,王寢,盜攻之,以戈擊王,王孫由于以背受之,中肩,王奔鄖,鍾建負季芊以從,由于徐蘇而從,鄖公辛之弟懷,將弒王曰,平王殺吾父,我殺其子,不亦可乎,辛曰,君討臣,誰敢讎之,君命天也,若死天命,將誰讎,詩曰,柔亦不茹,剛亦不吐,不侮矜寡,不畏彊禦,唯仁者能之,違彊陵弱,非勇也,乘人之約,非仁也,滅宗廢祀,非孝也,動無令名,非知也,必犯是,余將殺女,鬥辛與其弟巢,以王奔隨,吳人從之,謂隨人曰,周之子孫,在漢川者,楚實盡之,天誘其衷,致罰於楚,而君又竄之,周室何罪,君若顧報周室,施及寡人,以獎天衷,君之惠也,漢陽之田,君實有之,楚子在公宮之北,吳人在其南,子期似王,逃王,而己為王,曰,以我與之,王必免,隨人卜與之,不吉,乃辭吳曰,以隨之辟小,而密邇於楚,楚實存之,世有盟誓,至于今未改,若難而棄之,何以事君,執事之患,不唯一人,若鳩楚竟,敢不聽命,吳人乃退,鑪金初官於子期氏,實與隨人耍言,王使見辭曰,不敢以約為利,王割子期之心,以與隨人盟,初,伍員與申包胥友,其亡也,謂申包胥曰,我必復楚國,申包胥曰,勉之,子能復之,我必能興之,及昭王在隨,申包胥如秦乞師,曰,吳為封豕長蛇,以荐食上國,虐始於楚,寡君失守社稷,越在草莽,使下臣告急曰,夷德無厭,若鄰於君,疆場之患也,逮吳之未定,君其取分焉,若楚之遂亡,君之土也,若以君靈,撫之,世以事君,秦伯使辭焉,曰,寡人聞命矣,子姑就館,將圖而告,對曰,寡君越在草莽,未獲所伏,下臣何敢即安,立依於庭牆而哭,日夜不絕聲,勺飲不入口,七日,秦哀公為之賦無衣,九頓首而坐,秦師乃出。

五年,春,王三月,辛亥,朔,日有食之。

夏,歸粟于蔡。

於越入吳。

六月,丙申,季孫意如卒。

秋,七月,壬子,叔孫不敢卒。

冬,晉士鞅帥師圍鮮虞。

五年,春,王人殺子朝于楚,夏,歸粟於蔡,以周亟,矜無資,越入吳,吳在楚也。

六月,季平子行東野,還,未至,丙申,卒于房,陽虎將以璵璠斂,仲梁懷弗與,曰,改步改玉,陽虎欲逐之,告公山不狃,不狃曰,彼為君也,子何怨焉,既葬,桓子行東野,及費子洩為費宰,逆勞於郊,桓子敬之,勞仲梁懷,仲梁懷弗敬,子洩怒,謂陽虎,子行之乎。

申包胥以秦師至,秦子蒲,子虎,帥車五百乘以救楚,子蒲曰,吾未知吳道,使楚人先與吳人戰,而自稷會之,大敗夫概王于沂,吳人獲薳射於柏舉,其子帥奔徒以從,子西敗吳師于軍祥,秋,七月,子期,子蒲,滅唐,九月,夫概王歸,自立也,以與王戰而敗,奔楚為堂谿氏,吳師敗楚師于雍澨,秦師又敗吳師,吳師居麇,子期將焚之,子西曰,父兄親暴骨焉,不能收,又焚之,不可,子期曰,國亡矣,死者若有知也,可以歆舊祀,豈憚焚之,焚之而又戰,吳師敗,又戰于公婿之谿,吳師大敗,吳子乃歸,囚闉輿罷,闉輿罷請先,遂逃歸,葉公諸梁之弟后臧,從其母於吳,不待而歸,葉公終不正視。

乙亥,陽虎囚季桓子,及公父文伯,而逐仲梁,懷,冬,十月,丁亥,殺公何藐,己丑,盟桓子于稷門之內,庚寅,大詛逐公父歜及秦遄,皆奔齊。

楚子入于郢,初,鬥辛聞吳人之爭宮也,曰,吾聞之,不讓則不和,不和不可以遠征,吳爭於楚,必有亂,有亂則必歸,焉能定楚,王之奔隨也,將涉於成臼,藍尹亹涉其帑,不與王舟,及寧,王欲殺之,子西曰,子常唯思舊怨以敗,君何效焉,王曰,善,使復其所,吾以志前惡,王賞鬥辛,王孫由于,王孫圉,鍾建,鬥巢,申包胥,王孫賈,宋木,鬥懷,子西曰,請舍懷也,王曰,大德滅小怨,道也,申包胥曰,吾為君也,非為身也,君既定矣,又何求,且吾尤子旗,其又為諸,遂逃賞,王將嫁季芊,季芊辭曰,所以為女子,遠丈夫也,鍾建負我矣,以妻鍾建,以為樂尹,王之在隨也,子西為王輿服,以保路,國于脾洩,聞王所在,而後從王,王使由于城麇,復命子西問高厚焉,弗知,子西曰,不能如辭,城不知高厚小大,何知,對曰,固辭不能,子使余也,人各有能有不能,王遇盜於雲中,余受其戈,其所猶在,袒而示之背,曰,此余所能也,脾洩之事,余亦弗能也,晉士鞅圍鮮虞,報觀虎之役也。

六年,春,王正月,癸亥,鄭游速帥師滅許,以許男歸。

二月,公侵鄭公,至自侵鄭。

夏。

季孫斯,仲孫何忌,如晉。

秋,晉人執宋行人樂祁犁。

冬,城中城。

季孫斯,仲孫忌,帥師圍鄆。

六年,春,鄭滅許,因楚敗也。

二月,公侵鄭,取匡,為晉討鄭之伐胥靡也,往不假道於衛,及還,陽虎使季孟自南門入,出自東門,舍於豚澤,衛侯怒,使彌子瑕追之,公叔文子老矣,輦而如公,曰,尤人而效之,非禮也,昭公之難,君將以文之舒鼎,成之昭兆,定之鞶鑑,苟可以納之,擇用一焉,公子與二三臣之子,諸侯苟憂之,將以為之質,此群臣之所聞也,今將以小忿蒙舊德,無乃不可乎,大姒之子,唯周公康叔為相睦也,而效小人以棄之,不亦誣乎,天將多陽虎之罪以斃之,君姑待之,若何,乃止。

夏,季桓子如晉,獻鄭俘也,陽虎強使孟懿子,往報夫人之幣,晉人兼享之,孟孫立于房外,謂范獻子曰,陽虎若不能居魯,而息肩於晉所,不以為中軍司馬者,有如先君,獻子曰,寡君有官,將使其人,鞅何知焉,獻子謂簡子曰,魯人患陽虎矣,孟孫知其釁,以為必適晉,故強為之請,以取入焉。

四月,己丑,吳大子終纍敗楚舟師,獲潘子臣,小惟子,及大夫七人,楚國大惕,懼亡子期,又以陵師敗于繁揚,令尹子西喜曰,乃今可為矣,於是乎遷郢於鄀,而改紀其政,以定楚國。

周儋翩率王子朝之徒,因鄭人將以作亂于周,鄭於是乎伐馮,滑,胥靡,負黍,狐人,闕外,六月,晉閻沒戍周,且城胥靡。

秋,八月,宋樂祁言於景公曰,諸侯唯我事晉,今使不往,晉其憾矣,樂祁告其宰陳寅,陳寅曰,必使子往,他日,公謂樂祁曰,唯寡人說子之言,子必往,陳寅曰,子立後而行,吾室亦不亡,唯君亦以我為知難而行也,見溷而行,趙簡子逆而飲之酒於綿上,獻楊楯六十於簡子,陳寅曰,昔吾主范氏,今子主趙氏,又有納焉,以楊楯賈禍,弗可為也已,然子死,晉國子孫,必得志於宋,范獻子言於晉侯曰,以君命越疆而使,未致使而私飲酒,不敬二君,不可不討也,乃執樂祁。

陽虎又盟公及三桓於周社,盟國人于亳社,詛于五父之衢。

冬,十二月,天王處于姑蕕,辟儋翩之亂也。

七年,春,王正月。

夏,四月。

秋,齊侯,鄭伯,盟于鹹。

齊人執衛行人,北宮結,以侵衛,齊侯,衛侯,盟于沙。

大雩。

齊國夏帥師伐我西鄙。

九月,大雩,冬,十月。

七年,春,二月,周儋翩入于儀栗以叛。

齊人歸鄆陽關,陽虎居之以為政。

夏,四月,單武公,劉桓公,敗尹氏于窮谷。

秋,齊侯,鄭伯,盟于鹹,徵會于衛,衛侯欲叛晉,諸大夫不可,使北宮結如齊,而私於齊侯,曰,執結以侵我,齊侯從之,乃盟于瑣,齊國夏伐我,陽虎御季桓子,公斂處父御孟懿子,將宵軍齊師,齊師聞之,墮伏而待之,處父曰,虎不圖禍,而必死,苫夷曰,虎陷二子於難,不待有司,余必殺女,虎懼,乃還,不敗。

冬,十一月,戊午,單子,劉子,逆王于慶氏,晉籍秦逆王,己巳,王入于王城,館于公族黨氏,而後朝于莊宮。

八年,春,王正月,公侵齊,公至自侵齊。

二月,公侵齊,三月,公至自侵齊。

曹伯露卒。

夏,齊國夏帥師伐我西鄙。

公會晉師于瓦,公至自瓦。

秋,七月,戊辰,陳侯柳卒。

晉士鞅帥師侵鄭,遂侵衛。

葬曹靖公。

九月,葬陳懷公。

季孫斯,仲孫何忌,帥師侵衛。

冬,衛侯,鄭伯,盟于曲濮。

從祀先公。

盜竊寶玉大弓。

八年,春,王正月,公侵齊,門于陽州,士皆坐列,曰,顏高之弓六鈞,皆取而傳觀之,陽州人出,顏高奪人弱弓,籍丘子鉏擊之,與一人俱斃,偃且射子鉏,中頰,殪,顏息射人中眉,退曰,我無勇,吾志其目也,師退,冉猛偽傷足而先,其兄會乃呼曰,猛也殿。

三月,己丑,單子伐穀城,劉子伐儀栗,辛卯,單子伐簡城,劉子伐盂,以定王室。

趙鞅言於晉侯曰,諸侯唯宋事晉,好逆其使,猶懼不至,今又執之,是絕諸侯也,將歸樂祁,士鞅曰,三年止之,無故而歸之,宋必叛晉,獻子私謂子梁曰,寡君懼不得事宋君,是以止子,子姑使溷代子,子梁以告陳寅,陳寅曰,宋將叛晉,是棄溷也,不如待之,樂祁歸卒於大行,士鞅曰,宋必叛,不如止其尸,以求成焉,乃止諸州。

公侵齊,攻廩丘之郛,主人焚衝,或濡馬褐以救之,遂毀之,主人出,師奔,陽虎偽不見冉猛者,曰,猛在此必敗,猛逐之,顧而無繼,偽顛,虎曰,盡客氣也。

苫越生子,將待事而名之,陽州之役獲焉,名之曰陽州。

夏,齊國夏,高張,伐我西鄙,晉士鞅,趙鞅,荀寅,救我,公會晉師于瓦,范獻子執羔,趙簡子,中行文子,皆執鴈,魯於是始尚羔。

晉師將盟衛侯于鄟澤,趙簡子曰。

群臣誰敢盟衛君者,涉佗成何曰,我能盟之,衛人請執牛耳,成何曰,衛,吾溫原也,焉得視諸侯,將歃,涉佗捘衛侯之手及捥,衛侯怒,王孫賈趨進曰,盟以信禮也,有如衛君,其敢不唯禮是事,而受此盟也,衛侯欲叛晉,而患諸大夫,王孫賈使次于郊,大夫問故,公以晉詬語之,且曰,寡人辱社稷,其改卜嗣,寡人從焉,大夫曰,是衛之禍,豈君之過也,公曰,又有患焉,謂寡人必以而子,與大夫之子為質,大夫曰,苟有益也。公子則往,群臣之子,敢不皆負羈絏以從?將行,王孫賈曰,苟衛國有難,工商未嘗不為患,使皆行而後可,公以告大夫,乃皆將行之,行有日,公朝國人,使賈問焉,曰,若衛叛晉,晉五伐我,病何如矣,皆曰,五伐我,猶可以能戰,賈曰,然則如叛之,病而後質焉,何遲之有,乃叛晉,晉人請改盟,弗許。

秋,晉士鞅會成桓公侵鄭,圍蟲牢,報伊闕也,遂侵衛。

九月,師侵衛,晉故也。

季寤,公鉏極,公山不狃,皆不得志於季氏,叔孫輒無寵於叔孫氏,叔仲志不得志於魯,故五人因陽虎,陽虎欲去三桓,以季寤更季氏,以叔孫輒更叔孫氏,己更孟氏,冬,十月,順祀先公而祈焉,辛卯,禘于僖公,壬辰,將享季氏于蒲圃而殺之,戒都車曰,癸巳至,成宰公斂處父告孟孫曰,季氏戒都車,何故,孟孫曰,吾弗聞,處父曰,然則亂也,必及於子,先備諸,與孟孫以壬辰為期,陽虎前驅,林楚御桓子,虞人以鈹盾夾之,陽越殿,將如蒲圃,桓子咋謂林楚曰,而先皆季氏之良也,爾以是繼之,對曰,臣聞命後,陽虎為政,魯國服焉,違之徵死,死無益於主,桓子曰,何後之有,而能以我適孟氏乎,對曰,不敢愛死,懼不免主,桓子曰,往也,孟氏選圉人之壯者三百人,以為公期築室於門外,林楚怒,馬及衢而騁,陽越射之不中,築者闔門,有自門間射陽越,殺之,陽虎劫公與武叔以伐孟氏,公斂處父,帥成人,自上東門入與陽氏戰于南門之內,弗勝,又戰于棘下,陽氏敗,陽氏說甲如公宮,取寶玉大弓以出,舍于五父之衢,寢而為食其徒曰,追其將至,虎曰,魯人聞余出,喜於徵死,何暇追余,從者曰,嘻,速駕,公斂陽在,公斂陽請追之,孟孫弗許,陽欲殺桓子,孟孫懼而歸之,子言辨舍爵於季氏之廟而出,陽虎入于讙陽關以叛。

鄭駟歂嗣子大叔為政。

九年,春,王正月。

夏,四月,戊申,鄭伯蠆卒。

得寶玉大弓。

六月,葬鄭獻公。

秋,齊侯,衛侯,次于五民。

秦伯卒。

冬,葬秦哀公。

九年,春,宋公使樂大心盟于晉,且逆樂祁之尸,辭,偽有疾,乃使向巢如晉盟,且逆子梁之尸,子明謂桐門,右師出曰,吾猶衰絰,而子擊鍾,何也,右師曰,喪不在此故也,既而告人曰,已衰絰而生子,余何故舍,鍾子明聞之,怒言於公曰,右師將不利戴氏,不肯適晉,將作亂也,不然無疾,乃逐桐門右師。

鄭駟歂殺鄧析,而用其竹刑,君子謂子然於是不忠,苟有可以加於國家者,棄其邪可也,靜女之三章,取彤管焉,竿旄何以告之,取其忠也,故用其道不棄其人,詩云,蔽芾甘棠,勿翦勿伐,召伯所茇,思其人,猶愛其樹,況用其道,而不恤其人乎,子然無以勸能矣。

夏,陽虎歸寶玉大弓,書曰得,器用也,凡獲器用曰得,得用焉曰獲,六月,伐陽關,陽虎使焚萊門,師驚,犯之而出,奔齊,請師以伐魯,曰,三加必取之,齊侯將許之,鮑文子諫曰,臣嘗為隸於施氏矣,魯未可取也,上下猶和,眾庶猶睦,能事大國,而無天菑,若之何取之,陽虎欲勤齊師也,齊師罷,大臣必多死亡,已於是乎奮其詐謀,夫陽虎有寵於季氏,而將殺季孫,以不利魯國而求容焉,親富不親仁,君焉用之,君富於季氏,而大於魯國,茲陽虎所欲傾覆也,魯免其疾,而君又收之,無乃害乎,齊侯執陽虎,將東之,陽虎願東,乃囚諸西鄙,盡借邑人之車,鍥其軸,麻約而歸之,載蔥靈,寢於其中而逃,追而得之,囚於齊,又以蔥靈逃,奔宋,遂奔晉,適趙氏,仲尼曰,趙氏其世有亂乎。

秋,齊侯伐晉夷儀,敝無存之父將室之,辭,以與其弟,曰,此役也,不死,反必取於高國,先登,求自門出,死於霤下,東郭書讓登,犁彌從之,曰,子讓而左,我讓而右,使登者絕而後下,書左,彌先下,書與王猛息,猛曰,我先登,書斂甲曰,曩者之難,今又難焉,猛笑曰,吾從子,如驂之靳,晉車千乘,在中牟,衛侯將如五氏,卜過之,龜焦,衛侯曰,可也,衛車當其半,寡人當其半,敵矣,乃過中牟,中牟人欲伐之,衛褚師圃亡在中牟,曰,衛雖小,其君在焉,未可勝也,齊師克城而驕,其帥又賤,遇必敗之,不如從齊,乃伐齊師,敗之,齊侯致禚,媚,杏,於衛,齊侯賞犁彌,犁彌辭曰,有先登者,臣從之,皙幘而衣貍製,公使視東郭書,曰,乃夫子也,吾貺子,公賞東郭書,辭曰,彼賓旅也,乃賞犁彌,齊師之在夷儀也,齊侯謂夷儀人曰,得敝無存者,以五家免,乃得其尸,公三襚之,與之犀軒,與直蓋而先歸之,坐引者以師哭之,親推之三。

十年,春,王三月,及齊平。

夏,公會齊侯于夾谷,公至自夾谷。

晉趙鞅帥師圍衛。

齊人來歸鄆,讙,龜陰,田。

叔孫州仇,仲孫何忌,帥師圍郈。

秋,叔孫州仇,仲孫何忌,帥師圍郈。

宋樂大心出奔曹,宋公子地出奔陳。

冬,齊侯,衛侯,鄭游速,會于安甫。

叔孫州仇如齊。

宋公之弟辰,暨仲佗,石彄,出奔陳。

十年,春,及齊平。

夏,公會齊侯于祝其,實夾谷,孔丘相,犁彌言於齊侯曰,孔丘知禮而無勇,若使萊人以兵劫魯侯,必得志焉,齊侯從之,孔丘以公退,曰,士兵之,兩君合好,而裔夷之俘,以兵亂之,非齊君所以命諸侯也,裔不謀夏,夷不亂華,俘不干盟,兵不偪好,於神為不祥,於德為愆義,於人為失禮,君必不然,齊侯聞之,遽辟之,將盟,齊人加於載書曰,齊師出竟,而不以甲車三百乘從我者,有如此盟,孔丘使茲無還揖對曰,而不反我汶陽之田,吾以共命者,亦如之,齊侯將享公,孔丘謂梁丘,據,曰,齊魯之故,吾子何不聞焉,事既成矣,而又享之,是勤執事也,且犧象不出門,嘉樂不野合,饗而既具,是棄禮也,若其不具,用秕稗也,用秕稗君辱,棄禮名惡,子盍圖之,夫享所以昭德也,不昭不如其已也,乃不果享,齊人來歸鄆讙龜陰之田。

晉趙鞅圍衛,報夷儀也,初,衛侯伐邯鄲午於寒氏,城其西北而守之,宵熸,及晉圍衛,午以徒七十人門於衛西門,殺人於門中,曰,請報寒氏之役,涉佗曰,夫子則勇矣,然我往,必不敢啟門,亦以徒七十人,旦門焉,步左右,皆至而立,如植,日中不啟門,乃退,反役,晉人討衛之叛故,曰,由涉佗成何,於是執涉佗以求成於衛,衛人不許,晉人遂殺涉佗,成何奔燕,君子曰,此之謂棄禮,必不鈞,詩曰,人而無禮,胡不遄死,涉佗亦遄矣哉。

初,叔孫成子欲立武叔,公若藐固諫曰,不可,成子立之而卒,公南使賊射之,不能殺,公南為馬正,使公若為郈宰,武叔既定,使郈馬正侯犯,殺公若,不能,其圉人曰,吾以劍過朝,公若必曰,誰之劍也,吾稱子以告,必觀之,吾偽固,而授之未,則可殺也,使如之,公若曰,爾欲吳王我乎,遂殺公若,侯犯以郈叛,武叔懿子圍郈,弗克,秋,二子及齊師復圍郈,弗克,叔孫謂郈工師駟赤曰,郈非唯叔孫氏之憂,社稷之患也,將若之何,對曰,臣之業,在揚水卒章之四言矣,叔孫稽首,駟赤謂侯犯曰,居齊魯之際而無事,必不可矣,子盍求事於齊以臨民,不然,將叛,侯犯從之,齊使至,駟赤與郈人為之宣言於郈中,曰,侯犯將以郈易于齊,齊人將遷郈民,眾兇懼,駟赤謂侯犯曰,眾言異矣,子不如易於齊,與其死也,猶是郈也,而得紓焉,何必此,齊人欲以此偪魯,必倍與子地,且盍多舍甲於子之門,以備不虞,侯犯曰,諾,乃多舍甲焉,侯犯請易於齊,齊有司觀郈將至,駟赤使周走呼曰,齊師至矣,郈人大駭,介侯犯之門甲,以圍侯犯,駟赤將射之,侯犯止之,曰,謀免我,侯犯請行,許之,駟赤先如宿,侯犯殿,每出一門,郈人閉之,及郭門,止之曰,子以叔孫氏之甲出,有司若誅之,群臣懼死,駟赤曰,叔孫氏之甲有物,吾未敢以出,犯謂駟赤曰,子止而與之數,駟赤止而納魯人,侯犯奔齊,齊人乃致郈。

宋公子地嬖蘧富臘,十一分其室,而以其五與之,公子地有白馬四,公嬖向魋,魋欲之,公取而朱其尾鬣以與之,地怒,使其徒抶魋而奪之,魋懼將走,公閉門而泣之目盡腫,母弟辰曰,子分室以與獵也,而獨卑魋,亦有頗焉,子為君禮,不過出竟,君必止子,公子地出奔陳,公弗止,辰為之請,弗聽,辰曰,是我迋吾兄也,吾以國人出,君誰與處,冬,母弟辰,暨仲佗,石彄,出奔陳。

武叔聘于齊,齊侯享之,曰,子叔孫,若使郈在君之他竟,寡人何知焉,屬與敝邑際,故敢助君憂之,對曰,非寡君之望也,所以事君,封疆社稷是以,敢以家隸,勤君之執事,夫不令之臣,天下之所惡也,君豈以為寡君賜。

十有一年,春,宋公之弟辰,及仲佗,石彄,公子地,自陳入于蕭以叛。

夏,四月。

秋,宋樂大心自曹入于蕭。

冬,及鄭平,叔還如鄭蒞盟。

十一年,春,宋公母弟辰,暨仲佗,石彄,公子地,入于蕭以叛,秋,樂大心從之,大為宋患,寵向魋故也。

冬,及鄭平,始叛晉也。

十有二年,春,薛伯定卒。

夏,葬薛襄公。

叔孫州仇帥師墮郈。

衛公孟彄帥師伐曹。

季孫斯,仲孫何忌,帥師墮費。

秋,大雩。

冬,十月,癸亥,公會齊侯盟于黃。

十有一月,丙寅,朔,日有食之。

公至自黃。

十有二月,公圍成,公至自圍成。

十二年,夏,衛公孟彄伐曹,克郊還,滑羅殿,未出,不退于列,其御曰,殿而在列,其為無勇乎,羅曰,與其素厲,寧為無勇。

仲由為季氏宰。將墮三都,於是叔孫氏墮郈,季氏將墮費,公山不狃,叔孫輒,帥費人以襲魯,公與三子入于季氏之宮,登武子之臺,費人攻之弗克,入及公側,仲尼命申句須,樂頎,下伐之,費人北,國人追之,敗諸姑蔑,二子奔齊,遂墮費,將墮成,公斂處父謂孟孫,墮成,齊人必至于北門,且成,孟氏之保障也,無成是無孟氏也,子偽不知,我將不墜,冬,十二月,公圍成弗克。

十有三年,春,齊侯,衛侯,次于垂葭。

夏,築蛇淵囿。

大蒐于比蒲。

衛公孟彄帥師伐曹。

晉趙鞅入于晉陽以叛。

冬,晉荀寅,士吉射,入于朝歌以叛,晉趙鞅歸于晉。

薛弒其君比。

十三年,春,齊侯,衛侯,次于垂葭,實郥氏,使師伐晉,將濟河,諸大夫皆曰不可,邴意茲曰可,銳師伐河內,傳必數日而後及絳,絳不三月,不能出河,則我既濟水矣,乃伐河內,齊侯皆斂諸大夫之軒,唯邴意茲乘軒,齊侯欲與衛侯乘,與之宴,而駕乘廣,載甲焉,使告曰,晉師至矣。齊侯曰,比君之駕也。寡人請攝,乃介而與之乘,驅之。或告曰,無晉師,乃止。

晉趙鞅謂邯鄲午曰,歸我衛貢五百家,吾舍諸晉陽,午許諾,歸告其父兄,父兄皆曰,不可,衛是以為邯鄲,而寘諸晉陽,絕衛之道也,不如侵齊而謀之,乃如之,而歸之于晉陽,趙孟怒,召午而囚諸晉陽,使其從者說劍而入,涉賓不可,乃使告邯鄲人曰,吾私有討於午也,二三子唯所欲立,遂殺午,趙稷,涉賓,以邯鄲叛,夏,六月,上軍司馬籍秦圍邯鄲,邯鄲午荀寅之甥也,荀寅范吉射之姻也,而相與睦,故不與圍,邯鄲將作亂,董安于聞之,告趙孟曰,先備諸,趙孟曰,晉國有命,始禍者死,為後可也,安于曰,與其害於民,寧我獨死,請以我說,趙孟不可,秋,七月,范氏中行氏伐趙氏之宮,趙鞅奔晉陽,晉人圍之,范皋夷無寵於范吉射,而欲為亂於范氏,梁嬰父嬖於知文子,文子欲以為卿,韓簡子與中行文子相惡,魏襄子亦與范昭子相惡,故五子謀將逐荀寅,而以梁嬰父代之,逐范吉射而以范皋夷代之,荀躒言於晉侯曰,君命大臣,始禍者死,載書在河,今三臣始禍,而獨逐鞅,刑已不鈞矣,請皆逐之,冬,十一月,荀躒,韓不信,魏曼多,奉公以伐范氏,中行氏,弗克,二子將伐公,齊高彊曰,三折肱知為良醫,唯伐君為不可,民弗與也,我以伐君在此矣,三家未睦,可盡克也,克之,君將誰與,若先伐君,是使睦也,弗聽,遂伐公,國人助公,二子敗,從而伐之,丁未,荀寅,士吉射,奔朝歌,韓魏以趙氏為請,十二月,辛未,趙鞅入于絳,盟于公宮。

初,衛公叔文子朝而請享靈公,退見史鰌而告之,史鰌曰,子必禍矣,子富而君貪,其及子乎,文子曰,然,吾不先告子,是吾罪也,君既許我矣,其若之何,史鰌曰,無害,子臣,可以免,富而能臣,必免於難,上下同之,戌也驕,其亡乎,富而不驕者,鮮,吾唯子之見,驕而不亡者,未之有也,戌必與焉,及文子卒,衛侯始惡於公叔戌,以其富也,公叔戌又將去夫人之黨,夫人愬之曰,戌將為亂。

十有四年,春,衛公叔戌來奔,衛趙陽出奔宋。

二月,辛巳,楚公子結,陳公孫佗人,帥師滅頓,以頓子牂歸。

夏,衛北宮結來奔。

五月,於越敗吳于檇李。

吳子光卒。

公會齊侯,衛侯,于牽,公至自會。

秋,齊侯,宋公,會于洮。

天王使石尚來歸脤。

衛世子蒯瞶出奔宋。

衛公孟彄出奔鄭。

宋公之弟辰。

自蕭來奔。

大蒐于比蒲。

邾子來會公。

城莒父及霄。

十四年,春,衛侯逐公叔戌與其黨,故趙陽奔宋,戌來奔。

梁嬰父惡董安于謂知文子曰,不殺安于,使終為政於趙氏,趙氏必得晉國,盍以其先發難也,討於趙氏,文子使告於趙孟曰,范中行氏雖信為亂,安于則發之,是安于與謀亂也,晉國有命,始禍者死,二子既伏其罪矣,敢以告,趙孟患之,安于曰,我死而晉國寧,趙氏定,將焉用生,人誰不死,吾死莫矣,乃縊而死,趙孟尸諸市,而告於知氏曰,主命戮罪人,安于既伏其罪矣,敢以告,知伯從趙孟盟,而後趙氏定,祀安于於廟。

頓子牂欲事晉,背楚而絕陳好,二月,楚滅頓。

夏,衛北宮結來奔,公叔戌之故也。

吳伐越,越子勾踐禦之,陳于檇李,勾踐患吳之整也,使死士再禽焉,不動,使罪人三行,屬劍於頸,而辭曰,二君有治,臣奸旗鼓,不敏於君之行前,不敢逃刑,敢歸死,遂自剄也,師屬之目,越子因而伐之,大敗之,靈姑浮以戈擊闔廬,闔廬傷將指,取其一屨還,卒於陘,去檇李七里,夫差使人立於庭,苟出入,必謂己曰,夫差,而忘越王之殺而父乎,則對曰,唯不敢忘,三年乃報越。

晉人圍朝歌,公會齊侯,衛侯,于脾上梁之間,謀救范中行氏,析成鮒,小王桃甲,率狄師以襲晉,戰于絳中,不克而還,士鮒奔周,小王桃甲入于朝歌,秋,齊侯,宋公,會于洮,范氏故也,衛侯為夫人南子召宋朝,會于洮,大子蒯聵獻盂于齊,過宋野,野人歌之曰,既定爾婁豬,盍歸吾艾豭,大子羞之,謂戲陽速曰,從我而朝少君,少君見我,我顧乃殺之,速曰諾,乃朝夫人,夫人見大子,大子三顧,速不進,夫人見其色,啼而走曰,蒯聵將殺余,公執其手以登臺,大子奔宋,盡逐其黨,故公孟彄出奔鄭,自鄭奔齊,大子告人曰,戲陽速禍余,戲陽速告人曰,大子則禍余,大子無道,使余殺其母,余不許,將戕於余,若殺夫人,將以余說,余是故許而弗為,以紓余死,諺曰,民保於信,吾以信義也。

冬,十二月,晉人敗范中行氏之師於潞,獲籍秦,高彊又敗鄭師,及范氏之師于百泉。

十有五年,春,王正月,邾子來朝。

鼷鼠食郊牛,牛死,改卜牛。

二月,辛丑,楚子滅胡,以胡子豹歸。

夏,五月,辛亥郊。

壬申,公薨于高寢。

鄭罕達帥師伐宋。

齊侯,衛侯,次于渠蒢。

邾子來奔喪。

秋,七月,壬申,姒氏卒。

八月,庚辰,朔,日有食之。

九月,滕子來會葬。

丁巳,葬我君定公,雨不克葬,戊午,日下吳,乃克葬,辛巳,葬定姒。

冬,城漆。

十五年,春,邾隱公來朝,子貢觀焉,邾子執玉高,其容仰,公受玉卑,其容俯,子貢曰,以禮觀之,二君者皆有死亡焉,夫禮,死生存亡之體也,將左右周旋,進退俯仰,於是乎取之,朝祀喪戎,於是乎觀之,今正月相朝,而皆不度,心巳亡矣,嘉事不體,何以能久,高仰,驕也,卑俯,替也,驕近亂,替近疾,君為主,其先亡乎。

吳之入楚也,胡子盡俘楚邑之近胡者,楚既定,胡子豹又不事楚,曰,存亡有命,事楚何為,多取費焉,二月,楚滅胡。

夏,五月,壬申,公薨,仲尼曰,賜不幸言而中,是使賜多言者也。

鄭罕達敗宋師于老丘。

齊侯,衛侯,次于蘧挐,謀救宋也。

秋,七月,壬申,姒氏卒,不稱夫人,不赴且不祔也,葬定公,雨,不克襄事,禮也,葬定姒,不稱小君,不成喪也。

冬,城漆,書不時告也。


\end{pinyinscope}