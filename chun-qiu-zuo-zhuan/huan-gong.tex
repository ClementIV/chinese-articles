\article{桓公}

\begin{pinyinscope}
元年,春,王正月,公即位。

三月,公會鄭伯于垂,鄭伯以璧假許田。

夏,四月,丁未,公及鄭伯盟于越。

秋,大水。

冬,十月。

元年,春,公即位,修好于鄭,鄭人請復祀周公,卒易祊田,公許之,三月,鄭伯以璧假許田,為周公祊故也。

夏,四月,丁未,公及鄭伯盟于越,結祊成也,盟曰,渝盟無享國。

秋,大水,凡平原出水為大水。

冬,鄭伯拜盟。

宋華父督見孔父之妻于路,目逆而送之,曰,美而豔。

二年,春,王正月,戊申,宋督弒其君與夷,及其大夫孔父。

滕子來朝。

三月,公會齊侯,陳侯,鄭伯,于稷,以成宋亂。

夏,四月,取郜大鼎于宋,戊申,納于大廟。

秋,七月,杞侯來朝。

蔡侯,鄭伯,會于鄧。

九月,入杞。

公及戎盟于唐,冬,公至自唐。

二年,春,宋督攻孔氏,殺孔父而取其妻,公怒,督懼,遂弒殤公,君子以督為有無君之心,而後動於惡,故先書弒其君,會于稷,以成宋亂,為賂故,立華氏也,宋殤公立,十年十一戰,民不堪命,孔父嘉為司馬,督為大宰,故因民之不堪命,先宣言曰,司馬則然,已殺孔父而弒殤公,召莊公于鄭而立之,以親鄭,以郜大鼎賂公,齊陳鄭皆有賂,故遂相宋公,夏,四月,取郜大鼎于宋,戊申,納于大廟,非禮也,臧哀伯諫曰,君人者,將昭德塞違,以臨照百官,猶懼或失之,故昭令德以示子孫,是以清廟茅屋,大路越席,大羹不致,粢食不鑿,昭其儉也,袞,冕,黻,珽,帶,裳,幅,舄,衡,紞,紘,綖,昭其度也,藻率,鞞,鞛,鞶,厲,游,纓,昭其數也,火,龍,黼,黻,昭其文也,五色比象,昭其物也,鍚,鸞,和,鈴,昭其聲也,三辰旂旗,昭其明也,夫德,儉而有度,登降有數,文物以紀之,聲明以發之,以臨照百官,百官於是乎戒懼,而不敢易紀律,今滅德立違,而寘其賂器於大廟,以明示百官,百官象之,其又何誅焉。國家之敗,由官邪也。官之失德,寵賂章也,郜鼎在廟,章孰甚焉,武王克商,遷九鼎于雒邑,義士猶或非之,而況將昭違亂之賂器於大廟,其若之何,公不聽,周內史聞之曰,臧孫達其有後於魯乎,君違,不忘諫之以德。

秋,七月,杞侯來朝,不敬,杞侯歸,乃謀伐之。

蔡侯,鄭伯,會于鄧,始懼楚也。

九月,入杞,討不敬也。

公及戎盟于唐,脩舊好也。

冬,公至自唐,告于廟也。凡公行,告于宗廟,反行飲至,舍爵策勳焉。禮也,特相會,往來稱地,讓事也,自參以上,則往稱地,來稱會,成事也。

初,晉穆侯之夫人姜氏,以條之役生太子,命之曰仇,其弟以千畝之戰生,命之曰成師。師服曰,異哉君之名子也,夫名以制義,義以出禮,禮以體政,政以正民,是以政成而民聽,易則生亂,嘉耦曰妃,怨耦曰仇,古之命也,今君命大子曰仇,弟曰成師,始兆亂矣,兄其替乎,惠之二十四年,晉始亂,故封桓叔于曲沃,靖侯之孫欒賓傅之,師服曰,吾聞國家之立也,本大而末小,是以能固,故天子建國,諸侯立家,卿置側室,大夫有貳宗,士有隸子弟,庶人工商,各有分親,皆有等衰,是以民服事其上,而下無覬覦,今晉,甸侯也,而建國,本既弱矣,其能久乎,惠之三十年,晉潘父弒昭侯而立桓叔,不克,晉人立孝侯,惠之四十五年,曲沃莊伯伐翼,弒孝侯,翼人立其弟鄂侯,鄂侯生哀侯,哀侯侵陘庭之田,陘庭南鄙,啟曲沃伐翼。

三年,春,正月,公會齊侯于嬴。

夏,齊侯,衛侯,胥命于蒲。

六月,公會杞侯于郕。

秋,七月,壬辰朔,日有食之,既。

公子翬如齊逆女,九月,齊侯送姜氏于讙,公會齊侯于讙,夫人姜氏至自齊。

冬,齊侯使其弟年來聘。

有年。

三年,春,曲沃武公伐翼,次于陘庭,韓萬御戎,梁弘為右,逐翼侯于汾隰,驂絓而止,夜獲之,及欒共叔。

會于嬴,成昏于齊也。

夏,齊侯,衛侯,胥命于蒲,不盟也。

公會杞侯于郕,杞求成也。

秋,公子翬如齊逆女,脩先君之好,故曰公子,齊侯送姜氏,非禮也,凡公女嫁于敵國,姊妹則上卿送之,以禮於先君,公子則下卿送之,於大國,雖公子,亦上卿送之,於天子,則諸卿皆行,公不自送,於小國,則上大夫送之。

冬,齊仲年來聘,致夫人也。

芮伯萬之母芮姜,惡芮伯之多寵人也,故逐之,出居于魏。

四年,春,正月,公狩于郎。

夏,天王使宰渠伯糾來聘。

四年,春,正公,月狩于郎,書時禮也。

夏,周宰渠伯糾來聘,父在故名。

秋,秦師侵芮,敗焉,小之也。

冬王師秦師,圍魏,執芮伯以歸。

五年,春,正月,甲戌,己丑,陳侯鮑卒。

夏,齊侯,鄭伯,如紀。

天王使仍叔之子來聘。

葬陳桓公。

城祝丘。

秋,蔡人,衛人,陳人,從王伐鄭。

大雩。

螽。

冬,州公如曹。

五年,春,正月,甲戌,己丑,陳侯鮑卒,再赴也,於是陳亂,文公子佗殺太子免而代之,公疾病而亂作,國人分散,故再赴。

夏,齊侯,鄭伯,朝于紀,欲以襲之,紀人知之。

王奪鄭伯政,鄭伯不朝。

秋,王以諸侯伐鄭,鄭伯禦之,王為中軍,虢公林父將右軍,蔡人,衛人,屬焉,周公黑肩將左軍,陳人屬焉,鄭子元請為左拒,以當蔡人,衛人,為右拒,以當陳人,曰,陳亂,民莫有鬥心,若先犯之,必奔,王卒顧之,必亂,蔡衛不枝,固將先奔,既而萃於王卒,可以集事,從之,曼伯為右拒,祭仲足為左拒,原繁,高渠彌,以中軍奉公為魚麗之陳,先偏後伍,伍承彌縫,戰于繻葛,命二拒曰,旝動而鼓,蔡衛陳皆奔,王卒亂,鄭師合以攻之,王卒大敗,祝聃射王中肩,王亦能軍,祝聃請從之,公曰,君子不欲多上人,況敢陵天子乎,苟自救也,社稷無隕多矣,夜,鄭伯使祭足勞王,且問左右。

仍叔之子,弱也。

秋,大雩,書不時也,凡祀,啟蟄而郊,龍見而雩,始殺而嘗,閉蟄而烝,過則書。

冬,淳于公如曹,度其國危,遂不復。

六年,春,正月,寔來。

夏,四月,公會紀侯于成。

秋,八月,壬午,大閱,蔡人殺陳佗。

九月,丁卯,子同生。

冬,紀侯來朝。

六年,春,自曹來朝,書曰,寔來,不復其國也。

楚武王侵隨,使薳章求成焉,軍於瑕以待之,隨人使少師董成,鬥伯比言于楚子曰,吾不得志於漢東也,我則使然。我張吾三軍而被吾甲兵,以武臨之。彼則懼而協以謀我,故難間也,漢東之國,隨為大,隨張,必棄小國,小國離,楚之利也,少師侈,請羸師以張之,熊率且比曰,季梁在,何益,鬥伯比曰,以為後圖,少師得其君,王毀軍而納少師,少師歸,請追楚師,隨侯將許之,季梁止之曰,天方授楚,楚之羸,其誘我也,君何急焉,臣聞小之能敵大也,小道大淫,所謂道,忠於民而信於神也,上思利民,忠也,祝史正辭,信也,今民餒而君逞欲,祝史矯舉以祭,臣不知其可也,公曰,吾牲牷肥腯,粢盛豐備,何則不信,對曰,夫民,神之主也,是以聖王先成民,而後致力於神,故奉牲以告曰,博碩肥腯,謂民力之普存也,謂其畜之碩大蕃滋也,謂其不疾瘯蠡也,謂其備腯咸有也,奉盛以告曰,絜粢豐盛,謂其三時不害,而民和年豐也,奉酒醴以告曰,嘉栗旨酒,謂其上下皆有嘉德,而無違心也,所謂馨香,無讒慝也,故務其三時,脩其五教,親其九族,以致其禋祀,於是乎民和而神降之福,故動則有成,今民各有心,而鬼神乏主,君雖獨豐,其何福之有,君姑脩政而親兄弟之國,庶免於難,隨侯懼而脩政,楚不敢伐。

夏,會于成,紀來諮謀齊難也。

北戎伐齊,齊使乞師于鄭,鄭大子忽帥師救齊,六月,大敗戎師,獲其二帥,大良,少良,甲首三百,以獻於齊,於是諸侯之大夫戍齊,齊人饋之餼,使魯為其班,後鄭,鄭忽以其有功也,怒,故有郎之師,公之未昏於齊也,齊侯欲以文姜妻鄭大子,忽大子忽辭,人問其故,大子曰,人各有耦,齊大,非吾耦也,詩云,自求多福,在我而已,大國何為,君子曰,善自為謀,及其敗戎師也,齊侯又請妻之,固辭,人問其故,大子曰,無事於齊,吾猶不敢,今以君命,奔齊之急,而受室以歸,是以師昏也,民其謂我何,遂辭諸鄭伯。

秋,大閱,簡車馬也。

九月,丁卯,子同生,以大子生之禮舉之,接以大牢,卜士負之,士妻食之,公與文姜宗婦命之,公問名於申繻,對曰,名有五,有信,有義,有象,有假,有類,以名生為信,以德名為義,以類命為象,取於物為假,取於父為類,不以國,不以官,不以山川,不以隱疾,不以畜牲,不以器幣,周人以諱事神,名,終將諱之,故以國則廢名,以官則廢職,以山川則廢主,以畜牲則廢祀,以器幣則廢禮,晉以僖侯廢司徒,宋以武公廢司空,先君獻武廢二山,是以大物不可以命,公曰,是其生也,與吾同物,命之曰同。

冬,紀侯來朝,請王命以求成于齊,公告不能。

七年,春,二月,己亥,焚咸丘。

夏,穀伯綏來朝。

鄧侯吾離來朝。

七年,春,穀伯,鄧侯,來朝,名,賤之也。

夏盟,向,求成于鄭,既而背之。

秋,鄭人,齊人,衛人,伐盟,向,王遷盟向之民于郟。

冬,曲沃伯誘晉小子侯殺之。

八年,春,正月,己卯,烝。

天王使家父來聘。

夏,五月,丁丑,烝。

秋,伐邾。

冬,十月,雨雪。

祭公來,遂逆王后于紀。

八年,春,滅翼。

隨少師有寵楚,鬥伯比曰,可矣,讎有釁,不可失也,夏,楚子合諸侯于沈鹿,黃隨不會,使薳章讓黃,楚子伐隨,軍於漢淮之間,季梁請下之,弗許而後戰,所以怒我而怠寇也,少師謂隨侯曰,必速戰,不然,將失楚師,隨侯禦之,望楚師,季梁曰,楚人上左,君必左,無與王遇,且攻其右,右無良焉,必敗,偏敗,眾乃攜矣,少師曰,不當王,非敵也,弗從,戰于速杞,隨師敗績,隨侯逸,鬥丹獲其戎車,與其戎右,少師,秋,隨及楚平,楚子將不許。鬥伯比曰:天去其疾矣,隨未可克也,乃盟而還。

冬王命虢仲立晉哀侯之弟緡于晉。

祭公來,遂逆王后于紀,禮也。

九年,春,紀季姜歸于京師。

夏,四月。

秋,七月。

冬,曹伯使其世子射姑來朝。

九年,春,紀季姜歸于京師。凡諸侯之女行,唯王后書。

巴子使韓服告于楚,請與鄧為好,楚子使道朔將巴客以聘於鄧,鄧南鄙鄾人,攻而奪之幣,殺道朔,及巴行人,楚子使薳章讓於鄧,鄧人弗受,夏,楚使鬥廉帥師,及巴師圍鄾,鄧養甥,聃甥,帥師救鄾,三逐巴師不克,鬥廉衡陳其師於巴師之中,以戰而北,鄧人逐之,背巴師,而夾攻之,鄧師大敗,鄾人宵潰。

秋,虢仲,芮伯,梁伯,荀侯,賈伯,伐曲沃。

冬,曹大子來朝,賓之以上卿,禮也,享曹太子,初獻樂,奏而歎,施父曰,曹大子其有憂乎,非歎所也。

十年,春,王正月,庚申,曹伯終生卒。

夏,五月,葬曹桓公。

秋,公會衛侯于桃丘,弗遇。

冬,十有二月,丙午,齊侯,衛侯,鄭伯,來戰于郎。

十年,春,曹桓公卒。

虢仲譖其大夫詹父於王,詹父有辭,以王師伐虢。

夏,虢公出奔虞。

秋,秦人納芮伯萬于芮。

初,虞叔有玉,虞公求旃,弗獻,既而悔之曰,周諺有之,匹夫無罪,懷璧其罪,吾焉用此,其以賈害也,乃獻,又求其寶劍,叔曰,是無厭也,無厭將及我,遂伐虞公,故虞公出奔共池。

冬,齊衛鄭來戰于郎,我有辭也,初,北戎病齊,諸侯救之,鄭公子忽有功焉,齊人餼諸侯,使魯次之,魯以周班後鄭,鄭人怒,請師於齊,齊人以衛師助之,故不稱侵伐,先書齊,衛王爵也。

十有一年,春,正月,齊人,衛人,鄭人,盟于惡曹。

夏,五月,癸未,鄭伯寤生卒。

秋,七月,葬鄭莊公。

九月,宋人執鄭祭仲,突歸于鄭,鄭忽出奔衛。

柔會宋公,陳侯,蔡叔,盟于折。

公會宋公,于夫鍾,冬,十有二月,公會宋公于闞。

十一年,春,齊,衛,鄭,宋,盟于惡曹。

楚屈瑕將盟貳軫,鄖人軍於蒲騷,將與隨,絞,州,蓼,伐楚師,莫敖患之,鬥廉曰,鄖人軍其郊,必不誡,且日虞四邑之至也,君次於郊郢以禦四邑,我以銳師宵加於鄖,鄖有虞心而恃其城,莫有鬥志,若敗鄖師,四邑必離,莫敖曰,盍請濟師於王,對曰,師克在和,不在眾,商周之不敵,君之所聞也,成軍以出,又何濟焉,莫敖曰,卜之,對曰,卜以決疑,不疑何卜,遂敗鄖師於蒲騷,卒盟而還。

鄭昭公之敗北戎也,齊人將妻之,昭公辭,祭仲曰,必取之,君多內寵,子無大援,將不立,三公子皆君也,弗從。

夏,鄭莊公卒,初,祭封人仲足有寵於莊公,莊公使為卿,為公娶鄧曼,生昭公,故祭仲立之,宋雍氏女於鄭莊公,曰雍姞,生厲公,雍氏宗有寵於宋莊公,故誘祭仲而執之,曰,不立突,將死,亦執厲公而求賂焉,祭仲與宋人盟,以厲公歸而立之。

秋,九月,丁亥,昭公奔衛,已亥,厲公立。

十有二年,春,正月。

夏,六月,壬寅,公會杞侯,莒子,盟于曲池。

秋,七月,丁亥,公會宋公,燕人,盟于穀丘。

八月,壬辰,陳侯躍卒。

公會宋公于虛。

冬,十有一月,公會宋公于龜。

丙戌,公會鄭伯,盟于武父。

丙戌,衛侯晉卒。

十有二月,及鄭師伐宋,丁未,戰于宋。

十二年,夏,盟于曲池,平杞莒也。

公欲平宋鄭,秋,公及宋公盟于句瀆之丘,宋成未可知也,故又會于虛,冬,又會于龜,宋公辭平,故與鄭伯盟于武父,遂帥師而伐宋,戰焉,宋無信也,君子曰,苟信不繼,盟無益也。詩云:君子屢盟,亂是用長,無信也。

楚伐絞,軍其南門,莫敖屈瑕曰,絞小而輕,輕則寡謀,謀無扞采樵者以誘之,從之,絞人獲三十人,明日,絞人爭出,驅楚役徒於山中,楚人坐其北門,而覆諸山下,大敗之,為城下之盟而還,伐絞之役,楚師分涉於彭,羅人欲伐之,使伯嘉諜之,三巡數之。

十有三年,春,二月,公會紀侯,鄭伯,己巳,及齊侯,宋公,衛侯,燕人,戰,齊師,宋師,衛師,燕師,敗績。

三月,葬衛宣公。

夏,大水。

秋,七月。

冬,十月。

十三年,春,楚屈瑕伐羅,鬥伯比送之還,謂其御曰,莫敖必敗,舉趾高,心不固矣,遂見楚子曰,必濟師,楚子辭焉,入告夫人鄧曼,鄧曼曰,大夫其非眾之謂,其謂君撫小民以信,訓諸司以德,而威莫敖以刑也,莫敖狃於蒲騷之役,將自用也,必小羅,君若不鎮撫,其不設備乎,夫固謂君訓眾而好鎮撫之,召諸司而勸之以令德見莫敖而告諸天之不假易也,不然,夫豈不知楚師之盡行也,楚子使賴人追之,不及,莫敖使徇于師曰,諫者有刑,及鄢,亂次以濟,遂無次,且不設備,及羅,羅與盧戎兩軍之,大敗之,莫敖縊于荒谷,群帥囚于冶父,以聽刑,楚子曰,孤之罪也,皆免之。

宋多責賂於鄭,鄭不堪命,故以紀魯,及齊,與宋衛燕戰,不書所戰,後也。

鄭人來請脩好。

十有四年,春,正月,公會鄭伯于曹。

無冰。

夏,五。

鄭伯使其弟語來盟。

秋,八月,壬申,御廩災,乙亥,嘗。

冬,十有二月,丁巳,齊侯祿父卒。

宋人以齊人,蔡人,衛人,陳人,伐鄭。

十四年,春,會于曹,曹人致餼,禮也。

夏,鄭子人來尋盟,且脩曹之會。

秋,八月,壬申,御廩災,乙亥,嘗,書不害也。

冬,宋人以諸侯伐鄭,報宋之戰也,焚渠門,入及大逵,伐東郊,取牛首,以大宮之椽,歸為盧門之椽。

十有五年,春,二月,天王使家父來求車。

三月,乙未,天王崩,夏,四月,己巳,葬齊僖公,五月,鄭伯突出奔,蔡,鄭世子忽復歸于鄭。

許叔入于許。

公會齊侯于艾。

邾人,牟人,葛人,來朝。

秋,九月,鄭伯突入于櫟。

冬,十有一月,公會宋公,衛侯,陳侯,于袲,伐鄭。

十五年,春,天王使家父來求車,非禮也,諸侯不貢車服,天子不私求財。

祭仲專,鄭伯患之,使其婿雍糾殺之,將享諸郊。雍姬知之,謂其母曰,父與夫孰親,其母曰,人盡夫也,父一而已,胡可比也?遂告祭仲曰,雍氏舍其室,而將享子於郊,吾惑之,以告,祭仲殺雍糾,尸諸周氏之汪,公載以出,曰,謀及婦人,宜其死也。

夏,厲公出奔蔡。

六月,乙亥,昭公入。

許叔入于許。

公會齊侯于艾,謀定許也。

秋,鄭伯因櫟人,殺檀伯,而遂居櫟。

冬,會于袲,謀伐鄭,將納厲公也,弗克而還。

十有六年,春,正月,公會宋公,蔡侯,衛侯,于曹。

夏,四月,公會宋公,衛侯,陳侯。

侯,伐鄭。

秋,七月,公至自伐鄭。

冬,城向。

十有一月,衛侯朔出奔齊。

十六年,春,正月,會于曹,謀伐鄭也。

夏,伐鄭,秋,七月,公至自伐鄭,以飲至之禮也。

冬,城向,書時也。

初,衛宣公烝於夷姜,生急子,屬諸右公子,為之娶於齊而美,公取之,生壽,及朔,屬壽於左公子,夷姜縊,宣姜與公子朔構急子,公使諸齊,使盜待諸莘,將殺之,壽子告之,使行,不可,曰,棄父之命,惡用子矣,有無父之國則可也,及行,飲以酒,壽子載其旌以先,盜殺之,急子至曰,我之求也,此何罪,請殺我乎,又殺之,二公子故怨惠公,十一月,左公子洩,右公子職,立公子黔牟,惠公奔齊。

十有七年,春,正月丙辰,公會齊侯、紀侯盟于黃。

二月,丙午,公會邾儀父,盟于趡。

夏五月,丙午,及齊師戰于奚。

六月,丁丑,蔡侯封人卒。

秋八月,蔡季自陳歸于蔡。

癸巳,葬蔡桓侯。

及宋人、衛人伐邾。

冬十月朔,日有食之。

十七年,春,盟于黃,平齊紀,且謀衛故也。

及邾儀父盟于趡,尋蔑之盟也。

夏,及齊師戰于奚,疆事也,於是齊人侵魯疆,疆吏來告,公曰,疆場之事,慎守其一,而備其不虞,姑盡所備焉,事至而戰,又何謁焉。

蔡桓侯卒,蔡人召蔡季于陳。

秋,蔡季自陳歸于蔡,蔡人嘉之也。

伐邾,宋志也。

冬,十月朔,日有食之,不書,日官失之也,天子有日官,諸侯有日御,日官居卿以底日,禮也,日御不失日,以授百官于朝。

初,鄭伯將以高渠彌為卿,昭公惡之,固諫不聽,昭公立,懼其殺己也,辛卯,弒昭公而立公子亹,君子謂昭公知所惡矣,公子達曰,高伯其為戮乎,復惡已甚矣。

十有八年,春,王正月,公會齊侯于濼。

公與夫人姜氏遂如齊。

夏,四月,丙子,公薨于齊,丁酉,公之喪至自齊。

秋,七月。

冬,十有二月,己丑,葬我君桓公。

十八年,春,公將有行,遂與姜氏如齊,申繻曰,女有家,男有室,無相瀆也,謂之有禮,易此必敗,公會齊侯于濼,遂及文姜如齊,齊侯通焉,公謫之,以告。

夏,四月,丙子,享公,使公子彭生乘公,公薨于車,魯人告于齊曰,寡君畏君之威,不敢寧居,來脩舊好,禮成而不反,無所歸咎,惡於諸侯,請以彭生除之,齊人殺彭生。

秋,齊侯師于首止,子亹會之,高渠彌相,七月,戊戌,齊人殺子亹,而轘高渠彌,祭仲逆鄭子于陳而立之,是行也,祭仲知之,故稱疾不往,人曰,祭仲以知免,仲曰,信也。

周公欲弒莊王,而立王子克,辛伯告王,遂與王殺周公黑肩,王子克奔燕,初,子儀有寵於桓王,桓王,屬諸周公,辛伯諫曰,並后,匹嫡,兩政,耦國,亂之本也,周公弗從,故及。


\end{pinyinscope}