\article{僖公}

\begin{pinyinscope}
元年,春,王正月。

齊師,宋師,曹伯,次于聶北,救邢。

夏,六月,邢遷于夷儀,齊師,宋師,曹師,城邢。

秋,七月,戊辰,夫人姜氏薨于夷,齊人以歸。

楚人伐鄭。

八月,公會齊侯,宋公,鄭伯,曹伯,邾人,于檉。

九月,公敗邾師于偃。

冬,十月,壬午,公子友帥師敗莒師于酈,獲莒拏。

十有二月,丁已,夫人氏之喪至自齊。

元年,春,不稱即位,公出故也,公出復入不書,諱之也,諱國惡,禮也。

諸侯救邢,邢人潰,出奔師,師遂逐狄人,具邢器用而遷之,師無私焉。

夏,邢遷于夷儀,諸侯城之,救患也,凡侯伯,救患,分災,討罪,禮也。

秋,楚人伐鄭,鄭即齊故也,盟于犖,謀救鄭也。

九月,公敗邾師于偃,虛丘之戍將歸者也。

冬,莒人來求賂,公子友敗諸酈,獲莒子之弟拏,非卿也,嘉獲之也,公賜季友汶陽之田,及費。

夫人氏之喪至自齊,君子以齊人殺哀姜也,為已甚矣,女子從人者也。

二年,春,王正月,城楚丘。

夏,五月,辛已,葬我小君哀姜。

虞師,晉師,滅下陽。

秋,九月,齊侯,宋公,江人,黃人,盟于貫。

冬,十月,不雨。

楚人侵鄭。

二年,春,諸侯城楚丘而封衛焉,不書,所會後也。

晉荀息請以屈產之乘,與垂棘之璧,假道於虞以伐虢,公曰,是吾寶也,對曰,若得道於虞,猶外府也,公曰,宮之奇存焉,對曰,宮之奇之為人也,懦而不能強諫,且少長於君,君暱之,雖諫,將不聽,乃使荀息假道於虞,曰,冀為不道,入自顛軨,伐鄍三門,冀之既病,則亦唯君故,今虢為不道,保於逆旅,以侵敝邑之南鄙,敢請假道以請罪于虢,虞公許之,且請先伐虢,宮之奇諫,不聽,遂起師,夏,晉里克,荀息,帥師會虞師伐虢,滅下陽,先書虞,賄故也。

秋,盟于貫,服江黃也。

齊寺人貂始漏師于多魚。

虢公敗戎于桑田,晉卜偃曰,虢必亡矣,亡下陽不懼,而又有功,是天奪之鑒,而益其疾也,必易晉而不撫其民矣,不可以五稔。

冬,楚人伐鄭,鬥章囚鄭聃伯。

三年,春,王正月,不雨,夏,四月,不雨。

徐人取舒。

六月,雨。

秋,齊侯,宋公,江人,黃人,會于陽穀。

冬,公子友如齊蒞盟。

楚人伐鄭。

三年,春,不雨,夏,六月,雨,自十月不雨,至于五月,不曰旱,不為災也。

秋,會于陽穀,謀伐楚也。

齊侯為陽穀之會,來尋盟,冬,公子友如齊蒞盟。

楚人伐鄭,鄭伯欲成,孔叔不可曰,齊方勤我,棄德不祥。

齊侯與蔡姬乘舟于囿,蕩公,公懼,變色,禁之不可,公怒,歸之,未絕之也,蔡人嫁之。

四年,春,王正月,公會齊侯,宋公,陳侯,衛侯,鄭伯,許男,曹伯,侵蔡,蔡潰,遂伐楚,次于陘。

夏,許男新臣卒。

楚屈完來盟于師,盟于召陵。

齊人執陳轅濤塗。

秋,及江人,黃人,伐陳。

八月,公至自伐楚。

葬許穆公。

冬,十有二月,公孫茲帥師會齊人,宋人,衛人,鄭人,許人,曹人,侵陳。

四年,春,齊侯以諸侯之師侵蔡,蔡潰,遂伐楚。楚子使與師言曰:君處北海,寡人處南海,唯是風馬牛不相及也。不虞君之涉吾地也,何故,管仲對曰,昔召康公命我先君大公曰,五侯九伯,女實征之,以夾輔周室,賜我先君履。東至于海,西至于河,南至于穆陵,北至于無棣。爾貢包茅不入,王祭不共,無以縮酒,寡人是徵。昭王南征而不復,寡人是問。對曰,貢之不入,寡君之罪也,敢不共給,昭王之不復,君其問諸水濱,師進,次于陘。夏,楚子使屈完如師。師退,次于召陵,齊侯陳諸侯之師,與屈完乘而觀之,齊侯曰,豈不穀是為,先君之好是繼,與不穀同好如何,對曰,君惠徼福於敝邑之社稷,辱收寡君,寡君之願也,齊侯曰,以此眾戰,誰能禦之,以此攻城,何城不克,對曰,君若以德綏諸侯,誰敢不服,君若以力楚國方城以為城,漢水以為池,雖眾,無所用之,屈完及諸侯盟。

陳轅濤塗謂鄭申侯曰,師出於陳鄭之間,國必甚病,若出於東方,觀兵於東夷,循海而歸,其可也,申侯曰善,濤塗以告齊侯,許之,申侯見曰,師老矣,若出於東方而遇敵,懼不可用也,若出於陳鄭之間,共其資糧屝屨,其可也,齊侯說,與之虎牢,執轅濤塗。

秋,伐陳,討不忠也。

許穆公卒于師,葬之以侯,禮也,凡諸侯薨于朝會,加一等,死王事,加二等,於是有以袞斂。

冬,叔孫戴伯帥師,會諸侯之師侵陳,陳成,歸轅濤塗。

初,晉獻公欲以驪姬為夫人,卜之不吉,筮之吉,公曰,從筮,卜人曰,筮短龜長,不如從長,且其繇曰,專之渝,攘公之羭,一薰一蕕,十年尚猶有臭,必不可,弗聽,立之,生奚齊,其娣生卓子,及將立奚齊,既與中大夫成謀,姬謂大子曰,君夢齊姜,必速祭之,大子祭于曲沃,歸胙于公,公田,姬寘諸宮,六日,公至,毒而獻之,公祭之地,地墳,與犬,犬斃,與小臣,小臣亦斃,姬泣曰,賊由大子,大子奔新城,公殺其傅杜原款,或謂大子,子辭,君必辯焉。大子曰:君非姬氏,居不安。食不飽,我辭,姬必有罪,君老矣。吾又不樂,曰,子其行乎,大子曰,君實不察其罪,被此名也以出,人誰納我,十二月,戊申,縊于新城,姬遂譖二公子曰,皆知之,重耳奔蒲,夷吾奔屈。

五年,春,晉侯殺其世子申生。

杞伯姬來,朝其子。

夏,公孫茲如牟。

公及齊侯,宋公,陳侯,衛侯,鄭伯,許男,曹伯,會王世子于首止。

秋,八月,諸侯盟于首止,鄭伯逃歸,不盟。

楚人滅弦,弦子奔黃。

九月,戊申朔,日有食之。

冬,晉人執虞公。

五年,春,王正月,辛亥朔,日南至,公既視朔,遂登觀臺以望,而書,禮也。凡分,至,啟,閉,必書雲物,為備故也。

晉侯使以殺大子申生之故來告,初,晉侯使士蒍為二公子築蒲與屈,不慎,寘薪焉,夷吾訴之,公使讓之,士蒍稽首而對曰,臣聞之,無喪而慼,憂必讎焉,無戎而城,讎必保焉,寇讎之保,又何慎焉,守官廢命,不敬,固讎之保,不忠,失忠與敬,何以事君,詩云,懷德惟寧,宗子惟城,君其脩德而固宗子,何城如之,三年將尋師焉,焉用慎,退而賦曰,狐裘尨茸,一國三公,吾誰適從,及難,公使寺人披伐蒲,重耳曰,君父之命不校,乃徇曰,校者,吾讎也,踰垣而走,披斬其袪,遂出奔翟。

夏,公孫茲如牟,娶焉。

會于首止,會王大子鄭,謀寧周也。

陳轅宣仲怨鄭申侯之反己於召陵,故勸之城其賜邑,曰,美城之,大名也,子孫不忘,吾助子請,乃為之請於諸侯而城之,美遂譖諸鄭伯曰,美城其賜邑,將以叛也,申侯由是得罪。

秋,諸侯盟,王使周公召鄭伯曰,吾撫女以從楚,輔之以晉,可以少安,鄭伯喜於王命,而懼其不朝於齊也,故逃歸不盟,孔叔止之曰,國君不可以輕,輕則失親,失親患必至,病而乞盟,所喪多矣,君必悔之,弗聽,逃其師而歸。

楚鬥穀於菟滅弦,弦子奔黃,於是江,黃,道,柏,方睦於齊,皆弦姻也,弦子恃之而不事楚,又不設備,故亡。

晉侯復假道於虞以伐虢,宮之奇諫曰,虢,虞之表也,虢亡,虞必從之,晉不可啟,寇不可翫。一之謂甚,其可再乎。諺所謂輔車相依,脣亡齒寒者,其虞虢之謂也。公曰,晉,吾宗也,豈害我哉,對曰,大伯,虞仲,大王之昭也,大伯不從,是以不嗣,虢仲,虢叔,王季之穆也,為文王卿士,勳在王室,藏於盟府,將虢是滅,何愛於虞,且虞能親於桓莊乎,其愛之也。桓莊之族何罪,而以為戮,不唯偪乎,親以寵偪,猶尚害之,況以國乎?公曰,吾享祀豐絜,神必據我。對曰,臣聞之,鬼神非人實親,惟德是依,故《周書》曰:「皇天無親,惟德是輔」,又曰:「黍稷非馨,明德惟馨」,又曰:「民不易物,惟德緊物。」如是則非德,民不和,神不享矣。神所馮依,將在德矣,若晉取虞,而明德以薦馨香,神其吐之乎,弗聽,許晉使。宮之奇以其族行。曰,虞不臘矣,在此行也,晉不更舉矣,八月,甲午,晉侯圍上陽問於卜偃曰,吾其濟乎,對曰,克之,公曰,何時,對曰,童謠云,丙之晨,龍尾伏辰,均服振振,取虢之旂,鶉之賁賁,天策焞焞,火中成軍,虢公其奔,其九月十月之交乎。丙子旦,日在尾,月在策,鶉火中,必是時也。冬,十二月,丙子朔,晉滅虢,虢公醜奔京師,師還館于虞,遂襲虞,滅之,執虞公,及其大夫井伯,以媵秦穆姬,而脩虞祀,且歸其職貢於王,故書曰,晉人執虞公,罪虞,且言易也。

六年,春,王正月。

夏,公會齊侯,宋公,陳侯,衛侯,曹伯,伐鄭,圍新城。

秋,楚人圍許,諸侯遂救許。

冬,公至自伐鄭。

六年,春,晉侯使賈華伐屈,夷吾不能守,盟而行,將奔狄,郤芮曰,後出同走,罪也,不如之梁,梁近秦而幸焉,乃之梁。

夏,諸侯伐鄭,以其逃首止之盟故也,圍新密,鄭所以不時城也。

秋,楚子圍許,以救鄭,諸侯救許,乃還。

冬,蔡穆侯將許僖公以見楚子於武城,許男面縛銜璧,大夫衰絰,士輿櫬,楚子問諸逢伯,對曰,昔武王克殷,微子啟如是,武王親釋其縛,受其璧而祓之,焚其櫬,禮而命之,使復其所,楚子從之。

七年,春,齊人伐鄭。

夏,小邾子來朝。

鄭殺其大夫申侯。

秋,七月,公會齊侯,宋公,陳世子款,鄭世子華,盟于甯母。

曹伯班卒。

公子友如齊。

冬,葬曹昭公。

七年,春,齊人伐鄭,孔叔言於鄭伯曰,諺有之曰,心則不競,何憚於病,既不能彊,又不能弱,所以斃也,國危矣,請下齊以救國。公曰,吾知其所由來矣,姑少待我。對曰,朝不及夕,何以待君。

夏,鄭殺申侯以說于齊。且用陳轅濤塗之譖也。初,申侯,申出也,有寵於楚文王,文王將死,與之璧,使行。曰:唯我知女,女專利而不厭。予取予求,不女疵瑕也,後之人,將求多於女,女必不免,我死,女必速行,無適小國,將不女容焉,既葬,出奔鄭,又有寵於厲公,子文聞其死也,曰,古人有言曰,知臣莫若君,弗可改也已。

秋,盟于甯毌,謀鄭故也,管仲言於齊侯曰,臣聞之,招攜以禮,懷遠以德,德禮不易,無人不懷,齊侯脩禮於諸侯,諸侯官受方物,鄭伯使大子華聽命於會,言於齊侯曰,洩氏,孔氏,子人氏,三族,實違君命,若君去之以為成,我以鄭為內臣,君亦無所不利焉,齊侯將許之,管仲曰,君以禮與信屬諸侯,而以姦終之,無乃不可乎,子父不奸之謂禮,守命共時之謂信,違此二者,姦莫大焉,公曰,諸侯有討於鄭,未捷,今苟有釁,從之,不亦可乎,對曰,君若綏之以德,加之以訓辭,而帥諸侯以討鄭,鄭將覆亡之不暇,豈敢不懼,若摠其罪人以臨之,鄭有辭矣,何懼,且夫合諸侯以崇德也,會而列姦,何以示後嗣,夫諸侯之會,其德刑禮義,無國不記,記姦之位,君盟替矣,作而不記,非盛德也,君其勿許,鄭必受盟,夫子華既為大子,而求介於大國,以弱其國,亦必不免,鄭有叔詹,堵叔,師叔,三良為政,未可間也,齊侯辭焉,子華由是得罪於鄭。

冬,鄭伯使,請盟于齊。

閏月,惠王崩,襄王惡大叔帶之難,懼不立,不發表,而告難于齊。

八年,春,王正月,公會王人,齊侯,宋公,衛侯,許男,曹伯,陳世子款,盟于洮,鄭伯乞盟。

夏,狄伐晉。

秋,七月,禘于大廟,用致夫人。

冬,十有二月,丁未,天王崩。

八年,春,盟于洮,謀王室也,鄭伯乞盟,請服也,襄王定位而後發喪。

晉里克帥師,梁由靡御,虢射為右,以敗狄于采桑,梁由靡晰,狄無恥,從之必大克,里克曰,懼之而已,無速眾狄,虢射曰,期年狄必至,示之弱矣。

夏,狄伐晉,報采桑之役也,復期月。

秋,禘而致哀姜焉,非禮也,凡夫人不薨于寢,不殯于廟,不赴于同,不祔于姑,則弗致也。

冬,王人來告喪,難故也,是以緩。

宋公疾,大子茲父固請曰,目夷長且仁,君其立之,公命子魚,子魚辭曰,能以國讓,仁孰大焉,臣不及也,且又不順,遂走而退。

九年,春,王三月,丁丑,宋公御說卒。

夏,公會宰周公,齊侯,宋子,衛侯,鄭伯,許男,曹伯,于葵丘。

秋,七月,乙酉,伯姬卒。

九月,戊辰,諸侯盟于葵丘。

甲子,晉侯佹諸卒。

冬,晉里奚克殺其君之子奚齊。

九年,春,宋桓公卒,未葬,而襄公會諸侯,故曰子,凡在喪,王曰小童,公侯曰子。

夏,會于葵丘,尋盟,且脩好,禮也。

王使宰孔賜齊侯胙,曰,天子有事于文武,使孔賜伯舅胙,齊侯將下拜,孔曰,且有後命,天子使孔曰,以伯舅耋老,加勞賜一級,無下拜,對曰,天威不違顏咫尺,小白,余敢貪天子之命,無下拜,恐隕越于下,以遺天子羞,敢不下拜,下拜登受。

秋,齊侯盟諸侯于葵丘,曰,凡我同盟之人,既盟之後,言歸于好,宰孔先歸,遇晉侯曰,可無會也,齊侯不務德而勤遠略,故北伐山戎,南伐楚,西為此會也,東略之不知,西則否矣,其在亂乎,君務靖亂,無勤於行,晉侯乃還。

九月,晉獻公卒,里克、㔻鄭,欲納文公,故以三公子之徒作亂,初,獻公使荀息傅奚齊,公疾,召之曰,以是藐諸孤,辱在大夫,其若之何,稽首而對曰,臣竭其股肱之力,加之以忠貞,其濟,君之靈也,不濟,則以死繼之,公曰,何謂忠貞,對曰,公家之利,知無不為,忠也。送往事居,耦俱無猜,貞也。及里克將殺奚齊,先告荀息曰,三怨將作,秦晉輔之,子將何如,荀息曰,將死之,里克曰,無益也,荀叔曰,吾與先君言矣,不可以貳,能欲復言,而愛身乎,雖無益也。

將焉辟之,且人之欲善,誰不如我,我欲無貳,而能謂人已乎。

冬,十月,里克殺奚齊于次,書曰,殺其君之子,未葬也,荀息將死之,人曰,不如立卓子而輔之,荀息立公子卓以葬,十一月,里克殺公子卓于朝,荀息死之,君子曰,詩所謂白圭之玷,尚可磨也,斯言之玷,不可為也,荀息有焉。

齊侯以諸侯之師伐晉,及高梁而還,討晉亂也,令不及魯,故不書。

晉郤芮使夷吾重賂秦以求入,曰,人實有國,我何愛焉,入而能民,土於何有,從之,齊隰朋帥師會秦師,納晉惠公,秦伯謂郤芮曰,公子誰恃,對曰,臣聞亡人無黨,有黨必有讎,夷吾弱不好弄,能鬥不過,長亦不改,不識其他,公謂公孫枝曰,夷吾其定乎,對曰,臣聞之,唯則定國,詩曰,不識不知,順帝之則,文王之謂也,又曰,不僭不賊,鮮不為則,無好無惡,不忌不克之謂也,今其言多忌克,難哉,公曰,忌則多怨,又焉能克,是吾利也。

宋襄公即位,以公子目夷為仁,使為左師以聽政,於是宋治,故魚氏世為左師。

十年,春,王正月,公如齊。

狄滅溫,溫子奔衛。

晉里克弒其君卓,及其大夫荀息。

夏,齊侯許男伐北戎。

晉殺其大夫里克。

秋,七月。

冬,大雨雪。

十年,春,狄滅溫,蘇子無信也,蘇子叛王即狄,又不能於狄,狄人伐之,王不救,故滅,蘇子奔衛。

夏,四月,周公忌父,王子黨,會齊隰朋,立晉侯,晉侯殺里克以說,將殺里克,公使謂之曰,微子則不及此,雖然,子弒二君與一大夫,為子君者,不亦難乎,對曰,不有廢也,君何以興,欲加之罪,其無辭乎,臣聞命矣,伏劍而死,於是平鄭聘于秦,且謝緩賂,故不及。

晉侯改葬共大子,秋,狐突適下國,遇大子,大子使登僕,而告之曰,夷吾無禮,余得請於帝矣,將以晉畀秦,秦將祀余,對曰,臣聞之,神不歆非類,民不祀非族,君祀無乃殄乎,且民何罪,失刑乏祀,君其圖之,君曰,諾,吾將復請,七日,新城西偏,將有巫者而見我焉,許之,遂不見,及期而往,告之曰,帝許我罰有罪矣,敝於韓,平鄭之如秦也,言於秦伯曰,呂甥,郤稱,冀芮,實為不從,若重問以召之,臣出晉君,君納重耳,蔑不濟矣。

冬,秦伯使泠至報問,且召三子,郤芮曰,幣重而言甘,誘我也,遂殺平鄭,祁舉,及七輿大夫,左行共華,右行賈華,叔堅,騅歂,纍虎,特宮,山祁,皆里平之黨也平豹奔秦,言於秦伯曰,晉侯背大主而忌小怨,民弗與也,伐之必出,公曰,失眾,焉能殺,違禍,誰能出君。

十有一年,春,晉殺其大夫平鄭父。

夏,公及夫人姜氏,會齊侯于陽穀。

秋,八月,大雩。

冬,楚人伐黃。

十一年,春,晉侯使以平鄭之亂來告。

天王使召武公,內史過,賜晉侯命,受玉惰,過歸告王曰,晉侯其無後乎,王賜之命,而惰於受瑞,先自棄也已,其何繼之有,禮,國之幹也,敬,禮之輿也,不敬則禮不行,禮不行則上下昏,何以長世。

夏,揚拒泉皋伊雒之戎,同伐京師,入王城,焚東門,王子帶召之也,秦晉伐戎以救周,秋,晉侯平戎于王。

黃人不歸楚貢,冬,楚人伐黃。

十有二年,春,王三月,庚午,日有食之。

夏,楚人滅黃。

秋,七月。

冬,十有二月,丁丑,陳侯杵臼卒。

十二年,春,諸侯城衛楚丘之郛,懼狄難也。

黃人恃諸侯之睦于齊也,不共楚職,曰,自郢及我九百里,焉能害我,夏,楚滅黃。

王以戎難故,討王子帶,秋,王子帶奔齊。

冬,齊侯使管夷吾平戎于王,使隰朋平戎于晉,王以上卿之禮饗管仲,管仲辭曰,臣,賤有司也,有天子之二守國高在,若節春秋,來承王命,何以禮焉,陪臣敢辭,王曰,舅氏,余嘉乃勳,應乃懿德,謂督不忘,往踐乃職,無逆朕命,管仲受下卿之禮而還,君子曰,管氏之世祀也宜哉,讓不忘其上,詩曰,愷悌君子,神所勞矣。

十有三年,春,狄侵衛。

夏,四月,葬陳宣公。

公會齊侯,宋公,陳侯,衛侯,鄭伯,許男,曹伯,于鹹。

秋,九月,大雩。

冬,公子友如齊。

十三年,春,齊侯使仲孫湫聘于周,且言王子帶,事畢,不與王言,歸復命曰,未可,王怒未怠,其十年乎,不十年,王弗召也。

夏,會于鹹,淮夷病杞故,且謀王室也。

秋,為戎難故,諸侯戍周,齊仲孫湫致之。

冬,晉荐饑,使乞糴于秦,秦伯謂子桑與諸乎。對曰,重施而報,君將何求,重施而不報,其民必攜,攜而討焉,無眾必敗,謂百里與諸乎。對曰,天災流行,國家代有,救災恤鄰,道也,行道有福,平鄭之子豹在秦,請伐晉。秦伯曰,其君是惡,其民何罪,秦於是乎輸粟于晉,自雍及絳相繼,命之曰,汎舟之役。

十有四年,春,諸侯城緣陵。

夏,六月,季姬及鄫子遇于防,使鄫子來朝。

秋,八月,辛卯,沙鹿崩。

狄侵鄭。

冬,蔡侯肸卒。

十四年,春,諸侯城緣陵而遷杞焉,不書,其人有闕也。

鄫季姬來寧,公怒止之,以鄫子之不朝也,夏,遇于防,而使來朝。

秋,八月,辛卯,沙鹿崩,晉卜偃曰,期年將有大咎,幾亡國。

冬,秦饑,使乞糴于晉,晉人弗與,慶鄭曰,背施無親,幸災不仁,貪愛不祥,怒鄰不義,四德皆失,何以守國,虢射曰,皮之不存,毛將安傅,慶鄭曰,棄信背鄰,患孰恤之,無信患作,失援必斃,是則然矣,虢射曰,無損於怨,而厚於寇,不如勿與,慶鄭曰,背施幸災,民所棄也,近猶讎之,況怨敵乎,弗聽,退曰,君其悔是哉。

十有五年,春,王正月,公如齊。

楚人伐徐。

三月,公會齊侯,宋公,陳侯,衛侯,鄭伯,許男,曹伯,盟于牡丘,遂次于匡,公孫敖帥師,及諸侯之大夫救徐。

夏,五月,日有食之。

秋,七月,齊師,曹師,伐厲,八月,螽。

九月,公至自會。

季姬歸于鄫。

己卯晦,震夷伯之廟。

冬,宋人伐曹,楚人敗徐于婁林。

十有一月,壬戌,晉侯及秦伯戰于韓,獲晉侯。

十五年,春,楚人伐徐,徐即諸夏故也,三月,盟于牡丘,尋葵丘之盟,且救徐也,孟穆伯帥師,及諸侯之師救徐,諸侯次于匡以待之。

夏,五月,日有食之,不書朔與日,官失之也。

秋,伐厲,以救徐也。

晉侯之入也,秦穆姬屬賈君焉,且曰,盡納群公子,晉侯烝於賈君,又不納群公子,是以穆姬怨之,晉侯許賂中大夫,既而皆背之,賂秦伯以河外列城五,東盡虢略,南及華山,內及解梁城,既而不與,晉饑,秦輸之粟,秦饑,晉閉之糴,故秦伯伐晉,卜徒父筮之,吉,涉河,侯車敗,詰之,對曰,乃大吉也。三敗,必獲晉君,其卦遇蠱。曰,千乘三去,三去之餘,獲其雄狐。夫狐蠱,必其君也,蠱之貞,風也,其悔,山也,歲云秋矣,我落其實,而取其材,所以克也,實落材亡,不敗何待,三敗及韓,晉侯謂慶鄭曰,寇深矣,若之何,對曰,君實深之,可若何,公曰,不孫,卜右,慶鄭吉,弗使,步揚御戎,家僕徒為右,乘小駟,鄭入也,慶鄭曰,古者大事,必乘其產,生其水土,而知其人心,安其教訓,而服習其道,唯所納之,無不如志,今乘異產以從戎事,及懼而變,將與人易,亂氣狡憤,陰血周作,張脈僨興,外彊中乾,進退不可,周旋不能,君必悔之,弗聽,九月,晉侯逆秦師,使韓簡視師,復曰,師少於我,鬥士倍我,公曰,何故,對曰,出因其資,入用其寵,饑食其粟,三施而無報,是以來也,今又擊之,我怠秦奮,倍猶未也,公曰,一夫不可狃,況國乎,遂使請戰,曰,寡人不佞,不能合其眾,而不能離也,君若不還,無所逃命,秦伯使公孫枝對曰,君之未入,寡人懼之,入而未定列,猶吾憂也,苟列定矣,敢不承命,韓簡退曰,吾幸而得囚,壬戌,戰于韓原,晉戎馬還濘而止,公號慶鄭,慶鄭曰,愎諫違卜,固敗是求,又何逃焉,遂去之,梁由靡御韓簡,虢射為右,輅秦伯,將止之,鄭以救公誤之,遂失秦伯,秦獲晉侯以歸,晉大夫反首拔舍,從之,秦伯使辭焉,曰,二三子何其慼也,寡人之從君而西也,亦晉之妖夢是踐,豈敢以至,晉大夫三拜稽首曰,君履后土而戴皇天,皇天后土,實聞君之言,群臣敢在下風,穆姬聞晉侯將至,以大子罃,弘,與女簡璧,登臺而履薪焉,使以免服衰絰逆,且告曰,上天降災,使我兩君匪以玉帛相見,而以興戎,若晉君朝以入,則婢子夕以死,夕以入,則朝以死,唯君裁之,乃舍諸靈臺,大夫請以入,公曰,獲晉侯,以厚歸也,既而喪歸焉用之,大夫其何有焉,且晉人慼憂以重我,天地以要我,不圖晉憂,重其怒也,我食吾言,背天地也,重怒難任,背天不祥,必歸晉君,公子縶曰,不如殺之,無聚慝焉,子桑曰,歸之而質其大子,必得大成,晉未可滅,而殺其君,祇以成惡,且史佚有言曰,無始禍,無怙亂,無重怒,重怒難任,陵人不祥,乃許晉平,晉侯使郤乞告瑕呂飴甥,且召之,子金教之言曰,朝國人而以君命賞,且告之曰,孤雖歸,辱社稷矣,其卜貳圉也,眾皆哭,晉於是乎作爰田,呂甥曰,君亡之不恤,而群臣是憂,惠之至也,將若君何,眾曰,何為而可,對曰,征繕以輔孺子,諸侯聞之,喪君有君,群臣輯睦,甲兵益多,好我者勸,惡我者懼,庶有益乎,眾說,晉於是乎作州兵,初,晉獻公筮嫁伯姬於秦,遇歸妹之睽,史蘇占之,曰,不吉,其繇曰,士刲羊,亦無衁也,女承筐,亦無貺也,西鄰責言,不可償也,歸妹之睽,猶無相也,震之離,亦離之震,為雷為火,為嬴敗姬,車說其輹,火焚其旗,不利行師,敗于宗丘,歸妹睽孤,寇張之弧,姪其從姑,六年其逋,逃歸其國,而棄其家,明年,其死於高梁之虛,及惠公在秦,曰,先君若從史蘇之占,吾不及此夫,韓簡侍曰,龜,象也,筮,數也,物生而後有象,象而後有滋,滋而後有數,先君之敗德,及可數乎,史蘇是占,勿從何益,詩曰,下民之孽,匪降自天,僔沓背憎,職競由人。

震夷伯之廟,罪之也,於是展氏有隱慝焉。

冬,宋人伐曹,討舊怨也。

楚敗徐于婁林,徐恃救也。

十月,晉陰飴甥會秦伯,盟于王城,秦伯曰,晉國和乎,對曰,不和,小人恥失其君,而悼喪其親,不憚征繕,以立圉也,曰,必報讎,寧事戎狄,君子愛其君,而知其罪,不憚征繕以待秦命,曰,必報德,有死無二,以此不和,秦伯曰,國謂君何,對曰,小人慼,謂之不免,君子恕,以為必歸,小人曰,我毒秦,秦豈歸君,君子曰,我知罪矣,秦必歸君,貳而執之,服而舍之,德莫厚焉,刑莫威焉,服者懷德,貳者畏刑,此一役也,秦可以霸,納而不定,廢而不立,以德為怨,秦不其然,秦伯曰,是吾心也,改館晉侯,饋七牢焉蛾析謂慶鄭曰,盍行乎,對曰,陷君於敗,敗而不死,又使失刑,非人臣也,臣而不臣,行將焉入,十一月,晉侯歸,丁丑,殺慶鄭而後入,是歲,晉又饑,秦伯又餼之粟,曰,吾怨其君而矜其民,且吾聞唐叔之封也,箕子曰,其後必大,晉其庸可冀乎,姑樹德焉,以待能者,於是秦始征晉河東,置官司焉。

十有六年,春,王正月,戊申朔,隕石于宋五,是月,六鷁退飛,過宋都。

三月,壬申,公子季友卒。

夏,四月,丙申,鄫季姬卒。

秋,七月,甲子,公孫茲卒。

冬,十有二月,公會齊侯,宋公,陳侯,衛侯,鄭伯,許男,邢侯,曹伯,于淮。

十六年,春,隕石于宋五,隕星也,六鷁退飛,過宋都,風也,周內史叔興聘于宋,宋襄公問焉,曰,是何祥也,吉凶焉在,對曰,今茲魯多大喪,明年齊有亂,君將得諸侯而不終,退而告人曰,君失問,是陰陽之事,非吉凶所生也,吉凶由人,吾不敢逆君故也。

夏,齊伐厲,不克,救徐而還。

秋,狄侵晉,取狐廚,受鐸,涉汾,及昆都,因晉敗也。

王以戎難告于齊,齊徵諸侯而戍周。

冬,十一月,乙卯,鄭殺子華。

十二月,會于淮,謀鄫,且東略也,城鄫,役人病,有夜登丘而呼曰,齊有亂,不果城而還。

十有七年,春,齊人,徐人,伐英氏。

夏,滅項。

秋,夫人姜氏會齊侯于卞。

九月,公至自會。

冬,十有二月,乙亥,齊侯小白卒。

十七年,春,齊人為徐伐英氏,以報婁林之役也。

夏,晉大子圉為質於秦,秦歸河東而妻之,惠公之在梁也,梁伯妻之,梁嬴孕過期,卜招父與其子卜之,其子曰,將生一男一女,招曰,然,男為人臣,女為人妾,故名男曰圉,女曰妾,及子圉西質,妾為宦女焉。

師滅項,淮之會,公有諸侯之事,未歸而取項,齊人以為討而止公。

秋,聲姜以公故,會齊侯于卞,九月,公至,書曰,至自會,猶有諸侯之事焉,且諱之也。

齊侯之夫人三,王姬,徐嬴,蔡姬,皆無子,齊侯好內,多內寵,內嬖如夫人者六人,長衛姬生武孟,少衛姬生惠公,鄭姬生孝公,葛嬴生昭公,密姬生懿公,宋華子生公子雍,公與管仲屬孝公於宋襄公,以為大子,雍巫有寵於衛共姬,因寺人貂以薦羞於公,亦有寵,公許之,立武孟,管仲卒,五公子皆求立,冬,十月,乙亥,齊桓公卒,易牙入,與寺人貂因內寵以殺群吏,而立公子無虧,孝公奔宋,十二月,乙亥,赴,辛已,夜殯。

十有八年,春,王正月,宋公,曹伯,衛人,邾人,伐齊。

夏,師救齊。

五月,戊寅,宋師及齊師戰于甗,齊師敗績。

狄救齊。

秋,八月,丁亥,葬齊桓公。

冬,邢人,狄人,伐衛。

十八年,春,宋襄公以諸侯伐齊,三月,齊人殺無虧。

鄭伯始朝于楚,楚子賜之金,既而悔之,與之盟曰,無以鑄兵,故以鑄三鍾。

齊人將立孝公,不勝四公子之徒,遂與宋人戰,夏,五月,宋敗齊師于甗,立孝公而還。

秋八月,葬齊桓公。

冬,邢人,狄人,伐衛,圍菟圃,衛侯以國讓父兄子弟,及朝眾曰,苟能治之,燬請從焉,眾不可,而從師于訾婁,狄師還。

梁伯益其國而不能實也,命曰新里,秦取之。

十有九年,春,王三月,宋人執滕子嬰齊。

夏,六月,宋公,曹人,邾人,盟于曹南,鄫子會盟于邾,己酉,邾人執鄫子用之。

秋,宋人圍曹,衛人伐邢。

冬,會陳人,蔡人,楚人,鄭人,盟于齊。

梁亡。

十九年,春,遂城而居之。

宋人執滕宣公。

夏宋公使邾文公,用鄫子于次睢之社,欲以屬東夷,司馬子魚曰,古者六畜不相為用,小事不用大牲,而況敢用人乎,祭祀以為人也,民,神之主也,用人,其誰饗之,齊桓公存三亡國,以屬諸侯,義士猶曰薄德,今一會而虐二國之君,又用諸淫昏之鬼,將以求霸,不亦難乎,得死為幸。

秋,衛人伐邢,以報菟圃之役,於是衛大旱,卜有事於山川,不吉,甯莊子曰,昔周饑,克殷而年豐,今邢方無道,諸侯無伯,天其或者,欲使衛討邢乎,從之,師興而雨。

宋人圍曹,討不服也,子魚言於宋公曰,文王聞崇德亂而伐之,軍三旬而不降,退脩教而復伐之,因壘而降,詩曰,刑于寡妻,至于兄弟,以御于家邦,今君德無乃猶有所闕,而以伐人,若之何,盍姑內省德乎,無闕而後動。

陳穆公請脩好於諸侯,以無忘齊桓之德,冬,盟于齊,脩桓公之好也。

梁亡,不書其主,自取之也,初,梁伯好土功,亟城而弗處,民罷而弗堪,則曰,某寇將至,乃溝公宮,曰,秦將襲我,民懼而潰,秦遂取梁。

二十年,春,新作南門。

夏,郜子來朝。

五月,乙巳,西宮災。

鄭人入滑。

秋,齊人,狄人,盟于邢。

冬,楚人伐隨。

二十年,春,新作南門,書不時也,凡啟塞從時。

滑人,叛鄭,而服於衛,夏,鄭公子士,洩堵寇,帥師入滑。

秋,齊狄盟于邢,為邢謀衛難也,於是衛方病邢。

隨以漢東諸侯叛楚,冬,楚鬥穀於菟帥師伐隨,取成而還,君子曰,隨之見伐,不量力也,量力而動,其過鮮矣,善敗由己,而由人乎哉,詩曰,豈不夙夜,謂行多露。

宋襄公欲合諸侯,臧文仲聞之曰,以欲從人則可,以人從欲鮮濟。

二十有一年,春狄侵衛。

宋人,齊人,楚人,盟于鹿上。

夏,大旱。

秋,宋公,楚子,陳侯,蔡侯,鄭伯,許男,曹伯,會于盂,執宋公以伐宋。

冬,公伐邾。

楚人使宜申來獻捷。

十有二月,癸丑,公會諸侯盟于薄,釋宋公。

二十一年,春,宋人為鹿上之盟,以求諸侯於楚,楚人許之。公子目夷曰,小國爭盟,禍也。宋其亡乎,幸而後敗。

夏,大旱,公欲焚巫尪,臧文仲曰,非旱備也,脩城郭,貶食省用,務穡勸分,此其務也,巫尪何為,天欲殺之,則如勿生。

若能為旱,焚之滋甚,公從之,是歲也,饑而不害。

秋,諸侯會宋公于盂,子魚曰,禍其在此乎,君欲已甚,其何以堪之,於是楚執宋公以伐宋,冬,會于薄以釋之,子魚曰,禍猶未也,未足以懲君,任,宿,須句,顓臾,風姓也,實司大皞與有濟之祀,以服事諸夏,邾人滅須句,須句子來奔,因成風也,成風為之言於公曰,崇明祀,保小寡,周禮也,蠻夷猾夏,周禍也,若封須句,是崇皞濟而脩祀紓禍也。

二十有二年,春,公伐邾,取須句。

夏,宋公,衛侯,許男,滕子,伐鄭。

秋,八月,丁未,及邾人戰于升陘。

冬,十有一月,己已,朔,宋公及楚人戰于泓,宋師敗績。

二十二年,春,伐邾,取須句,反其君焉,禮也。

三月,鄭伯如楚。

夏,宋公伐鄭,子魚曰,所謂禍在此矣。

初,平王之東遷也,辛有適伊川,見被髮而祭於野者,曰,不及百年,此其戎乎,其禮先亡矣,秋,秦晉遷陸渾之戎于伊川。

晉大子圉為質於秦,將逃歸,謂嬴氏曰,與子歸乎,對曰,子,晉大子,而辱於秦,子之欲歸,不亦宜乎,寡君之使婢子侍執巾櫛,以固子也,從子而歸,棄君命也,不敢從,亦不敢言,遂逃歸。

富辰言於王曰,請召大叔詩曰,協比其鄰,昏姻孔云,吾兄弟之不協,焉能怨諸侯之不睦,王說,王子帶自齊復歸于京師,王召之也。

邾人以須句故出師。

公卑邾,不設備而禦之。臧文仲曰,國無小,不可易也,無備雖眾,不可恃也。詩曰:「戰戰兢兢,如臨深淵,如履薄冰」,又曰:「敬之敬之,天惟顯思,命不易哉」,先王之明德,猶無不難也,無不懼也,況我小國乎,君其無謂邾小,蜂蠆有毒,而況國乎,弗聽。八月,丁未,公及邾師戰于升陘,我師敗績,邾人獲公冑,縣諸魚門。

楚人伐宋以救鄭,宋公將戰,大司馬固諫曰,天之棄商久矣,君將興之,弗可赦也已,弗聽。

冬,十一月,己已,朔,宋公及楚人戰于泓,宋人既成列。

楚人未既濟,司馬曰,彼眾我寡,及其未既濟也,請擊之。公曰,不可,既濟而未成列,又以告,公曰,未可,既陳而後擊之,宋師敗績,公傷股,門官殲焉,國人皆咎公,公曰,君子不重傷,不禽二毛,古之為軍也,不以阻隘也,寡人雖亡國之餘,不鼓不成列,子魚曰,君未知戰,勍敵之人,隘而不列,天贊我也,阻而鼓之,不亦可乎,猶有懼焉,且今之勍者,皆吾敵也,雖及胡耇,獲則取之,何有於二毛,明恥教戰,求殺敵也,傷未及死,如何勿重。若愛重傷,則如勿傷。愛其二毛,則如服焉,三軍以利用也,金鼓以聲氣也,利而用之,阻隘可也聲盛致志,鼓儳可也。

丙子晨,鄭文夫人芊氏,姜氏,勞楚子於柯澤,楚子使師縉示之俘馘,君子曰,非禮也。婦人送迎不出門,見兄弟不踰閾。戎事不邇女器,丁丑,楚子入饗于鄭,九獻,庭實旅百,加籩豆六品,饗畢,夜出,文芊送于軍,取鄭二姬以歸,叔詹曰,楚王其不沒乎,為禮卒於無別,無別不可謂禮,將何以沒,諸侯是以知其不遂霸也。

二十二年,春,伐邾,取須句,反其君焉,禮也。

三月,鄭伯如楚。

夏,宋公伐鄭,子魚曰,所謂禍在此矣。

初,平王之東遷也,辛有適伊川,見被髮而祭於野者,曰,不及百年,此其戎乎,其禮先亡矣,秋,秦晉遷陸渾之戎于伊川。

晉大子圉為質於秦,將逃歸,謂嬴氏曰,與子歸乎,對曰,子,晉大子,而辱於秦,子之欲歸,不亦宜乎,寡君之使婢子侍執巾櫛,以固子也,從子而歸,棄君命也,不敢從,亦不敢言,遂逃歸。

富辰言於王曰,請召大叔詩曰,協比其鄰,昏姻孔云,吾兄弟之不協,焉能怨諸侯之不睦,王說,王子帶自齊復歸于京師,王召之也。

邾人以須句故出師。

公卑邾,不設備而禦之,臧文仲曰,國無小,不可易也,無備雖眾,不可恃也,詩曰,戰戰兢兢,如臨深淵,如履薄冰,又曰,敬之敬之,天惟顯思,命不易哉,先王之明德,猶無不難也,無不懼也,況我小國乎,君其無謂邾小,蜂蠆有毒,而況國乎,弗聽,八月,丁未,公及邾師戰于升陘,我師敗績,邾人獲公冑,縣諸魚門。

楚人伐宋以救鄭,宋公將戰,大司馬固諫曰,天之棄商久矣,君將興之,弗可赦也已,弗聽。

冬,十一月,己已,朔,宋公及楚人戰于泓,宋人既成列。

楚人未既濟,司馬曰,彼眾我寡,及其未既濟也,請擊之,公曰,不可,既濟而未成列,又以告,公曰,未可,既陳而後擊之,宋師敗績,公傷股,門官殲焉,國人皆咎公,公曰,君子不重傷,不禽二毛,古之為軍也,不以阻隘也,寡人雖亡國之餘,不鼓不成列,子魚曰,君未知戰,勍敵之人,隘而不列,天贊我也,阻而鼓之,不亦可乎,猶有懼焉,且今之勍者,皆吾敵也,雖及胡耇,獲則取之,何有於二毛,明恥教戰,求殺敵也,傷未及死,如何勿重,若愛重傷,則如勿傷,愛其二毛,則如服焉,三軍以利用也,金鼓以聲氣也,利而用之,阻隘可也聲盛致志,鼓儳可也。

丙子晨,鄭文夫人芊氏,姜氏,勞楚子於柯澤,楚子使師縉示之俘馘,君子曰,非禮也,婦人送迎不出門,見兄弟不踰閾,戎事不邇女器,丁丑,楚子入饗于鄭,九獻,庭實旅百,加籩豆六品,饗畢,夜出,文芊送于軍,取鄭二姬以歸,叔詹曰,楚王其不沒乎,為禮卒於無別,無別不可謂禮,將何以沒,諸侯是以知其不遂霸也。

二十有三年,春,齊侯伐宋,圍緡。

夏,五月,庚寅,宋公茲父卒。

秋,楚人伐陳。

冬,十有一月,杞子卒。

二十三年,春,齊侯伐宋,圍緡,以討其不與盟于齊也。

夏,五月,宋襄公卒,傷於泓故也。

秋,楚成得臣帥師伐陳,討其貳於宋也,遂取焦夷,城頓而還,子文以為之功使為令尹,叔伯曰,子若國何,對曰,吾以靖國也,夫有大功而無貴仕,其人能靖者與,有幾。

九月,晉惠公卒,懷公命無從亡人,期期而不至,無赦,狐突之子毛,及偃,從重耳在秦,弗召,冬,懷公執狐突曰,子來則免,對曰,子之能仕,父教之忠,古之制也,策名委質,貳乃辟也,今臣之子,名在重耳,有年數矣,若又召之,教之貳也,父教子貳,何以事君,刑之不濫,君之明也,臣之願也,淫刑以逞,誰則無罪,臣聞命矣,乃殺之,卜偃稱疾不出,曰,《周書》有之,乃大明服,己則不明,而殺人以逞,不亦難乎。民不見德,而唯戮是聞,其何後之有。

十一月,杞成公卒,書曰,子,杞,夷也,不書名,未同盟也,凡諸侯同盟,死則赴以名,禮也,赴以名,則亦書之,不然則否,辟不敏也。

晉公子重耳之及於難也,晉人伐諸蒲城,蒲城人欲戰,重耳不可,曰,保君父之命,而享其生祿,於是乎得人,有人而校,罪莫大焉,吾其奔也,遂奔狄,從者狐偃,趙衰,顛頡,魏武子,司空季子,狄人伐廧咎如,獲其二女,叔隗,季隗,納諸公子,公子取季隗,生伯鯈,叔劉,以叔隗妻趙衰,生盾,將適齊,謂季隗曰,待我二十五年不來而後嫁,對曰,我二十五年矣,又如是而嫁,則就木焉,請待子,處狄十二年而行,過衛,衛文公不禮焉,出於五鹿,乞食於野人,野人與之塊,公子怒,欲鞭之,子犯曰,天賜也,稽首受而載之。及齊,齊桓公妻之,有馬二十乘,公子安之。從者以為不可,將行,謀於桑下,蠶妾在其上,以告姜氏,姜氏殺之,而謂公子曰,子有四方之志,其聞之者,吾殺之矣,公子曰,無之,姜曰,行也,懷與安,實敗名,公子不可,姜與子犯謀,醉而遣之,醒以戈逐子犯,及曹,曹共公聞其駢脅,欲觀其裸,浴,薄而觀之,僖負羈之妻曰,吾觀晉公子之從者,皆足以相國,若以相,夫子必反其國,反其國,必得志於諸侯,得志於諸侯,而誅無禮,曹其首也,子盍蚤自貳焉,乃饋盤飧寘璧焉。公子受飧反璧。及宋,宋襄公贈之以馬二十乘,及鄭,鄭文公亦不禮焉,叔詹諫曰,臣聞天之所啟,人弗及也,晉公子有三焉,天其或者將建諸,君其禮焉,男女同姓,其生不蕃,晉公子,姬出也,而至于今,一也,離外之患,而天下不靖,晉國殆將啟之,二也,有三士足以上人,而從之,三也,晉鄭同儕,其過子弟,固將禮焉,況天之所啟乎,弗聽,及楚,楚子饗之,曰,公子若反晉國,則何以報不穀,對曰,子女玉帛,則君有之。羽毛齒革,則君地生焉。其波及晉國者,君之餘也,其何以報。君曰,雖然,何以報我,對曰,若以君之靈,得反晉國,晉楚治兵,遇於中原,其辟君三舍,若不獲命,其左執鞭弭,右屬櫜鞬,以與君周旋,子玉請殺之。楚子曰,晉公子廣而儉,文而有禮,其從者肅而寬,忠而能力。晉侯無親,外內惡之,吾聞姬姓,唐叔之後,其後衰者也,其將由晉公子乎,天將興之,誰能廢之,違天必有大咎,乃送諸秦,秦伯納女五人,懷嬴與焉,奉匜沃盥,既而揮之,怒曰,秦晉匹也,何以卑我,公子懼,降服而囚,他日,公享之,子犯曰,吾不如衰之文也,請使衰從,公子賦河水,公賦六月,趙衰曰,重耳拜賜,公子降拜稽首,公降一級,而辭焉,衰曰,君稱所以佐天子者命重耳,重耳敢不拜。

二十有四年,春,王正月。

夏,狄伐鄭。

秋,七月。

冬,天王出居于鄭,晉侯夷吾卒。

二十四年,春,王正月,秦伯納之,不書,不告入也,及河,子犯以璧授公子曰,臣負羈絏,從君巡於天下,臣之罪甚多矣,臣猶知之,而況君乎,請由此亡。公子曰:所不與舅氏同心者,有如白水,投其璧于河。濟河,圍令狐,入桑泉,取臼衰,二月,甲午,晉師軍于廬柳,秦伯使公子縶如晉師,師退,軍于郇,辛丑,狐偃及秦晉之大夫盟于郇,壬寅,公子入于晉師,丙午,入于曲沃,丁未,朝于武宮,戊申,使殺懷公于高梁,不書,亦不告也,呂郤畏偪,將焚公宮,而弒晉侯,寺人披請見,公使讓之,且辭焉,曰,蒲城之役,君命一宿,女即至,其後余從狄君以田渭濱,女為惠公來求殺余,命女三宿,女中宿至,雖有君命,何其速也,夫袪猶在,女其行乎,對曰,臣謂君之入也,其知之矣,若猶未也,又將及難,君命無二,古之制也,除君之惡,唯力是視,蒲人狄人,余何有焉,今君即位,其無蒲狄乎。齊桓公置射鉤而使管仲相。君若易之,何辱命焉,行者甚眾,豈唯刑臣,公見之,以難告。三月,晉侯潛會秦伯于王城,己丑,晦,公宮火,瑕甥,郤芮,不獲公,乃如河上,秦伯誘而殺之,晉侯逆夫人嬴氏以歸,秦伯送衛於晉三千人,實紀綱之僕,初,晉侯之豎頭須,守藏者也,其出也,竊藏以逃,盡用以求納之,及入,求見,公辭焉以沐,謂僕人曰,沐則心覆,心覆則圖反,宜吾不得見也,居者為社稷之守,行者為羈絏之僕,其亦可也,何必罪居者,國君而讎匹夫,懼者甚眾矣,僕人以告,公遽見之,狄人歸季隗于晉,而請其二子,文公妻趙衰,生原同,屏括,摟嬰,趙姬請逆盾,與其母,子餘辭,姬曰,得寵而忘舊,何以使人,必逆之,固請,許之,來,以盾為才,固請于公,以為嫡子,而使其三子下之,以叔隗為內子,而己下之。

晉侯賞從亡者,介之推不言祿,祿亦弗及,推曰,獻公之子九人,唯君在矣,惠懷無親,外內棄之,天未絕晉,必將有主,主晉祀者,非君而誰,天實置之,而二三子以為己力,不亦誣乎,竊人之財,猶謂之盜,況貪天之功,以為己力乎,下義其罪,上賞其姦,上下相蒙,難與處矣,其母曰,盍亦求之,以死誰懟,對曰,尤而效之,罪又甚焉,且出怨言,不食其食,其母曰,亦使知之,若何,對曰,言,身之文也,身將隱,焉用文之,是求顯也,其母曰,能如是乎,與女偕隱,遂隱而死,晉侯求之不獲,以綿上為之田,曰,以志吾過,且旌善人。

鄭之入滑也,滑人聽命,師還,又即衛,鄭公子士,洩堵俞彌,帥師伐滑,王使伯服,游孫伯,如鄭請滑,鄭伯怨惠王之入,而不與厲公爵也,又怨襄王之與衛滑也,故不聽王命,而執二子,王怒,將以狄伐鄭,富辰諫曰,不可,臣聞之,大上以德撫民,其次親親,以相及也,昔周公弔二叔之不咸,故封建親戚,以蕃屏周,管,蔡,郕,霍,魯,衛,毛,聃,郜,雍,曹,滕,畢,原,酆,郇,文之昭也,邘,晉,應,韓,武之穆也,凡,蔣,邢,茅,胙,祭,周公之胤也,召穆公思周德之不類,故糾合宗族于成周,而作詩,曰,常棣之華,鄂不韡韡,凡今之人,莫如兄弟,其四章曰,兄弟鬩于牆,外禦其侮,而是則兄弟雖有小忿,不廢懿親,今天子不忍小忿,以棄鄭親,其若之何,庸勳親親,暱,近尊賢,德之大者也,即聾,從昧,與頑,用嚚,姦之大者也,棄德崇姦,禍之大者也,鄭有平惠之勳,又有厲宣之親,棄嬖寵而用三良,於諸姬為近,四德具矣,耳不聽五聲之和為聾,目不別五色之章為昧,心不則德義之經為頑,口不道忠信之言為嚚,狄皆則之,四姦具矣,周之有懿德也,猶曰莫如兄弟,故封建之,其懷柔天下也,猶懼有外侮,扞禦侮者,莫如親親,故以親屏周,召穆公亦云,今周德既衰,於是乎又渝周召,以從諸姦,無乃不可乎,民未忘禍,王又興之,其若文武何,王弗聽,使頹叔,桃子,出狄師。

夏,狄伐鄭,取櫟,王德狄人,將以其女為后,富辰諫曰,不可,臣聞之曰,報者倦矣,施者未厭,狄固貪惏,王又啟之,女德無極,婦怨無終,狄必為患,王又弗聽,初,甘昭公有寵於惠后,惠后將立之,未及而卒,昭公奔齊,王復之,又通於隗氏,王替隗氏,頹叔桃子曰,我實使狄,狄其怨我,遂奉大叔,以狄師攻王,王御士將禦之,王曰,先后其謂我何,寧使諸侯圖之,王遂出,及坎欿,國人納之,秋,頹叔桃子奉大叔以狄師伐周,大敗周師,獲周公忌父,原伯,毛伯,富辰,王出適鄭,處于氾,大叔以隗氏居于溫。

鄭子華之弟子臧出奔宋,好聚鷸冠,鄭伯聞而惡之,使盜誘之,八月,盜殺之于陳宋之間,君子曰,服之不衷,身之災也,詩曰,彼己之子,不稱其服,子臧之服,不稱也夫,詩曰,自詒伊慼,其子臧之謂矣,夏書曰,地平天成,稱也。

宋及楚平,宋成公如楚,還,入於鄭,鄭伯將享之,問禮於皇武子,對曰,宋,先代之後也,於周為客,天子有事膰焉,有喪拜焉,豐厚可也,鄭伯從之,享宋公有加,禮也。

冬,王使來告難曰,不穀不德,得罪于母弟之寵子帶,鄙在鄭地氾,敢告叔父,臧文仲對曰,天子蒙塵于外,敢不奔問官守,王使簡師父告于晉,使左鄢父告于秦,天子無出,書曰,天王出居于鄭,辟母弟之難也,天子凶服降名,禮也。

鄭伯與孔將鉏,石甲父,侯宣多,省視官具于氾,而後聽其私政,禮也。

衛人將伐邢,禮至曰,不得其守,國不可得也,我請昆弟仕焉,乃往得仕。

二十有五年,春,王正月,丙午,衛侯燬滅邢。

夏,四月,癸酉,衛侯燬卒。

宋蕩伯姬來逆婦。

宋殺其大夫。

秋,楚人圍陳,納頓子于頓。

葬衛文公。

冬,十有二月,癸亥,公會衛子,莒慶,盟于洮。

二十五年,春,衛人伐邢,二禮從國子巡城,掖以赴外,殺之,正月,丙午,衛侯燬滅邢,同姓也,故名,禮至為銘曰,余掖殺國子,莫余敢止。

秦伯師于河上,將納王,狐偃言於晉侯曰,求諸侯莫如勤王,諸侯信之,且大義也,繼文之業,而信宣於諸侯,今為可矣,使卜偃卜之,曰,吉,遇黃帝戰于阪泉之兆,公曰,吾不堪也,對曰,周禮未改,今之王,古之帝也,公曰,筮之,筮之,遇大有之睽,曰吉,遇公用享于天子之卦也,戰克而王饗,吉孰大焉,且是卦也,天為澤以當日,天子降心以逆公,不亦可乎,大有去睽而復,亦其所也,晉侯辭秦師而下,三月,甲辰,次于陽樊,右師圍溫,左師逆王。

夏,四月,丁巳,王入于王城,取大叔于溫,殺之于隰城,戊午,晉侯朝王,王饗醴,命之宥,請隧,弗許,曰,王章也,未有代德,而有二王,亦叔父之所惡也,與之陽樊,溫原,欑,茅,之田,晉於是始起南陽,陽樊不服,圍之,蒼葛呼曰,德以柔中國,刑以威四夷,宜吾不敢服也,此誰非王之親姻,其俘之也,乃出其民。

秋,秦晉伐鄀,楚鬥克,屈禦寇,以申息之師戍商密,秦人過析,隈入而係輿人,以圍商密,昏而傅焉,宵坎血加書,偽與子儀子邊盟者,商密人懼曰,秦取析矣,戍人反矣,乃降秦師,囚申公子儀,息公子邊,以歸,楚令尹子玉追秦師,弗及,遂圍陳,納頓子于頓。

冬,晉侯圍原,命三日之糧,原不降,命去之,諜出,曰,原將降矣,軍吏曰,請待之,公曰,信,國之寶也,民之所庇也,得原失信,何以庇之,所亡滋多,退一舍而原降,遷原伯貫于冀,趙衰為原大夫,狐溱為溫大夫。

衛人平莒于我,十二月,盟于洮,脩衛文公之好,且及莒平也。

晉侯問原守於寺人勃鞮,對曰,昔趙衰以壺飧從徑,餒而弗食,故使處原。

二十有六年,春,王正月,己未,公會莒子,衛甯速,盟于向。

齊人侵我西鄙,公追齊師至酅,不及。

夏,齊人伐我北鄙。

衛人伐齊。

公子遂如楚乞師。

秋,楚人滅夔子歸。

冬,楚人伐宋,圍緡,公以楚師伐齊,取穀,公至自伐齊。

二十六年,春,王正月,公會莒茲平公,甯莊子盟于向,尋洮之盟也。

齊師侵我西鄙,討是二盟也。

夏,齊孝公伐我北鄙,衛人伐齊,洮之盟故也,公使展喜犒師,使受命于展禽,齊侯未入竟,展喜從之,曰,寡君聞君親舉玉趾,將辱於敝邑,使下臣犒執事,齊侯曰,魯人恐乎,對曰,小人恐矣,君子則否,齊侯曰,室如縣罄,野無青草,何恃而不恐,對曰,恃先王之命,昔周公,大公,股肱周室,夾輔成王,成王勞之,而賜之盟曰,世世子孫,無相害也,載在盟府,大師職之,桓公是以糾合諸侯,而謀其不協,彌縫其闕,而匡救其災,昭舊職也,及君即位,諸侯之望曰,其率桓之功,我敝邑用不敢保聚,曰,豈其嗣世九年,而棄命廢職,其若先君何,君必不然,恃此以不恐,齊侯乃還。

東門襄仲,臧文仲,如楚乞師,臧孫見子玉,而道之伐齊宋,以其不臣也。

夔子不祀祝融,與鬻熊,楚人讓之,對曰,我先王熊摯有疾,鬼神弗赦,而自竄于夔,吾是以失楚,又何祀焉,秋,楚成得臣,鬥宜申,帥師滅夔,以夔子歸。

宋以其善於晉侯也,叛楚即晉,冬,楚令尹子玉,司馬子西,帥師伐宋,圍緡,公以楚師伐齊,取穀,凡師,能左右之曰以,寘桓公子雍於穀,易牙奉之,以為魯援,楚申公叔侯戍之,桓公之子七人,為七大夫於楚。

二十有七年,春,杞子來朝。

夏,六月,庚寅,齊侯昭卒。

秋,八月,乙未,葬齊孝公,乙巳,公子遂帥師入杞。

冬,楚人,陳侯,蔡侯,鄭伯,許男,圍宋,十有二月,甲戌,公會諸侯盟于宋。

二十七年,春,杞桓公來朝,用夷禮,故曰子,公卑杞,杞不共也。

夏,齊孝公卒,有齊怨,不廢喪紀,禮也。

秋,入杞,責無禮也。

楚子將圍宋,使子文治兵於睽,終朝而畢,不戮一人,子玉復治兵於蒍,終日而畢,鞭七人,貫三人耳,國老皆賀子文,子文飲之酒,蒍賈尚幼,後至,不賀,子文問之,對曰,不知所賀,子之傳政於子玉,曰以靖國也,靖諸內而敗諸外,所獲幾何,子玉之敗,子之舉也,舉以敗國,將何賀焉,子玉剛而無禮,不可以治民,過三百乘,其不能以入矣,苟入而賀,何後之有。

冬,楚子及諸侯圍宋,宋公孫固如晉告急,先軫曰,報施救患,取威定霸,於是乎在矣,狐偃曰,楚始得曹,而新昏於衛,若伐曹衛,楚必救之,則齊宋免矣,於是乎蒐于被廬,作三軍,謀元帥,趙衰曰,郤縠可,臣亟聞其言矣,說禮樂而敦詩書,詩書,義之府也,禮樂,德之則也,德義,利之本也,夏書曰,賦納以言,明試以功,車服以庸,君其試之,乃使郤縠將中軍,郤溱佐之,使狐偃將上軍,讓於狐毛而佐之,命趙衰為卿,讓於欒枝,先軫,使欒枝將下軍,先軫佐之,荀林父御戎,魏犨為右,晉侯始入而教其民,二年,欲用之,子犯曰,民未知義,未安其居,於是乎出定襄王,入務利民,民懷生矣,將用之,子犯曰,民未知信,未宣其用,於是乎伐原以示之信,民易資者,不求豐焉,明徵其辭,公曰,可矣乎,子犯曰,民未知禮,未生其共,於是乎大蒐以示之禮,作執秩以正其官,民聽不惑,而後用之,出穀戍,釋宋圍,一戰而霸,文之教也。

二十有八年,春,晉侯侵曹,晉侯伐衛。

公子買戍衛,不卒戍,刺之,楚人救衛。

三月,丙午,晉侯入曹,執曹伯,畀宋人。

夏,四月,己巳,晉侯,齊師,宋師,秦師,及楚人戰于城濮,楚師敗績,楚殺其大夫得臣。

衛侯出奔楚。

五月,癸丑,公會晉侯,齊侯,宋公蔡侯,鄭伯,衛子,莒子,盟于踐土,陳侯如會,公朝于王所。

六月,衛侯鄭自楚復歸于衛,衛元咺出奔晉。

陳侯款卒。

秋,杞伯姬來。

公子遂如齊。

冬,公會晉侯,齊侯,宋公,蔡侯,鄭伯,陳子,莒子,邾人,秦人,于溫。

天王狩于河陽,壬申,公朝于王所。

晉人執衛侯歸之于京師,衛元咺自晉復歸于衛。

諸侯遂圍許,曹伯襄復歸于曹,遂會諸侯圍許。

二十八年,春,晉侯將伐曹,假道于衛,衛人弗許,還自河南濟,侵曹,伐衛,正月,戊申,取五鹿,二月,晉郤縠卒,原蹫將中軍,胥臣佐下軍,上德也,晉侯,齊侯,盟于斂盂,衛侯請盟,晉人弗許,衛侯欲與楚,國人不欲,故出其君,以說于晉,衛侯出居于襄牛。

公子買戍衛,楚人救衛,不克,公懼於晉,殺子叢以說焉,謂楚人曰,不卒戍也。

晉侯圍曹,門焉多死,曹人尸諸城上,晉侯患之,聽輿人之謀曰,稱舍於墓,師遷焉,曹人兇懼,為其所得者,棺而出之,因其兇也而攻之,三月,丙午,入曹,數之以其不用僖負羈,而乘軒者三百人也,且曰,獻狀,令無入僖負羈之宮,而免其族,報施也,魏犨,顛頡,怒曰,勞之不圖,報於何有,爇僖負羈氏氏,魏犨傷於胸,公欲殺之,而愛其材,使問,且視之病,將殺之,魏犨束胸,見使者曰,以君之靈,不有寧也,距躍三百,曲踊三百,乃舍之,殺顛頡以徇于師,立舟之僑以為戎右,宋人使門尹般如晉師告急,公曰,宋人告急,舍之則絕,告楚不許,我欲戰矣,齊秦未可,若之何,先軫曰,使宋舍我而賂齊秦,藉之告楚,我執曹君,而分曹衛之田,以賜宋人,楚愛曹衛,必不許也,喜賂怒頑,能無戰乎,公說,執曹伯,分曹衛之田,以畀宋人,楚人入居于申,使申叔去穀,使子玉去宋,曰,無從晉師,晉侯在外,十九年矣,而果得晉國,險阻艱難,備嘗之矣,民之情偽,盡知之矣,天假之年,而除其害,天之所置,其可廢乎,軍志曰,允當則歸,又曰,知難而退,又曰,有德不可敵,此三志者,晉之謂矣,子玉使伯棼請戰,曰,非敢必有功也,願以間執讒慝之口,王怒,少與之師,唯西廣東宮,與若敖之六卒,實從之,子玉使宛春告於晉師,曰,請復衛侯,而封曹,臣亦釋宋之圍。子犯曰:子玉無禮哉,君取一,臣取二,不可失矣。先軫曰,子與之,定人之謂禮,楚一言而定三國,我一言而亡之,我則無禮,何以戰乎,不許楚言,是棄宋也,救而棄之,謂諸侯何,楚有三施,我有三怨,怨讎已多,將何以戰,不如私許復曹衛以攜之,執宛春以怒楚,既戰而後圖之,公說,乃拘宛春於衛,且私許復曹衛,曹衛告絕於楚,子玉怒,從晉師,晉師退,軍吏曰,以君辟臣,辱也,且楚師老矣,何故退,子犯曰,師直為壯,曲為老,豈在久乎,微楚之惠不及此,退三舍辟之,所以報也,背惠食言,以亢其讎,我曲楚直,其眾素飽,不可謂老,我退而楚還,我將何求,若其不還,君退臣犯,曲在彼矣,退三舍,楚眾欲止,子玉不可,夏,四月,戊辰,晉侯,宋公,齊國歸父,崔夭,秦小子憖,次于城濮,楚師背酅而舍,晉侯患之,聽輿人之誦,曰,原田每每,舍其舊而新是謀,公疑焉,子犯曰,戰也,戰而捷,必得諸侯,若其不捷,表裡山河,必無害也,公曰,若楚惠何,欒貞子曰,漢陽諸姬,楚實盡之,思小惠而忘大恥,不如戰也,晉侯夢與楚子搏,楚子伏己而盬其腦,是以懼,子犯曰,吉,我得天,楚伏其罪,吾且柔之矣,子玉使鬥勃請戰,曰,請與君之士戲,君馮軾而觀之,得臣與寓目焉晉侯使欒枝對曰,寡君聞命矣,楚君之惠,未之敢忘,是以在此,為大夫退,其敢當君乎,既不獲命矣,敢煩大夫,謂二三子。戒爾車乘,敬爾君事,詰朝將見。晉車七百乘,韅靷鞅靽,晉侯登有莘之虛以觀師,曰,少長有禮,其可用也,遂伐其木,以益其兵,己巳,晉師陳于莘北,胥臣以下軍之佐,當陳蔡,子玉以若敖之六卒,將中軍,曰,今日必無晉矣,子西將左,子上將右,胥臣蒙馬以虎皮,先犯陳蔡,陳蔡奔,楚右師潰,狐毛設二旆而退之,欒枝使輿曳柴而偽遁,楚師馳之,原軫,郤溱,以中軍公族橫擊之,狐毛,狐偃,以上軍夾攻子西,楚左師潰,楚師敗績,子玉收其卒而止,故不敗,晉師三日館穀,及癸酉而還,甲午,至于衡雍,作王宮于踐土,鄉役之三月,鄭伯如楚,致其師,為楚師既敗而懼,使子人九行成于晉,晉欒枝入盟鄭伯,五月,丙午,晉侯及鄭伯盟于衡雍,丁未,獻楚俘于王,駟介百乘,徒兵千,鄭伯傅王,用平禮也,已酉,王享醴,命晉侯宥,王命尹氏,及王子虎,內史叔興父策命晉侯為侯伯,賜之大輅之服,戎輅之服,彤弓一,彤矢百,玈弓矢千,秬鬯一卣,虎賁三百人,曰,王謂叔父,敬服王命,以綏四國,糾逖王慝,晉侯三辭,從命,曰,重耳敢再拜稽首,奉揚天子之丕顯休命,受策以出,出入三覲。

衛侯聞楚師敗,懼,出奔楚,遂適陳,使元咺奉叔武以受盟,癸亥,王子虎盟諸侯于王庭,要言曰,皆獎王室,無相害也,有渝此盟,明神殛之,俾隊其師,無克祚國,及其玄孫,無有老幼,君子謂是盟也信,謂晉於是役也,能以德攻,初,楚子玉自為瓊弁玉纓,未之服也,先戰,夢河神謂己曰,畀余,余賜女孟諸之麋,弗致也,大心與子西,使榮黃諫,弗聽,榮季曰,死而利國,猶或為之,況瓊玉乎,是糞土也,而可以濟師,將何愛焉,弗聽,出告二子曰,非神敗令尹,令尹其不勤民,實自敗也,既敗,王使謂之曰,大夫若入,其若申息之老何,子西,孫伯,曰,得臣將死,二臣止之曰,君其將以為戮,及連穀而死,晉侯聞之,而後喜可知也,曰,莫余毒也已,蒍呂臣實為令尹,奉己而已,不在民矣。

或訴元咺於衛侯曰,立叔武矣,其子角從公,公使殺之,咺不廢命,奉夷叔以入守,六月,晉人復衛侯,甯武子與衛人盟于宛濮,曰,天禍衛國,君臣不協,以及此憂也,今天誘其衷,使皆降心以相從也,不有居者,誰守社稷,不有行者,誰扞牧圉,不協之故,用昭乞盟于爾大神,以誘天衷,自今日以往,既盟之後,行者無保其力,居者無懼其罪,有渝此盟,以相及也,明神先君,是糾是殛,國人聞此盟也,而後不貳,衛侯先期入,甯子先,長牂守門,以為使也,與之乘而入,公子歂犬,華仲,前驅,叔孫將沐,聞君至,喜,捉髮走出,前驅射而殺之,公知其無罪也,枕之股而哭之,歂犬走出,公使殺之,元咺出奔晉。

城濮之戰,晉中軍風于澤,亡大旆之左旃,祁瞞奸命,司馬殺之,以徇于諸侯,使茅茷代之,師還,壬午,濟河,舟之僑先歸,士會攝右,秋,七月,丙申,振旅愷以入于晉,獻俘授馘,飲至大賞,徵會討貳,殺舟之僑以徇于國,民於是大服。君子謂文公其能刑矣。三罪而民服,《詩》云:「惠此中國,以綏四方。」不失賞刑之謂也。

冬,會于溫,討不服也,衛侯與元咺訟,甯武子為輔,鍼莊子為坐,士榮為大士,衛侯不勝,殺士榮,刖鍼莊子,謂甯俞忠而免之,執衛侯,歸之于京師,寘諸深室,甯子職納橐饘焉,元咺歸于衛,立公子瑕。

是會也,晉侯召王,以諸侯見,且使王狩,仲尼曰,以臣召君,不可以訓,故書曰,天王狩于河陽,言非其地,也且明德也,壬申,公朝于王所,丁丑,諸侯圍許,晉侯有疾,曹伯之豎侯,獳貨筮史,使曰,以曹為解,齊桓公為會而封異姓,今君為會,而滅同姓,曹叔振鐸,文之昭也,先君唐叔,武之穆也,且合諸侯而滅兄弟,非禮也,與衛偕命,而不與偕復,非信也,同罪異罰,非刑也,禮以行義,信以守禮,刑以正邪,舍此三者,君將若之何,公說,復曹伯,遂會諸侯于許,晉侯作三行以禦狄,荀林父將中行,屠擊將右行,先篾將左行。

二十有九年,春,介葛盧來。

公至自圍許。

夏,六月,會王人,晉人,宋人,齊人,陳人,蔡人,秦人,盟于翟泉。

秋,大雨雹。

冬,介葛盧來。

二十九年,春,葛盧來朝,舍于昌衍之上,公在會,饋之芻米,禮也。

夏,公會王子虎,晉孤偃,宋公孫固,齊國歸父,陳轅濤塗,秦小子憖,盟于翟泉,尋踐土之盟,且謀伐鄭也,卿不書,罪之也,在禮,卿不會公侯,會伯子男可也。

秋,大雨雹,為災也。

冬,介葛盧來,以未見公,故復來朝,禮之,加燕好,介葛盧聞牛鳴,曰,是生三犧,皆用之矣,其音云,問之而信。

三十年,春,王正月。

夏狄侵齊。

秋,衛殺其大夫元咺,及公子瑕,衛侯鄭歸于衛。

晉人,秦人,圍鄭,介人侵蕭,冬,天王使宰周公來聘,公子遂如京師,遂如晉。

三十年,春,晉人侵鄭,以觀其可攻與否,狄間晉之有鄭虞也,夏,狄侵齊。

晉侯使醫衍酖衛侯,甯俞貨醫,使薄其酖,不死,公為之請,納玉於王,與晉侯,皆十榖,王許之,秋,乃釋衛侯。

衛侯使賂周歂,冶廑,曰,苟能納我,吾使爾為卿,周冶殺元咺,及子適,子儀,公入祀先君,周冶既服將命,周歂先入,及門,遇疾而死,冶廑辭卿。

九月,甲午,晉侯,秦伯,圍鄭,以其無禮於晉,且貳於楚也,晉軍函陵,秦軍汜南,佚之狐言於鄭伯曰,國危矣,若使燭之武見秦君,師必退,公從之,辭曰,臣之壯也,猶不如人,今老矣,無能為也已。公曰:吾不能早用子,今急而求子,是寡人之過也。然鄭亡,子亦有不利焉,許之,夜縋而出,見秦伯曰,秦晉圍鄭,鄭既知亡矣,若亡鄭而有益於君,敢以煩執事,越國以鄙遠,君知其難也,焉用亡鄭以倍鄰,鄰之厚,君之薄也,若舍鄭以為東道主,行李之往來,共其乏困,君亦無所害,且君嘗為晉君賜矣,許君焦瑕,朝濟而夕設版焉,君之所知也,夫晉何厭之有,既東封鄭,又欲肆其西封。若不闕秦,將焉取之?闕秦以利晉,唯君圖之。秦伯說,與鄭人盟,使杞子逢孫楊孫戍之,乃還,子犯謂擊之,公曰,不可。微夫人之力不及此。因人之力而敝之,不仁,失其所與,不知。以亂易整,不武。吾其還也,亦去之。

初,鄭公子蘭出奔晉,從於晉侯伐鄭,請無與圍鄭,許之,使待命于東,鄭石甲父,侯宣多,逆以為大子,以求成于晉,晉人許之。

冬,王使周公閱來聘,饗有昌歜,白,黑,形鹽,辭曰,國君文足昭也,武可畏也,則有備物之饗,以象其德,薦五味,羞嘉穀,鹽虎形,以獻其功,吾何以堪之。

東門襄仲將聘于周,遂初聘于晉。

三十有一年,春取濟西田。

公子遂如晉。

夏,四月,四卜郊不從,乃免牲,猶三望。

秋,七月。

冬,杞伯姬來求婦。

狄圍衛,十有二月,衛遷于帝丘。

三十一年,春,取濟西田,分曹地也,使臧文仲往,宿於重館,重館人告曰,晉新得諸侯,必親其共,不速行,將無及也,從之,分曹地自洮以南,東傅于濟,盡曹地也,襄仲如晉,拜曹田也。

夏,四月,四卜郊,不從,乃免牲,非禮也,猶三望,亦非禮也,禮不卜常祀,而卜其牲日,牛卜日曰牲,牲成而卜郊,上怠慢也,望,郊之細也,不郊,亦無望可也。

秋,晉蒐于清原,作五軍似禦狄,趙衰為卿。

冬,狄圍衛,衛遷于帝丘,卜曰,三百年,衛成公夢康叔曰,相奪予享,公命祀相,甯武子不可,曰,鬼神非其族類,不歆其祀,杞鄫何事,相之不享,於此久矣,非衛之罪也,不可以間成王周公之命祀,請改祀命。

鄭洩駕惡公子瑕,鄭伯亦惡之,故公子瑕出奔楚。

三十有二年,春,王正月。

夏,四月,己丑,鄭伯捷卒。

衛人侵狄。

秋,衛人及狄盟。

冬,十有二月,己卯,晉侯重耳卒。

三十二年,春,楚鬥章請平于晉,晉陽處父報之,晉楚始通。

夏,狄有亂,衛入侵狄。

狄請平焉。

秋,衛人及狄盟。

冬,晉文公卒。庚辰,將殯于曲沃,出絳柩,有聲如牛,卜偃使大夫拜曰,君命大事,將有西師過軼我,擊之必大捷焉,杞子自鄭使告于秦曰,鄭人使我掌其北門之管,若潛師以來,國可得也,穆公訪諸蹇叔。蹇叔曰,勞師以襲遠,非所聞也。師勞力竭,遠主備之,無乃不可乎,師之所為,鄭必知之,勤而無所,必有悖心,且行千里,其誰不知,公辭焉,召孟明,西乞,白乙,使出師於東門之外,蹇叔哭之曰,孟子,吾見師之出,而不見其入也,公使謂之曰,爾何知,中壽,爾墓之木拱矣。蹇叔之子與師。哭而送之曰,晉人禦師必於殽,殽有二陵焉,其南陵,夏后皋之墓也,其北陵,文王之所辟風雨也,必死是間,余收爾骨焉,秦師遂東。

三十有三年,春,王二月,秦人入滑。

齊侯使國歸父來聘。

夏,四月,辛巳,晉人及姜戎敗秦師于殽。

癸巳,葬晉文公。

狄侵齊。

公伐邾,取訾婁。

秋,公子遂帥師伐邾。

晉人敗狄于箕。

冬,十月,公如齊,十有二月,公至自齊。

乙巳,公薨于小寢。

隕霜不殺草,李梅實。

晉人,陳人,鄭人,伐許。

三十三年,春,晉秦師過周北門,左右免冑而下,超乘者三百乘。王孫滿尚幼,觀之,言於王曰:「秦師輕而無禮,必敗。輕則寡謀,無禮則脫,入險而脫,又不能謀,能無敗乎?」及滑,鄭商人弦高,將市於周,遇之。以乘韋先,牛十二,犒師,曰:「寡君聞吾子,將步師出於敝邑,敢犒從者;不腆敝邑,為從者之淹,居則具一日之積,行則備一夕之衛。」且使遽告于鄭。鄭穆公使視客館,則束載,厲兵,秣馬矣。使皇武子辭焉,曰:「吾子淹久於敝邑,唯是脯資,餼牽竭矣。為吾子之將行也,鄭之有原圃,猶秦之有具囿也;吾子取其麋鹿,以閒敝邑,若何?」杞子奔齊,逢孫楊孫奔宋。孟明曰:「鄭有備矣,不可冀也。攻之不克,圍之不繼,吾其還也。」滅滑而還。

齊國莊子來聘,自郊勞至于贈賄,禮成而加之以敏,臧文仲言於公曰,國子為政,齊猶有禮,君其朝焉,臣聞之,服於有禮,社稷之衛也。

晉原軫曰,秦違蹇叔而以貪勤民,天奉我也,奉不可失,敵不可縱,縱敵患生,違天不祥,必伐秦師,欒枝曰,未報秦施而伐其師,其為死君乎,先軫曰,秦不哀吾喪,而伐吾同姓,秦則無禮,何施之為,吾聞之,一日縱敵,數世之患也,謀及子孫,可謂死君乎,遂發命,遽興姜戎,子,墨衰絰,梁弘御戎,萊駒為右。

夏,四月,辛巳,敗秦師于殽,獲百里孟明視,西乞術,白乙丙,以歸。遂墨以葬文公,晉於是始墨。文嬴請三帥,曰,彼實構吾二君,寡君若得而食之,不厭,君何辱討焉,使歸就戮于秦,以逞寡君之志,若何,公許之,先軫朝,問秦囚,公曰,夫人請之,吾舍之矣,先軫怒曰,武夫力而拘諸原,婦人暫而免諸國,墮軍實而長寇讎,亡無日矣,不顧而唾。公使陽處父追之,及諸河,則在舟中矣。釋左驂,以公命,贈孟明,孟明稽首曰,君之惠,不以纍臣釁鼓,使歸就戮于秦,寡君之以為戮,死且不朽,若從君惠而免之,三年將拜君賜,秦伯素服郊次,鄉師而哭曰,孤違蹇叔,以辱二三子,孤之罪也,不替孟明,孤之過也,大夫何罪,且吾不以一眚掩大德。

狄侵齊,因晉喪也。

公伐邾,取訾婁,以報升陘之役,邾人不設備,秋,襄仲復伐邾。

狄伐晉,及箕,八月,戊子,晉侯敗狄于箕,郤缺獲白狄子,先軫曰,匹夫逞志於君,而無討,敢不自討乎,免冑入狄師,死焉,狄人歸其元,面如生,初,臼季使過冀,見冀缺耨,其妻饁之,敬,相待如賓,與之歸,言諸文公曰,敬,德之聚也,能敬必有德,德以治民,君請用之,臣聞之,出門如賓,承事如祭,仁之則也,公曰,其父有罪,可乎,對曰,舜之罪也,殛鯀,其舉也興禹,管敬仲,桓之賊也,實相以濟,《康誥》曰:父不慈,子不祗,兄不友,弟不共,不相及也,《詩》曰:采葑采菲,無以下體。君取節焉可也,文公以為下軍大夫,反自箕,襄公以三命命先且居將,中軍,以再命命先茅之縣賞胥臣,曰,舉郤缺,子之功也,以一命命郤缺為卿,復與之冀,亦未有軍行。

冬,公如齊朝,且弔有狄師也,反,薨于小寢,即安也。

晉,陳,鄭,伐許,討其貳於楚也,楚令尹子上侵陳蔡,陳蔡成,遂伐鄭,將納公子瑕,門于桔柣之門,瑕覆于周氏之汪,外僕髡屯禽之以獻,文夫人斂而葬之鄶城之下。

晉陽處父侵蔡,楚子上救之,與晉師夾泜而軍,陽子患之,使謂子上曰,吾聞之,文不犯順,武不違敵,子若欲戰,則吾退舍,子濟而陳,遲速唯命,不然紓我,老師費財,亦無益也,乃駕以待,子上欲涉,大孫伯曰,不可,晉人無信,半涉而薄我,悔敗何及,不如紓之,乃退舍,陽子宣言曰,楚師遁矣,遂歸,楚師亦歸,大子商臣譖子上曰,受晉賂而辟之,楚之恥也,罪莫大焉,王殺子上。

葬僖公緩,作主,非禮也,凡君薨,卒哭而祔,祔而作主,特祀於主,烝嘗禘於廟。


\end{pinyinscope}