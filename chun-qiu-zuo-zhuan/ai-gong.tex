\article{哀公}

\begin{pinyinscope}
元年,春,王正月,公即位。

楚子,陳侯,隨侯,許男,圍蔡。

鼷鼠食郊牛,改卜牛。

夏,四月,辛巳郊。

秋,齊侯,衛侯,伐晉。

冬,仲孫何忌帥師伐邾。

元年,春,楚子圍蔡,報柏舉也,里而栽,廣丈,高倍,夫屯,晝夜九日,如子西之素,蔡人男女以辨,使疆于江汝之間,而還,蔡於是乎請遷于吳。

吳王夫差敗越于夫椒,報檇李也,遂入越,越子以甲楯五千,保于會稽,使大夫種因吳大宰嚭以行成,吳子將許之,伍員曰,不可,臣聞之,樹德莫如滋,去疾莫如盡,昔有過澆,殺斟灌以伐斟鄩,滅夏后相,后緡方娠,逃出自竇,歸于有仍,生少康焉,為仍牧正,惎澆能戒之,澆使椒求之,逃奔有虞,為之庖正,以除其害,虞思於是妻之以二姚,而邑諸綸,有田一成,有眾一旅,能布其德,而兆其謀,以收夏眾,撫其官職,使女艾諜澆,使季杼誘豷。遂滅過戈,復禹之績。祀夏配天,不失舊物,今吳不如過,而越大於少康,或將豐之,不亦難乎,句踐能親而務施,施不失人,親不棄勞,與我同壤,而世為仇讎,於是乎克而弗取,將又存之,違天而長寇讎,後雖悔之,不可食巳,姬之衰也,日可俟也,介在蠻夷,而長寇讎。以是求伯,必不行矣。弗聽,退而告人曰,越十年生聚,而十年教訓,二十年之外,吳其為沼乎,三月,越及吳平,吳入越不書,吳不告慶,越不告敗也。

夏,四月,齊侯,衛侯,救邯鄲,圍五鹿。

吳之入楚也,使召陳懷公,懷公朝國人而問焉,曰,欲與楚者右,欲與吳者左,陳人從田,無田從黨。逢滑當公而進。曰,臣聞國之興也以福,其亡也以禍,今吳未有福,楚未有禍,楚未可棄,吳未可從,而晉盟主也,若以晉辭,吳若何?公曰,國勝君亡,非禍而何,對曰,國之有是多矣,何必不復,小國猶復,況大國乎,臣聞國之興也,視民如傷,是其福也,其亡也,以民為土芥,是其禍也,楚雖無德,亦不艾殺其民,吳日敝於兵,暴骨如莽,而未見德焉,天其或者正訓楚也,禍之適吳,其何日之有,陳侯從之,及夫差克越,乃脩先君之怨,秋,八月,吳侵陳,脩舊怨也。

齊侯,衛侯,會于乾侯,救范氏也,師及齊師,衛孔圉鮮虞人伐晉,取棘蒲。

吳師在陳,楚大夫皆懼曰,闔廬惟能用其民,以敗我於柏舉,今聞其嗣又甚焉,將若之何,子西曰,二三子恤不相睦,無患吳矣,昔闔廬食不二味,居不重席,室不崇壇,器不彤鏤,宮室不觀,舟車不飾,衣服財用,擇不取費,在國,天有菑癘,親巡其孤寡,而共其乏困,在軍,熟食者分而後敢食,其所嘗者,卒乘與焉,勤恤其民,而與之勞逸,是以民不罷勞,死不知曠,吾先大夫子常易之,所以敗我也,今聞夫差,次有臺榭陂池焉,宿有妃嬙嬪御焉,一日之行,所欲必成,玩好必從,珍異是聚,觀樂是務,視民如讎,而用之日新,夫先自敗也已,安能敗我。

冬,十一月,晉趙鞅伐朝歌。

二年,春,王二月,季孫斯,叔孫州,仇仲孫何忌,帥師伐邾,取漷東田及沂西田,癸巳,叔孫州仇,仲孫何忌,及邾人盟于句繹。

夏,四月,丙子,衛侯元卒。

滕子來朝。

晉趙鞅帥師納衛世子蒯瞶于戚。

秋,八月,甲戌,晉趙鞅帥師,及鄭罕達帥師,戰于鐵,鄭師敗績。

冬,十月,葬衛靈公。

十有一月,蔡遷于州來,蔡殺其大夫公子駟。

二年,春,伐邾,將伐絞,邾人愛其土,故賂以漷沂之田而受盟。

初,衛侯遊于郊,子南僕,公曰,余無子,將立女,不對,他日又謂之,對曰,郢不足以辱社稷,君其改圖,君夫人在堂,三揖在下,君命祇辱,夏,衛靈公卒,夫人曰,命公子郢為大子,君命也,對曰,郢異於他子,且君沒於吾手,若有之,郢必聞之,且亡人之子輒在,乃立輒,六月,乙酉,晉趙鞅納衛大子于戚,宵迷,陽虎曰,右河而南必至焉,使大子絻,八人衰絰,偽自衛逆者,告於門,哭而入,遂居之。

秋,八月,齊人輸范氏粟,鄭子姚,子般,送之,士吉射逆之,趙鞅禦之,遇于戚,陽虎曰,吾車少,以兵車之旆,與罕駟兵車,先陳,罕駟自後隨而從之,彼見吾貌,必右懼心,於是乎會之,必大敗之,從之,卜戰,龜焦,樂丁曰,詩曰,爰始爰謀,爰契我龜,謀協以故,兆詢可也,簡子誓曰,范氏中行氏反易天明,斬艾百姓,欲擅晉國而滅其君,寡君恃鄭而保焉,今鄭為不道,棄君助臣,二三子順天明,從君命,經德義,除詬恥,在此行也,克敵者,上大夫受縣,下大夫受郡,士田十萬,庶人工商遂,人臣隸圉免,志父無罪,君實圖之,若其有罪,絞縊以戮,桐棺三寸,不設屬辟,素車樸馬,無入于兆,下卿之罰也,甲戌,將戰,郵無恤御簡子,衛太子為右,登鐵上,望見鄭師眾,大子懼,自投于車下,子良授大子綏而乘之,曰,婦人也,簡子巡列,曰,畢萬匹夫也,七戰皆獲,有馬百乘,死於牖下,群子勉之,死不在寇,繁羽御趙羅,宋勇為右,羅無勇麇之,吏詰之,御對曰,痁作而伏,衛大子禱曰,曾孫蒯聵,敢昭告皇祖文王,烈祖康叔,文祖襄公,鄭勝亂從,晉午在難,不能治亂,使鞅討之,蒯聵不敢自佚,備持矛焉,敢告無絕筋,無折骨,無面傷,以集大事,無作三祖羞,大命不敢請,佩玉不敢愛,鄭人擊簡子中肩,斃于車中,獲其蜂旗,大子救之以戈,鄭師北,獲溫大夫趙羅,大子復伐之,鄭師大敗,獲齊粟千車,趙孟喜曰,可矣,傅傁曰,雖克鄭,猶有知在,憂未艾也,初,周人與范氏田,公孫尨稅焉,趙氏得而獻之,吏請殺之,趙孟曰,為其主也,何罪,止而與之田,及鐵之戰,以徒五百人,宵攻鄭師,取蜂旗於子姚之幕下,獻曰,請報主德,追鄭師,姚,般,公孫林,殿而射,前列多死,趙孟曰,國無小,既戰,簡子曰,吾伏弢嘔血,鼓音不衰,今日我上也,大子曰,吾救主於車,退敵於下,我右之上也,郵良曰,我兩靷將絕,吾能止之,我御之上也,駕而乘材,兩靷皆絕。

吳洩庸如蔡納聘,而稍納師,師畢入,眾知之,蔡侯告大夫殺公子駟以說,哭而遷墓,冬,蔡遷于州來。

三年,春,齊國夏,衛石曼姑,帥師圍戚。

夏,四月,甲午,地震。

五月,辛卯,桓宮僖宮災。

季孫斯,叔孫州仇,帥師城啟陽。

宋樂髡帥師伐曹。

秋,七月,丙子,季孫斯卒。

蔡人放其大夫公孫獵于吳。

冬,十月,癸卯,秦伯卒。

叔孫州仇,仲孫何忌,帥師圍邾。

三年,春,齊衛圍戚,求援于中山。

夏,五月,辛卯,司鐸火,火踰公宮,桓僖災,救火者皆曰顧府,南宮敬叔至,命周人出御書,俟於宮曰,庀女而不在死,子服景伯至,命宰人出禮書,以待命,命不共,有常刑,校人乘馬,巾車脂轄,百官官備,府庫慎守,官人肅給,濟濡帷幕,鬱攸從之,蒙茸公屋,自大廟始,外內以悛,助所不給,有不用命,則有常刑,無赦,公父文伯至,命校人駕乘車,季桓子至,御公立于象魏之外,命救火者,傷人則止,財可為也,命藏象魏,曰,舊章不可亡也,富父槐至,曰,無備而官辦者,猶拾瀋也,於是乎去表之稿,道還公宮,孔子在陳,聞火,曰,其桓僖乎。

劉氏,范氏,世為婚姻,萇弘事劉文公,故周與范氏,趙鞅以為討,六月,癸卯,周人殺萇弘。

秋,季孫有疾,命正常曰,無死,南孺子之子,男也,則以告而立之,女也,則肥也可,季孫卒,康子即位,既葬,康子在朝,南氏生男正常,載以如朝,告曰,夫子有遺言,命其圉臣曰,南氏生男,則以告於君,與大夫,而立之,今生矣,男也,敢告,遂奔衛,康子請退,公使共劉視之,則或殺之矣,乃討之,召正常,正常不反。

冬,十月,晉趙鞅圍朝歌,師于其南,荀寅伐其郛,使其徒自北門入,己犯師而出,癸丑,奔邯鄲,十一月,趙鞅殺士皋夷,惡范氏也。

四年,春,王二月,庚戌,盜殺蔡侯申,蔡公孫辰出奔吳。

葬秦惠公。

宋人執小邾子。

夏,蔡殺其大夫公孫姓,公孫霍。

晉人執戎蠻子赤歸于楚。

城西郛。

六月,辛丑,亳社災。

秋,八月,甲寅,滕子結卒。

冬,十有二月,葬蔡昭公。

葬滕頃公。

四年,春,蔡昭侯將如吳,諸大夫恐其又遷也,承,公孫翩逐而射之,入於家人而卒,以兩矢門之,眾莫敢進,文之鍇後至,曰,如牆而進,多而殺二人,鍇執弓而先,翩射之,中肘,鍇遂殺之,故逐公孫辰,而殺公孫姓,公孫盱。

夏,楚人既克夷虎,乃謀北方,左司馬販,申公壽餘,葉公諸梁,致蔡於負函,致方城之外於繒關,曰,吳將泝江入郢,將奔命焉,為一昔之期,襲梁及霍,單浮餘圍蠻氏,蠻氏潰,蠻子赤奔晉陰地,司馬起豐析與狄戎,以臨上雒,左師軍于菟和,右師軍于倉野,使謂陰地之命大夫士蔑曰,晉楚有盟,好惡同之,若將不廢,寡君之願也,不然,將通於少,習以聽命,士蔑請諸趙孟,趙孟曰,晉國未寧,安能惡於楚,必速與之,士蔑乃致九州之戎,將裂田以與蠻子而城之,且將為之卜,蠻子聽卜,遂執之,與其五大夫,以畀楚師于三戶,司馬致邑立宗焉,以誘其遺民,而盡俘以歸。

秋,七月,齊陳乞,弦施,衛甯跪,救范氏,庚午,圍五鹿,九月,趙鞅圍邯鄲,冬,十一月,邯鄲降,荀寅奔鮮虞,趙稷奔臨,十二月,弦施逆之,遂墮臨,國夏伐晉,取邢,任,欒,鄗,逆畤,陰人,盂,壺口,會鮮虞,納荀寅于柏人。

五年,春,城毗。

夏,齊侯伐宋。

晉趙鞅帥師伐衛。

秋,九月,癸酉,齊侯杵臼卒。

冬,叔還如齊。

閏月,葬齊景公。

五年,春,晉圍柏人,荀寅,士吉射,奔齊,初,范氏之臣王生,惡張柳朔,言諸昭子,使為柏人,昭子曰,夫非而讎乎,對曰,私讎不及,公好不廢,過惡不去,善義之經也,臣敢違之,及范氏出,張柳朔謂其子,爾從主,勉之,我將止死,主生授我矣,吾不可以僭之,遂死於柏人。

夏,趙鞅伐衛,范氏之故也,遂圍中牟。

齊燕姬生子,不成而死,諸子,鬻姒之子荼嬖,諸大夫恐其為大子也,言於公曰,君之齒長矣,未有大子,若之何,公曰,二三子間於憂虞,則有疾疢,亦姑謀樂,何憂於無君,公疾,使國惠子,高昭子,立荼,寘群公子於萊,秋,齊景公卒,冬,十月,公子嘉,公子駒,公子黔,奔衛,公子鉏,公子陽生,來奔,萊人歌之曰,景公死乎不與埋,三軍之事乎不與謀,師乎師乎,何黨之乎。

鄭駟秦富而侈,嬖大夫也,而常陳卿之車服於其庭,鄭人惡而殺之,子思曰,詩曰,不解于位,民之攸塈,不守其位,而能久者鮮矣,商頌曰,不僭不濫,不敢怠皇,命以多福。

六年,春,城邾瑕。

晉趙鞅帥師伐鮮虞。

吳伐陳。

夏,齊國夏及高張來奔。

叔還會吳于柤。

秋,七月,庚寅,楚子軫卒。

齊陽生入于齊,齊陳乞弒其君荼。

冬,仲孫何忌帥師伐邾。

宋向巢帥師伐曹。

六年,春,晉伐鮮虞,治范氏之亂也。

吳伐陳,復脩舊怨也,楚子曰,吾先君與陳有盟,不可以不救,乃救陳,師于城父。

齊陳乞偽事高國者,每朝必驂乘焉,所從必言,諸大夫曰,彼皆偃蹇,將棄子之命,皆曰,高國得君,必偪我,盍去諸,固將謀子,子早圖之,圖之莫如盡滅之,需事之下也,及朝,則曰,彼虎狼也,見我在子之側,殺我無日矣,請就之位,又謂諸大夫曰,二子者禍矣,恃得君而欲謀二三子,曰,國之多難,貴寵之由,盡去之而後君定,既成謀矣,盍及其未作也,先諸作而後悔,亦無及也,大夫從之,夏,六月,戊辰,陳乞鮑牧及諸大夫以甲入于公宮,昭子聞之,與惠子乘如公,戰于莊,敗,國人追之,國夏奔莒,遂及高張晏圉,弦施來奔。

秋,七月,楚子在城父,將救陳,卜戰不吉,卜退不吉,王曰,然則死也,再敗楚師,不如死,棄盟逃讎,亦不如死,死一也,其死讎乎,命公子申為王,不可,則命公子結,亦不可,則命公子啟,五辭而後許,將戰,王有疾,庚寅,昭王攻大冥,卒于城父,子閭退曰,君王舍其子而讓,群臣敢忘君乎,從君之命,順也,立君之子,亦順也,二順不可失也,與子西,子期,謀潛師閉塗,逆越女之子章立之,而後還,是歲也,有雲如眾,赤鳥夾日以飛,三日,楚子使問諸周大史,周大史曰,其當王身乎,若禜之,可移於令尹,司馬王曰,除腹心之疾,而寘諸股肱何益,不穀不有大過,天其夭諸,有罪受罰,又焉移之,遂弗禜,初,昭王有疾,卜曰,河為祟,王弗祭,大夫請祭諸郊,王曰,三代命祀,祭不越望,江漢雎章,楚之望也,禍福之至,不是過也,不穀雖不德,河非所獲罪也,遂弗祭,孔子曰,楚昭王知大道矣,其不失國也宜哉,夏書曰,惟彼陶唐,帥彼天常,有此冀方,今失其行,亂其紀綱,乃滅而亡,又曰,允出茲在茲,由己率常可矣。

八月,齊邴意茲來奔,陳僖子使召公子陽生,陽生駕而見南郭且于,曰,嘗獻馬於季孫,不入於上乘,故又獻此,請與子乘之,出萊門而告之故,闞止知之,先待諸外,公子曰,事未可知,反與壬也處,戒之,遂行,逮夜至於齊,國人知之,僖子使子士之母養之,與饋者皆入,冬,十月,丁卯,立之,將盟,鮑子醉而往,其臣差車鮑點,曰,此誰之命也,陳子曰,受命于鮑子,遂誣鮑子曰,子之命也,鮑子曰,女忘君之為孺子牛,而折其齒乎,而背之也,悼公稽首曰,吾子奉義而行者也,若我可,不必亡一大夫,若我不可,不必亡一公子,義則進,否則退,敢不唯子是從,廢興無以亂,則所願也,鮑子曰,誰非君之子,乃受盟,使胡姬以安孺子如賴,去鬻姒,殺王甲,拘江說,囚王豹于句竇之丘,公使朱毛告於陳子曰,微子則不及此,然君異於器,不可以二,器二不匱,君二多難,敢布諸大夫,僖子不對而泣,曰,君舉不信群臣乎,以齊國之困,困又有憂,少君不可以訪,是以求長君,庶亦能容群臣乎,不然,夫孺子何罪,毛復命,公悔之,毛曰,君大訪於陳子,而圖其小,可也,使毛遷孺子於駘,不至,殺諸野幕之下,葬諸殳冒淳。

七年,春,宋皇瑗帥師侵鄭。

晉魏曼多帥師侵衛。

夏,公會吳于鄫。

秋,公伐邾,八月,己酉,入邾,以邾子益來。

宋人圍曹。

冬,鄭駟弘帥師救曹。

七年,春,宋師侵鄭,鄭叛晉故也,晉師侵衛,衛不服也。

夏,公會吳于鄫,吳來徵百牢,子服景伯對曰,先王未之有也,吳人曰,宋百牢我,魯不可以後宋,且魯牢晉大夫過十,吳王百牢,不亦可乎,景伯曰,晉范鞅貪而棄禮,以大國懼敝邑,故敝邑十一牢之,君若以禮命於諸侯,則有數矣,若亦棄禮,則有淫者矣,周之王也,制禮上物,不過十二,以為天之大數也,今棄周禮,而曰必百牢,亦唯執事,吳人弗聽,景伯曰,吳將亡矣,棄天而背本,不與,必棄疾於我,乃與之,太宰嚭召季康子,康子使子貢辭,大宰嚭曰,國君道長,而大夫不出門,此何禮也,對曰,豈以為禮,畏大國也,大國不以禮命於諸侯,苟不以禮,豈可量也,寡君既共命焉,其老豈敢棄其國,大伯端委以治周禮,仲雍嗣之,斷髮文身,臝以為飾,豈禮也哉,有由然也,反自鄫,以吳為無能為也。

季康子欲伐邾,乃饗大夫以謀之,子服景伯曰,小所以事大,信也,大所以保小,仁也,背大國不信,伐小國不仁,民保於城,城保於德,失二德者,危將焉保,孟孫曰,二三子以為何如,惡賢而逆之,對曰,禹合諸侯於塗山,執玉帛者萬國,今其存者,無數十焉,唯大不字小,小不事大也,知必危,何故不言,魯德如邾,而以眾加之,可乎,不樂而出,秋伐邾,及范門,猶聞鍾聲,大夫諫,不聽,茅成子請告於吳,不許,曰,魯擊柝聞於邾,吳二千里,不三月不至,何及於我,且國內豈不足,成子以茅叛,師遂入邾,處其公宮,眾師晝掠,邾眾保于繹,師宵掠,以邾子益來,獻于亳社,囚諸負瑕,負瑕故有繹,邾茅夷鴻以束帛乘韋,自請救於吳,曰,魯弱晉而遠吳,馮恃其眾而背君之盟,辟君之執事,以陵我小國,邾非敢自愛也,懼君威之不立,君威之不立,小國之憂也,若夏盟於鄫衍,秋而背之,成求而不違,四方諸侯,其何以事君,且魯賦八百乘,君之貳也,邾賦六百乘,君之私也,以私奉貳,唯君圖之,吳子從之。

宋人圍曹,鄭桓子思曰,宋人有曹,鄭之患也,不可以不救,冬,鄭師救曹,侵宋,初,曹人或夢眾君子立于社宮,而謀亡曹,曹叔振鐸請待公孫彊,許之,旦而求之曹,無之,戒其子曰,我死,爾聞公孫彊為政,必去之,及曹伯陽即位,好田弋,曹鄙人公孫彊好弋,獲白鴈,獻之,且言田弋之說,說之,因訪政事,大說之,有寵使為司城以聽政,夢者之子乃行,彊言霸說於曹伯,曹伯從之,乃背晉而奸宋,宋人伐之,晉人不救,築五邑於其郊,曰,黍丘,揖丘,大城,鍾,邘。

八年,春,王正月,宋公入曹,以曹伯陽歸。

吳伐我。

夏,齊人取讙及闡。

歸邾子益于邾。

秋,七月。

冬,十有二月,癸亥,杞伯過卒。

齊人歸讙及闡。

八年,春,宋公伐曹,將還,褚師子肥殿,曹人詬之,不行,師待之,公聞之怒,命反之,遂滅曹,執曹伯,及司城彊以歸,殺之。

吳為邾故,將伐魯,問於叔孫輒,叔孫輒對曰,魯有名而無情,伐之必得志焉,退而告公山不狃,公山不狃曰,非禮也,君子違不適讎國,未臣而有伐之,奔命焉,死之可也,所託也則隱,且夫人之行也,不以所惡廢鄉,今子以小惡而欲覆宗國,不亦難乎,若使子率,子必辭,王將使我子張疾之,王問於子洩,對曰,魯雖無與立,必有與斃,諸侯將救之,未可以得志焉,晉與齊楚輔之,是四讎也,夫魯,齊晉之脣,脣亡齒寒,君所知也,不救何為,三月,吳伐我,子洩率,故道險,從武城,初武城人或有因於吳竟田焉,拘鄫人之漚菅者,曰,何故使吾水滋,及吳師至,拘者道之,以伐武城,克之,王犯嘗為之宰,澹臺子羽之父好焉,國人懼,懿子謂景伯,若之何,對曰,吳師來,斯與之戰,何患焉,且召之而至,又何求焉,吳師克東陽而進,舍於五梧,明日,舍於蠶室,公賓庚,公甲叔子,與戰于夷,獲叔子與析朱鉏,獻於王,王曰,此同車,必使能,國未可望也,明日舍于庚宗,遂次于泗上,微虎欲宵攻王舍,私屬徒七百人,三踊於幕庭,卒三百人,有若與焉,及稷門之內,或謂季孫曰,不足以害吳,而多殺國士,不如已也,乃止之,吳子聞之,一夕三遷,吳人行成,將盟,景伯曰,楚人圍宋,易子而食,析骸而爨,猶無城下之盟,我未及虧,而有城下之盟,是棄國也,吳輕而遠,不能久,將歸矣,請少待之,弗從,景伯負載,造於萊門,乃請釋子服何於吳。吳人許之,以王子姑曹當之,而後止,吳人盟而還。

齊悼公之來也,季康子以其妹妻之,即位而逆之,季魴侯通焉,女言其情,弗敢與也,齊侯怒,夏,五月,齊鮑牧帥師伐我,取讙及闡,或譖胡姬於齊侯,曰,安孺子之黨也,六月,齊侯殺胡姬,齊侯使如吳請師,將以伐我,乃歸邾子,邾子又無道,吳子使大宰子餘討之,囚諸樓臺栫之以棘,使諸大夫奉大子革以為政。

秋,及齊平,九月,臧賓如如齊蒞盟,齊閭丘明來蒞盟,且逆季姬以歸,嬖,鮑牧又謂群公子曰,使女有馬千乘乎,公子愬之,公謂鮑子,或譖子,子姑居於潞以察之,若有之,則分室以行,若無之,則反子之所,出門,使以三分之一行,半道,使以二乘,及潞,麇之以入,遂殺之。

冬,十二月,齊人歸讙及闡,季姬嬖故也。

九年,春,王二月,葬杞僖公。

宋皇瑗帥師取鄭師于雍丘。

夏,楚人伐陳。

秋,宋公伐鄭。

冬,十月。

九年,春,齊侯使公孟綽辭師于吳,吳子曰,昔歲寡人聞命,今又革之,不知所從,將進受命於君。

鄭武子賸之嬖,許瑕求邑,無以與之,請外取,許之,故圍宋雍丘,宋皇瑗圍鄭師,每日遷舍,壘合,鄭師哭,子姚救之,大敗,二月,甲戌,宋取鄭師于雍丘,使有能者無死,以郟張與鄭羅歸。

夏,楚人伐陳,陳即吳故也。

宋公伐鄭。

秋,吳城邗,溝通江淮。

晉趙鞅卜救鄭,遇水適火,占諸史趙,史墨,史龜,史龜曰,是謂沈陽,可以興兵,利以伐姜,不利子商,伐齊則可,敵宋不吉,史墨曰,盈,水名也,子,水位也,名位敵,不可干也,炎帝為火師,姜姓其後也,水勝火,伐姜則可,史趙曰,是謂如川之滿,不可游也,鄭方有罪,不可救也,救鄭則不吉,不知其他,陽虎以周易筮之,遇泰之需曰,宋方吉不可與也,微子啟,帝乙之,元子也,宋,鄭,甥舅也,祉,祿也,若帝乙之元子,歸妹而有吉祿,我安得吉焉,乃止。

冬,吳子使,來儆師伐齊。

十年,春,王二月,邾子益來奔。

公會吳伐齊。

三月,戊戌,齊侯陽生卒。

夏,宋人伐鄭。

晉趙鞅帥師侵齊。

五月,公至自伐齊。

葬齊悼公。

衛公孟彄自齊歸于衛。

薛伯夷卒。

秋,葬薛惠公。

冬,楚公子結帥師伐陳。

吳救陳。

十年,春,邾隱公來奔,齊甥也,故遂奔齊,公會吳子,邾子,郯子,伐齊南鄙,師于鄎。

齊人弒悼公,赴于師,吳子三日哭于軍門之外,徐承帥舟師,將自海入齊,齊人敗之,吳師乃還。

夏趙鞅帥師伐齊,大夫請卜之,趙孟曰,吾卜於此起兵,事不再令,卜不襲吉,行也,於是乎取犁及轅,毀高唐之郭,侵及賴而還。

秋,吳子使來復儆師。

冬,楚子期伐陳,吳延州來季子救陳,謂子期曰,二君不務德,而力爭諸侯,民何罪焉,我請退,以為子名務德而安,民乃還。

十有一年,春,齊國書帥師伐我。

夏,陳轅頗出奔鄭。

五月,公會吳伐齊。

甲戌,齊國書帥師及吳戰于艾陵,齊師敗績獲齊國書。

秋,七月,辛酉,滕子虞母卒。

秋,十有一月,葬滕隱公。

衛世叔齊出奔宋。

十一年,春,齊為鄎故,國書,高無平,帥師伐我,及清,季孫謂其宰冉求,曰,齊師在清,必魯故也,若之何,求曰,一子守,二子從,公禦諸竟,季孫曰,不能,求曰,居封疆之間,季孫告二子,二子不可,求曰,若不可,則君無出,一子帥師,背城而戰,不屬者,非魯人也,魯之群室,眾於齊之兵車,一室敵車,優矣,子何患焉,二子之不欲戰也,宜政在季氏,當子之身,齊人伐魯,而不能戰,子之恥也,大不列於諸侯矣,季孫使從於朝,俟於黨氏之溝,武叔呼而問戰焉,對曰,君子有遠慮,小人何知,懿子強問之,對曰,小人慮材而言,量力而共者也,武叔曰,是謂我不成丈夫也,退而蒐乘,孟孺子洩帥右師,顏羽御,邴洩為右,冉求帥左師,管周父御,樊遲為右,季孫曰,須也弱,有子曰,就用命焉,季孫之甲七千,冉有以武城人三百,為己徒卒,老幼守宮,次于雩門之外,五日,右師從之,公叔務人見保者而泣曰,事充政重,上不能謀,士不能死,何以治民,吾既言之矣,敢不勉乎,師及齊師戰于郊,齊師自稷曲,師不踰溝,樊遲曰,非不能也,不信子也,請三刻而踰之,如之,眾從之,師入齊軍,右師奔,齊人從之,陳瓘,陳莊,涉泗,孟之側後入,以為殿,抽矢策其馬曰,馬不進也,林不狃之伍曰,走乎,不狃曰,誰不如,曰,然則止乎,不狃曰,惡賢,徐步而死,師獲甲首八十,齊人不能師,宵諜曰,齊人遁,冉有請從之,三季孫弗許,孟孺子語人曰,我不如顏羽,而賢於邴洩,子羽銳敏,我不欲戰而能默,洩曰,驅之,公為與其嬖僮汪錡乘,皆死皆殯,孔子曰,能執干戈以衛社稷,可無殤也,冉有用矛於齊師,故能入其軍,孔子曰,義也。

夏,陳轅頗出奔鄭,初轅頗為司徒,賦封田,以嫁公女,有餘,以為己大器,國人逐之,故出道渴,其族轅咺,進稻醴,粱糗,腶脯焉,喜曰,何其給也,對曰,器成而具,曰,何不吾諫,對曰,懼先行。

為郊戰,故公會吳子伐齊,五月,克博,壬申,至于嬴,中軍從王,胥門巢將上軍,王子姑曹將下軍,展如將右軍,齊國書將中軍,高無平將上軍,宗樓將下軍,陳僖子謂其弟書,爾死,我必得志,宗子陽與閭丘明相厲也,桑掩胥御國子,公孫夏曰,二子必死,將戰,公孫夏命其徒歌虞殯,陳子行命其徒具含玉,公孫揮命其徒曰,人尋約,吳髮短,東郭書曰,三戰必死,於此三矣,使問弦多以琴,曰,吾不復見子矣,陳書曰,此行也,吾聞鼓而已,不聞金矣,甲戌,戰于艾陵,展如敗高子,國子敗胥門巢,王卒助之,大敗齊師,獲國書,公孫夏,閭丘明,陳書,東郭書,革車八百乘,甲首三千,以獻于公,將戰,吳子呼叔孫曰,而事何也,對曰,從司馬王賜之甲劍鈹,曰,奉爾君事,敬無廢命,叔孫未能對,衛賜進曰,州仇奉甲從君而拜,公使大史固,歸國子之元,寘之新篋,褽之以玄纁,加組帶焉,寘書于其上,曰,天若不識不衷,何以使下國。

吳將伐齊,越子率其眾以朝焉,王及列士,皆有饋賂,吳人皆喜,唯子胥懼曰,是豢吳也夫,諫曰,越在我,心腹之疾也,壤地同面有欲於我,夫其柔服,求濟其欲也,不如早從事焉,得志於齊,猶獲石田也,無所用之,越不為沼,吳其泯矣,使醫除疾,而曰必遺類焉者,未之有也,盤庚之誥曰,其有顛越不共,則劓殄無遺育,無俾易種于茲邑,是商所以興也,今君易之,將以求大,不亦難乎,弗聽,使於齊,屬其子於鮑氏,為王孫氏,反役,王聞之,使賜之屬鏤以死,將死,曰,樹吾墓檟,檟可材也,吳其亡乎,三年,其始弱矣,盈必毀,天之道也。

秋,季孫命脩守備,曰,小勝大,禍也,齊至無日矣。

冬,衛大叔疾出奔宋,初疾娶于宋子朝,其娣嬖,子朝出,孔文子使疾出其妻而妻之,疾使侍人誘其初妻之娣,寘於犁,而為之一宮,如二妻,文子怒,欲攻之,仲尼止之,遂奪其妻,或淫于外州,外州人奪之軒以獻,恥是二者,故出,衛人立遺,使室孔姞,疾臣向魋,納美珠焉,與之城鉏,宋公求珠,魋不與,由是得罪,及桓氏出,城鉏人攻大叔疾,衛莊公復之,使處巢,死焉,殯於鄖,葬於少禘,初,晉悼公子憖亡在衛,使其女僕而田,大叔懿子止而飲之酒,遂聘之,生悼子,悼子即位,故夏戊為大夫,悼子亡,衛人翦夏戊,孔文子之將攻大叔也,訪於仲尼。仲尼曰,胡簋之事,則嘗學之矣,甲兵之事,未之聞也。退命駕而行,曰,鳥則擇木,木豈能擇鳥,文子遽止之,曰,圉豈敢度其私,訪衛國之難也,將止,魯人以幣召之,乃歸,季孫欲以田賦,使冉有訪諸仲尼,仲尼曰,丘,不識也,三發,卒曰,子為國老,待子而行,若之何子之不言也。仲尼不對,而私於冉有曰,君子之行也。度於禮,施取其厚,事舉其中,斂從其薄,如是則以丘亦足矣,若不度於禮,而貪冒無厭,則雖以田賦,將又不足,且子季孫若欲行而法,則周公之典在,若欲苟而行,又何訪焉,弗聽。

十有二年,春,用田賦。

夏,五月,甲辰,孟子卒。

公會吳于橐皋。

秋,公會衛侯,宋皇瑗,于鄖。

宋向巢帥師伐鄭。

冬,十有二月,螽。

十二年,春,王正月,用田賦。

夏,五月,昭夫人孟子卒,昭公娶于吳,故不書姓,死不赴,故不稱夫人,不反哭,故言不葬小君,孔子與弔,適季氏,季氏不絻,放絰而拜。

公會吳于橐皋,吳子使大宰嚭請尋盟,公不欲使,子貢對曰盟,所以周信也,故心以制之,玉帛以奉之,言以結之,明神以要之,寡君以為苟有盟焉,弗可改也巳,若猶可改,日盟何益,今吾子曰,必尋盟,若可尋也,亦可寒也,乃不尋盟。

吳徵會于衛,初,衛人殺吳行人且姚而懼,謀於行人子羽,子羽曰,吳方無道,無乃辱吾君,不如止也,子木曰,吳方無道,國無道,必棄疾於人,吳雖無道,猶足以患衛,往也,長木之,斃無不摽也,國狗之瘈,無不噬也,而況大國乎,秋,衛侯會吳于鄖,公及衛侯,宋皇瑗盟,而卒辭吳盟,吳人藩衛侯之舍,子服景伯謂子貢曰,夫諸侯之會,事既畢矣,侯伯致禮地主歸餼,以相辭也。今吳不行禮於衛,而藩其君舍以難之。子盍見大宰,乃請束錦以行,語及衛故,大宰嚭曰,寡君願事衛君,衛君之來也緩,寡君懼,故將止之,子貢曰,衛君之來,必謀於其眾,其眾或欲或否,是以緩來,其欲來者,子之黨也,其不欲來者,子之讎也,若執衛君,是墮黨而崇讎也,夫墮子者,得其志矣,且合諸侯而執衛君,誰敢不懼,墮黨崇讎,而懼諸侯,或者難以霸乎,大宰嚭說,乃舍衛侯,衛侯歸,效夷言,子之尚幼,曰,君必不免,其死於夷乎,執焉,而又說其言,從之固矣。

冬,十二月,螽,季孫問諸仲尼,仲尼曰,丘聞之,火伏而後蟄者畢,今火猶西流,司厤過也。

宋鄭之間,有隙地焉,曰,彌作,頃丘,玉暢,喦戈,鍚,子產與宋人為成曰,勿有是,及宋平元之族,自蕭奔鄭,鄭人為之城喦戈鍚,九月,宋向巢伐鄭,取鍚,殺元公之孫,遂圍喦,十二月,鄭罕達救喦,丙申,圍宋師。

十有三年,春,鄭罕達帥師取宋師于喦。

夏,許男成卒。

公會晉侯及吳子于黃池。

楚公子申帥師伐陳。

於越入吳。

秋,公至自會。

晉魏曼多帥師侵衛。

葬許元公。

九月,螽。

冬,十有一月,有星孛于東方。

盜殺陳夏區夫。

十有二月,螽。

十三年,春,宋向魋救其師,鄭子賸使徇曰,得桓魋者有賞,魋也逃歸,遂取宋師于喦,獲成讙,郜延,以六邑為虛。

夏,公會單平公,晉定公,吳夫差于黃池。

六月,丙子,越子伐吳,為二隧。疇無餘,謳陽,自南方先及郊,吳大子友,王子地,王孫彌庸,壽於姚,自泓上觀之,彌庸見姑蔑之旗,曰,吾父之旗也,不可以見讎而弗殺也,大子曰,戰而不克,將亡國,請待之,彌庸不可,屬徒五千,王子地助之,乙酉,戰,彌庸獲疇無餘,地獲謳陽,越子至,王子地守,丙戌,復戰,大敗吳師,獲大子友,王孫彌庸,壽於姚,丁亥,入吳,吳人告敗于王,王惡其聞也,自剄七人於幕下。

秋,七月,辛丑,盟,吳晉爭先,吳人曰,於周室,我為長,晉人曰,於姬姓,我為伯,趙鞅呼司馬寅曰,日旰矣,大事未成,二臣之罪也,建鼓整列,二臣死之,長幼必可知也,對曰,請姑視之。反曰,肉食者無墨。今吳王有墨,國勝乎,大子死乎,且夷德輕,不忍久,請少待之,乃先晉人,吳人將以公見晉侯,子服景伯對使者曰,王合諸侯,則伯帥侯牧以見於王,伯合諸侯,則侯帥子男以見於伯,自王以下,朝聘玉帛不同,故敝邑之職貢於吳,有豐於晉,無不及焉,以為伯也,今諸侯會,而君將以寡君見晉君,則晉成為伯矣,敝邑將改職貢,魯賦於吳八百乘,若為子男,則將半邾,以屬於吳,而如邾以事晉,且執事以伯召諸侯,而以侯終之,何利之有焉,吳人乃止,既而悔之,將囚景伯,景伯曰,何也,立後於魯矣,將以二乘,與六人從,遲速唯命,遂囚以還,及戶牖,謂大宰曰,魯將以十月上辛,有事於上帝先王,季辛而畢,何,世有職焉,自襄以來,未之改也,若不會,祝宗將曰,吳實然,且謂魯不共,而執其賤者七人,何損焉,大宰嚭言於王曰,無損於魯,而祗為名,不如歸之,乃歸景伯,吳申叔儀,乞糧於公孫有山氏,曰,佩玉繠兮,余無所繫之,旨酒一盛兮,余與褐之父睨之,對曰,梁則無矣,麤則有之,若登首山以呼曰,庚癸乎,則諾。王欲伐宋殺其丈夫,而囚其婦人。大宰嚭曰,可勝也,而弗能居也,乃歸,冬,吳及越平。

十有四年,春,西狩獲麟。

小邾射以句繹來奔。

夏四月,齊陳恆執其君,寘于舒州。

庚戌叔還卒。

五月,庚申朔,日有食之。

陳宗豎出奔楚。

宋向魋入于曹以叛。

莒子狂卒。

六月,宋向魋自曹出奔衛,宋向巢來奔。

齊人弒其君壬于舒州。

秋,晉趙鞅帥師伐衛,八月,辛丑,仲孫何忌卒。

冬,陳宗豎自楚復入于陳,陳人殺之,陳轅買出奔楚。

有星孛。

饑。

十四年,春,西狩于大野,叔孫氏之車子鉏商獲麟,以為不祥,以賜虞人,仲尼觀之,曰,麟也,然後取之。

小邾射以句繹來奔,曰使季路要我,吾無盟矣,使子路,子路辭,季康子使冉有謂之曰,千乘之國,不信其盟,而信子之言,子何辱焉,對曰,魯有事于小邾,不敢問故,死其城下可也,彼不臣而濟其言,是義之也,由弗能。

齊簡公之在魯也,闞止有寵焉,及即位,使為政,陳成子憚之,驟顧諸朝,諸御鞅言於公曰,陳闞不可並也,君其擇焉,弗聽,子我夕,陳逆殺人,逢之,遂執以入,陳氏方睦,使疾,而遺之潘沐,備酒肉焉,饗守囚者,醉而殺之,而逃,子我盟諸陳於陳宗,初,陳豹欲為子我臣,使公孫言已,已有喪而止,既而言之,曰,有陳豹者,長而上僂,望視,事君子必得志,欲為子臣,吾憚其為人也,故緩以告,子我曰,何害,是其在我也,使為臣,他日,與之言政,說遂有寵,謂之曰,我盡逐陳氏,而立女,若何,對曰,我遠於陳氏矣,且其違者,不過數人,何盡逐焉,遂告陳氏,子行曰,彼得君,弗先,必禍子,子行舍於公宮,夏,五月,壬申,成子兄弟,四乘如公,子我在幄,出逆之,遂入,閉門,侍人禦之,子行殺侍人,公與婦人飲酒於檀臺,成子遷諸寢,公執戈,將擊之,大史子餘曰,非不利也,將除害也,成子出舍于庫,聞公猶怒,將出曰,何所無君,子行抽劍曰,需,事之賊也,誰非陳宗,所不殺子者,有如陳宗,乃止,子我歸,屬徒攻闈與大門,皆不勝,乃出,陳氏追之,失道於弇中,適豐丘,豐丘人執之以告,殺諸郭關,成子將殺大陸子方,陳逆請而免之,以公命取車於道,及耏,眾知而東之,出雍門,陳豹與之車,弗受,曰,逆為余請,豹與余車,余有私焉,事子我,而有私於其讎,何以見魯衛之士,東郭賈奔衛,庚辰,陳恆執公于舒州,公曰,吾早從鞅之言,不及此。

宋桓魋之寵,害於公,公使夫人驟請享焉,而將討之,未及,魋先謀公,請以鞍易薄,公曰,不可,薄,宗邑也,乃益鞍七邑,而請享公焉,以日中為期,家備盡往,公知之,告皇野曰,余長魋也,今將禍余,請即救,司馬子仲曰,有臣不順,神之所惡也,而況人乎,敢不承命,不得左師不可,請以君命召之,左師每食擊鍾,聞鍾聲,公曰,夫子將食,既食,又奏,公曰,可矣,以乘車往,曰,跡人來告,曰,逢澤有介麇焉,公曰,雖魋未來,得左師吾與之田,若何,君憚告子,野曰,嘗私焉,君欲速,故以乘車逆子,與之乘,至,公告之故,拜不能起,司馬曰,君與之言。公曰:所難子者,上有天,下有先君。對曰:魋之不共,宋之禍也,敢不唯命是聽,司馬請瑞焉,以命其徒攻桓氏。其父兄故臣曰,不可,其新臣曰,從吾君之命,遂攻之子,頎騁而告桓司馬,司馬欲入,子車止之,曰,不能事君,而又伐國,民不與也,祇取死焉,向魋遂入于曹,以叛,六月,使左師巢伐之,欲質大夫以入焉,不能,亦入于曹取質,魋曰,不可,既不能事君,又得罪于民,將若之何,乃舍之,民遂叛之,向魋奔衛,向巢來奔,宋公使止之,曰寡人與子有言矣,不可以絕向氏之祀,辭曰,臣之罪大,盡滅桓氏可也,若以先臣之故,而使有後,君之惠也,若臣則不可以入矣,司馬牛致其邑與珪焉,而適齊,向魋出於衛地,公文氏攻之,求夏后氏之璜焉,與之他玉,而奔齊,陳成子使為次卿,司馬牛又致其邑焉,而適吳,吳人惡之而反,趙簡子召之,陳成子亦召之,卒於魯郭門之外,阬氏葬諸丘輿。

甲午,齊陳恆弒其君壬于舒州,孔丘三日齊,而請伐齊,三,公曰,魯為齊弱久矣,子之伐之,將若之何,對曰,陳恆弒其君,民之不與者半,以魯之眾,加齊之半,可克也,公曰,子告季孫,孔子辭,退而告人曰,吾以從大夫之後也,故不敢不言。

初,孟孺子洩將圉馬於成,成宰公孫宿不受,曰孟孫為成之病,不圉馬焉,孺子怒,襲成,從者不得入,乃反,成有司使,孺子鞭之,秋,八月,辛丑,孟懿子卒,成人奔喪,弗內,袒免哭于衢,聽共,弗許,懼,不歸。

十有五年,春,王正月,成叛。

夏,五月,齊高無平出奔北燕。

鄭伯伐宋。

秋,八月,大雩。

晉趙鞅帥師伐衛。

冬,晉侯伐鄭。

及齊平。

衛公孟彄出奔齊。

十五年,春,成叛于齊,武伯伐成,不克,遂城輸。

夏,楚子西,子期,伐吳,及桐汭,陳侯使公孫貞子弔焉,及良而卒,將以尸入,吳子使大宰嚭勞,且辭曰,以水潦之不時,無乃廩然隕大夫之尸,以重寡君之憂,寡君敢辭上介,芋尹蓋對曰,寡君聞楚為不道,荐伐吳國,滅厥民人,寡君使蓋備使,弔君之下吏,無祿,使人逢天之慼,大命隕隊,絕世于良,廢日共積,一日遷次,今君命逆使人曰,無以尸造于門,是我寡君之命委于草莽也,且臣聞之,曰事死如事生,禮也,於是乎有朝聘而終,以尸將事之禮,又有朝聘而遭喪之禮,若不以尸將命,是遭喪而還也,無乃不可乎,以禮防民,猶或踰之,今大夫曰,死而棄之,是棄禮也,其何以為諸侯主,先民有言曰,無穢虐士,備使奉尸將命,苟我寡君之命,達于君所,雖隕于深淵,則天命也,非君與涉人之過也,吳人內之。

秋,齊陳瓘如楚,過衛,仲由見之,曰,天或者以陳氏為斧斤,既斲喪公室,而他人有之,不可知也,其使終饗之,亦不可知也,若善魯以待時,不亦可乎,何必惡焉,子玉曰,然,吾受命矣,子使告我弟。

冬,及齊平,子服景伯如齊,子贛為介,見公孫成,曰人皆臣人,而有背人之心,況齊人雖為子役,其有不貳乎,子,周公之孫也,多饗大利,猶思不義,利不可得,而喪宗國,將焉用之,成曰,善哉,吾不早聞命,陳成子館客,曰,寡君使恆告曰,寡君願事君如事衛君,景伯揖子贛而進之,對曰,寡君之願也,昔晉人伐衛,齊為衛故,伐晉冠氏,喪車五百,因與衛地,自濟以西,禚媚杏,以南,書社五百,吳人加敝邑以亂,齊因其病,取讙與闡,寡君是以寒心,若得視衛君之事君也,則固所願也,成子病之,乃歸成,公孫宿以其兵甲入于嬴。

衛孔圉取大子蒯聵之姊,生悝,孔氏之豎渾良夫,長而美,孔文子卒,通於內,大子在戚,孔姬使之焉,大子與之言曰,苟使我入獲國,服冕乘軒,三死無與,與之盟,為請於伯姬,閏月,良夫與大子入,舍於孔氏之外圃,昏,二人蒙衣而乘,寺人羅御,如孔氏,孔氏之老欒寧問之,稱姻妾以告,遂入,適伯姬氏,既食,孔伯姬杖戈而先,大子與五人介,輿豭從之,迫孔悝於廁強盟之,遂劫以登臺,欒寧將飲酒,炙未熟,聞亂,使告季子,召獲駕乘車,行爵食炙,奉衛侯輒來奔,季子將入,遇子羔將出,曰,門已閉矣,季子曰,吾姑至焉,子羔曰,弗及,不踐其難,季子曰,食焉,不辟其難,子羔遂出,子路入,及門,公孫敢門焉,曰,無入為也,季子曰,是公孫也,求利焉而逃其難,由,不然,利其祿,必救其患,有使者出,乃入,曰大子焉用孔悝,雖殺之,必或繼之,且曰,大子無勇,若燔臺半,必舍孔叔大子聞之懼,下石乞,盂黶敵子路,以戈擊之,斷纓,子路曰,君子死,冠不免,結纓而死,孔子聞衛亂,曰,柴也,其來由也死矣,孔悝立莊公,莊公害故政,欲盡去之,先謂司徒瞞成曰,寡人離病於外久矣,子請亦嘗之,歸告褚師比,欲與之伐公,不果。

十有六年,春,王正月,己卯,衛世子蒯聵自戚入于衛,衛侯輒來奔。

二月,衛子還成出奔宋。

夏,四月,己丑,孔丘卒。

十六年,春,瞞成,褚師比,出奔宋,衛侯使鄢武子告于周,曰蒯聵得罪于君父君母,逋竄于晉,晉以王室之故,不棄兄弟,寘諸河上,天誘其衷,獲嗣守封焉,使下臣肸,敢告執事,王使單平公對曰,肸以嘉命,來告余一人,往謂叔父,余嘉乃成世。復爾祿次,敬之哉,方天之休,弗敬弗休,悔其可追。

夏,四月,己丑,孔丘卒,公誄之曰,旻天不弔,不憖遺一老,俾屏余一人以在位,煢煢余在疚,嗚呼,哀哉,尼父無自律,子贛曰,君其不沒於魯乎,夫子之言曰,禮失則昏,名失則愆,失志為昏,失所為愆,生不能用,死而誄之,非禮也,稱一人,非名也,君兩失之。

六月,衛侯飲孔悝酒於平陽,重酬之,大夫皆有納焉,醉而送之,夜半而遣之,載伯姬於平陽而行,及西門,使貳車反祏於西圃,子伯季子,初為孔氏臣,新登于公,請追之,遇載祏者,殺而乘其車,許公為,反祏,遇之曰,與不仁人爭,明無不勝,必使先射,射三發,皆遠許為,許為射之,殪,或以其車從,得祏於橐中,孔悝出奔宋,楚大子建之遇讒也,自城父奔宋,又辟華氏之亂於鄭,鄭人甚善之,又適晉,與晉人謀襲鄭,乃求復焉,鄭人復之如初,晉人使諜於子木,請行而期焉,子木暴虐於其私邑,邑人訴之,鄭人省之,得晉諜焉,遂殺子木,其子曰勝,在吳,子西欲召之,葉公曰,吾聞勝也,詐而亂,無乃害乎,子西曰,吾聞勝也,信而勇,不為不利,舍諸邊竟,使衛藩焉,葉公曰,周仁之謂信,率義之謂勇,吾聞勝也,好復言,而求死士,殆有私乎,復言非信也,期死非勇也,子必悔之,弗從,召之使處吳竟,為白公,請伐鄭,子西曰,楚未節也,不然,吾不忘也,他日又請,許之,未起師,晉人伐鄭,楚救之,與之盟,勝怒曰,鄭人在此,讎不遠矣,勝自厲劍,子期之子平見之,曰,王孫何自厲也,曰,勝以直聞,不告女,庸為直乎,將以殺爾父,平以告子西,子西曰,勝如卵,余翼而長之,楚國第,我死,令尹司馬,非勝而誰,勝聞之曰,令尹之狂也,得死乃非我,子西不悛,勝謂石乞,曰王與二卿士,皆五百人當之,則可矣,乞曰,不可得也,曰,市南有熊宜僚者,若得之,可以當五百人矣,乃從白公而見之,與之言,說,告之故,辭,承之以劍,不動,勝曰,不為利諂,不為威惕,不洩人言,以求媚者,去之,吳人伐慎,白公敗之,請以戰備獻,許之,遂作亂,秋,七月,殺子西,子期,于朝,而劫惠王,子西以袂掩面而死,子期曰,昔者吾以力事君,不可以弗終,抉豫章以殺人,而後死,石乞曰,焚庫弒王,不然不濟,白公曰,不可,殺王不祥,焚庫無聚,將何以守矣,乞曰,有楚國而治其民,以敬事神,可以得祥,且有聚矣,何患弗從,葉公在蔡,方城之外皆曰,可以入矣,子高曰,吾聞之,以險徼幸者,其求無饜,偏重必離,聞其殺齊管脩也,而後入,白公欲以子閭為王,子閭不可,遂劫以兵,子閭曰,王孫若安靖楚國,匡正王室,而後庇焉,啟之願也,敢不聽從,若將專利,以傾王室,不顧楚國,有死不能,遂殺之,而以王如高府,石乞尹門,圉公陽穴宮,負王以如昭夫人之宮,葉公亦至,及北門,或遇之曰,君胡不冑,國人望君,如望慈父母焉,盜賊之矢若傷君,是絕民望也,若之何不冑,乃冑而進,又遇一人曰,君胡冑,國人望君,如望歲焉,日日以幾,若見君面,是得艾也,民知不死,其亦夫有奮心,猶將旌君以徇於國,而反掩面以絕民望,不亦甚乎,乃免冑而進,遇箴尹固,帥其屬將與白公,子高曰,微二子者,楚不國矣,棄德從賊,其可保乎,乃從葉公,使與國人以攻白公,白公奔山而縊,其徒微之生拘石乞,而問白公之死焉,對曰,余知其死所,而長者使余勿言,曰,不言將烹,乞曰,此事克則為卿,不克則烹,固其所也,何害,乃烹石乞,王孫燕奔頯黃氏,諸梁兼二事,國寧,乃使寧為令尹,使寬為司馬,而老於葉。

衛侯占夢嬖人,求酒於大叔僖子,不得,與卜人比,而告公曰,君有大臣在西南隅,弗去,懼害,乃逐大叔遺,遺奔晉。

衛侯謂渾良夫曰,吾繼先君,而不得其器,若之何,良夫代執火者而言,曰,疾與亡君,皆君之子也,召之,而擇材焉,可也,若不材,器可得也,豎告大子,大子使五人輿豭從己,劫公而強盟之,且請殺良夫,公曰,其盟免三死,曰,請三之後,有罪殺之,公曰,諾哉。

十七年,春,衛侯為虎幄於藉圃,成求令名者,而與之始食焉,大子請使良夫,良夫乘衷甸,兩牡,紫衣狐裘,至,袒裘不釋劍而食,大子使牽以退,數之以三罪,而殺之。

三月,越子伐吳,吳子禦之笠澤,夾水而陳,越子為左右句卒,使夜或左或右,鼓譟而進,吳師分以御之,越子以三軍潛涉,當吳中軍而鼓之,吳師大亂,遂敗之。

晉趙鞅使告于衛曰,君之在晉也,志父為主,請君若大子來,以免志父,不然,寡君其曰,志父之為也,衛侯辭以難,大子又使椓之,夏,六月,趙鞅圍衛,齊國觀,陳瓘,救衛,得晉人之致師者,子玉使服而見之,曰,國子實執齊柄,而命瓘曰,無辟晉師,豈敢廢命,子又何辱,簡子曰,我卜伐衛,未卜與齊戰,乃還。

楚白公之亂,陳人恃其聚而侵楚,楚既寧,將取陳麥,楚子問帥於大師子穀,與葉公諸梁,子穀曰,右領差車,與左史老,皆相令尹司馬以伐陳,其可使也,子高曰,率賤,民慢之,懼不用命焉,子穀曰,觀丁父,鄀俘也,武王以為軍率,是以克州蓼,服隨唐,大啟群蠻,彭仲爽,申俘也,文王以為令尹,實縣申息,朝陳蔡,封畛於汝,唯其任也,何賤之有,子高曰,天命不諂,令尹有憾於陳,天若亡之,其必令尹之子是與,君盍舍焉,臣懼右領與左史,有二俘之賤,而無其令德也,王卜之,武城尹吉,使帥師取陳麥,陳人御之,敗,遂圍陳,秋,七月,己卯,楚公孫朝帥師滅陳,王與葉公枚卜子良,以為令尹,沈尹朱曰,吉,過於其志,葉公曰,王子而相國,過將何為,他日改卜子國,而使為令尹。

衛侯夢于北宮,見人登昆吾之觀,被髮北面而譟曰,登此昆吾之虛,綿綿生之瓜,余為渾良夫,叫天無辜,公親筮之,胥彌赦占之,曰,不害,與之邑,寘之,而逃奔宋,衛侯貞卜,其繇曰,如魚竀尾,衡流而方羊。裔焉大國,滅之,將亡。闔門塞竇,乃自後踰,冬,十月,晉復伐衛,入其郛,將入城,簡子曰,止,叔向有言曰,怙亂滅國者無後,衛人出莊公,而與晉平,晉立襄公之孫般師而還,十一月,衛侯自鄄入,般師出,初,公登城以望,見戎州,問之,以告,公曰,我姬姓也,何戎之有焉,翦之,公使匠久,公欲逐石圃,未及而難作,辛巳,石圃因匠氏攻公,公闔門而請,弗許,踰于北方而隊,折股,戎州人攻之,大子疾,公子青,踰從公,戎州人殺之,公入于戎州己氏,初,公自城上,見己氏之妻髮美,使髡之,以為呂姜髢,既入焉,而示之璧,曰,活我,吾與女璧,己氏曰,殺女,璧其焉往,遂殺之,而取其璧,衛人復公孫般師而立之,十二月,齊人伐衛,衛人請平,立公子起,執般師以歸,舍諸潞。

公會齊侯盟于蒙,孟武伯相,齊侯稽首,公拜,齊人怒,武伯曰,非天子,寡君無所稽首,武伯問於高柴曰,諸侯盟,誰執牛耳,季羔曰,鄫衍之役,吳公子姑曹,發陽之役,衛石魋,武伯曰,然則彘也。

宋皇瑗之子麇,有友曰田丙,而奪其兄劖般邑,以與之,劖般慍而行,告桓司馬之臣子儀克,子儀克適宋,告夫人曰,麇將納桓氏,公問諸子仲,初,子仲將以杞姒之子,非我為子,麇曰,必立伯也,是良材,子仲怒,弗從,故對曰,右師則老矣,不識麇也,公執之,皇瑗奔晉,召之。

十八年,春,宋殺皇瑗,公聞其情,復皇氏之族,使皇緩為右師。

巴人伐楚,圍鄾,初,右司馬子國之卜也,觀瞻曰,如志,故命之,及巴師至,將卜師,王曰,寧如志,何卜焉,使帥師而行,請承,王曰,寢尹工尹,勤先君者也,三月,楚公孫寧,吳由于,薳固,敗巴師于鄾,故封子國於析,君子曰,惠王知志,夏書曰,官占,唯能蔽志,昆命于元龜,其是之謂乎,志曰,聖人不煩卜筮,惠王其有焉。

夏,衛石圃逐其君起,起奔齊,衛侯輒自齊復歸,逐石圃,而復石魋,與大叔遺。

十九年,春,越人侵楚,以誤吳也。

夏,楚公子慶,公孫寬,追越師,至冥,不及,乃還。

秋,楚沈諸梁伐東夷,三夷男女,及楚師盟于敖。

冬,叔青如京師,敬王崩故也。

二十年,春,齊人來徵會,夏,會于廩丘,為鄭故,謀伐晉,鄭人辭諸侯,秋,師還。

吳公子慶忌驟諫吳子曰,不改必亡,弗聽,出居于艾,遂適楚,聞越將伐吳,冬,請歸平越,遂歸欲除不忠者以說于越,吳人殺之。

十一月,越圍吳,趙孟降於喪食,楚隆曰,三年之喪,親暱之極也,主又降之,無乃有故乎,趙孟曰,黃池之役,先主與吳王有質,曰,好惡同之,今越圍吳,嗣子不廢舊業,而敵之,非晉之所能及也,吾是以為降,楚隆曰,若使吳王知之,若何,趙孟曰,可乎,隆曰,請嘗之,乃往,先造于越軍曰,吳犯間上國多矣,聞君親討焉,諸夏之人,莫不欣喜,唯恐君志之不從,請入視之,許之,告于吳王曰,寡人之老無恤,使陪臣隆,敢展謝其不共,黃池之役,君之先臣志父,得承齊盟,曰,好惡同之,今君在難,無恤不敢憚勞,非晉國之所能及也,使陪臣敢展布之,王拜稽首曰,寡人不佞,不能事越,以為大夫憂,拜命之辱,與之一簞珠,使問趙孟,曰,句踐將生憂寡人,寡人死之不得矣,王曰,溺人必笑,吾將有問也,史黯何以得為君子,對曰,黯也,進不見惡,退無謗言,王曰,宜哉。

二十一年,夏,五月,越人始來。

秋,八月,公及齊侯,邾子,盟于顧,齊人責稽首,因歌之,曰,魯人之皋,數年不覺,使我高蹈,唯其儒書,以為二國憂,是行也,公先至于陽穀,齊閭丘息曰,君辱舉玉趾,以在寡君之軍,群臣將傳遽以告寡君,比其復也,君無乃勤,為僕人之未次,請除館於舟道,辭曰,敢勤僕人。

二十二年,夏,四月,邾隱公自齊奔越,曰,吳為無道,執父立子,越人歸之,大子革奔越。

冬,十一月,丁卯,越滅吳,請使吳王居甬東,辭曰,孤老矣,焉能事君,乃縊,越人以歸。

二十三年,春,宋景曹卒,季康子使冉有弔,且送葬,曰,敝邑有社稷之事,使肥與有職競焉,是以不得助執紼,使求從輿人,曰,以肥之得備彌甥也,有不腆先人之產馬,使求薦諸夫人之宰,其可以稱旌繁乎。

夏,六月,晉荀瑤伐齊,高無平帥師御之,知伯視齊師,馬駭遂驅之,曰,齊人知余旗,其謂余畏而反也,及壘而還,將戰,長武子請卜,知伯曰,君告於天子,而卜之以守龜於宗祧,吉矣,吾又何卜焉,且齊人取我英丘,君命瑤,非敢耀武也,治英丘也,以辭伐罪足矣,何必卜,壬辰,戰于犁丘,齊師敗績,知伯親禽顏庚。

秋,八月,叔青如越,始使越也,越諸鞅來聘,報叔青也。

二十四年,夏,四月,晉侯將伐齊,使來乞師,曰,昔臧文仲以楚師伐齊取穀,宣叔以晉師伐齊,取汶陽,寡君欲徼福於周公,願乞靈於臧氏,臧石帥師會之,取廩丘,軍吏令繕,將進,萊章曰,君卑政暴,往歲克敵,今又勝都,天奉多矣,又焉能進,是躗言也,役將班矣,晉師乃還,餼臧石牛,大史謝之,曰,以寡君之在行,牢禮不度,敢展謝之。

邾子又無道,越人執之以歸,而立公子何,何亦無道。

公子荊之母嬖,將以為夫人,使宗人釁夏獻其禮,對曰,無之,公怒曰,女為宗司,立夫人,國之大禮也,何故無之,對曰,周公及武公娶於薛,孝,惠,娶於商,自桓以下娶於齊,此禮也,則有若以妾為夫人,則固無其禮也,公卒立之,而以荊為大子,國人始惡之。

閏月,公如越,得大子適郢,將妻公,而多與之地,公孫有山,使告于季孫,季孫懼,使因大宰嚭,而納賂焉,乃止。

二十五年,夏,五月,庚辰,衛侯出奔宋,衛侯為靈臺于藉圃,與諸大夫飲酒焉,褚師聲子韤而登席,公怒,辭曰,臣有疾異於人,若見之,君將嘔之,是以不敢,公愈怒,大夫辭之,不可,褚師出,公戟其手,曰,必斷而足,聞之,褚師與司寇亥乘,曰,今日幸而後亡,公之入也,奪南氏邑,而奪司寇亥政,公使侍人納公文懿子之車于池,初,衛人翦夏丁氏,以其帑賜彭封彌子,彌子飲公酒,納夏戊之女,嬖,以為夫人,其弟期大叔疾之從孫甥也,少畜於公,以為司徒,夫人寵衰,期得罪公使三匠久,公使優狡盟拳彌,而甚近信之,故褚師比,公孫彌牟,公文要,司寇亥,司徒期,因三匠與拳彌以作亂,皆執利兵,無者執斤,使拳彌入于公宮,而自大子疾之宮,譟以攻公,鄄子士請禦之,彌援其手曰,子則勇矣,將若君何,不見先君乎,君何所不逞欲,且君嘗在外矣,豈必不反,當今不可,眾怒難犯,休而易間也,乃出,將適蒲,彌曰,晉無信,不可,將適鄄,彌曰,齊晉爭我,不可,將適泠,彌曰,魯不足與,請適城鉏,以鉤越,越有君,乃適城鉏,彌曰,衛盜不可知也,請速,自我始,乃載寶以歸,公為支離之卒,因祝史揮以侵衛,衛人病之,懿子知之,見子之,請逐揮,文子曰,無罪,懿子曰,彼好專利而妄,夫見君之入也,將先道焉,若逐之,必出於南門,而適君所,夫越新得諸侯,將必請師焉,揮在朝,使吏遣諸其室,揮出,信弗內,五日,乃館諸外里,遂有寵,使如越請師。

六月,公至自越,季康子,孟武伯,逆於五梧,郭重僕,見二子曰,惡言多矣,君請盡之,公宴於五梧,武伯為祝,惡郭重曰,何肥也,季孫曰,請飲彘也,以魯國之密邇仇讎,臣是以不獲從君,克免於大行,又謂重也肥,公曰,是食言多矣,能無肥乎,飲酒不樂,公與大夫始有惡。

二十六年,夏,五月,叔孫舒帥師會越皋如,后庸,宋樂茷,納衛侯,文子欲納之,懿子曰,君愎而虐,少待之,必毒於民,乃睦於子矣,師侵外州,大獲,出禦之,大敗,掘褚師定子之墓,焚之于平莊之上,文子使王孫齊私於皋如,曰,子將大滅衛乎,抑納君而已乎,皋如曰,寡君之命無他,納衛君而已,文子致眾而問焉,曰,君以蠻夷伐國,國幾亡矣,請納之,眾曰,勿納,曰,彌牟亡而有益,請自北門出,眾曰,勿出,重賂越人,申開守陴而納公,公不敢入,師還,立悼公,南氏相之,以城鉏與越人,公曰,期則為此,令苟有怨於夫人者報之,司徒期聘於越,公攻而奪之幣,期告王,王命取之,期以眾取之,公怒,殺期之甥之為大子者,遂卒于越。

宋景公無子,取公孫周之子得,與啟,畜諸公宮,未有立焉,於是皇緩為右師,皇非我為大司馬,皇懷為司徒,靈不緩為左師,樂茷為司城,樂朱鉏為大司寇,六卿三族降聽政,因大尹以達,大尹常不告,而以其欲,稱君命以令,國人惡之,司城欲去大尹,左師曰,縱之,使盈其罪,重而無基,能無敝乎,冬十月,公游于空澤,辛巳,卒于連中,大尹興空澤之士千甲,奉公自空桐入,如沃宮,使召六子曰,聞下有師,君請六子畫,六子至,以甲劫之,曰君有疾病,請二三子盟,乃盟于少寢之庭,曰無為公室不利,大尹立啟,奉喪殯于大宮,三日而後國人知之,司城茷使宣言于國曰,大尹惑蠱其君而專其利,令君無疾而死,死又匿之,是無他矣,大尹之罪也,得夢啟,北首而寢於廬門之外,已為鳥而集於其上,咮加於南門,尾加於桐門,曰,余夢美,必立,大尹謀,曰,我不在盟,無乃逐我,復盟之乎,使祝為載書,六子在唐孟,將盟之,祝襄以載書告皇非我,皇非我因子潞,門尹得,左師謀曰,民與我,逐之乎,皆歸授甲,使徇于國曰,大尹惑蠱其君,以陵虐公室,與我者,救君者也,眾曰,與之,大尹徇曰,戴氏,皇氏,將不利公室,與我者,無憂不富,眾曰,無別,戴氏皇氏欲伐公,樂得曰,不可,彼以陵公有罪,我伐公,則甚焉,使國人施于大尹,大尹奉啟以奔楚,乃立得司城為上卿,盟曰,三族共政,無相害也。

衛出公自城鉏,使以弓問子贛,且曰,吾其入乎,子贛稽首受弓,對曰,臣不識也,私於使者曰,昔成公孫於陳,甯武子,孫莊子,為宛濮之盟,而君入,獻公孫於衛齊,子鮮,子展,為夷儀之盟,而君入,今君再在孫矣,內不聞獻之親,外不聞成之卿,則賜不識所由入也,詩曰,無競惟人,四方其順之,若得其人,四方以為主,而國於何有。

二十七年,春,越子使后庸來聘,且言邾田,封于駘上,二月,盟于平陽,三子皆從,康子病之,言及子贛,曰,若在此,吾不及此夫,武伯曰,然何不召,曰,固將召之,文子曰,他日請念。

夏,四月,己亥,季康子卒,公弔焉,降禮。

晉荀瑤帥師伐鄭,次于桐丘,鄭駟弘請救于齊,齊師將興,陳子屬孤子,三日朝,設乘車兩馬,繫五邑焉,召顏涿聚之子晉,曰,隰之役,而父死焉,以國之多難,未女恤也,今君命女以是邑也,服車而朝,毋廢前勞,乃救鄭,及留舒,違穀七里,穀人不知,及濮,雨不涉,子思曰,大國在敝邑之宇下,是以告急,今師不行,恐無及也,成子衣製杖戈,立於阪上,馬不出者,助之鞭之,知伯聞之,乃還,曰,我卜伐鄭,不卜敵齊,使謂成子曰,大夫陳子,陳之自出,陳之不祀,鄭之罪也,故寡君使瑤察陳衷焉,謂大夫其恤陳乎,若利本之顛,瑤何有焉,成子怒曰,多陵人者皆不在,知伯其能久乎,中行文子告成子,曰,有自晉師告寅者,將為輕車千乘,以厭齊師之門,則可盡也,成子曰,寡君命恆曰,無及寡,無畏眾,雖過千乘,敢辟之乎,將以子之命告寡君,文子曰,吾乃今知所以亡,君子之謀也,始衷終皆舉之,而後入焉,今我三不知而入之,不亦難乎,公患三桓之侈也,欲以諸侯去之,三桓亦患公之妄也,故君臣多間,公游于陵阪,遇孟武伯於孟氏之衢,曰,請有問於子,余及死乎,對曰,臣無由知之,三問,卒辭不對,公欲以越伐魯,而去三桓,秋,八月,甲戌,公如公孫有陘氏,因孫於邾,乃遂如越,國人施公孫有山氏。

悼之四年,晉荀瑤帥師圍鄭,未至鄭駟弘曰,知伯愎而好勝,早下之,則可行也,乃先保南里以待之,知伯入南里,門于桔柣之門,鄭人俘酅魁壘,賂之以知政,閉其口而死,將門,知伯謂趙孟,入之,對曰,主在此,知伯曰,惡而無勇,何以為子,對曰,以能忍恥,庶無害趙宗乎知伯不悛,趙襄子由是惎知伯,遂喪之,知伯貪而愎,故韓魏反而喪之。


\end{pinyinscope}