\article{襄公}

\begin{pinyinscope}
元年,春,王正月,公即位。

仲孫蔑會晉欒黶,宋華元,衛甯殖,曹人,莒人,邾人,滕人,宋人,薛人,圍宋彭城。

夏,晉韓厥帥師伐鄭。

仲孫蔑會齊崔杼,曹人,邾人,杞人,次于鄫。

秋,楚公子壬夫帥師侵宋。

九月,辛酉,天王崩。

邾子來朝。

冬,衛侯使公孫剽來聘。

晉侯使荀罃來聘。

元年,春,己亥,圍宋彭城,非宋地,追書也,於是為宋討魚石,故稱宋,且不登叛人也,謂之宋志,彭城降晉,晉人以宋五大夫在彭城者歸,寘諸瓠丘,齊人不會彭城,晉人以為討,二月,齊大子光為質於晉。

夏,五月,晉韓厥,荀偃,帥諸侯之師伐鄭,入其郛,敗其徒兵於洧上,於是東諸侯之師,次于鄫以待晉師,晉師自鄭,以鄫之師侵楚焦夷,及陳,晉侯,衛侯,次于戚,以為之援。

秋,楚子辛救鄭,侵宋呂留,鄭子然侵宋,取犬丘。

九月,邾子來朝,禮也。

冬,衛子叔,晉知武子,來聘,禮也,凡諸侯即位,小國朝之,大國聘焉,以繼好結信,謀事補闕,禮之大者也。

二年,春,王正月,葬簡王。

鄭師伐宋。

夏,五月,庚寅,夫人姜氏薨。

六月,庚辰,鄭伯睔卒。

晉師,宋師,衛甯殖,侵鄭。

秋,七月,仲孫蔑會晉荀罃,宋華元,衛孫林父,曹人,邾人,于戚,己丑,葬我小君齊姜。

叔孫豹如宋。

冬,仲孫蔑會晉荀罃,齊崔杼,宋華元,衛孫林父,曹人,邾人,滕人,薛人,小邾人,于戚,遂城虎牢。

楚殺其大夫公子申。

二年,春,鄭師侵宋,楚令也。

齊侯伐萊,萊人使正輿子,賂夙沙衛以索馬牛,皆百匹,齊師乃還,君子是以知齊靈公之為靈也。

夏,齊姜薨,初,穆姜使擇美檟,以自為櫬,與頌琴,季文子取以葬,君子曰,非禮也,禮無所逆,婦養姑者也,虧姑以成婦,逆莫大焉,詩曰,其惟哲人,告之話言,順德之行,季孫於是為不哲矣,且姜氏,君之妣也,詩曰,為酒為醴,烝畀祖妣,以洽百禮,降福孔偕。

齊侯使諸姜宗婦來送葬,召萊子,萊子不會,故晏弱城東陽以偪之。

鄭成公疾,子駟請息肩於晉,公曰,楚君以鄭故,親集矢於其目,非異人任,寡人也,若背之,是棄力與言,其誰暱我,免寡人,唯二三子。

秋,七月,庚辰,鄭伯睔卒,於是子罕當國,子駟為政,子國為司馬,晉師侵鄭,諸大夫欲從晉,子駟曰,官命未改,會于戚,謀鄭故也,孟獻子曰,請城虎牢以偪鄭,知武子曰,善鄫之會,吾子聞崔子之言,今不來矣,滕薛小邾之不至,皆齊故也,寡君之憂,不唯鄭,罃將復於寡君,而請於齊,得請而告,吾子之功也,若不得請,事將在齊,吾子之請,諸侯之福也,豈唯寡君賴之。

穆叔聘于宋,通嗣君也。

冬,復會于戚,齊崔武子,及滕薛小邾之大夫皆會,知武子之言故也,遂城虎牢,鄭人乃成。

楚公子申為右司馬,多受小國之賂,以偪子重,子辛,楚人殺之,故書曰,楚殺其大夫公子申。

三年,春,楚公子嬰齊帥師伐吳,公如晉。

夏,四月,壬戌,公及晉侯盟于長樗,公至自晉。

六月,公會單子,晉侯,宋公,衛侯,鄭伯,莒子,邾子,齊世子光,己未,同盟于雞澤,陳侯使袁僑如會,戊寅,叔孫豹,及諸侯之大夫,及陳袁僑盟。

秋,公至自會。

冬,晉荀罃帥師伐許。

三年,春,楚子重伐吳,為簡之師,克鳩茲,至于衡山,使鄧廖帥組甲三百,被練三千以侵吳。吳人要而擊之。獲鄧廖,其能免者,組甲八十,被練三百而已,子重歸,既飲至,三日,吳人伐楚,取駕,駕,良邑也,鄧廖,亦楚之良也,君子謂子重於是役也,所獲不如所亡,楚人以是咎子重,子重病之,遂遇心病而卒。

公如晉,始朝也。

夏,盟于長樗,孟獻子相,公稽首。知武子曰,天子在,而君辱稽首,寡君懼矣。孟獻子曰,以敝邑介在東表,密邇仇讎,寡君將君是望,敢不稽首。

晉為鄭服故,且欲脩吳好,將合諸侯,使士丐告于齊曰,寡君使丐以歲之不易,不虞之不戒,寡君願與一二兄弟相見,以謀不協,請君臨之使,丐乞盟,齊侯欲勿許,而難為不協,乃盟於耏外。

祁奚請老,晉侯問嗣焉,稱解狐,其讎也,將立之而卒,又問焉。對曰,午也可。於是羊舌職死矣,晉侯曰,孰可以代之。對曰,赤也可,於是使祁午為中軍尉,羊舌赤佐之,君子謂祁奚於是能舉善矣。稱其讎,不為諂,立其子,不為比,舉其偏,不為黨。商書曰,無偏無黨,王道蕩蕩,其祁奚之謂矣,解狐得舉,祁午得位,伯華得官,建一官而三物成,能舉善也,夫唯善,故能舉其類,詩云,惟其有之,是以似之,祁奚有焉。

六月,公會單頃公及諸侯,己未,同盟于雞澤,晉侯使荀會逆吳子于淮上,吳子不至。

楚子辛為令尹,侵欲於小國,陳成公使袁僑如會,求成,晉侯使和組父告于諸侯,秋,叔孫豹及諸侯之大夫,及陳袁僑盟,陳請服也。

晉侯之弟揚干,亂行於曲梁,魏絳戮其僕,晉侯怒謂羊舌赤曰,合諸侯以為榮也,揚干為戮,何辱如之,必殺魏絳,無失也,對曰,絳無貳志,事君不辟難,有罪不逃刑,其將來辭,何辱命焉,言終,魏絳至,授僕人書,將伏劍,士魴,張老,止之,公讀其書曰,日君乏使,使臣斯司馬,臣聞師眾以順為武,軍事有死無犯為敬,君合諸侯,臣敢不敬,君師不武,執事不敬,罪莫大焉,臣懼其死,以及揚干,無所逃罪,不能致訓,至於用鉞,臣之罪重,敢有不從,以怒君心,請歸死於司寇,公跣而出,曰,寡人之言,親愛也,吾子之討,軍禮也,寡人有弟,弗能教訓,使干大命,寡人之過也,子無重寡人之過,敢以為請,晉侯以魏絳為能,以刑佐民矣,反役,與之禮食,使佐新軍,張老為中軍司馬,士富為候奄。

楚司馬公子何忌侵陳,陳叛故也。

許靈公事楚,不會于雞澤,冬,晉知武子帥師伐許。

四年,春,王三月,己酉,陳,侯午卒。

夏,叔孫豹如晉。

秋,七月,戊子,夫人姒氏薨,葬陳成公,八月,辛亥,葬我小君定姒。

冬,公如晉。

陳人圍頓。

四年,春,楚師為陳叛故,猶在繁陽,韓獻子患之,言於朝曰,文王帥殷之叛國,以事紂,唯知時也,今我易之,難哉,三月,陳成公卒,楚人將伐陳,聞喪乃止,陳人不聽命,臧武仲聞之曰,陳不服於楚必亡,大國行禮焉而不服,在大猶有咎,而況小乎,夏,楚彭名侵陳,陳無禮故也,穆叔如晉,報知武子之聘也,晉侯享之,金奏肆夏之三,不拜,工歌文王之三,又不拜,歌鹿鳴之三,三拜,韓獻子使行人子員問之,曰,子以君命辱於敝邑,先君之禮,藉之以樂,以辱吾子,吾子舍其大而重,拜其細,敢問何禮也,對曰,三夏,天子所以享元侯也,使臣弗敢與聞,文王,兩君相見之樂也,臣不敢及,鹿鳴,君所以嘉寡君也,敢不拜嘉,四牡,君所以勞使臣也,敢不重拜,皇皇者華,君教使臣曰,必諮於周,臣聞之,訪問於善為咨,咨親為詢,咨禮為度,咨事為諏,咨難為謀,臣獲五善,敢不重拜。

秋,定姒薨,不殯于廟,無櫬,不虞,匠慶謂季文子曰,子為正卿,而小君之喪不成,不終君也,君長,誰受其咎,初,季孫為己樹六檟於蒲圃東門之外,匠慶請木,季孫曰,略,匠慶用蒲圃之檟,季孫不御,君子曰,志所謂多行無禮,必自及也,其是之謂乎。

冬,公如晉聽政,晉侯享公,公請屬鄫,晉侯不許,孟獻子曰,以寡君之密邇於仇讎,而願固事君,無失官命,鄫無賦於司馬,為執事朝夕之命敝邑,敝邑褊小,闕而為罪寡君,寡君是以願借助焉,晉侯許之。

楚人使頓間陳,而侵伐之,故陳人圍頓。

無終子嘉父使孟樂如晉,因魏莊子納虎豹之皮,以請和諸戎,晉侯曰,戎狄無親而貪,不如伐之,魏絳曰,諸侯新服,陳新來和,將觀於我,我德則睦,否則攜貳,勞師於戎,而楚伐陳,必弗能救,是棄陳也,諸華必叛,戎禽獸也,獲戎失華,無乃不可乎,夏訓有之曰,有窮后羿,公曰,后羿何如,對曰,昔有夏之方衰也,后羿自鉏遷于窮石,因夏民以代夏政,恃其射也,不脩民事,而淫于原獸,棄武羅,伯困,熊髡,尨圉,而用寒浞,寒浞,伯明氏之讒子弟也,伯明后寒棄之,夷羿收之,信而使之,以為己相,浞行媚于內,而施賂于外,愚弄其民,而虞羿于田,樹之詐慝,以取其國家,外內咸服,羿猶不悛,將歸自田,家眾殺而亨之,以食其子,其子不忍食諸,死于窮門,靡奔有鬲氏,浞因羿室,生澆及豷,恃其讒慝詐偽而不德于民,使澆用師,滅斟灌及斟尋氏,處澆于過,處豷于戈,靡自有鬲氏,收二國之燼以滅浞,而立少康,少康滅澆于過,后杼滅豷于戈,有窮由是遂亡,失人故也,昔周辛甲之為大史也,命百官,官箴王闕,於虞人之箴曰,芒芒禹跡,畫為九州,經啟九道,民有寢廟,獸有茂草,各有攸處,德用不擾,在帝夷羿,冒于原獸,忘其國恤,而思其麀牡,武不可重,用不恢于夏家,獸臣司原,取告僕夫,虞箴如是,可不懲乎,於是晉侯好田,故魏絳及之,公曰,然則莫如和戎乎,對曰,和戎有五利焉,戎狄荐居,貴貨易土,土可賈焉,一也,邊鄙不聳,民狎其野,穡人成功,二也,戎狄事晉,四鄰振動,諸侯威懷,三也,以德綏戎,師徒不動,甲兵不頓,四也,鑒于后羿,而用德度,遠至邇安,五也,君其圖之,公說,使魏絳盟諸戎,脩民事,田以時。

冬,十月,邾人,莒人,伐鄫,臧紇救鄫,侵邾,敗于狐駘,國人逆喪者皆髽,魯於是乎始髽,國人誦之曰,臧之狐裘,敗我於狐駘,我君小子,朱儒是使,朱儒朱儒,使我敗於邾。

五年,春,公至自晉。

夏,鄭伯使公子發來聘。

叔孫豹鄫,世子巫,如晉。

仲孫蔑,衛孫林父,會吳于善道。

秋,大雩。

楚殺其大夫公子壬夫。

公會晉侯,宋公,陳侯,衛侯,鄭伯,曹伯,莒子,邾子,滕子,薛伯,齊世子光,吳人,鄫人,于戚,公至自會。

冬,戍陳

楚公子貞帥師伐陳,公會晉侯,宋公,衛侯,鄭伯,曹伯,齊世子光,救陳,十有二月,公至自救陳。

辛未,季孫行父卒。

五年,春,公至自晉。

王使王叔陳生愬戎于晉,晉人執之,士魴如京師,言王叔之貳於戎也。

夏,鄭子國來聘,通嗣君也。

穆叔覿鄫大子于晉,以成屬鄫。書曰,叔孫豹,鄫大子巫,如晉,言比諸魯大夫也。

吳子使壽越如晉,辭不會于雞澤之故,且請聽諸侯之好,晉人將為之合諸侯,使魯衛先會吳,且告會期,故孟獻子,孫文子,會吳于善道。

秋,大雩,旱也。

楚人討陳叛故,曰,由令尹子辛,實侵欲焉,乃殺之,書曰,楚殺其大夫公子壬夫,貪也,君子謂楚共王於是不刑,詩曰,周道挺挺,我心扃扃,講事不令,集人來定,已則無信,而殺人以逞,不亦難乎,夏書曰,成允成功。

九月,丙午,盟于戚,會吳,且命戍陳也,穆叔以屬鄫為不利,使鄫大夫聽命于會。

楚子囊為令尹,范宣子曰,我喪陳矣,楚人討貳,而立子囊,必改行,而疾討陳,陳近于楚,民朝夕急,能無往乎,有陳,非吾事也,無之而後可,冬,諸侯戍陳,子囊伐陳,十一月,甲午,會于城棣以救之。

季文子卒,大夫入斂,公在位,宰庀家器為葬備,無衣帛之妾,無食粟之馬,無藏金玉,無重器備,君子是以知季文子之忠於公室也,相三君矣,而無私積,可不謂忠乎。

六年,春,王三月,壬午,杞伯姑容卒。

夏,宋華弱來奔。

秋,葬杞桓公。

滕子來朝。

莒人滅鄫。

冬,叔孫豹如邾。

季孫宿如晉。

十有二月,齊侯滅萊。

六年,春,杞桓公卒,始赴以名,同盟故也,宋華弱與樂轡少相狎,長相優,又相謗也,子蕩怒,以弓梏華弱于朝,平公見之,曰,司武而梏於朝,難以勝矣,遂逐之,夏,宋華弱來奔,司城子罕曰,同罪異罰,非刑也,專戮於朝,罪孰大焉,亦逐子蕩,子蕩射子罕之門曰,幾日而不我從,子罕善之如初。

秋,滕成公來朝,始朝公也。

莒人滅鄫,鄫恃賂也。

冬,穆叔如邾,聘且脩平。

晉人以鄫故來討曰,何故亡鄫,季武子如晉見,且聽命。

十一月,齊侯滅萊,萊恃謀也,於鄭子國之來聘也,四月,晏弱城東陽,而遂圍萊,甲寅,堙之,環城,傅於堞,及杞桓公卒之月,乙未,王湫帥師及正輿子,棠人,軍齊師,齊師大敗之,丁未,入萊,萊共公浮柔奔棠,正輿子王湫奔莒,莒人殺之,四月,陳無宇獻萊宗器于襄公,晏弱圍棠,十一月,丙辰,而滅之,遷萊于郳,高厚,崔杼,定其田。

七年,春,郯子來朝。

夏,四月,三卜郊不從,乃免牲。

小邾子來朝。

城費。

秋,季孫宿如衛。

八月,螽。

冬,十月,衛侯使孫林父來聘,壬戌,及孫林父盟,楚公子貞帥師圍陳。

十有二月,公會晉侯,宋公,陳侯,衛侯,曹伯,莒子,邾子,于鄬,鄭伯髡頑如會,未見諸侯,丙戌,卒于鄵,陳侯逃歸,

七年,春,郯子來朝,始朝公也。

夏,四月,三卜郊不從,乃免牲,孟獻子曰,吾乃今而後知有卜筮,夫郊祀后稷,以祈農事也,是故啟蟄而郊,郊而後耕,今既耕而卜郊,宜其不從也。

南遺為費宰,叔仲昭伯為隧正,欲善季氏,而求媚於南遺,謂遺請城費,吾多與而役,故季氏城費。

小邾穆公來朝,亦始朝公也。

秋,季武子如衛,報子叔之聘,且辭緩報,非貳也。

冬,十月,晉韓獻子告老,公族穆子有廢疾,將立之,辭曰,詩曰,豈不夙夜,謂行多露,又曰,弗躬弗親,庶民弗信,無忌不才,讓其可乎,請立起也,與田蘇游,而曰好仁,詩曰,靖共爾位,好是正直,神之聽之,介爾景福,恤民為德,正直為正,正曲為直,參和為仁,如是則神聽之,介福降之,立之,不亦可乎,庚戌,使宣子朝,遂老,晉侯謂韓無忌仁,使掌公族大夫。

衛孫文子來聘,且拜武子之言,而尋孫桓子之盟,公登亦登,叔孫穆子相,趨進曰,諸侯之會,寡君未嘗後衛君,今吾子不後寡君,寡君未知所過。吾子其少安。孫子無辭,亦無悛容,穆叔曰,孫子必亡,為臣而君,過而不悛,亡之本也,詩曰,退食自公,委蛇委蛇,謂從者也,衡而委蛇必折。

楚子囊圍陳,會于鄬以救之。

鄭僖公之為大子也,於成之十六年,與子罕適晉,不禮焉,又與子豐適楚,亦不禮焉,及其元年,朝于晉,子豐欲愬諸晉而廢之,子罕止之,及將會于鄬,子駟相,又不禮焉,侍者諫,不聽,又諫,殺子,及鄵,子駟使賊夜弒僖公,而以瘧疾赴于諸侯,簡公生五年,奉而立之。

陳人患楚,慶虎,慶寅,謂楚人曰,吾使公子黃往而執之,楚人從之,二慶使告陳侯于會曰,楚人執公子黃矣,君若不來,群臣不忍社稷宗廟,懼有二圖,陳侯逃歸。

八年,春,王正月,公如晉。

夏,葬鄭僖公。

鄭人侵蔡,獲蔡公子燮。

季孫宿會晉侯,鄭伯,齊人,宋人,衛人,邾人,于邢丘。

公至自晉。

莒人伐我東鄙。

秋,九月,大雩。

冬,楚公子貞帥師伐鄭。

晉侯使士丐來聘。

八年,春,公如晉朝,且聽朝聘之數。

鄭群公子以僖公之死也,謀子駟,子駟先之,夏,四月,庚辰,辭殺子狐,子熙,子侯,子丁,孫擊,孫惡,出奔衛。

庚寅,鄭子國,子耳,侵蔡,獲蔡司馬公子燮,鄭人皆喜,唯子產不順,曰,小國無文德而有武功,禍莫大焉,楚人來討,能勿從乎,從之,晉師必至,晉楚伐鄭,自今鄭國,不四五年,弗得寧矣,子國怒之,曰,爾何知,國有大命,而有正卿,童子言焉,將為戮矣。

五月,甲辰,會于邢丘,以命朝聘之數,使諸侯之大夫聽命,季孫宿,齊高厚,宋向戌,衛甯殖,邾大夫會之,鄭伯獻捷于會,故親聽命,大夫不書,尊晉侯也。

莒人伐我東鄙,以疆鄫田。

秋,九月,大雩,旱也。

冬,楚子囊伐鄭,討其侵蔡也,子駟,子國,子耳,欲從楚,子孔,子蟜,子展,欲待晉,子駟曰,周詩有之曰,俟河之清,人壽幾何,兆云詢多,職競作羅,謀之多族,民之多違,事滋無成,民急矣,姑從楚以紓吾民,晉師至,吾又從之,敬共幣帛,以待來者,小國之道也,犧牲玉帛,待於二竟,以待彊者,而庇民焉,寇不為害,民不罷病,不亦可乎,子展曰,小所以事大,信也,小國無信,兵亂日至,亡無日矣,五會之信,今將背之,雖楚救我,將安用之,親我無成,鄙我是欲,不可從也,不如待晉,晉君方明,四軍無闕,八卿和睦,必不棄鄭,楚師遼遠,糧食將盡,必將速歸,何患焉,舍之聞之,杖莫如信,完守以老楚,杖信以待晉,不亦可乎,子駟曰,詩云,謀夫孔多,是用不集,發言盈庭,誰敢執其咎,如匪行邁謀,是用不得于道,請從楚,騑也受其咎,乃及楚平,使王子伯駢告于晉曰,君命敝邑,脩而車賦,儆而師徒,以討亂略,蔡人不從,敝邑之人,不敢寧處,悉索敝賦,以討于蔡,獲司馬燮,獻于邢丘,今楚來討曰,女何故稱兵于蔡,焚我郊保,馮陵我城郭,敝邑之眾,夫婦男女,不遑啟處,以相救也,翦焉傾覆,無所控告,民死亡者,非其父兄,即其子弟。夫人愁痛,不知所庇。民知窮困,而受盟于楚,孤也與其二三臣,不能禁止,不敢不告,知武子使行人子員對之曰,君有楚命,亦不使一介行李,告于寡君,而即安于楚,君之所欲也,誰敢違君,寡君將帥諸侯以見于城下,唯君圖之,晉范宣子來聘,且拜公之辱,告將用師于鄭,公享之,宣子賦摽有梅,季武子曰,誰敢哉,今譬於草木,寡君在君,君之臭味也,歡以承命,何時之有,武子賦角弓,賓將出,武子賦彤弓,宣子曰,城濮之役,我先君文公獻功于衡雍,受彤弓于襄王,以為子孫藏,匄也,先君守官之嗣也,敢不承命,君子以為知禮。

九年,春,宋災。

夏,季孫宿如晉。

五月,辛酉,夫人姜氏薨。

秋,八月,癸未,葬我小君穆姜。

冬,公會晉侯,宋公,衛侯,曹伯,莒子,邾子,滕子,薛伯,杞伯,小邾子,齊世子光,伐鄭,十有二月,己亥,同盟于戲,楚子伐鄭。

九年,春,宋災,樂喜為司城,以為政,使伯氏司里,火所未至,徹小屋,塗大屋,陳畚挶,具綆缶,備水器,量輕重,蓄水潦,積土塗。巡丈城,繕守備。表火道,使華臣具正徒,令隧正,納郊保,奔火所,使華閱討右官,官庀其司,向戌討左,亦如之,使樂遄庀刑器,亦如之,使皇鄖命校正出馬,工正出車,備甲兵,庀武守,使西鉏吾庀府守,令司宮巷伯儆宮,二師令四鄉正敬享,祝宗用馬于四墉,祀盤庚于西門之外,晉侯問於士弱曰,吾聞之,宋災,於是乎知有天道,何故,對曰,古之火正,或食於心,或食於咮,以出內火,是故咮為鶉火,心為大火,陶唐氏之火正閼伯,居商丘,祀大火,而火紀時焉,相土因之,故商主大火,商人閱其禍敗之釁,必始於火,是以日知其有天道也,公曰,可必乎,對曰,在道,國亂無象,不可知也。

夏,季武子如晉,報宣子之聘也。

穆姜薨於東宮,始往而筮之,遇艮之八,史曰,是謂艮之隨,隨其出也,君必速出,姜曰,亡,是於周易,曰,隨元亨利貞,咎,元,體之長也,亨,嘉之會也,利,義之和也,貞,事之幹也,體仁足以長人,嘉德足以合禮,利物足以和義,貞固足以幹事,然故不可誣也,是以雖隨无咎,今我婦人而與於亂,固在下位,而有不仁,不可謂元,不靖國家,不可謂亨,作而害身,不可謂利,棄位而姣,不可謂貞,有四德者,隨而無咎,我皆無之,豈隨也哉,我則取惡,能無咎乎,必死於此,弗得出矣。

秦景公使士雃乞師于楚,將以伐晉,楚子許之,子囊曰,不可,當今吾不能與晉爭,晉君類能而使之,舉不失選,官不易方,其卿讓於善,其大夫不失守,其士競於教,其庶人力於農穡,商工皁隸,不知遷業,韓厥老矣,知罃稟焉,以為政,范丐少於中行偃而上之,使佐中軍,韓起少於欒黶,而欒黶士魴上之,使佐上軍,魏絳多功,以趙武為賢而為之佐,君明臣忠,上讓下競,當是時也,晉不可敵,事之而後可,君其圖之,王曰,吾既許之矣,雖不及晉,必將出師,秋,楚子師于武城,以為秦援,秦人侵晉,晉饑,弗能報也。

冬,十月,諸侯伐鄭,庚午,季武子,齊崔杼,宋皇鄖,從荀罃,士

丐,門于鄟門,衛北宮括,曹人,邾人,從荀偃,韓起,門于師之梁,滕人,薛人,從欒黶,士魴,門于北門,杞人,郳人,從趙武,魏絳,斬行栗。甲戌,師于氾,令於諸侯曰:脩器備,盛餱糧,歸老幼,居疾于虎牢。肆眚圍鄭,鄭人恐,乃行成,中行獻子曰,遂圍之,以待楚人之救也,而與之戰,不然無成,知武子曰,許之盟而還師,以敝楚人,吾三分四軍,與諸侯之銳,以逆來者,於我未病,楚不能矣,猶愈於戰,暴骨以逞,不可以爭,大勞未艾,君子勞心,小人勞力,先王之制也,諸侯皆不欲戰,乃許鄭成,十一月,己亥,同盟于戲,鄭服也,將盟,鄭六卿公子騑,公子發,公子嘉,公孫輒,公孫蠆,公孫舍之,及其大夫門子皆從鄭伯,晉士莊子為載書,曰,自今日既盟之後,鄭國而不唯晉命是聽,而或有異志者,有如此盟,公騑趨進曰,天禍鄭國,使介居二大國之閒,大國不加德音,而亂以要之,使其鬼神不獲歆其禋祀,其民人不獲享其土利,夫婦辛苦墊隘,無所底告,自今日既盟之後,鄭國而不唯有禮與彊,可以庇民者是從,而敢有異志者,亦如之,荀偃曰,改載書。公孫舍之曰,昭大神要言焉,若可改也,大國亦可叛也。知武子謂獻子曰,我實不德,而要人以盟,豈禮也哉,非禮何以主盟,姑盟而退,脩德息師而來,終必獲鄭,何必今日,我之不德,民將棄我,豈唯鄭,若能休和,遠人將至,何恃於鄭,乃盟而還。

晉人不得志於鄭,以諸侯復伐之,十二月,癸亥,門其三門,閏月,戊寅,濟于陰阪,侵鄭,次于陰口而還,子孔曰,晉師可擊也,師老而勞,且有歸志,必大克之,子展曰,不可。

公送晉侯,晉侯以公宴于河上,問公年,季武子對曰,會于沙隨之歲,寡君以生,晉侯曰,十二年矣,是謂一終,一星終也,國君十五而生子,冠而生子,禮也,君可以冠矣,大夫盍為冠具,武子對曰,君冠,必以祼享之禮行之,以金石之樂節之,以先君之祧處之,今寡君在行,未可具也,請及兄弟之國,而假備焉,晉侯曰,諾,公還及衛,冠于成公之廟,假鍾磬焉,禮也。

楚子伐鄭,子駟將及楚平,子孔,子蟜,曰,與大國盟,口血未乾而背之,可乎,子駟,子展,曰,吾盟固云唯彊是從,今楚師至,晉不我救,則楚彊矣,盟誓之言,豈敢背之,且要盟無質,神弗臨也,所臨唯信,信者言之瑞也,善之主也,是故臨之,明神不蠲要盟,背之可也,乃及楚平,公子罷戎入盟,同盟于中分,楚莊夫人卒,王未能定鄭而歸。

晉侯歸,謀所以息民,魏絳請施舍,輸積聚以貸,自公以下,苟有積者,盡出之,國無滯積,亦無困人,公無禁利,亦無貪民,祈以幣更,賓以特牲,器用不作,車服從給,行之期年,國乃有節,三駕而楚不能與爭。

十年,春,公會晉侯,宋公,衛侯,曹伯,莒子,邾子,滕子,薛伯,杞伯,小邾子,齊世子光,會吳于柤。

夏,五月,甲午,遂滅偪陽,公至自會。

楚公子貞,鄭公孫輒,帥師伐宋。

晉師伐秦。

秋,莒人伐我東鄙。

公會晉侯,宋公,衛侯,曹伯,莒子,邾子,齊世子光,滕子,薛伯,杞,伯,小邾子,伐鄭。

冬,盜殺鄭公子騑,公子發,公孫輒。

戍鄭虎牢。

楚公子貞帥師救鄭。

公至自伐鄭。

十年,春,會于柤,會吳子壽夢也,三月,癸丑,齊高厚相大子光,以先會諸侯于鍾離,不敬,士莊子曰,高子相大子以會諸侯,將社稷是衛,而皆不敬,棄社稷也,其將不免乎,夏,四月,戊午,會于柤。

晉荀偃,士丐,請伐偪陽而封宋向戌焉,荀罃曰,城小而固,勝之不武,弗勝為笑,固請,丙寅,圍之,弗克,孟氏之臣秦堇父,輦重如役,偪陽人啟門,諸侯之士門焉,縣門發,郰人紇抉之,以出門者,狄虒彌建大車之輪,而蒙之以甲,以為櫓,左執之,右拔戟,以成一隊,孟獻子曰,詩所謂有力如虎者也,主人縣布,堇父登之,及堞而絕之,隊則又縣之,蘇而復上者三,主人辭焉,乃退,帶其斷以徇於軍三日,諸侯之師,久於偪陽,荀偃,士丐,請於荀罃曰,水潦將降,懼不能歸,請班師,知伯怒,投之以机,出於其間曰,女成二事而後告余,余恐亂命,以不女違,女既勤君而興諸侯,牽帥老夫,以至于此,既無武守,而又欲易余罪,曰是實班師,不然克矣,余羸老也,可重任乎,七日不克,必爾乎取之,五月,庚寅,荀偃,士丐,帥卒攻偪陽,親受矢石,甲午,滅之,書曰,遂滅偪陽,言自會也,以與向戌,向戌辭曰,君若猶辱鎮撫宋國,而以偪陽光啟寡君,群臣安矣,其何貺如之,若專賜臣,是臣興諸侯以自封也,其何罪大焉,敢以死請,乃予宋公。

宋公享晉侯于楚丘,請以桑林,荀罃辭,荀偃,士丐,曰,諸侯宋魯,於是觀禮,魯有禘樂,賓祭用之,宋以桑林享君,不亦可乎,舞師題以旌夏,晉侯懼而退,入于房,去旌,卒享而還,及著雍,疾,卜,桑林見,荀偃,士丐,欲奔請禱焉,荀罃不可曰,我辭禮矣,彼則以之,猶有鬼神,於彼加之,晉侯有間,以偪陽子歸,獻于武宮,謂之夷俘,偪陽,妘姓也,使周內史選其族嗣,納諸霍人,禮也,師歸,孟獻子以秦堇父為右,生秦丕茲,事仲尼。

六月,楚子囊,鄭子耳,伐宋,師于訾毋,庚午,圍宋門于桐門。

晉荀罃伐秦,報其侵也。

衛侯救宋,師于襄牛,鄭子展曰,必伐衛,不然,是不與楚也,得罪於晉,又得罪於楚,國將若之何,子駟曰,國病矣,子展曰,得罪於二大國必亡,病不猶愈於亡乎,諸大夫皆以為然,故鄭皇耳帥師侵衛,楚令也,孫文子卜追之,獻兆於定姜,姜氏問繇曰,兆如山陵,有夫出征,而喪其雄,姜氏曰,征者喪雄,禦寇之利也,大夫圖之,衛人追之,孫蒯獲鄭皇耳于犬丘。

秋,七月,楚子囊,鄭子耳,伐我西鄙,還圍蕭,八月,丙寅,克之,九月,子耳侵宋北鄙,孟獻子曰,鄭其有災乎,師競已甚,周猶不堪競,況鄭乎,有災,其執政之三士乎。

莒人間諸侯之有事也,故伐我東鄙。

諸侯伐鄭,齊崔杼使大子光先至于師,故長於滕,己酉,師于牛首。

初子駟與尉止有爭,將禦諸侯之師,而黜其車,尉止獲,又與之爭,子駟抑尉止曰,爾車非禮也,遂弗使獻,初,子駟為田洫,司氏,堵氏,侯氏,子師氏,皆喪田焉,故五族聚群不逞之人,因公子之徒以作亂,於是子駟當國,子國為司馬,子耳為司空,子孔為司徒,冬,十月,戊辰,尉止,司臣,侯晉,堵女父,子師僕,帥賊以入,晨攻執政于西宮之朝,殺子駟,子國,子耳,劫鄭伯,以如北宮,子孔知之,故不死,書曰盜,言無大夫焉,子西聞盜,不儆而出,尸而追盜,盜入於北宮,乃歸授甲,臣妾多逃,器用多喪,子產聞盜為門者,庀群司,閉府庫,慎閉藏,完守備,成列而後出,兵車十七乘,尸而攻盜於北宮,子蟜帥國人助之,殺尉止,子師僕,盜眾盡死,侯晉奔晉,堵女父,司臣,尉翩,司齊,奔宋,子孔當國,為載書,以位序聽政辟,大夫諸司門子弗順,將誅之,子產止之,請為之焚書,子孔不可,曰,為書以定國,眾怒而焚之,是眾為政也,國不亦難乎,子產曰,眾怒難犯,專欲難成,合二難以安國,危之道也,不如焚書以安眾,子得所欲,眾亦得安,不亦可乎,專欲無成,犯眾興禍,子必從之,乃焚書於倉門之外,眾而後定。

諸侯之師,城虎牢而戍之,晉師城梧及制,士魴,魏絳,戍之,書曰,戍鄭虎牢,非鄭地也,言將歸焉,鄭及晉平。

楚子囊救鄭,十一月,諸侯之師還鄭而南,至於陽陵,楚師不退,知武子欲退,曰,今我逃楚,楚必驕,驕則可與戰矣,欒黶曰,逃楚,晉之恥也,合諸侯以益恥,不如死,我將獨進,師遂進。己亥,與楚師夾潁而軍,子蟜曰,諸侯既有成行,必不戰矣,從之將退,不從亦退,退楚必圍我。猶將退也,不如從楚,亦以退之。霄涉潁,與楚人盟,欒黶欲伐鄭師,荀罃不可,曰,我實不能禦楚,又不能庀鄭,鄭何罪,不如致怨焉而還,今伐其師,楚必救之。戰而不克,為諸侯笑;克不可命,不如還也。丁未,諸侯之師還,侵鄭北鄙而歸,楚人亦還。

王叔陳生與伯輿爭政,王右伯輿,王叔陳生怒而出奔,及河,王復之,殺史狡以說焉,不入,遂處之,晉侯使士丐平王室,王叔與伯輿訟焉,王叔之宰,與伯輿之大夫瑕禽,坐獄於王庭,士丐聽之,王叔之宰曰,篳門閨竇之人,而皆陵其上,其難為上矣,瑕禽曰,昔平王東遷,吾七姓從王,牲用備具,王賴之,而賜之騂旄之盟,曰,世世無失職,若篳門閨竇,其能來東底乎,且王何賴焉,今自王叔之相也,政以賄成,而刑放於寵,官之師旅,不勝其富,吾能無篳門閨竇乎,唯大國圖之,下而無直,則何謂正矣,范宣子曰,天子所右,寡君亦右之,所左亦左之,使王叔氏與伯輿合要,王叔氏不能舉其契,王叔奔晉,不書,不告也,單靖公為卿士,以相王室。

十有一年,春,王正月,作三軍。

夏,四月,四卜郊不從,乃不郊。

鄭公孫舍之帥師侵宋。

公會晉侯,宋公,衛侯,曹伯,齊世子光,莒子,邾子,滕子,薛伯,杞伯,小邾子,伐鄭。

秋,七月,己未,同盟于亳城北。

公至自伐鄭。

楚子,鄭伯,伐宋。

公會晉侯,宋公,衛侯,曹伯,齊世子光,莒子,邾子,滕子,薛伯,杞伯,小邾子,伐鄭,會于蕭魚,公至自會。

楚執鄭行人,良霄

冬,秦人伐晉。

十一年,春,季武子將作三軍,告叔孫穆子曰,請為三軍,各征其軍,穆子曰,政將及子,子必不能,武子固請之,穆子曰,然則盟諸,乃盟諸僖閎,詛諸五父之衢,正月,作三軍,三分公室,而各有其一,三子各毀其乘,季氏使其乘之人,以其役邑入者無征,不入者倍征,孟氏使半為臣,若子若弟,叔孫氏使盡為臣,不然不舍。

鄭人患晉楚之故,諸大夫曰,不從晉,國幾亡,楚弱於晉,晉不吾疾也,晉疾,楚將辟之,何為而使晉師致死於我,楚弗敢敵,而後可固與也,子展曰,與宋為惡,諸侯必至,吾從之盟,楚師至,吾又從之,則晉怒甚矣,晉能驟來楚將不能,吾乃固與晉,大夫說之,使疆場之司,惡於宋,宋向戌侵鄭,大獲,子展曰,師而伐宋可矣,若我伐宋,諸侯之伐我必疾,吾乃聽命焉,且告於楚,楚師至,吾乃與之盟,而重賂晉師,乃免矣,夏,鄭子展侵宋。

四月,諸侯伐鄭,己亥,齊太子光,宋向戌,先至于鄭,門于東門,其莫,晉荀罃至于西郊,東侵舊許,衛孫林父侵其北鄙,六月,諸侯會于北林,師于向右,還次于瑣,圍鄭,觀兵于南門,西濟于濟隧,鄭人懼,乃行成,秋,七月,同盟于亳,范宣子曰,不慎,必失諸侯,諸侯道敝而無成,能無貳乎,乃盟,載書曰,凡我同盟,毋薀年,毋壅利,毋保姦,毋留慝,救災患,恤禍亂,同好惡,獎王室,或間茲命,司慎司盟,名山名川,群神群祀,先王先公,七姓十二國之祖,明神殛之,俾失其民,隊命亡氏,踣其國家。

楚子囊乞旅于秦,秦右大夫詹帥師從楚子,將以伐鄭,鄭伯逆之,丙子,伐宋。

九月,諸侯悉師以復伐鄭,鄭人使良霄,大宰石蓖,如楚,告將服于晉,曰,孤以社稷之故,不能懷君,君若能以玉帛綏晉,不然,則武震以攝威之,孤之願也,楚人執之,書曰行人,言使人也。

諸侯之師觀兵于鄭東門,鄭人使王子伯駢行成,甲戌,晉趙武入盟鄭伯,冬,十月,丁亥,鄭子展出盟晉侯,十二月,戊寅,會于蕭魚,庚辰,赦鄭囚,皆禮而歸之,納斥候,禁侵掠,晉侯使叔肸告于諸侯,公使臧孫紇對曰,凡我同盟,小國有罪,大國致討,苟有以藉手,鮮不赦宥,寡君聞命矣。

鄭人賂晉侯以師悝,師觸,師蠲,廣車,軘車,淳十五乘,甲兵備,凡兵車百乘,歌鐘二肆,及其鎛磬,女樂二八,晉侯以樂之半賜魏絳,曰,子教寡人,和諸戎狄,以正諸華,八年之中,九合諸侯,如樂之和,無所不諧,請與子樂之,辭曰,夫和戎狄,國之福也,八年之中,九合諸侯,諸侯無慝,君之靈也,二三子之勞也,臣何力之有焉,抑臣願君安其樂而思其終也,詩曰,樂只君子,殿天子之邦,樂只君子,福祿攸同,便蕃左右,亦是帥從,夫樂以安德,義以處之,禮以行之,信以守之,仁以厲之,而後可以殿邦國,同福祿,來遠人,所謂樂也,書曰,居安思危,思則有備,有備無患,敢以此規,公曰,子之教,敢不承命,抑微子,寡人無以待戎,不能濟河,夫賞,國之典也,藏在盟府,不可廢也,子其受之,魏絳於是乎始有金石之樂,禮也。

秦庶長鮑,庶長武,帥師伐晉,以救鄭,鮑先入晉地,士魴禦之,少秦師而弗設備,壬午,武濟自輔氏,與鮑交伐晉師,己丑,秦晉戰于櫟,晉師敗績,易秦故也。

十有二年,春,王二月,莒人伐我東鄙,圖台,季孫宿帥師救台,遂入鄆。

夏,晉侯使士魴來聘。

秋,九月,吳子乘卒。

冬,楚公子貞帥師侵宋。

公如晉。

十二年,春,莒人伐我東鄙,圍台,季武子救台,遂入鄆,取其鐘以為公盤。

夏,晉士魴來聘,且拜師。

秋,吳子壽夢卒,臨於周廟,禮也,凡諸侯之喪,異姓臨於外,同姓於宗廟,同宗於祖廟,同族於禰廟,是故魯為諸姬,臨於周廟,為邢,凡,蔣,茅,胙,祭,臨於周公之廟。

冬,楚子囊,秦庶長無地,伐宋,師于楊梁,以報晉之取鄭也。

靈王求后于齊。

齊侯問對於晏桓子,桓子對曰,先王之禮辭有之,天子求后於諸侯,諸侯對曰,夫婦所生若而人,妾婦之子若而人,無女而有姊妹,及姑姊妹,則曰,先守某公之遺女若而人,齊侯許昏,王使陰里逆之。

公如晉,朝,且拜士魴之辱,禮也。

秦嬴歸于楚,楚司馬子庚聘于秦,為夫人寧,禮也。

十有三年,春,公至自晉。

夏,取邿。

秋,九月,庚辰,楚子審卒。

冬,城防。

十三年春,公至自晉,孟獻子書勞于廟,禮也。

夏,邿亂,分為三師救邿,遂取之。凡書取,言易也。用大師焉曰滅,弗地曰入。

荀罃,士魴,卒,晉侯蒐于綿上以治兵,使士丐將中軍,辭曰,伯游長,昔臣習於知伯,是以佐之,非能賢也,請從伯游,荀偃將中軍,士丐佐之,使韓起將上軍,辭以趙武,又使欒黶辭曰,臣不如韓起,韓起願上趙武,君其聽之,使趙武將上軍,韓起佐之,欒黶將下軍,魏絳佐之,新軍無帥,晉侯難其人,使其什吏,率其卒乘官屬,以從於下軍,禮也,晉國之民,是以大和,諸侯遂睦,君子曰,讓,禮之主也,范宣子讓,其下皆讓,欒黶為汰,弗敢違也,晉國以平,數世賴之,刑善也夫。一人刑善,百姓休和。可不務乎,書曰,一人有慶,兆民賴之,其寧惟永,其是之謂乎,周之興也,其詩曰,儀刑文王,萬邦作孚,言刑善也,及其衰也,其詩曰,大夫不均,我從事獨賢,言不讓也,世之治也,君子尚能而讓其下,小人農力以事其上,是以上下有禮,而讒慝黜遠,由不爭也,謂之懿德,及其亂也,君子稱其功以加小人,小人伐其技以馮君子,是以上下無禮,亂虐並生,由爭善也,謂之昏德,國家之敝,恆必由之。

楚子疾,告大夫曰,不穀不德,少主社稷,生十年而喪先君,未及習師保之教訓,而應受多福,是以不德,而亡師于鄢,以辱社稷,為大夫憂,其弘多矣,若以大夫之靈,獲保首領,以歿於地,唯是春秋窀穸之事,所以從先君於禰廟者,請為靈若厲,大夫擇焉,莫對,及五命,乃許,秋,楚共王卒,子囊謀諡,大夫曰,君有命矣,子囊曰,君命以共,若之何毀之,赫赫楚國,而君臨之,撫有蠻夷,奄征南海,以屬諸夏,而知其過,可不謂共乎,請謚之共,大夫從之。

吳侵楚,養由基奔命,子庚以師繼之,養叔曰,吳乘我喪,謂我不能師也,必易我而不戒,子為三覆以待我,我請誘之,子庚從之,戰于康浦,大敗吳師,獲公子黨,君子以吳為不弔,詩曰,不弔昊天,亂靡有定。

冬,城防,書事時也,於是將早城,臧武仲請俟畢農事,禮也。

鄭良霄,大宰石蓖,猶在楚,石蓖言於子囊曰,先王卜征五年,而歲習其祥,祥習則行,不習則增,脩德而改卜,今楚實不競,行人何罪,止鄭一卿,以除其偪,使睦而疾楚,以固於晉焉,用之使歸,而廢其使,怨其君以疾其大夫,而相牽引也,不猶愈乎,楚人歸之。

十有四年,春,王正月,季孫宿叔老會晉士丐,齊人,宋人,衛人,鄭公孫蠆,曹人,莒人,邾人,滕人,薛人,杞人,小邾人,會吳于向。

二月,乙未朔,日有食之。

夏,四月,叔孫豹會晉荀偃,齊人,宋人,衛北宮括,鄭公孫蠆,曹人,莒人,邾人,滕人,薛人,杞人,小邾人,伐秦。

己未,衛侯出奔齊。

莒人侵我東鄙。

秋,楚公子貞帥師伐吳。

冬,季孫宿會晉士丐,宋華閱,衛孫林父,鄭公孫蠆,莒人,邾人,于戚。

十四年,春,吳告敗于晉,會于向,為吳謀楚故也,范宣子數吳之不德也,以退吳人,執莒公子務婁,以其通楚使也,將執戎子駒支,范宣子親數諸朝,曰,來,姜戎氏,昔秦人迫逐乃祖吾離于瓜州,乃祖吾離被苫蓋,蒙荊棘,以來歸我先君,我先君惠公有不腆之田,與女剖分而食之,今諸侯之事我寡君,不如昔者,蓋言語漏洩,則職女之由,詰朝之事,爾無與焉,與將執女,對曰,昔秦人負恃其眾,貪于土地,逐我諸戎,惠公蠲其大德,謂我諸戎,是四獄之裔冑也,毋是翦棄,賜我南鄙之田,狐狸所居,豺狼所嗥,我諸戎除翦其荊棘,驅其狐狸豺狼,以為先君不侵不叛之臣,至于今不貳,昔文公與秦伐鄭,秦人竊與鄭盟,而舍戍焉,於是乎有殽之師,晉禦其上,戎亢其下,秦師不復,我諸戎實然,譬如捕鹿,晉人角之,諸戎掎之,與晉踣之,戎何以不免,自是以來,晉之百役,與我諸戎,相繼于時,以從執政,猶殽志也,豈敢離逷,今官之師旅,無乃實有所闕,以攜諸侯,而罪我諸戎,我諸戎飲食衣服,不與華同,贄幣不通,言語不達,何惡之能為,不與於會,亦無瞢焉,賦青蠅而退,宣子辭焉,使即事於會,成愷悌也,於是子叔齊子為季武子介以會,自是晉人輕魯幣,而益敬其使。

吳子諸樊既除喪,將立季札,季札辭曰,曹宣公之卒也,諸侯與曹人不義曹君,將立子臧,子臧去之,遂弗為也,以成曹君,君子曰,能守節,君義嗣也,誰敢奸君,有國非吾節也,札雖不才,願附於子臧,以無失節,固立之,棄其室而耕,乃舍之。

夏,諸侯之大夫從晉侯伐秦,以報櫟之役也,晉侯待于竟,使六卿帥諸侯之師以進,及涇不濟,叔向見叔孫穆子,穆子賦匏有苦葉,叔向退而具舟,魯人,莒人,先濟,鄭子蟜見衛北宮懿子曰,與人而不固,取惡莫甚焉,若社稷何,懿子說,二子見諸侯之師,而勸之濟,濟涇而次,秦人毒涇上流,師入多死,鄭司馬子蟜帥鄭師以進,師皆從之,至于棫林,不獲成焉,荀偃令曰,雞鳴而駕,塞井夷灶,唯余馬首是瞻,欒黶曰,晉國之命,未是有也余馬首欲東,乃歸,下軍從之,左史謂魏莊子曰,不待中行伯乎,莊子曰,夫子命從帥,欒伯吾帥也,吾將從之,從帥所以待夫子也,伯游曰,吾今實過,悔之何及,多遺秦禽,乃命大還,晉人謂之遷延之役,欒鍼曰,此役也,報櫟之敗也,役又無功,晉之恥也,吾有二位於戎路,敢不恥乎,與士鞅馳秦師死焉,士鞅反,欒黶謂士丐曰,余弟不欲往而子召之,余弟死而子來,是而子殺余之弟也,弗逐,余亦將殺之,士鞅奔秦,於是齊崔杼,宋華閱,仲江,會伐秦,不書,惰也,向之會,亦如之,衛北宮括不書於向,書於伐秦,攝也,秦伯問於士鞅曰,晉大夫其誰先亡,對曰,其欒氏乎,秦伯曰,以其汰乎,對曰,然,欒黶汰虐已甚,猶可以免,其在盈乎,秦伯曰,何故,對曰,武子之德在民,如周人之思召公焉,愛其甘棠,況其子乎,欒黶死,盈之善未能及人,武子所施沒矣,而黶之怨實章,將於是乎在,秦伯以為知言,為之請於晉而復之。

衛獻公戒孫文子,寧惠子食,皆服而朝,日旰不召,而射鴻於囿,二子從之,不釋皮冠而與之言,二子怒,孫文子知戚,孫蒯入使,公飲之酒,使大師歌巧言之卒章,大師辭,師曹請為之,初,公有嬖妾,使師曹誨之琴,師曹鞭之,公怒,鞭師曹三百,故師曹欲歌之,以怒孫子,以報公,公使歌之,遂誦之,蒯懼,告文子,文子曰,君忌我矣,弗先,必死,并帑於戚,而入見蘧伯玉曰,君之暴虐,子所知也,大懼社稷之傾覆,將若之何,對曰,君制其國,臣敢奸之,雖奸之,庸知愈乎,遂行,從近關出,公使子蟜,子伯,子皮,與孫子盟于丘宮,孫子皆殺之,四月,己未,子展奔齊,公如鄄,使子行於孫子,孫子又殺之,公出奔齊,孫氏追之,敗公徒于河澤,鄄人執之,初,尹公佗學射於庾公差,庾公差學射於公孫丁,二子追公,公孫丁御公,子魚曰,射為背師,不射為戮,射為禮乎,射兩軥而還,尹公佗曰,子為師,我則遠矣,乃反之,公孫丁授公轡而射之,貫臂,子鮮從公及竟,公使祝宗告亡,且告無罪,定姜曰,無神何告,若有,不可誣也,有罪若何告無,舍大臣而與小臣謀,一罪也,先君有冢卿以為師保,而蔑之,二罪也,余以巾櫛事先君,而暴妾使余,三罪也,告亡而已,無告無罪,公使厚成叔弔于衛曰,寡君使瘠聞君不撫社稷,而越在他竟,若之何不弔,以同盟之故,使瘠敢私於執事,曰,有君不弔,有臣不敏,君不赦宥,臣亦不帥職,增淫發洩,其若之何,衛人使大叔儀對曰,群臣不佞,得罪於寡君,寡君不以即刑而悼棄之,以為君憂,君不忘先君之好,辱弔群臣,又重恤之,敢拜君命之辱,重拜大貺,厚孫歸復命,語臧武仲曰,衛君其必歸乎,有大叔儀以守,有母弟鱄以出,或撫其內,或營其外,能無歸乎,齊人以郲寄衛侯,及其復也,以郲糧歸,右宰穀從而逃歸,衛人將殺之,辭曰,余不說初矣,余狐裘而羔袖,乃赦之,衛人立公孫剽,孫林父,甯殖,相之,以聽命於諸侯,衛侯在郲,臧紇如齊唁衛侯,與之言虐,退而告其人曰,衛侯其不得入矣,其言糞土也,亡而不變,何以復國,子展,子鮮,聞之,見臧紇與之言道,臧孫說,謂其人曰,衛君必入,夫二子者,或輓之,或推之,欲無入得乎。

師歸自伐秦,晉侯舍新軍,禮也,成國不過半天子之軍,周為六軍,諸侯之大者,三軍可也,於是知朔生盈而死,盈生六年而武子卒,彘裘亦幼,皆未可立也,新軍無帥,故舍之,師曠侍於晉侯,晉侯曰,衛人出其君,不亦甚乎,對曰,或者其君實甚,良君將賞善而刑淫,養民如子,蓋之如天,容之如地,民奉其君,愛之如父母,仰之如日月,敬之如神明,畏之如雷霆,其可出乎,夫君,神之主也,民之望也,若困民之主,匱神乏祀,百姓絕望,社稷無主,將安用之,弗去何為,天生民而立之君,使司牧之,勿使失性,有君而為之貳,使師保之,勿使過度,是故天子有公,諸侯有卿,卿置側室,大夫有貳,宗士有朋友,庶人工商皂隸牧圉,皆有親暱,以相輔佐也,善則賞之,過則匡之,患則救之,失則革之,自王以下,各有父兄子弟,以補察其政,史為書,瞽為詩,工誦箴諫,大夫規誨,士傳言,庶人謗,商旅于市,百工獻藝,故夏書曰,遒人以木鐸徇于路,官師相規,工執藝事以諫,正月孟春,於是乎有之,諫失常也,天之愛民甚矣,豈其使一人肆於民上,以從其淫,而棄天地之性,必不然矣。

秋,楚子為庸浦之役故,子囊師于棠以伐吳,吳不出而還,子囊殿,以吳為不能而弗儆,吳人自皋舟之隘要而擊之,楚人不能相救,吳人敗之,獲楚公子宜穀。

王使劉定公賜齊侯,命曰,昔伯舅大公,右我先王,股肱周室,師保萬民,世胙大師,以表東海,王室之不壞,繄伯舅是賴,今余命女環,茲率舅氏之典,纂乃祖考,無忝乃舊,敬之哉,無廢朕命。

晉侯問衛故於中行獻子,對曰,不如因而定之,衛有君矣,伐之,未可以得志,而勤諸侯,史佚有言曰,因重而撫之,仲虺有言曰,亡者侮之,亂者取之,推亡固存,國之道也,君其定衛以待時乎,冬,會于戚,謀定衛也。

范宣子假羽毛於齊而弗歸,齊人始貳。

楚子囊還自伐吳,卒,將死,遺言謂子庚必城郢,君子謂子囊忠,君薨不忘增其名,將死不忘衛社稷,可不謂忠乎,忠,民之望也,詩曰,行歸于周,萬民所望,忠也。

十有五年,春,宋公使同戌來聘,二月,己亥,及向戌盟于劉。

劉夏逆王后于齊。

夏,齊侯伐我北鄙,圍成,公救成,至遇。

季孫宿,叔孫豹,帥師城成郛。

秋,八月,丁巳,日有食之。

邾人伐我南鄙。

冬,十有一月,癸亥,晉侯周卒。

十五年,春,宋向戌來聘,且尋盟,見孟獻子,尤其室曰,子有令聞,而美其室,非所望也,對曰,我在晉,吾兄為之,毀之重勞,且不敢間。

官師從單靖公,逆王后于齊,卿不行,非禮也。

楚公子午為令尹,公子罷戎為右尹,蒍子馮為大司馬,公子橐師為右司馬,公子成為左司馬,屈到為莫敖,公子追舒為箴尹,屈蕩為連尹,養由基為宮廄尹,以靖國人,君子謂楚於是乎能官人,官人,國之急也,能官人,則民無覦心,詩云,嗟我懷人,寘彼周行,能官人也,王及公侯,伯,子,男,甸,采,衛大夫,各居其列,所謂周行也。

鄭尉氏,司氏,之亂其餘盜在宋,鄭人以子西,伯有,子產,之故,納賂于宋,以馬四十乘,與師茷,師慧,三月,公孫黑為質焉,司城子罕以堵女父,尉翩,司齊,與之,良司臣而逸之,託諸季武子,武子寘諸卞,鄭人醢之,三人也,師慧過宋朝,將私焉,其相曰,朝也,慧曰,無人焉,相曰,朝也,何故無人,慧曰,必無人焉,若猶有人,豈其以千乘之相,易淫樂之矇,必無人焉故也,子罕聞之,固請而歸之。

夏,齊侯圍成,貳於晉故也,於是乎城成郛。

秋,邾人伐我南鄙,使告于晉,晉將為會,以討邾莒,晉侯有疾,乃止,冬,晉悼公卒,遂不克會。

鄭公孫夏如晉奔喪,子蟜送葬。

宋人或得玉,獻諸子罕,子罕弗受,獻玉者曰,以示玉人,玉人以為寶也,故敢獻之,子罕曰,我以不貪為寶爾,以玉為寶,若以與我,皆喪寶也,不若人有其寶,稽首而告曰,小人懷璧,不可以越鄉,納此以請死也,子罕寘諸其里,使玉人為之攻之,富而後使復其所。

十二月,鄭人奪堵狗之妻,而歸諸范氏。

十有六年,春,王正月,葬晉悼公。

三月,公會晉侯,宋公,衛侯,鄭伯,曹伯,莒子,邾子,薛伯,杞伯,小邾子,于溴梁,戊寅,大夫盟,晉人執莒子,邾子,以歸。

齊侯伐我北鄙。

夏,公至自會。

五月,甲子,地震。

叔老會鄭伯,晉荀偃,衛甯殖,宋人,伐許。

秋,齊侯伐我北鄙,圍郕。

大雩。

冬,叔孫豹如晉。

十六年,春,葬晉悼公,平公即位,羊舌肸為傅,張君臣為中軍司馬,祁奚,韓襄,欒盈,士鞅,為公族大夫,虞丘書為乘馬御,改服脩官,烝于曲沃,警守而下,會于溴梁,命歸侵田,以我故,執邾宣公,莒犁比公,且曰,通齊楚之使,晉侯與諸侯宴于溫,使諸大夫舞,曰,歌詩必類,齊高厚之詩不類,荀偃怒,且曰,諸侯有異志矣,使諸大夫盟高厚,高厚逃歸,於是叔孫豹,晉荀偃,宋向戌,衛甯殖,鄭公孫蠆,小邾之大夫,盟曰,同討不庭。

許男請遷于晉,諸侯遂遷許,許大夫不可,晉人歸諸侯,鄭子蟜聞將伐許,遂相鄭伯以從諸侯之師,穆叔從公,齊子帥師會晉荀偃,書曰,會鄭伯,為夷故也,夏,六月,次于棫林,庚寅,伐許,次于函氏。

晉荀偃,欒黶,帥師伐楚,以報宋揚梁之役,楚公子格帥師,及晉師戰于湛阪,楚師敗績,晉師遂侵方城之外,復伐許而還。

秋,齊侯圍郕,孟孺子速徼之,齊侯曰,是好勇,去之以為之名,速遂塞海陘而還。

冬,穆叔如晉聘,且言齊故,晉人曰,以寡君之未禘祀,與民之未息,不然,不敢忘,穆叔曰,以齊人之朝夕釋憾於敝邑之地,是以大請,敝邑之急,朝不及夕,引領西望曰,庶幾乎比執事之間,恐無及也,見中行獻子賦圻父,獻子曰,偃知罪矣,敢不從執事,以同恤社稷,而使魯及此,見范宣子,賦鴻鴈之卒章,宣子曰,丐在此,敢使魯無鳩乎。

十有七年,春,王二月,庚午,邾子牼卒。

宋人伐陳。

夏,衛石買帥師伐曹。

秋,齊侯伐我北鄙,圍桃。

高厚帥師伐我北鄙,圍防。

九月,大雩。

宋華臣出奔陳。

冬,邾人伐我南鄙。

十七年,春,宋莊朝伐陳,獲司徒卬,卑宋也。

衛孫蒯田于曹隧,飲馬于重丘,毀其瓶,重丘人閉門而詢之,曰,親逐而君,爾父為厲,是之不憂,而何以田為,夏,衛石買,孫蒯,伐曹,取重丘,曹人愬于晉,齊人以其未得志于我故,秋,齊侯伐我北鄙,圍桃,高厚圍臧紇于防,師自陽關逆臧孫,至于旅松,郰叔紇,臧疇,臧賈,帥甲三百,宵犯齊師,送之而復,齊師去之,齊人獲臧堅,齊侯使夙沙衛唁之,且曰無死,堅稽首曰,拜命之辱,抑君賜不終,姑又使其刑臣禮於士,以杙抉其傷而死。

冬,邾人伐我南鄙,為齊故也。

宋華閱卒,華臣弱皋比之室,使賊殺其宰華吳,賊六人以鈹殺諸盧門,合左師之後,左師懼曰,老夫無罪,賊曰,皋比私有討於吳,遂幽其妻,曰,畀余而大璧,宋公聞之,曰,臣也,不唯其宗室是暴,大亂宋國之政,必逐之,左師曰,臣也亦卿也,大臣不順,國之恥也,不如蓋之,乃舍之,左師為己短策,苟過華臣之門,必騁,十一月,甲午,國人逐瘈狗,瘈狗入於華臣氏,國人從之,華臣懼,遂奔陳。

宋皇國父為大宰,為平公築臺,妨於農功,子罕請俟農功之畢,公弗許,築者謳曰,澤門之皙,實興我役,邑中之黔,實慰我心,子罕聞之,親執扑,以行築者,而抶其不勉者,曰吾儕小人,皆有闔廬,以辟燥濕寒暑,今君為一臺而不速成,何以為役,謳者乃止,或問其故,子罕曰,宋國區區,而有詛有祝,禍之本也。

齊晏桓子卒,晏嬰麤縗斬,苴絰帶,杖,菅屨,食鬻,居倚廬,寢苫,枕草,其老曰,非大夫之禮也,曰,唯卿為大夫。

十有八年,春,白狄來。

夏,晉人執衛行人石買。

秋,齊師伐我北鄙。

冬,十月,公會晉侯,宋公,衛侯,鄭伯,曹伯,莒子,邾子,滕子,薛伯,杞伯,小邾子,同圍齊,曹伯負芻卒于師。

楚公子午帥師伐鄭。

十八年,春,白狄始來。

夏,晉人執衛行人石買于長子,執孫蒯于純留,為曹故也。

秋,齊侯伐我北鄙,中行獻子將伐齊,夢與厲公訟,弗勝,公以戈擊之,首隊於前,跪而戴之,奉之以走,見梗陽之巫皋,他日見諸道,與之言同,巫曰,今茲主必死,若有事於東方,則可以逞,獻子許諾,晉侯伐齊,將濟河,獻子以朱絲係玉二榖而禱曰,齊環怙恃其險,負其眾庶,棄好背盟,陵虐神主,曾臣彪將率諸侯以討焉,其官臣偃實先後之,苟捷有功,無作神羞,官臣偃無敢復濟,唯爾有神裁之,沈玉而濟。

冬,十月,會于魯濟,尋溴梁之言,同伐齊,齊侯禦諸平陰,塹防門而守之廣里,夙沙衛曰,不能戰,莫如守險,弗聽,諸侯之士門焉,齊人多死,范宣子告析文子曰,吾知子敢匿情乎,魯人,莒人,皆請以車千乘,自其鄉入,既許之矣,若入,君必失國,子盍圖之,子家以告公,公恐,晏嬰聞之曰,君固無勇,而又聞是,弗能久矣,齊侯登巫山以望晉師,晉人使司馬斥山澤之險,雖所不至,必旆而疏陳之,使乘車者,左實右偽,以斾先,輿曳柴而從之,齊侯見之,畏其眾也,乃脫歸,丙寅晦,齊師夜遁。師曠告晉侯曰,鳥烏之聲樂,齊師其遁,邢伯告中行伯曰,有班馬之聲,齊師其遁,叔向告晉侯曰,城上有烏,齊師其遁。十一月,丁卯,朔,入平陰,遂從齊師,夙沙衛連大車以塞隧而殿,殖綽,郭最,曰,子殿國師,齊之辱也,子姑先乎,乃代之殿,衛殺馬於隘以塞道,晉州綽及之,射殖綽中肩,兩矢夾脰。曰,止,將為三軍獲,不止,將取其衷,顧曰,為私誓,州綽曰,有如日,乃弛弓而自後縛之,其右具丙,亦舍兵而縛郭最,皆衿甲而縛,坐于中軍之鼓下,晉人欲逐歸者,魯衛請攻險,己卯,荀偃,士丐以中軍克京茲,乙酉,魏絳,欒盈,以下軍克邿,趙武,韓起,以上軍圍廬,弗克。十二月,戊戌,及秦周伐雍門之萩,范鞅門于雍門,其御追喜,以戈殺犬于門中,孟莊子斬其橁,以為公琴,己亥,焚雍門,及西郭,南郭,劉難,士弱,率諸侯之師,焚申池之竹木,壬寅,焚東郭,北郭,范鞅門于揚門,州綽門于東閭,左驂迫,還于東門中,以枚數闔,齊侯駕,將走郵棠,大子與郭榮扣馬曰,師速而疾,略也,將退矣,君何懼焉,且社稷之主,不可以輕,輕則失眾,君必待之,將犯之,大子抽劍斷鞅,乃止,甲辰,東侵及濰,南及沂。

鄭子孔欲去諸大夫,將叛晉,而起楚師以去之,使告子庚,子庚弗許,楚子聞之,使楊豚尹宜告子庚曰,國人謂不穀主社稷,而不出師,死不從禮,不穀即位,於今五年,師徒不出,人其以不穀為自逸而忘先君之業矣,大夫圖之,其若之何,子庚歎曰,君王其謂午懷安乎,吾以利社稷也,見使者,稽首而對曰,諸侯方睦於晉,臣請嘗之,若何,君而繼之,不可,收師而退,可以無害,君亦無辱,子庚帥師治兵於汾,於是子蟜,伯有,子張,從鄭伯伐齊,子孔,子展,子西守,二子知子孔之謀,完守入保,子孔不敢會楚師,楚師伐鄭,次於魚陵,右師城上棘,遂涉潁,次于旃然,蒍子馮,公子格,率銳師侵費滑,胥靡,獻于,雍梁,右回梅山,侵鄭東北,至于蟲牢而反,子庚門于純門,信于城下,而還涉於魚齒之下,甚雨及之,楚師多凍,役徒幾盡,晉人聞有楚師,師曠曰,不害,吾驟歌北風,又歌南風,南風不競,多死聲,楚必無功,董叔曰,天道多在西北,南師不時,必無功,叔向曰,在其君之德也。

十有九年,春,王正月,諸侯盟于祝柯,晉人執邾子,公至自伐齊,取邾田自漷水。

季孫宿如晉,葬曹成公。

夏,衛孫林父帥師伐齊。

秋,七月,辛卯,齊侯環卒。

晉士丐帥師侵齊,至穀,聞齊侯卒,乃還。

八月,丙辰,仲孫蔑卒。

齊殺其大夫高厚。

鄭殺其大夫公子嘉。

冬,葬齊靈公。

城西郛。

叔孫豹會晉士丐于柯。

城武城。

十九年,春,諸侯還自沂上,盟于督揚曰,大毋侵小,執邾悼公,以其伐我故。

遂次于泗上,疆我田,取邾田,自漷水,歸之于我,晉侯先歸,公享晉六卿于蒲圃,賜之三命之服,軍尉,司馬,司空,輿尉,候奄,皆受一命之服,賄荀偃東錦,加璧乘馬,先吳壽夢之鼎,荀偃癉疽,生瘍於頭,濟河,及著雍,病,目出,大夫先歸者皆反,士丐請見,弗內,請後,曰鄭甥可,二月,甲寅,卒,而視,不可含,宣子盥而撫之曰,事吳敢不如事主,猶視,欒懷子曰,其為未卒事於齊故也乎,乃復撫之曰,主苟終,所不嗣事于齊者,有如河,乃瞑受含,宣子出曰,吾淺之為丈夫也。

晉欒魴帥師從衛孫文子伐齊。

季武子如晉拜師,晉侯享之,范宣子為政,賦黍苗,季武子興,再拜稽首曰,小國之仰大國也,如百穀之仰膏雨焉,若常膏之,其天下輯睦,豈唯敝邑,賦六月,季武子以所得於齊之兵,作林鍾,而銘魯功焉,臧武仲謂季孫曰,非禮也,夫銘,天子令德,諸侯言時計功,大夫稱伐,今稱伐,則下等也,計功,則借人也,言時,則妨民多矣,何以為銘,且夫大伐小,取其所得以作彝器,銘其功烈,以示子孫,昭明德而懲無禮也,今將借人之力,以救其死,若之何銘之,小國幸於大國,而昭所獲焉,以怒之,亡之道也。

齊侯娶于魯,曰顏懿姬,無子,其姪鬷聲姬生光,以為大子,諸子,仲子,戎子,戎子嬖,仲子生牙,屬諸戎子,戎子請以為大子,許之,仲子曰,不可,廢常不祥,間諸侯難,光之立也,列於諸侯矣,今無故而廢之,是專黜諸侯,而以難犯不祥也,君必悔之,公曰,在我而已,遂東大子光,使高厚傅牙以為大子,夙沙衛為少傅,齊侯疾,崔杼微逆光,疾病而立之,光殺戎子,尸諸朝,非禮也,婦人無刑,雖有刑,不在朝市。夏,五月,壬辰,晦,齊靈公卒,莊公即位,執公子牙於句瀆之丘,以夙沙衛易已,衛奔高唐以叛。

晉士丐侵齊及穀,聞喪而還,禮也。

於四月,丁未,鄭公孫蠆卒,赴於晉,大夫范宣子言於晉侯,以其善於伐秦也,六月,晉侯請於王,王追賜之大路,使以行禮也。

秋,八月,齊崔杼殺高厚於灑藍,而兼其室,書曰,齊殺其大夫,從君於昏也。

鄭子孔之為政也專,國人患之,乃討西宮之難,與純門之師,子孔當罪,以其甲及子革,子良氏之甲守,甲辰,子展,子西,率國人伐之,殺子孔而分其室,書曰,鄭殺其大夫,專也,子然,子孔,宋子之子也,士子孔,圭媯之子也,圭媯之班,亞宋子而相親也,士子孔,亦相親也,僖之四年,子然卒,簡之元年,士子孔卒,司徒孔實相子革,子良,之室,三室如一,故及於難,子革,子良,出奔楚,子革,為右尹,鄭人使子展當國,子西聽政,立子產為卿。

齊慶封圍高唐弗克,冬,十一月,齊侯圍之,見衛在城上,號之乃下,問守備焉,以無備告,揖之,乃登,聞師將傅,食高唐也,殖綽,工僂,會,夜縋納師,醢衛于軍。

城西郛,懼齊也。

齊及晉平,盟于大隧,故穆叔會范宣子于柯,穆叔見叔向,賦載馳之四章,叔向曰,肸敢不承命,穆叔曰,齊猶未也,不可以不懼,乃城武城。

衛石共子卒,悼子不哀,孔成子曰,是謂蹙其本,必不有其宗。

二十年,春,王正月,辛亥,仲孫速會莒人盟于向。

夏,六月,庚申,公會晉侯,齊侯,宋公,衛侯,鄭伯,曹伯,莒子,邾子。

滕子,薛伯,杞伯,小邾子,盟于澶淵,秋,公至自會。

仲孫速帥師伐邾。

蔡殺其大夫公子燮,蔡公子履出奔楚。

陳侯之弟黃出奔楚,叔老如齊。

冬,十月,丙辰朔,日有食之,季孫宿如宋。

二十年,春,及莒平,孟莊子會莒人,盟于向,督揚之盟故也。

夏,盟于澶淵,齊成故也。

邾人驟至,以諸侯之事,弗能報也,秋,孟莊子伐邾以報之。

蔡公子燮欲以蔡之晉,蔡人殺之,公子履其母弟也,故出奔楚。

陳慶虎,慶寅,畏公子黃之偪,愬諸楚曰,與蔡司馬同謀,楚人以為討,公子黃出奔楚,初,蔡文侯欲事晉,曰,先君與於踐土之盟,晉不可棄,且兄弟也,畏楚不能行而卒,楚人使蔡無常,公子燮,求從先君以利蔡,不能而死,書曰,蔡殺其大夫公子燮,言不與民同欲也,陳侯之弟黃出奔楚,言非其罪也,公子黃將出奔,呼於國曰,慶氏無道,求專陳國,暴蔑其君,而去其親,五年不滅,是無天也。

齊子初聘于齊,禮也。

冬,季武子如宋,報向戌之聘也,褚師段逆之以受享,賦常棣之七章以卒,宋人重賄之,歸復命,公享之,賦魚麗之卒章,公賦南山有臺,武子去所曰,臣不堪也。

衛甯惠子疾,召悼子曰,吾得罪於君,悔而無及也,名藏在諸侯之策,曰孫林父,甯殖,出其君,君入則掩之,若能掩之,則吾子也,若不能。猶有鬼神,吾有餒而已,不來食矣。悼子許諾,惠子遂卒。

二十有一年,春,王正月,公如晉。

邾庶其以漆閭丘來奔。

夏,公至自晉。

秋,晉欒盈出奔楚。

九月,庚戌朔,日有食之。

冬,十月,庚辰朔,日有食之。

曹伯來朝。

公會晉侯,齊侯,宋公,衛侯,鄭伯,曹伯,莒子,邾子,于商任。

二十一年,春,公如晉拜師,及取邾田也。

邾庶其以漆閭丘來奔,季武子以公姑姊妻之,皆有賜於其從者,於是魯多盜,季孫謂臧武仲曰,子盍詰盜,武仲曰,不可詰也,紇又不能,季孫曰,我有四封,而詰其盜,何故不可,子為司寇,將盜是務去,若之何不能,武仲曰子召外盜而大禮焉,何以止吾盜,子為正卿,而來外盜,使紇去之,將何以能,庶其竊邑於邾以來,子以姬氏妻之,而與之邑,其從者皆有賜焉,若大盜禮焉,以君之姑姊與其大邑,其次皁牧輿馬,其小者衣裳劍帶,是賞盜也,賞而去之,其或難焉,紇也聞之,在上位者洒濯其心,壹以待人,軌度其信,可明徵也,而後可以治人,夫上之所為,民之歸也,上所不為,而民或為之,是以加刑罰焉,而莫敢不懲,若上之所為而民亦為之,乃其所也,又可禁乎,夏書曰,念茲在茲,釋茲在茲,名言茲在茲,允出茲在茲,惟帝念功,將謂由已壹也,信由已壹,而後功可念也,庶其非卿也,以地來,雖賤必書,重地也。

齊侯使慶佐為大夫,復討公子牙之黨,執公子買于句瀆之丘,公子鉏來奔,叔孫還奔燕。

夏,楚子庚卒,楚子使薳子馮為令尹,訪於申叔豫,叔豫曰,國多寵而王弱,國不可為也,遂以疾辭,方暑,闕地下冰而床焉,重繭衣裘,鮮食而寢,楚子使醫視之,復曰,瘠則甚矣,而血氣未動,乃使子南為令尹。

欒桓子娶於范宣子,生懷子。范鞅以其亡也,怨欒氏,故與欒盈為公族大夫,而不相能。桓子卒,欒祁與其老州賓通,幾亡室矣,懷子患之,祁懼其討也。愬諸宣子曰,盈將為亂,以范氏為死桓主而專政矣,曰,吾父逐鞅也,不怒,而以寵報之,又與吾同官而專之,吾父死而益富,死吾父而專於國,有死而已,吾蔑從之矣,其謀如是,懼害於主,吾不敢不言,范鞅為之徵,懷子好施,士多歸之,宣子畏其多士也,信之,懷子為下卿,宣子使城著而遂逐之,秋,欒盈出奔楚,宣子殺箕遺,黃淵,嘉父,司空靖,邴豫,董叔,邴師,申書,羊舌虎,叔羆,囚伯華,叔向,籍偃,人謂叔向曰,子離於罪,其為不知乎,叔向曰,與其死亡若何,詩曰,優哉游哉,聊以卒歲,知也,樂王鮒見叔向曰,吾為子請,叔向弗應,出不拜,其人皆咎叔向,叔向曰,必祁大夫,室老聞之曰,樂王鮒言於君,無不行,求赦吾子,吾子不許,祁大夫所不能也,而曰必由之,何也,叔向曰,樂王鮒,從君者也,何能行,祁大夫外舉不棄讎,內舉不失親,其獨遺我乎,詩曰,有覺德行,四國順之,夫子覺者也,晉侯問叔向之罪於樂王鮒,對曰,不棄其親,其有焉,於是祁奚老矣,聞之,乘馹而見宣子曰,詩曰,惠我無疆,子孫保之,書曰,聖有暮勳,明徵定保,夫謀而鮮過,惠訓不倦者,叔向有焉,社稷之固也,猶將十世宥之,以勸能者,今壹不免其身,其棄社稷,不亦惑乎,鯀殛而禹興,伊尹放大甲而相之,卒無怨色,管蔡為戮,周公右王,若之何其以虎也棄社稷,子為善,誰敢不勉,多殺何為,宣子說,與之乘以言諸公而免之,不見叔向而歸,叔向亦不告免焉而朝,初,叔向之母妒叔虎之母美而不使,其子皆諫其母,其母曰,深山大澤,實生龍蛇,彼美,余懼其生龍蛇以禍女,女敝族也,國多大寵,不仁人間之,不亦難乎,余何愛焉,使往視寢,生叔虎,美而有勇力,欒懷子嬖之,故羊舌氏之族及於難,欒盈過於周,周西鄙掠之,辭於行人曰,天子陪臣盈,得罪於王之守臣,將逃罪,罪重於郊甸,無所伏竄,敢布其死,昔陪臣書能輸力於王室,王施惠焉,其子黶不能保任其父之勞,大君若不棄書之力,亡臣猶有所逃,若棄書之力,而思黶之罪,臣戮餘也,將歸死於尉氏,不敢還矣,敢布四體,惟大君命焉,王曰,尤而效之,其又甚焉,使司徒禁掠欒氏者,歸所取焉,使候出諸轘轅。

冬,曹武公來朝,始見也。

會於商任,錮欒氏也,齊侯,衛侯,不敬,叔向曰,二君者必不免,會朝禮之經也,禮政之輿也,政身之守也,怠禮失政,失政不立,是以亂也。

知起,中行喜,州綽,邢蒯,出奔齊,皆欒氏之黨也,樂王鮒謂范宣子曰,盍反州綽,邢蒯,勇士也,宣子曰,彼欒氏之勇也,余何獲焉,王鮒曰,子為彼欒氏,乃亦子之勇也。

齊莊公朝指殖綽,郭最,曰,是寡人之雄也,州綽曰,君以為雄,誰敢不雄,然臣不敏,平陰之役,先二子鳴,莊公為勇爵,殖綽,郭最,欲與焉,州綽曰,東閭之役,臣左驂迫,還於門中,識其枚數,其可以與於此乎,公曰,子為晉臣也,對曰,臣為隸新,然二子者,譬於禽獸,臣食其肉,而寢處其皮矣。

二十有二年,春,王正月,公至自會。

夏,四月。

秋,七月,辛酉,叔老卒。

冬,公會晉侯,齊侯,宋公,衛侯,鄭伯,曹伯,莒子,邾子,薛伯,杞伯,小邾子,于沙隨,公至自會。

楚殺其大夫公子追舒。

二十二年,春,臧武仲如晉,雨,過御叔,御叔在其邑,將飲酒,曰,焉用聖人,我將飲酒而已,雨行,何以聖為,穆叔聞之,曰,不可使也,而傲使人,國之蠹也,令倍其賦。

夏,晉人徵朝于鄭,鄭人使少正公孫僑對曰,在晉先君悼公九年,我寡君於是即位,即位八月,而我先大夫子駟,從寡君以朝于執事,執事不禮於寡君,寡君懼因是行也,我二年六月,朝于楚,晉是以有戲之役,楚人猶競而申禮於敝邑,敝邑欲從執事,而懼為大尤,曰,晉其謂我不共有禮,是以不敢攜貳於楚,我四年三月,先大夫子蟜又從寡君以觀釁於楚,晉於是乎有蕭魚之役,謂我敝邑,邇在晉國,譬諸草木,吾臭味也,而何敢差池,楚亦不競,寡君盡其土,實重之以宗,器以受齊,盟遂帥群,臣隨于執,事以會歲終,貳於楚者,子侯,石盂,歸而討之,溴梁之明年,子蟜老矣,公孫夏從寡君以朝于君,見於嘗酎,與執燔焉,間二年,聞君將靖東夏,四月,又朝以聽事,期不朝之間,無歲不聘,無役不從,以大國政令之無常,國家罷病,不虞荐至,無日不惕,豈敢忘職,大國若安定之,其朝夕在庭,何辱命焉,若不恤其患,而以為口實,其無乃不堪任命,而翦為仇讎,敝邑是懼,其敢忘君命,委諸執事,執事實重圖之。

秋,欒盈自楚適齊,晏平仲言於齊侯曰,商任之會,受命於晉,今納欒氏,將安用之,小所以事大,信也,失信不立,君其圖之,弗聽,退告陳文子曰,君人執信,臣人執共,忠信篤敬,上下同之,天之道也,君自棄也,弗能久矣。

九月,鄭公孫黑肱有疾,歸邑于公,召室老宗人,立段而使黜官薄祭,祭以特羊,殷以少牢,足以共祀,盡歸其餘邑曰,吾聞之,生於亂世,貴而能貧,民無求焉,可以後亡,敬共事君,與二三子,生在敬戒,不在富也,己巳,伯張卒,君子曰善戒,詩曰,慎爾侯度,用戒不虞,鄭子張其有焉。

冬,會于沙隨,復錮欒氏也,欒盈猶在齊,晏子曰,禍將作矣,齊將伐晉,不可以不懼。

楚觀起有寵於令尹子南,未益祿而有馬數十乘,楚人患之,王將討焉,子南之子棄疾為王御士,王每見之必泣,棄疾曰,君三泣臣矣,敢問誰之罪也,王曰,令尹之不能,爾所知也,國將討焉,爾其居乎,對曰,父戮子居,君焉用之,洩命重刑,臣亦不為,王遂殺子南於朝,轘觀起於四竟,子南之臣謂棄疾,請徙子尸於朝,曰君臣有禮,唯二三子,三日,棄疾請尸,王許之,既葬,其徒曰,行乎,曰吾與殺吾父,行將焉入,曰然則臣王乎,曰棄父事讎,吾弗忍也,遂縊而死,復使薳子馮為令尹,公齮子為司馬,屈建為莫敖,有寵於薳子者八人,皆無祿而多馬,他日朝,與申叔豫言,弗應而退,從之,入於人中,又從之,遂歸,退朝見之,曰,子三困我於朝,吾懼不敢不見,吾過,子姑告我,何疾我也,對曰,吾不免是懼,何敢告子,曰,何故,對曰,昔觀起有寵於子南,子南得罪,觀起車裂,何故不懼,自御而歸,不能當道,至,謂八人者曰,吾見申叔夫子,所謂生死而肉骨也,知我者,如夫子則可,不然謂止。辭八人者,而後王安之。

十二月,鄭游販將歸晉,未出竟,遭逆妻者。

奪之,以館于邑,丁巳,其夫攻子明殺之,以其妻行,子展廢良而立大叔,曰國卿,君之貳也,民之主也,不可以苟,請舍子明之類,求亡妻者,使復其所,使游氏勿怨,曰,無昭惡也。

二十有三年,春,王二月,癸酉朔,日有食之。

三月,己巳,杞伯丐卒。

夏,邾畀我來奔。

葬杞孝公。

陳殺其大夫慶虎及慶寅。

陳侯之弟黃,自楚歸于陳。

晉欒盈復入于晉,入于曲沃。

秋,齊侯伐衛,遂伐晉。

八月,叔孫豹帥師救晉,次于雍榆。

己卯,仲孫速卒。

冬,十月,乙亥,臧孫紇出奔邾。

晉人殺欒盈。

齊侯襲莒。

二十三年,春,杞孝公卒,晉悼夫人喪之,平公不徹樂,非禮也,禮為鄰國闕。

陳侯如楚,公子黃愬二慶於楚,楚人召之,使慶樂往殺之,慶氏以陳叛,夏,屈建從陳侯圍陳,陳人城板隊而殺人,役人相命,各殺其長,遂殺慶虎,慶寅,楚人納公子黃,君子謂慶氏不義,不可肆也,故書曰,惟命不于常。

晉將嫁女于吳,齊侯使析歸父媵之,以藩載欒盈,及其士,納諸曲沃,欒盈夜見胥午而告之,對曰,不可,天之所廢,誰能興之,子必不免,吾非愛死也,知不集也,盈曰,雖然,因子而死,吾無悔矣,我實不天,子無咎焉,許諾,伏之而觴曲沃人,樂作,午言曰,今也得欒孺子何如,對曰,得主而為之死,猶不死也,皆歎,有泣者,爵行,又言,皆曰,得主何貳之有,盈出,遍拜之,四月,欒盈帥曲沃之甲,因魏獻子以晝入絳,初,欒盈佐魏莊子於下軍,獻子私焉,故因之,趙氏以原屏之難怨欒氏,韓趙方睦,中行氏以伐秦之役怨欒氏,而固與范氏和親,知悼子少而聽於中行氏,程鄭嬖於公,唯魏氏及七輿大夫與之,樂王鮒侍坐於范宣子,或告曰,欒氏至矣,宣子懼,桓子曰,奉君以走固宮,必無害也,且欒氏多怨,子為政,欒氏自外,子在位,其利多矣,既有利權,又執民柄,將何懼焉,欒氏所得,其唯魏氏乎,而可強取也,夫克亂在權,子無懈矣,公有姻喪,王鮒使宣子墨縗冒絰,二婦人輦以如公,奉公以如固宮,范鞅逆魏舒,則成列既乘,將逆欒氏矣,趨進曰,欒氏帥賊以入,鞅之父與二三子在君所矣,使鞅逆吾子,鞅請驂乘持帶,遂超乘,右撫劍,左援帶,命驅之出,僕請,鞅曰,之公,宣子逆諸階,執其手,賂之以曲沃,初,斐豹隸也,著於丹書,欒氏之力臣曰督戎,國人懼之,斐豹謂宣子曰,苟焚丹書,我殺督戎,宣子喜曰,而殺之,所不請於君焚丹書者,有如日,乃出豹而閉之,督戎從之,踰隱而待之,督戎踰入,豹自後擊而殺之,范氏之徒在臺後,欒氏乘公門,宣子謂鞅曰,矢及君屋死之,鞅用劍以帥卒,欒氏退,攝車從之,遇欒樂曰,樂免之,死將訟女於天,樂射之不中,又注則乘槐本而覆,或以戟鉤之,斷肘而死,欒魴傷,欒盈奔曲沃,晉人圍之。

秋,齊侯伐衛,先驅,穀榮御王孫揮,召揚為右,申驅,成秩御莒恆,申鮮虞之傅摯為右,曹開御戎,晏父戎為右,貳廣,上之登御邢公,盧蒲癸為右,啟,牢成御襄罷師,狼蘧疏為右,胠,商子車御侯朝,桓跳為右,大殿,商子游御夏之御寇,崔如為右,燭庸之越駟乘,自衛將遂伐晉,晏平仲曰,君恃勇力以伐盟主,若不濟,國之福也,不德而有功,憂必及君,崔杼諫曰,不可,臣聞之,小國間大國之敗而毀焉,必受其咎,君其圖之,弗聽,陳文子見崔武子曰,將如君何,武子曰,吾言於君,君弗聽也,以為盟主,而利其難,群臣若急,君於何有,子姑止之,文子退,告其人曰,崔子將死乎,謂君甚,而又過之,不得其死,過君以義,猶自抑也,況以惡乎,齊侯遂伐晉,取朝歌為二隊,入孟門,登大行,張武軍於熒庭,戍郫邵,封少水,以報平陰之役,乃還,趙勝帥東陽之師以追之,獲晏氂,八月叔孫豹帥師救晉,次于雍榆,禮也。

季武子無適子,公彌長,而愛悼子,欲立之,訪於申豐曰,彌與紇,吾皆愛之,欲擇才焉而立之,申豐趨退,歸,盡室將行,他日又訪焉,對曰,其然,將具敝車而行,乃止,訪於臧紇,臧紇曰,飲我酒,吾為子立之,季氏飲大夫酒,臧紇為客,既獻,臧孫命北面重席,新樽絜之,召悼子,降逆之,大夫皆起,及旅,而召公鉏,使與之齒,季孫失色,季氏以公鉏為馬正,慍而不出,閔子馬見之曰,子無然,禍福無門,唯人所召。為人子者,患不孝,不患無所。敬其父命,何常之有,若能孝敬,富倍季氏可也,姦回不軌,禍倍下民可也,公鉏然之,敬共朝夕,恪居官次,季孫喜,使飲己酒,而以具往,盡舍㫋,故公鉏氏富,又出為公左宰,孟孫惡臧孫,季孫愛之,孟氏之御騶豐點,好羯也,曰,從余言,必為孟孫,再三云,羯從之,孟莊子疾,豐點謂公鉏,苟立羯,請讎臧氏,公鉏謂季孫曰,孺子秩固其所也,若羯立,則季氏信有力於臧氏矣,弗應,己卯,孟孫卒,公鉏奉羯立于戶側,季孫至,入哭而出,曰,秩焉在,公鉏曰,羯在此矣,季孫曰,孺子長,公鉏曰,何長之有,唯其才也,且夫子之命也,遂立羯,秩奔邾,臧孫入哭,甚哀多涕,出,其御曰,孟孫之惡子也,而哀如是,季孫若死,其若之何,臧孫曰,季之愛我,疾疢也,孟孫之惡我,藥石也,美疢不如惡石,夫石猶生我,疢之美,其毒滋多,孟孫死,吾亡無日矣,孟氏閉門,告於季孫曰,臧氏將為亂,不使我葬,季孫不信,臧孫聞之戒,冬,十月,孟氏將辟,藉除於臧氏,臧孫使正夫助之,除於東門甲,從已而視之,孟氏又告季孫,季孫怒,命攻臧氏,乙亥,臧紇斬鹿門之關,以出奔,邾。初,臧宣叔娶于鑄,生賈及為而死,繼室以其姪,穆姜之姨子也。生紇,長於公宮,姜氏愛之,故立之,臧賈臧為出在鑄,臧武仲自邾使告臧賈,且致大蔡焉,曰,紇不佞,失守宗祧,敢告不弔,紇之罪不及不祀,子以大蔡納請,其可,賈曰,是家之禍也,非子之過也,賈聞命矣,再拜受龜,使為以納請,遂自為也,臧孫如防,使來告曰,紇非能害也,知不足也,非敢私請,苟守先祀,無廢二勳,敢不辟邑,乃立臧為,臧紇致防而奔齊,其人曰,其盟我乎,臧孫曰,無辭,將盟臧氏,季孫召外史掌惡臣,而問盟首焉,對曰,盟東門氏也,曰:毋或如東門遂。不聽公命,殺適立庶。盟叔孫氏也,曰,毋或如叔孫僑如欲廢國常,蕩覆公室,季孫曰,臧孫之罪,皆不及此,孟椒曰,盍以其犯門斬關,季孫用之,乃盟臧氏曰,無或如臧孫紇,干國之紀,犯門斬關。臧孫聞之曰,國有人焉,誰居?其孟椒乎?

晉人克欒盈于曲沃,盡殺欒氏之族黨,欒魴出奔宋,書曰,晉人殺欒盈,不言大夫,言自外也。

齊侯還自晉,不入,遂襲莒,門于且于,傷股而退,明日將復戰,期于壽舒,杞,殖華還,載甲夜入且于之隧,宿於莒郊,明日,先遇莒子於蒲侯氏,莒子重賂之,使無死,曰,請有盟,華周對曰,貪貨棄命,亦君所惡也,昏而受命,日未中而棄之,何以事君,莒子親鼓之,從而伐之,獲杞梁,莒人,行成,齊侯歸,遇杞梁之妻於郊,使弔之,辭曰,殖之有罪,何辱命焉,若免於罪,猶有先人之敝廬在,下妾不得與郊弔,齊侯弔諸其室。

齊侯將為臧紇田,臧孫聞之,見齊侯,與之言伐晉,對曰,多則多矣,抑君似鼠,夫鼠晝伏夜動,不穴於寢廟,畏人故也,今君聞晉之亂,而後作焉,寧將事之,非鼠如何,乃弗與田,仲尼曰,知之難也,有臧武仲之知,而不容於魯國,抑有由也,作不順而施不恕也,夏書曰,念茲在茲,順事恕施也。

二十有四年,春,叔孫豹如晉。

仲孫羯帥師侵齊。

夏,楚子伐吳。

秋,七月,甲子,朔,日有食之,既。

齊崔杼帥師伐莒。

大水。

八月,癸巳朔,日有食之。

公會晉侯,宋公,衛侯,鄭伯,曹伯,莒子,邾子,滕子,薛伯,杞伯,小邾子,于夷儀。

冬,楚子,蔡侯,陳侯,許男,伐鄭,公至自會。

陳鍼宜咎出奔楚。

叔孫豹如京師。

大饑。

二十四年,春,穆叔如晉,范宣子逆之問焉,曰,古人有言曰,死而不朽,何謂也,穆叔未對,宣子曰,昔丐之祖,自虞以上為陶唐氏,在夏為御龍氏,在商為豕韋氏,在周為唐杜氏,晉主夏盟為范氏,其是之謂乎,穆叔曰,以豹所聞,此之謂世祿,非不朽也,魯有先大夫曰臧文仲,既沒,其言立,其是之謂乎,豹聞之。大上有立德,其次有立功,其次有立言。雖久不廢,此之謂不朽,若夫保姓受氏,以守宗祊,世不絕祀,無國無之,祿之大者,不可謂不朽。

范宣子為政,諸侯之幣重,鄭人病之,二月,鄭伯如晉,子產寓書於子西,以告宣子曰,子為晉國,四鄰諸侯,不聞令德,而聞重幣,僑也惑之,僑聞君子長國家者,非無賄之患,而無令名之難,夫諸侯之賄,聚於公室,則諸侯貳,若吾子賴之,則晉國貳,諸侯貳則晉國壞,晉國貳則子之家壞,何沒沒也,將焉用賄,夫令名,德之輿也,德,國家之基也,有基無壞,無亦是務乎,有德則樂,樂則能久,《詩》云:「樂只君子,邦家之基。」有令德也,夫上帝臨女,無貳爾心,有令名也,夫恕,思以明德,則令名載而行之,是以遠至邇安,毋寧使人謂子,子實生我,而謂子浚我以生乎,象有齒以焚其身,賄也,宣子說,乃輕幣,是行也,鄭伯朝晉,為重幣故,且請伐陳也,鄭伯稽首,宣子辭,子西相曰,以陳國之介,恃大國而陵虐於敝邑,寡君是以請罪焉,敢不稽首。

孟孝伯侵齊,晉故也。

夏,楚子為舟師以伐吳,不為軍政,無功而還。

齊侯既伐晉而懼,將欲見楚子,楚子使薳啟彊如齊聘,且請期,齊社蒐軍實,使客觀之,陳文子曰,齊將有寇,吾聞之,兵不戢,必取其族。

秋,齊侯聞將有晉師,使陳無宇從薳啟彊如楚辭,且乞師,崔杼帥師送之,遂伐莒,侵介根,會于夷儀,將以伐齊,水不克。

冬,楚子伐鄭以救齊,門于東門,次于棘澤,諸侯還救鄭,晉侯使張骼,輔躒,致楚師,求御于鄭,鄭人卜宛射犬吉,子大叔戒之曰,大國之人,不可與也,對曰無有眾寡,其上一也,大叔曰,不然,部婁無松柏,二子在幄,坐射犬于外,既食而後食之,使御廣車而行,已皆乘乘車,將及楚師,而後從之乘,皆踞轉而鼓琴,近不告而馳之,皆取冑於櫜而冑,入壘皆下,搏人以投,收禽挾囚,弗待而出,皆超乘,抽弓而射,既免,復踞轉而鼓琴,曰,公孫同乘,兄弟也,故再不謀,對曰,曩者志入而已,今則怯也,皆笑曰,公孫之亟也。

楚子自棘澤還,使薳啟彊帥師送陳無宇。

吳人為楚舟師之役故,召舒鳩人,舒鳩人叛楚,楚子師于荒浦,使沈尹壽與師祁犁讓之,舒鳩子敬逆二子,而告無之,且請受盟,二子復命,王欲伐之,薳子曰,不可,彼告不叛,且請受盟,而又伐之,伐無罪也,姑歸息民,以待其卒,卒而不貳,吾又何求,若猶叛我,無辭,有庸,乃還。

陳人復討慶氏之黨,鍼宜咎出奔楚。

齊人城郟,穆叔如周聘,且賀城,王嘉其有禮也,賜之大路。

晉侯嬖程鄭,使佐下軍,鄭行人公孫揮如晉聘,程鄭問焉,曰,敢問降階何由,子羽不能對,歸以語然明,然明曰,是將死矣,不然將亡,貴而知懼,懼而思降,乃得其階,下人而已,又何問焉,且夫既登而求降階者,知人也,不在程鄭,其有亡釁乎,不然,其有惑疾,將死而憂也。

二十有五年,春,齊崔杼帥師伐我北鄙。

夏,五月,乙亥,齊崔杼弒其君光。

公會晉侯,宋公,衛侯,鄭伯,曹伯,莒子,邾子,滕子,薛伯,杞伯,小邾子,于夷儀。

六月,壬子,鄭公孫舍之帥師入陳。

秋,八月,己巳,諸侯同盟于重丘,公至自會。

衛侯入于夷儀。

楚屈建帥師滅舒鳩。

冬,鄭公孫夏帥師伐陳。

十有二月,吳子遏伐楚,門于巢,卒。

二十五年,春,齊崔杼帥師伐我北鄙,以報孝伯之師也,公患之,使告于晉,孟公綽曰,崔子將有大志,不在病我,必速歸,何患焉,其來也不寇,使民不嚴,異於他日,齊師徒歸。

齊棠公之妻,東郭偃之姊也,東郭偃臣崔武子,棠公死,偃御武子以弔焉,見棠姜而美之,使偃取之,偃曰,男女辨姓,今君出自丁,臣出自桓,不可,武子筮之,遇困之大過,史皆曰吉,示陳文子,文子曰,夫從風,風隕妻,不可聚也,且其繇曰,困于石,據于蒺梨,入于其宮,不見其妻,凶,困于石,往不濟也,據于蒺梨,可恃傷也,入于其宮,不見其妻,凶,無所歸也,崔子曰,嫠也何害,先夫當之矣,遂取之,莊公通焉,驟如崔氏,以崔子之冠賜人,侍者曰不可,公曰,不為崔子,其無冠乎,崔子因是,又以其間伐晉也,曰晉必將報,欲弒公以說于晉,而不獲間,公鞭侍人賈舉,而又近之,乃為崔子間公,夏,五月,莒子為且于之役故,莒子朝于齊,甲戌,饗諸北郭,崔子稱疾不視事,乙亥,公問崔子,遂從姜氏,姜入于室,與崔子自側戶出,公拊楹而歌,侍人賈舉止眾從者,而入閉門,甲興,公登臺而請,弗許,請盟,弗許,請自刃於廟,勿許,皆曰,君之臣杼疾病,不能聽命,近於公宮,陪臣干掫有淫者,不知二命,公踰牆,又射之,中股,反隊,遂弒之,賈舉,州綽,邴師,公孫敖,封具,鐸父,襄伊,僂堙,皆死,祝佗父祭於高唐,至復命,不說弁而死於崔氏,申蒯侍漁者,退謂其宰曰,爾以帑免,我將死,其宰曰,免,是反子之義也,與之皆死,崔氏殺鬷蔑于平陰,晏子立於崔氏之門外,其人曰,死乎,曰,獨吾君也乎哉,吾死也,曰,行乎,曰,吾罪也乎哉,吾亡也,曰,歸乎,曰,君死安歸,君民者,豈以陵民,社稷是主,臣君者,豈為其口實,社稷是養,故君為社稷死,則死之,為社稷亡,則亡之,若為己死而己亡,非其私暱,誰敢任之,且人有君而弒之,吾焉得死之,而焉得亡之,將庸何歸,門啟而入,枕尸股而哭,興,三踊而出,人謂崔子必殺之,崔子曰,民之望也,舍之得民,盧蒲癸奔晉,王何奔莒,叔孫宣伯之在齊也,叔孫還納其女於靈公,嬖,生景公,丁丑,崔杼立而相之,慶封為左相,盟國人於大宮曰,所不與崔慶者,晏子仰天歎曰,嬰所不唯忠於君,利社稷者是與,有如上帝,乃歃,辛巳,公與大夫及莒子盟,大史書曰,崔杼弒其君,崔子殺之,其弟嗣書,而死者二人,其弟又書,乃舍之,南史氏聞大史盡死,執簡以往,聞既書矣,乃還,閭丘嬰以帷縳其妻而載之,與申鮮虞乘而出,鮮虞推而下之曰,君昏不能匡,危不能救,死不能死,而知匿其暱,其誰納之,行及弇中,將舍,嬰曰,崔慶其追我。鮮虞曰:一與一,誰能懼我?遂舍,枕轡而寢,食馬而食,駕而行,出弇中,謂嬰曰,速驅之,崔慶之眾,不可當也,遂來奔,崔氏側莊公于北郭,丁亥,葬諸士孫之里,四翣,不蹕,下車七乘,不以兵甲。

晉侯濟自泮,會于夷儀,伐齊以報朝歌之役,齊人以莊公說,使隰鉏請成,慶封如師,男女以班,賂晉侯以宗器樂器,自六正,五吏,三十師,三軍之大夫,百官之正長,師旅,及處守者,皆有賂,晉侯許之,使叔向告於諸侯,公使子服惠伯對曰,君舍有罪,以靖小國,君之惠也,寡君聞命矣。

晉侯使魏舒,宛沒,逆衛侯,將使衛與之夷儀,崔子止其帑,以求五鹿。

初,陳侯會楚子伐鄭,當陳隧者,井堙木刊,鄭人怨之六月,鄭子展,子產,帥車七百乘伐陳,宵突陳城,遂入之,陳侯扶其大子偃師奔墓,遇司馬桓子曰,載余,曰將巡城,遇賈獲載其母妻,下之而授公車,公曰,舍而母,辭曰,不祥,與其妻扶其母以奔墓,亦免,子展命師無入公宮,與子產親御諸門,陳侯使司馬桓子賂以宗器,陳侯免,擁社,使其眾男女別而纍,以待於朝,子展執縶而見,再拜稽首,承飲而進獻,子美入,數俘而出,祝祓社,司徒致民,司馬致節,司空致地,乃還。

秋,七月,己巳,同盟于重丘,齊成故也。

趙文子為政,令薄諸侯之幣,而重其禮,穆叔見之,謂穆叔曰,自今以往,兵其少弭矣,齊崔慶新得政,將求善於諸侯,武也,知楚令尹,若敬行其禮,道之以文辭,以靖諸侯,兵可以弭。

楚薳子馮卒,屈建為令尹,屈蕩為莫敖,舒鳩人卒叛,楚令尹子木伐之,及離城,吳人救之,子木遽以右師先,子彊息桓,子捷,子駢,子盂,帥左師以退,吳人居其間七日,子彊曰,久將墊隘,隘乃禽也,不如速戰,請以其私卒誘之,簡師陳以待我,我克則進,奔則亦視之,乃可以免,不然,必為吳禽,從之,五人以其私卒,先擊吳師,吳師奔,登山以望,見楚師不繼,復逐之,傅諸其軍,簡師會之,吳師大敗,遂圍舒鳩,舒鳩潰,八月,楚滅舒鳩。

衛獻公入于夷儀。

鄭子產獻捷于晉,戎服將事,晉人問陳之罪。對曰:昔虞閼父為周陶正,以服事我先王。我先王賴其利器用也,與其神明,之後也,庸以元女大姬,配胡公而封之陳,以備三恪,則我周之自出,至于今是賴,桓公之亂,蔡人欲立其出,我先君莊公奉五父而立之,蔡人殺之,我又與蔡人奉戴厲公,至於莊宣,皆我之自立,夏氏之亂,成公播蕩,又我之自入,君所知也,今陳忘周之大德,蔑我大惠,棄我姻親,介恃楚眾,以憑陵我,敝邑,不可億逞,我是以有往年之告,未獲成命,則有我東門之役,當陳隧者,井堙木刊,敝邑大懼不競,而恥大姬,天誘其衷,啟敝邑之心,陳知其罪,授手于我,用敢獻功,晉人曰,何故侵小,對曰,先王之命,唯罪所在,各致其辟,且昔天子之地一圻,列國一同,自是以衰,今大國多數圻矣,若無侵小,何以至焉,晉人曰,何故戎服,對曰,我先君武莊為平桓卿士,城濮之役,文公布命曰,各復舊職,命我文公,戎服輔王,以授楚捷,不敢廢王命故也,士莊伯不能詰,復於趙文子,文子曰,其辭順,犯順不祥,乃受之,冬,十月,子展相鄭伯如晉,拜陳之功,子西復伐陳,陳及鄭平,仲尼曰,志有之,言以足志,文以足言,不言誰知其志。言之無文,行而不遠。晉為伯鄭入陳,非文辭不為功,慎辭也。

楚蒍掩為司馬,子木使庀賦,數甲兵,甲午,蒍掩書土田,度山林,鳩藪澤,辨京陵,表淳鹵,數疆潦,規偃豬,町原防,牧隰皋,井衍沃,量入脩賦,賦車,籍馬,賦車兵,徒卒,甲楯之數,既成,以授子木,禮也。

十二月,吳子諸樊伐楚,以報舟師之役,門于巢,巢牛臣曰,吳王勇而輕,若啟之,將親門,我獲射之,必殪,是君也死,彊其少安,從之,吳子門焉,牛臣隱於短牆以射之,卒。

楚子以滅舒鳩賞子木,辭曰,先大夫蒍子之功也,以與蒍掩。

晉程鄭卒,子產始知然明,問為政焉,對曰,視民如子,見不仁者誅之,如鷹鸇之逐鳥雀也,子產喜以語子大叔,且曰,他日吾見蔑之面而已,今吾見其心矣,子大叔問政於子產,子產曰,政如農功,日夜思之,思其始而成其終,朝夕而行之,行無越思,如農之有畔,其過鮮矣。

衛獻公自夷儀使與甯喜言,甯喜許之,大叔文子聞之曰,烏呼,詩所謂我躬不說,皇恤我後者,甯子可謂不恤其後矣,將可乎哉,殆必不可,君子之行,思其終也,思其復也,書曰,慎始而敬終,終以不困,詩曰,夙夜匪解,以事一人,今甯子視君,不如弈棋,其何,以免乎,弈者舉棋不定,不勝其耦,而況置君,而弗定乎,必不免矣,九世之卿族,一舉而滅之,可哀也哉,傳會于夷儀之歲,齊人城郟,其五月,秦晉為成,晉韓起如秦蒞盟,秦伯車如晉蒞盟,成而不結。

二十有六年,春,王二月,辛卯,衛甯喜弒其君剽,衛孫林父入于戚以叛,甲午,衛侯衎復歸于衛。

夏,晉侯使荀吳來聘。

公會晉人,鄭良霄,宋人,曹人,于澶淵。

秋,宋公殺其世子痤。

晉人執衛甯喜。

八月,壬午,許男甯卒于楚。

冬,楚子,蔡侯,陳侯,伐鄭。

葬許靈公。

二十六年,春,秦伯之弟鍼如晉脩成,叔向命召行人子員,行人子朱曰,朱也當御,三云,叔向不應,子朱怒曰,班爵同,何以黜朱於朝,撫劍從之,叔向曰,秦晉不和久矣,今日之事,幸而集,晉國賴之,不集,三軍暴骨,子員道二國之言無私,子常易子,姦以事君者,吾所能御也,拂衣從之,人救之,平公曰,晉其庶乎,吾臣之所爭者大,師曠曰,公室懼卑,臣不心競而力爭,不務德而爭善,私欲已侈,能無卑乎。

衛獻公使子鮮為復,辭,敬姒強命之,對曰,君無信,臣懼不免,敬姒曰,雖然,以吾故也,許諾,初,獻公使與甯喜言,甯喜曰,必子鮮在,不然,必敗,故公使子鮮,子鮮不獲命於敬姒,以公命與甯喜言曰,苟反,政由甯氏,祭則寡人,甯喜告蘧伯玉,伯玉曰,瑗不得聞君之出,敢聞其入,遂行,從近關出,告右宰穀,右宰穀曰,不可。獲罪於兩君,天下誰畜之。悼子曰,吾受命於先人,不可以貳,穀曰,我請使焉而觀之,遂見公於夷儀,反曰,君淹恤在外,十二年矣,而無憂色,亦無寬言,猶夫人也,若不已,死無日矣,悼子曰,子鮮在,右宰穀曰,子鮮在何益,多而能亡,於我何為,悼子曰,雖然,不可以已,孫文子在戚,孫嘉聘於齊,孫襄居守,二月庚寅,甯喜右宰穀伐孫氏,不克,伯國傷,甯子出舍於郊,伯國死,孫氏夜哭國人召甯子,甯子復攻孫氏,克之,辛卯,殺子叔及大子角,書曰,甯喜弒其君剽,言罪之在甯氏也,孫林父以戚如晉書曰,入于戚以叛,罪孫氏也,臣之祿,君實有之,義則進,否則奉身而退,專祿以周旋,戮也,甲午,衛侯入,書曰,復歸國,納之也,大夫逆於竟者,執其手而與之言道,逆者自車揖之,逆於門者頷之而已,公至,使讓大叔,文子曰,寡人淹恤在外,二三子皆使寡人,朝夕聞衛國之言。吾子獨不在寡人,古人有言曰非所怨勿怨寡人怨矣。對曰,臣知罪矣,臣不佞,不能負羈絏以從扞牧圉,臣之罪一也,有出者,有居者,臣不能貳,通外內之言以事君,臣之罪二也,有二罪,敢忘其死,乃行,從近關出,公使止之。

衛人侵戚東鄙,孫氏愬于晉,晉戍茅氏,殖綽伐茅氏,殺晉戍三百人,孫蒯追之,弗敢擊,文子曰,厲之不如,遂從衛師,敗之圉雍鉏,獲殖綽,復愬于晉。

鄭伯賞入陳之功,三月,甲寅,朔,享子展,賜之先路三命之服,先八邑,賜子產次路再命之服,先六邑,子產辭邑,曰,自上以下,隆殺,以兩,禮也,臣之位在四,且子展之功也,臣不敢及賞禮,請辭邑,公固予之,乃受三邑,公孫揮曰,子產其將知政矣,讓不失禮。

晉人為孫氏故,召諸侯,將以討衛也,夏,中行穆子來聘,召公也。

楚子,秦人,侵吳,及雩婁,聞吳有備而還,遂侵鄭,五月,至于城麇,鄭皇頡戍之,出與楚師戰,敗,穿封戌囚皇頡,公子圍與之爭之,正於伯州犁,伯州犁曰,請問於囚,乃立囚,伯州犁曰,所爭,君子也,其何不知,上其手曰,夫子為王子圍,寡君之貴介弟也,下其手曰,此子為穿封戌,方城外之縣尹也,誰獲子,囚曰,頡遇王子弱焉,戌怒,抽戈逐王子圍,弗及,楚人以皇頡歸,印堇父與皇頡戍城麇,楚人囚之,以獻於秦,鄭人取貨於印氏以請之,子大叔為令正,以為請,子產曰,不獲,受楚之功,而取貨於鄭,不可謂國,秦不其然,若曰拜君之勤鄭國,微君之惠,楚師其猶在敝邑之城下,其可,弗從,遂行,秦人不予,更幣,從子產,而後獲之。

六月,公會晉趙武,宋向戌,鄭良霄,曹人,于澶淵,以討衛,疆戚田,取衛西鄙懿氏六十,以與孫氏,趙武不書,尊公也,向戌不書,後也,鄭先宋,不失所也,於是衛侯會之,晉人執寧喜,北宮遺,使女齊以先歸,衛侯如晉,晉人執而囚之,於士弱氏,秋,七月,齊侯,鄭伯,為衛侯故如晉,晉侯兼享之,晉侯賦嘉樂,國景子相齊侯,賦蓼蕭,子展相鄭伯,賦緇衣,叔向命晉侯拜二君,曰,寡君敢拜齊君之安,我先君之宗祧也,敢拜鄭君之不貳也,國子使晏平仲私於叔向,曰,晉君宣其明德於諸侯,恤其患而補其闕,正其違而治其煩,所以為盟主也,今為臣執君,若之何,叔向告趙文子,文子以告晉侯,晉侯言衛侯之罪,使叔向告二君,國子賦轡之柔矣,子展賦將仲子兮,晉侯乃許歸衛侯,叔向曰,鄭七穆,罕氏其後亡者也,子展儉而壹。

初,宋芮司徒生女子,赤而毛,棄諸堤下,共姬之妾,取以入,名之曰棄,長而美,平公入夕,共姬與之食,公見棄也而視之尤,姬納諸御,嬖,生佐,惡而婉,大子痤美而很,合左師畏而惡之,寺人惠牆伊戾,為大子內師,而無寵,秋,楚客聘於晉,過宋,大子知之,請野享之,公使往,伊戾請從之,公曰,夫不惡女乎,對曰,小人之事君子也,惡之不敢遠,好之不敢近,敬以待命,敢有貳心乎,縱有共其外,莫共其內,臣請往也,遣之,至則欿用牲,加書徵之,而騁告公曰,大子將為亂,既與楚客盟矣,公曰,為我子,又何求,對曰,欲速,公使視之,則信有焉,問諸夫人與左師,則皆曰固聞之,公囚大子,大子曰,唯佐也能免我,召而使請,曰日中不來,吾知死矣,左師聞之,聒而與之語,過期,乃縊而死,佐為大子,公徐聞其無罪也,乃亨伊戾,左師見夫人之步馬者問之,對曰,君夫人氏也,左師曰,誰為君夫人,余胡弗知,圉人歸以告夫人,夫人使饋之錦與馬,先之以玉,曰,君之妾棄,使某獻,左師改命曰,君夫人,而後再拜稽首受之。

鄭伯歸自晉,使子西如晉聘,辭曰,寡君來煩執事,懼不免於戾,使夏謝不敏,君子曰,善事大國。

初楚伍參與蔡太師子朝友,其子伍舉與聲子相善也,伍舉娶於王子牟,王子牟為申公而亡,楚人曰,伍舉實送之,伍舉奔鄭,將遂奔晉,聲子將如晉,遇之於鄭郊,班荊相與食,而言復故,聲子曰,子行也,吾必復子,及宋向戌將平晉楚,聲子通使於晉,還如楚,令尹子木與之語,問晉故焉,且曰,晉大夫與楚孰賢,對曰,晉卿不如楚,其大夫則賢,皆卿材也,如杞梓皮革,自楚往也,雖楚有材,晉實用之,子木曰,夫獨無族姻乎,對曰,雖有,而用楚材實多,歸生聞之,善為國者,賞不僭而刑不濫,賞僭則懼及淫人,刑濫則懼及善人,若不幸而過,寧僭無濫,與其失善,寧其利淫,無善人,則國從之,詩曰,人之云亡,邦國殄瘁,無善人之謂也。故《夏書》曰:「與其殺不辜,寧失不經」。懼失善也,商頌有之曰,不僭不濫,不敢怠皇,命于下國,封建厥福,此湯所以獲天福也,古之治民者,勸賞而畏刑,恤民不倦,賞以春夏,刑以秋冬,是以將賞為之加膳,加膳則飫賜,此以知其勸賞也,將刑為之不舉,不舉則徹樂,此以知其畏刑也,夙興夜寐,朝夕臨政,此以知其恤民也,三者禮之大節也,有禮無敗,今楚多淫刑,其大夫逃死於四方,而為之謀主,以害楚國,不可救療,所謂不能也,子儀之亂,析公奔晉,晉人寘諸戎車之殿,以為謀主,繞角之役,晉將遁矣,析公曰,楚師輕窕,易震蕩也,若多鼓鈞聲以夜軍之,楚師必遁晉人從之,楚師宵潰,晉遂侵蔡襲沈,獲其君,敗申息之師於桑隧,獲申麗而還,鄭於是不敢南面,楚失華夏,則析公之為也,雍子之父兄譖雍子,君與大夫不善是也,雍子奔晉,晉人與之鄐,以為謀主,彭城之役,晉楚遇於靡角之谷,晉將遁矣,雍子發命於軍曰,歸老幼,反孤疾,二人役歸一人,簡兵蒐乘,秣馬蓐食,師陳焚次,明日將戰,行歸者,而逸楚囚,楚師宵潰,晉降彭城,而歸諸宋,以魚石歸,楚失東夷,子辛死之,則雍子之為也,子反與子靈爭夏姬,而雍害其事,子靈奔晉,晉人與之邢,以為謀主,扞禦北狄,通吳於晉,教吳叛楚,教之乘車,射御,驅侵,使其子狐庸,為吳行人焉,吳於是伐巢,取駕,克棘,入州來,楚罷於奔命,至今為患,則子靈之為也,若敖之亂,伯賁之子賁皇奔晉,晉人與之苗,以為謀主,鄢陵之役,楚晨壓晉軍而陳,晉將遁矣,苗賁皇曰,楚師之良在其中軍王族而已,若塞井夷灶,成陳以當之,欒范易行以誘之,中行二郤,必克二穆,吾乃四萃於其王族,必大敗之,晉人從之,楚師大敗,王夷師熸,子反死之,鄭叛吳興,楚失諸侯,則苗賁皇之為也,子木曰,是皆然矣,聲子曰,今又有甚於此,椒舉聚於申公子牟,子牟得戾而亡,君大夫謂椒舉,女實遣之,懼而奔鄭,引領南望曰,庶幾赦余,亦弗圖也,今在晉矣,晉人將與之縣,以比叔向,彼若謀害楚國,豈不為患,子木懼,言諸王,益其祿爵而復之,聲子使椒鳴逆之。

許靈公如楚,請伐鄭,曰,師不興,孤不歸矣,八月,卒于楚,楚子曰,不伐鄭,何以求諸侯,冬,十月,楚子伐鄭,鄭人將禦之,子產曰,晉楚將平,諸侯將和,楚王是故昧於一來,不如使逞而歸,乃易成也,夫小人之性,釁於勇,嗇於禍,以足其性,而求名焉者,非國家之利也,若何從之,子展說,不禦寇,十二月,乙酉,入南里,墮其城,涉於樂氏,門于師之梁,縣門發,獲九人焉,涉于氾而歸,而後葬許靈公。

衛人歸衛姬于晉,乃釋衛侯,君子是以知平公之失政也。

晉韓宣子聘于周,王使請事對曰,晉士起將歸時事於宰旅,無他事矣,王聞之曰,韓氏其昌阜於晉乎,辭不失舊。

齊人城郟之歲,其夏,齊烏餘以廩丘奔晉,襲衛羊角取之。遂襲我高魚,有大雨自其竇入。介于其庫,以登其城,克而取之,又取邑于宋,於是范宣子卒,諸侯弗能治也,及趙文子為政,乃卒治之,文子言於晉侯曰,晉為盟主,諸侯或相侵也,則討而使歸其地,今烏餘之邑,皆討類也,而貪之,是無以為盟主也,請歸之,公曰,諾,孰可使也,對曰,胥梁帶能無用師,晉侯使往。

二十有七年,春,齊侯使慶封來聘。

夏,叔孫豹會晉趙武,楚屈建,蔡公孫歸生,衛石惡,陳孔奐,鄭良霄,許人,曹人,于宋。

衛殺其大夫甯喜。

衛侯之弟鱄出奔晉。

秋,七月,辛巳,豹及諸侯之大夫盟于宋。

冬,十有二月,乙卯,朔,日有食之。

二十七年,春,胥梁帶使諸喪邑者,具車徒以受地,必周使烏餘具車徒以受封,烏餘以眾出使諸侯,偽效烏餘之封者,而遂執之,盡獲之,皆取其邑而歸諸侯,諸侯是以睦於晉。

齊慶封來聘,其車美,孟孫謂叔孫曰,慶季之車,不亦美乎,叔孫曰,豹聞之,服美不稱,必以惡終,美車何為,叔孫與慶封食不敬,為賦相鼠,亦不知也。

衛甯喜專,公患之,公孫免餘請殺之,公曰,微甯子不及此,吾與之言矣,事未可知,祗成惡名,止也,對曰,臣殺之,君勿與知,乃與公孫無地,公孫臣謀,使攻甯氏,弗克,皆死。公曰,臣也無罪,父子死余矣,夏,免餘復攻甯氏,殺甯喜及右宰穀,尸諸朝,石惡將會宋之盟,受命而出,衣其尸,枕之股而哭之,欲斂以亡,懼不免,且曰,受命矣,乃行。子鮮曰,逐我者出,納我者死,賞罰無章,何以沮勸。君失其信,而國無刑,不亦難乎。且鱄實使之,遂出奔晉,公使止之,不可,及河,又使止之,止使者而盟於河,託於木門,不鄉衛國而坐,木門大夫勸之仕,不可,曰,仕而廢其事,罪也,從之,昭吾所以出也,將誰愬乎,吾不可以立於人之朝矣,終身不仕,公喪之,如稅服終身,公與免餘邑六十,辭曰,唯卿備百邑,臣六十矣,下有上祿,亂也,臣弗敢聞,且甯子唯多邑故死,臣懼死之速及也,公固與之,受其半以為少師,公使為卿,辭曰,大叔儀不貳,能贊大事,君其命之,乃使文子為卿。

宋向戌善於趙文子,又善於令尹子木,欲弭諸侯之兵以為名,如晉,告趙孟,趙孟謀於諸大夫,韓宣子曰,兵,民之殘也,財用之蠹,小國之大菑也,將或弭之,雖曰不可,必將許之,弗許,楚將許之,以召諸侯,則我失為盟主矣,晉人許之,如楚,楚亦許之,如齊,齊人難之,陳文子曰,晉楚許之,我焉得已,且人曰,弭兵,而我弗許,則固攜吾民矣,將焉用之,齊人許之,告於秦,秦亦許之,皆告於小國,為會於宋,五月甲辰,晉趙武至於宋,丙午,鄭良霄至,六月丁未朔,宋人享趙文子,叔向為介,司馬置折俎,禮也,仲尼使舉是,禮也,以為多文辭,戊申,叔孫豹,齊慶封,陳須無,衛石惡,至,甲寅,晉荀盈從趙武至,丙辰,邾悼公至壬戌,楚公子黑肱先至,成言於晉,丁卯,宋戌如陳,從子木成言於楚,戊辰,滕成公至,子木謂向戌,請晉楚之從,交相見也,庚午,向戌復於趙孟,趙孟曰,晉,楚,齊,秦,匹也,晉之不能於齊,猶楚之不能於秦也,楚君若能使秦君辱於敝邑,寡君敢不固請於齊,壬申,左師復言於子木,子木使馹謁諸王,王曰,釋齊秦,他國請相見也,秋,七月,戊寅,左師至,是夜也,趙孟及子皙盟,以齊言,庚辰,子木至自陳,陳孔奐,蔡公孫歸生,至,曹許之大夫皆至,以藩為軍,晉楚各處其偏,伯夙謂趙孟曰,楚氛甚惡,懼難,趙孟曰,吾左還入於宋,若我何,辛巳,將盟於宋西門之外,楚人衷甲伯州犁,曰,合諸侯之師,以為不信,無乃不可乎,夫諸侯望信於楚,是以來服,若不信,是棄其所以服諸侯也,固請釋甲,子木曰,晉楚無信久矣,事利而已,苟得志焉,焉用有信,大宰退告人曰,令尹將死矣,不及三年,求逞志而棄信,志將逞乎,志以發言,言以出信,信以立志,參以定之,信亡何以及三,趙孟患楚衷甲,以告叔向,叔向曰,何害也,匹夫一為不信,猶不可,單斃其死,若合諸侯之卿,以為不信,必不捷矣,食言者不病,非子之患也,夫以信召人,而以僭濟之,必莫之與也,安能害我,且吾因宋以守病,則夫能致死,與宋致死,雖倍楚可也,子何懼焉,又不及是,曰,弭兵以召諸侯,而稱兵以害我,吾庸多矣,非所患也,季武子使謂叔孫以公命曰,視邾滕,既而齊人請邾,宋人請滕,皆不與盟,叔孫曰,邾,滕,人之私也,我列國也,何故視之,宋,衛,吾匹也,乃盟,故不書其族,言違命也,晉楚爭先,晉人曰,晉固為諸侯盟主,未有先晉者也,楚人曰,子言晉楚匹也,若晉常先,是楚弱也,且晉楚狎主諸侯之盟也久矣,豈專在晉,叔向謂趙孟曰,諸侯歸晉之德只,非歸其尸盟也,子務德,無爭先,且諸侯盟,小國固必有尸盟者,楚為晉細,不亦可乎,乃先楚人,書先晉,晉有信也,壬午,宋公兼享晉楚之大夫,趙孟為客,子木與之言,弗能對,使叔向侍言焉,子木亦不能對也,乙酉,宋公及諸侯之大夫盟于蒙門之外。子木問於趙孟曰:范武子之德何如?對曰:夫子之家事治,言於晉國無隱情,其祝史陳信於鬼神,無愧辭,子木歸以語王。王曰,尚矣哉,能歆神人,宜其光輔五君,以為盟主也,子木又語王曰,宜晉之伯也,有叔向以佐其卿,楚無以當之,不可與爭,晉荀寅遂如楚蒞盟。

鄭伯享趙孟于垂隴,子展,伯有,子西,子產,子大叔,二子石,從,趙孟曰,七子從君,以寵武也,請皆賦以卒君貺,武亦以觀七子之志,子展賦草蟲,趙孟曰,善哉,民之主也,抑武也不足以當之,伯有賦鶉之賁賁,趙孟曰,床笫之言不踰閾,況在野乎,非使人之所得聞也,子西賦黍苗之四章,趙孟曰,寡君在,武何能焉,子產賦隰桑,趙孟曰,武請受其卒章,子大叔賦野有蔓草,趙孟曰,吾子之惠也,印段賦蟋蟀,趙孟曰,善哉保家之主也,吾有望矣,公孫段賦桑扈,趙孟曰,匪交匪敖,福將焉往,若保是言也,欲辭福祿得乎,卒享,文子告叔向曰,伯有將為戮矣,詩以言志,志誣其上,而公怨之,以為賓榮,其能久乎,幸而後亡,叔向曰,然,已侈所謂,不及五稔者,夫子之謂矣,文子曰,其餘皆數世之主也,子展其後亡者也,在上不忘降,印氏其次也,樂而不荒,樂以安民,不淫以使之,後亡不亦可乎。

宋左師請賞,曰,請免死之邑,公與之邑六十,以示子罕,子罕曰,凡諸侯小國,晉楚所以兵威之,畏而後上下慈和,慈和而後能安靖其國家,以事大國,所以存也,無威則驕,驕則亂生,亂生必滅,所以亡也,天生五材,民並用之,廢一不可,誰能去兵,兵之設久矣,所以威不軌而昭文德也,聖人以興,亂人以廢,廢興存亡,昏明之術,皆兵之由也,而子求之,不亦誣乎,以誣道蔽諸侯,罪莫大焉,縱無大討,而又求賞,無厭之甚也,削而投之,左師辭邑,向氏欲攻司城,左師曰,我將亡,夫子存我,德莫大焉,又可攻乎,君子曰,彼己之子,邦之司直,樂喜之謂乎,何以恤我,我其收之,向戌之謂乎。

齊崔杼生成,及彊,而寡。娶東郭姜,生明,東郭姜以孤入,曰,棠無咎,與東郭偃相崔氏,崔成有病而廢之,而立明,成請老于崔,崔子許之,偃與?咎弗予,曰崔宗邑也,必在宗主,成與彊怒,將殺之,告慶封曰,夫子之身,亦子所知也,唯?咎與偃是從,父兄莫得進矣,大恐害夫子,敢以告,慶封曰,子姑退,吾圖之,告盧蒲嫳,盧蒲嫳曰,彼君之讎也,天或者將棄彼矣,彼實家亂,子何病焉,崔之薄,慶之厚也,他日又告,慶封曰,苟利夫子,必去之,難吾助女,九月,庚辰,崔成崔彊殺東郭偃,棠?咎,於崔氏之朝,崔子怒而出,其眾皆逃,求人使駕,不得,使圉人駕,寺人御而出,且曰崔氏有福,止余猶可,遂見慶封,慶封曰,崔慶一也,是何敢然,請為子討之,使盧蒲嫳帥甲以攻崔氏,崔氏堞其宮而守之,弗克,使國人助之,遂滅崔氏,殺成與彊而盡俘其家,其妻縊,嫳復命於崔子,且御而歸之,至則無歸矣,乃縊,崔明夜辟諸大墓,辛巳,崔明來奔,慶封當國。

楚薳罷如晉蒞盟,晉侯享之,將出,賦既醉,叔向曰,薳氏之有後於楚國也,宜哉,承君命,不忘敏,子蕩將知政矣,敏以事君,必能養民,政其焉往。

崔氏之亂,申鮮虞來奔,僕賃於野,以喪莊公,冬,楚人召之,遂如楚為右尹。

十一月,乙亥,朔,日有食之,辰在申,司曆過也,再閏失矣。

二十有八年,春,無冰。

夏,衛石惡出奔晉。

邾子來朝。

秋,八月,大雩。

仲孫羯如晉。

冬,齊慶封來奔。

十有一月,公如楚。

十有二月甲寅,天王崩。

乙未,楚子昭卒。

二十八年,春,無冰。梓慎曰,今茲宋鄭其饑乎,歲在星紀,而淫於玄枵,以有時菑,陰不堪陽。蛇乘龍。龍,宋鄭之星也。宋鄭必饑,玄枵,虛中也。枵,秏名也,土虛而民秏,不饑何為。

夏,齊侯,陳侯,蔡侯,北燕伯,杞伯,胡子,沈子,白狄,朝于晉,宋之盟故也,齊侯將行,慶封曰,我不與盟,何為於晉,陳文子曰,先事後賄,禮也,小事大,未獲事焉,從子,如志,禮也,雖不與盟,敢叛晉乎,重丘之盟,未可忘也,子其勸行。

衛人討甯氏之黨,故石惡出奔晉,衛人立其從之圃,以守石氏之祀,禮也。

邾悼公來朝,時事也。

秋,八月,大雩,旱也。

蔡侯歸自晉,入于鄭,鄭伯享之,不敬,子產曰,蔡侯其不免乎,日其過此也,君使子展迋勞於東門之外而傲,吾曰,猶將更之,今還受享而惰,乃其心也,君小國事大國,而惰傲以為己心,將得死乎,若不免,必由其子,其為君也,淫而不父,僑聞之,如是者恆有子禍。

孟孝伯如晉,告將為宋之盟故如楚也,蔡侯之如晉也,鄭伯使游吉如楚,及漢,楚人還之,曰,宋之盟,君實親辱,今吾子來,寡君謂吾子姑還,吾將使馹奔問諸晉,而以告,子大叔曰,宋之盟,君命將利小國,而亦使安定其社稷,鎮撫其民人,以禮承天之休,此君之憲令,而小國之望也,寡君是故使吉奉其皮幣,以歲之不易,聘於下執事,今執事有命曰,女何與政令之有,必使而君,棄而封守,跋涉山川,蒙犯霜露,以逞君心。小國將君是望,敢不唯命是聽。無乃非盟載之言,以闕君德,而執事有不利焉,小國是懼,不然,其何勞之敢憚,子大叔歸復命,告子展曰,楚子將死矣,不脩其政德,而貪昧於諸侯,以逞其願,欲久得乎,周易有之,在復之頤曰,迷復凶,其楚子之謂乎,欲復其願,而棄其本,復歸無所,是謂迷復,能無凶乎,君其往也,送葬而歸,以快楚心,楚不幾十年,未能恤諸侯也,吾乃休吾民矣,裨灶曰,今茲周王及楚子皆將死,歲棄其次,而旅於明年之次,以害鳥帑,周楚惡之。

九月,鄭游吉如晉,告將朝于楚,以從宋之盟,子產相鄭伯以如楚,舍不為壇,外僕言曰,昔先大夫相先君適四國,未嘗不為壇,自是至今,亦皆循之,今子草舍,無乃不可乎,子產曰,大適小,則為壇,小適大,苟舍而已,焉用壇,僑聞之,大適小,有五美,宥其罪戾,赦其過失,救其菑患,賞其德刑,教其不及,小國不困,懷服如歸,是故作壇以昭其功,宣告後人,無怠於德,小適大有五惡,說其罪戾,請其不足,行其政事,共其職貢,從其時命,不然則重其幣帛,以賀其福而弔其凶,皆小國之禍也,焉用作壇,以昭其禍,所以告子孫,無昭禍焉可也。

齊慶封好田而耆酒,與慶舍政,則以其內實,遷于盧蒲嫳氏,易內而飲酒數日國遷朝焉,使諸亡人得賊者,以告而反之,故反盧蒲癸,癸臣子之,有寵,妻之,慶舍之士,謂盧蒲癸曰,男女辨姓,子不辟宗,何也,曰,宗不余辟,余獨焉辟之,賦詩斷章,余取所求焉,惡識宗,癸言王何而反之,二人皆嬖,使執寢戈而先後之,公膳日雙雞,饔人竊更之以鶩,御者知之,則去其肉,而以其洎饋,子雅,子尾,怒,慶封告盧蒲嫳。盧蒲嫳曰,譬之如禽獸,吾寢處之矣,使析歸父告晏平仲。平仲曰:嬰之眾不足用也。知無能謀也,言弗敢出,有盟可也。子家曰,子之言云,又焉用盟,告北郭子車。子車曰,人各有以事君,非佐之所能也,陳文子謂桓子曰,禍將作矣,吾其何得,對曰,得慶氏之木百車於莊文子曰,可慎守也已,盧蒲癸,王何,卜,攻慶氏,示子之兆,曰,或卜攻讎,敢獻其兆,子之曰,克,見血,冬,十月,慶封田于萊,陳無宇從,丙辰,文子使召之,請曰,無宇之母疾病,請歸,慶季卜之,示之兆,曰,死,奉龜而泣,乃使歸,慶嗣聞之,曰,禍將作矣,謂子家速歸,禍作必於嘗,歸猶可及也,子家弗聽,亦無悛志,子息曰,亡矣,幸而獲在吳越,陳無宇濟水,而戕舟發梁,盧蒲姜謂癸曰,有事而不告我,必不捷矣,癸告之,姜曰,夫子愎,莫之止,將不出,我請止之,癸曰,諾,十一月,乙亥,嘗于大公之廟,慶舍蒞事,盧蒲姜告之,且止之,弗聽,曰,誰敢者,遂如公,麻嬰為尸,慶奊為上獻,盧蒲癸,王何,執寢戈,慶氏以其甲環公宮,陳氏,鮑氏,之圉人為優,慶氏之馬善驚,士皆釋甲束馬而飲酒,且觀優,至於魚里,欒高陳鮑之徒,介慶氏之甲,子尾抽桷擊扉三,盧蒲癸自後刺子之,王何以戈擊之,解其左肩,猶援廟桷動於甍,以俎壺投殺人而後死,遂殺慶繩麻嬰,公懼,鮑國曰,群臣為君故也陳須無以公歸,稅服而如內宮,慶封歸,遇告亂者,丁亥,伐西門,弗克,還伐北門,克之,入伐內宮,弗克,反陳于嶽,請戰弗許,遂來奔,獻車於季武子,美澤可以鑑,展莊叔見之,曰,車甚澤,人必瘁,宜其亡也,叔孫穆子食慶封,慶封氾祭,穆子不說,使工為之誦茅鴟,亦不知,既而齊人來讓,奔吳,吳句餘予之朱方,聚其族焉而居之,富於其舊,子服惠伯謂叔孫曰,天殆富淫人,慶封又富矣,穆子曰善人富謂之賞,淫人富謂之殃,天其殃之也,其將聚而殲旃。

癸巳,天王崩,未來赴,亦未書,禮也。

崔氏之亂,喪群公子,故鉏在魯,叔孫還在燕,賈在句瀆之丘,及慶氏亡,皆召之,具其器用,而反其邑焉,與晏子邶殿,其鄙六十,弗受,子尾曰,富,人之所欲也,何獨弗欲,對曰,慶氏之邑,足欲故亡,吾邑不足欲也,益之以邶殿,乃足欲,足欲,亡無日矣,在外,不得宰吾一邑,不受邶殿,非惡富也,恐失富也,且夫富如布帛之有幅焉,為之制度,使無遷也,夫民生厚而用利,於是乎正德以幅之,使無黜嫚,謂之幅利,利過則為敗,吾不敢貪多,所謂幅也,與北郭佐邑六十,受之,與子雅邑,辭多受少,與子尾邑,受而稍致之,公以為忠,故有寵,釋盧蒲嫳于北竟,求崔杼之尸,將戮之,不得,叔孫穆子曰,必得之,武王有亂臣十人,崔杼其有乎,不十人,不足以葬,既崔氏之臣曰,與我其拱璧,吾獻其柩,於是得之,十二月,乙亥,朔齊人選莊公殯于大寢,以其棺尸崔杼於市,國人猶知之,皆曰崔子也。

為宋之盟故,公及宋公,陳侯,鄭伯,許男,如楚,公過鄭,鄭伯不在伯有迋勞於黃崖,不敬,穆叔曰,伯有無戾於鄭,鄭必有大咎,敬,民之主也,而棄之,何以承守,鄭人不討,必受其辜。濟澤之阿。行潦之蘋藻,寘諸宗室,季蘭尸之,敬也,敬可棄乎,及漢,楚康王卒,公欲反,叔仲昭伯曰,我楚國之為,豈為一人行也,子服惠伯曰,君子有遠慮,小人從邇,飢寒之不恤,誰遑其後,不如姑歸也,叔孫穆子曰,叔仲子,專之矣,子服子,始學者也,榮成伯曰,遠圖者,忠也,公遂行,宋向戌曰,我一人之為,非為楚也,飢寒之不恤,誰能恤楚,姑歸而息民,待其立君而為之備,宋公遂反。

楚屈建卒,趙文子喪之如同盟,禮也。

王人來告喪,問崩日,以甲寅告,故書之,以徵過也。

二十有九年,春,王正月,公在楚。

夏,五月,公至自楚。

庚午,衛侯衎卒。

閽弒吳子餘祭。

仲孫羯會晉荀盈,齊高止,宋華定,衛世叔儀,鄭公孫段,曹人,莒人,滕人,薛人,小邾人,城杞。

晉侯使士鞅來聘。

杞子來盟。

吳子使札來聘。

秋,九月,葬衛獻公。

齊高止出奔北燕。

冬,仲孫羯如晉。

二十九年,春,王正月,公在楚,釋不朝,正于廟也,楚人使公親襚,公患之,穆叔曰,袚殯而襚,則布幣也,乃使巫以桃茢先袚殯,楚人弗禁,既而悔之。

二月,癸卯,齊人葬莊公於北郭。

夏,四月,葬楚康王,公及陳侯,鄭伯,許男,送葬,至於西門之外,諸侯之大夫,皆至于墓,楚郟敖即位,王子圍為令尹,鄭行人子羽曰,是謂不宜,必代之昌,松柏之下,其草不殖,公還及方城,季武子取卞,使公冶問,璽書追而與之,曰,聞守卞者將叛,臣帥徒以討之,既得之矣,敢告,公冶致使而退,及舍而後聞取卞,公曰,欲之而言叛,祇見疏也,公謂公冶曰,吾可以入乎,對曰,君實有國,誰敢違君,公與公冶冕服,固辭,強之而後受,公欲無入,榮成伯賦式微,乃歸,五月,公至自楚,公冶致其邑於季氏,而終不入焉,曰,欺其君,何必使余,季孫見之,則言季氏如他日,不見,則終不言季氏及疾,聚其臣曰,我死,必無以冕服斂,非德賞也,且無使季氏葬我。

葬靈王,鄭上卿有事,子展使印段往,伯有曰,弱,不可,子展曰,與其莫往,弱不猶愈乎,詩云,王事靡盬,不遑啟處,東西南北,誰敢寧處,堅事晉楚,以蕃王室也,王事無曠,何常之有,遂使印段如周。

吳人伐楚,獲俘焉,以為閽,使守舟,吳子餘祭觀舟,閽以刀弒之。

鄭子展卒,子皮即位,於是鄭饑而未及麥,民病,子皮以子展之命,餼國人粟,戶一鍾,是以得鄭國之民,故罕氏常掌國政,以為上卿,宋司城子罕聞之,曰,鄰於善,民之望也,宋亦饑,請於平公,出公粟以貸,使大夫皆貸,司城氏貸而不書,為大夫之無者貸,宋無飢人,叔向聞之,曰,鄭之罕,宋之樂,其後亡者也,二者其皆得國乎,民之歸也,施而不德,樂氏加焉,其以宋升降乎。

晉平公,杞出也,故治杞,六月,知悼子合諸侯之大夫以城杞,孟孝伯會之,鄭子大叔與伯石往,子大叔見大叔文子,與之語,文子曰,甚乎其城杞也,子大叔曰,若之何哉,晉國不恤周宗之闕,而夏肄是屏,其棄諸姬,亦可知也已,諸姬是棄,其誰歸之,吉也聞之,棄同即異,是謂離德,詩曰,協比其鄰,昏姻孔云,晉不鄰矣,其誰云之。

齊高子容,與宋司徒,見知伯,女齊相禮,賓出,司馬侯言於知伯曰,二子皆將不免,子容專,司徒侈,皆亡家之主也,知伯曰,何如,對曰,專則速及,侈將以其力斃,專則人實斃之,將及矣。

范獻子來聘,拜城杞也,公享之,展莊叔執幣,射者三耦,公臣不足,取於家臣,家臣,展瑕,展玉父,為一耦,公臣,公巫,召伯仲,顏莊叔,為一耦,鄫鼓父,黨叔,為一耦。

晉侯使司馬女叔侯來治杞田,弗盡歸也,晉悼夫人慍曰,齊也取貨,先君若有知也,不尚取之,公告叔侯,叔侯曰,虞,虢,焦,滑,霍,揚,韓,魏,皆姬姓也,晉是以大,若非侵小,將何所取,武獻以下,兼國多矣,誰得治之,杞,夏餘也,而即東夷,魯,周公之後也,而睦於晉,以杞封魯,猶可,而何有焉,魯之於晉也,職貢不乏,玩好時至,公卿大夫,相繼於朝,史不絕書,府無虛月,如是可矣,何必瘠魯以肥杞,且先君而有知也,毋寧夫人,而焉用老臣。

杞文公來盟,書曰子,賤之也。

吳公子札來聘,見叔孫穆子,說之,謂穆子曰,子其不得死乎,好善而不能擇人,吾聞君子務在擇人,吾子為魯宗卿,而任其大政,不慎舉,何以堪之,禍必及子,請觀於周樂,使工為之歌周南召南,曰,美哉,始基之矣,猶未也,然勤而不怨矣,為之歌邶,鄘,衛,曰,美哉,淵乎,憂而不困者也。吾聞衛康叔武公之德如是,是其衛風乎,為之歌王,曰,美哉思而不懼,其周之東乎,為之歌鄭,曰,美哉,其細已甚,民弗堪也,是其先亡乎,為之歌齊。曰,美哉,泱泱乎,大風也哉,表東海者,其大公乎,國未可量也,為之歌豳,曰,美哉,蕩乎,樂而不淫,其周公之東乎,為之歌秦,曰,此之謂夏聲,夫能夏,則大,大之至乎其周之舊也,為之歌魏,曰,美哉,渢楓乎,大而婉,險而易,行以德輔,此則明主也,為之歌唐。曰,思深哉,其有陶唐氏之遺民乎,不然,何憂之遠也,非令德之後,誰能若是,為之歌陳。曰,國無主,其能久乎,自鄶以下,無譏焉。為之歌小雅曰,美哉,思而不貳,怨而不言,其周德之衰乎,猶有先王之遺民焉,為之歌大雅,曰,廣哉,熙熙乎,曲而有直體,其文王之德乎,為之歌頌,曰,至矣哉,直而不倨,曲而不屈,邇而不偪,遠而不攜,遷而不淫,復而不厭,哀而不愁,樂而不荒,用而不匱,廣而不宣,施而不費,取而不貪,處而不底,行而不流,五聲和,八風平,節有度,守有序,盛德之所同也,見舞象箾南籥者,曰,美哉,猶有憾,見舞大武者,曰,美哉,周之盛也,其若此乎,見舞韶濩者,曰,聖人之弘也,而猶有慚德,聖人之難也,見舞大夏者,曰,美哉,勤而不德,非禹其誰能脩之,見舞韶箾者,曰,德至矣哉,大矣,如天之無不幬也,如地之無不載也,雖甚盛德,其蔑以加於此矣,觀止矣,若有他樂,吾不敢請已,其出聘也,通嗣君也,故遂聘于齊,說晏平仲,謂之曰,子速納邑與政,無邑無政,乃免於難,齊國之政,將有所歸,未獲所歸,難未歇也,故晏子因陳桓子以納政與邑,是以免於欒高之難,聘於鄭,見子產,如舊相識,與之縞帶,子產獻紵衣焉,謂子產曰,鄭之執政侈,難將至矣,政必及子,子為政,慎之以禮,不然,鄭國將敗,適衛,說蘧瑗,史狗,史鰌,公子荊,公叔發,公子朝,曰,衛多君子,未有患也,自衛如晉,將宿於戚,聞鍾聲焉,曰,異哉,吾聞之也,辯而不德,必加於戮,夫子獲罪於君以在此,懼猶不足,而又何樂,夫子之在此也,猶燕之巢於幕上,君又在殯,而可以樂乎,遂去之,文子聞之,終身不聽琴瑟,適晉說趙文子,韓宣子,魏獻子,曰,晉國其萃於三族乎,說叔向,將行,謂叔向曰,吾子勉之,君侈而多良,大夫皆富,政將在家,吾子好直,必思自免於難。

秋九月,齊公孫蠆,公孫灶,放其大夫高止於北燕,乙未出,書曰出奔,罪高止也,高止好以事自為功,且專,故難及之。

冬孟孝伯如晉,報范叔也,為高氏之難故,高豎以盧叛,十月,庚寅,閭丘嬰帥師圍盧,高豎曰,苟請高氏有後,請致邑,齊人立敬仲之曾孫酀,良敬仲也,十一月,乙卯,高豎致盧而出奔晉,晉人城綿而寘旃。

鄭伯有使公孫黑如楚,辭曰,楚鄭方惡而使余往,是殺余也,伯有曰,世行也,子皙曰,可則往,難則已,何世之有,伯有將強使之,子皙怒,將伐伯有氏,大夫和之,十二月,己巳。鄭大夫盟於伯有氏,裨諶曰,是盟也,其與幾何。詩曰:君子屢盟,亂是用長,今是長,亂之道也,禍未歇也。必三年而後能紓然明曰,政將焉往,裨諶曰,善之代不善,天命也,其焉辟,子產舉不踰等,則位班也,擇善而舉,則世隆也,天又除之,奪伯有魄,子西即世,將焉辟之,天禍鄭久矣,其必使子產息之,乃猶可以戾,不然,將亡矣。

三十年,春,王正月,楚子使薳罷來聘。

夏,四月,蔡世子般弒其君固。

五月,甲午,宋災。

宋伯姬卒。

天王殺其弟佞夫。

王子瑕奔晉。

秋,七月,叔弓如宋,葬宋共姬。

鄭良霄出奔許,自許入于鄭,鄭人殺良霄。

冬,十月,葬蔡景公。

晉人,齊人,宋人,衛人,鄭人,曹人,莒人,邾人,滕人,薛人,杞人,小邾人,會于澶淵,宋災故。

三十年,春,王正月,楚子使薳罷來聘,通嗣君也,穆叔問王子之為政何如,對曰吾儕小人,食而聽事,猶懼不給命,而不免於戾,焉與知政,固問焉,不告,穆叔告大夫曰,楚令尹將有大事子蕩將與焉,助之匿其情矣。

子產相鄭伯以如晉,叔向問鄭國之政焉,對曰,吾得見與否,在此歲也,駟良方爭,未知所成若有所成,吾得見,乃可知也,叔向曰,不既和矣乎,對曰,伯有侈而愎,子皙好在人上,莫能相下也,雖其和也,猶相積惡也,惡至無日矣。

三月,癸未,晉悼夫人食輿人之城杞者,絳縣人或年長矣,無子,而往與於食,有與疑年,使之年,曰臣小人也,不知紀年,臣生之歲,正月甲子朔,四百有四十五,甲子矣,其季於今,三之一也,吏走問諸朝,師曠曰,魯叔仲惠伯會郤成子于承匡之歲也,是歲也,狄伐魯,叔孫莊叔於是乎敗狄于鹹,獲長狄僑如,及虺也豹也,而皆以名其子,七十三年矣,史趙曰,亥有二首六身,下二如身,是其日數也,士文伯曰,然則二萬二千六百有六旬也,趙孟問其縣大夫,則其屬也,召之而謝過焉,曰,武不才,任君之大事。以晉國之多虞,不能由吾子。使吾子辱在泥塗久矣,武之罪也,敢謝不才,遂仕之,使助為政,辭以老,與之田,使為君復陶,以為絳縣師,而廢其輿尉,於是魯使者在晉,歸以語諸大夫,季武子曰,晉未可媮也,有趙孟以為大夫,有伯瑕以為佐,有史趙師曠而咨度焉,有叔向女齊以師保其君,其朝多君子,其庸可媮乎,勉事之而後可。

夏,四月,己亥,鄭伯及其大夫盟,君子是以知鄭難之不已也。

蔡景侯為大子般娶于楚,通焉,大子弒景侯。

初,王儋季卒,其子括將見王而歎,單公子愆期為靈王御士,過諸廷,聞其歎而言曰,烏乎,必有此夫,入以告王,且曰,必殺之,不慼而願大,視躁而足高,心在他矣,不殺必害,王曰,童子何知,及靈王崩,儋括欲立王子佞夫,佞夫弗知,戊子,儋括圍蒍,逐成愆,成愆奔平畤,五月,癸巳,尹言多,劉毅,單蔑,甘過,鞏成,殺佞夫,括瑕廖奔晉,書曰,天王殺其弟佞夫,罪在王也。

或呌于宋大廟曰,譆譆出出,鳥鳴于亳社,如曰譆譆,甲午,宋大災,宋伯姬卒,待姆也,君子謂宋共姬女而不婦,女待人,婦義事也。

六月,鄭子產如陳蒞盟,歸復命,告大夫曰,陳亡國也,不可與也,聚禾粟,繕城郭,恃此二者而不撫其民,其君弱植,公子侈,大子卑,大夫敖,政多門,以介於大國,能無亡乎,不過十年矣。

秋,七月,叔弓如宋,葬共姬也。

鄭伯有耆酒,為窟室,而夜飲酒,擊鍾焉,朝至未己,朝者曰,公焉在,其人曰,吾公在壑谷,皆自朝布路而罷,既而朝,則又將使子皙如楚,歸而飲酒,庚子,子皙以駟氏之甲,伐而焚之,伯有奔雍梁,醒而後知之,遂奔許,大夫聚謀,子皮曰,仲虺之志云,亂者取之,亡者侮之,推亡固存,國之利也,罕,駟,豐,同生,伯有汏侈,故不免,人謂子產,就直助彊,子產曰,豈為我徒,國之禍難,誰知所儆,或主彊直,難乃不生,姑成吾所,辛丑,子產斂伯有氏之死者而殯之,不及謀而遂行印段從之,子皮止之,眾曰,人不我順,何止焉,子皮曰,夫子禮於死者,況生者乎,遂自止之,壬寅,子產入,癸卯,子石入,皆受盟于子皙氏,乙巳,鄭伯及其大夫盟于大宮,盟國人于師之梁之外,伯有聞鄭人之盟己也,怒,聞子皮之甲,不與攻己也,喜,曰,子皮與我矣,癸丑,晨,自墓門之瀆入,因馬師頡介于襄庫,以伐舊北門,駟帶率國人以伐之,皆召子產,子產曰兄弟而及此,吾從天所與,伯有死於羊肆,子產襚之,枕之股而哭之,斂而殯諸,伯有之臣在市側者,既而葬諸斗城,子駟氏欲攻子產,子皮怒之,曰,禮,國之幹也,殺有禮,禍莫大焉,乃止,於是游吉如晉還,聞難不入,復命于介,八月,甲子,奔晉,駟帶追之,及酸棗,與子上盟,用兩珪質于河,使公孫肸入盟大夫,己巳,復歸,書曰,鄭人殺良霄,不稱大夫,言自外入也,於子蟜之卒也,將葬,公孫揮與裨灶晨會事焉,過伯有氏,其門上生莠,子羽曰,其莠猶在乎,於是歲在降婁,降婁中而旦,裨灶指之曰,猶可以終歲,歲不及此次也已,及其亡也,歲在娵訾之口,其明年,乃及降婁,僕展從伯有,與之皆死,羽頡出奔晉,為任大夫,雞澤之會,鄭樂成奔楚,遂適晉,羽頡因之,與之比而事趙文子,言伐鄭之說焉,以宋之盟故,不可,子皮以公孫鉏為馬師。

楚公子圍殺大司馬蒍掩而取其室,申無宇曰,王子必不免,善人,國之主也,王子相楚國,將善是封,殖而虐之,是禍國也,且司馬令尹之偏,而王之四體也,絕民之主,去身之偏,艾王之體,以禍其國,無不祥大焉,何以得免。

為宋災故,諸侯之大夫會,以謀歸宋財,冬,十月,叔孫豹會晉趙武,齊公孫蠆,宋向戌,衛北宮佗,鄭罕虎,及小邾之大夫,會于澶淵,既而無歸於宋,故不書其人,君子曰,信其不可不慎乎,澶淵之會,卿不書,不信也夫,諸侯之上卿,會而不信,寵名皆棄,不信之不可也如是,詩曰,文王陟降,在帝左右,信之謂也,又曰,淑慎爾止,無載爾偽,不信之謂也,書曰,某人某人會于澶淵,宋災故,尤之也,不書魯大夫,諱之也。

鄭子皮授子產政,辭曰,國小而偪,族大寵多,不可為也,子皮曰,虎帥以聽,誰敢犯子,子善相之,國無小,小能事大,國乃寬,子產為政,有事伯石,賂與之邑,子大叔曰,國皆其國也,奚獨賂焉,子產曰,無欲實難,皆得其欲,以從其事,而要其成,非我有成,其在人乎,何愛於邑,邑將焉往,子大叔曰,若四國何,子產曰,非相違也,而相從也,四國何尤焉,鄭書有之曰,安定國家,必大焉先,姑先安大,以待其所歸,既伯石懼而歸邑,卒與之,伯有既死,使大史命伯石為卿,辭,大史退,則請命焉,復命之,又辭,如是三,乃受策入拜,子產是以惡其為人也,使次己位,子產使都鄙有章,上下有服,田有封洫,廬井有伍,大人之忠儉者,從而與之,泰侈者因而斃之,豐卷將祭,請田焉,弗許,曰,唯君用鮮,眾給而已,子張怒,退而徵役,子產奔晉,子皮止之,而逐豐卷,豐卷奔晉,子產請其田里,三年而復之,反其田里,及其入焉,從政一年,輿人誦之曰,取我衣冠而褚之,取我田疇而伍之,孰殺子產,吾其與之,及三年,又誦之曰,我有子弟,子產誨之,我有田疇,子產殖之,子產而死,誰其嗣之。

三十有一年,春,王正月。

夏,六月,辛巳,公薨于楚宮。

秋,九月,癸巳,子野卒。

己亥,仲孫羯卒。

冬,十月,滕子來會葬。

癸酉,葬我君襄公。

十有一月,莒人弒其君密州。

三十一年,春,王正月,穆叔至自會,見孟孝伯,語之曰,趙孟將死矣,其語偷,不似民主,且年未盈五十,而諄諄焉如八九十者,弗能久矣,若趙孟死,為政者其韓子乎,吾子盍與季孫言之,可以樹善,君子也,晉君將失政矣,若不樹焉,使早備魯,既而政在大夫,韓子懦弱,大夫多貪,求欲無厭,齊楚未足與也,魯其懼哉,孝伯曰,人生幾何,誰能無偷,朝不及夕,將安用樹,穆叔出而告人曰,孟孫將死矣,吾語諸趙孟之偷也,而又甚焉,又與季孫語晉故,季孫不從,及趙文子卒,晉公室卑,政在侈家,韓宣子為政,不能圖諸侯,魯不堪晉求,讒慝弘多,是以有平丘之會。

齊子尾害閭丘嬰,欲殺之,使帥師以伐陽州,我問師故,夏,五月,子尾殺閭丘嬰以說于我,師工僂,灑渻灶,孔虺,賈寅,出奔莒,出群公子。

公作楚宮,穆叔曰,大誓云,民之所欲,天必從之,君欲楚也夫,故作其宮,若不復適楚,必死是宮也,六月,辛巳,公薨于楚宮,叔仲帶竊其拱璧以與御人,納諸其懷,而從取之,由是得罪,立胡女敬歸之子子野,次于季氏,秋,九月,癸巳卒,毀也。

己亥,孟孝伯卒,立敬歸之娣齊歸之子公子裯,穆叔不欲,曰,大子死,有母弟則立之,無則長立,年鈞擇賢,義鈞則卜,古之道也,非適嗣,何必娣之子,且是人也,居喪而不哀,在慼而有嘉容,是謂不度,不度之人,鮮不為患,若果立之,必為季氏憂,武子不聽,卒立之,比及葬,三易衰,衰衽如故衰,於是昭公十九年矣,猶有童心,君子是以知其不能終也。

冬,十月,滕成公來會葬,惰而多涕,子服惠伯曰,滕君將死矣,怠於其位,而哀已甚,兆於死所矣,能無從乎。

癸酉,葬襄公,公薨之月,子產相鄭伯以如晉,晉侯以我喪故,未之見也,子產使盡壞其館之垣,而納車馬焉,士文伯讓之曰,敝邑以政刑之不脩,寇盜充斥,無若諸侯之屬,辱在寡君者何,是以令吏人完客所館,高其閈閎,厚其牆垣,以無憂客使,今吾子壞之,雖從者能戒,其若異客何,以敝邑之為盟主,繕完葺牆,以待賓客,若皆毀之,其何以共命,寡君使丐請命,對曰,以敝邑褊小,介於大國,誅求無時,是以不敢寧居,悉索敝賦,以來會時事,逢執之不間,而未得見,又不獲聞命,未知見時,不敢輸幣,亦不敢暴露,其輸之,則君之府實也,非薦陳之,不敢輸也,其暴露之,則恐燥濕之不時,而朽蠹以重敝邑之罪,僑聞文公之為盟主也,宮室卑庳,無觀臺榭,以崇大諸侯之館,館如公寢,庫廄繕脩,司空以時平易道路,圬人以時塓館宮室,諸侯賓至,甸設庭燎,僕人巡宮,車馬有所,賓從有代,巾車脂轄,隸人牧圉,各瞻其事,百官之屬,各展其物,公不留賓,而亦無廢事,憂樂同之,事則巡之,教其不知,而恤其不足,賓至如歸,無寧菑患,不畏寇盜,而亦不患燥濕,今銅鞮之宮數里,而諸侯舍於隸人,門不容車,而不可踰越,盜賊公行,而夭厲不戒,賓見無時,命不可知,若又勿壞,是無所藏幣以重罪也,敢請執事,將何以命之,雖君之有魯喪,亦敝邑之憂也,若獲薦幣,脩垣而行,君之惠也,敢憚勤勞,文伯復命,趙文子曰,信我實不德,而以隸人之垣以贏諸侯,是吾罪也,使士文伯謝不敏焉,晉侯見鄭伯,有加禮,厚其宴好而歸之,乃築諸侯之館,叔向曰,辭之不可以已也如是夫,子產有辭,諸侯賴之,若之何其釋辭也,詩曰,辭之輯矣,民之協矣,辭之繹矣,民之莫矣,其知之矣。

鄭子皮使印段如楚,以適晉告,禮也。

莒犁比公生去疾,及展輿,既立展輿,又廢之,犁比公虐,國人患之,十一月,展輿因國人以攻莒子,弒之,乃立,去疾奔齊,齊出也,展輿吳出也,書曰,莒人弒其君買朱鉏,言罪之在也。

吳子使屈狐庸聘于晉,通路也,趙文子問焉,曰,延州來季子,其果立乎,巢隕諸樊,閽戕戴吳,天似啟之,何如,對曰,不立,是二王之命也,非啟季子也,若天所啟,其在今嗣君乎,甚德而度,德不失民,度不失事,民親而事有序,其天所啟也,有吳國者,必此君之子孫實終之,季子守節者也,雖有國不立。

十二月,北宮文子相衛襄公以如楚,宋之盟故也,過鄭,印段迋勞于棐林,如聘禮而以勞辭,文子入聘,子羽為行人,馮簡子與子大叔逆客,事畢而出,言於衛侯曰,鄭有禮,其數世之福也,其無大國之討乎,詩曰,誰能執熱,逝不以濯,禮之於政,如熱之有濯也,濯以救熱,何患之有,子產之從政也,擇能而使之,馮簡子能斷大事,子大叔美秀而文,公孫揮能知四國之為,而辨於其大夫之族姓,班位貴賤能否,而又善為辭令,裨諶能謀,謀於野則獲,謀於邑則否,鄭國將有諸侯之事,子產乃問四國之為於子羽,且使多為辭令,與裨諶乘以適野,使謀可否,而告馮簡子使斷之,事成,乃授子大叔使行之,以應對賓客,是以鮮有敗事,北宮文子所謂有禮也。

鄭人游于鄉校,以論執政,然明謂子產曰,毀鄉校何如,子產曰,何為,夫人朝夕退而游焉,以議執政之善否,其所善者,吾則行之,其所惡者,吾則改之,是吾師也,若之何毀之,我聞忠善以損怨,不聞作威以防怨,豈不遽止,然猶防川,大決所犯,傷人必多,吾不克救也,不如小決,使道不如,吾聞而藥之也,然明曰,蔑也今而後知吾子之信可事也,小人實不才,若果行此,其鄭國實賴之,豈唯二三臣,仲尼聞是語也,曰,以是觀之,人謂子產不仁,吾不信也。

子皮欲使尹何為邑,子產曰,少,未知可否,子皮曰,愿吾愛之,不吾叛也,使夫往而學焉,夫亦愈知治矣,子產曰,不可,人之愛人,求利之也,今吾子愛人則以政,猶未能操刀而使割也,其傷實多,子之愛人,傷之而已,其誰敢求愛於子,子於鄭國,棟也,棟折榱崩,僑將厭焉,敢不盡言,子有美錦,不使人學製焉,大官大邑,身之所庇也,而使學者製焉,其為美錦,不亦多乎,僑聞學而後入政,未聞以政學者也,若果行此,必有所害,譬如田獵,射御貫,則能獲禽,若未嘗登車射御,則敗績厭覆是懼,何暇思獲,子皮曰,善哉,虎不敏,吾聞君子務知大者遠者,小人務知小者近者,我小人也,衣服附在吾身,我知而慎之,大官大邑,所以庇身也,我遠而慢之,微子之言,吾不知也,他日,我曰子為鄭國,我為吾家,以庇焉其可也,今而後知不足,自今請雖吾家聽子而行,子產曰,人心之不同,如其面焉,吾豈敢謂子面如吾面乎,抑心所謂危,亦以告也,子皮以為忠,故委政焉,子產是以能為鄭國。

衛侯在楚,北宮文子見令尹圍之威儀,言於衛侯曰,令尹似君矣,將有他志,雖獲其志,不能終也,《詩》云:『靡不有初,鮮克有終。』終之實難,令尹其將不免,公曰,子何以知之,對曰,《詩》云:『敬慎威儀,惟民之則。』令尹無威儀,民無則焉,民所不則,以在民上,不可以終,公曰,善哉,何謂威儀,對曰,有威而可畏,謂之威,有儀而可象,謂之儀,君有君之威儀,其臣畏而愛之,則而象之,故能有其國家,令聞長世,臣有臣之威儀,其下畏而愛之,故能守其官職,保族宜家,順是以下,皆如是,是以上下能相固也,衛詩曰,威儀棣棣,不可選也,言君臣上下,父子兄弟,內外大小,皆有威儀也,周詩曰,朋友攸攝,攝以威儀,言朋友之道,必相教訓,以威儀也,周書數文王之德曰,大國畏其力,小國懷其德,言畏而愛之也,《詩》云:『不識不知,順帝之則。』言則而象之也,紂囚文王七年,諸侯皆從之囚,紂於是乎懼而歸之,可謂愛之,文王伐崇,再駕而降為臣,蠻夷帥服,可謂畏之,文王之功,天下誦而歌舞之,可謂則之,文王之行,至今為法,可謂象之,有威儀也,故君子在位可畏,施舍可愛,進退可度,周旋可則,容止可觀,作事可法,德行可象,聲氣可樂,動作有文,言語有章,以臨其下,謂之有威儀也。


\end{pinyinscope}