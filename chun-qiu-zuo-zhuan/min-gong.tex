\article{閔公}

\begin{pinyinscope}
元年,春,王正月。

齊人救邢。

夏,六月,辛酉,葬我君莊公。

秋,八月,公及齊侯盟于落姑,季子來歸。

冬,齊仲孫來。

元年,春,不書即位,亂故也。

狄人伐邢,管敬仲言於齊侯曰,戎狄豺狼,不可厭也,諸夏親暱,不可棄也,宴安酖毒,不可懷也,《詩》云:「豈不懷歸,畏此簡書。」簡書,同惡相恤之謂也,請救邢以從簡書,齊人救邢。

夏,六月,葬莊公,亂故,是以緩。

秋,八月,公及齊侯盟于落姑,請復季友也,齊侯許之,使召諸陳,公次于郎以待之,季子來歸,嘉之也。

冬,齊仲孫湫來省難,書曰,仲孫,亦嘉之也,仲孫歸曰,不去慶父,魯難未巳,公曰,若之何而去之,對曰,難不巳,將自斃,君其待之,公曰,魯可取乎,對曰,不可,猶秉周禮,周禮,所以本也,臣聞之,國將亡,本必先顛,而後枝葉從之,魯不棄周禮,未可動也,君其務寧魯難而親之,親有禮,因重固,間攜貳,覆昏亂,霸王之器也。

晉侯作二軍,公將上軍,大子申生將下軍,趙夙御戎,畢萬為右,以滅耿,滅霍,滅魏,還為大子城曲沃,賜趙夙耿,賜畢萬魏,以為大夫,士蒍曰,大子不得立矣,分之都城,而位以卿,先為之極,又焉得立,不如逃之,無使罪至,為吳大伯,不亦可乎,猶有令名,與其及也,且諺曰,心苟無瑕,何恤乎無家,天若祚大子,其無晉乎,卜偃曰,畢萬之後必大,萬,盈數也,魏,大名也,以是始賞,天啟之矣,天子曰兆民,諸侯曰萬民,今名之大,以從盈數,其必有眾,初,畢萬筮仕於晉,遇屯之比,辛廖占之,曰,吉,屯固比入,吉孰大焉,其必蕃昌,震為土,車從馬,足居之,兄長之,母覆之,眾歸之,六體不易,合而能固,安而能殺,公侯之卦也,公侯之子孫,必復其始。

二年,春,王正月,齊人遷陽。

夏,五月,乙酉禘于莊公。

秋,八月,辛丑,公薨。

九月,夫人姜氏孫于邾。

公子慶父出奔莒。

冬,齊高子來盟。

十有二月,狄入衛。

鄭棄其師。

二年,春,虢公敗犬戎于渭汭,舟之僑曰,無德而祿,殃也,殃將至矣,遂奔晉。

夏,吉禘于莊公,速也。

初,公傅奪卜齮田,公不禁,秋,八月,辛丑,共仲使卜齮賊公于武闈,成季以僖公適邾,共仲奔莒,乃入立之,以賂求共仲于莒,莒人歸之,及密,使公子魚請,不許,哭而往,共仲曰,奚斯之聲也,乃縊,閔公,哀姜之娣,叔姜之子也,故齊人立之,共仲通於哀姜,哀姜欲立之,閔公之死也,哀姜與知之,故孫于邾,齊人取而殺之,于夷以其尸歸,僖公請而葬之。

成季之將生也,桓公使卜,楚丘之父卜之,曰,男也,其名曰友,在公之右,間于兩社,為公室輔,季氏亡則魯不昌,又筮之,遇大有之乾,曰,同復于父,敬如君所,及生,有文在其手曰友,遂以命之。

冬,十二月,狄人伐衛,衛懿公好鶴,鶴有乘軒者,將戰,國人受甲者,皆曰使鶴。鶴實有祿位,余焉能戰?公與石祁子玦,與甯莊子矢,使守,曰,以此贊國,擇利而為之,與夫人繡衣,曰,聽於二子,渠孔御戎,子伯為右,黃夷前驅,孔嬰齊殿,及狄人,戰于熒澤,衛師敗績,遂滅衛,衛侯不去其旗,是以甚敗,狄人囚史華龍滑,與禮孔,以逐衛人,二人曰,我大史也,實掌其祭,不先,國不可得也,乃先之,至則告守曰,不可待也,夜與國人出,狄入衛,遂從之,又敗諸河,初,惠公之即位也,少,齊人使昭伯烝於宣姜,不可,強之,生齊子,戴公,文公,宋桓夫人,許穆夫人,文公為衛之多患也,先適齊,及敗,宋桓公逆諸河,宵濟,衛之遺民,男女七百有三十人,益之以共滕之民,為五千人,立戴公以廬于曹,許穆夫人賦載馳,齊侯使公子無虧帥車三百乘,甲士三千人,以戍曹,歸公乘馬,祭服五稱,牛,羊,豕,雞,狗,皆三百,與門材,歸夫人魚軒,重錦三十兩。

鄭人惡高克,使帥師次于河上,久而弗召,師潰而歸,高克奔陳,鄭人為之賦清人。

晉侯使大子申生伐東山皋落氏,里克諫曰,大子奉冢祀社稷之粢盛,以朝夕視君膳者也,故曰,冢子,君行則守,有守則從,從曰撫軍,守曰監國,古之制也,夫帥師,專行謀,誓軍旅,君與國政之所圖也,非大子之事也,師在制命而已,稟命則不威,專命則不孝,故君之嗣適,不可以帥師,君失其官,帥師不威,將焉用之,且臣聞皋落氏將戰,君其舍之,公曰,寡人有子,未知其誰立焉,不對而退,見大子,大子曰,吾其廢乎,對曰,告之以臨民,教之以軍旅,不共是懼,何故廢乎,且子懼不孝,無懼弗得立,脩己而不責人,則免於難,大子帥師,公衣之偏衣,佩之金玦,狐突御戎,先友為右,梁餘子養御罕夷,先丹木為右,羊舌大夫為尉。先友曰,衣身之偏,握兵之要,在此行也,子其勉之,偏躬無慝,兵要遠災,親以無災,又何患焉。狐突歎曰,時,事之徵也,衣,身之章也,佩,衷之旗也,故敬其事則命以始,服其身則衣之純,用其衷則佩之度,今命以時卒,閟其事也,衣之尨服,遠其躬也,佩以金玦,棄其衷也,服以遠之,時以閟之,尨涼冬殺,金寒玦離,胡可恃也,雖欲勉之,狄可盡乎,梁餘子養曰,帥師者,受命於廟,受脤於社,有常服矣,不獲而尨,命可知也,死而不孝,不如逃之,罕夷曰,尨奇無常,金玦不復,雖復何為,君有心矣,先丹木曰,是服也,狂夫阻之,曰,盡敵而反,敵可盡乎,雖盡敵,猶有內讒,不如違之,狐突欲行,羊舌大夫曰,不可,違命不孝,棄事不忠,雖知其寒,惡不可取,子其死之,大子將戰,狐突諫曰,不可,昔辛伯諗周桓公云,內寵並后,外寵二政,嬖子配適,大都耦國,亂之本也,周公弗從,故及於難,今亂本成矣,立可必乎,孝而安民,子其圖之,與其危身以速罪也。

成風聞成季之繇,乃事之,而屬僖公焉,故成季立之。

僖之元年,齊桓公遷邢于夷儀,二年,封衛于楚丘,邢遷如歸,衛國忘亡。

衛文公大布之衣,大帛之冠,務材,訓農,通商,惠工,敬教,勸學,授方,任能,元年,革車三十乘,季年,乃三百乘。


\end{pinyinscope}