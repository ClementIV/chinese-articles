\article{隱公}

\begin{pinyinscope}
惠公元妃孟子,孟子卒,繼室以聲子,生隱公,宋武公生仲子,仲子生而有文在其手,曰為魯夫人,故仲子歸于我,生桓公而惠公薨,是以隱公立而奉之。

元年,春,王正月。

三月,公及邾儀父盟于蔑。

夏,五月,鄭伯克段于鄢。

秋,七月,天王使宰咺來歸惠公仲子之賵。

九月,及宋人盟于宿。

冬,十有二月,祭伯來。

公子益師卒。

元年,春,王周正月,不書即位,攝也。

三月,公及邾儀父盟于蔑,邾子克也,未王命,故不書爵,曰儀父,貴之也,公攝位,而欲求好於邾,故為蔑之盟。

夏四月,費伯帥師城郎,不書,非公命也。

初,鄭武公娶于申,曰武姜。生莊公及共叔段。莊公寤生,驚姜氏,故名曰寤生,遂惡之。愛共叔段,欲立之。亟請於武公,公弗許。及莊公即位,為之請制。公曰:「制,巖邑也,虢叔死焉,佗邑唯命。」請京,使居之,謂之「京城大叔」。祭仲曰:「都城過百雉,國之害也,先王之制:大都不過參國之一;中,五之一;小,九之一。今京不度,非制也,君將不堪。」公曰:「姜氏欲之,焉辟害?」對曰:「姜氏何厭之有!不如早為之所,無使滋蔓。蔓,難圖也。蔓草猶不可除,況君之寵弟乎!」公曰:「多行不義,必自斃,子姑待之。」既而大叔命西鄙、北鄙貳於己。公子呂曰:「國不堪貳,君將若之何?欲與大叔,臣請事之;若弗與,則請除之,無生民心。」公曰:「無庸,將自及。」大叔又收貳以為己邑,至于廩延。子封曰:「可矣,厚將得眾。」公曰:「不義不暱,厚將崩。」大叔完聚,繕甲兵,具卒乘,將襲鄭。夫人將啟之。公聞其期,曰:「可矣。」命子封帥車二百乘以伐京。京叛大叔段,段入于鄢,公伐諸鄢。五月辛丑,大叔出奔共。書曰:「鄭伯克段于鄢。」段不弟,故不言弟。如二君,故曰克。稱鄭伯,譏失教也。謂之鄭志,不言出奔,難之也。遂寘姜氏于城潁,而誓之曰:「不及黃泉,無相見也!」既而悔之。潁考叔為潁谷封人,聞之,有獻於公。公賜之食。食舍肉,公問之,對曰:「小人有母,皆嘗小人之食矣,未嘗君之羹。請以遺之。」公曰:「爾有母遺,繄我獨無!」潁考叔曰:「敢問何謂也?」公語之故,且告之悔,對曰:「君何患焉?若闕地及泉,隧而相見,其誰曰不然?」公從之。公入而賦:「大隧之中,其樂也融融。」姜出而賦:「大隧之外,其樂也洩洩。」遂為母子如初。君子曰:「潁考叔純孝也,愛其母,施及莊公。《詩》曰:『孝子不匱,永錫爾類。』其是之謂乎!」

秋,七月,天王使宰咺來歸惠公仲子之賵,緩,且子氏未薨,故名,天子七月而葬,同軌畢至,諸侯五月,同盟至,大夫三月,同位至,士踰月,外姻至,贈死不及尸,弔生不及哀,豫凶事,非禮也。

八月,紀人伐夷,夷不告,故不書,有蜚,不為災,亦不書,惠公之季年,敗宋師于黃,公立,而求成焉,九月,及宋人盟于宿,始通也。

冬十月,庚申,改葬惠公,公弗臨,故不書,惠公之薨也,有宋師,太子少,葬故有闕,是以改葬。

衛侯來會葬,不見公,亦不書。

鄭共叔之亂,公孫滑出奔衛,衛人為之伐鄭,取廩延,鄭人以王師,虢師,伐衛南鄙,請師於邾,邾子使私於公子豫,豫請往,公弗許。遂行,及邾人,鄭人,盟于翼,不書,非公命也。

新作南門,不書,亦非公命也。

十二月,祭伯來,非王命也。

眾父卒,公不與小斂,故不書日。

二年,春,公會戎于潛。

夏,五月,莒人入向。

無駭帥師入極。

秋,八月,庚辰,公及戎盟于唐。

九月,紀裂繻來逆女。

冬,十月,伯姬歸于紀。

紀子帛莒子,盟于密。

十有二月,乙卯,夫人子氏薨。

鄭人伐衛。

二年,春,公會戎于潛,修惠公之好也,戎請盟,公辭。

莒子娶于向,向姜不安莒而歸,夏,莒人入向,以姜氏還。

司空無駭入極,費庈父勝之。

戎請盟,秋,盟于唐,復修戎好也。

九月,紀裂繻來逆女,卿為君逆也。

冬紀子帛,莒子,盟于密,魯故也。

鄭人伐衛,討公孫滑之亂也。

三年,春,王二月,己巳,日有食之。

三月,庚戌,天王崩。

夏,四月,辛卯,君氏卒。

秋,武氏子來求賻。

八月,庚辰,宋公和卒。

冬,十有二月,齊侯鄭伯盟于石門。

癸未,葬宋穆公。

三年,春,王三月,壬戌,平王崩,赴以庚戌,故書之。

夏,君氏卒,聲子也,不赴於諸侯,不反哭于寢,不祔于姑,故不曰薨,不稱夫人,故不言葬,不書姓,為公故,曰君氏。

鄭武公,莊公,為平王卿士,王貳于虢,鄭伯怨王,王曰,無之,故周鄭交質,王子狐為質於鄭,鄭公子忽為質於周,王崩,周人將畀虢公政,四月,鄭祭足帥師取溫之麥,秋,又取成周之禾,周鄭交惡,君子曰,信不由中,質無益也,明恕而行,要之以禮,雖無有質,誰能間之,苟有明信,澗,谿,沼,沚,之毛,蘋,蘩,薀,藻,之菜,筐,筥,錡,釜,之,器,潢汙,行潦,之水,可薦於鬼神,可羞於王公,而況君子結二國之信,行之以禮,又焉用質,風有采繁,采蘋,雅有行葦,泂酌,昭忠信也。

武氏子來求賻,王未葬也。

宋穆公疾,召大司馬孔父而屬殤公焉。曰,先君舍與夷而立寡人,寡人弗敢忘,若以大夫之靈,得保首領以沒,先君若問與夷,其將何辭以對,請子奉之,以主社稷,寡人雖死,亦無悔焉,對曰,群臣願奉馮也,公曰,不可,先君以寡人為賢,使主社稷,若棄德不讓,是廢先君之舉也,豈曰能賢,光昭先君之令德,可不務乎,吾子其無廢先君之功,使公子馮出居於鄭,八月,庚辰,宋穆公卒,殤公即位,君子曰,宋宣公可謂知人矣,立穆公,其子饗之,命以義夫,商頌曰,殷受命咸宜,百祿是荷,其是之謂乎。

冬齊鄭盟于石門,尋盧之盟也,庚戌,鄭伯之車僨于濟。

衛莊公娶于齊東宮得臣之妹,曰莊姜,美而無子,衛人所為賦碩人也,又娶于陳,曰厲媯,生孝伯,早死,其娣戴媯,生桓公,莊姜以為己子,公子州吁,嬖人之子也,有寵而好兵,公弗禁,莊姜惡之,石碏諫曰,臣聞愛子,教之以義方,弗納于邪,驕奢淫泆,所自邪也,四者之來,寵祿過也,將立州吁,乃定之矣,若猶未也,階之為禍,夫寵而不驕,驕而能降,降而不憾,憾而能眕者,鮮矣,且夫賤妨貴,少陵長,遠間親,新間舊,小加大,淫破義,所謂六逆也,君義,臣行,父慈,子孝,兄愛,弟敬,所謂六順也,去順效逆,所以速禍也,君人者,將禍是務去,而速之,無乃不可乎,弗聽,其子厚與州吁游,禁之不可,桓公立,乃老。

四年,春,王二月,莒人伐杞,取牟婁。

戊申,衛州吁弒其君完。

夏,公及宋公遇于清。

宋公,陳侯,蔡人,衛人,伐鄭。

秋,翬帥師會宋公,陳侯,蔡人,衛人,伐鄭。

九月,衛人殺州吁于濮。

冬,十有二月,衛人立晉。

四年,春,衛州吁弒桓公而立,公與宋公為會,將尋宿之盟,未及期,衛人來告亂。

夏,公及宋公遇于清。

宋殤公之即位也,公子馮出奔鄭,鄭人欲納之,及衛州吁立,將脩先君之怨於鄭,而求寵於諸侯,以和其民,使告於宋曰,君若伐鄭,以除君害,君為主,敝邑以賦,與陳蔡從,則衛國之願也,宋人許之,於是陳蔡方睦於衛,故宋公,陳侯,蔡人,衛人,伐鄭,圍其東門,五日而還。公問於眾仲曰,衛州吁其成乎。對曰,臣聞以德和民,不聞以亂,以亂,猶治絲而棼之也,夫州吁阻兵而安忍,阻兵無眾,安忍無親,眾叛親離,難以濟矣。夫兵,猶火也。弗戢,將自焚也,夫州吁弒其君,而虐用其民,於是乎不務令德,而欲以亂成,必不免矣。

秋諸侯復伐鄭,宋公使來乞師,公辭之,羽父請以師會之,公弗許,固請而行,故書曰,翬帥師,疾之也,諸侯之師,敗鄭徒兵,取其禾而還。

州吁未能和其民,厚問定君於石子,石子曰,王覲為可,曰,何以得覲,曰,陳桓公方有寵於王,陳衛方睦,若朝陳使請,必可得也,厚從州吁如陳,石碏使告于陳曰,衛國褊小,老夫耄矣,無能為也,此二人者,實弒寡君,敢即圖之,陳人執之,而請蒞于衛,九月,衛人使右宰醜,蒞殺州吁于濮,石碏使其宰獳羊肩,蒞殺石厚于陳,君子曰,石碏,純臣也,惡州吁而厚與焉,大義滅親,其是之謂乎。

衛人逆公子晉于邢,冬十二月,宣公即位,書曰,衛人立晉,眾也。

五年,春,公矢魚于棠。

夏,四月,葬衛桓公。

秋,衛師入郕。

九月,考仲子之宮,初獻六羽。

邾人,鄭人,伐宋。

螟。

冬,十有二月,辛巳,公子彄卒。

宋人伐鄭,圍長葛。

五年,春,公將如棠觀魚者,臧僖伯諫曰,凡物不足以講大事,其材不足以備器用,則君不舉焉,君將納民於軌物者也,故講事以度軌量謂之軌,取材以章物采謂之物,不軌不物,謂之亂政,亂政亟行,所以敗也,故春蒐,夏苗,秋獮,冬狩,皆於農隙以講事也,三年而治兵,入而振旅,歸而飲至,以數軍實昭文章,明貴賤,辨等列,順少長,習威儀也,鳥獸之肉,不登於俎,皮革,齒牙,骨角,毛,羽,不登於器,則公不射,古之制也。若夫山林川澤之實,器用之資,皁隸之事,官司之守,非君所及也。公曰,吾將略地焉,遂往陳魚而觀之,僖伯稱疾不從,書曰,公矢魚于棠,非禮也,且言遠地也。

曲沃莊伯,以鄭人,邢人,伐翼,王使尹氏武氏助之,翼侯奔隨。

夏葬衛桓公,衛亂,是以緩。

四月,鄭人侵衛牧,以報東門之役,衛人以燕師伐鄭,鄭祭足,原繁,洩駕,以三軍軍其前,使曼伯與子元,潛軍軍其後,燕人畏鄭三軍,而不虞制人,六月,鄭二公子以制人,敗燕師于北制,君子曰,不備不虞,不可以師。

曲沃叛王,秋,王命虢公伐曲沃,而立哀侯于翼。

衛之亂也,郕人侵衛,故衛師入郕。

九月,考仲子之宮將萬焉,公問羽數於眾仲,對曰,天子用八,諸侯用六,大夫四,士二,夫舞所以節八音,而行八風,故自八以下,公從之,於是初獻六羽,始用六佾也。

宋人取邾田,邾人告於鄭曰,請君釋憾於宋,敝邑為道,鄭人以王師會之,伐宋,入其郛,以報東門之役,宋人使來告命,公聞其入郛也,將救之,問於使者曰,師何及,對曰,未及國,公怒,乃止,辭,使者曰,君命寡人,同恤社稷之難,今問諸使者,曰,師未及國,非寡人之所敢知也。

冬十二月,辛巳,臧僖伯卒,公曰,叔父有憾於寡人,寡人弗敢忘,葬之加一等。

宋人伐鄭,圍長葛,以報入郛之役也。

六年,春,鄭人來渝平。

夏,五月,辛酉,公會齊侯盟于艾。

秋,七月。

冬,宋人取長葛。

六年,春,鄭人來渝平,更成也。

翼九宗,五正,頃父之子嘉父,逆晉侯于隨,納諸鄂,晉人謂之鄂侯。

夏,盟于艾,始平于齊也。

五月,庚申,鄭伯侵陳,大獲,往歲,鄭伯請成于陳,陳侯不許,五父諫曰,親仁善鄰,國之寶也,君其許鄭。陳侯曰,宋衛實難,鄭何能為,遂不許。君子曰,善不可失,惡不可長,其陳桓公之謂乎,長惡不悛,從自及也,雖欲救之,其將能乎。商書曰,惡之易也,如火之燎于原,不可鄉邇,其猶可撲滅,周任有言曰,為國家者,見惡如農夫之務去草焉,芟夷蘊崇之,絕其本根,勿使能殖,則善者信矣。

秋,宋人取長葛。

冬,京師來告饑,公為之請糴於宋,衛,齊,鄭,禮也。

鄭伯如周,始朝桓王也,王不禮焉,周桓公言於王曰,我周之東遷,晉鄭焉依,善鄭以勸來者,猶懼不蔇,況不禮焉,鄭不來矣。

七年,春,王三月,叔姬歸于紀。

滕侯卒。

夏,城中丘。

齊侯使其弟年來聘。

秋,公伐邾。

冬,天王使凡伯來聘,戎伐凡伯于楚丘,以歸。

七年,春,滕侯卒,不書名,未同盟也,凡諸侯同盟,於是稱名,故薨則赴以名,告終,稱1嗣也,以繼好息民,謂之禮經。

夏,城中丘,書不時也。

齊侯使夷仲年來聘,結艾之盟也。

秋,宋及鄭平,七月,庚申,盟于宿,公伐邾,為宋討也。

初,戎朝于周,發幣于公卿,凡伯弗賓,冬,王使凡伯來聘,還,戎伐之于楚丘,以歸。

陳及鄭平,十二月,陳五父如鄭蒞盟,壬申,及鄭伯盟,歃如忘,洩伯曰,五父必不免,不賴盟矣,鄭良佐如陳蒞盟,辛巳,及陳侯盟,亦知陳之將亂也。

鄭公子忽在王所,故陳侯請妻之,鄭伯許之,乃成昏。1. 稱 : 舊脫。 據《春秋左傳正義》註釋:「稱」字原無,按阮校:「石經、宋本、岳本、足利本『終』下有『稱』字,是也。」

八年,春,宋公,衛侯遇于垂。

三月,鄭伯使宛來歸祊,庚寅,我入祊。

夏,六月,己亥,蔡侯考父卒,辛亥,宿男卒。

秋,七月,庚午,宋公,齊侯,衛侯,盟于瓦屋。

八月,葬蔡宣公。

九月,辛卯,公及莒人盟于浮來。

螟。

冬,十有二月,無駭卒。

八年,春,齊侯將平宋衛,有會期,宋公以幣請於衛,請先相見,衛侯許之,故遇于犬丘。

鄭伯請釋泰山之祀而祀周公,以泰山之祊易許田,三月,鄭伯使宛來歸祊,不祀泰山也。

夏虢公忌父始作卿士于周。

四月,甲辰,鄭公子忽如陳,逆婦媯,辛亥,以媯氏歸,甲寅,入于鄭,陳鍼子送女,先配而後祖,鍼子曰,是不為夫婦,誣其祖矣,非禮也,何以能育。

齊人卒平宋衛于鄭,秋,會于溫,盟于瓦屋,以釋東門之役,禮也。

八月,丙戌,鄭伯以齊人朝王,禮也。

公及莒人盟于浮來,以成紀好也。

冬齊侯使來告成三國,公使眾仲對曰,君釋三國之圖,以鳩其民,君之惠也,寡君聞命矣,敢不承受君之明德。

無駭卒,羽父請諡與族,公問族於眾仲,眾仲對曰,天子建德,因生以賜姓,胙之土而命之氏,諸侯以字,為諡,因以為族,官有世功,則有官族,邑亦如之,公命以字為展氏。

九年,春,天子使南季來聘。

三月,癸酉,大雨震電,庚辰,大雨雪,挾卒。

夏,城郎。

秋,七月。

冬,公會齊侯于防。

九年,春,王三月,癸酉,大雨霖以震,書始也,庚辰,大雨雪,亦如之,書時失也,凡雨,自三日以往為霖,平地尺為大雪。

夏城郎書不時也。

宋公不王,鄭伯為王左卿士,以王命討之,伐宋,宋以入郛之役怨公,不告命,公怒,絕宋使。

秋,鄭人以王命來告伐宋。

冬,公會齊侯于防,謀伐宋也。

北戎侵鄭,鄭伯禦之,患戎師曰,彼徒我車,懼其侵軼我也。公子突曰,使勇而無剛者,嘗寇,而速去之,君為三覆以待之,戎輕而不整,貪而無親,勝不相讓,敗不相救,先者見獲,必務進,進而遇覆,必速奔,後者不救,則無繼矣,乃可以逞,從之。戎人之前遇覆者,奔。祝聃逐之,衷戎師,前後擊之,盡殪,戎師大奔。十一月,甲寅,鄭人大敗戎師。

十年,春,王二月,公會齊侯,鄭伯,于中丘。

夏,翬帥師會齊人,鄭人,伐宋。

六月,壬戌,公敗宋師于菅,辛未,取郜,辛巳,取防。

秋,宋人,衛人,入鄭,宋人,蔡人,衛人,伐戴,鄭伯伐取之。

冬,十月,壬午,齊人,鄭人入郕。

十年,春,王正月,公會齊侯,鄭伯,于中丘,癸丑,盟于鄧,為師期。

夏,五月,羽父先會齊侯,鄭伯,伐宋。

六月,戊申,公會齊侯,鄭伯,于老桃,壬戌,公敗宋師于菅,庚午,鄭師入郜,辛未,歸于我,庚辰,鄭師入防,辛巳,歸于我,君子謂鄭莊公於是乎可謂正矣,以王命討不庭,不貪其土,以勞王爵,正之體也。

蔡人,衛人,郕人,不會王命。

秋,七月,庚寅,鄭師入郊,猶在郊,宋人,衛人,入鄭,蔡人從之,伐戴,八月,壬戌,鄭伯圍戴,癸亥,克之,取三師焉,宋衛既入鄭,而以伐戴召蔡人,蔡人怒,故不和而敗。

九月,戊寅,鄭伯入宋。

冬齊人,鄭人,入郕,討違王命也。

十有一年,春,滕侯,薛侯,來朝。

夏,公會鄭伯于時來。

秋,七月,壬午,公及齊侯,鄭伯,入許。

冬,十有一月,壬辰,公薨。

十一年,春,滕侯,薛侯,來朝,爭長,薛侯曰,我先封,滕侯曰,我周之卜正也,薛,庶姓也,我不可以後之,公使羽父請於薛侯曰,君與滕君,辱在寡人,周諺有之曰,山有木,工則度之,賓有禮,主則擇之,周之宗盟,異姓為後,寡人若朝于薛,不敢與諸任齒,君若辱貺寡人,則願以滕君為請,薛侯許之,乃長滕侯。

夏,公會鄭伯于郲,謀伐許也,鄭伯將伐許,五月,甲辰,授兵于大宮,公孫閼與潁考叔爭車,潁考叔挾輈以走,子都拔棘以逐之,及大逵,弗及,子都怒。

秋,七月,公會齊侯,鄭伯,伐許,庚辰,傅于許,潁考叔取鄭伯之旗蝥弧以先登,子都自下射之,顛,瑕叔盈又以蝥弧登,周麾而呼曰,君登矣,鄭師畢登,壬午,遂入許,許莊公奔衛,齊侯以許讓公,公曰,君謂許不共,故從君討之,許既伏其罪矣,雖君有命,寡人弗敢與聞,乃與鄭人,鄭伯使許大夫百里,奉許叔以居許東偏,曰,天禍許國,鬼神實不逞于許君,而假手于我寡人,寡人唯是一二父兄,不能共億,其敢以許自為功乎。寡人有弟,不能和協,而使餬其口於四方,其況能久有許乎?吾子其奉許叔,以撫柔此民也,吾將使獲也佐吾子,若寡人得沒于地,天其以禮悔禍于許,無寧茲許公,復奉其社稷,唯我鄭國之有請謁焉,如舊昏媾,其能降以相從也,無滋他族,實偪處此,以與我鄭國爭此土也,吾子孫其覆亡之不暇,而況能禋祀許乎,寡人之使吾子處此,不唯許國之為,亦聊以固吾圉也,乃使公孫獲處許西偏,曰,凡而器用財賄,無寘於許,我死,乃亟去之,吾先君新邑於此,王室而既卑矣,周之子孫,日失其序,夫許,大岳之胤也,天而既厭周德矣,吾其能與許爭乎,君子謂鄭莊公於是乎有禮,禮經國家,定社稷,序民人,利後嗣者也,許無刑而伐之,服而舍之,度德而處之,量力而行之,相時而動,無累後人,可謂知禮矣。

鄭伯使卒出豭,行出犬雞,以詛射潁考叔者。君子謂鄭莊公失政刑矣,政以治民,刑以正邪,既無德政,又無威刑,是以及邪,邪而詛之,將何益矣。

王取鄔,劉,蒍,邘,之田于鄭,而與鄭人蘇忿生之田,溫,原,絺,樊,隰郕,欑茅,向,盟,州,陘,隤,懷,君子是以知桓王之失鄭也,恕而行之,德之則也,禮之經也,己弗能有,而以與人,人之不至,不亦宜乎。

鄭息有違言,息侯伐鄭,鄭伯與戰于竟,息師大敗而還,君子是以知息之將亡也,不度德,不量力,不親親,不徵辭,不察有罪,犯五不韙,而以伐人,其喪師也,不亦宜乎。

冬,十月,鄭伯以虢師伐宋。壬戌,大敗宋師,以報其入鄭也。宋不告命,故不書。凡諸侯有命告則書,不然則否。師出臧否,亦如之。雖及滅國,滅不告敗,勝不告克,不書于策。

羽父請殺桓公,將以求大宰,公曰,為其少故也,吾將授之矣,使營菟裘,吾將老焉,羽父懼,反譖公于桓公,而請弒之,公之為公子也,與鄭人戰于狐壤,止焉,鄭人囚諸尹氏,賂尹氏,而禱於其主鍾巫,遂與尹氏歸,而立其主,十一月,公祭鍾巫,齊于社圃,館于寪氏,壬辰,羽父使賊弒公于寪氏,立桓公,而討寪氏,有死者,不書葬,不成喪也。


\end{pinyinscope}