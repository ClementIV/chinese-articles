\article{成公}

\begin{pinyinscope}
元年,春,王正月,公即位。

二月,辛酉,葬我君宣公。

無冰。

三月,作丘甲。

夏,臧孫許,及晉侯,盟于赤棘。

秋,王師敗績于茅戎。

冬,十月。

元年,春,晉侯使瑕嘉平戎于王,單襄公如晉拜成,劉康公徼戎,將遂伐之,叔服曰,背盟而欺大國,此必敗,背盟不祥,欺大國不義,神人弗助,將何以勝,不聽,遂伐茅戎,三月,癸未,敗績于徐吾氏。

為齊難故,作丘甲。

聞齊將出楚師,夏,盟于赤棘。

秋,王人來告敗。

冬,臧宣叔令脩賦繕完,具守備,曰,齊楚結好,我新與晉盟,晉楚爭盟,齊師必至,雖晉人伐齊,楚必救之,是齊楚同我也,知難而有備,乃可以逞。

二年,春,齊侯伐我北鄙。

夏,四月,丙戌,衛孫良夫帥師,及齊師戰于新築,衛師敗績。

六月,癸酉,季孫行父,臧孫許,叔孫僑如,公孫嬰齊,帥師會晉郤克,衛孫良夫,曹公子首,及齊侯戰于鞍,齊師敗績。

秋,七月,齊侯使國佐如師,己酉,及國佐盟于袁婁。

八月,壬午,宋公鮑卒。

庚寅,衛侯速卒。

取汶陽田。

冬,楚師,鄭師,侵衛。

十有一月,公會楚公子嬰齊于蜀。

丙申,公及楚人,秦人,宋人,陳人,衛人,鄭人,齊人,曹人,邾人,薛人,鄫人,盟于蜀。

二年,春,齊侯伐我北鄙,圍龍。頃公之嬖人盧蒲就魁,門焉,龍人囚之。齊侯曰,勿殺,吾與而盟,無入而封,弗聽,殺而膊諸城上,齊侯親鼓,士陵城,三日。取龍,遂南侵,及巢丘。

衛侯使孫良夫,石稷,甯相,向禽,將侵齊,與齊師遇,石子欲還,孫子曰,不可,以師伐人,遇其師而還,將謂君何,若知不能,則如無出,今既遇矣,不如戰也,夏,有,石成子曰,師敗矣,子不少須,眾懼盡,子喪師徒,何以復命,皆不對,又曰,子國卿也,隕子,辱矣,子以眾退,我此乃止,且告車來甚眾,齊師乃止,次于鞫居,新築人仲叔于奚救孫桓子,桓子是以免,既,衛人賞之以邑,辭,請曲縣繁纓以朝,許之,仲尼聞之曰,惜也,不如多與之邑,唯器與名,不可以假人,君之所司也,名以出信,信以守器,器以藏禮,禮以行義,義以生利,利以平民,政之大節也,若以假人,與人政也,政亡,則國家從之,弗可止也已,孫桓子還於新築,不入,遂如晉乞師。

臧宣叔亦如晉乞師,皆主郤獻子,晉侯許之七百乘。郤子曰:此城濮之賦也,有先君之明,與先大夫之肅,故捷。克於先大夫,無能為役,請八百乘,許之,郤克將中軍,士燮將上軍,欒書將下軍,韓厥為司馬,以救魯衛,臧宣叔逆晉師,且道之,季文子帥師會之,及衛地,韓獻子將斬人,郤獻子馳將救之,至,則既斬之矣,郤子使速以徇,告其僕曰,吾以分謗也,師從齊師于莘,六月,壬申,師至于靡笄之下,齊侯使請戰,曰,子以君師,辱於敝邑,不腆敝賦,詰朝請見,對曰,晉與魯衛,兄弟也,來告曰,大國朝夕釋憾於敝邑之地,寡君不忍,使群臣請於大國,無令輿師,淹於君地,能進不能退,君無所辱命,齊侯曰,大夫之許,寡人之願也,若其不許,亦將見也,齊高固入晉師,桀石以投人,禽之,而乘其車,繫桑本焉,以徇齊壘,曰,欲勇者,賈余餘勇,癸酉,師陳于鞍,邴夏御齊侯,逢丑父為右,晉解張御郤克,鄭丘緩為右,齊侯曰,余姑翦滅此而朝食,不介馬而馳之。郤克傷於矢。流血及屨,未絕鼓音。曰:余病矣。張侯曰:自始合,而矢貫余手及肘,余折以御,左輪朱殷,豈敢言病,吾子忍之。緩曰,自始合,苟有險,余必下推車,子豈識之,然子病矣,張侯曰,師之耳目,在吾旗鼓,進退從之,此車,一人殿之,可以集事,若之何其以病,敗君之大事也,擐甲執兵,固即死也,病未及死,吾子勉之,左并轡,右援枹而鼓,馬逸不能止,師從之。齊師敗績,逐之,三周華不注。韓厥夢子輿謂己曰,且辟左右,故中御而從齊侯,邴夏曰,射其御者,君子也,公曰,謂之君子而射之,非禮也,射其左,越于車下,射其右,斃于車中,綦毋張喪車,從韓厥曰,請寓乘,從左右,皆肘之,使立於後,韓厥俛定其右。逢丑父與公易位,將及華泉,驂絓於木而止。丑父寢於轏中,蛇出於其下,以肱擊之,傷而匿之,故不能推車而及。韓厥執縶馬前,再拜稽首,奉觴加璧以進,曰,寡君使群臣為魯衛請,曰,無令輿師,陷入君地,下臣不幸,屬當戎行,無所逃隱,且懼奔辟,而忝兩君,臣辱戎士,敢告不敏,攝官承乏,丑父使公下如華泉取飲,鄭周父御佐車,宛茷為右,載齊侯以免,韓厥獻丑父,郤獻子將戮之,呼曰,自今無有代其君任患者,有一於此,將為戮乎,郤子曰,人不難以死免其君,我戮之不祥,赦之以勸事君者,乃免之,齊侯免,求丑父,三入三出,每出齊師以帥退,入于狄卒,狄卒皆抽戈楯冒之以入于衛師,衛師免之,遂自徐關入,齊侯見保者曰勉之,齊師敗矣,辟女子,女子曰,君免乎,曰,免矣,曰,銳司徒免乎,曰,免矣,曰,苟君與吾父免矣,可若何,乃奔,齊侯以為有禮,既而問之,辟司徒之妻也,予之石窌,晉師從齊師,入自丘,輿擊馬陘,齊侯使賓媚人,賂以紀甗,玉磬,與地,不可,則聽客之所為,賓媚人致賂,晉人不可,曰,必以蕭同叔子為質,而使齊之封內,盡東其畝,對曰,蕭同叔子非他,寡君之母也,若以匹敵,則亦晉君之母也,吾子布大命於諸侯,而曰必質其母以為信,其若王命何,且是以不孝令也,詩曰,孝子不匱,永錫爾類,若以不孝令於諸侯,其無乃非德類也乎,先王疆理天下,物土之宜而布其利,故詩曰,我疆我理,南東其畝,今吾子疆理諸侯,而曰盡東其畝而已,唯吾子戎車是利,無顧土宜,其無乃非先王之命也乎,反先王則不義,何以為盟主,其晉實有闕,四王之王也,樹德而濟同欲焉,五伯之霸也,勤而撫之,以役王命,今吾子求合諸侯,以逞無疆之欲,詩曰,布政優優,百祿是遒,子實不優,而棄百祿,諸侯何害焉,不然,寡君之命使臣,則有辭矣,曰,子以君師辱於敝邑,不腆敝賦,以犒從者,畏君之震,師徒橈敗,吾子惠徼齊國之福,不泯其社稷,使繼舊好,唯是先君之敝器土地不敢愛,子又不許,請收合餘燼,背城借一,敝邑之幸,亦云從也。況其不幸,敢不唯命是聽。魯衛諫曰,齊疾我矣,其死亡者,皆親暱也,子若不許,讎我必甚,唯子則又何求,子得其國寶,我亦得地而紓於難,其榮多矣,齊晉亦唯天所授,豈必晉。晉人許之,對曰,群臣帥賦輿,以為魯衛請,若苟有以藉口,而復於寡君,君之惠也,敢不唯命是聽,禽鄭自師逆公。秋,七月,晉師及齊國佐盟于爰婁,使齊人歸我汶陽之田,公會晉師于上鄍,三帥先路三命之服,司馬,司空,輿帥,候正,亞旅,皆受一命之服。

八月,宋文公卒,始厚葬,用蜃炭,益車馬,始用殉,重器備。槨有四阿,棺有翰檜。君子謂華元,樂舉,於是乎不臣,臣,治煩去惑者也,是以伏死而爭,今二子者,君生則縱其惑,死又益其侈,是棄君於惡也,何臣之為。

九月,衛穆公卒,晉二子自役弔焉,哭於大門之外,衛人逆之,婦人哭於門內,送亦如之,遂常以葬。

楚之討陳夏氏也,莊王欲納夏姬,申公巫臣曰,不可,君召諸侯,以討罪也,今納夏姬,貪其色也,貪色為淫,淫為大罰,《周書》曰,明德慎罰,文王所以造周也,明德,務崇之之謂也,慎罰,務去之之謂也,若興諸侯,以取大罰,非慎之也,君其圖之,王乃止,子反欲取之,巫臣曰,是不祥人也,是夭子蠻,殺御叔,弒靈侯,戮夏南,出孔儀,喪陳國,何不祥如是,人生實難,其有不獲死乎,天下多美婦人,何必是,子反乃止,王以予連尹襄老,襄老死於邲,不獲其尸,其子黑要烝焉,巫臣使道焉,曰,歸,吾聘女,又使自鄭召之,曰,尸可得也,必來逆之,姬以告王,王問諸屈巫,對曰,其信,知罃之父,成公之嬖也,而中行伯之季弟也,新佐中軍,而善鄭皇戌,甚愛此子,其必因鄭而歸王子,與襄老之尸,以求之,鄭人懼於邲之役,而欲求媚於晉,其必許之,王遣夏姬歸,將行,謂送者曰,不得尸,吾不反矣,巫臣聘諸鄭,鄭伯許之,及共王即位,將為陽橋之役,使屈巫聘於齊,且告師期,巫臣盡室以行,申叔跪從其父將適郢,遇之,曰,異哉,夫子有三軍之懼,而又有桑中之喜,宜將竊妻以逃者也,及鄭,使介反幣,而以夏姬行,將奔齊,齊師新敗,曰,吾不處不勝之國,遂奔晉,而因郤至,以臣於晉,晉人使為邢大夫,子反請以重幣錮之,王曰,止,其自為謀也則過矣,其為吾先君謀也則忠,忠,社稷之固也,所蓋多矣,且彼若能利國家,雖重幣,晉將可乎,若無益於晉,晉將棄之,何勞錮焉。

晉師歸,范文子後入,武子曰,無為吾望爾也乎,對曰,師有功,國人喜以逆之,先入,必屬耳目焉,是代帥受名也,故不敢,武子曰,吾知免矣,郤伯見公曰,子之力也夫,對曰,君之訓也,二三子之力也,臣何力之有焉,范叔見,勞之如郤伯,對曰,庚所命也,克之制也,燮何力之有焉,欒伯見,公亦如之,對曰,燮之詔也,士用命也,書何力之有焉。

宣公使求好于楚,莊王卒,宣公薨,不克作好,公即位,受盟于晉,會晉伐齊,衛人不行使于楚,而亦受盟于晉,從於伐齊,故楚令尹子重為陽橋之役以救齊,將起師,子重曰,君弱,群臣不如先大夫,師眾而後可,詩曰,濟濟多士,文王以寧,夫文王猶用眾,況吾儕乎,且先君莊王屬之曰,無德以及遠方,莫如惠恤其民而善用之,乃大戶,已責,逮鰥,救乏,赦罪,悉師,王卒盡行,彭名御戎,蔡景公為左,許靈公為右,二君弱,皆強冠之,冬,楚師侵衛,遂侵我師于蜀,使臧孫往,辭曰,楚遠而久,固將退矣,無功而受名,臣不敢,楚侵及陽橋,孟孫請往賂之,以執斲執鍼織紝,皆百人,公衡為質,以請盟,楚人許平,十一月,公及楚公子嬰齊,蔡侯,許男,秦右大夫說,宋華元,陳公孫寧,衛孫良夫,鄭公子去疾,及齊國之大夫,盟于蜀,卿不書匱盟也,於是乎畏晉而竊與楚盟,故曰匱盟,蔡侯許男不書,乘楚車也,謂之失位,君子曰,位其不可不慎也乎,蔡許之君,一失其位,不得列於諸侯,況其下乎,詩曰,不解于位,民之攸塈,其是之謂矣。

楚師及宋,公衡逃歸,臧宣叔曰,衡父不忍數年之不宴,以棄魯國,國將若之何,誰居,後之人必有任是夫,國棄矣,是行也,晉辟楚,畏其眾也,君子曰,眾之不可已也,大夫為政,猶以眾克,況明君而善用其眾乎,大誓所謂商兆民離,周十人同者,眾也。

晉侯使鞏朔獻齊捷于周,王弗見,使單襄公辭焉,曰,蠻夷戎狄,不式王命,淫湎毀常,王命伐之,則有獻捷,王親受而勞之,所以懲不敬,勸有功也,兄弟甥舅,侵敗王略,王命伐之,告事而已,不獻其功,所以敬親暱,禁淫慝也,今叔父克遂有功于齊,而不使命卿鎮撫王室,所使來撫余一人,而鞏伯實來,未有職司於王室,又奸先王之禮,余雖欲於鞏伯,其敢廢舊典以忝叔父,夫齊,甥舅之國也,而大師之後也,寧不亦淫從其欲,以怒叔父,抑豈不可諫誨,士莊伯不能對,王使委於三吏,禮之如侯伯克敵,使大夫告慶之禮,降於卿禮一等,王以鞏伯宴,而私賄之,使相告之曰,非禮也,勿籍。

三年,春,王正月,公會晉侯,宋公,衛侯,曹伯,伐鄭。

辛亥,葬衛穆公。

二月,公至自伐鄭。

甲子,新宮災,三日哭。

乙亥,葬宋文公。

夏,公如晉。

鄭公子去疾帥師伐許。

公至自晉。

秋,叔孫僑如帥師圍棘。

大雩。

晉郤克,衛孫良夫,伐廧咎如。

冬,十有一月,晉侯使荀庚來聘。

衛侯使孫良夫來聘。

丙午,及荀庚盟。

丁未,及孫良夫盟,鄭伐許。

三年,春,諸侯伐鄭,次于伯牛,討邲之役也,遂東侵鄭,鄭公子偃帥師禦之,使東鄙覆諸鄤,敗諸丘輿,皇戌如楚獻捷。

夏,公如晉拜汶陽之田。

許恃楚而不事鄭,鄭子良伐許。

晉人歸楚公子穀臣,與連尹襄老之尸于楚,以求知罃,於是荀首佐中軍矣,故楚人許之,王送知罃,曰,子其怨我乎,對曰,二國治戎,臣不才,不勝其任,以為俘馘,執事不以釁鼓,使歸即戮,君之惠也,臣實不才,又誰敢怨,王曰,然則德我乎,對曰,二國圖其社稷,而求紓其民,各懲其忿,以相宥也,兩釋纍囚,以成其好,二國有好,臣不與及,其誰敢德,王曰,子歸何以報我,對曰,臣不任受怨,君亦不任受德,無怨無德,不知所報,王曰,雖然,必告不穀,對曰,以君之靈,纍臣得歸骨於晉,寡君之以為戮,死且不朽,若從君之惠而免之,以賜君之外臣首,首其請於寡君,而以戮於宗,亦死且不朽,若不獲命,而使嗣宗職,次及於事,而帥偏師以脩封疆,雖遇執事,其弗敢違,其竭力致死,無有二心,以盡臣禮,所以報也,王曰,晉未可與爭,重為之禮而歸之。

秋,叔孫僑如圍棘,取汶陽之田,棘不服,故圍之。

晉郤克,衛孫良夫,伐廧咎如,討赤狄之餘焉,廧咎如潰,上失民也。

冬,十一月,晉侯使荀庚來聘,且尋盟,衛侯使孫良夫來聘,且尋盟,公問諸臧宣叔曰,中行伯之於晉也,其位在三,孫子之於衛也,位為上卿,將誰先。對曰,次國之上卿,當大國之中,中當其下,下當其上大夫,小國之上卿,當大國之下卿,中當其上大夫,下當其下大夫,上下如是,古之制也。衛在晉,不得為次國,晉為盟主,其將先之。丙午,盟晉。丁未,盟衛,禮也。

十二月,甲戌,晉作六軍,韓厥,趙括,鞏朔,韓穿,荀騅,趙旃,皆為卿,賞鞍之功也。

齊侯朝于晉,將授玉,郤克趨進曰,此行也。君為婦人之笑辱也,寡君未之敢任,晉侯享齊侯,齊侯視韓厥。韓厥曰,君知厥也乎?齊侯曰,服改矣。韓厥登舉爵曰,臣之不敢愛死,為兩君之在此堂也。

荀罃之在楚也,鄭賈人有將窴諸褚中以出,既謀之,未行,而楚人歸之,賈人如晉,荀罃善視之,如實出已,賈人曰,吾無其功,敢有其實乎,吾小人,不可以厚誣君子,遂適齊。

四年,春,宋公使華元來聘。

三月,壬申,鄭伯堅卒。

杞伯來朝。

夏,四月,甲寅,臧孫許卒。

公如晉。

葬鄭襄公。

秋,公至自晉。

冬,城鄆。

鄭伯伐許。

四年,春,宋華元來聘,通嗣君也。

杞伯來朝,歸叔姬故也。

夏,公如晉,晉侯見公不敬,季文子曰,晉侯必不免,詩曰,敬之敬之,天惟顯思,命不易哉,夫晉侯之命,在諸侯矣,可不敬乎。

秋,公至自晉,欲求成于楚,而叛晉,季文子曰,不可,晉雖無道,未可叛也,國大臣睦,而邇於我,諸侯聽焉,未可以貳,史佚之志有之曰,非我族類,其心必異,楚雖大,非吾族也,其肯字我乎,公乃止。

冬,十一月,鄭公孫申帥師疆許田,許人敗諸展陂,鄭伯伐許,取鉏任泠敦之田。

晉欒書將中軍,荀首佐之,士燮佐上軍,以救許,伐鄭,取汜祭,楚子反救鄭,鄭伯與許男訟焉,皇戌攝鄭伯之辭,子反不能決也,曰,君若辱在寡君,寡君與其二三臣,共聽兩君之所欲,成其可知也,不然,側不足以知二國之成。

晉趙嬰通于趙莊姬。

五年,春,王正月,杞叔姬來歸。

仲孫蔑如宋。

夏,叔孫僑如會晉荀首于穀。

梁山崩。

秋,大水。

冬,十有一月,己酉,天王崩。

十有二月,己丑,公會晉侯,齊侯,宋公,衛侯,鄭伯,曹伯,邾子,杞伯,同盟于蟲牢。

五年,春,原屏放諸齊,嬰曰,我在,故欒氏不作,我亡,吾二昆其憂哉,且人各有能有不能,舍我何害,弗聽,嬰夢天使謂己祭余,余福女,使問諸士貞伯,貞伯曰,不識也,既而告其人曰,神福仁而禍淫,淫而無罰,福也,祭其得亡乎,祭之之明日而亡。

孟獻子如宋,報華元也。

夏,晉荀首如齊逆女,故宣伯餫諸穀。

梁山崩,晉侯以傳召伯宗伯,宗辟重,曰,辟傳,重人曰,待我,不如捷之速也,問其所,曰,絳人也,問絳事焉,曰,梁山崩,將召伯宗謀之,問將若之何,曰,山有朽壤而崩,可若何,國主山川,故山崩川竭,君為之不舉,降服,乘縵,徹樂,出次,祝幣,史辭,以禮焉,其如此而已,雖伯宗,若之何,伯宗請見之,不可,遂以告而從之。

許靈公愬鄭伯于楚,六月,鄭悼公如楚訟,不勝,楚人執皇戌及子國,故鄭伯歸,使公子偃請成于晉,秋,八月,鄭伯及晉趙同盟于垂棘。

宋公子圍龜為質于楚而歸,華元享之,請鼓譟以出,鼓譟以復入,曰,習攻華氏,宋公殺之。

冬,同盟于蟲牢,鄭服也,諸侯謀復會,宋公使向為人辭以子靈之難。

十一月,己酉,定王崩。

六年,春,王正月,公至自會。

二月,辛巳,立武宮。

取鄟。

衛孫良夫帥師侵宋。

夏,六月,邾子來朝。

公孫嬰齊如晉。

壬申,鄭伯費卒。

秋,仲孫蔑,叔孫僑如,帥師侵宋。

楚公子嬰齊帥師伐鄭。

冬,季孫行父如晉。

晉欒書帥師救鄭。

六年,春,鄭伯如晉拜成,子游相,授玉于東楹之東,士貞伯曰,鄭伯其死乎,自棄也已,視流而行速,不安其位,宜不能久。

二月,季文子以鞍之功立武宮,非禮也,聽於人以救其難,不可以立武,立武由己,非由人也。

取鄟,言易也。

三月,晉伯宗,夏陽說,衛孫良夫,甯相,鄭人,伊雒之戎,陸渾,蠻氏,侵宋,以其辭會也,師于鍼,衛人不保,說欲襲衛,曰,雖不可入,多俘而歸,有罪不及死,伯宗曰,不可,衛唯信晉,故師在其郊,而不設備,若襲之,是棄信也,雖多衛俘而晉無信,何以求諸侯,乃止,師還,衛人登陴。

晉人謀去故絳,諸大夫皆曰,必居郇瑕氏之地,沃饒而近盬,國利君樂,不可失也,韓獻子將新中軍,且為僕大夫,公揖而入,獻子從公立於寢庭,謂獻子曰,何如,對曰,不可。郇瑕氏土薄水淺,其惡易覯。易覯則民愁,民愁則墊隘,於是乎有沈溺重膇之疾,不如新田。土厚水深,居之不疾,有汾澮以流其惡。且民從教,十世之利也,夫山澤林盬,國之寶也,國饒,則民驕佚,近寶,公室乃貧,不可謂樂,公說,從之,夏,四月,丁丑,晉遷于新田。

六月,鄭悼公卒。

子叔聲伯如晉,命伐宋,秋,孟獻子,叔孫宣伯,侵宋,晉命也。

楚子重伐鄭,鄭從晉故也。

冬,季文子如晉,賀遷也。

晉欒書救鄭,與楚師遇於繞角,楚師還,晉師遂侵蔡,楚公子申,公子成,以申息之師救蔡,禦諸桑隧,趙同,趙括,欲戰,請於武子,武子將許之,知莊子,范文子,韓獻子,諫曰,不可,吾來救鄭,楚師去我,吾遂至於此,是遷戮也,戮而不已,又怒楚師,戰必不克,雖克不令,成師以出,而敗楚之二縣,何榮之有焉,若不能敗,為辱已甚,不如還也,乃遂還,於是軍師之欲戰者眾,或謂欒武子曰,聖人與眾同欲,是以濟事,子盍從眾,子為大政,將酌於民者也,子之佐十一人,其不欲戰者,三人而已,欲戰者可謂眾矣,商書曰,三人占,從二人,眾故也,武子曰,善鈞從眾,夫善,眾之主也,三卿為主,可謂眾矣,從之,不亦可乎。

七年,春,王正月,鼷鼠食郊牛角,改卜牛,鼷鼠又食其角,乃免牛。

吳伐郯。

夏,五月,曹伯來朝。

不郊猶三望。

秋,楚公子嬰齊帥師伐鄭。

公會晉侯,齊侯,宋公,衛侯,曹伯,莒子,邾子,杞伯,救鄭,八月,戊辰,同盟于馬陵,公至自會。

吳入州來。

冬,大雩。

衛孫林父出奔晉。

七年,春,吳伐郯,郯成,季文子曰,中國不振旅,蠻夷入伐,而莫之或恤,無弔者也,夫,詩曰,不弔昊天,亂靡有定,其此之謂乎,有上不弔,其誰不受亂,吾亡無日矣,君子曰,知懼如是,斯不亡矣。

鄭子良相成公以如晉見,且拜師。

夏,曹宣公來朝。

秋,楚子重伐鄭,師于氾,諸侯救鄭,鄭共仲,侯羽,軍楚師,囚鄖公鍾儀,獻諸晉,八月,同盟于馬陵,尋蟲牢之盟,且莒服故也。

晉人以鍾儀歸,囚諸軍府。

楚圍宋之役,師還,子重請取於申呂,以為賞田,王許之,申公巫臣曰,不可,此申呂所以邑也,是以為賦,以御北方,若取之,是無申呂也,晉鄭必至于漢,王乃止,子重是以怨巫臣,子反欲取夏姬,巫臣止之,遂取以行,子反亦怨之,及共王即位,子重,子反,殺巫臣之族子閻,子蕩,及清尹弗忌,及襄老之子黑要,而分其室,子重取子閻之室,使沈尹,與王子罷,分子蕩之室,子反取黑要,與清尹之室,巫臣自晉遺二子書曰,爾以讒慝貪惏事君,而多殺不辜,余必使爾罷於奔命以死,巫臣請使於吳,晉侯許之,吳子壽夢說之,乃通吳于晉,以兩之一卒適吳,舍偏兩之一焉,與其射御,教吳乘車,教之戰陳,教之叛楚,寘其子狐庸焉,使為行人於吳,吳始伐楚,伐巢,伐徐,子重奔命,馬陵之會,吳入州來,子重自鄭奔命,子重,子反,於是乎一歲七奔命,蠻夷屬於楚者,吳盡取之,是以始大,通吳於上國。

衛定公惡孫林父,冬,孫林父出奔晉,衛侯如晉,晉反戚焉。

八年,春,晉侯使韓穿來言汶陽之田,歸之于齊。

晉欒書帥師侵蔡。

公孫嬰齊如莒。

宋公使華元來聘。

夏,宋公使公孫壽來納幣。

晉殺其大夫趙同,趙括。

秋,七月,天子使召伯來賜公命。

冬,十月,癸卯,杞叔姬卒。

晉侯使士燮來聘。

叔孫僑如會晉士燮,齊人,邾人,伐剡。

衛人來媵。

八年,春,晉侯使韓穿來言汶陽之田,歸之于齊,季文子餞之,私焉,曰,大國制義,以為盟主,是以諸侯懷德畏討,無有貳心,謂汶陽之田,敝邑之舊也,而用師於齊,使歸諸敝邑,今有二命,曰,歸諸齊,信以行義,義以成命,小國所望而懷也,信不可知,義無所立,四方諸侯,其誰不解體,詩曰,女也不爽,士貳其行,士也罔極,二三其德,七年之中,一與一奪,二三孰甚焉,士之二三,猶喪妃耦,而況霸主,霸主將德是以而二三之,其何以長有諸侯乎,詩曰,猶之未遠,是用大簡,行父懼晉之不遠猶,而失諸侯也,是以敢私言之。

晉欒書侵蔡,遂侵楚,獲申驪,楚師之還也,晉侵沈,獲沈子揖,初從知范韓也,君子曰,從善如流,宜哉,詩曰,愷悌君子,遐不作人,求善也夫,作人斯有功績矣,是行也,鄭伯將會晉師,門于許東門,大獲焉。

聲伯如莒,逆也。

宋華元來聘,聘共姬也,夏,宋公使公孫壽來納幣,禮也。

晉趙莊姬為趙嬰之亡故,譖之于晉侯,曰,原屏將為亂,欒郤為徵,六月,晉討趙同,趙括,武從姬氏畜于公宮,以其田與祁奚,韓厥言於晉侯曰,成季之勳,宣孟之忠,而無後,為善者其懼矣,三代之令王,皆數百年保天之祿,夫豈無辟王,賴前哲以免也,《周書》曰,不敢侮鰥寡,所以明德也,乃立武而反其田焉。

秋,召桓公來賜公命。

晉侯使申公巫臣如吳,假道于莒,與渠丘公立於池上,曰,城已惡,莒子曰,辟陋在夷,其孰以我為虞,對曰,夫狡焉思啟封疆,以利社稷者,何國蔑有,唯然,故多大國矣,唯或思或縱也,勇夫重閉,況國乎。

冬,杞叔姬卒,來歸自杞,故書。

晉士燮來聘,言伐郯也,以其事吳故,公賂之,請緩師,文子不可,曰,君命無貳,失信不立,禮無加貨,事無二成,君後諸侯,是寡君不得事君也,燮將復之,季孫懼,使宣伯帥師會伐郯。

衛人來媵,共姬,禮也,凡諸侯嫁女,同姓媵之,異姓則否。

九年,春,王正月,杞伯來逆叔姬之喪以歸。

公會晉侯,齊侯,宋公,衛侯,鄭伯,曹伯,莒子,杞伯,同盟于蒲,公至自會。

二月,伯姬歸于宋。

夏,季孫行父如宋致女,晉人來媵。

秋,七月,丙子,齊侯無野卒。

晉人執鄭伯,晉欒書帥師伐鄭。

冬,十有一月,葬齊頃公。

楚公子嬰齊帥師伐莒,庚申,莒潰,楚人入鄆。

秦人白狄伐晉。

鄭人圍許,城中城。

九年,春,杞桓公來逆叔姬之喪,請之也,杞叔姬卒,為杞故也,逆叔姬,為我也。

為歸汶陽之田,故諸侯貳於晉,晉人懼,會於蒲,以尋馬陵之盟,季文子謂范文子曰,德則不競,尋盟何為,范文子曰,勤以撫之,寬以待之,堅疆以御之,明神以要之,柔服而伐貳,德之次也,是行也,將始會吳,吳人不至。

二月,伯姬歸于宋。

楚人以重賂求鄭,鄭伯會楚公子成于鄧。

夏季文子如宋致女,復命,公享之,賦韓奕之五章,穆姜出于房,再拜曰,大夫勤辱,不忘先君,以及嗣君,施及未亡人,先君猶有望也,敢拜大夫之重勤,又賦綠衣之卒章而入。

晉人來媵,禮也。

秋,鄭伯如晉,晉人討其貳於楚也,執諸銅鞮,欒書伐鄭,鄭人使伯蠲行成,晉人殺之,非禮也,兵交,使在其間可也,楚子重侵陳以救鄭,晉侯觀于軍府,見鍾儀,問之曰,南冠而縶者,誰也,有司對曰,鄭人所獻楚囚也,使稅之,召而弔之,再拜稽首,問其族,對曰,泠人也,公曰,能樂乎,對曰,先父之職官也,敢有二事,使與之琴,操南音,公曰,君王何如,對曰,非小人之所得知也,固問之,對曰,其為大子也,師保奉之,以朝于嬰齊,而夕于側也,不知其他,公語范文子,文子曰,楚囚,君子也,言稱先職,不背本也。樂操土風,不忘舊也。稱大子,抑無私也,名其二卿,尊君也,不背本,仁也,不忘舊,信也,無私,忠也,尊君,敏也,仁以接事,信以守之,忠以成之,敏以行之,事雖大必濟,君盍歸之,使合晉楚之成,公從之,重為之禮,使歸求成。

冬,十一月,楚子重自陳伐莒,圍渠丘,渠丘城惡,眾潰,奔莒,戊申,楚入渠丘,莒人囚楚公子平,楚人曰,勿殺,吾歸而俘,莒人殺之,楚師圍莒,莒城亦惡,庚申,莒潰,楚遂入鄆,莒無備故也,君子曰,恃陋而不備,罪之大者也,備豫不虞,善之大者也,莒恃其陋,而不脩城郭,浹辰之間,而楚克其三都,無備也夫,詩曰,雖有絲麻,無棄菅蒯,雖有姬姜,無棄蕉萃,凡百君子,莫不代匱,言備之不可以已也。

秦人白狄伐晉,諸侯貳故也。

鄭人圍許,示晉不急君也,是則公孫申謀之曰,我出師以圍許,為將改立君者,而紓晉使,晉必歸君。

城中城,書時也。

十二月,楚子使公子辰如晉,報鍾儀之使,請修好結成。

十年,春,衛侯之弟黑背,帥師侵鄭。

夏,四月,五卜郊不從,乃不郊。

五月,公會晉侯,齊侯,宋公,衛侯,曹伯,伐鄭。

齊人來媵。

丙午,晉侯獳卒。

秋,七月,公如晉。

冬,十月。

十年,春,晉侯使糴茷如楚,報大宰子商之使也。

衛子叔黑背侵鄭,晉命也。

鄭公子班聞叔申之謀,三月,子如立公子繻,夏,四月,鄭人殺繻,立髡頑,子如奔許,欒武子曰,鄭人立君,我執一人焉何益,不如伐鄭,而歸其君,以求成焉,晉侯有疾,五月,晉立大子州蒲以為君,而會諸侯伐鄭,鄭子罕賂以襄鍾,子然盟于脩澤,子駟為質,辛巳,鄭伯歸。

晉侯夢大厲,被髮及地,搏膺而踊曰,殺余孫不義,余得請於帝矣,壞大門及寢門而入,公懼,入于室,又壞戶,公覺,召桑田巫,巫言如夢,公曰,何如,曰,不食新矣,公疾病,求醫于秦,秦伯使醫緩為之,未至,公夢疾為二豎子曰,彼良醫也,懼傷我,焉逃之,其一曰,居肓之上,膏之下,若我何,醫至,曰,疾不可為也,在肓之上,膏之下,攻之不可,達之不及,藥不至焉,不可為也,公曰,良醫也,厚為之禮而歸之,六月,丙午,晉侯欲麥,使甸人獻麥,饋人為之,召桑田巫,示而殺之,將食,張,如廁,陷而卒,小臣有晨夢負公以登天,及日中,負晉侯出諸廁,遂以為殉。

鄭伯討立君者,戊申,殺叔申,叔禽,君子曰,忠為令德,非其人猶不可,況不令乎。

秋,公如晉,晉人止公,使送葬,於是糴茷未反,冬,葬晉景公,公送葬,諸侯莫在,魯人辱之,故不書,諱之也。

十有一年,春,王三月,公至自晉。

晉侯使郤犨來聘,己丑,及郤犨盟。

夏,季孫行父如晉。

秋,叔孫僑如如齊。

冬,十月。

十一年,春,王三月,公至自晉,晉人以公為貳於楚,故止公,公請受盟,而後使歸。

郤犨來聘,且蒞盟。

聲伯之母不聘,穆姜曰,吾不以妾為姒,生聲伯而出之,嫁於齊管于奚,生二子而寡,以歸聲伯,聲伯以其外弟為大夫,而嫁其外妹於施孝叔,郤犨來聘,求婦於聲伯,聲伯奪施氏婦以與之,婦人曰,鳥獸猶不失儷,子將若何,曰,吾不能死亡,婦人遂行,生二子於郤氏,郤氏亡,晉人歸之施氏,施氏逆諸河,沈其二子,婦人怒曰,已不能庇其伉儷而亡之,又不能字人之孤而殺之,將何以終,遂誓施氏。

夏,季文子如晉報聘,且蒞盟也。

周公楚惡惠襄之偪也,且與伯與爭政,不勝,怒而出,及陽樊,王使劉子復之,盟于鄄而入,三日,復出奔晉。

秋,宣伯聘于齊,以脩前好。

晉郤至與周爭鄇田,王命劉康公單襄公訟諸晉,郤至曰,溫,吾故也,故不敢失,劉子單子曰,昔周克商,使諸侯撫封,蘇忿生以溫為司寇,與檀伯達封于河,蘇氏即狄,又不能於狄,而奔衛,襄王勞文公,而賜之溫,狐氏,陽氏,先處之,而後及子,若治其故,則王官之邑也,子安得之,晉侯使郤至勿敢爭。

宋華元善於令尹子重,又善於欒武子,聞楚人既許晉糴茷成,而使歸復命矣,冬,華元如楚,遂如晉,合晉楚之成。

秦晉為成,將會于令狐,晉侯先至焉,秦伯不肯涉河,次于王城,使史顆盟晉侯于河東,晉郤犨盟秦伯于河西,范文子曰,是盟也,何益,齊盟,所以質信也,會所信之始也,始之不從,其何質乎,秦伯歸而背晉成。

十有二年,春,周公出奔晉。

夏,公會晉侯,衛侯,于瑣澤。

秋,晉人敗狄于交剛。

冬,十月。

十二年,春,王使以周公之難來告,書曰,周公出奔晉,凡自周無出,周公自出故也。

宋華元克合晉楚之成,夏,五月,晉士燮會楚公子罷,許偃,癸亥,盟于宋西門之外,曰,凡晉楚無相加戎,好惡同之,同恤菑危,備救凶患,若有害楚,則晉伐之,在晉,楚亦如之,交贄往來,道路無壅,謀其不協,而討不庭,有渝此盟,明神殛之,俾隊其師,無克胙國,鄭伯如晉聽成,會于瑣澤,成故也。

狄人間宋之盟以侵晉,而不設備,秋,晉人敗狄于交剛。

晉郤至如楚聘,且蒞盟,楚子享之,子反相,為地室而縣焉,郤至將登,金奏作於下,驚而走出。子反曰,日云莫矣,寡君須矣,吾子其入也。賓曰,君不忘先君之好,施及下臣,貺之以大禮,重之以備樂,如天之福,兩君相見,何以代此,下臣不敢,子反曰,如天之福,兩君相見,無亦唯是一矢以相加遺,焉用樂,寡君須矣,吾子其入也,賓曰,若讓之以一矢,禍之大者,其何福之為,世之治也,諸侯間於天子之事,則相朝也,於是乎有享宴之禮,享以訓共儉,宴以示慈惠,共儉以行禮,而慈惠以布政,政以禮成,民是以息,百官承事,朝而不夕,此公侯之所以扞城其民也,故詩曰,赳赳武夫,公侯干城,及其亂也,諸侯貪冒,侵欲不忌,爭尋常以盡其民,略其武夫,以為己腹心股肱爪牙,故詩曰,赳赳武夫,公侯腹心,天下有道,則公侯能為民干城,而制其腹心,亂則反之,今吾子之言,亂之道也,不可以為法,然吾子主也,至敢不從,遂入卒事,歸以語范文子,文子曰,無禮必食言,吾死無日矣夫,冬,楚公子罷如晉聘,且蒞盟,十二月,晉侯及楚公子罷盟于赤棘。

十有三年,春,晉侯使郤錡來乞師。

三月,公如京師。

公,五月,公自京師,遂會晉侯,齊侯,宋公,衛侯,鄭伯,曹伯,邾人,滕人,伐秦。

曹伯盧卒于師。

秋,七月,公至自伐秦。

冬,葬曹宣公。

十三年,春,晉侯使郤錡來乞師,將事不敬,孟獻子曰,郤氏其亡乎,禮,身之幹也,敬,身之基也,郤子無基,且先君之嗣卿也,受命以求師,將社稷是衛,而惰棄君命也,不亡何為。

三月,公如京師,宣伯欲賜,請先使,王以行人之禮,禮焉,孟獻子從,王以為介,而重賄之,公及諸侯朝王,遂從劉康公,成肅公,會晉侯伐秦,成子受脤于社,不敬,劉子曰,吾聞之,民受天地之中以生,所謂命也,是以有動作禮義威儀之則,以定命也,能者養之以福,不能者敗以取禍,是故君子勤禮,小人盡力,勤禮莫如致敬,盡力莫如敦篤,敬在養神,篤在守業,國之大事,在祀與戎,祀有執膰,戎有受脤,神之大節也,今成子惰棄其命矣,其不反乎。

夏,四月,戊午,晉侯使呂相絕秦,曰,昔逮我獻公,及穆公相好,戮力同心,申之以盟誓,重之以昏姻,天禍晉國,文公如齊,惠公如秦,無祿,獻公即世,穆公不忘舊德,俾我惠公,用能奉祀于晉,又不能成大勳,而為韓之師,亦悔于厥心,用集我文公,是穆之成也,文公躬擐甲冑,跋履山川,踰越險阻,征東之諸侯,虞夏商周之胤,而朝諸秦,則亦既報舊德矣,鄭人怒君之疆場,我文公帥諸侯及秦圍鄭,秦大夫不詢于我寡君,擅及鄭盟,諸侯疾之,將致命于秦,文公恐懼,綏靜諸侯,秦師克還無害,則是我有大造于西也,無祿,文公即世,穆為不弔,蔑死我君,寡我襄公,迭我殽地,奸絕我好,伐我保城,殄滅我費滑,散離我兄弟,撓亂我同盟,傾覆我國家,我襄公未忘君之舊勳,而懼社稷之隕,是以有殽之師,猶願赦罪于穆公,穆公弗聽,而即楚謀我,天誘其衷,成王隕命,穆公是以不克逞志于我,穆襄即世,康靈即位,康公我之自出,又欲闕翦我公室,傾覆我社稷,帥我蝥賊,以來蕩搖我邊疆,我是以有令狐之役,康猶不悛,入我河曲,伐我涑川,俘我王官,翦我羈馬,我是以有河曲之戰,東道之不通,則是康公絕我好也,及君之嗣也,我君景公,引領西望曰,庶撫我乎,君亦不惠稱盟,利吾有狄難,入我河縣,焚我箕郜,芟夷我農功,虔劉我邊陲,我是以有輔氏之聚,君亦悔禍之延,而欲徼福于先君獻穆,使伯車來命我景公曰,吾與女同好棄惡,復脩舊德,以追念前勳,言誓未就,景公即世,我寡君是以有令狐之會,君又不祥,背棄盟誓,白狄及君同州,君之仇讎,而我之昏姻也,君來賜命曰,吾與女伐狄,寡君不敢顧昏姻,畏君之威,而受命于吏,君有二心於狄,曰,晉將伐女,狄應且憎,是用告我,楚人惡君之二三其德也,亦來告我曰,秦背令狐之盟,而來求盟于我,昭告昊天上帝,秦三公,楚三王,曰,余雖與晉出入,余唯利是視,不穀惡其無成德,是用宣之,以懲不壹,諸侯備聞此言,斯是用痛心疾首,暱就寡人,寡人帥以聽命,唯好是求,君若惠顧諸侯,矜哀寡人,而賜之盟,則寡人之願也,其承寧諸侯以退,豈敢徼亂,君若不施大惠,寡人不佞,其不能諸侯退矣,敢盡布之執事,俾執事實圖利之,秦桓公既與晉厲公為令狐之盟,而又召狄與楚,欲道以伐晉,諸侯是以睦於晉,晉欒書將中軍,荀庚佐之,士燮將上軍,郤錡佐之,韓厥將下軍,荀罃佐之,趙旃將新軍,郤至佐之,郤毅御戎,欒鍼為右,孟獻子曰,晉帥乘和,師必有大功,五月,丁亥,晉師以諸侯之師,及秦師戰于麻隧,秦師敗績,獲秦成差,及不更女父,曹宣公卒于師,師遂濟涇,及侯麗而還,迓晉侯于新楚,成肅公卒于瑕。

六月,丁卯,夜,鄭公子班自訾求入于大宮,不能,殺子印,子羽,反軍于市,己巳,子駟帥國人盟于大宮,遂從而盡焚之,殺子如,子駹,孫叔,孫知。

曹人使公子負芻守,使公子欣時逆曹伯之喪,秋,負芻殺其大子而自立也,諸侯乃請討之,晉人以其役之勞,請俟他年,冬,葬曹宣公,既葬,子臧將亡,國人皆將從之,成公乃懼,告罪,且請焉,乃反而致其邑。

十有四年,春,王正月,莒子朱卒。

夏,衛孫林父自晉歸于衛。

秋,叔孫僑如如齊逆女。

鄭公子喜帥師伐許。

九月,僑如以夫人婦姜氏至自齊。

冬,十月,庚寅,衛侯臧卒。

秦伯卒。

十四年,春,衛侯如晉,晉侯強見孫林父焉,定公不可,夏,衛侯既歸,晉侯使郤犨送孫林父而見之,衛侯欲辭,定姜曰,不可,是先君宗卿之嗣也,大國又以為請,不許,將亡,雖惡之,不猶愈於亡乎,君其忍之,安民而宥宗卿,不亦可乎,衛侯見而復之,衛侯饗苦成叔,甯惠子相,苦成叔傲,甯子曰,苦成家其亡乎,古之為享食也,以觀威儀,省禍福也,故詩曰,兕觥其觩,旨酒思柔,彼交匪傲,萬福來求,今夫子傲,取禍之道也。

秋,宣伯如齊逆女,稱族,尊君命也。

八月,鄭子罕伐許,敗焉,戊戌,鄭伯復伐許,庚子,入其郛,許人平以叔申之封。

九月,僑如以夫人婦姜氏至自齊,舍族,尊夫人也,故君子曰,春秋之稱微而顯,志而晦,婉而成章,盡而不汙,懲惡而勸善,非聖人誰能脩之。

衛侯有疾,使孔成子,甯惠子,立敬姒之子衎,以為大子,冬,十月,衛定公卒,夫人姜氏既哭而息,見大子之不哀也,不內酌飲,歎曰,是夫也,將不唯衛國之敗,其必始於未亡人,烏呼,天禍衛國也,夫吾不獲鱄也,使主社稷,大夫聞之,無不聳懼,孫文子自是不敢舍其重器於衛,盡寘諸戚,而甚善晉大夫。

十有五年,春,王二月,葬衛定公。

三月,乙巳,仲嬰齊卒。

癸丑,公會晉侯,衛侯,鄭伯,曹伯,宋世子成,齊國佐,邾人,同盟于戚,晉侯執曹伯,歸于京師,公至自會。

夏,六月,宋公固卒。

楚子伐鄭。

秋,八月,庚辰,葬宋共公。

宋華元,出奔晉,宋華元自晉歸于宋,宋殺其大夫山,宋魚石出奔楚。

冬,十有一月,叔孫僑如會晉士燮,齊高無咎,宋華元,衛孫林父,鄭公子鰌,邾人,會吳于鍾離。

許遷于葉。

十五年,春,會于戚,討曹成公也,執而歸諸京師,書曰,晉侯執曹伯,不及其民也,凡君不道於其民,諸侯討而執之,則曰,某人執某侯,不然則否,諸侯將見子臧於王,而立之,子臧辭曰,前志有之曰,聖達節,次守節,下失節,為君,非吾節也,雖不能聖,敢失守乎,遂逃奔宋。

夏,六月,宋共公卒。

楚將北師,子囊曰:「新與晉盟而背之,無乃不可乎?」子反曰:敵利則進,何盟之有,申叔時老矣,在申,聞之曰,子反必不免,信以守禮,禮以庇身,信禮之亡,欲免得乎,楚子侵鄭,及暴隧,遂侵衛,及首止,鄭子罕侵楚,取新石,欒武子欲報楚,韓獻子曰,無庸,使重其罪,民將叛之,無民孰戰。

秋,八月,葬宋共公,於是華元為右師,魚石為左師,蕩澤為司馬,華喜為司徒,公孫師為司城,向為人為大司寇,鱗朱為少司寇,向帶為大宰,魚府為少宰,蕩澤弱公室,殺公子肥,華元曰,我為右師,君臣之訓,師所司也,今公室卑而不能正,吾罪大矣,不能治官,敢賴寵乎,乃出奔晉,二華,戴族也,司城,莊族也,六官者,皆桓族也,魚石將止華元,魚府曰,右師反必討,是無桓氏也,魚石曰,右師苟獲反,雖許之討,必不敢,且多大功,國人與之,不反,懼桓氏之無祀於宋也,右師討,猶有戌在,桓氏雖亡,必偏,魚石自止華元于河上,請討,許之,乃反,使華喜,公孫師,帥國人攻蕩氏,殺子山,書曰,宋殺大夫山,言背其族也,魚石,向為人,鱗朱,向帶,魚府,出舍於雎上,華元使止之,不可,冬,十月,華元自止之,不可,乃反,魚府曰,今不從,不得入矣,右師視速而言疾,有異志焉,若不我納,今將馳矣,登丘而望之,則馳聘而從之,則決睢澨,閉門登陴矣,左師,二司寇,二宰,遂出奔楚,華元使向戌為左師,老佐為司馬,樂裔為司寇,以靖國人。

晉三郤害伯宗,譖而殺之,及欒弗忌,伯州犁奔楚,韓獻子曰,郤氏其不免乎,善人,天地之紀也,而驟絕之,不亡何待,初,伯宗每朝,其妻必戒之曰,盜憎主人,民惡其上,子好直言,必及於難。

十一月,會吳于鍾離,始通吳也。

許靈公畏偪于鄭,請遷于楚,辛丑,楚公子申遷許于葉。

十有六年,春,王正月,雨木冰。

夏,四月,辛未,滕子卒。

鄭公子喜帥師侵宋。

六月,丙寅朔,日有食之。

晉侯使欒黶來乞師。

甲午,晦,晉侯及楚子,鄭伯,戰于鄢陵,楚子鄭師敗績,楚殺其大夫公子側。

秋,公會晉侯,齊侯,衛侯,宋華元,邾人,于沙隨,不見公,公至自會。

公會尹子,晉侯,齊國佐,邾人,伐鄭。

曹伯歸自京師。

九月,晉人執季孫行父,舍之于苕丘,冬,十月,乙亥,叔孫僑如出奔齊。

十有二月,乙丑,季孫行父及晉郤犨盟于扈。

公至自會。

乙酉,刺公子偃。

十六年,春,楚子自武城使公子成,以汝陰之田,求成于鄭,鄭叛晉,子駟從楚子盟于武城。

夏,四月,滕文公卒。

鄭子罕伐宋,宋將鉏,樂懼,敗諸汋陂,退舍於夫渠,不儆,鄭人覆之,敗諸汋陵,獲將鉏樂懼,宋恃勝也。

衛侯伐鄭,至于鳴鴈,為晉故也。

晉侯將伐鄭。范文子曰:若逞吾願,諸侯皆叛,晉可以逞,若唯鄭叛,晉國之憂,可立俟也。欒武子曰:不可以當吾世而失諸侯,必伐鄭,乃興師,欒書將中軍,士燮佐之,郤錡將上軍,荀偃佐之,韓厥將下軍,郤至佐新軍,荀罃居守,郤犨如衛,遂如齊,皆乞師焉,欒黶來乞師。孟獻子曰:有勝矣,戊寅,晉師起,鄭人聞有晉師,使告于楚,姚句耳與往,楚子救鄭,司馬將中軍,令尹將左,右尹子辛將右過申,子反入見申叔時,曰,師其何如,對曰,德,刑,詳,義,禮,信,戰之器也,德以施惠,刑以正邪,詳以事神,義以建利,禮以順時,信以守物,民生厚而德正,用利而事節,時順而物成,上下和睦,周旋不逆,求無不具,各知其極,故詩曰,立我烝民,莫匪爾極,是以神降之福,時無災害,民生敦厖,和同以聽。莫不盡力以從上命,致死以補其闕,此戰之所由克也。今楚內棄其民,而外絕其好,瀆齊盟,而食話言,奸時以動,而疲民以逞,民不知信,進退罪也,人恤所底,其誰致死,子其勉之,吾不復見子矣,姚句耳先歸,子駟問焉,對曰,其行速,過險而不整,速則失志,不整喪列,志失列喪,將何以戰,楚懼不可用也,五月,晉師濟河,聞楚師將至,范文子欲反,曰,我偽逃楚,可以紓憂,夫合諸侯,非吾所能也,以遺能者,我若群臣輯睦以事君,多矣,武子曰,不可,六月,晉楚遇於鄢陵,范文子不欲戰,郤至曰,韓之戰,惠公不振旅,箕之役,先軫不反命,邲之師,荀伯不復從,皆晉之恥也,子亦見先君之事矣,今我辟楚,又益恥也,文子曰,吾先君之亟戰也有故,秦狄齊楚皆彊,不盡力,子孫將弱,今三彊服矣,敵楚而已,唯聖人能外內無患。自非聖人,外寧必有內憂。盍釋楚以為外懼乎,甲午,晦,楚晨壓晉軍而陳,軍吏患之,范匄趨進曰,塞井夷灶,陳於軍中而疏行首,晉楚唯天所授,何患焉,文子執戈逐之,曰,國之存亡,天也,童子何知焉,欒書曰,楚師輕窕,固壘而待之,三日必退,退而擊之,必獲勝焉,郤至曰,楚有六間,不可失也,其二卿相惡,王卒以舊,鄭陳而不整,蠻軍而不陳,陳不違晦,在陳而囂,合而加囂,各顧其後,莫有鬥心,舊不必良,以犯天忌,我必克之,楚子登巢車以望晉軍,子重使大宰伯州犁侍于王後,王曰,騁而左右,何也,曰,召軍吏也,皆聚於中軍矣,曰,合謀也,張幕矣,曰,虔卜於先君也,徹幕矣,曰,將發命也,甚囂且塵上矣,曰,將塞井夷灶而為行也,皆乘矣,左右執兵而下矣,曰,聽誓也,戰乎,曰,未可知也,乘而左右皆下矣,曰,戰禱也,伯州犁以公卒告王,苗賁皇在晉侯之側,亦以王卒告,皆曰,國士在,且厚,不可當也,苗賁皇言於晉侯曰,楚之良,在其中軍王族而已,請分良以擊其左右,而三軍萃於王卒,必大敗之,公筮之,史曰,吉,其卦遇復,曰,南國蹙,射其元,王中厥目,國蹙王傷,不敗何待,公從之,有淖於前,乃皆左右,相違於淖,步毅御晉厲公,欒鍼為右,彭名御楚共王,潘黨為右,石首御鄭成公,唐苟為右,欒范以其族夾公行,陷於淖,欒書將載晉侯,鍼曰,書退,國有大任,焉得專之,且侵官,冒也,失官,慢也,離局,姦也,有三罪焉,不可犯也,乃掀公以出於淖,癸巳,潘尪之黨,與養由基,蹲甲而射之,徹七札焉,以示王,曰,君有二臣如此,何憂於戰,王怒曰,大辱國,詰朝,爾射死藝,呂錡夢射月,中之,退入於泥,占之曰,姬姓,日也,異姓,月也,必楚王也,射而中之,退入於泥,亦必死矣,及戰,射共王中目,王召養由基,與之兩矢,使射呂錡,中項伏弢,以一矢復命,郤至三遇楚子之卒,見楚子必下,免冑而趨風。楚子使工尹襄問之以弓。曰,方事之殷也,有𩎟韋之跗注,君子也,識見不穀而趨,無乃傷乎?郤至見客,免冑承命曰:「君之外臣至,從寡君之戎事,以君之靈,間蒙甲冑,不敢拜命,敢告不寧,君命之辱,為事之故,敢肅使者。」三肅使者而退。晉韓厥從鄭伯,其御杜溷羅曰,速從之,其御屢顧,不在馬,可及也,韓厥曰,不可以再辱國君,乃止,郤至從鄭伯,其右茀翰胡曰,諜輅之,余從之乘,而俘以,下郤至,曰傷國君有,刑亦,止石首,曰衛懿公唯不去其,旗是以敗於,熒乃內旌於弢,中唐苟謂石首,曰子在君側,敗者壹大,我不如子,子以君免,我請止,乃死,楚師薄於險,叔山冉謂養由基曰,雖君有命,為國故,子必射,乃射,再發盡殪,叔山冉搏人以役,中車折軾,晉師乃止,囚楚公子茷,欒鍼見子重之旌,請曰,楚人謂夫旌,子重之麾也,彼其子重也,日臣之使於楚也,子重問晉國之勇,臣對曰,好以眾整,曰,又何如,臣對曰,好以暇,今兩國治戎,行人不使,不可謂整,臨事而食言,不可謂暇,請攝飲焉,公許之,使行人執榼承飲,造于子重曰,寡君乏使,使鍼御持矛,是以不得犒從者,使某攝飲,子重曰,夫子嘗與吾言於楚,必是故也,不亦識乎,受而飲之,免使者而復鼓,旦而戰。見星未已,子反命軍吏察夷傷,補卒乘,繕甲兵,展車馬,雞鳴而食,唯命是聽。晉人患之,苗賁皇徇曰,蒐乘補卒,秣馬利兵。脩陳固列,蓐食申禱,明日復戰,乃逸楚囚,王聞之,召子反謀,穀陽豎獻飲於子反,子反醉而不能見,王曰,天敗楚也夫,余不可以待,乃宵遁,晉入楚軍,三日穀,范文子立於戎馬之前,曰,君幼,諸臣不佞,何以及此,君其戒之,《周書》曰,惟命不于常,有德之謂,楚師還,及瑕,王使謂子反曰,先大夫之覆師徒者,君不在,子無以為過,不穀之罪也,子反再拜稽首曰,君賜臣死,死且不朽,臣之卒實奔,臣之罪也,子重復謂子反曰,初隕師徒者,而亦聞之矣,盍圖之,對曰,雖微先大夫有之,大夫命側,側敢不義,側亡君師,敢忘其死,王使止之,弗及而卒,戰之日,齊國佐,高無咎,至于師,衛侯出于衛,公出于壞隤,宣伯通於穆姜,欲去季孟,而取其室,將行,穆姜送公,而使逐二子,公以晉難告,曰,請反而聽命,姜怒,公子偃,公子鉏,趨過指之曰,女不可,是皆君也,公待於壞隤,申宮儆備,設守而後行,是以後,使孟獻子守于公宮,秋,會干沙隨,謀伐鄭也,宣伯使告郤犨曰,魯侯待于壞隤,以待勝者,郤犨將新軍,且為公族大夫,以主東諸侯,取貨于宣伯,而訴公于晉侯,晉侯不見公。

曹人請于晉曰,自我先君宣公即位,國人曰,若之何,憂猶未弭,而又討我寡君,以亡曹國社稷之鎮公子,是大泯曹也,先君無乃有罪乎,若有罪,則君列諸會矣,君唯不遺德刑,以伯諸侯,豈獨遺諸敝邑,敢私布之。

七月,公會尹武公,及諸侯伐鄭,將行,姜又命公如初,公又申守而行,諸侯之師,次于鄭西,我師次于督揚,不敢過鄭,子叔聲伯使叔孫豹請逆于晉師,為食於鄭郊,師逆以至,聲伯四日不食以待之,食使者,而後食。

諸侯遷于制田,知武子佐下軍,以諸侯之師侵陳,至于鳴鹿,遂侵蔡未反,諸侯遷于潁上,戊午,鄭子罕宵軍之宋齊,衛皆失軍,曹人復請于晉,晉侯謂子臧,反,吾歸而君,子臧反,曹伯歸,子臧盡致其邑與卿,而不出。

宣伯使告郤犨曰,魯之有季孟,猶晉之有欒范也,政令於是乎成,今其謀曰,晉政多門,不可從也,寧事齊楚,有亡而已,蔑從晉矣,若欲得志於魯,請止行父而殺之,我斃蔑也,而事晉,蔑有貳矣,魯不貳,小國必睦,不然,歸必叛矣,九月,晉人執季文子于苕丘,公還,待于鄆,使子叔聲伯請季孫于晉,郤犨曰,苟去仲孫蔑而止季孫行父,吾與子國,親於公室,對曰,僑如之情,子必聞之矣,若去蔑與行父,是大棄魯國,而罪寡君也,若猶不棄,而惠徼周公之福,使寡君得事晉君,則夫二人者,魯國社稷之臣也,若朝亡之,魯必夕亡,以魯之密邇仇讎,亡而為讎,治之何及,郤犨曰,吾為子請邑,對曰,嬰齊,魯之常隸也,敢介大國,以求厚焉,承寡君之命以請,若得所請,吾子之賜多矣,又何求,范文子謂欒武子曰,季孫於魯,相二君矣,妾不衣帛,馬不食粟,可不謂忠乎,信讒慝而棄忠良,若諸侯何,子叔嬰齊奉君命無私,謀國家不貳,圖其身不忘其君,若虛其請,是棄善人也,子其圖之,乃許魯平,赦季孫,冬,十月,出叔孫僑如而盟之,僑如奔齊,十二月,季孫及郤犨盟于扈,歸刺公子偃,召叔孫豹于齊而立之。

齊聲孟子通僑如,使立於高國之閒,僑如曰,不可以再罪,奔衛,亦閒於卿。

晉侯使郤至獻楚捷于周,與單襄公語,驟稱其伐,單子語諸大夫曰,溫季其亡乎,位於七人之下,而求掩其上,怨之所聚,亂之本也,多怨而階亂,何以在位,夏書曰,怨豈在明,不見是圖,將慎其細也,今而明之,其可乎。

十有七年,春,衛北宮括帥師侵鄭。

夏,公會尹子,單子,晉侯,齊侯,宋公,衛侯,曹伯,邾人,伐鄭。

六月,乙酉,同盟于柯陵。

秋,公至自會。

齊高無咎出奔莒。

九月,辛丑,用郊。

晉侯使荀罃來乞師。

冬,公會單子,晉侯,宋公,衛侯,曹伯,齊人,邾人,伐鄭。

十有一月,公至自伐鄭。

壬申,公孫嬰卒于貍脤,十有二月丁已朔,日有食之。

邾子貜且卒。

晉殺其大夫郤錡,郤犨,郤至。

楚人滅舒庸。

十七年,春,王正月,鄭子駟侵晉虛滑,衛北宮括救晉侵鄭,至于高氏,夏,五月,鄭大子髡,頑侯獳,為質於楚,楚公子成,公子寅,戍鄭。

公會尹武公,單襄公,及諸侯伐鄭,自戲童至于曲洧。

晉范文子反自鄢陵,使其祝宗祈死曰,君驕侈而克敵,是天益其疾也,難將作矣,愛我者唯祝我使我速死,無及於難,范氏之福也,六月,戊辰,士燮卒。

乙酉,同盟于柯陵,尋戚之盟也。

楚子重救鄭師于首止,諸侯還。

齊慶克通于聲孟子,與婦人蒙衣乘輦,而入于閎,鮑牽見之,以告國武子,武子召慶克而謂之,慶克久不出,而告夫人曰,國子謫我,夫人怒,國子相靈公以會高鮑,處守及還,將至,閉門而索客,孟子訴之曰,高鮑將不納君而立公子角,國子知之,秋,七月,壬寅,刖鮑牽而逐高無咎,無咎奔莒,高弱以盧叛,齊人來召鮑國而立之,初,鮑國去鮑氏而來,為施孝叔臣,施氏卜宰,匡句須吉,施氏之宰,有百室之邑與匡句須邑,使為宰以讓鮑國而致邑焉,施孝叔曰,子實吉,對曰,能與忠良,吉孰大焉,鮑國相施氏忠,故齊人取以為鮑氏,後仲尼曰,鮑莊子之知不如葵,葵猶能衛其足。

冬,諸侯伐鄭。十月,庚午,圍鄭,楚公子申救鄭,師于汝上。十一月,諸侯還。

初,聲伯夢涉洹,或與己瓊瑰食之,泣而為瓊瑰,盈其懷,從而歌之曰,濟洹之水,贈我以瓊瑰,歸乎歸乎,瓊瑰盈吾懷乎,懼不敢占也,還自鄭,壬申,至于貍脤而占之,曰,余恐死,故不敢占也,今眾繁而從余三年矣,無傷也,言之之莫而卒。

齊侯使崔杼為大夫,使慶克佐之,帥師圍盧,國佐從諸侯圍鄭,以難請而歸,遂如盧師,殺慶克以穀叛,齊侯與之盟于徐關而復之,十二月,盧降,使國勝告難于晉,待命于清。

晉厲公侈,多外嬖,反自鄢陵,欲盡去群大夫而立其左右,胥童以胥克之廢也,怨郤氏,而嬖於厲公,郤錡奪夷陽五田,五亦嬖於厲公,郤犨與長魚矯爭田,執而梏之,與其父母妻子同一轅,既矯,亦嬖於厲公,欒書怨郤至,以其不從己而敗楚師也,欲廢之,使楚公子茷告公曰,此戰也,郤至實召寡君,以東師之未至也,與軍帥之不具也,曰此必敗,吾因奉孫周以事君,公告欒書,書曰,其有焉,不然,豈其死之不恤,而受敵使乎,君盍嘗使諸周而察之,郤至聘于周,欒書使孫周見之,公使覘之信,遂怨郤至,厲公田,與婦人先殺而飲酒,後使大夫殺,郤至奉豕,寺人孟張奪之,郤至,射而殺之,公曰,季子欺余,厲公將作難,胥童曰,必先三郤,族大多怨,去大族不偪,敵多怨有庸,公曰,然郤氏聞之,郤錡欲攻公,曰雖死,君必危,郤至曰,人所以立,信知勇也,信不叛君,知不害民,勇不作亂,失茲三者,其誰與我,死而多怨,將安用之,君實有臣而殺之,其謂君何,我之有罪,吾死後矣,若殺不辜,將失其民,欲安得乎,待命而已,受君之祿,是以聚黨,有黨而爭命,罪孰大焉,壬午,胥童,夷羊五,帥甲八百,將攻郤氏,長魚矯請無用眾,公使清沸魋助之,抽戈結衽而偽訟者,三郤將謀於榭,矯以戈殺駒伯苦成叔於其位,溫季曰,逃威也,遂趨,矯及諸其車,以戈殺之,皆尸諸朝,胥童以甲劫欒書中行偃於朝,矯曰,不殺二子,憂必及君,公曰,一朝而尸三卿,余不忍益也,對曰,人將忍君,臣聞亂在外為姦,在內為軌,御姦以德,御軌以刑,不施而殺,不可謂德,臣偪而不討,不可謂刑,德刑不立,姦軌並至,臣請行,遂出奔狄,公使辭於二子曰,寡人有討於郤氏,郤氏既伏其辜矣,大夫無辱,其復職位,皆再拜稽首曰,君討有罪而免臣於死,君之惠也,二臣雖死,敢忘君德,乃皆歸,公使胥童為卿,公遊于匠麗氏,欒書,中行偃,遂執公焉,召士匄,士匄辭,召韓厥,韓厥辭,曰,昔吾畜於趙氏,孟姬之讒,吾能違兵,古人有言曰,殺老牛莫之敢尸,而況君乎,二三子不能事君,焉用厥也。

舒庸人以楚師之敗也,道吳人圍巢,伐駕,圍釐虺,遂恃吳而不設備,楚公子橐師襲舒庸,滅之。

閏月,乙卯,晦,欒書中行偃殺胥童,民不與郤氏,胥童道君為亂,故皆書曰,晉殺其大夫。

十有八年,春,王正月,晉殺其大夫胥童。

庚申,晉弒其君州蒲。

齊殺其大夫國佐。

公如晉。

夏,楚子,鄭伯,伐宋,宋魚石復入于彭城。

公至自晉。

晉侯使士匄來聘。

秋,杞伯來朝。

八月,邾子來朝。

築鹿囿。

己丑,公薨于路寢。

冬,楚人鄭人侵宋。

晉侯使士魴來乞師。

十有二月,仲孫蔑會晉侯,宋公,衛侯,邾子,齊崔杼,同盟于虛朾。

丁未,葬我君成公。

十八年,春,王正月,庚申,晉欒書,中行偃,使程滑弒厲公,葬之于翼東門之外,以車一乘,使荀罃,士魴,逆周子于京師而立之,生十四年矣,大夫逆于清原,周子曰,孤始願不及此,雖及此,豈非天乎,抑人之求君,使出命也,立而不從將安用君,二三子用我今日,否亦今日,共而從君,神之所福也。對曰:群臣之願也,敢不唯命是聽,庚午,盟而入,館于伯子同氏辛巳,朝于武宮,逐不臣者七人。周子有兄而無慧,不能辨菽麥,故不可立。

齊為慶氏之難故,甲申晦,齊侯使士華免以戈殺國佐于內宮之朝,師逃于夫人之宮,書曰,齊殺其大夫國佐,棄命專殺,以穀叛故也,使清人殺國勝,國弱來奔,王湫奔萊,慶封為大夫,慶佐為司寇,既,齊侯反國弱,使嗣國氏,禮也。

二月,乙酉朔,晉侯悼公即位于朝,始命百官,施舍己責,逮鰥寡,振廢滯,匡乏困,救災患,禁淫慝,薄賦斂,宥罪戾,節器用,時用民,欲無犯,時使魏相,士魴,魏頡,趙武為卿,荀家,荀會,欒黶,韓無忌為公族大夫,使訓卿之子弟,共儉孝弟,使士渥濁為大傅,使脩范武子之法,右行辛為司空,使脩士蒍之法,弁糾御戎,校正屬焉,使訓諸御知義,荀賓為右,司士屬焉,使訓勇力之士,時使卿無共御,立軍尉以攝之,祁奚為中軍尉,羊舌職佐之,魏絳為司馬,張老為候奄,鐸遏寇為上軍尉,籍偃為之司馬,使訓卒乘,親以聽命,程鄭為乘馬御,六騶屬焉,使訓群騶知禮,凡六官之長,皆民譽也,舉不失職,官不易方,爵不踰德,師不陵正,旅不偪師,民無謗言,所以復霸也。

公如晉,朝嗣君也。

夏,六月,鄭伯侵宋,及曹門外,遂會楚子伐宋,取朝郟,楚子辛,鄭皇辰,侵城郜,取幽丘,同伐彭城,納宋魚石,向為人,鱗朱,向帶,魚府焉,以三百乘戍之而還,書曰復入,凡去其國,國逆而立之曰入,復其位曰復歸,諸侯納之曰歸,以惡曰復入,宋人患之,西鉏吾曰,何也,若楚人與吾同惡,以德於我,吾固事之也,不敢貳矣,大國無厭,鄙我猶憾,不然,而收吾憎,使贊其政,以間吾釁,亦吾患也,今將崇諸侯之姦,而披其地,以塞夷庚,逞姦而攜服,毒諸侯而懼吳晉,吾庸多矣,非吾憂也,且事晉何為,晉必恤之。

公至自晉,晉范宣子來聘,且拜朝也,君子謂晉於是乎有禮。

秋,杞桓公來朝,勞公,且問晉故,公以晉君語之,杞伯於是驟朝于晉,而請為昏。

七月,宋老佐,華喜,圍彭城,老佐卒焉,八月,邾宣公來朝,即位而來見也。

築鹿囿,書不時也。

己丑,公薨于路寢,言道也。

冬十一月,楚子重救彭城伐宋,宋華元如晉告急,韓獻子為政,曰,欲求得人,必先勤之,成霸安彊,自宋始矣,晉侯師于台谷以救宋,遇楚師于靡角之谷,楚師還,晉士魴來乞師,季文子問師數於臧武仲,對曰,伐鄭之役,知伯實來,下軍之佐也,今彘季亦佐下軍,如伐鄭可也,事大國無失班爵,而加敬焉禮也,從之。

十二月,孟獻子會于虛朾,謀救宋也,宋人辭諸侯,而請師以圍彭城,孟獻子請于諸侯而先歸會葬。

丁未,葬我君成公,書順也。


\end{pinyinscope}