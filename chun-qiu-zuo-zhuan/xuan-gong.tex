\article{宣公}

\begin{pinyinscope}
元年,春,王正月,公即位。

公子遂如齊逆女,三月,遂以夫人婦姜至自齊。

夏,季,孫行父如齊。

晉放其大夫胥甲父于衛。

公會齊侯于平州。

公子遂如齊。

六月,齊人取濟西田。

秋,邾子來朝。

楚子,鄭人,侵陳,遂侵宋,晉趙盾帥師救陳,宋公,陳侯,衛侯,曹伯,會晉師于棐林,伐鄭。

冬,晉趙穿帥師侵崇。

晉人,宋人,伐鄭。

元年,春,王正月,公子遂如齊逆女,尊君命也,三月,遂以夫人婦姜至自齊,尊夫人也。

夏,季文子如齊,納賂,以請會。

晉人討不用命者,放胥甲父于衛,而立胥克,先辛奔齊。

會于平州,以定公位。

東門襄仲如齊拜成。

六月,齊人取濟西之田,為立公故,以賂齊也。

宋人之弒昭公也,晉荀林父以諸侯之師伐宋,宋及晉平,宋文公受盟于晉,又會諸侯于扈,將為魯討齊,皆取賂而還,鄭穆公曰,晉不足與也,遂受盟于楚,陳共公之卒,楚人不禮焉,陳靈公受盟于晉,秋,楚子侵陳,遂侵宋,晉趙盾帥師救陳宋,會于棐林,以伐鄭也,楚蒍賈救鄭,遇于北林,囚晉解揚,晉人乃還。

晉欲求成於秦,趙穿曰,我侵崇,秦急崇必救之,吾以求成焉,冬,趙穿侵崇,秦弗與成。

晉人伐鄭,以報北林之役,於是晉侯侈,趙宣,子為政,驟諫而不入,故不競於楚。

二年,春,王二月,壬子,宋華元帥師,及鄭公子歸生帥師,戰于大棘,宋師敗績,獲宋華元。

秦師伐晉。

夏晉人,宋人,衛人,陳人,侵鄭。

秋,九月,乙丑,晉趙盾弒其君夷皋。

冬,十月,乙亥,天王崩。

二年,春,鄭公子歸生受命于楚,伐宋,宋華元,樂呂,御之,二月,壬子,戰于大棘,宋師敗績,囚華元,獲樂呂,及甲車四百六十乘,俘二百五十人,馘百人,狂狡輅鄭人,鄭人入于井,倒戟而出之,獲狂狡,君子曰,失禮違命,宜其為禽也,戎昭果毅以聽之,之謂禮,殺敵為果,致果為毅,易之戮也,將戰,華元殺羊食士,其御羊斟不與,及戰,曰,疇昔之羊,子為政,今日之事,我為政,與入鄭師,故敗,君子謂羊斟非人也,以其私憾,敗國殄民,於是刑孰大焉,詩所謂人之無良者,其羊斟之謂乎,殘民以逞。

宋人以兵車百乘,文馬百駟,以贖華元于鄭,半入,華元逃歸,立于門外,告而入,見叔牂,曰,子之馬然也。對曰,非馬也,其人也,既合而來奔,宋城,華元為植,巡功,城者謳曰,睅其目,皤其腹,棄甲而復,于思于思,棄甲復來,使其驂乘,謂之曰,牛則有皮,犀兕尚多,棄甲則那。役人曰,從其有皮,丹漆若何?華元曰,去之,夫其口眾我寡。

秦師伐晉,以報崇也,遂圍焦,夏,晉趙盾救焦,遂自陰地,及諸侯之師侵鄭,以報大棘之役,楚鬥椒救鄭,曰,能欲諸侯而惡其難乎,遂次于鄭,以待晉師,趙盾曰,彼宗競于楚,殆將斃矣,姑益其疾,乃去之。

晉靈公不君,厚斂以彫牆,從臺上彈人,而觀其辟丸也,宰夫胹熊蹯不熟,殺之,寘諸畚,使婦人載以過朝,趙盾,士季,見其手,問其故,而患之,將諫,士季曰,諫而不入,則莫之繼也,會請先,不入,則子繼之,三進及溜,而後視之,曰,吾知所過矣,將改之。稽首而對曰,人誰無過,過而能改,善莫大焉。《詩》曰:『靡不有初,鮮克有終』,夫如是,則能補過者鮮矣。君能有終,則社稷之固也,豈惟群臣賴之,又曰,袞職有闕,惟仲山甫補之,能補過也。君能補過,袞不廢矣。猶不改,宣子驟諫,公患之,使鉏麑賊之,晨往,寢門闢矣,盛服將朝,尚早,坐而假寐,麑退,歎而言曰,不忘恭敬,民之主也,賊民之主,不忠,棄君之命,不信,有一於此,不如死也,觸槐而死。

秋,九月,晉侯飲趙盾酒,伏甲將攻之,其右提彌明知之,趨登曰,臣侍君宴,過三爵,非禮也,遂扶以下,公嗾夫獒焉,明搏而殺之,盾曰,棄人用犬,雖猛何為,鬥且出,提彌明死之,初,宣子田於首山,舍于翳桑,見靈輒餓,問其病,曰,不食三日矣,食之,舍其半,問之。曰:宦三年矣,未知母之存否。今近焉,請以遺之,使盡之,而為之簞食與肉,寘諸橐以與之,既而與為公介,倒戟以禦公徒,而免之,問何故,對曰,翳桑之餓人也。問其名居,不告而退,遂自亡也。乙丑,趙穿攻靈公於桃園,宣子未出山而復,大史書曰,趙盾弒其君,以示於朝,宣子曰,不然,對曰,子為正卿。亡不越竟,反不討賊。非子而誰,宣子曰,嗚呼,我之懷矣,自詒伊慼,其我之謂矣。孔子曰:董狐,古之良史也。書法不隱,趙宣子,古之良大夫也,為法受惡,惜也,越竟乃免,宣子使趙穿逆公子黑臀于周,而立之,壬申,朝于武宮。

初,麗姬之亂,詛無畜群公子,自是晉無公族,及成公即位,乃宦卿之適子,而為之田,以為公族,又宦其餘子,亦為餘子,其庶子為公行,晉於是有公族,餘子,公行,趙盾請以括為公族,曰,君姬氏之愛子也,微君姬氏,則臣狄人也,公許之,冬,趙盾為旄車之族,使屏季以其故族為公族大夫。

三年,春,王正月,郊牛之口傷,改卜牛,牛死,乃不郊,猶三望。

葬匡王。

楚子伐陸渾之戎。

夏,楚人侵鄭。

秋,赤狄侵齊。

宋師圍曹。

冬,十月,丙戌,鄭伯蘭卒。

葬鄭穆公。

三年,春,不郊而望,皆非禮也,望,郊之屬也,不郊,亦無望可也。

晉侯伐鄭,及郔,鄭及晉平,士會入盟。

楚子伐陸渾之戎,遂至于雒,觀兵于周疆,定王使王孫滿勞楚子,楚子問鼎之大小輕重焉,對曰,在德不在鼎,昔夏之方有德也,遠方圖物,貢金九牧,鑄鼎象物,百物而為之備,使民知神姦,故民入川澤山林,不逢不若,螭魅罔兩,莫能逢之,用能協于上下,以承天休,桀有昏德,鼎遷于商,載祀六百,商紂暴虐,鼎遷于周,德之休明,雖小,重也,其姦回昏亂,雖大,輕也,天祚明德,有所底止,成王定鼎于郟鄏,卜世三十,卜年七百,天所命也,周德雖衰,天命未改,鼎之輕重,未可問也。

夏,楚人侵鄭,鄭即晉,故也。

宋文公即位三年,殺母弟須,及昭公子,武氏之謀也,使戴桓之族,攻武氏於司馬子伯之館,盡逐武穆之族,武穆之族,以曹師伐宋,秋,宋師圍曹,報武氏之亂也。

冬,鄭穆公卒,初,鄭文公有賤妾,曰燕姞,夢天使與己蘭,曰,余為伯鯈,余而祖也,以是為而子,以蘭有國香,人服媚之如是,既而文公見之,與之蘭而御之,辭曰,妾不才,幸而有子,將不信,敢徵蘭乎,公曰,諾,生穆公,名之曰蘭,文公報鄭子之妃,曰陳媯,生子華,子臧,子臧得罪而出,誘子華而殺之南里,使盜殺子臧於陳宋之間,又娶于江,生公子士,朝于楚,楚人酖之,及葉而死,又娶于蘇,生子瑕,子俞彌,俞彌早卒,洩駕惡瑕,文公亦惡之,故不立也,公逐群公子,公子蘭奔晉,從晉文公伐鄭,石癸曰,吾聞姬姞耦,其子孫必蕃,姞,吉人也,后稷之元妃也,今公子蘭,姞甥也,天或啟之,必將為君,其後必蕃,先納之,可以亢寵,與孔將鉏,侯宣多,納之,盟于大宮,而立之,以與晉平,穆公有疾,曰,蘭死,吾其死乎,吾所以生也,刈蘭而卒。

四年,春,王正月,公及齊侯平莒及郯,莒人不肯,公伐莒,取向。

秦伯稻卒。

夏,六月,乙酉,鄭公子歸生弒其君夷。

赤狄侵齊。

秋,公如齊,公至自齊。

冬,楚子伐鄭。

四年,春,公及齊侯平莒及郯,莒人不肯,公伐莒取向,非禮也,平國以禮,不以亂,伐而不治,亂也,以亂平亂,何治之有無治,何以行禮。

楚人獻黿於鄭靈公,公子宋,與子家將見,子公之食指動,以示子家,曰,他日我如此,必嘗異味,及入,宰夫將解黿,相視而笑,公問之,子家以告,及食大夫黿,召子公而弗與也,子公怒,染指於鼎,嘗之而出,公怒,欲殺子公,子公與子家謀先,子家曰,畜老猶憚殺之,而況君乎,反譖子家,子家懼而從之,夏弒靈公,書曰,鄭公子歸生弒其君夷,權不足也,君子曰,仁而不武,無能達也,凡弒君稱君,君無道也,稱臣,臣之罪也,鄭人立子良,辭曰,以賢則去疾不足,以順,則公子堅長,乃立襄公,襄公將去穆氏,而舍子良,子良不可曰,穆氏宜存,則固願也,若將亡之,則亦皆亡,去疾何為,乃舍之,皆為大夫。

初,楚司馬子良,生子越椒,子文曰,必殺之,是子也,熊虎之狀,而豺狼之聲,弗殺,必滅若敖氏矣,諺曰,狼子野心,是乃狼也,其可畜乎,子良不可,子文以為大慼,及將死,聚其族曰,椒也知政,乃速行矣,無及於難,且泣曰,鬼猶求食。若敖氏之鬼,不其餒而!及令尹子文卒,鬥般為令尹,子越為司馬,蒍賈為工正,譖子揚而殺之,子越為令尹,已為司馬,子越又惡之,乃以若敖氏之族,圄伯嬴於轑陽,而殺之,遂處烝野,將攻王,王以三王之子為質焉,弗受,師于漳澨,秋,七月,戊戌,楚子與若敖氏戰于皋滸,伯棼射王,汰輈,及鼓跗,著於丁寧,又射,汰輈,以貫笠轂,師懼,退,王使巡師曰,吾先君文王克息,獲三矢焉,伯棼竊其二,盡於是矣,鼓而進之,遂滅若敖氏,初,若敖娶於䢵,生鬥伯比,若敖卒,從其母畜於邧,淫於邧子之女,生子文焉,邧夫人使棄諸夢中,虎乳之,邧子田,見之,懼而歸,夫人以告,遂使收之。楚人謂乳穀,謂虎於菟。故命之曰鬥穀於菟,以其女妻伯比,實為令尹子文,其孫箴尹克黃,使於齊,還及宋,聞亂,其人曰,不可以人矣,箴尹曰,棄君之命,獨誰受之,君,天也,天可逃乎,遂歸復命,而自拘於司敗,王思子文之治楚國也,曰,子文無後,何以勸善,使復其所,改命曰生。

冬,楚子伐鄭,鄭未服也。

五年,春,公如齊。

夏,公至自齊。

秋,九月,齊高固來逆叔姬。

叔孫得臣卒。

冬,齊高固及子叔姬來。

楚人伐鄭。

五年,春,公如齊,高固使齊侯止公,請叔姬焉。

夏,公至自齊,書過也。

秋,九月,齊高固來逆女,自為也,故書曰,逆叔姬,即自逆也。

冬,來,反馬也。

楚子伐鄭,陳及楚平,晉荀林父救鄭伐陳。

五年,春,公如齊,高固使齊侯止公,請叔姬焉。

夏,公至自齊,書過也。

秋,九月,齊高固來逆女,自為也,故書曰,逆叔姬,即自逆也。

冬,來,反馬也。

楚子伐鄭,陳及楚平,晉荀林父救鄭伐陳。

六年,春,晉趙盾,衛孫免,侵陳。

夏,四月。

秋,八月,螽。

冬,十月。

六年,春,晉衛侵陳,陳即楚故也。

夏,定王使子服求后于齊。

秋,赤狄伐晉,圍懷,及邢丘,晉侯欲伐之,中行桓子曰,使疾其民,以盈其貫,將可殪也,《周書》曰:殪戎殷,此類之謂也。

冬,召桓公逆王后于齊。

楚人伐鄭,取成而還。

鄭公子曼滿,與王子伯廖語,欲為卿伯廖告人曰,無德而貪,其在周易豐之離,弗過之矣,間一歲,鄭人殺之。

七年,春,衛侯使孫良夫來盟。

夏,公會齊侯伐萊。

秋,公至自伐萊。

大旱。

冬,公會晉侯,宋公,衛侯,鄭伯,曹伯,于黑壤。

七年,春,衛孫桓子來盟,始通,且謀會晉也。

夏,公會齊侯伐萊,不與謀也,凡師出,與謀曰及,不與謀曰會。

赤狄侵晉,取向陰之禾。

鄭及晉平,公子宋之謀也,故相鄭伯以會,冬,盟于黑壤,王叔桓公臨之,以謀不睦。

晉侯之立也,公不朝焉,又不使大夫聘晉人止公于會,盟于黃父,公不與盟,以賂免,故黑壤之盟不書,諱之也。

八年,春,公至自會。

夏,六月,公子遂如齊,至黃,乃復。

辛巳,有事于大廟,仲遂卒于垂,壬午,猶繹,萬入,去籥。

戊子,夫人嬴氏薨。

晉師,白狄,伐秦。

楚人滅舒蓼。

秋,七月,甲子,日有食之,既。

冬,十月,己丑,葬我小君敬嬴,雨不克葬,庚寅,日中而克葬。

城平陽。

楚師伐陳。

八年,春,白狄及晉平,夏,會晉伐秦,晉人獲秦諜,殺諸絳市,六日而蘇。

有事于大廟,襄仲卒而繹,非禮也。

楚為眾舒叛故,伐舒蓼,滅之,楚子疆之。

及滑汭,盟吳越而還。

晉胥克有蠱疾,郤缺為政,秋,廢胥克,使趙朔佐下軍。

冬,葬敬嬴,旱無麻,始用葛茀,雨不克葬,禮也,禮,卜葬先遠日,辟不懷也。

城平陽,書時也。

陳及晉平,楚師伐陳,取成而還。

八年,春,白狄及晉平,夏,會晉伐秦,晉人獲秦諜,殺諸絳市,六日而蘇。

有事于大廟,襄仲卒而繹,非禮也。

楚為眾舒叛故,伐舒蓼,滅之,楚子疆之。

及滑汭,盟吳越而還。

晉胥克有蠱疾,郤缺為政,秋,廢胥克,使趙朔佐下軍。

冬,葬敬嬴,旱無麻,始用葛茀,雨不克葬,禮也,禮,卜葬先遠日,辟不懷也。

城平陽,書時也。

陳及晉平,楚師伐陳,取成而還。

九年,春,王正月,公如齊,公至自齊。

夏,仲孫蔑如京師。

齊侯伐萊。

秋,取根牟。

八月,滕子卒。

九月,晉侯,宋公,衛侯,鄭伯,曹伯,會于扈。

晉荀林父帥師伐陳。

辛酉,晉侯黑臀卒于扈。

冬,十月,癸酉,衛侯鄭卒。

宋人圍滕。

楚子伐鄭。

晉郤缺帥師救鄭。

陳殺其大夫洩冶。

九年,春,王使來徵聘,夏,孟獻子聘於周,王以為有禮,厚賄之。

秋,取根牟,言易也。

滕昭公卒。

會于扈,討不睦也,陳侯不會,晉荀林父以諸侯之師伐陳,晉侯卒于扈,乃還。

冬,宋人圍滕,因其喪也。

陳靈公與孔寧,儀行父,通於夏姬,皆衷其衵服以戲于朝,洩冶諫曰,公卿宣淫,民無效焉,且聞不令,君其納之,公曰,吾能改矣,公告二子,二子請殺之,公弗禁,遂殺洩冶,孔子曰:「《詩》云:『民之多辟,無自立辟』,其洩冶之謂乎。」

楚子為厲之役故,伐鄭。

晉郤缺救鄭,鄭伯敗楚師于柳棼,國人皆喜,唯子良憂,曰,是國之災也,吾死無日矣。

十年,春,公如齊,公至自齊。

齊人歸我濟西田。

夏,四月,丙辰,日有食之。

己巳,齊侯元卒。

齊崔氏出奔衛。

公如齊,五月,公至自齊。

癸巳,陳夏徵舒弒其君平國。

六月,宋師伐滕。

公孫歸父如齊。

葬齊惠公,晉人,宋人,衛人,曹人,伐鄭。

秋,天王使王季子來聘。

公孫歸父帥師伐邾,取繹。

大水。

季孫行父如齊。

冬,公孫歸父如齊,齊侯使國佐來聘。

饑,楚子伐鄭。

十年,春,公如齊,齊侯以我服故,歸濟西之田。

夏,齊惠公卒,崔杼有寵於惠公,高,國畏其偪也,公卒而逐之,奔衛,書曰,崔氏,非其罪也,且告以族,不以名,凡諸侯之大夫違,告於諸侯曰,某氏之守臣某,失守宗廟,敢告,所有玉帛之使者則告,不然則否。

公如齊奔喪。

陳靈公與孔寧,儀行父,飲酒於夏氏,公謂行父曰,徵舒似女,對曰,亦似君,徵舒病之,公出,自其廄射而殺之,二子奔楚。

滕人恃晉而不事宋,六月,宋師伐滕。

鄭及楚平,諸侯之師伐鄭,取成而還。

秋,劉康公來報聘。

師伐邾,取繹。

季文子初聘于齊。

冬子家如齊,伐邾故也,國武子來報聘。

楚子伐鄭,晉士會救鄭,逐楚師于潁北,諸侯之師戍鄭,鄭子家卒,鄭人討幽公之亂,斲子家之棺而逐其族,改葬幽公,諡之曰靈。

十有一年,春,王正月。

夏,楚子,陳侯,鄭伯,盟于辰陵。

公孫歸父會齊人伐莒。

秋,晉侯會狄于欑函。

冬,十月,楚人殺陳夏徵舒。

丁亥,楚子入陳,納公孫寧儀行父于陳。

十一年,春楚子伐鄭,及櫟,子良曰,晉楚不務德而兵爭,與其來者可也,晉楚無信,我焉得有信,乃從楚,夏,楚盟于辰陵,陳鄭服也。

楚左尹子重侵宋,王待諸郔。

令尹蒍艾獵城沂,使封人慮事,以授司徒,量功命日,分財用,平板榦,稱畚築,程土物,議遠邇,略基趾,具餱糧,度有司,事三旬而成,不愆于素。

晉郤成子求成于眾狄,眾狄疾赤狄之役,遂服于晉,秋,會于欑函,眾狄服也,是行也,諸大夫欲召狄,郤成子曰,吾聞之,非德莫如勤,非勤何以求人,能勤有繼,其從之也,詩曰,文王既勤止,文王猶勤,況寡德乎。

冬,楚子為陳夏氏亂故,伐陳,謂陳人無動,將討於少西氏,遂入陳,殺夏徵舒,轘諸栗門,因縣陳,陳侯在晉,申叔時使於齊反,復命而退,王使讓之曰,夏徵舒為不道,弒其君,寡人以諸侯討而戮之,諸侯縣公皆慶寡人,女獨不慶寡人,何故,對曰,猶可辭乎,王曰,可哉,曰,夏徵舒弒其君,其罪大矣,討而戮之,君之義也,抑人亦有言曰,牽牛以蹊人之田,而奪之牛,牽牛以蹊者,信有罪矣,而奪之牛,罰已重矣,諸侯之從也,曰,討有罪也,今縣陳,貪其富也,以討召諸侯,而以貪歸之,無乃不可乎,王曰,善哉,吾未之聞也,反之,可乎,對曰,吾儕小人,所謂取諸其懷而與之也,乃復封陳,鄉取一人焉以歸,謂之夏州,故書曰,楚子入陳,納公孫寧,儀行父,于陳,書有禮也。

厲之役,鄭伯逃歸,自是楚未得志焉,鄭既受盟于辰陵,又徼事于晉。

十有二年,春,葬陳靈公。

楚子圍鄭。

夏,六月,乙卯,晉荀林父帥師及楚子戰于邲,晉師敗績。

秋,七月。

冬,十有二月,戊寅,楚子滅蕭。

晉人,宋人,衛人,曹人,同盟于清丘,宋師伐陳,衛人救陳。

十二年,春,楚子圍鄭,旬有七日,鄭人卜行成不吉,卜臨于大宮,且巷出車,吉,國人大臨,守陴者皆哭,楚子退師,鄭人脩城,進復圍之。三月,克之,入自皇門,至于逵路,鄭伯肉袒牽羊以逆,曰,孤不天,不能事君,使君懷怒,以及敝邑,孤之罪也,敢不唯命是聽。其俘諸江南,以實海濱,亦唯命,其翦以賜諸侯,使臣妾之,亦唯命,若惠顧前好,徼福於厲,宣,桓,武,不泯其社稷,使改事君,夷於九縣,君之惠也,孤之願也,非所敢望也,敢布腹心,君實圖之,左右曰,不可許也,得國無赦。王曰,其君能下人,必能信用其民矣,庸可幾乎,退三十里,而許之平,潘尪入盟,子良出質。

夏,六月,晉師救鄭,荀林父將中軍,先縠佐之,士會將上軍,郤克佐之,趙朔將下軍,欒書佐之,趙括,趙嬰齊,為中軍大夫,鞏朔,韓穿,為上軍大夫,荀首,趙同,為下軍大夫,韓厥為司馬及河,聞鄭既及楚平,桓子欲還,曰,無及於鄭,而勦民,焉用之,楚歸而動,不後,隨武子曰,善,會聞用師觀釁而動,德,刑,政,事,典禮,不易,不可敵也,不為是征,楚軍討鄭,怒其貳而哀其卑,叛而伐之,服而舍之,德刑成矣,伐叛,刑也,柔服,德也,二者立矣,昔歲入陳,今茲入鄭,民不罷勞,君無怨讟,政有經矣,荊尸而舉,商農工賈,不敗其業。而卒乘輯睦,事不奸矣。蒍敖為宰,擇楚國之令典,軍行,右轅,左追蓐,前茅慮無,中權,後勁,百官象物而動,軍政不戒而備,能用典矣,其君之舉也,內姓選於親,外姓選於舊,舉不失德,賞不失勞,老有加惠,旅有施舍,君子小人,物有服章,貴有常尊,賤有等威,禮不逆矣,德立刑行,政成事時,典從禮順,若之何敵之,見可而進,知難而退,軍之善政也,兼弱攻昧,武之善經也,子姑整軍而經武乎,猶有弱而昧者,何必楚,仲虺有言曰,取亂侮亡,兼弱也,汋曰,於鑠王師,遵養時晦,耆昧也,武曰,無競惟烈,撫弱耆昧,以務烈所,可也,彘子曰,不可,晉所以霸,師武臣力也,今失諸侯,不可謂力,有敵而不從,不可謂武,由我失霸,不如死。且成師以出,聞敵彊而退,非夫也。命有軍師,而卒以非夫,唯群子能,我弗為也,以中軍佐濟。

知莊子曰,此師殆哉,周易有之,在師之臨曰,師出以律,否臧凶,執事順成為臧,逆為否,眾散為弱,川壅為澤,有律以如己也,故曰,律否臧,且律竭也,盈而以竭,夭且不整,所以凶也,不行謂之臨,有帥而不從,臨孰甚焉,此之謂矣,果遇必敗,彘子尸之,雖免而歸,必有大咎,韓獻子謂桓子曰,彘子以偏師陷,子罪大矣,子為元帥,師不用命,誰之罪也,失屬亡師,為罪已重,不如進也,事之不捷,惡有所分,與其專罪,六人同之,不猶愈乎,師遂濟。

楚子北師次於郔,沈尹將中軍,子重將左,子反將右,將飲馬於河而歸,聞晉師既濟,王欲還,嬖人伍參欲戰,令尹孫叔敖弗欲,曰,昔歲入陳,今茲入鄭,不無事矣,戰而不捷,參之肉,其足食乎,參曰,若事之捷,孫叔為無謀矣,不捷,參之肉,將在晉軍,可得食乎,令尹南轅反旆,伍參言於王曰,晉之從政者新,未能行令,其佐先縠,剛愎不仁,未肯用命,其三帥者,專行不獲,聽而無上,眾誰適從,此行也,晉師必敗,且君而逃臣,若社稷何,王病之,告令尹,改乘轅而北之,次于管以待之。

晉師在敖鄗之間,鄭皇戌使如晉師曰,鄭之從楚,社稷之故也,未有貳心,楚師驟勝而驕,其師老矣,而不設備,子擊之,鄭師為承,楚師必敗,彘子曰,敗楚服鄭,於此在矣,必許之。欒武子曰:楚自克庸以來,其君無日不討國人而訓之于民生之不易,禍至之無日,戒懼之不可以怠。在軍,無日不討軍實而申儆之,于勝之不可保,紂之百克,而卒無後,訓之以若敖,蚡冒,篳路藍縷,以啟山林,箴之曰,民生在勤,勤則不匱,不可謂驕,先大夫子犯有言曰,師直為壯,曲為老,我則不德,而徼怨于楚,我曲楚直,不可謂老,其君之戎,分為二廣,廣有一卒,卒偏之兩,右廣初駕,數及日中,左則受之,以至于昏,內官序當其夜,以待不虞,不可謂無備,子良,鄭之良也,師叔,楚之崇也,師叔入盟,子良在楚,楚鄭親矣,來勸我戰,我克則來,不克遂往,以我卜也,鄭不可從,趙括,趙同,曰,率師以來,唯敵是求,克敵得屬,又何俟,必從彘子,知季曰,原屏,咎之徒也。趙莊子曰,欒伯,善哉,實其言,必長晉國。

楚少宰如晉師。曰,寡君少遭閔凶,不能文,聞二先君之出入此行也,將鄭是訓定,豈敢求罪于晉,二三子無淹久,隨季對曰,昔平王命我先君文侯曰,與鄭夾輔周室,毋廢王命,今鄭不率,寡君使群臣問諸鄭,豈敢辱候人,敢拜君命之辱,彘子以為諂,使趙括從而更之曰,行人失辭,寡君使群臣遷大國之跡於鄭,曰,無辟敵,群臣無所逃命,楚子又使求成于晉,晉人許之,盟有日矣,楚許伯御樂伯,攝叔為右,以致晉師,許伯曰,吾聞致師者,御靡旌,摩壘而還,樂伯曰,吾聞致師者,左射以菆,代御執轡,御下兩馬,掉鞅而還,攝叔曰,吾聞致師者,右入壘,折馘,執俘而還,皆行其所聞而復,晉人逐之,左右角之,樂伯左射馬而右射人,角不能進,矢一而已。麋興於前,射麋麗龜。晉鮑癸當其後,使攝叔奉麋獻焉,曰,以歲之非時,獻禽之未至,敢膳諸從者,鮑癸止之曰,其左善射,其右有辭,君子也,既免。

晉魏錡求,公族未得,而怒,欲敗晉師,請致師,弗許,請使,許之,遂往請戰而還,楚潘黨逐之,及熒澤,見六麋,射一麋以顧獻。曰:子有軍事,獸人無乃不給於鮮,敢獻於從者,叔黨命去之。趙旃求卿未得,且怒於失楚之致師者,請挑戰,弗許,請召盟,許之,與魏錡皆命而往,郤獻子曰,二憾往矣,弗備必敗,彘子曰,鄭人勸戰,弗敢從也,楚人求成,弗能好也,師無成命,多備何為,士季曰,備之善,若二子怒楚,楚人乘我,喪師無日矣,不如備之,楚之無惡,除備而盟,何損於好,若以惡來,有備不敗,且雖諸侯相見,軍衛不徹,警也,彘子不可。

士季使鞏朔,韓穿,帥七覆于敖前,故上軍不敗,趙嬰齊使其徒先具舟于河,故敗而先濟。

潘黨既逐魏錡,趙旃夜至於楚軍,席於軍門之外,使其徒入之,楚子為乘,廣三十乘,分為左右,右廣,雞鳴而駕,日中而說,左則受之,日入而說,許偃御右廣,養由基為右,彭名御左廣,屈蕩為右,乙卯,王乘左廣,以逐趙旃,趙旃棄車而走林,屈蕩搏之,得其甲裳,晉人懼二子之怒楚師也,使軘車逆之,潘黨望其塵,使騁而告曰,晉師至矣,楚人亦懼王之入晉軍也,遂出陳,孫叔曰,進之,寧我薄人,無人薄我,詩云,元戎十乘,以先啟行,先人也,軍志曰,先人有奪人之心,薄之也。遂疾進師,車馳卒奔,乘晉軍。桓子不知所為,鼓於軍中,曰,先濟者有賞,中軍下軍爭舟,舟中之指可掬也。

晉師右移,上軍未動,工尹齊,將右拒卒,以逐下軍,楚子使唐狡,與蔡鳩居,告唐惠侯,曰,不穀不德而貪,以遇大敵,不穀之罪也,然楚不克,君之羞也,敢藉君靈,以濟楚師,使潘黨率游闕四十乘,從唐侯以為左拒,以從上軍,駒伯曰,待諸乎,隨季曰,楚師方壯,若萃於我,吾師必盡,不如收而去之,分謗生民,不亦可乎,殿其卒而退,不敗。

王見右廣,將從之乘,屈蕩尸之曰,君以此始,亦必以終,自是楚之乘,廣先左,晉人或以廣隊,不能進,楚人惎之脫扃,少進,馬還,又惎之拔旆投衡,乃出,顧曰,吾不如大國之數奔也,趙旃以其良馬二,濟其兄與叔父,以他馬反,遇敵不能去,棄車而走林,逢大夫與其二子乘,謂其二子無顧,顧曰,趙傁在後,怒之,使下,指木曰,尸女於是,授趙旃綏以免,明日以表尸之,皆重獲在木下,楚熊負羈囚知罃,知莊子以其族反之,廚武子御,下軍之士多從之,每射,抽矢菆,納諸廚子之房,廚子怒曰,非子之求,而蒲之愛,董澤之蒲,可勝既乎,知季曰,不以人子,吾子其可得乎,吾不可以苟射故也,射連尹襄老,獲之,遂載其尸,射公子穀臣,囚之,以二者還,及昏,楚師軍於邲,晉之餘師不能軍,宵濟,亦終夜有聲,丙辰,楚重至於邲,遂次于衡雍,潘黨曰,君盍築武軍,而收晉尸以為京觀,臣聞克敵,必示子孫,以無忘武功,楚子曰,非爾所知也,夫文,止戈為武,武王克商,作頌曰,載戢干戈,載櫜弓矢,我求懿德,肆于時夏,允王保之,又作武,其卒章曰,耆定爾功,其三曰,鋪時繹思,我徂維求定,其六曰,綏萬邦,屢豐年,夫武,禁暴,戢兵,保大,定功,安民,和眾,豐財,者也,故使子孫無忘其章,今我使二國暴骨,暴矣,觀兵以威諸侯,兵不戢矣,暴而不戢,安能保大,猶有晉在,焉得定功,所違民欲猶多,民何安焉,無德而強爭諸侯,何以和眾,利人之幾,而安人之亂,以為己榮,何以豐財,武有七德,我無一焉,何以示子孫,其為先君宮,告成事而已,武非吾功也,古者明王,伐不敬,取其鯨鯢而封之,以為大戮,於是乎有京觀,以懲淫慝,今罪無所,而民皆盡忠,以死君命,又何以為京觀乎,祀于河作先君宮,告成事而還。

是役也,鄭石制,實入楚師,將以分鄭,而立公子魚臣,辛未,鄭殺僕叔及子服,君子曰,史佚所謂毋怙亂者,謂是類也,詩曰,亂離瘼矣,爰其適歸,歸於怙亂者也夫,鄭伯,許男,如楚,秋,晉師歸,桓子請死,晉侯欲許之,士貞子諫曰,不可,城濮之役,晉師三日穀,文公猶有憂色,左右曰,有喜而憂,如有憂而喜乎,公曰,得臣猶在,憂未歇也,困獸猶鬥,況國相乎,及楚殺子玉,公喜而後可知也,曰,莫余毒也已,是晉再克,而楚再敗也,楚是以再世不競,今天或者大警晉也,而又殺林父以重楚勝,其無乃久不競乎,林父之事君也,進思盡忠,退思補過,社稷之衛也,若之何殺之,夫其敗也,如日月之食焉,何損於明,晉侯使復其位。

冬,楚子伐蕭,宋華椒以蔡人救蕭,蕭人囚熊相宜僚,及公子丙,王曰,勿殺,吾退,蕭人殺之,王怒,遂圍蕭,蕭潰,申公巫臣曰,師人多寒,王巡三軍,拊而勉之,三軍之士,皆如挾纊,遂傅於蕭,還無社與司馬卯言,號申叔展,叔展曰,有麥麴乎,曰,無,有山鞠窮乎,曰,無,河魚腹疾奈何,曰,目於眢井而拯之,若為茅絰,哭井則已,明日,蕭潰,申叔視其井,則茅絰存焉,號而出之。

晉原縠,宋華椒,衛孔達,曹人,同盟于清丘。曰,恤病討貳,於是卿不書,不實其言也,宋為盟故,伐陳,衛人救之。孔達曰,先君有約言焉,若大國討,我則死之。

十有三年,春,齊師伐莒。

夏,楚子伐宋。

秋,螽,冬,晉殺其大夫先縠。

十三年,春,齊師伐莒,莒恃晉而不事齊故也。

夏,楚子伐宋,以其救蕭也,君子曰,清丘之盟,唯宋可以免焉。

秋,赤狄伐晉,及清,先縠召之也。

冬,晉人討邲之敗,與清之師,歸罪於先縠而殺之,盡滅其族,君子曰,惡之來也,己則取之,其先縠之謂乎。

清丘之盟,晉以衛之救陳也,討焉,使人弗去,曰,罪無所歸,將加而師,孔達曰苟利社稷,請以我說,罪我之由,我則為政,而亢大國之討,將以誰任,我則死之。

十有四年,春,衛殺其大夫孔達,夏,五月,壬申,曹伯壽卒。

晉侯伐鄭。

秋,九月,楚子圍宋。

葬曹文公。

冬,公孫歸父會齊侯于穀。

十四年,春,孔達縊而死,衛人以說于晉,而免,遂告于諸侯曰,寡君有不令之臣達,構我敝邑于大國,既伏其罪矣,敢告,衛人以為成勞,復室其子,使復其位。

夏,晉侯伐鄭,為邲故也,告於諸侯,蒐焉而還,中行桓子之謀也,曰,示之以整,使謀而來,鄭人懼,使子張代子良于楚,鄭伯如楚,謀晉故也,鄭以子良為有禮,故召之。

楚子使申舟聘于齊,曰,無假道于宋,亦使公子馮聘于晉,不假道于鄭,申舟以孟諸之役惡宋,曰,鄭昭,宋聾,晉使不害,我則必死,王曰,殺女,我伐之。見犀而行。及宋,宋人止之,華元曰,過我而不假道,鄙我也,鄙我,亡也,殺其使者,必伐我,伐我,亦亡也,亡一也,乃殺之,楚子聞之,投袂而起,屨及於窒皇,劍及於寢門之外,車及於蒲胥之市,秋,九月,楚子圍宋。

冬,公孫歸父會齊侯于穀,見晏桓子,與之言魯樂,桓子告高宣子,曰,子家其亡乎,懷於魯矣,懷必貪,貪必謀人,謀人,人亦謀己,一國謀之,何以不亡。

孟獻子言於公曰,臣聞小國之免於大國也,聘而獻物,於是有庭實旅百,朝而獻功,於是有容貌,采章,嘉淑,而有加貨,謀其不免也,誅而薦賄,則無及也,今楚在宋,君其圖之,公說。

十有五年,春,公孫歸父會楚子于宋。

夏,五月,宋人及楚人平。

六月,癸卯,晉師滅赤狄潞氏,以潞子嬰兒歸。

秦人伐晉。

王札子殺召伯,毛伯。

秋螽。

仲孫蔑,會齊高固于無婁。

初稅畝。

冬,蝝生。

饑。

十五年,春,公孫歸父會楚子于宋。

宋人使樂嬰齊告急于晉,晉侯欲救之,伯宗曰,不可,古人有言曰,雖鞭之長,不及馬腹,天方授楚,未可與爭,雖晉之彊,能違天乎,諺曰,高下在心,川澤納汙,山藪藏疾,瑾瑜匿瑕,國君含垢,天之道也,君其待之,乃止,使解揚如宋,使無降楚,曰,晉師悉起,將至矣,鄭人囚而獻諸楚,楚子厚賂之,使反其言,不許,三而許之,登諸樓車,使呼宋而告之,遂致其君命,楚子將殺之,使與之言曰,爾既許不穀,而反之,何故,非我無信,女則棄之,速即爾刑,對曰,臣聞之,君能制命為義,臣能承命為信,信載義而行之為利,謀不失利,以衛社稷,民之主也,義無二信,信無二命,君之賂臣,不知命也,受命以出,有死無霣,又可賂乎,臣之許君,以成命也,死而成命,臣之祿也,寡君有信臣,下臣獲考,死又何求,楚子舍之以歸。

夏,五月,楚師將去宋,申犀稽首於王之馬前,曰,毋畏知死,而不敢廢王命,王棄言焉,王不能荅,申叔時僕曰,築室反耕者,宋必聽命,從之,宋人懼,使華元夜入楚師,登子反之床,起之曰,寡君使元以病告,曰,敝邑易子而食,析骸以爨。雖然,城下之盟,有以國斃,不能從也,去我三十里,唯命是聽,子反懼,與之盟,而告王。退三十里,宋及楚平,華元為質。盟曰:我無爾詐,爾無我虞。

潞子嬰兒之夫人,晉景公之姊也,酆舒為政而殺之,又傷潞子之目,晉侯將伐之,諸大夫皆曰,不可,酆舒有三雋才,不如待後之人,伯宗曰,必伐之,狄有五罪,雋才雖多,何補焉,不祀,一也,耆酒,二也,棄仲章而奪黎氏地,三也,虐我伯姬,四也,傷其君目,五也,怙其雋才,而不以茂德,茲益罪也,後之人,或者將敬奉德義,以事神人,而申固其命,若之何待之,不討有罪,曰,將待後,後有辭而討焉,毋乃不可乎,夫恃才與眾,亡之道也,商紂由之故滅。天反時為災,地反物為妖,民反德為亂。亂則妖災生,故文反正為乏,盡在狄矣,晉侯從之,六月,癸卯,晉荀林父敗赤狄于曲梁,辛亥,滅潞,酆舒奔衛,衛人歸諸晉,晉人殺之。

王孫蘇與召氏,毛氏,爭政,使王子捷殺召戴公,及毛伯衛,卒立召襄。

秋,七月,秦桓公伐晉,次于輔氏,壬午,晉侯治兵于稷,以略狄土,立黎侯而還,及雒,魏顆敗秦師于輔氏,獲杜回,秦之力人也,初,魏武子有嬖妾,無子,武子疾,命顆曰,必嫁是,疾病則曰,必以為殉,及卒,顆嫁之,曰,疾病則亂,吾從其治也,及輔氏之役,顆見老人,結草以亢杜回,杜回躓而顛,故獲之,夜夢之曰,余,而所嫁婦人之父也,爾用先人之治命,余是以報。

晉侯賞桓子狄臣千室,亦賞士伯以瓜衍之縣,曰,吾獲狄土,子之功也,微子,吾喪伯氏矣,羊舌職說是賞也,曰,《周書》所謂庸庸祗祗者,謂此物也夫,士伯庸中行伯,君信之,亦庸士伯,此之謂明德矣,文王所以造周,不是過也,故《詩》曰:陳錫哉周,能施也,率是道也,其何不濟。

晉侯使趙同,獻狄俘于周,不敬,劉康公曰,不及十年,原叔必有大咎,天奪之魄矣。

初,稅畝,非禮也,穀出不過藉,以豐財也。

冬蝝生,饑,幸之也。

十有六年,春,王正月,晉人滅赤狄甲氏,及留吁。

夏,成周宣榭火。

秋,郯伯姬來歸。

冬,大有年。

十六年,春,晉士會帥師滅赤狄甲氏,及留吁,鐸辰,三月,獻狄俘晉侯,請于王,戊申,以黻冕命士會將中軍,且為大傅,於是晉國之盜,逃奔于秦,羊舌職曰,吾聞之,禹稱善人,不善人遠,此之謂也,夫詩曰,戰戰兢兢,如臨深淵,如履薄冰,善人在上也,善人在上,則國無幸民,諺曰,民之多幸,國之不幸也,是無善人之謂也。

夏,成周宣榭火,人火之也,凡火,人火曰火,天火曰災。

秋,郯伯姬來歸,出也。

為毛召之難故,王室復亂,王孫蘇奔晉,晉人復之。

冬,晉侯使士會平王室,定王享之,原襄公相禮,殽烝,武子私問其故,王聞之,召武子曰,季氏,而弗聞乎,王享有體薦,晏有折俎,公當享,卿當宴,王室之禮也,武子歸而講求典禮,以脩晉國之法。

十有七年,春,王正月,庚子,許男錫我卒,丁未,蔡侯申卒。

夏,葬許昭公,葬蔡文公。

六月,癸卯,日有食之。

己未,公會晉侯,衛侯,曹伯,邾子,同盟于斷道。

秋,公至自會。

冬,十有一月,壬午,公弟叔肸卒。

十七年,春,晉侯使郤克徵會于齊,齊頃公帷婦人使觀之,郤子登,婦人笑於房。獻子怒,出而誓曰:「所不此報,無能涉河。」獻子先歸,使欒京廬待命于齊,曰,不得齊事,無復命矣,郤子至,請伐齊,晉侯弗許,請以其私屬,又弗許,齊侯使高固,晏弱,蔡朝,南郭偃,會,及斂盂,高固逃歸,夏,會于斷道,討貳也,盟于卷楚,辭齊人,晉人執晏弱于野王,執蔡朝于原,執南郭偃于溫,苗賁皇使,見晏桓子,歸言於晉侯曰,夫晏子何罪,昔者諸侯事吾先君,皆如不逮,舉言群臣不信,諸侯皆有貳志,齊君恐不得禮,故不出,而使四子來,左右或沮之,曰,君不出,必執吾使,故高子及斂盂而逃,夫三子者曰,若絕君好,寧歸死焉,為是犯難而來,吾若善逆彼,以懷來者,吾又執之,以信齊沮,吾不既過矣乎,過而不改,而又久之,以成其悔,何利之有焉,使反者得辭,而害來者,以懼諸侯,將焉用之,晉人緩之,逸。

秋,八月,晉師還。

范武子將老,召文子曰,燮乎,吾聞之,喜怒以類者鮮,易者實多,詩曰,君子如怒,亂庶遄沮,君子如祉,亂庶遄已,君子之喜怒,以已亂也,弗已者,必益之,郤子其或者欲已亂於齊乎,不然,余懼其益之也,余將老,使郤子逞其志,庶有豸乎,爾從二三子,唯敬,乃請老,郤獻子為政。

冬,公弟叔肸卒,公母弟也,凡大子之母,弟公在曰公子,不在曰弟,凡稱弟,皆母弟也。

十有八年,春,晉侯,衛世子臧,伐齊。

公伐杞。

夏,四月。

秋,七月,邾人戕鄫子于鄫。

甲戌,楚子旅卒。

公孫歸父如晉。

冬,十月,壬戌,公薨于路寢。

歸父還自晉,至笙,遂奔齊。

十八年,春,晉侯,衛大子臧,伐齊,至于陽穀,齊侯會晉侯盟于繒,以公子彊為質于晉,晉師還,蔡朝,南郭偃,逃歸。

夏,公使如楚乞師,欲以伐齊。

秋,邾人戕鄫子于鄫,凡自虐其君曰弒,自外曰戕。

楚莊王卒,楚師不出,既而用晉師,楚於是乎有蜀之役。

公孫歸父以襄仲之立公也,有寵,欲去三桓,以張公室,與公謀而聘于晉,欲以晉人去之,冬,公薨,季文子言於朝曰,使我殺適立庶,以失大援者,仲也夫,臧宣叔怒曰,當其時,不能治也,後之人何罪,子欲去之,許請去之,遂逐東門氏,子家還及笙,壇帷,復命於介,既復命,袒括髮,即位哭,三踊而出,遂奔齊,書曰,歸父還自晉,善之也。


\end{pinyinscope}