\article{昭公}

\begin{pinyinscope}
元年春,王正月,公即位。

叔孫豹會晉趙武,楚公子圍,齊國弱,宋向戌,衛齊惡,陳公子招,蔡公孫歸生,鄭罕虎,許人,曹人,于虢。

三月取鄆。

夏,秦伯之弟鍼出奔晉。

六月丁巳,邾子華卒。

晉荀吳帥師敗狄于大鹵。

秋,莒去疾自齊入于莒,莒展輿出奔吳。

叔弓帥師疆鄆田。

葬邾悼公。

冬,十有一月,己酉,楚子麇卒。

公子比出奔晉。

元年春,楚公子圍聘于鄭,且娶於公孫段氏,伍舉為介,將入館,鄭人惡之,使行人子羽與之言,乃館於外,既聘,將以眾逆,子產患之,使子羽辭曰,以敝邑褊小,不足以容從者,請墠聽命,令尹命大,宰伯州犁對曰,君辱貺寡大夫圍,謂圍將使豐氏,撫有而室,圍布几筵,告於莊共之廟而來,若野賜之,是委君貺於草莽也,是寡大夫不得列於諸卿也,不寧唯是,又使圍蒙其先君,將不得為寡君老其蔑以復矣,唯大夫圖之,子羽曰,小國無罪,恃實其罪,將恃大國之安靖己,而無乃包藏禍心以圖之,小國失恃,而懲諸侯,使莫不憾者,距違君命,而有所壅塞不行是懼,不然,敝邑館人之屬也,其敢愛豐氏之祧,伍舉知其有備也,請垂櫜而入,許之,正月乙未,入逆而出,遂會於虢,尋宋之盟也,祁午謂趙文子曰,宋之盟,楚人得志於晉,今令尹之不信,諸侯之所聞也,子弗戒,懼又如宋,子木之信,稱於諸侯,猶詐晉而駕焉,況不信之尤者乎,楚重得志於晉,晉之恥也,子相晉國,以為盟主,於今七年矣,再合諸侯,三合大夫,服齊狄,寧東夏,平秦亂,城淳于,師徒不頓,國家不罷,民無謗讟,諸侯無怨,天無大災,子之力也,有令名矣,而終之以恥午也是懼,吾子其不可以不戒,文子曰,武受賜矣,然宋之盟,子木有禍人之心,武有仁人之心,是楚所以駕於晉也,今武猶是心也,楚又行僭,非所害也,武將信以為本,循而行之,譬如農夫,是穮是蔉,雖有饑饉,必有豐年,且吾聞之,能信不為人下,吾未能也,詩曰,不僭不賊,鮮不為則,信也,能為人則者,不為人下矣,吾不能是難,楚不為患,楚令尹圍請用牲,讀舊書,加于牲上而已,晉人許之,三月,甲辰,盟,楚公子圍設服離衛,叔孫穆子曰,楚公子美矣,君哉,鄭子皮曰,二執戈者前矣,蔡子家曰,蒲宮有前,不亦可乎,楚伯州犁曰,此行也,辭而假之寡君,鄭行人揮曰,假不反矣,伯州犁曰,子姑憂子皙之欲背誕也,子羽曰,當璧猶在,假而不反,子其無憂乎,齊國子曰,吾代二子愍矣,陳公子招曰,不憂何成,二子樂矣,衛齊子曰,苟或知之,雖憂何害,宋合左師曰,大國令,小國共,吾知共而已,晉樂王鮒曰,小旻之卒章善矣,吾從之,退會,子羽謂子皮曰,叔孫絞而婉,宋左師簡而禮,樂王鮒字而敬,子與子家持之,皆保世之主也,齊衛陳大夫,其不免乎,國子代人憂,子招樂憂,齊子雖憂弗害,夫弗及而憂,與可憂而樂,與憂而弗害,皆取憂之道也,憂必及之,大誓曰,民之所欲,天必從之,三大夫兆憂,能無至乎,言以知物,其是之謂矣。

季武子伐莒,取鄆,莒人告於會,楚告於晉曰,尋盟未退,而魯伐莒,瀆齊盟,請戮其使,樂桓子相趙文子,欲求貨於叔孫,而為之請,使請帶焉弗與,梁其脛曰,貨以藩身,子何愛焉,叔孫曰,諸侯之會,衛社稷也,我以貨免,魯必受師,是禍之也,何衛之為,人之有牆,以蔽惡也,牆之隙壞,誰之咎也,衛而惡之,吾又甚焉,雖怨季孫,魯國何罪,叔出季處,有自來矣,吾又誰怨,然鮒也賄,弗與不已,召使者裂裳帛而與之,曰帶其褊矣,趙孟聞之曰,臨患不忘國,忠也,思難不越官,信也,圖國忘死,貞也,謀主三者,義也,有是四者,又可戮乎,乃請諸楚,曰,魯雖有罪,其執事不辟難,畏威而敬命矣,子若免之,以勸左右可也,若子之群吏,處不辟污,出不逃難,其何患之有,患之所生,污而不治,難而不守,所由來也,能是二者,又何患焉,不靖其能,其誰從之,魯叔孫豹可謂能矣,請免之以靖能者,子會而赦有罪,又賞其賢,諸侯其誰不欣焉,望楚而歸之,視遠如邇,疆場之邑,一彼一此,何常之有,王伯之令也,引其封疆,而樹之官,舉之表旗,而著之制令,過則有刑,猶不可壹,於是乎虞有三苗,夏有觀扈,商有姺邳,周有徐奄,自無令王諸侯逐進,狎主齊盟,其又可壹乎,恤大舍小,足以為盟主,又焉用之,封疆之削,何國蔑有。主齊盟者,誰能辯焉?吳濮有釁,楚之執事,豈其顧盟,莒之疆事,楚勿與知,諸侯無煩,不亦可乎,莒魯爭鄆,為日久矣,苟無大害於其社稷,可無亢也,去煩宥善,莫不競勸,子其圖之,固請諸楚,楚人許之,乃免叔孫,令尹享趙孟,賦大明之首章,趙孟賦小宛之二章,事畢,趙孟謂叔向曰,令尹自以為王矣,何如,對曰,王弱,令尹疆,其可哉,雖可不終,趙孟曰,何故,對曰,彊以克弱而安之,彊不義也,不義而彊,其斃必速,詩曰,赫赫宗周,褒姒滅之,彊不義也,令尹為王,必求諸侯,晉少懦矣,諸侯將往,若獲諸侯,其虐滋甚,民弗堪也,將何以終,夫以彊取,不義而克,必以為道,道以淫虐,弗可久已矣。

夏,四月,趙孟,叔孫豹,曹大夫,入于鄭,鄭伯兼享之,子皮戒趙孟,禮終,趙孟賦瓠葉,子皮遂戒穆叔,且告之,穆叔曰,趙孟欲一獻,子其從之,子皮曰,敢乎,穆叔曰,夫人之所欲也,又何不敢,及享,具五獻之籩豆於幕下,趙孟辭,私於子產曰,武請於冢宰矣,乃用一獻,趙孟為客,禮終乃宴,穆叔賦鵲巢,趙孟曰,武不堪也,又賦采蘩,曰,小國為蘩,大國省穡而用之,其何實非命,子皮賦野有死麇之卒章,趙孟賦常棣,且曰吾兄弟比以安,尨也可使無吠,穆叔,子皮,及曹大夫,興拜,舉兕爵曰,小國賴子,知免於戾矣,飲酒樂,趙孟出,曰,吾不復此矣,天王使劉定公勞趙孟於潁,館於雒汭,劉子曰,美哉禹功,明德遠矣,微禹,吾其魚乎,吾與子弁冕端委,以治民臨諸侯,禹之力也,子盍亦遠績禹功,而大庇民乎,對曰,老夫罪戾是懼,焉能恤遠,吾儕偷食,朝不謀夕,何其長也,劉子歸以語王曰,諺所為老將知而耄及之者,其趙孟之謂乎,為晉正卿,以主諸侯,而儕於隸入,朝不謀夕,棄神人矣,神怒民叛,何以能久,趙孟不復年矣,神怒不歆其祀,民叛不即其事,祀事不從,又何以年。

叔孫歸,曾夭御季孫以勞之,旦及日中,不出,曾夭謂曾阜曰,旦及日中,吾知罪矣,魯以相忍為國也,忍其外,不忍其內,焉用之,阜曰,數月於外,一旦於是,庸何傷,賈而欲贏,而惡囂乎,阜謂叔孫曰,可以出矣,叔孫指楹曰,雖惡是,其可去乎,乃出見之。

鄭徐吾犯之妹美,公孫楚聘之矣,公孫黑又使強委禽焉,犯懼,告子產,子產曰,是國無政,非子之患也,唯所欲與,犯請於二子,請使女擇焉,皆許之,子皙盛飾入,布幣而出,子南戎服入,左右射,超乘而出,女自房觀之,曰,子皙信美矣,抑子南夫也,夫夫婦婦,所謂順也,適子南氏,子皙怒,既而櫜甲以見子南,欲殺之,而取其妻,子南知之,執戈逐之,及衝,擊之以戈,子皙傷而歸,告大夫曰,我好見之,不知其有異志也,故傷,大夫皆謀之,子產曰,直鈞幼賤,有罪,罪在楚也,乃執子南而數之曰,國之大節有五,女皆奸之,畏君之威,聽其政,尊其貴,事其長,養其親,五者所以為國也,今君在國,女用兵焉,不畏威也,奸國之紀,不聽政也,子皙上大夫,女嬖大夫,而弗下之,不尊貴也,幼而不忌,不事長也,兵其從兄,不養親也,君曰,余不女忍殺,宥女以遠。勉速行乎!無重而罪!五月,庚辰,鄭放游楚於吳,將行子南,子產咨於大叔,大叔曰,吉不能亢身,焉能亢宗,彼國政也,非私難也,子圖鄭國,利則行之,又何疑焉,周公殺管叔而蔡蔡叔,夫豈不愛,王室故也,吉若獲戾,子將行之,何有於諸游。

秦后子有寵於桓,如二君於景。其母曰:『弗去,懼選。』癸卯,鍼適晉,其車千乘,書曰,秦伯之弟鍼出奔晉,罪秦伯也,后子享晉侯,造舟于河,十里舍車,自雍及絳,歸取酬幣,終事八反,司馬侯問焉,曰,子之車盡於此而已乎,對曰,此之謂多矣,若能少此,吾何以得見,女叔齊以告公,且曰,秦公子必歸,臣聞君子能知其過,必有令圖,令圖,天所贊也,后子見趙孟,趙孟曰,吾子其曷歸,對曰,鍼懼選於寡君,是以在此,將待嗣君,趙孟曰,秦君何如,對曰,無道,趙孟曰,亡乎,對曰,何為,一世無道,國未艾也,國於天地,有與立焉,不數世淫,弗能斃也,趙孟曰,天乎,對曰,有焉,趙孟曰,其幾何,對曰,鍼聞之,國無道而年穀和熟,天贊之也,鮮不五稔,趙孟視蔭曰,朝夕不相及,誰能待五,后子出而告人曰,趙孟將死矣,主民,翫歲而愒日,其與幾何。

鄭為游楚亂故,六月,丁巳,鄭伯及其大夫盟于公孫段氏,罕虎,公孫僑,公孫段,印段,游吉,駟帶,私盟于閨門之外,實薰隧,公孫黑強與於盟,使大史書其名,且曰七子,子產弗討。

晉中行穆子敗無終及群狄于大原,崇卒也,將戰,魏舒曰,彼徒我車,所遇又阨,以什共車,必克,困諸阨,又克,請皆卒,自我始,乃毀車以為行,五乘為三伍,荀吳之嬖人不肯即卒,斬以徇,為五陳以相離,兩於前,伍於後,專為右角,參為左角,偏為前拒,以誘之,翟人笑之,未陳而薄之,大敗之。

莒展輿立,而奪群公子秩,公子召去疾于齊,秋,齊公子鉏納去疾,展輿奔吳,叔弓帥師疆鄆田,因莒亂也,於是莒務婁,瞀胡,及公子滅明,以大厖,與常儀靡,奔齊,君子曰,莒展之不立,棄人也夫,人可棄乎,詩曰,無競維人,善矣。

晉侯有疾,鄭伯使公孫僑如晉聘,且問疾,叔向問焉,曰,寡君之疾病,卜人曰,實沈臺駘為祟,史莫之知,敢問此何神也,子產曰,昔高辛氏有二子,伯曰閼伯,季曰實沈,居于曠林,不相能也,日尋干戈,以相征討,后帝不臧,遷閼伯于商丘,主辰,商人是因,故辰為商星,遷實沈于大夏,主參,唐人是因,以服事夏商,其季世曰唐叔虞,當武王邑姜,方震大叔,夢帝謂已,余命而子曰虞,將與之唐,屬諸參而蕃育其子孫,及生有文在其手,曰虞,遂以命之,及成王滅唐而封大叔焉,故參為晉星,由是觀之,則實沈,參神也,昔金天氏有裔子曰昧,為玄冥師,生允格,臺駘,臺駘能業其官,宣汾洮,障大澤,以處大原,帝用嘉之,封諸汾川,沈,姒,蓐,黃,實守其祀,今晉主汾而滅之矣,由是觀之,則臺駘,汾神也,抑此二者,不及君身,山川之神,則水旱癘疫之災,於是乎禜之,日月星辰之神,則雪霜風雨之不時,於是乎禜之,若君身,則亦出入飲食哀樂之事也,山川星辰之神,又何為焉,僑聞之,君子有四時,朝以聽政,晝以訪問,夕以脩令,夜以安身,於是乎節宣其氣,勿使有所壅閉湫底,以露其體,茲心不爽,而昏亂百度,今無乃壹之,則生疾矣,僑又聞之,內官不及同姓,其生不殖,美先盡矣,則相生疾,君子是以惡之,故志曰,買妾不知其姓,則卜之,違此二者,古之所慎也,男女辨姓,禮之大司也,今君內實有四姬焉,其無乃是也乎,若由是二者,弗可為也已,四姬有省,猶可無則,必生疾矣,叔向曰,善哉,肸未之聞也,此皆然矣,叔向出,行人揮送之,叔向問鄭故焉,且問子皙,對曰,其與幾何,無禮而好陵人,怙富而卑其上,弗能久矣,晉侯聞子產之言曰,博物君子也,重賄之,晉侯求醫於秦,秦伯使醫和視之,曰,疾不可為也,是謂近女室,疾如蠱,非鬼非食,惑以喪志,良臣將死,天命不祐,公曰,女不可近乎對曰,節之,先王之樂,所以節百事也,故有五節遲速本末以相及,中聲以降,五降之後,不容彈矣,於是有煩手淫聲,慆堙心耳,乃忘平和,君子弗聽也,物亦如之,至於煩,乃舍也已,無以生疾,君子之近琴瑟,以儀節也,非以慆心也,天有六氣,降生五味,發為五色,徵為五聲,淫生六疾,六氣曰陰,陽,風,雨,晦明也,分為四時,序為五節,過則為菑,陰淫寒疾,陽淫熱疾,風淫末疾,雨淫腹疾,晦淫惑疾,明淫心疾,女陽物而晦時,淫則生內熱惑蠱之疾,今君不節不時能無及此乎,出告趙孟,趙孟曰,誰當良臣,對曰,主是謂矣,主相晉國,於今八年,晉國無亂,諸侯無闕,可謂良矣,和聞之,國之大臣,榮其寵祿,任其寵節,有菑禍興而無改焉,必受其咎,今君至於淫以生疾,將不能圖恤社稷,禍孰大焉,主不能禦,吾是以云也,趙孟曰,何謂蠱,對曰,淫溺惑亂之所生也,於文,皿蟲為蠱,穀之飛亦為蠱,在周易,女惑男,風落山,謂之蠱,皆同物也,趙孟曰,良醫也,厚其禮而歸之。

楚公子圍使公子黑肱,伯州犁,城犨,櫟,郟,鄭人懼,子產曰,不害,令尹將行大事,而先除二子也,禍不及鄭,何患焉,冬,楚公子圍將聘于鄭,伍舉為介,未出竟,聞王有疾而還,伍舉遂聘,十一月,己酉,公子圍至,入問王疾,縊而弒之,遂殺其二子幕及平夏,右尹子干出奔晉,宮廄尹子皙出奔鄭,殺大宰伯州犁于郟,葬王于郟,謂之郟敖,使赴于鄭,伍舉問應為後之辭焉,對曰,寡大夫圍,伍舉更之曰,共王之子圍為長,子干奔晉,從車五乘,叔向使與秦公子同食,皆百人之餼,趙文子曰,秦公子富,叔向曰,底祿以德,德鈞以年,年同以尊,公子以國,不聞以富,且夫以千乘去其國,彊禦已甚,詩曰,不侮鰥寡,不畏彊禦,秦楚匹也,使后子與子干齒辭。曰,鍼懼選,楚公子不獲,是以皆來,亦唯命,且臣與羇齒,無乃不可乎。史佚有言曰,非羇何忌,楚靈王即位,薳罷為令尹,薳啟彊為大宰,鄭游吉如楚葬郟敖,且聘立君,歸,謂子產曰,具行器矣,楚王汰侈,而自說其事,必合諸侯,吾往無日矣,子產曰,不數年,未能也,十二月,晉既烝,趙孟適南陽將會孟子餘,甲辰朔,烝于溫,庚戌,卒,鄭伯如晉,弔及雍乃復。

二年,春,晉侯使韓起來聘。

夏,叔弓如晉。

秋,鄭殺其大夫公孫黑。

冬,公如晉,至河乃復。

季孫宿如晉。

二年,春,晉侯使韓宣子來聘,且告為政,而來見禮也,觀書於大史氏,見易象與魯春秋,曰,周禮盡在魯矣,吾乃今知周公之德,與周之所以王也,公享之,季武子賦綿之卒章,韓子賦角弓,季武子拜曰,敢拜子之彌縫敝邑,寡君有望矣,武子賦節之卒章,既享,宴于季氏,有嘉樹焉,宣子譽之,武子曰,宿敢不封殖此樹以無忘角弓,遂賦甘棠,宣子曰,起不堪也,無以及召公,宣子遂如齊納幣,見子雅,子雅召子旗,使見宣子,宣子曰,非保家之主也,不臣,見子尾,子尾見彊,宣子謂之如子旗,大夫多笑之,唯晏子信之,曰,夫子君子也,君子有信,其有以知之矣,自齊聘於衛,衛侯享之,北宮文子賦淇澳,宣子賦木瓜,夏,四月,韓須如齊逆女,齊陳無宇送女,致少姜,少姜有寵於晉侯,晉侯謂之少齊,謂陳無宇非卿,執諸中都,少姜為之請曰,送從逆班,畏大國也,猶有所易,是以亂作。

叔弓聘于晉,報宣子也,晉侯使郊勞,辭曰,寡君使弓來繼舊好,固曰,女無敢為賓,徹命於執事,敝邑弘矣,敢辱郊使,請辭,致館,辭曰,寡君命下臣來繼舊好,好合使成,臣之祿也,敢辱大館,叔向曰,子叔子知禮哉,吾聞之曰,忠信,禮之器也,卑讓,禮之宗也,辭不忘國,忠信也,先國後己,卑讓也,詩曰,敬慎威儀,以近有德,夫子近德矣。

秋,鄭公孫黑將作亂,欲去游氏而代其位,傷疾作而不果,駟氏與諸大夫欲殺之,子產在鄙,聞之,懼弗及,乘遽而至,使吏數之,曰,伯有之亂,以大國之事,而未爾討也,爾有亂心無厭,國不女堪,專伐伯有,而罪一也,昆弟爭室,而罪二也,薰隧之盟,女矯君位,而罪三也,有死罪三,何以堪之,不速死,大刑將至,再拜稽首辭曰,死在朝夕,無助天為虐,子產曰,人誰不死,凶人不終,命也,作凶事,為凶人,不助天,其助凶人乎,請以印為褚師,子產曰,印也若才,君將任之,不才,將朝夕從女,女罪之不恤,而又何請焉,不速死,司寇將至,七月,壬寅,縊,尸諸周氏之衢,加木焉。

晉少姜卒,公如晉,及河,晉侯使士,文伯來辭曰,非伉儷也,請君無辱,公還,季孫宿遂致服焉叔,向言陳無宇於晉侯曰,彼何罪,君使公族逆之,齊使上大夫送之,猶曰不共,君求以貪,國則不共,而執其使,君刑己頗,何以為盟主,且少姜有辭,冬,十月,陳無宇歸。

十一月,鄭印段如晉弔。

三年,春,王正月,丁未,滕子原卒。

夏,叔弓如滕。

五月,葬滕成公。

秋,小邾子來朝。

八月,大雩。

冬,大雨雹。

北燕伯款出奔齊。

三年,春,王正月,鄭游吉如晉,至少姜之葬,梁丙與張趯見之,梁丙曰,甚矣哉,子之為此來也,子大叔曰,將得已乎,昔文襄之霸也,其務不煩諸侯,令諸侯三歲而聘,五歲而朝,有事而會,不協而盟,君薨大夫弔,卿共葬事,夫人士弔,大夫送葬,足以昭禮命,事謀闕而已,無加命矣,今嬖寵之喪,不敢擇位,而數於守適,唯懼獲戾,豈敢憚煩,少姜有寵而死,齊必繼室,今茲吾又將來賀,不唯此行也,張趯曰,善哉,吾得聞此數也,然自今子其無事矣,譬如火焉,火中,寒暑乃退,此其極也,能無退乎,晉將失諸侯,諸侯求煩不獲,二大夫退,子大叔告人曰,張趯有知,其猶在君子之後乎。

丁未,滕子原卒,同盟,故書名。

齊侯使晏嬰請繼室於晉,曰,寡君使嬰曰,寡人願事君,朝夕不倦,將奉質幣,以無失時,則國家多難,是以不獲,不腆先君之適,以備內官,焜燿寡人之望,則又無祿,早世隕命,寡人失望,君若不忘先君之好,惠顧齊國,辱收寡人,徼福於大公丁公,照臨敝邑,鎮撫其社稷,則猶有先君之適,及遺姑姊妹若而人,君若不棄敝邑,而辱使董振擇之,以備嬪嬙,寡人之望也,韓宣子使叔向對曰,寡君之願也,寡君不能獨任其社稷之事,未有伉儷,在縗絰之中,是以未敢請,君有辱命,惠莫大焉,若惠顧敝邑,撫有晉國,賜之內主,豈惟寡君,舉群臣實受其貺其自唐叔以下,實寵嘉之,既成昏,晏子受禮,叔向從之晏,相與語,叔向曰,齊其何如,晏子曰,此季世也,吾弗知,齊其為陳氏矣,公棄其民,而歸於陳氏,齊舊四量,豆,區,釜,鍾,四升為豆,各自其四,以登於釜,釜十則鍾,陳氏三量,皆登一焉,鍾乃大矣,以家量貸,而以公量收之,山木如市,弗加於山,魚鹽蜃蛤,弗加於海,民參其力,二入於公,而衣食其一,公聚朽蠹,而三老凍餒,國之諸市,屨賤踊貴,民人痛疾,而或燠休之,其愛之如父母,而歸之如流水,欲無獲民,將焉辟之,箕伯,直柄,虞遂,伯戲,其相胡公大姬,已在齊矣,叔向曰,然。雖吾公室,今亦季世也,戎馬不駕,卿無軍行,公乘無人,卒列無長,庶民罷敝,而宮室滋侈,道殣相望,而女富溢尤。民聞公命,如逃寇讎,欒,郤,胥,原,狐,續,慶,伯,降在皁隸,政在家門,民無所依,君日不悛,以樂慆憂,公室之卑,其何日之有,讒鼎之銘曰,昧旦丕顯,後世猶怠,況日不悛,其能久乎,晏子曰,子將若何,叔向曰,晉之公族盡矣,肸聞之,公室將卑,其宗族枝葉先落,則公從之,肸之宗十一族,唯羊舌氏在而已,肸又無子,公室無度,幸而得死,豈其獲祀,初,景公欲更晏子之宅,曰,子之宅近市,湫隘囂塵,不可以居,請更諸爽塏者,辭曰,君之先臣容焉,臣不足以嗣之,於臣侈矣,且小人近市,朝夕得所求,小人之利也,敢煩里旅,公笑曰,子近市,識貴賤乎,對曰,既利之,敢不識乎,公曰,何貴何賤,於是景公繁於刑,有鬻踊者,故對曰,踊貴屨賤,既已告於君,故與叔向語而稱之,景公為是省於刑,君子曰,仁人之言,其利博哉,晏子一言而齊侯省刑,詩曰,君子如祉,亂庶遄已,其是之謂乎,及晏子如晉,公更其宅,反則成矣,既拜乃毀之,而為里室,皆如其舊,則使宅人反之,且諺曰,非宅是卜,唯鄰是卜,二三子先卜鄰矣,違卜不祥,君子不犯非禮,小人不犯不祥,古之制也,吾敢違諸乎,卒復其舊宅,公弗許,因陳桓子以請,乃許之。

夏,四月,鄭伯如晉,公孫段相,甚敬而卑,禮無違者。晉侯嘉焉,授之以策,曰:「子豐有勞於晉國,余聞而弗忘,賜女州田,以胙乃舊勳。」伯石再拜稽首,受策以出。君子曰,禮其人之急也乎,伯石之汏也,一為禮於晉,猶荷其祿,況以禮終始乎,詩曰,人而無禮,胡不遄死,其是之謂乎,初,州縣欒豹之邑也,及欒氏亡,范宣子,趙文子,韓宣子,皆欲之,文子曰,溫吾縣也,二宣子曰,自郤稱以別三傳矣,晉之別縣,不唯州,誰獲治之,文子病之,乃舍之,二子曰,吾不可以正議而自與也,皆舍之,及文子為政,趙獲曰,可以取州矣,文子曰,退,二子之言義也,違義禍也,余不能治余縣,又焉用州,其以徼禍也,君子曰,弗知實難,知而弗從,禍莫大焉,有言州必死,豐氏故主,韓氏伯石之獲州也,韓宣子為之請之,為其復取之之故。

五月,叔弓如滕,葬滕成公,子服椒為介,及郊,遇懿伯之忌,敬子不入,惠伯曰,公事有公利,無私忌,椒請先入,乃先受館,敬子從之。

晉韓起如齊逆女,公孫蠆為少姜之有寵也,以其子更公女,而嫁公子,人謂宣子,子尾欺晉,晉胡受之,宣子曰,我欲得齊而遠其寵,寵將來乎。

秋,七月,鄭罕虎如晉,賀夫人,旦告曰,楚人日徵敝邑,以不朝立王之故,敝邑之往,則畏執事,其謂寡君,而固有外心,其不往,則宋之盟云,進退罪也,寡君使虎布之,宣子使叔向對曰,君若辱有寡君,在楚何害,脩宋盟也,君苟思盟,寡君乃知免於戾矣,君若不有寡君,雖朝夕辱於敝邑,寡君猜焉,君實有心,何辱命焉,君其往也,苟有寡君,在楚猶在晉也,張趯使謂大叔曰,自子之歸也,小人糞除先人之敝廬,曰,子其將來,今子皮實來,小人失望,大叔曰,吉賤不獲來,畏大國尊夫人也,且孟曰而將無事,吉庶幾焉。

小邾穆公來朝,季武子欲卑之,穆叔曰,不可,曹滕二邾,實不忘我,好敬以逆之,猶懼其貳,又卑一睦焉,逆群好也,其如舊而加敬焉志曰,能敬無災,又曰,敬逆來者,天所福也,季孫從之。

八月,大雩,旱也。

齊侯田於莒,盧蒲嫳見,泣且請曰,余髮如此種種,余奚能為,公曰,諾,吾告二子,歸而告之,子尾欲復之,子雅不可,曰,彼其髮短而心甚長,其或寢處我矣,九月,子雅放盧蒲嫳于北燕,燕簡公多嬖寵,欲去諸大夫,而立其寵人,冬,燕大夫比以殺公之外嬖,公懼奔齊,書曰,北燕伯款出奔齊,罪之也。

十月,鄭伯如楚,子產相,楚子享之,賦吉日既享,子產乃具田備,王以田江南之夢。

齊公孫灶卒,司馬灶見晏子曰,又喪子雅矣,晏子曰,惜也子旗不免,殆哉姜族弱矣,而媯將始昌,二惠競爽,猶可,又弱一個焉,姜其危哉。

四年,春,王正月,大雨雹。

夏,楚子,蔡侯,陳侯,鄭伯,許男,徐子,滕子,頓子,胡子,沈子,小邾子,宋世子,佐淮夷會于申。

楚人執徐子。

秋,七月,楚子,蔡侯,陳侯,許男,頓子,胡子,沈子,淮夷,伐吳,執齊慶封,殺之,遂滅賴。

九月,取鄫。

冬,十有二月,乙卯,叔孫豹卒。

四年,春,王正月,許男如楚,楚子止之,遂止鄭伯,復田江南,許男與焉,使椒舉如晉求諸侯,二君待之椒舉致命曰,寡君使舉曰,日君有惠,賜盟于宋,曰,晉楚之從,交相見也,以歲之不易,寡人願結驩於二三君,使舉請間,君若苟無四方之虞,則願假寵以請於諸侯,晉侯欲勿許,司馬侯曰,不可,楚王方侈,天或者欲逞其心,以厚其毒而降之罰,未可知也,其使能終,亦未可知也,晉楚唯天,所相不可與爭,君其許之,而脩德以待其歸,若歸於德,吾猶將事之,況諸侯乎,若適淫虐,楚將棄之,吾又誰與爭,曰,晉有三不殆,其何敵之有,國險而多馬,齊楚多難,有是三者,何鄉而不濟,對曰,恃險與馬,而虞鄰國之難,是三殆也,四獄三塗,陽城大室,荊山中南,九州之險也,是不一姓,冀之北土,馬之所生,無興國焉,恃險與馬,不可以為固也,從古以然,是以先王務脩德音,以亨神人,不聞其務險與馬也,鄰國之難,不可虞也,或多難以固其國,啟其疆土,或無難以喪其國,失其守宇,若何虞難,齊有仲孫之難,而獲桓公,至今賴之,晉有里平之難,而獲文公,是以為盟主,衛邢無難,敵亦喪之,故人之難,不可虞也,恃此三者,而不脩政德,亡於不暇,又何能濟,君其許之,紂作淫虐,文王惠和,殷是以隕,周是以興,夫豈爭諸侯,乃許楚使,使叔向對曰,寡君有社稷之事,是以不獲春秋時見諸侯,君實有之,何辱命焉,椒舉遂請昏,晉侯許之,楚子問於子產曰,晉其許我諸侯乎,對曰,許君,晉君少安,不在諸侯,其大夫多求,莫匡其君,在宋之盟,又曰如一,若不許君,將焉用之,王曰,諸侯其來乎,對曰,必來,從宋之盟,承君之歡,不畏大國,何故不來,不來者,其魯衛曹邾乎,曹畏宋,邾畏魯,魯衛偪於齊而親於晉,唯是不來,其餘君之所及也,誰敢不至,王曰,然則吾所求者,無不可乎,對曰,求逞於人,不可,與人同欲,盡濟。

大雨雹,季武子問於申豐曰,雹可禦乎,對曰,聖人在上,無雹,雖有不為災,古者日在北陸,而藏冰西陸,朝覿而出之,其藏冰也,深山窮谷,固陰沍寒,於是乎取之,其出之也,朝之祿位,賓食喪祭,於是乎用之,其藏之也,黑牡秬黍,以享司寒,其出之也,桃弧棘矢,以除其災,其出入也,時食肉之祿,冰皆與焉,大大命婦,喪浴用冰,祭寒而藏之,獻羔而啟之,公始用之,火出而畢賦,自命夫命婦,至於老疾,無不受冰,山人取之,縣人傳之,輿人納之,隸人藏之,夫冰以風壯,而以風出,其藏之也周,其用之也遍,則冬無愆陽,夏無伏陰,春無凄風,秋無苦雨,雷出不震,無菑霜雹癘疾不降,民不夭札,今藏川池之冰,棄而不用,風不越而殺,雷不發而震,雹之為菑,誰能禦之,七月之卒章,藏冰之道也。

夏,諸侯如楚,魯,衛,曹,邾,不會,曹邾辭以難,公辭以時祭,衛侯辭以疾,鄭伯先待于申,六月丙午,楚子合諸侯于申,椒舉言於楚子曰,臣聞諸侯無歸,禮以為歸,今君始得諸侯,其慎禮矣,霸之濟否,在此會也,夏啟有鈞臺之享,商湯有景亳之命,周武有孟津之誓,成有岐陽之蒐,康有酆宮之朝,穆有塗山之會,齊桓有召陵之師,晉文有踐土之盟,君其何用,宋向戌,鄭公孫僑,在諸侯之良也,君其選焉,王曰,吾用齊桓,王使問禮於左師與子產,左師曰,小國習之,大國用之,敢不薦聞,獻公合諸侯之禮六,子產曰,小國共職,敢不薦守,獻伯子男會公之禮六,君子謂合左師善守先代,子產善相小國,王使椒舉侍於後以規過,卒事不規,王問其故,對曰,禮吾未見者有六焉,又何以規,宋大子佐後至,王田於武城,久而弗見,椒舉請辭焉,王使往曰,屬有宗祧之事於武城,寡君將墮幣焉,敢謝後見,徐子吳出也,以為貳焉,故執諸申,楚子示諸侯侈,椒舉曰,夫六王二公之事,皆所以示諸侯,禮也,諸侯所由用命也,夏桀為仍之會,有緡叛之,商紂為黎之蒐,東夷叛之,周幽為大室之盟,戎狄叛之,皆所以示諸侯,汏也,諸侯所由棄命也,今君以汏,無乃不濟乎,王弗聽,子產見左師曰,吾不患楚矣,汏而愎諫,不過十年,左師曰,然,不十年侈,其惡不遠,遠惡而後棄,善亦如之,德遠而後興。

秋七月,楚子以諸侯伐吳,宋大子鄭伯先歸,宋華費遂鄭大夫從,使屈申圍朱方,八月,甲申,克之,執齊慶封而盡滅其族,將戮慶封,椒舉曰,臣聞無瑕者可以戮人,慶封惟逆命,是以在此,其肯從於戮乎,播於諸侯,焉用之,王弗聽,負之釜鉞,以徇於諸侯,使言曰,無或如齊慶封,弒其君,弱其孤,以盟其大夫,慶封曰,無或如楚共王之庶子圍,弒其君兄之子,麇而代之,以盟諸侯,王使速殺之,遂以諸侯滅賴,賴子面縛銜璧,士袒輿櫬,從之,造於中軍,王問諸椒舉。對曰,成王克許,許僖公如是,王親釋其縛,受其璧,焚其櫬,王從之,遷賴於鄢,楚子欲遷許於賴,使鬥韋龜與公子棄疾,城之而還,申無宇曰,楚禍之首,將在此矣。召諸侯而來,伐國而克,城竟莫校,王心不違,民其居乎?民之不處,其誰堪之,不堪王命,乃禍亂也。

九月,取鄫,言易也,莒亂,著丘公立而不撫鄫,鄫叛而來,故曰取,凡克邑,不用師徒曰取。

鄭子產作丘賦,國人謗之,曰,其父死於路,己為蠆尾,以令於國,國將若之何,子寬以告,子產曰,何害,苟利社稷,死生以之,且吾聞為善者不改其度,故能有濟也,民不可逞,度不可改,詩曰,禮義不愆,何恤於人言,吾不遷矣,渾罕曰,國氏其先亡乎,君子作法於涼,其敝猶貪,作法於貪,敝將若之何,姬在列者,蔡及曹滕,其先亡乎,偪而無禮,鄭先衛亡,偪而無法,政不率法,而制於心,民各有心,何上之有。

冬,吳伐楚入,棘,櫟,麻,以報朱方之役,楚沈尹射奔命於夏汭,咸尹宜咎城鍾離,薳啟疆城巢,然丹城州來,東國水,不可以城,彭生罷賴之師。

初,穆子去叔孫氏,及庚宗,遇婦人,使私為食而宿焉,問其行,告之故,哭而送之,適齊娶於國氏,生孟丙仲壬,蔓天壓己,弗勝,顧而見人,黑而上僂,深目而豭喙,號之曰,牛助余,乃勝之,旦而皆召其徒,無之,且曰,志之,及宣伯奔齊,饋之,宣伯曰,魯以先子之故,將存吾宗,必召女,召女何如,對曰,願之久矣,魯人召之,不告而歸,既立,所宿庚宗之婦人,獻以雉,問其姓,對曰,余子長矣,能奉雉而從我矣,召而見之,則所夢也,未問其名,號之曰牛,曰唯,皆召其徒,使視之,遂使為豎,有寵,長使為政,公孫明知叔孫於齊,歸,未逆國姜,子明取之,故怒其子,長而後使逆之,田於丘蕕,遂遇疾焉,豎牛欲亂其室而有之,強與孟盟,不可,叔孫為孟鍾曰,爾未際,饗大夫以落之,既具,使豎牛請,日入弗謁,出命之日,及賓至,聞鍾聲,牛曰,孟有北婦人之客,怒將往,牛止之,賓出,使拘而殺諸外,牛又強與仲盟,不可,仲與公御萊書,觀於公,公與之環,使牛入示之,入不示,出命佩之,牛謂叔孫見仲而何,叔孫曰,何為,曰不見,既自見矣,公與之環而佩之矣,遂逐之,奔齊,疾急,命召仲,牛許而不召,杜洩見,告之飢渴,授之戈,對曰,求之而至,又何去焉,豎牛曰,夫子疾病,不欲見人,使寘饋于個而退,牛弗進,則置虛命徹,十二月,癸丑,叔孫不食,乙卯,卒,牛立昭子而相之,公使杜洩葬叔孫,豎牛賂叔仲昭子與南遺,使惡杜洩於季孫而去之,杜洩將以路葬,且盡卿禮,南遺謂季孫曰,叔孫未乘,路葬焉,用之,且冢卿無路,介卿以葬,不亦左乎,季孫曰,然,使杜洩舍路,不可,曰,夫子受命於朝,而聘于王,王思舊勳為賜之路,復命而致之君,君不敢逆王命,而復賜之,使三官書之,吾子為司徒,實書名,夫子為司馬,與工正書服,孟孫為司空以書勳,今死而弗以,是棄君命也,書在公府而弗以,是廢三官也,若命服,生弗敢服,死又不以,將焉用之,乃使以葬,季孫謀去中軍,豎牛曰,夫子固欲去之。

五年,春,王正月,舍中軍。

楚殺其大夫屈申。

公如晉。

夏,莒牟夷以牟婁及防茲來奔。

秋,七月,公至自晉。

戊辰,叔弓帥師敗莒師于蚡泉。

秦伯卒。

冬,楚子,蔡侯,陳侯,許男,頓子,沈子,徐人,越人,伐吳。

五年,春,王正月,舍中軍,卑公室也,毀中軍于施氏,成諸臧氏,初作中軍,三分公室而各有其一,季氏盡征之,叔孫氏臣其子弟,孟氏取其半焉,及其舍之也,四分公室,季氏檡二,二子各一,皆盡征之,而貢于公,以書使杜洩告於殯曰,子固欲毀中軍,既毀之矣,故告杜洩曰,夫子唯不欲毀也,故盟諸僖閎,詛諸五父之衢,受其書而投之,帥士而哭之。叔仲子謂季孫曰,帶受命於子叔孫曰,葬鮮者自西門,季孫命杜洩。杜洩曰,卿喪自朝,魯禮也,吾子為國政,未改禮而又遷之,群臣懼死,不敢自也,既葬而行,仲至自齊,季孫欲立之,南遺曰,叔孫氏厚,則季氏薄,彼實家亂,子勿與知,不亦可乎,南遺使國人助豎牛,以攻諸大庫之庭,司宮射之,中目而死,豎牛取東鄙三十邑,以與南遺,昭子即位。朝其家眾曰,豎牛禍叔孫氏,使亂大從,殺適立庶,又披其邑,將以赦罪,罪莫大焉,必速殺之。豎,牛懼,奔齊,孟仲之子,殺諸塞關之外,投其首於寧風之棘上,仲尼曰,叔孫昭子之不勞,不可能也,周任有言曰,為政者不賞私勞,不罰私怨,《詩》云:『有覺德行,四國順之。』初,穆子之生也,莊叔以周易筮之,遇明夷之謙,以示卜楚丘,曰,是將行,而歸為子祀,以讒人入,其名曰牛,卒以餒死,明夷,日也,日之數十,故有十時,亦當十位,自王已下,其二為公,其三為卿,日上其中,食日為二,旦日為三,明夷之謙,明而未融,其當旦乎,故曰為子祀,日之謙當鳥,故曰明夷于飛,明之未融,故曰垂其翼象,日之動,故曰君子于行,當三在旦,故曰三日不食,離,火也,艮,山也,離為火,火焚山,山敗,於人為言,敗言為讒,故曰有攸往,主人有言,言必讒也,純離為牛,世亂讒勝,勝將適離,故曰其名曰牛,謙不足,飛不翔,垂不峻,翼不廣,故曰其為子後乎,吾子亞卿也,抑少不終。

楚子以屈伸為貳於吳,乃殺之,以屈生為莫敖,使與令尹子蕩如晉逆女,過鄭,鄭伯勞子蕩于氾,勞屈生于菟氏,晉侯送女于邢丘,子產相鄭伯,會晉侯于邢丘。

公如晉,自郊勞至于贈賄,無失禮,晉侯謂女叔齊曰,魯侯不亦善於禮乎,對曰,魯侯焉知禮,公曰,何為,自郊勞至于贈賄,禮無違者,何故不知,對曰,是儀也,不可謂禮,禮所以守其國,行其政令,無失其民者也,今政令在家,不能取也,有子家羈,弗能用也,奸大國之盟,陵虐小國,利人之難,不知其私,公室四分,民食於他,思莫在公,不圖其終,為國君,難將及身,不恤其所,禮之本末,將於此乎在,而屑屑焉習儀以亟,言善於禮,不亦遠乎,君子謂叔侯於是乎知禮。

晉韓宣子如楚送女,叔向為介,鄭子皮,子大叔,勞諸索氏,大叔謂叔向曰,楚王汏侈已甚,子其戒之,叔向曰,汏侈已甚,身之災也,焉能及人,若奉吾幣帛,慎吾威儀,守之以信,行之以禮,敬始而思終,終無不復,從而不失儀,敬而不失威,道之以訓辭,奉之以舊法,考之以先王,度之以二國,雖汏侈若我何,及楚,楚子朝其大夫曰,晉,吾仇敵也,苟得志焉,無恤其他,今其來者,上卿上大夫也,若吾以韓起為閽,以羊舌肸為司宮,足以辱晉,吾亦得志矣,可乎,大夫莫對,薳啟彊曰,可,苟有其備,何故不可,恥匹夫不可以無備,況恥國乎,是以聖王務行禮,不求恥人,朝聘有珪,享覜有璋,小有述職,大有巡功,設机而不倚,爵盈而不飲,宴有好貨,飧有陪鼎,入有郊勞,出有贈賄,禮之至也,國家之敗,失之道也,則禍亂興,城濮之役,晉無楚備,以敗於邲,邲之役,楚無晉備,以敗於鄢,自鄢以來,晉不失備,而加之以禮,重之以睦,是以楚弗能報,而求親焉,既獲姻親,又欲恥之,以召寇讎,備之若何,誰其重此,若有其人,恥之可也,若其未有,君亦圖之,晉之事君,臣曰可矣,求諸侯而麇至,求昏而薦女,君親送之,上卿及上大夫致之,猶欲恥之,君其亦有備矣,不然奈何,韓起之下,趙成,中行吳,魏舒,范鞅,知盈,羊舌肸之下,祁午,張趯,籍談,女齊,梁丙,張骼,輔躒,苗賁皇,皆諸侯之選也,韓襄為公族大夫,韓須受命而使矣,箕襄,邢帶,叔禽,叔椒,子羽,皆大家也,韓賦七邑,皆成縣也,羊舌四族,皆彊家也,晉人若喪韓起,楊肸,五卿,八大夫,輔韓須,楊石,因其十家九縣,長轂九百,其餘四十縣,遺守四千,奮其武怒,以報其大恥,伯華謀之,中行伯魏舒帥之,其蔑不濟矣,君將以親易怨,實無禮以速寇,而未有其備,使群臣往遺之禽,以逞君心,何不可之有,王曰,不穀之過也,大夫無辱,厚為韓子禮,王欲敖叔向以其所不知而不能,亦厚其禮,韓起反,鄭伯勞諸圉,辭不敢見,禮也。

鄭罕虎如齊,娶於子尾氏,晏子驟見之,陳桓子問其故,對曰,能用善人,民之主也。

夏,莒牟夷以牟婁及防茲來奔,牟夷非卿而書,尊地也,莒人愬于晉,晉侯欲止公,范獻子曰,不可,人朝而執之,誘也,討不以師,而誘以成之,惰也,為盟主而犯此二者,無乃不可乎,請歸之,間而以師討焉,乃歸公,秋,七月,公至自晉,莒人來討,不設備,戊辰,叔弓敗諸蚡泉,莒未陳也。

冬,十月,楚子以諸侯及東夷伐吳,以報棘,櫟,麻,之役,薳射以繁揚之師,會於夏汭,越大夫常壽過,帥師會楚子于瑣,聞吳師出,薳啟彊帥師從之,遽不設備,吳人敗諸鵲岸,楚子以馹至於羅汭,吳子使其弟蹶由犒師,楚人執之,將以釁鼓,王使問焉,曰,女卜來吉乎,對曰,吉,寡君聞君將治兵於敝邑,卜之以守龜,曰,余亟使人犒師,請行以觀王怒之疾徐,而為之備,尚克知之,龜兆告吉,曰克可知也,君若驩焉,好逆使臣,滋敝邑休殆而忘其死,亡無日矣,今君奮焉,震電馮怒,虐執使臣,將以釁鼓,則吳知所備矣,敝邑雖羸,若早脩完,其可以息師,難易有備,可謂吉矣,且吳社稷是卜,豈為一人,使臣獲釁軍鼓,而敝邑知備,以禦不虞,其為吉孰大焉,國之守龜,其何事不卜,一臧一否,其誰能當之,城濮之兆,其報在邲,今此行也,其庸有報志,乃弗殺,楚師濟於羅汭,沈尹赤會楚子次於萊山,薳射帥繁揚之師,先入南懷,楚師從之,及汝清,吳不可入,楚子遂觀兵於坻箕之山,是行也,吳早設備,楚無功而還,以蹶由歸,楚子懼吳,使沈尹射待命于巢,薳啟彊待命于雩婁,禮也。

秦后子復歸於秦,景公卒故也。

六年,春,王正月,杞柏益姑卒。

葬秦景公。

夏,季孫宿如晉,葬杞文公。

宋華合比,出奔衛。

秋,九月,大雩。

楚薳罷帥師伐吳。

冬,叔弓如楚。

齊侯伐北燕。

六年,春,王正月,杞文公卒,弔如同盟,禮也,大夫如秦葬景公,禮也。

三月,鄭人鑄刑書,叔向使詒子產書曰,始吾有虞於子,今則已矣,昔先王議事以制,不為刑辟,懼民之有爭心也,猶不可禁禦,是故閑之以義,糾之以政,行之以禮,守之以信,奉之以仁,制為祿位,以勸其從,嚴斷刑罰,以威其淫,懼其未也,故誨之以忠,聳之以行,教之以務,使之以和,臨之以敬,蒞之以彊,斷之以剛,猶求聖哲之上,明察之官,忠信之長,慈惠之師,民於是乎可任使也,而不生禍亂,民知有辟,則不忌於上,並有爭心,以徵於書,而徼幸以成之,弗可為矣,夏有亂政而作禹刑,商有亂政而作湯刑,周有亂政而作九刑,三辟之興,皆叔世也,今吾子相鄭國,作封洫,立謗政,制參辟,鑄刑書,將以靖民,不亦難乎,詩曰,儀式刑文王之德,日靖四方,又曰,儀刑文王,萬邦作孚,如是何辟之有,民知爭端矣,將棄禮而徵於書,錐刀之末,將盡爭之,亂獄滋豐,賄賂並行,終子之世,鄭其敗乎,肸聞之,國將亡,必多制,其此之謂乎,復書曰,若吾子之言,僑不才,不能及子孫,吾以救世也,既不承命,敢忘大惠,士文伯曰,火見,鄭其火乎,火未出而作火,以鑄刑器,藏爭辟焉,火如象之,不火何為,夏,季孫宿如晉,拜莒田也,晉侯享之有加籩,武子退,使行人告曰,小國之事大國也,苟免於討,不敢求貺,得貺不過三獻,今豆有加,下臣弗堪,無乃戾也,韓宣子曰,寡君以為驩也,對曰,寡君猶未敢,況下臣,君之隸也,敢聞加貺,固請徹加,而後卒事,晉人以為知禮,重其好貨。

宋寺人柳有寵,大子佐惡之,華合比曰,我殺之,柳聞之,乃坎用牲埋書,而告公曰,合比將納亡人之族,既盟于北郭矣,公使視之,有焉,遂逐華合比,合比奔衛,於是華亥欲代右師,乃與寺人柳比,從為之徵曰,聞之久矣,公使代之,見於左師,左師曰,女夫也必亡,女喪而宗室,於人何有,人亦於女何有,詩曰,宗子維城,毋俾城壞,毋獨斯畏,女其畏哉。

六月,丙戌,鄭災。

楚公子棄疾如晉,報韓子也,過鄭,鄭罕虎,公孫僑,游吉,從鄭伯以勞諸柤,辭不敢見,固請見之,見如見王,以其乘馬八匹,私面見子皮,如上卿,以馬六匹,見子產以馬四匹,見子大叔以馬二匹,禁芻牧採樵,不入田,不樵樹,不采蓺,不抽屋,不強匄,誓曰,有犯命者,君子廢,小人降,舍不為暴,主不慁賓,往來如是,鄭三卿皆知其將為王也,韓宣子之適楚也,楚人弗逆,公子棄疾及晉竟,晉侯將亦弗逆,叔向曰,楚辟我衷,若何效辟,詩曰,爾之教矣,民胥效矣,從我而已焉,用效人之辟,書曰,聖作則,無寧以善人為則,而則人之辟乎,匹夫為善,民猶則之,況國君乎,晉侯說,乃逆之。

秋,七月,大雩,旱也。

徐儀楚聘于楚,楚子執之,逃歸,懼其叛也,使薳洩伐徐,吳人救之,令尹子蕩帥師伐吳,師于豫章,而次于乾谿,吳人敗其師於房鍾,獲宮廄尹棄疾,子蕩歸罪於薳洩而殺之。

冬,叔弓如楚聘,且弔敗也。

十一月,齊侯如晉,請伐北燕也,士匄相士鞅逆諸河,禮也,晉侯許之,十二月,齊侯遂伐北燕,將納簡公,晏子曰,不入,燕有君矣,民不貳,吾君賄,左右諂諛,作大事不以信,未嘗可也。

七年,春,王正月,暨齊平。

三月,公如楚。

叔孫婼如齊蒞盟。

夏,四月,甲辰,朔,日有食之。

秋,八月,戊辰,衛侯惡卒。

九月,公至自楚。

冬,十有一月,癸未,季孫宿卒。

十有二月,癸亥,葬衛襄公。

七年,春,王正月,暨齊平,齊求之也,癸巳,齊侯次于虢,燕人行成,曰,敝邑知罪,敢不聽命,先君之敝器,請以謝罪公孫皙曰,受服而退,俟釁而動,可也,二月,戊午,盟于濡上,燕人歸燕姬,賂以瑤罋玉櫝斝耳,不克而還。

楚子之為令尹也,為王旌以田,芊尹無宇斷之,曰,一國兩君,其誰堪之,及即位,為章華之宮,納亡人以實之,無宇之閽入焉,無宇執之,有司弗與,曰,執人於王宮,其罪大矣,執而謁諸王,王將飲酒,無宇辭曰,天子經略,諸侯正封,古之制也,封略之內,何非君土,食土之毛,誰非君臣。故詩曰,普天之下,莫非王土,率土之濱,莫非王臣,天有十日,人有十等,下所以事上,上所以共神也,故王臣公,公臣大夫,大夫臣士,士臣皁,皁臣輿,輿臣隸,隸臣僚,僚臣僕,僕臣臺,馬有圉,牛有牧,以待百事,今有司曰,女胡執人於王宮,將焉執之。周文王之法曰,有亡荒閱,所以得天下也。吾先君文王作僕區之法曰,盜所隱器,與盜同罪,所以封汝也。若從有司,是無所執逃臣也,逃而舍之,是無陪臺也,王事無乃闕乎,昔武王數紂之罪,以告諸侯曰,紂為天下逋逃主,萃淵藪,故夫致死焉,君王始求諸侯而則紂,無乃不可乎,若以二文之法取之,盜有所在矣,王曰,取而臣以往,盜有寵,未可得也,遂赦之,楚子成章華之臺,願以諸侯落之,大宰薳啟彊曰,臣能得魯侯,薳啟彊來召公,辭曰,昔先君成公,命我先大夫嬰齊曰,吾不忘先君之好,將使衡父照臨楚國,鎮撫其社稷,以輯寧爾民,嬰齊受命于蜀,奉承以來,弗敢失隕,而致諸宗祧曰,我先君共王,引領北望,日月以冀,傳序相授,於今四王矣,嘉惠未至,唯襄公之辱臨我喪,孤與其二三臣,悼心失圖,社稷之不皇,況能懷思君德,今君若步玉趾,辱見寡君,寵靈楚國,以信蜀之役,致君之嘉惠,是寡君既受貺矣,何蜀之敢望,其先君鬼神實嘉賴之,豈唯寡君,君若不來,使臣請問行期,寡君將承質幣而見于蜀,以請先君之貺,公將往,夢襄公祖,梓慎曰,君不果行,襄公之適楚也,夢周公祖而行,今襄公實祖,君其不行,子服惠伯曰,行,先君未嘗適楚,故周公祖以道之,襄公適楚矣,而祖以道,君不行何之,三月,公如楚,鄭伯勞于師之梁,孟僖子為介,不能相儀,及楚,不能荅郊勞。

夏,四月,甲辰,朔,日有食之,晉侯問於士文伯曰,誰將當日食,對曰,魯衛惡之,衛大魯小,公曰,何故,對曰,去衛地,如魯地,於是有災,魯實受之,其大咎,其衛君乎,魯將上卿,公曰,詩所謂彼日而食,于何不臧者,何也,對曰,不善政之謂也,國無政,不用善,則自取謫于日月之災,故政不可不慎也,務三而已,一曰擇人,二曰因民,三曰從時。

晉人來治杞田,季孫將以成與之,謝息為孟孫守,不可,曰,人有言曰,雖有挈缾之知,守不假器,禮也,夫子從君,而守臣喪邑,雖吾子亦有猜焉,季孫曰,君之在楚,於晉罪也,又不聽晉,魯罪重矣,晉師必至,吾無以待之,不如與之,間晉而取諸杞,吾與子桃,成反,誰敢有之,是得二成也,魯無憂,而孟孫益邑,子何病焉,辭以無山,與之萊柞,乃遷于桃,晉人為杞取成。

楚子享公于新臺,使長鬣者相,好以大屈,既而悔之,薳啟彊聞之,見公,公語之,拜賀,公曰,何賀,對曰,齊與晉越,欲此久矣,寡君無適與也,而傳諸君,君其備禦三鄰,慎守寶矣,敢不賀乎,公懼,乃反之。

鄭子產聘于晉,晉侯疾,韓宣子逆客,私焉,曰,寡君寢疾,於今三月矣,並走群望,有加而無瘳,今夢黃熊入于寢門,其何厲鬼也,對曰,以君之明,子為大政,其何厲之有,昔堯殛鯀于羽山,其神化為黃熊,以入于羽淵,實為夏郊,三代祀之,晉為盟主,其或者未之祀也乎,韓子祀夏郊,晉侯有間,賜子產莒之二方鼎,子產為豐施歸州田於韓宣子,曰,日君以夫公孫段,為能任其事,而賜之州田,今無祿早世,不獲久享君德,其子弗,敢有不敢以聞於君,私致諸子,宣子辭,子產曰,古人有言曰,其父析薪,其子弗克負荷,施將懼不能任其先人之祿,其況能任大國之賜,縱吾子為政而可,後之人若屬有疆埸之言,敝邑獲戾,而豐氏受其大討,吾子取州,是免敝邑於戾,而建置豐氏也,敢以為請,宣子受之,以告晉侯,晉侯以與宣子,宣子為初言,病有之,以易原縣於樂大心。

鄭人相驚以伯有,曰伯有至矣,則皆走,不知所往,鑄刑書之歲二月,或夢伯有介而行,曰壬子,余將殺帶也,明年壬寅,余又將殺段也,及壬子,駟帶卒,國人益懼,齊燕平之月,壬寅,公孫段卒,國人愈懼,其明月,子產立公孫洩及良止以撫之,乃止,子大叔問其故,子產曰,鬼有所歸,乃不為厲,吾為之歸也,大叔曰,公孫洩何為,子產曰,說也,為身無義而圖說,從政有所反之以取媚也,不媚不信,不信,民不從也,及子產適晉,趙景子問焉,曰,伯有猶能為鬼乎?子產曰:能。人生始化曰魄,既生魄,陽曰魂,用物精多,則魂魄強,是以有精爽,至於神明,匹夫匹婦強死,其魂魄猶能馮依於人,以為淫厲,況良霄。我先君穆公之冑,子良之孫,子耳之子,敝邑之卿,從政三世矣,鄭雖無腆,抑諺曰,蕞爾國,而三世執其政柄,其用物也弘矣,其取精也多矣,其族又大,所馮厚矣,而強死,能為鬼,不亦宜乎。

子皮之族,飲酒無度,故馬師氏與子皮氏有惡,齊師還自燕之月,罕朔殺罕魋,罕朔奔晉,韓宣子問其位於子產,子產曰,君之羈臣,苟得容以逃死,何位之敢擇,卿違,從大夫之位,罪人,以其罪降,古之制也,朔於敝邑,亞大夫也,其官馬師也,獲戾而逃,唯執政所寘之,得免其死,為惠大矣,又敢求位,宣子為子產之敏也,使從嬖大夫。

秋,八月,衛襄公卒,晉大夫言於范獻子曰,衛事晉為睦,晉不禮焉,庇其賊人,而取其地,故諸侯貳,詩曰,䳭鴒在原,兄弟急難,又曰,死喪之威,兄弟孔懷,兄弟之不睦,於是乎不弔,況遠人誰敢歸之,今又不禮於衛之嗣,衛必叛我,是絕諸侯也,獻子以告韓宣子,宣子說,使獻子如衛弔,且反戚田,衛齊惡告喪于周,且請命,王使臣簡公如衛弔,且追命襄公曰,叔父陟恪,在我先王之左右,以佐事上帝,余敢忘高圉,亞圉。

九月,公至自楚,孟僖子病不能相禮,乃講學之,苟能禮者從之,及其將死也,召其大夫曰,禮,人之幹也,無禮無以立,吾聞將有達者,曰孔丘,聖人之後也,而滅於宋,其祖弗父何,以有宋而授厲公,及正考父佐戴,武,宣,三命茲益共,故其鼎銘云,一命而僂,再命而傴,三命而俯,循牆而走,亦莫余敢侮,饘於是,鬻於是,以餬余口,其共也如是,臧孫紇有言曰,聖人有明德者,若不當世,其後必有達人,今其將在孔丘乎,我若獲沒必屬說與何忌於夫子,使事之而學禮焉,以定其位,故孟懿子,與南宮敬叔,師事仲尼,仲尼曰,能補過者,君子也,詩曰,君子是則是效,孟僖子可則效已矣。

單獻公棄親用羈,冬,十月,辛酉,襄頃之族,殺獻公而立成公。

十一月,季武子卒,晉侯謂伯瑕曰,吾所問日食從矣,可常乎,對曰,不可,六物不同,民心不壹,事序不類,官職不則,同始異終,胡可常也,詩曰,或燕燕居息,或憔悴事國,其異終也如是,公曰,何謂六物對曰,歲時日月星辰是謂也,公曰,多語寡人辰,而莫同,何謂辰,對曰,日月之會是謂辰,故以配日。

衛襄公,夫人姜氏無子,嬖人婤姶生孟縶,孔成子夢康叔謂己,立元,余使羈之孫圉與史苟相之,史朝亦夢康叔謂己,余將命而子苟,與孔烝鉏之曾孫圉,相元,史朝見成子,告之夢,夢協,晉韓宣子為政,聘于諸侯之歲,婤姶生子,名之曰元,孟縶之足不良,能行孔成子以周易筮之曰,元尚享衛國,主其社稷,遇屯,又曰,余尚立縶,尚克嘉之,遇屯之比以示史朝,史朝曰,元亨,又何疑焉,成子曰,非長之謂乎,對曰,康叔名之,可謂長矣,孟非人也,將不列於宗,不可謂長,且其繇曰,利建侯,嗣吉,何建,建非嗣也,二卦皆云子其建之,康叔命之,二卦告之,筮襲於夢,武王所用也,弗從何為,弱足者居,侯主社稷,臨祭祀,奉民人,事鬼神,從會朝,又焉得居,各以所利,不亦可乎,故孔成子立靈公,十二月,癸亥,葬衛襄公。

八年,春,陳侯之弟招,殺陳世子偃師。

夏,四月,辛丑,陳侯溺卒。

叔弓如晉。

楚人執陳行人干徵師,殺之,陳公子留,出奔鄭。

秋,蒐于紅。

陳人殺其大夫公子過。

大雩。

冬,十月,壬午,楚師滅陳,執陳公子招,放之于越,殺陳孔奐。

葬陳哀公。

八年,春,石言于晉魏榆,晉侯問於師曠曰,石何故言,對曰石不能言,或馮焉,不然,民聽濫也,抑臣又聞之曰,作事不時,怨讟動于民,則有非言之物而言,今宮室崇侈,民力彫盡,怨讟並作,莫保其性,石言不亦宜乎,於是晉侯方築虒祁之宮,叔向曰,子野之言君子哉,君子之言,信而有徵,故怨遠於其身,小人之言僭而無徵,故怨咎及之,詩曰,哀哉不能言,匪舌是出,唯躬是瘁,哿矣能言,巧言如流,俾躬處休,其是之謂乎,是宮也成,諸侯必叛,君必有咎,夫子知之矣。

陳哀公元妃鄭姬生悼大子偃師,二妃生公子留,下妃生公子勝,二妃嬖,留有寵,屬諸徒招與公子過,哀公有廢疾,三月甲申,公子招,公子過,殺悼大子偃師而立公子留。

夏,四月,辛亥,哀公縊,干徵師赴于楚,且告有立君,公子勝愬之于楚,楚人執而殺之,公子留奔鄭,書曰,陳侯之弟招殺陳世子偃師,罪在招也,楚人執陳行人干徵師殺之,罪不在行人也。

叔弓如晉,賀虒祁也,游吉相鄭伯以如晉,亦賀虒祁也,史趙見子大叔曰,甚哉其相蒙也,可弔也而又賀之,子大叔曰,若何弔也,其非唯我賀,將天下實賀。

秋,大蒐于紅,自根牟至于商衛,革車千乘。

七月,甲戌,齊子尾卒,子旗欲治其室,丁丑,殺梁嬰,八月,庚戌,逐子成,子工,子車,皆來奔,而立子良氏之宰,其臣曰,孺子長矣,而相吾室,欲兼我也,授甲將攻之,陳桓子善於子尾,亦授甲將助之,或告子旗,子旗不信,則數人告將往,又數人告於道,遂如陳氏,桓子將出矣,聞之而還,游服而逆之,請命,對曰,聞彊氏授甲將攻子,子聞諸,曰,弗聞,子盍亦授甲,無宇請從,子旗曰,子胡然,彼孺子也,吾誨之,猶懼其不濟,吾又寵秩之,其若先人何,子盍謂之,《周書》曰,惠不惠,茂不茂,康叔所以服弘大也,桓子稽顙曰,頃靈福子,吾猶有望,遂和之如初,陳公子招歸罪於公子過而殺之,九月,楚公子棄疾帥師奉孫吳圍陳,宋戴惡會之,冬,十一月,壬午,滅陳輿嬖,袁克殺馬毀玉以葬,楚人將殺之,請寘之,既又請私,私於幄,加絰於顙而逃,使穿封戌,為陳公曰,城麇之役不諂,侍飲酒於王,王曰,城麇之役,女知寡人之及此,女其辟寡人乎,對曰,若知君之及此,臣必致死禮以息楚,晉侯問於史趙曰,陳其遂亡乎,對曰,未也,公曰何故,對曰,陳顓頊之族也,歲在鶉火,是以卒滅,陳將如之。今在析木之津,猶將復由。且陳氏得政于齊,而後陳卒亡,自幕至于瞽瞍,無違命,舜重之以明德,寘德于遂,遂世守之,及胡公不淫,故周賜之姓,使祀虞帝,臣聞盛德必百世祀,虞之世數未也,繼守將在齊,其兆既存矣。

九年,春,叔弓會楚子于陳。

許遷于夷。

夏,四月,陳災。

秋,仲孫貜如齊。

冬,築郎囿。

九年,春,叔弓,宋華亥,鄭游吉,衛趙黶,會楚子于陳。

二月,庚申,楚公子棄疾,遷許于夷,實城父,取州來淮北之田以益之,伍舉授許男田,然丹遷城父人於陳,以夷濮西田益之,遷方城外人於許。

周甘人與晉閻嘉爭閻田,晉梁丙,張趯,率陰戎伐潁。王使詹桓伯辭於晉曰:我自夏以后稷,魏,駘,芮,岐,畢,吾西土也。及武王克商,蒲姑,商奄,吾東土也,巴濮,楚鄧,吾南土也,肅慎,燕,亳,吾北土也,吾何邇封之有,文武成康之建母弟,以蕃屏周,亦其廢隊是為,豈如弁髦,而因以敝之,先王居檮杌于四裔,以禦螭魅,故允姓之姦,居于瓜州,伯父惠公歸自秦,而誘以來,使偪我諸姬,入我郊甸,則戎焉取之,戎有中國,誰之咎也,后稷封殖天下,今戎制之,不亦難乎,伯父圖之。我在伯父,猶衣服之有冠冕,木水之有本原,民人之有謀主也。伯父若裂冠毀冕,拔本塞原,專棄謀主,雖戎狄其何有余一人,叔向謂宣子曰,文之伯也,豈能改物,翼戴天子,而加之以共,自文以來,世有衰德,而暴滅宗周,以宣示其侈,諸侯之貳,不亦宜乎,且王辭直,子其圖之,宣子說,王有姻喪,使趙成如周弔,且致閻田與襚,反潁俘,王亦使賓滑執甘大夫襄以說於晉,晉人禮而歸之。

夏,四月,陳災,鄭裨灶曰,五年,陳將復封,封五十二年而遂亡,子產問其故,對曰,陳,水屬也,火,水妃也,而楚所相也,今火出而火陳,逐楚而建陳也,妃以五成,故曰五年,歲五及鶉火,而後陳卒亡,楚克有之,天之道也,故曰五十二年。

晉荀盈如齊逆女,還,六月,卒于戲陽,殯于絳,未葬,晉侯飲酒樂,膳宰屠蒯趨入,請佐公使尊,許之,而遂酌以飲,工曰,女為君耳,將司聰也,辰在子卯,謂之疾日,君徹宴樂,學人舍業,為疾故也,君之卿佐,是謂股肱,股肱或虧,何痛如之,女弗聞而樂,是不聰也,又飲外嬖嬖叔曰,女為君目,將司明也,服以旌禮,禮以行事,事有其物,物有其容,今君之容,非其物也,而女不見,是不明也,亦自飲也,曰,味以行氣,氣以實志,志以定言,言以出令,臣實司味,二御失官,而君弗命,臣之罪也,公說,徹酒,初,公欲廢知氏而立其外嬖,為是悛而止,秋,八月,使荀躒佐下軍以說焉。

孟僖子如齊,殷聘禮也。

冬,築郎囿,書時也,季平子欲其速成也,叔孫昭子曰,詩曰,經始勿亟,庶民子來,焉用速成,其以勦民也,無囿猶可,無民其可乎。

十年,春,王正月。

夏,齊欒施來奔。

秋,七月,季孫意如叔弓,仲孫貜帥師伐莒。

戊子,晉侯彪卒。

九月,叔孫婼如晉。

葬晉平公。

十有二月,甲子,宋公成卒。

十年,春,王正月,有星出于婺女,鄭裨灶言於子產曰,七月,戊子,晉君將死,今茲歲在顓頊之虛,姜氏任氏,實守其地,居其維首,而有妖星焉,告邑姜也,邑姜,晉之妣也,天以七紀,戊子,逢公以登星斯於是乎出,吾是以譏之。

齊惠欒,高氏,皆耆酒,信內多怨,彊於陳鮑氏而惡之,夏,有告陳桓子曰,子旗,子良,將攻陳鮑,亦告鮑氏,桓子授甲而如鮑氏,遭子良醉而騁,遂見文子,則亦授甲矣,使視二子,則皆從飲酒,桓子曰,彼雖不信,聞我授甲,則必逐我,及其飲酒也,先伐諸,陳鮑方睦,遂伐欒高氏,子良曰,先得公,陳鮑焉往,遂伐虎門,晏平仲端委立于虎門之外,四族召之,無所往,其徒曰,助陳鮑乎,曰,何善焉,助欒高乎,曰,庸愈乎,然則歸乎,曰,君伐焉歸,公召之而後入,公卜使王黑以靈姑銔率,吉,請斷三尺焉而用之,五月,庚辰,戰于稷,欒高敗,又敗諸莊,國人追之,又敗諸鹿門,欒施,高彊,來奔,陳鮑分其室,晏子謂桓子,必致諸公,讓德之主也,謂懿德,凡有血氣,皆有爭心,故利不可強,思義為愈,義,利之本也,蘊利生孽,姑使無蘊乎,可以滋長,桓子盡致諸公,而請老于莒,桓子召子山,私具幄幕器用,從者之衣屨,而反棘焉,子商亦如之,而反其邑,子周亦如之,而與之夫于,反子城,子公,公孫捷,而皆益其祿,凡公子公孫之無祿者,私分之邑,國之貧約孤寡者,私與之粟,曰,詩云,陳錫載周,能施也,桓公是以霸,公與桓子莒之旁邑,辭,穆孟姬為之請高唐,陳氏始大。

秋,七月,平子伐莒,取郠,獻俘,始用人於亳社,臧武仲在齊,聞之,曰,周公其不饗魯祭乎,周公饗義,魯無義,詩曰,德音孔昭,視民不佻,佻之謂甚矣,而壹用之,將誰福哉。

戊子,晉平公卒,鄭伯如晉,及河,晉人辭之,游吉遂如晉,九月,叔孫婼,齊國弱,宋華定,衛北宮喜,鄭罕虎,許人,曹人,莒人,邾人,薛人,杞人,小邾人,如晉,葬平公也,鄭子皮將以幣行,子產曰,喪焉用幣,用幣必百兩,百兩必千人,千人至,將不行,不行,必盡用之,幾千人而國不亡,子皮固請以行,既葬,諸侯之大夫欲因見新君,叔孫昭子曰,非禮也,弗聽,叔向辭之,曰,大夫之事畢矣,而又命孤,孤斬焉在衰絰之中,其以嘉服見,則喪禮未畢,其以喪服見,是重受弔也,大夫將若之何,皆無辭以見,子皮盡用其幣,歸,謂子羽曰,非知之實難,將在行之,夫子知之矣,我則不足,書曰,欲敗度,縱敗禮,我之謂矣,夫子知度與禮矣,我實縱欲,而不能自克也,昭子至自晉,大夫皆見,高彊見而退,昭子語諸大夫曰,為人子,不可不慎也哉,昔慶封亡,子尾多受邑而稍致諸君,君以為忠,而甚寵之,將死,疾于公宮,輦而歸,君親推之,其子不能任,是以在此。忠為令德,其子弗能任,罪猶及之,難不慎也!喪夫人之力,棄德曠宗,以及其身,不害乎,詩曰,不自我先,不自我後,其是之謂乎。

冬,十二月,宋平公卒,初,元公惡寺人柳,欲殺之,及喪,柳熾炭于位,將至,則去之,比葬,又有寵。

十有一年,春,王二月,叔弓如宋。

葬宋平公。

夏,四月,丁巳,楚子虔誘蔡侯般,殺之于申,楚公子杞疾帥師圍蔡。

五月,甲申,夫人歸氏薨。

大蒐于比蒲。

仲孫貜會邾子盟于祲祥。

秋,季孫意如會,晉韓起,齊國弱,宋華亥,衛北宮佗,鄭罕虎,曹人,杞人,于厥憖。

九月,己亥,葬我小君齊歸。

冬,十有一月,丁酉,楚師滅蔡,執蔡世子有以歸用之。

十一年,春,王二月,叔弓如宋,葬平公也。

景王問於萇弘曰:「今茲諸侯,何實吉,何實凶?」對曰:蔡凶,此蔡侯般弒其君之歲也,歲在豕韋,弗過此矣,楚將有之然壅也,歲及大梁,蔡復楚凶,天之道也,楚子在申,召蔡靈侯,靈侯將往,蔡大夫曰,王貪而無信,唯蔡於感,今幣重而言甘,誘我也,不如無往,蔡侯不可,五月,丙申,楚子伏甲而饗蔡侯於申,醉而執之,夏,四月,丁巳,殺之,刑其士七十人,公子棄疾帥師圍蔡,韓宣子問於叔向曰,楚其克乎,對曰,克哉,蔡侯獲罪於其君,而不能其民,天將假手於楚以斃之,何故不克,然肸聞之,不信以幸,不可再也,楚王奉孫吳以討於陳曰,將定而國,陳人聽命,而遂縣之,今又誘蔡而殺其君,以圍其國,雖幸而克,必受其咎,弗能久矣,桀克有緡,以喪其國,紂克東夷,而隕其身,楚小位下,而亟暴於二王,能無咎乎,天之假助不善,非祚之也,厚其凶惡,而降之罰也,且譬之如天,其有五材,而將用之,力盡而敝之,是以無拯,不可沒振。

五月,齊歸薨大蒐于比蒲,非禮也。

孟僖子會邾莊公盟于祲祥,脩好,禮也,泉丘人有女,夢以其帷幕孟氏之廟,遂奔僖子,其僚從之,盟于清丘之社,曰有子,無相棄也,僖子使助薳氏之簉,反自祲祥,宿于薳氏,生懿子及南宮敬叔於泉丘人,其僚無子,使字敬叔。

楚師在蔡,晉荀吳謂韓宣子曰,不能救陳,又不能救蔡,物以無親,晉之不能,亦可知也,己為盟主,而不恤亡國,將焉用之。

秋,會于厥憖,謀救蔡也,鄭子皮將行,子產曰,行不遠,不能救蔡也,蔡小而不順,楚大而不德,天將棄蔡以壅楚,盈而罰之,蔡必亡矣,且喪君而能守者鮮矣,三年,王其有咎乎,美惡周必復,王惡周矣,晉人使狐父請蔡于楚,弗許。

單子會韓宣子于戚,視下言徐,叔向曰,單子其將死乎,朝有著定,會有表,衣有襘,帶有結,會朝之言,必聞于表著之位,所以昭事序也,視不過結襘之中,所以道容貌也,言以命之,容貌以明之,失則有闕,今單子為王官伯,而命事於會,視不登帶言不過步,貌不道容,而言不昭矣,不道不共,不昭不從,無守氣矣。

九月,葬齊歸,公不慼,晉士之送葬者,歸以語史趙,史趙曰,必為魯郊,侍者曰,何故,曰,歸,姓也,不思親,祖不歸也,叔向曰,魯公室其卑乎,君有大喪,國不廢蒐,有三年之喪,而無一日之慼,國不恤喪,不忌君也,君無慼容,不顧親也,國不忌君,君不顧親,能無卑乎,殆其失國。

冬,十一月,楚子滅蔡,用隱大子于岡山,申無宇曰,不祥,五牲不相為用,況用諸侯乎,王必悔之。

十二月,單成公卒。

楚子城陳蔡不羹,使棄疾為蔡公,王問於申無宇曰,棄疾在蔡何如,對曰,擇子莫如父,擇臣莫如君,鄭莊公城櫟而寘子元焉,使昭公不立,齊桓公城穀而寘管仲焉,至于今賴之,臣聞五大不在邊,五細不在庭,親不在外,羈不在內,今棄疾在外,鄭丹在內,君其少戒,王曰,國有大城何如,對曰,鄭京櫟實殺曼伯,宋蕭亳實殺子游,齊渠丘實殺無知,衛蒲戚實出獻公,若由是觀之,則害於國,末大必折,尾大不掉,君所知也。

十有二年,春,齊高偃帥師納北燕伯于陽。

三月,壬申,鄭伯嘉卒。

夏,宋公使華定來聘。

公如晉至河乃復。

五月,葬鄭簡公。

楚殺其大夫成熊。

秋,七月。

冬,十月,公子憖出奔齊。

楚子伐徐。

晉伐鮮虞。

十二年,春,齊高偃納北燕伯款于唐,因其眾也。

三月,鄭簡公卒,將為葬除,及游氏之廟,將毀焉,子大叔使其除徒執用以立,而無庸毀,曰,子產過女,而問何故不毀,乃曰,不忍廟也,諾,將毀矣,既如是,子產乃使辟之,司墓之室,有當道者,毀之,則朝而塴,弗毀,則日中而塴,子大叔請毀之,曰,無若諸侯之賓何,子產曰,諸侯之賓,能來會吾喪,豈憚日中,無損於賓,而民不害,何故不為,遂弗毀,日中而葬,君子謂子產於是乎知禮,禮無毀人,以自成也。

夏,宋華定來聘,通嗣君也,享之,為賦蓼蕭,弗知,又不荅賦,昭子曰,必亡,宴語之不懷,寵光之不宣,令德之不知,同福之不受,將何以在。

齊侯,衛侯,鄭伯,如晉,朝嗣君也。

公如晉,至河乃復,取郠之役,莒人愬于晉,晉有平公之喪,未之治也,故辭公,公子憖遂如晉,晉侯享諸侯,子產相鄭伯,辭於享,請免喪而後聽命,晉人許之,禮也,晉侯以齊侯宴,中行穆子相,投壺,晉侯先,穆子曰,有酒如淮,有肉如坻,寡君中此,為諸侯師,中之,齊侯舉矢曰,有酒如澠,有肉如陵,寡人中此,與君代興,亦中之,伯瑕謂穆子曰,子失辭,吾固師諸侯矣,壺何為焉,其以中雋也,齊君弱,吾君歸,弗來矣,穆子曰,吾軍帥彊禦,卒乘競勸,今猶古也,齊將何事,公孫傁趨進曰,日旰君勤,可以出矣,以齊侯出。

楚子謂成虎,若敖之餘也,遂殺之,或譖成虎於楚子,成虎知之,而不能行,書曰,楚殺其大夫成虎,懷寵也。

六月,葬鄭簡公。

晉荀吳偽會齊師者,假道於鮮虞,遂入昔陽,秋,八月,壬午,滅肥,以肥子綿皋歸。

周原伯絞虐其輿臣,使曹逃,冬,十月,壬申,朔,原輿人逐絞而立公子跪,尋絞奔郊。

甘簡公無子,立其弟過,過將去成景之族,成景之族賂劉獻公,丙申,殺甘悼公而立成公之孫鰌,丁酉,殺獻太子之傅,庾皮之子過,殺瑕辛于市,及宮嬖綽,王孫沒,劉州鳩,陰忌,老陽子。

季平子立而不禮於南蒯,南蒯謂子仲,吾出季氏,而歸其室於公,子更其位,我以費為公臣,子仲許之,南蒯語叔仲穆子,且告之故,季悼子之卒也,叔孫昭子以再命為卿,及平子伐莒,克之,更受三命,叔仲子欲構二家,謂平子曰,三命踰父兄,非禮也,平子曰,然,故使昭子。昭子曰,叔孫氏有家禍,殺適立庶。故婼也及此,若因禍以斃之,則聞命矣,若不廢君命,則固有著矣,昭子朝而命吏曰,婼將與季氏訟,書辭無頗,季孫懼,而歸罪於叔仲子,故叔仲小,南蒯,公子憖,謀季氏,憖告公,而遂從公如晉,南蒯懼不克,以費叛如齊,子仲還及衛,聞亂,逃介而先,及郊,聞費叛,遂奔齊,南蒯之將叛也,其鄉人或知之,過之而歎,且言曰,恤恤乎,湫乎攸乎,深思而淺謀,邇身而遠志,家臣而君圖,有人矣哉,南蒯枚筮之,遇坤之比曰,黃裳元吉,以為大吉也,示子服惠伯曰,即欲有事何如,惠伯曰,吾嘗學此矣,忠信之事則可,不然必敗,外彊內溫,忠也,和以率貞,信也,故曰黃裳元吉,黃,中之色也,裳,下之飾也,元,善之長也,中不忠,不得其色,下不共,不得其飾,事不善,不得其極。外內倡和為忠;率事以信為共;供養三德為善。非此三者弗當,且夫易,不可以占險,將何事也,且可飾乎,中美能黃,上美為元,下美則裳,參成可筮,猶有闕也,筮雖吉,未也,將適費,飲鄉人酒,鄉人或歌之曰,我有圃,生之杞乎,從我者子乎,去我者鄙乎,倍其鄰者恥乎,已乎已乎,非吾黨之士乎,平子欲使昭子逐叔仲小,小聞之,不敢朝,昭子命吏謂小待政於朝,曰,吾不為怨府。

楚子狩于州來,次于潁尾,使蕩侯,潘子,司馬督,囂尹,午陵,尹喜,帥師圍徐,以懼吳,楚子次于乾谿,以為之援,雨雪,王皮冠,秦復陶,翠被,豹舄,執鞭以出,僕析父從,右尹子革夕,王見之,去冠被舍鞭,與之語曰,昔我先王熊繹,與呂級,王孫牟,燮父,禽父,並事康王,四國皆有分,我獨無有,今吾使人於周,求鼎以為分,王其與我乎,對曰,與君王哉,昔我先王熊繹,辟在荊山,篳路藍縷,以處草莽,跋涉山林,以事天子,唯是桃弧棘矢,以共禦王事,齊王舅也,晉及魯衛,王母弟也,楚是以無分,而彼皆有。今周與四國,服事君王,將唯命是從,豈其愛鼎。王曰,昔我皇祖伯父昆吾,舊許是宅,今鄭人貪賴其田,而不我與我,若求之,其與我乎,對曰,與君王哉,周不愛鼎,鄭敢愛田,王曰,昔諸侯遠我而畏晉,今我大城,陳蔡不羹,賦皆千乘,子與有勞焉,諸侯其畏我乎,對曰,畏君王哉,是四國者,專足畏也,又加之以楚,敢不畏君王哉,工尹路請曰,君王命剝圭以為鏚柲,敢請命,王入視之,析父謂子革,吾子,楚國之望也,今與王言如響,國其若之何,子革曰,摩厲以須,王出,吾刃將斬矣,王出復語,左史倚相趨過,王曰,是良史也,子善視之,是能讀三墳五典,八索九丘,對曰,臣嘗問焉,昔穆王欲肆其心,周行天下,將皆必有車轍馬跡焉,祭公謀父作祈招之詩,以止王心,王是以獲沒於祗宮,臣問其詩而不知也,若問遠焉,其焉能知之,王曰,子能乎,對曰,能,其詩曰,祈招之愔愔,式昭德音,思我王度,式如玉,式如金,形民之力,而無醉飽之心,王揖而入,饋不食,寢不寐,數日不能自克,以及於難,仲尼曰,古也有志,克己復禮,仁也,信善哉,楚靈王若能如是,豈其辱於乾谿。

晉伐鮮虞,因肥之役也。

十有三年,春,叔弓帥師圍費。

夏,四月,楚公子比自晉歸于楚,弒其君虔于乾谿,楚公子棄疾殺公子比。

秋,公會劉子,晉侯,齊侯,宋公,衛侯,鄭伯,曹伯,莒子,邾子,滕子,薛伯,杞伯,小邾子,于平丘,八月,甲戌,同盟于平丘,公不與盟,晉人執季孫意如以歸,公至自會。

蔡侯廬歸于蔡,陳侯吳歸于陳。

冬,十月,葬蔡靈公。

公如晉,至河乃復。

吳滅州來。

十三年,春,叔弓圍費,弗克敗焉,平子怒,令見費人執之,以為囚俘,冶區夫曰,非也,若見費人,寒者衣之,飢者食之,為之令主,而共其乏困,費來如歸,南氏亡矣,民將叛之,誰與居邑,若憚之以威,懼之以怒,民疾而叛,為之聚也,若諸侯皆然,費人無歸,不親南氏,將焉入矣,平子從之,費人叛南氏。

楚子之為令尹也,殺大司馬薳掩而取其室,及即位,奪薳居田,遷許而質許圍,蔡洧有寵於王,王之滅蔡也,其父死焉,王使與於守,而行申之會,越大夫戮焉,王奪鬥韋龜中犨,又奪成然邑,而使為郊尹,蔓成然故事蔡公,故薳氏之族,及薳居,許圍,蔡洧,蔓成然,皆王所不禮也,因群喪職之族,啟越大夫常壽過作亂,圍固城,克息舟城而居之,觀起之死也,其子從在蔡,事朝吳曰,今不封蔡,蔡不封矣,我請試之,以蔡公之命召子干,子皙,及郊而告之情,強與之盟,入襲蔡,蔡公將食,見之而逃,觀從使子干飲坎用牲,加書而速行,已徇於蔡曰,蔡公召二子,將納之,與之盟而遣之矣,將師而從之,蔡人聚,將執之,辭曰,失賊成軍,而殺余何益,乃釋之,朝吳曰,二三子若能死亡,則如違之,以待所濟,若求安定,則如與之,以濟所欲,且違上何適而可,眾曰與之,乃奉蔡公召二子,而盟于鄧,依陳蔡人以國,楚公子比,公子黑肱,公子棄疾,蔓成然,蔡朝吳,帥陳,楚,不羹,許,葉,之師,因四族之徒以入楚,及郊,陳蔡欲為名,故請為武軍,蔡公知之,曰欲速,且役病矣,請藩而巳,乃藩為軍,蔡公使須務牟與史猈先入,因正僕人殺大子祿,及公子罷敵,公子比為王,公子黑肱為令尹,次于魚陂,公子棄疾為司馬,先除王宮,使觀從從師于乾谿,而遂告之,且曰,先歸復所,後者劓,師及訾梁而潰,王聞群公子之死也,自投于車下,曰,人之愛其子也,亦如余乎,侍者曰,甚焉,小人老而無子,知擠于溝壑矣,王曰,余殺人子多矣,能無及此乎,右尹子革曰,請待于郊,以聽國人,王曰,眾怒不可犯也,曰,若入於大都,而乞師於諸侯,王曰,皆叛矣,曰,若亡於諸侯,以聽大國之圖君也。王曰:大福不再,祇取辱焉。然丹乃歸于楚,王沿夏,將欲入鄢,芊尹無宇之子申亥曰,吾父再奸王命,王弗誅,惠孰大焉,君不可忍,惠不可棄,吾其從王,乃求王,遇諸棘圍,以歸,夏,五月,癸亥,王縊于芊尹申亥氏,申亥以其二女,殉而葬之,觀從謂子干曰,不殺棄疾,雖得國,猶受禍也,子干曰,余不忍也,子玉曰,人將忍子,吾不忍俟也,乃行,國每夜駭曰,王入矣,乙卯,夜,棄疾使周走而呼曰,王至矣,國人大驚,使蔓成然走告子干,子皙曰,王至矣,國人殺君,司馬將來矣,君若早自圖也,可以無辱,眾怒如水火焉,不可為謀,又有呼而走至者曰,眾至矣,二子皆自殺,丙辰,棄疾即位,名曰熊居,葬子于于訾實,訾敖殺囚,衣之王服,而流諸漢,乃取而葬之,以靖國人,使子旗為令尹,楚師還自徐,吳人敗諸豫章,獲其五帥,平王封陳蔡,復遷邑,致群賂,施舍寬民,宥罪舉職,召觀從王曰,唯爾所欲,對曰,臣之先佐開卜,乃使為卜尹,使枝如子躬聘于鄭,且致犨櫟之田,事畢,弗致,鄭人請曰,聞諸道路,將命寡君以犨櫟,敢請命,對曰,臣未聞命,既復,王問犨櫟,降服而對曰,臣過失命未之致也,王執其手曰,子毋勤,姑歸,不穀有事,其告子也,他年,芊尹申亥以王柩告,乃改葬之,初,靈王卜曰,余尚得天下,不吉,投龜詬天而呼曰,是區區者而不余畀,余必自取之,民患王之無厭也,故從亂如歸,初,共王無𠸄,適有寵子五人,無適立焉,乃大有事于群望而祈曰,請神擇於五人者,使主社稷,乃遍以璧見於群望曰,當璧而拜者,神所立也,誰敢違之,既乃與巴姬密埋璧於大室之庭,使五人齊而長入拜,康王跨之,靈王肘加焉,子干子皙皆遠之,平王弱,抱而入,再拜,皆厭紐,鬥韋龜屬成然焉,且曰,棄禮違命,楚其危哉,子干歸,韓宣子問於叔向曰,子干其濟乎,對曰,難,宣子曰,同惡相求,如市賈焉,何難,對曰,無與同好,誰與同惡,取國有五難,有寵而無人,一也,有人而無主,二也,有主而無謀,三也,有謀而無民,四也,有民而無德,五也,子干在晉,十三年矣,晉楚之從,不聞達者,可謂無人,族盡親叛,可謂無主,無釁而動,可謂無謀,為羇終世,可謂無民,亡無愛徵,可謂無德,王虐而不忌,楚君子干涉,五難以殺舊君,誰能濟之,有楚國者,其棄疾乎,君陳蔡,城外屬焉,苛慝不作,盜賊伏隱,私欲不違,民無怨心,先神命之,國民信之,芊姓有亂,必季實立,楚之常也,獲神,一也,有民,二也,令德,三也,寵貴,四也,居常,五也,有五利以去五難,誰能害之,子干之官,則右尹也,數其貴寵,則庶子也,以神所命,則又遠之,其貴亡矣,其寵棄矣,民無懷焉,國無與焉,將何以立,宣子曰,齊桓晉文,不亦是乎,對曰,齊桓,衛姬之子也,有寵於僖,有鮑叔牙,賓須無,隰朋,以為輔佐,有莒,衛,以為外主,有國,高,以為內主,從善如流,下善齊肅,不藏賄,不從欲,施舍不倦,求善不厭,是以有國,不亦宜乎,我先君文公,狐季姬之子也,有寵於獻,好學而不貳,生十七年,有士五人,有先大夫子餘,子犯,以為腹心,有魏犨,賈佗,以為股肱,有齊,宋,秦,楚,以為外主,有欒,郤,狐,先,以為內主,亡十九年,守志彌篤,惠懷棄民,民從而與之,獻無異親,民無異望,天方相晉,將何以代文,此二君者,異於子干,共有寵子,國有奧主,無施於民,無援於外,去晉而不送,歸楚而不逆,何以冀國。

晉成虒祁,諸侯朝而歸者,皆有貳心,為取郠故,晉將以諸侯來討,叔向曰,諸侯不可以不示威,乃並徵會告于吳,秋,晉侯會吳子干良,水道不可,吳子辭,乃還,七月,丙寅,治兵于邾,南甲車四千乘,羊舌鮒攝司馬,遂合諸侯于平丘,子產,子大叔,相鄭伯以會,子產以幄幕九張行,子大叔以四十,既而悔之,每舍損焉,及會亦如之,次于衛地,叔鮒求貨於衛,淫芻蕘者,衛人使屠伯饋叔向羹,與一篋錦,曰,諸侯事晉,未敢攜貳,況衛在君之宇下,而敢有異志,芻蕘者異於他日,敢請之,叔向受羹,反錦曰,晉有羊舌鮒者,瀆貨無厭,亦將及矣,為此役也,子若以君命賜之,其已,客從之,未退而禁之,晉人將尋盟,齊人不可,晉侯使叔向告劉獻公曰,抑齊人不盟,若之何,對曰,盟以厎信,君苟有信,諸侯不貳,何患焉,告之以文辭,董之以武師,雖齊不許,君庸多矣,天子之老,請帥王賦,元戎十乘,以先啟行,遲速唯君,叔向告于齊曰,諸侯求盟,已在此矣,今君弗利,寡君以為請,對曰,諸侯討貳,則有尋盟,若皆用命,何盟之尋,叔向曰,國家之敗,有事而無業,事則不經,有業而無禮,經則不序,有禮而無威,序則不共,有威而不昭,共則不明,不明棄共百事,不終所由傾覆也,是故明王之制,使諸侯歲聘以志業,間朝以講禮,再朝而會以示威,再會而盟以顯昭明,志業於好,講禮於等,示威於眾,昭明於神,自古以來,未之或失也,存亡之道,恆由是興,晉禮主盟,懼有不治,奉承齊犧,而布諸君,求終事也,君曰余必廢之,何齊之有,唯君圖之,寡君聞命矣,齊人懼,對曰,小國言之,大國制之,敢不聽從,既聞命矣,敬共以往,遲速唯君,叔向曰,諸侯有間矣,不可以不示眾,八月,辛未,治兵,建而不旆,壬申,復旆之,諸侯畏之,邾人,莒人,愬于晉曰,魯朝夕伐我,幾亡矣,我之不共,魯故之以,晉侯不見公,使叔向來辭曰,諸侯將以甲戌盟,寡君知不得事君矣,請君無勤,子服惠伯對曰,君信蠻夷之訴,以絕兄弟之國,棄周公之後,亦惟君,寡君聞命矣,叔向曰,寡君有甲車四千乘在,雖以無道,行之必可畏也,況其率道,其何敵之有,牛雖瘠,僨於豚上,其畏不死,南蒯子仲之憂,其庸可棄乎,若奉晉之眾,用諸侯之師,因邾莒杞鄫之怒,以討魯罪,間其二憂,何求而弗克,魯人懼聽命,甲戌,同盟于平丘,齊服也,令諸侯日中造于除,癸酉退朝,子產命外僕速張於除,子大叔止之,使待,明白,及夕,子產聞其未張也,使速往,乃無所張矣,及盟,子產爭承,曰,昔天子班貢,輕重以列,列尊貢重,周之制也,卑而貢重者,甸服也,鄭伯,男也,而使從公侯之貢,懼弗給也,敢以為請,諸侯靖兵,好以為事,行理之命,無月不至,貢之無藝,小國有闕,所以得罪也,諸侯脩盟存小國也,貢獻無極,亡可待也,存亡之制,將在今矣,自日中以爭,至于昏,晉人許之,既盟,子大叔咎之,曰,諸侯若討,其可瀆乎,子產曰,晉政多門,貳偷之不暇,何暇討國,不競亦陵,何國之為,公不與盟,晉人執季孫意如,以幕蒙之,使狄人守之,司鐸射懷錦奉壺飲冰,以蒲伏焉,守者御之,乃與之錦而入,晉人以平子歸,子服湫從,子產歸,未至,聞子皮卒,哭且曰,吾已無為為善矣,唯夫子知我,仲尼謂子產於是行也,足以為國基矣,詩曰,樂只君子,邦家之基,子產,君子之求樂者也,且曰,合諸侯,藝貢事,禮也。

鮮虞人聞晉師之悉起也,而不警邊,且不脩備,晉荀吳自著雍以上軍侵鮮虞,及中人,驅衝競,大獲而歸。

楚之滅蔡也,靈王遷許,胡,沈,道,房,申,於荊焉,平王即位,既封陳蔡,而皆復之,禮也,隱大子之子廬,歸于蔡,禮也,悼大子之子吳,歸于陳,禮也。

冬,十月,葬蔡靈公,禮也。

公如晉,荀吳謂韓宣子曰,諸侯相朝,講舊好也,執其卿而朝其君,有不好焉,不如辭之,乃使士景伯辭公于河。

吳滅州來,令尹子期請伐吳,王弗許,曰,吾未撫民人,未事鬼神,未脩守備,未定國家,而用民力,敗不可悔,州來在吳,猶在楚也,子始待之。

季孫猶在晉,子服惠伯私於中行穆子,曰,魯事晉何以不如夷之小國,魯,兄弟也,土地猶大,所命能具,若為夷棄之,使事齊楚,其何瘳於晉,親親與大,賞共罰否,所以為盟主也,子其圖之,諺曰,臣一主二,吾豈無大國,穆子告韓宣子,且曰,楚滅陳蔡,不能救而為夷執親,將焉用之,乃歸季孫,惠伯曰,寡君未知其罪,合諸侯而執其老,若猶有罪,死命可也,若曰無罪,而惠免之,諸侯不聞,是逃命也,何免之為,請從君惠於會,宣子患之,謂叔向曰,子能歸季孫乎,對曰不能,鮒也能乃使叔魚,叔魚見季孫曰,昔鮒也得罪於晉君,自歸於魯君,微武子之賜,不至於今,雖獲歸骨於晉,猶子則肉之,敢不盡情,歸子而不歸鮒也,聞諸吏將為子除館於西河,其若之何,且泣,平子懼,先歸,惠伯待禮。

十有四年,春,意如至自晉。

三月,曹伯滕卒。

夏,四月。

秋,葬曹武公。

八月,莒子去疾卒。

冬,莒殺其公子意恢。

十四年,春,意如至自晉,尊晉罪己也,尊晉罪己,禮也。

南蒯之將叛也,盟費人,司徒老祁,慮癸,偽廢疾,使請於南蒯曰,臣願受盟而疾興,若以君靈不死,請侍間而盟,許之,二子因民之欲叛也,請朝眾而盟,遂劫南蒯,曰,群臣不忘其君,畏子以及今,三年聽命矣,子若弗圖,費人不忍其君,將不能畏子矣,子何所不逞欲,請送子,請期五日,遂奔齊,侍飲酒於景公,公曰,叛夫,對曰,臣欲張公室也,子韓皙曰,家臣而欲張公室,罪莫大焉,司徒老祁,慮癸來歸費,齊侯使鮑文子致之。

夏,楚子使然丹簡上國之兵於宗丘,且撫其民,分貧振窮,長孤幼,養老疾,收介特,救災患,宥孤寡,赦罪戾,詰姦慝,舉淹滯,禮新敘舊,祿勳合親,任良物官,使屈罷簡東國之兵於召陵,亦如之,好於邊疆息,民五年,而後用師,禮也。

秋,八月,莒著丘公卒,郊公不慼,國人弗順,欲立著丘公之弟庚與,蒲餘侯惡公子意恢,而善於庚與,郊公惡公子鐸,而善於意恢,公子鐸因蒲餘侯而與之謀,曰,爾殺意恢,我出君而納庚與,許之。

楚令尹子旗有德於王,不知度與養氏比,而求無厭,王患之。九月,甲午,楚子殺鬥成然而滅養氏之族,使鬥辛居鄖,以無忘舊勳。

冬,十二月,蒲餘侯茲夫殺莒公子意恢,郊公奔齊,公子鐸逆庚與於齊,齊隰黨,公子鉏,送之,有賂田。

晉邢侯與雍子爭鄐田,久而無成,士景伯如楚,叔魚攝理,韓宣子命斷舊獄,罪在雍子,雍子納其女於叔魚,叔魚蔽罪邢侯,邢侯怒,殺叔魚,與雍子於朝,宣子問其罪於叔向,叔向曰,三人同罪,施生戮死,可也,雍子自知其罪,而賂以買直,鮒也鬻獄,邢侯專殺,其罪一也。已惡而掠美為昏,貪以敗官為墨,殺人不忌為賊。夏書曰,昏墨賊殺,皋陶之刑也,請從之,乃施邢侯,而尸雍子,與叔魚於市,仲尼曰,叔向,古之遺直也,治國制刑,不隱於親,三數叔魚之惡,不為末減,曰,義也夫,可謂直矣,平丘之會,數其賄也,以寬衛國,晉不為暴,歸魯季孫,稱其詐也,以寬魯國,晉不為虐,邢侯之獄,言其貪也,以正刑書,晉不為頗,三言而除,三惡加三利,殺親益榮,猶義也夫。

十有五年,春,王正月,吳子夷末卒。

二月,癸酉,有事于武宮籥入叔弓卒,去樂卒事。

夏,蔡朝吳出奔鄭。

六月,丁巳,朔日有食之。

秋,晉荀吳帥師伐鮮虞。

冬,公如晉。

十五年,春,將禘于武公,戒百官,梓慎曰,禘之日,其有咎乎,吾見赤墨之祲,非祭祥也,喪氛也,其在蒞事乎,二月,癸酉,禘,叔弓蒞事,籥入而卒,去樂卒事,禮也。

楚費無極害朝吳之在蔡也,欲去之,乃謂之曰,王唯信子,故處子於蔡,子亦長矣,而在下位,辱,必求之,吾助子請,又謂其上之人曰,王唯信吳,故處諸蔡,二三子莫之如也,而在其上,不亦難乎,弗圖,必及於難,夏,蔡人逐朝吳,朝吳出奔鄭,王怒曰,余唯信吳,故寘諸蔡,且微吳,吾不及此,女何故去之,無極對曰,臣豈不欲吳,然而前知其為人之異也,吳在蔡,蔡必速飛,去吳,所以翦其翼也。

六月,乙丑,王大子壽卒。

秋,八月,戊寅,王穆后崩。

晉荀吳帥師伐鮮虞圍,鼓鼓,人或請以城叛,穆子弗許,左右曰,師徒不勤,而可以獲城,何故不為,穆子曰,吾聞諸叔向曰,好惡不愆,民知所適,事無不濟,或以吾城叛,吾所甚惡也,人以城來,吾獨何好焉,賞所甚惡,若所好何,若其弗賞,是失信也,何以庇民,力能則進,否則退,量力而行,吾不可以,欲城而邇姦,所喪滋多,使鼓人殺叛人而繕守備,圍鼓三月,鼓人或請降,使其民見曰,猶有食色,姑脩而城,軍吏曰,獲城而弗取,勤民而頓兵,何以事君,穆子曰,吾以事君也,獲一邑而教民,怠將焉用,邑邑以賈怠,不如完舊,賈怠無卒,棄舊不祥,鼓人能事其君,我亦能事吾君,率義不爽,好惡不愆,城可獲而民知義,所有死命,而無二心,不亦可乎,鼓人告食竭力盡,而後取之,克鼓而反,不戮一人,以鼓子截鞮歸。

冬,公如晉,平丘之會故也。

十二月,晉荀躒如周葬穆后,籍談為介,既葬除喪,以文伯宴,樽以魯壺,王曰,伯氏,諸侯皆有以鎮撫王室,晉獨無有,何也,文伯揖籍談對曰,諸侯之封也,皆受明器於王室,以鎮撫其社稷,故能薦彝器於王,晉居深山,戎狄之與鄰,而遠於王室,王靈不及,拜戎不暇,其何以獻器,王曰,叔氏而忘諸乎,叔父唐叔,成王之母弟也,其反無分乎,密須之鼓,與其大路,文所以大蒐也,闕鞏之甲,武所以克商也,唐叔受之,以處參虛,匡有戎狄,其後襄之二路,鏚鉞秬鬯,彤弓虎賁,文公受之,以有南陽之田,撫征東夏,非分而何,夫有勳而不廢,有績而載,奉之以土田,撫之以彝器,旌之以車服,明之以文章,子孫不忘,所謂福也,福祚之不登,叔父焉在,且昔而高祖孫伯黶司晉之典籍,以為大政,故曰籍氏,及辛有之二子董之,晉於是乎有董史,女司典之後也,何故忘之,籍談不能對,賓出,王曰,籍父其無後乎,數典而忘其祖,籍談歸以告叔向,叔向曰,王其不終乎,吾聞之,所樂必卒焉,今王樂憂,若卒以憂,不可謂終,王一歲而有三年之喪二焉,於是乎以喪賓宴,又求彝器,樂憂甚矣,且非禮也,彝器之來,嘉功之由,非由喪也,三年之喪,雖貴遂服,禮也,王雖弗遂,宴樂以早,亦非禮也,禮,王之大經也,一動而失二禮,無大經矣,言以考典,典以志經,忘經而多言,舉典將焉用之。

十有六年,春,齊侯伐徐。

楚子誘戎蠻子殺之。

夏,公至自晉。

秋,八月,己亥,晉侯夷卒。

九月,大雩。

季孫意如如晉。

冬,十月,葬晉昭公。

十六年,春,王正月,公在晉,晉人止公,不書,諱之也。

齊侯伐徐,楚子聞蠻氏之亂也,與蠻子之無質也,使然丹誘戎蠻子嘉,殺之,遂取蠻氏,既而復立其子焉,禮也,二月,丙申,齊師至于蒲隧,徐人行成,徐子及郯人,莒人,會齊侯盟于蒲隧,賂以甲父之鼎,叔孫昭子曰,諸侯之無伯,害哉,齊君之無道也,興師而伐遠方,會之有成,而還莫之亢也,無伯也夫,詩曰,宗周既滅,靡所止戾,正大夫離居,莫知我肄,其是之謂乎。

二月,晉韓起聘于鄭,鄭伯享之,子產戒曰,苟有位於朝,無有不共恪,孔張後至,立於客間,執政禦之,適客後,又禦之,適縣間,客從而笑之,事畢,富子諫,曰,夫大國之人,不可不慎也,幾為之笑,而不陵我,我皆有禮,夫猶鄙我,國而無禮,何以求榮,孔張失位,吾子之恥也,子產怒曰,發命之不衷,出令之不信,刑之頗類,獄之放紛,會朝之不敬,使命之不聽,取陵於大國,罷民而無功,罪及而弗知,僑之恥也,孔張,君之昆孫,子孔之後也,執政之嗣也,為嗣大夫,承命以使,周於諸侯,國人所尊,諸侯所知,立於朝而祀於家,有祿於國,有賦於軍,喪祭有職,受脤歸賑,其祭在廟,已有著位,在位數世,世守其業,而忘其所僑,焉得恥之,辟邪之人,而皆及執政,是先王無刑罰也,子寧以他規我,宣子有環,其一在鄭商,宣子謁諸鄭伯,子產弗與,曰,非官府之守器也,寡君不知,子大叔,子羽,謂子產曰,韓子亦無幾求,晉國亦未可以貳,晉國韓子,不可偷也,若屬有讒人,交鬥其間,鬼神而助之,以興其凶怒,悔之何及,吾子何愛於一環,其以取憎於大國也,盍求而與之,子產曰,吾非偷晉而有二心,將終事之,是以弗與,忠信故也,僑聞君子非無賄之難立,而無令名之患,僑聞為國非不能事大,字小之難,無禮以定其位之患,夫大國之人,令於小國,而皆獲其求,將何以給之,一共一否,為罪滋大,大國之求,無禮以斥之,何饜之有,吾且為鄙邑,則失位矣,若韓子奉命以使而求玉焉,貪淫甚矣,獨非罪乎,出一玉以起二罪,吾又失位,韓子成貪,將焉用之,且吾以玉賈罪,不亦銳乎,韓子買諸賈人,既成賈矣,商人曰,必告君大夫,韓子請諸子產曰,日起請夫環,執政弗義,弗敢復也,今買諸商人,商人曰,必以聞,敢以為請,子產對曰,昔我先君桓公,與商人皆出自周,庸次比耦,以艾殺此地,斬之蓬蒿藜藿而共處之,世有盟誓,以相信也,曰爾無我叛,我無強賈,毋或匄奪,爾有利市寶賄,我勿與知,恃此質誓,故能相保,以至于今,今吾子以好來辱,而謂敝邑強奪商人,是教敝邑背盟誓也,毋乃不可乎,吾子得玉而失諸侯,必不為也,若大國令,而共無藝,鄭鄙邑也,亦弗為也僑若獻玉,不知所成,敢私布之,韓子辭玉曰,起不敏,敢求玉以徼二罪,敢辭之。

夏,四月,鄭六卿餞宣子於郊,宣子曰,二三君子請皆賦,起亦以知鄭志,子齹賦野有蔓草,宣子曰,孺子善哉,吾有望矣,子產賦鄭之羔裘,宣子曰,起不堪也,子大叔賦褰裳,宣子曰,起在此,敢勤子,至於他人乎,子大叔拜,宣子曰,善哉,子之言,是不有是事,其能終乎,子游賦風雨,子旗賦有女同車,子柳賦蘀兮,宣子喜曰,鄭其庶乎,二三君子,以君命貺起,賦不出鄭志,皆昵燕好也,二三君子,數世之主也,可以無懼矣,宣子皆獻馬焉,而賦我將,子產拜,使五卿皆拜,曰,吾子靖亂,敢不拜德,宣子私覲於子產,以玉與馬曰,子命起,舍夫玉,是賜我玉而免吾死也,敢藉手以拜。

公至自晉,子服昭伯語季平子曰,晉之公室,其將遂卑矣,君幼弱,六卿彊而奢傲,將因是以習,習實為常,能無卑乎,平子曰,爾幼,惡識國。

秋,八月,晉昭公卒。

九月,大雩,旱也,鄭大旱,使屠擊,祝款,竪柎,有事於桑山,斬其木不雨,子產曰,有事於山,蓺山林也,而斬其木,其罪大矣,奪之官邑。

冬,十月,季平子如晉,葬昭公,平子曰,子服回之言猶信,子服氏有子哉。

十有七年,春,小邾子來朝。

夏,六月,甲戌,朔,日有食之。

秋,郯子來朝。

八月,晉荀吳帥師滅陸渾之戎。

冬,有星孛于大辰。

楚人及吳戰于長岸。

十七年,春,小邾穆公來朝,公與之燕,季平子賦采叔,穆公賦菁菁者莪,昭子曰,不有以國,其能久乎。

夏,六月,甲戌,朔,日有食之,祝史請所用幣,昭子曰,日有食之,天子不舉,伐鼓於社,諸侯用幣於社,伐鼓於朝,禮也,平子禦之,曰,止也,唯正月朔,慝未作,日有食之,於是乎有伐鼓用幣,禮也,其餘則否,大史曰,在此月也,日過分而未至,三辰有災,於是乎百官降物,君不舉辟,移時樂奏鼓,祝用幣,史用辭,故夏書曰,辰不集于房,瞽奏鼓,嗇夫馳,庶人走,此月朔之謂也,當夏四月,是謂孟夏,平子弗從,昭子退曰,夫子將有異志,不君君矣。

秋,郯子來朝,公與之宴,昭子問焉,曰,少皞氏鳥名官,何故也,郯子曰,吾祖也,我知之,昔者黃帝氏以雲紀,故為雲師而雲名,炎帝氏以火紀,故為火師而火名,共工氏以水紀,故為水師而水名,大皞氏以龍紀,故為龍師而龍名,我高祖少皞,摯之立也,鳳鳥適至,故紀於鳥,為鳥師而鳥名,鳳鳥氏歷正也,玄鳥氏司分者也,伯趙氏司至者也,青鳥氏司啟者也,丹鳥氏司閉者也,祝鳩氏司徒也,鴡鳩氏司馬也,鳲鳩氏司空也,爽鳩氏司寇也,鶻鳩氏司事也,五鳩,鳩民者也,五雉為五工正,利器用,正度量,夷民者也,九扈為九農正,扈民無淫者也,自顓頊以來,不能紀遠,乃紀於近,為民師而命以民事,則不能故也,仲尼聞之,見於郯子而學之,既而告人曰,吾聞之,天子失官,學在四夷,猶信。

晉侯使屠蒯如周,請有事於雒,與三塗,萇弘謂劉子曰,客容猛,非祭也,其伐戎乎,陸渾氏甚睦於楚,必是故也,君其備之,乃警戎備,九月,丁卯,晉荀吳帥師,涉自棘津,使祭史先用牲于雒,陸渾人弗知,師從之,庚午,遂滅陸渾,數之以其貳於楚也,陸渾子奔楚,其眾奔甘鹿,周大獲,宣子夢文公攜荀吳,而授之陸渾,故使穆子帥師,獻俘于文宮。

冬,有星孛于大辰,西及漢,申須曰,彗所以除舊布新也,天事恆象,今除於火,火出必布焉,諸侯其有火災乎,梓慎曰,往年吾見之,是其徵也,火出而見,今茲火出而章,必火入而伏,其居火也久矣,其與不然乎,火出,於夏為三月於商為四月,於周為五月,夏數得天,若火作,其四國當之,在宋衛陳鄭乎,宋,大辰之虛也,陳,大皞之虛也,鄭,祝融之虛也,皆火房也,星孛天漢,漢,水祥也,衛,顓頊之虛也,故為帝丘,其星為大水,水火之牡也,其以丙子若壬午作乎,水火所以合也,若火入而伏,必以壬午,不過其見之月,鄭裨灶言於子產曰,宋衛陳鄭,將同日火,若我用瓘斝玉瓚,鄭必不火,子產弗與。

吳伐楚,陽匄為令尹,卜戰不吉,司馬子魚曰,我得上流,何故不吉,且楚故,司馬令龜,我請改卜,令曰,魴也,以其屬死之,楚師繼之,尚大克之,吉,戰于長岸,子魚先死,楚師繼之,大敗吳師,獲其乘舟餘皇,使隨人與後至者守之,環而塹之,及泉,盈其隧炭,陳以待命,吳公子光請於其眾曰,喪先王之乘舟,豈唯光之罪,眾亦有焉,請藉取之,以救死,眾許之,使長鬣者三人,潛伏於舟側,曰,我呼皇則對,師夜從之,三呼皆迭對,楚人從而殺之,楚師亂,吳人大敗之,取餘皇以歸。

十有八年,春,王三月,曹伯須卒。

夏,五月,壬午,宋,衛,陳,鄭,災。

六月,邾人入鄅。

秋,葬曹平公。

冬,許遷于白羽。

十八年,春,王二月,乙卯,周毛得殺毛伯過而代之,萇弘曰,毛得必亡,是昆吾稔之日也,侈故之以,而毛得以濟侈於王都,不亡何待。

三月,曹平公卒。

夏,五月,火始昏見,丙子,風,梓慎曰,是謂融風,火之始也,七日其火作乎,戊寅,風甚,壬午,大甚,宋衛,陳,鄭,皆火,梓慎登大庭氏之庫以望之,曰,宋,衛,陳,鄭,也,數日皆來告火,裨灶曰,不用吾言,鄭又將火,鄭人請用之,子產不可,子大叔曰,寶以保民也,若有火,國幾亡,可以救亡,子何愛焉,子產曰,天道遠,人道邇,非所及也,何以知之,灶焉知天道,是亦多言矣,豈不或信,遂不與,亦不復火,鄭之未災也,里析告子產曰,將有大祥,民震動,國幾亡,吾身泯焉,弗良及也,國遷,其可乎,子產曰,雖可,吾不足以定遷矣,及火,里析死矣,未葬,子產使輿三十人遷其柩,火作,子產辭晉公子公孫于東門,使司寇出新客,禁舊客,勿出於宮,使子寬,子上,巡群屏攝至于大宮,使公孫登徙大龜,使祝史徙主祏於周廟,告於先君,使府人,庫人,各儆其事,商成公,儆司宮,出舊宮人,寘諸火所不及,司馬,司寇,列居火道,行火所焮,城下之人,伍列登城,明日,使野司寇,各保其徵,郊人助祝史除於國北,禳火于玄冥回祿,祈于四鄘,書焚室而寬其征,與之材,三日哭,國不市,使行人告於諸侯,宋衛皆如是,陳不救火,許不弔災,君子是以知陳許之先亡也。

六月,鄅人鄅藉,稻邾人襲,鄅人將閉門,邾人羊羅,攝其首焉,遂入之,盡俘以歸,鄅子曰,余無歸矣,從帑於邾,邾莊公反鄅夫人,而舍其女。

秋,葬曹平公,往者見周原伯魯焉。

與之語,不說學,歸以語閔子馬,閔子馬曰,周其亂乎,夫必多有是說,而後及其大人,大人患失而惑,又曰,可以無學,無學不害,不害而不學,則苟而可,於是乎下陵上替,能無亂乎,夫學,殖也,不學將落,原氏其亡乎。

七月,鄭子產為火故,大為社,祓禳於四方,振除火災,禮也,乃簡,兵大蒐,將為蒐除,子大叔之廟,在道南,其寢在道北,其庭小,過期三日,使除徒陳於道南廟北曰,子產過女,而命速除,乃毀於而鄉,子產朝,過而怒之,除者南毀,子產及衝,使從者止之,曰,毀於北方,火之作也,子產授兵登陴,子大叔曰,晉無乃討乎,子產曰,吾聞之,小國忘守則危,況有災乎,國之不可小,有備故也,既晉之邊吏讓鄭曰,鄭國有災,晉君大夫不敢寧居,卜筮走望,不愛牲玉,鄭之有災,寡君之憂也,今執事撊然授兵登陴,將以誰罪,邊人恐懼,不敢不告,子產對曰,若吾子之言,敝邑之災,君之憂也,敝邑失政,天降之災,又懼讒慝之間謀之,以啟貪人,荐為敝邑不利,以重君之憂,幸而不亡,猶可說也,不幸而亡,君雖憂之,亦無及也,鄭有他竟,望走在晉,既事晉矣,其敢有二心。

楚左尹王子勝言於楚子曰,許於鄭,仇敵也,而居楚地,以不禮於鄭,晉鄭方睦,鄭若伐許,而晉助之,楚喪地矣,君盍遷許,許不專於楚,鄭方有令政,許曰,余舊國也,鄭曰,余俘邑也,葉在楚國,方城外之蔽也,土不可易,國不可小,許不可俘,讎不可啟,君其圖之,楚子說,冬,楚子使王子勝遷許於析,實白羽。

十有九年,春,宋公伐邾。

夏,五月,戊辰,許世子止弒其君買。

己卯,地震。

秋齊高發帥師伐莒。

冬,葬許悼公

十九年,春,楚工尹赤遷陰于下陰,令尹子瑕城郟,叔孫昭子曰,楚不在諸侯矣,其僅自完也,以持其世而已。

楚子之在蔡也,具陽封人之女奔之,生大子建,及即位,使伍奢為之師,費無極為少師,無寵焉,欲譖諸王,曰,建可室矣,王為之聘於秦,無極與逆,勸王取之,正月,楚夫人嬴氏至自秦。

鄅夫人,嬋向戌之女也,故向寧請師,二月,宋公伐邾圍蟲,三月取之,乃盡歸鄅俘。

夏,許悼公瘧,五月,戊辰,飲大子止之藥,卒,大子奔晉,書曰,弒其君,君子曰,盡心力以事君,舍藥物可也。

邾人,郳人,徐人,會宋公,乙亥,同盟于蟲,楚子為舟師以伐濮,費無極言於楚子曰,晉之伯也,邇於諸夏,而楚辟陋,故弗能與爭,若大城城父,而寘大子焉,以通北方,王收南方,是得天下也,王說,從之,故太子建居于城父,令尹子瑕聘于秦,拜夫人也。

秋,齊高發帥師伐莒,莒子奔紀鄣,使孫書伐之,初,莒有婦人,莒子殺其夫,己為嫠婦,及老,託於紀鄣,紡焉以度而去之,及師至,則投諸外,或獻諸子占,子占使師夜縋而登,登者六十人,縋絕,師鼓譟,城上之人亦譟,莒共公懼,啟西門而出,七月,丙子,齊師入紀。

是歲也,鄭駟偃卒,子游娶於晉大夫,生絲弱,其父兄立子瑕,子產憎其為人也,且以為不順,弗許,亦弗止,駟氏聳,他日,絲以告其舅,冬,晉人使以幣如鄭,問駟乞之立故,駟氏懼,駟乞欲逃,子產弗遣,請龜以卜,亦弗予,大夫謀對,子產不待而對客曰,鄭國不天,寡君之二三臣,札瘥夭昏,今又喪我先大夫偃,其子幼弱,其一二父兄,懼隊宗主,私族於謀,如立長親,寡君與其二三老曰,抑天實剝亂,是吾何知焉,諺曰,無過亂門,民有亂兵,猶憚過之,而況敢知天之所亂,今大夫將問其故,抑寡君實不敢知,其誰實知之,平丘之會,君尋舊盟曰,無或失職,若寡君之二三臣,其即世者,晉大夫而專制其位,是晉之縣鄙也,何國之為,辭客幣而報其使,晉人舍之。

楚人城州來,沈尹戌曰,楚人必敗,昔吳滅州來,子旗請伐之,王曰,吾未撫吾民,今亦如之,而城州來,以挑吳,能無敗乎,侍者曰,王施舍不倦,息民五年,可謂撫之矣,戌曰,吾聞撫民者,節用於內,而樹德於外,民樂其性,而無寇讎,今宮室無量,民人日駭,勞罷死轉,忘寢與食,非撫之也。

鄭大水,龍鬥于時門之外洧淵,國人請為禜焉,子產弗許,曰,我鬥,龍不我覿也,龍鬥,我獨何覿焉,禳之則彼其室也,吾無求於龍,龍亦無求於我,乃止也。

令尹子瑕言蹶由於楚子曰,彼何罪。諺所謂室於怒,市於色者,楚之謂矣。舍前之忿,可也,乃歸蹶由。

二十年,春,王正月。

夏,曹公孫會自鄸出奔宋。

秋,盜殺衛侯之兄縶。

冬,十月,宋華亥向寧,華定,出奔陳。

十有一月,辛卯,蔡侯盧卒。

二十年,春,王二月,己丑,日南至,梓慎望氛,曰,今茲宋有亂,國幾亡,三年而後弭,蔡有大喪,叔孫昭子曰,然則戴桓也,汏侈無禮,已甚,亂所在也。

費無極言於楚子曰,建與伍奢,將以方城之外叛,自以為猶宋鄭也,齊晉又交輔之,將以害楚,其事集矣,王信之,問伍奢,伍奢對曰,君一過多矣,何信於讒,王執伍奢,使城父司馬奮揚殺大子,未至而使遣之,三月,大子建奔宋,王召奮揚,奮揚使城父人執己以至,王曰,言出於余口,入於爾耳,誰告建也,對曰,臣告之,君王命臣曰,事建如事余,臣不佞,不能苟貳,奉初以還,不忍後命,故遣之,既而悔之,亦無及,巳,王曰,而敢來,何也,對曰,使而失命,召而不來,是再奸也,逃無所入王曰,歸,從政如他日,無極曰奢之子材,若在吳必憂楚國,盍以免其父召之,彼仁必來,不然將為患,王使召之,曰,來,吾免而父,棠君尚謂其弟員曰,爾適吳,我將歸死,吾知不逮,我能死,爾能報,聞免父之命,不可以莫之奔也,親戚為戮,不可以莫之報也,奔死免父,孝也,度功而行,仁也,擇任而往,知也,知死不辟,勇也,父不可棄,名不可廢,爾其勉之,相從為愈,伍尚歸,奢聞員不來,曰,楚君大夫其旰食乎,楚人皆殺之,員如吳,言伐楚之利於州于。公子光曰:是宗為戮,而欲反其讎,不可從也。員曰,彼將有他志,余姑為之求士,而鄙以待之,乃見鱄設諸焉,而耕於鄙。

宋元公無信多私而惡,華向,華定,華亥,與向寧謀曰,亡愈於死,先諸華亥,偽有疾以誘群公子,公子問之,則執之,夏,六月,丙申,殺公子寅,公子御戎,公子朱,公子固,公孫援,公孫丁,拘向勝,向行,於其廩,公如華氏請焉,弗許,遂劫之,癸卯,取大子欒,與母弟辰,公子地,以為質,公亦取華亥之子無慼,向寧之子羅,華定之子啟,與華氏盟以為質。

衛公孟縶狎齊豹,奪之司寇與鄄,有役則反之,無則取之,公孟惡北宮喜,褚師圃,欲去之,公子朝通于襄夫人宣姜,懼而欲以作亂,故齊豹,北宮喜,褚師圃,公子朝,作亂。初,齊豹見宗魯於公孟,為驂乘焉。將作亂,而謂之曰,公孟之不善,子所知也,勿與乘,吾將殺之。對曰,吾由子事公孟,子假吾名焉,故不吾遠也,雖其不善,吾亦知之,抑以利故,不能去,是吾過也,今聞難而逃,是僭子也,子行事乎,吾將死之,以周事子,而歸死於公孟,其可也,丙辰,衛侯在平壽,公孟有事於蓋獲之門外,齊子氏帷於門外,而伏甲焉,使祝蛙寘戈於車薪,以當門,使一乘從公孟以出,使華齊御公孟,宗魯驂乘,及閎中,齊氏用戈擊公孟,宗魯以背蔽之,斷肱,以中公孟之肩,殺皆之,公聞亂,乘驅自閱門入,慶比御公,公南楚驂乘,使華寅乘貳車,及公宮,鴻騮魋駟乘于公,公載寶以出,褚師子申,遇公于馬路之衢,遂從,過齊氏,使華寅肉袒執蓋,以當其闕,齊氏射公,中南楚之背,公遂出,寅閉郭門,踰而從公,公如死鳥,析朱鉏宵從竇出,徒行從公,齊侯使公孫青聘于衛,既出,聞衛亂,使請所聘,公曰,猶在竟內,則衛君也,乃將事焉,遂從諸死鳥,請將事,辭曰,亡人不佞,失守社稷,越在草莽,吾子無所辱君命。賓曰,寡君命下臣於朝曰:『阿下執事,臣不敢貳』。主人曰:君若惠顧先君之好,昭臨敝邑,鎮撫其社稷,則有宗祧在,乃止衛侯,固請見之,不獲命,以其良馬見,為未致使故也,衛侯以為乘馬,賓將掫,主人辭曰,亡人之憂,不可以及,吾子草莽之中,不足以辱從者,敢辭。賓曰,寡君之下臣,君之牧圉也,若不獲扞外役,是不有寡君也,臣懼不免於戾,請以除死,親執鐸,終夕與於燎,齊氏之宰渠子,召北宮子,北官氏之宰,不與聞謀,殺渠子,遂伐齊氏,滅之,丁巳,晦,公入,與北宮喜盟于彭水之上。秋,七月,戊午,朔,遂盟國人。八月,辛亥,公子朝,褚師圃,子玉霄,子高魴,出奔晉,閏月,戊辰,殺宣姜,衛侯賜北宮喜謚曰貞子,賜析朱鉏謚曰成子,而以齊氏之墓予之,衛侯告寧于齊,且言子石,齊侯將飲酒,遍賜大夫曰,二三子之教也,苑何忌辭曰,與於青之賞,必及于其罰,在《康誥》曰,父子兄弟,罪不相及,況在群臣,臣敢貪君賜,以干先王,琴張聞宗魯死,將往弔之,仲尼曰,齊豹之盜,而孟縶之賊,女何弔焉,君子不食姦,不受亂,不為利疚於回,不以回待人,不蓋不義,不犯非禮。

宋華向之亂,公子城,公孫忌,樂舍,司馬疆,向宜,向鄭,楚建,郳甲,出奔鄭,其徒與華氏戰于鬼閻,敗子城,子城適晉,華亥與其妻,必盥而食所質公子者,而後食,公與夫人,每日,必適華氏,食公子而後歸,華亥患之,欲歸公子,向寧曰,唯不信,故質其子,若又歸之,死無日矣,公請於華費,遂將攻華氏,對曰,臣不敢愛死,無乃求去憂而滋長乎,臣是以懼,敢不聽命,公曰,子死亡有命,余不忍其詢,冬,十月,公殺華向之質而攻之,戊辰,華向奔陳,華登奔吳,向寧欲殺大子,華亥曰,干君而出,又殺其子,其誰納我,且歸之有庸,使少司寇牼以歸,曰,子之齒長矣,不能事人,以三公子為質,必免,公子既入,華牼將自門行,公遽見之,執其手曰,余知而無罪也,入復而所。

齊侯疥,遂痁,期而不瘳,諸侯之賓問疾者多在,梁丘據與裔款言於公曰,吾事鬼神豐,於先君有加矣,今君疾病,為諸侯憂,是祝史之罪也,諸侯不知,其謂我不敬,君盍誅於祝固史嚚,以辭賓,公說,告晏子,晏子曰,日宋之盟,屈建問范會之德於趙武,趙武曰,夫子之家事治,言於晉國,竭情無私,其祝史祭祀,陳信不愧,其家事無猜,其祝史不祈,建以語康王,康王曰,神人無怨,宜夫子之光輔五君,以為諸侯主也,公曰,據與款謂寡人能事鬼神,故欲誅于祝史,子稱是語,何故,對曰,若有德之君,外內不廢,上下無怨,動無違事,其祝史薦信,無愧心矣,是以鬼神用饗,國受其福,祝史與焉,其所以蕃祉老壽者,為信君使也,其言忠信於鬼神,其適遇淫君,外內頗邪,上下怨疾,動作辟違,從欲厭私,高臺深池,撞鍾舞女,斬刈民力,輸掠其聚,以成其違,不恤後人,暴虐淫從,肆行非度,無所還忌,不思謗讟,不憚鬼神,神怒民痛,無悛於心,其祝史薦信,是言罪也,其蓋失數美,是矯誣也,進退無辭,則虛以求媚,是以鬼神不饗其國以禍之,祝史與焉,所以夭昏孤疾者,為暴君使也,其言僭嫚於鬼神,公曰,然則若之何,對曰,不可為也,山林之木,衡鹿守之,澤之萑蒲,舟鮫守之,藪之薪蒸,虞候守之,海之鹽蜃,祈望守之,縣鄙之人,入從其政,偪介之關,暴征其私,承嗣大夫,強易其賄,布常無藝,徵斂無度,宮室日更,淫樂不違,內寵之妾,肆奪於市,外寵之臣,僭令於鄙,私欲養求,不給則應,民人苦病,夫婦皆詛,祝有益也,詛亦有損,聊攝以東,姑尤以西,其為人也,多矣,雖其善祝,豈能勝億兆人之詛,君若欲誅於祝史,脩德而後可,公說,使有司寬政,毀關,去禁,薄斂,已責。

十二月,齊侯田于沛,招虞人以弓,不進,公使執之,辭曰,昔我先君之田也,旃以招大夫,弓以招士,皮冠以招虞人,臣不見皮冠,故不敢進,乃舍之仲尼曰,守道不如守官,君子同之,齊侯至自田,晏子侍于遄臺,子猶馳而造焉,公曰,唯據與我和夫,晏子對曰,據亦同也,焉得為和,公曰,和與同異乎,對曰異,和如羹焉,水火醯醢鹽梅,以烹魚肉,燀之以薪,宰夫和之,齊之以味,濟其不及,以洩其過,君子食之,以平其心,君臣亦然,君所謂可,而有否焉,臣獻其否,以成其可,君所謂否,而有可焉,臣獻其可,以去其否,是以政平而不干民無爭心,故詩曰,亦有和羹,既戒既平,鬷假無言,時靡有爭,先王之濟五味,和五聲也,以平其心,成其政也,聲亦如味,一氣,二體,三類,四物,五聲,六律,七音,八風,九歌,以相成也,清濁大小,長短疾徐,哀樂剛柔,遲速高下,出入周疏,以相濟也,君子聽之,以平其心,心平德和,故詩曰,德音不瑕,今據不然,君所謂可,據亦曰可,君所謂否,據亦曰否,若以水濟水,誰能食之,若琴瑟之專壹,誰能聽之,同之不可也如是,飲酒樂,公曰,古而無死,其樂若何,晏子對曰,古而無死,則古之樂也,君何得焉,昔爽鳩氏始居此地,季萴因之,有逢伯陵因之,蒲姑氏因之,而後大公因之,古者無死,爽鳩氏之樂,非君所願也。

鄭子產有疾,謂子大叔曰,我死,子必為政,唯有德者,能以寬服民,其次莫如猛,夫火烈,民望而畏之,故鮮死焉,水懦弱,民狎而翫之,則多死焉,故寬難,疾數月而卒,大叔為政,不忍猛而寬,鄭國多盜,取人於萑苻之澤,大叔悔之,曰,吾早從夫子,不及此,興徒兵以攻萑苻之盜,盡殺之,盜少止,仲尼曰,善哉,政寬則民慢,慢則糾之以猛,猛則民殘,殘則施之以寬,寬以濟猛,猛以濟寬,政是以和,詩曰,民亦勞之,汔可小康,惠此中國,以綏四方,施之以寬也,毋從詭隨,以謹無良,式遏寇虐,慘不畏明,糾之以猛也,柔遠能邇,以定我王,平之以和也,又曰,不競不絿,不剛不柔,布政優優,百祿是遒,和之至也,及子產卒,仲尼聞之,出涕曰,古之遺愛也。

二十有一年,春,王三月,葬蔡平公。

夏,晉侯使士鞅來聘。

宋華亥,向寧,華定,自陳入于宋南里以叛。

秋,七月,壬午,朔,日有食之。

八月,乙亥,叔輒卒。

冬,蔡侯朱出奔楚。

公如晉至河乃復。

二十一年,春,天王將鑄無射,泠州鳩曰,王其以心疾死乎,夫樂,天子之職也,夫音,樂之輿也,而鐘,音之器也,天子省風以作樂,器以鐘之,輿以行之,小者不窕,大者不槬,則和於物,物和則嘉成,故和聲入於耳,而藏於心,心億則樂,窕則不咸,總則不容,心是以感,感實生疾,今鐘槬矣,王心弗堪,其能久乎。

三月,葬蔡平公,蔡大子朱失位,位在卑,大夫送葬者,歸見昭子,昭子問蔡,故以告,昭子歎曰,蔡其亡乎,若不亡,是君也必不終,詩曰,不解于位,民之攸塈,今蔡侯始即位,而適卑,身將從之。

夏,晉士鞅來聘,叔孫為政,季孫欲惡諸晉,使有司以齊鮑國歸費之禮為士鞅,士鞅怒曰,鮑國之位下,其國小,而使鞅從其牢禮,是卑敝邑也,將復諸寡君,魯人恐,加四牢焉,為十一牢。

宋華費遂生華貙,華多僚,華登,貙為少司馬,多僚為御士,與貙相惡,乃譖諸公曰,貙將納亡人,亟言之,公曰,司馬以吾故,亡其良子,死亡有命,吾不可以再亡之,對曰,君若愛司馬,則如亡,死如可逃,何遠之有,公懼,使侍人召司馬之侍人宜僚,飲之酒,而使告司馬,司馬歎曰,必多僚也,吾有讒子,而弗能殺,吾又不死,抑君有命,可若何,乃與公謀,逐華貙,將使田孟諸而遣之,公飲之酒,厚酬之,賜及從者,司馬亦如之,張匄尤之,曰,必有故,使子皮承宜僚以劍而訊之,宜僚盡以告,張匄欲殺多僚,子皮曰,司馬老矣,登之謂甚,吾又重之,不如亡也,五月,丙申,子皮將見司馬而行,則遇多僚,御司馬而朝,張匄不勝其怒,遂與子皮,臼任,鄭翩,殺多僚,劫司馬以叛,而召亡人,壬寅,華向入,樂大心,豐愆,華牼,禦諸橫,華氏居盧門,以南里叛,六月,庚午,宋城舊鄘及桑林之門,而守之。

秋,七月,壬午,朔,日有食之,公問於梓慎曰,是何物也,禍福何為,對曰,二至二分,日有食之,不為災,日月之行也,分同道也,至相過也,其他月則為災,陽不克也,故常為水,於是叔輒哭日食,昭子曰,子叔將死,非所哭也,八月,叔輒卒。

冬,十月,華登以吳師救華氏,齊烏枝鳴戍宋,廚人濮曰,軍志有之,先人有奪人之心,後人有待其衰,盍及其勞,且未定也,伐諸,若入而固,則華氏眾矣,悔無及也,從之,丙寅,齊師,宋師,敗吳師于鴻口,獲其二帥,公子苦雂,偃州員,華登,帥其餘以敗宋師,公欲出,廚人濮曰,吾小人,可藉死而不能送亡,君請待之,乃徇曰,楊徽者,公徒也,眾從之,公自楊門見之,下而巡之,曰,國亡君死,二三子之恥也,豈專孤之罪也。齊烏枝鳴曰:用少莫如齊致死,齊致死莫如去備。彼多兵矣,請皆用劍,從之,華氏北,復即之,廚人濮以裳裹首而荷以走曰,得華登矣,遂敗華氏,于新里,翟僂新居于新里,既戰,說甲于公,而歸華姓居于公里,亦如之,十一月,癸未,公子城以晉師至,曹翰胡會晉荀吳,齊苑何忌,衛公子朝,救宋,丙戌,與華氏戰于赭丘,鄭翩願為鸛,其御願為鵝,子祿御公子城,莊堇為右,于犨御呂封人,華豹張匄為右,相遇,城還,華豹曰,城也,城怒,而反之,將注豹,則關矣,曰,平公之靈,尚輔相余,豹射出其間,將注,則又關矣,曰,不狎鄙,抽矢,城射之,殪,張匄抽殳而下,射之,折股,扶伏而擊之,折軫,又射之,死,干犨請一矢,城曰,余言汝於君,封曰,不死伍乘,軍之大刑也,干刑而從子,君焉用之,子速諸,乃射之,殪,大敗華氏,圍諸南里,華亥搏膺而呼,見華貙曰,吾為欒氏矣,貙曰,子無我迋,不幸而後亡,使華登如楚乞師,華貙以車十五乘,徒七十人,犯師而出,食於睢上,哭而送之,乃復入,楚薳越帥師,將逆華氏,大宰犯諫曰,諸侯唯宋事其君,今又爭國,釋君而臣是助,無乃不可乎,王曰,而告我也,後既許之矣。

蔡侯朱出奔楚,費無極取貨於東國,而謂蔡人曰,朱不用命於楚,君王將立東國,若不先從王欲,楚必圍蔡,蔡人懼,出朱而立東國,朱愬于楚,楚子將討蔡,無極曰,平侯與楚有盟,故封其子,有二心,故廢之,靈王殺隱大子,其子與君同惡,德君必甚,又使立之,不亦可乎,且廢置在君,蔡無他矣。

公如晉,及河鼓叛晉,晉將伐鮮虞,故辭公。

二十有二年,春,齊侯伐莒。

宋,華亥,向寧,華定,自宋南里出奔楚。

大蒐于昌間。

夏,四月,乙丑,天王崩,六月,叔鞅如京師葬景王,王室亂。

劉子,單子,以王猛居于皇。

秋,劉子,單子,以王猛入于王城。

冬,十月,王子猛卒。

十有二月,癸酉,朔,日有食之。

二十二年,春,王二月,甲子,齊北郭啟帥師伐莒,莒子將戰,苑羊牧之諫曰,齊帥賤,其求不多,不如下之,大國不可怒也,弗聽,敗齊師于壽餘,齊侯伐莒,莒子行成,司馬灶如莒蒞盟,莒子如齊蒞盟,盟于稷門之外,莒於是乎大惡其君。

楚薳越使告于宋曰,寡君聞君有不令之臣,為君憂,無寧以為宗羞,寡君請受而戮之,對曰,孤不佞不能媚於父兄,以為君憂,拜命之辱,抑君臣曰戰,君曰,余必臣是助,亦唯命,人有言曰,唯亂門之無過,君若惠保敝邑,無亢不衷,以獎亂人,孤之望也,唯君圖之,楚人患之,諸侯之戍謀曰,若華氏知困而致死,楚恥無功而疾戰,非吾利也,不如出之,以為楚功,其亦能無為也,巳,救宋而除其害,又何求,乃固請出之,宋人從之,己巳,宋華亥,向寧,華定,華貙,華登,皇奄傷,省臧,士平,出奔楚,宋公使公孫忌為大司馬,邊卬為大司徒,樂祁為司馬,仲幾為左師,樂大心為右師,樂輓為大司寇,以靖國人。

王子朝,賓起,有寵於景王,王與賓孟說之,欲立之,劉獻公之庶子伯蚡事單穆公,惡賓孟之為人也,願殺之,又惡王子朝之言,以為亂,願去之,賓孟適郊,見雄雞自斷其尾,問之侍者曰,自憚其犧也,遽歸告王,且曰,雞其憚為人用乎,人異於是,犧者實用人,人犧實難,己犧何害,王弗應,夏,四月,王田北山,使公卿皆從,將殺單子,劉子,王有心疾,乙丑,崩于榮錡氏,戊辰,劉子摯卒,無子,單子立劉蚡,五月,庚辰,見王,遂攻賓起,殺之,盟群王子于單氏。

晉之取鼓也,既獻而反鼓子焉,又叛於鮮虞,六月,荀吳略東陽,使師偽糴者,負甲以息於昔陽之門外,遂襲鼓滅之,以鼓子鳶鞮歸,使涉佗守之。

丁巳葬景王王子朝因舊官百工之喪職秩者,與靈景之族以作亂,帥郊要餞之甲,以逐劉子,壬戌,劉子奔揚,單子逆悼王于莊宮,以歸,王子還,夜取王以如莊宮,癸亥,單子出,王子還與召莊公謀曰,不殺單旗,不捷,與之重盟,必來背盟,而克者多矣,從之,樊頃子曰,非言也,必不克,遂奉王以追單子,及領,大盟而復,殺摯荒以說,劉子如劉,單子亡,乙丑,奔于平畤,群王子追之,單子殺還,姑,發,弱,鬷,延,定,稠,子朝奔京,丙寅,伐之,京人奔山,劉子入于王城,辛未,鞏簡公敗績于京,乙亥,甘平公亦敗焉,叔鞅至自京師,言王室之亂也,閔馬父曰,子朝必不克,其所與者,天所廢也,單子欲告急於晉,秋,七月,戊寅,以王如平畤,遂如圃車,次于皇,劉子如劉,單子使王子處守于王城,盟百工于平宮,辛卯,鄩肸伐皇,大敗,獲鄩肸,壬辰,焚諸王城之市,八月,辛酉,司徒醜以王師敗績于前城,百工叛,己巳,伐單氏之宮,敗焉,庚午,反伐之,辛未,伐東圉,冬,十月,丁巳,晉籍談,荀躒,帥九州之戎,及焦瑕溫原之帥,以納王于王城,庚申,單子劉蚡以王師敗績于郊前,城人敗陸渾于社,十一月,乙酉,王子猛卒,不成喪也,己丑,敬王即位,館于子旅氏。

十二月,庚戌,晉籍談,荀躒,賈辛,司馬督,帥師軍于陰,于侯氏,于谿泉,次于社,王師軍于氾,于解,次于任人,閏月,晉箕遺,樂徵,右行詭,濟師,取前城,軍其東南,王師軍于京楚,辛丑,伐京,毀其西南。

二十有三年,春,王正月,叔孫婼如晉。

癸丑,叔鞅卒。

晉人執我行人叔孫婼。

晉人圍郊。

夏,六月,蔡侯東國卒于楚。

秋,七月,莒子庚輿來奔。

戊辰,吳敗頓胡,沈,蔡,陳,許,之師于雞父,胡子髡,沈子逞,滅獲陳夏齧。

天王居于狄泉,尹氏立王子朝。

八月,乙未,地震。

冬,公如晉,至河,有疾,乃復。

二十三年,春,王正月,壬寅,朔,二師圍郊,癸卯,郊鄩潰,丁未,晉師在平陰,王師在澤邑,王使告間,庚戌還。

邾人城翼,還,將自離姑,公孫鉏曰,魯將御我,欲自武城還,循山而南,徐鉏,丘弱,茅地,曰,道下遇雨,將不出,是不歸也,遂自離姑,武城人塞其前,斷其後之木而弗殊,邾師過之,乃推而蹙之,遂取邾師,獲鉏弱地,邾人愬于晉,晉人來討,叔孫婼如晉,晉執人之,書曰,晉人執我行人叔孫婼,言使人也,晉人使與邾大夫坐,叔孫曰,列國之卿,當小國之君,固周制也,邾又夷也,寡君之命介子服回在,請使當之,不敢廢周制故也,乃不果坐,韓宣子使邾人取其眾,將以叔孫與之,叔孫聞之,去眾與兵而朝,士彌牟謂韓宣子曰,子弗良圖,而以叔孫與其讎,叔孫必死之,魯亡叔孫,必亡邾,邾君亡國,將焉歸,子雖悔之,何及,所謂盟主,討違命也,若皆相執,焉用盟主,乃弗與,使各居一館,士伯聽其辭,而愬諸宣子,乃皆執之,士伯御叔孫,從者四人,過邾館以如吏,先歸邾子,士伯曰,以芻蕘之難,從者之病,將館子於都,叔孫旦而立,期焉,乃館諸箕,舍子服昭伯於他邑,范獻子求貨於叔孫,使請冠焉,取其冠法,而與之兩冠,曰,盡矣,為叔孫故,申豐以貨如晉,叔孫曰,見我,吾告女所行貨,見而不出,吏人之與叔孫居於箕者,請其吠狗,弗與,及將歸,殺而與之食之,叔孫所館者,雖一日,必葺其牆屋,去之如始至。

夏,四月,乙酉,單子取訾,劉子取牆人,直人,六月,壬午,王子朝入于尹,癸未,尹圉誘劉佗殺之,丙戌,單子從阪道,劉子從尹道,伐尹,單子先至而敗,劉子還,己丑,召伯奐,南宮極,以成周人戍尹,庚寅,單子,劉子,樊齊,以王如劉,甲午,王子朝入于王城,次于左巷,秋,七月,戊申,鄩羅納諸莊宮,尹辛敗劉師于唐,丙辰,又敗諸鄩,甲子,尹辛取西闈,丙寅,攻蒯,蒯潰。

莒子庚輿虐而好劍,苟鑄劍,必試諸人,國人患之,又將叛,齊烏存帥國人以逐之,庚輿將出,聞烏存執殳而立於道左,懼,將止死,苑羊牧之曰,君過之,烏存以力聞可矣,何必以弒君成名,遂來奔,齊人納郊公。

吳人伐州來,楚薳越帥師,及諸侯之師,奔命救州來,吳人禦諸鍾離,子瑕卒,楚師熸,吳公子光曰,諸侯從於楚者眾,而皆小國也,畏楚而不獲巳,是以來,吾聞之曰,作事威克其愛,雖小必濟,胡沈之君幼而狂,陳大夫齧壯而頑,頓與許蔡疾楚政,楚令尹死,其師熸,帥賤多寵,政令不壹,七國同役而不同心,帥賤而不能整,無大威命,楚可敗也,若分師先以犯胡沈與陳,必先奔,三國敗,諸侯之師乃搖心矣,諸侯乖亂,楚必大奔,請先者去備薄威,後者敦陳整旅,吳子從之,戊辰,晦,戰于雞父,吳子以罪人三千,先犯胡沈與陳,三國爭之,吳為三軍以繫於後,中軍從王,光帥右,掩餘帥左,吳罪之人,或奔或止,三國亂,吳師擊之,三國敗,獲胡沈之君,及陳大夫,舍胡沈之囚,使奔許與蔡頓,曰,吾君死矣,師譟而從之,三國奔,楚師大奔,書曰,胡子髡,沈子逞,滅,獲陳夏齧,君臣之辭也,不言戰,楚未陳也。

八月,丁酉,南宮極震,萇弘謂劉文公曰,君其勉之,先君之力可濟也,周之亡也,其三川震,今西王之大臣亦震,天棄之矣,東王必大克。

楚大子建之母在郹,召吳人而啟之,冬,十月,甲申,吳大子諸樊入郹,取楚夫人,與其寶器以歸,楚司馬薳越追之,不及,將死,眾曰,請遂伐吳以徼之,薳越曰,再敗君師,死且有罪,亡君夫人,不可以莫之死也,乃縊於薳澨。

公為叔孫故如晉,及河,有疾而復。

楚囊瓦為令尹,城郢,沈尹戌曰,子常必亡郢,苟不能衛,城無益也,古者天子守在四夷,天子卑,守在諸侯,諸侯守在四鄰,諸侯卑,守在四竟,慎其四竟,結其四援,民狎其野,三務成功,民無內憂,而又無外懼,國焉用城,今吳是懼,而城於郢,守巳小矣,卑之不獲,能無亡乎,昔梁伯溝其公宮而民潰,民棄其上,不亡何待,夫正其疆埸,脩其土田,險其走集,親其民人,明其伍候,信其鄰國,慎其官守,守其交禮,不僭不貪,不懦不耆,完其守備,以待不虞,又何畏矣,詩曰,無念爾祖,聿脩厥德,無亦監乎,若敖蚡冒,至于武文,土不過同,慎其四竟,猶不城郢,今土數圻而郢是城,不亦難乎。

二十四年,春,王三月,丙戌,仲孫貜卒。

婼至自晉。

夏,五月,乙未,朔,日有食之。

秋,八月,大雩。

丁酉,杞伯郁釐卒。

冬,吳滅巢。

葬杞平公。

二十四年,春,王正月,辛丑,召簡公,南宮嚚,以甘桓公見王子朝,劉子謂萇弘曰,甘氏又往矣,對曰,何害,同德度義,大誓曰,紂有億兆夷人,亦有離德,余有亂臣十人,同心同德,此周所以興也,君其務德,無患無人,戊午,王子朝入于鄔。

晉士彌牟逆叔孫于箕,叔孫使梁其脛待于門內,曰,余左顧而欬,乃殺之,右顧而笑,乃止,叔孫見士伯,士伯曰,寡君以為盟主之故,是以久子,不腆敝邑之禮,將致諸從者,使彌牟逆吾子,叔孫受禮而歸,二月,婼至自晉,尊晉也。

三月,庚戌,晉侯使士景伯蒞問周故,士伯立于乾祭,而問於介眾,晉人乃辭王子朝,不納其使。

夏,五月,乙未,朔,日有食之,梓慎曰,將水,昭子曰,旱也,日過分,無陽猶不克,克必甚,能無旱乎,陽不克莫,將積聚也。

六月,壬申,王子朝之師,攻瑕及杏,皆潰,鄭伯如晉,子大叔相,見范獻子,獻子曰,若王室何,對曰,老夫其國家不能恤,敢及王室,抑人亦有言曰,嫠不恤其緯,而憂宗周之隕,為將及焉,今王室實蠢蠢焉,吾小國懼矣,然大國之憂也,吾儕何知焉,吾子其早圖之,詩曰,缾之罄矣,惟罍之恥,王室之不寧,晉之恥也,獻子懼,而與宣子圖之,乃徵會於諸侯,期以明年,秋,八月,大雩,旱也。

冬,十月,癸酉,王子朝用成周之寶珪于河,甲戌,津人得諸河上,陰不佞以溫人南侵,拘得玉者,取其玉,將賣之,則為石,王定而獻之,與之東訾。

楚子為舟師,以略吳疆,沈尹戌曰,此行也,楚必亡邑,不撫民而勞之,吳不動而速之,吳踵楚,而疆埸無備,邑能無亡乎,越大夫胥犴勞王於豫章之汭,越公子倉歸王乘舟,倉及壽夢帥師從王,王及圉陽而還,吳人踵楚,而邊人不備,遂滅巢及鍾離而還,沈尹戌曰,亡郢之始,於此在矣,王壹動而亡二姓之帥,幾如是而不及郢,詩曰,誰生厲階,至今為梗,其王之謂乎。

二十有五年,春,叔孫婼如宋。

夏,叔詣會晉趙鞅,宋樂大心,衛北宮喜,鄭游吉,曹人,邾人,滕人,薛人,小邾人,于黃父。

有鸜鵒來巢。

秋,七月,上辛,大雩,季辛,又雩。

九月,巳月己亥,公孫于齊,次于陽州,齊侯唁公于野井。

冬,十月,戊辰,叔孫婼卒。

十有一月,己亥,宋公佐卒于曲棘,十有二月,齊侯取鄆。

二十五年,春,叔孫婼聘于宋,桐門右師見之,語卑宋大夫,而賤司城氏,昭子告其人曰,右師其亡乎,君子貴其身,而後能及人,是以有禮,今夫子卑其大夫,而賤其宗,是賤其身也,能有禮乎,無禮必亡,宋公享昭子,賦新宮,昭子賦車轄,明日宴,飲酒樂,宋公使昭子右坐,語相泣也,樂祁佐退而告人曰,今茲君與叔孫,其皆死乎。吾聞之,哀樂而樂哀,皆喪心也,心之精爽,是謂魂魄,魂魄去之,何以能久。

季公若之姊為小邾夫人,生宋元夫人,生子,以妻季平子,昭子如宋聘,且逆之,公若從,謂曹氏勿與,魯將逐之,曹氏告公,公告樂祁,樂祁曰,與之如是,魯君必出,政在季氏三世矣,魯君喪政四公矣,無民而能逞其志者,未之有也,國君是以鎮撫其民,詩曰,人之云亡,心之憂矣,魯君失民焉,焉得逞其志,靖以待命猶可,動必憂。

夏,會于黃父,謀王室也,趙簡子令諸侯之大夫輸王粟,具戌人,曰,明年將納王,子大叔見趙簡子,簡子問揖讓周旋之禮焉,對曰,是儀也,非禮也,簡子曰,敢問何謂禮,對曰,吉也聞諸先大夫子產曰,夫禮,天之經也,地之義也,民之行也,天地之經,而民實則之,則天之明,因地之性,生其六氣,用其五行,氣為五味,發為五色,章為五聲,淫則昏亂,民失其性,是故為禮以奉之,為六畜,五牲,三犧,以奉五味,為九文,六采,五章,以奉五色,為九歌,八風,七音,六律,以奉五聲,為君臣上下,以則地義,為夫婦外內,以經二物,為父子,兄弟,姑姊,甥舅,昏媾,姻亞,以象天明,為政事,庸力行務,以從四時,為刑罰,威獄,使民畏忌,以類其震曜殺戮,為溫,慈,惠,和,以效天之生殖,長育,民有好惡喜怒哀樂,生于六氣,是故審則宜類,以制六志,哀有哭泣,樂有歌舞,喜有施舍,怒有戰鬥,喜生於好,怒生於惡,是故審行信令,禍福賞罰,以制死生,生,好物也,死,惡物也,好物樂也,惡物哀也,哀樂不失,乃能協于天地之性,是以長久,簡子曰,甚哉禮之大也,對曰,禮上下之紀,天地之經緯也,民之所以生也,是以先王尚之,故人之能自曲直以赴禮者,謂之成人,大不亦宜乎,簡子曰,鞅也,請終身守此言也,宋樂大心曰,我不輸粟,我於周為客,若之何使客,晉士伯曰,自踐土以來,宋何役之不會,而何盟之不同,曰,同恤王室,子焉得辟之,子奉君命以會大事,而宋背盟,無乃不可乎,右師不敢對,受牒而退,士伯告簡子曰,宋右師必亡,奉君命以使,而欲背盟以干盟主,無不祥大焉。

有鸜鵒來巢,書所無也,師己曰,異哉,吾聞文武之世,童謠有之曰,鴝之鵒之,公出辱之,鴝鵒之羽,公在外野,往饋之馬,鴝鵒跦跦,公在乾侯,徵褰與襦,鴝鵒之巢,遠哉遙遙,稠父喪勞,宋父以驕,鴝鵒鴝鵒,往歌來哭,童謠有是,今鴝鵒來巢,其將及乎。

秋,書再雩,旱甚也,初,季公鳥娶妻於齊,鮑文子,生甲,公鳥死,季公亥,與公思展,與公鳥之臣申夜姑,相其室,及季姒與饔人檀通,而懼,乃使其妾抶己,以示秦遄之妻,曰,公若欲使余,余不可而抶余,又訴於公甫,曰,展與夜姑將要余,秦姬以告公之,公之與公甫告平子,平子拘展於卞,而執夜姑,將殺之,公若泣而哀之,曰,殺是,是殺余也,將為之請,平子使豎勿內,日中不得請,有司逆命,公之使速殺之,故公若怨平子,季郈之雞鬥,季氏介其雞,郈氏為之金距,平子怒,益宮於郈氏,且讓之,故郈昭伯亦怨平子,臧昭伯之從弟會,為讒於臧氏,而逃於季氏,臧氏執旃,平子怒,拘臧氏老,將褅於襄公,萬者二人,其眾萬於季氏,臧孫曰,此之謂不能庸先君之廟,大夫遂怨平子,公若獻弓於公為,且與之出射於外,而謀去季氏,公為告公果,公賁,公果,公賁,使侍人僚柤告公,公寢,將以戈擊之,乃走,公曰,執之,亦無命也,懼而不出,數月不見,公不怒,又使言,公執戈以懼之,乃走,又使言,公曰,非小人之所及也,公果自言,公以告臧孫,臧孫以難,告郈孫,郈孫以可勸,告子家懿伯,懿伯曰,讒人以君徼幸,事若不克,君受其名,不可為也,舍民數世以求克,事不可必也,且政在焉,其難圖也,公退之,辭曰,臣與聞命矣,言若洩,臣不獲死,乃館於公,叔孫昭子如闞,公居於長府,九月,戊戌,伐季氏,殺公之于門,遂入之,平子登臺而請,曰,君不察臣之罪,使有司討臣以干戈,臣請待於沂上以察罪弗許,請囚于費,弗許,請以五乘亡,弗許,子家子曰,君其許之,政自之出久矣,隱民多取食焉,為之徒者眾矣,日入慝作,弗可知也,眾怒不可蓄也,蓄而弗治,將薀,薀蓄民將生心,生心同求將合,君必悔之,弗聽,郈孫曰,必殺之,公使郈孫逆孟懿子,叔孫氏之司馬鬷戾,言於其眾,曰,若之何,莫對,又曰,我家臣也,不敢知國,凡有季氏與無於我孰利,皆曰,無季氏,是無叔孫氏也,鬷戾曰,然則救諸,帥徒以往,陷西北隅以入,公徒釋甲,執冰而踞,遂逐之,孟氏使登西北隅,以望季氏,見叔孫氏之旌,以告,孟氏執郈昭伯,殺之于南門之西,遂伐公徒,子家子曰,諸臣偽劫君者,而負罪以出,君止,意如之事君也,不敢不改,公曰,余不忍也,與臧孫如墓謀,遂行,己亥,公孫于齊,次于陽州,齊侯將唁公于平陰,公先至于野井,齊侯曰,寡人之罪也,使有司待於平陰,為近故也,書曰,公孫于齊次于陽州,齊侯唁公于野井,禮也,將求於人,則先下之,禮之善物也,齊侯曰,自莒疆以西,請致千社,以待君命。寡人將帥敝賦,以從執事,唯命是聽。君之憂,寡人之憂也,公喜,子家子曰,天祿不再,天若胙君,不過周公,以魯足矣,失魯而以千社為臣,誰與之立,且齊君無信,不如早之,晉弗從,臧昭伯率從者將盟。載書曰:戮力壹心,好惡同之,信罪之有無。繾綣從公,無通外內,以公命示子家子,子家子曰,如此,吾不可以盟羈也,不佞,不能與二三子同心,而以為皆有罪,或欲通外內,且欲去君,二三子好亡而惡定,焉可同也,陷君於難,罪孰大焉,通外內而去君,君將速入,弗通何為,而何守焉,乃不與盟,昭子自闞歸,見平子,平子稽顙曰,子若我何,昭子曰,人誰不死,子以逐君成名,子孫不忘,不亦傷乎,將若子何,平子曰,苟使意如得改事君,所謂生死而肉骨也,昭子從公于齊,與公言,子家子命適公館者執之,公與昭子言於幄內,曰,將安眾而納公,公徒將殺昭子,伏諸道,左師展告公,公使昭子自鑄歸,平子有異志,冬,十月,辛酉,昭子齊於其寢,使祝宗祈死,戊辰,卒,左師展將以公乘馬而歸,公徒執之。

壬申,尹文公涉于鞏,焚東訾,弗克。

十一月,宋公元公將為公故如晉,夢大子欒即位於廟,已與平公,服而相之,旦召六卿,公曰,寡人不佞,不能事父兄,以為二三子憂,寡人之罪也,若以群子之靈,獲保首領以歿,唯是楄柎所以藉幹者,請無及先君,仲幾對曰,君若以社稷之故,私降昵宴,群臣弗敢知,若夫宋國之法,死生之度,先君有命矣,群臣以死守之,弗敢失隊,臣之失職,常刑不赦,臣不忍其死,君命祗辱,宋公遂行,己亥,卒于曲棘。

十二月,庚辰,齊侯圍鄆。

初,臧昭伯如晉,臧會竊其寶龜僂句,以卜為信與僭,僭吉,臧氏老將如晉問,會請往,昭伯問家故,盡對,及內子與母弟叔孫,則不對,再三問,不對,歸,及郊,會逆,問,又如初,至,次於外而察之,皆無之,執而戮之,逸奔郈,郈魴假使為賈正焉,計於季氏,臧氏使五人,以戈楯伏諸桐汝之閭,會出逐之,反奔,執諸季氏中門之外,平子怒曰,何故以兵入吾門,拘臧氏老,季臧有惡,及昭伯從公,平子立臧會,會曰,僂句不余欺也。

楚子使薳射城州屈,復茄人焉,城丘皇,遷訾人焉,使熊相禖郭巢,季然郭卷,子大叔聞之,曰,楚王將死矣,使民不安其土,民必憂,憂將及王,弗能久矣。

二十有六年,春,王正月,葬宋元公。

三月,公至自齊,居于鄆。

夏,公圍成。

秋,公會齊侯,莒子,邾子,杞伯,盟于鄟陵,公至自會,居于鄆。

九月,庚申,楚子居卒。

冬,十月,天王入于成周,尹氏,召伯,毛伯,以王子朝奔楚。

二十六年,春,王正月,庚申,齊侯取鄆。

葬宋元公,如先君,禮也。

三月,公至自齊,處于鄆,言魯地也,夏,齊侯將納公,命無受魯貨,申豐從女賈,以幣錦二兩,縛一如瑱,適齊師,謂子猶之人高齮,能貨子猶,為高氏後,粟五千庾,高齮以錦示子猶,子猶欲之,齮曰,魯人買之,百兩一布,以道之不通,先入幣財,子猶受之,言於齊侯曰,群臣不盡力于魯君者,非不能事君也,然據有異焉,宋元公為魯君如晉,卒於曲棘,叔孫昭子求納其君,無疾而死,不知天之棄魯耶,抑魯君有罪於鬼神,故及此也,君若待于曲棘,使群臣從魯君以卜焉,若可,師有濟也,君而繼之,茲無敵矣,若其無成,君無辱焉,齊侯從之,使公子鉏帥師從公成大夫,公孫朝謂平子曰,有都以衛國也,請我受師,許之,請納質,弗許,曰,信女足矣,告於齊師曰,孟氏,魯之敝室也,用成已甚,弗能忍也,請息肩于齊,齊師圍成,成人伐齊師之飲馬于淄者,曰,將以厭眾,魯成備而後告,曰,不勝眾,師及齊師戰于炊鼻,齊子淵捷從洩聲子,射之中楯瓦,繇朐汏,輈匕入者三寸,聲子射其馬,斬鞅,殪,改駕人以為鬷戾也,而助之,子車曰,齊人也,將擊子車,子車射之,殪,其御曰,又之,子車曰,眾可懼也,而不可怒也,子囊帶從野洩,叱之,洩曰,軍無私怒,報乃私也,將亢子,又叱之,亦叱之,冉豎射陳武子,中手,失弓而罵,以告平子曰,有君子白皙,鬒鬚眉,甚口,平子曰,必子彊也,無乃亢諸,對曰,謂之君子,何敢亢之,林雍羞為顏鳴右,下,苑何忌取其耳,顏鳴去之,苑子之御曰,視下顧,苑子刜林雍,斷其足,鑋而乘於他車以歸,顏鳴三入齊師,呼曰,林雍乘。

四月,單子如晉告急,五月,戊午,劉人敗王城之師于尸氏,戊辰,王城人,劉人,戰于施谷,劉師敗績。

秋,盟于剸陵,謀納公也。

七月,己巳,劉子以王出,庚午,次于渠,王城人焚劉,丙子,王宿于褚氏,丁丑,王次于萑谷,庚辰,王入于胥靡,辛巳,王次于滑,晉知躒,趙鞅,帥師納王,使汝寬守關塞。

九月,楚平王卒,令尹子常欲立子西,曰,大子壬弱,其母非適也,王子建實聘之,子西長而好善,立長則順,建善則治,王順國治,可不務乎,子西怒曰,是亂國而惡君王也,國有外援,不可瀆也,王有適嗣,不可亂也,敗親速讎亂嗣不祥,我受其名,賂吾以天下,吾滋不從也,楚國何為,必殺令尹,令尹懼,乃立昭王。

冬,十月,丙申,王起師于滑,辛丑,在郊,遂次于尸,十一月,辛酉,晉師克鞏,召伯盈逐王子朝,王子朝及召氏之族,毛氏得,尹氏固,南宮嚚,奉周之典籍,以奔楚,陰忌奔莒以叛,召伯逆王于尸,及劉子單子盟,遂軍圉澤,次于隄上,癸酉,王入于成周,甲戌,盟于襄宮,晉師成公般戍周而還,十二月,癸未,王入于莊宮,王子朝使告于諸侯曰,昔武王克殷,成王靖四方,康王息民,並建母弟,以蕃屏周,亦曰,吾無專享文武之功,且為後人之迷敗傾覆,而溺入于難,則振救之,至于夷王,王愆于厥身,諸侯莫不並走其望,以祈王身,至于厲王,王心戾虐,萬民弗忍,居王于彘,諸侯釋位,以間王政,宣王有志,而後效官,至于幽王,天不弔周,王昏不若,用愆厥位,攜王奸命,諸侯替之,而建王嗣,用遷郟鄏,則是兄弟之能用力於王室也,至于惠王,天不靖周,生頹禍心,施于叔帶,惠襄辟難,越去王都,則有晉鄭,咸黜不端,以綏定王家,則是兄弟之能率先王之命也,在定王六年,秦人降妖,曰,周其有髭王,亦克能脩其職,諸侯服享,二世共職,王室其有間王位,諸侯不圖,而受其亂災,至于靈王,生而有髭,王甚神聖,無惡於諸侯,靈王景王,克終其世,今王室亂,單旗,劉狄,剝亂天下,壹行不若,謂先王何常之有,唯余心所命,其誰敢請之,帥群不弔之人,以行亂于王室,侵欲無厭,規求無度,貫瀆鬼神,慢棄刑法,倍奸齊盟,傲很威儀,矯誣先王,晉為不道,是攝是贊,思肆其罔極,茲不穀震盪播越,竄在荊蠻,未有攸厎,若我一二兄弟甥舅,獎順天法,無助狡猾,以從先王之命,毋速天罰,赦圖不穀,則所願也,敢盡布其腹心,及先王之經,而諸侯實深圖之,昔先王之命曰,王后無適,則擇立長,年鈞以德,德鈞以卜,王不立愛,公卿無私,古之制也,穆后及大子壽,早天即世,單劉贊私立少,以間先王,亦唯伯仲叔季圖之,閔馬父聞子朝之辭曰,文辭以行禮也,子朝干景之命,遠晉之大,以專其志,無禮甚矣,文辭何為。

齊有彗星,齊侯使禳之,晏子曰,無益也,祇取誣焉,天道不諂不貳,其命若之何,禳之,且天之有彗也,以除穢也,君無穢德,又何禳焉,若德之穢,禳之何損,詩曰,惟此文王,小心翼翼,昭事上帝,聿懷多福。厥德不回。以受方國,君無違德,方國將至,何患於彗,詩曰,我無所監,夏后及商,用亂之故,民卒流亡,若德回亂,民將流亡,祝史之為,無能補也,公說,乃止,齊侯與晏子坐于路寢,公歎曰,美哉室,其誰有此乎,晏子曰,敢問何謂也,公曰,吾以為在德,對曰,如君之言,其陳氏乎,陳氏雖無大德,而有施於民,豆區釜鍾之數,其取之公也薄,其施之民也厚,公厚斂焉,陳氏厚施焉,民歸之矣,詩曰,雖無德與女,式歌且舞,陳氏之施,民歌舞之矣,後世若少惰陳氏而不亡,則國其國也已,公曰,善哉,是可若何,對曰,唯禮可以已之,在禮家施不及國,民不遷晨,不移工,賈不變士,不濫官,不滔大夫,不收公利,公曰,善哉,我不能矣,吾今而後知禮之可以為國也,對曰,禮之可以為國也久矣,與天地並,君令臣共,父慈子孝,兄愛弟敬,夫和妻柔,姑慈婦聽,禮也,君令而不違,臣共而不貳,父慈而教,子孝而箴,兄愛而友,弟敬而順,夫和而義,妻柔而正,姑慈而從,婦聽而婉,禮之善物也,公曰善哉,寡人今而後聞此,禮之上也,對曰,先王所稟於天地,以為其民也,是以先王上之。

二十有七年,春,公如齊,公至自齊,居于鄆。

夏,四月,吳弒其君僚。

楚殺其大夫郤宛。

秋,晉士鞅,宋樂祁犁,衛北宮喜,曹人,邾人,滕人,會于扈。

冬,十月,曹伯午卒。

邾快來奔。

公如齊,公至自齊,居于鄆。

二十七年,春,公如齊,公至自齊,處于鄆,言在外也。

吳子欲因楚喪而伐之,使公子掩餘,公子燭庸,帥師圍潛。

使延州來季子聘于上國,遂聘于晉,以觀諸侯,楚莠尹然,工尹麇,帥師救潛,左司馬沈尹戌,帥都君子與王馬之屬,以濟師,與吳師遇于窮,令尹子常以舟師及沙汭而還,左尹郤宛,工尹壽,帥師至于潛,吳師不能退,吳公子光曰,此時也,弗可失也,告鱄設諸曰,上國有言曰,不索何獲,我王嗣也,吾欲求之,事若克,季子雖至,不吾廢也,鱄設諸曰,王可弒也,母老子弱,是無若我何,光曰,我爾身也,夏,四月,光伏甲於堀室而享王,王使甲坐於道,及其門,門階戶席,皆王親也,夾之以鈹,羞者獻體改服於門外,執羞者坐行而入,執鈹者夾承之,及體以相授也,光偽足疾,入于堀室,鱄設諸寘劍於魚中以進,抽劍刺王,鈹交於胸,遂弒王,闔廬以其子為卿,季子至曰,苟先君無廢祀,民人無廢主,社稷有奉,國家無傾,乃吾君也,吾誰敢怨,哀死事生,以待天命,非我生亂,立者從之,先人之道也,復命哭墓,復位而待,吳公子掩餘奔徐,公子燭庸奔鍾吾,楚師聞吳亂而還。

郤宛直而和,國人說之,鄢將師為右領,與費無極比而惡之,令尹子常賄而信讒,無極譖郤宛焉,謂子常曰,子惡欲飲子酒,又謂子惡,令尹欲飲酒於子氏,子惡曰,我賤人也,不足以辱令尹,令尹將必來辱,為惠巳甚,吾無以酬之,若何,無極曰,令尹好甲兵,子出之,吾擇焉,取五甲五兵,曰,寘諸門,令尹至,必觀之,而從以酬之,及饗日,帷諸門左,無極謂令尹曰,吾幾禍子,子惡將為子不利,甲在門矣,子必無往,且此役也,吳可以得志,子惡取賂焉而還,又誤群帥,使退其師,曰,乘亂不祥,吳乘我喪,我乘其亂,不亦可乎,令尹使視郤氏,則有甲焉,不往,召鄢將師而告之,將師退,遂令攻郤氏,且爇之,子惡聞之,遂自殺也,國人弗爇,令曰,不爇郤氏,與之同罪,或取一編菅焉,或取一秉稈焉,國人投之,遂弗爇也,令尹炮之,盡滅郤氏之族黨,殺陽令終,與其弟完,及佗,與晉,陳,及其子弟,晉陳之族,呼於國曰,鄢氏費氏,自以為王,專禍楚國,弱寡王室,蒙王與令尹,以自利也,令尹盡信之矣,國將如何,令尹病之。

秋,會于扈,令成周,且謀納公也,宋衛皆利納公,固請之,范獻子取貨於季孫,謂司城子梁與北宮貞子曰,季孫未知其罪,而君伐之,請囚請亡,於是乎不獲,君又弗克,而自出也,夫豈無備而能出君乎,季氏之復,天救之也,休公徒之怒,而啟叔孫氏之心,不然,豈其伐人而說甲執冰以游,叔孫氏懼禍之濫,而自同於季氏,天之道也,魯君守齊,三年而無成,季氏甚得其民,淮夷與之,有十年之備,有齊楚之援,有天之贊,有民之助,有堅守之心,有列國之權,而弗敢宣也,事君如在國,故鞅以為難,二子皆圖國者也,而欲納魯君,鞅之願也,請從二子以圍魯,無成,死之,二子懼,皆辭,乃辭小國,而以難復,孟懿子陽虎伐鄆,鄆人將戰,子家子曰,天命不慆久矣,使君亡者,必此眾也,天既禍之,而自福也,不亦難乎,猶有鬼神,此必敗也,嗚呼,為無望也夫,其死於此乎,公使子家子如晉,公徒敗于且知。

楚郤宛之難,國言未已,進胙者莫不謗令尹,沈尹戌言於子常曰,夫左尹與中廄尹,莫知其罪,而子殺之,以興謗讟,至于今不已,戌也惑之,仁者殺人以掩謗,猶弗為也,今吾子殺人以興謗,而弗圖,不亦異乎,夫無極,楚之讒人也,民莫不知,去朝吳,出蔡侯宋,喪太子建,殺連尹奢,屏王之耳目,使不聰明,不然,平王之溫惠共儉,有過成莊,無不及焉,所以不獲諸侯,邇無及也,今又殺三不辜,以興大謗,幾及子矣,子而不圖,將焉用之,夫鄢將師矯子之命,以滅三族,國之良也,而不愆位,吳新有君,疆埸日駭,楚國若有大事,子其危哉,知者除讒以自安也,今子愛讒以自危也,甚矣其惑也,子常曰,是瓦之罪,敢不良圖,九月,己未,子常殺費無極與鄢將師,盡滅其族,以說于國,謗言乃止。

冬,公如齊,齊侯請饗之子家子曰,朝夕立於其朝,又何饗焉,其飲酒也,乃飲酒,使宰獻而請安,子仲之子曰重,為齊侯夫人,曰,請使重見,子家子乃以君出,十二月,晉籍秦致諸侯之戍于周,魯人辭以難。

二十有八年,春,王三月,葬曹悼公。

公如晉,次于乾侯。

夏,四月,丙戌,鄭伯寧卒。

六月,葬鄭定公。

秋,七月,癸巳,滕子寧卒,冬,葬滕悼公。

二十八年,春,公如晉,將如乾侯,子家子曰,有求於人,而即其安,人孰矜之,其造於竟,弗聽,使請逆於晉,晉人曰,天禍魯國,君淹恤在外,君亦不使一個,辱在寡人,而即安於甥舅,其亦使逆君,使公復于竟而後逆之,晉祁勝與鄔臧通室,祁盈將執之,訪於司馬叔游,叔游曰,鄭書有之,惡直醜正,實蕃有徒,無道立矣,子懼不免,詩曰,民之多辟,無自立辟,姑已若何,盈曰,祁氏私有討,國何有焉,遂執之,祁勝賂荀躒,荀躒為之言於晉侯,晉侯執祁盈,祁盈之臣曰,鈞將皆死,憖使吾君聞勝與臧之死也以為快,乃殺之,夏,六月,晉殺祁盈及楊食我,食我,祁盈之黨也,而助亂,故殺之,遂滅祁氏,羊舌氏,初,叔向欲娶於申公巫臣氏,其母欲娶其黨,叔向曰,吾母多而庶鮮,吾懲舅氏矣,其母曰,子靈之妻,殺三夫,一君,一子,而亡一國,兩卿矣,可無懲乎,吾聞之,甚美必有甚惡,是鄭穆少妃,姚子之子,子貉之妹也,子貉早死無後,而天鍾美於是,將必以是,大有敗也,昔有仍氏生女,黰黑,而甚美,光可以鑑,名曰玄妻,樂正后夔取之,生伯封,實有豕心,貪惏無饜,忿纇無期,謂之封豕,有窮后羿滅之,夔是以不祀,且三代之亡,共子之廢,皆是物也。女何以為哉?夫有尤物,足以移人,苟非德義,則必有禍,叔向懼,不敢取,平公強使取之,生伯石,伯石始生,子容之母走謁諸姑,曰,長叔姒生男,姑視之,及堂,聞其聲而還,曰,是豺狼之聲也,狼子野心,非是,莫喪羊舌氏矣,遂弗視。

秋,晉韓宣子卒,魏獻子為政,分祁氏之田,以為七縣,分羊舌氏之田,以為三縣,司馬彌牟為鄔大夫,賈辛為祁大夫,司馬烏為平陵大夫,魏戊為梗陽大夫,知徐吾為塗水大夫,韓固為馬首大夫,孟丙為盂大夫,樂霄為銅鞮大夫,趙朝為平陽大夫,僚安為楊氏大夫,謂賈辛,司馬烏,為有力於王室,故舉之,謂知徐吾,趙朝,韓固,魏戊,餘子之不失職,能守業者也,其四人者,皆受縣而後見於魏子,以賢舉也,魏子謂成鱄,吾與戊也縣,人其以我為黨乎,對曰,何也,戊之為人也,遠不忘君,近不偪同,居利思義,在約思純,有守心而無淫行,雖與之縣不亦可乎,昔武王克商,光有天下,其兄弟之國者,十有五人,姬姓之國者,四十人,皆舉親也,夫舉無他,唯善所在,親疏一也,詩曰,唯此文王,帝度其心,莫其德音,其德克明,克明克類,克長克君,王此大國,克順克比,比于文王,其德靡悔,既受帝祉,施于孫子,心能制義曰度,德正應和曰莫,照臨四方曰明,勤施無私曰類,教誨不倦曰長,賞慶刑威曰君,慈和遍服曰順,擇善而從之曰比,經緯天地曰文,九德不愆,作事無悔,故襲天祿,子孫賴之,主之舉也,近文德矣,所及其遠哉,賈辛將適其縣,見於魏子,魏子曰,辛來,昔叔向適鄭,鬷蔑惡欲觀叔向,從使之收器者,而往立於堂下,一言而善,叔向將飲酒,聞之曰,必鬷明也,下執其手,以上曰,昔賈大夫惡,娶妻而美,三年不言不笑,御以如皋,射雉獲之,其妻始笑而言,賈大夫曰,才之不可以已,我不能射,女遂不言不笑夫,今子少不颺,子若無言,吾幾失子矣,言不可以已也如是,遂如故知,今女有力於王室,吾是以舉女,行乎敬之哉,毋墮乃力,仲尼聞魏子之舉也,以為義,曰,近不失親,遠不失舉,可謂義矣,又聞其命賈辛也,以為忠,詩曰,永言配命,自求多福,忠也,魏子之舉也,義其命也,忠其長有後於晉國乎。

冬,梗陽人有獄,魏戊不能斷,以獄上其大宗,賂以女樂,魏子將受之,魏戊謂閻沒女寬曰,主以不賄,聞於諸侯,若受梗陽,人賄莫甚焉,吾子必諫,皆許諾,退朝待於庭,饋入召之,比置三歎,既食使坐,魏子曰,吾聞諸伯叔諺曰,唯食忘憂,吾子置食之間,三歎何也,同辭而對曰,或賜二小人酒,不夕食,饋之始至,恐其不足,是以歎,中置自咎曰,豈將軍食之,而有不足,是以再歎,及饋之畢,願以小人之腹,為君子之心,屬厭而已,獻子辭梗陽人。

二十有九年,春,公至自乾侯,居于鄆,齊侯使高張來唁公。

公如晉,次于乾侯。

夏,四月,庚子,叔詣卒。

秋,七月。

冬,十月,鄆潰。

二十九年,春,公至自乾侯,處于鄆,齊侯使高張來唁公,稱主君,子家子曰,齊卑君矣,君祇辱焉,公如乾侯。

三月,己卯,京師殺召伯盈,尹氏固,及原伯魯之子,尹固之復也,有婦人遇之周郊,尤之曰,處則勸人為禍,行則數日而反,是夫也,其過三歲乎,夏,五月,庚寅,王子趙車入于鄻以叛,陰不佞敗之。

平子每歲賈馬,具從者之衣屨,而歸之于乾侯,公執歸馬者賣之,乃不歸馬,衛侯來獻其乘馬,曰啟服,塹而死,公將為之櫝,子家子曰,從者病矣,請以食之,乃以幃裹之,公賜公衍羔裘,使獻龍輔於齊侯,遂入羔裘,齊侯喜,與之陽穀,公衍,公為,之生也,其母偕出,公衍先生,公為之母曰,相與偕出,請相與偕告,三日,公為生,其母先以告,公為為兄,公私喜於陽穀,而思於魯,曰,務人為此,禍也,且後生而為兄,其誣也久矣,乃黜之而以公,衍為大子。

秋,龍見于絳郊,魏獻子問於蔡墨曰,吾聞之,蟲莫知於龍,以其不生得也,謂之知,信乎,對曰,人實不知,非龍實知,古者畜龍,故國有豢龍氏,有御龍氏,獻子曰,是二氏者,吾亦聞之,而知其故,是何謂也,對曰,昔有飂叔安有裔子,曰董父實,甚好龍,能求其耆欲以飲食之,龍多歸之,乃擾畜龍以服事帝舜,帝賜之姓,曰董氏,曰豢龍,封諸鬷川,鬷夷氏其後也,故帝舜氏世有畜龍,及有夏孔甲,擾于有帝,帝賜之乘龍,河漢各二,各有雌雄,孔甲不能食,而未獲豢龍氏,有陶唐氏既衰,其後有劉累學擾龍于豢龍氏,以事孔甲,能飲食之,夏后嘉之,賜氏曰御龍,以更豕韋之後,龍一雌死,潛醢以食,夏后,夏后饗之,既而使求之,懼而遷于魯縣,范氏其後也,獻子曰,今何故無之,對曰,夫物物有其官,官脩其方,朝夕思之,一日失職,則死及之,失官不食,官宿其業,其物乃至,若泯棄之,物乃坻伏,鬱湮不育,故有五行之官,是謂五官,實列受氏姓,封為上公,祀為貴神,社稷五祀,是尊是奉,木正曰句芒,火正曰祝融,金正曰蓐收,水正曰玄冥,土正曰后土,龍,水物也,水官棄矣,故龍不生得,不然,周易有之,在乾之姤曰,潛龍勿用,其同人曰,見龍在田,其大有曰,飛龍在天,其夬曰,亢龍有悔,其坤曰,見群龍無首,吉,坤之剝曰,龍戰于野,若不朝夕見,誰能物之,獻子曰,社稷五祀,誰氏之五官也,對曰,少皞氏有四叔,曰重,曰該,曰脩,曰熙,實能金木及水,使重為句芒,該為蓐收,脩及熙為玄冥,世不失職,遂濟窮桑,此其三祀也,顓頊氏有子曰犁,為祝融,共工氏有子曰句龍,為后土,此其二祀也,后土為社,稷,田正也,有烈山氏之子曰柱,為稷,自夏以上祀之,周棄亦為稷,自商以來祀之。

冬,晉趙鞅,荀寅,帥師城汝濱,遂賦晉國一鼓鐵,以鑄刑鼎,著范宣子所謂刑書焉,仲尼曰,晉其亡乎,失其度矣,夫晉國將守唐叔之所受法度,以經緯其民,卿大夫以序守之,民是以能尊其貴,貴是以能守其業,貴賤不愆,所謂度也,文公是以作執秩之官,為被廬之法,以為盟主,今棄是度也,而為刑鼎,民在鼎矣,何以尊貴,貴何業之守,貴賤無序,何以為國,且夫宣子之刑,夷之蒐也,晉國之亂制也,若之何以為法,蔡史墨曰,范氏,中行氏,其亡乎,中行寅為下卿,而干上令,擅作刑器,以為國法,是法姦也,又加范氏,焉易之,亡也,其及趙氏,趙孟與焉,然不得已,若德可以免。

三十年,春,王正月,公在乾侯。

夏,六月,庚辰,晉侯去疾卒。

秋,八月,葬晉頃公。

冬,十有二月,吳滅徐,徐子章羽奔楚。

三十年,春,王正月,公在乾侯,不先書鄆與乾侯,非公,且徵過也。

夏,六月,晉頃公卒,秋,八月,葬鄭游吉弔,且送葬,魏獻子使士景伯詰之,曰,悼公之喪,子西弔,子蟜送葬,今吾子無貳,何故,對曰,諸侯所以歸晉君,禮也,禮也者,小事大,大字小之謂,事大在共其時命,字小在恤其所無,以敝邑居大國之間,共其職貢,與其備御,不虞之患,豈忘共命,先王之制,諸侯之喪,士弔,大夫送葬,唯嘉好聘享,三軍之事,於是乎使卿,晉之喪事,敝邑之間,先君有所,助執紼矣,若其不間,雖士大夫,有所不獲數矣,大國之惠,亦慶其加,而不討其乏,明厎其情,取備而已,以為禮也,靈王之喪,我先君簡公在楚,我先大夫印段實往,敝邑之少卿也,王吏不討,恤所無也,今大夫曰,女盍從舊,舊有豐有省,不知所從,從其豐,則寡君幼弱,是以不共,從其省,則吉在此矣,唯大夫圖之,晉人不能詰。

吳子使徐人執掩餘,使鍾吾人執燭庸,二公子奔楚,楚子大封而定其徙,使監馬尹大心逆吳公子,使居養,莠尹然,左司馬沈尹,戌城之,取於城父與胡田以與之,將以害吳也,子西諫曰,吳光新得國而親其民,視民如子,辛苦同之,將用之也,若好吳邊疆,使柔服焉,猶懼其至,吾又疆其讎,以重怒之,無乃不可乎,吳,周之冑裔也,而棄在海濱,不與姬通,今而始大,比于諸華,光又甚文,將自同於先王,不知天將以為虐乎,使翦喪吳國,而封大異姓乎,其抑亦將卒以祚吳乎,其終不遠矣,我盍姑億吾鬼神,而寧吾族姓,以待其歸,將焉用自播揚焉,王弗聽,吳子怒,冬,十二月,吳子執鍾吳子,遂伐徐,防山以水之,己卯,滅徐,徐子章禹斷其髮,攜其夫人,以逆吳子,吳子唁而送之,使其邇臣從之,遂奔楚,楚沈尹戌帥師救徐,弗及,遂城夷,使徐子處之,吳子問於伍員曰,初而言伐楚,余知其可也,而恐其使余往也,又惡人之有余之功也,今余將自有之矣,伐楚何如,對曰,楚執政眾而乖,莫適任患,若為三師之肄焉,一師至,彼必皆出,彼出則歸,彼歸則出,楚必道敝,亟肄以罷之,多方以誤之,既罷而後以三軍繼之,必大克之,闔廬從之,楚於是乎始病。

三十有一年,春,王正月,公在乾侯。

季孫意如會晉荀躒於適歷。

夏,四月,丁巳,薛伯穀卒。

晉侯使荀躒唁公于乾侯。

秋,葬薛獻公。

冬,黑肱以濫來奔。

十有二月,辛亥,朔,日有食之。

三十一年,春,王正月,公在乾侯,言不能外內也。

晉侯將以師納公,范獻子曰,若召季孫而不來,則信不臣矣,然後伐之,若何,晉人召季孫,獻子使私焉,曰,子必來,我受其無咎,季孫意如會晉荀躒于適歷,荀躒曰,寡君使躒謂吾子,何故出君,有君不事,周有常刑,子其圖之,季孫練冠麻衣跣行,伏而對曰,事君,臣之所不得也,敢逃刑命,君若以臣為有罪,請囚于費,以待君之察也,亦唯君,若以先臣之故,不絕季氏,而賜之死,若弗殺弗亡,君之惠也,死且不朽,若得從君而歸,則固臣之願也,敢有異心,夏,四月,季孫從知伯如乾侯,子家子曰,君與之歸,一慚之不忍,而終身慚乎,公曰,諾,眾曰,在一言矣,君必逐之,荀躒以晉侯之命唁公,且曰,寡君使躒以君命討於意如,意如不敢逃死,君其入也,公曰,君惠顧先君之好,施及亡人,將使歸,糞除宗祧,以事君,則不能見夫人已,所能見夫人者,有如河,荀躒掩耳而走,曰,寡君其罪之恐,敢與知魯國之難,臣請復於寡君,退而謂季孫,君怒未怠,子姑歸祭,子家子曰,君以一乘入于魯師,季孫必與君歸,公欲從之,眾從者脅公不得歸。

薛伯穀卒,同盟故書。

秋,吳人侵楚,伐夷,侵潛六,楚沈尹戌帥師救潛,吳師還,楚師遷潛於南岡而還,吳師圍弦,左司馬戌,右司馬稽,帥師救弦,及豫章,吳師還,始用子胥之謀也。

冬,邾黑肱以濫來奔,賤而書名,重地故也,君子曰,名之不可不慎也如是夫,有所有名,而不如其已,以地叛,雖賤必書,地以名其人,終為不義,弗可滅已,是故君子動則思禮,行則思義,不為利回,不為義疚,或求名而不得,或欲蓋而名章,懲不義也,齊豹為衛司寇,守嗣大夫,作而不義,其書為盜,邾庶其,莒牟夷,邾黑肱,以土地出,求食而已,不求其名,賤而必書,此二物者,所以懲肆而去貪也,若艱難其身,以險危大人,而有名章徹,攻難之士,將奔走之,若竊邑叛君,以徼大利,而無名,貪冒之民,將寘力焉,是以春秋書齊豹曰,盜,三叛人名,以懲不義,數惡無禮,其善志也,故曰,春秋之稱,微而顯,婉而辨,上之人能使昭明,善人勸焉,淫人懼焉,是以君子貴之。

十二月,辛亥,朔,日有食之,是夜也,趙簡子夢童子臝而轉以歌,旦占諸史墨曰,吾夢如是,今而日食,何也,對曰,六年,及此月也,吳其入郢乎,終亦弗克,入郢必以庚辰,日月在辰尾,庚午之日,日始有謫,火勝金,故弗克。

三十有二年,春,王正月,公在乾侯取闞。

夏,吳伐越。

秋,七月。

冬,仲孫何忌會晉韓不信,齊高張,宋仲幾,衛世叔申,鄭國參,曹人,莒人,薛人,杞人,小邾人,城成周,十有二月,己未,公薨于乾侯。

三十二年,春,王正月,公在乾侯,言不能外內,又不能用其人也。

夏,吳伐越,始用師於越也,史墨曰,不及四十年,越其有吳乎,越得歲而吳伐之,必受其凶。

秋,八月,王使富辛與石張如晉,請城成周,天子曰,天降禍于周,俾我兄弟,並有亂心,以為伯父憂,我一二親昵甥舅,不皇啟處,於今十年,勤戍五年,余一人無日忘之,閔閔焉如農夫之望歲,懼以待時,伯父若肆大惠,復二文之業,弛周室之憂,徼文武之福,以固盟主,宣昭令名,則余一人有大願矣,昔成王合諸侯,城成周,以為東都,崇文德焉,今我欲徼福假靈于成王,脩成周之城,俾戍人無勤,諸侯用寧,蝥賊遠屏,晉之力也,其委諸伯父,使伯父實重圖之,俾我一人,無徵怨于百姓,而伯父有榮,施先王庸之,范獻子謂魏獻子曰,與其戍周,不如城之,天子實云,雖有後事,晉勿與知可也,從王命以紓諸侯,晉國無憂,是之不務,而又焉從事,魏獻子曰善,使伯音對,曰,天子有命,敢不奉承,以奔告於諸侯,遲速衰序,於是焉在,冬,十一月,晉魏舒,韓不信,如京師,合諸侯之大夫于狄泉,尋盟,且令城成周,魏子南面,衛彪徯曰,魏子必有大咎,干位以令大事,非其任也,詩曰,敬天之怒,不敢戲豫,敬天之渝,不敢馳驅,況敢干位,以作大事乎,己丑,士彌牟營成周,計丈數,揣高卑,度厚薄,仞溝洫,物土方,議遠邇,量事期,計徒庸,慮財用,書餱糧,以令役於諸侯,屬役賦丈,書以授帥,而效諸劉子,韓簡子臨之,以為成命。

十二月,公疾,遍賜大夫,大夫不受,賜子家子雙琥,一環,一璧,輕服,受之,大夫皆受其賜,己未,公薨,子家子反賜於府人曰,吾不敢逆君命也,大夫皆反其賜,書曰,公薨于乾侯,言失其所也,趙簡子問於史墨曰,季氏出其君,而民服焉,諸侯與之,君死於外,而莫之或罪也,對曰,物生有兩,有三有五,有陪貳,故天有三辰,地有五行,體有左右,各有妃耦,王有公,諸侯有卿,皆有貳也,天生季氏,以貳魯侯,為日久矣,民之服焉,不亦宜乎,魯君世從其失,季氏世脩其勤,民忘君矣,雖死於外,其誰矜之,社稷無常奉,君臣無常位,自古以然,故詩曰,高岸為谷,深谷為陵,三后之姓,於今為庶,王所知也,在易卦,雷乘乾曰,大壯,天之道也,昔成季友,桓之季也,文姜之愛子也,始震而卜,卜人謁之曰,生有嘉聞,其名曰友,為公室輔,及生如卜人之言,有文在其手,曰友,遂以名之,既而有大功於魯,受費以為上卿,至於文子,武子。世增其業,不費舊績,魯文公薨,而東門遂殺適立庶。魯君於是乎失國,政在季氏,於此君也,四公矣,民不知君,何以得國,是以為君,慎器與名,不可以假人。


\end{pinyinscope}