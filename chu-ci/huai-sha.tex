\article{懷沙}

\begin{pinyinscope}
滔滔孟夏兮,草木莽莽。傷懷永哀兮,汨徂南土。眴兮杳杳,孔靜幽默。鬱結紆軫兮,離愍而長鞠。撫情效志兮,冤屈而自抑。刓方以為圜兮,常度未替。易初本迪兮,君子所鄙。章畫志墨兮,前圖未改。內厚質正兮,大人所盛。巧倕不斲兮,孰察其撥正。玄文處幽兮,矇瞍謂之不章;離婁微睇兮,瞽以為無明。變白以為黑兮,倒上以為下。鳳皇在笯兮,雞鶩翔舞。同糅玉石兮,一概而相量。夫惟黨人鄙固兮,羌不知余之所臧。任重載盛兮,陷滯而不濟。懷瑾握瑜兮,窮不知所示。邑犬之群吠兮,吠所怪也。非俊疑傑兮,固庸態也。文質疏內兮,眾不知余之異採。材樸委積兮,莫知余之所有。重仁襲義兮,謹厚以為豐。重華不可牾兮,孰知余之從容!古固有不並兮,豈知其何故?湯禹久遠兮,邈而不可慕。懲違改忿兮,抑心而自彊。離湣而不遷兮,願志之有像。進路北次兮,日昧昧其將暮。舒憂娛哀兮,限之以大故。

亂曰:浩浩沅﹑湘,分流汨兮。脩路幽蔽,道遠忽兮。曾吟恆悲,永歎慨兮。世既莫吾知,人心不可謂兮。懷質抱情,獨無正兮。伯樂既沒,驥焉程兮?萬民之生,各有所錯兮。定心廣志,余何畏懼兮?曾傷爰哀,永歎喟兮。世溷濁莫吾知,人心不可謂兮。知死不可讓,願勿愛兮。明告君子,吾將以為類兮。


\end{pinyinscope}