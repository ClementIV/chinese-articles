\article{離騷經}

\begin{pinyinscope}

\begin{shici}

帝高陽之苗裔兮,\\
朕皇考曰伯庸。\\
攝提貞於孟陬兮,\\
惟庚寅吾以降。\\
皇覽揆余初度兮,\\
肇錫余以嘉名。\\
名余曰正則兮,\\
字余曰靈均。\\
紛吾既有此內美兮,\\
又重之以脩能。\\
扈江離與闢芷兮,\\
紉秋蘭以為佩。\\
汩余若將不及兮,\\
恐年歲之不吾與。\\
朝搴阰之木蘭兮,\\
夕攬洲之宿莽。\\
日月忽其不淹兮,\\
春與秋其代序。\\
惟草木之零落兮,\\
恐美人之遲暮。\\
不撫壯而棄穢兮,\\
何不改此度?\\
乘騏驥以馳騁兮,\\
來吾道夫先路。\\
昔三后之純粹兮,\\
固衆芳之所在。\\
雜申椒與菌桂兮,\\
豈維紉夫蕙茝?\\
彼堯舜之耿介兮,\\
既遵道而得路。\\
何桀紂之猖披兮,\\
夫唯捷徑以窘步。\\
惟夫黨人之偷樂兮,\\
路幽昧以險隘。\\
豈余身之憚殃兮,\\
恐皇輿之敗績。\\
忽奔走以先後兮,\\
及前王之踵武。\\
荃不察余之中情兮,\\
反信讒而齌怒。\\
余固知謇謇之為患兮,\\
忍而不能舍也。\\
指九天以為正兮,\\
夫唯靈脩之故也。\\
曰黃昏以為期兮,\\
羌中道而改路。\\
初既與余成言兮,\\
後悔遁而有他。\\
余既不難夫離別兮,\\
傷靈脩之數化。\\
余既滋蘭之九畹兮,\\
又樹蕙之百畝。\\
畦留夷與揭車兮,\\
雜杜衡與芳芷。\\
冀枝葉之峻茂兮,\\
願竢時乎吾將刈。\\
雖萎絕其亦何傷兮,\\
哀衆芳之蕪穢。\\
衆皆競進以貪婪兮,\\
憑不猒乎求索。\\
羌內恕己以量人兮,\\
各興心而嫉妒。\\
忽馳騖以追逐兮,\\
非余心之所急。\\
老冉冉其將至兮,\\
恐脩名之不立。\\
朝飲木蘭之墜露兮,\\
夕餐秋菊之落英。\\
苟余情其信姱以練要兮,\\
長顑頷亦何傷?\\
擥木根以結茝兮,\\
貫薜荔之落蕊。\\
矯菌桂以紉蕙兮,\\
索胡繩之纚纚。\\
謇吾法夫前脩兮,\\
非世俗之所服。\\
雖不周於今之人兮,\\
願依彭咸之遺則。\\
長太息以掩涕兮,\\
哀民生之多艱。\\
余雖好脩姱以鞿羈兮,\\
謇朝誶而夕替。\\
既替余以蕙纕兮,\\
又申之㠯攬茝。\\
亦余心之所善兮,\\
雖九死其猶未悔。\\
怨靈脩之浩蕩兮,\\
終不察夫民心。\\
衆女嫉余之蛾眉兮,\\
謠諑謂余以善淫。\\
固時俗之工巧兮,\\
偭規矩而改錯。\\
背繩墨以追曲兮,\\
競周容以為度。\\
忳鬱邑余侘傺兮,\\
吾獨窮困乎此時也。\\
寧溘死以流亡兮,\\
余不忍為此態也。\\
鷙鳥之不羣兮,\\
自前世而固然。\\
何方圜之能周兮,\\
夫孰異道而相安。\\
屈心而抑志兮,\\
忍尤而攘詬。\\
伏清白以死直兮,\\
固前聖之所厚。\\
悔相道之不察兮,\\
延佇乎吾將反。\\
回朕車以復路兮,\\
及行迷之未遠。\\
步余馬於蘭臯兮,\\
馳椒丘且焉止息。\\
進不入以離尤兮,\\
退將復脩吾初服。\\
製芰荷以為衣兮,\\
集芙蓉以為裳。\\
不吾知其亦已兮,\\
苟余情其信芳。\\
高余冠之岌岌兮,\\
長余佩之陸離。\\
芳與澤其雜糅兮,\\
唯昭質其猶未虧。\\
忽反顧以遊目兮,\\
將往觀乎四荒。\\
佩繽紛其繁飾兮,\\
芳菲菲其彌章。\\
民生各有所樂兮,\\
余獨好脩以為常。\\
雖體解吾猶未變兮,\\
豈余心之可懲。\\
女嬃之嬋媛兮,\\
申申其詈予。\\
曰鯀婞直以亡身兮,\\
終然殀乎羽之野。\\
汝何博謇而好脩兮,\\
紛獨有此姱節。\\
薋菉葹以盈室兮,\\
判獨離而不服。\\
衆不可戶說兮,\\
孰云察余之中情。\\
世並舉而好朋兮,\\
夫何煢獨而不予聽。\\
依前聖以節中兮,\\
喟憑心而歷玆。\\
濟沅湘以南征兮,\\
就重華而敶詞:\\
啟《九辯》與《九歌》兮,\\
夏康娛以自縱。\\
不顧難以圖後兮,\\
五子用失乎家巷。\\
羿淫遊以佚畋兮,\\
又好射夫封狐。\\
固亂流其鮮終兮,\\
浞又貪夫厥家。\\
澆身被服強圉兮,\\
縱欲而不忍。\\
日康娛而自忘兮,\\
厥首用夫顛隕。\\
夏桀之常違兮,\\
乃遂焉而逢殃。\\
后辛之菹醢兮,\\
殷宗用而不長。\\
湯禹儼而祗敬兮,\\
周論道而莫差。\\
舉賢而授能兮,\\
循繩墨而不頗。\\
皇天無私阿兮,\\
覽民德焉錯輔。\\
夫維聖哲以茂行兮,\\
苟得用此下土。\\
瞻前而顧後兮,\\
相觀民之計極。\\
夫孰非義而可用兮,\\
孰非善而可服。\\
阽余身而危死兮,\\
覽余初其猶未悔。\\
不量鑿而正枘兮,\\
固前脩以菹醢。\\
曾歔欷余鬱邑兮,\\
哀朕時之不當。\\
攬茹蕙以掩涕兮,\\
霑余襟之浪浪。\\
跪敷衽以陳辭兮,\\
耿吾既得此中正;\\
駟玉虯以乘鷖兮,\\
溘埃風余上徵。\\
朝發軔於蒼梧兮,\\
夕余至乎縣圃;\\
欲少留此靈瑣兮,\\
日忽忽其將暮。\\
吾令羲和弭節兮,\\
望崦嵫而勿迫。\\
路曼曼其脩遠兮,\\
吾將上下而求索。\\
飲余馬於咸池兮,\\
總余轡乎扶桑。\\
折若木以拂日兮,\\
聊逍遙以相羊。\\
前望舒使先驅兮,\\
後飛廉使奔屬。\\
鸞皇為余先戒兮,\\
雷師告余以未具。\\
吾令鳳鳥飛騰兮,\\
繼之以日夜。\\
飄風屯其相離兮,\\
帥雲霓而來御。\\
紛緫緫其離合兮,\\
斑陸離其上下。\\
吾令帝閽開關兮,\\
倚閶闔而望予。\\
時曖曖其將罷兮,\\
結幽蘭而延佇。\\
世溷濁而不分兮,\\
好蔽美而嫉妒。\\
朝吾將濟於白水兮,\\
登閬風而緤馬。\\
忽反顧以流涕兮,\\
哀高丘之無女。\\
溘吾遊此春宮兮,\\
折瓊枝以繼佩。\\
及榮華之未落兮,\\
相下女之可詒。\\
吾令豐隆椉雲兮,\\
求宓妃之所在。\\
解佩纕以結言兮,\\
吾令蹇脩以為理。\\
紛緫緫其離合兮,\\
忽緯繣其難遷。\\
夕歸次於窮石兮,\\
朝濯髮乎洧盤。\\
保厥美以驕傲兮,\\
日康娛以淫遊。\\
雖信美而無禮兮,\\
來違棄而改求。\\
覽相觀於四極兮,\\
周流乎天余乃下。\\
望瑤臺之偃蹇兮,\\
見有娀之佚女。\\
吾令鴆為媒兮,\\
鴆告余以不好。\\
雄鳩之鳴逝兮,\\
余猶惡其佻巧。\\
心猶豫而狐疑兮,\\
欲自適而不可。\\
鳳皇既受詒兮,\\
恐高辛之先我。\\
欲遠集而無所止兮,\\
聊浮遊以逍遙。\\
及少康之未家兮,\\
留有虞之二姚。\\
理弱而媒拙兮,\\
恐導言之不固。\\
世溷濁而嫉賢兮,\\
好蔽美而稱惡。\\
閨中既以邃遠兮,\\
哲王又不寤。\\
懷朕情而不發兮,\\
余焉能忍與此終古。\\
索藑茅以筳篿兮,\\
命靈氛為余占之。\\
曰兩美其必合兮,\\
孰信脩而慕之?\\
思九州之博大兮,\\
豈唯是其有女?\\
曰勉遠逝而無狐疑兮,\\
孰求美而釋女?\\
何所獨無芳草兮,\\
爾何懷乎故宇?\\
世幽昧以昡曜兮,\\
孰云察余之善惡。\\
民好惡其不同兮,\\
惟此黨人其獨異。\\
戶服艾以盈要兮,\\
謂幽蘭其不可佩。\\
覽察草木其猶未得兮,\\
豈珵美之能當?\\
蘇糞壤㠯充幃兮,\\
謂申椒其不芳。\\
欲從靈氛之吉占兮,\\
心猶豫而狐疑。\\
巫咸將夕降兮,\\
懷椒糈而要之。\\
百神翳其備降兮,\\
九疑繽其並迎。\\
皇剡剡其揚靈兮,\\
告余以吉故。\\
日勉陞降以上下兮,\\
求榘矱之所同。\\
湯禹嚴而求合兮,\\
摯咎繇而能調。\\
苟中情其好脩兮,\\
又何必用夫行媒。\\
說操築於傳巖兮,\\
武丁用而不疑。\\
呂望之鼓刀兮,\\
遭周文而得舉。\\
甯戚之謳歌兮,\\
齊桓聞以該輔。\\
及年歲之未宴兮,\\
時亦猶其未央。\\
恐鵜鴃之先鳴兮,\\
使夫百草為之不芳。\\
何瓊佩之偃蹇兮,\\
衆薆然而蔽之。\\
惟此黨人之不諒兮,\\
恐嫉妒而折之。\\
時繽紛其變易兮,\\
又何可以淹留。\\
蘭芷變而不芳兮,\\
荃蕙化而為茅。\\
何昔日之芳草兮,\\
今直為此蕭艾也。\\
豈其有他故兮,\\
莫好脩之害也。\\
余以蘭為何恃兮,\\
羌無實而容長。\\
委厥美以從俗兮,\\
苟得列乎衆芳。\\
椒專佞以慢慆兮,\\
樧又欲充夫佩幃。\\
既乾進而務入兮,\\
又何芳之能祗。\\
固時俗之流從兮,\\
又孰能無變化。\\
覽椒蘭其若玆兮,\\
又況揭車與江離。\\
惟玆佩之可貴兮,\\
委厥美而歷玆。\\
芳菲菲而難虧兮,\\
芬至今猶未沫。\\
和調度以自娛兮,\\
聊浮游而求女。\\
及余飾之方壯兮,\\
周流觀乎上下。\\
靈氛既告余以吉占兮,\\
歷吉日乎吾將行。\\
折瓊枝以為羞兮,\\
精瓊爢以為粻。\\
為余駕飛龍兮,\\
雜瑤象以為車。\\
何離心之可同兮,\\
吾將遠逝以自疏。\\
邅吾道夫崑崙兮,\\
路脩遠以周流。\\
揚雲霓之晻藹兮,\\
鳴玉鸞之啾啾。\\
朝發軔於天津兮,\\
夕余至乎西極。\\
鳳皇翼其承旂兮,\\
高翱翔之翼翼。\\
忽吾行此流沙兮,\\
遵赤水而容與。\\
麾蛟龍使梁津兮,\\
詔西皇使涉予。\\
路脩遠以多艱兮,\\
騰衆車使徑待。\\
路不周以左轉兮,\\
指西海以為期。\\
屯余車其千乘兮,\\
齊玉軑而並馳。\\
駕八龍之婉婉兮,\\
載雲旗之委蛇。\\
抑志而弭節兮,\\
神高馳之邈邈。\\
奏《九歌》而舞《韶》兮,\\
聊假日以媮樂。\\
陟陞皇之赫戲兮,\\
忽臨睨夫舊鄉。\\
僕夫悲余馬懷兮,\\
蜷局顧而不行。\\
亂曰:已矣哉,\\
國無人莫我知兮,\\
又何懷乎故都?\\
既莫足與為美政兮,\\
吾將從彭咸之所居。\\

\end{shici}

\end{pinyinscope}