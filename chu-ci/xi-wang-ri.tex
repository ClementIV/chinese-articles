\article{昔往日}

\begin{pinyinscope}
惜往日之曾信兮,受命詔以昭詩。奉先功以照下兮,明法度之嫌疑。國富強而法立兮,屬貞臣而日娭。秘密事之載心兮,雖過失猶弗治。心純庬而不泄兮,遭讒人而嫉之。君含怒而待臣兮,不清澈其然否。蔽晦君之聰明兮,虛惑誤又以欺。弗參驗以考實兮,遠遷臣而弗思。信讒諛之溷濁兮,盛氣志而過之。何貞臣之無罪兮,被離謗而見尤。慚光景之誠信兮,身幽隱而備之。臨沅湘之玄淵兮,遂自忍而沈流。卒沒身而絕名兮,惜壅君之不昭。君無度而弗察兮,使芳草為藪幽。焉舒情而抽信兮,恬死亡而不聊。獨鄣壅而蔽隱兮,使貞臣為無由。聞百裡之為虜兮,伊尹烹於庖廚。呂望屠於朝歌兮,甯戚歌而飯牛。不逢湯武與桓繆兮,世孰雲而知之。吳信讒而弗味兮,子胥死而後憂。介子忠而立枯兮,文君寤而追求。封介山而為之禁兮,報大德之優游。思久故之親身兮,因縞素而哭之。或忠信而死節兮,或訑謾而不疑。弗省察而按實兮,聽讒人之虛辭。芳與澤其雜糅兮,孰申旦而別之?何芳草之早殀兮,微霜降而下戒。諒聰不明而蔽壅兮,使讒諛而日得。自前世之嫉賢兮,謂蕙若其不可佩。妒佳冶之芬芳兮,嫫母姣而自好。雖有西施之美容兮,讒妒入以自代。願陳情以白行兮,得罪過之不意。情冤見之日明兮,如列宿之錯置。乘騏驥而馳騁兮,無轡銜而自載;乘氾泭以下流兮,無舟楫而自備。背法度而心治兮,闢與此其無異。寧溘死而流亡兮,恐禍殃之有再。不畢辭而赴淵兮,惜壅君之不識。


\end{pinyinscope}