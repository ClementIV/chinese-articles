\article{怨世}

\begin{pinyinscope}
世沈淖而難論兮,俗岒峨而嵾嵯。清泠泠而殲滅兮,溷湛湛而日多。梟鴞既以成群兮,玄鶴弭翼而屏移。蓬艾親入御於床笫兮,馬蘭踸踔而日加。棄捐葯芷與杜衡兮,余柰世之不知芳何?何周道之平易兮,然蕪穢而險戲。高陽無故而委塵兮,唐虞點灼而毀議。誰使正其真是兮,雖有八師而不可為。皇天保其高兮,後土持其久。服清白以逍遙兮,偏與乎玄英異色。西施媞媞而不得見兮,嫫母勃屑而日侍。桂蠹不知所淹留兮,蓼蟲不知徙乎葵菜。處湣湣之濁世兮,今安所達乎吾志。意有所載而遠逝兮,固非眾人之所識。驥躊躇於弊輂兮,遇孫陽而得代。呂望窮困而不聊生兮,遭周文而舒志。甯戚飯牛而商歌兮,桓公聞而弗置。路室女之方桑兮,孔子過之以自侍。吾獨乖剌而無當兮,心悼怵而耄思。思比干之恲恲兮,哀子胥之慎事。悲楚人之和氏兮,獻寶玉以為石。遇厲武之不察兮,羌兩足以畢斮。小人之居勢兮,視忠正之何若?改前聖之法度兮,喜囁嚅而妄作。親讒諛而疏賢聖兮,訟謂閭娵為醜惡。愉近習而蔽遠兮,孰知察其黑白?卒不得效其心容兮,安眇眇而無所歸薄。專精爽以自明兮,晦冥冥而壅蔽。年既已過太半兮,然埳軻而留滯。欲高飛而遠集兮,恐離罔而滅敗。獨冤抑而無極兮,傷精神而壽夭。皇天既不純命兮,余生終無所依。願自沈於江流兮,絕橫流而徑逝。寧為江海之泥塗兮,安能久見此濁世?


\end{pinyinscope}