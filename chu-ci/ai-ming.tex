\article{哀命}

\begin{pinyinscope}

\begin{shici}

哀時命之不合兮,\\
傷楚國之多憂。\\
內懷情之潔白兮,\\
遭亂世而離尤。\\
惡耿介之直行兮,\\
世溷濁而不知。\\
何君臣之相失兮,\\
上沅湘而分離。\\
測汨羅之湘水兮,\\
知時固而不反。\\
傷離散之交亂兮,\\
遂側身而既遠。\\
處玄舍之幽門兮,\\
穴巖石而窟伏。\\
從水蛟而為徙兮,\\
與神龍乎休息。\\
何山石之嶄巖兮,\\
靈魂屈而偃蹇。\\
含素水而蒙深兮,\\
日眇眇而既遠。\\
哀形體之離解兮,\\
神罔兩而無舍。\\
惟椒蘭之不反兮,\\
魂迷惑而不知路。\\
願無過之設行兮,\\
雖滅沒之自樂。\\
痛楚國之流亡兮,\\
哀靈修之過到。\\
固時俗之溷濁兮,\\
志瞀迷而不知路。\\
念私門之正匠兮,\\
遙涉江而遠去。\\
念女嬃之嬋媛兮,\\
涕泣流乎於悒。\\
我決死而不生兮,\\
雖重追吾何及。\\
戲疾瀨之素水兮,\\
望高山之蹇產。\\
哀高丘之赤岸兮,\\
遂沒身而不反。\\

\end{shici}

\end{pinyinscope}