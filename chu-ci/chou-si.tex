\article{抽思}

\begin{pinyinscope}
心鬱鬱之憂思兮,獨永歎乎增傷。思蹇產之不釋兮,曼遭夜之方長。悲秋風之動容兮,何回極之浮浮。數惟蓀之多怒兮,傷余心之懮懮。願搖起而橫奔兮,覽民尤以自鎮。結微情以陳詞兮,矯以遺夫美人。昔君與我誠言兮,曰黃昏以為期。羌中道而回畔兮,反既有此他志。憍吾以其美好兮,覽余以其脩姱。與余言而不信兮,蓋為余而造怒。願承閒而自察兮,心震悼而不敢;悲夷猶而冀進兮,心怛傷之憺憺。玆歷情以陳辭兮,蓀詳聾而不聞。固切人之不媚兮,眾果以我為患。初吾所陳之耿著兮,豈至今其庸亡?何毒藥之謇謇兮?願蓀美之可完。望三五以為像兮,指彭咸以為儀。夫何極而不至兮,故遠聞而難虧。善不由外來兮,名不可以虛作。孰無施而有報兮,孰不實而有穫?

少歌曰:與美人抽怨兮,並日夜而無正。憍吾以其美好兮,敖朕辭而不聽。

倡曰:有鳥自南兮,來集漢北。好姱佳麗兮,牉獨處此異域。既惸獨而不群兮,又無良媒在其側。道卓遠而日忘兮,願自申而不得。望北山而流涕兮,臨流水而太息。望孟夏之短夜兮,何晦明之若歲!惟郢路之遼遠兮,魂一夕而九逝。曾不知路之曲直兮,南指月與列星。願徑逝而未得兮,魂識路之營營。何靈魂之信直兮,人之心不與吾心同!理弱而媒不通兮,尚不知余之從容。

亂曰:長瀨湍流,泝江潭兮。狂顧南行,聊以娛心兮。軫石崴嵬,蹇吾願兮。超回志度,行隱進兮。低佪夷猶,宿北姑兮。煩冤瞀容,實沛徂兮。愁歎苦神,靈遙思兮。路遠處幽,又無行媒兮。道思作頌,聊以自救兮。憂心不遂,斯言誰告兮。


\end{pinyinscope}