\article{涉江}

\begin{pinyinscope}
余幼好此奇服兮,年既老而不衰。帶長鋏之陸離兮,冠切雲之崔嵬。被明月兮珮寶璐。世溷濁而莫余知兮,吾方高馳而不顧。駕青虯兮驂白螭,吾與重華遊兮瑤之圃。登崑崙兮食玉英,與天地兮同壽,與日月兮同光。哀南夷之莫吾知兮,旦余濟乎江湘。乘鄂渚而反顧兮,欸秋冬之緒風。步余馬兮山皋,邸余車兮方林。乘舲船余上沅兮,齊吳榜以擊汰。船容與而不進兮,淹回水而疑滯。朝發枉陼兮,夕宿辰陽。苟余心其端直兮,雖僻遠之何傷。入漵浦余儃佪兮,迷不知吾所如。深林杳以冥冥兮,猿狖之所居。山峻高以蔽日兮,下幽晦以多雨。霰雪紛其無垠兮,雲霏霏而承宇。哀吾生之無樂兮,幽獨處乎山中。吾不能變心而從俗兮,固將愁苦而終窮。接輿髡首兮,桑扈臝行。忠不必用兮,賢不必以。伍子逢殃兮,比干菹醢。與前世而皆然兮,吾又何怨乎今之人!余將董道而不豫兮,固將重昏而終身!

亂曰:鸞鳥鳳皇,日以遠兮。燕雀烏鵲,巢堂壇兮。露申辛夷,死林薄兮。腥臊並御,芳不得薄兮。陰陽易位,時不當兮。懷信侘傺,忽乎吾將行兮!


\end{pinyinscope}