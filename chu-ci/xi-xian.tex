\article{惜賢}

\begin{pinyinscope}
覽屈氏之離騷兮,心哀哀而怫鬱。聲嗷嗷以寂寥兮,顧僕夫之憔悴。撥諂諛而匡邪兮,切淟涊之流俗。盪渨涹之姦咎兮,夷蠢蠢之溷濁。懷芬香而挾蕙兮,佩江蘺之菲菲。握申椒與杜若兮,冠浮雲之峨峨。登長陵而四望兮,覽芷圃之蠡蠡。游蘭皋與蕙林兮,睨玉石之嵾嵯。揚精華以眩燿兮,芳鬱渥而純美。結桂樹之旖旎兮,紉荃蕙與辛夷。芳若茲而不御兮,捐林薄而菀死。

驅子僑之奔走兮,申徒狄之赴淵。若由夷之純美兮,介子推之隱山。晉申生之離殃兮,荊和氏之泣血。吳申胥之抉眼兮,王子比干之橫廢。欲卑身而下體兮,心隱惻而不置。方圜殊而不合兮,鉤繩用而異態。欲俟時於須臾兮,日陰曀其將暮。時遲遲其日進兮,年忽忽而日度。妄周容而入世兮,內距閉而不開。俟時風之清激兮,愈氛霧其如塺。進雄鳩之耿耿兮,讒介介而蔽之。默順風以偃仰兮,尚由由而進之。心懭悢以冤結兮,情舛錯以曼憂。搴薜荔於山野兮,採撚支於中洲。望高丘而歎涕兮,悲吸吸而長懷。孰契契而委棟兮,日晻晻而下頹。

歎曰:江湘油油長流汩兮,挑揄揚汰盪迅疾兮,憂心展轉愁怫鬱兮,冤結未舒長隱忿兮,丁時逢殃可奈何兮,勞心悁悁涕滂沱兮。


\end{pinyinscope}