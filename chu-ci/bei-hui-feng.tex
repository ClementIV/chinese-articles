\article{悲回風}

\begin{pinyinscope}
悲回風之搖蕙兮,心冤結而內傷。物有微而隕性兮,聲有隱而先倡。夫何彭咸之造思兮,暨志介而不忘!萬變其情豈可蓋兮,孰虛偽之可長!鳥獸鳴以號群兮,草苴比而不芳。魚葺鱗以自別兮,蛟龍隱其文章。故荼薺不同畝兮,蘭茞幽而獨芳。惟佳人之永都兮,更統世而自貺。眇遠志之所及兮,憐浮雲之相羊。介眇志之所惑兮,竊賦詩之所明。惟佳人之獨懷兮,折若椒以自處。曾歔欷之嗟嗟兮,獨隱伏而思慮。涕泣交而凄凄兮,思不眠以至曙。終長夜之曼曼兮,掩此哀而不去。寤從容以周流兮,聊逍遙以自恃。傷太息之愍憐兮,氣於邑而不可止。糾思心以為纕兮,編愁苦以為膺。折若木以蔽光兮,隨飄風之所仍。存髣彿而不見兮,心踴躍其若湯。撫珮衽以案志兮,超惘惘而遂行。歲曶曶其若頹兮,時亦冉冉而將至。薠蘅槁而節離兮,芳以歇而不比。憐思心之不可懲兮,證此言之不可聊。寧溘死而流亡兮,不忍為此之常愁。孤子吟而抆淚兮,放子出而不還。孰能思而不隱兮,照彭咸之所聞。登石巒以遠望兮,路眇眇之默默。入景響之無應兮,聞省想而不可得。愁鬱鬱之無快兮,居戚戚而不可解。心鞿羈而不形兮,氣繚轉而自締。穆眇眇之無垠兮,莽芒芒之無儀。聲有隱而相感兮,物有純而不可為。藐蔓蔓之不可量兮,縹綿綿之不可紆。愁悄悄之常悲兮,翩冥冥之不可娛。淩大波而流風兮,託彭咸之所居。上高巖之峭岸兮,處雌蜺之標顛。據青冥而攄虹兮,遂儵忽而捫天。吸湛露之浮源兮,漱凝霜之雰雰。依風穴以自息兮,忽傾寤以嬋媛。馮崑崙以瞰霧兮,隱岷山以清江。憚涌湍之磕磕兮,聽波聲之洶洶。紛容容之無經兮,罔芒芒之無紀。軋洋洋之無從兮,馳委移之焉止。漂翻翻其上下兮,翼遙遙其左右。氾潏潏其前後兮,伴張弛之信期。觀炎氣之相仍兮,窺煙液之所積。悲霜雪之俱下兮,聽潮水之相擊。借光景以往來兮,施黃棘之枉策。求介子之所存兮,見伯夷之放跡。心調度而弗去兮,刻著志之無適。曰:吾怨往昔之所冀兮,悼來者之悐悐。浮江淮而入海兮,從子胥而自適。望大河之洲渚兮,悲申徒之抗跡。驟諫君而不聽兮,重任石之何益。心絓結而不解兮,思蹇產而不釋。


\end{pinyinscope}