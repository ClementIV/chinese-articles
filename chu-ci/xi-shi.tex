\article{惜誓}

\begin{pinyinscope}
惜余年老而日衰兮,歲忽忽而不反。登蒼天而高舉兮,歷眾山而日遠。觀江河之紆曲兮,離四海之霑濡。攀北極而一息兮,吸沆瀣以充虛。飛朱鳥使先驅兮,駕太一之象輿。蒼龍蚴虯於左驂兮,白虎騁而為右騑。建日月以為蓋兮,載玉女於後車。馳騖於杳冥之中兮,休息虖崑崙之墟。樂窮極而不厭兮,願從容虖神明。涉丹水而駝騁兮,右大夏之遺風。

黃鵠之一舉兮,知山川之紆曲。再舉兮,睹天地之圜方。臨中國之眾人兮,託回飆乎尚羊。乃至少原之野兮,赤鬆、王喬皆在旁。二子擁瑟而調均兮,余因稱乎清商。澹然而自樂兮,吸眾氣而翱翔。念我長生而久僊兮,不如反余之故鄉。

黃鵠後時而寄處兮,鴟梟群而制之。神龍失水而陸居兮,為螻蟻之所裁。夫黃鵠神龍猶如此兮,況賢者之逢亂世哉。壽冉冉而日衰兮,固儃回而不息。俗流從而不止兮,眾枉聚而矯直。或偷合而苟進兮,或隱居而深藏。苦稱量之不審兮,同權概而就衡。或推迻而苟容兮,或直言之諤諤。傷誠是之不察兮,並紉茅絲以為索。方世俗之幽昏兮,眩白黑之美惡。放山淵之龜玉兮,相與貴夫礫石。梅伯數諫而至醢兮,來革順志而用國。悲仁人之盡節兮,反為小人之所賊。比干忠諫而剖心兮,箕子被髮而佯狂。水背流而源竭兮,木去根而不長。非重軀以慮難兮,惜傷身之無功。

已矣哉!獨不見夫鸞鳳之高翔兮,乃集大皇之野。循四極而回周兮,見盛德而後下。彼聖人之神德兮,遠濁世而自藏。使麒麟可得羈而係兮,又何以異虖犬羊?


\end{pinyinscope}