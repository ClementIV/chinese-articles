\article{卜居}

\begin{pinyinscope}
屈原既放,三年不得復見。竭知盡忠,而蔽鄣於讒。心煩慮亂,不知所從。往見太卜鄭詹尹,曰:「余有所疑,願因先生決之。」詹尹乃端策拂龜,曰:「君將何以教之?」

屈原曰:「吾寧悃悃款款,朴以忠乎?將送往勞來,斯無窮乎?寧誅鋤草茅,以力耕乎?將游大人,以成名乎?寧正言不諱,以危身乎?將從俗富貴,以媮生乎?寧超然高舉,以保真乎?將哫訾栗斯,喔咿儒兒,以事婦人乎?寧廉潔正直,以自清乎?將突梯滑稽,如脂如韋,以潔楹乎?寧昂昂若千里之駒乎?將氾氾若水中之鳧乎,與波上下,偷以全吾軀乎?寧與騏驥亢軛乎?將隨駑馬之跡乎?寧與黃鵠比翼乎?將與雞鶩爭食乎?此孰吉孰凶?何去何從?世溷濁而不清,蟬翼為重,千鈞為輕;黃鐘毀棄,瓦釜雷鳴;讒人高張,賢士無名。吁嗟默默兮,誰知吾之廉貞!」

詹尹乃釋策而謝,曰:「夫尺有所短,寸有所長,物有所不足,智有所不明,數有所不逮,神有所不通。用君之心,行君之意,龜策誠不能知事。」


\end{pinyinscope}