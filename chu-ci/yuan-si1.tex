\article{怨思}

\begin{pinyinscope}
惟鬱鬱之憂毒兮,志坎壈而不違。身憔悴而考旦兮,日黃昏而長悲。閔空宇之孤子兮,哀枯楊之冤鶵。孤雌吟於高墉兮,鳴鳩棲於桑榆。玄蝯失於潛林兮,獨偏棄而遠放。征夫勞於周行兮,處婦憤而長望。申誠信而罔違兮,情素潔於紐帛。光明齊於日月兮,文采燿於玉石。傷壓次而不發兮,思沈抑而不揚。芳懿懿而終敗兮,名靡散而不彰。

背玉門以奔騖兮,蹇離尤而乾詬。若龍逢之沈首兮,王子比干之逢醢。念社稷之幾危兮,反為讎而見怨。思國家之離沮兮,躬獲愆而結難。若青蠅之偽質兮,晉驪姬之反情。恐登階之逢殆兮,故退伏於末庭。孽臣之號咷兮,本朝蕪而不治。犯顏色而觸諫兮,反蒙辜而被疑。菀蘼蕪與菌若兮,漸槁芋本於洿瀆。淹芳芷於腐井兮,棄雞駭於筐簏。執棠谿以刜蓬兮,秉乾將以割肉。筐澤瀉以豹鞹兮,破荊和以繼築。時溷濁猶未清兮,世殽亂猶未察。欲容與以俟時兮,懼年歲之既晏。顧屈節以從流兮,心鞏鞏而不夷。寧浮沅而馳騁兮,下江湘以邅迴。

歎曰:山中檻檻余傷懷兮,征夫皇皇其孰依兮,經營原野杳冥冥兮,乘騏騁驥舒吾情兮,歸骸舊邦莫誰語兮,長辭遠逝乘湘去兮。


\end{pinyinscope}