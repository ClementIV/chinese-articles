\article{謬諫}

\begin{pinyinscope}
怨靈修之浩蕩兮,夫何執操之不固?悲太山之為隍兮,孰江河之可涸?願承閒而效志兮,恐犯忌而乾諱。卒撫情以寂寞兮,然怊悵而自悲。玉與石其同匱兮,貫魚眼與珠璣。駑駿雜而不分兮,服罷牛而驂驥。年滔滔而自遠兮,壽冉冉而愈衰。心悇憛而煩冤兮,蹇超搖而無冀。固時俗之工巧兮,滅規矩而改錯。卻騏驥而不乘兮,策駑駘而取路。當世豈無騏驥兮,誠無王良之善馭。見執轡者非其人兮,故駒跳而遠去。不量鑿而正枘兮,恐矩矱之不同。不論世而高舉兮,恐操行之不調。弧弓弛而不張兮,孰雲知其所至?無傾危之患難兮,焉知賢士之所死?俗推佞而進富兮,節行張而不著。賢良蔽而不群兮,朋曹比而黨譽。邪說飾而多曲兮,正法弧而不公。直士隱而避匿兮,讒諛登乎明堂。棄彭咸之娛樂兮,滅巧倕之繩墨。菎蕗雜於黀蒸兮,機蓬矢以射革。駕蹇驢而無策兮,又何路之能極?以直鍼而為釣兮,又何魚之能得?伯牙之絕弦兮,無鍾子期而聽之。和抱璞而泣血兮,安得良工而剖之?同音者相和兮,同類者相似。飛鳥號其群兮,鹿鳴求其友。故叩宮而宮應兮,彈角而角動。虎嘯而谷風至兮,龍舉而景雲往。音聲之相和兮,言物類之相感也。夫方圜之異形兮,勢不可以相錯。列子隱身而窮處兮,世莫可以寄託。眾鳥皆有行列兮,鳳獨翔翔而無所薄。經濁世而不得志兮,願側身巖穴而自託。欲闔口而無言兮,嘗被君之厚德。獨便悁而懷毒兮,愁鬱鬱之焉極?念三年之積思兮,願壹見而陳辭。不及君而騁說兮,世孰可為明之?身寢疾而日愁兮,情沈抑而不揚。眾人莫可與論道兮,悲精神之不通。

亂曰:鸞皇孔鳳日以遠兮,畜鳧駕鵝。雞鶩滿堂壇兮,鼁黽游乎華池。要褭奔亡兮,騰駕橐駝。鉛刀進御兮,遙棄太阿。拔搴玄芝兮,列樹芋荷。橘柚萎枯兮,苦李旖旎。甂甌登於明堂兮,周鼎潛乎深淵。自古而固然兮,吾又何怨乎今之人。


\end{pinyinscope}