\article{遠遊}

\begin{pinyinscope}
悲余性之不可改兮,屢懲艾而不迻。服覺皓以殊俗兮,貌揭揭以巍巍。譬若王僑之乘雲兮,載赤霄而淩太清。欲與天地參壽兮,與日月而比榮。登崑崙而北首兮,悉靈圉而來謁。選鬼神於太陰兮,登閶闔於玄闕。回朕車俾西引兮,褰虹旗於玉門。馳六龍於三危兮,朝西靈於九濱。結余軫於西山兮,橫飛谷以南征。絕都廣以直指兮,歷祝融於朱冥。枉玉衡於炎火兮,委兩館於咸唐。貫澒濛以東朅兮,維六龍於扶桑。

周流覽於四海兮,志升降以高馳。徵九神於回極兮,建虹採以招指。駕鸞鳳以上遊兮,從玄鶴與鷦明。孔鳥飛而送迎兮,騰群鶴於瑤光。排帝宮與羅囿兮,升縣圃以眩滅。結瓊枝以雜佩兮,立長庚以繼日。淩驚雷以軼駭電兮,綴鬼谷於北辰。鞭風伯使先驅兮,囚靈玄於虞淵。泝高風以低佪兮,覽周流於朔方。就顓頊而敶辭兮,考玄冥於空桑。旋車逝於崇山兮,奏虞舜於蒼梧。濟楊舟於會稽兮,就申胥於五湖。見南郢之流風兮,殞余躬於沅湘。望舊邦之黯黮兮,時溷濁其猶未央。懷蘭茞之芬芳兮,妒被離而折之。張絳帷以襜襜兮,風邑邑而蔽之。日暾暾其西舍兮,陽焱焱而復顧。聊假日以須臾兮,何騷騷而自故。

歎曰:譬彼蛟龍乘雲浮兮,汎淫澒溶紛若霧兮,潺湲轇轕雷動電發馺高舉兮,升虛淩冥沛濁浮清入帝宮兮,搖翹奮羽馳風騁雨游無窮兮。


\end{pinyinscope}