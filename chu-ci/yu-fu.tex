\article{漁父}

\begin{pinyinscope}
屈原既放,遊於江潭,行吟澤畔,顏色憔悴,形容枯槁。漁父見而問之曰:「子非三閭大夫與?何故至於斯!」屈原曰:「舉世皆濁我獨清,眾人皆醉我獨醒,是以見放!」漁父曰:「聖人不凝滯於物,而能與世推移。世人皆濁,何不淈其泥而揚其波?眾人皆醉,何不餔其糟而歠其釃?何故深思高舉,自令放為?」屈原曰:「吾聞之,新沐者必彈冠,新浴者必振衣;安能以身之察察,受物之汶汶者乎!寧赴湘流,葬於江魚之腹中。安能以皓皓之白,而蒙世俗之塵埃乎!」漁父莞爾而笑,鼓枻而去,乃歌曰:「滄浪之水清兮,可以濯吾纓。滄浪之水濁兮,可以濯吾足。」遂去不復與言。


\end{pinyinscope}