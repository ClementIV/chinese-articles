\article{哀時命}

\begin{pinyinscope}
哀時命之不及古人兮,夫何予生之不遘時!往者不可扳援兮,徠者不可與期。志憾恨而不逞兮,杼中情而屬詩。夜炯炯而不寐兮,懷隱憂而歷茲。心鬱鬱而無告兮,眾孰可與深謀!欿愁悴而委惰兮,老冉冉而逮之。居處愁以隱約兮,志沈抑而不揚。道壅塞而不通兮,江河廣而無梁。願至崑崙之懸圃兮,採鍾山之玉英。攬瑤木之橝枝兮,望閬風之板桐。弱水汨其為難兮,路中斷而不通。勢不能凌波以徑度兮,又無羽翼而高翔。然隱憫而不達兮,獨徙倚而彷徉。悵惝罔以永思兮,心紆軫宗而增傷。倚躊躇以淹留兮,日饑饉而絕糧。廓抱景而獨倚兮,超永思乎故鄉。廓落寂而無友兮,誰可與玩此遺芳?白日晼晼其將入兮,哀余壽之弗將。車既弊而馬罷兮,蹇邅徊而不能行。身既不容於濁世兮,不知進退之宜當。

冠崔嵬而切雲兮,劍淋離而從橫。衣攝葉以儲與兮,左袪掛於榑桑;右衽拂於不周兮,六合不足以肆行。上同鑿枘於伏戲兮,下合矩矱於虞唐。願尊節而式高兮,志猶卑夫禹、湯。雖知困其不改操兮,終不以邪枉害方。世並舉而好朋兮,壹鬥斛而相量。眾比周以肩迫兮,賢者遠而隱藏。為鳳皇作鶉籠兮,雖翕翅其不容。靈皇其不寤知兮,焉陳詞而效忠。俗嫉妒而蔽賢兮,孰知余之從容?願舒志而抽馮兮,庸詎知其吉凶?璋珪雜於甑窐兮,隴廉與孟娵同宮。舉世以為恆俗兮,固將愁苦而終窮。幽獨轉而不寐兮,惟煩懣而盈匈。魂眇眇而馳騁兮,心煩冤之忡忡。志欿憾而不憺兮,路幽昧而甚難。

塊獨守此曲隅兮,然欿切而永歎。愁修夜而宛轉兮,氣涫沸沛其若波。握剞劂而不用兮,操規矩而無所施。騁騏驥於中庭兮,焉能極夫遠道?置猿狖於櫺檻兮,夫何以責其捷巧?駟跛虌而上山兮,吾固知其不能陞。釋管晏而任臧獲兮,何權衡之能稱?菎簬雜於黀蒸兮,機蓬矢以射革。負檐荷以丈尺兮,欲伸要而不可得。外迫脅於機臂兮,上牽聯於矰隿。肩傾側而不容兮,固愜腹而不得息。務光自投於深淵兮,不獲世之塵垢。孰魁摧之可久兮,願退身而窮處。鑿山楹而為室兮,下被衣於水渚。霧露濛濛其晨降兮,雲依斐而承宇。虹霓紛其朝霞兮,夕淫淫而淋雨。怊茫茫而無歸兮,悵遠望此曠野。下垂釣於谿谷兮,上要求於僊者。與赤鬆而結友兮,比王僑而為耦。使梟楊先導兮,白虎為之前後。浮雲霧而入冥兮,騎白鹿而容與。

魂聇聇以寄獨兮,汨徂往而不歸。處卓卓而日遠兮,志浩蕩而傷懷。鸞鳳翔於蒼雲兮,故矰繳而不能加。蛟龍潛於旋淵兮,身不掛於罔羅。知貪餌而近死兮,不如下游乎清波。寧幽隱以遠禍兮,孰侵辱之可為。子胥死而成義兮,屈原沈於汨羅。雖體解其不變兮,豈忠信之可化。志怦怦而內直兮,履繩墨而不頗。執權衡而無私兮,稱輕重而不差。摡塵垢之枉攘兮,除穢累而反真。形體白而質素兮,中皎潔而淑清。時厭飫而不用兮,且隱伏而遠身。聊竄端而匿跡兮,嗼寂默而無聲。獨便悁而煩毒兮,焉發憤而筊抒。時曖曖其將罷兮,遂悶歎而無名。伯夷死於首陽兮,卒夭隱而不榮。太公不遇文王兮,身至死而不得逞。懷瑤象而佩瓊兮,願陳列而無正。生天墜之若過兮,忽爛漫而無成。邪氣襲余之形體兮,疾憯怛而萌生。願壹見陽春之白日兮,恐不終乎永年。


\end{pinyinscope}