\article{遠逝}

\begin{pinyinscope}
志隱隱而鬱怫兮,愁獨哀而冤結。腸紛紜以繚轉兮,涕漸漸其若屑。情慨慨而長懷兮,信上皇而質正。合五嶽與八靈兮,訊九鬿與六神。指列宿以白情兮,訴五帝以置辭。北斗為我折中兮,太一為余聽之。雲服陰陽之正道兮,御後土之中和。佩蒼龍之蚴虯兮,帶隱虹之逶蛇。曳彗星之皓旰兮,撫朱爵與鵔鸃。游清靈之颯戾兮,服雲衣之披披。杖玉策與朱旗兮,垂明月之玄珠。舉霓旌之墆翳兮,建黃纁之總旄。躬純粹而罔愆兮,承皇考之妙儀。

惜往事之不合兮,橫汨羅而下厲。乘隆波而南渡兮,逐江湘之順流。赴陽侯之潢洋兮,下石瀨而登洲。陸魁堆以蔽視兮,雲冥冥而闇前。山峻高以無垠兮,遂曾閎而迫身。雪雰雰而薄木兮,雲霏霏而隕集。阜隘狹而幽險兮,石嵾嵯以翳日。悲故鄉而發忿兮,去余邦之彌久。背龍門而入河兮,登大墳而望夏首。橫舟航而濟湘兮,耳聊啾而戃慌。波淫淫而周流兮,鴻溶溢而滔蕩。路曼曼其無端兮,周容容而無識。引日月以指極兮,少須臾而釋思。水波遠以冥冥兮,眇不睹其東西。順風波以南北兮,霧宵晦以紛紛。日杳杳以西頹兮,路長遠而窘迫。欲酌醴以娛憂兮,蹇騷騷而不釋。

歎曰:飄風蓬龍埃坲坲兮,屮木搖落時槁悴兮,遭傾遇禍不可救兮,長吟永欷涕究究兮,舒情敶詩冀以自免兮,頹流下隕身日遠兮。


\end{pinyinscope}