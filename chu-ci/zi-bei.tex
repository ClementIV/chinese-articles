\article{自悲}

\begin{pinyinscope}
居愁懃其誰告兮,獨永思而憂悲。內自省而不俟兮,操愈堅而不衰。隱三年而無決兮,歲忽忽其若頹。憐余身不足以卒意兮,冀一見而復歸。哀人事之不幸兮,屬天命而委之咸池。身被疾而不閒兮,心沸熱其若湯。冰炭不可以相並兮,吾固知乎命之不長。哀獨苦死之無樂兮,惜余年之未央。悲不反余之所居兮,恨離予之故鄉。鳥獸驚而失群兮,猶高飛而哀鳴。狐死必首丘兮,夫人孰能不反其真情?故人疏而日忘兮,新人近而俞好。莫能行於杳冥兮,孰能施於無報?苦眾人之皆然兮,乘回風而遠游。淩恆山其若陋兮,聊愉娛以忘憂。悲虛言之無實兮,苦眾口之鑠金。遇故鄉而一顧兮,泣歔欷而霑衿。厭白玉以為面兮,懷琬琰以為心。邪氣入而感內兮,施玉色而外淫。何青雲之流瀾兮,微霜降之矇矇。徐風至而徘徊兮,疾風過之湯湯。聞南藩樂而欲往兮,至會稽而且止。見韓眾而宿之兮,問天道之所在?借浮雲以送予兮,載雌霓而為旌。駕青龍以馳騖兮,班衍衍之冥冥。忽容容其安之兮,超慌忽其焉如?苦眾人之難信兮,願離群而遠舉。登巒山而遠望兮,好桂樹之冬榮。觀天火之炎煬兮,聽大壑之波聲。引八維以自道兮,含沆瀣以長生。居不樂以時思兮,食草木之秋實。飲菌若之朝露兮,構桂木而為室。雜橘柚以為囿兮,列新夷與椒楨。鵾鶴孤而夜號兮,哀居者之誠貞。


\end{pinyinscope}