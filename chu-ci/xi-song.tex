\article{惜誦}

\begin{pinyinscope}
惜誦以致愍兮,發憤以抒情。所作忠而言之兮,指蒼天以為正。令五帝以㭊中兮,戒六神與嚮服。俾山川以備御兮,命咎繇使聽直。竭忠誠以事君兮,反離群而贅肬。忘儇媚以背眾兮,待明君其知之。言與行其可跡兮,情與貌其不變。故相臣莫若君兮,所以證之不遠。吾誼先君而後身兮,羌眾人之所仇。專惟君而無他兮,又眾兆之所讎。壹心而不豫兮,羌不可保也。疾親君而無他兮,有招禍之道也。思君其莫我忠兮,忽忘身之賤貧。事君而不貳兮,迷不知寵之門。忠何罪以遇罰兮,亦非余心之所志。行不群以巔越兮,又眾兆之所咍。紛逢尤以離謗兮,謇不可釋。情沈抑而不達兮,又蔽而莫之白。心鬱邑余侘傺兮,又莫察余之中情。固煩言不可結詒兮,願陳志而無路。退靜默而莫余知兮,進號呼又莫吾聞。申侘傺之煩惑兮,中悶瞀之忳忳。昔余夢登天兮,魂中道而無杭。吾使厲神占之兮,曰:「有志極而無旁。」「終危獨以離異兮,曰:「君可思而不可恃。故眾口其鑠金兮,初若是而逢殆。懲於羹者而吹虀兮,何不變此志也?欲釋階而登天兮,猶有曩之態也。眾駭遽以離心兮,又何以為此伴也?同極而異路兮,又何以為此援也?晉申生之孝子兮,父信讒而不好。行婞直而不豫兮,鯀功用而不就。」吾聞作忠以造怨兮,忽謂之過言。九折臂而成醫兮,吾至今而知其信然。矰弋機而在上兮,罻羅張而在下。設張闢以娛君兮,願側身而無所。欲儃佪以乾傺兮,恐重患而離尤。欲高飛而遠集兮,君罔謂汝何之?欲橫奔而失路兮,堅志而不忍。背膺牉以交痛兮,心鬱結而紆軫。檮木蘭以矯蕙兮,鑿申椒以為糧。播江離與滋菊兮,願春日以為糗芳。恐情質之不信兮,故重著以自明。矯玆媚以私處兮,願曾思而遠身。


\end{pinyinscope}