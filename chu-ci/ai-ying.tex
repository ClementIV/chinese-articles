\article{哀郢}

\begin{pinyinscope}
皇天之不純命兮,何百姓之震愆?民離散而相失兮,方仲春而東遷。去故鄉而就遠兮,遵江夏以流亡。出國門而軫懷兮,甲之朝吾以行。發郢都而去閭兮,荒忽其焉極?楫齊揚以容與兮,哀見君而不再得。望長楸而太息兮,涕淫淫其若霰。過夏首而西浮兮,顧龍門而不見。心嬋媛而傷懷兮,眇不知其所蹠。順風波以從流兮,焉洋洋而為客。淩陽侯之氾濫兮,忽翱翔之焉薄。心絓結而不解兮,思蹇產而不釋。將運舟而下浮兮,上洞庭而下江。去終古之所居兮,今逍遙而來東。羌靈魂之欲歸兮,何須臾而忘反。背夏浦而西思兮,哀故都之日遠。登大墳以遠望兮,聊以舒吾憂心。哀州土之平樂兮,悲江介之遺風。當陵陽之焉至兮,淼南渡之焉如?曾不知夏之為丘兮,孰兩東門之可蕪?心不怡之長久兮,憂與愁其相接。惟郢路之遼遠兮,江與夏之不可涉。忽若去不信兮,至今九年而不復。慘鬱鬱而不通兮,蹇侘傺而含慼。外承歡之汋約兮,諶荏弱而難持。忠湛湛而願進兮,妒被離而鄣之。堯舜之抗行兮,瞭杳杳而薄天。眾讒人之嫉妒兮,被以不慈之偽名。憎慍惀之脩美兮,好夫人之康慨。眾踥蹀而日進兮,美超遠而逾邁。

亂曰:曼余目以流觀兮,冀壹反之何時。鳥飛反故鄉兮,狐死必首丘。信非吾罪而棄逐兮,何日夜而忘之!


\end{pinyinscope}