\article{離世}

\begin{pinyinscope}
靈懷其不吾知兮,靈懷其不吾聞。就靈懷之皇祖兮,愬靈懷之鬼神。靈懷曾不吾與兮,即聽夫人之諛辭。余辭上參於天墜兮,旁引之於四時。指日月使延照兮,撫招搖以質正。立師曠俾端辭兮,命咎繇使並聽。兆出名曰正則兮,卦發字曰靈均。余幼既有此鴻節兮,長愈固而彌純。不從俗而詖行兮,直躬指而信志。不枉繩以追曲兮,屈情素以從事。端余行其如玉兮,述皇輿之踵跡。群阿容以晦光兮,皇輿覆以幽闢。輿中塗以回畔兮,駟馬驚而橫奔。執組者不能制兮,必折軛而摧轅。斷鑣銜以馳騖兮,暮去次而敢止。路蕩蕩其無人兮,遂不禦乎千里。

身衡陷而下沈兮,不可獲而復登。不顧身之卑賤兮,惜皇輿之不興。出國門而端指兮,冀壹寤而錫還。哀僕夫之坎毒兮,屢離憂而逢患。九年之中不吾反兮,思彭咸之水游。惜師延之浮渚兮,赴汨羅之長流。遵江曲之逶移兮,觸石碕而衡游。波澧澧而揚澆兮,順長瀨之濁流。淩黃沱而下低兮,思還流而復反。玄輿馳而並集兮,身容與而日遠。櫂舟杭以橫濿兮,濟湘流而南極。立江界而長吟兮,愁哀哀而累息。情慌忽以忘歸兮,神浮游以高厲。心蛩蛩而懷顧兮,魂眷眷而獨逝。

歎曰:余思舊邦心依違兮,日暮黃昏羌幽悲兮,去郢東遷余誰慕兮,讒夫黨旅其以茲故兮,河水淫淫情所願兮,顧瞻郢路終不返兮。


\end{pinyinscope}