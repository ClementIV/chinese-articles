\article{初放}

\begin{pinyinscope}
平生於國兮,長於原野。言語訥譅兮,又無彊輔。淺智褊能兮,聞見又寡。數言便事兮,見怨門下。王不察其長利兮,卒見棄乎原野。伏念思過兮,無可改者。群眾成朋兮,上浸以惑。巧佞在前兮,賢者滅息。堯、舜聖已沒兮,孰為忠直?高山崔巍兮,水流湯湯。死日將至兮,與麋鹿如坑。塊兮鞠,當道宿,舉世皆然兮,余將誰告?斥逐鴻鵠兮,近習鴟梟,斬伐橘柚兮,列樹苦桃。便娟之修竹兮,寄生乎江潭。上葳蕤而防露兮,下泠泠而來風。孰知其不合兮?若竹柏之異心。往者不可及兮,來者不可待。悠悠蒼天兮,莫我振理。竊怨君之不寤兮,吾獨死而後已。


\end{pinyinscope}