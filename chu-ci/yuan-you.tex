\article{遠遊}

\begin{pinyinscope}
悲時俗之近阨兮,願輕舉而遠遊。質菲薄而無因兮,焉托乘而上浮?遭沈濁而污穢兮,獨鬱結其誰語!夜耿耿而不寐兮,魂營營而至曙。惟天地之無窮兮,哀人生之長勤,往者余弗及兮,來者吾不聞,步徙倚而遙思兮,怊惝怳而乖懷。意荒忽而流蕩兮,心愁淒而增悲。神倏忽而不反兮,形枯槁而獨留。內惟省以端操兮,還應正氣之所由。漠虛靜以恬愉兮,澹無為而自得。聞赤鬆之清塵兮,願承風乎遺則。貴真人之休德兮,美往世之登仙,與化去而不風兮,名聲著而日延。奇傅說之托辰星兮,羨韓眾之得一。形穆穆以浸遠兮,離人群而遁逸。因氣變而遂曾舉兮,忽神奔而鬼怪。時彷彿以遙見兮,精皎皎以往來。超氛埃而淑郵兮,終不反其故都。免眾患而不懼兮,世莫知其所如。恐天時之代序兮,耀靈曄而西征。微霜降而下淪兮,悼芳草之先零。聊仿佯而逍遙兮,永歷年而無成。誰可與玩斯遺芳兮?長向風而舒情。高陽邈以遠兮,余將焉所程?

重曰:春秋忽其不淹兮,奚久留此故居。軒轅不可攀援兮,吾將從王喬而娛戲。餐六氣而飲沆瀣兮,漱正陽而含朝霞。保神明之清澄兮,精氣入而粗穢除。順凱風以從游兮,至南巢而一息。見王子而宿之兮,審一氣之和德。曰:「道可受兮,不可傳;其小無內兮,其大夫垠;毋滑而魂兮,彼將自然;一氣孔神兮,於中夜存;虛以待之存,無以為先;庶類以成兮,此德之門。」聞至貴而遂徂兮,忽乎吾將行。仍羽人於丹丘,留不死之舊鄉。朝濯發於湯谷兮,夕晞余身兮九陽。吸飛泉之微液兮,懷琬琰之華英。玉色頩以脕顏兮,精醇粹而始壯。質銷鑠以汋約兮,神要眇以淫放。嘉南州之炎德兮,麗桂樹之冬榮;山蕭條而無獸兮,野寂漠其無人。載營魄而登霞兮,掩浮雲而上徵。命天閽其開關兮,排閶闔而望予。如豐隆使先導兮,問太微之所居。集重陽入帝宮兮,造旬始而觀清都。朝發軔於太儀兮,夕始臨乎於微閭。屯余車之萬乘兮,紛容與而並馳。駕八龍之婉婉兮,載雲旗之逶蛇。建雄虹之採旄兮,五色雜而炫耀。服偃蹇以低昂兮,驂連蜷以驕驁。騎膠葛以雜亂兮,斑漫衍而方行。撰余轡而正策兮,吾將過乎句芒。歷太皓以右轉兮,前飛廉以啟路。陽杲杲其未光兮,凌天地以徑度。風伯為作先驅兮,氛埃闢而清涼。鳳凰翼其承旗兮,遇蓐收乎西皇。攬慧星以為旍兮,舉鬥柄以為麾。叛陸離其上下兮,游驚霧之流波。時暖曃其曭莽兮,召玄武而奔屬。後文昌使掌行兮,選署眾神以並轂。路漫漫其修遠兮,徐弭節而高厲。左雨師使徑侍兮,右雷公以為衛。欲度世以忘歸兮,意姿睢以擔撟。內欣欣而自美兮,聊愉娛以淫樂。涉青雲以氾濫游兮,忽臨睨夫舊鄉。僕夫懷余心悲兮,邊馬顧而不行。思舊故以想像兮,長太息而掩涕。泛容與而遐舉兮,聊抑志而自弭。指炎神而直馳兮,吾將往乎南疑。覽方外之荒忽兮,沛𣶈瀁而自浮。祝融戒而蹕御兮,騰告鸞鳥迎宓妃。張《咸池》奏《承雲》兮,二女御《九韶》歌。使湘靈鼓瑟兮,令海若舞馮夷。玄螭蟲象並出進兮,形蟉虯而逶蛇。雌蜺便娟以增撓兮,鸞鳥軒翥而翔飛。音樂博衍無終極兮,焉及逝以徘徊。舒並節以馳騖兮,逴絕垠乎寒門。軼迅風天清源兮,從顓瑣乎增冰。歷玄冥以邪徑兮,乘間維以反顧。召黔贏而見之兮,為余先乎平路。經營四方兮,周流六漠。上至列缺兮,降望大壑。下崢嶸而無地兮,上寥廓而無天。視倏忽而無見兮,聽惝恍而無聞。超無為以至清兮,與泰初而為鄰。


\end{pinyinscope}