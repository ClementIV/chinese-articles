\article{思古}

\begin{pinyinscope}
冥冥深林兮,樹木鬱鬱。山參差以嶄巖兮,阜杳杳以蔽日。悲余心之悁悁兮,目眇眇而遺泣。風騷屑以搖木兮,雲吸吸以湫戾。悲余生之無歡兮,愁倥傯於山陸。旦徘徊於長阪兮,夕仿偟而獨宿。髮披披以鬤鬤兮,躬劬勞而瘏悴。魂俇俇而南行兮,泣霑襟而濡袂。心嬋媛而無告兮,口噤閉而不言。違郢都之舊閭兮,回湘、沅而遠遷。念余邦之橫陷兮,宗鬼神之無次。閔先嗣之中絕兮,心惶惑而自悲。聊浮游於山峽兮,步周流於江畔。臨深水而長嘯兮,且倘佯而氾觀。

興《離騷》之微文兮,冀靈修之壹悟。還余車於南郢兮,復往軌於初古。道修遠其難遷兮,傷余心之不能已。背三五之典刑兮,絕洪範之闢紀。播規矩以背度兮,錯權衡而任意。操繩墨而放棄兮,傾容幸而侍側。甘棠枯於豐草兮,藜棘樹於中庭。西施斥於北宮兮,仳倠倚於彌楹。烏獲戚而驂乘兮,燕公操於馬圉。蒯聵登於清府兮,咎繇棄而在野外。蓋見茲以永歎兮,欲登階而狐疑。乘白水而高騖兮,因徙弛而長辭。

歎曰:倘佯壚阪沼水深兮,容與漢渚涕淫淫兮,鍾牙已死誰為聲兮?纖阿不御焉舒情兮,曾哀悽欷心離離兮,還顧高丘泣如灑兮。


\end{pinyinscope}