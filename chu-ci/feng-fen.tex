\article{逢紛}

\begin{pinyinscope}
伊伯庸之末胄兮,諒皇直之屈原。雲余肇祖於高陽兮,惟楚懷之嬋連。原生受命於貞節兮,鴻永路有嘉名。齊名字於天地兮,並光明於列星。吸精粹而吐氛濁兮,橫邪世而不取容。行叩誠而不阿兮,遂見排而逢讒。後聽虛而黜實兮,不吾理而順情。腸憤悁而含怒兮,志遷蹇而左傾。心戃慌其不我與兮,躬速速其不吾親。辭靈修而隕志兮,吟澤畔之江濱。椒桂羅以顛覆兮,有竭信而歸誠。讒夫藹藹而漫著兮,曷其不舒予情?

始結言於廟堂兮,信中塗而叛之。懷蘭蕙與衡芷兮,行中野而散之。聲哀哀而懷高丘兮,心愁愁而思舊邦。願承閒而自恃兮,徑淫曀而道壅。顏黴黧以沮敗兮,精越裂而衰耄。裳襜襜而含風兮,衣納納而掩露。赴江湘之湍流兮,順波湊而下降。徐徘徊於山阿兮,飄風來之洶洶。馳余車兮玄石,步余馬兮洞庭。平明發兮蒼梧,夕投宿兮石城。芙蓉蓋而蔆華車兮,紫貝闕而玉堂。薜荔飾而陸離薦兮,魚鱗衣而白傭蜺。登逢龍而下隕兮,違故鄉之漫漫。思南郢之舊俗兮,腸一夕而九運。揚流波之潢潢兮,體溶溶而東回。心怊悵以永思兮,意晻晻而日頹。白露紛以塗塗兮,秋風瀏以蕭蕭。身永流而不還兮,魂長逝而常愁。

歎曰:譬彼流水紛揚磕兮,波逢洶涌濆壅滂兮。揄揚滌盪飄流隕往觸崟石兮,龍卬脟圈繚戾宛轉阻相薄兮,遭紛逢凶蹇離尤兮,垂文揚採遺將來兮。


\end{pinyinscope}