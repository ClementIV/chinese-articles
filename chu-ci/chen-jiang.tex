\article{沈江}

\begin{pinyinscope}
惟往古之得失兮,覽私微之所傷。堯、舜聖而慈仁兮,後世稱而弗忘。齊桓失於專任兮,夷吾忠而名彰。晉獻惑於孋姬兮,申生孝而被殃。偃王行其仁義兮,荊文寤而徐亡。紂暴虐以失位兮,周得佐乎呂望。修往古以行恩兮,封比干之丘壟。賢俊慕而自附兮,日浸淫而合同。明法令而修理兮,蘭芷幽而有芳。苦眾人之妒予兮,箕子寤而佯狂。不顧地以貪名兮,心怫鬱而內傷。聯蕙芷以為佩兮,過鮑肆而失香。正臣端其操行兮,反離謗而見攘。世俗更而變化兮,伯夷餓於首陽。獨廉潔而不容兮,叔齊久而逾明。浮雲陳而蔽晦兮,使日月乎無光。忠臣貞而欲諫兮,讒諛毀而在旁。秋草榮其將實兮,微霜下而夜降。商風肅而害生兮,百草育而不長。眾並諧以妒賢兮,孤聖特而易傷。懷計謀而不見用兮,巖穴處而隱藏。成功隳而不卒兮,子胥死而不葬。世從俗而變化兮,隨風靡而成行。信直退而毀敗兮,虛偽進而得當。追悔過之無及兮,豈盡忠而有功。廢制度而不用兮,務行私而去公。終不變而死節兮,惜年齒之未央。將方舟而下流兮,冀幸君之發矇。痛忠言之逆耳兮,恨申子之沈江。願悉心之所聞兮,遭值君之不聰。不開寤而難道兮,不別橫之與縱。聽姦臣之浮說兮,絕國家之久長。滅規矩而不用兮,背繩墨之正方。離憂患而乃寤兮,若縱火於秋蓬。業失之而不救兮,尚何論乎禍凶。彼離畔而朋黨兮,獨行之士其何望?日漸染而不自知兮,秋毫微哉而變容。眾輕積而折軸兮,原咎雜而累重。赴湘、沅之流澌兮,恐逐波而復東。懷沙礫而自沈兮,不忍見君之蔽壅。


\end{pinyinscope}