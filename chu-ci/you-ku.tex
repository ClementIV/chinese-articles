\article{憂苦}

\begin{pinyinscope}
悲余心之悁悁兮,哀故邦之逢殃。辭九年而不復兮,獨煢煢而南行。思余俗之流風兮,心紛錯而不受。遵野莽以呼風兮,步從容於山廋。巡陸夷之曲衍兮,幽空虛以寂寞。倚石巖以流涕兮,憂憔悴而無樂。登巑岏以長企兮,望南郢而闚之。山修遠其遼遼兮,塗漫漫其無時。聽玄鶴之晨鳴兮,於高岡之峨峨。獨憤積而哀娛兮,翔江洲而安歌。三鳥飛以自南兮,覽其志而欲北。願寄言於三鳥兮,去飄疾而不可得。

欲遷志而改操兮,心紛結其未離。外彷徨而游覽兮,內惻隱而含哀。聊須臾以時忘兮,心漸漸其煩錯。願假簧以舒憂兮,志紆鬱其難釋。歎《離騷》以揚意兮,猶未殫於《九章》。長噓吸以於悒兮,涕橫集而成行。傷明珠之赴泥兮,魚眼璣之堅藏。同駑鸁與乘駔兮,雜斑駮與闒茸。葛藟虆於桂樹兮,鴟鴞集於木蘭。偓促談於廊廟兮,律魁放乎山間。惡虞氏之簫韶兮,好遺風之激楚。潛周鼎於江淮兮,爨土鬵於中宇。且人心之持舊兮,而不可保長。邅彼南道兮,征夫宵行。思念郢路兮,還顧睠睠。涕流交集兮,泣下漣漣。

登山長望中心悲兮,菀彼青青泣如頹兮,留思北顧涕漸漸兮,折銳摧矜凝氾濫兮,念我煢煢魂誰求兮,僕夫慌悴散若流兮。


\end{pinyinscope}