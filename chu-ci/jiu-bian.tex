\article{九辯}

\begin{pinyinscope}
悲哉!秋之為氣也。蕭瑟兮,草木搖落而變衰。憭慄兮,若在遠行。登山臨水兮,送將歸。泬寥兮,天高而氣清;寂寥兮,收潦而水清。憯悽增欷兮,薄寒之中人;愴怳懭悢兮,去故而就新;坎廩兮,貧士失職而志不平;廓落兮,羈旅而無友生;惆悵兮,而私自憐。燕翩翩其辭歸兮,蟬寂漠而無聲。鴈廱廱而南遊兮,鶤雞啁哳而悲鳴。獨申旦而不寐兮,哀蟋蟀之宵徵。時亹亹而過中兮,蹇淹留而無成。

悲憂窮戚兮獨處廓,有美一人兮心不繹;去鄉離家兮徠遠客,超逍遙兮今焉薄!專思君兮不可化,君不知兮可柰何!蓄怨兮積思,心煩憺兮忘食事。願一見兮道余意,君之心兮與余異。車既駕兮朅而歸,不得見兮心傷悲。倚結軨兮長太息,涕潺湲兮下霑軾。慷慨絕兮不得,中瞀亂兮迷惑。私自憐兮何極?心怦怦兮諒直。皇天平分四時兮,竊獨悲此廩秋。白露既下百草兮,奄離披此梧楸。去白日之昭昭兮,襲長夜之悠悠。離芳藹之方壯兮,余萎約而悲愁。秋既先戒以白露兮,冬又申之以嚴霜。收恢台之孟夏兮,然欿傺而沈藏。葉菸邑而無色兮,枝煩挐而交橫。顏淫溢而將罷兮,柯彷彿而萎黃。萷櫹槮之可哀兮,形銷鑠而瘀傷。惟其紛糅而將落兮,恨其失時而無當。攬騑轡而下節兮,聊逍遙以相佯。歲忽忽而遒盡兮,恐余壽之弗將。悼余生之不時兮,逢此世之俇攘。澹容與而獨倚兮,蟋蟀鳴此西堂。心怵惕而震盪兮,何所憂之多方。卬明月而太息兮,步列星而極明。

竊悲夫蕙華之曾敷兮,紛旖旎乎都房。何曾華之無實兮,從風雨而飛颺!以為君獨服此蕙兮,羌無以異於眾芳。閔奇思之不通兮,將去君而高翔。心閔憐之慘悽兮,願一見而有明。重無怨而生離兮,中結軫而增傷。豈不鬱陶而思君兮?君之門以九重!猛犬狺狺而迎吠兮,關梁閉而不通。皇天淫溢而秋霖兮,後土何時而得漧?塊獨守此無澤兮,仰浮雲而永歎!

何時俗之工巧兮?背繩墨而改錯!卻騏驥而不乘兮,策駑駘而取路。當世豈無騏驥兮,誠莫之能善御。見執轡者非其人兮,故駒跳而遠去。鳧鴈皆唼夫梁藻兮,鳳愈飄翔而高舉。圜鑿而方枘兮,吾固知其鉏鋙而難入。眾鳥皆有所登棲兮,鳳獨遑遑而無所集。願銜枚而無言兮,嘗被君之渥洽。太公九十乃顯榮兮,誠未遇其匹合。謂騏驥兮安歸?謂鳳皇兮安棲?變古易俗兮世衰,今之相者兮舉肥。騏驥伏匿而不見兮,鳳皇高飛而不下。鳥獸猶知褱德兮,何雲賢士之不處?驥不驟進而求服兮,鳳亦不貪餧而妄食。君棄遠而不察兮,雖願忠其焉得?欲寂漠而絕端兮,竊不敢忘初之厚德。獨悲愁其傷人兮,馮鬱鬱其何極?

霜露慘悽而交下兮,心尚幸其弗濟。霰雪雰糅其增加兮,乃知遭命之將至。願徼幸而有待兮,泊莽莽與野草同死。願自往而徑游兮,路壅絕而不通。欲循道而平驅兮,又未知其所從。然中路而迷惑兮,自厭按而學誦。性愚陋以褊淺兮,信未達乎從容。

竊美申包胥之氣晟兮,恐時世之不固。何時俗之工巧兮?滅規矩而改鑿!獨耿介而不隨兮,願慕先聖之遺教。處濁世而顯榮兮,非余心之所樂。與其無義而有名兮,寧窮處而守高。食不媮而為飽兮,衣不苟而為溫。竊慕詩人之遺風兮,願託志乎素餐。蹇充倔而無端兮,泊莽莽而無垠。無衣裘以御冬兮,恐溘死不得見乎陽春。

靚杪秋之遙夜兮,心繚悷而有哀。春秋逴逴而日高兮,然惆悵而自悲。四時遞來而卒歲兮,陰陽不可與儷偕。白日晼晚其將入兮,明月銷鑠而減毀。歲忽忽而遒盡兮,老冉冉而愈弛。心搖悅而日幸兮,然怊悵而無冀。中憯惻之悽愴兮,長太息而增欷。年洋洋以日往兮,老嵺廓而無處。事亹亹而覬進兮,蹇淹留而躊躇。

何氾濫之浮雲兮?猋廱蔽此明月。忠昭昭而願見兮,然霠曀而莫達。願皓日之顯行兮,雲矇矇而蔽之。竊不自料而願忠兮,或黕點而汙之。堯舜之抗行兮,瞭冥冥而薄天。

何險巇之嫉妒兮?被以不慈之偽名。彼日月之照明兮,尚黯黮而有瑕。何況一國之事兮,亦多端而膠加。被荷裯之晏晏兮,然潢洋而不可帶。既驕美而伐武兮,負左右之耿介。憎慍惀之修美兮,好夫人之慷慨。眾踥蹀而日進兮,美超遠而逾邁。農夫輟耕而容與兮,恐田野之蕪穢。事綿綿而多私兮,竊悼後之危敗。世雷同而炫曜兮,何毀譽之昧昧!今修飾而窺鏡兮,後尚可以竄藏。願寄言夫流星兮,羌儵忽而難當。卒廱蔽此浮雲,下暗漠而無光。

堯舜皆有所舉任兮,故高枕而自適。諒無怨於天下兮,心焉取此怵惕?乘騏驥之瀏瀏兮,馭安用夫強策?諒城郭之不足恃兮,雖重介之何益?邅翼翼而無終兮,忳惛惛而愁約。生天地之若過兮,功不成而無嶜。願沈滯而不見兮,尚欲布名乎天下。然潢洋而不遇兮,直怐愚而自苦。莽洋洋而無極兮,忽翱翔之焉薄?國有驥而不知乘兮,焉皇皇而更索?甯戚謳於車下兮,桓公聞而知之。無伯樂之相善兮,今誰使乎譽之?罔流涕以聊慮兮,惟著意而得之。紛純純之願忠兮,妒被離而鄣之。

亂曰:願賜不肖之軀而別離兮,放遊志乎雲中。乘精氣之摶摶兮,騖諸神之湛湛。驂白霓之習習兮,歷群靈之豐豐。左朱雀之苃苃兮,右蒼龍之躣躣。屬雷師之闐闐兮,通飛廉之衙衙。前輕輬之鏘鏘兮,後輜乘之從從。載雲旗之委蛇兮,扈屯騎之容容。計專專之不可化兮,願遂推而為臧。賴皇天之厚德兮,還及君之無恙。


\end{pinyinscope}