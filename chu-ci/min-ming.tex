\article{愍命}

\begin{pinyinscope}
昔皇考之嘉志兮,喜登能而亮賢。情純潔而罔薉兮,姿盛質而無愆。放佞人與諂諛兮,斥讒夫與便嬖。親忠正之悃誠兮,招貞良與明智。心溶溶其不可量兮,情澹澹其若淵。回邪辟而不能入兮,誠願藏而不可遷。逐下袟於後堂兮,迎虙妃於伊雒。刜讒賊於中廇兮,選呂管於榛薄。叢林之下無怨士兮,江河之畔無隱夫。三苗之徒以放逐兮,伊皋之倫以充廬。

今反表以為裏兮,顛裳以為衣。戚宋萬於兩楹兮,廢周邵於遐夷。卻騏驥以轉運兮,騰驢鸁以馳逐。蔡女黜而出帷兮,戎婦入而綵繡服。慶忌囚於阱室兮,陳不占戰而赴圍。破伯牙之號鍾兮,挾人箏而彈緯。藏瑉石於金匱兮,捐赤瑾於中庭。韓信蒙於介胄兮,行夫將而攻城。莞芎棄於澤洲兮,瓟瓥蠹於筐簏。麒麟奔於九皋兮,熊羆群而逸囿。折芳枝與瓊華兮,樹枳棘與薪柴。掘荃蕙與射干兮,耘藜藿與蘘荷。惜今世其何殊兮,遠近思而不同。或沈淪其無所達兮,或清激其無所通。哀余生之不當兮,獨蒙毒而逢尤。雖謇謇以申志兮,君乖差而屏之。誠惜芳之菲菲兮,反以茲為腐也。懷椒聊之蔎蔎兮,乃逢紛以罹詬也。

歎曰:嘉皇既歿終不返兮,山中幽險郢路遠兮,讒人諓諓孰可愬兮,征夫罔極誰可語兮?行吟累欷聲喟喟兮,懷憂含戚何侘傺兮。


\end{pinyinscope}