\article{金滕}

\begin{pinyinscope}
武王有疾,周公作《金滕》。

既克商二年,王有疾,弗豫。二公曰:「我其為王穆卜。」周公曰:「未可以戚我先王?」公乃自以為功,為三壇同墠。為壇於南方,北面,周公立焉。植璧秉珪,乃告太王、王季、文王。

史乃冊,祝曰:「惟爾元孫某,遘厲虐疾。若爾三王是有丕子之責于天,以旦代某之身。予仁若考能,多材多藝,能事鬼神。乃元孫不若旦多材多藝,不能事鬼神。乃命于帝庭,敷佑四方,用能定爾子孫于下地。四方之民罔不祗畏。嗚呼!無墜天之降寶命,我先王亦永有依歸。今我即命于元龜,爾之許我,我其以璧與珪歸俟爾命;爾不許我,我乃屏璧與珪。」

乃卜三龜,一習吉。啟籥見書,乃並是吉。公曰:「體!王其罔害。予小子新命于三王,惟永終是圖;茲攸俟,能念予一人。」

公歸,乃納冊于金滕之匱中。王翼日乃瘳。

武王既喪,管叔及其群弟乃流言於國,曰:「公將不利於孺子。」周公乃告二公曰:「我之弗辟,我無以告我先王。」周公居東二年,則罪人斯得。于後,公乃為詩以貽王,名之曰《鴟鴞》。王亦未敢誚公。

秋,大熟,未獲,天大雷電以風,禾盡偃,大木斯拔,邦人大恐。王與大夫盡弁以啟金滕之書,乃得周公所自以為功代武王之說。二公及王乃問諸史與百執事。對曰:「信。噫!公命我勿敢言。」

王執書以泣,曰:「其勿穆卜!昔公勤勞王家,惟予沖人弗及知。今天動威以彰周公之德,惟朕小子其新逆,我國家禮亦宜之。」王出郊,天乃雨,反風,禾則盡起。二公命邦人凡大木所偃,盡起而築之。歲則大熟。


\end{pinyinscope}