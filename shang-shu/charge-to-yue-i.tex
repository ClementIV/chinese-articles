\article{說命上}

\begin{pinyinscope}
高宗夢得說,使百工營求諸野,得諸傅巖,作《說命》三篇。

王宅憂,亮陰三祀。既免喪,其惟弗言,群臣咸諫于王曰:「嗚呼!知之曰明哲,明哲實作則。天子惟君萬邦,百官承式,王言惟作命,不言臣下罔攸稟令。」

王庸作書以誥曰:「以台正于四方,惟恐德弗類,茲故弗言。恭默思道,夢帝賚予良弼,其代予言。」乃審厥象,俾以形旁求于天下。說築傅巖之野,惟肖。爰立作相。王置諸其左右。

命之曰:「朝夕納誨,以輔台德。若金,用汝作礪;若濟巨川,用汝作舟楫;若歲大旱,用汝作霖雨。啟乃心,沃朕心,若藥弗瞑眩,厥疾弗瘳;若跣弗視地,厥足用傷。惟暨乃僚,罔不同心,以匡乃辟。俾率先王,迪我高后,以康兆民。嗚呼!欽予時命,其惟有終。」

說復于王曰:「惟木從繩則正,后從諫則聖。后克聖,臣不命其承,疇敢不祗若王之休命?」


\end{pinyinscope}