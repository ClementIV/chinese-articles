\article{微子}

\begin{pinyinscope}
殷既錯天命,微子作誥父師、少師。

微子若曰:「父師、少師!殷其弗或亂正四方。我祖厎遂陳于上,我用沈酗于酒,用亂敗厥德于下。殷罔不小大好草竊奸宄。卿士師師非度。凡有辜罪,乃罔恆獲,小民方興,相為敵仇。今殷其淪喪,若涉大水,其無津涯。殷遂喪,越至于今!」

曰:「父師、少師,我其發出狂?吾家耄遜于荒?今爾無指告,予顛隮,若之何其?」

父師若曰:「王子!天毒降災荒殷邦,方興沈酗于酒,乃罔畏畏,咈其耇長舊有位人。今殷民乃攘竊神祇之犧牷牲用以容,將食無災。降監殷民,用乂仇斂,召敵仇不怠。罪合于一,多瘠罔詔。商今其有災,我興受其敗;商其淪喪,我罔為臣僕。詔王子出,迪我舊云刻子。王子弗出,我乃顛隮。自靖,人自獻于先王,我不顧行遯。」


\end{pinyinscope}