\article{盤庚上}

\begin{pinyinscope}
盤庚五遷,將治亳殷,民咨胥怨。作《盤庚》三篇。

盤庚遷于殷,民不適有居,率籲眾慼出,矢言曰:「我王來,既爰宅于茲,重我民,無盡劉。不能胥匡以生,卜稽,曰其如台?先王有服,恪謹天命,茲猶不常寧;不常厥邑,于今五邦。今不承于古,罔知天之斷命,矧曰其克從先王之烈?若顛木之有由蘗,天其永我命于茲新邑,紹復先王之大業,厎綏四方。」

盤庚斆于民,由乃在位以常舊服,正法度。曰:「無或敢伏小人之攸箴!」王命眾,悉至于庭。

王若曰:「格汝眾,予告汝訓:汝猷黜乃心,無傲從康。古我先王,亦惟圖任舊人共政。王播告之,修不匿厥指。王用丕欽;罔有逸言,民用丕變。今汝聒聒,起信險膚,予弗知乃所訟。非予自荒茲德,惟汝含德,不惕予一人。予若觀火,予亦拙謀作乃逸。」

若網在綱,有條而不紊;若農服田,力穡乃亦有秋。汝克黜乃心,施實德于民,至于婚友,丕乃敢大言汝有積德。乃不畏戎毒于遠邇,惰農自安,不昏作勞,不服田畝,越其罔有黍稷。

汝不和吉言于百姓,惟汝自生毒,乃敗禍姦宄,以自災于厥身。乃既先惡于民,乃奉其恫,汝悔身何及!相時憸民,猶胥顧于箴言,其發有逸口,矧予制乃短長之命!汝曷弗告朕,而胥動以浮言,恐沈于眾?若火之燎于原,不可嚮邇,其猶可撲滅。則惟汝眾自作弗靖,非予有咎。

遲任有言曰:『人惟求舊,器非求舊,惟新。』古我先王暨乃祖乃父胥及逸勤,予敢動用非罰?世選爾勞,予不掩爾善。茲予大享于先王,爾祖其從與享之。作福作災,予亦不敢動用非德。

「予告汝于難,若射之有志。汝無侮老成人,無弱孤有幼。各長于厥居。勉出乃力,聽予一人之作猷。無有遠邇,用罪伐厥死,用德彰厥善。邦之臧,惟汝眾;邦之不臧,惟予一人有佚罰。凡爾眾,其惟致告:自今至于後日,各恭爾事,齊乃位,度乃口。 罰及爾身,弗可悔。」


\end{pinyinscope}