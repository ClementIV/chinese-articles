\article{洛誥}

\begin{pinyinscope}
召公既相宅,周公往營成周,使來告卜,作《洛誥》。

周公拜手稽首曰:「朕復子明辟。王如弗敢及天基命定命,予乃胤保大相東土,其基作民明辟。予惟乙卯,朝至于洛師。我卜河朔黎水,我乃卜澗水東,瀍水西,惟洛食;我又卜瀍水東,亦惟洛食。伻來以圖及獻卜。」

王拜手稽首曰:「公不敢不敬天之休,來相宅,其作周匹,休!公既定宅,伻來,來,視予卜,休恆吉。我二人共貞。公其以予萬億年敬天之休。拜手稽首誨言。」

周公曰:「王,肇稱殷禮,祀于新邑,咸秩無文。予齊百工,伻從王于周,予惟曰:『庶有事。』今王即命曰:『記功,宗以功作元祀。』惟命曰:『汝受命篤弼,丕視功載,乃汝其悉自教工。』

孺子其朋,孺子其朋,其往!無若火始焰焰;厥攸灼敘,弗其絕。厥若彝及撫事如予,惟以在周工往新邑。伻向即有僚,明作有功,惇大成裕,汝永有辭。」

公曰:「已!汝惟沖子,惟終。汝其敬識百辟享,亦識其有不享。享多儀,儀不及物,惟曰不享。惟不役志于享,凡民惟曰不享,惟事其爽侮。乃惟孺子頒,朕不暇聽。

朕教汝于棐民,彝汝乃是不蘉,乃時惟不永哉!篤敘乃正父罔不若予,不敢廢乃命。汝往敬哉!茲予其明農哉!彼裕我民,無遠用戾。」

王若曰:「公!明保予沖子。公稱丕顯德,以予小子揚文武烈,奉答天命,和恆四方民,居師;惇宗將禮,稱秩元祀,咸秩無文。惟公德明光于上下,勤施于四方,旁作穆穆,迓衡不迷。文武勤教,予沖子夙夜毖祀。」王曰:「公功棐迪,篤罔不若時。」

王曰:「公!予小子其退,即辟于周,命公後。四方迪亂未定,于宗禮亦未克敉,公功迪將,其後監我士師工,誕保文武受民,亂為四輔。」王曰:「公定,予往已。公功肅將祗歡,公無困哉!我惟無斁其康事,公勿替刑,四方其世享。」

周公拜手稽首曰:「王命予來承保乃文祖受命民,越乃光烈考武王弘朕恭。孺子來相宅,其大惇典殷獻民,亂為四方新辟,作周恭先。曰其自時中乂,萬邦咸休,惟王有成績。予旦以多子越御事篤前人成烈,答其師,作周孚先。』考朕昭子刑,乃單文祖德。

伻來毖殷,乃命寧予以秬鬯二卣。曰明禋,拜手稽首休享。予不敢宿,則禋于文王、武王。惠篤敘,無有遘自疾,萬年厭于乃德,殷乃引考。王伻殷乃承敘萬年,其永觀朕子懷德。」

戊辰,王在新邑烝,祭歲,文王騂牛一,武王騂牛一。王命作冊逸祝冊,惟告周公其後。

王賓殺禋咸格,王入太室,祼。王命周公後,作冊逸誥,在十有二月。惟周公誕保文武受命,惟七年。


\end{pinyinscope}