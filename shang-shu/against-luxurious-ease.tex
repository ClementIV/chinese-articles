\article{無逸}

\begin{pinyinscope}
周公作《無逸》。

周公曰:「嗚呼!君子所,其無逸。先知稼穡之艱難,乃逸,則知小人之依。相小人,厥父母勤勞稼穡,厥子乃不知稼穡之艱難,乃逸乃諺。既誕,否則侮厥父母曰:『昔之人無聞知。』」

周公曰:「嗚呼!我聞曰:昔在殷王中宗,嚴恭寅畏,天命自度,治民祗懼,不敢荒寧。肆中宗之享國七十有五年。其在高宗,時舊勞于外,爰暨小人。作其即位,乃或亮陰,三年不言。其惟不言,言乃雍。不敢荒寧,嘉靖殷邦。至于小大,無時或怨。肆高宗之享國五十年有九年。其在祖甲,不義惟王,舊為小人。作其即位,爰知小人之依,能保惠于庶民,不敢侮鰥寡。肆祖甲之享國三十有三年。自時厥後立王,生則逸,生則逸,不知稼穡之艱難,不聞小人之勞,惟耽樂之從。自時厥後,亦罔或克壽。或十年,或七八年,或五六年,或四三年。」

周公曰:「嗚呼!厥亦惟我周太王、王季,克自抑畏。文王卑服,即康功田功。徽柔懿恭,懷保小民,惠鮮鰥寡。自朝至于日中昃,不遑暇食,用咸和萬民。文王不敢盤于游田,以庶邦惟正之供。文王受命惟中身,厥享國五十年。」

周公曰:「嗚呼!繼自今嗣王,則其無淫于觀、于逸、于游、于田,以萬民惟正之供。無皇曰:『今日耽樂。』乃非民攸訓,非天攸若,時人丕則有愆。無若殷王受之迷亂,酗于酒德哉!」

周公曰:「嗚呼!我聞曰:『古之人猶胥訓告,胥保惠,胥教誨,民無或胥譸張為幻。』此厥不聽,人乃訓之,乃變亂先王之正刑,至于小大。民否則厥心違怨,否則厥口詛祝。」

周公曰:「嗚呼!自殷王中宗及高宗及祖甲及我周文王,茲四人迪哲。厥或告之曰:『小人怨汝詈汝。』則皇自敬德。厥愆,曰:『朕之愆。』允若時,不啻不敢含怒。此厥不聽,人乃或譸張為幻,曰小人怨汝詈汝,則信之,則若時,不永念厥辟,不寬綽厥心,亂罰無罪,殺無辜。怨有同,是叢于厥身。」

周公曰:「嗚呼!嗣王其監于茲。」


\end{pinyinscope}