\article{益稷}

\begin{pinyinscope}
帝曰:「來,禹!汝亦昌言。」禹拜曰:「都!帝,予何言?予思日孜孜。」皋陶曰:「吁!如何?」禹曰:「洪水滔天,浩浩懷山襄陵,下民昏墊。予乘四載,隨山刊木,暨益奏庶鮮食。予決九川,距四海,濬畎澮距川;暨稷播,奏庶艱食鮮食。懋遷有無,化居。烝民乃粒,萬邦作乂。」皋陶曰:「俞!師汝昌言。」

禹曰:「都!帝,慎乃在位。」帝曰:「俞!禹曰:「安汝止,惟幾惟康。其弼直,惟動丕應。徯志以昭受上帝,天其申命用休。」

帝曰:「吁!臣哉鄰哉!鄰哉臣哉!」禹曰:「俞!」

帝曰:「臣作朕股肱耳目。予欲左右有民,汝翼。予欲宣力四方,汝為。予欲觀古人之象,日、月、星辰、山、龍、華蟲作會;宗彝、藻、火、粉米、黼、黻,絺繡,以五采彰施于五色,作服,汝明。予欲聞六律五聲八音,在治忽,以出納五言,汝聽。予違,汝弼,汝無面從,退有後言。欽四鄰!庶頑讒說,若不在時,侯以明之,撻以記之,書用識哉,欲並生哉!工以納言,時而颺之,格則承之庸之,否則威之。」

禹曰:「俞哉!帝光天之下,至于海隅蒼生,萬邦黎獻,共惟帝臣,惟帝時舉。敷納以言,明庶以功,車服以庸。誰敢不讓,敢不敬應?帝不時敷,同,日奏,罔功。」

帝曰:1「無若丹朱傲,惟慢遊是好,傲虐是作。罔晝夜頟頟,罔水行舟。朋淫于家,用殄厥世。予創若時,娶于塗山,辛壬癸甲。啟呱呱而泣,予弗子,惟荒度土功。弼成五服,至于五千。州十有二師,外薄四海,咸建五長,各迪有功,苗頑弗即工,帝其念哉!」帝曰:「迪朕德,時乃功,惟敘。」皋陶方祗厥敘,方施象刑,惟明。1. 帝曰: : 舊脫。 孫星衍《尚書今古文注疏》

夔曰:「戛擊鳴球、搏拊、琴、瑟、以詠。」祖考來格,虞賓在位,群后德讓。下管鼗鼓,合止柷敔,笙鏞以閒。鳥獸蹌蹌;《簫韶》九成,鳳皇來儀。夔曰:「於!予擊石拊石,百獸率舞,庶尹允諧。」

帝庸作歌,曰:「勑天之命,惟時惟幾。」乃歌曰:「股肱喜哉!元首起哉!百工熙哉!」皋陶拜手稽首颺言曰:「念哉!率作興事,慎乃憲,欽哉!屢省乃成,欽哉!」乃賡載歌曰:「元首明哉,股肱良哉,庶事康哉!」又歌曰:「元首叢脞哉,股肱惰哉,萬事墮哉!」帝拜曰:「俞,往欽哉!」


\end{pinyinscope}