\article{酒誥}

\begin{pinyinscope}
王若曰:「明大命于妹邦。乃穆考文王肇國在西土。厥誥毖庶邦庶士越少正御事,朝夕曰:『祀茲酒。惟天降命,肇我民,惟元祀。天降威,我民用大亂喪德,亦罔非酒惟行;越小大邦用喪,亦罔非酒惟辜。』

文王誥教小子有正有事:無彝酒。越庶國:飲惟祀,德將無醉。惟曰我民迪小子惟土物愛,厥心臧。聰聽祖考之遺訓,越小大德,小子惟一。

妹土,嗣爾股肱,純其藝黍稷,奔走事厥考厥長。肇牽車牛,遠服賈用。孝養厥父母,厥父母慶,自洗腆,致用酒。

庶士、有正越庶伯、君子,其爾典聽朕教!爾大克羞耇惟君,爾乃飲食醉飽。丕惟曰爾克永觀省,作稽中德,爾尚克羞饋祀。爾乃自介用逸,茲乃允惟王正事之臣。茲亦惟天若元德,永不忘在王家。」

王曰:「封,我西土棐徂,邦君御事小子尚克用文王教,不腆于酒,故我至于今,克受殷之命。」

王曰:「封,我聞惟曰:『在昔殷先哲王迪畏天顯小民,經德秉哲。自成湯咸至于帝乙,成王畏相惟御事,厥棐有恭,不敢自暇自逸,矧曰其敢崇飲?越在外服,侯甸男衛邦伯,越在內服,百僚庶尹惟亞惟服宗工越百姓里居,罔敢湎于酒。不惟不敢,亦不暇,惟助成王德顯越,尹人祗辟。』

我聞亦惟曰:『在今後嗣王,酣,身厥命,罔顯于民祗,保越怨不易。誕惟厥縱,淫泆于非彝,用燕喪威儀,民罔不衋傷心。惟荒腆于酒,不惟自息乃逸,厥心疾很,不克畏死。辜在商邑,越殷國滅,無罹。弗惟德馨香祀,登聞于天;誕惟民怨,庶群自酒,腥聞在上。故天降喪于殷,罔愛于殷,惟逸。天非虐,惟民自速辜。』」

王:「封,予不惟若茲多誥。古人有言曰:『人無於水監,當於民監。』今惟殷墜厥命,我其可不大監撫于時!

予惟曰:「汝劼毖殷獻臣、侯、甸、男、衛,矧太史友、內史友、越獻臣百宗工,矧惟爾事服休,服采,矧惟若疇,圻父薄違,農夫若保,宏父定辟,矧汝,剛制于酒。』

厥或誥曰:『群飲。』汝勿佚。盡執拘以歸于周,予其殺。又惟殷之迪諸臣惟工,乃湎于酒,勿庸殺之,姑惟教之。有斯明享,乃不用我教辭,惟我一人弗恤弗蠲,乃事時同于殺。」

王曰:「封,汝典聽朕毖,勿辯乃司民湎于酒。」


\end{pinyinscope}