\article{呂刑}

\begin{pinyinscope}
呂命穆王訓夏贖刑,作《呂刑》。

惟呂命,王享國百年,耄,荒度作刑,以詰四方。

王曰:「若古有訓,蚩尤惟始作亂,延及于平民,罔不寇賊,鴟義,奸宄,奪攘,矯虔。苗民弗用靈,制以刑,惟作五虐之刑曰法。殺戮無辜,爰始淫為劓、刵、椓、黥。越茲麗刑並制,罔差有辭。民興胥漸,泯泯棼棼,罔中于信,以覆詛盟。虐威庶戮,方告無辜于上。上帝監民,罔有馨香德,刑發聞惟腥。皇帝哀矜庶戮之不辜,報虐以威,遏絕苗民,無世在下。乃命重、黎,絕地天通,罔有降格。群后之逮在下,明明棐常,鰥寡無蓋。

皇帝清問下民鰥寡有辭于苗。德威惟畏,德明惟明。乃命三后,恤功于民。伯夷降典,折民惟刑;禹平水土,主名山川;稷降播種,家殖嘉谷。三后成功,惟殷于民。士制百姓于刑之中,以教祗德。穆穆在上,明明在下,灼于四方,罔不惟德之勤,故乃明于刑之中,率乂于民棐彝。典獄非訖于威,惟訖于富。敬忌,罔有擇言在身。惟克天德,自作元命,配享在下。」

王曰:「嗟!四方司政典獄,非爾惟作天牧?今爾何監?非時伯夷播刑之迪?其今爾何懲?惟時苗民匪察于獄之麗,罔擇吉人,觀于五刑之中;惟時庶威奪貨,斷制五刑,以亂無辜,上帝不蠲,降咎于苗,苗民無辭于罰,乃絕厥世。」

王曰:「嗚呼!念之哉。伯父、伯兄、仲叔、季弟、幼子、童孫,皆聽朕言,庶有格命。今爾罔不由慰曰勤,爾罔或戒不勤。天齊于民,俾我一日,非終惟終,在人。爾尚敬逆天命,以奉我一人!雖畏勿畏,雖休勿休。惟敬五刑,以成三德。一人有慶,兆民賴之,其寧惟永。」

王曰:「吁!來,有邦有土,告爾祥刑。在今爾安百姓,何擇,非人?何敬,非刑?何度,非及?兩造具備,師聽五辭。五辭簡孚,正于五刑。五刑不簡,天于五罰;五罰不服,正于五過。五過之疵:惟官,惟反,惟內,惟貨,惟來。其罪惟均,其審克之!

五刑之疑有赦,五罰之疑有赦,其審克之!簡孚有眾,惟貌有稽。無簡不聽,具嚴天威。墨辟疑赦,其罰百鍰,閱實其罪。劓辟疑赦,其罪惟倍,閱實其罪。剕辟疑赦,其罰倍差,閱實其罪。宮辟疑赦,其罰六百鍰,閱實其罪。大辟疑赦,其罰千鍰,閱實其罪。墨罰之屬千。劓罰之屬千,剕罰之屬五百,宮罰之屬三百,大辟之罰其屬二百。五刑之屬三千。

上下比罪,無僭亂辭,勿用不行,惟察惟法,其審克之!上刑適輕,下服;下刑適重,上服。輕重諸罰有權。刑罰世輕世重,惟齊非齊,有倫有要。罰懲非死,人極于病。非佞折獄,惟良折獄,罔非在中。察辭于差,非從惟從。哀敬折獄,明啟刑書胥占,咸庶中正。其刑其罰,其審克之。獄成而孚,輸而孚。其刑上備,有並兩刑。」

王曰:「嗚呼!敬之哉!官伯族姓,朕言多懼。朕敬于刑,有德惟刑。今天相民,作配在下。明清于單辭,民之亂,罔不中聽獄之兩辭,無或私家于獄之兩辭!獄貨非寶,惟府辜功,報以庶尤。永畏惟罰,非天不中,惟人在命。天罰不極,庶民罔有令政在于天下。」

王曰:「嗚呼!嗣孫,今往何監,非德?于民之中,尚明聽之哉!哲人惟刑,無疆之辭,屬于五極,咸中有慶。受王嘉師,監于茲祥刑。」


\end{pinyinscope}