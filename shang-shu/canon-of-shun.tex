\article{舜典}

\begin{pinyinscope}
虞舜側微,堯聞之聰明,將使嗣位,歷試諸難,作《舜典》。

曰若稽古帝舜,曰重華協于帝。濬哲文明,溫恭允塞,玄德升聞,乃命以位。

慎徽五典,五典克從;納于百揆,百揆時敘;賓于四門,四門穆穆;納于大麓,烈風雷雨弗迷。帝曰:「格!汝舜。詢事考言,乃言厎可績,三載。汝陟帝位。」舜讓于德,弗嗣。正月上日,受終于文祖。

在璿璣玉衡,以齊七政。肆類于上帝,禋于六宗,望于山川,徧于群神。輯五瑞。既月乃日,覲四岳群牧,班瑞于群后。

歲二月,東巡守,至于岱宗,柴。望秩于山川,肆覲東后。協時月正日,同律度量衡。修五禮、五玉、三帛、二生、一死贄。如五器,卒乃復。五月南巡守,至于南岳,如岱禮。八月西巡守,至于西岳,如初。十有一月朔巡守,至于北岳,如西禮。歸,格于藝祖,用特。五載一巡守,群后四朝。敷奏以言,明試以功,車服以庸。

肇十有二州,封十有二山,濬川。

象以典刑,流宥五刑,鞭作官刑,扑作教刑,金作贖刑。眚災肆赦,怙終賊刑。欽哉,欽哉,惟刑之恤哉!流共工于幽洲,放驩兜于崇山,竄三苗于三危,殛鯀于羽山,四罪而天下咸服。

二十有八載,帝乃殂落。百姓如喪考妣,三載,四海遏密八音。

月正元日,舜格于文祖,詢于四岳,闢四門,明四目,達四聰。「咨,十有二牧!」曰,「食哉惟時!柔遠能邇,惇德允元,而難任人,蠻夷率服。」

舜曰:「咨,四岳!有能奮庸熙帝之載,使宅百揆亮采,惠疇?」僉曰:「伯禹作司空。」帝曰:「俞,咨!禹,汝平水土,惟時懋哉!」禹拜稽首,讓于稷、契暨皋陶。帝曰:「俞,汝往哉!」

帝曰:「棄,黎民阻飢,汝后稷,播時百穀。」

帝曰:「契,百姓不親,五品不遜。汝作司徒,敬敷五教,在寬。」

帝曰:「皋陶,蠻夷猾夏,寇賊姦宄。汝作士,五刑有服,五服三就。五流有宅,五宅三居。惟明克允!」

帝曰:「疇若予工?」僉曰:「垂哉!」帝曰:「俞,咨!垂,汝共工。」垂拜稽首,讓于殳斨暨伯與。」帝曰:「俞,往哉!汝諧。」

帝曰:「疇若予上下草木鳥獸?」僉曰:「益哉!」帝曰:「俞,咨!益,汝作朕虞。」益拜稽首,讓于朱虎、熊羆。帝曰:「俞,往哉!汝諧。」

帝曰:「咨!四岳,有能典朕三禮?」僉曰:「伯夷!」帝曰:「俞,咨!伯,汝作秩宗。夙夜惟寅,直哉惟清。」伯拜稽首,讓于夔、龍。帝曰:「俞,往,欽哉!」

帝曰:「夔!命汝典樂,教冑子,直而溫,寬而栗,剛而無虐,簡而無傲。詩言志,歌永言,聲依永,律和聲。八音克諧,無相奪倫,神人以和。」夔曰:「於!予擊石拊石,百獸率舞。」

帝曰:「龍,朕堲讒說殄行,震驚朕師。命汝作納言,夙夜出納朕命,惟允!」

帝曰:「咨!汝二十有二人,欽哉!惟時亮天功。」

三載考績,三考,黜陟幽明,庶績咸熙。分北三苗。

舜生三十徵庸,三十在位。五十載,陟方乃死。

帝釐下土,方設居方,別生分類。作《汨作》、《九共》九篇、《槀飫》。


\end{pinyinscope}