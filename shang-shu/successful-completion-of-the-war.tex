\article{武成}

\begin{pinyinscope}
武王伐殷。往伐歸獸,識其政事,作《武成》。

惟一月壬辰,旁死魄。越翼日,癸巳,王朝步自周,于征伐商。

厥四月,哉生明,王來自商,至于豐。乃偃武修文,歸馬于華山之陽,放牛于桃林之野,示天下弗服。

丁未,祀于周廟,邦甸、侯、衛,駿奔走,執豆、籩。越三日,庚戌,柴、望,大告武成。

既生魄,庶邦塚君暨百工,受命于周。

王若曰:「嗚呼,群后!惟先王建邦啟土,公劉克篤前烈,至于大王肇基王跡,王季其勤王家。我文考文王克成厥勳,誕膺天命,以撫方夏。大邦畏其力,小邦懷其德。惟九年,大統未集,予小子其承厥志。

厎商之罪,告于皇天、后土、所過名山、大川,曰:『惟有道曾孫周王發,將有大正于商。今商王受無道,暴殄天物,害虐烝民,為天下逋逃主,萃淵藪。予小子既獲仁人,敢祗承上帝,以遏亂略。華夏蠻貊,罔不率俾。恭天成命,肆予東征,綏厥士女。惟其士女,篚厥玄黃,昭我周王。天休震動,用附我大邑周。惟爾有神,尚克相予以濟兆民,無作神羞!」

既戊午,師逾孟津。癸亥,陳于商郊,俟天休命。甲子昧爽,受率其旅若林,會于牧野。罔有敵于我師,前徒倒戈,攻于後以北,血流漂杵。一戎衣,天下大定。乃反商政,政由舊。釋箕子囚,封比干墓,式商容閭。散鹿臺之財,發鉅橋之粟,大賚于四海,而萬姓悅服。

列爵惟五,分土惟三。建官惟賢,位事惟能。重民五教,惟食、喪、祭。惇信明義,崇德報功。垂拱而天下治。


\end{pinyinscope}