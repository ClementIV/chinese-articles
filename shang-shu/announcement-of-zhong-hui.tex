\article{仲虺之誥}

\begin{pinyinscope}
湯歸自夏,至于大坰,仲虺作誥。

成湯放桀于南巢,惟有慚德。曰:「予恐來世以台為口實。」

仲虺乃作誥,曰:「嗚呼!惟天生民有欲,無主乃亂,惟天生聰明時乂,有夏昏德,民墜塗炭,天乃錫王勇智,表正萬邦,纘禹舊服。茲率厥典,奉若天命。夏王有罪,矯誣上天,以布命于下。帝用不臧,式商受命,用爽厥師。簡賢附勢,寔繁有徒。肇我邦于有夏,若苗之有莠,若粟之有秕。小大戰戰,罔不懼于非辜。矧予之德,言足聽聞。惟王不邇聲色,不殖貨利。德懋懋官,功懋懋賞。用人惟己,改過不吝。克寬克仁,彰信兆民。乃葛伯仇餉,初征自葛,東征,西夷怨;南征,北狄怨,曰:『奚獨後予?』攸徂之民,室家相慶,曰:『徯予后,后來其蘇。』民之戴商,厥惟舊哉!佑賢輔德,顯忠遂良,兼弱攻昧,取亂侮亡,推亡固存,邦乃其昌。德日新,萬邦惟懷;志自滿,九族乃離。王懋昭大德,建中于民,以義制事,以禮制心,垂裕後昆。予聞曰:『能自得師者王,謂人莫已若者亡。好問則裕,自用則小』。嗚呼!慎厥終,惟其始。殖有禮,覆昏暴。欽崇天道,永保天命。」


\end{pinyinscope}