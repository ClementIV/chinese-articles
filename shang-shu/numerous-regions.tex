\article{多方}

\begin{pinyinscope}
成王歸自奄,在宗周,誥庶邦,作《多方》。

惟五月丁亥,王來自奄,至于宗周。

周公曰:王若曰:「猷告爾四國多方,惟爾殷侯尹民。我惟大降爾命,爾罔不知。洪惟圖天之命,弗永寅念于祀,惟帝降格于夏。有夏誕厥逸,不肯慼言于民,乃大淫昏,不克終日勸于帝之迪,乃爾攸聞。厥圖帝之命,不克開于民之麗,乃大降罰,崇亂有夏。因甲于內亂,不克靈承于旅。罔丕惟進之恭,洪舒于民。亦惟有夏之民叨懫日欽,劓割夏邑。天惟時求民主,乃大降顯休命于成湯,刑殄有夏。惟天不畀純,乃惟以爾多方之乂民不克永于多享。惟夏之恭多士大不克明保享于民,乃胥惟虐于民,至于百為,大不克開。乃惟成湯克以爾多方簡,代夏作民主。慎厥麗,乃勸。厥民刑,用勸。以至于帝乙,罔不明德慎罰,亦克用勸。要囚殄戮多罪,亦克用勸。開釋無辜,亦克用勸。今至于爾辟,弗克以爾多方享天之命,嗚呼!」

王若曰:「誥告爾多方,非天庸釋有夏,非天庸釋有殷。乃惟爾辟以爾多方大淫,圖天之命屑有辭。乃惟有夏圖厥政,不集于享,天降時喪,有邦間之。乃惟爾商後王逸厥逸,圖厥政不蠲烝,天惟降時喪。

惟聖罔念作狂,惟狂克念作聖。天惟五年須暇之子孫,誕作民主,罔可念聽。天惟求爾多方,大動以威,開厥顧天。惟爾多方罔堪顧之。惟我周王靈承于旅,克堪用德,惟典神天。天惟式教我用休,簡畀殷命,尹爾多方。

今我曷敢多誥?我惟大降爾四國民命。爾曷不忱裕之于爾多方?爾曷不夾介乂我周王享天之命?今爾尚宅爾宅,畋爾田,爾曷不惠王熙天之命?爾乃迪屢不靜,爾心未愛。爾乃不大宅天命,爾乃屑播天命,爾乃自作不典,圖忱于正。我惟時其教告之,我惟時其戰要囚之,至于再,至于三。乃有不用我降爾命,我乃其大罰殛之!非我有周秉德不康寧,乃惟爾自速辜。」

王曰:「嗚呼!猷告爾有方多士暨殷多士,今爾奔走臣我監五祀,越惟有胥伯小大多正,爾罔不克臬。自作不和,爾惟和哉!爾室不睦,爾惟和哉!爾邑克明,爾惟克勤乃事。爾尚不忌于凶德,亦則以穆穆在乃位,克閱于乃邑謀介。爾乃自時洛邑,尚永力畋爾田,天惟畀矜爾,我有周惟其大介賚爾,迪簡在王庭。尚爾事,有服在大僚。」

王曰:「嗚呼!多士,爾不克勸忱我命,爾亦則惟不克享,凡民惟曰不享。爾乃惟逸惟頗,大遠王命,則惟爾多方探天之威,我則致天之罰,離逖爾土。」

王曰:「我不惟多誥,我惟祇告爾命。」又曰:「時惟爾初,不克敬于和,則無我怨。」


\end{pinyinscope}