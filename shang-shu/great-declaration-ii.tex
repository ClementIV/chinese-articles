\article{泰誓中}

\begin{pinyinscope}
惟戊午,王次于河朔,群后以師畢會。王乃徇師而誓曰:「嗚呼!西土有眾,咸聽朕言。

我聞吉人為善,惟日不足。凶人為不善,亦惟日不足。今商王受,力行無度,播棄犁老,暱比罪人。淫酗肆虐,臣下化之,朋家作仇,脅權相滅。無辜籲天,穢德彰聞。

惟天惠民,惟辟奉天。有夏桀弗克若天,流毒下國。天乃佑命成湯,降黜夏命。惟受罪浮于桀。剝喪元良,賊虐諫輔。謂己有天命,謂敬不足行,謂祭無益,謂暴無傷。厥監惟不遠,在彼夏王。天其以予乂民,朕夢協朕卜,襲于休祥,戎商必克。

受有億兆夷人,離心離德。予有亂臣十人,同心同德。雖有周親,不如仁人。天視自我民視,天聽自我民聽。百姓有過,在予一人,今朕必往。我武維揚,侵于之疆,取彼凶殘。我伐用張,于湯有光。勖哉夫子!罔或無畏,寧執非敵。百姓懍懍,若崩厥角。嗚呼!乃一德一心,立定厥功,惟克永世。」


\end{pinyinscope}