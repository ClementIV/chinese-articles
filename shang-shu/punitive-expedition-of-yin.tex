\article{胤征}

\begin{pinyinscope}
羲和湎淫,廢時亂日,胤往征之,作《胤征》。

惟仲康肇位四海,胤侯命掌六師。羲和廢厥職,酒荒于厥邑,胤后承王命徂征。告于眾曰:「嗟予有眾,聖有謨訓,明徵定保,先王克謹天戒,臣人克有常憲,百官修輔,厥后惟明明,每歲孟春,遒人以木鐸徇于路,官師相規,工執藝事以諫,其或不恭,邦有常刑。」「惟時羲和顛覆厥德,沈亂于酒,畔官離次,俶擾天紀,遐棄厥司,乃季秋月朔,辰弗集于房,瞽奏鼓,嗇夫馳,庶人走,羲和尸厥官罔聞知,昏迷于天象,以干先王之誅,《政典》曰:『先時者殺無赦,不及時者殺無赦。』今予以爾有眾,奉將天罰。爾眾士同力王室,尚弼予欽承天子威命。火炎崑岡,玉石俱焚。天吏逸德,烈于猛火。殲厥渠魁,脅從罔治,舊染污俗,咸與維新。嗚呼!威克厥愛,允濟;愛克厥威,允罔功。其爾眾士懋戒哉!」

自契至于成湯八遷,湯始居亳,從先王居。作《帝告》、《釐沃》。

湯征諸侯,葛伯不祀,湯始征之,作《湯征》。

伊尹去亳適夏,既丑有夏,復歸于亳。入自北門,乃遇汝鳩、汝方。作《汝鳩》、《汝方》。


\end{pinyinscope}