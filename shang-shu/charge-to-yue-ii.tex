\article{說命中}

\begin{pinyinscope}
惟說命總百官,乃進于王曰:「嗚呼!明王奉若天道,建邦設都,樹后王君公,承以大夫師長,不惟逸豫,惟以亂民。惟天聰明,惟聖時憲,惟臣欽若,惟民從乂。惟口起羞,惟甲冑起戎,惟衣裳在笥,惟干戈省厥躬。王惟戒茲,允茲克明,乃罔不休。惟治亂在庶官。官不及私昵,惟其能;爵罔及惡德,惟其賢。慮善以動,動惟厥時。有其善,喪厥善;矜其能,喪厥功。惟事事,乃其有備,有備無患。無啟寵納侮,無恥過作非。惟厥攸居,政事惟醇。黷予祭祀,時謂弗欽。禮煩則亂,事神則難。」

王曰:「旨哉!說。乃言惟服。乃不良于言,予罔聞于行。」說拜稽首曰:「非知之艱,行之惟艱。王忱不艱,允協于先王成德,惟說不言有厥咎。」


\end{pinyinscope}