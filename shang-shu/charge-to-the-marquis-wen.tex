\article{文侯之命}

\begin{pinyinscope}
平王錫晉文侯秬鬯、圭瓚,作《文侯之命》。

王若曰:「父義和!丕顯文、武,克慎明德,昭升于上,敷聞在下;惟時上帝,集厥命于文王。亦惟先正克左右昭事厥辟,越小大謀猷罔不率從,肆先祖懷在位。嗚呼!閔予小子嗣,造天丕愆。殄資澤于下民,侵戎我國家純。即我御事,罔或耆壽俊在厥服,予則罔克。曰惟祖惟父,其伊恤朕躬!嗚呼!有績予一人永綏在位。父義和!汝克紹乃顯祖,汝肇刑文、武,用會紹乃辟,追孝于前文人。汝多修,扞我于艱,若汝,予嘉。」

王曰:「父義和!其歸視爾師,寧爾邦。用賚爾秬一鬯卣,彤弓一,彤矢百,盧弓一,盧矢百,馬四匹。父往哉!柔遠能邇,惠康小民,無荒寧。簡恤爾都,用成爾顯德。」


\end{pinyinscope}