\article{太甲中}

\begin{pinyinscope}
惟三祀十有二月朔,伊尹以冕服奉嗣王歸于亳,作書曰:「民非后,罔克胥匡以生;后非民,罔以辟四方。皇天眷佑有商,俾嗣王克終厥德,實萬世無疆之休。」

王拜手稽首曰:「予小子不明于德,自厎1不類。欲敗度,縱敗禮,以速戾于厥躬。天作孽,猶可違;自作孽,不可逭。既往背師保之訓,弗克于厥初,尚賴匡救之德,圖惟厥終。」1. 厎 : 原作「底」。《尚書正義》曰:「類,善也。闇於德,故自致不善。」又曰:「底,之履反」。

伊尹拜手稽首曰:「修厥身,允德協于下,惟明后。先王子惠困窮,民服厥命,罔有不悅。並其有邦厥鄰,乃曰:『徯我后,后來無罰。』王懋乃德,視乃厥祖,無時豫怠。奉先思孝,接下思恭。視遠惟明;聽德惟聰。朕承王之休無斁。」


\end{pinyinscope}