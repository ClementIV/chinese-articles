\article{堯典}

\begin{pinyinscope}
昔在帝堯,聰明文思,光宅天下。將遜于位,讓于虞舜,作《堯典》。

曰若稽古帝堯,曰放勳,欽、明、文、思、安安,允恭克讓,光被四表,格于上下。克明俊德,以親九族。九族既睦,平章百姓。百姓昭明,協和萬邦。黎民於變時雍。

乃命羲和,欽若昊天,歷象日月星辰,敬授人時。分命羲仲,宅嵎夷,曰暘谷。寅賓出日,平秩東作。日中,星鳥,以殷仲春。厥民析,鳥獸孳尾。申命羲叔,宅南交。平秩南訛,敬致。日永,星火,以正仲夏。厥民因,鳥獸希革。分命和仲,宅西,曰昧谷。寅餞納日,平秩西成。宵中,星虛,以殷仲秋。厥民夷,鳥獸毛毨。申命和叔,宅朔方,曰幽都。平在朔易。日短,星昴,以正仲冬。厥民隩,鳥獸氄毛。帝曰:「咨!汝羲暨和。朞三百有六旬有六日,以閏月定四時,成歲。允釐百工,庶績咸熙。」

帝曰:「疇咨若時登庸?」放齊曰:「胤子朱啟明。」帝曰:「吁!嚚訟可乎?」

帝曰:「疇咨若予采?」驩兜曰:「都!共工方鳩僝功。」帝曰:「吁!靜言庸違,象恭滔天。」

帝曰:「咨!四岳,湯湯洪水方割,蕩蕩懷山襄陵,浩浩滔天。下民其咨,有能俾乂?」僉曰:「於!鯀哉。」帝曰:「吁!咈哉,方命圮族。」岳曰:「异哉!試可乃已。」

帝曰,「往,欽哉!」九載,績用弗成。

帝曰:「咨!四岳。朕在位七十載,汝能庸命,巽朕位?」岳曰:「否德忝帝位。」曰:「明明揚側陋。」師錫帝曰:「有鰥在下,曰虞舜。」帝曰:「俞?予聞,如何?」岳曰:「瞽子,父頑,母嚚,象傲;克諧以孝,烝烝乂,不格姦。」帝曰:「我其試哉!女于時,觀厥刑于二女。」釐降二女于媯汭,嬪于虞。帝曰:「欽哉!」


\end{pinyinscope}