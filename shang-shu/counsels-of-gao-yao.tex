\article{皋陶謨}

\begin{pinyinscope}
曰若稽古。皋陶曰:「允迪厥德,謨明弼諧。」

禹曰:「俞!如何?」

皋陶曰:「都!慎厥身,修思永。惇敘九族,庶明勵翼,邇可遠,在茲。」

禹拜昌言曰:「俞!」

皋陶曰:「都!在知人,在安民。」

禹曰:「吁!咸若時,惟帝其難之。知人則哲,能官人安民則惠。黎民懷之,能哲而惠,何憂乎驩兜?何遷乎有苗?何畏乎巧言令色孔壬?」

皋陶曰:「都!亦行有九德,亦言其人有德,乃言曰,載采采。」

禹曰:「何?」

皋陶曰:「寬而栗。柔而立,愿而恭,亂而敬,擾而毅,直而溫,簡而廉,剛而塞,彊而義,彰厥有常。吉哉!日宣三德,夙夜浚明有家。日嚴祗敬六德,亮采有邦,翕受敷施。九德咸事,俊乂在官。百僚師師,百工惟時。撫于五辰,庶績其凝。無教逸欲。有邦兢兢業業,一日二日萬幾。無曠庶官,天工人其代之。天敘有典,勑我五典五惇哉!天秩有禮,自我五禮有庸哉!同寅協恭和衷哉!天命有德,五服五章哉!天討有罪,五刑五用哉!政事懋哉懋哉!天聰明,自我民聰明,天明畏自我民明威。達于上下,敬哉有土。」

皋陶曰:「朕言惠,可厎1行。」

禹曰:「俞!乃言厎可績。」

皋陶曰:「予未有知,思曰贊贊襄哉!」 1. 厎 : 原作「底」。據下文及《四部叢刊》本等改。


\end{pinyinscope}