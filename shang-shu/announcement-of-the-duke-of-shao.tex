\article{召誥}

\begin{pinyinscope}
成王在豐,欲宅洛邑,使召公先相宅,作《召誥》。

惟二月既望,越六日乙未,王朝步自周,則至于豐。

惟太保先周公相宅,越若來三月,惟丙午朏。越三日戊申,太保朝至于洛,卜宅。厥既得卜,則經營。越三日庚戌,太保乃以庶殷攻位于洛汭。越五日甲寅,位成。

若翼日乙卯,周公朝至于洛,則達觀于新邑營。越三日丁巳,用牲于郊,牛二。越翼日戊午,乃社于新邑,牛一,羊一,豕一。

越七日甲子,周公乃朝用書命庶殷侯甸男邦伯。厥既命殷庶,庶殷丕作。

太保乃以庶邦塚君出取幣,乃復入錫周公。曰:「拜手稽首,旅王若公誥告庶殷越自乃御事:嗚呼!皇天上帝,改厥元子茲大國殷之命。惟王受命,無疆惟休,亦無疆惟恤。嗚呼!曷其奈何弗敬?

天既遐終大邦殷之命,茲殷多先哲王在天,越厥後王后民,茲服厥命。厥終,智藏瘝在。夫知保抱攜持厥婦子,以哀籲天,徂厥亡,出執。嗚呼!天亦哀于四方民,其眷命用懋。王其疾敬德!

相古先民有夏,天迪從子保,面稽天若;今時既墜厥命。今相有殷,天迪格保,面稽天若;今時既墜厥命。今沖子嗣,則無遺壽耇,曰其稽我古人之德,矧曰其有能稽謀自天?

嗚呼!有王雖小,元子哉。其丕能諴于小民。今休:王不敢後,用顧畏于民碞;王來紹上帝,自服于土中。旦曰:『其作大邑,其自時配皇天,毖祀于上下,其自時中乂;王厥有成命治民。』今休,王先服殷御事,比介于我有周御事,節性惟日其邁。王敬作所,不可不敬德。

我不可不監于有夏,亦不可不監于有殷。我不敢知曰,有夏服天命,惟有歷年;我不敢知曰,不其延。惟不敬厥德,乃早墜厥命。我不敢知曰,有殷受天命,惟有歷年;我不敢知曰,不其延。惟不敬厥德,乃早墜厥命。今王嗣受厥命,我亦惟茲二國命,嗣若功。

王乃初服。嗚呼!若生子,罔不在厥初生,自貽哲命。今天其命哲,命吉凶,命歷年;知今我初服,宅新邑。肆惟王其疾敬德?王其德之用,祈天永命。

其惟王勿以小民淫用非彝,亦敢殄戮用乂民,若有功。其惟王位在德元,小民乃惟刑用于天下,越王顯。上下勤恤,其曰我受天命,丕若有夏歷年,式勿替有殷歷年。欲王以小民受天永命。」

拜手稽首,曰:「予小臣敢以王之仇民百君子越友民,保受王威命明德。王末有成命,王亦顯。我非敢勤,惟恭奉幣,用供王能祈天永命。」


\end{pinyinscope}