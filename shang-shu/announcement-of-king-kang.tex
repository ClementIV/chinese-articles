\article{康王之誥}

\begin{pinyinscope}
康王既尸天子,遂誥諸侯,作《康王之誥》。

王出,在應門之內,太保率西方諸侯入應門左,畢公率東方諸侯入應門右,皆布乘黃朱。賓稱奉圭兼幣,曰:「一二臣衛,敢執壤奠。」皆再拜稽首。王義嗣,德答拜。

太保暨芮伯咸進,相揖。皆再拜稽首曰:「敢敬告天子,皇天改大邦殷之命,惟周文武誕受羑若,克恤西土。惟新陟王畢協賞罰,戡定厥功,用敷遺後人休。今王敬之哉!張惶六師,無壞我高祖寡命。」

王若曰:「庶邦侯、甸、男、衛,惟予一人釗報誥。昔君文武丕平,富不務咎,厎至齊信,用昭明于天下。則亦有熊羆之士,不二心之臣,保乂王家,用端命于上帝。皇天用訓厥道,付畀四方。乃命建侯樹屏,在我後之人。今予一二伯父尚胥暨顧,綏爾先公之臣服于先王。雖爾身在外,乃心罔不在王室,用奉恤厥若,無遺鞠子羞!」

群公既皆聽命,相楫,趨出。王釋冕,反喪服。


\end{pinyinscope}