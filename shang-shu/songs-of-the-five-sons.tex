\article{五子之歌}

\begin{pinyinscope}
太康失邦,昆弟五人須于洛汭,作《五子之歌》。

太康尸位,以逸豫滅厥德,黎民咸貳,乃盤遊無度,畋于有洛之表,十旬弗反。有窮后羿因民弗忍,距于河,厥弟五人御其母以從,徯于洛之汭。五子咸怨,述大禹之戒以作歌。

其一曰:「皇祖有訓,民可近,不可下,民惟邦本,本固邦寧。予視天下愚夫愚婦一能勝予,一人三失,怨豈在明,不見是圖。予臨兆民,懍乎若朽索之馭六馬,為人上者,柰何不敬?」

其二曰:「訓有之,內作色荒,外作禽荒。甘酒嗜音,峻宇彫牆。有一于此,未或不亡。」

其三曰:「惟彼陶唐,有此冀方。今失厥道,亂其紀綱,乃厎滅亡。」

其四曰:「明明我祖,萬邦之君。有典有則,貽厥子孫。關石和鈞,王府則有。荒墜厥緒,覆宗絕祀!」

其五曰:「嗚呼曷歸?予懷之悲。萬姓仇予,予將疇依?郁陶乎予心,顏厚有忸怩。弗慎厥德,雖悔可追?」


\end{pinyinscope}