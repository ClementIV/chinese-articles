\article{君陳}

\begin{pinyinscope}
周公既沒,命君陳分正東郊成周,作《君陳》。

王若曰:「君陳,惟爾令德孝恭。惟孝友于兄弟,克施有政。命汝尹茲東郊,敬哉!昔周公師保萬民,民懷其德。往慎乃司,茲率厥常,懋昭周公之訓,惟民其乂。我聞曰:『至治馨香,感于神明。黍稷非馨,明德惟馨爾。』尚式時周公之猷訓,惟日孜孜,無敢逸豫。凡人未見聖,若不克見;既見聖,亦不克由聖,爾其戒哉!爾惟風,下民惟草。圖厥政,莫或不艱,有廢有興,出入自爾師虞,庶言同則繹。爾有嘉謀嘉猷,則入告爾后于內,爾乃順之于外,曰:『斯謀斯猷,惟我后之德。』嗚呼!臣人咸若時,惟良顯哉!」

王曰:「君陳,爾惟弘周公丕訓,無依勢作威,無倚法以削,寬而有制,從容以和。殷民在辟,予曰辟,爾惟勿辟;予曰宥,爾惟勿宥,惟厥中。有弗若于汝政,弗化于汝訓,辟以止辟,乃辟。狃于奸宄,敗常亂俗,三細不宥。爾無忿疾于頑,無求備于一夫。必有忍,其乃有濟;有容,德乃大。簡厥修,亦簡其或不修。進厥良,以率其或不良。惟民生厚,因物有遷。違上所命,從厥攸好。爾克敬典在德,時乃罔不變。允升于大猷,惟予一人膺受多福,其爾之休,終有辭於永世。」


\end{pinyinscope}