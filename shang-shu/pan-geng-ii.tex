\article{盤庚中}

\begin{pinyinscope}
盤庚作,惟涉河以民遷。乃話民之弗率,誕告用亶。其有眾咸造,勿褻在王庭,盤庚乃登進厥民。曰:「明聽朕言,無荒失朕命!嗚呼!古我前后,罔不惟民之承保。后胥戚鮮,以不浮于天時。殷降大虐,先王不懷厥攸作,視民利用遷。汝曷弗念我古后之聞?承汝俾汝惟喜康共,非汝有咎比于罰。予若籲懷茲新邑,亦惟汝故,以丕從厥志。

今予將試以汝遷,安定厥邦。汝不憂朕心之攸困,乃咸大不宣乃心,欽念以忱動予一人。爾惟自鞠自苦,若乘舟,汝弗濟,臭厥載。爾忱不屬,惟胥以沈。不其或稽,自怒曷瘳?汝不謀長以思乃災,汝誕勸憂。今其有今罔後,汝何生在上?今予命汝,一無起穢以自臭,恐人倚乃身,迂乃心。予迓續乃命于天,予豈汝威,用奉畜汝眾。

予念我先神后之勞爾先,予丕克羞爾,用懷爾,然。失于政,陳于茲,高后丕乃崇降罪疾,曰『曷虐朕民?』汝萬民乃不生生,暨予一人猷同心,先后丕降與汝罪疾,曰:『曷不暨朕幼孫有比?』故有爽德,自上其罰汝,汝罔能迪。

「古我先后既勞乃祖乃父,汝共作我畜民,汝有戕則在乃心!我先后綏乃祖乃父,乃祖乃父乃斷棄汝,不救乃死。茲予有亂政同位,具乃貝玉。乃祖乃父丕乃告我高后曰:『作丕刑于朕孫!』迪高后丕乃崇降弗祥。」

嗚呼!今予告汝:不易!永敬大恤,無胥絕遠!汝分猷念以相從,各設中于乃心。乃有不吉不迪,顛越不恭,暫遇姦宄,我乃劓殄滅之,無遺育,無俾易種于茲新邑。

往哉!生生!今予將試以汝遷,永建乃家。」


\end{pinyinscope}