\article{君奭}

\begin{pinyinscope}
召公為保,周公為師,相成王為左右。召公不說,周公作《君奭》。

周公若曰:「君奭!弗弔天降喪于殷,殷既墜厥命,我有周既受。我不敢知曰厥基永孚于休。若天棐忱,我亦不敢知曰其終出于不祥。嗚呼!君已曰時我,我亦不敢寧于上帝命,弗永遠念天威越我民;罔尤違,惟人。在我後嗣子孫,大弗克恭上下,遏佚前人光在家,不知天命不易,天難諶,乃其墜命,弗克經歷。嗣前人,恭明德,在今予小子旦非克有正,迪惟前人光施于我沖子。」又曰:「天不可信,我道惟寧王德延,天不庸釋于文王受命。」

公曰:「君奭!我聞在昔成湯既受命,時則有若伊尹,格于皇天。在太甲,時則有若保衡。在太戊,時則有若伊陟、臣扈,格于上帝;巫咸乂王家。在祖乙,時則有若巫賢。在武丁,時則有若甘盤。率惟茲有陳,保乂有殷,故殷禮陟配天,多歷年所。天維純佑命,則商實百姓王人。罔不秉德明恤,小臣屏侯甸,矧咸奔走。惟茲惟德稱,用乂厥辟,故一人有事于四方,若卜筮罔不是孚。」

公曰:「君奭!天壽平格,保乂有殷,有殷嗣,天滅威。今汝永念,則有固命,厥亂明我新造邦。」

公曰:「君奭!在昔上帝割申勸寧王之德,其集大命于厥躬?惟文王尚克修和我有夏;亦惟有若虢叔,有若閎夭,有若散宜生,有若泰顛,有若南宮括。」

又曰:「無能往來,茲迪彝教,文王蔑德降于國人。亦惟純佑秉德,迪知天威,乃惟時昭文王迪見冒,聞于上帝。惟時受有殷命哉。武王惟茲四人尚迪有祿。後暨武王誕將天威,咸劉厥敵。惟茲四人昭武王惟冒,丕單稱德。今在予小子旦,若游大川,予往暨汝奭其濟。小子同未在位,誕無我責收,罔勖不及。耇造德不降我則,鳴鳥不聞,矧曰其有能格?」

公曰:「嗚呼!君肆其監于茲!我受命于疆惟休,亦大惟艱。告君,乃猷裕我,不以後人迷。」

公曰:「前人敷乃心,乃悉命汝,作汝民極。曰:『汝明勖偶王,在但乘茲大命,惟文王德丕承,無疆之恤!』」

公曰:「君!告汝,朕允保奭。其汝克敬以予監于殷喪大否,肆念我天威。予不允惟若茲誥,予惟曰:『襄我二人,汝有合哉?』言曰:『在時二人。』天休茲至,惟時二人弗戡。其汝克敬德,明我俊民,在讓後人于丕時。

嗚呼!篤棐時二人,我式克至于今日休?我咸成文王功于不怠。丕冒海隅出日,罔不率俾。」

公曰:「君!予不惠若茲多誥,予惟用閔于天越民。」

公曰:「嗚呼!君!惟乃知民德亦罔不能厥初,惟其終。祗若茲,往敬用治!」


\end{pinyinscope}