\article{畢命}

\begin{pinyinscope}
康王命作冊畢,分居里,成周郊,作《畢命》。

惟十有二年,六月庚午,朏。越三日壬申,王朝步自宗周,至于豐。以成周之眾,命畢公保釐東郊。

王若曰:「嗚呼!父師,惟文王、武王敷大德于天下,用克受殷命。惟周公左右先王,綏定厥家,毖殷頑民,遷于洛邑,密邇王室,式化厥訓。既歷三紀,世變風移,四方無虞,予一人以寧,道有升降,政由俗革,不臧厥臧,民罔攸勸。惟公懋德,克勤小物,弼亮四世,正色率下,罔不祗師言。嘉績多于先王,予小子垂拱仰成。」

王曰:「嗚呼!父師,今予祗命公以周公之事,往哉!旌別淑慝,表厥宅里,彰善癉惡,樹之風聲。弗率訓典,殊厥井疆,俾克畏慕。申畫郊圻,慎固封守,以康四海。政貴有恆,辭尚體要,不惟好異。商俗靡靡,利口惟賢,余風未殄,公其念哉!我聞曰:『世祿之家,鮮克由禮』。以蕩陵德,實悖天道。敝化奢麗,萬世同流。茲殷庶士,席寵惟舊,怙侈滅義,服美于人。驕淫矜侉,將由惡終。雖收放心,閑之惟艱。資富能訓,惟以永年。惟德惟義,時乃大訓。不由古訓,于何其訓。」

王曰:「嗚呼!父師,邦之安危,惟茲殷士。不剛不柔,厥德允修。惟周公克慎厥始,惟君陳克和厥中,惟公克成厥終。三后協心,同厎于道,道洽政治,澤潤生民,四夷左衽,罔不咸賴,予小子永膺多福。公其惟時成周,建無窮之基,亦有無窮之聞。子孫訓其成式,惟乂。嗚呼!罔曰弗克,惟既厥心;罔曰民寡,惟慎厥事。欽若先王成烈,以休于前政。」


\end{pinyinscope}