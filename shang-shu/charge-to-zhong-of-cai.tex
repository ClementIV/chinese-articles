\article{蔡仲之命}

\begin{pinyinscope}
蔡叔既沒,王命蔡仲,踐諸侯位,作《蔡仲之命》。

惟周公位塚宰,正百工,群叔流言。乃致辟管叔于商;囚蔡叔于郭鄰,以車七乘;降霍叔于庶人,三年不齒。蔡仲克庸只德,周公以為卿士。叔卒,乃命諸王邦之蔡。

王若曰:「小子胡,惟爾率德改行,克慎厥猷,肆予命爾侯于東土。往即乃封,敬哉!爾尚蓋前人之愆,惟忠惟孝;爾乃邁跡自身,克勤無怠,以垂憲乃後;率乃祖文王之遺訓,無若爾考之違王命。皇天無親,惟德是輔。民心無常,惟惠之懷。為善不同,同歸于治;為惡不同,同歸于亂。爾其戒哉!慎厥初,惟厥終,終以不困;不惟厥終,終以困窮。懋乃攸績,睦乃四鄰,以蕃王室,以和兄弟,康濟小民。率自中,無作聰明亂舊章。詳乃視聽,罔以側言改厥度。則予一人汝嘉。」王曰:「嗚呼!小子胡,汝往哉!無荒棄朕命!」

成王東伐淮夷,遂踐奄,作《成王政》。

成王既踐奄,將遷其君於蒲姑,周公告召公,作《將蒲姑》。


\end{pinyinscope}