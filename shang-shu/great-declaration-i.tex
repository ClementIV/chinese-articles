\article{泰誓上}

\begin{pinyinscope}
惟十有一年,武王伐殷。一月戊午,師渡孟津,作《泰誓》三篇。

惟十有三年春,大會于孟津。王曰:「嗟!我友邦塚君越我御事庶士,明聽誓。

惟天地萬物父母,惟人萬物之靈。但聰明,作元后,元后作民父母。今商王受,弗敬上天,降災下民。沈湎冒色,敢行暴虐,罪人以族,官人以世,惟宮室、台榭、陂池、侈服,以殘害于爾萬姓。焚炙忠良,刳剔孕婦。皇天震怒,命我文考,肅將天威,大勳未集。

肆予小子發,以爾友邦塚君,觀政于商。惟受罔有悛心,乃夷居,弗事上帝神祇,遺厥先宗廟弗祀。犧牲粢盛,既于凶盜。乃曰:『吾有民有命!』罔懲其侮。

天佑下民,作之君,作之師,惟其克相上帝,寵綏四方。有罪無罪,予曷敢有越厥志?

同力,度德;同德,度義。受有臣億萬,惟億萬心;予有臣三千,惟一心。商罪貫盈,天命誅之。予弗順天,厥罪惟鈞。

予小子夙夜祗懼,受命文考,類于上帝,宜于塚土,以爾有眾,厎天之罰。天矜于民,民之所欲,天必從之。爾尚弼予一人,永清四海,時哉弗可失!」


\end{pinyinscope}