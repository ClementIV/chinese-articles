\article{秦誓}

\begin{pinyinscope}
秦穆公伐鄭,晉襄公帥師敗諸崤,還歸,作《秦誓》。

公曰:「嗟!我士,聽無嘩!予誓告汝群言之首。古人有言曰:『民訖自若,是多盤。』責人斯無難,惟受責俾如流,是惟艱哉!我心之憂,日月逾邁,若弗雲來。

惟古之謀人,則曰未就予忌;惟今之謀人,姑將以為親。雖則云然,尚猷詢茲黃髮,則罔所愆。」番番良士,旅力既愆,我尚有之;仡仡勇夫,射御不違,我尚不欲。惟截截善諞言,俾君子易辭,我皇多有之!

昧昧我思之,如有一介臣,斷斷猗無他技,其心休休焉,其如有容。人之有技,若己有之。人之彥聖,其心好之,不啻若自其口出。是能容之,以保我子孫黎民,亦職有利哉!人之有技,冒疾以惡之;人之彥聖而違之,俾不達是不能容,以不能保我子孫黎民,亦曰殆哉!

邦之杌隉,曰由一人;邦之榮懷,亦尚一人之慶。」


\end{pinyinscope}