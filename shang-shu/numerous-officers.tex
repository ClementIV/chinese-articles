\article{多士}

\begin{pinyinscope}
成周既成,遷殷頑民,周公以王命誥,作《多士》。

惟三月,周公初于新邑洛,用告商王士。

王若曰:「爾殷遺多士,弗弔旻天,大降喪于殷,我有周佑命,將天明威,致王罰,敕殷命終于帝。肆爾多士!非我小國敢弋殷命。惟天不畀允罔固亂,弼我,我其敢求位?惟帝不畀,惟我下民秉為,惟天明畏。

我聞曰:「上帝引逸,有夏不適逸;則惟帝降格,向于時夏。弗克庸帝,大淫泆有辭。惟時天罔念聞,厥惟廢元命,降致罰;乃命爾先祖成湯革夏,俊民甸四方。自成湯至于帝乙,罔不明德恤祀。亦惟天丕建,保乂有殷,殷王亦罔敢失帝,罔不配天其澤。在今後嗣王,誕罔顯于天,矧曰其有聽念于先王勤家?誕淫厥泆,罔顧于天顯民祗,惟時上帝不保,降若茲大喪。惟天不畀不明厥德,凡四方小大邦喪,罔非有辭于罰。」

王若曰:「爾殷多士,今惟我周王丕靈承帝事,有命曰:『割殷,』告敕于帝。惟我事不貳適,惟爾王家我適。予其曰惟爾洪無度,我不爾動,自乃邑。予亦念天,即于殷大戾,肆不正。」

王曰:「猷!告爾多士,予惟時其遷居西爾,非我一人奉德不康寧,時惟天命。無違,朕不敢有後,無我怨。

惟爾知,惟殷先人有冊有典,殷革夏命。今爾又曰:『夏迪簡在王庭,有服在百僚。』予一人惟聽用德,肆予敢求爾于天邑商,予惟率肆矜爾。非予罪,時惟天命。」

王曰:「多士,昔朕來自奄,予大降爾四國民命。我乃明致天罰,移爾遐逖,比事臣我宗多遜。」

王曰:「告爾殷多士,今予惟不爾殺,予惟時命有申。今朕作大邑于茲洛,予惟四方罔攸賓,亦惟爾多士攸服奔走臣我多遜。爾乃尚有爾土,爾用尚寧干止,爾克敬,天惟畀矜爾;爾不克敬,爾不啻不有爾土,予亦致天之罰于爾躬!今爾惟時宅爾邑,繼爾居;爾厥有干有年于茲洛。爾小子乃興,從爾遷。」

王曰:「又曰時予,乃或言爾攸居。」


\end{pinyinscope}