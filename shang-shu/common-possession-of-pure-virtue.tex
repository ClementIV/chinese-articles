\article{咸有一德}

\begin{pinyinscope}
伊尹作《咸有一德》。

伊尹既復政厥辟,將告歸,乃陳戒于德。曰:「嗚呼!天難諶,命靡常。常厥德,保厥位。厥德匪常,九有以亡。夏王弗克庸德,慢神虐民。皇天弗保,監于萬方,啟迪有命,眷求一德,俾作神主。惟尹躬暨湯,咸有一德,克享天心,受天明命,以有九有之師,爰革夏正。非天私我有商,惟天祐于一德;非商求于下民,惟民歸于一德。德惟一,動罔不吉;德二三,動罔不凶。惟吉凶不僭在人,惟天降災祥在德。今嗣王新服厥命,惟新厥德。終始惟一,時乃日新。任官惟賢材,左右惟其人。臣為上為德,為下為民。其難其慎,惟和惟一。德無常師,主善為師。善無常主,協于克一。俾萬姓咸曰:『大哉王言。』又曰:『一哉王心』。克綏先王之祿,永厎烝民之生。嗚呼!七世之廟,可以觀德。萬夫之長,可以觀政。后非民罔使;民非后罔事。無自廣以狹人,匹夫匹婦,不獲自盡,民主罔與成厥功。」

沃丁既葬伊尹于亳,咎單遂訓伊尹事,作《沃丁》。

伊陟相大戊,亳有祥桑谷共生于朝。伊陟贊于巫咸,作《咸乂》四篇。

太戊贊于伊陟,作《伊陟》、《原命》。

仲丁遷于囂,作《仲丁》。

河但甲居相,作《河但甲》。

祖乙圯于耿,作《祖乙》。


\end{pinyinscope}