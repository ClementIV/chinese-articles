\article{伊訓}

\begin{pinyinscope}
成湯既沒,太甲元年,伊尹作《伊訓》、《肆命》、《徂后》。

惟元祀十有二月乙丑,伊尹祠于先王。奉嗣王祗見厥祖,侯、甸群后咸在,百官總已以聽冢宰。伊尹乃明言烈祖之成德,以訓于王。

曰:「嗚呼!古有夏先后,方懋厥德,罔有天災。山川鬼神,亦莫不寧,暨鳥獸魚鱉咸若。于其子孫弗率,皇天降災,假手于我有命,造攻自鳴條,朕哉自亳。惟我商王,布昭聖武,代虐以寬,兆民允懷。今王嗣厥德,罔不在初,立愛惟親,立敬惟長,始于家邦,終于四海。

嗚呼!先王肇修人紀,從諫弗咈,先民時若。居上克明,為下克忠,與人不求備,檢身若不及,以至于有萬邦,茲惟艱哉!敷求哲人,俾輔于爾後嗣,制官刑,儆于有位。曰:『敢有恆舞于宮,酣歌于室,時謂巫風,敢有殉于貨色,恆于游畋,時謂淫風。敢有侮聖言,逆忠直,遠耆德,比頑童,時謂亂風。惟茲三風十愆,卿士有一于身,家必喪;邦君有一于身,國必亡。臣下不匡,其刑墨,具訓于蒙士。』

嗚呼!嗣王祗厥身,念哉!聖謨洋洋,嘉言孔彰。惟上帝不常,作善降之百祥,作不善降之百殃。爾惟德罔小,萬邦惟慶;爾惟不德罔大,墜厥宗。」


\end{pinyinscope}