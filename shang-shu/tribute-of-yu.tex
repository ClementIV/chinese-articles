\article{禹貢}

\begin{pinyinscope}
禹別九州,隨山濬川,任土作貢。

禹敷土,隨山刊木,奠高山大川。

冀州:既載壺口,治梁及岐。既修太原,至于岳陽;覃懷厎績,至于衡漳。厥土惟白壤,厥賦惟上上錯,厥田惟中中。恆、衛既從,大陸既作。島夷皮服,夾右碣石入于河。

濟河惟兗州。九河既道,雷夏既澤,灉、沮會同。桑土既蠶,是降丘宅土。厥土黑墳,厥草惟繇,厥木惟條。厥田惟中下,厥賦貞,作十有三載乃同。厥貢漆絲,厥篚織文。浮于濟、漯,達于河。

海岱惟青州。嵎夷既略,濰、淄其道。厥土白墳,海濱廣斥。厥田惟上下,厥賦中上。厥貢鹽絺,海物惟錯。岱畎絲、枲、鉛、松、怪石。萊夷作牧。厥篚檿絲。浮于汶,達于濟。

海、岱及淮惟徐州。淮、沂其乂,蒙、羽其藝,大野既豬,東原厎平。厥土赤埴墳,草木漸包。厥田惟上中,厥賦中中。厥貢惟土五色,羽畎夏翟,嶧陽孤桐,泗濱浮磬,淮夷蠙珠暨魚。厥篚玄纖、縞。浮于淮、泗,達于河。

淮海惟揚州。彭蠡既豬,陽鳥攸居。三江既入,震澤厎定。篠、簜既敷,厥草惟夭,厥木惟喬。厥土惟塗泥。厥田唯下下,厥賦下上,上錯。厥貢惟金三品,瑤、琨、篠、簜、齒、革、羽、毛惟木。鳥夷卉服。厥篚織貝,厥包橘柚,錫貢。沿于江、海,達于淮、泗。

荊及衡陽惟荊州。江、漢朝宗于海,九江孔殷,沱、潛既道,雲土、夢作乂。厥土惟塗泥,厥田惟下中,厥賦上下。厥貢羽、毛、齒、革惟金三品,杶、榦、栝、柏,礪、砥、砮、丹,惟菌、簵、楛;三邦厎貢厥名。包匭菁茅,厥篚玄纁璣組,九江納錫大龜。浮于江、沱、潛、漢,逾于洛,至于南河。

荊河惟豫州。伊、洛、瀍、澗既入于河,滎波既豬。導菏澤,被孟豬。厥土惟壤,下土墳壚。厥田惟中上,厥賦錯上中。厥貢漆、枲,絺、紵,厥篚纖、纊,錫貢磬錯。浮于洛,達于河。

華陽、黑水惟梁州。岷、嶓既藝,沱、潛既道。蔡、蒙旅平,和夷厎績。厥土青黎,厥田惟下上,厥賦下中,三錯。厥貢璆、鐵、銀、鏤、砮磬、熊、羆、狐、狸、織皮,西傾因桓是來,浮于潛,逾于沔,入于渭,亂于河。

黑水、西河惟雍州。弱水既西,涇屬渭汭,漆沮既從,灃水攸同。荊、岐既旅,終南、惇物,至于鳥鼠。原隰厎績,至于豬野。三危既宅,三苗丕敘。厥土惟黃壤,厥田惟上上,厥賦中下。厥貢惟球、琳、琅玕。浮于積石,至于龍門、西河,會于渭汭。織皮崐崘、析支、渠搜,西戎即敘。

導岍及岐,至于荊山,逾于河;壺口、雷首至于太岳;厎柱、析城至于王屋;太行、恆山至于碣石,入于海。

西傾、朱圉、鳥鼠至于太華;熊耳、外方、桐柏至于陪尾。

導嶓冢,至于荊山;內方,至于大別。

岷山之陽,至于衡山,過九江,至于敷淺原。

導弱水,至于合黎,餘波入于流沙。

導黑水,至于三危,入于南海。

導河、積石,至于龍門;南至于華陰,東至于厎柱,又東至于孟津,東過洛汭,至于大伾;北過降水,至于大陸;又北,播為九河,同為逆河,入于海。

嶓冢導漾,東流為漢,又東,為滄浪之水,過三澨,至于大別,南入于江。東,匯澤為彭蠡,東,為北江,入于海。

岷山導江,東別為沱,又東至于澧;過九江,至于東陵,東迆北,會于匯;東為中江,入于海。

導沇水,東流為濟,入于河,溢為滎;東出于陶丘北,又東至于菏,又東北,會于汶,又北,東入于海。

導淮自桐柏,東會于泗、沂,東入于海。

導渭自鳥鼠同穴,東會于灃,又東會于涇,又東過漆沮,入于河。

導洛自熊耳,東北,會于澗、瀍;又東,會于伊,又東北,入于河。

九州攸同,四隩既宅,九山刊旅,九川滌源,九澤既陂,四海會同。六府孔修,庶土交正,厎慎財賦,咸則三壤成賦。中邦錫土、姓,祗台德先,不距朕行。

五百里甸服:百里賦納總,二百里納銍,三百里納秸服,四百里粟,五百里米。

五百里侯服:百里采,二百里男邦,三百里諸侯。

五百里綏服:三百里揆文教,二百里奮武衛。

五百里要服:三百里夷,二百里蔡。

五百里荒服:三百里蠻,二百里流。

東漸于海,西被于流沙,朔南暨聲教訖于四海。禹錫玄圭,告厥成功。


\end{pinyinscope}