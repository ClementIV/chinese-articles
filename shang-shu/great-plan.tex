\article{洪範}

\begin{pinyinscope}
武王勝殷,殺受,立武庚,以箕子歸。作《洪範》。

惟十有三祀,王訪于箕子。王乃言曰:「嗚呼!箕子。惟天陰騭下民,相協厥居,我不知其彝倫攸敘。」

箕子乃言曰:「我聞在昔,鯀堙洪水,汩陳其五行。帝乃震怒,不畀『洪範』九疇,彝倫攸斁。鯀則殛死,禹乃嗣興,天乃錫禹『洪範』九疇,彝倫攸敘。

初一曰五行,次二曰敬用五事,次三曰農用八政,次四曰協用五紀,次五曰建用皇極,次六曰乂用三德,次七曰明用稽疑,次八曰念用庶徵,次九曰嚮用五福,威用六極。

一、五行:一曰水,二曰火,三曰木,四曰金,五曰土。水曰潤下,火曰炎上,木曰曲直,金曰從革,土爰稼穡。潤下作鹹,炎上作苦,曲直作酸,從革作辛,稼穡作甘。

二、五事:一曰貌,二曰言,三曰視,四曰聽,五曰思。貌曰恭,言曰從,視曰明,聽曰聰,思曰睿。恭作肅,從作乂,明作哲,聰作謀,睿作聖。

三、八政:一曰食,二曰貨,三曰祀,四曰司空,五曰司徒,六曰司寇,七曰賓,八曰師。

四、五紀:一曰歲,二曰月,三曰日,四曰星辰,五曰歷數。

五、皇極:皇建其有極。斂時五福,用敷錫厥庶民。惟時厥庶民于汝極。錫汝保極:凡厥庶民,無有淫朋,人無有比德,惟皇作極。凡厥庶民,有猷有為有守,汝則念之。不協于極,不罹于咎,皇則受之。而康而色,曰:『予攸好德。』汝則錫之福。時人斯其惟皇之極。無虐煢獨而畏高明,人之有能有為,使羞其行,而邦其昌。凡厥正人,既富方谷,汝弗能使有好于而家,時人斯其辜。于其無好德,汝雖錫之福,其作汝用咎。無偏無陂,遵王之義;無有作好,遵王之道;無有作惡,尊王之路。無偏無黨,王道蕩蕩;無黨無偏,王道平平;無反無側,王道正直。會其有極,歸其有極。曰:皇,極之敷言,是彝是訓,于帝其訓,凡厥庶民,極之敷言,是訓是行,以近天子之光。曰:天子作民父母,以為天下王。

六、三德:一曰正直,二曰剛克,三曰柔克。平康,正直;彊弗友,剛克;燮友,柔克。沈潛,剛克;高明,柔克。惟闢作福,惟闢作威,惟辟玉食。臣無有作福、作威、玉食。臣之有作福、作威、玉食,其害于而家,凶于而國。人用側頗僻,民用僭忒。

七、稽疑:擇建立卜筮人,乃命十筮。曰雨,曰霽,曰蒙,曰驛,曰克,曰貞,曰悔,凡七。卜五,佔用二,衍忒。立時人作卜筮,三人占,則從二人之言。汝則有大疑,謀及乃心,謀及卿士,謀及庶人,謀及卜筮。汝則從,龜從,筮從,卿士從,庶民從,是之謂大同。身其康彊,子孫其逢,汝則從,龜從,筮從,卿士逆,庶民逆吉。卿士從,龜從,筮從,汝則逆,庶民逆,吉。庶民從,龜從,筮從,汝則逆,卿士逆,吉。汝則從,龜從,筮逆,卿士逆,庶民逆,作內吉,作外凶。龜筮共違于人,用靜吉,用作凶。

八、庶徵:曰雨,曰暘,曰燠,曰寒,曰風。曰時。五者來備,各以其敘,庶草蕃廡。一極備,凶;一極無,凶。曰休徵;曰肅、時雨若;曰乂,時暘若;曰晰,時燠若;曰謀,時寒若;曰聖,時風若。曰咎徵:曰狂,恆雨若;曰僭,恆暘若;曰豫,恆燠若;曰急,恆寒若;曰蒙,恆風若。曰王省惟歲,卿士惟月,師尹惟日。歲月日時無易,百穀用成,乂用明,俊民用章,家用平康。日月歲時既易,百穀用不成,乂用昏不明,俊民用微,家用不寧。庶民惟星,星有好風,星有好雨。日月之行,則有冬有夏。月之從星,則以風雨。

九、五福:一曰壽,二曰富,三曰康寧,四曰攸好德,五曰考終命。六極:一曰凶、短、折,二曰疾,三曰憂,四曰貧,五曰惡,六曰弱。

武王既勝殷,邦諸侯,班宗彝,作《分器》。


\end{pinyinscope}