\article{君牙}

\begin{pinyinscope}
穆王命君牙,為周大司徒,作《君牙》。

王若曰:「嗚呼!君牙,惟乃祖乃父,世篤忠貞,服勞王家,厥有成績,紀于太常。惟予小子嗣守文、武、成、康遺緒,亦惟先正之臣,克左右亂四方。心之憂危,若蹈虎尾,涉于春冰。今命爾予翼,作股肱心膂,纘乃舊服。無忝祖考,弘敷五典,式和民則。爾身克正,罔敢弗正,民心罔中,惟爾之中。夏暑雨,小民惟曰怨咨:冬祁寒,小民亦惟曰怨咨。厥惟艱哉!思其艱以圖其易,民乃寧。嗚呼!丕顯哉,文王謨!丕承哉,武王烈!啟佑我後人,咸以正罔缺。爾惟敬明乃訓,用奉若于先王,對揚文、武之光命,追配于前人。」

王若曰:「君牙,乃惟由先正舊典時式,民之治亂在茲。率乃祖考之攸行,昭乃辟之有乂。」


\end{pinyinscope}