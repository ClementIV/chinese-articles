\article{太甲上}

\begin{pinyinscope}
太甲既立,不明,伊尹放諸桐。三年復歸于亳,思庸,伊尹作《太甲》三篇。

惟嗣王不惠于阿衡,伊尹作書曰:「先王顧諟天之明命,以承上下神祇。社稷宗廟,罔不祗肅。天監厥德,用集大命,撫綏萬方。惟尹躬克左右厥辟,宅師,肆嗣王丕承基緒。惟尹躬先見于西邑夏,自周有終。相亦惟終;其後嗣王罔克有終,相亦罔終,嗣王戒哉!祗爾厥辟,辟不辟,忝厥祖。」

王惟庸罔念聞。伊尹乃言曰:「先王昧爽丕顯,坐以待旦。旁求俊彥,啟迪後人,無越厥命以自覆。慎乃儉德,惟懷永圖。若虞機張,往省括于度則釋。欽厥止,率乃祖攸行,惟朕以懌,萬世有辭。」

王未克變。伊尹曰:「茲乃不義,習與性成。予弗狎于弗順,營于桐宮,密邇先王其訓,無俾世迷。王徂桐宮居憂,克終允德。」


\end{pinyinscope}