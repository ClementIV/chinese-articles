\article{立政}

\begin{pinyinscope}
周公作《立政》。

周公若曰:「拜手稽首,告嗣天子王矣。」用咸戒于王曰:「王左右常伯、常任、準人、綴衣、虎賁。」

周公曰:「嗚呼!休茲知恤,鮮哉!古之人迪惟有夏,乃有室大競,籲俊尊上帝迪,知忱恂于九德之行。乃敢告教厥后曰:『拜手稽首后矣!』曰:『宅乃事,宅乃牧,宅乃準,茲惟后矣。謀面,用丕訓德,則乃宅人,茲乃三宅無義民。』

桀德,惟乃弗作往任,是惟暴德罔後。亦越成湯陟,丕釐上帝之耿命,乃用三有宅;克即宅,曰三有俊,克即俊。嚴惟丕式,克用三宅三俊,其在商邑,用協于厥邑;其在四方,用丕式見德。

嗚呼!其在受德暋,惟羞刑暴德之人,同于厥邦;乃惟庶習逸德之人,同于厥政。帝欽罰之,乃伻我有夏,式商受命,奄甸萬姓。

亦越文王、武王,克知三有宅心,灼見三有俊心,以敬事上帝,立民長伯。立政:任人、準夫、牧、作三事。虎賁、綴衣、趣馬、小尹、左右攜僕、百司庶府。大都小伯、藝人、表臣百司、太史、尹伯,庶常吉士。司徒、司馬、司空、亞、旅。夷、微、盧烝。三亳阪尹。

文王惟克厥宅心,乃克立茲常事司牧人,以克俊有德。文王罔攸兼于庶言;庶獄庶慎,惟有司之牧夫是訓用違;庶獄庶慎,文王罔敢知于茲。亦越武王,率惟敉功,不敢替厥義德,率惟謀從容德,以並受此丕丕基。」

嗚呼!孺子王矣!繼自今我其立政。立事、準人、牧夫,我其克灼知厥若,丕乃俾亂;相我受民,和我庶獄庶慎。時則勿有間之,自一話一言。我則末惟成德之彥,以乂我受民。

嗚呼!予旦已受人之徽言咸告孺子王矣。繼自今文子文孫,其勿誤于庶獄庶慎,惟正是乂之。

自古商人亦越我周文王立政,立事、牧夫、準人,則克宅之,克由繹之,茲乃俾乂,國則罔有。立政用憸人,不訓于德,是罔顯在厥世。繼自今立政,其勿以憸人,其惟吉士,用勵相我國家。

今文子文孫,孺子王矣!其勿誤于庶獄,惟有司之牧夫。其克詰爾戎兵以陟禹之跡,方行天下,至于海表,罔有不服。以覲文王之耿光,以揚武王之大烈。嗚呼!繼自今後王立政,其惟克用常人。」

周公若曰:「太史!司寇蘇公式敬爾由獄,以長我王國。茲式有慎,以列用中罰。」


\end{pinyinscope}