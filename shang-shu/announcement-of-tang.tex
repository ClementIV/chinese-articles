\article{湯誥}

\begin{pinyinscope}
湯既黜夏命,復歸于亳,作《湯誥》。

王歸自克夏,至于亳,誕告萬方。王曰:「嗟!爾萬方有眾,明聽予一人誥。惟皇上帝,降衷于下民。若有恆性,克綏厥猷惟后。夏王滅德作威,以敷虐于爾萬方百姓。爾萬方百姓,罹其凶害,弗忍荼毒,並告無辜于上下神祇。天道福善禍淫,降災于夏,以彰厥罪。肆台小子,將天命明威,不敢赦。敢用玄牡,敢昭告于上天神后,請罪有夏。聿求元聖,與之戮力,以與爾有眾請命。上天孚佑下民,罪人黜伏,天命弗僭,賁若草木,兆民允殖。俾予一人輯寧爾邦家,茲朕未知獲戾于上下,慄慄危懼,若將隕于深淵。凡我造邦,無從匪彝,無即慆淫,各守爾典,以承天休。爾有善,朕弗敢蔽;罪當朕躬,弗敢自赦,惟簡在上帝之心。其爾萬方有罪,在予一人;予一人有罪,無以爾萬方。嗚呼!尚克時忱,乃亦有終。」

咎單作《明居》。


\end{pinyinscope}