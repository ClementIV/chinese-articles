\article{康誥}

\begin{pinyinscope}
成王既伐管叔、蔡叔,以殷余民封康叔,作《康誥》、《酒誥》、《梓材》。

惟三月哉生魄,周公初基作新大邑于東國洛,四方民大和會。侯、甸、男邦、采、衛百工、播民,和見士于周。周公咸勤,乃洪大誥治。

王若曰:「孟侯,朕其弟,小子封。惟乃丕顯考文王,克明德慎罰;不敢侮鰥寡,庸庸,祗祗,威威,顯民,用肇造我區夏,越我一、二邦以修我西土。惟時怙冒,聞于上帝,帝休,天乃大命文王。殪戎殷,誕受厥命越厥邦民,惟時敘,乃寡兄勖。肆汝小子封在茲東土。

王曰:「嗚呼!封,汝念哉!今民將在祗遹乃文考,紹聞衣德言。往敷求于殷先哲王用保乂民,汝丕遠惟商耇成人宅心知訓。別求聞由古先哲王用康保民。弘于天,若德,裕乃身不廢在王命!」

王曰:「嗚呼!小子封,恫瘝乃身,敬哉!天畏棐忱;民情大可見,小人難保。往盡乃心,無康好逸豫,乃其乂民。我聞曰:『怨不在大,亦不在小;惠不惠,懋不懋。』

已!汝惟小子,乃服惟弘王應保殷民,亦惟助王宅天命,作新民。」

王曰:「嗚呼!封,敬明乃罰。人有小罪,非眚,乃惟終自作不典;式爾,有厥罪小,乃不可不殺。乃有大罪,非終,乃惟眚災:適爾,既道極厥辜,時乃不可殺。」

王曰:「嗚呼!封,有敘時,乃大明服,惟民其敕懋和。若有疾,惟民其畢棄咎。若保赤子,惟民其康乂。非汝封刑人殺人,無或刑人殺人。非汝封又曰劓刵人,無或劓刵人。」

王曰:「外事,汝陳時臬司師,茲殷罰有倫。」又曰:「要囚,服念五、六日至于旬時,丕蔽要囚。」

王曰:「汝陳時臬事罰。蔽殷彝,用其義刑義殺,勿庸以次汝封。乃汝盡遜曰時敘,惟曰未有遜事。

已!汝惟小子,未其有若汝封之心。朕心朕德,惟乃知。

凡民自得罪:寇攘奸宄,殺越人于貨,暋不畏死,罔弗憝。

王曰:「封,元惡大憝,矧惟不孝不友。子弗祗服厥父事,大傷厥考心;于父不能字厥子,乃疾厥子。于弟弗念天顯,乃弗克恭厥兄;兄亦不念鞠子哀,大不友于弟。惟弔茲,不于我政人得罪,天惟與我民彝大泯亂,曰:乃其速由文王作罰,刑茲無赦。

不率大戛,矧惟外庶子、訓人惟厥正人越小臣、諸節。乃別播敷,造民大譽,弗念弗庸,瘝厥君,時乃引惡,惟朕憝。已!汝乃其速由茲義率殺。

亦惟君惟長,不能厥家人越厥小臣、外正;惟威惟虐,大放王命;乃非德用乂。汝亦罔不克敬典,乃由裕民,惟文王之敬忌;乃裕民曰:『我惟有及。』則予一人以懌。」

王曰:「封,爽惟民迪吉康,我時其惟殷先哲王德,用康乂民作求。矧今民罔迪,不適;不迪,則罔政在厥邦。」

王曰:「封,予惟不可不監,告汝德之說于罰之行。今惟民不靜,未戾厥心,迪屢未同,爽惟天其罰殛我,我其不怨。惟厥罪無在大,亦無在多,矧曰其尚顯聞于天。」

王曰:「嗚呼!封,敬哉!無作怨,勿用非謀非彝蔽時忱。丕則敏德,用康乃心,顧乃德,遠乃猷,裕乃以;民寧,不汝瑕殄。」

王曰:「嗚呼!肆汝小子封。惟命不于常,汝念哉!無我殄享,明乃服命,高乃聽,用康乂民。」

王若曰:「往哉!封,勿替敬典,聽朕告,汝乃以殷民世享。」


\end{pinyinscope}