\article{說命下}

\begin{pinyinscope}
王曰:「來!汝說。台小子舊學于甘盤,既乃遯于荒野,入宅于河。自河徂亳,暨厥終罔顯。爾惟訓于朕志,若作酒醴,爾惟麴蘖;若作和羹,爾惟鹽梅。爾交修予,罔予棄,予惟克邁乃訓。」

說曰:「王,人求多聞,時惟建事,學于古訓乃有獲。事不師古,以克永世,匪說攸聞。惟學,遜志務時敏,厥修乃來。允懷于茲,道積于厥躬。惟斆學半,念終始典于學,厥德脩罔覺。監于先王成憲,其永無愆。惟說式克欽承,旁招俊乂,列于庶位。」

王曰:「嗚呼!說,四海之內,咸仰朕德,時乃風。股肱惟人,良臣惟聖。昔先正保衡作我先王,乃曰:『予弗克俾厥后惟堯舜,其心愧恥,若撻于市。』一夫不獲,則曰時予之辜。佑我烈祖,格于皇天。爾尚明保予,罔俾阿衡專美有商。惟后非賢不乂,惟賢非后不食。其爾克紹乃辟于先王,永綏民。」

說拜稽首曰:「敢對揚天子之休命。」


\end{pinyinscope}