\article{大禹謨}

\begin{pinyinscope}
皋陶矢厥謨,禹成厥功,帝舜申之。作《大禹》、《皋陶謨》、《益稷》。

曰若稽古大禹,曰文命,敷於四海,祗承于帝。曰:「后克艱厥后,臣克艱厥臣,政乃乂,黎民敏德。」

帝曰:「俞!允若茲,嘉言罔攸伏,野無遺賢,萬邦咸寧。稽于眾,舍己從人,不虐無告,不廢困窮,惟帝時克。」

益曰:「都,帝德廣運,乃聖乃神,乃武乃文。皇天眷命,奄有四海為天下君。」

禹曰:「惠迪吉,從逆凶,惟影響。」

益曰:「吁!戒哉!儆戒無虞,罔失法度。罔遊于逸,罔淫于樂。任賢勿貳,去邪勿疑。疑謀勿成,百志惟熙。罔違道以干百姓之譽,罔咈百姓以從己之欲。無怠無荒,四夷來王。」

禹曰:「於!帝念哉!德惟善政,政在養民。水、火、金、木、土、穀,惟修;正德、利用、厚生、惟和。九功惟敘,九敘惟歌。戒之用休,董之用威,勸之以九歌俾勿壞。」

帝曰:「俞!地平天成,六府三事允治,萬世永賴,時乃功。」

帝曰:「格,汝禹!朕宅帝位三十有三載,耄期倦于勤。汝惟不怠,總朕師。」

禹曰:「朕德罔克,民不依。皋陶邁種德,德乃降,黎民懷之。帝念哉!念茲在茲,釋茲在茲,名言茲在茲,允出茲在茲,惟帝念功。」

帝曰:「皋陶,惟茲臣庶,罔或干予正。汝作士,明于五刑,以弼五教。期于予治,刑期于無刑,民協于中,時乃功,懋哉。」

皋陶曰:「帝德罔愆,臨下以簡,御眾以寬;罰弗及嗣,賞延于世。宥過無大,刑故無小;罪疑惟輕,功疑惟重;與其殺不辜,寧失不經;好生之德,洽于民心,茲用不犯于有司。」

帝曰:「俾予從欲以治,四方風動,惟乃之休。」

帝曰:「來,禹!降水儆予,成允成功,惟汝賢。克勤于邦,克儉于家,不自滿假,惟汝賢。汝惟不矜,天下莫與汝爭能。汝惟不伐,天下莫與汝爭功。予懋乃德,嘉乃丕績,天之歷數在汝躬,汝終陟元后。人心惟危,道心惟微,惟精惟一,允執厥中。無稽之言勿聽,弗詢之謀勿庸。可愛非君?可畏非民?眾非元后,何戴?后非眾,罔與守邦?欽哉!慎乃有位,敬修其可願,四海困窮,天祿永終。惟口出好興戎,朕言不再。」

禹曰:「枚卜功臣,惟吉之從。」

帝曰:「禹!官占惟先蔽志,昆命于元龜。朕志先定,詢謀僉同,鬼神其依,龜筮協從,卜不習吉。」禹拜稽首,固辭。

帝曰:「毋!惟汝諧。」

正月朔旦,受命于神宗,率百官若帝之初。

帝曰:「咨,禹!惟時有苗弗率,汝徂征。」

禹乃會群后,誓于師曰;「濟濟有眾,咸聽朕命。蠢茲有苗,昏迷不恭,侮慢自賢,反道敗德,君子在野,小人在位,民棄不保,天降之咎,肆予以爾眾士,奉辭伐罪。爾尚一乃心力,其克有勳。」

三旬,苗民逆命。益贊于禹曰:「惟德動天,無遠弗屆。滿招損,謙受益,時乃天道。帝初于歷山,往于田,日號泣于旻天,于父母,負罪引慝。祗載見瞽瞍,夔夔齋慄,瞽亦允若。至諴感神,矧茲有苗。」

禹拜昌言曰:「俞!」班師振旅。帝乃誕敷文德,舞干羽于兩階,七旬有苗格。


\end{pinyinscope}