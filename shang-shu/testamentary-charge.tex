\article{顧命}

\begin{pinyinscope}
成王將崩,命召公、畢公率諸侯相康王,作《顧命》。

惟四月,哉生魄,王不懌。甲子,王乃洮頮水。相被冕服,憑玉几。乃同,召太保奭、芮伯、彤伯、畢公、衛侯、毛公、師氏、虎臣、百尹、御事。

王曰:「嗚呼!疾大漸,惟幾,病日臻。既彌留,恐不獲誓言嗣,茲予審訓命汝。昔君文王、武王宣重光,奠麗陳教,則肄肄不違,用克達殷集大命。在後之侗,敬迓天威,嗣守文、武大訓,無敢昏逾。今天降疾,殆弗興弗悟。爾尚明時朕言,用敬保元子釗弘濟于艱難,柔遠能邇,安勸小大庶邦。思夫人自亂于威儀。爾無以釗冒貢于非幾。」

茲既受命,還出綴衣于庭。越翼日乙丑,王崩。

太保命仲桓、南宮毛俾爰齊侯呂伋,以二干戈、虎賁百人逆子釗於南門之外。延入翼室,恤宅宗。丁卯,命作冊度。越七日癸酉,伯相命士須材。

狄設黼扆、綴衣。牖間南嚮,敷重篾席,黼純,華玉,仍几。西序東嚮,敷重厎席,綴純,文貝,仍几。東序西嚮,敷重豐席,畫純,雕玉,仍几。西夾南嚮,敷重筍席,玄紛純,漆,仍几。越玉五重,陳寶,赤刀、大訓、弘璧、琬琰、在西序。大玉、夷玉、天球、河圖,在東序。胤之舞衣、大貝、鼖鼓,在西房;兌之戈、和之弓、垂之竹矢,在東房。大輅在賓階面,綴輅在阼階面,先輅在左塾之前,次輅在右塾之前。

二人雀弁,執惠,立于畢門之內。四人綦弁,執戈上刃,夾兩階戺。一人冕,執劉,立于東堂,一人冕,執鉞,立于西堂。一人冕,執戣,立于東垂。一人冕,執瞿,立于西垂。一人冕,執銳,立于側階。

王麻冕黼裳,由賓階隮。卿士邦君麻冕蟻裳,入即位。太保、太史、太宗皆麻冕彤裳。太保承介圭,上宗奉同瑁,由阼階隮。太史秉書,由賓階隮,御王冊命。曰:「皇后憑玉几,道揚末命,命汝嗣訓,臨君周邦,率循大卞,燮和天下,用答揚文、武之光訓。」王再拜,興,答曰:「眇眇予末小子,其能而亂四方以敬忌天威。」乃受同瑁,王三宿,三祭,三吒。上宗曰:「饗!」太保受同,降,盥,以異同秉璋以酢。授宗人同,拜。王答拜。太保受同,祭,嚌,宅,授宗人同,拜。王答拜。太保降,收。諸侯出廟門俟。


\end{pinyinscope}