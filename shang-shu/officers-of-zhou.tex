\article{周官}

\begin{pinyinscope}
成王既黜殷命,滅淮夷,還歸在豐,作《周官》。

惟周王撫萬邦,巡侯、甸,四征弗庭,綏厥兆民。六服群辟,罔不承德。歸于宗周,董正治官。

王曰:「若昔大猷,制治于未亂,保邦于未危。」曰:「唐虞稽古,建官惟百。內有百揆四岳,外有州、牧、侯伯。庶政惟和,萬國咸寧。夏商官倍,亦克用乂。明王立政,不惟其官,惟其人。今予小子,祗勤于德,夙夜不逮。仰惟前代時若,訓迪厥官。

立太師、太傅、太保,茲惟三公。論道經邦,燮理陰陽。官不必備,惟其人。少師、少傅、少保,曰三孤。貳公弘化,寅亮天地,弼予一人。塚宰掌邦治,統百官,均四海。司徒掌邦教,敷五典,擾兆民。宗伯掌邦禮,治神人,和上下。司馬掌邦政,統六師,平邦國。司冠掌邦禁,詰奸慝,刑暴亂。司空掌邦土,居四民,時地利。六卿分職,各率其屬,以倡九牧,阜成兆民。六年,五服一朝。又六年,王乃時巡,考制度于四岳。諸侯各朝于方岳,大明黜陟。」

王曰:「嗚呼!凡我有官君子,欽乃攸司,慎乃出令,令出惟行,弗惟反。以公滅私,民其允懷。學古入官。議事以制,政乃不迷。其爾典常作之師,無以利口亂厥官。蓄疑敗謀,怠忽荒政,不學牆面,蒞事惟煩。

戒爾卿士,功崇惟志,業廣惟勤,惟克果斷,乃罔後艱。位不期驕,祿不期侈。恭儉惟德,無載爾偽。作德,心逸日休;作偽,心勞日拙。居寵思危,罔不惟畏,弗畏入畏。推賢讓能,庶官乃和,不和政龐。舉能其官,惟爾之能。稱匪其人,惟爾不任。」

王曰:「嗚呼!三事暨大無,敬爾有官,亂爾有政,以佑乃辟。永康兆民,萬邦惟無斁。

成王既伐東夷,肅慎來賀。王俾榮伯作《賄肅慎之命》。

周公在豐,將沒,欲葬成周。公薨,成王葬于畢,告周公,作《亳姑》。


\end{pinyinscope}