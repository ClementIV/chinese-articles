\article{梁惠王上}

\begin{pinyinscope}
孟子見梁惠王。王曰:「叟不遠千里而來,亦將有以利吾國乎?」

孟子對曰:「王何必曰利?亦有仁義而已矣。王曰『何以利吾國』?大夫曰『何以利吾家』?士庶人曰『何以利吾身』?上下交征利而國危矣。萬乘之國弒其君者,必千乘之家;千乘之國弒其君者,必百乘之家。萬取千焉,千取百焉,不為不多矣。苟為後義而先利,不奪不饜。未有仁而遺其親者也,未有義而後其君者也。王亦曰仁義而已矣,何必曰利?」

孟子見梁惠王,王立於沼上,顧鴻鴈麋鹿,曰:「賢者亦樂此乎?」

孟子對曰:「賢者而後樂此,不賢者雖有此,不樂也。《詩》云:『經始靈臺,經之營之,庶民攻之,不日成之。經始勿亟,庶民子來。王在靈囿,麀鹿攸伏,麀鹿濯濯,白鳥鶴鶴。王在靈沼,於牣魚躍。』文王以民力為臺為沼。而民歡樂之,謂其臺曰靈臺,謂其沼曰靈沼,樂其有麋鹿魚鼈。古之人與民偕樂,故能樂也。《湯誓》曰:『時日害喪?予及女偕亡。』民欲與之偕亡,雖有臺池鳥獸,豈能獨樂哉?」

梁惠王曰:「寡人之於國也,盡心焉耳矣。河內凶,則移其民於河東,移其粟於河內。河東凶亦然。察鄰國之政,無如寡人之用心者。鄰國之民不加少,寡人之民不加多,何也?」

孟子對曰:「王好戰,請以戰喻。填然鼓之,兵刃既接,棄甲曳兵而走。或百步而後止,或五十步而後止。以五十步笑百步,則何如?」

曰:「不可,直不百步耳,是亦走也。」

曰:「王如知此,則無望民之多於鄰國也。不違農時,穀不可勝食也;數罟不入洿池,魚鼈不可勝食也;斧斤以時入山林,材木不可勝用也。穀與魚鼈不可勝食,材木不可勝用,是使民養生喪死無憾也。養生喪死無憾,王道之始也。五畝之宅,樹之以桑,五十者可以衣帛矣;雞豚狗彘之畜,無失其時,七十者可以食肉矣;百畝之田,勿奪其時,數口之家可以無飢矣;謹庠序之教,申之以孝悌之義,頒白者不負戴於道路矣。七十者衣帛食肉,黎民不飢不寒,然而不王者,未之有也。

狗彘食人食而不知檢,塗有餓莩而不知發;人死,則曰:『非我也,歲也。』是何異於刺人而殺之,曰:『非我也,兵也。』王無罪歲,斯天下之民至焉。」

梁惠王曰:「寡人願安承教。」

孟子對曰:「殺人以梃與刃,有以異乎?」

曰:「無以異也。」

「以刃與政,有以異乎?」

曰:「無以異也。」

曰:「庖有肥肉,廐有肥馬,民有飢色,野有餓莩,此率獸而食人也。獸相食,且人惡之。為民父母,行政不免於率獸而食人。惡在其為民父母也?仲尼曰:『始作俑者,其無後乎!』為其象人而用之也。如之何其使斯民飢而死也?」

梁惠王曰:「晉國,天下莫強焉,叟之所知也。及寡人之身,東敗於齊,長子死焉;西喪地於秦七百里;南辱於楚。寡人恥之,願比死者一洒之,如之何則可?」

孟子對曰:「地方百里而可以王。王如施仁政於民,省刑罰,薄稅斂,深耕易耨。壯者以暇日修其孝悌忠信,入以事其父兄,出以事其長上,可使制梃以撻秦楚之堅甲利兵矣。彼奪其民時,使不得耕耨以養其父母,父母凍餓,兄弟妻子離散。彼陷溺其民,王往而征之,夫誰與王敵?故曰:『仁者無敵。』王請勿疑!」

孟子見梁襄王。出,語人曰:「望之不似人君,就之而不見所畏焉。卒然問曰:『天下惡乎定?』吾對曰:『定于一。』

『孰能一之?』對曰:『不嗜殺人者能一之。』

『孰能與之?』對曰:『天下莫不與也。王知夫苗乎?七八月之間旱,則苗槁矣。天油然作雲,沛然下雨,則苗浡然興之矣。其如是,孰能禦之?今夫天下之人牧,未有不嗜殺人者也,如有不嗜殺人者,則天下之民皆引領而望之矣。誠如是也,民歸之,由水之就下,沛然誰能禦之?』」

齊宣王問曰:「齊桓、晉文之事可得聞乎?」

孟子對曰:「仲尼之徒無道桓、文之事者,是以後世無傳焉。臣未之聞也。無以,則王乎?」

曰:「德何如,則可以王矣?」

曰:「保民而王,莫之能禦也。」

曰:「若寡人者,可以保民乎哉?」

曰:「可。」

曰:「何由知吾可也?」

曰:「臣聞之胡齕曰,王坐於堂上,有牽牛而過堂下者,王見之,曰:『牛何之?』對曰:『將以釁鐘。』王曰:『舍之!吾不忍其觳觫,若無罪而就死地。』對曰:『然則廢釁鐘與?』曰:『何可廢也?以羊易之!』不識有諸?」

曰:「有之。」

曰:「是心足以王矣。百姓皆以王為愛也,臣固知王之不忍也。」

王曰:「然。誠有百姓者。齊國雖褊小,吾何愛一牛?即不忍其觳觫,若無罪而就死地,故以羊易之也。」

曰:「王無異於百姓之以王為愛也。以小易大,彼惡知之?王若隱其無罪而就死地,則牛羊何擇焉?」

王笑曰:「是誠何心哉?我非愛其財。而易之以羊也,宜乎百姓之謂我愛也。」

曰:「無傷也,是乃仁術也,見牛未見羊也。君子之於禽獸也,見其生,不忍見其死;聞其聲,不忍食其肉。是以君子遠庖廚也。」

王說曰:「《詩》云:『他人有心,予忖度之。』夫子之謂也。夫我乃行之,反而求之,不得吾心。夫子言之,於我心有戚戚焉。此心之所以合於王者,何也?」

曰:「有復於王者曰:『吾力足以舉百鈞』,而不足以舉一羽;『明足以察秋毫之末』,而不見輿薪,則王許之乎?」

曰:「否。」

「今恩足以及禽獸,而功不至於百姓者,獨何與?然則一羽之不舉,為不用力焉;輿薪之不見,為不用明焉,百姓之不見保,為不用恩焉。故王之不王,不為也,非不能也。」

曰:「不為者與不能者之形何以異?」

曰:「挾太山以超北海,語人曰『我不能』,是誠不能也。為長者折枝,語人曰『我不能』,是不為也,非不能也。故王之不王,非挾太山以超北海之類也;王之不王,是折枝之類也。老吾老,以及人之老;幼吾幼,以及人之幼。天下可運於掌。《詩》云:『刑于寡妻,至于兄弟,以御于家邦。』言舉斯心加諸彼而已。故推恩足以保四海,不推恩無以保妻子。古之人所以大過人者無他焉,善推其所為而已矣。今恩足以及禽獸,而功不至於百姓者,獨何與?權,然後知輕重;度,然後知長短。物皆然,心為甚。王請度之!抑王興甲兵,危士臣,構怨於諸侯,然後快於心與?」

王曰:「否。吾何快於是?將以求吾所大欲也。」

曰:「王之所大欲可得聞與?」王笑而不言。

曰:「為肥甘不足於口與?輕煖不足於體與?抑為采色不足視於目與?聲音不足聽於耳與?便嬖不足使令於前與?王之諸臣皆足以供之,而王豈為是哉?」

曰:「否。吾不為是也。」

曰:「然則王之所大欲可知已。欲辟土地,朝秦楚,莅中國而撫四夷也。以若所為求若所欲,猶緣木而求魚也。」

王曰:「若是其甚與?」

曰:「殆有甚焉。緣木求魚,雖不得魚,無後災。以若所為,求若所欲,盡心力而為之,後必有災。」

曰:「可得聞與?」

曰:「鄒人與楚人戰,則王以為孰勝?」

曰:「楚人勝。」

曰:「然則小固不可以敵大,寡固不可以敵眾,弱固不可以敵彊。海內之地方千里者九,齊集有其一。以一服八,何以異於鄒敵楚哉?蓋亦反其本矣。今王發政施仁,使天下仕者皆欲立於王之朝,耕者皆欲耕於王之野,商賈皆欲藏於王之市,行旅皆欲出於王之塗,天下之欲疾其君者皆欲赴愬於王。其若是,孰能禦之?」

王曰:「吾惛,不能進於是矣。願夫子輔吾志,明以教我。我雖不敏,請嘗試之。」

曰:「無恆產而有恆心者,惟士為能。若民,則無恆產,因無恆心。苟無恆心,放辟,邪侈,無不為已。及陷於罪,然後從而刑之,是罔民也。焉有仁人在位,罔民而可為也?是故明君制民之產,必使仰足以事父母,俯足以畜妻子,樂歲終身飽,凶年免於死亡。然後驅而之善,故民之從之也輕。今也制民之產,仰不足以事父母,俯不足以畜妻子,樂歲終身苦,凶年不免於死亡。此惟救死而恐不贍,奚暇治禮義哉?王欲行之,則盍反其本矣。五畝之宅,樹之以桑,五十者可以衣帛矣;雞豚狗彘之畜,無失其時,七十者可以食肉矣;百畝之田,勿奪其時,八口之家可以無飢矣;謹庠序之教,申之以孝悌之義,頒白者不負戴於道路矣。老者衣帛食肉,黎民不飢不寒,然而不王者,未之有也。」


\end{pinyinscope}