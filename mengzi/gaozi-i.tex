\article{告子上}

\begin{pinyinscope}
告子曰:「性,猶杞柳也;義,猶桮棬也。以人性為仁義,猶以杞柳為桮棬。」

孟子曰:「子能順杞柳之性而以為桮棬乎?將戕賊杞柳而後以為桮棬也?如將戕賊杞柳而以為桮棬,則亦將戕賊人以為仁義與?率天下之人而禍仁義者,必子之言夫!」

告子曰:「性猶湍水也,決諸東方則東流,決諸西方則西流。人性之無分於善不善也,猶水之無分於東西也。」

孟子曰:「水信無分於東西。無分於上下乎?人性之善也,猶水之就下也。人無有不善,水無有不下。今夫水,搏而躍之,可使過顙;激而行之,可使在山。是豈水之性哉?其勢則然也。人之可使為不善,其性亦猶是也。」

告子曰:「生之謂性。」

孟子曰:「生之謂性也,猶白之謂白與?」曰:「然。」

「白羽之白也,猶白雪之白;白雪之白,猶白玉之白與?」曰:「然。」

「然則犬之性,猶牛之性;牛之性,猶人之性與?」

告子曰:「食色,性也。仁,內也,非外也;義,外也,非內也。」

孟子曰:「何以謂仁內義外也?」

曰:「彼長而我長之,非有長於我也;猶彼白而我白之,從其白於外也,故謂之外也。」

曰:「異於白馬之白也,無以異於白人之白也;不識長馬之長也,無以異於長人之長與?且謂長者義乎?長之者義乎?」

曰:「吾弟則愛之,秦人之弟則不愛也,是以我為悅者也,故謂之內。長楚人之長,亦長吾之長,是以長為悅者也,故謂之外也。」

曰:「耆秦人之炙,無以異於耆吾炙。夫物則亦有然者也,然則耆炙亦有外與?」

孟季子問公都子曰:「何以謂義內也?」曰:「行吾敬,故謂之內也。」

「鄉人長於伯兄一歲,則誰敬?」曰:「敬兄。」

「酌則誰先?」曰:「先酌鄉人。」

「所敬在此,所長在彼,果在外,非由內也。」

公都子不能答,以告孟子。孟子曰:「敬叔父乎?敬弟乎?彼將曰『敬叔父』。曰:『弟為尸,則誰敬?』彼將曰『敬弟。』子曰:『惡在其敬叔父也?』彼將曰『在位故也。』子亦曰:『在位故也。庸敬在兄,斯須之敬在鄉人。』」

季子聞之曰:「敬叔父則敬,敬弟則敬,果在外,非由內也。」

公都子曰:「冬日則飲湯,夏日則飲水,然則飲食亦在外也?」

公都子曰:「告子曰:『性無善無不善也。』或曰:『性可以為善,可以為不善;是故文武興,則民好善;幽厲興,則民好暴。』或曰:『有性善,有性不善;是故以堯為君而有象,以瞽瞍為父而有舜;以紂為兄之子且以為君,而有微子啟、王子比干。』今曰『性善』,然則彼皆非與?」

孟子曰:「乃若其情,則可以為善矣,乃所謂善也。若夫為不善,非才之罪也。惻隱之心,人皆有之;羞惡之心,人皆有之;恭敬之心,人皆有之;是非之心,人皆有之。惻隱之心,仁也;羞惡之心,義也;恭敬之心,禮也;是非之心,智也。仁義禮智,非由外鑠我也,我固有之也,弗思耳矣。故曰:『求則得之,舍則失之。』或相倍蓰而無算者,不能盡其才者也。《詩》曰:『天生蒸民,有物有則。民之秉夷,好是懿德。』孔子曰:『為此詩者,其知道乎!故有物必有則,民之秉夷也,故好是懿德。』」

孟子曰:「富歲,子弟多賴;凶歲,子弟多暴,非天之降才爾殊也,其所以陷溺其心者然也。今夫麰麥,播種而耰之,其地同,樹之時又同,浡然而生,至於日至之時,皆熟矣。雖有不同,則地有肥磽,雨露之養,人事之不齊也。故凡同類者,舉相似也,何獨至於人而疑之?聖人與我同類者。故龍子曰:『不知足而為屨,我知其不為蕢也。』屨之相似,天下之足同也。口之於味,有同耆也。易牙先得我口之所耆者也。如使口之於味也,其性與人殊,若犬馬之與我不同類也,則天下何耆皆從易牙之於味也?至於味,天下期於易牙,是天下之口相似也惟耳亦然。至於聲,天下期於師曠,是天下之耳相似也。惟目亦然。至於子都,天下莫不知其姣也。不知子都之姣者,無目者也。故曰:口之於味也,有同耆焉;耳之於聲也,有同聽焉;目之於色也,有同美焉。至於心,獨無所同然乎?心之所同然者何也?謂理也,義也。聖人先得我心之所同然耳。故理義之悅我心,猶芻豢之悅我口。」

孟子曰:「牛山之木嘗美矣,以其郊於大國也,斧斤伐之,可以為美乎?是其日夜之所息,雨露之所潤,非無萌櫱之生焉,牛羊又從而牧之,是以若彼濯濯也。人見其濯濯也,以為未嘗有材焉,此豈山之性也哉?雖存乎人者,豈無仁義之心哉?其所以放其良心者,亦猶斧斤之於木也,旦旦而伐之,可以為美乎?其日夜之所息,平旦之氣,其好惡與人相近也者幾希,則其旦晝之所為,有梏亡之矣。梏之反覆,則其夜氣不足以存;夜氣不足以存,則其違禽獸不遠矣。人見其禽獸也,而以為未嘗有才焉者,是豈人之情也哉?故苟得其養,無物不長;苟失其養,無物不消。孔子曰:『操則存,舍則亡;出入無時,莫知其鄉。』惟心之謂與?」

孟子曰:「無或乎王之不智也,雖有天下易生之物也,一日暴之、十日寒之,未有能生者也。吾見亦罕矣,吾退而寒之者至矣,吾如有萌焉何哉?今夫弈之為數,小數也;不專心致志,則不得也。弈秋,通國之善弈者也。使弈秋誨二人弈,其一人專心致志,惟弈秋之為聽。一人雖聽之,一心以為有鴻鵠將至,思援弓繳而射之,雖與之俱學,弗若之矣。為是其智弗若與?曰非然也。」

孟子曰:「魚,我所欲也;熊掌,亦我所欲也,二者不可得兼,舍魚而取熊掌者也。生,亦我所欲也;義,亦我所欲也,二者不可得兼,舍生而取義者也。生亦我所欲,所欲有甚於生者,故不為苟得也;死亦我所惡,所惡有甚於死者,故患有所不辟也。如使人之所欲莫甚於生,則凡可以得生者,何不用也?使人之所惡莫甚於死者,則凡可以辟患者,何不為也?由是則生而有不用也,由是則可以辟患而有不為也。是故所欲有甚於生者,所惡有甚於死者,非獨賢者有是心也,人皆有之,賢者能勿喪耳。一簞食,一豆羹,得之則生,弗得則死。嘑爾而與之,行道之人弗受;蹴爾而與之,乞人不屑也。萬鍾則不辨禮義而受之。萬鍾於我何加焉?為宮室之美、妻妾之奉、所識窮乏者得我與?鄉為身死而不受,今為宮室之美為之;鄉為身死而不受,今為妻妾之奉為之;鄉為身死而不受,今為所識窮乏者得我而為之,是亦不可以已乎?此之謂失其本心。」

孟子曰:「仁,人心也;義,人路也。舍其路而弗由,放其心而不知求,哀哉!人有雞犬放,則知求之;有放心,而不知求。學問之道無他,求其放心而已矣。」

孟子曰:「今有無名之指,屈而不信,非疾痛害事也,如有能信之者,則不遠秦楚之路,為指之不若人也。指不若人,則知惡之;心不若人,則不知惡,此之謂不知類也。」

孟子曰:「拱把之桐梓,人苟欲生之,皆知所以養之者。至於身,而不知所以養之者,豈愛身不若桐梓哉?弗思甚也。」

孟子曰:「人之於身也,兼所愛。兼所愛,則兼所養也。無尺寸之膚不愛焉,則無尺寸之膚不養也。所以考其善不善者,豈有他哉?於己取之而已矣。體有貴賤,有小大。無以小害大,無以賤害貴。養其小者為小人,養其大者為大人。今有場師,舍其梧檟,養其樲棘,則為賤場師焉。養其一指而失其肩背,而不知也,則為狼疾人也。飲食之人,則人賤之矣,為其養小以失大也。飲食之人無有失也,則口腹豈適為尺寸之膚哉?」

公都子問曰:「鈞是人也,或為大人,或為小人,何也?」

孟子曰:「從其大體為大人,從其小體為小人。」

曰:「鈞是人也,或從其大體,或從其小體,何也?」

曰:「耳目之官不思,而蔽於物,物交物,則引之而已矣。心之官則思,思則得之,不思則不得也。此天之所與我者,先立乎其大者,則其小者弗能奪也。此為大人而已矣。」

孟子曰:「有天爵者,有人爵者。仁義忠信,樂善不倦,此天爵也;公卿大夫,此人爵也。古之人修其天爵,而人爵從之。今之人修其天爵,以要人爵;既得人爵,而棄其天爵,則惑之甚者也,終亦必亡而已矣。」

孟子曰:「欲貴者,人之同心也。人人有貴於己者,弗思耳。人之所貴者,非良貴也。趙孟之所貴,趙孟能賤之。《詩》云:『既醉以酒,既飽以德。』言飽乎仁義也,所以不願人之膏粱之味也;令聞廣譽施於身,所以不願人之文繡也。」

孟子曰:「仁之勝不仁也,猶水勝火。今之為仁者,猶以一杯水,救一車薪之火也;不熄,則謂之水不勝火,此又與於不仁之甚者也。亦終必亡而已矣。」

孟子曰:「五穀者,種之美者也;苟為不熟,不如荑稗。夫仁亦在乎熟之而已矣。」

孟子曰:「羿之教人射,必志於彀;學者亦必志於彀。大匠誨人,必以規矩;學者亦必以規矩。」


\end{pinyinscope}