\article{離婁上}

\begin{pinyinscope}
孟子曰:「離婁之明,公輸子之巧,不以規矩,不能成方員:師曠之聰,不以六律,不能正五音;堯舜之道,不以仁政,不能平治天下。今有仁心仁聞而民不被其澤,不可法於後世者,不行先王之道也。

「故曰,徒善不足以為政,徒法不能以自行。《詩》云:『不愆不忘,率由舊章。』遵先王之法而過者,未之有也。聖人既竭目力焉,繼之以規矩準繩,以為方員平直,不可勝用也;既竭耳力焉,繼之以六律,正五音,不可勝用也;既竭心思焉,繼之以不忍人之政,而仁覆天下矣。

「故曰,為高必因丘陵,為下必因川澤。為政不因先王之道,可謂智乎?是以惟仁者宜在高位。不仁而在高位,是播其惡於眾也。上無道揆也。下無法守也,朝不信道,工不信度,君子犯義,小人犯刑,國之所存者幸也。

「故曰,城郭不完,兵甲不多,非國之災也;田野不辟,貨財不聚,非國之害也。上無禮,下無學,賊民興,喪無日矣。《詩》曰:『天之方蹶,無然泄泄。』泄泄,猶沓沓也。事君無義,進退無禮,言則非先王之道者,猶沓沓也。故曰:責難於君謂之恭,陳善閉邪謂之敬,吾君不能謂之賊。」

孟子曰:「規矩,方員之至也;聖人,人倫之至也。欲為君盡君道,欲為臣盡臣道,二者皆法堯舜而已矣。不以舜之所以事堯事君,不敬其君者也;不以堯之所以治民治民,賊其民者也。孔子曰:『道二:仁與不仁而已矣。』暴其民甚,則身弒國亡;不甚,則身危國削。名之曰『幽厲』,雖孝子慈孫,百世不能改也。《詩》云『殷鑒不遠,在夏后之世』,此之謂也。」

孟子曰:「三代之得天下也以仁,其失天下也以不仁。國之所以廢興存亡者亦然。天子不仁,不保四海;諸侯不仁,不保社稷;卿大夫不仁,不保宗廟;士庶人不仁,不保四體。今惡死亡而樂不仁,是猶惡醉而強酒。」

孟子曰:「愛人不親反其仁,治人不治反其智,禮人不答反其敬。行有不得者,皆反求諸己,其身正而天下歸之。《詩》云:『永言配命,自求多福。』」

孟子曰:「人有恆言,皆曰『天下國家』。天下之本在國,國之本在家,家之本在身。」

孟子曰:「為政不難,不得罪於巨室。巨室之所慕,一國慕之;一國之所慕,天下慕之;故沛然德教溢乎四海。」

孟子曰:「天下有道,小德役大德,小賢役大賢;天下無道,小役大,弱役強。斯二者天也。順天者存,逆天者亡。齊景公曰:『既不能令,又不受命,是絕物也。』涕出而女於吳。

今也小國師大國而恥受命焉,是猶弟子而恥受命於先師也。如恥之,莫若師文王。師文王,大國五年,小國七年,必為政於天下矣。《詩》云:『商之孫子,其麗不億。上帝既命,侯于周服。侯服于周,天命靡常。殷士膚敏,祼將于京。』孔子曰:『仁不可為眾也。夫國君好仁,天下無敵。』今也欲無敵於天下而不以仁,是猶執熱而不以濯也。《詩》云:『誰能執熱,逝不以濯?』」

孟子曰:「不仁者可與言哉?安其危而利其菑,樂其所以亡者。不仁而可與言,則何亡國敗家之有?有孺子歌曰:『滄浪之水清兮,可以濯我纓;滄浪之水濁兮,可以濯我足。』孔子曰:『小子聽之!清斯濯纓,濁斯濯足矣,自取之也。』夫人必自侮,然後人侮之;家必自毀,而後人毀之;國必自伐,而後人伐之。《太甲》曰:『天作孽,猶可違;自作孽,不可活。』此之謂也。」

孟子曰:「桀紂之失天下也,失其民也;失其民者,失其心也。得天下有道:得其民,斯得天下矣;得其民有道:得其心,斯得民矣;得其心有道:所欲與之聚之,所惡勿施爾也。民之歸仁也,猶水之就下、獸之走壙也。故為淵敺魚者,獺也;為叢敺爵者,鸇也;為湯武敺民者,桀與紂也。今天下之君有好仁者,則諸侯皆為之敺矣。雖欲無王,不可得已。今之欲王者,猶七年之病求三年之艾也。苟為不畜,終身不得。苟不志於仁,終身憂辱,以陷於死亡。《詩》云『其何能淑,載胥及溺』,此之謂也。」

孟子曰:「自暴者,不可與有言也;自棄者,不可與有為也。言非禮義,謂之自暴也;吾身不能居仁由義,謂之自棄也。仁,人之安宅也;義,人之正路也。曠安宅而弗居,舍正路而不由,哀哉!」

孟子曰:「道在爾而求諸遠,事在易而求之難。人人親其親、長其長而天下平。」

孟子曰:「居下位而不獲於上,民不可得而治也。獲於上有道:不信於友,弗獲於上矣;信於友有道:事親弗悅,弗信於友矣;悅親有道:反身不誠,不悅於親矣;誠身有道:不明乎善,不誠其身矣。是故誠者,天之道也;思誠者,人之道也。至誠而不動者,未之有也;不誠,未有能動者也。」

孟子曰:「伯夷辟紂,居北海之濱,聞文王作,興曰:『盍歸乎來!吾聞西伯善養老者。』太公辟紂,居東海之濱,聞文王作,興曰:『盍歸乎來!吾聞西伯善養老者。』二老者,天下之大老也,而歸之,是天下之父歸之也。天下之父歸之,其子焉往?諸侯有行文王之政者,七年之內,必為政於天下矣。」

孟子曰:「求也為季氏宰,無能改於其德,而賦粟倍他日。孔子曰:『求非我徒也,小子鳴鼓而攻之可也。』由此觀之,君不行仁政而富之,皆棄於孔子者也。況於為之強戰?爭地以戰,殺人盈野;爭城以戰,殺人盈城。此所謂率土地而食人肉,罪不容於死。故善戰者服上刑,連諸侯者次之,辟草萊、任土地者次之。」

孟子曰:「存乎人者,莫良於眸子。眸子不能掩其惡。胸中正,則眸子瞭焉;胸中不正,則眸子眊焉。聽其言也,觀其眸子,人焉廋哉?」

孟子曰:「恭者不侮人,儉者不奪人。侮奪人之君,惟恐不順焉,惡得為恭儉?恭儉豈可以聲音笑貌為哉?」

淳于髡曰:「男女授受不親,禮與?」

孟子曰:「禮也。」

曰:「嫂溺則援之以手乎?」

曰:「嫂溺不援,是豺狼也。男女授受不親,禮也;嫂溺援之以手者,權也。」

曰:「今天下溺矣,夫子之不援,何也?」

曰:「天下溺,援之以道;嫂溺,援之以手。子欲手援天下乎?」

公孫丑曰:「君子之不教子,何也?」

孟子曰:「勢不行也。教者必以正;以正不行,繼之以怒;繼之以怒,則反夷矣。『夫子教我以正,夫子未出於正也。』則是父子相夷也。父子相夷,則惡矣。古者易子而教之。父子之間不責善。責善則離,離則不祥莫大焉。」

孟子曰:「事孰為大?事親為大;守孰為大?守身為大。不失其身而能事其親者,吾聞之矣;失其身而能事其親者,吾未之聞也。孰不為事?事親,事之本也;孰不為守?守身,守之本也。曾子養曾皙,必有酒肉。將徹,必請所與。問有餘,必曰『有』。曾皙死,曾元養曾子,必有酒肉。將徹,不請所與。問有餘,曰『亡矣』。將以復進也。此所謂養口體者也。若曾子,則可謂養志也。事親若曾子者,可也。」

孟子曰:「人不足與適也,政不足間也。惟大人為能格君心之非。君仁莫不仁,君義莫不義,君正莫不正。一正君而國定矣。」

孟子曰:「有不虞之譽,有求全之毀。」

孟子曰:「人之易其言也,無責耳矣。」

孟子曰:「人之患在好為人師。」

樂正子從於子敖之齊。樂正子見孟子。孟子曰:「子亦來見我乎?」曰:「先生何為出此言也?」曰:「子來幾日矣?」曰:「昔昔。」曰:「昔昔,則我出此言也,不亦宜乎?」曰:「舍館未定。」曰:「子聞之也,舍館定,然後求見長者乎?」曰:「克有罪。」

孟子謂樂正子曰:「子之從於子敖來,徒餔啜也。我不意子學古之道,而以餔啜也。」

孟子曰:「不孝有三,無後為大。舜不告而娶,為無後也,君子以為猶告也。」

孟子曰:「仁之實,事親是也;義之實,從兄是也。智之實,知斯二者弗去是也;禮之實,節文斯二者是也;樂之實,樂斯二者,樂則生矣;生則惡可已也,惡可已,則不知足之蹈之、手之舞之。」

孟子曰:「天下大悅而將歸己。視天下悅而歸己,猶草芥也。惟舜為然。不得乎親,不可以為人;不順乎親,不可以為子。舜盡事親之道而瞽瞍厎豫,瞽瞍厎豫而天下化,瞽瞍厎豫而天下之為父子者定,此之謂大孝。」


\end{pinyinscope}