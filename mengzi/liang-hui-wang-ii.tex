\article{梁惠王下}

\begin{pinyinscope}
莊暴見孟子,曰:「暴見於王,王語暴以好樂,暴未有以對也。」曰:「好樂何如?」

孟子曰:「王之好樂甚,則齊國其庶幾乎!」

他日,見於王曰:「王嘗語莊子以好樂,有諸?」王變乎色,曰:「寡人非能好先王之樂也,直好世俗之樂耳。」

曰:「王之好樂甚,則齊其庶幾乎!今之樂猶古之樂也。」曰:「可得聞與?」

曰:「獨樂樂,與人樂樂,孰樂?」曰:「不若與人。」

曰:「與少樂樂,與眾樂樂,孰樂?」曰:「不若與眾。」

「臣請為王言樂:今王鼓樂於此,百姓聞王鐘鼓之聲,管籥之音,舉疾首蹙頞而相告曰:『吾王之好鼓樂,夫何使我至於此極也?父子不相見,兄弟妻子離散。』今王田獵於此,百姓聞王車馬之音,見羽旄之美,舉疾首蹙頞而相告曰:『吾王之好田獵,夫何使我至於此極也?父子不相見,兄弟妻子離散。』此無他,不與民同樂也。」

「今王鼓樂於此,百姓聞王鐘鼓之聲,管籥之音,舉欣欣然有喜色而相告曰:『吾王庶幾無疾病與?何以能鼓樂也?』今王田獵於此,百姓聞王車馬之音,見羽旄之美,舉欣欣然有喜色而相告曰『吾王庶幾無疾病與?何以能田獵也?』此無他,與民同樂也。今王與百姓同樂,則王矣。」

齊宣王問曰:「文王之囿方七十里,有諸?」

孟子對曰:「於傳有之。」

曰:「若是其大乎?」

曰:「民猶以為小也。」

曰:「寡人之囿方四十里,民猶以為大,何也?」

曰:「文王之囿方七十里,芻蕘者往焉,雉兔者往焉,與民同之。民以為小,不亦宜乎?臣始至於境,問國之大禁,然後敢入。臣聞郊關之內有囿方四十里,殺其麋鹿者如殺人之罪。則是方四十里,為阱於國中。民以為大,不亦宜乎?」

齊宣王問曰:「交鄰國有道乎?」

孟子對曰:「有。惟仁者為能以大事小,是故湯事葛,文王事昆夷;惟智者為能以小事大,故大王事獯鬻,句踐事吳。以大事小者,樂天者也;以小事大者,畏天者也。樂天者保天下,畏天者保其國。《詩》云:『畏天之威,于時保之。』」

王曰:「大哉言矣!寡人有疾,寡人好勇。」

對曰:「王請無好小勇。夫撫劍疾視曰,『彼惡敢當我哉』!此匹夫之勇,敵一人者也。王請大之!《詩》云:『王赫斯怒,爰整其旅,以遏徂莒,以篤周祜,以對于天下。』此文王之勇也。文王一怒而安天下之民。《書》曰:『天降下民,作之君,作之師。惟曰其助上帝,寵之四方。有罪無罪,惟我在,天下曷敢有越厥志?』一人衡行於天下,武王恥之。此武王之勇也。而武王亦一怒而安天下之民。今王亦一怒而安天下之民,民惟恐王之不好勇也。」

齊宣王見孟子於雪宮。王曰:「賢者亦有此樂乎?」

孟子對曰:「有。人不得,則非其上矣。不得而非其上者,非也;為民上而不與民同樂者,亦非也。樂民之樂者,民亦樂其樂;憂民之憂者,民亦憂其憂。樂以天下,憂以天下,然而不王者,未之有也。

「昔者齊景公問於晏子曰:『吾欲觀於轉附、朝儛,遵海而南,放于琅邪。吾何脩而可以比於先王觀也?』晏子對曰:『善哉問也!天子適諸侯曰巡狩,巡狩者巡所守也;諸侯朝於天子曰述職,述職者述所職也。無非事者。春省耕而補不足,秋省斂而助不給。夏諺曰:「吾王不遊,吾何以休?吾王不豫,吾何以助?一遊一豫,為諸侯度。」今也不然:師行而糧食,飢者弗食,勞者弗息。睊睊胥讒,民乃作慝。方命虐民,飲食若流。流連荒亡,為諸侯憂。從流下而忘反謂之流,從流上而忘反謂之連,從獸無厭謂之荒,樂酒無厭謂之亡。先王無流連之樂,荒亡之行。惟君所行也。』景公說,大戒於國,出舍於郊。於是始興發補不足。召大師曰:『為我作君臣相說之樂!』蓋徵招角招是也。其詩曰:『畜君何尤?』畜君者,好君也。」

齊宣王問曰:「人皆謂我毀明堂。毀諸?已乎?」

孟子對曰:「夫明堂者,王者之堂也。王欲行王政,則勿毀之矣。」

王曰:「王政可得聞與?」

對曰:「昔者文王之治岐也,耕者九一,仕者世祿,關市譏而不征,澤梁無禁,罪人不孥。老而無妻曰鰥。老而無夫曰寡。老而無子曰獨。幼而無父曰孤。此四者,天下之窮民而無告者。文王發政施仁,必先斯四者。《詩》云:『哿矣富人,哀此煢獨。』」

王曰:「善哉言乎!」

曰:「王如善之,則何為不行?」

王曰:「寡人有疾,寡人好貨。」

對曰:「昔者公劉好貨,《詩》云:『乃積乃倉,乃裹餱糧,于橐于囊。思戢用光。弓矢斯張,干戈戚揚,爰方啟行。』故居者有積倉,行者有裹糧也,然後可以爰方啟行。王如好貨,與百姓同之,於王何有?」

王曰:「寡人有疾,寡人好色。」

對曰:「昔者大王好色,愛厥妃。《詩》云:『古公亶甫,來朝走馬,率西水滸,至于岐下。爰及姜女,聿來胥宇。』當是時也,內無怨女,外無曠夫。王如好色,與百姓同之,於王何有?」

孟子謂齊宣王曰:「王之臣有託其妻子於其友,而之楚遊者。比其反也,則凍餒其妻子,則如之何?」

王曰:「棄之。」

曰:「士師不能治士,則如之何?」

王曰:「已之。」

曰:「四境之內不治,則如之何?」

王顧左右而言他。

孟子見齊宣王曰:「所謂故國者,非謂有喬木之謂也,有世臣之謂也。王無親臣矣,昔者所進,今日不知其亡也。」

王曰:「吾何以識其不才而舍之?」

曰:「國君進賢,如不得已,將使卑踰尊,疏踰戚,可不慎與?左右皆曰賢,未可也;諸大夫皆曰賢,未可也;國人皆曰賢,然後察之;見賢焉,然後用之。左右皆曰不可,勿聽;諸大夫皆曰不可,勿聽;國人皆曰不可,然後察之;見不可焉,然後去之。左右皆曰可殺,勿聽;諸大夫皆曰可殺,勿聽;國人皆曰可殺,然後察之;見可殺焉,然後殺之。故曰,國人殺之也。如此,然後可以為民父母。」

齊宣王問曰:「湯放桀,武王伐紂,有諸?」

孟子對曰:「於傳有之。」

曰:「臣弒其君可乎?」

曰:「賊仁者謂之賊,賊義者謂之殘,殘賊之人謂之一夫。聞誅一夫紂矣,未聞弒君也。」

孟子見齊宣王曰:「為巨室,則必使工師求大木。工師得大木。則王喜,以為能勝其任也。匠人斵而小之,則王怒,以為不勝其任矣。夫人幼而學之,壯而欲行之。王曰『姑舍女所學而從我』,則何如?今有璞玉於此,雖萬鎰,必使玉人彫琢之。至於治國家,則曰『姑舍女所學而從我』,則何以異於教玉人彫琢玉哉?」

齊人伐燕,勝之。宣王問曰:「或謂寡人勿取,或謂寡人取之。以萬乘之國伐萬乘之國,五旬而舉之,人力不至於此。不取,必有天殃。取之,何如?」

孟子對曰:「取之而燕民悅,則取之。古之人有行之者,武王是也。取之而燕民不悅,則勿取。古之人有行之者,文王是也。以萬乘之國伐萬乘之國,簞食壺漿,以迎王師。豈有他哉?避水火也。如水益深,如火益熱,亦運而已矣。」

齊人伐燕,取之。諸侯將謀救燕。宣王曰:「諸侯多謀伐寡人者,何以待之?」

孟子對曰:「臣聞七十里為政於天下者,湯是也。未聞以千里畏人者也。《書》曰:『湯一征,自葛始。』天下信之。『東面而征,西夷怨;南面而征,北狄怨。曰,奚為後我?』民望之,若大旱之望雲霓也。歸市者不止,耕者不變。誅其君而弔其民,若時雨降,民大悅。《書》曰:『徯我后,后來其蘇。』

「今燕虐其民,王往而征之。民以為將拯己於水火之中也,簞食壺漿,以迎王師。若殺其父兄,係累其子弟,毀其宗廟,遷其重器,如之何其可也?天下固畏齊之彊也。今又倍地而不行仁政,是動天下之兵也。王速出令,反其旄倪,止其重器,謀於燕眾,置君而後去之,則猶可及止也。」

鄒與魯鬨。穆公問曰:「吾有司死者三十三人,而民莫之死也。誅之,則不可勝誅;不誅,則疾視其長上之死而不救,如之何則可也?」

孟子對曰:「凶年饑歲,君之民老弱轉乎溝壑,壯者散而之四方者,幾千人矣;而君之倉廩實,府庫充,有司莫以告,是上慢而殘下也。曾子曰:『戒之戒之!出乎爾者,反乎爾者也。』夫民今而後得反之也。君無尤焉。君行仁政,斯民親其上、死其長矣。」

滕文公問曰:「滕,小國也,間於齊楚。事齊乎?事楚乎?」

孟子對曰:「是謀非吾所能及也。無已,則有一焉:鑿斯池也,築斯城也,與民守之,效死而民弗去,則是可為也。」

滕文公問曰:「齊人將築薛,吾甚恐。如之何則可?」

孟子對曰:「昔者大王居邠,狄人侵之,去之岐山之下居焉。非擇而取之,不得已也。苟為善,後世子孫必有王者矣。君子創業垂統,為可繼也。若夫成功,則天也。君如彼何哉?彊為善而已矣。」

滕文公問曰:「滕,小國也。竭力以事大國,則不得免焉。如之何則可?」

孟子對曰:「昔者大王居邠,狄人侵之。事之以皮幣,不得免焉;事之以犬馬,不得免焉;事之以珠玉,不得免焉。乃屬其耆老而告之曰:『狄人之所欲者,吾土地也。吾聞之也:君子不以其所以養人者害人。二三子何患乎無君?我將去之。』去邠,踰梁山,邑于岐山之下居焉。邠人曰:『仁人也,不可失也。』從之者如歸市。或曰:『世守也,非身之所能為也。效死勿去。』君請擇於斯二者。」

魯平公將出。嬖人臧倉者請曰:「他日君出,則必命有司所之。今乘輿已駕矣,有司未知所之。敢請。」公曰:「將見孟子。」曰:「何哉?君所為輕身以先於匹夫者,以為賢乎?禮義由賢者出。而孟子之後喪踰前喪。君無見焉!」公曰:「諾。」

樂正子入見,曰:「君奚為不見孟軻也?」曰:「或告寡人曰,『孟子之後喪踰前喪』,是以不往見也。」曰:「何哉君所謂踰者?前以士,後以大夫;前以三鼎,而後以五鼎與?」曰:「否。謂棺槨衣衾之美也。」曰:「非所謂踰也,貧富不同也。」

樂正子見孟子,曰:「克告於君,君為來見也。嬖人有臧倉者沮君,君是以不果來也。」曰:「行或使之,止或尼之。行止,非人所能也。吾之不遇魯侯,天也。臧氏之子焉能使予不遇哉?」


\end{pinyinscope}