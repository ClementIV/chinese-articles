\article{滕文公上}

\begin{pinyinscope}
滕文公為世子,將之楚,過宋而見孟子。孟子道性善,言必稱堯舜。

世子自楚反,復見孟子。孟子曰:「世子疑吾言乎?夫道一而已矣。成覸謂齊景公曰:『彼丈夫也,我丈夫也,吾何畏彼哉?』顏淵曰:『舜何人也?予何人也?有為者亦若是。』公明儀曰:『文王我師也,周公豈欺我哉?』今滕,絕長補短,將五十里也,猶可以為善國。《書》曰:『若藥不瞑眩,厥疾不瘳。』」

滕定公薨。世子謂然友曰:「昔者孟子嘗與我言於宋,於心終不忘。今也不幸至於大故,吾欲使子問於孟子,然後行事。」

然友之鄒問於孟子。孟子曰:「不亦善乎!親喪固所自盡也。曾子曰:『生事之以禮;死葬之以禮,祭之以禮,可謂孝矣。』諸侯之禮,吾未之學也;雖然,吾嘗聞之矣。三年之喪,齊疏之服,飦粥之食,自天子達於庶人,三代共之。」

然友反命,定為三年之喪。父兄百官皆不欲,曰:「吾宗國魯先君莫之行,吾先君亦莫之行也,至於子之身而反之,不可。且志曰:『喪祭從先祖。』」曰:「吾有所受之也。」謂然友曰:「吾他日未嘗學問,好馳馬試劍。今也父兄百官不我足也,恐其不能盡於大事,子為我問孟子。」

然友復之鄒問孟子。孟子曰:「然。不可以他求者也。孔子曰:『君薨,聽於冢宰。歠粥,面深墨。即位而哭,百官有司,莫敢不哀,先之也。』上有好者,下必有甚焉者矣。『君子之德,風也;小人之德,草也。草尚之風必偃。』是在世子。」

然友反命。世子曰:「然。是誠在我。」五月居廬,未有命戒。百官族人可謂曰知。及至葬,四方來觀之,顏色之戚,哭泣之哀,弔者大悅。

滕文公問為國。孟子曰:「民事不可緩也。《詩》云:『晝爾于茅,宵爾索綯;亟其乘屋,其始播百穀。』民之為道也,有恆產者有恆心,無恆產者無恆心。苟無恆心,放辟邪侈,無不為已。及陷乎罪,然後從而刑之,是罔民也。焉有仁人在位,罔民而可為也?是故賢君必恭儉禮下,取於民有制。陽虎曰:『為富不仁矣,為仁不富矣。』

「夏后氏五十而貢,殷人七十而助,周人百畝而徹,其實皆什一也。徹者,徹也;助者,藉也。龍子曰:『治地莫善於助,莫不善於貢。貢者校數歲之中以為常。樂歲,粒米狼戾,多取之而不為虐,則寡取之;凶年,糞其田而不足,則必取盈焉。為民父母,使民盻盻然,將終歲勤動,不得以養其父母,又稱貸而益之。使老稚轉乎溝壑,惡在其為民父母也?』夫世祿,滕固行之矣。《詩》云:『雨我公田,遂及我私。』惟助為有公田。由此觀之,雖周亦助也。

「設為庠序學校以教之:庠者,養也;校者,教也;序者,射也。夏曰校,殷曰序,周曰庠,學則三代共之,皆所以明人倫也。人倫明於上,小民親於下。有王者起,必來取法,是為王者師也。《詩》云『周雖舊邦,其命惟新』,文王之謂也。子力行之,亦以新子之國。」

使畢戰問井地。孟子曰:「子之君將行仁政,選擇而使子,子必勉之!夫仁政,必自經界始。經界不正,井地不鈞,穀祿不平。是故暴君汙吏必慢其經界。經界既正,分田制祿可坐而定也。夫滕壤地褊小,將為君子焉,將為野人焉。無君子莫治野人,無野人莫養君子。請野九一而助,國中什一使自賦。卿以下必有圭田,圭田五十畝。餘夫二十五畝。死徙無出鄉,鄉田同井。出入相友,守望相助,疾病相扶持,則百姓親睦。方里而井,井九百畝,其中為公田。八家皆私百畝,同養公田。公事畢,然後敢治私事,所以別野人也。此其大略也。若夫潤澤之,則在君與子矣。」

有為神農之言者許行,自楚之滕,踵門而告文公曰:「遠方之人聞君行仁政,願受一廛而為氓。」文公與之處,其徒數十人,皆衣褐,捆屨、織席以為食。

陳良之徒陳相與其弟辛,負耒耜而自宋之滕,曰:「聞君行聖人之政,是亦聖人也,願為聖人氓。」陳相見許行而大悅,盡棄其學而學焉。

陳相見孟子,道許行之言曰:「滕君,則誠賢君也;雖然,未聞道也。賢者與民並耕而食,饔飧而治。今也滕有倉廩府庫,則是厲民而以自養也,惡得賢?」

孟子曰:「許子必種粟而後食乎?」曰:「然。」

「許子必織布而後衣乎?」曰:「否。許子衣褐。」

「許子冠乎?」曰:「冠。」

曰:「奚冠?」曰:「冠素。」

曰:「自織之與?」曰:「否。以粟易之。」

曰:「許子奚為不自織?」曰:「害於耕。」

曰:「許子以釜甑爨,以鐵耕乎?」曰:「然。」

「自為之與?」曰:「否。以粟易之。」

「以粟易械器者,不為厲陶冶;陶冶亦以其械器易粟者,豈為厲農夫哉?且許子何不為陶冶。舍皆取諸其宮中而用之?何為紛紛然與百工交易?何許子之不憚煩?」曰:「百工之事,固不可耕且為也。」

「然則治天下獨可耕且為與?有大人之事,有小人之事。且一人之身,而百工之所為備。如必自為而後用之,是率天下而路也。故曰:或勞心,或勞力;勞心者治人,勞力者治於人;治於人者食人,治人者食於人:天下之通義也。

「當堯之時,天下猶未平,洪水橫流,氾濫於天下。草木暢茂,禽獸繁殖,五穀不登,禽獸偪人。獸蹄鳥跡之道,交於中國。堯獨憂之,舉舜而敷治焉。舜使益掌火,益烈山澤而焚之,禽獸逃匿。禹疏九河,瀹濟漯,而注諸海;決汝漢,排淮泗,而注之江,然後中國可得而食也。當是時也,禹八年於外,三過其門而不入,雖欲耕,得乎?后稷教民稼穡。樹藝五穀,五穀熟而民人育。人之有道也,飽食、煖衣、逸居而無教,則近於禽獸。聖人有憂之,使契為司徒,教以人倫:父子有親,君臣有義,夫婦有別,長幼有序,朋友有信。放勳曰:『勞之來之,匡之直之,輔之翼之,使自得之,又從而振德之。』聖人之憂民如此,而暇耕乎?

「堯以不得舜為己憂,舜以不得禹、皋陶為己憂。夫以百畝之不易為己憂者,農夫也。分人以財謂之惠,教人以善謂之忠,為天下得人者謂之仁。是故以天下與人易,為天下得人難。孔子曰:『大哉堯之為君!惟天為大,惟堯則之,蕩蕩乎民無能名焉!君哉舜也!巍巍乎有天下而不與焉!』堯舜之治天下,豈無所用其心哉?亦不用於耕耳。

「吾聞用夏變夷者,未聞變於夷者也。陳良,楚產也。悅周公、仲尼之道,北學於中國。北方之學者,未能或之先也。彼所謂豪傑之士也。子之兄弟事之數十年,師死而遂倍之。昔者孔子沒,三年之外,門人治任將歸,入揖於子貢,相向而哭,皆失聲,然後歸。子貢反,築室於場,獨居三年,然後歸。他日,子夏、子張、子游以有若似聖人,欲以所事孔子事之,彊曾子。曾子曰:『不可。江漢以濯之,秋陽以暴之,皜皜乎不可尚已。』今也南蠻鴃舌之人,非先王之道,子倍子之師而學之,亦異於曾子矣。吾聞出於幽谷遷于喬木者,末聞下喬木而入於幽谷者。《魯頌》曰:『戎狄是膺,荊舒是懲。』周公方且膺之,子是之學,亦為不善變矣。」

「從許子之道,則市賈不貳,國中無偽。雖使五尺之童適市,莫之或欺。布帛長短同,則賈相若;麻縷絲絮輕重同,則賈相若;五穀多寡同,則賈相若;屨大小同,則賈相若。」曰:「夫物之不齊,物之情也;或相倍蓰,或相什伯,或相千萬。子比而同之,是亂天下也。巨屨小屨同賈,人豈為之哉?從許子之道,相率而為偽者也,惡能治國家?」

墨者夷之,因徐辟而求見孟子。孟子曰:「吾固願見,今吾尚病,病愈,我且往見,夷子不來!」

他日又求見孟子。孟子曰:「吾今則可以見矣。不直,則道不見;我且直之。吾聞夷子墨者。墨之治喪也,以薄為其道也。夷子思以易天下,豈以為非是而不貴也?然而夷子葬其親厚,則是以所賤事親也。」

徐子以告夷子。夷子曰:「儒者之道,古之人『若保赤子』,此言何謂也?之則以為愛無差等,施由親始。」

徐子以告孟子。孟子曰:「夫夷子,信以為人之親其兄之子為若親其鄰之赤子乎?彼有取爾也。赤子匍匐將入井,非赤子之罪也。且天之生物也,使之一本,而夷子二本故也。蓋上世嘗有不葬其親者。其親死,則舉而委之於壑。他日過之,狐狸食之,蠅蚋姑嘬之。其顙有泚,睨而不視。夫泚也,非為人泚,中心達於面目。蓋歸反虆梩而掩之。掩之誠是也,則孝子仁人之掩其親,亦必有道矣。」

徐子以告夷子。夷子憮然為閒曰:「命之矣。」


\end{pinyinscope}