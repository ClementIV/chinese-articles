\article{盡心下}

\begin{pinyinscope}
孟子曰:「不仁哉,梁惠王也!仁者以其所愛及其所不愛,不仁者以其所不愛及其所愛。」

公孫丑問曰:「何謂也?」

「梁惠王以土地之故,糜爛其民而戰之,大敗,將復之,恐不能勝,故驅其所愛子弟以殉之,是之謂以其所不愛及其所愛也。」

孟子曰:「《春秋》無義戰。彼善於此,則有之矣。征者上伐下也,敵國不相征也。」

孟子曰:「盡信《書》,則不如無《書》。吾於《武成》,取二三策而已矣。仁人無敵於天下。以至仁伐至不仁,而何其血之流杵也?」

孟子曰:「有人曰:『我善為陳,我善為戰。』大罪也。國君好仁,天下無敵焉。南面而征北狄怨,東面而征西夷怨。曰:『奚為後我?』武王之伐殷也,革車三百兩,虎賁三千人。王曰:『無畏!寧爾也,非敵百姓也。』若崩厥角稽首。征之為言正也,各欲正己也,焉用戰?」

孟子曰:「梓匠輪輿能與人規矩,不能使人巧。」

孟子曰:「舜之飯糗茹草也,若將終身焉;及其為天子也,被袗衣,鼓琴,二女果,若固有之。」

孟子曰:「吾今而後知殺人親之重也:殺人之父,人亦殺其父;殺人之兄,人亦殺其兄。然則非自殺之也,一閒耳。」

孟子曰:「古之為關也,將以禦暴。今之為關也,將以為暴。」

孟子曰:「身不行道,不行於妻子;使人不以道,不能行於妻子。」

孟子曰:「周于利者,凶年不能殺;周于德者,邪世不能亂。」

孟子曰:「好名之人,能讓千乘之國;苟非其人,簞食豆羹見於色。」

孟子曰:「不信仁賢,則國空虛。無禮義,則上下亂。無政事,則財用不足。」

孟子曰:「不仁而得國者,有之矣;不仁而得天下,未之有也。」

孟子曰:「民為貴,社稷次之,君為輕。是故得乎丘民而為天子,得乎天子為諸侯,得乎諸侯為大夫。諸侯危社稷,則變置。犧牲既成,粢盛既潔,祭祀以時,然而旱乾水溢,則變置社稷。」

孟子曰:「聖人,百世之師也,伯夷、柳下惠是也。故聞伯夷之風者,頑夫廉,懦夫有立志;聞柳下惠之風者,薄夫敦,鄙夫寬。奮乎百世之上。百世之下,聞者莫不興起也。非聖人而能若是乎,而況於親炙之者乎?」

孟子曰:「仁也者,人也。合而言之,道也。」

孟子曰:「孔子之去魯,曰:『遲遲吾行也。』去父母國之道也。去齊,接淅而行,去他國之道也。」

孟子曰:「君子之戹於陳蔡之閒,無上下之交也。」

貉稽曰:「稽大不理於口。」

孟子曰:「無傷也。士憎茲多口。《詩》云:『憂心悄悄,慍于群小。』孔子也。『肆不殄厥慍,亦不隕厥問。』文王也。」

孟子曰:「賢者以其昭昭,使人昭昭;今以其昬昬,使人昭昭。」

孟子謂高子曰:「山徑之蹊閒,介然用之而成路。為閒不用,則茅塞之矣。今茅塞子之心矣。」

高子曰:「禹之聲,尚文王之聲。」

孟子曰:「何以言之?」

曰:「以追蠡。」

曰:「是奚足哉?城門之軌,兩馬之力與?」

齊饑。陳臻曰:「國人皆以夫子將復為發棠,殆不可復。」

孟子曰:「是為馮婦也。晉人有馮婦者,善搏虎,卒為善士。則之野,有眾逐虎。虎負嵎,莫之敢攖。望見馮婦,趨而迎之。馮婦攘臂下車。眾皆悅之,其為士者笑之。」

孟子曰:「口之於味也,目之於色也,耳之於聲也,鼻之於臭也,四肢之於安佚也,性也,有命焉,君子不謂性也。仁之於父子也,義之於君臣也,禮之於賓主也,智之於賢者也,聖人之於天道也,命也,有性焉,君子不謂命也。」

浩生不害問曰:「樂正子,何人也?」

孟子曰:「善人也,信人也。」

「何謂善?何謂信?」

曰:「可欲之謂善,有諸己之謂信。充實之謂美,充實而有光輝之謂大,大而化之之謂聖,聖而不可知之之謂神。樂正子,二之中,四之下也。」

孟子曰:「逃墨必歸於楊,逃楊必歸於儒。歸,斯受之而已矣。今之與楊墨辯者,如追放豚,既入其苙,又從而招之。」

孟子曰:「有布縷之征,粟米之征,力役之征。君子用其一,緩其二。用其二而民有殍,用其三而父子離。」

孟子曰:「諸侯之寶三:土地,人民,政事。寶珠玉者,殃必及身。」

盆成括仕於齊。孟子曰:「死矣盆成括!」

盆成括見殺。門人問曰:「夫子何以知其將見殺?」

曰:「其為人也小有才,未聞君子之大道也,則足以殺其軀而已矣。」

孟子之滕,館於上宮。有業屨於牖上,館人求之弗得。或問之曰:「若是乎從者之廀也?」

曰:「子以是為竊屨來與?」

曰:「殆非也。夫子之設科也,往者不追,來者不距。苟以是心至,斯受之而已矣。」

孟子曰:「人皆有所不忍,達之於其所忍,仁也;人皆有所不為,達之於其所為,義也。人能充無欲害人之心,而仁不可勝用也;人能充無穿踰之心,而義不可勝用也。人能充無受爾汝之實,無所往而不為義也。士未可以言而言,是以言餂之也;可以言而不言,是以不言餂之也,是皆穿踰之類也。」

孟子曰:「言近而指遠者,善言也;守約而施博者,善道也。君子之言也,不下帶而道存焉。君子之守,修其身而天下平。人病舍其田而芸人之田,所求於人者重,而所以自任者輕。」

孟子曰:「堯舜,性者也;湯武,反之也。動容周旋中禮者,盛德之至也;哭死而哀,非為生者也;經德不回,非以干祿也;言語必信,非以正行也。君子行法,以俟命而已矣。」

孟子曰:「說大人,則藐之,勿視其巍巍然。堂高數仞,榱題數尺,我得志弗為也;食前方丈,侍妾數百人,我得志弗為也;般樂飲酒,驅騁田獵,後車千乘,我得志弗為也。在彼者,皆我所不為也;在我者,皆古之制也,吾何畏彼哉?」

孟子曰:「養心莫善於寡欲。其為人也寡欲,雖有不存焉者,寡矣;其為人也多欲,雖有存焉者,寡矣。」

曾皙嗜羊棗,而曾子不忍食羊棗。公孫丑問曰:「膾炙與羊棗孰美?」

孟子曰:「膾炙哉!」

公孫丑曰:「然則曾子何為食膾炙而不食羊棗?」

曰:「膾炙所同也,羊棗所獨也。諱名不諱姓,姓所同也,名所獨也。」

萬章問曰:「孔子在陳曰:『盍歸乎來!吾黨之士狂簡,進取,不忘其初。』孔子在陳,何思魯之狂士?」

孟子曰:「孔子『不得中道而與之,必也狂獧乎!狂者進取,獧者有所不為也』。孔子豈不欲中道哉?不可必得,故思其次也。」

「敢問何如斯可謂狂矣?」

曰:「如琴張、曾皙、牧皮者,孔子之所謂狂矣。」

「何以謂之狂也?」

曰:「其志嘐嘐然,曰『古之人,古之人』。夷考其行而不掩焉者也。狂者又不可得,欲得不屑不潔之士而與之,是獧也,是又其次也。孔子曰:『過我門而不入我室,我不憾焉者,其惟鄉原乎!鄉原,德之賊也。』」

曰:「何如斯可謂之鄉原矣?」

曰:「『何以是嘐嘐也?言不顧行,行不顧言,則曰:古之人,古之人。行何為踽踽涼涼?生斯世也,為斯世也,善斯可矣。』閹然媚於世也者,是鄉原也。」

萬子曰:「一鄉皆稱原人焉,無所往而不為原人,孔子以為德之賊,何哉?」

曰:「非之無舉也,刺之無刺也;同乎流俗,合乎汙世;居之似忠信,行之似廉潔;眾皆悅之,自以為是,而不可與入堯舜之道,故曰德之賊也。孔子曰:『惡似而非者:惡莠,恐其亂苗也;惡佞,恐其亂義也;惡利口,恐其亂信也;惡鄭聲,恐其亂樂也;惡紫,恐其亂朱也;惡鄉原,恐其亂德也。』君子反經而已矣。經正,則庶民興;庶民興,斯無邪慝矣。」

孟子曰:「由堯舜至於湯,五百有餘歲,若禹、皋陶,則見而知之;若湯,則聞而知之。由湯至於文王,五百有餘歲,若伊尹、萊朱則見而知之;若文王,則聞而知之。由文王至於孔子,五百有餘歲,若太公望、散宜生,則見而知之;若孔子,則聞而知之。由孔子而來至於今,百有餘歲,去聖人之世,若此其未遠也;近聖人之居,若此其甚也,然而無有乎爾,則亦無有乎爾。」


\end{pinyinscope}