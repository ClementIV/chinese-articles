\article{公孫丑上}

\begin{pinyinscope}
公孫丑問曰:「夫子當路於齊,管仲、晏子之功,可復許乎?」

孟子曰:「子誠齊人也,知管仲、晏子而已矣。或問乎曾西曰;『吾子與子路孰賢?』曾西蹴然曰:『吾先子之所畏也。』曰:『然則吾子與管仲孰賢?』曾西艴然不悅,曰:『爾何曾比予於管仲?管仲得君,如彼其專也;行乎國政,如彼其久也;功烈,如彼其卑也。爾何曾比予於是?』」曰:「管仲,曾西之所不為也,而子為我願之乎?」

曰:「管仲以其君霸,晏子以其君顯。管仲、晏子猶不足為與?」

曰:「以齊王,由反手也。」

曰:「若是,則弟子之惑滋甚。且以文王之德,百年而後崩,猶未洽於天下;武王、周公繼之,然後大行。今言王若易然,則文王不足法與?」

曰:「文王何可當也?由湯至於武丁,賢聖之君六七作。天下歸殷久矣,久則難變也。武丁朝諸侯有天下,猶運之掌也。紂之去武丁未久也,其故家遺俗,流風善政,猶有存者;又有微子、微仲、王子比干、箕子、膠鬲皆賢人也,相與輔相之,故久而後失之也。尺地莫非其有也,一民莫非其臣也,然而文王猶方百里起,是以難也。齊人有言曰:『雖有智慧,不如乘勢;雖有鎡基,不如待時。』

「今時則易然也。夏后、殷、周之盛,地未有過千里者也,而齊有其地矣;雞鳴狗吠相聞,而達乎四境,而齊有其民矣。地不改辟矣,民不改聚矣,行仁政而王,莫之能禦也。且王者之不作,未有疏於此時者也;民之憔悴於虐政,未有甚於此時者也。飢者易為食,渴者易為飲。孔子曰:『德之流行,速於置郵而傳命。』當今之時,萬乘之國行仁政,民之悅之,猶解倒懸也。故事半古之人,功必倍之,惟此時為然。」

公孫丑問曰:「夫子加齊之卿相,得行道焉,雖由此霸王不異矣。如此,則動心否乎?」

孟子曰:「否。我四十不動心。」

曰:「若是,則夫子過孟賁遠矣。」

曰:「是不難,告子先我不動心。」

曰:「不動心有道乎?」

曰:「有。北宮黝之養勇也,不膚撓,不目逃,思以一豪挫於人,若撻之於市朝。不受於褐寬博,亦不受於萬乘之君。視刺萬乘之君,若刺褐夫。無嚴諸侯。惡聲至,必反之。孟施舍之所養勇也,曰:『視不勝猶勝也。量敵而後進,慮勝而後會,是畏三軍者也。舍豈能為必勝哉?能無懼而已矣。』孟施舍似曾子,北宮黝似子夏。夫二子之勇,未知其孰賢,然而孟施舍守約也。昔者曾子謂子襄曰:『子好勇乎?吾嘗聞大勇於夫子矣:自反而不縮,雖褐寬博,吾不惴焉;自反而縮,雖千萬人,吾往矣。』孟施舍之守氣,又不如曾子之守約也。」

曰:「敢問夫子之不動心,與告子之不動心,可得聞與?」

「告子曰:『不得於言,勿求於心;不得於心,勿求於氣。』不得於心,勿求於氣,可;不得於言,勿求於心,不可。夫志,氣之帥也;氣,體之充也。夫志至焉,氣次焉。故曰:『持其志,無暴其氣。』」

「既曰『志至焉,氣次焉』,又曰『持其志無暴其氣』者,何也?」

曰:「志壹則動氣,氣壹則動志也。今夫蹶者趨者,是氣也,而反動其心。」

「敢問夫子惡乎長?」

曰:「我知言,我善養吾浩然之氣。」

「敢問何謂浩然之氣?」

曰:「難言也。其為氣也,至大至剛,以直養而無害,則塞于天地之閒。其為氣也,配義與道;無是,餒也。是集義所生者,非義襲而取之也。行有不慊於心,則餒矣。我故曰,告子未嘗知義,以其外之也。必有事焉而勿正,心勿忘,勿助長也。無若宋人然:宋人有閔其苗之不長而揠之者,芒芒然歸。謂其人曰:『今日病矣,予助苗長矣。』其子趨而往視之,苗則槁矣。天下之不助苗長者寡矣。以為無益而舍之者,不耘苗者也;助之長者,揠苗者也。非徒無益,而又害之。」

「何謂知言?」

曰:「詖辭知其所蔽,淫辭知其所陷,邪辭知其所離,遁辭知其所窮。生於其心,害於其政;發於其政,害於其事。聖人復起,必從吾言矣。」

「宰我、子貢善為說辭,冉牛、閔子、顏淵善言德行。孔子兼之,曰:『我於辭命則不能也。』然則夫子既聖矣乎?」

曰:「惡!是何言也?昔者子貢、問於孔子曰:『夫子聖矣乎?』孔子曰:『聖則吾不能,我學不厭而教不倦也。』子貢曰:『學不厭,智也;教不倦,仁也。仁且智,夫子既聖矣!』夫聖,孔子不居,是何言也?」

「昔者竊聞之:子夏、子游、子張皆有聖人之一體,冉牛、閔子、顏淵則具體而微。敢問所安。」

曰:「姑舍是。」

曰:「伯夷、伊尹何如?」

曰:「不同道。非其君不事,非其民不使;治則進,亂則退,伯夷也。何事非君,何使非民;治亦進,亂亦進,伊尹也。可以仕則仕,可以止則止,可以久則久,可以速則速,孔子也。皆古聖人也,吾未能有行焉;乃所願,則學孔子也。」

「伯夷、伊尹於孔子,若是班乎?」

曰:「否。自有生民以來,未有孔子也。」

曰:「然則有同與?」

曰:「有。得百里之地而君之,皆能以朝諸侯有天下。行一不義、殺一不辜而得天下,皆不為也。是則同。」

曰:「敢問其所以異?」

曰:「宰我、子貢、有若智足以知聖人。汙,不至阿其所好。宰我曰:『以予觀於夫子,賢於堯舜遠矣。』子貢曰:『見其禮而知其政,聞其樂而知其德。由百世之後,等百世之王,莫之能違也。自生民以來,未有夫子也。』有若曰:『豈惟民哉?麒麟之於走獸,鳳凰之於飛鳥,太山之於丘垤,河海之於行潦,類也。聖人之於民,亦類也。出於其類,拔乎其萃,自生民以來,未有盛於孔子也。』」

孟子曰:「以力假仁者霸,霸必有大國,以德行仁者王,王不待大。湯以七十里,文王以百里。以力服人者,非心服也,力不贍也;以德服人者,中心悅而誠服也,如七十子之服孔子也。《詩》云:『自西自東,自南自北,無思不服。』此之謂也。」

孟子曰:「仁則榮,不仁則辱。今惡辱而居不仁,是猶惡溼而居下也。如惡之,莫如貴德而尊士,賢者在位,能者在職。國家閒暇,及是時明其政刑。雖大國,必畏之矣。《詩》云:『迨天之未陰雨,徹彼桑土,綢繆牖戶。今此下民,或敢侮予?』孔子曰:『為此詩者,其知道乎!能治其國家,誰敢侮之?』今國家閒暇,及是時般樂怠敖,是自求禍也。禍褔無不自己求之者。《詩》云:『永言配命,自求多褔。』《太甲》曰:『天作孽,猶可違;自作孽,不可活。』此之謂也。」

孟子曰:「尊賢使能,俊傑在位,則天下之士皆悅而願立於其朝矣。市廛而不征,法而不廛,則天下之商皆悅而願藏於其市矣。關譏而不征,則天下之旅皆悅而願出於其路矣。耕者助而不稅,則天下之農皆悅而願耕於其野矣。廛無夫里之布,則天下之民皆悅而願為之氓矣。信能行此五者,則鄰國之民仰之若父母矣。率其子弟,攻其父母,自生民以來,未有能濟者也。如此,則無敵於天下。無敵於天下者,天吏也。然而不王者,未之有也。」

孟子曰:「人皆有不忍人之心。先王有不忍人之心,斯有不忍人之政矣。以不忍人之心,行不忍人之政,治天下可運之掌上。所以謂人皆有不忍人之心者,今人乍見孺子將入於井,皆有怵惕惻隱之心。非所以內交於孺子之父母也,非所以要譽於鄉黨朋友也,非惡其聲而然也。由是觀之,無惻隱之心,非人也;無羞惡之心,非人也;無辭讓之心,非人也;無是非之心,非人也。惻隱之心,仁之端也;羞惡之心,義之端也;辭讓之心,禮之端也;是非之心,智之端也。人之有是四端也,猶其有四體也。有是四端而自謂不能者,自賊者也;謂其君不能者,賊其君者也。凡有四端於我者,知皆擴而充之矣,若火之始然,泉之始達。苟能充之,足以保四海;苟不充之,不足以事父母。」

孟子曰:「矢人豈不仁於函人哉?矢人唯恐不傷人,函人唯恐傷人。巫匠亦然,故術不可不慎也。孔子曰:『里仁為美。擇不處仁,焉得智?』夫仁,天之尊爵也,人之安宅也。莫之禦而不仁,是不智也。不仁、不智、無禮、無義,人役也。人役而恥為役,由弓人而恥為弓,矢人而恥為矢也。如恥之,莫如為仁。仁者如射,射者正己而後發。發而不中,不怨勝己者,反求諸己而已矣。」

孟子曰:「子路,人告之以有過則喜。禹聞善言則拜。大舜有大焉,善與人同。舍己從人,樂取於人以為善。自耕、稼、陶、漁以至為帝,無非取於人者。取諸人以為善,是與人為善者也。故君子莫大乎與人為善。」

孟子曰:「伯夷,非其君不事,非其友不友。不立於惡人之朝,不與惡人言。立於惡人之朝,與惡人言,如以朝衣朝冠坐於塗炭。推惡惡之心,思與鄉人立,其冠不正,望望然去之,若將浼焉。是故諸侯雖有善其辭命而至者,不受也。不受也者,是亦不屑就已。柳下惠,不羞汙君,不卑小官。進不隱賢,必以其道。遺佚而不怨,阨窮而不憫。故曰:『爾為爾,我為我,雖袒裼裸裎於我側,爾焉能浼我哉?』故由由然與之偕而不自失焉,援而止之而止。援而止之而止者,是亦不屑去已。」

孟子曰:「伯夷隘,柳下惠不恭。隘與不恭,君子不由也。」


\end{pinyinscope}