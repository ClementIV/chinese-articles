\article{萬章下}

\begin{pinyinscope}
孟子曰:「伯夷,目不視惡色,耳不聽惡聲。非其君不事,非其民不使。治則進,亂則退。橫政之所出,橫民之所止,不忍居也。思與鄉人處,如以朝衣朝冠坐於塗炭也。當紂之時,居北海之濱,以待天下之清也。故聞伯夷之風者,頑夫廉,懦夫有立志。

「伊尹曰:『何事非君?何使非民?』治亦進,亂亦進。曰:『天之生斯民也,使先知覺後知,使先覺覺後覺。予,天民之先覺者也;予將以此道覺此民也。』思天下之民匹夫匹婦有不與被堯舜之澤者,若己推而內之溝中,其自任以天下之重也。

「柳下惠,不羞汙君,不辭小官。進不隱賢,必以其道。遺佚而不怨,阨窮而不憫。與鄉人處,由由然不忍去也。『爾為爾,我為我,雖袒裼裸裎於我側,爾焉能浼我哉?』故聞柳下惠之風者,鄙夫寬,薄夫敦。

「孔子之去齊,接淅而行;去魯,曰:『遲遲吾行也。』去父母國之道也。可以速而速,可以久而久,可以處而處,可以仕而仕,孔子也。」

孟子曰:「伯夷,聖之清者也;伊尹,聖之任者也;柳下惠,聖之和者也;孔子,聖之時者也。孔子之謂集大成。集大成也者,金聲而玉振之也。金聲也者,始條理也;玉振之也者,終條理也。始條理者,智之事也;終條理者,聖之事也。智,譬則巧也;聖,譬則力也。由射於百步之外也,其至,爾力也;其中,非爾力也。」

北宮錡問曰:「周室班爵祿也,如之何?」

孟子曰:「其詳不可得聞也。諸侯惡其害己也,而皆去其籍。然而軻也,嘗聞其略也。天子一位,公一位,侯一位,伯一位,子、男同一位,凡五等也。君一位,卿一位,大夫一位,上士一位,中士一位,下士一位,凡六等。

「天子之制,地方千里,公侯皆方百里,伯七十里,子、男五十里,凡四等。不能五十里,不達於天子,附於諸侯,曰附庸。天子之卿受地視侯,大夫受地視伯,元士受地視子、男。

「大國地方百里,君十卿祿,卿祿四大夫,大夫倍上士,上士倍中士,中士倍下士,下士與庶人在官者同祿,祿足以代其耕也。次國地方七十里,君十卿祿,卿祿三大夫,大夫倍上士,上士倍中士,中士倍下士,下士與庶人在官者同祿,祿足以代其耕也。小國地方五十里,君十卿祿,卿祿二大夫,大夫倍上士,上士倍中士,中士倍下士,下士與庶人在官者同祿,祿足以代其耕也。耕者之所獲,一夫百畝。百畝之糞,上農夫食九人,上次食八人,中食七人,中次食六人,下食五人。庶人在官者,其祿以是為差。」

萬章問曰:「敢問友。」

孟子曰:「不挾長,不挾貴,不挾兄弟而友。友也者,友其德也,不可以有挾也。孟獻子,百乘之家也,有友五人焉:樂正裘、牧仲,其三人,則予忘之矣。獻子之與此五人者友也,無獻子之家者也。此五人者,亦有獻子之家,則不與之友矣。非惟百乘之家為然也。雖小國之君亦有之。費惠公曰:『吾於子思,則師之矣;吾於顏般,則友之矣;王順、長息則事我者也。』非惟小國之君為然也,雖大國之君亦有之。晉平公之於亥唐也,入云則入,坐云則坐,食云則食。雖疏食菜羹,未嘗不飽,蓋不敢不飽也。然終於此而已矣。弗與共天位也,弗與治天職也,弗與食天祿也,士之尊賢者也,非王公之尊賢也。舜尚見帝,帝館甥于貳室,亦饗舜,迭為賓主,是天子而友匹夫也。用下敬上,謂之貴貴;用上敬下,謂之尊賢。貴貴、尊賢,其義一也。」

萬章問曰:「敢問交際何心也?」

孟子曰:「恭也。」

曰:「卻之卻之為不恭,何哉?」

曰:「尊者賜之,曰『其所取之者,義乎,不義乎」,而後受之,以是為不恭,故弗卻也。」

曰:「請無以辭卻之,以心卻之,曰『其取諸民之不義也』,而以他辭無受,不可乎?」

曰:「其交也以道,其接也以禮,斯孔子受之矣。」

萬章曰:「今有禦人於國門之外者,其交也以道,其餽也以禮,斯可受禦與?」

曰:「不可。《康誥》曰:『殺越人于貨,閔不畏死,凡民罔不譈。』是不待教而誅者也。殷受夏,周受殷,所不辭也。於今為烈,如之何其受之?」

曰:「今之諸侯取之於民也,猶禦也。苟善其禮際矣,斯君子受之,敢問何說也?」

曰:「子以為有王者作,將比今之諸侯而誅之乎?其教之不改而後誅之乎?夫謂非其有而取之者盜也,充類至義之盡也。孔子之仕於魯也,魯人獵較,孔子亦獵較。獵較猶可,而況受其賜乎?」

曰:「然則孔子之仕也,非事道與?」

曰:「事道也。」

「事道奚獵較也?」

曰:「孔子先簿正祭器,不以四方之食供簿正。」

曰:「奚不去也?」

曰:「為之兆也。兆足以行矣,而不行,而後去,是以未嘗有所終三年淹也。孔子有見行可之仕,有際可之仕,有公養之仕也。於季桓子,見行可之仕也;於衛靈公,際可之仕也;於衛孝公,公養之仕也。」

孟子曰:「仕非為貧也,而有時乎為貧;娶妻非為養也,而有時乎為養。為貧者,辭尊居卑,辭富居貧。辭尊居卑,辭富居貧,惡乎宜乎?抱關擊柝。孔子嘗為委吏矣,曰『會計當而已矣』。嘗為乘田矣,曰『牛羊茁壯,長而已矣』。位卑而言高,罪也;立乎人之本朝,而道不行,恥也」

萬章曰:「士之不託諸侯,何也?」

孟子曰:「不敢也。諸侯失國,而後託於諸侯,禮也;士之託於諸侯,非禮也。」

萬章曰:「君餽之粟,則受之乎?」

曰:「受之。」

「受之何義也?」

曰:「君之於氓也,固周之。」

曰:「周之則受,賜之則不受,何也?」

曰:「不敢也。」

曰:「敢問其不敢何也?」

曰:「抱關擊柝者,皆有常職以食於上。無常職而賜於上者,以為不恭也。」

曰:「君餽之,則受之,不識可常繼乎?」

曰:「繆公之於子思也,亟問,亟餽鼎肉。子思不悅。於卒也,摽使者出諸大門之外,北面稽首再拜而不受。曰:『今而後知君之犬馬畜伋。』蓋自是臺無餽也。悅賢不能舉,又不能養也,可謂悅賢乎?」

曰:「敢問國君欲養君子,如何斯可謂養矣?」

曰:「以君命將之,再拜稽首而受。其後廩人繼粟,庖人繼肉,不以君命將之。子思以為鼎肉,使己僕僕爾亟拜也,非養君子之道也。堯之於舜也,使其子九男事之,二女女焉,百官牛羊倉廩備,以養舜於畎畝之中,後舉而加諸上位。故曰:王公之尊賢者也。」

萬章曰:「敢問不見諸侯,何義也?」

孟子曰:「在國曰市井之臣,在野曰草莽之臣,皆謂庶人。庶人不傳質為臣,不敢見於諸侯,禮也。」

萬章曰:「庶人,召之役,則往役;君欲見之,召之,則不往見之,何也?」

曰:「往役,義也;往見,不義也。且君之欲見之也,何為也哉?」

曰:「為其多聞也,為其賢也。」

曰:「為其多聞也,則天子不召師,而況諸侯乎?為其賢也,則吾未聞欲見賢而召之也。繆公亟見於子思,曰:『古千乘之國以友士,何如?』子思不悅,曰:『古之人有言:曰事之云乎,豈曰友之云乎?』子思之不悅也,豈不曰:『以位,則子,君也;我,臣也。何敢與君友也?以德,則子事我者也。奚可以與我友?』千乘之君求與之友,而不可得也,而況可召與?齊景公田,招虞人以旌,不至,將殺之。志士不忘在溝壑,勇士不忘喪其元。孔子奚取焉?取非其招不往也。」

曰:「敢問招虞人何以?」

曰:「以皮冠。庶人以旃,士以旂,大夫以旌。以大夫之招招虞人,虞人死不敢往。以士之招招庶人,庶人豈敢往哉。況乎以不賢人之招招賢人乎?欲見賢人而不以其道,猶欲其入而閉之門也。夫義,路也;禮,門也。惟君子能由是路,出入是門也。《詩》云:『周道如砥,其直如矢;君子所履,小人所視。』」

萬章曰:「孔子,君命召,不俟駕而行。然則孔子非與?」

曰:「孔子當仕有官職,而以其官召之也。」

孟子謂萬章曰:「一鄉之善士,斯友一鄉之善士;一國之善士,斯友一國之善士;天下之善士,斯友天下之善士。以友天下之善士為未足,又尚論古之人。頌其詩,讀其書,不知其人,可乎?是以論其世也。是尚友也。」

齊宣王問卿。孟子曰:「王何卿之問也?」

王曰:「卿不同乎?」

曰:「不同。有貴戚之卿,有異姓之卿。」

王曰:「請問貴戚之卿。」

曰:「君有大過則諫,反覆之而不聽,則易位。」

王勃然變乎色。曰:「王勿異也。王問臣,臣不敢不以正對。」

王色定,然後請問異姓之卿。曰:「君有過則諫,反覆之而不聽,則去。」


\end{pinyinscope}