\article{告子下}

\begin{pinyinscope}
任人有問屋廬子曰:「禮與食孰重?」曰:「禮重。」

「色與禮孰重?」曰:「禮重。」

曰:「以禮食,則飢而死;不以禮食,則得食,必以禮乎?親迎,則不得妻;不親迎,則得妻,必親迎乎!」

屋廬子不能對,明日之鄒以告孟子。孟子曰:「於答是也何有?不揣其本而齊其末,方寸之木可使高於岑樓。金重於羽者,豈謂一鉤金與一輿羽之謂哉?取食之重者,與禮之輕者而比之,奚翅食重?取色之重者,與禮之輕者而比之,奚翅色重?往應之曰:『紾兄之臂而奪之食,則得食;不紾,則不得食,則將紾之乎?踰東家牆而摟其處子,則得妻;不摟,則不得妻,則將摟之乎?』」

曹交問曰:「人皆可以為堯舜,有諸?」

孟子曰:「然。」

「交聞文王十尺,湯九尺,今交九尺四寸以長,食粟而已,如何則可?」

曰:「奚有於是?亦為之而已矣。有人於此,力不能勝一匹雛,則為無力人矣;今曰舉百鈞,則為有力人矣。然則舉烏獲之任,是亦為烏獲而已矣。夫人豈以不勝為患哉?弗為耳。徐行後長者謂之弟,疾行先長者謂之不弟。夫徐行者,豈人所不能哉?所不為也。堯舜之道,孝弟而已矣。子服堯之服,誦堯之言,行堯之行,是堯而已矣;子服桀之服,誦桀之言,行桀之行,是桀而已矣。」

曰:「交得見於鄒君,可以假館,願留而受業於門。」

曰:「夫道,若大路然,豈難知哉?人病不求耳。子歸而求之,有餘師。」

公孫丑問曰:「高子曰:『《小弁》,小人之詩也。』」

孟子曰:「何以言之?」

曰:「怨。」

曰:「固哉,高叟之為《詩》也!有人於此,越人關弓而射之,則己談笑而道之;無他,疏之也。其兄關弓而射之,則己垂涕泣而道之;無他,戚之也。小弁之怨,親親也。親親,仁也。固矣夫,高叟之為《詩》也!」

曰:「《凱風》何以不怨?」

曰:「《凱風》,親之過小者也;《小弁》,親之過大者也。親之過大而不怨,是愈疏也;親之過小而怨,是不可磯也。愈疏,不孝也;不可磯,亦不孝也。孔子曰:『舜其至孝矣,五十而慕。』」

宋牼將之楚,孟子遇於石丘。曰:「先生將何之?」

曰:「吾聞秦楚構兵,我將見楚王說而罷之。楚王不悅,我將見秦王說而罷之,二王我將有所遇焉。」

曰:「軻也請無問其詳,願聞其指。說之將何如?」

曰:「我將言其不利也。」

曰:「先生之志則大矣,先生之號則不可。先生以利說秦楚之王,秦楚之王悅於利,以罷三軍之師,是三軍之士樂罷而悅於利也。為人臣者懷利以事其君,為人子者懷利以事其父,為人弟者懷利以事其兄。是君臣、父子、兄弟終去仁義,懷利以相接,然而不亡者,未之有也。先生以仁義說秦楚之王,秦楚之王悅於仁義,而罷三軍之師,是三軍之士樂罷而悅於仁義也。為人臣者懷仁義以事其君,為人子者懷仁義以事其父,為人弟者懷仁義以事其兄,是君臣、父子、兄弟去利,懷仁義以相接也。然而不王者,未之有也。何必曰利?」

孟子居鄒,季任為任處守,以幣交,受之而不報。處於平陸,儲子為相,以幣交,受之而不報。他日由鄒之任,見季子;由平陸之齊,不見儲子。屋廬子喜曰:「連得閒矣。」問曰:「夫子之任見季子,之齊不見儲子,為其為相與?」

曰:「非也。《書》曰:『享多儀,儀不及物曰不享,惟不役志于享。』為其不成享也。」

屋廬子悅。或問之。屋廬子曰:「季子不得之鄒,儲子得之平陸。」

淳于髡曰:「先名實者,為人也;後名實者,自為也。夫子在三卿之中,名實未加於上下而去之,仁者固如此乎?」

孟子曰:「居下位,不以賢事不肖者,伯夷也;五就湯,五就桀者,伊尹也;不惡汙君,不辭小官者,柳下惠也。三子者不同道,其趨一也。一者何也?曰:仁也。君子亦仁而已矣,何必同?」

曰:「魯繆公之時,公儀子為政,子柳、子思為臣,魯之削也滋甚。若是乎賢者之無益於國也!」曰:「虞不用百里奚而亡,秦穆公用之而霸。不用賢則亡,削何可得與?」

曰:「昔者王豹處於淇,而河西善謳;緜駒處於高唐,而齊右善歌;華周、杞梁之妻善哭其夫,而變國俗。有諸內必形諸外。為其事而無其功者,髡未嘗覩之也。是故無賢者也,有則髡必識之。」

曰:「孔子為魯司寇,不用,從而祭,燔肉不至,不稅冕而行。不知者以為為肉也。其知者以為為無禮也。乃孔子則欲以微罪行,不欲為苟去。君子之所為,眾人固不識也。」

孟子曰:「五霸者,三王之罪人也;今之諸侯,五霸之罪人也;今之大夫,今之諸侯之罪人也。天子適諸侯曰巡狩,諸侯朝於天子曰述職。春省耕而補不足,秋省斂而助不給。入其疆,土地辟,田野治,養老尊賢,俊傑在位,則有慶,慶以地。入其疆,土地荒蕪,遺老失賢,掊克在位,則有讓。一不朝,則貶其爵;再不朝,則削其地;三不朝,則六師移之。是故天子討而不伐,諸侯伐而不討。五霸者,摟諸侯以伐諸侯者也,故曰:五霸者,三王之罪人也。

「五霸,桓公為盛。葵丘之會諸侯,束牲、載書而不歃血。初命曰:『誅不孝,無易樹子,無以妾為妻。』再命曰:『尊賢育才,以彰有德。』三命曰:『敬老慈幼,無忘賓旅。』四命曰:『士無世官,官事無攝,取士必得,無專殺大夫。』五命曰:『無曲防,無遏糴,無有封而不告。』曰:『凡我同盟之人,既盟之後,言歸于好。』今之諸侯,皆犯此五禁,故曰:今之諸侯,五霸之罪人也。

「長君之惡其罪小,逢君之惡其罪大。今之大夫,皆逢君之惡,故曰:今之大夫,今之諸侯之罪人也。」

魯欲使慎子為將軍。孟子曰:「不教民而用之,謂之殃民。殃民者,不容於堯舜之世。一戰勝齊,遂有南陽,然且不可。」

慎子勃然不悅曰:「此則滑釐所不識也。」

曰:「吾明告子。天子之地方千里;不千里,不足以待諸侯。諸侯之地方百里;不百里,不足以守宗廟之典籍。周公之封於魯,為方百里也;地非不足,而儉於百里。太公之封於齊也,亦為方百里也;地非不足也,而儉於百里。今魯方百里者五,子以為有王者作,則魯在所損乎?在所益乎?徒取諸彼以與此,然且仁者不為,況於殺人以求之乎?君子之事君也,務引其君以當道,志於仁而已。」

孟子曰:「今之事君者曰:『我能為君辟土地,充府庫。』今之所謂良臣,古之所謂民賊也。君不鄉道,不志於仁,而求富之,是富桀也。『我能為君約與國,戰必克。』今之所謂良臣,古之所謂民賊也。君不鄉道,不志於仁,而求為之強戰,是輔桀也。由今之道,無變今之俗,雖與之天下,不能一朝居也。」

白圭曰:「吾欲二十而取一,何如?」

孟子曰:「子之道,貉道也。萬室之國,一人陶,則可乎?」

曰:「不可,器不足用也。」

曰:「夫貉,五穀不生,惟黍生之。無城郭、宮室、宗廟、祭祀之禮,無諸侯幣帛饔飧,無百官有司,故二十取一而足也。今居中國,去人倫,無君子,如之何其可也?陶以寡,且不可以為國,況無君子乎?欲輕之於堯舜之道者,大貉小貉也;欲重之於堯舜之道者,大桀小桀也。」

白圭曰:「丹之治水也愈於禹。」

孟子曰:「子過矣。禹之治水,水之道也。是故禹以四海為壑,今吾子以鄰國為壑。水逆行,謂之洚水。洚水者,洪水也,仁人之所惡也。吾子過矣。」

孟子曰:「君子不亮,惡乎執?」

魯欲使樂正子為政。

孟子曰:「吾聞之,喜而不寐。」

公孫丑曰:「樂正子強乎?」曰:「否。」

「有知慮乎?」曰:「否。」

「多聞識乎?」曰:「否。」

「然則奚為喜而不寐?」曰:「其為人也好善。」

「好善足乎?」曰:「好善優於天下,而況魯國乎?夫苟好善,則四海之內,皆將輕千里而來告之以善。夫苟不好善,則人將曰:『訑訑,予既已知之矣。』訑訑之聲音顏色,距人於千里之外。士止於千里之外,則讒諂面諛之人至矣。與讒諂面諛之人居,國欲治,可得乎?」

陳子曰:「古之君子何如則仕?」

孟子曰:「所就三,所去三。迎之致敬以有禮,言將行其言也,則就之;禮貌未衰,言弗行也,則去之。其次,雖未行其言也,迎之致敬以有禮,則就之;禮貌衰,則去之。其下,朝不食,夕不食,飢餓不能出門戶。君聞之曰:『吾大者不能行其道,又不能從其言也,使飢餓於我土地,吾恥之。』周之,亦可受也,免死而已矣。」

孟子曰:「舜發於畎畝之中,傅說舉於版築之閒,膠鬲舉於魚鹽之中,管夷吾舉於士,孫叔敖舉於海,百里奚舉於市。故天將降大任於是人也,必先苦其心志,勞其筋骨,餓其體膚,空乏其身,行拂亂其所為,所以動心忍性,曾益其所不能。人恒過,然後能改;困於心,衡於慮,而後作;徵於色,發於聲,而後喻。入則無法家拂士,出則無敵國外患者,國恒亡。然後知生於憂患而死於安樂也。」

孟子曰:「教亦多術矣,予不屑之教誨也者,是亦教誨之而已矣。」


\end{pinyinscope}