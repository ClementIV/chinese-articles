\article{離婁下}

\begin{pinyinscope}
孟子曰:「舜生於諸馮,遷於負夏,卒於鳴條,東夷之人也。文王生於岐周,卒於畢郢,西夷之人也。地之相去也,千有餘里;世之相後也,千有餘歲。得志行乎中國,若合符節。先聖後聖,其揆一也。」

子產聽鄭國之政,以其乘輿濟人於溱洧。孟子曰:「惠而不知為政。歲十一月徒杠成,十二月輿梁成,民未病涉也。君子平其政,行辟人可也。焉得人人而濟之?故為政者,每人而悅之,日亦不足矣。」

孟子告齊宣王曰:「君之視臣如手足,則臣視君如腹心;君之視臣如犬馬,則臣視君如國人;君之視臣如土芥,則臣視君如寇讎。」

王曰:「禮,為舊君有服,何如斯可為服矣?」

曰:「諫行言聽,膏澤下於民;有故而去,則君使人導之出疆,又先於其所往;去三年不反,然後收其田里。此之謂三有禮焉。如此,則為之服矣。今也為臣。諫則不行,言則不聽;膏澤不下於民;有故而去,則君搏執之,又極之於其所往;去之日,遂收其田里。此之謂寇讎。寇讎何服之有?」

孟子曰:「無罪而殺士,則大夫可以去;無罪而戮民,則士可以徙。」

孟子曰:「君仁莫不仁,君義莫不義。」

孟子曰:「非禮之禮,非義之義,大人弗為。」

孟子曰:「中也養不中,才也養不才,故人樂有賢父兄也。如中也棄不中,才也棄不才,則賢不肖之相去,其閒不能以寸。」

孟子曰:「人有不為也,而後可以有為。」

孟子曰:「言人之不善,當如後患何?」

孟子曰:「仲尼不為已甚者。」

孟子曰:「大人者,言不必信,行不必果,惟義所在。」

孟子曰:「大人者,不失其赤子之心者也。」

孟子曰:「養生者不足以當大事,惟送死可以當大事。」

孟子曰:「君子深造之以道,欲其自得之也。自得之,則居之安;居之安,則資之深;資之深,則取之左右逢其原,故君子欲其自得之也。」

孟子曰:「博學而詳說之,將以反說約也。」

孟子曰:「以善服人者,未有能服人者也;以善養人,然後能服天下。天下不心服而王者,未之有也。」

孟子曰:「言無實不祥。不祥之實,蔽賢者當之。」

徐子曰:「仲尼亟稱於水,曰:『水哉,水哉!』何取於水也?」

孟子曰:「原泉混混,不舍晝夜。盈科而後進,放乎四海,有本者如是,是之取爾。苟為無本,七八月之閒雨集,溝澮皆盈;其涸也,可立而待也。故聲聞過情,君子恥之。」

孟子曰:「人之所以異於禽於獸者幾希,庶民去之,君子存之。舜明於庶物,察於人倫,由仁義行,非行仁義也。」

孟子曰:「禹惡旨酒而好善言。湯執中,立賢無方。文王視民如傷,望道而未之見。武王不泄邇,不忘遠。周公思兼三王,以施四事;其有不合者,仰而思之,夜以繼日;幸而得之,坐以待旦。」

孟子曰:「王者之迹熄而詩亡,詩亡然後春秋作。晉之乘,楚之檮杌,魯之春秋,一也。其事則齊桓、晉文,其文則史。孔子曰:『其義則丘竊取之矣。』」

孟子曰:「君子之澤五世而斬,小人之澤五世而斬。予未得為孔子徒也,予私淑諸人也。」

孟子曰:「可以取,可以無取,取傷廉;可以與,可以無與,與傷惠;可以死,可以無死,死傷勇。」

逄蒙學射於羿,盡羿之道,思天下惟羿為愈己,於是殺羿。孟子曰:「是亦羿有罪焉。」公明儀曰:「宜若無罪焉。」曰:「薄乎云爾,惡得無罪?鄭人使子濯孺子侵衛,衛使庾公之斯追之。子濯孺子曰:『今日我疾作,不可以執弓,吾死矣夫!』問其僕曰:『追我者誰也?』其僕曰:『庾公之斯也。』曰:『吾生矣。』其僕曰:『庾公之斯,衛之善射者也,夫子曰「吾生」,何謂也?』曰:『庾公之斯學射於尹公之他,尹公之他學射於我。夫尹公之他,端人也,其取友必端矣。』庾公之斯至,曰:『夫子何為不執弓?』曰:『今日我疾作,不可以執弓。』曰:『小人學射於尹公之他,尹公之他學射於夫子。我不忍以夫子之道反害夫子。雖然,今日之事,君事也,我不敢廢。』抽矢扣輪,去其金,發乘矢而後反。」

孟子曰:「西子蒙不潔,則人皆掩鼻而過之。雖有惡人,齊戒沐浴,則可以祀上帝。」

孟子曰:「天下之言性也,則故而已矣。故者以利為本。所惡於智者,為其鑿也。如智者若禹之行水也,則無惡於智矣。禹之行水也,行其所無事也。如智者亦行其所無事,則智亦大矣。天之高也,星辰之遠也,苟求其故,千歲之日至,可坐而致也。」

公行子有子之喪,右師往弔,入門,有進而與右師言者,有就右師之位而與右師言者。孟子不與右師言,右師不悅曰:「諸君子皆與驩言,孟子獨不與驩言,是簡驩也。」

孟子聞之,曰:「禮,朝廷不歷位而相與言,不踰階而相揖也。我欲行禮,子敖以我為簡,不亦異乎?」

孟子曰:「君子所以異於人者,以其存心也。君子以仁存心,以禮存心。仁者愛人,有禮者敬人。愛人者人恆愛之,敬人者人恆敬之。有人於此,其待我以橫逆,則君子必自反也:我必不仁也,必無禮也,此物奚宜至哉?其自反而仁矣,自反而有禮矣,其橫逆由是也,君子必自反也:我必不忠。自反而忠矣,其橫逆由是也,君子曰:『此亦妄人也已矣。如此則與禽獸奚擇哉?於禽獸又何難焉?』是故君子有終身之憂,無一朝之患也。乃若所憂則有之:舜人也,我亦人也。舜為法於天下,可傳於後世,我由未免為鄉人也,是則可憂也。憂之如何?如舜而已矣。若夫君子所患則亡矣。非仁無為也,非禮無行也。如有一朝之患,則君子不患矣。」

禹、稷當平世,三過其門而不入,孔子賢之。顏子當亂世,居於陋巷。一簞食,一瓢飲。人不堪其憂,顏子不改其樂,孔子賢之。孟子曰:「禹、稷、顏回同道。禹思天下有溺者,由己溺之也;稷思天下有飢者,由己飢之也,是以如是其急也。禹、稷、顏子易地則皆然。今有同室之人鬬者,救之,雖被髮纓冠而救之,可也。鄉鄰有鬬者,被髮纓冠而往救之,則惑也,雖閉戶可也。」

公都子曰:「匡章,通國皆稱不孝焉。夫子與之遊,又從而禮貌之,敢問何也?」

孟子曰:「世俗所謂不孝者五:惰其四支,不顧父母之養,一不孝也;博弈好飲酒,不顧父母之養,二不孝也;好貨財,私妻子,不顧父母之養,三不孝也;從耳目之欲,以為父母戮,四不孝也;好勇鬥很,以危父母,五不孝也。章子有一於是乎?夫章子,子父責善而不相遇也。責善,朋友之道也;父子責善,賊恩之大者。夫章子,豈不欲有夫妻子母之屬哉?為得罪於父,不得近。出妻屏子,終身不養焉。其設心以為不若是,是則罪之大者,是則章子已矣。

曾子居武城,有越寇。或曰:「寇至,盍去諸?」曰:「無寓人於我室,毀傷其薪木。」寇退,則曰:「修我牆屋,我將反。」寇退,曾子反。左右曰:「待先生,如此其忠且敬也。寇至則先去以為民望,寇退則反,殆於不可。」沈猶行曰:「是非汝所知也。昔沈猶有負芻之禍,從先生者七十人,未有與焉。」

子思居於衛,有齊寇。或曰:「寇至,盍去諸?」子思曰:「如伋去,君誰與守?」

孟子曰:「曾子、子思同道。曾子,師也,父兄也;子思,臣也,微也。曾子、子思易地則皆然。」

儲子曰:「王使人瞷夫子,果有以異於人乎?」

孟子曰:「何以異於人哉?堯舜與人同耳。」

齊人有一妻一妾而處室者,其良人出,則必饜酒肉而後反。其妻問所與飲食者,則盡富貴也。其妻告其妾曰:「良人出,則必饜酒肉而後反;問其與飲食者,盡富貴也,而未嘗有顯者來,吾將瞷良人之所之也。」

蚤起,施從良人之所之,遍國中無與立談者。卒之東郭墦閒,之祭者,乞其餘;不足,又顧而之他,此其為饜足之道也。其妻歸,告其妾曰:「良人者,所仰望而終身也。今若此。」與其妾訕其良人,而相泣於中庭。而良人未之知也,施施從外來,驕其妻妾。

由君子觀之,則人之所以求富貴利達者,其妻妾不羞也,而不相泣者,幾希矣。


\end{pinyinscope}