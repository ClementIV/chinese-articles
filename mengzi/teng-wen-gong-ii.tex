\article{滕文公下}

\begin{pinyinscope}
陳代曰:「不見諸侯,宜若小然;今一見之,大則以王,小則以霸。且志曰:『枉尺而直尋』,宜若可為也。」

孟子曰:「昔齊景公田,招虞人以旌,不至,將殺之。志士不忘在溝壑,勇士不忘喪其元。孔子奚取焉?取非其招不往也,如不待其招而往,何哉?且夫枉尺而直尋者,以利言也。如以利,則枉尋直尺而利,亦可為與?昔者趙簡子使王良與嬖奚乘,終日而不獲一禽。嬖奚反命曰:『天下之賤工也。』或以告王良。良曰:『請復之。』彊而後可,一朝而獲十禽。嬖奚反命曰:『天下之良工也。』簡子曰:『我使掌與女乘。』謂王良。良不可,曰:『吾為之範我馳驅,終日不獲一;為之詭遇,一朝而獲十。《詩》云:「不失其馳,舍矢如破。」我不貫與小人乘,請辭。』御者且羞與射者比。比而得禽獸,雖若丘陵,弗為也。如枉道而從彼,何也?且子過矣,枉己者,未有能直人者也。」

景春曰:「公孫衍、張儀豈不誠大丈夫哉?一怒而諸侯懼,安居而天下熄。」

孟子曰:「是焉得為大丈夫乎?子未學禮乎?丈夫之冠也,父命之;女子之嫁也,母命之,往送之門,戒之曰:『往之女家,必敬必戒,無違夫子!』以順為正者,妾婦之道也。居天下之廣居,立天下之正位,行天下之大道。得志與民由之,不得志獨行其道。富貴不能淫,貧賤不能移,威武不能屈。此之謂大丈夫。」

周霄問曰:「古之君子仕乎?」

孟子曰:「仕。傳曰:『孔子三月無君,則皇皇如也,出疆必載質。』公明儀曰:『古之人三月無君則弔。』」

「三月無君則弔,不以急乎?」

曰:「士之失位也,猶諸侯之失國家也。禮曰:『諸侯耕助,以供粢盛;夫人蠶繅,以為衣服。犧牲不成,粢盛不潔,衣服不備,不敢以祭。惟士無田,則亦不祭。』牲殺器皿衣服不備,不敢以祭,則不敢以宴,亦不足弔乎?」

「出疆必載質,何也?」

曰:「士之仕也,猶農夫之耕也,農夫豈為出疆舍其耒耜哉?」

曰:「晉國亦仕國也,未嘗聞仕如此其急。仕如此其急也,君子之難仕,何也?」

曰:「丈夫生而願為之有室,女子生而願為之有家。父母之心,人皆有之。不待父母之命、媒妁之言,鑽穴隙相窺,踰牆相從,則父母國人皆賤之。古之人未嘗不欲仕也,又惡不由其道。不由其道而往者,與鑽穴隙之類也。」

彭更問曰:「後車數十乘,從者數百人,以傳食於諸侯,不以泰乎?」

孟子曰:「非其道,則一簞食不可受於人;如其道,則舜受堯之天下,不以為泰,子以為泰乎?」

曰:「否。士無事而食,不可也。」

曰:「子不通功易事,以羡補不足,則農有餘粟,女有餘布;子如通之,則梓匠輪輿皆得食於子。於此有人焉,入則孝,出則悌,守先王之道,以待後之學者,而不得食於子。子何尊梓匠輪輿而輕為仁義者哉?」

曰:「梓匠輪輿,其志將以求食也;君子之為道也,其志亦將以求食與?」

曰:「子何以其志為哉?其有功於子,可食而食之矣。且子食志乎?食功乎?」

曰:「食志。」

曰:「有人於此,毀瓦畫墁,其志將以求食也,則子食之乎?」

曰:「否。」

曰:「然則子非食志也,食功也。」

萬章問曰:「宋,小國也。今將行王政,齊楚惡而伐之,則如之何?」

孟子曰:「湯居亳,與葛為鄰,葛伯放而不祀。湯使人問之曰:『何為不祀?』曰:『無以供犧牲也。』湯使遺之牛羊。葛伯食之,又不以祀。湯又使人問之曰:『何為不祀?』曰:『無以供粢盛也。』湯使亳眾往為之耕,老弱饋食。葛伯率其民,要其有酒食黍稻者奪之,不授者殺之。有童子以黍肉餉,殺而奪之。《書》曰:『葛伯仇餉。』此之謂也。為其殺是童子而征之,四海之內皆曰:『非富天下也,為匹夫匹婦復讎也。』『湯始征,自葛載』,十一征而無敵於天下。東面而征,西夷怨;南面而征,北狄怨,曰:『奚為後我?』民之望之,若大旱之望雨也。歸市者弗止,芸者不變,誅其君,弔其民,如時雨降。民大悅。《書》曰:『徯我后,后來其無罰。』『有攸不惟臣,東征,綏厥士女,匪厥玄黃,紹我周王見休,惟臣附于大邑周。』其君子實玄黃于匪以迎其君子,其小人簞食壺漿以迎其小人,救民於水火之中,取其殘而已矣。《太誓》曰:『我武惟揚,侵于之疆,則取于殘,殺伐用張,于湯有光。』不行王政云爾,苟行王政,四海之內皆舉首而望之,欲以為君。齊楚雖大,何畏焉?」

孟子謂戴不勝曰:「子欲子之王之善與?我明告子。有楚大夫於此,欲其子之齊語也,則使齊人傅諸?使楚人傅諸?」

曰:「使齊人傅之。」

曰:「一齊人傅之,眾楚人咻之,雖日撻而求其齊也,不可得矣;引而置之莊嶽之間數年,雖日撻而求其楚,亦不可得矣。子謂薛居州,善士也。使之居於王所。在於王所者,長幼卑尊,皆薛居州也,王誰與為不善?在王所者,長幼卑尊,皆非薛居州也,王誰與為善?一薛居州,獨如宋王何?」

公孫丑問曰:「不見諸侯何義?」

孟子曰:「古者不為臣不見。段干木踰垣而辟之,泄柳閉門而不內,是皆已甚。迫,斯可以見矣。陽貨欲見孔子而惡無禮,大夫有賜於士,不得受於其家,則往拜其門。陽貨矙孔子之亡也,而饋孔子蒸豚;孔子亦矙其亡也,而往拜之。當是時,陽貨先,豈得不見?曾子曰:『脅肩諂笑,病于夏畦。』子路曰:『未同而言,觀其色赧赧然,非由之所知也。』由是觀之,則君子之所養可知已矣。」

戴盈之曰:「什一,去關市之征,今茲未能。請輕之,以待來年,然後已,何如?」

孟子曰:「今有人日攘其鄰之雞者,或告之曰:『是非君子之道。』曰:『請損之,月攘一雞,以待來年,然後已。』如知其非義,斯速已矣,何待來年。」

公都子曰:「外人皆稱夫子好辯,敢問何也?」

孟子曰:「予豈好辯哉?予不得已也。天下之生久矣,一治一亂。當堯之時,水逆行,氾濫於中國。蛇龍居之,民無所定。下者為巢,上者為營窟。《書》曰:『洚水警余。』洚水者,洪水也。使禹治之,禹掘地而注之海,驅蛇龍而放之菹。水由地中行,江、淮、河、漢是也。險阻既遠,鳥獸之害人者消,然後人得平土而居之。

「堯、舜既沒,聖人之道衰。暴君代作,壞宮室以為汙池,民無所安息;棄田以為園囿,使民不得衣食。邪說暴行又作,園囿、汙池、沛澤多而禽獸至。及紂之身,天下又大亂。周公相武王,誅紂伐奄,三年討其君,驅飛廉於海隅而戮之。滅國者五十,驅虎、豹、犀、象而遠之。天下大悅。《書》曰:『丕顯哉,文王謨!丕承哉,武王烈!佑啟我後人,咸以正無缺。』

「世衰道微,邪說暴行有作,臣弒其君者有之,子弒其父者有之。孔子懼,作《春秋》。《春秋》,天子之事也。是故孔子曰:『知我者其惟春秋乎!罪我者其惟春秋乎!』

「聖王不作,諸侯放恣,處士橫議,楊朱、墨翟之言盈天下。天下之言,不歸楊,則歸墨。楊氏為我,是無君也;墨氏兼愛,是無父也。無父無君,是禽獸也。公明儀曰:『庖有肥肉,廄有肥馬,民有飢色,野有餓莩,此率獸而食人也。』楊墨之道不息,孔子之道不著,是邪說誣民,充塞仁義也。仁義充塞,則率獸食人,人將相食。吾為此懼,閑先聖之道,距楊墨,放淫辭,邪說者不得作。作於其心,害於其事;作於其事,害於其政。聖人復起,不易吾言矣。

「昔者禹抑洪水而天下平,周公兼夷狄驅猛獸而百姓寧,孔子成《春秋》而亂臣賊子懼。《詩》云:『戎狄是膺,荊舒是懲,則莫我敢承。』無父無君,是周公所膺也。我亦欲正人心,息邪說,距詖行,放淫辭,以承三聖者;豈好辯哉?予不得已也。能言距楊墨者,聖人之徒也。」

匡章曰:「陳仲子豈不誠廉士哉?居於陵,三日不食,耳無聞,目無見也。井上有李,螬食實者過半矣,匍匐往將食之,三咽,然後耳有聞,目有見。」

孟子曰:「於齊國之士,吾必以仲子為巨擘焉。雖然,仲子惡能廉?充仲子之操,則蚓而後可者也。夫蚓,上食槁壤,下飲黃泉。仲子所居之室,伯夷之所築與?抑亦盜跖之所築與?所食之粟,伯夷之所樹與?抑亦盜跖之所樹與?是未可知也。」

曰:「是何傷哉?彼身織屨,妻辟纑,以易之也。」

曰:「仲子,齊之世家也。兄戴,蓋祿萬鍾。以兄之祿為不義之祿而不食也,以兄之室為不義之室而不居也,辟兄離母,處於於陵。他日歸,則有饋其兄生鵝者,己頻顣曰:『惡用是鶃鶃者為哉?』他日,其母殺是鵝也,與之食之。其兄自外至,曰:『是鶃鶃之肉也。』出而哇之。以母則不食,以妻則食之;以兄之室則弗居,以於陵則居之。是尚為能充其類也乎?若仲子者,蚓而後充其操者也。」


\end{pinyinscope}