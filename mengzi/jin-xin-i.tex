\article{盡心上}

\begin{pinyinscope}
孟子曰:「盡其心者,知其性也。知其性,則知天矣。存其心,養其性,所以事天也。殀壽不貳,修身以俟之,所以立命也。」

孟子曰:「莫非命也,順受其正。是故知命者,不立乎巖牆之下。盡其道而死者,正命也。桎梏死者,非正命也。」

孟子曰:「求則得之,舍則失之,是求有益於得也,求在我者也。求之有道,得之有命,是求無益於得也,求在外者也。」

孟子曰:「萬物皆備於我矣。反身而誠,樂莫大焉。強恕而行,求仁莫近焉。」

孟子曰:「行之而不著焉,習矣而不察焉,終身由之而不知其道者,眾也。」

孟子曰:「人不可以無恥。無恥之恥,無恥矣。」

孟子曰:「恥之於人大矣。為機變之巧者,無所用恥焉。不恥不若人,何若人有?」

孟子曰:「古之賢王好善而忘勢,古之賢士何獨不然?樂其道而忘人之勢。故王公不致敬盡禮,則不得亟見之。見且猶1不得亟,而況得而臣之乎?」1. 猶 : 或作「由」。《孟子正義》作「由」。

孟子謂宋句踐曰:「子好遊乎?吾語子遊。人知之,亦囂囂;人不知,亦囂囂。」

曰:「何如斯可以囂囂矣?」

曰:「尊德樂義,則可以囂囂矣。故士窮不失義,達不離道。窮不失義,故士得己焉;達不離道,故民不失望焉。古之人,得志,澤加於民;不得志,脩身見於世。窮則獨善其身,達則兼善天下。」

孟子曰:「待文王而後興者,凡民也。若夫豪傑之士,雖無文王猶興。」

孟子曰:「附之以韓魏之家,如其自視欿然,則過人遠矣。」

孟子曰:「以佚道使民,雖勞不怨;以生道殺民,雖死不怨殺者。」

孟子曰:「霸者之民,驩虞如也;王者之民,皞皞如也。殺之而不怨,利之而不庸,民日遷善而不知為之者。夫君子所過者化,所存者神,上下與天地同流,豈曰小補之哉?」

孟子曰:「仁言,不如仁聲之入人深也。善政,不如善教之得民也。善政民畏之,善教民愛之;善政得民財,善教得民心。」

孟子曰:「人之所不學而能者,其良能也;所不慮而知者,其良知也。孩提之童,無不知愛其親者;及其長也,無不知敬其兄也。親親,仁也;敬長,義也。無他,達之天下也。」

孟子曰:「舜之居深山之中,與木石居,與鹿豕遊,其所以異於深山之野人者幾希。及其聞一善言,見一善行,若決江河,沛然莫之能禦也。」

孟子曰:「無為其所不為,無欲其所不欲,如此而已矣。」

孟子曰:「人之有德慧術知者,恒存乎疢疾。獨孤臣孽子,其操心也危,其慮患也深,故達。」

孟子曰:「有事君人者,事是君則為容悅者也。有安社稷臣者,以安社稷為悅者也。有天民者,達可行於天下而後行之者也。有大人者,正己而物正者也。」

孟子曰:「君子有三樂,而王天下不與存焉。父母俱存,兄弟無故,一樂也。仰不愧於天,俯不怍於人,二樂也。得天下英才而教育之,三樂也。君子有三樂,而王天下不與存焉。」

孟子曰:「廣土眾民,君子欲之,所樂不存焉。中天下而立,定四海之民,君子樂之,所性不存焉。君子所性,雖大行不加焉,雖窮居不損焉,分定故也。君子所性,仁義禮智根於心。其生色也,睟然見於面,盎於背,施於四體,四體不言而喻。」

孟子曰:「伯夷辟紂,居北海之濱,聞文王作興,曰:『盍歸乎來!吾聞西伯善養老者。』太公辟紂,居東海之濱,聞文王作興,曰:『盍歸乎來!吾聞西伯善養老者。』天下有善養老,則仁人以為己歸矣。五畝之宅,樹牆下以桑,匹婦蠶之,則老者足以衣帛矣。五母雞,二母彘,無失其時,老者足以無失肉矣。百畝之田,匹夫耕之,八口之家足以無飢矣。所謂西伯善養老者,制其田里,教之樹畜,導其妻子,使養其老。五十非帛不煖,七十非肉不飽。不煖不飽,謂之凍餒。文王之民,無凍餒之老者,此之謂也。」

孟子曰:「易其田疇,薄其稅斂,民可使富也。食之以時,用之以禮,財不可勝用也。民非水火不生活,昏暮叩人之門戶,求水火,無弗與者,至足矣。聖人治天下,使有菽粟如水火。菽粟如水火,而民焉有不仁者乎?」

孟子曰:「孔子登東山而小魯,登太山而小天下。故觀於海者難為水,遊於聖人之門者難為言。觀水有術,必觀其瀾。日月有明,容光必照焉。流水之為物也,不盈科不行;君子之志於道也,不成章不達。」

孟子曰:「雞鳴而起,孳孳為善者,舜之徒也。雞鳴而起,孳孳為利者,蹠之徒也。欲知舜與蹠之分,無他,利與善之閒也。」

孟子曰:「楊子取為我,拔一毛而利天下,不為也。墨子兼愛,摩頂放踵利天下,為之。子莫執中,執中為近之,執中無權,猶執一也。所惡執一者,為其賊道也,舉一而廢百也。」

孟子曰:「飢者甘食,渴者甘飲,是未得飲食之正也,飢渴害之也。豈惟口腹有飢渴之害?人心亦皆有害。人能無以飢渴之害為心害,則不及人不為憂矣。」

孟子曰:「柳下惠不以三公易其介。」

孟子曰:「有為者辟若掘井,掘井九軔而不及泉,猶為棄井也。」

孟子曰:「堯舜,性之也;湯武,身之也;五霸,假之也。久假而不歸,惡知其非有也。」

公孫丑曰:「伊尹曰:『予不狎于不順。』放太甲于桐,民大悅。太甲賢。又反之,民大悅。賢者之為人臣也,其君不賢,則固可放與?」

孟子曰:「有伊尹之志,則可;無伊尹之志,則篡也。」

公孫丑曰:「《詩》曰:『不素餐兮』,君子之不耕而食,何也?」

孟子曰:「君子居是國也,其君用之,則安富尊榮;其子弟從之,則孝弟忠信。『不素餐兮』,孰大於是?」

王子墊問曰:「士何事?」

孟子曰:「尚志。」

曰:「何謂尚志?」

曰:「仁義而已矣。殺一無罪,非仁也;非其有而取之,非義也。居惡在?仁是也;路惡在?義是也。居仁由義,大人之事備矣。」

孟子曰:「仲子,不義與之齊國而弗受,人皆信之,是舍簞食豆羹之義也。人莫大焉亡親戚、君臣、上下。以其小者信其大者,奚可哉?」

桃應問曰:「舜為天子,皋陶為士,瞽瞍殺人,則如之何?」

孟子曰:「執之而已矣。」

「然則舜不禁與?」

曰:「夫舜惡得而禁之?夫有所受之也。」

「然則舜如之何?」

曰:「舜視棄天下,猶棄敝蹝也。竊負而逃,遵海濱而處,終身訢然,樂而忘天下。」

孟子自范之齊,望見齊王之子。喟然歎曰:「居移氣,養移體,大哉居乎!夫非盡人之子與?」

孟子曰:「王子宮室、車馬、衣服多與人同,而王子若彼者,其居使之然也;況居天下之廣居者乎?魯君之宋,呼於垤澤之門。守者曰:『此非吾君也,何其聲之似我君也?』此無他,居相似也。」

孟子曰:「食而弗愛,豕交之也;愛而不敬,獸畜之也。恭敬者,幣之未將者也。恭敬而無實,君子不可虛拘。」

孟子曰:「形色,天性也;惟聖人,然後可以踐形。」

齊宣王欲短喪。公孫丑曰:「為朞之喪,猶愈於已乎?」

孟子曰:「是猶或紾其兄之臂,子謂之姑徐徐云爾,亦教之孝弟而已矣。」

王子有其母死者,其傅為之請數月之喪。公孫丑曰:「若此者,何如也?」

曰:「是欲終之而不可得也。雖加一日愈於已,謂夫莫之禁而弗為者也。」

孟子曰:「君子之所以教者五:有如時雨化之者,有成德者,有達財者,有答問者,有私淑艾者。此五者,君子之所以教也。」

公孫丑曰:「道則高矣,美矣,宜若登天然,似不可及也。何不使彼為可幾及而日孳孳也?」

孟子曰:「大匠不為拙工改廢繩墨,羿不為拙射變其彀率。君子引而不發,躍如也。中道而立,能者從之。」

孟子曰:「天下有道,以道殉身;天下無道,以身殉道。未聞以道殉乎人者也。」

公都子曰:「滕更之在門也,若在所禮。而不答,何也?」

孟子曰:「挾貴而問,挾賢而問,挾長而問,挾有勳勞而問,挾故而問,皆所不答也。滕更有二焉。」

孟子曰:「於不可已而已者,無所不已;於所厚者薄,無所不薄也。其進銳者,其退速。」

孟子曰:「君子之於物也,愛之而弗仁;於民也,仁之而弗親。親親而仁民,仁民而愛物。」

孟子曰:「知者無不知也,當務之為急;仁者無不愛也,急親賢之為務。堯舜之知而不遍物,急先務也;堯舜之仁不遍愛人,急親賢也。不能三年之喪,而緦小功之察;放飯流歠,而問無齒決,是之謂不知務。」


\end{pinyinscope}