\article{公孫丑下}

\begin{pinyinscope}
孟子曰:「天時不如地利,地利不如人和。三里之城,七里之郭,環而攻之而不勝。夫環而攻之,必有得天時者矣;然而不勝者,是天時不如地利也。城非不高也,池非不深也,兵革非不堅利也,米粟非不多也;委而去之,是地利不如人和也。故曰:域民不以封疆之界,固國不以山谿之險,威天下不以兵革之利。得道者多助,失道者寡助。寡助之至,親戚畔之;多助之至,天下順之。以天下之所順,攻親戚之所畔;故君子有不戰,戰必勝矣。」

孟子將朝王,王使人來曰:「寡人如就見者也,有寒疾,不可以風。朝將視朝,不識可使寡人得見乎?」對曰:「不幸而有疾,不能造朝。」

明日,出吊於東郭氏,公孫丑曰:「昔者辭以病,今日吊,或者不可乎!」曰:「昔者疾,今日愈,如之何不吊?」

王使人問疾,醫來。孟仲子對曰:「昔者有王命,有采薪之憂,不能造朝。今病小愈,趨造於朝,我不識能至否乎?」使數人要於路,曰:「請必無歸,而造於朝!」

不得已而之景丑氏宿焉。景子曰:「內則父子,外則君臣,人之大倫也。父子主恩,君臣主敬。丑見王之敬子也,未見所以敬王也。」

曰:「惡!是何言也!齊人無以仁義與王言者,豈以仁義為不美也?其心曰『是何足與言仁義也』云爾,則不敬莫大乎是。我非堯舜之道,不敢以陳於王前,故齊人莫如我敬王也。」

景子曰:「否,非此之謂也。禮曰:『父召,無諾;君命召,不俟駕。』固將朝也,聞王命而遂不果,宜與夫禮若不相似然。」

曰:「豈謂是與?曾子曰:『晉楚之富,不可及也。彼以其富,我以吾仁;彼以其爵,我以吾義,吾何慊乎哉?』夫豈不義而曾子言之?是或一道也。天下有達尊三:爵一,齒一,德一。朝廷莫如爵,鄉黨莫如齒,輔世長民莫如德。惡得有其一,以慢其二哉?故將大有為之君,必有所不召之臣。欲有謀焉,則就之。其尊德樂道,不如是不足與有為也。故湯之於伊尹,學焉而後臣之,故不勞而王;桓公之於管仲,學焉而後臣之,故不勞而霸。今天下地醜德齊,莫能相尚。無他,好臣其所教,而不好臣其所受教。湯之於伊尹,桓公之於管仲,則不敢召。管仲且猶不可召,而況不為管仲者乎?」

陳臻問曰:「前日於齊,王餽兼金一百而不受;於宋,餽七十鎰而受;於薛,餽五十鎰而受。前日之不受是,則今日之受非也;今日之受是,則前日之不受非也。夫子必居一於此矣。」

孟子曰:「皆是也。當在宋也,予將有遠行。行者必以贐,辭曰:『餽贐。』予何為不受?當在薛也,予有戒心。辭曰:『聞戒。』故為兵餽之,予何為不受?若於齊,則未有處也。無處而餽之,是貨之也。焉有君子而可以貨取乎?」

孟子之平陸。謂其大夫曰:「子之持戟之士,一日而三失伍,則去之否乎?」

曰:「不待三。」

「然則子之失伍也亦多矣。凶年饑歲,子之民,老羸轉於溝壑,壯者散而之四方者,幾千人矣。」

曰:「此非距心之所得為也。」

曰:「今有受人之牛羊而為之牧之者,則必為之求牧與芻矣。求牧與芻而不得,則反諸其人乎?抑亦立而視其死與?」

曰:「此則距心之罪也。」

他日,見於王曰:「王之為都者,臣知五人焉。知其罪者,惟孔距心。為王誦之。」王曰:「此則寡人之罪也。」

孟子謂蚔鼃曰:「子之辭靈丘而請士師,似也,為其可以言也。今既數月矣,未可以言與?」

蚔鼃諫於王而不用,致為臣而去。齊人曰:「所以為蚔鼃,則善矣;所以自為,則吾不知也。」

公都子以告。曰:「吾聞之也:有官守者,不得其職則去;有言責者,不得其言則去。我無官守,我無言責也,則吾進退,豈不綽綽然有餘裕哉?」

孟子為卿於齊,出吊於滕,王使蓋大夫王驩為輔行。王驩朝暮見,反齊滕之路,未嘗與之言行事也。

公孫丑曰:「齊卿之位,不為小矣;齊滕之路,不為近矣。反之而未嘗與言行事,何也?」

曰:「夫既或治之,予何言哉?」

孟子自齊葬於魯,反於齊,止於嬴。

充虞請曰:「前日不知虞之不肖,使虞敦匠事。嚴,虞不敢請。今願竊有請也,木若以美然。」

曰:「古者棺槨無度,中古棺七寸,槨稱之。自天子達於庶人。非直為觀美也,然後盡於人心。不得,不可以為悅;無財,不可以為悅。得之為有財,古之人皆用之,吾何為獨不然?且比化者,無使土親膚,於人心獨無恔乎?吾聞之君子:不以天下儉其親。」

沈同以其私問曰:「燕可伐與?」

孟子曰:「可。子噲不得與人燕,子之不得受燕於子噲。有仕於此,而子悅之,不告於王而私與之吾子之祿爵;夫士也,亦無王命而私受之於子,則可乎?何以異於是?」

齊人伐燕。或問曰:「勸齊伐燕,有諸?」

曰:「未也。沈同問『燕可伐與』?吾應之曰『可』,彼然而伐之也。彼如曰『孰可以伐之』?則將應之曰:『為天吏,則可以伐之。』今有殺人者,或問之曰『人可殺與』?則將應之曰『可』。彼如曰『孰可以殺之』?則將應之曰:『為士師,則可以殺之。』今以燕伐燕,何為勸之哉?」

燕人畔。王曰:「吾甚慚於孟子。」

陳賈曰:「王無患焉。王自以為與周公,孰仁且智?」

王曰:「惡!是何言也?」

曰:「周公使管叔監殷,管叔以殷畔。知而使之,是不仁也;不知而使之,是不智也。仁智,周公未之盡也,而況於王乎?賈請見而解之。」

見孟子,問曰:「周公何人也?」

曰:「古聖人也。」

曰:「使管叔監殷,管叔以殷畔也,有諸?」

曰:「然。」

曰:「周公知其將畔而使之與?」

曰:「不知也。」

「然則聖人且有過與?」

曰:「周公,弟也;管叔,兄也。周公之過,不亦宜乎?且古之君子,過則改之;今之君子,過則順之。古之君子,其過也,如日月之食,民皆見之;及其更也,民皆仰之。今之君子,豈徒順之,又從為之辭。」

孟子致為臣而歸。王就見孟子,曰:「前日願見而不可得,得侍,同朝甚喜。今又棄寡人而歸,不識可以繼此而得見乎?」對曰:「不敢請耳,固所願也。」

他日,王謂時子曰:「我欲中國而授孟子室,養弟子以萬鍾,使諸大夫國人皆有所矜式。子盍為我言之?」

時子因陳子而以告孟子,陳子以時子之言告孟子。孟子曰:「然。夫時子惡知其不可也?如使予欲富,辭十萬而受萬,是為欲富乎?季孫曰:『異哉子叔疑!使己為政,不用,則亦已矣,又使其子弟為卿。人亦孰不欲富貴?而獨於富貴之中,有私龍斷焉。』古之為市也,以其所有易其所無者,有司者治之耳。有賤丈夫焉,必求龍斷而登之,以左右望而罔市利。人皆以為賤,故從而征之。征商,自此賤丈夫始矣。

孟子去齊,宿於晝。

有欲為王留行者,坐而言。不應,隱几而臥。客不悅曰:「弟子齊宿而後敢言,夫子臥而不聽,請勿復敢見矣。」

曰:「坐!我明語子。昔者魯繆公無人乎子思之側,則不能安子思;泄柳、申詳,無人乎繆公之側,則不能安其身。子為長者慮,而不及子思,子絕長者乎?長者絕子乎?」

孟子去齊。尹士語人曰:「不識王之不可以為湯武,則是不明也;識其不可,然且至,則是干澤也。千里而見王,不遇故去。三宿而後出晝,是何濡滯也?士則茲不悅。」

高子以告。曰:「夫尹士惡知予哉?千里而見王,是予所欲也;不遇故去,豈予所欲哉?予不得已也。予三宿而出晝,於予心猶以為速。王庶幾改之。王如改諸,則必反予。夫出晝而王不予追也,予然後浩然有歸志。予雖然,豈舍王哉?王由足用為善。王如用予,則豈徒齊民安,天下之民舉安。王庶幾改之,予日望之。予豈若是小丈夫然哉?諫於其君而不受,則怒,悻悻然見於其面。去則窮日之力而後宿哉?」

尹士聞之曰:「士誠小人也。」

孟子去齊。充虞路問曰:「夫子若有不豫色然。前日虞聞諸夫子曰:『君子不怨天,不尤人。』」

曰:「彼一時,此一時也。五百年必有王者興,其間必有名世者。由周而來,七百有餘歲矣。以其數則過矣,以其時考之則可矣。夫天,未欲平治天下也;如欲平治天下,當今之世,舍我其誰也?吾何為不豫哉?」

孟子去齊,居休。

公孫丑問曰:「仕而不受祿,古之道乎?」

曰:「非也。於崇,吾得見王。退而有去志,不欲變,故不受也。繼而有師命,不可以請。久於齊,非我志也。」


\end{pinyinscope}