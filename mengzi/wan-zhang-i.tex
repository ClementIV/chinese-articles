\article{萬章上}

\begin{pinyinscope}
萬章問曰:「舜往于田,號泣于旻天,何為其號泣也?」

孟子曰:「怨慕也。」

萬章曰:「父母愛之,喜而不忘;父母惡之,勞而不怨。然則舜怨乎?」

曰:「長息問於公明高曰:『舜往于田,則吾既得聞命矣;號泣于旻天,于父母,則吾不知也。』公明高曰:『是非爾所知也。』夫公明高以孝子之心,為不若是恝,我竭力耕田,共為子職而已矣,父母之不我愛,於我何哉?帝使其子九男二女,百官牛羊倉廩備,以事舜於畎畝之中。天下之士多就之者,帝將胥天下而遷之焉。為不順於父母,如窮人無所歸。天下之士悅之,人之所欲也,而不足以解憂;好色,人之所欲,妻帝之二女,而不足以解憂;富,人之所欲,富有天下,而不足以解憂;貴,人之所欲,貴為天子,而不足以解憂。人悅之、好色、富貴,無足以解憂者,惟順於父母,可以解憂。人少,則慕父母;知好色,則慕少艾;有妻子,則慕妻子;仕則慕君,不得於君則熱中。大孝終身慕父母。五十而慕者,予於大舜見之矣。」

萬章問曰:「《詩》云:『娶妻如之何?必告父母。』信斯言也,宜莫如舜。舜之不告而娶,何也?」

孟子曰:「告則不得娶。男女居室,人之大倫也。如告,則廢人之大倫,以懟父母,是以不告也。」

萬章曰:「舜之不告而娶,則吾既得聞命矣;帝之妻舜而不告,何也?」

曰:「帝亦知告焉則不得妻也。」

萬章曰:「父母使舜完廩,捐階,瞽瞍焚廩。使浚井,出,從而揜之。象曰:『謨蓋都君咸我績。牛羊父母,倉廩父母,干戈朕,琴朕,弤朕,二嫂使治朕棲。』象往入舜宮,舜在床琴。象曰:『鬱陶思君爾。』忸怩。舜曰:『惟茲臣庶,汝其于予治。』不識舜不知象之將殺己與?」

曰:「奚而不知也?象憂亦憂,象喜亦喜。」

曰:「然則舜偽喜者與?」

曰:「否。昔者有饋生魚於鄭子產,子產使校人畜之池。校人烹之,反命曰:『始舍之圉圉焉,少則洋洋焉,攸然而逝。』子產曰『得其所哉!得其所哉!』校人出,曰:『孰謂子產智?予既烹而食之,曰:得其所哉?得其所哉。』故君子可欺以其方,難罔以非其道。彼以愛兄之道來,故誠信而喜之,奚偽焉?」

萬章問曰:「象日以殺舜為事,立為天子,則放之,何也?」

孟子曰:「封之也,或曰放焉。」

萬章曰:「舜流共工于幽州,放驩兜于崇山,殺三苗于三危,殛鯀于羽山,四罪而天下咸服,誅不仁也。象至不仁,封之有庳。有庳之人奚罪焉?仁人固如是乎?在他人則誅之,在弟則封之。」

曰:「仁人之於弟也,不藏怒焉,不宿怨焉,親愛之而已矣。親之欲其貴也,愛之欲其富也。封之有庳,富貴之也。身為天子,弟為匹夫,可謂親愛之乎?」

「敢問或曰放者,何謂也?」

曰:「象不得有為於其國,天子使吏治其國,而納其貢稅焉,故謂之放,豈得暴彼民哉?雖然,欲常常而見之,故源源而來。『不及貢,以政接于有庳』,此之謂也。」

咸丘蒙問曰:「語云:『盛德之士,君不得而臣,父不得而子。』舜南面而立,堯帥諸侯北面而朝之,瞽瞍亦北面而朝之。舜見瞽瞍,其容有蹙。孔子曰:『於斯時也,天下殆哉,岌岌乎!』不識此語誠然乎哉?」

孟子曰:「否。此非君子之言,齊東野人之語也。堯老而舜攝也。《堯典》曰:『二十有八載,放勳乃徂落,百姓如喪考妣,三年,四海遏密八音。』孔子曰:『天無二日,民無二王。』舜既為天子矣,又帥天下諸侯以為堯三年喪,是二天子矣。」

咸丘蒙曰:「舜之不臣堯,則吾既得聞命矣。《詩》云:『普天之下,莫非王土;率土之濱,莫非王臣。』而舜既為天子矣,敢問瞽瞍之非臣,如何?」

曰:「是詩也,非是之謂也;勞於王事,而不得養父母也。曰:『此莫非王事,我獨賢勞也。』故說《詩》者,不以文害辭,不以辭害志。以意逆志,是為得之。如以辭而已矣,《雲漢》之詩曰:『周餘黎民,靡有孑遺。』信斯言也,是周無遺民也。孝子之至,莫大乎尊親;尊親之至,莫大乎以天下養。為天子父,尊之至也;以天下養,養之至也。《詩》曰:『永言孝思,孝思維則。』此之謂也。《書》曰:『祗載見瞽瞍,夔夔齊栗,瞽瞍亦允若。』是為父不得而子也。」

萬章曰:「堯以天下與舜,有諸?」

孟子曰:「否。天子不能以天下與人。」

「然則舜有天下也,孰與之?」

曰:「天與之。」

「天與之者,諄諄然命之乎?」

曰:「否。天不言,以行與事示之而已矣。」

曰:「以行與事示之者如之何?」

曰:「天子能薦人於天,不能使天與之天下;諸侯能薦人於天子,不能使天子與之諸侯;大夫能薦人於諸侯,不能使諸侯與之大夫。昔者堯薦舜於天而天受之,暴之於民而民受之,故曰:天不言,以行與事示之而已矣。」

曰:「敢問薦之於天而天受之,暴之於民而民受之,如何?」

曰:「使之主祭而百神享之,是天受之;使之主事而事治,百姓安之,是民受之也。天與之,人與之,故曰:天子不能以天下與人。舜相堯二十有八載,非人之所能為也,天也。堯崩,三年之喪畢,舜避堯之子於南河之南。天下諸侯朝覲者,不之堯之子而之舜;訟獄者,不之堯之子而之舜;謳歌者,不謳歌堯之子而謳歌舜,故曰天也。夫然後之中國,踐天子位焉。而居堯之宮,逼堯之子,是篡也,非天與也。《太誓》曰:『天視自我民視,天聽自我民聽』,此之謂也。」

萬章問曰:「人有言:『至於禹而德衰,不傳於賢而傳於子。』有諸?」

孟子曰:「否,不然也。天與賢,則與賢;天與子,則與子。昔者舜薦禹於天,十有七年,舜崩。三年之喪畢,禹避舜之子於陽城。天下之民從之,若堯崩之後,不從堯之子而從舜也。禹薦益於天,七年,禹崩。三年之喪畢,益避禹之子於箕山之陰。朝覲訟獄者不之益而之啟,曰:『吾君之子也。』謳歌者不謳歌益而謳歌啟,曰:『吾君之子也。』丹朱之不肖,舜之子亦不肖。舜之相堯,禹之相舜也,歷年多,施澤於民久。啟賢,能敬承繼禹之道。益之相禹也,歷年少,施澤於民未久。舜、禹、益相去久遠,其子之賢不肖,皆天也,非人之所能為也。莫之為而為者,天也;莫之致而至者,命也。匹夫而有天下者,德必若舜禹,而又有天子薦之者,故仲尼不有天下。繼世以有天下,天之所廢,必若桀紂者也,故益、伊尹、周公不有天下。伊尹相湯以王於天下。湯崩,太丁未立,外丙二年,仲壬四年。太甲顛覆湯之典刑,伊尹放之於桐。三年,太甲悔過,自怨自艾,於桐處仁遷義;三年,以聽伊尹之訓己也,復歸于亳。周公之不有天下,猶益之於夏,伊尹之於殷也。孔子曰:『唐虞禪,夏后、殷、周繼,其義一也。』」

萬章問曰:「人有言『伊尹以割烹要湯』有諸?」

孟子曰:「否,不然。伊尹耕於有莘之野,而樂堯舜之道焉。非其義也,非其道也,祿之以天下,弗顧也;繫馬千駟,弗視也。非其義也,非其道也,一介不以與人,一介不以取諸人,湯使人以幣聘之,囂囂然曰:『我何以湯之聘幣為哉?我豈若處畎畝之中,由是以樂堯舜之道哉?』湯三使往聘之,既而幡然改曰:『與我處畎畝之中,由是以樂堯舜之道,吾豈若使是君為堯舜之君哉?吾豈若使是民為堯舜之民哉?吾豈若於吾身親見之哉?天之生此民也,使先知覺後知,使先覺覺後覺也。予,天民之先覺者也;予將以斯道覺斯民也。非予覺之,而誰也?』思天下之民匹夫匹婦有不被堯舜之澤者,若己推而內之溝中。其自任以天下之重如此,故就湯而說之以伐夏救民。吾未聞枉己而正人者也,況辱己以正天下者乎?聖人之行不同也,或遠或近,或去或不去,歸潔其身而已矣。吾聞其以堯舜之道要湯,未聞以割烹也。伊訓曰:『天誅造攻自牧宮,朕載自亳。』」

萬章問曰:「或謂孔子於衛主癰疽,於齊主侍人瘠環,有諸乎?」

孟子曰:「否,不然也。好事者為之也。於衛主顏讎由。彌子之妻與子路之妻,兄弟也。彌子謂子路曰:『孔子主我,衛卿可得也。』子路以告。孔子曰:『有命。』孔子進以禮,退以義,得之不得曰『有命』。而主癰疽與侍人瘠環,是無義無命也。孔子悅於魯衛,遭宋桓司馬將要而殺之,微服而過宋。是時孔子當阨,主司城貞子,為陳侯周臣。吾聞觀近臣,以其所為主;觀遠臣,以其所主。若孔子主癰疽與侍人瘠環,何以為孔子?」

萬章問曰:「或曰:『百里奚自鬻於秦養牲者,五羊之皮,食牛,以要秦穆公。』信乎?」

孟子曰:「否,不然。好事者為之也。百里奚,虞人也。晉人以垂棘之璧與屈產之乘,假道於虞以伐虢。宮之奇諫,百里奚不諫。知虞公之不可諫而去,之秦,年已七十矣,曾不知以食牛干秦穆公之為汙也,可謂智乎?不可諫而不諫,可謂不智乎?知虞公之將亡而先去之,不可謂不智也。時舉於秦,知穆公之可與有行也而相之,可謂不智乎?相秦而顯其君於天下,可傳於後世,不賢而能之乎?自鬻以成其君,鄉黨自好者不為,而謂賢者為之乎?」


\end{pinyinscope}