\article{孝順孝沖孝質帝紀}

\begin{pinyinscope}
孝順皇帝諱保,安帝之子也。母李氏,為閻皇后所害。永寧元年,立為皇太子。延光三年,安帝乳母王聖、大長秋江京、中常侍樊豐譖太子乳母王男、廚監邴吉,殺之,太子數為歎息。王聖等懼有後禍,遂與豐、京共搆陷太子,太子坐廢為濟陰王。明年三月,安帝崩,北鄉侯立,濟陰王以廢黜,不得上殿親臨梓宮,悲號不食,內外群僚莫不哀之。及北鄉侯薨,車騎將軍閻顯及江京,與中常侍劉安、陳達等白太后,祕不發喪,而更徵立諸國王子,乃閉宮門,屯兵自守。

十一月丁巳,京師及郡國十六地震。是夜,中黃門孫程等十九人共斬江京、劉安、陳達等,迎濟陰王於德陽殿西鍾下,即皇帝位,年十一。近臣尚書以下,從輦到南宮,登雲臺,召百官。尚書令劉光等奏言:「孝安皇帝聖德明茂,早棄天下。陛下正統,當奉宗廟,而姦臣交搆,遂令陛下龍潛蕃國,群僚遠近莫不失望。天命有常,北鄉不永,漢德盛明,福祚孔章。近臣建策,左右扶翼,內外同心,稽合神明。陛下踐祚,奉遵鴻緒,為郊廟主,承續祖宗無窮之烈,上當天心,下猒民望。而即位倉卒,典章多缺,請條案禮儀,分別具奏。」制曰:「可。」乃召公卿百僚,使虎賁、羽林士屯南、北宮諸門。閻顯兄弟聞帝立,率兵入北宮,尚書郎鎮與交鋒刃,遂斬顯弟衛尉景。戊午,遣使者入省,奪得璽綬,乃幸嘉德殿,遣侍御史持節收閻顯及其弟城門校尉耀、執金吾晏,並下獄誅。己未,開門,罷屯兵。壬戌,詔司隸校尉:「惟閻顯、江京近親當伏辜誅,其餘務崇寬貸。」壬申,謁高廟。癸酉,謁光武廟。

乙亥,詔益州刺史罷子午道,通褒斜路。

己卯,葬少帝以諸王禮。司空劉授免。賜公卿以下錢穀各有差。十二月甲申,以少府河南陶敦為司空。

其令郡國守相視事未滿歲者,一切得舉孝廉吏。

癸卯,尚書奏請下有司,收還延光三年九月丁酉以皇太子為濟陰王詔書。奏可。

京師大疫。

辛亥,詔公卿、郡守、國相,舉賢良方正、能直言極諫之士各一人。尚書令以下從輦幸南宮者,皆增秩賜布各有差。

永建元年春正月甲寅,詔曰:「先帝聖德,享祚未永,早棄鴻烈。姦慝緣閒,人庶怨讟,上干和氣,疫癘為災。朕奉承大業,未能寧濟。蓋至理之本,稽私德惠,蕩滌宿惡,與人更始,其大赦天下。賜男子爵,人二級,為父後、三老、孝悌、力田三級,流民欲自占者一級;鰥、寡、孤、獨、篤缮、貧不能自存者粟,人五斛;貞婦帛,人三匹。坐法當徙,勿徙;亡徒當傳,勿傳。宗室以罪絕,皆復屬籍。其與閻顯、江京等交通者,悉勿考。勉修厥職,以康我民。」

辛未,皇太后閻氏崩。

辛巳,太傅馮石、太尉劉熹、司徒李郃免。

二月甲申,葬安思皇后。

丙戌,太常桓焉為太傅;大鴻臚朱寵為太尉,參錄尚書事;長樂少府九江朱倀為司徒。賜百官隨輦宿衛及拜除者布各有差。

隴西鐘羌叛,護羌校尉馬賢討破之。

夏五月丁丑,詔幽、并、涼州刺史,使各實二千石以下至黃綬,年老劣弱不任軍事者,上名。嚴敕障塞,繕設屯備,立秋之後,簡習戎馬。

六月己亥,封濟南王錯子顯為濟南王。

秋七月庚午,衛尉來歷為車騎將軍。

八月,鮮卑寇代郡,代郡太守李超戰歿。

九月辛亥,初令三公、尚書入奏事。

冬十月辛巳,詔減死罪以下徙邊;其亡命贖,各有差。

丁亥,司空陶敦免。

鮮卑犯邊。庚寅,遣黎陽營兵出屯中山北界。告幽州刺史,其令緣邊郡增置步兵,列屯塞下。調五營弩師,郡舉五人,令教習戰射。

壬寅,廷尉張皓為司空。

甲辰,詔以疫癘水潦,令人半輸今年田租;傷害什四以上,勿收責;不滿者,以實除之。

十二月辛巳,賜王、主、貴人、公卿以下布各有差。

二年春正月戊申,樂安王鴻來朝。

丁卯,常山王章薨。

二月,鮮卑寇遼東、玄菟。

甲辰,詔稟貸荊、豫、兗、冀四州流冗貧人,所在安業之;疾病致醫藥。

護烏桓校尉耿曄率南單于擊鮮卑,破之。

三月,旱,遣使者錄囚徒。

疏勒國遣使奉獻。

夏六月乙酉,追尊謚皇妣李氏為恭愍皇后,葬于恭北陵。

西域長史班勇、敦煌太守張朗討焉耆、尉犁、危

須三國,破之;並遣子貢獻。

秋七月甲戌朔,日有食之。

壬午,太尉朱寵、司徒朱倀罷。庚子,太常劉光為太尉,錄尚書事;光祿勳許敬為司徒。

辛丑,下邳王成薨。

三年春正月丙子,京師地震,漢陽地陷裂。甲午,詔實覈傷害者,賜年七歲以上錢,人二千;一家被害,郡縣為收斂。乙未,詔勿收漢陽今年田租、口賦。

夏四月癸卯,遣光祿大夫案行漢陽及河內、魏郡、陳留、東郡,稟貸貧人。

六月,旱。遣使者錄囚徒,理輕繫。

甲寅,濟南王顯薨。

秋七月丁酉,茂陵園寑災,帝縞素避正殿。辛亥,使太常王龔持節告祠茂陵。

九月,鮮卑寇漁陽。

冬十二月己亥,太傅桓焉免。

是歲,車騎將軍來歷罷。

四年春正月丙寅,詔曰:「朕託王公之上,涉道日寡,政失厥中,陰陽氣隔,寇盜肆暴,庶獄彌繁,憂悴永歎,疢如疾首。《詩》云:『君子如祉,亂庶遄已。』三朝之會,朔旦立春,嘉與海內洗心自新。其赦天下。從甲寅赦令已來復秩屬籍,三年正月已來還贖。其閻顯、江京等知識婚姻禁錮,一原除之。務崇寬和,敬順時令,遵典去苛,以稱朕意。」

丙子,帝加元服。賜王、主、貴人、公卿以下金帛各有差。賜男子爵及流民欲占者人一級,為父後、三老、孝悌、力田人二級;鰥、寡、孤、獨、篤缮、不能自存帛,一匹。

二月戊戌,詔以民入山鑿石,發洩藏氣,敕有司檢察所當禁絕,如建武、永平故事。

夏五月壬辰,詔曰:「海內頗有災異,朝廷修政,太官減膳,珍玩不御。而桂陽太守文礱,不惟竭忠,宣暢本朝,而遠獻大珠,以求幸媚,今封以還之。」

五州雨水。秋八月庚子,遣使實覈死亡,收斂稟賜。

丁巳,太尉劉光、司空張皓免。

九月,復安定、北地、上郡歸舊土。

癸酉,大鴻臚龐參為太尉,錄尚書事。太常王龔為司空。

冬十一月庚辰,司徒許敬免。

鮮卑寇朔方。

十二月乙卯,宗正劉崎為司徒。

是歲,分會稽為吳郡。拘彌國遣使貢獻。

五年春正月,疏勒王遣侍子,及大宛、莎車王皆奉使貢獻。

夏四月,京師旱。辛巳,詔郡國貧人被災者,勿收責今年過更。京師及郡國十二蝗。

冬十月丙辰,詔郡國中都官死罪繫囚皆減罪一等,詣北地、上郡、安定戍。

乙亥,定遠侯班始坐殺其妻陰城公主,腰斬,同產皆棄市。

六年春二月庚午,河閒王開薨。

三月辛亥,復伊吾屯田,復置伊吾司馬一人。

秋九月辛巳,繕起太學。

護烏桓校尉耿曄遣兵擊鮮卑,破之。

丁酉,于闐王遣侍子貢獻。

冬十一月辛亥,詔曰:「連年災潦,冀部尤甚。比蠲除實傷,贍恤窮匱,而百姓猶有棄業,流亡不絕。疑郡縣用心怠惰,恩澤不宣。易美『損上益下』,書稱『安民則惠』。其令冀部勿收今年田租、芻稿。」

十二月,日南徼外葉調國、撣國遣使貢獻。

壬申,客星出牽牛。

于闐王遣侍子詣闕貢獻。

陽嘉元年春正月乙巳,立皇后梁氏。賜爵,人二級,三老、孝悌、力田三級,爵過公乘,得移與子若同產、同產子,民無名數及流民欲占著者人一級;鰥、寡、孤、獨、篤缮、貧不能自存者粟,人五斛。

二月,海賊曾旌等寇會稽,殺句章、鄞、鄮三縣長,攻會稽東部都尉。詔緣海縣各屯兵戍。

丁巳,皇后謁高廟、光武廟,詔稟甘陵貧人,大小口各有差。

京師旱。庚申,敕郡國二千石各禱名山岳瀆,遣大夫、謁者詣嵩高、首陽山,并祠河、洛,請雨。戊辰,雩。

以冀部比年水潦,民食不贍,詔案行稟貸,勸農功,賑乏絕。

甲戌,詔曰:「政失厥和,陰陽隔并,冬鮮宿雪,春無澍雨。分禱祈請,靡神不禜。深恐在所慢違『如在』之義,今遣侍中王輔等,持節分詣岱山、東海、滎陽、河、洛,盡心祈焉。」

三月,楊州六郡妖賊章河等寇四十九縣,殺傷長吏。

庚寅,帝臨辟雍饗射,大赦天下,改元陽嘉。詔宗室絕屬籍者,一切復籍;稟冀州尤貧民,勿收今年更、租、口賦。

夏五月戊寅,阜陵王恢薨。

秋七月,史官始作候風地動銅儀。

丙辰,以太學新成,試明經下第者補弟子,增甲、乙科員各十人。除郡國耆儒九十人補郎、舍人。

九月,詔郡國中都官繫囚皆減死一等,亡命者贖,各有差。

鮮卑寇遼東。

冬十一月甲申,望都、蒲陰狼殺女子九十七人,詔賜狼所殺者錢,人三千。

辛卯,初令郡國舉孝廉,限年四十以上,諸生通章句,文吏能牋奏,乃得應選;其有茂才異行,若顏淵、子奇,不拘年齒。

十二月丁未,東平王敞薨。

庚戌,復置玄菟郡屯田六郡。

閏月丁亥,令諸以詔除為郎,年四十以上課試如孝廉科者,得參廉選,歲舉一人。

戊子,客星出天苑。

辛卯,詔曰:「閒者以來,吏政不勤,故災咎屢臻,盜賊多有。退省所由,皆以選舉不實,官非其人,是以天心未得,人情多怨。書歌股肱,詩刺三事。今刺史、二千石之選,歸任三司。其簡序先後,精覈高下,歲月之次,文武之宜,務存厥衷。」

庚子,恭陵百丈廡災。

是歲,起西苑,修飾宮殿。

二年春二月甲申,詔以吳郡、會稽飢荒,貸人種糧。

三月,使匈奴中郎將王稠率左骨都侯等擊鮮卑,破之。

辛酉,除京師耆儒年六十以上四十八人補郎、舍人及諸王國郎。

夏四月,復置隴西南部都尉官。

己亥,京師地震。五月庚子,詔曰:「朕以不德,統奉鴻業,無以奉順乾坤,協序陰陽,災眚屢見,咎徵仍臻。地動之異,發自京師,矜矜祗畏,不知所裁。群公卿士將何以匡輔不逮,奉荅戒異?異不空設,必有所應,其各悉心直言厥咎,靡有所諱。」

戊午,司空王龔免。六月辛未,太常魯國孔扶為司空。

疏勒國獻師子、封牛。

丁丑,洛陽地陷。是月,旱。

秋七月己未,太尉龐參免。八月己巳,大鴻臚沛國施延為太尉。

鮮卑寇代郡。

冬十月庚午,行禮辟雍,奏應鍾,始復黃鍾,作樂器隨月律。

三年春二月己丑,詔以久旱,京師諸獄無輕重皆且勿考竟,須得澍雨。

三月庚戌,益州盜賊劫質令長,殺列侯。

夏四月丙寅,車師後部司馬率後部王加特奴等掩擊匈奴,大破之,獲其季母。

五月戊戌,制詔曰:「昔我太宗,丕顯之德,假于上下,儉以恤民,政致康乂。朕秉事不明,政失厥道,天地譴怒,大變仍見。春夏連旱,寇賊彌繁,元元被害,朕甚愍之。嘉與海內洗心更始。其大赦天下,自殊死以下謀反大逆諸犯不當得赦者,皆赦除之。賜民年八十以上米,一斛,肉二十斤,酒五斗;九十以上加賜帛,人二匹,絮三斤。」

秋七月庚戌,鍾羌寇隴西、漢陽。冬十月,護羌校尉馬續擊破之。

十一月壬寅,司徒劉崎、司空孔扶免。乙巳,大司農南郡黃尚為司徒,光祿勳河東王卓為司空。

丙午,武都塞上屯羌及外羌攻破屯官,驅略人畜。

四年春二月丙子,初聽中官得以養子為後,世襲封爵。

自去冬旱,至于是月。

謁者馬賢擊鍾羌,大破之。

夏四月甲子,太尉施延免。戊寅,執金吾梁商為大將軍,前太尉龐參為太尉。

六月己未,梁王匡薨。秋七月己亥,濟北王登薨。

閏月丁亥朔,日有食之。

冬十月,烏桓寇雲中。十一月,圍度遼將軍耿曄於蘭池,發諸郡兵救之,烏桓退走。

十二月甲寅,京師地震。

永和元年春正月,夫餘王來朝。

乙卯,詔曰:「朕秉政不明,災眚屢臻。典籍所忌,震食為重。今日變方遠,地搖京師,咎徵不虛,必有所應。群公百僚其各上封事,指陳得失,靡有所諱。」

己巳,宗祀明堂,登靈臺,改元永和,大赦天下。

秋七月,偃師蝗。

冬十月丁亥,承福殿火,帝避御雲臺。

十一月丙子,太尉龐參罷。

十二月,象林蠻夷叛。

乙巳,以前司空王龔為太尉。

二年春正月,武陵蠻叛,圍充縣,又寇夷道。

二月,廣漢屬國都尉擊破白馬羌。

武陵太守李進擊叛蠻,破之。

三月辛亥,北海王翼薨。

乙卯,司空王卓薨。丁丑,光祿勳馮翊郭虔為司空。

夏四月丙申,京師地震。

五月,日南叛蠻攻郡府。

秋七月,九真、交阯二郡兵反。

八月庚子,熒惑犯南斗。

江夏盜賊殺邾長。

冬十月甲申,行幸長安,所過鰥、寡、孤、獨、貧不能自存者賜粟,人五斛。庚子,幸未央宮,會三輔郡守、都尉及官屬,勞賜作樂。十一月丙午,祠高廟。丁未,遂有事十一陵。丁卯,京師地震。十二月乙亥,至自長安。

三年春二月乙亥,京師及金城、隴西地震,二郡山岸崩,地陷。戊子,太白犯熒惑。

夏四月,九江賊蔡伯流寇郡界,及廣陵,殺江都長。

戊戌,遣光祿大夫案行金城、隴西,賜壓死者年七歲以上錢,人二千;一家皆被害,為收斂之。除今年田租,尤甚者勿收口賦

閏月,蔡伯流等率眾詣徐州刺史應志降。

己酉,京師地震。

五月,吳郡丞羊珍反,攻郡府,太守王衡破斬之。

六月辛丑,琅邪王遵薨。

九真太守祝良、交阯刺史張喬慰誘日南叛蠻,降之,嶺外平。

秋七月丙戌,濟北王多薨。

八月己未,司徒黃尚免。九月己酉,光祿勳長沙劉壽為司徒。

丙戌,令大將軍、三公各舉故刺史、二千石及見令、長、郎、謁者、四府掾屬剛毅武猛有謀謨任將帥者各二人,特進、卿、校尉各一人。

冬十月,燒當羌寇金城,護羌校尉馬賢擊破之,羌遂相招而叛。

十二月戊戌朔,日有食之。

四年春正月庚辰,中常侍張逵、蘧政、楊定等有罪誅,連及弘農太守張鳳、安平相楊皓,下獄死。

三月乙亥,京師地震。

夏四月癸卯,護羌校尉馬賢討燒當羌,大破之。

戊午,大赦天下。賜民爵及粟帛各有差。

五月戊辰,封故濟北惠王壽子安為濟北王。

秋八月,太原郡旱,民庶流冗。癸丑,遣光祿大夫案行稟貸,除更賦。

冬十月戊午,校獵上林苑,歷函谷關而還。十一月丙寅,幸廣成苑。

五年春二月戊申,京師地震。

夏四月庚子,中山王弘薨。

南匈奴左部句龍大人吾斯、車紐等叛,圍美稷。

五月,度遼將軍馬續討吾斯、車紐,破之,使匈奴中郎將陳龜迫殺南單于。

己丑晦,日有食之。

且凍羌寇三輔,殺令長。

丁丑,令死罪以下及亡命贖,各有差。

九月,令扶風、漢陽築隴道塢三百所,置屯兵。

辛未,太尉王龔罷。

且凍羌寇武都,燒隴關。

壬午,太常桓焉為太尉。

丁亥,徙西河郡居離石,上郡居夏陽,朔方居五原。

句龍吾斯等東引烏桓,西收羌胡,寇上郡,立車紐為單于。冬十一月辛巳,遣使匈奴中郎將張耽擊破之,車紐降。

六年春正月丙子,征西將軍馬賢與且凍羌戰于射姑山,賢軍敗沒,安定太守郭璜下獄死。

詔貸王、侯國租一歲。

閏月,鞏唐羌寇隴西,遂及三輔。

二月丁巳,有星孛于營室。

三月,武都太守趙沖討鞏唐羌,破之。

庚子,司空郭虔免。

丁巳,河閒王政薨。

丙午,太僕趙戒為司空。

夏五月庚子,齊王無忌薨。

使匈奴中郎將張耽大破烏桓、羌胡於天山。

鞏唐羌寇北地。

秋七月甲午,詔假民有貲者戶錢一千。

八月丙辰,大將軍梁商薨;壬戌,河南尹梁冀為大將軍。

九月,諸種羌寇武威。

辛亥晦,日有食之。

冬十月癸丑,徙安定居扶風,北地居馮翊。

十一月庚子,以執金吾張喬行車騎將軍事,將兵屯三輔。

漢安元年春正月癸巳,宗祀明堂,大赦天下,改元漢安。

二月丙辰,詔大將軍、公、卿舉賢良方正、能探賾索隱者各一人。

秋七月,始置承華廄。

八月,南匈奴左部大人句龍吾斯與薁鞬臺耆等反叛。

丁卯,遣侍中杜喬、光祿大夫周舉、守光祿大夫郭遵、馮羨、欒巴、張綱、周栩、劉班等八人分行州郡,班宣風化,舉實臧否。

九月庚寅,廣陵盜賊張嬰等寇郡縣。

冬十月辛未,太尉桓焉、司徒劉壽免。甲戌,行車騎將軍張喬罷。十一月壬午,司隸校尉趙峻為太尉,大司農胡廣為司徒。

癸卯,詔大將軍、三公選武猛試用有效驗任為將校者各一人。

是歲,廣陵賊張嬰等詣太守張綱降。

二年春二月丙辰,鄯善國遣使貢獻。

夏四月庚戌,護羌校尉趙沖與漢陽太守張貢擊燒當羌於參讀,破之。

六月乙丑,熒惑犯鎮星。

丙寅,立南匈奴守義王兜樓儲為南單于。

冬十月辛丑,令郡國中都官繫囚殊死以下出縑贖,各有差;其不能入贖者,遣詣臨羌縣居作二歲。

甲辰,減百官奉。丙午,禁沽酒,又貸王、侯國租一歲。

閏月,趙沖擊燒當羌於河陽,破之。

十一月,使匈奴中郎將馬寔遣人刺殺句龍吾斯。

十二月,楊、徐盜賊攻燒城寺,殺略吏民。

是歲,涼州地百八十震。

建康元年春正月辛丑,詔曰:「隴西、漢陽、張掖、北地、武威、武都,自去年九月已來,地百八十震,山谷坼裂,壞敗城寺,殺害民庶。夷狄叛逆,賦役重數,內外怨曠,惟咎歎息。其遣光祿大夫案行,宣暢恩澤,惠此下民,勿為煩擾。」

三月庚子,沛王廣薨。

領護羌校尉衛琚追討叛羌,破之。

南郡、江夏盜賊寇掠城邑,州郡討平之。

夏四月,使匈奴中郎將馬寔擊南匈奴左部,破之,於是胡羌、烏桓悉詣寔降。

辛巳,立皇子炳為皇太子,改年建康,大赦天下。賜人爵各有差。

秋七月丙午,清河王延平薨。

八月,楊、徐盜賊范容、周生等寇掠城邑,遣御史中丞馮赦督州郡兵討之。

庚午,帝崩于玉堂前殿,時年三十。遺詔無起寑廟,斂以故服,珠玉玩好皆不得下。

論曰:古之人君,離幽放而反國祚者有矣,莫不矯鑒前違,審識情偽,無忘在外之憂,故能中興其業。觀夫順朝之政,殆不然乎?何其傚僻之多與?

孝沖皇帝諱炳,順帝之子也。母曰虞貴人。

建康元年立為皇太子,其年八月庚午,即皇帝位,年二歲。尊皇后曰皇太后。太后臨朝。

丁丑,以太尉趙峻為太傅;大司農李固為太尉,參錄尚書事。

九月丙午,葬孝順皇帝于憲陵,廟曰敬宗。

是日,京師及太原、鴈門地震,三郡水涌土裂。

庚戌,詔三公、特進、侯、卿、校尉,舉賢良方正、幽逸修道之士各一人,百僚皆上封事。

己未,九江太守丘騰有罪,下獄死。

楊州刺史尹耀、九江太守鄧顯討賊范容等於歷陽

,軍敗,耀、顯為賊所歿。

冬十月,日南蠻夷攻燒城邑,交阯刺史夏方招誘降之。

壬申,常山王儀薨。

己卯,零陵太守劉康坐殺無辜,下獄死。

十一月,九江盜賊徐鳳、馬勉等稱「無上將軍」,攻燒城邑。

己酉,令郡國中都官繫囚減死一等,徙邊;謀反大逆,不用此令。

十二月,九江賊黃虎等攻合肥。

是歲,群盜發憲陵。護羌校尉趙沖追擊叛羌於鸇陰河,戰歿。

永嘉元年春正月戊戌,帝崩于玉堂前殿,年三歲。清河王蒜徵至京師。

孝質皇帝諱纘,肅宗玄孫。曾祖父千乘貞王伉,祖父樂安夷王寵,父勃海孝王鴻,母陳夫人。沖帝不豫,大將軍梁冀徵帝到洛陽都亭。及沖帝崩,皇太后與冀定策禁中,丙辰,使冀持節,以王青蓋車迎帝入南宮。丁巳,封為建平侯,其日即皇帝位,年八歲。

己未,葬孝沖皇帝于懷陵。

廣陵賊張嬰等復反,攻殺堂邑、江都長。九江賊徐鳳等攻殺曲陽、東城長。

甲申,謁高廟,乙酉,謁光武廟。

二月,豫章太守虞續坐贓,下獄死。

乙酉,大赦天下,賜人爵及粟帛各有差。還王侯所削戶邑。

彭城王道薨。

叛羌詣左馮翊梁並降。

三月,九江賊馬勉稱「黃帝」。九江都尉滕撫討馬勉、范容、周生、大破斬之。

夏四月壬申,雩。

庚辰,濟北王安薨。

丹陽賊陸宮等圍城,燒亭寺,丹陽太守江漢擊破之。

五月甲午,詔曰:「朕以不德,託母天下,布政不明,每失厥中。自春涉夏,大旱炎赫,憂心京京,故得禱祈明祀,冀蒙潤澤。前雖得雨,而宿麥頗傷;比日陰雲,還復開霽。寤寐永歎,重懷慘結。將二千石、令長不崇寬和,暴刻之為乎?其令中都官繫囚罪非殊死考未竟者,一切任出,以須立秋。郡國有名山大澤能興雲雨者,二千石長吏各絜齊請禱,謁誠盡禮。又兵役連年,死亡流離,或支骸不斂,或停棺莫收,朕甚愍焉。昔文王葬枯骨,人賴其德。今遣使者案行,若無家屬及貧無資者,隨宜賜卹,以慰孤魂。」

是月,下邳人謝安應募擊徐鳳等,斬之。

丙辰,詔曰:「孝殤皇帝雖不永休祚,而即位踰年,君臣禮成。孝安皇帝承襲統業,而前世遂令恭陵在康陵之上,先後相踰,失其次序,非所以奉宗廟之重,垂無窮之制。昔定公追正順祀,春秋善之。其令恭陵次康陵,憲陵次恭陵,以序親秩,為萬世法。」

六月,鮮卑寇代郡。

秋七月庚寅,阜陵王代薨。

廬江盜賊攻尋陽,又攻盱台,滕撫遣司馬王章擊破之。

九月庚戌,太傅趙峻薨。

冬十一月己丑,南陽太守韓昭坐贓下獄死。

丙午,中郎將滕撫擊廣陵賊張嬰,破之。

丁未,中郎將趙序坐事棄巿。

歷陽賊華孟自稱「黑帝」,攻殺九江太守楊岑,滕撫率諸將擊孟等,大破斬之。

本初元年春正月丙申,詔曰:「昔堯命四子,以欽天道,鴻範九疇,休咎有象。夫瑞以和降,異因逆感,禁微應大,前聖所重。頃者,州郡輕慢憲防,競逞殘暴,造設科條,陷入無罪。或以喜怒驅逐長吏,恩阿所私,罰枉仇隙,至令守闕訴訟,前後不絕。送故迎新,人離其害,怨氣傷和,以致災眚。書云:『明德慎罰。』方春東作,育微敬始。其敕有司,罪非殊死,且勿案驗,以崇在寬。」

壬子,廣陵太守王喜坐討賊逗留,下獄死。

二月庚辰,詔曰:「九江、廣陵二郡數離寇害,殘夷最甚。生者失其資業,死者委尸原野。昔之為政,一物不得其所,若己為之,況我元元,嬰此困毒。方春戒節,賑濟乏厄,掩骼埋胔之時。其調比郡見穀,出稟窮弱,收葬枯骸,務加埋卹,以稱朕意。」

夏四月庚辰,令郡國舉明經,年五十以上、七十以下詣太學。自大將軍至六百石,皆遣子受業,歲滿課試,以高第五人補郎中,次五人太子舍人。又千石、六百石、四府掾屬、三署郎、四姓小侯先能通經者,各令隨家法,其高第者上名牒,當以次賞進。

五月庚寅,徙樂安王為勃海王。

海水溢。戊申,使謁者案行,收葬樂安、北海人為水所漂沒死者,又稟給貧羸。

庚戌,太白犯熒惑。

六月丁巳,大赦天下,賜民爵及粟帛各有差。

閏月甲申,大將軍梁冀潛行鴆弒,帝崩于玉堂前殿,年九歲。

丁亥,太尉李固免。戊子,司徒胡廣為太尉,司空趙戒為司徒,與梁冀參錄尚書事。太僕袁湯為司空。

贊曰:孝順初立,時髦允集。匪砥匪革,終淪嬖習。保阿傳土,后家世及。沖夭未識,質弒以聰。陵折在運,天緒三終。


\end{pinyinscope}