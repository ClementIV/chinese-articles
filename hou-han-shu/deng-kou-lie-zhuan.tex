\article{鄧寇列傳}

\begin{pinyinscope}
鄧禹字仲華,南陽新野人也。年十三,能誦詩,受業長安。時光武亦游學京師,禹年雖幼,而見光武知非常人,遂相親附。數年歸家。

及漢兵起,更始立,豪桀多薦舉禹,禹不肯從。及聞光武安集河北,即杖策北渡,追及於鄴。光武見之甚歡,謂曰:「我得專封拜,生遠來,寧欲仕乎?」禹曰:「不願也。」光武曰:「即如是,何欲為?」禹曰:「但願明公威德加於四海,禹得效其尺寸,垂功名於竹帛耳。」光武笑,因留宿閒語。禹進說曰:「更始雖都關西,今山東未安,赤眉、青犢之屬,動以萬數,三輔假號,往往群聚。更始既未有所挫,而不自聽斷,諸將皆庸人屈起,志在財幣,爭用威力,朝夕自快而已,非有忠良明智,深慮遠圖,欲尊主安民者也。四方分崩離析,形埶可見。明公雖建藩輔之功,猶恐無所成立。於今之計,莫如延攬英雄,務悅民心,立高祖之業,救萬民之命。以公而慮天下,不足定也。」光武大悅,因令左右號禹曰鄧將軍。常宿止於中,與定計議。

及王郎起兵,光武自薊至信都,使禹發奔命,得數千人,令自將之,別攻拔樂陽。從至廣阿,光武舍城樓上,披輿地圖,指示禹曰:「天下郡國如是,今始乃得其一。子前言以吾慮天下不足定,何也?」禹曰:「方今海內殽亂,人思明君,猶赤子之慕慈母。古之興者,在德薄厚,不以大小。」光武悅。時任使諸將,多訪於禹,禹每有所舉者,皆當其才,光武以為知人。使別將騎,與蓋延等擊銅馬於清陽。延等先至,戰不利,還保城,為賊所圍。禹遂進與戰,破之,生獲其大將。從光武追賊至滿陽,連大克獲,北州略定。

及赤眉西入關,更始使定國上公王匡、襄邑王成丹、抗威將軍劉均及諸將,分據河東、弘農以拒之。赤眉眾大集,王匡等莫能當。光武籌赤眉必破長安,欲乘舋并關中,而方自事山東,未知所寄,以禹沈深有大度,故授以西討之略。乃拜為前將軍持節,中分麾下精兵二萬人,遣西入關,令自選偏裨以下可與俱者。於是以韓歆為軍師,李文、李春、程慮為祭酒,馮愔為積弩將軍,樊崇為驍騎將軍,宗歆為車騎將軍,鄧尋為建威將軍,耿訢為赤眉將軍,左于為軍師將軍,引而西。

建武元年正月,禹自箕關將入河東,河東都尉守關不開,禹攻十日,破之,獲輜重千餘乘。進圍安邑,數月未能下。更始大將軍樊參將數萬人,度大陽欲攻禹,禹遣諸將逆擊於解南,大破之,斬參首。於是王匡、成丹、劉均等合軍十餘萬,復共擊禹,禹軍不利,樊崇戰死。會日暮,戰罷,軍師韓歆及諸將見兵埶已摧,皆勸禹夜去,禹不聽。明日癸亥,匡等以六甲窮日不出,禹因得更理兵勒眾。明旦,匡悉軍出攻禹,禹令軍中無得妄動;既至營下,因傳發諸將鼓而並進,大破之。匡等皆棄軍亡走,禹率輕騎急追,獲劉均及河東太守楊寶、持節中郎將弭彊,皆斬之,收得節六,印綬五百,兵器不可勝數,遂定河東。承制拜李文為河東太守,悉更置屬縣令長以鎮撫之。是月,光武即位於鄗,使使者持節拜禹為大司徒。策曰:「制詔前將軍禹:深執忠孝,與朕謀謨帷幄,決勝千里。孔子曰:『自吾有回,門人日親。』斬將破軍,平定山西,功效尤著。百姓不親,五品不訓,汝作司徒,敬敷五教,五教在寬。今遣奉車都尉授印綬,封為酇侯,食邑萬戶。敬之哉!」禹時年二十四。

遂渡汾陰河,入夏陽。更始中郎將左輔都尉公乘歙,引其眾十萬,與左馮翊兵共拒禹於衙,禹復破走之,而赤眉遂入長安。是時三輔連覆敗,赤眉所過殘賊,百姓不知所歸。聞禹乘勝獨剋而師行有紀,皆望風相攜負以迎軍,降者日以千數,眾號百萬。禹所止輒停車住節,以勞來之,父老童稚,垂髮戴白,滿其車下,莫不感悅,於是名震關西。帝嘉之,數賜書褒美。

諸將豪傑皆勸禹徑攻長安。禹曰:「不然。今吾眾雖多,能戰者少,前無可仰之積,後無轉饋之資。赤眉新拔長安,財富充實,鋒銳未可當也。夫盜賊群居,無終日之計,財穀雖多,變故萬端,寧能堅守者也?上郡、北地、安定三郡,土廣人稀,饒穀多畜,吾且休兵北道,就糧養士,以觀其弊,乃可圖也。」於是引軍北至栒邑。禹所到,擊破赤眉別將諸營保,郡邑皆開門歸附。西河太守宗育遣子奉檄降,禹遣詣京師。

帝以關中未定,而禹久不進兵,下敕曰:「司徒,堯也;亡賊,桀也。長安吏人,遑遑無所依歸。宜以時進討,鎮慰西京,繫百姓之心。」禹猶執前意,乃分遣將軍別攻上郡諸縣,更徵兵引穀,歸至大要。遣馮愔、宗歆守栒邑。二人爭權相攻,愔遂殺歆,因反擊禹,禹遣使以聞帝。帝問使人:「愔所親愛為誰」,對曰:「護軍黃防。」帝度愔、防不能久和,執必相忤,因報禹曰:「縛馮愔者,必黃防也。」乃遣尚書宗廣持節降之。後月餘,防果執愔,將其眾歸罪。更始諸將王匡、胡殷成丹等皆詣廣降,與共東歸。至安邑,道欲亡,廣悉斬之。愔至洛陽,赦不誅。

二年春,遣使者更封禹為梁侯,食四縣。時赤眉西走扶風,禹乃南至長安,軍昆明池,大饗士卒。率諸將齋戒,擇吉日,修禮謁祠高廟,收十一帝神主,遣使奉詣洛陽,因循行園陵,為置吏士奉守焉。

禹引兵與延岑戰於藍田,不克,復就穀雲陽。漢中王劉嘉詣禹降。嘉相李寶倨慢無禮,禹斬之。寶弟收寶部曲擊禹,殺將軍耿訢。自馮愔反後,禹威稍損,又乏食,歸附者離散。而赤眉復還入長安,禹與戰,敗走,至高陵,軍士飢餓者,皆食棗菜。帝乃徵禹還,敕曰:「赤眉無穀,自當來東,吾折捶笞之,非諸將憂也。無得復妄進兵。」禹慚於受任而功不遂,數以飢卒徼戰,輒不利。三年春,與車騎將軍鄧弘擊赤眉,遂為所敗,眾皆死散。事在馮異傳。獨與二十四騎還詣宜陽,謝上大司徒、梁侯印綬。有詔歸侯印綬。數月,拜右將軍。

延岑自敗於東陽,遂與秦豐合。四年春,復寇順陽閒。遣禹護復漢將軍鄧曄、輔漢將軍于匡,擊破岑於鄧;追至武當,復破之。岑奔漢中,餘黨悉降。

十三年,天下平定,諸功臣皆增戶邑,定封禹為高密侯,食高密、昌安、夷安、淳于四縣。帝以禹功高,封弟寬為明親侯。其後左右將軍官罷,以特進奉朝請。禹內文明,篤行淳備,事母至孝。天下既定,常欲遠名埶。有子十三人,各使守一蓺。修整閨門,教養子孫,皆可以為後世法。資用國邑,不修產利。帝益重之。中元元年,復行司徒事。從東巡狩,封岱宗。

顯宗即位,以禹先帝元功,拜為太傅,進見東向,甚見尊寵。居歲餘,寢疾。帝數自臨問,以子男二人為郎。永平元年,年五十七薨,謚曰元侯。

帝分禹封為三國:長子震為高密侯,襲為昌安侯,珍為夷安侯。

禹少子鴻,好籌策。永平中,以為小侯。引入與議邊事,帝以為能,拜將兵長史,率五營士屯鴈門。肅宗時,為度遼將軍。永元中,與大將軍竇憲俱出擊匈奴,有功,徵行車騎將軍。出塞追畔胡逢侯,坐逗留,下獄死。

高密侯震卒,子乾嗣。乾尚顯宗女沁水公主。永元十四年,陰皇后巫蠱事發,乾從兄奉以后舅被誅,乾從坐,國除。元興元年,和帝復封乾本國,拜侍中。乾卒,子成嗣。成卒,子褒嗣。褒尚安帝妹舞陰長公主,桓帝時為少府。褒卒,長子某嗣。少子昌襲母爵為舞陰侯,拜黃門侍郎。

昌安侯襲嗣子藩,亦尚顯宗女平皋長公主,和帝時為侍中。

夷安侯珍子康,少有操行。兄良襲封,無後,永初六年,紹封康為夷安侯。時諸紹封者皆食故國半租,康以皇太后戚屬,獨三分食二,以侍祠侯為越騎校尉。康以太后久臨朝政,宗門盛滿,數上書長樂宮諫爭,宜崇公室,自損私權,言甚切至。太后不從。康心懷畏懼,永寧元年,遂謝病不朝。太后使內侍者問之。時宮人出入,多能有所毀譽,其中耆宿皆稱中大人。所使者乃康家先婢,亦自通中大人。康聞,詬之曰:「汝我家出,亦敢爾邪!」婢怨恚,還說康詐疾而言不遜。太后大怒,遂免康官,遣歸國,絕屬籍。及從兄騭誅,安帝徵康為侍中。順帝立,為太僕,有方正稱,名重朝廷。以病免,加位特進。陽嘉三年卒,謚曰義侯。

論曰:夫變通之世,君臣相擇,斯最作事謀始之幾也。鄧公嬴糧徒步,觸紛亂而赴光武,可謂識所從會矣。於是中分麾下之軍,以臨山西之隙,至使關河響動,懷赴如歸。功雖不遂,而道亦弘矣!及其威損栒邑,兵散宜陽,褫龍章於終朝,就侯服以卒歲,榮悴交而下無二色,進退用而上無猜情,使君臣之美,後世莫闚其閒,不亦君子之致為乎!

訓字平叔,禹第六子也。少有大志,不好文學,禹常非之。顯宗即位,初以為郎中。訓樂施下士,士大夫多歸之。

永平中,理虖沱、石臼河,從都慮至羊腸倉,欲令通漕。太原吏人苦役,連年無成,轉運所經三百八十九隘,前後沒溺死者不可勝筭。建初三年,拜訓謁者,使監領其事。訓考量隱括,知大功難立,具以上言。肅宗從之,遂罷其役,更用驢輦,歲省費億萬計,全活徒士數千人。

會上谷太守任興欲誅赤沙烏桓,怨恨謀反,詔訓將黎陽營兵屯狐奴,以防其變。訓撫接邊民,為幽部所歸。六年,遷護烏桓校尉,黎陽故人多攜將老幼,樂隨訓徙邊。鮮卑聞其威恩,皆不敢南近塞下。八年,舞陰公主子梁扈有罪,訓坐私與扈通書,徵免歸閭里。

元和三年,盧水胡反畔,以訓為謁者,乘傳到武威,拜張掖太守。

章和二年,護羌校尉張紆誘誅燒當種羌迷吾等,由是諸羌大怒,謀欲報怨,朝廷憂之。公卿舉訓代紆為校尉。諸羌激忿,遂相與解仇結婚,交質盟詛,眾四萬餘人,期冰合度河攻訓。先是小月氏胡分居塞內,勝兵者二三千騎,皆勇健富彊,每與羌戰,常以少制多。雖首施兩端,漢亦時收其用。時迷吾子迷唐,別與武威種羌合兵萬騎,來至塞下,未敢攻訓,先欲脅月氏胡。訓擁衛稽故,令不得戰。議者咸以羌胡相攻,縣官之利,以夷伐夷,不宜禁護。訓曰:「不然。今張紆失信,眾羌大動,經常屯兵,不下二萬,轉運之費,空竭府帑,涼州吏人,命縣絲髮。原諸胡所以難得意者,皆恩信不厚耳。今因其迫急,以德懷之,庶能有用。」遂令開城及所居園門,悉驅群胡妻子內之,嚴兵守衛。羌掠無所得,又不敢逼諸胡,因即解去。由是湟中諸胡皆言「漢家常欲鬥我曹,今鄧使君待我以恩信,開門內我妻子,乃得父母」。咸歡喜叩頭曰:「唯使君所命。」訓遂撫養其中少年勇者數百人,以為義從。

羌胡俗恥病死,每病臨困,輒以刃自刺。訓聞有困疾者,輒拘持縛束,不與兵刃,使醫藥療之,愈者非一,小大莫不感悅。於是賞賂諸羌種,使相招誘。迷唐伯父號吾乃將其母及種人八百戶,自塞外來降。訓因發湟中秦、胡、羌兵四千人,出塞掩擊迷唐於寫谷,斬首虜六百餘人,得馬牛羊萬餘頭。迷唐乃去大、小榆,居頗巖谷,眾悉破散。其春,復欲歸故地就田業,訓乃發湟中六千人,令長史任尚將之,縫革為船,置於箄上以度河,掩擊迷唐廬落大豪,多所斬獲。復追逐奔北,會尚等夜為羌所攻,於是義從羌胡并力破之,斬首前後一千八百餘級,獲生口二千人,馬牛羊三萬餘頭,一種殆盡。迷唐遂收其餘部,遠徙廬落,西行千餘里,諸附落小種皆背畔之。燒當豪帥東號稽顙歸死,餘皆款塞納質。於是綏接歸附,威信大行。遂罷屯兵,各令歸郡。唯置弛刑徒二千餘人,分以屯田,為貧人耕種,修理城郭塢壁而已。

永元二年,大將軍竇憲將兵鎮武威,憲以訓曉羌胡方略,上求俱行。訓初厚於馬氏,不為諸竇所親,及憲誅,故不離其禍。

訓雖寬中容眾,而於閨門甚嚴,兄弟莫不敬憚,諸子進見,未嘗賜席接以溫色。四年冬,病卒官,時年五十三。吏人羌胡愛惜,旦夕臨者日數千人。戎俗父母死,恥悲泣,皆騎馬歌呼。至聞訓卒,莫不吼號,或以刀自割,又刺殺其犬馬牛羊,日「鄧使君已死,我曹亦俱死耳」。前烏桓吏士皆奔走道路,至空城郭。吏執不聽,以狀白校尉徐傿。傿歎息曰:「此義也。」乃釋之。遂家家為訓立祠,每有疾病,輒此請禱求福。

元興元年,和帝以訓皇后之父,使謁者持節至訓墓,賜策追封,謚曰平壽敬侯。中宮自臨,百官大會。

訓五子:騭,京,悝,弘,閶。

騭字昭伯,少辟大將軍竇憲府。及女弟為貴人,騭兄弟皆除郎中。及貴人立,是為和熹皇后。騭三遷虎賁中郎將,京、悝、弘、閶皆黃門侍郎。京卒於官。延平元年,拜騭車騎將軍、儀同三司。始自騭也。悝虎賁中郎將,弘、閶皆侍中。

殤帝崩,太后與騭等定策立安帝,悝遷城門校尉,弘虎賁中郎將。自和帝崩後,騭兄弟常居禁中。騭謙遜不欲久在內,連求還第,歲餘,太后乃許之。

永初元年,封騭上蔡侯,悝葉侯,弘西平侯,閶西華侯,食邑各萬戶。騭以定策功,增邑三千戶。騭等辭讓不獲,遂逃避使者,閒關詣闕,上疏自陳曰:「臣兄弟汙濊,無分可採,過以外戚,遭值明時,託日月之末光,被雲雨之渥澤,並統列位,光昭當世。不能宣贊風美,補助清化,誠慚誠懼,無以處心。陛下躬天然之姿,體仁聖之德,遭國不造,仍離大憂,開日月之明,運獨斷之慮,援立皇統,奉承大宗。聖策定於神心,休烈垂於不朽,本非臣等所能萬一,而猥推嘉美,並享大封,伏聞詔書,驚惶慚怖。追觀前世傾覆之誡,退自惟念,不寒而慄。臣等雖無逮及遠見之慮,猶有庶幾戒懼之情。常母子兄弟,內相敕厲,冀以端愨畏慎,一心奉戴,上全天恩,下完性命。刻骨定分,有死無二。終不敢橫受爵土,以增罪累。惶窘征營,昧死陳乞。」太后不聽。騭頻上疏,至於五六,乃許之。

其夏,涼部畔羌搖蕩西州,朝廷憂之。於是詔騭將左右羽林、北軍五校士及諸部兵擊之,車駕幸平樂觀餞送。騭西屯漢陽,使征西校尉任尚、從事中郎司馬鈞與羌戰,大敗。時以轉輸疲弊,百姓苦役。冬,徵騭班師。朝廷以太后故,遣五官中郎將迎拜騭為大將軍。軍到河南,使大鴻臚親迎,中常侍齎牛酒郊勞,王、主以下候望於道。既至,大會群臣,賜束帛乘馬,寵靈顯赫,光震都鄙。

時遭元二之災,人士荒飢,死者相望,盜賊群起,四夷侵畔。騭等崇節儉,罷力役,推進天下賢士何熙、祋諷、羊浸、李郃、陶敦等列於朝廷,辟楊震、朱寵、陳禪置之幕府,故天下復安。

四年,母新野君寢病,騭兄弟並上書求還侍養。太后以閶最少,孝行尤著,特聽之,賜安車駟馬。及新野君薨,騭等復乞身行服,章連上,太后許之。騭等既還里第,並居冢次。閶至孝骨立,有聞當時。及服闋,詔喻騭還輔朝政,更授前封。騭等叩頭固讓,乃止,於是並奉朝請,位次在三公下,特進、侯上。其有大議,乃詣朝堂,與公卿參謀。

元初二年,弘卒。太后服齊衰,帝絲麻,並宿幸其第。弘少治歐陽尚書,授帝禁中,諸儒多歸附之。初疾病,遺言悉以常服,不得用錦衣玉匣。有司奏贈弘驃騎將軍,位特進,封西平侯。太后追思弘意,不加贈位衣服,但賜錢千萬,布萬匹,騭等復辭不受。詔大鴻臚持節,即弘殯封子廣德為西平侯。將葬,有司復奏發五營輕車騎士,禮儀如霍光故事,太后皆不聽,但白蓋雙騎,門生輓送。後以帝師之重,分西平之都鄉封廣德弟甫德為都鄉侯。四年,又封京子黃門侍郎珍為陽安侯,邑三千五百戶。

五年,悝、閶相繼並卒,皆遺言薄葬,不受爵贈,太后並從之。乃封悝子廣宗為葉侯,閶子忠為西華侯。

自祖父禹教訓子孫,皆遵法度,深戒竇氏,檢敕宗族,闔門靜居。騭子侍中鳳,嘗與尚書郎張龕書,屬郎中馬融宜在臺閣。又中郎將任尚嘗遺鳳馬,後尚坐斷盜軍糧,檻車徵詣廷尉,鳳懼事泄,先自首於騭。騭畏太后,遂髡妻及鳳以謝,天下稱之。

建光元年,太后崩,未及大斂,帝復申前命,封騭為上蔡侯,位特進。帝少號聰敏,及長多不德,而乳母王聖見太后久不歸政,慮有廢置,常與中黃門李閏候伺左右。及太后崩,宮人先有受罰者,懷怨恚,因誣告悝、弘、閶先從尚書鄧訪取廢帝故事,謀立平原王得。帝聞,追怒,令有司奏悝等大逆無道,遂廢西平侯廣德、葉侯廣宗、西華侯忠、陽安侯珍、都鄉侯甫德皆為庶人。騭以不與謀,但免特進,遣就國。宗族皆免官歸故郡,沒入騭等貲財田宅,徙鄧訪及家屬於遠郡。郡縣逼迫,廣宗及忠皆自殺。又徙封騭為羅侯,騭與子鳳並不食而死。騭從弟河南尹豹、度遼將軍舞陽侯遵、將作大匠暢皆自殺,唯廣德兄弟以母閻后戚屬得留京師。

大司農朱寵痛騭無罪遇禍,乃肉袒輿櫬,上疏追訟騭曰:「伏惟和熹皇后聖善之德,為漢文母。兄弟忠孝,同心憂國,宗廟有主,王室是賴。功成身退,讓國遜位,歷世外戚,無與為比。當享積善履謙之祐,而橫為宮人單辭所陷。利口傾險,反亂國家,罪無申證,獄不訊鞠,遂令騭等罹此酷濫。一門七人,並不以命,屍骸流離,怨魂不反,逆天感人,率土喪氣。宜收還冢次,寵樹遺孤,奉承血祀,以謝亡靈。」寵知其言切,自致廷尉,詔免官歸田里。眾庶多為騭稱枉,帝意頗悟,乃譴讓州郡,還葬洛陽北芒舊塋,公卿皆會喪,莫不悲傷之。詔遣使者祠以中牢,諸從昆弟皆歸京師。及順帝即位,追感太后恩訓,龟騭無辜,乃詔宗正復故大將軍鄧騭宗親內外,朝見皆如故事。除騭兄弟子及門從十二人悉為郎中,擢朱寵為太尉,錄尚書事。

寵字仲威,京兆人,初辟騭府,稍遷潁川太守,治理有聲。及拜太尉,封安鄉侯,甚加優禮。

廣德早卒。甫德更召徵為開封令。學傳父業。喪母,遂不仕。

閶妻耿氏有節操,痛鄧氏誅廢,子忠早卒,乃養河南尹豹子嗣為閶後。耿氏教之書學,遂以通博稱。永壽中,與伏無忌、延篤著書東觀,官至屯騎校尉。

禹曾孫香子女為桓帝后,帝又紹封度遼將軍遵子萬世為南鄉侯,拜河南尹。及后廢,萬世下獄死,其餘宗親皆復歸故郡。

鄧氏自中興後,累世寵貴,凡侯者二十九人,公二人,大將軍以下十三人,中二千石十四人,列校二十二人,州牧、郡守四十八人,其餘侍中、將、大夫、郎、謁者不可勝數,東京莫與為比。

論曰:漢世外戚,自東、西京十有餘族,非徒豪橫盈極,自取災故,必於貽釁後主,以至顛敗者,其數有可言焉。何則?恩非己結,而權已先之;情疏禮重,而枉性圖之;來寵方授,地既害之;隙開埶謝,讒亦勝之。悲哉!騭、悝兄弟,委遠時柄,忠勞王室,而終莫之免,斯樂生所以泣而辭燕也!

寇恂字子翼,上谷昌平人也,世為著姓。恂初為郡功曹,太守耿況甚重之。

王莽敗,更始立,使使者徇郡國,曰「先降者復爵位」。恂從耿況迎使者於界上,況上印綬,使者納之,一宿無還意。恂勒兵入見使者,就請之。使者不與,曰:「天王使者,功曹欲脅之邪?」恂曰:「非敢脅使君,竊傷計之不詳也。今天下初定,國信未宣,使君建節銜命,以臨四方,郡國莫不延頸傾耳,望風歸命。今始至上谷而先墮大信,沮向化之心,生離畔之隙,將復何以號令它郡乎?且耿府君在上谷,久為吏人所親,今易之,得賢則造次未安,不賢則秖更生亂。為使君計,莫若復之以安百姓。」使者不應,恂叱左右以使者命召況。況至,恂進取印綬帶況。使者不得已,乃承制詔之,況受而歸。

及王郎起,遣將徇上谷,急況發兵。恂與門下掾閔業共說況曰:「邯鄲拔起,難可信向。昔王莽時,所難獨有劉伯升耳。今聞大司馬劉公,伯升母弟,尊賢下士,士多歸之,可攀附也。」況曰:「邯鄲方盛,力不能獨拒,如何?」恂對曰:「今上谷完實,控弦萬騎,舉大郡之資,可以詳擇去就。恂請東約漁陽,齊心合眾,邯鄲不足圖也。」況然之,乃遣恂到漁陽,結謀彭寵。恂還,至昌平,襲擊邯鄲使者,殺之,奪其軍,遂與況子弇等俱南及光武於廣阿。拜恂為偏將軍,號承義侯,從破群賊。數與鄧禹謀議,禹奇之,因奉牛酒共交歡。

光武南定河內,而更始大司馬朱鮪等盛兵據洛陽。又并州未安,光武難其守,問於鄧禹曰:「諸將誰可使守河內者?」禹曰:「昔高祖任蕭何於關中,無復西顧之憂,所以得專精山東,終成大業。今河內帶河為固,戶口殷實,北通上黨,南迫洛陽。寇恂文武備足,有牧人御眾之才,非此子莫可使也。」乃拜恂河內太守,行大將軍事。光武謂恂曰:「河內完富,吾將因是而起。昔高祖留蕭何鎮關中,吾今委公以河內,堅守轉運,給足軍糧,率厲士馬,防遏它兵,勿令北度而已。」光武於是復北征燕、代。恂移書屬縣,講兵肄射,伐淇園之竹,為矢百餘萬,養馬二千匹,收租四百萬斛,轉以給軍。

朱鮪聞光武北而河內孤,使討難將軍蘇茂、副將賈彊將兵三萬餘人,度鞏河攻溫。檄書至,恂即勒軍馳出,並移告屬縣,發兵會於溫下。軍吏皆諫曰:「今洛陽兵度河,前後不絕,宜待眾軍畢集,乃可出也。」恂曰:「溫,郡之藩蔽,失溫則郡不可守。」遂馳赴之。旦日合戰,而偏將軍馮異遣救及諸縣兵適至,士馬四集,幡旗蔽野。恂乃令士卒乘城鼓噪,大呼言曰:「劉公兵到!」蘇茂軍聞之,陳動,恂因奔擊,大破之,追至洛陽,遂斬賈彊。茂兵自投河死者數千,生獲萬餘人。恂與馮異過河而還。自是洛陽震恐,城門晝閉。時光武傳聞朱鮪破河內,有頃恂檄至,大喜曰:「吾知寇子翼可任也!」諸將軍賀,因上尊號,於是即位。

時軍食急乏,恂以輦車驪駕轉輸,前後不絕,尚書升斗以稟百官。帝數策書勞問恂,同門生茂陵董崇說恂曰:「上新即位,四方未定,而君侯以此時據大郡,內得人心,外破蘇茂,威震鄰敵,功名發聞,此讒人側目怨禍之時也。昔蕭何守關中,悟鮑生之言而高祖悅。今君所將,皆宗族昆弟也,無乃當以前人為鏡戒。」恂然其言,稱疾不視事。帝將攻洛陽,先至河內,恂求從軍。帝曰:「河內未可離也。」數固請,不聽,乃遣兄子寇張、姊子谷崇將突騎願為軍鋒。帝善之,皆以為偏將軍。

建武二年,恂坐繫考上書者免。是時潁川人嚴終、趙敦聚眾萬餘,與密人賈期連兵為寇。恂免數月,復拜潁川太守,與破姦將軍侯進俱擊之。數月,斬期首,郡中悉平定。封恂雍奴侯,邑萬戶。

執金吾賈復在汝南,部將殺人於潁川,恂捕得繫獄。時尚草創,軍營犯法,率多相容,恂乃戮之於市。復以為恥,歎。還過潁川,謂左右曰:「吾與寇恂並列將帥,而今為其所陷,大丈夫豈有懷侵怨而不決之者乎?今見恂,必手劍之!」恂知其謀,不欲與相見。谷崇曰:「崇,將也,得帶劍侍側。卒有變,足以相當。」恂曰:「

不然。昔藺相如不畏秦王而屈於廉頗者,為國也。區區之趙,尚有此義,吾安可以忘之乎?」乃敕屬縣盛供具,儲酒醪,執金吾軍入界,一人皆兼二人之饌。恂乃出迎於道,稱疾而還。賈復勒兵欲追之,而吏士皆醉,遂過去。恂遣谷崇以狀聞,帝乃徵恂。恂至引見,時復先在坐,欲起相避。帝曰:「天下未定,兩虎安得私鬥?今日朕分之。」於是並坐極歡,遂共車同出,結友而去。

恂歸潁川。三年,遣使者即拜為汝南太守,又使驃騎將軍杜茂將兵助恂討盜賊。盜賊清靜,郡中無事。恂素好學,乃修鄉校,教生徒,聘能為左氏春秋者,親受學焉。七年,代朱浮為執金吾。明年,從車駕擊隗囂,而潁川盜賊群起,帝乃引軍還,謂恂曰:「潁川迫近京師,當以時定。惟念獨卿能平之耳,從九卿復出,以憂國可知也。」恂對曰:「潁川剽輕,聞陛下遠踰阻險,有事隴、蜀,故狂狡乘閒相詿誤耳。如聞乘輿南向,賊必惶怖歸死,臣願執銳前驅。」即日車駕南征,恂從至潁川,盜賊悉降,而竟不拜郡。百姓遮道曰:「願從陛下復借寇君一年。」四乃留恂長社,鎮撫吏人,受納餘降。

初,隗囂將安定高峻,擁兵萬人,據高平第一,帝使待詔馬援招降峻,由是河西道開。中郎將來歙承制拜峻通路將軍,封關內侯,後屬大司馬吳漢,共圍囂於冀。及漢軍退,峻亡歸故營,復助囂拒隴阺。及囂死,峻據高平,畏誅堅守。建威大將軍耿弇率太中大夫竇士、武威太守梁統等圍之,一歲不拔。十年,帝入關,將自征之,恂時從駕,諫曰:「長安道里居中,應接近便,安定、隴西必懷震懼,此從容一處可以制四方也。今士馬疲倦,方履險阻,非萬乘之固,前年潁川,可為至戒。」帝不從。進軍及汧,峻猶不下,帝議遣使降之,乃謂恂曰:「卿前止吾此舉,今為吾行也。若峻不即降,引耿弇等五營擊之。」恂奉璽書至第一,峻遣軍師皇甫文出謁,辭禮不屈。恂怒,將誅文。諸將諫曰:「高峻精兵萬人,率多彊弩,西遮隴道,連年不下。今欲降之而反戮其使,無乃不可乎?」恂不應,遂斬之。遣其副歸告峻曰:「軍師無禮,已戮之矣。欲降,急降;不欲,固守。」峻惶恐,即日開城門降。諸將皆賀,因曰:「敢問殺其使而降其城,何也?」恂曰:「皇甫文,峻之腹心,其所取計者也。今來,辭意不屈,必無降心。全之則文得其計,殺之則峻亡其膽,是以降耳。」諸將皆曰:「非所及也。」遂傳峻還洛陽。

恂經明行修,名重朝廷,所得秩奉,厚施朋友、故人及從吏士。常曰:「吾因士大夫以致此,其可獨享之乎!」時人歸其長者,以為有宰相器。

十二年卒,謚曰威侯。子損嗣。恂同產弟及兄子、姊子以軍功封列侯者凡八人,終其身,不傳於後。

初所與謀閔業者,恂數為帝言其忠,賜爵關內侯,官至遼西太守。

十三年,復封損庶兄壽為洨侯。後徙封損扶柳侯。損卒,子釐嗣,徙封商鄉侯。釐卒,子襲嗣。

恂女孫為大將軍鄧騭夫人,由是寇氏得志於永初閒。

恂曾孫榮。

論曰:傳稱「喜怒以類者鮮矣」。夫喜而不比,怒而思難者,其唯君子乎!子曰:「伯夷、叔齊,不念舊惡,怨是用希。」於寇公而見之矣。

榮少知名,桓帝時為侍中。性矜絜自貴,於人少所與,以此見害於權寵。而從兄子尚帝妹益陽長公主,帝又聘其從孫女於後宮,左右益惡之。延熹中,遂陷以罪辟,與宗族免歸故郡,吏承望風旨,持之浸急,榮恐不免,奔闕自訟。未至,刺史張敬追劾榮以擅去邊,有詔捕之。榮逃竄數年,會赦令,不得除,積窮困,乃自亡命中上書曰:

臣聞天地之於萬物也好生,帝王之於萬人也慈愛。陛下統天理物,為萬國覆,作人父母,先慈愛,後威武,先寬容,後刑辟,自生齒以上,咸蒙德澤。而臣兄弟獨以無辜為專權之臣所見批扺,青蠅之人所共搆會。以臣婚姻王室,謂臣將撫其背,奪其位,退其身,受其埶。於是遂作飛章以被於臣,欲使墜萬仞之阬,踐必死之地,令陛下忽慈母之仁,發投杼之怒。尚書背繩墨,案空劾,不復質确其過,寘於嚴棘之下,便奏正臣罪。司隸校尉馮羨佞邪承旨,廢於王命,驅逐臣等,不得旋踵。臣奔走還郡,沒齒無怨。臣誠恐卒為豺狼橫見噬食,故冒死欲詣闕,披肝膽,布腹心。

刺史張敬好為諂諛,張設機網,復令陛下興雷電之怒。司隸校尉應奉、河南尹何豹、洛陽令袁騰並驅爭先,若赴仇敵,罰及死沒,髡剔墳墓,但未掘壙出尸,剖棺露胔耳。昔文王葬枯骨,公劉敦行葦,世稱其仁。今殘酷容媚之吏,無折中處平之心,不顧無辜之害,而興虛誣之誹,欲使嚴朝必加濫罰。是以不敢觸突天威,而自竄山林,以俟陛下發神聖之聽,啟獨睹之明,拒讒慝之謗,絕邪巧之言,救可濟之人,援沒溺之命。不意滯怒不為春夏息,淹恚不為順時怠,遂馳使郵驛,布告遠近,嚴文剋剝,痛於霜雪,張羅海內,設罝萬里,逐臣者窮人纹,追臣者極車軌,雖楚購伍員,漢求季布,無以過也。

臣遇罰以來,三赦再贖,無驗之罪,足以蠲除。而陛下疾臣愈深,有司咎臣甫力,止則見埽滅,行則為亡虜,苟生則為窮人,極死則為冤鬼,天廣而無以自覆,地厚而無以自載,蹈陸土而有沈淪之憂,遠巖牆而有鎮壓之患。精誠足以感於陛下,而哲王未肯悟。如臣犯元惡大憝,足以陳於原野,備刀鋸,陛下當班布臣之所坐,以解眾論之疑。臣思入國門,坐於胏石之上,使三槐九棘平臣之罪。而閶闔九重,陷阱步設,舉趾觸罘罝,動行絓羅網,無綠至萬乘之前,永無見信之期矣。

國君不可讎匹夫,讎之則一國盡懼。臣奔走以來,三離寒暑,陰陽易位,當煖反寒,春常凄風,夏降霜雹,又連年大風,折拔樹木。風為號令,春夏布德,議獄緩死之時。願陛下思帝堯五教在寬之德,企成湯避遠讒夫之誡,以寧風旱,以弭災兵。臣聞勇者不逃死,智者不重困,固不為明朝惜垂盡之命,願赴湘、沅之波,從屈原之悲,沈江湖之流,弔子胥之哀。臣功臣苗緒,生長王國,懼獨含恨以葬江魚之腹,無以自別於世,不勝狐死首丘之情,營魂識路之懷。犯冒王怒,觸突帝禁,伏於兩觀,陳訴毒痛,然後登金鑊,入沸湯,糜爛於熾爨之下,九死而未悔。

悲夫,久生亦復何聊!蓋忠臣殺身以解君怒,孝子殞命以寧親怨,故大舜不避塗廩浚井之難,申生不辭姬氏讒邪之謗。臣敢忘斯議,不自斃以解明朝之忿哉!乞以身塞重責。願陛下饨兄弟死命,使臣一門頗有遺類,以崇陛下寬饒之惠。先死陳情,臨章涕泣,泣血連如。

帝省章愈怒,遂誅榮。寇氏由是衰廢。

贊曰:元侯淵謨,乃作司徒。明啟帝略,肇定秦都。勳成智隱,靜其如愚。子翼守溫,蕭公是埒,係兵轉食,以集鴻烈。誅文屈賈。有剛有折。


\end{pinyinscope}