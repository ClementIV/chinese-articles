\article{銚期王霸祭遵列傳}

\begin{pinyinscope}
銚期字次況,潁川郟人也。長八尺二寸,容貌絕異,矜嚴有威。父猛,為桂陽太守,卒,期服喪三年,鄉里稱之。光武略地潁川,聞期志義,召署賊曹掾,從徇薊。時王郎檄書到薊,薊中起兵應郎。光武趨駕出,百姓聚觀,諠呼滿道,遮路不得行,期騎馬奮戟,瞋目大呼左右曰「旧」,眾皆披靡。及至城門,門已閉,攻之得出。行至信都,以期為裨將,與傅寬、呂晏俱屬鄧禹。徇傍縣,又發房子兵。禹以期為能,獨拜偏將軍,授兵二千人,寬、晏各數百人。還言其狀,光武甚善之。使期別徇真定宋子,攻拔樂陽、槁、肥纍。

從擊王郎將兒宏、劉奉於鉅鹿下,期先登陷陳,手殺五十餘人,被創中額,攝幩復戰,遂大破之。王郎滅,拜期虎牙大將軍。乃因閒說光武曰:「河北之地,界接邊塞,人習兵戰,號為精勇。今更始失政,大統危殆,海內無所歸往。明公據河山之固,擁精銳之眾,以順萬人思漢之心,則天下誰敢不從?」光武笑曰:「卿欲遂前䟆邪?」時銅馬數十萬眾入清陽、博平,期與諸將迎擊之,連戰不利,期乃更背水而戰,所殺傷甚多。會光武救至,遂大破之,追至館陶,皆降之。從擊青犢、赤眉於射犬,賊襲期輜重,期還擊之,手殺傷數十人,身被三創,而戰方力,遂破走之。

光武即位,封安成侯,食邑五千戶。時檀鄉、五樓賊入繁陽、內黃,又魏郡大姓數反覆,而更始將卓京謀欲相率反鄴城。帝以期為魏郡太守,行大將軍事。期發郡兵擊卓京,破之,斬首六百餘級。京亡入山,追斬其將校數十人,獲京妻子。進擊繁陽、內黃,復斬數百級,郡界清平。督盜賊李熊,鄴中之豪,而熊弟陸謀欲反城迎檀鄉。或以告期,期不應,告者三四,期乃召問熊。熊叩頭首服,願與老母俱就死。期曰:「為吏儻不若為賊樂者,可歸與老母往就陸也。」使吏送出城。熊行求得陸,將詣鄴城西門。陸不勝愧感,自殺以謝期。期嗟歎,以禮葬之,而還熊故職。於是郡中服其威信。

建武五年,行幸魏郡,以期為太中大夫。從還洛陽,又拜衛尉。

期重於信義,自為將,有所降下,未嘗虜掠。及在朝廷,憂國愛主,其有不得於心,必犯顏諫諍。帝嘗輕與期門近出,期頓首車前曰:「臣聞古今之戒,變生不意,誠不願陛下微行數出。」帝為之回輿而還。十年卒,帝親臨襚斂,贈以衛尉、安成侯印綬,謚曰忠侯。

子丹嗣。復封丹弟統為建平侯。後徙封丹葛陵侯。丹卒,子舒嗣。舒卒,子羽嗣。羽卒,子蔡嗣。

王霸字元伯,潁川潁陽人也。世好文法,父為郡決曹掾,霸亦少為獄吏。常慷慨不樂吏職,其父奇之。遣西學長安。漢兵起,光武過潁陽,霸率賓客上謁,曰:「將軍興義兵,竊不自知量,貪慕威德,願充行伍。」光武曰:「夢想賢士,共成功業,豈有二哉!」遂從擊破王尋、王邑於昆陽,還休鄉里。

及光武為司隸校尉,道過潁陽,霸請其父,願從。父曰:「吾老矣,不任軍旅,汝往,勉之!」霸從至洛陽。及光武為大司馬,以霸為功曹令史,從度河北。賓客從霸者數十人,稍稍引去。光武謂霸曰:「潁川從我者皆逝,而子獨留。努力!疾風知勁草。」

及王郎起,光武在薊,郎移檄購光武。光武令霸至市中募人,將以擊郎。市人皆大笑,舉手邪揄之,霸慚懅而還。光武即南馳至下曲陽。傳聞王郎兵在後,從者皆恐。及至虖沱河,候吏還白河水流澌,無船,不可濟。官屬大懼。光武令霸往視之。霸恐驚眾,欲且前,阻水,還即詭曰:「冰堅可度。」官屬皆喜。光武笑曰:「候吏果妄語也。」遂前。比至河,河冰亦合,乃令霸護度,未畢數騎而冰解。光武謂霸曰:「安吾眾得濟免者,卿之力也。」霸謝曰:「此明公至德,神靈之祐,雖武王白魚之應,無以加此。」武光謂官屬曰:「王霸權以濟事,殆天瑞也。」以為軍正,爵關內侯。既至信都,發兵攻拔邯鄲。霸追斬王郎,得其璽綬。封王鄉侯。

從平河北,常與臧宮、傅俊共營,霸獨善撫士卒,死者脫衣以斂之,傷者躬親以養之。光武即位,以霸曉兵愛士,可獨任,拜為偏將軍,并將臧宮、傅俊兵,而以宮、俊為騎都尉。建武二年,更封富波侯。

四年秋,帝幸譙,使霸與捕虜將軍馬武東討周建於垂惠。蘇茂將五校兵四千餘人救建,而先遣精騎遮擊馬武軍糧,武往救之。建從城中出兵夾擊武,武恃霸之援,戰不甚力,為茂、建所敗。武軍奔過霸營,大呼求救。霸曰:「賊兵盛,出必兩敗,努力而已。」乃閉營堅壁。軍吏皆爭之。霸曰:「茂兵精銳,其眾又多,吾吏士心恐,而捕虜與吾相恃,兩軍不一,此敗道也。今閉營固守,示不相援,賊必乘勝輕進;捕虜無救,其戰自倍。如此,茂眾疲勞,吾承其弊,乃可剋也。」茂、建果悉出攻武。合戰良久,霸軍中壯士路潤等數十人斷髮請戰。霸知士心銳,乃開營後,出精騎襲其背。茂、建前後受敵,驚亂敗走,霸、武各歸營。賊復聚眾挑戰,霸堅臥不出,方饗士作倡樂。茂雨射營中,中霸前酒樽,霸安坐不動。軍吏皆曰:「茂前日已破;今易擊也。」霸曰:「不然。蘇茂客兵遠來,糧食不足,故數挑戰,以儌一切之勝。今閉營休士,所謂不戰而屈人之兵,善之善者也。」茂、建既不得戰,乃引還營。其夜,建兄子誦反,閉城拒之,茂、建遁去,誦以城降。

五年春,帝使太中大夫持節拜霸為討虜將軍。六年,屯田新安。八年,屯函谷關。擊滎陽、中牟盜賊,皆平之。

九年,霸與吳漢及橫野大將軍王常、建義大將軍朱祐、破姦將軍侯進等五萬餘人,擊盧芳將賈覽、閔堪於高柳。匈奴遣騎助芳,漢車遇雨,戰不利。吳漢還洛陽,令朱祐屯常山,王常屯涿郡,侯進屯漁陽。璽書拜霸上谷太守,領屯兵如故,捕擊胡虜,無拘郡界。明年,霸復與吳漢等四將軍六萬人出高柳擊賈覽,詔霸與漁陽太守陳訢將兵為諸軍鋒。匈奴左南將軍將數千騎救覽,霸等連戰於平城下,破之,追出塞,斬首數百級。霸及諸將還入鴈門,與驃騎大將軍杜茂會攻盧芳將尹由於崞、繁畤,不剋。

十三年,增邑戶,更封向侯。是時,盧芳與匈奴、烏桓連兵,寇盜尤數,緣邊愁苦。詔霸將弛刑徒六千餘人,與杜茂治飛狐道,堆石布土,築起亭障,自代至平城三百餘里。凡與匈奴、烏桓大小數十百戰,頗識邊事,數上書言宜與匈奴結和親,又陳委輸可從溫水漕,以省陸轉輸之勞,事皆施行。後南單于、烏桓降服,北邊無事。霸在上谷二十餘歲。三十年,定封淮陵侯。永平二年,以病免,後數月卒。

子符嗣,徙封軑侯。符卒,子度嗣。度尚顯宗女浚儀長公主,為黃門郎。度卒,子歆嗣。

祭遵字弟孫,潁川潁陽人也。少好經書。家富給,而遵恭儉,惡衣服。喪母,負土起墳。嘗為部吏所侵,結客殺之。初,縣中以其柔也,既而皆憚焉。

及光武破王尋等,還過潁陽,遵以縣吏數進見,光武愛其容儀,署為門下史。從征河北,為軍市令。舍中兒犯法,遵格殺之。光武怒,命收遵。時主簿陳副諫曰:「明公常欲眾軍整齊,今遵奉法不避,是教令所行也。」光武乃貰之,以為刺姦將軍。謂諸將曰:「當備祭遵!吾舍中兒犯法尚殺之,必不私諸卿也。」尋拜為偏將軍,從平河北,以功封列侯。

建武二年春,拜征虜將軍,定封潁陽侯。與驃騎大將軍景丹、建義大將軍朱祐、漢忠將軍王常、騎都尉王梁、臧宮等入箕關,南擊弘農、厭新、柏華蠻中賊。弩中遵口,洞出流血,眾見遵傷,稍引退,遵呼叱止之,士卒戰皆自倍,遂大破之。時新城蠻中山賊張滿,屯結險隘為人害,詔遵攻之。遵絕其糧道,滿數挑戰,遵堅壁不出。而厭新、柏華餘賊復與滿合,遂攻得霍陽聚,遵乃分兵擊破降之。明年春,張滿飢困,城拔,生獲之。初,滿祭祀天地,自云當王,既執,歎曰:「讖文誤我!」乃斬之,夷其妻子。遵引兵南擊鄧奉弟終於杜衍,破之。

時涿郡太守張豐執使者舉兵反,自稱無上大將軍,與彭寵連兵。四年,遵與朱祐及建威大將軍耿弇、驍騎將軍劉喜俱擊之。遵兵先至,急攻豐,豐功曹孟锾執豐降。初,豐好方術,有道士言豐當為天子,以五綵囊裹石繫豐肘,云石中有玉璽。豐信之,遂反。既執當斬,猶曰:「肘石有玉璽。」遵為椎破之,豐乃知被詐,仰天歎曰:「當死無所恨!」諸將皆引還,遵受詔留屯良鄉拒彭寵。因遣護軍傅玄襲擊寵將李豪於潞,大破之,斬首千餘級。相拒歲餘,數挫其鋒,黨與多降者。及寵死,遵進定其地。

六年春,詔遵與建威大將軍耿弇、虎牙大將軍蓋延、漢忠將軍王常、捕虜將軍馬武、驍騎將軍劉歆、武威將軍劉尚等從天水伐公孫述。師次長安,時車駕亦至,而隗囂不欲漢兵上隴,辭說解故。帝召諸將議。皆曰:「可且延囂日月之期,益封其將帥,以消散之。」遵曰:「囂挾姦久矣。今若按甲引時,則使其詐謀益深,而蜀警增備,固不如遂進。」帝從之,乃遣遵為前行。隗囂使其將王元拒隴坻,遵進擊,破之,追至新關。及諸將到,與囂戰,並敗,引退下隴。乃詔遵軍汧,耿弇軍漆,征西大將軍馮異軍栒邑,大司馬吳漢等還屯長安。自是後遵數挫隗囂。事已見馮異傳。

八年秋,復從車駕上隴。及囂破,帝東歸過汧,幸遵營,勞饗士卒,作黃門武樂,良夜乃罷。時遵有疾,詔賜重茵,覆以御蓋。復令進屯隴下。及公孫述遣兵救囂,吳漢、耿弇等悉奔還,遵獨留不卻。九年春,卒於軍。

遵為人廉約小心,克己奉公,賞賜輒盡與士卒,家無私財,身衣韋恊,布被,夫人裳不加緣,帝以是重焉。及卒,愍悼之尤甚。遵喪至河南縣,詔遣百官先會喪所,車駕素服臨之,望哭哀慟。還幸城門,過其車騎,涕泣不能已。喪禮成,復親祠以太牢,如宣帝臨霍光故事。詔大長秋、謁者、河南尹護喪事,大司農給費。博士范升上疏,追稱遵曰:「臣聞先王崇政,尊美屏惡。昔高祖大聖,深見遠慮,班爵割地,與下分功,著錄勳臣,頌其德美。生則寵以殊禮,奏事不名,入門不趨。死則疇其爵邑,世無絕嗣,丹書鐵券,傳於無窮。斯誠大漢厚下安人長久之德,所以累世十餘,歷載數百,廢而復興,絕而復續者也。陛下以至德受命,先明漢道,褒序輔佐,封賞功臣,同符祖宗。征虜將軍潁陽侯遵,不幸早薨。陛下仁恩,為之感傷,遠迎河南,惻怛之慟,形於聖躬,喪事用度,仰給縣官,重賜妻子,不可勝數。送死有以加生,厚亡有以過存,矯俗厲化,卓如日月。古者臣疾君視,臣卒君弔,德之厚者也。陵遲已來久矣。及至陛下,復興斯禮,群下感動,莫不自勵。臣竊見遵修行積善,竭忠於國,北平漁陽,西拒隴、蜀,先登坻上,深取略陽。眾兵既退,獨守衝難。制御士心,不越法度。所在吏人,不知有軍。清名聞於海內,廉白著於當世。所得賞賜,輒盡與吏士,身無奇衣,家無私財。同產兄午以遵無子,娶妾送之,遵乃使人逆而不受,自以身任於國,不敢圖生慮繼嗣之計。臨死遺誡牛車載喪,薄葬洛陽。問以家事,終無所言。任重道遠,死而後已。遵為將軍,取士皆用儒術,對酒設樂,必雅歌投壺。又建為孔子立後,奏置五經大夫。雖在軍旅,不忘俎豆,可為好禮悅樂,守死善道者也。禮,生有爵,死有謚,爵以殊尊卑,謚以明善惡。臣愚以為宜因遵薨,論敘眾功,詳案謚法,以禮成之。顯章國家篤古之制,為後嗣法。」帝乃下升章以示公卿。至葬,車駕復臨,贈以將軍、侯印綬,朱輪容車,介士軍陳送葬,謚曰成侯。既葬,車駕復臨其墳,存見夫人室家。其後會朝,帝每歎曰:「安得憂國奉公之臣如祭征虜者乎!」遵之見思若此。

無子,國除。兄午,官至酒泉太守。從弟肜。

肜字次孫,早孤,以至孝見稱。遇天下亂,野無煙火,而獨在冢側。每賊過,見其尚幼而有志節,皆奇而哀之。

光武初以遵故,拜肜為黃門侍郎,常在左右。及遵卒無子,帝追傷之,以肜為偃師長,令近遵墳墓,四時奉祠之。肜有權略,視事五歲,縣無盜賊,課為第一,遷襄賁令。時天下郡國尚未悉平,襄賁盜賊白日公行。肜至,誅破姦猾,殄其支黨,數年,襄賁政清。璽書勉勵,增秩一等,賜縑百匹。

當是時,匈奴、鮮卑及赤山烏桓連和彊盛,數入塞殺略吏人。朝廷以為憂,益增緣邊兵,郡有數千人,又遣諸將分屯障塞。帝以肜為能,建武十七年,拜遼東太守。至則勵兵馬,廣斥候。肜有勇力,能貫三百斤弓。虜每犯塞,常為士卒鋒,數破走之。二十一年秋,鮮卑萬餘騎寇遼東,肜率數千人迎擊之,自被甲陷陳,虜大奔,投水死者過半,遂窮追出塞,虜急,皆棄兵祼身散走,斬首三千餘級,獲馬數千匹。自是後鮮卑震怖,畏肜不敢復闚塞。肜以三虜連和,卒為邊害,二十五年,乃使招呼鮮卑,示以財利。其大都護偏何遣使奉獻,願得歸化,肜慰納賞賜,稍復親附。其異種滿離、高句驪之屬,遂駱驛款塞,上貂裘好馬,帝輒倍其賞賜。其後偏何邑落諸豪並歸義,願自效。肜曰:「審欲立功,當歸擊匈奴,斬送頭首乃信耳。」偏何等皆仰天指心曰:「必自效!」即擊匈奴左伊袟訾部,斬首二千餘級,持頭詣郡。其後歲歲相攻,輒送首級受賞賜。自是匈奴衰弱,邊無寇警,鮮卑、烏桓並入朝貢。

肜為人質厚重毅,體貌絕眾。撫夷狄以恩信,皆畏而愛之,故得其死力。初,赤山烏桓數犯上谷,為邊害,詔書設購賞,功責州郡,不能禁。肜乃率勵偏何,遣往討之。永平元年,偏何擊破赤山,斬其魁帥,持首詣肜,塞外震讋。肜之威聲,暢於北方,西自武威,東盡玄菟及樂浪,胡夷皆來內附,野無風塵。乃悉罷緣邊屯兵。

十二年,徵為太僕。肜在遼東幾三十年,衣無兼副。顯宗既嘉其功,又美肜清約,拜日,賜錢百萬,馬三匹,衣被刀劍下至居室什物,大小無不悉備。帝每見肜,常歎息以為可屬以重任。後從東巡狩,過魯,坐孔子講堂,顧指子路室謂左右曰:「此太僕之室。太僕,吾之禦侮也。」

十六年,使肜以太僕將萬餘騎與南單于左賢王信伐北匈奴,期至涿邪山。信初有嫌於肜,行出高闕塞九百餘里,得小山,乃妄言以為涿邪山。肜到不見虜而還,坐逗留畏懦下獄免。肜性沈毅內重,自恨見詐無功,出獄數日,歐血死。臨終謂其子曰:「吾蒙國厚恩,奉使不稱,微績不立,身死誠慚恨。義不可以無功受賞,死後,若悉簿上所得賜物,身自詣兵屯,效死前行,以副吾心。」既卒,其子逢上疏具陳遺言。帝雅重肜,方更任用,聞之大驚,召問逢疾狀,嗟歎者良久焉。烏桓、鮮卑追思肜無已,每朝賀京師,常過冢拜謁,仰天號泣乃去。遼東吏人為立祠,四時奉祭焉。

肜既葬,子參遂詣奉車都尉竇固,從軍擊車師有功,稍遷遼東太守。永元中,鮮卑入郡界,參坐沮敗,下獄死。肜子孫多為邊吏者,皆有名稱。

論曰:祭肜武節剛方,動用安重,雖條侯、穰苴之倫,不能過也。且臨守偏海,政移獷俗,徼人請符以立信,胡貊數級於郊下,至乃臥鼓邊亭,滅烽幽障者將三十年。古所謂「必世而後仁」,豈不然哉!而一眚之故,以致感憤,惜哉,畏法之敝也!

贊曰:期啟燕門,霸冰虖河。祭遵好禮,臨戎雅歌。肜抗遼左,邊廷懷和。


\end{pinyinscope}