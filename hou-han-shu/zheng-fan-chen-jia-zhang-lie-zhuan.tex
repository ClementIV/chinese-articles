\article{鄭范陳賈張列傳}

\begin{pinyinscope}
鄭興字少贛,河南開封人也。少學公羊春秋。晚善左氏傳,遂積精深思,通達其旨,同學者皆師之。天鳳中,將門人從劉歆講正大義,歆美興才,使撰條例、章句、傳詁,及校三統歷。

更始立,以司直李松行丞相事,先入長安,松以興為長史,令還奉迎遷都。更始諸將皆山東人,咸勸留洛陽。興說更始曰:「陛下起自荊楚,權政未施,一朝建號,而山西雄桀爭誅王莽,開關郊迎者,何也?此天下同苦王氏虐政,而思高祖之舊德也。今久不撫之,臣恐百姓離心,盜賊復起矣。春秋書『齊小白入齊』,不稱侯,未朝廟故也。今議者欲先定赤眉而後入關,是不識其本而爭其末,恐國家之守轉在函谷,雖臥洛陽,庸得安枕乎?」更始曰:「朕西決矣。」拜興為諫議大夫,使安集關西及朔方、涼、益三州,還拜涼州刺史。會天水有反者,攻殺郡守,興坐免。

時赤眉入關,東道不通,興乃西歸隗囂,虛心禮請,而興恥為之屈,稱疾不起。囂矜己自飾,常以為西伯復作,乃與諸將議自立為王。興聞而說囂曰:「春秋傳云:『口不道忠信之言為嚚,耳不聽五聲之和為聾。』閒者諸將集會,無乃不道忠信之言;大將軍之聽,無乃阿而不察乎?昔文王承積德之緒,加之以睿聖,三分天下,尚服事殷。及武王即位,八百諸侯不謀同會,皆曰『紂可伐矣』,武王以未知天命,還兵待時。高祖征伐累年,猶以沛公行師。今令德雖明,世無宗周之祚,威略雖振,未有高祖之功,而欲舉未可之事,昭速禍患,無乃不可乎?惟將軍察之。」囂竟不稱王。後遂廣置職位,以自尊高。興復止囂曰:「夫中郎將、太中大夫、使持節官皆王者之器,非人臣所當制也。孔子曰:『唯器與名,不可以假人。』不可以假人者,亦不可以假於人也。無益於實,有損於名,非尊上之意也。」囂病之而止。

及囂遣子恂入侍,將行,興因恂求歸葬父母,囂不聽而徙興舍,益其秩禮。興入見囂曰:「前遭赤眉之亂,以將軍僚舊,故敢歸身明德。幸蒙覆載之恩,復得全其性命。興聞事親之道,生事之以禮,死葬之以禮,祭之以禮,奉以周旋,弗敢失墜。今為父母未葬,請乞骸骨,若以增秩徙舍,中更停留,是以親為餌,無禮甚矣。將軍焉用之!」囂曰:「囂將不足留故邪?」興曰:「將軍據七郡之地,擁羌胡之眾,以戴本朝,德莫厚焉,威莫重焉。居則為專命之使,入必為鼎足之臣。興,從俗者也,不敢深居屏處,因將軍求進,不患不達,因將軍求入,何患不親,此興之計不逆將軍者也。興業為父母請,不可以已,願留妻子獨歸葬,將軍又何猜焉?」囂曰:「幸甚。」促為辨裝,遂令與妻子俱東。時建武六年也。

侍御史杜林先與興同寓隴右,乃薦之曰:「竊見河南鄭興,執義堅固,敦悅詩書,好古博物,見疑不惑,有公孫僑、觀射父之德,宜侍帷幄,典職機密。昔張仲在周,燕翼宣王,而詩人悅喜。惟陛下留聽少察,以助萬分。」乃徵為太中大夫。

明年三月晦,日食。興因上疏曰:

春秋以天反時為災,地反物為妖,人反德為亂,亂則妖災生。往年以來,謫咎連見,意者執事頗有闕焉。案春秋『昭公十七年夏六月甲戌朔,日有食之』。傳曰:『日過分而未至,三辰有災,於是百官降物,君不舉,避移時,樂奏鼓,祝用幣,史用辭。』今孟夏,純乾用事,陰氣未作,其災尤重。夫國無善政,則謫見日月,變咎之來,不可不慎,其要在因人之心,擇人處位也。堯知鯀不可用而用之者,是屈己之明,因人之心也。齊桓反政而相管仲,晉文歸國而任郤縠者,是不私其私,擇人處位也。今公卿大夫多舉漁陽太守郭伋可大司空者,而不以時定,道路流言,咸曰「朝廷欲用功臣」,功臣用則人位謬矣。願陛下上師唐、虞,下覽齊、晉,以成屈己從眾之德,以濟群臣讓善之功。

夫日月交會,數應在朔,而頃年日食,每多在晦。先時而合,皆月行疾也。日君象而月臣象,君亢急則臣下促迫,故行疾也。今年正月繁霜,自爾以來,率多寒日,此亦急咎之罰。天於賢聖之君,猶慈父之於孝子也,丁寧申戒,欲其反政,故災變仍見,此乃國之福也。今陛下高明而群臣惶促,宜留思柔剋之政,垂意洪範之法,博採廣謀,納群下之策。

書奏,多有所納。

帝嘗問興郊祀事,曰:「吾欲以讖斷之,何如?」興對曰:「臣不為讖。」帝怒曰:「卿之不為讖,非之邪?」興惶恐曰:「臣於書有所未學,而無所非也。」帝意乃解。興數言政事,依經守義,文章溫雅,然以不善讖故不能任。

九年,使監征南、積弩營於津鄉,會征南將軍岑彭為刺客所殺,興領其營,遂與大司馬吳漢俱擊公孫述。述死,詔興留屯成都。頃之,侍御史舉奏興奉使私買奴婢,坐左轉蓮勺令。是時喪亂之餘,郡縣殘荒,興方欲築城郭,修禮教以化之,會以事免。

興好古學,尤明左氏、周官,長於歷數,自杜林、桓譚、衛宏之屬,莫不斟酌焉。世言左氏者多祖於興,而賈逵自傳其父業,故有鄭、賈之學。興去蓮勺,後遂不復仕,客授閿鄉,三公連辟不肯應,卒于家。子眾。

眾字仲師。年十二,從父受左氏春秋,精力於學,明三統歷,作春秋難記條例,兼通易、詩,知名於世。建武中,皇太子及山陽王荊,因虎賁中郎將梁松以縑帛聘請眾,欲為通義,引籍出入殿中。眾謂松曰:「太子儲君,無外交之義,漢有舊防,蕃王不宜私通賓客。」遂辭不受。松復風眾以「長者意,不可逆」。眾曰:「犯禁觸罪,不如守正而死。」太子及荊聞而奇之,亦不強也。及梁氏事敗,賓客多坐之,唯眾不染於辭。

永平初,辟司空府,以明經給事中,再遷越騎司馬,復留給事中。是時北匈奴遣使求和親。八年,顯宗遣眾持節使匈奴。眾至北庭,虜欲令拜,眾不為屈。單于大怒,圍守閉之,不與水火,欲脅服眾。眾拔刀自誓,單于恐而止,乃更發使隨眾還京師。朝議復欲遣使報之,眾上疏諫曰:「臣伏聞北單于所以要致漢使者,欲以離南單于之眾,堅三十六國之心也。又當揚漢和親,誇示鄰敵,令西城欲歸化者局促狐疑,懷土之人絕望中國耳。漢使既到,便偃蹇自信。若復遣之,虜必自謂得謀,其群臣駮議者不敢復言。如是,南庭動搖,烏桓有離心矣。南單于久居漢地,具知形埶,萬分離析,旋為邊害。今幸有度遼之眾揚威北垂,雖勿報荅,不敢為患。」帝不從,復遣眾。眾因上言:「臣前奉使不為匈奴拜,單于恚恨,故遣兵圍臣。今復銜命,必見陵折。臣誠不忍持大漢節對氈裘獨拜。如令匈奴遂能服臣,將有損大漢之強。」帝不聽,眾不得已,既行,在路連上書固爭之。詔切責眾,追還繫廷尉,會赦歸家。

其後帝見匈奴來者,問眾與單于爭禮之狀,皆言匈奴中傳眾意氣壯勇,雖蘇武不過。乃復召眾為軍司馬,使與虎賁中郎將馬廖擊車師。至敦煌,拜為中郎將,使護西域。會匈奴脅車師,圍戊己校尉,眾發兵救之。遷武威太守,謹修邊備,虜不敢犯。遷左馮翊,政有名跡。

建初六年,代鄧彪為大司農。是時肅宗議復鹽鐵官,眾諫以為不可。詔數切責,至被奏劾,眾執之不移。帝不從。在位以清正稱。其後受詔作春秋刪十九篇。八年,卒官。

子安世,亦傳家業,為長樂、未央廄令。延光中,安帝廢太子為濟陰王,安世與太常桓焉、太僕來歷等共正議諫爭。及順帝立,安世已卒,追賜錢帛,除子亮為郎。眾曾孫公業,自有傳。

范升字辯卿。代郡人也。少孤,依外家居。九歲通論語、孝經,及長,習梁丘易、老子,教授後生。

王莽大司空王邑辟升為議曹史。時莽頻發兵役,徵賦繁興,升乃奏記邑曰:「升聞子以人不閒於其父母為孝,臣以下不非其君上為忠。今眾人咸稱朝聖,皆曰公明。蓋明者無不見,聖者無不聞。今天下之事,昭昭於日月,震震於雷霆,而朝云不見,公云不聞,則元元焉所呼天?公以為是而不言,則過小矣;知而從令,則過大矣。二者於公無可以免,宜乎天下歸怨於公矣。朝以遠者不服為至念,升以近者不悅為重憂。今動與時戾,事與道反,馳騖覆車之轍,探湯敗事之後,後出益可怪,晚發愈可懼耳。方春歲首,而動發遠役,藜藿不充,田荒不耕,穀價騰躍,斛至數千,吏人陷於湯火之中,非國家之人也。如此,則胡、貊守關,青、徐之寇在於帷帳矣。升有一言,可以解天下倒縣,免元元之急,不可書傳,願蒙引見,極陳所懷。」邑雖然其言,而竟不用。升稱病乞身,邑不聽,令乘傳使上黨。升遂與漢兵會,因留不還。

建武二年,光武徵詣懷宮,拜議郎,遷博士,上疏讓曰:「臣與博士梁恭、山陽太守呂羌俱修梁丘易。二臣年並耆艾,經學深明,而臣不以時退,與恭並立,深知羌學,又不能達,慚負二老,無顏於世。誦而不行,知而不言,不可開口以為人師,願推博士以避恭、羌。」帝不許,然由是重之,數詔引見,每有大議,輒見訪問。

時尚書令韓歆上疏,欲為費氏易、左氏春秋立博士,詔下其議。四年正月,朝公卿、大夫、博士,見於雲臺。帝曰:「范博士可前平說。」升起對曰:「左氏不祖孔子,而出於丘明,師徒相傳,又無其人,且非先帝所存,無因得立。」遂與韓歆及太中大夫許淑等互相辯難,日中乃罷。升退而奏曰:「臣聞主不稽古,無以承天;臣不述舊,無以奉君。陛下愍學微缺,勞心經蓺,情存博聞,故異端競進。近有司請置京氏易博士,群下執事,莫能據正。京氏既立,費氏怨望,左氏春秋復以比類,亦希置立。京、費已行,次復高氏,春秋之家,又有騶、夾。如令左氏、費氏得置博士,高氏、騶、夾,五經奇異,並復求立,各有所執,乖戾分爭,從之則失道,不從則失人,將恐陛下必有猒倦之聽。孔子曰:『博學約之,弗叛矣夫。』夫學而不約,必叛道也。顏淵曰:『博我以文,約我以禮。』孔子可謂知教,顏淵可謂善學矣。老子曰:『學道日損。』損猶約也。又曰:『絕學無憂。』絕末學也。今費、左二學,無有本師,而多反異,先帝前世,有疑於此,故京氏雖立,輒復見廢。疑道不可由,疑事不可行。詩書之作,其來已久。孔子尚周流遊觀,至于知命,自衛反魯,乃正雅、頌。今陛下草創天下,紀綱未定,雖設學官,無有弟子,詩書不講,禮樂不修,奏立左、費,非政急務。孔子日:『攻乎異端,斯害也已。』傳曰:『聞疑傳疑,聞信傳信,而堯舜之道存。』願陛下疑先帝之所疑,信先帝之所信,以示反本,明不專己。天下之事所以異者,以不一本也。《易》曰:『天下之動,貞夫一也。』又曰:『正其本,萬事理。』五經之本自孔子始,謹奏左氏之失凡十四事。」時難者以太史公多引左氏,升又上太史公違戾五經,謬孔子言,及左氏春秋不可錄三十一事。詔以下博士。

後升為出妻所告,坐繫,得出,還鄉里。永平中,為聊城令,坐事免,卒於家。

陳元字長孫,蒼梧廣信人也。父欽,習左氏春秋,事黎陽賈護,與劉歆同時而別自名家。王莽從欽受左氏學,以欽為猒難將軍。元少傳父業,為之訓詁,銳精覃思,至不與鄉里通。以父任為郎。

建武初,元與桓譚、杜林、鄭興俱為學者所宗。時議欲立左氏傳博士,范升奏以為左氏淺末,不宜立。元聞之,乃詣闕上疏曰:

陛下撥亂反正,文武並用,深愍經蓺謬雜,真偽錯亂,每臨朝日,輒延群臣講論聖道。知丘明至賢,親受孔子,而公羊、穀梁傳聞於後世,故詔立左氏,博詢可否,示不專己,盡之群下也。今論者沈溺所習,翫守舊聞,固執虛言傳受之辭,以非親見實事之道。左氏孤學少與,遂為異家之所覆冒。夫至音不合眾聽,故伯牙絕弦;至寶不同眾好,故卞和泣血。仲尼聖德,而不容於世,況於竹帛餘文,其為雷同者所排,固其宜也。非陛下至明,孰能察之!

臣元竊見博士范升等所議奏左氏春秋不可立

,及太史公違戾凡四十五事。案升為所言,前後相違,皆斷涞小文,媟黷微辭,以年數小差,掇為巨謬,遺脫纖微,指為大尤,抉瑕擿釁,掩其弘美,所謂「小辯破言,小言破道」者也。升等又曰:「先帝不以左氏為經,故不置博士,後主所宜因襲。」臣愚以為若先帝所行而後主必行者,則盤庚不當遷于殷,周公不當營洛邑,陛下不當都山東也。往者,孝武皇帝好公羊,衛太子好穀梁,有詔詔太子受公羊,不得受穀梁。孝宣皇帝在人閒時,聞衛太子好穀梁,於是獨學之。及即位,為石渠論而穀梁氏興,至今與公羊並存。此先帝後帝各有所立,不必其相因也。孔子曰,純,儉,吾從眾;至於拜下,則違之。夫明者獨見,不惑於朱紫,聽者獨聞,不謬於清濁,故離朱不為巧眩移目,師曠不為新聲易耳。方今干戈少弭,戎事略戰,留思聖蓺,眷顧儒雅,採孔子拜下之義,卒淵聖獨見之旨,分明白黑,建立左氏,解釋先聖之積結,洮汰學者之累惑,使基業垂於萬世,後進無復狐疑,則天下幸甚。

臣元愚鄙,嘗傳師言。如得以褐衣召見,俯伏庭下,誦孔氏之正道,理丘明之宿冤;若辭不合經,事不稽古,退就重誅,雖死之日,生之年也。

書奏,下其議,范升復與元相辯難,凡十餘上。帝卒立左氏學,太常選博士四人,元為第一。帝以元新忿爭,乃用其次司隸從事李封,於是諸儒以左氏之立,論議讙譁,自公卿以下,數廷爭之。會封病卒,左氏復廢。

元以才高著名,辟司空李通府。時大司農江馮上言,宜令司隸校尉督察三公。事下三府。元上疏曰:「臣聞師臣者帝,賓臣者霸。故武王以太公為師,齊桓以夷吾為仲父。孔子曰:『百官總己聽於冢宰。』近則高帝優相國之禮,太宗假宰輔之權。及亡新王莽,遭漢中衰,專操國柄,以偷天下,況己自喻,不信群臣。奪公輔之任,損宰相之威,以刺舉為明,徼訐為直。至乃陪僕告其君長,子弟變其父兄,罔密法峻,大臣無所措手足。然不能禁董忠之謀,身為世戮。故人君患在自驕,不患驕臣;失在自任,不在任人。是以文王有日昃之勞,周公執吐握之恭,不聞其崇刺舉,務督察也。方今四方尚擾,天下未一,百姓觀聽,咸張耳目。陛下宜修文武之聖典,襲祖宗之遺德,勞心下士,屈節待賢,誠不宜使有司察公輔之名。」帝從之,宣下其議。

李通罷,元後復辟司徒歐陽歙府,數陳當世便事、郊廟之禮,帝不能用。以病去,年老,卒於家。子堅卿,有文章。

賈逵字景伯,扶風平陵人也。九世祖誼,文帝時為梁王太傅。曾祖父光,為常山太守,宣帝時以吏二千石自洛陽徙焉。父徽,從劉歆受左氏春秋,兼習國語、周官,又受古文尚書於塗惲,學毛詩於謝曼卿,作左氏條例二十一篇。

逵悉傳父業,弱冠能誦左氏傳及五經本文,以大夏侯尚書教授,雖為古學,兼通五家穀梁之說。自為兒童,常在太學,不通人閒事。身長八尺二寸,諸儒為之語曰:「問事不休賈長頭。」性愷悌,多智思,俶儻有大節。尤明左氏傳、國語,為之解詁五十一篇,永平中,上疏獻之。顯宗重其書,寫藏祕館。

時有神雀集宮殿官府,冠羽有五采色,帝異之,以問臨邑侯劉復,復不能對,薦逵博物多識,帝乃召見逵,問之。對曰:「昔武王終父之業,鸑鷟在岐,宣帝威懷戎狄,神雀仍集,此胡降之徵也。」帝敕蘭臺給筆札,使作神雀頌,拜為郎,與班固並校祕書,應對左右。

肅宗立,降意儒術,特好古文尚書、左氏傳。建初元年,詔逵入講北宮白虎觀、南宮雲臺。帝善逵說,使發出左氏傳大義長於二傳者。逵於是具條奏之曰:

臣謹擿出左氏三十事尤著明者,斯皆君臣之正義,父子之紀綱。其餘同公羊者什有七八,或文簡小異,無害大體。至如祭仲、紀季、伍子胥、叔術之屬,左氏義深於君父,父羊多任於權變,其相殊絕,固以甚遠,而冤抑積久,莫肯分明。

臣以永平中上言左氏與圖讖合者,先帝不遺芻蕘,省納臣言,寫其傳詁,藏之祕書。建平中,侍中劉歆欲立左氏,不先暴論大義,而輕移太常,恃其義長,詆挫諸儒,諸儒內懷不服,相與排之。孝哀皇帝重逆眾心,故出歆為河內太守。從是攻擊左氏,遂為重讎。至光武皇帝,奮獨見之明,興立左氏、穀梁,會二家先師不曉圖讖,故令中道而廢。凡所以存先王之道者,要在安上理民也。今左氏崇君父,卑臣子,彊幹弱枝,勸善戒惡,至明至切,至直至順。且三代異物,損益隨時,故先帝博觀異家,各有所採。易有施、孟,復立梁丘,尚書歐陽,復有大小夏侯,今三傳之異亦猶是也。又五經家皆無以證圖讖明劉氏為堯後者,而左氏獨有明文。五經家皆言顓頊代黃帝,而堯不得為火德。左氏以為少昊代黃帝,即圖讖所謂帝宣也。如令堯不得為火,則漢不得為赤。其所發明,補益實多。

陛下通天然之明,建大聖之本,改元正歷,垂萬世則,是以麟鳳百數,嘉瑞雜遝。猶朝夕恪勤,遊情六蓺,研機綜微,靡不審覈。若復留意廢學,以廣聖見,庶幾無所遺失矣。

書奏,帝嘉之,賜布五百匹,衣一襲,令逵自選公羊嚴、顏諸生高才者二十人,教以左氏,與簡紙經傳各一通。

逵母常有疾,帝欲加賜,以校書例多,特以錢二十萬,使潁陽侯馬防與之。謂防曰:「賈逵母病,此子無人事於外,屢空則從孤竹之子於首陽山矣。」

逵數為帝言古文尚書與經傳爾雅詁訓相應,詔令撰歐陽、大小夏侯尚書古文同異。逵集為三卷,帝善之。復令撰齊、魯、韓詩與毛氏異同。并作周官解故。遷逵為衛士令。八年,乃詔諸儒各選高才生,受左氏、穀梁春秋、古文尚書、毛詩,由是四經遂行於世。皆拜逵所選弟子及門生為千乘王國郎,朝夕受業黃門署,學者皆欣欣羨慕焉。

和帝即位,永元三年,以逵為左中郎將。八年,復為侍中,領騎都尉。內備帷幄,兼領祕書近署,甚見信用。

逵薦東萊司馬均、陳國汝郁,帝即徵之,並蒙優禮。均字少賓,安貧好學,隱居教授,不應辟命。信誠行乎州里,鄉人有所計爭,輒令祝少賓,不直者終無敢言。位至侍中,以老病乞身,帝賜以大夫祿,歸鄉里。郁字叔異,性仁孝,及親歿,遂隱處山澤。後累遷為魯相,以德教化,百姓稱之,流人歸者八九千戶。

逵所著經傳義詁及論難百餘萬言,又作詩、頌、誄、書、連珠、酒令凡九篇,學者宗之,後世稱為通儒。然不修小節,當世以此頗譏焉,故不至大官。永元十三年卒,時年七十二。朝廷愍惜,除兩子為太子舍人。

論曰:鄭、賈之學,行乎數百年中,遂為諸儒宗,亦徒有以焉爾。桓譚以不善讖流亡,鄭興以遜辭僅免,賈逵能附會文致,最差貴顯。世主以此論學,悲矣哉!

張霸字伯饒,蜀郡成都人也。年數歲而知孝讓,雖出入飲食,自然合禮,鄉人號為「張曾子。」七歲通春秋,復欲進餘經,父母曰「汝小未能也」,霸曰「我饒為之」,故字曰「饒」焉。

後就長水校尉樊儵受嚴氏公羊春秋,遂博覽五經。諸生孫林、劉固、段著等慕之,各市宅其傍,以就學焉。

舉孝廉光祿主事,稍遷,永元中為會稽太守,表用郡人處士顧奉、公孫松等。奉後為潁川太守,松為司隸校尉,並有名稱。其餘有業行者,皆見擢用。郡中爭厲志節,習經者以千數,道路但聞誦聲。

初,霸以樊儵刪嚴氏春秋猶多繁辭,乃減定為二十萬言,更名張氏學。

霸始到越,賊未解,郡界不寧,乃移書開購,明用信賞,賊遂束手歸附,不煩士卒之力。童謠曰:「棄我戟,捐我矛,盜賊盡,吏皆休。」視事三年,謂掾史曰:「太守起自孤生,致位郡守。蓋日中則移,月滿則虧。老氏有言:『知足不辱。』」遂上病。

後徵,四遷為侍中。時皇后兄虎賁中郎將鄧騭,當朝貴盛,聞霸名行,欲與為交,霸逡巡不荅,眾人笑其不識時務。後當為五更,會疾卒,年七十。遺敕諸子曰:「昔延州使齊,子死嬴、博,因坎路側,遂以葬焉。今蜀道阻遠,不宜歸塋,可止此葬,足藏髮齒而已。務遵速朽,副我本心。人生一世,但當畏敬於人,若不善加己,直為受之。」諸子承命,葬於河南梁縣,因遂家焉。將作大匠翟酺等與諸儒門人追錄本行,謚曰憲文。中子楷。

楷字公超,通嚴氏春秋、古文尚書,門徒常百人。賓客慕之,自父黨夙儒,偕造門焉。車馬填街,徒從無所止,黃門及貴戚之家,皆起舍巷次,以候過客往來之利。楷疾其如此,輒徙避之。家貧無以為業,常乘驢車至縣賣藥,足給食者,輒還鄉里。司隸舉茂才,除長陵令,不至官。隱居弘農山中,學者隨之,所居成市,後華陰山南遂有公超市。五府連辟,舉賢良方正,不就。

漢安元年,順帝特下詔告河南尹曰:「故長陵令張楷行慕原憲,操擬夷、齊,輕貴樂賤,竄跡幽藪,高志確然,獨拔群俗。前比徵命,盤桓未至,將主者翫習於常,優賢不足,使其難進歟?郡時以禮發遣。」楷復告疾不到。

性好道術,能作五里霧。時關西人裴優亦能為三里霧,自以不如楷,從學之,楷避不肯見。桓帝即位,優遂行霧作賊,事覺被考,引楷言從學術,楷坐繫廷尉詔獄,積二年,恆諷誦經籍,作尚書注。後以事無驗,見原還家。建和三年,下詔安車備禮聘之,辭以篤疾不行。年七十,終於家。子陵。

陵字處沖,官至尚書。元嘉中,歲首朝賀,大將軍梁冀帶劍入省,陵呵叱令出,敕羽林、虎賁奪冀劍。冀跪謝,陵不應,即劾奏冀,請廷尉論罪,有詔以一歲俸贖,而百僚肅然。

初,冀弟不疑為河南尹,舉陵孝廉。不疑疾陵之奏冀,因謂曰:「昔舉君,適所以自罰也。」陵對曰:「明府不以陵不肖,誤見擢序,今申公憲,以報私恩。」不疑有愧色。陵弟玄。

玄字處虛,沈深有才略,以時亂不仕。司空張溫數以禮辟,不能致。中平二年,溫以車騎將軍出征涼州賊邊章等,將行,玄自田廬被褐帶索,要說溫曰:「天下寇賊雲起,豈不以黃門常侍無道故乎?聞中貴人公卿已下當出祖道於平樂觀,明公總天下威重,握六師之要,若於中坐酒酣,鳴金鼓,整行陣,召軍正執有罪者誅之,引兵還屯都亭,以次翦除中官,解天下之倒縣,報海內之怨毒,然後顯用隱逸忠正之士,則邊章之徒宛轉股掌之上矣。」溫聞大震,不能對,良久謂玄曰:「處虛,非不悅子之言,顧吾不能行,如何!」玄乃歎曰:「

事行則為福,不行則為賊。今與公長辭矣。」即仰藥欲飲之。溫前執其手曰:「子忠於我,我不能用,是吾罪也,子何為當然!且出口入耳之言,誰今知之!」玄遂去,隱居魯陽山中。及董卓秉政,聞之,辟以為掾,舉侍御史,不就。卓臨之以兵,不得已彊起,至輪氏,道病終。

贊曰:中世儒門,賈、鄭名學。眾馳一介,爭禮氈幄。升、元守經,義偏情較,霸貴知止,辭交戚里。公超善術,所舍成市。


\end{pinyinscope}