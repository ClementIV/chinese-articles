\article{西域傳}

\begin{pinyinscope}
武帝時,西域內屬,有三十六國。漢為置使者、校尉領護之。宣帝改曰都護。元帝又置戊己二校尉,屯田於車師前王庭。哀平閒,自相分割為五十五國。王莽篡位,貶易侯王,由是西域怨叛,與中國遂絕,並復役屬匈奴。匈奴斂稅重刻,諸國不堪命,建武中,皆遣使求內屬,願請都護。光武以天下初定,未遑外事,竟不許之。會匈奴衰弱,莎車王賢誅滅諸國,賢死之後,遂更相攻伐。小宛、精絕、戎廬、且末為鄯善所并。渠勒、皮山為于窴所統,悉有其地。郁立、單桓、孤胡、烏貪訾離為車師所滅。後其國並復立。永平中,北虜乃脅諸國共寇河西郡縣,城門晝閉。十六年,明帝乃命將帥,北征匈奴,取伊吾盧地,置宜禾都尉以屯田,遂通西域,于窴諸國皆遣子入侍。西域自絕六十五載,乃復通焉。明年,始置都護、戊己校尉。及明帝崩,焉耆、龜茲攻沒都護陳睦,悉覆其眾,匈奴、車師圍戊己校尉。建初元年春,酒泉太守段彭大破車師於交河城。章帝不欲疲敝中國以事夷狄,乃迎還戊己校尉,不復遣都護。二年,復罷屯田伊吾,匈奴因遣兵守伊吾地。時軍司馬班超留于窴,綏集諸國。和帝永元元年,大將軍竇憲大破匈奴。二年,憲因遣副校尉閻槃將二千餘騎掩擊伊吾,破之。三年,班超遂定西域,因以超為都護,居龜茲。復置戊己校尉,領兵五百人,居車師前部高昌壁,又置戊部候,居車師後部候城,相去五百里。六年,班超復擊破焉耆,於是五十餘國悉納質內屬。其條支、安息諸國至于海瀕四萬里外,皆重譯貢獻。九年,班超遣掾甘英窮臨西海而還。皆前世所不至,山經所未詳,莫不備其風土,傳其珍怪焉。於是遠國蒙奇、兜勒皆來歸服,遣使貢獻。

及孝和晏駕,西域背畔。安帝永初元年,頻攻圍都護任尚、段禧等,朝廷以其險遠,難相應赴,詔罷都護。自此遂棄西域。北匈奴即復收屬諸國,共為邊寇十餘歲。敦煌太守曹宗患其暴害,元初六年,乃上遣行長史索班,將千餘人屯伊吾以招撫之,於是車師前王及鄯善王來降。數月,北匈奴復率車師後部王共攻沒班等,遂擊走其前王。鄯善逼急,求救於曹宗,宗因此請出兵擊匈奴,報索班之恥,復欲進取西域。鄧太后不許,但令置護西域副校尉,居敦煌,復部營兵三百人,羈縻而已。其後北虜連與車師入寇河西,朝廷不能禁,議者因欲閉玉門、陽關,以絕其患。

延光二年,敦煌太守張璫上書陳三策,以為「北虜呼衍王常展轉蒲類、秦海之閒,專制西域,共為寇鈔。今以酒泉屬國吏士二千餘人集昆侖塞,先擊呼衍王,絕其根本,因發鄯善兵五千人脅車師後部,此上計也。若不能出兵,可置軍司馬,將士五百人,四郡供其犁牛、穀食,出據柳中,此中計也。如又不能,則宜棄交河城,收鄯善等悉使入塞,此下計也」。朝廷下其議。尚書陳忠上疏曰:「臣聞八蠻之寇,莫甚北虜。漢興,高祖窘平城之圍,太宗屈供奉之恥。故孝武憤怒,深惟久長之計,命遣虎臣,浮河絕漠,窮破虜庭。當斯之役,黔首隕於狼望之北,財幣縻於盧山之壑,府庫單竭,杼柚空虛,筭至舟車,貲及六畜。夫豈不懷,慮久故也。遂開河西四郡,以隔絕南羌,收三十六國,斷匈奴右臂。是以單于孤特,鼠竄遠藏。至於宣、元之世,遂備蕃臣,關徼不閉,羽檄不行。由此察之,戎狄可以威服,難以化狎。西域內附日久,區區東望扣關者數矣,此其不樂匈奴慕漢之效也。今北虜已破車師,埶必南攻鄯善,棄而不救,則諸國從矣。若然,則虜財賄益增,膽埶益殖,威臨南羌,與之交連。如此,河西四郡危矣。河西既危,不得不救,則百倍之役興,不訾之費發矣。議者但念西域絕遠,卹之煩費,不見先世苦心勤勞之意也。方今邊境守禦之具不精,內郡武衛之備不脩,敦煌孤危,遠來告急,復不輔助,內無以慰勞吏民,外無以威示百蠻。蹙國減土,經有明誡。臣以為敦煌宜置校尉,案舊增四郡屯兵,以西撫諸國。庶足折衝萬里,震怖匈奴。」帝納之,乃以班勇為西域長史,將弛刑士五百人,西屯柳中。勇遂破平車師。自建武至于延光,西域三絕三通。順帝永建二年,勇復擊降焉耆。於是龜茲、疏勒、于窴、莎車等十七國皆來服從,而烏孫、蔥領已西遂絕。六年,帝以伊吾舊膏腴之地,傍近西域,匈奴資之,以為鈔暴,復令開設屯田如永元時事,置伊吾司馬一人。自陽嘉以後,朝威稍損,諸國驕放,轉相陵伐。元嘉二年,長史王敬為于窴所沒。永興元年,車師後王復反攻屯營。雖有降首,曾莫懲革,自此浸以疏慢矣。班固記諸國風土人俗,皆已詳備前書。今撰建武以後其事異於先者,以為西域傳,皆安帝末班勇所記云。

西域內屬諸國,東西六千餘里,南北千餘里,東極玉門、陽關,西至蔥領。其東北與匈奴、烏孫相接。南北有大山,中央有河。其南山東出金城,與漢南山屬焉。其河有兩源,一出蔥領東流,一出于窴南山下北流,與蔥領河合,東注蒲昌海。蒲昌海一名鹽澤,去玉門三百餘里。

自敦煌西出玉門、陽關,涉鄯善,北通伊吾千餘里,自伊吾北通車師前部高昌壁千二百里,自高昌壁北通後部金滿城五百里。此其西域之門戶也,故戊己校尉更互屯焉。伊吾地宜五穀、桑麻、蒲萄。其北又有柳中,皆膏腴之地。故漢常與匈奴爭車師、伊吾,以制西域焉。

自鄯善踰蔥領出西諸國,有兩道。傍南山北,陂河西行至莎車,為南道。南道西踰蔥領,則出大月氏、安息之國也。自車師前王庭隨北山,陂河西行至疏勒,為北道。北道西踰蔥領,出大宛、康居、奄蔡焉耆。

出玉門,經鄯善、且末、精絕三千餘里至拘彌。

拘彌國居寧彌城,去長史所居柳中四千九百里,去洛陽萬二千八百里。領戶二千一百七十三,口七千二百五十一,勝兵千七百六十人。

順帝永建四年,于窴王放前殺拘彌王興,自立其子為拘彌王,而遣使者貢獻於漢。敦煌太守徐由上求討之,帝赦于窴罪,令歸拘彌國,放前不肯。陽嘉元年,徐由遣疏勒王臣槃發二萬人擊于窴,破之,斬首數百級,放兵大掠,更立興宗人成國為拘彌王而還。至靈帝熹平四年,于窴王安國攻拘彌,大破之,殺其王,死者甚眾,戊己校尉、西域長史各發兵輔立拘彌侍子定興為王。時人眾裁有千口。其國西接于窴三百九十里。

于窴國居西城,去長史所居五千三百里,去洛陽萬一千七百里。領戶三萬二千,口八萬三千,勝兵三萬餘人。

建武末,莎車王賢強盛,攻并于窴,徙其王俞林為驪歸王。明帝永平中,于窴將休莫霸反莎車,自立為于窴王。休莫霸死,兄子廣德立,後遂滅莎車,其國轉盛。從精絕西北至疏勒十三國皆服從。而鄯善王亦始強盛。自是南道自蔥領以東,唯此二國為大。

順帝永建六年,于窴王放前遣侍子詣闕貢獻。元嘉元年,長史趙評在于窴病癰死,評子迎喪,道經拘彌。拘彌王成國與于窴王建素有隙,乃語評子云:「于窴王令胡醫持毒藥著創中,故致死耳。」評子信之,還入塞,以告敦煌太守馬達。明年,以王敬代為長史,達令敬隱覈其事。敬先過拘彌,成國復說云:「于窴國人欲以我為王,今可因此罪誅建,于窴必服矣。」敬貪立功名,且受成國之說,前到于窴,設供具請建,而陰圖之。或以敬謀告建,建不信,曰:「我無罪,王長史何為欲殺我?」旦日,建從官屬數十人詣敬。坐定,建起行酒,敬叱左右執之,吏士並無殺建意,官屬悉得突走。時成國主簿秦牧隨敬在會,持刀出曰:「大事已定,何為復疑?」即前斬建。于窴侯將輸僰等遂會兵攻敬,敬持建頭上樓宣告曰:「天子使我誅建耳。」于窴侯將遂焚營舍,燒殺吏士,上樓斬敬,懸首於巿。輸僰欲自立為王,國人殺之,而立建子安國焉。馬達聞之,欲將諸郡兵出塞擊于窴,桓帝不聽,徵達還,而以宋亮代為敦煌太守。亮到,開募于窴,令自斬輸僰。時輸僰死已經月,乃斷死人頭送敦煌,而不言其狀。亮後知其詐,而竟不能出兵。于窴恃此遂驕。

自于窴經皮山,至西夜、子合、德若焉。

西夜國一名漂沙,去洛陽萬四千四百里。戶二千五百,口萬餘,勝兵三千人。地生白草,有毒,國人煎以為藥,傅箭鏃,所中即死。漢書中誤云西夜、子合是一國,今各自有王。

子合國居呼鞬谷。去疏勒千里。領戶三百五十,口四千,勝兵千人。

德若國領戶百餘,口六百七十,勝兵三百五十人。東去長史居三千五百三十里,去洛陽萬二千一百五十里,與子合相接。其俗皆同。

自皮山西南經烏秅,涉懸度,歷罽賓,六十餘日行至烏弋山離國,地方數千里,時改名排持。

復西南馬行百餘日至條支。

條支國城在山上,周回四十餘里。臨西海,海水曲環其南及東北,三面路絕,唯西北隅通陸道。土地暑溼,出師子、犀牛、封牛、孔雀、大雀。大雀其卵如甕。

轉北而東,復馬行六十餘日至安息。後役屬條支,為置大將,監領諸小城焉。

安息國居和櫝城,去洛陽二萬五千里。北與康居接,南與烏弋山離接。地方數千里,小城數百,戶口勝兵最為殷盛。其東界木鹿城,號為小安息,去洛陽二萬里。

章帝章和元年,遣使獻師子、符拔。符拔形似麟而無角。和帝永元九年,都護班超遣甘英使大秦,抵條支。臨大海欲度,而安息西界船人謂英曰:「海水廣大,往來者逢善風三月乃得度,若遇遲風,亦有二歲者,故入海人皆齎三歲糧。海中善使人思土戀慕,數有死亡者。」英聞之乃止。十三年,安息王滿屈復獻師子及條支大鳥,時謂之安息雀。

自安息西行三千四百里至阿蠻國。從阿蠻西行三千六百里至斯賓國。從斯賓南行度河,又西南至于羅國九百六十里,安息西界極矣。自此南乘海,乃通大秦。其土多海西珍奇異物焉。

大秦國一名犁鞬,以在海西,亦云海西國。地方數千里,有四百餘城。小國役屬者數十。以石為城郭。列置郵亭,皆堊塈之。有松柏諸木百草。人俗力田作,多種樹蠶桑。皆髡頭而衣文繡,乘輜軿白蓋小車,出入擊鼓,建旌旗幡幟。

所居城邑,周圜百餘里。城中有五宮,相去各十里。宮室皆以水精為柱,食器亦然。其王日游一宮,聽事五日而後遍。常使一人持囊隨王車,人有言事者,即以書投囊中,王至宮發省,理其枉直。各有官曹文書。置三十六將,皆會議國事。其王無有常人,皆簡立賢者。國中災異及風雨不時,輒廢而更立,受放者甘黜不怨。其人民皆長大平正,有類中國,故謂之大秦。

土多金銀奇寶,有夜光璧、明月珠、駭雞犀、珊瑚、虎魄、琉璃、琅玕、朱丹、青碧。刺金縷繡,織成金縷罽、雜色綾。作黃金塗、火浣布。又有細布,或言水羊毳,野蠶繭所作也。合會諸香,煎其汁以為蘇合。凡外國諸珍異皆出焉。

以金銀為錢,銀錢十當金錢一。與安息、天竺交巿於海中,利有十倍。其人質直,巿無二價。穀食常賤,國用富饒。鄰國使到其界首者,乘驛詣王都,至則給以金錢。其王常欲通使於漢,而安息欲以漢繒綵與之交市,故遮閡不得自達。至桓帝延熹九年,大秦王安敦遣使自日南徼外獻象牙、犀角、玳瑁,始乃一通焉。其所表貢,並無珍異,疑傳者過焉。

或云其國西有弱水、流沙,近西王母所居處,幾於日所入也。漢書云「從條支西行二百餘日,近日所入」,則與今書異矣。前世漢使皆自烏弋以還,莫有至條支者也。又云「從安息陸道繞海北行出海西至大秦,人庶連屬,十里一亭,三十里一置,終無盜賊寇警。而道多猛虎、師子,遮害行旅,不百餘人,齎兵器,輒為所食」。又言「有飛橋數百里可度海北」。諸國所生奇異玉石諸物,譎怪多不經,故不記云。

大月氏國居藍氏城,西接安息,四十九日行,東去長史所居六千五百三十七里,去洛陽萬六千三百七十里。戶十萬,口四十萬,勝兵十餘萬人。

初,月氏為匈奴所滅,遂遷於大夏,分其國為休密、雙靡、貴霜、驸頓、都密,凡五部臓侯。後百餘歲,貴霜臓侯丘就卻攻滅四臓侯,自立為王,國號貴霜王。侵安息,取高附地。又滅濮達、罽賓,悉有其國。丘就卻年八十餘死,子閻膏珍代為王。復滅天竺,置將一人監領之。月氏自此之後,最為富盛,諸國稱之皆曰貴霜王。漢本其故號,言大月氏云。

高附國在大月氏西南,亦大國也。其俗似天竺,而弱,易服。善賈販,內富於財。所屬無常,天竺、罽賓、安息三國強則得之,弱則失之,而未嘗屬月氏。漢書以為五臓侯數,非其實也。後屬安息。及月氏破安息,始得高附。

天竺國一名身毒,在月氏之東南數千里。俗與月氏同,而卑溼暑熱。其國臨大水。乘象而戰。其人弱於月氏,脩浮圖道,不殺伐,遂以成俗。從月氏、高附國以西,南至西海,東至磐起國,皆身毒之地。身毒有別城數百,城置長。別國數十,國置王。雖各小異,而俱以身毒為名,其時皆屬月氏。月氏殺其王而置將,令統其人。土出象、犀、玳瑁、金、銀、銅、鐵、鉛、錫,西與大秦通,有大秦珍物。又有細布、好毾卫、諸香、石蜜、胡椒、薑、黑鹽。

和帝時,數遣使貢獻,後西域反畔,乃絕。至桓帝延熹二年、四年,頻從日南徼外來獻。

世傳明帝夢見金人,長大,頂有光明,以問群臣。或曰:「西方有神,名曰佛,其形長丈六尺而黃金色。」帝於是遣使天竺問佛道法,遂於中國圖畫形像焉。楚王英始信其術,中國因此頗有奉其道者。後桓帝好神,數祀浮圖、老子,百姓稍有奉者,後遂轉盛。

東離國居沙奇城,在天竺東南三千餘里,大國也。其土氣、物類與天竺同。列城數十,皆稱王。大月氏伐之,遂臣服焉。男女皆長八尺,而怯弱。乘象、駱駝,往來鄰國。有寇,乘象以戰。

栗弋國屬康居。出名馬牛羊、蒲萄眾果,其土水美,故蒲萄酒特有名焉。

嚴國在奄蔡北,屬康居,出鼠皮以輸之。

奄蔡國改名阿蘭聊國,居地城,屬康居。土氣溫和,多楨松、白草。民俗衣服與康居同。

莎車國西經蒲犁、無雷至大月氏,東去洛陽萬九百五十里。

匈奴單于因王莽之亂,略有西域,唯莎車王延最強,不肯附屬。元帝時,嘗為侍子,長於京師,慕樂中國,亦復參其典法。常敕諸子,當世奉漢家,不可負也。天鳳五年,延死,謚忠武王,子康代立。

光武初,康率傍國拒匈奴,擁衛故都護吏士妻子千餘口,檄書河西,問中國動靜,自陳思慕漢家。建武五年,河西大將軍竇融乃承制立康為漢莎車建功懷德王、西域大都尉,五十五國皆屬焉

九年,康死,謚宣成王。弟賢代立,攻破拘彌、西夜國,皆殺其王,而立其兄康兩子為拘彌、西夜王。十四年,賢與鄯善王安並遣使詣闕貢獻,於是西域始通。蔥領以東諸國皆屬賢。十七年,賢復遣使奉獻,請都護。天子以問大司空竇融,以為賢父子兄弟相約事漢,款誠又至,宜加號位以鎮安之。帝乃因其使,賜賢西域都護印綬,及車旗黃金錦繡。敦煌太守裴遵上言:「夷狄不可假以大權,又令諸國失望。」詔書收還都護印綬,更賜賢以漢大將軍印綬。其使不肯易,遵迫奪之,賢由是始恨。而猶詐稱大都護,移書諸國,諸國悉服屬焉,號賢為單于。賢浸以驕橫,重求賦稅,數攻龜茲諸國,諸國愁懼。

二十一年冬,車師前王、鄯善、焉耆等十八國俱遣子入侍,獻其珍寶。及得見,皆流涕稽首,願得都護。天子以中國初定,北邊未服,皆還其侍子,厚賞賜之。是時賢自負兵強,欲并兼西域,攻擊益甚。諸國聞都護不出,而侍子皆還,大憂恐,乃與敦煌太守檄,願留侍子以示莎車,言侍子見留,都護尋出,冀且息其兵。裴遵以狀聞,天子許之。二十二年,賢知都護不至,遂遺鄯善王安書,令絕通漢道。安不納而殺其使。賢大怒,發兵攻鄯善。安迎戰,兵敗,亡入山中。賢殺略千餘人而去。其冬,賢復攻殺龜茲王,遂兼其國。鄯善、焉耆諸國侍子久留敦煌,愁思,皆亡歸。鄯善王上書,願復遣子入侍,更請都護。都護不出,誠迫於匈奴。天子報曰:「今使者大兵未能得出,如諸國力不從心,東西南北自在也。」於是鄯善、車師復附匈奴,而賢益橫。

媯塞王自以國遠,遂殺賢使者,賢擊滅之,立其國貴人駟鞬為媯塞王。賢又自立其子則羅為龜茲王。賢以則羅年少,乃分龜茲為烏壘國,徙駟鞬為烏壘王,又更以貴人為媯塞王。數歲,龜茲國人共殺則羅、駟鞬,而遣使匈奴,更請立王。匈奴立龜茲貴人身毒為龜茲王,龜茲由是屬匈奴。

賢以大宛貢稅滅少,自將諸國兵數萬人攻大宛,大宛王延留迎降,賢因將還國,徙拘彌王橋塞提為大宛王。而康居數攻之,橋塞提在國歲餘,亡歸,賢復以為拘彌王,而遣延留還大宛,使貢獻如常。賢又徙于窴王俞林為驪歸王,立其弟位侍為于窴王。歲餘,賢疑諸國欲畔,召位侍及拘彌、姑墨、子合王,盡殺之,不復置王,但遣將鎮守其國。位侍子戎亡降漢,封為守節侯。

莎車將君得在于窴暴虐,百姓患之。明帝永平三年,其大人都末出城,見野豕,欲射之。豕乃言曰:「無射我,我乃為汝殺君得。」都末因此即與兄弟共殺君得。而大人休莫霸復與漢人韓融等殺都末兄弟,自立為于窴王,復與拘彌國人攻殺莎車將在皮山者,引兵歸。於是賢遣其太子、國相,將諸國兵二萬人擊休莫霸,霸迎與戰,莎車兵敗走,殺萬餘人。賢復發諸國數萬人,自將擊休莫霸,霸復破之,斬殺過半,賢脫身走歸國。休莫霸進圍莎車,中流矢死,兵乃退。

于窴國相蘇榆勒等共立休莫霸兄子廣德為王。匈奴與龜茲諸國共攻莎車,不能下。廣德承莎車之敝,使弟輔國侯仁將兵攻賢。賢連被兵革,乃遣使與廣德和。先是廣德父拘在莎車數歲,於是賢歸其父,而以女妻之,結為昆弟,廣德引兵去。明年,莎車相且運等患賢驕暴,密謀反城降于窴。于窴王廣德乃將諸國兵三萬人攻莎車。賢城守,使使謂廣德曰:「我還汝父,與汝婦,汝來擊我何為?」廣德曰:「王,我婦父也,久不相見,願各從兩人會城外結盟。」賢以問且運,且運曰:「廣德女婿至親,宜出見之。」賢乃輕出,廣德遂執賢。而且運等因內于窴兵,虜賢妻子而并其國。鎖賢將歸,歲餘殺之。

匈奴聞廣德滅莎車,遣五將發焉耆、尉黎、龜茲十五國兵三萬餘人圍于窴,廣德乞降,以其太子為質,約歲給罽絮。冬,匈奴復遣兵將賢質子不居徵立為莎車王,廣德又攻殺之,更立其弟齊黎為莎車王,章帝元和三年。時長史班超發諸國兵擊莎車,大破之,由是遂降漢。事已具班超傳。

莎車東北至疏勒。

疏勒國去長史所居五千里,去洛陽萬三百里。領戶二萬一千,勝兵三萬餘人。

明帝永平十六年,龜茲王建攻殺疏勒王成,自以龜茲左侯兜題為疏勒王。冬,漢遣軍司馬班超劫縛兜題,而立成之兄子忠為疏勒王。忠後反畔,超擊斬之。事已具超傳。

安帝元初中,疏勒王安國以舅臣磐有罪,徙於月氏,月氏王親愛之。後安國死,無子,母持國政,與國人共立臣磐同產弟子遺腹為疏勒王。臣磐聞之,請月氏王曰:「安國無子,種人微弱,若立母氏,我乃遺腹叔父也,我當為王。」月氏乃遣兵送還疏勒。國人素敬愛臣磐,又畏憚月氏,即共奪遺腹印綬,迎臣磐立為王,更以遺腹為磐稿城侯。後莎車畔于窴,屬疏勒,疏勒以強,故得與龜茲、于窴為敵國焉。

順帝永建二年,臣磐遣使奉獻,帝拜臣磐為漢大都尉,兄子臣勳為守國司馬。五年,臣磐遣侍子與大宛、莎車使俱詣闕貢獻。陽嘉二年,臣磐復獻師子、封牛。至靈帝建寧元年,疏勒王漢大都尉於獵中為其季父和得所射殺,和得自立為王。五年,涼州刺史孟佗遣從事任涉將敦煌兵五百人,與戊己司馬曹寬、西域長史張晏,將焉耆、龜茲、車師前後部,合三萬餘人,討疏勒,攻楨中城,四十餘日不能下,引去。其後疏勒王連相殺害,朝廷亦不能禁。

東北經尉頭、溫宿、姑墨、龜茲至焉耆。

焉耆國王居南河城,北去長史所居八百里,東去洛陽八千二百里。戶萬五千,口五萬二千,勝兵二萬餘人。其國四面有大山,與龜茲相連,道險阨易守。有海水曲入四山之內,周匝其城三十餘里。

永平末,焉耆與龜茲共攻沒都護陳睦、副校尉郭恂,殺吏士二千餘人。至永元六年,都護班超發諸國兵討焉耆、危須、尉黎、山國,遂斬焉耆、尉黎二王首,傳送京師,縣蠻夷邸。超乃立焉耆左侯元孟為王,尉黎、危須、山國皆更立其王。至安帝時,西域背畔。延光中,超子勇為西域長史,復討定諸國。元孟與尉黎、危須不降。永建二年,勇與敦煌太守張朗擊破之,元孟乃遣子詣闕貢獻。

蒲類國居天山西疏榆谷,東南去長史所居千二百九十里,去洛陽萬四百九十里。戶八百餘,口二千餘,勝兵七百餘人。廬帳而居,逐水草,頗知田作。有牛、馬、駱锓、羊畜。能作弓矢。國出好馬。

蒲類本大國也,前西域屬匈奴,而其王得罪單于,單于怒,徙蒲類人六千餘口,內之匈奴右部阿惡地,因號曰阿惡國。南去車師後部馬行九十餘日。人口貧羸,逃亡山谷閒,故留為國云。

移支國居蒲類地。戶千餘,口三千餘,勝兵千餘人。其人勇猛敢戰,以寇鈔為事。皆被髮,隨畜逐水草,不知田作。所出皆與蒲類同。

東且彌國東去長史所居八百里,去洛陽九千二百五十里。戶三千餘,口五千餘,勝兵二千餘人。廬帳居,逐水草,頗田作。其所出有亦與蒲類同。所居無常。

車師前王居交河城。河水分流繞城,故號交河。去長史所居柳中八十里,東去洛陽九千一百二十里。領戶千五百餘,口四千餘,勝兵二千人。

後王居務塗谷,去長史所居五百里,去洛陽九千六百二十里。領戶四千餘,口萬五千餘,勝兵三千餘人。

前後部及東且彌、卑陸、蒲類、移支,是為車師六國,北與匈奴接。前部西通焉耆北道,後部西通烏孫。

建武二十一年,與鄯善、焉耆遣子入侍,光武遣還之,乃附屬匈奴。明帝永平十六年,漢取伊吾盧,通西域,車師始復內屬。匈奴遣兵擊之,復降北虜。和帝永元二年,大將軍竇憲破北匈奴,車師震慴,前後王各遣子奉貢入侍,並賜印綬金帛。八年,戊己校尉索頵欲廢後部王涿鞮,立破虜侯細致。涿鞮忿前王尉卑大賣己,因反擊尉卑大,獲其妻子。明年,漢遣將兵長史王林,發涼州六郡兵及羌虜胡二萬餘人,以討涿鞮,獲首虜千餘人。涿鞮入北匈奴,漢軍追擊,斬之,立涿鞮弟農奇為王。至永寧元年,後王軍就及母沙麻反畔,殺後部司馬及敦煌行事。至安帝延光四年,長史班勇擊軍就,大破,斬之。

順帝永建元年,勇率後王農奇子加特奴及八滑等,發精兵擊北虜呼衍王,破之。勇於是上立加特奴為後王,八滑為後部親漢侯。陽嘉三年夏,車師後部司馬率加特奴等千五百人,掩擊北匈奴於閶吾陸谷,壞其廬落,斬數百級,獲單于母、季母及婦女數百人,牛羊十餘萬頭,車千餘兩,兵器什物甚眾。四年春,北匈奴呼衍王率兵侵後部,帝以車師六國接近北虜,為西域蔽扞,乃令敦煌太守發諸國兵,及玉門關候、伊吾司馬,合六千三百騎救之,掩擊北虜於勒山,漢軍不利。秋,呼衍王復將二千人攻後部,破之。桓帝元嘉元年,呼衍王將三千餘騎寇伊吾,伊吾司馬毛愷遣吏兵五百人於蒲類海東與呼衍王戰,悉為所沒,呼衍王遂攻伊吾屯城。夏,遣敦煌太守司馬達將敦煌、酒泉、張掖屬國吏士四千餘人救之,出塞至蒲類海,呼衍王聞而引去,漢軍無功而還。

永興元年,車師後部王阿羅多與戊部候嚴皓不相得,遂忿戾反畔,攻圍漢屯田且固城,殺傷吏士。後部候炭遮領餘人畔阿羅多詣漢吏降。阿羅多迫急,將其母妻子從百餘騎亡走北匈奴中,敦煌太守宋亮上立後部故王軍就質子卑君為後部王。後阿羅多復從匈奴中還,與卑君爭國,頗收其國人。戊校尉閻詳慮其招引北虜,將亂西域,乃開信告示,許復為王,阿羅多乃詣詳降。於是收奪所賜卑君印綬,更立阿羅多為王,仍將卑君還敦煌,以後部人三百帳別屬役之,食其稅。帳者,猶中國之戶數也。

論曰:西域風土之載,前古未聞也。漢世張騫懷致遠之略,班超奮封侯之志,終能立功西遐,羈服外域。自兵威之所肅服,財賂之所懷誘,莫不獻方奇,納愛質,露頂肘行,東向而朝天子。故設戊己之官,分任其事;建都護之帥,總領其權。先馴則賞継金而賜龜綬,後服則繫頭顙而釁北闕。立屯田於膏腴之野,列郵置於要害之路。馳命走驛,不絕於時月;商胡販客,日款於塞下。其後甘英乃抵條支而歷安息,臨西海以望大秦,拒玉門、陽關者四萬餘里,靡不周盡焉。若其境俗性智之優薄,產載物類之區品,川河領障之基源,氣節涼暑之通隔,梯山棧谷繩行沙度之道,身熱首痛風災鬼難之域,莫不備寫情形,審求根實。至於佛道神化,興自身毒,而二漢方志莫有稱焉。張騫但著地多暑溼,乘象而戰,班勇雖列其奉浮圖,不殺伐,而精文善法導達之功靡所傳述。余聞之後說也,其國則殷乎中土,玉燭和氣,靈聖之所集,賢懿之所挺生,神跡詭怪,則理絕人區,感驗明顯,則事出天外。而騫、超無聞者,豈其道閉往運,數開叔葉乎?不然,何誣異之甚也!漢自楚英始盛齋戒之祀,桓帝又修華蓋之飾。將微義未譯,而但神明之邪?詳其清心釋累之訓,空有兼遣之宗,道書之流也。且好仁惡殺,蠲敝崇善,所以賢達君子多愛其法焉。然好大不經,奇譎無已,雖鄒衍談天之辯,莊周蝸角之論,尚未足以概其萬一。又精靈起滅,因報相尋,若曉而昧者,故通人多惑焉。蓋導俗無方,適物異會,取諸同歸,措夫疑說,則大道通矣。

贊曰:纯矣西胡,天之外區。土物琛麗,人性淫虛。不率華禮,莫有典書。若微神道,何恤何拘。


\end{pinyinscope}