\article{百官四}

\begin{pinyinscope}
將作大匠城門校尉北軍中候

司隸校尉

執金吾一人,中二千石。本注曰:掌宮外戒司非常水火之事。月三繞行宮外,及主兵器。吾猶禦也。丞一人,比千石。緹騎二百人。本注曰:無秩,比吏食奉。

武庫令一人,六百石。本注曰:主兵器。丞一人。

右屬執金吾。本注曰:本有式道、左右中候三人,六百石。車駕出,掌在前清道,還持麾至宮門,宮門乃開。中興但一人,又不常置,每出,以郎兼式道候,事已罷,不復屬執金吾。又省中壘、寺互、都船令、丞、尉及左右京輔都尉。

太子太傅一人,中二千石。本注曰:職掌輔導太子。禮如師,不領官屬。

太長秋一人,二千石。本注曰:承秦將行,宦者。景帝更為大長秋,或用士人。中興常用宦者,職掌奉宣中宮命。凡給賜宗親,及宗親當謁見者關通之,中宮出則從。丞一人,六百石。本注曰:宦者。

中宮僕一人,千石。本注曰:宦者。主馭。本注曰:太僕,秩二千石,中興省「太」,減秩千石,以屬長秋。

中宮謁者令一人,六百石。本注曰:宦者。中宮謁者三人,四百石。本注曰:宦者。主報中章。

中宮尚書五人,六百石。本注曰:宦者。主中文書。

中宮私府令一人,六百石。本注曰:宦者。主中藏幣帛諸物,裁衣被補浣者皆主之。丞一人。本注曰:宦者。

中宮永巷令一人,六百石。本注曰:宦者。主宮人。丞一人。本注曰:宦者。

中宮黃門冗從僕射一人,六百石。本注曰:宦者。主中黃門冗從。

中宮署令一人,六百石。本注曰:宦者。主中宮清署天子數。女騎六人,丞、復道丞各一人。本注曰:宦者。復道丞主中閣道。

中宮藥長一人,四百石。本注曰:宦者。

右屬大長秋。本注曰:承秦,有詹事一人,位在長秋上,亦宦者,主中諸官。成帝省之,以其職并長秋。是後皇后當法駕出,則中謁、中宦者職吏權兼詹事奉引,訖罷。宦者誅後,尚書選兼職吏一人奉引云。其中長信、長樂宮者,置少府一人,職如長秋,及餘吏皆以宮名為號,員數秩次如中宮。本注曰:帝祖母稱長信宮,故有長信少府,長樂少府,位在長秋上,及職吏皆宦者,秩次如中宮。長樂又有衛尉,僕為太僕,皆二千石,在少府上。其崩則省,不常置。

太子少傅,二千石。本注曰:亦以輔導為職,悉主太子官屬。

太子率更令一人,千石。本注曰:主庶子、舍人更直,職似光祿。

太子庶子,四百石。本注曰:無員,如三署中郎。

太子舍人,二百石。本注曰:無員,更直宿衛,如三署郎中。

太子家令一人,千石。本注曰:主倉穀飲食,職似司農、少府。

太子倉令一人,六百石。本注曰:主倉穀。

太子食官令一人,六百石。本注曰:主飲食。

太子僕一人,千石。本注曰:主車馬,職如太僕。

太子廄長一人,四百石。本注曰:主車馬。

太子門大夫,六百石。本注曰:舊注云職比郎將。舊有左右戶將,別主左右戶直郎,建武以來省之。

太子中庶子,六百石。本注曰:員五人,職如侍中。

太子洗馬,比六百石。本注曰:舊注云員十六人,職如謁者。太子出,則當直者在前導威儀。

太子中盾一人,四百石。本注曰:主周衛徼循。

太子衛率一人,四百石。本注曰:主門衛士。

右屬太子少傅。本注曰:凡初即位,未有太子,官屬皆罷,唯舍人不省,領屬少府。

將作大匠一人,二千石。本注曰:承秦,曰將作少府,景帝改為將作大匠。掌修作宗廟、路寢、宮室、陵園木土之功,并樹桐梓之類列于道側。丞一人,六百石。

左校令一人,六百石。本注曰:掌左工徒。丞一人。

右校令一人,六百石。本注曰:掌右工徒。丞一人。

右屬將作大匠。

城門校尉一人,比二千石。本注曰:掌雒陽城門十二所。

司馬一人,千石。本注曰:主兵。城門每門候一人,六百石。本注曰:雒陽城十二門,其正南一門曰平城門,北宮門,屬衛尉。其餘上西門,雍門,廣陽門,津門,小苑門,開陽門,秏門,中東門,上東門,穀門,夏門,凡十二門。

右屬城門校尉。

北軍中候一人,六百石。本注曰:掌監五營。

屯騎校尉一人,比二千石。本注曰:掌宿衛兵。司馬一人,千石。

越騎校尉一人,比二千石。本注曰:掌宿衛兵。司馬一人,千石。

步兵校尉一人,比二千石。本注曰:掌宿衛兵。司馬一人,千石。

長水校尉一人,比二千石。本注曰:掌宿衛兵。司馬、胡騎司馬各一人,千石。本注曰:掌宿衛,主烏桓騎。

射聲校尉一人,比二千石。本注曰:掌宿衛兵。司馬一人,千石。

右屬北軍中候。本注曰:舊有中壘校尉,領北軍營壘之事。有胡騎、虎賁校尉,皆武帝置。中興省中壘,但置中候,以監五營。胡騎并長水。虎賁主輕車,并射聲。

凡中二千石,丞比千石。真二千石,丞、長史六百石。比二千石,丞比六百石。令、相千石,丞、尉四百石;其六百石,丞、尉三百石。長、相四百石及三百石,丞、尉皆二百石。諸侯、公主家丞,秩皆比百石。諸邊鄣塞尉、諸陵校尉長,皆二百石。有常例者不署秩。

司隸校尉一人,比二千石。本注曰:孝武帝初置,持節,掌察舉百官以下,及京師近郡犯法者。元帝去節,成帝省,建武中復置,并領一州。從事史十二人。本注曰:都官從事,主察舉百官犯法者。功曹從事,主州選署及眾事。別駕從事,校尉行部則奉引,錄眾事。簿曹從事,主財穀簿書。其有軍事,則置兵曹從事,主兵事。其餘部郡國從事,每郡國各一人,主督促文書,察舉非法,皆州自辟除,故通為百石云。假佐二十五人。本注曰:主簿錄閤下事,省文書。門亭長主州正。門功曹書佐主選用。孝經師主監試經。月令師主時節祠祀。律令師主平法律。簿曹書佐主簿書。其餘都官書佐及每郡國,各有典郡書佐一人,各主一郡文書,以郡吏補,歲滿一更。司隸所部郡七。

河南尹一人,主京都,特奉朝請。其京兆尹、左馮翊、右扶風三人,漢初都長安,皆秩中二千石,謂之三輔。中興都雒陽,更以河南郡為尹,以三輔陵廟所在,不改其號,但減其秩。其餘弘農、河內、河東三郡。其置尹,馮翊、扶風及太守丞奉之本位,在地理志。


\end{pinyinscope}