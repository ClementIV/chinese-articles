\article{顯宗孝明帝紀}

\begin{pinyinscope}
顯宗孝明皇帝諱莊,光武第四子也。母陰皇后。帝生而豐下,十歲能通春秋,光武奇之。建武十五年封東海公,十七年進爵為王,十九年立為皇太子。師事博士桓榮,學通尚書。

中元二年二月戊戌,即皇帝位,年三十。尊皇后曰皇太后。

三月丁卯,葬光武皇帝於原陵。有司奏上尊廟曰世祖。

夏四月丙辰,詔曰:「予末小子,奉承聖業,夙夜震畏,不敢荒寧。先帝受命中興,德侔帝王,協和萬邦,假於上下,懷柔百神,惠於鰥寡。朕承大運,繼體守文,不知稼穡之艱難,懼有廢失。聖恩遺戒,顧重天下,以元元為首。公卿百僚,將何以輔朕不逮?其賜天下男子爵,人二級;三老、孝悌、力田人三級;爵過公乘,得移與子若同產、同產子;及流人無名數欲自占者人一級;鰥、寡、孤、獨、篤缮粟,人十斛。其施刑及郡國徒,在中元元年四月己卯赦前所犯而後捕繫者,悉免其刑。又邊人遭亂為內郡人妻,在己卯赦前,一切遣還邊,恣其所樂。中二千石下至黃綬,貶秩贖論者,悉皆復秩還贖。方今上無天子,下無方伯,若涉淵水而無舟楫。夫萬乘至重而壯者慮輕,實賴有德左右小子。高密侯禹元功之首,東平王蒼寬博有謀,並可以受六尺之託,臨大節而不撓。其以禹為太傅,蒼為驃騎將軍。太尉憙告謚南郊,司徒訢奉安梓宮,司空魴將校復土。其封憙為節鄉侯,訢為安鄉侯,魴為楊邑侯。」

秋九月,燒當羌寇隴西,敗郡兵於允街。赦隴西囚徒,減罪一等,勿收今年租調。又所發天水三千人,亦復是歲更賦。遣謁者張鴻討叛羌於允吾,鴻軍大敗,戰歿。冬十一月,遣中郎將竇固監捕虜將軍馬武等二將軍討燒當羌。

十二月甲寅,詔曰:「方春戒節,人以耕桑。其敕有司務順時氣,使無煩擾。天下亡命殊死以下,聽得贖論:死罪入縑二十匹,右趾至髡鉗城旦舂十匹,完城旦舂至司寇作三匹。其未發覺,詔書到先自告者,半入贖。今選舉不實,邪佞未去,權門請託,殘吏放手,百姓愁怨,情無告訴。有司明奏罪名,并正舉者。又郡縣每因徵發,輕為姦利,詭責羸弱,先急下貧。其務在均平,無令枉刻。」

永平元年春正月,帝率公卿已下朝於原陵,如元會儀。

夏五月,太傅鄧禹薨。

戊寅,東海王彊薨,遣司空馮魴持節視喪事,賜升龍旄頭、鑾輅、龍旂。

六月乙卯,葬東海恭王。

秋七月,捕虜將軍馬武等與燒當羌戰,大破之。募士卒戍隴右,賜錢人三萬。

八月戊子,徙山陽王荊為廣陵王,遣就國。

是歲,遼東太守祭肜使鮮卑擊赤山烏桓,大破之,斬其渠帥。越巂姑復夷叛,州郡討平之。

二年春正月辛末,宗祀光武皇帝於明堂,帝及公卿列侯始服冠冕、衣裳、玉佩、絇屨以行事。禮畢,登靈臺。使尚書令持節詔驃騎將軍、三公曰:「今令月吉日,宗祀光武皇帝於明堂,以配五帝。禮備法物,樂和八音,詠祉福,舞功德,其班時令,敕群后。事畢,升靈臺,望元氣,吹時律,觀物變。群僚藩輔,宗室子孫,眾郡奉計,百蠻貢職,烏桓、濊貊咸來助祭,單于侍子、骨都侯亦皆陪位。斯固聖祖功德之所致也。朕以闇陋,奉承大業,親執珪璧,恭祀天地。仰惟先帝受命中興,撥亂反正,以寧天下,封泰山,建明堂,立辟雍,起靈臺,恢弘大道,被之八極;而胤子無成康之質,群臣無呂旦之謀,盥洗進爵,踧踖惟慚。素性頑鄙,臨事益懼,故『君子坦蕩蕩,小人長戚戚』。其令天下自殊死已下,謀反大逆,皆赦除之。百僚師尹,其勉修厥職,順行時令,敬若昊天,以綏兆人。」

三月,臨辟雍,初行大射禮。

秋九月,沛王輔、楚王英、濟南王康、淮陽王延、東海王政來朝。

冬十月壬子,幸辟雍,初行養老禮。詔曰:「光武皇帝建三朝之禮,而未及臨饗。眇眇小子,屬當聖業。閒暮春吉辰,初行大射;令月元日,復踐辟雍。尊事三老,兄事五更,安車狈輪,供綏執授。侯王設醬,公卿饌珍,朕親袒割,執爵而酳。祝哽在前,祝噎在後。升歌鹿鳴,下管新宮,八佾具脩,萬舞於庭。朕固薄德,何以克當?易陳負乘,詩刺彼己,永念慚疚,無忘厥心。三老李躬,年耆學明。五更桓榮,授朕尚書。《詩》曰:『無德不報,無言不酬。』其賜榮爵關內侯,食邑五千戶。三老、五更皆以二千石祿養終厥身。其賜天下三老酒人一石,肉四十斤。有司其存耆环,恤幼孤,惠鰥寡,稱朕意焉。」

中山王焉始就國。

甲子,西巡狩,幸長安,祠高廟,遂有事於十一陵。歷覽館邑,會郡縣吏,勞賜作樂。十一月甲申,遣使者以中牢祠蕭何、霍光。帝謁陵園,過式其墓。進幸河東,所過賜二千石、令長已下至於掾史,各有差。癸卯,車駕還宮。

十二月,護羌校尉竇林下獄死。

是歲,始迎氣於五郊。少府陰就子豐殺其妻酈邑公主,就坐自殺。

三年春正月癸巳,詔曰:「朕奉郊祀,登靈臺,見史官,正儀度。夫春者,歲之始也。始得其正,則三時有成。比者水旱不節,邊人食寡,政失於上,人受其咎。有司其勉順時氣,勸督農桑,去其螟蜮,以及蝥賊;詳刑慎罰,明察單辭,夙夜匪懈,以稱朕意。」

二月甲寅,太尉趙憙、司徒李訢免。丙辰,左馮翊郭丹為司徒。己未,南陽太守虞延為太尉。

甲子,立貴人馬氏為皇后,皇子炟為皇太子。賜天下男子爵,人二級;三老、孝悌、力田人三級;流人無名數欲占者人一級;鰥、寡、孤、獨、篤缮、貧不能自存者粟,人五斛。

夏四月辛酉,封皇子建為千乘王,羨為廣平王。

六月丁卯,有星孛于天船北。

秋八月戊辰,改大樂為大予樂。

壬申晦,日有蝕之。詔曰:「朕奉承祖業,無有善政。日月薄蝕,彗孛見天,水旱不節,稼穡不成,人無宿儲,下生愁墊。雖夙夜勤思,而智能不逮。昔楚莊無災,以致戒懼;魯哀禍大,天不降譴。今之動變,儻尚可救。有司勉思厥職,以匡無德。古者卿士獻詩,百工箴諫。其言事者,靡有所諱。」

冬十月,蒸祭光武廟,初奏文始、五行、武德之舞。

甲子,車駕從皇太后幸章陵,觀舊廬。十二月戊辰,至自章陵。

是歲,起北宮及諸官府。京師及郡國七大水。

四年春二月辛亥,詔曰:「朕親耕藉田,以祈農事。京師冬無宿雪,春不燠沐,煩勞群司,積精禱求。而比再得時雨,宿麥潤澤。其賜公卿半奉。有司勉遵時政,務平刑罰。」

秋九月戊寅,千乘王建薨。

冬十月乙卯,司徒郭丹、司空馮魴免。丙辰,河南尹范遷為司徒,太僕伏恭為司空。

十二月,陵鄉侯梁松下獄死。

五年春二月庚戌,驃騎將軍東平王蒼罷歸藩;琅邪王京就國。

冬十月,行幸鄴。與趙王栩會鄴。常山三老言於帝曰:「上生於元氏,願蒙優復。」詔曰:「豐、沛、濟陽,受命所由,加恩報德,適其宜也。今永平之政,百姓怨結,而吏人求復,令人愧笑,重逆此縣之拳拳,其復元氏縣田租更賦六歲,勞賜縣掾史,及門闌走卒。」至自鄴。

十一月,北匈奴寇五原;十二月,寇雲中,南單于擊卻之。

是歲,發遣邊人在內郡者,賜裝錢人二萬。

六年春正月,沛王輔、楚王英、東平王蒼、淮陽王延、琅邪王京、東海王政、趙王盱、北海王興、齊王石來朝。

二月,王雒山出寶鼎,廬江太守獻之。夏四月甲子,詔曰:「昔禹收九牧之金,鑄鼎以象物,使人知神姦,不逢惡氣。遭德則興,遷于商、周;周德既衰,鼎乃淪亡。祥瑞之降,以應有德。方今政化多僻,何以致茲?易曰鼎象三公,豈公卿奉職得其理邪?太常其以礿祭之日,陳鼎於廟,以備器用。賜三公帛五十匹,九卿、二千石半之。先帝詔書,禁人上事言聖,而閒者章奏頗多浮詞,自今若有過稱虛譽,尚書皆宜抑而不省,示不為諂子蚩也。」

冬十月,行幸魯,祠東海恭王陵;會沛王輔、楚王英、濟南王康、東平王蒼、淮陽王延、琅邪王京、東海王政。十二月,還,幸陽城,遣使者祠中岳。壬午,車駕還宮。東平王蒼、琅邪王京從駕來朝皇太后。

七年春正月癸卯,皇太后陰氏崩。二月庚申,葬光烈皇后。

秋八月戊辰,北海王興薨。

是歲,北匈奴遣使乞和親。

八年春正月己卯,司徒范遷薨。三月辛卯,太尉虞延為司徒,衛尉趙憙行太尉事。

遣越騎司馬鄭眾報使北匈奴。初置度遼將軍,屯五原曼柏。

秋,郡國十四雨水。

冬十月,北宮成。

丙子,臨辟雍,養三老、五更。禮畢,詔三公募郡國中都官死罪繫囚,減罪一等,勿笞,詣度遼將軍營,屯朔方、五原之邊縣;妻子自隨,便占著邊縣;父母同產欲相代者,恣聽之。其大逆無道殊死者,一切募下蠶室。亡命者令贖罪各有差。凡徙者,賜弓弩衣糧。

壬寅晦,日有食之,既。詔曰:「朕以無德,奉承大業,而下貽人怨,上動三光。日食之變,其災尤大,春秋圖讖所為至譴。永思厥咎,在予一人。群司勉修職事,極言無諱。」於是在位者皆上封事,各言得失。帝覽章,深自引咎,乃以所上班示百官。詔曰:「群僚所言,皆朕之過。人冤不能理,吏黠不能禁;而輕用人力,繕修宮宇,出入無節,喜怒過差。昔應門失守,關雎刺世;飛蓬隨風,微子所歎。永覽前戒,竦然兢懼。徒恐薄德,久而致怠耳。」

北匈奴寇西河諸郡。

九年春三月辛丑,詔郡國死罪囚減罪,與妻子詣五原、朔方占著,所在死者皆賜妻父若男同產一人復終身;其妻無父兄獨有母者,賜其母錢六萬,又復其口筭。

夏四月甲辰,詔郡國以公田賜貧人各有差。令司隸校尉、部刺史歲上墨綬長吏視事三歲已上理狀尤異者各一人,與計偕上。及尤不政理者,亦以聞。

是歲,大有年。為四姓小侯開立學校,置五經師。

十年春二月,廣陵王荊有罪,自殺,國除。

夏四月戊子,詔曰:「昔歲五穀登衍,今茲蠶麥善收,其大赦天下。方盛夏長養之時,蕩滌宿惡,以報農功。百姓勉務桑稼,以備災害。吏敬厥職,無令愆墯。」

閏月甲午,南巡狩,幸南陽,祠章陵。日北至,又祠舊宅。禮畢,召校官弟子作雅樂,奏鹿鳴,帝自御塤篪和之,以娛嘉賓。還,幸南頓,勞饗三老、官屬。

冬十一月,徵淮陽王廷會平輿,徵沛王輔會睢陽。

十二月甲午,車駕還宮。

十一年春正月,沛王輔、楚王英、濟南王康、東平王蒼、淮陽王延、中山王焉、琅邪王京、東海王政來朝。

秋七月,司隸校尉郭霸下獄死。

是歲,漅湖出黃金,廬江太守以獻。時麒麟、白雉、醴泉、嘉禾所在出焉。

十二年春正月,益州徼外夷哀牢王相率內屬,於是置永昌郡,罷益州西部都尉。

夏四月,遣將作謁者王吳修汴渠,自滎陽至于千乘海口。

五月丙辰,賜天下男子爵,人二級,三老、孝悌、力田人三級,流民無名數欲占者人一級;鰥、寡、孤、獨、篤缮、貧無家屬不能自存者粟,人三斛。詔曰:「昔曾、閔奉親,竭觀致養,仲尼葬子,有棺無槨。喪貴致哀,禮存寧儉。今百姓送終之制,競為奢靡。生者無擔石之儲,而財力盡於墳土。伏臘無糟糠,而牲牢兼於一奠。糜破積世之業,以供終朝之費,子孫飢寒,絕命於此,豈祖考之意哉!又車服制度,恣極耳目。田荒不耕,游食者眾。有司其申明科禁,宜於今者,宣下郡國。」

秋七月乙亥,司空伏恭罷。乙未,大司農牟融為司空。

冬十月,司隸校尉王康下獄死。

是歲,天下安平,人無傜役,歲比登稔,百姓殷富,粟斛三十,牛羊被野。

十三年春二月,帝耕於藉田。禮畢,賜觀者食。

三月,河南尹薛昭下獄死。

夏四月,汴渠成。辛巳,行幸滎陽,巡行河渠。乙酉,詔曰:「自汴渠決敗,六十餘歲,加頃年以來,雨水不時,汴流東侵,日月益甚,水門故處,皆在河中,漭瀁廣溢,莫測圻岸,蕩蕩極望,不知綱紀。今兗、豫之人,多被水患,乃云縣官不先人急,好興它役。又或以為河流入汴,幽、冀蒙利,故曰左隄彊則右隄傷,左右俱彊則下方傷,宜任水埶所之,使人隨高而處,公家息壅塞之費,百姓無陷溺之患。議者不同,南北異論,朕不知所從,久而不決。今既築隄理渠,絕水立門,河、汴分流,復其舊跡,陶丘之北,漸就壤墳,故薦嘉玉絜牲,以禮河神。東過洛汭,歎禹之績。今五土之宜,反其正色,濱渠下田,賦與貧人,無令豪右得固其利,庶繼世宗瓠子之作。」因遂度河,登太行,進幸上黨。壬寅,車駕還宮。

冬十月壬辰晦,日有食之。三公免冠自劾。制曰:「冠履勿劾。災異屢見,咎在朕躬,憂懼遑遑,未知其方。將有司陳事,多所隱諱,使君上壅蔽,下有不暢乎?昔衛有忠臣,靈公得守其位。今何以和穆陰陽,消伏災譴?刺史、太守詳刑理冤,存恤鰥孤,勉思職焉。」

十一月,楚王英謀反,廢,國除,遷於涇縣,所連及死徙者數千人。

是歲,齊王石薨。

十四年春三月甲戌,司徒虞延免,自殺。夏四月丁巳,鉅鹿太守南陽邢穆為司徒。

前楚王英自殺。

夏五月,封故廣陵王荊子元壽為廣陵侯。

初作壽陵。

十五年春二月庚子,東巡狩。辛丑,幸偃師。詔亡命自殊死以下贖:死罪縑四十匹,右趾至髡鉗城旦舂十匹,完城旦至司寇五匹;犯罪未發覺,詔書到日自告者,半入贖。徵沛王輔會睢陽。進幸彭城。癸亥,帝耕于下邳。

三月,徵琅邪王京會良成,徵東平王蒼會陽都,又徵廣陵侯及其三弟會魯。祠東海恭王陵。還,幸孔子宅,祠仲尼及七十二弟子。親御講堂,命皇太子、諸王說經。又幸東平。辛卯,進幸大梁,至定陶,祠定陶恭王陵。夏四月庚子,車駕還宮。

改信都為樂成國,臨淮為下邳國。封皇子恭為鉅鹿王,黨為樂成王,衍為下邳王,暢為汝南王,疠為常山王,長為濟陰王。賜天下男子爵,人三級;郎、從官二十歲已上帛百匹,十歲已上二十匹,十歲已下十匹,官府吏五匹,書佐、小史三匹。令天下大酺五日。乙巳,大赦天下,其謀反大逆及諸不應宥者,皆赦除之。

冬,車騎校獵上林苑。

十二月,遣奉車都尉竇固、駙馬都尉耿秉屯涼州。

十六年春二月,遣太僕祭肜出高闕,奉車都尉竇固出酒泉,駙馬都尉耿秉出居延,騎都尉來苗出平城,伐北匈奴。竇固破呼衍王於天山,留兵屯伊吾盧城。耿秉、來苗、祭肜並無功而還。

夏五月,淮陽王延謀反,發覺。癸丑,司徒邢穆、駙馬都尉韓光坐事下獄死,所連及誅死者甚眾。

戊午晦,日有食之。

六月丙寅,大司農西河王敏為司徒。

秋七月,淮陽王延徙封阜陵王。

九月丁卯,詔令郡國中都官死罪繫囚減死罪一等,勿笞,詣軍營,屯朔方、敦煌;妻子自隨,父母同產欲求從者,恣聽之;女子嫁為人妻,勿與俱。謀反大逆無道不用此書。

是歲,北匈奴寇雲中,雲中太守廉范擊破之。

十七年春正月,甘露降於甘陵。北海王睦薨。

二月乙巳,司徒王敏薨。三月癸丑,汝南太守鮑昱為司徒。

是歲,甘露仍降,樹枝內附,芝草生殿前,神雀五色翔集京師。西南夷哀牢、儋耳、僬僥、槃木、白狼、動黏諸種,前後慕義貢獻;西域諸國遣子入侍。夏五月戊子,公卿百官以帝威德懷遠,祥物顯應,乃並集朝堂,奉觴上壽。制曰:「天生神物,以應王者;遠人慕化,實由有德。朕以虛薄,何以享斯?唯高祖、光武聖德所被,不敢有辭。其敬舉觴,太常擇吉日策告宗廟。其賜天下男子爵,人二級,三老、孝悌、力田人三級,流人無名數欲占者人一級;鰥、寡、孤、獨、篤缮、貧不能自存者粟,人三斛;郎、從官視事十歲以上者,帛十匹。中二千石、二千石下至黃綬,貶秩奉贖,在去年以來皆還贖。」

秋八月丙寅,令武威、張掖、酒泉、敦煌及張掖屬國,繫囚右趾已下任兵者,皆一切勿治其罪,詣軍營。

冬十一月,遣奉車都尉竇固、駙馬都尉耿秉、騎都尉劉張出敦煌昆侖塞,擊破白山虜於蒲類海上,遂入車師。初置西域都護、戊己校尉。

是歲,改天水為漢陽郡。

十八年春三月丁亥,詔曰:「其令天下亡命,自殊死已下贖:死罪縑三十匹,右趾至髡鉗城旦舂十匹,完城旦至司寇五匹;吏人犯罪未發覺,詔書到自告者,半入贖。」

夏四月己未,詔曰:「自春已來,時雨不降,宿麥傷旱,秋種未下,政失厥中,憂懼而已。其賜天下男子爵,人二級,及流民無名數欲占者人一級;鰥、寡、孤、獨、篤缮,貧不能自存者粟,人三斛。理冤獄,錄輕繫。二千石分禱五岳四瀆。郡界有名山大川能興雲雨者,長吏各絜齋禱請,冀蒙嘉澍。」

六月己未,有星孛於太微。

焉耆、龜茲攻西域都護陳睦,悉沒其眾。北匈奴及車師後王圍戊己校尉耿恭。

秋八月壬子,帝崩於東宮前殿。年四十八。遺詔無起寑廟,藏主於光烈皇后更衣別室。帝初作壽陵,制令流水而已,石槨廣一丈二尺,長二丈五尺,無得起墳。萬年之後,埽地而祭,杅水脯糒而已。過百日,唯四時設奠,置吏卒數人供給灑埽,勿開修道。敢有所興作者,以擅議宗廟法從事。

帝遵奉建武制度,無敢違者。後宮之家,不得封侯與政。館陶公主為子求郎,不許,而賜錢千萬。謂群臣曰:「郎官上應列宿,出宰百里,有非其人,則民受其殃,是以難之。」故吏稱其官,民安其業,遠近肅服,戶口滋殖焉。

論曰:明帝善刑理,法令分明。日晏坐朝,幽枉必達。內外無倖曲之私,在上無矜大之色。斷獄得情,號居前代十二。故後之言事者,莫不先建武、永平之政。而鍾離意、宋均之徒,常以察慧為言,夫豈弘人之度未優乎?

贊曰:顯宗丕承,業業兢兢。危心恭德,政察姦勝。備章朝物,省薄墳陵。永懷廢典,下身遵道。登臺觀雲,臨雍拜老。懋惟帝績,增光文考。


\end{pinyinscope}