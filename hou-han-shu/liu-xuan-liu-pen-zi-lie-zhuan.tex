\article{劉玄劉盆子列傳}

\begin{pinyinscope}
劉玄字聖公,光武族兄也,弟為人所殺,聖公結客欲報之。客犯法,聖公避吏於平林。吏繫聖公父子張。聖公詐死,使人持喪歸舂陵,吏乃出子張,聖公因自逃匿。

王莽末,南方飢饉,人庶群入野澤,掘鳧茈而食之,更相侵奪。新市人王匡、王鳳為平理諍訟,遂推為渠帥,眾數百人。於是諸亡命馬武、王常、成丹等往從之;共攻離鄉聚,臧於綠林中,數月閒至七八千人。地皇二年,荊州牧某發奔命二萬人攻之,匡等相率迎擊於雲杜。大破牧軍,殺數千人,盡獲輜重,遂攻拔竟陵。轉擊雲杜、安陸,多略婦女,還入綠林中,至有五萬餘口,州郡不能制。

三年,大疾疫,死者且半,乃各分散引去。王常、成丹西入南郡,號下江兵;王匡、王鳳、馬武及其支黨朱鮪、張卬等北入南陽,號新市兵:皆自稱將軍。七月,匡等進攻隨,未能下。平林人陳牧、廖湛復聚眾千餘人,號平林兵,以應之。聖公因往從牧等,為其軍安集掾。

是時光武及兄伯升亦起舂陵,與諸部合兵而進。四年正月,破王莽前隊大夫甄阜、屬正梁丘賜,斬之,號聖公為更始將軍。眾雖多而無所統一,諸將遂共議立更始為天子。二月辛巳,設壇場於淯水上沙中,陳兵大會。更始即帝位,南面立,朝群臣。素懦弱,羞愧流汗,舉手不能言。於是大赦天下,建元曰更始元年。悉拜置諸將,以族父良為國三老,王匡為定國上公,王鳳成國上公,朱鮪大司馬,伯升大司徒,陳牧大司空,餘皆九卿、將軍。五月,伯升拔宛。六月,更始入都宛城,盡封宗室及諸將,為列侯者百餘人。

更始忌伯升威名,遂誅之,以光祿勳劉賜為大司徒。前鍾武侯劉望起兵,略有汝南。時王莽納言將軍嚴尤、秩宗將軍陳茂既敗於昆陽,往歸之。八月,望遂自立為天子,以尤為大司馬,茂為丞相。王莽使太師王匡、國將哀章守洛陽。更始遣定國上公王匡攻洛陽,西屏大將軍申屠建、丞相司直李松攻武關,三輔震動。是時海內豪桀翕然響應,皆殺其牧守,自稱將軍,用漢年號,以待詔命,旬月之閒,遍於天下。

長安中起兵攻未央宮。九月,東海人公賓就斬王莽於漸臺,收璽綬,傳首詣宛。更始時在便坐黃堂,取視之,喜曰:「莽不如是,當與霍光等。」寵姬韓夫人笑曰:「若不如是,帝焉得之乎?」更始悅,乃懸莽首於宛城市。是月,拔洛陽,生縛王匡、哀章,至,皆斬之。十月,使奮威大將軍劉信擊殺劉望於汝南,并誅嚴尤、陳茂。更始遂北都洛陽,以劉賜為丞相。申屠建、李松自長安傳送乘輿服御,又遣中黃門從官奉迎遷都。二年二月,更始自洛陽而西。初發,李松奉引,馬驚奔,觸北宮鐵柱,三馬皆死。

初,王莽敗,唯未央宮被焚而已,其餘宮館一無所毀。宮女數千,備列後庭,自鍾鼓、帷帳、輿輦、器服、太倉、武庫、官府、市里,不改於舊。更始既至,居長樂宮,升前殿,郎吏以次列庭中。更始羞怍,俛首刮席不敢視。諸將後至者,更始問虜掠得幾何,左右侍官皆宮省久吏,各驚相視。

李松與棘陽人趙萌說更始,宜悉王諸功臣。朱鮪爭之,以為高祖約,非劉氏不王。更始乃先封宗室太常將軍劉祉為定陶王,劉賜為宛王,劉慶為燕王,劉歙為元氏王,大將軍劉嘉為漢中王,劉信為汝陰王;後遂立王匡為比陽王,王鳳為宜城王,朱鮪為膠東王,衛尉大將軍張卬為淮陽王,廷尉大將軍王常為鄧王,執金吾大將軍廖湛為穰王,申屠建為平氏王,尚書胡殷為隨王,桂天大將軍李通為西平王,五威中郎將李軼為舞陰王,水衡大將軍成丹為襄邑王,大司空陳牧為陰平王,驃騎大將軍宋佻為潁陰王,尹尊為郾王。唯朱鮪辭曰:「臣非劉宗,不敢干典。」遂讓不受。乃徙鮪為左大司馬,劉賜為前大司馬,使與李軼、李通、王常等鎮撫關東。以李松為丞相,趙萌為右大司馬,共秉內任。

更始納趙萌女為夫人,有寵,遂委政於萌,日夜與婦人飲讌後庭。群臣欲言事,輒醉不能見,時不得已,乃令侍中坐帷內與語。諸將識非更始聲,出皆怨曰:「成敗未可知,遽自縱放若此!」韓夫人尤嗜酒,每侍飲,見常侍奏事,輒怒曰:「帝方對我飲,正用此時持事來乎!」起,扺破書案。趙萌專權,威福自己。郎吏有說萌放縱者,更始怒,拔劍擊之。自是無復敢言。萌私忿侍中,引下斬之,更始救請,不從。時李軼、朱鮪擅命山東,王匡、張卬橫暴三輔。其所授官爵者,皆群小賈豎,或有膳夫庖人,多著繡面衣、錦蔥、襜褕、諸于,罵詈道中。長安為之語曰:「灶下養,中郎將。爛羊胃,騎都尉。爛羊頭,關內侯。」

軍帥將軍豫章李淑上書諫曰:「方今賊寇始誅,王化未行,百官有司宜慎其任。夫三公上應台宿,九卿下括河海,故天工人其代之。陛下定業,雖因下江、平林之埶,斯蓋臨時濟用,不可施之既安。宜釐改制度,更延英俊,因才授爵,以匡王國。今公卿大位莫非戎陳,尚書顯官皆出庸伍,資亭長、賊捕之用,而當輔佐綱維之任。唯名與器,聖人所重。今以所重加非其人,望其毗益萬分,興化致理,譬猶緣木求魚,升山採珠。海內望此,有以闚度漢祚。臣非有憎疾以求進也,但為階下惜此舉厝。敗材傷錦,所宜至慮。惟割既往謬妄之失,思隆周文濟濟之美。」更始怒,繫淑詔獄。自是關中離心,四方怨叛。諸將出征,各自專置牧守,州郡交錯,不知所從。

十二月,赤眉西入關。

三年正月,平陵人方望立前孺子劉嬰為天子。初,望見更始政亂,度其必敗,謂安陵人弓林等曰:「前定安公嬰,平帝之嗣,雖王莽篡奪,而嘗為漢主。今皆云劉氏真人,當更受命,欲共定大功,何如?」林等然之,乃於長安求得嬰,將至臨涇立之。聚黨數千人,望為丞相,林為大司馬。更始遣李松與討難將軍蘇茂等擊破,皆斬之。又使蘇茂拒赤眉於弘農,茂軍敗,死者千餘人。

三月,遣李松會朱鮪與赤眉戰於蓩鄉,松等大敗,棄軍走,死者三萬餘人。

時王匡、張卬守河東,為鄧禹所破,還奔長安。卬與諸將議曰:「赤眉近在鄭、華陰閒,旦暮且至。今獨有長安,見滅不久,不如勒兵掠城中以自富,轉攻所在,東歸南陽,收宛王等兵。事若不集,復入湖池中為盜耳。」申屠建、廖湛等皆以為然,共入說更始。更始怒不應,莫敢復言。及赤眉立劉盆子,更始使王匡、陳牧、成丹、趙萌屯新豐,李松軍掫,以拒之。

張卬、廖湛、胡殷、申屠建等與御史大夫隗囂合

謀,欲以立秋日貙膢時共劫更始,俱成前計。侍中劉能卿知其謀,以告之。更始託病不出,召張卬等。卬等皆入,將悉誅之,唯隗囂不至。更始狐疑,使卬等四人且待於外廬。卬與湛、殷疑有變,遂突出,獨申屠建在,更始斬之。卬與湛、殷遂勒兵掠東西市。昏時,燒門入,戰於宮中,更始大敗。明旦,將妻子車騎百餘,東奔趙萌於新豐。

更始復疑王匡、陳牧、成丹與張卬等同謀,乃並召入。牧、丹先至,即斬之。王匡懼,將兵入長安,與張卬等合。李松還從更始,與趙萌共攻匡、卬於城內。連戰月餘,匡等敗走,更始徙居長信宮。赤眉至高陵,匡等迎降之,遂共連兵而進。更始守城,使李松出戰,敗,死者二千餘人,赤眉生得松。時松弟汎為城門校尉,赤眉使使謂之曰:「開城門,活汝兄。」汎即開門。九月,赤眉入城。更始單騎走,從廚城門出。諸婦女從後連呼曰:「陛下,當下謝城!」更始即下拜,復上馬去。

初,侍中劉恭以赤眉立其弟盆子,自繫詔獄;聞更始敗,乃出,步從至高陵,止傳舍。右輔都尉嚴本恐失更始為赤眉所誅,將兵在外,號為屯衛而實囚之。赤眉下書曰:「聖公降者,封長沙王。過二十日,勿受。」更始遣劉恭請降,赤眉使其將謝祿往受之。十月更始遂隨祿肉袒詣長樂宮,上璽綬於盆子。赤眉坐更始,置庭中,將殺之。劉恭、謝祿為請,不能得,遂引更始出。劉恭追呼曰:「臣誠力極,請得先死。」拔劍欲自刎,赤眉帥樊崇等遽共救止之,乃赦更始,封為畏威侯。劉恭復為固請,竟得封長沙王。更始常依謝祿居,劉恭亦擁護之。

三輔苦赤眉暴虐,皆憐更始,而張卬等以為慮,謂祿曰:「今諸營長多欲篡聖公者。一旦失之,合兵攻公,自滅之道也。」於是祿使從兵與更始共牧馬於郊下,因令縊殺之。劉恭夜往收臧其屍。光武聞而傷焉,詔大司徒鄧禹葬之於霸陵。

有三子:求,歆,鯉。明年夏,求兄弟與母東詣洛陽,帝封求為襄邑侯,奉更始祀;歆為穀孰侯,鯉為壽光侯。求後徙封成陽侯。求卒,子巡嗣,復徙封灌澤侯。巡卒,子姚嗣。

論曰:周武王觀兵孟津,退而還師,以為紂未可伐,斯時有未至者也。漢起,驅輕黠烏合之眾,不當天下萬分之一,而旌旃之所撝及,書文之所通被,莫不折戈頓顙,爭受職命。非唯漢人餘思,固亦幾運之會也。夫為權首,鮮或不及。陳、項且猶未興,況庸庸者乎!

劉盆子者,太山式人,城陽景王章之後也。祖父憲,元帝時封為式侯,父萌嗣。王莽篡位,國除,因為式人焉。

天鳳元年,琅邪海曲有呂母者,子為縣吏,犯小罪,宰論殺之。呂母怨宰,密聚客,規以報仇。母家素豐,貲產數百萬,乃益釀醇酒,買刀劍衣服。少年來酤者,皆賒與之,視其乏者,輒假衣裳,不問多少。數年,財用稍盡,少年欲相與償之。呂母垂泣曰:「所以厚諸君者,非欲求利,徒以縣宰不道,枉殺吾子,欲為報怨耳。諸君寧肯哀之乎!」少年壯其意,又素受恩,皆許諾。其中勇士自號猛虎,遂相聚得數十百人,因與呂母入海中,招合亡命,眾至數千。呂母自稱將軍,引兵還攻破海曲,執縣宰。諸吏叩頭為宰請。母曰:「吾子犯小罪,不當死,而為宰所殺。殺人當死,又何請乎?」遂斬之,以其首祭子冢,復還海中。

後數歲,琅邪人樊崇起兵於莒,眾百餘人,轉入太山,自號三老。時青、徐大飢,寇賊蜂起,眾盜以崇勇猛,皆附之,一歲閒至萬餘人。崇同郡人逄安,東海人徐宣、謝祿、楊音,各起兵,合數萬人,復引從崇。共還攻莒,不能下,轉掠至姑幕,因擊王莽探湯侯田況,大破之,殺萬餘人,遂北入青州,所過虜掠。還至太山,留屯南城。初,崇等以困窮為寇,無攻城徇地之計。眾既寖盛,乃相與為約:殺人者死,傷人者償創。以言辭為約束,無文書、旌旗、部曲、號令。其中最尊者號三老,次從事,次卒吏,汎相稱曰臣人。王莽遣平均公廉丹、太師王匡擊之。崇等欲戰,恐其眾與莽兵亂,乃皆朱其眉以相識別,由是號曰赤眉。赤眉遂大破丹、匡軍,殺萬餘人,追至無鹽,廉丹戰死,王匡走。崇又引其兵十餘萬,復還圍莒,數月。或說崇曰:「莒,父母之國,奈何攻之?」乃解去。時呂母病死,其眾分入赤眉、青犢、銅馬中。赤眉遂寇東海,與王莽沂平大尹戰,敗,死者數千人,乃引去,掠楚、沛、汝南、潁川,還入陳留,攻拔魯城,轉至濮陽。

會更始都洛陽,遣使降崇。崇等聞漢室復興,即留其兵,自將渠帥二十餘人,隨使者至洛陽降更始,皆封為列侯。崇等即未有國邑,而留眾稍有離叛,乃遂亡歸其營,將兵入潁川,分其眾為二部,崇與逄安為一部,徐宣、謝祿、楊音為一部。崇、安攻拔長社,南擊宛,斬縣令;而宣、祿等亦拔陽翟,引之梁,擊殺河南太守。赤眉眾雖數戰勝,而疲敝厭兵,皆日夜愁泣,思欲東歸。崇等計議,慮眾東向必散,不如西攻長安。更始二年冬,崇、安自武關,宣等從陸渾關,兩道俱入。三年正月,俱至弘農,與更始諸將連戰剋勝,眾遂大集。乃分萬人為一營,凡三十營,營置三老、從事各一人。進至華陰。

軍中常有齊巫鼓舞祠城陽景王,以求福助。巫狂言景王大怒,曰:「當為縣官,何故為賊?」有笑巫者輒病,軍中驚動。時方望弟陽怨更始殺其兄,乃逆說崇等曰:「更始荒亂,政令不行,故使將軍得至於此。今將軍擁百萬之眾,西向帝城,而無稱號,名為群賊,不可以久。不如立宗室,挾義誅伐。以此號令,誰敢不服?」崇等以為然,而巫言益甚。前及鄭,乃相與議曰:「今迫近長安,而鬼神如此,當求劉氏共尊立之。」六月,遂立盆子為帝,自號建世元年。

初,赤眉過式,掠盆子及二兄恭、茂,皆在軍中。恭少習尚書,略通大義。及隨崇等降更始,即封為式侯。以明經數言事,拜侍中,從更始在長安。盆子與茂留軍中,屬右校卒吏劉俠卿,主芻牧牛,號曰牛吏。及崇等欲立帝,求軍中景王後者,得七十餘人,唯盆子與茂及前西安侯劉孝最為近屬。崇等議曰:「聞古天子將兵稱上將軍。」乃書札為符曰「上將軍」,又以兩空札置笥中,遂於鄭北設壇場,祠城陽景王。諸三老、從事皆大會陛下,列盆子等三人居中立,以年次探札。盆子最幼,後探得符,諸將乃皆稱臣拜。盆子時年十五,被髮徒跣,敝衣赭汗,見眾拜,恐畏欲啼。茂謂曰:「善藏符。」盆子即齧折棄之,復還依俠卿。俠卿為制絳單衣、半頭赤幘、直綦履,乘軒車大馬,赤屏泥,絳襜絡,而猶從牧兒遨。

崇雖起勇力而為眾所宗,然不知書數。徐宣故縣獄吏,能通易經。遂共推宣為丞相,崇御史大夫,逄安左大司馬,謝祿右大司馬,自楊音以下皆為列卿。

軍及高陵,與更始叛將張卬等連和,遂攻東都門,入長安城,更始來降。

盆子居長樂宮,諸將日會論功,爭言讙呼,拔劍擊柱,不能相一。三輔郡縣營長遣使貢獻,兵士輒剽奪之。又數虜暴吏民,百姓保壁,由是皆復固守。至臘日,崇等乃設樂大會,盆子坐正殿,中黃門持兵在後,公卿皆列坐殿上。酒未行,其中一人出刀筆書謁欲賀,其餘不知書者起請之,各各屯聚,更相背向。大司農楊音按劍罵曰:「諸卿皆老傭也!今日設君臣之禮,反更殽亂,兒戲尚不如此,皆可格殺!」更相辯鬥,而兵眾遂各踰宮斬關,入掠酒肉,互相殺傷。衛尉諸葛稚聞之,勒兵入,格殺百餘人,乃定。盆子惶恐,日夜啼泣,獨與中黃門共臥起,唯得上觀閣而不聞外事。

時掖庭中宮女猶有數百千人,自更始敗後,幽閉殿內,掘庭中蘆菔根,捕池魚而食之,死者因相埋於宮中。有故祠甘泉樂人,尚共擊鼓歌舞,衣服鮮明,見盆子叩頭言飢。盆子使中黃門稟之米,人數斗。後盆子去,皆餓死不出。

劉恭見赤眉眾亂,知其必敗,自恐兄弟俱禍,密教盆子歸璽綬,習為辭讓之言。建武二年正月朔,崇等大會,劉恭先曰:「諸君共立恭弟為帝,德誠深厚。立且一年,肴亂日甚,誠不足以相成。恐死而無所益,願得退為庶人,更求賢知,唯諸君省察。」崇等謝曰:「此皆崇等罪也。」恭復固請。或曰:「此寧式侯事邪!」恭惶恐起去。盆子乃下床解璽綬,叩頭曰:「今設置縣官而為賊如故。吏人貢獻,輒見剽劫,流聞四方,莫不怨恨,不復信向。此皆立非其人所致,願乞骸骨,避賢聖。必欲殺盆子以塞責者,無所離死。誠冀諸君肯哀憐之耳!」因涕泣噓唏。崇等及會者數百人,莫不哀憐之,乃皆避席頓首曰:「臣無狀,負陛下。請自今已後,不敢復放縱。」因共抱持盆子,帶以璽綬。盆子號呼不得已。既罷出,各閉營自守,三輔翕然,稱天子聰明。百姓爭還長安,市里且滿。

得二十餘日,赤眉貪財物,復出大掠。城中糧食盡,遂收載珍寶,因大縱火燒宮室,引兵而西。過祠南郊,車甲兵馬最為猛盛,眾號百萬。盆子乘王車,駕三馬,從數百騎。乃自南山轉掠城邑,與更始將軍嚴春戰於郿,破春,殺之,遂入安定、北地。至陽城、番須中,逢大雪,坑谷皆滿,士多凍死,乃復還,發掘諸陵,取其寶貨,遂汙辱呂后屍。凡賊所發,有玉匣殮者率皆如生,故赤眉得多行婬穢。大司徒鄧禹時在長安,遣兵擊之於郁夷,反為所敗,禹乃出之雲陽。九月,赤眉復入長安,止桂宮。

時漢中賊延岑出散關,屯杜陵,逄安將十餘萬人擊之。鄧禹以逄安精兵在外,唯盆子與羸弱居城中,乃自往攻之。會謝祿救至,夜戰槁街中,禹兵敗走。延岑及更始將軍李寶合兵數萬人,與逄安戰於杜陵。岑等大敗,死者萬餘人,寶遂降安,而延岑收散卒走。寶乃密使人謂岑曰:「子努力還戰,吾當於內反之,表裏合勢,可大破也。」岑即還挑戰,安等空營擊之,寶從後悉拔赤眉旌幟,更立己幡旗。安等戰疲還營,見旗幟皆白,大驚亂走,自投川谷,死者十餘萬,逄安與數千人脫歸長安。時三輔大飢,人相食,城郭皆空,白骨蔽野,遺人往往聚為營保,各堅守不下。赤眉虜掠無所得,十二月,乃引而東歸,眾尚二十餘萬,隨道復散。

光武乃遣破姦將軍侯進等屯新安,建威大將軍耿弇等屯宜陽,分為二道,以要其還路。敕諸將曰:「賊若東走,可引宜陽兵會新安;賊若南走,可引新安兵會宜陽。」明年正月,鄧禹自河北度,擊赤眉於湖,禹復敗走,赤眉遂出關南向。征西大將軍馮異破之於崤底。帝聞,乃自將幸宜陽,盛兵以邀其走路。

赤眉忽遇大軍,驚震不知所為,乃遣劉恭乞降,曰:「盆子將百萬眾降,陛下何以待之?」帝曰:「待汝以不死耳。」樊崇乃將盆子及丞相徐宣以下三十餘人肉袒降。上所得傳國璽綬,更始七尺寶劍及玉璧各一。積兵甲宜陽城西,與熊耳山齊。帝令縣廚賜食,眾積困餧,十餘萬人皆得飽飫。明旦,大陳兵馬臨洛水,令盆子君臣列而觀之。謂盆子曰:「自知當死不?」對曰:「罪當應死,猶幸上憐赦之耳。」帝笑曰:「兒大黠,宗室無蚩者。」又謂崇等曰:「得無悔降乎?朕今遣卿歸營勒兵,鳴鼓相攻,決其勝負,不欲強相服也。」徐宣等叩頭曰:「臣等出長安東都門,君臣計議,歸命聖德。百姓可與樂成,難與圖始,故不告眾耳。今日得降,猶去虎口歸慈母,誠歡誠喜,無所恨也。」帝曰:「卿所謂鐵中錚錚,傭中佼佼者也。」又曰:「諸卿大為無道,所過皆夷滅老弱,溺社稷,汙井灶。然猶有三善:攻破城邑,周遍天下,本故妻婦無所改易,是一善也;立君能用宗室,是二善也;餘賊立君,迫急皆持其首降,自以為功,諸卿獨完全以付朕,是三善也。」乃令各與妻子居洛陽,賜宅人一區,田二頃。

其夏,樊崇、逄安謀反,誅死。楊音在長安時,遇趙王良有恩,賜爵關內侯,與徐宣俱歸鄉里,卒於家。劉恭為更始報殺謝祿,自繫獄,赦不誅。

帝憐盆子,賞賜甚厚,以為趙王郎中。後病失明,賜滎陽均輸官地,以為列肆,使食其稅終身。

贊曰:聖公靡聞,假我風雲。始順歸歷,終然崩分。赤眉阻亂,盆子探符。雖盜皇器,乃食均輸。


\end{pinyinscope}