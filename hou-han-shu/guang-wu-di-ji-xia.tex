\article{光武帝紀下}

\begin{pinyinscope}
六年春正月丙辰,改舂陵鄉為章陵縣。世世復傜役,比豐、沛,無有所豫。

辛酉,詔曰:「往歲水旱蝗蟲為災,穀價騰躍,人用困乏。朕惟百姓無以自贍,惻然愍之。其命郡國有穀者,給稟高年、鰥、寡、孤、獨及篤缮、無家屬貧不能自存者,如律。二千石勉加循撫,無令失職。」

揚武將軍馬成等拔舒,獲李憲。

二月,大司馬吳漢拔朐,獲董憲、龐萌,山東悉平。諸將還京師,置酒賞賜。

三月,公孫述遣將任滿寇南郡。

夏四月丙子,幸長安,始謁高廟,遂有事十一陵。

遣虎牙大將軍蓋延等七將軍從隴道伐公孫述。

五月己未,至自長安。

隗囂反,蓋延等因與囂戰於隴阺,諸將敗績。

辛丑,詔曰:「惟天水、隴西、安定、北地吏人為隗囂所詿誤者,又三輔遭難赤眉,有犯法不道者,自殊死以下,皆赦除之。」

六月辛卯,詔曰:「夫張官置吏,所以為人也。今百姓遭難,戶口耗少,而縣官吏職所置尚繁,其令司隸、州牧各實所部,省減吏員。縣國不足置長吏可并合者,上大司徒、大司空二府。」於是條奏并省四百餘縣,吏職減損,十置其一。

代郡太守劉興擊盧芳將賈覽於高柳,戰歿。

初,樂浪人王調據郡不服。秋,遣樂浪太守王遵擊之,郡吏殺調降。

遣前將軍李通率二將軍,與公孫述將戰於西城,破之。

夏,蝗。

秋九月庚子,赦樂浪謀反大逆殊死已下。

丙寅晦,日有食之。

冬十月丁丑,詔曰:「吾德薄不明,寇賊為害,彊弱相陵,元元失所。《詩》云:『日月告凶,不用其行。』永念厥咎,內疚於心。其敕公卿舉賢良、方正各一人;百僚並上封事,無有隱諱;有司修職,務遵法度。」

十一月丁卯,詔王莽時吏人沒入為奴婢不應舊法者,皆免為庶人。

十二月壬辰,大司空宋弘免。

癸巳,詔曰:「頃者師旅未解,用度不足,故行什一之稅。今軍士屯田,糧儲差積。其令郡國收見田租三十稅一,如舊制。」

隗囂遣將行巡寇扶風,征西大將軍馮異拒破之。

是歲,初罷郡國都尉官。始遣列侯就國。匈奴遣使來獻,使中郎將報命。

七年春正月丙申,詔中都官、三輔、郡、國出繫囚,非犯殊死,皆一切勿案其罪。見徒免為庶民。耐罪亡命,吏以文除之。

又詔曰:「世以厚葬為德,薄終為鄙,至于富者奢僭,貧者單財,法令不能禁,禮義不能止,倉卒乃知其咎。其布告天下,令知忠臣、孝子、慈兄、悌弟薄葬送終之義。」

二月辛巳,罷護漕都尉官。

三月丁酉,詔曰:「今國有眾軍,並多精勇,宜且罷輕車、騎士、材官、樓船士及軍假吏,令還復民伍。」

公孫述立隗囂為朔寧王。

癸亥晦,日有食之,避正殿,寑兵,不聽事五日。詔曰:「吾德薄致災,謫見日月,戰慄恐懼,夫何言哉!今方念愆,庶消厥咎。其令有司各修職任,奉遵法度,惠茲元元。百僚各上封事,無有所諱。其上書者,不得言聖。」

夏四月壬午,詔曰:「比陰陽錯謬,日月薄食。百姓有過,在予一人,大赦天下。公、卿、司隸、州牧舉賢良、方正各一人,遣詣公車,朕將覽試焉。」

五月戊戌,前將軍李通為大司空。

甲寅,詔吏人遭饑亂及為青、徐賊所略為奴婢下妻,欲去留者,恣聽之。敢拘制不還,以賣人法從事。

是夏,連雨水。

漢忠將軍王常為橫野大將軍。

八月丁亥,封前河閒王邵為河閒王。

隗囂寇安定,征西大將軍馮異、征虜將軍祭遵擊卻之。

冬,盧芳所置朔方太守田颯、雲中太守喬扈各舉郡降。

是歲,省長水、射聲二校尉官。

八年春正月,中郎將來歙襲略陽,殺隗囂守將而據其城。

夏四月,司隸校尉傅抗下獄死。

隗囂攻來歙,不能下。閏月,帝自征囂,河西太守竇融率五郡太守與車駕會高平。隴右潰,隗囂奔西城,遣大司馬吳漢、征南大將軍岑彭圍之;進幸上邽,不降,命虎牙大將軍蓋延、建威大將軍耿弇攻之。

潁川盜賊寇沒屬縣,河東守守兵亦叛,京師騷動。

秋,大水。

八月,帝自上邽晨夜東馳。九月乙卯,車駕還宮。

庚申,帝自征潁川盜賊,皆降。

安丘侯張步叛歸琅邪,琅邪太守陳俊討獲之。

戊寅,至自潁川。

冬十月丙午,幸懷。十一月乙丑,至自懷。

公孫述遣兵救隗囂,吳漢、蓋延等還軍長安。天水、隴西復反歸囂。

十二月,高句麗王遣使奉貢。

是歲大水。

九年春正月,隗囂病死,其將王元、周宗復立囂子純為王。

徙鴈門吏人於太原。

三月辛亥,初置青巾左校尉官。

公孫述遣將田戎、任滿據荊門。

夏六月丙戌,幸緱氏,登轘轅。

遣大司馬吳漢率四將軍擊盧芳將賈覽於高柳,戰不利。

秋八月,遣中郎將來歙監征西大將軍馮異等五將軍討隗純於天水。

驃騎大將軍杜茂與賈覽戰於繁畤,茂軍敗績。

是歲,省關都尉,復置護羌校尉官。

十年春正月,大司馬吳漢率捕虜將軍王霸等五將軍擊賈覽於高柳,匈奴遣騎救覽,諸將與戰,卻之。

修理長安高廟。

夏,征西大將軍馮異破公孫述將趙匡於天水,斬之。征西大將軍馮異薨。

秋八月己亥,幸長安,祠高廟,遂有事十一陵。

戊戌,進幸汧。隗囂將高峻降。

冬十月,中郎將來歙等大破隗純於落門,其將王元奔蜀,純與周宗降,隴右平。

先零羌寇金城、隴西,來歙率諸將擊羌於五谿,大破之。

庚寅,車駕還宮。

是歲,省定襄郡,徙其民於西河。泗水王歙薨。淄川王終薨。

十一年春二月己卯,詔曰:「天地之性人為貴。其殺奴婢,不得減罪。」

己酉,幸南陽;還,幸章陵,祠園陵。

城陽王祉薨。

庚午,車駕還宮。

閏月,征南大將軍岑彭率三將軍與公孫述將田戎、任滿戰於荊門,大破之,獲任滿。威虜將軍馮駿圍田戎於江州,岑彭遂率舟師伐公孫述,平巴郡。

夏四月丁卯,省大司徒司直官。

先零羌寇臨洮。

六月,中郎將來歙率揚武將軍馬成破公孫述將王元、環安於下辯。安遣閒人刺殺中郎將來歙。帝自將征公孫述。秋七月,次長安。八月,岑彭破公孫述將侯丹於黃石。輔威將軍臧宮與公孫述將延岑戰於沈水,大破之。王元降。至自長安。

癸亥,詔曰:「敢灸灼奴婢,論如律,免所灸灼者為庶民。」

冬十月壬午,詔除奴婢射傷人棄市律。

公孫述遣閒人刺殺征南大將軍岑彭。

馬成平武都,因隴西太守馬援擊破先零羌,徙致天水、隴西、扶風。

十二月,大司馬吳漢率舟師伐公孫述。

是歲,省朔方牧,并并州。初斷州牧自還奏事。

十二年春正月,大司馬吳漢與公孫述將史興戰於武陽,斬之。

三月癸酉,詔隴、蜀民被略為奴婢自訟者,及獄官未報,一切免為庶民。

夏,甘露降南行唐。六月,黃龍見東阿。

秋七月,威虜將軍馮駿拔江州,獲田戎。九月,吳漢大破公孫述將謝豐于廣都,斬之。輔威將軍臧宮拔涪城,斬公孫恢。

大司空李通罷。

冬十一月戊寅,吳漢、臧宮與公孫述戰於成都,大破之。述被創,夜死。辛巳,吳漢屠成都,夷述宗族及延岑等。

十二月辛卯,揚武將軍馬成行大司空事。

是歲,九真徼外蠻夷張遊率種人內屬,封為歸漢里君。省金城郡屬隴西。參狼羌寇武都,隴西太守馬援討降之。詔邊吏力不足戰則守,追虜料敵不拘以逗留法。橫野大將軍王常薨。遣驃騎大將軍杜茂將眾郡施刑屯北邊,築亭候,修烽燧。

十三年春正月庚申,大司徒侯霸薨。

戊子,詔曰:「往年已敕郡國,異味不得有所獻御,今猶未止,非徒有豫養導擇之勞,至乃煩擾道上,疲費過所。其令太官勿復受。明敕下以遠方口實所以薦宗廟,自如舊制。」

二月,遣捕虜將軍馬武屯虖沱河以備匈奴。盧芳自五原亡入匈奴。

丙辰,詔曰:「長沙王興、真定王得、河閒王邵、中山王茂,皆襲爵為王,不應經義。其以興為臨湘侯,得為真定侯,邵為樂成侯,茂為單父侯。」其宗室及絕國封侯者凡一百三十七人。丁巳,降趙王良為趙公,太原王章為齊公,魯王興為魯公。庚午,以殷紹嘉公孔安為宋公,周承休公姬常為衛公。省并西京十三國:廣平屬鉅鹿,真定屬常山,河閒屬信都,城陽屬琅邪,泗水屬廣陵,淄川屬高密,膠東屬北海,六安屬廬江,廣陽屬上谷。

三月辛未,沛郡太守韓歆為大司徒。丙子,行大司空馬成罷。

夏四月,大司馬吳漢自蜀還京師,於是大饗將士,班勞策勳。功臣增邑更封,凡三百六十五人。其外戚恩澤封者四十五人。罷左右將軍官。建威大將軍耿弇罷。

益州傳送公孫述瞽師、郊廟樂器、葆車、輿輦、

於是法物始備。時兵革既息,天下少事,文書調役,務從簡寡,至乃十存一焉。

甲寅,冀州牧竇融為大司空。

五月,匈奴寇河東。

秋七月,廣漢徼外白馬羌豪率種人內屬。

九月,日南徼外蠻夷獻白雉、白兔。

冬十二月甲寅,詔益州民自八年以來被略為奴婢者,皆一切免為庶民;或依託為人下妻,欲去者,恣聽之;敢拘留者,比青、徐二州以略人法從事。

復置金城郡。

十四年春正月,起南宮前殿。

匈奴遣使奉獻,使中郎將報命。

夏四月辛巳,封孔子後志為褒成侯。

越巂人任貴自稱太守,遣使奉計。

秋九月,平城人賈丹殺盧芳將尹由來降。

是歲,會稽大疫。莎車國、鄯善國遣使奉獻。

十二月癸卯,詔益、涼二州奴婢,自八年以來自訟在所官,一切免為庶民,賣者無還直。

十五年春正月辛丑,大司徒韓歆免,自殺。

丁未,有星孛於昴。

汝南太守歐陽歙為大司徒。建義大將軍朱祐罷。

丁未,有星孛於營室。

二月,徙鴈門、代郡、上谷三郡民,置常關、居庸關以東。

初,巴蜀既平,大司馬吳漢上書請封皇子,不許,重奏連歲。三月,乃詔群臣議。大司空融、固始侯通、膠東侯復、高密侯禹、太常登等奏議曰:「古者封建諸侯,以藩屏京師。周封八百,同姓諸姬並為建國,夾輔王室,尊事天子,享國永長,為後世法。故《詩》云:『大啟爾宇,為周室輔。』高祖聖德,光有天下,亦務親親,封立兄弟諸子,不違舊章。陛下德橫天地,興復宗統,褒德賞勳,親睦九族,功臣宗室,咸蒙封爵,多受廣地,或連屬縣。今皇子賴天,能勝衣趨拜,陛下恭謙克讓,抑而未議,群臣百姓,莫不失望。宜因盛夏吉時,定號位,以廣藩輔,明親親,尊宗廟,重社稷,應古合舊,厭塞眾心。臣請大司空上輿地圖,太常擇吉日,具禮儀。」制曰:「可。」

夏四月戊申,以太牢告祠宗廟。丁巳,使大司空融告廟,封皇子輔為右翊公,英為楚公,陽為東海公,康為濟南公,蒼為東平公,延為淮陽公,荊為山陽公,衡為臨淮公,焉為左翊公,京為琅邪公。癸丑,追謚兄伯升為齊武公,兄仲為魯哀公。

六月庚午,復置屯騎、長水、射聲三校尉官;改青巾左校尉為越騎校尉。

詔下州郡檢覈墾田頃畝及戶口年紀,又考實二千石長吏阿枉不平者。

冬十一月甲戌,大司徒歐陽歙下獄死。十二月庚午,關內侯戴涉為大司徒。

盧芳自匈奴入居高柳。

是歲,驃騎大將軍杜茂免。虎牙大將軍蓋延薨。

十六年春二月,交阯女子徵側反,略有城邑。

三月辛丑晦,日有蝕之。

秋九月,河南尹張伋及諸郡守十餘人,坐度田不實,皆下獄死。

郡國大姓及兵長、群盜處處並起,攻劫在所,害殺長吏。郡縣追討,到則解散,去復屯結。青、徐、幽、冀四州尤甚。冬十月,遣使者下郡國,聽群盜自相糾擿,五人共斬一人者,除其罪。吏雖逗留回避故縱者,皆勿問,聽以禽討為效。其牧守令長坐界內盜賊而不收捕者,又以畏懦捐城委守者,皆不以為負,但取獲賊多少為殿最,唯蔽匿者乃罪之。於是更相追捕,賊並解散。徙其魁帥於它郡,賦田受稟,使安生業。自是牛馬放牧,邑門不閉。

盧芳遣使乞降。十二月甲辰,封芳為代王。

初,王莽亂後,貨幣雜用布、帛、金、粟。是歲,始行五銖錢。

十七年春正月,趙公良薨。

二月乙亥晦,日有食之。

夏四月乙卯,南巡狩,皇太子及右翊公輔、楚公英、東海公陽、濟南公康、東平公蒼從,幸潁川,進幸葉、章陵。五月乙卯,車駕還宮。

六月癸巳,臨淮公衡薨。

秋七月,妖巫李廣等群起據皖城,遣虎賁中郎將馬援、驃騎將軍段志討之。九月,破皖城,斬李廣等。

冬十月辛巳,廢皇后郭氏為中山太后,立貴人陰氏為皇后。進右翊公輔為中山王,食常山郡。其餘九國公,皆即舊封進爵為王。

甲申,幸章陵。脩園廟,祠舊宅,觀田廬,置酒作樂,賞賜。時宗室諸母因酣悅,相與語曰:「文叔少時謹信,與人不款曲,唯直柔耳。今乃能如此!」帝聞之,大笑曰:「吾理天下,亦欲以柔道行之。」乃悉為舂陵宗室起祠堂。有五鳳皇見於潁川之郟縣。十二月,至自章陵。

是歲,莎車國遣使貢獻。

十八年春二月,蜀郡守將史歆叛,遣大司馬吳漢率二將軍討之,圍成都。

甲寅,西巡狩,幸長安。三月壬午,祠高廟,遂有事十一陵。歷馮翊界,進幸蒲阪,祠后土。夏四月甲戌車駕還宮。

癸酉,詔曰:「今邊郡盜穀五十斛,罪至於死,開殘吏妄殺之路,其蠲除此法,同之內郡。」

遣伏波將軍馬援率樓船將軍段志等擊交阯賊徵側

等。

戊申,幸河內。戊子,至自河內。

五月,旱。

盧芳復亡入匈奴。

秋七月,吳漢拔成都,斬史歆等。壬戌,赦益州所部殊死已下。

冬十月庚辰,幸宜城。還,祠章陵。十二月乙丑,車駕還宮。

是歲,罷州牧,置刺史。

十九年春正月庚子,追尊孝宣皇帝曰中宗。始祠昭帝、元帝於太廟,成帝、哀帝、平帝於長安,舂陵節侯以下四世於章陵。

妖巫單臣、傅鎮等反,據原武,遣太中大夫臧宮圍之。夏四月,拔原武,斬臣、鎮等。

伏波將軍馬援破交阯,斬徵側等。因擊破九真賊都陽等,降之。

閏月戊申,進趙、齊、魯三國公爵為王。

六月戊申,詔曰:「春秋之義,立子以貴。東海王陽,皇后之子,宜承大統。皇太子彊,崇執謙退,願備藩國。父子之情,重久違之。其以彊為東海王,立陽為皇太子,改名莊。」

秋九月,南巡狩。壬申,幸南陽,進幸汝南南頓縣舍,置酒會,賜吏人,復南頓田租歲。父老前叩頭言:「皇考居此日久,陛下識知寺舍,每來輒加厚恩,願賜復十年。」帝曰:「天下重器,常恐不任,日復一日,安敢遠期十歲乎?」吏人又言:「陛下實惜之,何言謙也?」帝大笑,復增一歲。進幸淮陽、梁、沛。

西南夷寇益州郡,遣武威將軍劉尚討之。越巂太守任貴謀叛,十二月,劉尚襲貴,誅之。

是歲,復置函谷關都尉。修西京宮室。

二十年春二月戊子,車駕還宮。

夏四月庚辰,大司徒戴涉下獄死。大司空竇融免。

五月辛亥,大司馬吳漢薨。

匈奴寇上黨、天水,遂至扶風。

六月庚寅,廣漢太守蔡茂為大司徒,太僕朱浮為大司空。壬辰,左中郎將劉隆為驃騎將軍,行大司馬事。

乙未,徙中山王輔為沛王。

秋,東夷韓國人率眾詣樂浪內附。

冬十月,東巡狩。甲午,幸魯,進幸東海、楚、沛國。

十二月,匈奴寇天水。

壬寅,車駕還宮。

是歲,省五原郡,徙其吏人置河東。復濟陽縣傜役六歲。

二十一年春正月,武威將軍劉尚破益州夷,平之。

夏四月,安定屬國胡叛,屯聚青山,遣將兵長史陳訢討平之。

秋,鮮卑寇遼東,遼東太守祭肜大破之。

冬十月,遣伏波將軍馬援出塞擊烏桓,不克。

匈奴寇上谷、中山。

其冬,鄯善王、車師王等十六國皆遣子入侍奉獻,願請都護。帝以中國初定,未遑外事,乃還其侍子,厚加賞賜。

二十二年春閏月丙戌,幸長安,祠高廟,遂有事十一陵。二月己巳,至自長安。

夏五月乙未晦,日有食之。

秋七月,司隸校尉蘇鄴下獄死。

九月戊辰,地震裂。制詔曰:「日者地震,南陽尤甚。夫地者,任物至重,靜而不動者也。而今震裂,咎在君上。鬼神不順無德,災殃將及吏人,朕甚懼焉。其令南陽勿輸今年田租芻稿。遣謁者案行,其死罪繫囚在戊辰以前,減死罪一等;徒皆弛解鉗,衣絲絮。賜郡中居人壓死者棺錢,人三千。其口賦逋稅而廬宅尤破壞者,勿收責。吏人死亡,或在壞垣毀屋之下,而家羸弱不能收拾者,其以見錢穀取傭,為尋求之。」

冬十月壬子,大司空朱浮免。癸丑,光祿勳杜林為大司空。

是歲,齊王章薨。青州蝗。匈奴薁鞬日逐王比遣使詣漁陽請和親,使中郎將李茂報命。烏桓擊破匈奴,匈奴北徙,幕南地空。詔罷諸邊郡亭候吏卒。

二十三年春正月,南郡蠻叛,遣武威將軍劉尚討破之,徙其種人於江夏。

夏五月丁卯,大司徒蔡茂薨。

秋八月丙戌,大司空杜林薨。

九月辛未,陳留太守玉況為大司徒。

冬十月丙申,太僕張純為大司空。

高句麗率種人詣樂浪內屬。

十二月,武陵蠻叛,寇掠郡縣,遣劉尚討之,戰於沅水,尚軍敗歿。

是歲,匈奴薁鞬日逐王比率部曲遣使詣西河內附。

二十四年春正月乙亥,大赦天下。

匈奴薁鞬日逐王比遣使款五原塞,求扞禦北虜。

秋七月,武陵蠻寇臨沅,遣謁者李嵩、中山太守馬成討蠻,不克,於是伏波將軍馬援率四將軍討之。

詔有司申明舊制阿附蕃王法。

冬十月,匈奴薁鞬日逐王比自立為南單于,於是分為南、北匈奴。

二十五春正月,遼東徼外貊人寇右北平、漁陽、上谷、太原,遼東太守祭肜招降之。烏桓大人來朝。

南單于遣使詣闕貢獻,奉蕃稱臣;又遣其左賢王擊破北匈奴,卻地千餘里。三月,南單于遣子入侍。

戊申晦,日有食之。

伏波將軍馬援等破武陵蠻於臨沅。冬十月,叛蠻悉降。

夫餘王遣使奉獻。

是歲,烏桓大人率眾內屬,詣闕朝貢。

二十六年正月,詔有司增百官奉。其千石已上,減於西京舊制;六百石已下,增於舊秩。

初作壽陵。將作大匠竇融上言園陵廣袤,無慮所用。帝曰:「古者帝王之葬,皆陶人瓦器,木車茅馬,使後世之人不知其處。太宗識終始之義,景帝能述遵孝道,遭天下反覆,而霸陵獨完受其福,豈不美哉!今所制地不過二三頃,無為山陵,陂池裁令流水而已。」

遣中郎將段郴授南單于璽綬,令入居雲中,始置使匈奴中郎將,將兵衛護之。南單于遣子入侍,奉奏詣闕。於是雲中、五原、朔方、北地、定襄、鴈門、上谷、代八郡民歸於本土。遣謁者分將施刑補理城郭。發遣邊民在中國者,布還諸縣,皆賜以裝錢,轉輸給食。

二十七年夏四月戊午,大司徒玉況薨。

五月丁丑,詔曰:「昔契作司徒,禹作司空,皆無『大』名,其令二府去『大』。」又改大司馬為太尉。驃騎大將軍行大司馬劉隆即日罷,太僕趙憙為太尉,大司農馮勤為司徒。

益州郡徼外蠻夷率種人內屬。

北匈奴遣使詣武威乞和親。

冬,魯王興、齊王石始就國。

二十八年春正月己巳,徙魯王興為北海王,以魯國益東海。賜東海王彊虎賁、旄頭、鍾虡之樂。

夏六月丁卯,沛太后郭氏薨,因詔郡縣捕王侯賓客,坐死者數千人。

秋八月戊寅,東海王彊、沛王輔、楚王英、濟南王康、淮陽王延始就國。

冬十月癸酉,詔死罪繫囚皆一切募下蠶室,其女子宮。

北匈奴遣使貢獻,乞和親。

二十九年春二月丁巳朔,日有食之。遣使者舉冤獄,出繫囚。

庚申,賜天下男子爵,人二級;鰥、寡、孤、獨、篤缮、貧不能自存者粟,人五斛。

夏四月乙丑,詔令天下繫囚自殊死已下及徒各減本罪一等,其餘贖罪輸作各有差。

三十年春正月,鮮卑大人內屬,朝賀。

二月,東巡狩。甲子,幸魯,進幸濟南。閏月癸丑,車駕還宮。

有星孛于紫宮。

夏四月戊子,徙左翊王焉為中山王。

五月,大水。

賜天下男子爵,人二級;鰥、寡、孤、獨、篤缮、貧不能自存者粟,人五斛。

秋七月丁酉,幸魯國。復濟陽縣是年傜役。冬十一月丁酉,至自魯。

三十一年夏五月,大水。

戊辰,賜天下男子爵,人二級;鰥、寡、孤、獨、篤缮、貧不能自存者粟,人六斛。

癸酉晦,日有食之。

是夏,蝗。

秋九月甲辰,詔令死罪繫囚皆一切募下蠶室,其女子宮。

是歲,陳留雨穀,形如稗實。北匈奴遣使奉獻。

中元元年春正月,東海王彊、沛王輔、楚王英、濟南王康、淮陽王延、趙王盱皆來朝。

丁卯,東巡狩。二月己卯,幸魯,進幸太山。北海王興、齊王石朝于東嶽。辛卯,柴望岱宗,登封太山;甲午,禪于梁父。

三月戊辰,司空張純薨。

夏四月癸酉,車駕還宮。己卯,大赦天下。復嬴、博、梁父、奉高,勿出今年田租芻稿。改年為中元。

行幸長安。戊子,祀長陵。五月乙丑,至自長安。

六月辛卯,太僕馮魴為司空。

乙未,司徒馮勤薨。

是夏,京師醴泉涌出,飲之者固疾皆愈,惟眇、蹇者不瘳。又有赤草生於水崖。郡國頻上甘露。群臣奏言:「地祇靈應而朱草萌生。孝宣帝每有嘉瑞,輒以改元,神爵、五鳳、甘露、黃龍,列為年紀,蓋以感致神祇,表彰德信。是以化致升平,稱為中興。今天下清寧,靈物仍降。陛下情存損挹,推而不居,豈可使祥符顯慶,沒而無聞?宜令太史撰集,四以傳來世。」帝不納。常自謙無德,每郡國所上,輒抑而不當,故史官罕得記焉。

秋,郡國三蝗。

冬十月辛未,司隸校尉東萊李訢為司徒。

甲申,使司空告祠高廟曰:「高皇帝與群臣約,非劉氏不王。呂太后賊害三趙,專王呂氏,賴社稷之靈,祿、產伏誅,天命幾墜,危朝更安。呂太后不宜配食高廟,同祧至尊。薄太后母德慈仁,孝文皇帝賢明臨國,子孫賴福,延祚至今。其上薄太后尊號曰高皇后,配食地祇。遷呂太后廟主于園,四時上祭。」

十一月甲子晦,日有食之。

是歲,初起明堂、靈臺、辟雍,及北郊兆域。宣布圖讖於天下。復濟陽、南頓是年傜役。參狼羌寇武都,敗郡兵,隴西太守劉盱遣軍救之,及武都郡兵討叛羌,皆破之。

二年春正月辛未,初立北郊,祀后土。

東夷倭奴國王遣使奉獻。

二月戊戌,帝崩於南宮前殿,年六十二。遺詔曰:「朕無益百姓,皆如孝文皇帝制度,務從約省。刺史、二千石長吏皆無離城郭,無遣吏及因郵奏。」

初,帝在兵閒久,厭武事,且知天下疲秏,思樂息肩。自隴、蜀平後,非儆急,未嘗復言軍旅。皇太子嘗問攻戰之事,帝曰:「昔衛靈公問陳,孔子不對,此非爾所及。」每旦視朝,日仄乃罷。數引公卿、郎、將講論經理,夜分乃寐。皇太子見帝勤勞不怠,承閒諫曰:「陛下有禹湯之明,而失黃老養性之福,願頤愛精神,優游自寧。」帝曰:「我自樂此,不為疲也。」雖身濟大業,兢兢如不及,故能明慎政體,總髓權綱,量時度力,舉無過事。退功臣而進文吏,戢弓矢而散馬牛,雖道未方古,斯亦止戈之武焉。

論曰:皇考南頓君初為濟陽令,以建平元年十二月甲子夜生光武於縣舍,有赤光照室中。欽異焉,使卜者王長占之。長辟左右曰:「此兆吉不可言。」是歲縣界有嘉禾生,一莖九穗,因名光武曰秀。明年,方士有夏賀良者,上言哀帝,云漢家歷運中衰,當再受命。於是改號為太初元年,稱「陳聖劉太平皇帝」,以厭勝之。及王莽篡位,忌惡劉氏,以錢文有金刀,故改為貨泉。或以貨泉字文為「白水真人」。後望氣者蘇伯阿為王莽使至南陽,遙望見舂陵郭,唶曰:「氣佳哉!鬱鬱蔥蔥然。」及始起兵還舂陵,遠望舍南,火光赫然屬天,有頃不見。初,道士西門君惠、李守等亦云劉秀當為天子。其王者受命,信有符乎?不然,何以能乘時龍而御天哉!

贊曰:炎正中微,大盜移國。九縣飆回,三精霧塞。人厭淫詐,神思反德。光武誕命,靈貺自甄。沈幾先物,深略緯文。尋、邑百萬,貔虎為群。長轂雷野,高鋒彗雲。英威既振,新都自焚。虔劉庸、代,紛紜梁、趙。三河未澄,四關重擾。神旌乃顧,遞行天討。金湯失險,車書共道。靈慶既啟,人謀咸贊。明明廟謨,赳赳雄斷。於赫有命,系隆我漢。


\end{pinyinscope}