\article{孝桓帝紀}

\begin{pinyinscope}
孝桓皇帝諱志,肅宗曾孫也。祖父河閒孝王開,父蠡吾侯翼,母匽氏。翼卒,帝襲爵為侯。

本初元年,梁太后徵帝到夏門亭,將妻以女弟。會質帝崩,太后遂與兄大將軍冀定策禁中,閏月庚寅,使冀持節,以王青蓋車迎帝入南宮,其日即皇帝位,時年十五。太后猶臨朝政。

秋七月乙卯,葬孝質皇帝于靜陵。

齊王喜薨。

辛巳,謁高廟、光武廟。

丙戌,詔曰:「孝廉、廉吏皆當典城牧民,禁姦舉善,興化之本,恆必由之。詔書連下,分明懇惻,而在所翫習,遂至怠慢,選舉乖錯,害及元元。頃雖頗繩正,猶未懲改。方今淮夷未殄,軍師屢出,百姓疲悴,困於徵發。庶望群吏,惠我勞民,蠲滌貪穢,以祈休祥。其令秩滿百石,十歲以上,有殊才異行,乃得參選。臧吏子孫,不得察舉。杜絕邪偽請託之原,令廉白守道者得信其操。各明守所司,將觀厥後。」

九月戊戌,追尊皇祖河閒孝王曰孝穆皇,夫人趙氏曰孝穆皇后,皇考蠡吾侯曰孝崇皇。冬十月甲午,尊皇母匽氏為孝崇博園貴人。

建和元年春正月辛亥朔,日有食之。詔三公、九卿、校尉各言得失。

戊午,大赦天下。賜吏更勞一歲;男子爵,人二級,為父後及三老、孝悌、力田人三級;鰥、寡、孤、獨、篤缮、貧不能自存者粟,人五斛;貞婦帛,人三匹。災害所傷什四以上,勿收田租;其不滿者,以實除之。

二月,荊揚二州人多餓死,遣四府掾分行賑給。

沛國言黃龍見譙。

夏四月庚寅,京師地震。詔大將軍、公、卿、校尉舉賢良方正、能直言極諫者各一人。又命列侯、將、大夫、御史、謁者、千石、六百石、博士、議郎、郎官各上封事,指陳得失。又詔大將軍、公、卿、郡、國舉至孝篤行之士各一人。

壬辰,詔州郡不得迫脅驅逐長吏。長吏臧滿三十萬而不糾舉者,刺史、二千石以縱避為罪。若有擅相假印綬者,與殺人同棄巿論。

丙午,詔郡國繫囚減死罪一等,勿笞。唯謀反大逆,不用此書。又詔曰:「比起陵塋,彌歷時歲,力役既廣,徒隸尤勤。頃雨澤不沾,密雲復散,儻或在茲。其令徒作陵者減刑各六月。」

是月,立阜陵王代兄勃遒亭侯便為阜陵王。

郡國六地裂,水涌井溢。芝草生中黃藏府。

六月,太尉胡廣罷,大司農杜喬為太尉。

秋七月,勃海王鴻薨,立帝弟蠡吾侯悝為勃海王。

乙未,立皇后梁氏。

九月丁卯,京師地震。

太尉杜喬免,冬十月,司徒趙戒為太尉,司空袁湯為司徒,前太尉胡廣為司空。

十一月,濟陰言有五色大鳥見于己氏。

戊午,減天下死罪一等,戍邊。

清河劉文反,殺國相射暠,欲立清河王蒜為天子;事覺伏誅。蒜坐貶為尉氏侯,徙桂陽,自殺。

前太尉李固、杜喬皆下獄死。

陳留盜賊李堅自稱皇帝,伏誅。

二年春正月甲子,皇帝加元服。庚午,大赦天下。賜河閒、勃海二王黃金各百斤,彭城諸國王各五十斤;公主、大將軍、三公、特進、侯、中二千石、二千石、將、大夫、郎吏、從官、四姓及梁鄧小侯、諸夫人以下帛,各有差。年八十以上賜米、酒、肉,九十以上加帛二匹,綿三斤。

三月戊辰,帝從皇太后幸大將軍梁冀府。

白馬羌寇廣漢屬國,殺長吏,益州刺史率板楯蠻討破之。

夏四月丙子,封帝弟顧為平原王,奉孝崇皇祀。尊孝崇皇夫人馬氏為孝崇園貴人。

嘉禾生大司農帑藏。五月癸丑,北宮掖廷中德陽殿及左掖門火,車駕移幸南宮。

六月,改清河為甘陵,立安平王得子經侯理為甘陵王。

秋七月,京師大水。河東言木連理。

冬十月,長平陳景自號「黃帝子」,署置官屬,又南頓管伯亦稱「真人」,並圖舉兵,悉伏誅。

三年春三月甲申,彭城王定薨。

夏四月丁卯晦,日有食之。五月乙亥,詔曰:「蓋聞天生蒸民,不能相理,為之立君,使司牧之。君道得於下,則休祥著乎上;庶事失其序,則咎徵見乎象。閒者,日食毀缺,陽光晦暗,朕祗懼潛思,匪遑啟處。傳不云乎:『日食修德,月食修刑。』昔孝章帝愍前世禁徙,故建初之元,並蒙恩澤,流徙者使還故郡,沒入者免為庶民。先皇德政,可不務乎!其自永建元年迄乎今歲,凡諸妖惡,支親從坐,及吏民減死徙邊者,悉歸本郡;唯沒入者不從此令。」

六月庚子,詔大將軍、三公、特進、侯,其與卿、校尉舉賢良方正、能直言極諫之士各一人。

乙卯,震憲陵寑屋。秋七月庚申,廉縣雨肉。八月乙丑,有星孛于天巿。京師大水。九月己卯,地震。庚寅,地又震。詔死罪以下及亡命者贖,各有差。郡國五山崩。

冬十月,太尉趙戒免。司徒袁湯為太尉,大司農河內張歆為司徒。

十一月甲申,詔曰:「朕攝政失中,災眚連仍,三光不明,陰陽錯序。監寐寤歎,疢如疾首。今京師廝舍,死者相枕,郡縣阡陌,處處有之,甚違周文掩胔之義。其有家屬而貧無以葬者,給直,人三千,喪主布三匹;若無親屬,可於官壖地葬之,表識姓名,為設祠祭。又徒在作部,疾病致醫藥,死亡厚埋藏。民有不能自振及流移者,稟穀如科。州郡檢察,務崇恩施,以康我民。」

和平元年春正月甲子,大赦天下,改元和平。

己丑,詔曰:「曩者遭家不造,先帝早世。永惟大宗之重,深思嗣續之福,詢謀台輔,稽之兆占。既建明哲,克定統業,天人協和,萬國咸寧。元服已加,將即委付,而四方盜竊,頗有未靜,故假延臨政,以須安謐。幸賴股肱禦侮之助,殘醜消蕩,民和年稔,普天率土,遐邇洽同。遠覽『復子明辟』之義,近慕先姑歸授之法,及今令辰,皇帝稱制。群公卿士,虔恭爾位,戮力一意,勉同斷金。『展也大成』,則所望矣。」

二月扶風妖賊裴優自稱皇帝,伏誅。

甲寅,皇太后梁氏崩。

三月,車駕徙幸北宮。

甲午,葬順烈皇后。

夏五月庚辰,尊博園匽貴人曰孝崇皇后。

秋七月,梓潼山崩。

冬十一月辛巳,減天下死罪一等,徙邊戍。

元嘉元年春正月,京師疾疫,使光祿大夫將醫藥案行。

癸酉,大赦天下,改元元嘉。

二月,九江、廬江大疫。

甲午,河閒王建薨。夏四月己丑,安平王得薨。

京師旱。任城、梁國飢,民相食。

司徒張歆罷,光祿勳吳雄為司徒。

秋七月,武陵蠻叛。

冬十月,司空胡廣罷。

十一月辛巳,京師地震。

閏月庚午,任城王崇薨。太常黃瓊為司空。

二年春正月,西城長史王敬為于窴國所殺。

丙辰,京師地震。

夏四月甲寅,孝崇皇后匽氏崩。庚午,常山王豹薨。五月辛卯,葬孝崇皇后于博陵。

秋七月庚辰,日有食之。八月,濟陰言黃龍見句陽,金城言黃龍見允街。冬十月乙亥,京師地震。

十一月,司空黃瓊免。十二月,特進趙戒為司空。

右北平太守和旻坐臧,下獄死。

永興元年春二月,張掖言白鹿見。

三月丁亥,幸鴻池。

夏五月丙申,大赦天下,改元永興。

丁酉,濟南王廣薨,無子,國除。

秋七月,郡國三十二蝗。河水溢。百姓飢窮,流冗道路,至有數十萬戶,冀州尤甚。詔在所賑給乏絕,安慰居業。

冬十月,太尉袁湯免,太常胡廣為太尉。司徒吳雄罷,司空趙戒免;以太僕黃瓊為司徒,光祿勳房植為司空。

十一月丁丑,詔減天下死罪一等,徙邊戍。

是歲,武陵太守應奉招誘叛蠻,降之。

二年春正月甲午,大赦天下。

二月辛丑,初聽刺史、二千石行三年喪服。

癸卯,京師地震,詔公、卿、校尉舉賢良方正、能直言極諫者各一人。詔曰:「比者星辰謬越,坤靈震動,災異之降,必不空發。敕己修政,庶望有補。其輿服制度有踰侈長飾者,皆宜損省。郡縣務存儉約,申明舊令,如永平故事。」

六月,彭城泗水增長逆流。詔司隸校尉、部刺史曰:「蝗災為害,水變仍至,五穀不登,人無宿儲。其令所傷郡國種蕪菁以助人食。」

京師蝗。東海朐山崩。

九月丁卯朔,日有食之。詔曰:「朝政失中,雲漢作旱,川靈涌水,蝗螽孳蔓,殘我百穀,太陽虧光,飢饉荐臻。其不被害郡縣,當為飢餒者儲。天下一家,趣不糜爛,則為國寶。其禁郡國不得賣酒,祠祀裁足。」

太尉胡廣免,司徒黃瓊為太尉。閏月,光祿勳尹頌為司徒。

減天下死罪一等,徙邊戍。

蜀郡李伯詐稱宗室,當立為「太初皇帝」,伏誅。

冬十一月甲辰,校獵上林苑,遂至函谷關,賜所過道傍年九十以上錢,各有差。

太山、琅邪賊公孫舉等反叛,殺長吏。

永壽元年春正月戊申,大赦天下,改元永壽。

二月,司隸、冀州飢,人相食。敕州郡賑給貧弱。若王侯吏民有積穀者,一切貣十分之三,以助稟貸;其百姓吏民者,以見錢雇直。王侯須新租乃償。

夏四月,白烏見齊國。

六月,洛水溢,壤鴻德苑。南陽大水。

司空房植免,太常韓縯為司空。

詔太山、琅邪遇賊者,勿收租、賦,復更、筭三年。又詔被水死流失屍骸者,令郡縣鉤求收葬;及所唐突壓溺物故,七歲以上賜錢,人二千。壞敗廬舍,亡失穀食,尤貧者稟,人二斛。

巴郡、益州郡山崩。

秋七月,初置太山、琅邪都尉官。

南匈奴左臺、且渠伯德等叛,寇美稷,安定屬國都尉張奐討除之。

二年春正月,初聽中官得行三年服。

二月甲申,東海王臻薨。

三月,蜀郡屬國夷叛。

秋七月,鮮卑寇雲中。太山賊公孫舉等寇青、兗、徐三州,遣中郎將段熲討,破斬之。

冬十一月,置太官右監丞官。

十二月,京師地震。

三年春正月己未,大赦天下。

夏四月,九真蠻夷叛,太守兒式討之,戰歿;遣九真都尉魏朗擊破之。復屯據日南。

閏月庚辰晦,日有食之。

六月,初以小黃門為守宮令,置冗從右僕射官。

京師蝗。秋七月,河東地裂。

冬十一月,司徒尹頌薨。

長沙蠻叛,寇益陽。

司空韓縯為司徒,太常北海孫朗為司空。

延熹元年春三月己酉,初置鴻德苑令。

夏五月己酉,大會公卿以下,賞賜各有差。

甲戌晦,日有食之。京師蝗。

六月戊寅,大赦天下,改元延熹。

丙戌,分中山置博陵郡,以奉孝崇皇園陵。大雩。

秋七月己巳,雲陽地裂。

甲子,太尉黃瓊免,太常胡廣為太尉。

冬十月,校獵廣成,遂幸上林苑。

十二月,鮮卑寇邊,使匈奴中郎將張奐率南單于擊破之。

二年春二月,鮮卑寇鴈門。

己亥,阜陵王便薨。

蜀郡夷寇蠶陵,殺縣令。

三月,復斷刺史、二千石行三年喪。

夏,京師雨水。

六月,鮮卑寇遼東。

秋七月,初造顯陽苑,置丞。

丙午,皇后梁氏崩。乙丑,葬懿獻皇后于懿陵。

大將軍梁冀謀為亂。八月丁丑,帝御前殿,詔司隸校尉張彪將兵圍冀第,收大將軍印綬,冀與妻皆自殺。衛尉梁淑、河南尹梁胤、屯騎校尉梁讓、越騎校尉梁忠、長水校尉梁戟等,及中外宗親數十人,皆伏誅。太尉胡廣坐免。司徒韓縯、司空孫朗下獄。

壬午,立皇后鄧氏,追廢懿陵為貴人冢。詔曰:「梁冀姦暴,濁亂王室。孝質皇帝聰敏早茂,冀心懷忌畏,私行殺毒。永樂太后親尊莫二,冀又遏絕,禁還京師,使朕離母子之愛,隔顧復之恩。禍害深大,罪釁日滋。賴宗廟之靈,及中常侍單超、徐璜、具瑗、左悺、唐衡、尚書令尹勳等激憤建策,內外協同,漏刻之閒,桀逆梟夷。斯誠社稷之祐,臣下之力,宜班慶賞,以酬忠勳。其封超等五人為縣侯,勳等七人為亭侯。」於是舊故恩私,多受封爵。

大司農黃瓊為太尉,光祿大夫中山祝恬為司徒,大鴻臚梁國盛允為司空。初置祕書監官。

冬十月壬申,行幸長安。乙酉,幸未央宮。甲午,祠高廟。十一月庚子,遂有事十一陵。

壬寅,中常侍單超為車騎將軍。

十二月己巳,至自長安,賜長安民粟人十斛,園陵人五斛,行所過縣三斛。

燒當等八種羌叛,寇隴右,護羌校尉段熲追擊於羅亭,破之。

天竺國來獻。

三年春正月丙申,大赦天下。

丙午,車騎將軍單超薨。

閏月,燒何羌叛,寇張掖,護羌校尉段熲追擊於積石,大破之。

白馬令李雲坐直諫,下獄死。

夏四月,上郡言甘露降。五月甲戌,漢中山崩。

六月辛丑,司徒祝恬薨。秋七月,司空盛允為司徒,太常虞放為司空。

長沙蠻寇郡界。

九月,太山、琅邪賊勞丙等復叛,寇掠百姓,遣御史中丞趙某持節督州郡討之。

丁亥,詔無事之官權絕奉,豐年如故。

冬十一月,日南蠻賊率眾詣郡降。

勒姐羌圍允街,段熲擊破之。

太山賊叔孫無忌攻殺都尉侯章。十二月,遣中郎將宗資討破之。

武陵蠻寇江陵,車騎將軍馮緄討,皆降散。荊州刺史度尚討長沙蠻,平之。

四年春正月辛酉,南宮嘉德殿火。戊子,丙署火。大疫。二月壬辰,武庫火。

司徒盛允免,大司農种暠為司徒。三月,省冗從右僕射官。太尉黃瓊免。夏四月,太常劉矩為太尉。

甲寅,封河閒王開子博為任城王。

五月辛酉,有星孛于心。丁卯,原陵長壽門火。己卯,京師雨雹。六月,京兆、扶風及涼州地震。庚子,岱山及博尤來山並穨裂。

己酉,大赦天下。

司空虞放免,前太尉黃瓊為司空。

犍為屬國夷寇鈔百姓,益州刺史山昱擊破之。

零吾羌與先零諸種並叛,寇三輔。

秋七月,京師雩。

減公卿以下奉,貣王侯半租。占賣關內侯、虎賁、羽林、緹騎營士、五大夫錢各有差。

九月,司空黃瓊免,大鴻臚劉寵為司空。

冬十月,天竺國來獻。

南陽黃武與襄城惠得、昆陽樂季訞言相署,皆伏誅。

先零沈氐羌與諸種羌寇并涼二州,十一月,中郎將皇甫規擊破之。

十二月,夫餘王遣使來獻。

五年春正月,省太官右監丞。

壬午,南宮丙署火。

三月,沈氐羌寇張掖、酒泉。

壬午,濟北王次薨。

夏四月,長沙賊起,寇桂陽、蒼梧。

驚馬逸象突入宮殿。乙丑。恭陵東闕火。戊辰,虎賁掖門火。己巳,太學西門自壞。五月,康陵園寑火。

長沙、零陵賊起,攻桂陽、蒼梧、南海、交阯,遣御史中丞盛脩督州郡討之,不克。

乙亥,京師地震。詔公、卿各上封事。甲申,中藏府承祿署火。秋七月己未,南宮承善闥火。

鳥吾羌寇漢陽、隴西、金城,諸郡兵討破之。

八月庚子,詔減虎賁、羽林住寺不任事者半奉,勿與冬衣;其公卿以下給冬衣之半。

艾縣賊焚燒長沙郡縣,寇益陽,殺令。又零陵蠻亦叛,寇長沙。

己卯,罷琅邪都尉官。

冬十月,武陵蠻叛,寇江陵,南郡太守李肅坐奔北棄市;辛丑,以太常馮緄為車騎將軍,討之。假公卿以下奉。又換王侯租以助軍糧,出濯龍中藏錢還之。十一月,馮緄大破叛蠻於武陵。

京兆虎牙都尉宗謙坐臧,下獄死。

滇那羌寇武威、張掖、酒泉。

太尉劉矩免,太常楊秉為太尉。

六年春二月戊午,司徒种暠薨。

三月戊戌,大赦天下。

衛尉潁川許栩為司徒。

夏四月辛亥,康陵東署火。

五月,鮮卑寇遼東屬國。

秋七月甲申,平陵園寑火。

桂陽盜賊李研等寇郡界。

武陵蠻復叛,太守陳奉與戰,大破降之。

隴西太守孫羌討滇那羌,破之。

八月,車騎將軍馮緄免。

冬十月丙辰,校獵廣成,遂幸函谷關、上林苑。

十一月,司空劉寵免。

南海賊寇郡界。

十二月,衛尉周景為司空。

七年春正月庚寅,沛王榮薨。

三月癸亥,隕石于鄠。

夏四月丙寅,梁王成薨。

五月己丑,京師雨雹。

秋七月辛卯,趙王乾薨。

野王山上有死龍。

荊州刺史度尚擊零陵、桂陽盜賊及蠻夷,大破平之。

冬十月壬寅,南巡狩。庚申,幸章陵,祠舊宅,遂有事于園廟,賜守令以下各有差。戊辰,幸雲夢,臨漢水;還,幸新野,祠湖陽、新野公主、魯哀王、壽張敬侯廟。

護羌校尉段熲擊當煎羌,破之。

十二月辛丑,車駕還宮。

八年春正月,遣中常侍左悺之苦縣,祠老子。

勃海王悝謀反,降為癭陶王。

丙申晦,日有食之。詔公、卿、校尉舉賢良方正。

己酉,南宮嘉德署黃龍見。千秋萬歲殿火。

太僕左稱有罪自殺。

癸亥,皇后鄧氏廢。河南尹鄧萬世、虎賁中郎將鄧會下獄死。

護羌校尉段熲擊罕姐羌,破之。

三月辛巳,大赦天下。

夏四月甲寅,安陵園寢火。

丁巳,壞郡國諸房祀。

濟陰、東郡、濟北河水清。

五月壬申,罷太山都尉官。丙戌,太尉楊秉薨。

丙辰,緱氏地裂。

桂陽胡蘭、朱蓋等復反,攻沒郡縣,轉寇零陵,零陵太守陳球拒之;遣中郎將度尚、長沙太守抗徐等擊蘭、蓋,大破斬之。蒼梧太守張敘為賊所執,又桂陽太守任胤背敵畏儒,皆棄市。

閏月甲午,南宮長秋和歡殿後鉤楯、掖庭、朔平署火。

六月,段熲擊當煎羌於湟中,大破之。

秋七月,太中大夫陳蕃為太尉。

八月戊辰,初令郡國有田者畝斂稅錢。

九月丁未,京師地震。

冬十月,司空周景免,太常劉茂為司空。

辛巳,立貴人竇氏為皇后。

勃海妖賊蓋登等稱「太上皇帝」,有玉印、珪、璧、鐵券,相署置,皆伏誅。

十一月壬子,德陽殿西閤、黃門北寺火,延及廣義、神虎門,燒殺人。

使中常侍管霸之苦縣,祠老子。

九年春正月辛亥朔,日有食之。詔公、卿、校尉、郡國舉至孝。

沛國戴異得黃金印,無文字,遂與廣陵人龍尚等共祭井,作符書,稱「太上皇」,伏誅。

己酉,詔曰:「比歲不登,民多飢窮,又有水旱疾疫之困。盜賊徵發,南州尤甚。災異日食,譴告累至。政亂在予,仍獲咎徵。其令大司農絕今歲調度徵求,及前年所調未畢者,勿復收責。其災旱盜賊之郡,勿收租,餘郡悉半入。」

三月癸巳,京師有火光轉行,人相驚譟。

司隸、豫州飢死者什四五,至有滅戶者,遣三府掾賑稟之。

陳留太守韋毅坐臧自殺。

夏四月,濟陰、東郡、濟北、平原河水清。

司徒許栩免。五月,太常胡廣為司徒。

六月,南匈奴及烏桓、鮮卑寇緣邊九郡。

秋七月,沈氐羌寇武威、張掖。詔舉武猛,三公各二人,卿、校尉各一人。

太尉陳蕃免。

庚午,祠黃、老於濯龍宮。

遣使匈奴中郎將張奐擊南匈奴、烏桓、鮮卑。

九月,光祿勳周景為太尉。

南陽太守成档、太原太守劉質,並以譖棄市。

司空劉茂免。

大秦國王遣使奉獻。

冬十二月,洛城傍竹柏枯傷。

光祿勳汝南宣酆為司空。

南匈奴、烏桓率眾詣張奐降。

司隸校尉李膺等二百餘人受誣為黨人,並坐下獄,書名王府。

永康元年春正月,先零羌寇三輔,中郎將張奐破平之。當煎羌寇武威,護羌校尉段熲追擊於鸞鳥,大破之。西羌悉平。

夫餘王寇玄菟,太守公孫域與戰,破之。

夏四月,先零羌寇三輔。

五月丙申,京師及上黨地裂。

廬江賊起,寇郡界。

壬子晦,日有食之。詔公、卿、校尉舉賢良方正。

六月庚申,大赦天下,悉除黨錮,改元永康。

丙寅,阜陵王統薨。

秋八月,魏郡言嘉禾生,甘露降。巴郡言黃龍見。

六州大水,勃海海溢。詔州郡賜溺死者七歲以上錢,人二千;一家皆被害者,悉為收斂;其亡失穀食,稟人三斛。

冬十月,先零羌寇三輔,使匈奴中郎將張奐擊破之。

十一月,西河言白菟見。

十二月壬申,復癭陶王悝為勃海王。

丁丑,帝崩于德陽前殿,年三十六。戊寅,尊皇后曰皇太后,太后臨朝。

是歲,復博陵、河閒二郡,比豐、沛。

論曰:前史稱桓帝好音樂,善琴笙。飾芳林而考濯龍之宮,設華蓋以祠浮圖、老子,斯將所謂「聽於神」乎!及誅梁冀,奮威怒,天下猶企其休息。而五邪嗣虐,流衍四方。自非忠賢力爭,屢折姦鋒,雖願依斟流彘,亦不可得已。

贊曰:桓自宗支,越躋天祿。政移五倖,刑淫三獄。傾宮雖積,皇身靡續。


\end{pinyinscope}