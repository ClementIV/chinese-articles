\article{孝安帝紀}

\begin{pinyinscope}
恭宗孝安皇帝諱祜,肅宗孫也。父清河孝王慶,母左姬。帝自在邸第,數有神光照室,又有赤蛇盤於床笫之閒。年十歲,好學史書,和帝稱之,數見禁中。

延平元年,慶始就國,鄧太后特詔留帝清河邸。

八月,殤帝崩,太后與兄車騎將軍鄧騭定策禁中。其夜,使騭持節,以王青蓋車迎帝,齋于殿中。皇太后御崇德殿,百官皆吉服,群臣陪位,引拜帝為長安侯。皇太后詔曰:「先帝聖德淑茂,早棄天下。朕奉皇帝,夙夜瞻仰日月,冀望成就。豈意卒然顛沛,天年不遂,悲痛斷心。朕惟平原王素被痼疾,念宗廟之重,思繼嗣之統,唯長安侯祜質性忠孝,小心翼翼,能通詩、論,篤學樂古,仁惠愛下。年已十三,有成人之志。親德係後,莫宜於祜。禮『昆弟之子猶己子』;春秋之義,為人後者為之子,不以父命辭王父命。其以祜為孝和皇帝嗣,奉承祖宗,案禮儀奏。」又作策命曰:「惟延平元年秋八月癸丑,皇太后曰:咨長安侯祜:孝和皇帝懿德巍巍,光于四海;大行皇帝不永天年。朕惟侯孝章帝世嫡皇孫,謙恭慈順,在孺而勤,宜奉郊廟,承統大業。今以侯嗣孝和皇帝後。其審君漢國,允執其中『一人有慶,萬民賴之。』皇帝其勉之哉!」讀策畢,太尉奉上璽綬,即皇帝位,年十三。太后猶臨朝。

九月庚子,謁高廟。辛丑,謁光武廟。

六州大水。己未,遣謁者分行虛實,舉災害,賑乏絕。

丙寅,葬孝殤皇帝于康陵。

乙亥,隕石于陳留。

西域諸國叛,攻都護任尚,遣副校尉梁慬救尚,擊破之。

冬十月,四州大水,雨雹。詔以宿麥不下,賑賜貧人。

十二月甲子,清河王薨,使司空持節弔祭,車騎將軍鄧騭護喪事。

乙酉,罷魚龍曼延百戲。

永初元年春正月癸酉朔,大赦天下。

蜀郡徼外羌內屬。

戊寅,分犍為南部為屬國都尉。

稟司隸、兗、豫、徐、冀、并州貧民。

二月丙午,以廣成游獵地及被災郡國公田假與貧民。

丁卯,分清河國封帝弟常保為廣川王。

庚午,司徒梁鮪薨。

三月癸酉,日有食之。詔公卿內外眾官、郡國守相,舉賢良方正、有道術之士,明政術、達古今、能直言極諫者,各一人。

己卯,永昌徼外僬僥種夷貢獻內屬。

甲申,葬清河孝王,贈龍旗、虎賁。

夏五月甲戌,長樂衛尉魯恭為司徒。

丁丑,詔封北海王睦孫壽光侯普為北海王。

九真徼外夜郎蠻夷舉土內屬。

六月戊申,爵皇太后母陰氏為新野君。

丁巳,河東地陷。

壬戌,罷西域都護。

先零種羌叛,斷隴道,大為寇掠,遣車騎將軍鄧騭、征西校尉任尚討之。丁卯,赦除諸羌相連結謀叛逆者罪。

秋九月庚午,詔三公明申舊令,禁奢侈,無作浮巧之物,殫財厚葬。

是日,太尉徐防免。辛未,司空尹勤免。

癸酉,調揚州五郡租米,贍給東郡、濟陰、陳留、梁國、下邳、山陽。

丁丑,詔曰:「自今長吏被考竟未報,自非父母喪無故輒去職者,劇縣十歲、平縣五歲以上,乃得次用。」

壬午,詔太僕、少府減黃門鼓吹,以補羽林士;廄馬非乘輿常所御者,皆減半食;諸所造作,非供宗廟園陵之用,皆且止。

丙戌,詔死罪以下及亡命贖,各有差。

庚寅,太傅張禹為太尉,太常周章為司空。

冬十月,倭國遣使奉獻。

辛酉,新城山泉水大出。

十一月丁亥,司空周章密謀廢立,策免,自殺。

戊子,敕司隸校尉、冀并二州刺史:「民訛言相驚,棄捐舊居,老弱相攜,窮困道路。其各敕所部長吏,躬親曉喻。若欲歸本郡,在所為封長檄;不欲,勿強。」

十二月乙卯,潁川太守張敏為司空。

是歲,郡國十八地震;四十一雨水,或山水暴至;二十八大風,雨雹。

二年春正月,稟河南、下邳、東萊、河內貧民。

車騎大將軍鄧騭為種羌所敗於冀西。

二月乙丑,遣光祿大夫樊準、呂倉分行冀兗二州,稟貸流民。

夏四月甲寅,漢陽城中火,燒殺三千五百七十人。

五月,旱。丙寅,皇太后幸洛陽寺及若盧獄,錄囚徒,賜河南尹、廷尉、卿及官屬以下各有差,即日降雨。

六月,京師及郡國四十大水,大風,雨雹。

秋七月戊辰,詔曰:「昔在帝王,承天理民,莫不據琁機玉衡,以齊七政。朕以不德,遵奉大業,而陰陽差越,變異並見,萬民飢流,羌貊叛戾。夙夜克己,憂心京京。閒令公卿郡國舉賢良方正,遠求博選,開不諱之路,冀得至謀,以鑒不逮,而所對皆循尚浮言,無卓爾異聞。其百僚及郡國吏人,有道術明習災異陰陽之度琁機之數者,各使指變以聞。二千石長吏明以詔書,博衍幽隱,朕將親覽,待以不次,冀獲嘉謀,以承天誡。」

閏月辛丑,廣川王常保薨,無子,國除。

癸未,蜀郡徼外羌舉土內屬。

九月庚子,詔王主官屬墨綬下至郎、謁者,其經明任博士,居鄉里有廉清孝順之稱,才任理人者,國相歲移名,與計偕上尚書,公府通調,令得外補。

冬十月庚寅,稟濟陰、山陽、玄菟貧民。

征西校尉任尚與先零羌戰于平襄,尚軍敗績。

十一月辛酉,拜鄧騭為大將軍,徵還京師,留任尚屯隴右。先零羌滇零稱天子於北地,遂寇三輔,東犯趙、魏,南入益州,殺漢中太守董炳。

十二月辛卯,稟東郡、鉅鹿、廣陽、安定、定襄、沛國貧民。

廣漢塞外參狼羌降,分廣漢北部為屬國都尉。

是歲,郡國十二地震。

三年春正月庚子,皇帝加元服。大赦天下。賜王、主、貴人、公、卿以下金帛各有差;男子為父後,及三老、孝悌、力田爵,人二級,流民欲占者人一級。

遣騎都尉任仁討先零羌,不利,羌遂破沒臨洮。

高句驪遣使貢獻。

三月,京師大飢,民相食。壬辰,公卿詣闕謝。詔曰:「朕以幼沖,奉承鴻業,不能宣流風化,而感逆陰陽,至令百姓飢荒,更相噉食。永懷悼歎,若墜淵水。咎在朕躬,非群司之責,而過自貶引,重朝廷之不德。其務思變復,以助不逮。」癸巳,詔以鴻池假與貧民。

壬寅,司徒魯恭免。夏四月丙寅,大鴻臚九江夏勤為司徒。

三公以國用不足,奏令吏人入錢穀,得為關內侯、虎賁羽林郎、五大夫、官府吏、緹騎、營士各有差。

己巳,詔上林、廣成苑可墾闢者,賦與貧民。

甲申,清河王虎威薨。五月丙申,封樂安王寵子延平為清河王。

丁酉,沛王正薨。

癸丑,京師大風,

六月,烏桓寇代郡、上谷、涿郡。

秋七月,海賊張伯路等寇略緣海九郡,遣侍御史龐雄督州郡兵討破之。

庚子,詔長吏案行在所,皆令種宿麥蔬食,務盡地力,其貧者給種餉。

九月,鴈門烏桓及鮮卑叛,敗五原郡兵於高渠谷。

冬十月,南單于叛,圍中郎將耿种於美稷。十一月,遣行車騎將軍何熙討之。

十二月辛酉,郡國九地震。乙亥,有星孛于天苑。

是歲,京師及郡國四十一雨水雹。并涼二州大飢,人相食。

四年春正月元日,會,徹樂,不陳充庭車。

辛卯,詔以三輔比遭寇亂,人庶流冗,除三年逋租、過更、口筭、芻稿;稟上郡貧民各有差。

海賊張伯路復與勃海、平原劇賊劉文河、周文光

等攻厭次,殺縣令,遣御史中丞王宗督青州刺史法雄討破之。

度遼將軍梁慬、遼東太守耿夔討破南單于於屬國

故城。

丙午,詔減百官及州郡縣奉各有差。

二月丁巳,稟九江貧民。

南匈奴寇常山。

乙丑,初置長安、雍二營都尉官。

乙亥,詔自建初以來,諸祅言它過坐徙邊者,各歸本郡;其沒入官為奴婢者,免為庶人。

詔謁者劉珍及五經博士,校定東觀五經、諸子、傳記、百家蓺術,整齊脫誤,是正文字。

三月,南單于降。

先零羌寇褒中,漢中太守鄭勤戰歿。徙金城郡都襄武。

戊子,杜陵園火。癸巳,郡國九地震,夏四月,六州蝗。丁丑,大赦天下,秋七月乙酉,三郡大水。

巳卯,騎都尉任仁下獄死,

九月甲申,益州郡地震。

冬十月甲戌,新野君陰氏薨,使司空持節護喪事。

大將軍鄧騭罷。

五年春正月庚辰朔,日有食之。丙戌,郡國十地震。

己丑,太尉張禹免。甲申,光祿勳李脩為太尉。

二月丁卯,詔省減郡國貢獻太官口食。

先零羌寇河東,遂至河內。

三月,詔隴西徙襄武,安定徙美陽,北地徙池陽,上郡徙衙

夫餘夷犯塞,殺傷吏人。

閏月丁酉,赦涼州河西四郡。

戊戌,詔曰:「朕以不德,奉郊廟,承大業,不能興和降善,為人祈福。災異蜂起,寇賊縱橫,夷狄猾夏,戎事不息,百姓匱乏,疲於徵發。重以蝗蟲滋生,害及成麥,秋稼方收,甚可悼也。朕以不明,統理失中,亦未獲忠良以毗闕政。傳曰:『顛而不扶,危而不持,則將焉用彼相矣。』公卿大夫將何以匡救,濟斯艱厄,承天誡哉?蓋為政之本,莫若得人,褒賢顯善,聖制所先。『濟濟多士,文王以寧。』思得忠良正直之臣,以輔不逮。其令三公、特進、侯、中二千石、二千石、郡守、諸侯相舉賢良方正、有道術、達於政化、能直言極諫之士各一人,及至孝與眾卓異者,并遣詣公車,朕將親覽焉。」

六月甲辰,樂成王巡薨。

秋七月己巳,詔三公、特進、九卿、校尉,舉列將子孫明曉戰陳任將帥者。

九月,漢陽人杜琦、王信叛,與先零諸種羌攻陷上邽城。十二月,漢陽太守趙博遣客刺殺杜琦。

是歲,九州蝗,郡國八雨水。

六年春正月庚申,詔越巂置長利、高望、始昌三苑,又令益州郡置萬歲苑,犍為置漢平苑。

三月,十州蝗。

夏四月乙丑,司空張敏罷。

己卯,太常劉凱為司空。

五月,旱。

丙寅,詔令中二千石下至黃綬,一切復秩還贖,賜爵各有差。

戊辰,皇太后幸雒陽寺,錄囚徒,理冤獄。

六月壬辰,豫章、員谿、原山崩。

辛巳,大赦天下。

遣侍御史唐喜討漢陽賊王信,破斬之。

冬十一月辛丑,護烏桓校尉吳祉下獄死。

是歲,先零羌滇零死,子零昌復襲偽號。

七年春正月庚戌,皇太后率大臣命婦謁宗廟。

二月丙午,郡國十八地震。

夏四月乙未,平原王勝薨。

丙申晦,日有食之。五月庚子,京師大雩。

秋,護羌校尉侯霸、騎都尉馬賢破先零羌。

八月丙寅,京師大風,蝗蟲飛過洛陽。詔賜民爵。郡國被蝗傷稼十五以上,勿收今年田租;不滿者,以實除之。

九月,調零陵、桂陽、丹陽、豫章、會稽租米,賑給南陽、廣陵、下邳、彭城、山陽、廬江、九江飢民;又調濱水縣穀輸敖倉。

元初元年春正月甲子,改元元初。賜民爵,人二級,孝悌、力田人三級,爵過公乘,得移與子若同產、同產子,民脫無名數及流民欲占者人一級;鰥、寡、孤、獨、篤缮、不能自存者穀,人三斛,貞婦帛,人一匹。

二月己卯,日南地坼。三月癸酉,日有食之。

夏四月丁酉,大赦天下。

京師及郡國五旱、蝗。

詔三公、特進、列侯、中二千石、二千石、郡守

舉敦厚質直者,各一人。

五月,先零羌寇雍城。

六月丁巳,河東地陷。

秋七月,蜀郡夷寇蠶陵,殺縣令。

九月乙丑,太尉李脩罷。

先零羌寇武都、漢中,絕隴道。

辛未,大司農山陽司馬苞為太尉。

冬十月戊子朔,日有食之。

先零羌敗涼州刺史皮陽於狄道。

乙卯,詔除三輔三歲田租、更賦、口筭。

十一月。是歲,郡國十五地震。

二年春正月,詔稟三輔及并、涼六郡流冗貧人。

蜀郡青衣道夷奉獻內屬。

修理西門豹所分漳水為支渠,以溉民田。

二月戊戌,遣中謁者收葬京師客死無家屬及棺槨朽敗者,皆為設祭;其有家屬,尤貧無以葬者,賜錢人五千。

辛酉,詔三輔、河內、河東、上黨、趙國、太原各修理舊渠,通利水道,以溉公私田疇。

三月癸亥,京師大風。

先零羌寇益州,遣中郎將尹就討之。

夏四月丙午,立貴人閻氏為皇后。

五月,京師旱,河南及郡國十九蝗。甲戌,詔曰:「朝廷不明,庶事失中,災異不息,憂心悼懼。被蝗以來,七年于茲,而州郡隱匿,裁言頃畝。今群飛蔽天,為害廣遠,所言所見,寧相副邪?三司之職,內外是監,既不奏聞,又無舉正。天災至重,欺罔罪大。今方盛夏,且復假貸,以觀厥後。其務消救災眚,安輯黎元。」

六月丙戌,太尉司馬苞薨。

洛陽新城地裂。

秋七月辛巳,太僕太山馬英為太尉。

八月,遼東鮮卑圍無慮縣。九月,又攻夫犁營,殺縣令。

壬午晦,日有食之。

冬十月,遣中郎將任尚屯三輔。

詔郡國中都官繫囚減死一等。勿笞,詣馮翊、扶風屯,妻子自隨,占著所在;女子勿輸。亡命死罪以下贖,各有差。其吏人聚為盜賊,有悔過者,除其罪。

乙未,右扶風仲光、安定太守杜恢、京兆虎牙都尉耿溥與先零羌戰於丁奚城,光等大敗,並沒。左馮翊司馬鈞下獄,自殺。

十一月庚申,郡國十地震。

十二月,武陵澧中蠻叛,州郡擊破之。

己酉,司徒夏勤罷。庚戍,司空劉愷為司徒,光祿勳袁敞為司空。

三年春正月甲戌,修理太原舊溝渠,溉灌官私田。

東平陸上言木連理。

蒼梧、鬱林、合浦蠻夷反叛,二月,遣侍御史任逴督州郡兵討之。

郡國十地震。三月辛亥,日有食之。

丙辰,赦蒼梧、鬱林、合浦、南海吏人為賊所迫者。

夏四月,京師旱。

五月,武陵蠻復叛,州郡討破之。

癸酉,度遼將軍鄧遵率南匈奴擊先零羌於靈州,破之。

越巂徼外夷舉種內屬。

六月,中郎將任尚遣兵擊破先零羌於丁奚城。

秋七月,武陵蠻復叛,州郡討平之。

緱氏地坼。

九月辛巳,趙王宏薨。

冬十一月,蒼梧、鬱林、合浦蠻夷降。

丙戌,初聽大臣、二千石、刺史行三年喪。

癸卯,郡國九地震。

十二月丁巳,任尚遣兵擊破先零羌於北地。

四年春二月乙巳朔,日有食之。乙卯,大赦天下。壬戌,武庫災。

夏四月戊申,司空袁敞薨。

己巳,鮮卑寇遼西,遼西郡兵與烏桓擊破之。

五月丁丑,太常李郃為司空。

六月戊辰,三郡雨雹。

秋七月辛丑,陳王鈞薨。

京師及郡國十雨水。詔曰:「今年秋稼茂好,垂可收穫,而連雨未霽,懼必淹傷。夕惕惟憂,思念厥咎。夫霖雨者,人怨之所致。其武吏以威暴下,文吏妄行苛刻,鄉吏因公生姦,為百姓所患苦者,有司顯明其罰。又月令『仲秋養衰老,授几杖,行糜粥。』。方今案比之時,郡縣多不奉行。雖有糜粥,糠秕相半,長吏怠事,莫有躬親,甚違詔書養老之意。其務崇仁恕,賑護寡獨,稱朕意焉。」

九月,護羌校尉任尚使客刺殺叛羌零昌。

冬十一月己卯,彭城王恭薨。

十二月,越巂夷寇遂久,殺縣令。

甲子,任尚及騎都尉馬賢與先零羌戰于富平上河,大破之。虔人羌率眾降,隴右平。

是歲,郡國十三地震。

五年春正月,越巂夷叛。

二月壬戌,中山王憲薨。

三月,京師及郡國五旱,詔稟遭旱貧人。

夏六月,高句驪與穢貊寇玄菟。

秋七月,越巂蠻夷及旄牛豪叛,殺長吏。

丙子,詔曰:「舊令制度,各有科品,欲令百姓務崇節約。遭永初之際,人離荒厄,朝廷躬自菲薄,去絕奢飾,食不兼味,衣無二綵。比年雖獲豐穰,尚乏儲積,而小人無慮,不圖久長,嫁娶送終,紛華靡麗,至有走卒奴婢被綺縠,著珠璣。京師尚若斯,何以示四遠?設張法禁,懇惻分明,而有司惰任,訖不奉行。秋節既立,鷙鳥將用,且復重申,以觀後效。」

八月丙申朔,日有食之。

鮮卑寇代郡,殺長吏。冬十月,鮮卑寇上谷。

十二月丁巳,中郎將任尚有罪,棄市。

是歲,郡國十四地震。

六年春二月乙巳,京師及郡國四十二地震,或坼裂,水泉涌出。

壬子,詔三府選掾屬高第,能惠利牧養者各五人,光祿勳與中郎將選孝廉郎寬博有謀,清白行高者五十人,出補令、長、丞、尉。

乙卯,詔曰:「夫政,先京師,後諸夏。月令仲春『養幼小,存諸孤』,季春『賜貧窮,賑乏絕,省婦使,表貞女』,所以順陽氣,崇生長也。其賜人尤貧困、孤弱、單獨穀,人三斛;貞婦有節義十斛,甄表門閭,旌顯厥行。」

三月庚辰,始立六宗,祀於洛城西北。

夏四月,會稽大疫,遣光祿大夫將太醫循行疾病,賜棺木,除田租、口賦。

沛國、勃海大風,雨雹。五月,京師旱。

六月丁丑,樂成王賓薨。丙戌,平原王得薨。

秋七月,鮮卑寇馬城,度遼將軍鄧遵率南單于擊破之。

九月癸巳,陳王竦薨

十二月戊午朔,日有食之,既。郡國八地震。

是歲,永昌、益州蜀郡夷叛,與越雟夷殺長吏,燔城邑,益州刺史張喬討破降之。

永寧元年春正月甲辰,任城王安薨。三月丁酉,濟北王壽薨。

車師後王叛,殺部司馬。

沈氐羌寇張掖。

夏四月丙寅,立皇子保為皇太子,改元永寧,大赦天下。賜王、主、三公、列侯下至郎吏、從官金帛;又賜民爵及布粟各有差。

己巳,紹封陳王羨子崇為陳王,濟北王子萇為樂成王,河閒王子翼為平原王。

壬午,琅邪王壽薨。

六月,沈氐種羌叛,寇張掖,護羌校尉馬賢討沈氐羌,破之。

秋七月乙酉朔,日有食之。

冬十月己巳,司空李郃免。癸酉,衛尉廬江陳褒為司空。

自三月至是月,京師及郡國三十三大風,雨水。

十二月,永昌徼外撣國遣使貢獻。

戊辰,司徒劉愷罷。

遼西鮮卑降。

癸酉,太常楊震為司徒。

是歲,郡國二十三地震。夫餘王遣子詣闕貢獻。燒當羌叛。

建光元年春正月,幽州刺史馮煥率二郡太守討高句驪、穢貊,不克。

二月癸亥,大赦天下。賜諸園貴人、王、主、公、卿以下錢布各有差。以公、卿、校尉、尚書子弟一人為郎、舍人。

三月癸巳,皇太后鄧氏崩。丙午,葬和熹皇后。

丁未,樂安王寵薨。

戊申,追尊皇考清河孝王曰孝德皇,皇妣左氏曰孝德皇后,祖妣宋貴人曰敬隱皇后。

夏四月,穢貊復與鮮卑寇遼東,遼東太守蔡諷追擊,戰歿。

丙辰,以廣川并清河國。

丁巳,尊孝德皇元妃耿氏為甘陵大貴人。

甲子,樂成王萇有罪,廢為臨湖侯。

己巳,令公、卿、特進、侯、中二千石、二千石、郡國守相,舉有道之士各一人。賜鰥、寡、孤、獨、貧不能自存者穀,人三斛。

甲戌,遼東屬國都尉龐奮,承偽璽書殺玄菟太守姚光。

五月庚辰,特進鄧騭及度遼將軍鄧遵,並以譖自殺。

丙申,貶平原王翼為都鄉侯。

秋七月己卯,改元建光,大赦天下。

壬寅,太尉馬英薨。

八月,護羌校尉馬賢討燒當羌於金城,不利。

甲子,前司徒劉愷為太尉。

鮮卑寇居庸關,九月,雲中太守成嚴擊之,戰歿。鮮卑圍烏桓校尉於馬城,度遼將軍耿夔救之。

戊子,幸衛尉馮石府。

是秋,京師及郡國二十九雨水。

冬十一月己丑,郡國三十五地震,或坼裂。詔三公以下,各上封事陳得失。遣光祿大夫案行,賜死者錢,人二千。除今年田租。其被災甚者,勿收口賦。

鮮卑寇玄菟。

庚子,復斷大臣二千石以上服三年喪。

癸卯,詔三公、特進、侯、卿、校尉,舉武猛堪將帥者各五人。

丙午,詔京師及郡國被水雨傷稼者,隨頃畝減田租。

甲子,初置漁陽營兵。

冬十二月,高句驪、馬韓、穢貊圍玄菟城,夫餘王遣子與州郡并力討破之。

延光元年春二月,夫餘王遣子將兵救玄菟,擊高句驪、馬韓、穢貊,破之,遂遣使貢獻。

三月丙午,改元延光。大赦天下。還徙者,復戶邑屬籍。賜民爵及三老、孝悌、力田,人二級;加賜鰥、寡、孤、獨、篤缮、貧不能自存者粟,人三斛;貞婦帛,人二匹。

夏四月癸未,京師郡國二十一雨雹。

癸巳,司空陳褒免。

五月庚戌,宗正彭城劉授為司空。

己巳,改樂成國為安平,封河閒王開子得為安平王。

六月,郡國蝗。秋七月癸卯,京師及郡國十三地震。

高句驪降。

虔人羌叛,攻穀羅城,度遼將軍耿夔討破之。

八月戊子,陽陵園寑火。辛卯,九真言黃龍見無功。

己亥,詔三公、中二千石,舉刺史、二千石、令、長、相,視事一歲以上至十歲,清白愛利,能敕身率下,防姦理煩,有益於人者,無拘官簿。刺史舉所部,郡國太守相舉墨綬,隱親悉心,勿取浮華。

九月甲戌,郡國二十七地震。

冬十月,鮮卑寇鴈門、定襄。十一月,鮮卑寇太原。

燒當羌豪降。

十二月,九真徼外蠻夷貢獻內屬。

是歲,京師及郡國二十七雨水,大風,殺人。詔賜壓溺死者年七歲以上錢,人二千;其壞敗廬舍、失亡穀食,粟,人三斛;又田被淹傷者,一切勿收田租;若一家皆被災害而弱小存者,郡縣為收斂之。虔人羌反攻穀羅城,度遼將軍耿夔討破之。

二年春正月,旄牛夷叛,寇靈關,殺縣令。益州刺史蜀郡西部都尉討之。

詔選三署郎及吏人能通古文尚書、毛詩、

穀梁春秋各一人。

丙辰,河東、潁川大風。夏六月壬午,郡國十一大風。九真言嘉禾生。

丙申,北海王普薨。

秋七月,丹陽山崩。

八月庚午,初令三署郎通達經術任牧民者,視事三歲以上,皆得察舉。

九月,郡國五雨水。

冬十月辛未,太尉劉愷罷。甲戌,司徒楊震為太尉,光祿勳東萊劉熹為司徒。

十一月甲辰,校獵上林苑。

鮮卑敗南匈奴於曼柏。

是歲,分蜀郡西部為屬國都尉。京師及郡國三地震。

三年春二月丙子,東巡狩。丁丑,告陳留太守,祠南頓君、光武皇帝于濟陽,復濟陽今年田租、芻稿。庚寅,遣使者祠唐堯於成陽。

戊子,濟南上言,鳳皇集臺縣丞霍收舍樹上。賜臺長帛五十匹,丞二十匹,尉半之,吏卒人三匹。鳳皇所過亭部,無出今年田租。賜男子爵,人二級。辛卯,幸太山,柴告岱宗。齊王無忌、北海王普、樂安王延來朝。壬辰,宗祀五帝于汶上明堂。癸巳,告祀二祖、六宗,勞賜郡縣,作樂。

三月甲午,陳王崇薨。戊戌,祀孔子及七十二弟子於闕里,自魯相、令、丞、尉及孔氏親屬、婦女、諸生悉會,賜褒成侯以下帛各有差。還,幸東平,至東郡,歷魏郡、河內。壬戌,車駕還京師,幸太學。是日,太尉楊震免。

夏四月乙丑,車駕入宮。假于祖禰。壬戌,沛國言甘露降豐縣。戊辰,光祿勳馮石為太尉。

五月,南匈奴左日逐王叛,使匈奴中郎將馬翼討破之。

日南徼外蠻夷內屬。

六月,鮮卑寇玄菟。

庚午,閬中山崩。辛未,扶風言白鹿見雍。

辛巳,遣侍御史分行青冀二州災害,督錄盜賊。

秋七月丁酉,初復右校令、左校丞官。

日南徼外蠻豪帥詣闕貢獻。

馮翊言甘露降頻陽、衙。潁川上言木連理。白鹿、麒麟見陽翟。

鮮卑寇高柳。

梁王堅薨。

八月辛巳,大鴻臚耿寶為大將軍。

戊子,潁川上言麒麟一、白虎二見陽翟。

九月丁酉,廢皇太子保為濟陰王。

乙巳,詔郡國中都官死罪繫囚減罪一等,詔敦煌、隴西及度遼營;其右趾以下及亡命者贖,各有差。

辛亥,濟南上言黃龍見歷城。庚申晦,日有食之。

冬十月,行幸長安。壬午,新豐上言鳳皇集西界亭。丁亥,會三輔守、令、掾史於長安,作樂。閏月乙未,祠高廟,遂有事十一陵,歷觀上林、昆明池。遣使者祠太上皇于萬年,以中牢祠蕭何、曹參、霍光。十一月乙丑,至自長安。

十二月乙未,琅邪言黃龍見諸縣。

是歲,京師及諸郡國二十三地震;三十六雨水,疾風,雨雹。

四年春正月壬午,東郡言黃龍二、麒麟一見濮陽。

二月乙亥,下邳王衍薨。

甲辰,南巡狩。

三月戊午朔,日有食之。

庚申,幸宛,帝不豫。辛酉,令大將軍耿寶行太尉事。祠章陵園廟,告長沙、零陵太守,祠定王、節侯、鬱林府君。乙丑,自宛還。丁卯,幸葉,帝崩于乘輿,年三十二。祕不敢宣,所在上食問起居如故。庚午,還宮。辛未夕,乃發喪。尊皇后為皇太后。太后臨朝,以后兄大鴻臚閻顯為車騎將軍,定策禁中,立章帝孫濟北惠王壽子北鄉侯懿。

甲戌,濟南王香薨。

乙酉,北鄉侯即皇帝位。

夏四月丁酉,太尉馮石為太傅,司徒劉熹為太尉,參錄尚書事;前司空李郃為司徒。

辛卯,大將軍耿寶、中常侍樊豐、侍中謝惲、周廣、乳母野王君王聖,坐相阿黨,豐、惲、廣下獄死,寶自殺,聖徙鴈門。

己酉,葬孝安皇帝于恭陵。廟曰恭宗。

六月乙巳,大赦天下。詔先帝巡狩所幸,皆半入今年田租。

秋七月,西域長史班勇擊車師後王,斬之。

丙午,東海王肅薨。

冬十月丙午,越巂山崩。

辛亥,少帝薨。

是冬,京師大疫。

論曰:孝安雖稱尊享御,而權歸鄧氏,至乃損徹膳服,克念政道。然令自房帷,威不逮遠,始失根統,歸成陵敝。遂復計金授官,移民逃寇,推咎台衡,以荅天眚。既云哲婦,亦「惟家之索」矣。

贊曰:安德不升,秕我王度。降奪儲嫡,開萌邪蠹。馮石承歡,楊公逢怒。彼日而微,遂祲天路。


\end{pinyinscope}