\article{袁張韓周列傳}

\begin{pinyinscope}
袁安字邵公,汝南汝陽人也。祖父良,習孟氏易,平帝時舉明經,為太子舍人;建武初,至成武令。

安少傳良學。為人嚴重有威,見敬於州里。初為縣功曹,奉檄詣從事,從事因安致書於令。安曰:「公事自有郵驛,私請則非功曹所持。」辭不肯受,從事懼然而止。後舉孝廉,除陰平長、任城令,所在吏人畏而愛之。

永平十三年,楚王英謀為逆,事下郡覆考。明年,三府舉安能理劇,拜楚郡太守。是時英辭所連及繫者數千人,顯宗怒甚,吏案之急,迫痛自誣,死者甚眾。安到郡,不入府,先往案獄,理其無明驗者,條上出之。府丞掾史皆叩頭爭,以為阿附反虜,法與同罪,不可。安曰:「如有不合,太守自當坐之,不以相及也。」遂分別具奏。帝感悟,即報許,得出者四百餘家。歲餘,徵為河南尹。政號嚴明,然未曾以臧罪鞠人。常稱曰:「凡學仕者,高則望宰相,下則希牧守。錮人於聖世,尹所不忍為也。」聞之者皆感激自勵。在職十年,京師肅然,名重朝廷。建初八年,遷太僕。

元和二年,武威太守孟雲上書:「北虜既已和親,而南部復往抄掠,北單于謂漢欺之,謀欲犯邊。宜還其生口,以安慰之。」詔百官議朝堂。公卿皆言夷狄譎詐,求欲無猒,既得生口,當復妄自誇大,不可開許。安獨曰:「北虜遣使奉獻和親,有得邊生口者,輒以歸漢,此明其畏威,而非先違約也。雲以大臣典邊,不宜負信於戎狄,還之足示中國優貸,而使邊人得安,誠便。」司徒桓虞改議從安。太尉鄭弘、司空第五倫皆恨之。弘因大言激勵虞曰:「諸言當還生口者,皆為不忠。」虞廷叱之,倫及大鴻臚韋彪各作色變容,司隸校尉舉奏,安等皆上印綬謝。肅宗詔報曰:「久議沈滯,各有所志。蓋事以議從,策由眾定,誾誾衎衎,得禮之容,寢嘿抑心,更非朝廷之福。君何尤而深謝?其各冠履。」帝竟從安議。明年,代第五倫為司空。章和元年,代桓虞為司徒。

和帝即位,竇太后臨朝,后兄車騎將軍憲北擊匈奴,安與太尉宋由、司空任隗及九卿詣朝堂上書諫,以為匈奴不犯邊塞,而無故勞師遠涉,損費國用,徼功萬里,非社稷之計。書連上輒寢。宋由懼,遂不敢復署議,而諸卿稍自引止。唯安獨與任隗守正不移,至免冠朝堂固爭者十上。太后不聽,眾皆為之危懼,安正色自若。竇憲既出,而弟衛尉篤、執金吾景各專威權,公於京師使客遮道奪人財物。景又擅使乘驛施檄緣邊諸郡,發突騎及善騎射有才力者,漁陽、鴈門、上谷三郡各遣吏將送詣景第。有司畏憚,莫敢言者。安乃劾景擅發邊兵,驚惑吏人,二千石不待符信而輒承景檄,當伏顯誅。又奏司隸校尉、河南尹阿附貴戚,無盡節之義,請免官案罪。並寢不報。憲、景等日益橫,盡樹其親黨賓客於名都大郡,皆賦斂吏人,更相賂遺,其餘州郡,亦復望風從之。安與任隗舉奏諸二千石,又它所連及貶秩免官者四十餘人,竇氏大恨。但安、隗素行高,亦未有以害之。

時竇憲復出屯武威。明年,北單于為耿夔所破,遁走烏孫,塞北地空,餘部不知所屬。憲日矜己功,欲結恩北虜,乃上立降者左鹿蠡王阿佟為北單于,置中郎將領護,如南單于故事。事下公卿議,太尉宋由、太常丁鴻、光祿勳耿秉等十人議可許。安與任隗奏,以為「光武招懷南虜,非謂可永安內地,正以權時之筭,可得扞禦北狄故也。今朔漠既定,宜令南單于反其北庭,并領降眾,無緣復更立阿佟,以增國費」。宗正劉方、大司農尹睦同安議。事奏,未以時定。安懼憲計遂行,乃獨上封事曰:「臣聞功有難圖,不可豫見;事有易斷,較然不疑。伏惟光武皇帝本所以立南單于者,欲安南定北之策也,恩德甚備,故匈奴遂分,邊境無患。孝明皇帝奉承先意,不敢失墜,赫然命將,爰伐塞北。至乎章和之初,降者十餘萬人,議者欲置之濱塞,東至遼東,太尉宋由、光祿勳耿秉皆以為失南單于心,不可,先帝從之。陛下奉承洪業,大開疆宇,大將軍遠師討伐,席卷北庭,此誠宣明祖宗,崇立弘勳者也。宜審其終,以成厥初。伏念南單于屯,先父舉眾歸德,自蒙恩以來,四十餘年。三帝積累,以遺陛下。陛下深宜遵述先志,成就其業。況屯首唱大謀,空盡北虜,輟而弗圖,更立新降,以一朝之計,違三世之規,失信於所養,建立於無功。由、秉實知舊議,而欲背棄先恩。夫言行君子之樞機,賞罰理國之綱紀。論語曰:『言忠信,行篤敬,雖蠻貊行焉。』今若失信於一屯,則百蠻不敢復保誓矣。又烏桓、鮮卑新殺北單于,凡人之情,咸畏仇讎,今立其弟,則二虜懷怨。兵、食可廢,信不可去。且漢故事,供給南單于費直歲一億九十餘萬,西域歲七千四百八十萬。今北庭彌遠,其費過倍,是乃空盡天下,而非建策之要也。」詔下其議。安又與憲更相難折。憲險急負埶,言辭驕訐,至詆毀安,稱光武誅韓歆、戴涉故事,安終不移。憲竟立匈奴降者右鹿蠡王於除鞬為單于,後遂反叛,卒如安策。

安以天子幼弱,外戚擅權,每朝會進見,及與公卿言國家事,未嘗不噫嗚流涕。自天子及大臣皆恃賴之。四年春,薨,朝廷痛惜焉。

後數月,竇氏敗,帝始親萬機,追思前議者邪正之節,乃除安子賞為郎。策免宋由,以尹睦為太尉,劉方為司空。睦,河南人,薨於位。方,平原人,後坐事免歸,自殺。

初,安父沒,母使安訪求葬地,道逢三書生,問安何之,安為言其故,生乃指一處,云「葬此地,當世為上公」。須臾不見,安異之。於是遂葬其所占之地,故累世隆盛焉。安子京、敞最知名。

京字仲譽。習孟氏易,作難記三十萬言。初拜郎中,稍遷侍中,出為蜀郡太守。

子彭,字伯楚。少傳父業,歷廣漢、南陽太守。順帝初,為光禒勳。行至清,為吏麤袍糲食,終於議郎。尚書胡廣等追表其有清絜之美,比前朝貢禹、第五倫。未蒙顯贈,當時皆嗟歎之。

彭弟湯,字仲河,少傳家學,諸儒稱其節,多歷顯位。桓帝初為司空,以豫議定策封安國亭侯,食邑五百戶。累遷司徒、太尉,以災異策免。卒,謚曰康侯。

湯長子成,左中郎。早卒,次子逢嗣。

逢字周陽,以累世三公子,寬厚篤信,著稱於時。靈帝立,逢以太僕豫議,增封三百戶。後為司空,卒於執金吾。朝廷以逢嘗為三老,特優禮之,賜以珠畫特詔祕器,飯含珠玉二十六品,使五官中郎將持節奉策,贈以車騎將軍印綬,加號特進,謚曰宣文侯。子基嗣,位至太僕。

逢弟隗,少歷顯官,先逢為三公。時中常侍袁赦,隗之宗也,用事於中。以逢、隗世宰相家,推崇以為外援。故袁氏貴寵於世,富奢甚,不與它公族同。獻帝初,隗為太傳。

成子紹,逢子術,自有傳。董卓忿紹、術背己,遂誅隗及術兄基男女二十餘人。

敞字叔平,少傳易經教授,以父任為太子舍人。和帝時,歷位將軍、大夫、侍中,出為東郡太守,徵拜太僕、光祿勳。元初三年,代劉愷為司空。明年,坐子與尚書郎張俊交通,漏洩省中語,策免。敞廉勁不阿權貴,失鄧氏旨,遂自殺。

張俊者,蜀郡人,有才能,與兄龕並為尚書郎,年少勵鋒氣。郎朱濟、丁盛立行不脩,俊欲舉奏之,二人聞,恐,因郎陳重、雷義往請俊,俊不聽,因共私賂侍史,使求俊短,得其私書與敞子,遂封上之,皆下獄,當死。俊自獄中占獄吏上書自訟,書奏而俊獄已報。廷尉將出穀門,臨行刑,鄧太后詔馳騎以減死論。俊假名上書謝曰:「臣孤恩負義,自陷重刑,情斷意訖,無所復望。廷尉鞠遣,歐刀在前,棺絮在後,魂魄飛揚,形容已枯。陛下聖澤,以臣嘗在近密,識其狀貌,傷其眼目,留心曲慮,特加遍覆。喪車復還,白骨更肉,披棺發槨,起見白日。天地父母能生臣俊,不能使臣俊當死復生。陛下德過天地,恩重父母,誠非臣俊破碎骸骨,舉宗腐爛,所報萬一。臣俊徒也,不得上書;不勝去死就生,驚喜踊躍,觸冒拜章。」當時皆哀其文。

朝廷由此薄敞罪而隱其死,以三公禮葬之,復其官。子盱。

盱後至光祿勳。時大將軍梁冀擅朝,內外莫不阿附,唯盱與廷尉邯鄲義正身自守。及桓帝誅冀,使盱持節收其印綬,事已具梁冀傳。

閎字夏甫,彭之孫也。少勵操行,苦身脩節。父賀,為彭城相。閎往省謁,變名姓,徒行無旅。既至府門,連日吏不為通,會阿母出,見閎驚,入白夫人,乃密呼見。既而辭去,賀遣車送之,閎稱眩疾不肯乘,反,郡界無知者。及賀卒郡,閎兄弟迎喪,不受賻贈,縗絰扶柩,冒犯寒露,禮貌枯毀,手足血流,見者莫不傷之。服闋,累徵聘舉召,皆不應。居處仄陋,以耕學為業。從父逢、隗並貴盛,數饋之,無所受。

閎見時方險亂,而家門富盛,常對兄弟歎曰:「吾先公福祚,後世不能以德守之,而競為驕奢,與亂世爭權,此即晉之三郤矣。」延熹末,黨事將作,閎遂散髮絕世,欲投跡深林。以母老不宜遠遁,乃築土室,四周於庭,不為戶,自牖納飲食而已。旦於室中東向拜母。母思閎,時往就視,母去,便自掩閉,兄弟妻子莫得見也。及母歿,不為制服設位,時莫能名,或以為狂生。潛身十八年,黃巾賊起,攻沒郡縣,百姓驚散,閎誦經不移。賊相約語不入其閭,卿人就閎避難,皆得全免。年五十七,卒於土室。二弟忠、弘,節操皆亞於閎。

忠字正甫,與同郡范滂為友,俱證黨事得釋,語在滂傳。初平中,為沛相,乘葦車到官,以清亮稱。及天下大亂,忠棄官客會稽上虞。一見太守王朗徒從整飾,心嫌之,遂稱病自絕。後孫策破會稽,忠等浮海南投交阯。獻帝都許,徵為衛尉,未到,卒。

弘字邵甫,恥其門族貴埶,乃變姓名,徒步師門,不應徵辟,終於家。

忠子祕,為郡門下議生。黃巾起,祕從太守趙謙擊之,軍敗,祕與功曹封觀等七人以身扞刃,皆死於陳,謙以得免。詔祕等門閭號曰「七賢」。

封觀者,有志節,當舉孝廉,以兄名位未顯,恥先受之,遂稱風疾,喑不能言。火起觀屋,徐出避之。忍而不告。後數年,兄得舉,觀乃稱損而仕郡焉。

論曰:陳平多陰謀,而知其後必廢;邴吉有陰德,夏侯勝識其當封及子孫。終陳掌不侯,而邴昌紹國,雖有不類,未可致詰,其大致歸然矣。袁公竇氏之閒,乃情帝室,引義雅正,可謂王臣之烈。及其理楚獄,未嘗鞫人於臧罪,其仁心足以覃乎後昆。子孫之盛,不亦宜乎?

張酺字孟侯,汝南細陽人,趙王張敖之後也。敖子壽,封細陽之池陽鄉,後廢,因家焉。

酺少從祖父充受尚書,能傳其業。又事太常桓榮。勤力不怠,聚徒以百數。永平九年,顯宗為四姓小侯開學於南宮,置五經師。酺以尚書教授,數講於御前。以論難當意,除為郎,賜車馬衣裳,遂令入授皇太子。

酺為人質直,守經義,每侍講閒隙,數有匡正之辭,以嚴見憚。及肅宗即位,擢酺為侍中、虎賁中郎將。數月,出為東郡太守。酺自以嘗經親近,未悟見出,意不自得,上疏辭曰:「臣愚以經術給事左右,少不更職,不曉文法,猥當剖符典郡,班政千里,必有負恩辱位之咎。臣竊私自分,殊不慮出城闕,冀蒙留恩,託備冗官,群僚所不安,耳目所聞見,不敢避好醜。」詔報曰:「經云:『身雖在外,乃心不離王室。』典城臨民,益所以報效也。好醜必上,不在遠近。今賜裝錢三十萬,其亟之官。」酺雖儒者,而性剛斷。下車擢用義勇,搏擊豪彊。長安有殺盜徒者,酺輒案之,以為令長受臧,猶不至死,盜徒皆飢寒傭保,何足窮其法乎!

郡吏王青者,祖父翁,與前太守翟義起兵攻王莽,及義敗,餘眾悉降,翁獨守節力戰,莽遂燔燒之。父隆,建武初為都尉功曹,青為小史。與父俱從都尉行縣,道遇賊,隆以身衛全都尉,遂死於難;青亦被矢貫咽,音聲流喝。前郡守以青身有金夷,竟不能舉。酺見之,歎息曰:「豈有一門忠義而爵賞不及乎?」遂擢用極右曹,乃上疏薦青三世死節,宜蒙顯異。奏下三公,由此為司空所辟。

自酺出後,帝每見諸王師傅,常言:「張酺前入侍講,屢有諫正,誾誾惻惻,出於誠心,可謂有史魚之風矣。」元和二年,東巡狩,幸東郡,引酺及門生並郡縣掾史並會庭中。帝先備弟子之儀,使酺講尚書一篇,然後脩君臣之禮。賞賜殊特,莫不沾洽。

酺視事十五年,和帝初,遷魏郡太守。郡人鄭據時為司隸校尉,奏免執金吾竇景。景後復位,遣掾夏猛私謝酺曰:「鄭據小人,為所侵冤。聞其兒為吏,放縱狼藉。取是曹子一人,足以驚百。」酺大怒,即收猛繫獄,檄言執金吾府,疑猛與據子不平,矯稱卿意,以報私讎。會有贖罪令,猛乃得出。頃之,徵入為河南尹。竇景家人復擊傷市卒,吏捕得之,景怒,遣緹騎侯海等五百人歐傷市丞。酺部吏楊章等窮究,正海罪,徙朔方。景忿怨,乃移書辟章等六人為執金吾吏,欲因報之。章等惶恐,入白酺,願自引臧罪,以辭景命。酺即上言其狀。竇太后詔報:「自今執金吾辟吏,皆勿遣。」

及竇氏敗,酺乃上疏曰:「臣實愚惷,不及大體,以為竇氏雖伏厥辜,而罪刑未著,後世不見其事,但聞其誅,非所以垂示國典,貽之將來。宜下理官,與天下平之。方憲等寵貴,群臣阿附唯恐不及,皆言憲受顧命之託,懷伊、呂之忠,至乃復比鄧夫人於文母。今嚴威既行,皆言當死,不復顧其前後,考折厥衷。臣伏見夏陽侯瑰,每存忠善,前與臣言,常有盡節之心,檢敕賓客,未嘗犯法。臣聞王政骨肉之刑,有三宥之義,過厚不過薄。今議者為瑰選嚴能相,恐其迫切,必不完免,宜裁加貸宥,以崇厚德。」和帝感酺言,徙瑰封,就國而己。

永元五年,遷酺為太僕。數月,代尹睦為太尉。數上疏以疾乞身,薦魏郡太守徐防自代。帝不許,使中黃門問病,加以珍羞,賜錢三十萬。酺遂稱篤。時子蕃以郎侍講,帝因令小黃門敕蕃曰:「陰陽不和,萬人失所,朝廷望公思惟得失,與國同心,而託病自絜,求去重任,誰當與吾同憂責者?非有望於斷金也。司徒固疾,司空年老,公其傴僂,勿露所敕。」酺惶恐詣闕謝,還復視事。酺雖在公位,而父常居田里,酺每有遷職,輒一詣京師。嘗來候酺,適會歲節,公卿罷朝,俱詣酺府奉酒上壽,極歡卒日,眾人皆慶羨之。及父卒,既葬,詔遣使齎牛酒為釋服。

後以事與司隸校尉晏稱會於朝堂,酺從容謂稱曰:「三府辟吏,多非其人。」稱歸,即奏令三府各實其掾史。酺本以私言,不意稱奏之,甚懷恨。會復共謝闕下,酺因責讓於稱。稱辭言不順,酺怒,遂廷叱之,稱乃劾奏酺有怨言。天子以酺先帝師,有詔公卿、博士、朝臣會議。司徒呂蓋奏酺位居三司,知公門有儀,不屏氣鞠躬以須詔命,反作色大言,怨讓使臣,不可以示四遠。於是策免。

酺歸里舍,謝遣諸生,閉門不通賓客。左中郎將何敞及言事者多訟酺公忠,帝亦雅重之。十五年,復拜為光祿勳。數月,代魯恭為司徒。月餘薨。乘輿縞素臨弔,賜冢塋地,賵贈恩寵異於它相。酺病臨危,敕其子曰:「顯節陵埽地露祭,欲率天下以儉。吾為三公,既不能宣揚王化,令吏人從制,豈可不務節約乎?其無起祠堂,可作稿蓋廡,施祭其下而已。」

曾孫濟,好儒學,光和中至司空,病罷。及卒,靈帝以舊恩贈車騎將軍、關內侯印綬。其年,追濟侍講有勞,封子根為蔡陽鄉侯。

濟弟喜,初平中為司空。

韓棱字伯師,潁川舞陽人,弓高侯穨當之後也。世為鄉里著姓。父尋,建武中為隴西太守。

棱四歲而孤,養母弟以孝友稱。及壯,推先父餘財數百萬與從昆弟,鄉里益高之。初為郡功曹,太守葛興中風,病不能聽政,棱陰代興視事,出入二年,令無違者。興子嘗發教欲署吏,棱拒執不從,因令怨者章之。事下案驗,吏以棱掩蔽興病,專典郡職,遂致禁錮。顯宗知其忠,後詔特原之。由是徵辟,五遷為尚書令,與僕射郅壽、尚書陳寵,同時俱以才能稱。肅宗嘗賜諸尚書劍,唯此三人特以寶劍,自手署其名曰:「韓棱楚龍淵,郅壽蜀漢文,陳寵濟南椎成。」時論者為之說:以棱淵深有謀,故得龍淵;壽明達有文章,故得漢文;寵敦朴,善不見外,故得椎成。

和帝即位,侍中竇憲使人刺殺齊殤王子都鄉侯暢於上東門,有司畏憲,咸委疑於暢兄弟。詔遣侍御史之齊案其事。棱上疏以為賊在京師,不宜捨近問遠,恐為姦臣所笑。竇太后怒,以切責棱,棱固執其議。及事發,果如所言。憲惶恐,白太后求出擊北匈奴以贖罪。棱復上疏諫,太后不從。及憲有功,還為大將軍,威震天下,復出屯武威。會帝西祠園陵,詔憲與車駕會長安。及憲至,尚書以下議欲拜之,伏稱萬歲。棱正色曰:「夫上交不諂,下交不黷,禮無人臣稱萬歲之制。」議者皆慚而止。尚書左丞王龍私奏記上牛酒於憲,棱舉奏龍,論為城旦。棱在朝數薦舉良吏應順、呂章、周紆等,皆有名當時。及竇氏敗,棱典案其事,深竟黨與,數月不休沐。帝以為憂國忘家,賜布三百匹。

遷南陽太守,特聽棱得過家上冢,鄉里以為榮。棱發擿姦盜,郡中震慄,政號嚴平。數歲,徵入為太僕。九年冬,代張奮為司空。明年薨。

子輔,安帝時至趙相。

棱孫演,順帝時為丹陽太守,政有能名。桓帝時為司徒。大將軍梁冀被誅,演坐阿黨抵罪,以減死論,遣歸本郡。後復徵拜司隸校尉。

周榮字平孫,廬江舒人也。肅宗時,舉明經,辟司徒袁安府。安數與論議,甚器之。及安舉奏竇景及與竇憲爭立北單于事,皆榮所具草。竇氏客太尉掾徐齮深惡之,脅榮曰:「子為袁公腹心之謀,排奏竇氏,竇氏悍士刺客滿城中,謹備之矣!」榮曰:「榮江淮孤生,蒙先帝大恩,以歷宰二城。今復得備宰士,縱為竇氏所害,誠所甘心。」故常敕妻子,若卒遇飛禍,無得殯斂,冀以區區腐身覺悟朝廷。及竇氏敗,榮由此顯名。自郾令擢為尚書令。出為潁川太守,坐法,當下獄,和帝思榮忠節,左轉共令。歲餘,復以為山陽太守。所歷郡縣,皆見稱紀。以老病乞身,卒于家,詔特賜錢二十萬,除子男興為郎中。

興少有名譽,永寧中,尚書陳忠上疏薦興曰:「臣伏惟古者帝王有所號令,言必弘雅,辭必溫麗,垂於後世,列於典經。故仲尼嘉唐虞之文章,從周室之郁郁。臣竊見光祿郎周興,孝友之行,著於閨門,清厲之志,聞於州里。蘊櫝古今,博物多聞,三墳之篇,五典之策,無所不覽。屬文著辭,有可觀採。尚書出納帝命,為王喉舌。臣等既愚闇,而諸郎多文俗吏,鮮有雅才,每為詔文,宣示內外,轉相求請,或以不能而專己自由,辭多鄙固。興抱奇懷能,隨輩栖遲,誠可歎惜。」詔乃拜興為尚書郎。卒。興子景。

景字仲饗。辟大將軍梁冀府,稍遷豫州刺史、河內太守。好賢愛士,其拔才薦善,常恐不及。每至歲時,延請舉吏入上後堂,與共宴會,如此數四,乃遣之。贈送什物,無不充備。既而選其父兄子弟,事相優異。常稱曰:「臣子同貫,若之何不厚!」先是司徒韓演在河內,志在無私,舉吏當行,一辭而已,恩亦不及其家。曰:「我舉若可矣,豈可令遍積一門!」故當時論者議此二人。

景後徵入為將作大匠。及梁冀誅,景以故吏免官禁錮。朝廷以景素著忠正,頃之,復引拜尚書令。遷太僕、衛尉。六年,代劉寵為司空。是時宦官任人及子弟充塞列位。景初視事,與太尉楊秉舉奏諸姦猾,自將軍牧守以下,免者五十餘人。遂連及中常侍防東侯覽、東武陽侯具瑗,皆坐黜。朝廷莫不稱之。視事二年,以地震策免。歲餘,復代陳蕃為太尉。建寧元年薨。以豫議定策立靈帝,追封安陽鄉侯。

長子崇嗣,至甘陵相。

中子忠,少歷列位,累遷大司農。忠子暉,前為洛陽令,去官歸。兄弟好賓客,雄江淮閒,出入從車常百餘乘。及帝崩,暉聞京師不安,來候忠,董卓聞而惡之,使兵劫殺其兄弟。忠後代皇甫嵩為太尉,錄尚書事,以災異免。復為衛尉,從獻帝東歸洛陽。

贊曰:袁公持重,誠單所奉。惟德不忘,延世承寵。孟侯經博,侍言帝幙。棱、榮事君,志同鸇雀。


\end{pinyinscope}