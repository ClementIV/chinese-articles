\article{耿弇列傳}

\begin{pinyinscope}
耿弇字伯昭,扶風茂陵人也。其先武帝時,以吏二千石自鉅鹿徙焉。父況,字俠游,以明經為郎,與王莽從弟伋共學老子於安丘先生,後為朔調連率。弇少好學,習父業。常見郡尉試騎士,建旗鼓,肄馳射,由是好將帥之事,

及王莽敗,更始立,諸將略地者,前後多擅威權,輒改易守、令。況自以莽之所置,懷不自安。時弇年二十一,乃辭況奉奏詣更始,因齎貢獻,以求自固之宜。及至宋子,會王郎詐稱成帝子子輿,起兵邯鄲,弇從吏孫倉、衛包於道共謀曰:「劉子輿成帝正統,捨此不歸,遠行安之?」弇按劍曰:「子輿弊賊,卒為降虜耳。我至長安,與國家陳漁陽、上谷兵馬之用,還出太原、代郡,反覆數十日,歸發突騎以轔烏合之眾,如摧枯折腐耳。觀公等不識去就,族滅不久也。」倉、包不從,遂亡降王郎。

弇道聞光武在盧奴,乃馳北上謁,光武留署門下吏。弇因說護軍朱祐,求歸發兵,以定邯鄲。光武笑曰:「小兒曹乃有大意哉!」因數召見加恩慰。弇因從光武北至薊。聞邯鄲兵方到,光武將欲南歸,召官屬計議。弇曰:「今兵從南來,不可南行。漁陽太守彭寵,公之邑人;上谷太守,即弇父也。發此兩郡,控弦萬騎,邯鄲不足慮也。」光武官屬腹心皆不肯,曰:「死尚南首,柰何北行入囊中?」光武指弇曰:「是我北道主人也。」會薊中亂,光武遂南馳,官屬各分散。弇走昌平就況,因說況使寇恂東約彭寵,各發突騎二千匹,步兵千人。弇與景丹、寇恂及漁陽兵合軍而南,所過擊斬王郎大將、九卿、校尉以下四百餘級,得印綬百二十五,節二,斬首三萬級,定涿郡、中山、鉅鹿、清河、河閒凡二十二縣,遂及光武於廣阿。是時光武方攻王郎,傳言二郡兵為邯鄲來,眾皆恐。既而悉詣營上謁。光武見弇等,說,曰:「當與漁陽、上谷士大夫共此大功。」乃皆以為偏將軍,使還領其兵。加況大將軍、興義侯,得自置偏屓。弇等遂從拔邯鄲。

時更始徵代郡太守趙永,而況勸永不應召,令詣于光武。光武遣永復郡。永北還,而代令張曄據城反畔,乃招迎匈奴、烏桓以為援助。光武以弇弟舒為復胡將軍,使擊曄,破之。永乃得復郡。時五校賊二十餘萬北寇上谷,況與舒連擊破之,賊皆退走。

更始見光武威聲日盛,君臣疑慮,乃遣使立光武為蕭王,令罷兵與諸將有功者還長安;遣苗曾為幽州牧,韋順為上谷太守,蔡充為漁陽太守,並北之部。時光武居邯鄲宮,晝臥溫明殿。弇入造床下請閒,因說曰:「今更始失政,君臣淫亂,諸將擅命於畿內,貴戚縱橫於都內。天子之命,不出城門,所在牧守,輒自遷易,百姓不知所從,士人莫敢自安。虜掠財物,劫掠婦女,懷金玉者,至不生歸。元元叩心,更思莽朝。又銅馬、赤眉之屬數十輩,輩數十百萬,聖公不能辦也。其敗不久。公首事南陽,破百萬之軍;今定河北,北據天府之地。以義征伐,發號響應,天下可傳檄而定。天下至重,不可令它姓得之。聞使者從西方來,欲罷兵,不可從也。今吏士死亡者多,弇願歸幽州,益發精兵,以集其大計。」光武大說,乃拜弇為大將軍,與吳漢北發幽州十郡兵。弇到上谷,收韋順、蔡充斬之;漢亦誅苗曾。於是悉發幽州兵,引而南,從光武擊破銅馬、高湖、赤眉、青犢,又追尤來、大槍、五幡於元氏,弇常將精騎為軍鋒,輒破走之。光武乘勝戰慎水上,虜危急,殊死戰。時軍士疲弊,遂大敗奔還,壁范陽,數日乃振,賊亦退去,從追至容城、小廣陽、安次,連戰破之。光武還薊,復遣弇與吳漢、景丹、蓋延、朱祐、邳彤、耿純、劉植、岑彭、祭遵、堅鐔、王霸、陳俊、馬武十三將軍,追賊至潞東,及平谷,再戰,斬首萬三千餘級,遂窮追於右北平無終、土垠之閒,至浚靡而還。賊散入遼西、遼東,或為烏桓、貊人所鈔擊,略盡。

光武即位,拜弇為建威大將軍。與驃騎大將軍景丹、彊弩將軍陳俊攻厭新賊於敖倉,皆破降之。建武二年,更封好畤侯。食好畤、美陽二縣。三年,延岑自武關出攻南陽,下數城。穰人杜弘率其眾以從岑。弇與岑等戰於穰,大破之,斬首三千餘級,生獲其將士五千餘人,得印綬三百。杜弘降,岑與數騎遁走東陽。

弇從幸舂陵,因見自請北收上谷兵未發者,定彭寵於漁陽,取張豐於涿郡,還收富平、獲索,東攻張步,以平齊地。帝壯其意,乃許之。四年,詔弇進攻漁陽,弇以父據上谷,本與彭寵同功,又兄弟無在京師者,自疑,不敢獨進,上書求詣洛陽。詔報曰:「將軍出身舉宗為國,所向陷敵,功效尤著,何嫌何疑,而欲求徵?且與王常共屯涿郡,勉思方略。」況聞弇求徵,亦不自安,遣舒弟國入侍。帝善之,進封況為隃麋侯。乃命弇與建義大將軍朱祐、漢忠將軍王常等擊望都、故安西山賊十餘營,皆破之。時征虜將軍祭遵屯良鄉,驍騎將車劉喜屯陽鄉,以拒彭寵。寵遣弟純將匈奴二千餘騎,寵自引兵數萬,分為兩道以擊遵、喜。胡騎經軍都,舒襲破其眾,斬匈奴兩王,寵乃退走。況復與舒攻寵,取軍都。五年,寵死,天子嘉況功,使光祿大夫持節迎況,賜甲第,奉朝請。封牟平侯。遣弇與吳漢擊富平、獲索賊於平原,大破之,降者四萬餘人。

因詔弇進討張步。弇悉收集降卒,結部曲,置將吏,率騎都尉劉歆、太山太守陳俊引兵而東,從朝陽橋濟河以度。張步聞之,乃使其大將軍費邑軍歷下,又分兵屯祝阿,別於太山鐘城列營數十以侍弇。弇度河先擊祝阿,自旦攻城,未中而拔之,故開圍一角,令其眾得奔歸鐘城。鐘城人聞祝阿已潰,大恐懼,遂空壁亡去。費邑分遣弟敢守巨里。弇進兵先脅巨里,使多伐樹木,揚言以填塞阬塹。數日,有降者言邑聞弇欲攻巨里,謀來救之。弇乃嚴令軍中趣修攻具,宣敕諸部,後三日當悉力攻巨里城。陰緩生口,令得亡歸。歸者以弇期告邑,邑至日果自將精兵三萬餘人來救之。弇喜,謂諸將曰:「吾所以修攻具者,欲誘致邑耳。今來,適其所求也。」即分三千人守巨里,自行精兵上岡阪,乘高合戰,大破之,臨陳斬邑。既而收首級以示巨里城中,城中兇懼,費敢悉眾亡歸張步。弇復收其積聚,縱兵擊諸未下者,平四十餘營,遂定濟南。

時張步都劇,使其弟藍將精兵二萬守西安,諸郡太守合萬餘人守臨淄,相去四十里。盒進軍畫中,居二城之閒。弇視西安城小而堅,且藍兵又精,臨淄名雖大而實易攻,乃敕諸校會,後五日攻西安。藍聞之,晨夜儆守。至期夜半,弇敕諸將皆蓐食,會明至臨淄城。護軍荀梁等爭之,以為宜速攻西安。弇曰:「

不然。西安聞吾欲攻之,日夜為備;臨淄出不意而至,必驚擾,吾攻之一日必拔。拔臨淄即西安孤,張藍與步隔絕,必復亡去,所謂擊一而得二者也。若先攻西安,不卒下,頓兵堅城,死傷必多。縱能拔之,藍引軍還奔臨淄,并兵合埶,觀人虛實,吾深入敵地,後無轉輸,旬月之閒,不戰而困。諸君之言,未見其宜。」遂攻臨淄,半日拔之,入據其城。張藍聞懼,遂將其眾亡歸劇。

弇乃令軍中無得妄掠劇下,須張步至乃取之,以激怒步。步聞大笑曰:「以尤來、大彤十餘萬眾,吾皆即其營而破之。今大耿兵少於彼,又皆疲勞,何足懼乎!」乃與三弟藍、弘、壽及故大彤渠帥重異等兵號二十萬,至臨淄大城東,將攻弇。弇先出淄水上,與重異遇,突騎欲縱,弇恐挫其鋒,令步不敢進,故示弱以盛其氣,乃引歸小城,陳兵於內。步氣盛,直攻弇營,與劉歆等合戰,弇升王宮壞臺望之,視歆等鋒交,乃自引精兵以橫突步陳於東城下,大破之。飛矢中弇股,以佩刀截之,左右無知者。至暮罷。弇明旦復勒兵出。是時帝在魯,聞弇為步所攻,自往救之,未至。陳俊謂弇曰:「劇虜兵盛,可且閉營休士,以須上來。」弇曰:「乘輿且到,臣子當擊牛釃酒以待百官,反欲以賊虜遺君父邪?」乃出兵大戰,自旦及昏,復大破之,殺傷無數,城中溝塹皆滿。弇知步困將退,豫置左右翼為伏以待之。人定時,步果引去,伏兵起縱擊,追至鉅昧水上,八九十里僵尸相屬,收得輜重二千餘兩。步還劇,兄弟各分兵散去。

後數日,車駕至臨淄自勞軍,群臣大會。帝謂弇曰:「昔韓信破歷下以開基,今將軍攻祝阿以發跡,此皆齊之西界,功足相方。而韓信襲擊已降,將軍獨拔勍敵,其功乃難於信也。又田橫亨酈生,及田橫降,高帝詔衛尉不聽為仇。張步前亦殺伏隆,若步來歸命,吾當詔大司徒釋其怨,又事尤相類也。將軍前在南陽建此大策,常以為落落難合,有志者事竟成也!」弇因復追步,步奔平壽,乃肉袒負斧鑕於軍門。弇傳步詣行在所,而勒兵入據其城。樹十二郡旗鼓,令步兵各以郡人詣旗下,眾尚十餘萬,輜重七千餘兩,皆罷遣歸鄉里。弇復引兵至城陽,降五校餘黨,齊地悉平。振旅還京師。

六年,西拒隗囂,屯兵於漆。八年,從上隴。明年,與中郎將來歙分部徇安定、北地諸營保,皆下之。

弇凡所平郡四十六,屠城三百,未常挫折。

十二年,況疾病,乘輿數自臨幸。復以國弟廣、舉並為中郎將。弇兄弟六人皆垂青紫,省侍醫藥,當代以為榮。及況卒,謚烈侯,少子霸襲況爵。

十三年,增弇戶邑,上大將軍印綬,罷,以列侯奉朝請。每有四方異議,輒召入問籌策。年五十六,永平元年卒,謚曰愍侯。

子忠嗣。忠以騎都尉擊匈奴於天山,有功。忠卒,子馮嗣。馮卒,子良嗣,一名無禁。延光中,尚安帝妹濮陽長公主,位至侍中。良卒,子協嗣。

隃麋侯霸卒,子文金嗣。文金卒,子喜嗣。喜卒,子顯嗣,為羽林左監。顯卒,子援嗣。尚桓帝妹長社公主,為河陽太守。後曹操誅耿氏,唯援孫弘存焉。

牟平侯舒卒,子襲嗣。尚顯宗女隆慮公主。襲卒,子寶嗣。

寶女弟為清河孝王妃。及安帝立,尊孝王,母為孝德皇后,以妃為甘園大貴人。帝以寶元舅之重,使監羽林左車騎,位至大將軍。而附事內寵,與中常侍樊豐、帝乳母王聖等譖廢皇太子為濟陰王,及排陷太尉楊震,議者怨之。寶弟子承襲公主爵為林慮侯,位至侍中。安帝崩,閻太后以寶等阿附嬖倖,共為不道,策免寶及承,皆貶爵為亭侯,遣就國。寶於道自殺,國除。大貴人數為耿氏請,陽嘉三年,順帝遂詔封寶子箕牟平侯,為侍中。以恆為陽亭侯,承為羽林中郎將。其後貴人薨,大將軍梁冀從承求貴人珍玩,不能得,冀怒,風有司奏奪其封。承惶恐,遂亡匿於穰。數年,冀推跡得之,乃并族其家十餘人。

論曰:淮陰廷論項王,審料成埶,則知高祖之廟勝矣。弇決策河北,定計南陽,亦見光武之業成矣。然弇自剋拔全齊,而無尺寸功。夫豈不懷?將時之度數,不足以相容乎?三世為將,道家所忌,而耿氏累葉以功名自終。將其用兵欲以殺止殺乎?何其獨能隆也!

國字叔慮,建武四年初入侍,光武拜為黃門侍郎,應對左右,帝以為能,遷射聲校尉。七年,射聲官罷,拜駙馬都尉。父況卒,國於次當嗣,上疏以先侯愛少子霸,固自陳讓,有詔許焉。後歷頓丘、陽翟、上蔡令,所在吏人稱之。徵為五官中郎將。

是時烏桓、鮮卑屢寇外境,國素有籌策,數言邊事,帝器之。及匈奴薁鞬日逐王比自立為呼韓邪單于,款塞稱藩,願扞禦北虜。事下公卿。議者皆以為天下初定,中國空虛,夷狄情偽難知,不可許。國獨曰:「臣以為宜如孝宣故事受之,令東扞鮮卑,北拒匈奴,率厲四夷,完復邊郡,使塞下無晏開之警,萬世有安寧之策也。」帝從其議,遂立比為南單于。由是烏桓、鮮卑保塞自守,北虜遠遁,中國少事。二十七年,代馮勤為大司馬。又上言宜置度遼將軍,左右校尉,屯五原以防逃亡。永平元年卒官。顯宗追思國言,後遂置度遼將軍,左右校尉,如其議焉。

國二子:秉,夔。

秉字伯初,有偉體,腰帶八圍。博通書記,能說司馬兵法,尤好將帥之略。以父任為郎,數上言兵事。常以中國虛費,邊陲不寧,其患專在匈奴。以戰去戰,盛王之道。顯宗既有志北伐,陰然其言。永平中,召詣省闥,問前後所上便宜方略,拜謁者僕射,遂見親幸。每公卿會議,常引秉上殿,訪以邊事,多簡帝心。

十五年,拜駙馬都尉。十六年,以騎都尉秦彭為副,與奉車都尉竇固等俱伐北匈奴。虜皆奔走,不戰而還。

十七年夏,詔秉與固合兵萬四千騎,復出白山擊車師。車師有後王、前王,前王即後王之子,其廷相去五百餘里。固以後王道遠,山谷深,士卒寒苦,欲攻前王。秉議先赴後王,以為并力根本,則前王自服。固計未決。秉奮身而起曰:「請行前。」乃上馬,引兵北入,眾軍不得已,遂進。並縱兵抄掠,斬首數千級,收馬牛十餘萬頭。後王安得震怖,從數百騎出迎秉。而固司馬蘇安欲全功歸固,即馳謂安得曰:「漢貴將獨有奉車都尉,天子姊婿,爵為通侯,當先降之。」安得乃還,更令其諸將迎秉。秉大怒,被甲上馬,麾其精騎徑造固壁。言曰:「車師王降,訖今不至,請往梟其首。」固大驚曰:「且止,將敗事!」秉厲聲曰:「受降如受敵。」遂馳赴之。安得惶恐,走出門,脫帽抱馬足降。秉將以詣固。其前王亦歸命,遂定車師而還。

明年秋,肅宗即位,拜秉征西將軍。遣案行涼州邊境,勞賜保塞羌胡,進屯酒泉,救戊己校尉。

建初元年,拜度遼將軍。視事七年,匈奴懷其恩信。徵為執金吾,甚見親重。帝每巡郡國及幸宮觀,秉常領禁兵宿衛左右。除三子為郎。章和二年,復拜征西將軍,副車騎將軍竇憲擊北匈奴,大破之。事并事憲傳。封秉美陽侯。食邑三千戶。

秉性勇壯而簡易於事,軍行常自被甲在前,休止不結營部,然遠斥候,明要誓,有警,軍陳立成,士卒皆樂為死。永元二年,代桓虞為光祿勳。明年夏卒,時年五十餘。賜以朱棺、玉衣,將作大匠穿冢,假鼓吹,五營騎士三百餘人送葬。謚曰桓侯。匈奴聞秉卒,舉國號哭,或至災面流血。

長子沖嗣。及竇憲敗,以秉竇氏黨,國除。沖官至漢陽太守。

曾孫紀,少有美名,辟公府,曹操甚敬異之,稍遷少府。紀以操將篡漢,建安二十三年,與大醫令吉桧、丞相司直韋況晃曄謀起兵誅操,不克,夷三族。于時衣冠盛門坐紀罹禍滅者眾矣。

夔字定公。少有氣決。永元初,為車騎將軍竇憲假司馬,北擊匈奴,轉車騎都尉。三年,憲復出河西,以夔為大將軍左校尉。將精騎八百,出居延塞,直奔北單于廷,於金微山斬閼氏、名王已下五千餘級,單于與數騎脫亡,盡獲其匈奴珍寶財畜,去塞五千餘里而還,自漢出師所未嘗至也。乃封夔粟邑侯。會北單于弟左鹿蠡王於除鞬自立為單于,眾八部二萬餘人,來居蒲類海上,遣使款塞。以夔為中郎將,持節衛護之。及竇憲敗,夔亦免官奪爵土。

後復為長水校尉,拜五原太守,遷遼東太守。元興元年,貊人寇郡界,夔追擊,斬其渠帥。永初三年,南單于檀反畔,使夔率鮮卑及諸郡兵屯鴈門,與車騎將軍何熙共擊之。熙推夔為先鋒,而遣其司馬耿溥、劉祉將二千人與夔俱進。到屬國故城,單于遣薁鞬日逐王三千餘人遮漢兵。夔自擊其左,令鮮卑攻其右,虜遂敗走,追斬千餘級,殺其名王六人,獲穹廬車重千餘兩,馬畜生口甚眾。鮮卑馬多羸病,遂畔出塞。夔不能獨進,以不窮追,左轉雲中太守,後遷行度遼將軍事。

夔勇而有氣,數侵陵匈奴中郎將鄭戩。元初元年,坐徵下獄,以減死論,笞二百。建光中,復拜度遼將軍。時鮮卑攻殺雲中太守成嚴,圍烏桓校尉徐常於馬城。夔與幽州刺史龐參救之,追虜出塞而還。後坐法免,卒於家。

恭字伯宗,國弟廣之子也。少孤。慷慨多大略,有將帥才。永平十七年冬,騎都尉劉張出擊車師,請恭為司馬,與奉車都尉竇固及從弟駙馬都尉秉破降之。始置西域都護、戊己校尉,乃以恭為戊己校尉,屯後王部金蒲城,謁者關寵為戊己校尉,屯前王柳中城,屯各置數百人。恭至部,移檄烏孫,示漢威德,大昆彌已下皆歡喜,遣使獻名馬,及奉宣帝時所賜公主博具,願遣子入侍。恭乃發使齎金帛,迎其侍子。

明年三月,北單于遣左鹿蠡王二萬騎擊車師。恭遣司馬將兵三百人救之,道逢匈奴騎多,皆為所歿。匈奴遂破殺後王安得,而攻金蒲城。恭乘城搏戰,以毒藥傅矢。傳語匈奴曰:「漢家箭神,其中瘡者必有異。」因發彊弩射之。虜中矢者,視創皆沸,遂大驚。會天暴風雨,隨雨擊之,殺傷甚眾。匈奴震怖,相謂曰:「漢兵神,真可畏也!」遂解去。恭以疏勒城傍有澗水可固,五月,乃引兵據之。七月,匈奴復來攻恭,恭募先登數千人直馳之,胡騎散走,匈奴遂於城下擁絕澗水。恭於城中穿井十五丈不得水,吏士渴乏,笮馬糞汁而飲之。恭仰歎曰:「聞昔貳師將軍拔佩刀刺山,飛泉涌出;今漢德神明,豈有窮哉。」乃整衣服向井再拜,為吏士禱。有頃,水泉奔出,眾皆稱萬歲。乃令吏士揚水以示虜。虜出不意,以為神明,遂引去。

時焉耆、龜茲攻歿都護陳睦,北虜亦圍關寵於柳中。會顯宗崩,救兵不至,車師復畔,與匈奴共攻恭。恭厲士眾擊走之。後王夫人先世漢人,常私以虜情告恭,又給以糧餉。數月,食盡窮困,乃煮鎧弩,食其筋革。恭與士推誠同死生,故皆無二心,而稍稍死亡,餘數十人。單于知恭已困,欲必降之。復遣使招恭曰:「若降者,當封為白屋王,妻以女子。」恭乃誘其使上城,手擊殺之,炙諸城上。虜官屬望見,號哭而去。單于大怒,更益兵圍恭,不能下。

初,關寵上書求救,時肅宗新即位,乃詔公卿會議。司空第五倫以為不宜救。司徒鮑昱議曰:「今使人於危難之地,急而棄之,外則縱蠻夷之暴,內則傷死難之臣。誠令權時後無邊事可也,匈奴如復犯塞為寇,陛下將何以使將?又二部兵人裁各數十,匈奴圍之,歷旬不下,是其寡弱盡力之效也。可令敦煌、酒泉太守各將精騎二千,多其幡幟,倍道兼行,以赴其急。匈奴疲極之兵,必不敢當,四十日閒,足還入塞。」帝然之。乃遣征西將軍耿秉屯酒泉,行太守事;遣秦彭與謁者王蒙、皇甫援發張掖、酒泉、敦煌三郡及鄯善兵,合七千餘人,建初元年正月,會柳中擊車師,攻交河城,斬首三千八百級,獲生口三千餘人,駝驢馬牛羊三萬七千頭。北虜驚走,車師復降。

會關寵已歿,蒙等聞之,便欲引兵還。先是恭遣軍吏范羌至敦煌迎兵士寒服,羌因隨王蒙軍俱出塞。羌固請迎恭,諸將不敢前,乃分兵二千人與羌,從山北迎恭,遇大雪丈餘,軍僅能至。城中夜聞兵馬聲,以為虜來,大驚。羌乃遙呼曰:「我范羌也。漢遣軍迎校尉耳。」城中皆稱萬歲。開門,共相持涕泣。明日,遂相隨俱歸。虜兵追之,且戰且行。吏士素飢困,發疏勒時尚有二十六人,隨路死沒,三月至玉門,唯餘十三人。衣屨穿決,形容枯槁。中郎將鄭眾為恭已下洗沐易衣冠。上疏曰:「耿恭以單兵固守孤城,當匈奴之衝,對數萬之眾,連月踰年,心力困盡。鑿山為井,煮弩為糧,出於萬死無一生之望。前後殺傷醜虜數千百計,卒全忠勇,不為大漢恥。恭之節義,古今未有。宜蒙顯爵,以厲將帥。」及恭至雒陽,鮑昱奏恭節過蘇武,宜蒙爵賞。於是拜為騎都尉,以恭司馬石修為雒陽市丞,張封為雍營司馬,軍吏范羌為共丞,餘九人皆補羽林。恭母先卒,及還,追行喪制,有詔使五官中郎將齎牛酒釋服。

明年,遷長水校尉。其秋,金城、隴西羌反。恭上疏言方略,詔召入問狀。乃遣恭將五校士三千人,副車騎將軍馬防討西羌。恭屯枹罕,數與羌接戰。明年秋,燒當羌降,防還京師,恭留擊諸未服者,首虜千餘人,獲牛羊四萬餘頭,勒姐、燒何羌等十三種數萬人,皆詣恭降。初,恭出隴西,上言「故安豐侯竇融昔在西州,甚得羌胡腹心。今大鴻臚固,即其子孫。前擊白山,功冠三軍。宜奉大使,鎮撫涼部。令車騎將軍防屯軍漢陽,以為威重」。由是大忤於防。及防還,監營謁者李譚承旨奏恭不憂軍事,被詔怨望。坐徵下獄,免官歸本郡,卒於家。

子溥,為京兆虎牙都尉。元初二年,擊畔羌於丁奚城,軍敗,遂歿。詔拜溥子宏、嘩並為郎。

曄字季遇。順帝初,為烏桓校尉。時鮮卑寇緣邊,殺代郡太守。曄率烏桓及諸郡卒出塞討擊,大破之。鮮卑震怖,數萬人詣遼東降。自後頻出輒克獲,威振北方。遷度遼將軍。

耿氏自中興已後迄建安之末,大將軍二人,將軍九人,卿十三人,尚公主三人,列侯十九人,中郎將、護羌校尉及刺史、二千石數十百人,遂與漢興衰云。

論曰:余初讀蘇武傳,感其茹毛窮海,不為大漢羞。後覽耿恭疏勒之事,喟然不覺涕之無從。嗟哉,義重於生,以至是乎!昔曹子抗質於柯盟,相如申威於河表,蓋以決一旦之負,異乎百死之地也。以為二漢當疏高爵,宥十世。而蘇君恩不及嗣,恭亦終填牢戶。追誦龍蛇之章,以為歎息。

贊曰:好畤經武,能畫能兵。往收燕卒,來集漢營。請閒趙殿,釃酒齊城。況、舒率從,亦既有成。國圖久策,分此凶狄。秉洽胡情,夔單虜跡。慊慊伯宗,枯泉飛液。


\end{pinyinscope}