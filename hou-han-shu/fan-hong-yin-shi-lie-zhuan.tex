\article{樊宏陰識列傳}

\begin{pinyinscope}
樊宏字靡卿,南陽湖陽人也,世祖之舅。其先周仲山甫,封于樊,因而氏焉,為鄉里著姓。父重,字君雲,世善農稼,好貨殖。重性溫厚,有法度,三世共財,子孫朝夕禮敬,常若公家。其營理產業,物無所棄,課役童隸,各得其宜,故能上下戮力,財利歲倍,至乃開廣田土三百餘頃。其所起廬舍,皆有重堂高閣,陂渠灌注。又池魚牧畜,有求必給。嘗欲作器物,先種梓漆,時人嗤之,然積以歲月,皆得其用,向之笑者咸求假焉。貲至巨萬,而賑贍宗族,恩加鄉閭。外孫何氏兄弟爭財,重恥之,以田二頃解其忿訟。縣中稱美,推為三老。年八十餘終。其素所假貸人閒數百萬,遺令焚削文契。責家聞者皆慚,爭往償之,諸子從敕,竟不肯受。

宏少有志行。王莽末,義兵起,劉伯升與族兄賜俱將兵攻湖陽,城守不下。賜女弟為宏妻,湖陽由是收繫宏妻子,令出譬伯升,宏因留不反。湖陽軍帥欲殺其妻子,長吏以下共相謂曰:「樊重子父,禮義恩德行於鄉里,雖有罪,且當在後。」會漢兵日盛,湖陽惶急,未敢殺之,遂得免脫。更始立,欲以宏為將,宏叩頭辭曰:「書生不習兵事。」竟得免歸,與宗家親屬作營塹自守,老弱歸之者千餘家。時赤眉賊掠唐子鄉,多所殘殺,欲前攻宏營,宏遣人持牛酒米穀,勞遺赤眉。赤眉長老先聞宏仁厚,皆稱曰:「樊君素善,且今見待如此,何心攻之。」引兵而去,遂免寇難。

世祖即位,拜光祿大夫,位特進,次三公。建武五年,封長羅侯。十三年,封弟丹為射陽侯,兄子尋玄鄉侯,族兄忠更父侯。十五年,定封宏壽張侯。十八年,帝南祠章陵,過湖陽,祠重墓,追爵謚為壽張敬侯,立廟於湖陽。車駕每南巡,常幸其墓,賞賜大會。

宏為人謙柔畏慎,不求苟進。常戒其子曰:「富貴盈溢,未有能終者。吾非不喜榮埶也,天道惡滿而好謙,前世貴戚皆明戒也。保身全己,豈不樂哉!」每當朝會,輒迎期先到,俯伏待事,時至乃起。帝聞之,常敕騶騎臨朝乃告,勿令豫到。宏所上便宜及言得失,輒手自書寫,毀削草本。公朝訪逮,不敢眾對。宗族染其化,未嘗犯法。帝甚重之。及病困,車駕臨視,留宿,問其所欲言。宏頓首自陳:「無功享食大國,誠恐子孫不能保全厚恩,令臣魂神慚負黃泉,願還壽張,食小鄉亭。」帝悲傷其言,而竟不許。

二十七年,卒。遺敕薄葬,一無所用,以為棺柩一臧,不宜復見,如有腐敗,傷孝子之心,使與夫人同墳異臧。帝善其令,以書示百官,因曰:「今不順壽張侯意,無以彰其德。且吾萬歲之後,欲以為式。」賻錢千萬,布萬匹,謚為恭侯,贈以印綬,車駕親送葬。子鯈嗣。帝悼宏不已,復封少子茂為平望侯。樊氏侯者凡五國。明年,賜鯈弟鮪及從昆弟七人合錢五千萬。

論曰:昔楚頃襄王問陽陵君曰:「君子之富何如

?」對曰:「假人不德不責,食人不使不役,親戚愛之,眾人善之。」若乃樊重之折契止訟,其庶幾君子之富乎!分地以用天道,實廩以崇禮節,取諸理化,則亦可以施於政也。與夫愛而畏者,何殊閒哉!

鯈字長魚,謹約有父風。事後母至孝,及母卒,哀思過禮,毀病不自支,世祖常遣中黃門朝暮送饘粥。服闋,就侍中丁恭受公羊嚴氏春秋。建武中,禁網尚闊,諸王既長,各招引賓客,以鯈外戚,爭遣致之,而鯈清靜自保,無所交結。及沛王輔事發,貴戚子弟多見收捕,鯈以不豫得免。帝崩,鯈為復土校尉。

永平元年,拜長水校尉,與公卿雜定郊祠禮儀,以讖記正五經異說。北海周澤、琅邪承宮並海內大儒,鯈皆以為師友而致之於朝。上言郡國舉孝廉,率取年少能報恩者,耆宿大賢多見廢棄,宜敕郡國簡用良俊。又議刑辟宜須秋月,以順時氣。顯宗並從之。二年,以壽張國益東平王,徙封鯈燕侯。其後廣陵王荊有罪,帝以至親悼傷之,詔鯈與羽林監南陽任隗雜理其獄。事竟,奏請誅荊。引見宣明殿,帝怒曰:「諸卿以我弟故,欲誅之,即我子,卿等敢爾邪!」鯈仰而對曰:「天下高帝天下,非陛下之天下也。春秋之義,『君親無將,將而誅焉』。是以周公誅弟,季友鴆兄,經傳大之。臣等以荊屬託母弟,陛下留聖心,加惻隱,故敢請耳。如令陛下子,臣等專誅而已。」帝歎息良久。鯈益以此知名。其後弟鮪為子賞求楚王英女敬鄉公主,鯈聞而止之,曰:「建武時,吾家並受榮寵,一宗五侯。時特進一言,女可以配王,男可以尚主,但以貴寵過盛,即為禍患,故不為也。且爾一子,柰何棄之於楚乎?」鮪不從。

十年鯈卒,賵贈甚厚,謚曰哀侯。帝遣小黃門張音問所遺言。先是河南縣亡失官錢,典負者坐死及罪徙者甚眾,遂委責於人,以償其秏。鄉部吏司因此為姦,鯈常疾之。又野王歲獻甘醪、膏坛,每輒擾人,吏以為利。鯈並欲奏罷之,疾病未及得上。音歸,具以聞,帝覽之而悲歎,敕二郡並令從之。

長子汜嗣,以次子郴、梵為郎。其後楚事發覺,帝追念鯈謹恪,又聞其止鮪婚事,故其諸子得不坐焉。

梵字文高,為郎二十餘年,三署服其重慎。悉推財物二千餘萬與孤兄子,官至大鴻臚。

汜卒,子時嗣。時卒,子建嗣。建卒,無子,國絕。永寧元年,鄧太后復封建弟盼。盼卒,子尚嗣。

初,鯈刪定公羊嚴氏春秋章句,世號「樊侯學」,教授門徒前後三千餘人。弟子潁川李脩、九江夏勤,皆為三公。勤字伯宗,為京、宛二縣令,零陵太守,所在有理能稱。安帝時,位至司徒。

準字幼陵,宏之族曾孫也。父瑞,好黃老言,清靜少欲。準少勵志行,修儒術,以先父產業數百萬讓孤兄子。永元十五年,和帝幸南陽,準為郡功曹,召見,帝器之,拜郎中,從車駕還宮,特補尚書郎。鄧太后臨朝,儒學陵替,準乃上疏曰:

臣聞賈誼有言,「人君不可以不學」。故雖大舜聖德,孳孳為善;成王賢主,崇明師傅。及光武皇帝受命中興,群雄崩擾,旌旗亂野,東西誅戰,不遑啟處,然猶投戈講蓺,息馬論道。至孝明皇帝,兼天地之姿,用日月之明,庶政萬機,無不簡心,而垂情古典,游意經蓺,每饗射禮畢,正坐自講,諸儒並聽,四方欣欣。雖闕里之化,矍相之事,誠不足言。又多徵名儒,以充禮官,如沛國趙孝、琅邪承宮等,或安車結駟,告歸鄉里;或豐衣博帶,從見宗廟。其餘以經術見優者,布在廊廟。故朝多皤皤之良,華首之老。每讌會,則論難衎衎,共求政化。詳覽群言,響如振玉。朝者進而思政,罷者退而備問。小大隨化,雍雍可嘉。朝門羽林介冑之士,悉通孝經。博士議郎,一人開門,徒眾百數。化自聖躬,流及蠻荒,匈奴遣伊秩訾王大車且渠來入就學。八方肅清,上下無事。是以議者每稱盛時,咸言永平。

今學者蓋少,遠方尤甚。博士倚席不講,儒者競論浮麗,忘謇謇之忠,習諓諓之辭。文吏則去法律而學詆欺,銳錐刀之鋒,斷刑辟之重,德陋俗薄,以致苛刻。昔孝文竇后性好黃老,而清靜之化流景武之閒。臣愚以為宜下明詔,博求幽隱,發揚巖穴,寵進儒雅,有如孝、宮者,徵詣公車,以俟聖上講習之期。公卿各舉明經及舊儒子孫,進其爵位,使纘其業。復召郡國書佐,使讀律令。如此,則延頸者日有所見,傾耳者月有所聞。伏願陛下推述先帝進業之道。

太后深納其言,是後屢舉方正、敦樸、仁賢之士。

準再遷御史中丞。永初之初,連年水旱災異,郡國多被飢困,準上疏曰:

臣聞傳曰:「飢而不損茲曰太,厥災水。」春秋穀梁傳曰:「五穀不登,謂之大侵。大侵之禮,百官備而不製,群神禱而不祠。」由是言之,調和陰陽,寔在儉節。朝廷雖勞心元元,事從省約,而在職之吏,尚未奉承。夫建化致理,由近及遠,故《詩》曰「京師翼翼,四方是則」。今可先令太官、尚方、考功、上林池烃諸官,實減無事之物,五府調省中都官吏京師作者。如此,則化及四方,人勞省息。

伏見被災之郡,百姓凋殘,恐非賑給所能勝贍,雖有其名,終無其實。可依征和元年故事,遣使持節慰安。尤困乏者,徙置荊、揚孰郡,既省轉運之費,且令百姓各安其所。今雖有西屯之役,宜先東州之急。如遣使者與二千石隨事消息,悉留富人守其舊土,轉尤貧者過所衣食,誠父母之計也。願以臣言下公卿平議。

太后從之,悉以公田賦與貧人。即擢準與議郎呂倉並守光祿大夫,準使冀州,倉使兗州。準到部,開倉稟食,慰安生業,流人咸得蘇息。還,拜鉅鹿太守。時飢荒之餘,人庶流迸,家戶且盡,準課督農桑,廣施方略,期年閒,穀粟豐賤數十倍。而趙、魏之郊數為羌所鈔暴,準外禦寇虜,內撫百姓,郡境以安。

五年,轉河內太守。時羌復屢入郡界,準輒將兵討逐,修理塢壁,威名大行。視事三年,以疾徵,三轉為尚書令,明習故事,遂見任用。元初三年,代周暢為光祿勳。五年,卒於官。

陰識字次伯,南陽新野人也,光烈皇后之前母兄也。其先出自管仲,管仲七世孫修,自齊適楚,為陰大夫,因而氏焉。秦漢之際,始家新野。

及劉伯升起義兵,識時游學長安,聞之,委業而歸,率子弟、宗族、賓客千餘人往詣伯升。伯升乃以識為校尉。更始元年,遷偏將軍,從攻宛,別降新野、淯陽、杜衍、冠軍、胡陽。二年,更始封識陰德侯,行大將軍事。

建武元年,光武遣使迎陰貴人於新野,并徵識。識隨貴人至,以為騎都尉,更封陰鄉侯。二年,以征伐軍功增封,識叩頭讓曰:「天下初定,將帥有功者眾,臣託屬掖廷,仍加爵邑,不可以示天下。」帝甚美之,以為關都尉,鎮函谷。遷侍中,以母憂辭歸。十五年,定封原鹿侯。及顯宗立為皇太子,以識守執金吾,輔導東宮。帝每巡郡國,識常留鎮守京師,委以禁兵。入雖極言正議,及與賓客語,未嘗及國事。帝敬重之,常指識以敕戒貴戚,激厲左右焉。識所用掾史皆簡賢者,如虞延、傅寬、薛愔等,多至公卿校尉。

顯宗即位,拜為執金吾,位特進。永平二年,卒,贈以本官印綬,謚曰貞侯。

子躬嗣。躬卒,子璜嗣。永初七年,為奴所殺,無子,國絕。永寧元年,鄧太后以璜弟淑紹封。淑卒,子鮪嗣。

躬弟子綱女為和帝皇后,封綱吳房侯,位特進,三子軼、輔、敞,皆黃門侍郎。后坐巫蠱事廢,綱自殺,輔下獄死,軼、敞徙日南。識弟興。

興字君陵,光烈皇后母弟也,為人有膂力。建武二年,為黃門侍郎,守期門僕射,典將武騎,從征伐,平定郡國。興每從出入,常操持小蓋,障翳風雨,躬履塗泥,率先期門。光武所幸之處,輒先入清宮,甚見親信。雖好施接賓,然門無俠客。與同郡張宗、上谷鮮于裒不相好,知其有用,猶稱所長而達之;友人張汜、杜禽與興厚善,以為華而少實,但私之以財,終不為言:是以世稱其忠平。第宅苟完,裁蔽風雨。

九年,遷侍中,賜爵關內侯。帝後召興,欲封之,置印綬於前,興固讓曰:「臣未有先登陷陣之功,而一家數人並蒙爵土,令天下觖望,誠為盈溢。臣蒙陛下、貴人恩澤至厚,富貴已極,不可復加,至誠不願。」帝嘉興之讓,不奪其志。貴人問其故,興曰:「貴人不讀書記邪?『亢龍有悔。』夫外戚家苦不知謙退,嫁女欲配侯王,取婦眄睨公主,愚心實不安也。富貴有極,人當知足,夸奢益為觀聽所譏。」貴人感其言,深自降挹,卒不為宗親求位。十九年,拜衛尉,亦輔導皇太子。明年夏,帝風眩疾甚,後以興領侍中,受顧命於雲臺廣室。會疾瘳,召見興,欲以代吳漢為大司馬。興叩頭流涕,固讓曰:「臣不敢惜身,誠虧損聖德,不可苟冒。」至誠發中,感動左右,帝遂聽之。

二十三年,卒,時年三十九。興素與從兄嵩不相能,然敬其威重。興疾病,帝親臨,問以政事及群臣能不。興頓首曰:「臣愚不足以知之。然伏見議郎席廣、謁者陰嵩,並經行明深,踰於公卿。」興沒後,帝思其言,遂擢廣為光祿勳;嵩為中郎將,監羽林十餘年,以謹敕見幸。顯宗即位,拜長樂衛尉,遷執金吾。

永平元年詔曰:「故侍中衛尉關內侯興,典領禁兵,從平天下,當以軍功顯受封爵,又諸舅比例,應蒙恩澤,興皆固讓,安乎里巷。輔導朕躬,有周昌之直,在家仁孝,有曾、閔之行,不幸早卒,朕甚傷之。賢者子孫,宜加優異。其以汝南之鮦陽封興子慶為鮦陽侯,慶弟博為濦強侯。」博弟員、丹並為郎,慶推田宅財物悉與員、丹。帝以慶義讓,擢為黃門侍郎。慶卒,子琴嗣。建初五年,興夫人卒,肅宗使五官中郎將持節即墓賜策,追謚興曰翼侯。琴卒,子萬全嗣。萬全卒,子桂嗣。

興弟就,嗣父封宣恩侯,後改封為新陽侯。就善談論,朝臣莫及,然性剛傲,不得眾譽。顯宗即位,以就為少府,位特進。就子豐尚酈邑公主。公主嬌妒,豐亦狷急。永平二年,遂殺主,被誅,父母當坐,皆自殺,國除。帝以舅氏故,不極其刑。

陰氏侯者凡四人。初,陰氏世奉管仲之祀,謂為「相君」。宣帝時,陰子方者,至孝有仁恩,臘日晨炊而灶神形見,子方再拜受慶。家有黃羊,因以祀之。自是已後,暴至巨富,田有七百餘頃,輿馬僕隸,比於邦君。子方常言「我子孫必將彊大」,至識三世而遂繁昌,故後常以臘日祀灶,而薦黃羊焉。

贊曰:權族好傾,后門多毀。樊氏世篤,陰亦戒侈。恂恂苗胤,傳龜襲紫。


\end{pinyinscope}