\article{百官一}

\begin{pinyinscope}
太傅太尉司徒司空將軍

漢之初興,承繼大亂,兵不及戢,法度草創,略依秦制,後嗣因循。至景帝,感吳楚之難,始抑損諸侯王。及至武帝,多所改作,然而奢廣,民用匱乏。世祖中興,務從節約,并官省職,費減億計,所以補復殘缺,及身未改,而四海從風,中國安樂者也。

昔周公作周官,分職著明,法度相持,王室雖微,猶能久存。今其遺書,所以觀周室牧民之德既至,又其有益來事之範,殆未有所窮也。故新汲令王隆作小學漢官篇,諸文倜說,較略不究。唯班固著百官公卿表,記漢承秦置官本末,訖于王莽,差有條貫;然皆孝武奢廣之事,又職分未悉。世祖節約之制,宜為常憲,故依其官簿,粗注職分,以為百官志。凡置官之本,及中興所省,無因復見者,既在漢書百官表,不復悉載。

太傅,上公一人。本注曰:掌以善導,無常職。世祖以卓茂為太傅,薨,因省。其後每帝初即位,輒置太傅錄尚書事,薨,輒省。

太尉,公一人。本注曰:掌四方兵事功課,歲盡即奏其殿最而行賞罰。凡郊祀之事,掌亞獻;大喪則告謚南郊。凡國有大造大疑,則與司徒、司空通而論之。國有過事,則與二公通諫爭之。世祖即位,為大司馬。建武二十七年,改為太尉。

長史一人,千石。本注曰:署諸曹事。

掾史屬二十四人。本注曰:漢舊注東西曹掾比四百石,餘掾比三百石,屬比二百石,故曰公府掾,比古元士三命者也。或曰,漢初掾史辟,皆上言之,故有秩比命士。其所不言,則為百石屬。其後皆自辟除,故通為百石云。西曹主府史署用。東曹主二千石長吏遷除及軍吏。戶曹主民戶、祠祀、農桑。奏曹主奏議事。辭曹主辭訟事。法曹主郵驛科程事。尉曹主卒徒轉運事。賊曹主盜賊事。決曹主罪法事。兵曹主兵事。金曹主貨幣、鹽、鐵事。倉曹主倉穀事。黃閤主簿錄省眾事。

令史及御屬二十三人。本注曰:漢舊注公令史百石,自中興以後,注不說石數。御屬主為公御。閤下令史主閤下威儀事。記室令史主上章表報書記。門令史主府門。其餘令史,各典曹文書。

司徒,公一人。本注曰:掌人民事。凡教民孝悌、遜順、謙儉,養生送死之事,則議其制,建其度。凡四方民事功課,歲盡則奏其殿最而行賞罰。凡郊祀之事,掌省牲視濯,大喪則掌奉安梓宮。凡國有大疑大事,與太尉同。世祖即位,為大司徒,建武二十七年,去「大」。

長史一人,千石。掾屬三十一人。令史及御屬三十六人。本注曰:世祖即位,以武帝故事,置司直,居丞相府,助督錄諸州,建武十八年省也。

司空,公一人。本注曰:掌水土事。凡營城起邑、浚溝洫、修墳防之事,則議其利,建其功。凡四方水土功課,歲盡則奏其殿最而行賞罰。凡郊祀之事,掌掃除樂器,大喪則掌將校復土。凡國有大造大疑,諫爭,與太尉同。世祖即位,為大司空,建武二十七年,去「大」。

屬長史一人,千石。掾屬二十九人。令史及御屬四十二人。

將軍,不常置。本注曰:掌征伐背叛。比公者四:第一大將軍,次驃騎將軍,次車騎將軍,次衛將軍。又有前、後、左、右將軍。

初,武帝以衛青數征伐有功,以為大將軍,欲尊寵之。以古尊官唯有三公,皆將軍始自秦、晉,以為卿號,故置大司馬官號以冠之。其後霍光、王鳳等皆然。成帝綏和元年,賜大司馬印綬,罷將軍官。世祖中興,吳漢以大將軍為大司馬,景丹為驃騎大將軍,位在公下,及前、後、左、右雜號將軍眾多,皆主征伐,事訖皆罷。明帝初即位,以弟東平王蒼有賢才,以為驃騎將軍;以王故,位在公上,數年後罷。章帝即位,西羌反,故以舅馬防行車騎將軍征之,還後罷。和帝即位,以舅竇憲為車騎將軍,征匈奴,位在公下;還復有功,遷大將軍,位在公上;復征西羌,還免官,罷。安帝即位,西羌寇亂,復以舅鄧騭為車騎將軍征之,還遷大將軍,位如憲,數年復罷。自安帝政治衰缺,始以嫡舅耿寶為大將軍,常在京都。順帝即位,又以皇后父、兄、弟相繼為大將軍,如三公焉。

長史、司馬皆一人,千石。本注曰:司馬主兵,如太尉。從事中郎二人,六百石。本注曰:職參謀議。掾屬二十九人。令史及御屬三十一人。本注曰:此皆府員職也。又賜官騎三十人,及鼓吹。

其領軍皆有部曲。大將軍營五部,部校尉一人,比二千石;軍司馬一人,比千石。部下有曲,曲有軍候一人,比六百石。曲下有純,純長一人,比二百石。其不置校尉部,但軍司馬一人。又有軍假司馬、假候,皆為副貳。其別營領屬為別部司馬,其兵多少各隨時宜。門有門候。其餘將軍,置以征伐,無員職,亦有部曲、司馬、軍候以領兵。其職吏部集各一人,總知營事。兵曹掾史主兵事器械。稟假掾史主稟假禁司。又置外刺、刺姦,主罪法。

明帝初置度遼將軍,以衛南單于眾新降有二心者,後數有不安,遂為常守。


\end{pinyinscope}