\article{光武帝紀上}

\begin{pinyinscope}
世祖光武皇帝諱秀,字文叔,南陽蔡陽人,高祖九世之孫也,出自景帝生長沙定王發。發生舂陵節侯買,買生鬱林太守外,外生鉅鹿都尉回,回生南頓令欽,欽生光武。光武年九歲而孤,養於叔父良。身長七尺三寸,美須眉,大口,隆準,日角。性勤於稼穡,而兄伯升好俠養士,常非笑光武事田業,比之高祖兄仲。王莽天鳳中,乃之長安,受尚書,略通大義。

莽末,天下連歲災蝗,寇盜鋒起。地皇三年,南陽荒饑,諸家賓客多為小盜。光武避吏新野,因賣穀於宛。宛人李通等以圖讖說光武云:「劉氏復起,李氏為輔。」光武初不敢當,然獨念兄伯升素結輕客,必舉大事,且王莽敗亡已兆,天下方亂,遂與定謀,於是乃市兵弩。十月,與李通從弟軼等起於宛,時年二十八。

十一月,有星孛于張。光武遂將賓客還舂陵。時伯升已會眾起兵。初,諸家子弟恐懼,皆亡逃自匿,曰「伯升殺我」。及見光武絳衣大冠,皆驚曰「謹厚者亦復為之」,乃稍自安。伯升於是招新市、平林兵,與其帥王鳳、陳牧西擊長聚。光武初騎牛,殺新野尉乃得馬。進屠唐子鄉,又殺湖陽尉。軍中分財物不均,眾恚恨,欲反攻諸劉。光武斂宗人所得物,悉以與之,眾乃悅。進拔棘陽,與王莽前隊大夫甄阜、屬正梁丘賜戰於小長安,漢軍大敗,還保棘陽。

更始元年正月甲子朔,漢軍復與甄阜、梁丘賜戰於沘水西,大破之,斬阜、賜。伯升又破王莽納言將軍嚴尤、秩宗將軍陳茂於淯陽,進圍宛城。

二月辛巳,立劉聖公為天子,以伯升為大司徒,光武為太常偏將軍。

三月,光武別與諸將徇昆陽、定陵、郾,皆下之。多得牛馬財物,穀數十萬斛,轉以饋宛下。莽聞阜、賜死,漢帝立,大懼,遣大司徒王尋、大司空王邑將兵百萬,其甲士四十二萬人,五月,到潁川,復與嚴尤、陳茂合。初,光武為舂陵侯家訟逋租於尤,尤見而奇之。及是時,城中出降尤者言光武不取財物,但會兵計策。尤笑曰:「是美須眉者邪?何為乃如是!」

初,王莽徵天下能為兵法者六十三家數百人,並以為軍吏;選練武衛,招募猛士,旌旗輜重,千里不絕。時有長人巨無霸,長一丈,大十圍,以為壘尉;又驅諸猛獸虎豹犀象之屬,以助威武。自秦、漢出師之盛,未嘗有也。光武將數千兵,徼之於陽關。諸將見尋、邑兵盛,反走,馳入昆陽,皆惶怖,憂念妻孥,欲散歸諸城。光武議曰:「今兵穀既少,而外寇彊大,并力禦之,功庶可立;如欲分散,埶無俱全。且宛城未拔,不能相救,昆陽即破,一日之閒,諸部亦滅矣。今不同心膽共舉功名,反欲守妻子財物邪?」諸將怒曰:「劉將軍何敢如是!」光武笑而起。會候騎還,言大兵且至城北,軍陳數百里,不見其後。諸將遽相謂曰:「更請劉將軍計之。」光武復為圖畫成敗。諸將憂迫,皆曰「諾」。時城中唯有八九千人,光武乃使成國上公王鳳、廷尉大將軍王常留守,夜自與驃騎大將軍宗佻、五威將軍李軼等十三騎,出城南門,於外收兵。時莽軍到城下者且十萬,光武幾不得出。既至郾、定陵,悉發諸營兵,而諸將貪惜財貨,欲分留守之。光武曰:「今若破敵,珍寶萬倍,大功可成;如為所敗,首領無餘,何財物之有!」眾乃從。

嚴尤說王邑曰:「昆陽城小而堅,今假號者在宛,亟進大兵,彼必奔走;宛敗,昆陽自服。」邑曰:「吾昔以虎牙將軍圍翟義,坐不生得,以見責讓。今將百萬之眾,遇城而不能下,何謂邪?」遂圍之數十重,列營百數,雲車十餘丈,瞰臨城中,旗幟蔽野,埃塵連天,鉦鼓之聲聞數百里。或為地道,衝輣橦城。積弩亂發,矢下如雨,城中負戶而汲。王鳳等乞降,不許。尋、邑自以為功在漏刻,意氣甚逸。夜有流星墜營中,晝有雲如壞山,當營而隕,不及地尺而散,吏士皆厭伏。

六月己卯,光武遂與營部俱進,自將步騎千餘,前去大軍四五里而陳。尋、邑亦遣兵數千合戰。光武奔之,斬首數十級。諸部喜曰:「劉將軍平生見小敵怯,今見大敵勇,甚可怪也,且復居前。請助將軍!」光武復進,尋、邑兵卻,諸部共乘之,斬首數百千級。連勝,遂前。時伯升拔宛已三日,而光武尚未知,乃偽使持書報城中,云「宛下兵到」,而陽墯其書。尋、邑得之,不憙。諸將既經累捷,膽氣益壯,無不一當百。光武乃與敢死者三千人,從城西水上衝其中堅,尋、邑陳亂,乘銳崩之,遂殺王尋。城中亦鼓譟而出,中外合埶,震呼動天地,莽兵大潰,走者相騰踐,奔殪百餘里閒。會大雷風,屋瓦皆飛,雨下如注,滍川盛溢,虎豹皆股戰,士卒爭赴,溺死者以萬數,水為不流。王邑、嚴尤、陳茂輕騎乘死人度水逃去。盡獲其軍實輜重,車甲珍寶,不可勝筭,舉之連月不盡,或燔燒其餘。

光武因復徇下潁陽。會伯升為更始所害,光武自父城馳詣宛謝。司徒官屬迎弔光武,光武難交私語,深引過而已。未嘗自伐昆陽之功,又不敢為伯升服喪,飲食言笑如平常。更始以是慚,拜光武為破虜大將軍,封武信侯。

九月庚戌,三輔豪桀共誅王莽,傳首詣宛。

更始將北都洛陽,以光武行司隸校尉,使前整修宮府。於是置僚屬,作文移,從事司察,一如舊章。時三輔吏士東迎更始,見諸將過,皆冠幘,而服婦人衣,諸于繡镼,莫不笑之,或有畏而走者。及見司隸僚屬,皆歡喜不自勝。老吏或垂涕曰:「不圖今日復見漢官威儀!」由是識者皆屬心焉。

及更始至洛陽,乃遣光武以破虜將軍行大司馬事。十月,持節北度河,鎮慰州郡。所到部縣,輒見二千石、長吏、三老、官屬,下至佐史,考察黜陟,如州牧行部事。輒平遣囚徒,除王莽苛政,復漢官名。吏人喜悅,爭持牛酒迎勞。

進至邯鄲,故趙繆王子林說光武曰:「赤眉今在河東,但決水灌之,百萬之眾可使為魚。」光武不荅,去之真定。林於是乃詐以卜者王郎為成帝子子輿,十二月,立郎為天子,都邯鄲,遂遣使者降下郡國。

二年正月,光武以王郎新盛,乃北徇薊。王郎移檄購光武十萬戶,而故廣陽王子劉接起兵薊中以應郎,城內擾亂,轉相驚恐,言邯鄲使者方到,二千石以下皆出迎。於是光武趣駕南轅,晨夜不敢入城邑,舍食道傍。至饒陽,官屬皆乏食。光武乃自稱邯鄲使者,入傳舍。傳吏方進食,從者飢,爭奪之。傳吏疑其偽,乃椎鼓數十通,紿言邯鄲將軍至,官屬皆失色。光武升車欲馳;既而懼不免,徐還坐,曰:「請邯鄲將軍入。」久乃駕去。傳中人遙語門者閉之。門長曰:「天下詎可知,而閉長者乎?」遂得南出。晨夜兼行,蒙犯霜雪,天時寒,面皆破裂。至呼沱河,無船,適遇冰合,得過,未畢數車而陷。進至下博城西,遑惑不知所之。有白衣老父在道旁,指曰:「努力!信都郡為長安守,去此八十里。」光武即馳赴之,信都太守任光開門出迎。世祖因發旁縣,得四千人,先擊堂陽、貰縣,皆降之。王莽和戎卒正邳彤亦舉郡降。又昌城人劉植,宋子人耿純,各率宗親子弟,據其縣邑,以奉光武。於是北降下曲陽,眾稍合,樂附者至有數萬人。

復北擊中山,拔盧奴。所過發奔命兵,移檄邊部,共擊邯鄲,郡縣還復響應。南擊新市、真定、元氏、防子,皆下之,因入趙界。

時王郎大將李育屯柏人,漢兵不知而進,前部偏將朱浮、鄧禹為育所破,亡失輜重。光武在後聞之,收浮、禹散卒,與育戰於郭門,大破之,盡得其所獲。育還保城,攻之不下,於是引兵拔廣阿。會上谷太守耿況、漁陽太守彭寵各遣其將吳漢、寇恂等將突騎來助擊王郎,更始亦遣尚書僕射謝躬討郎,光武因大饗士卒,遂東圍鉅鹿。王郎守將王饒堅守,月餘不下。郎遣將倪宏、劉奉率數萬人救鉅鹿,光武逆戰於南讀,斬首數千級。四月,進圍邯鄲,連戰破之。五月甲辰,拔其城,誅王郎。收文書,得吏人與郎交關謗毀者數千章。光武不省,會諸將軍燒之,曰:「令反側子自安。」

更始遣侍御史持節立光武為蕭王,悉令罷兵詣行在所。光武辭以河北未平,不就徵。自是始貳於更始。

是時長安政亂,四方背叛。梁王劉永擅命睢陽,公孫述稱王巴蜀,李憲自立為淮南王,秦豐自號楚黎王,張步起琅邪,董憲起東海,延岑起漢中,田戎起夷陵,並置將帥,侵略郡縣。又別號諸賊銅馬、大肜、高湖、重連、鐵脛、大搶、尤來、上江、青犢、五校、檀鄉、五幡、五樓、富平、獲索等,各領部曲,眾合數百萬人,所在寇掠。

光武將擊之,先遣吳漢北發十郡兵。幽州牧苗曾不從,漢遂斬曾而發其眾。秋,光武擊銅馬於鄡,吳漢將突騎來會清陽。賊數挑戰,光武堅營自守;有出鹵掠者,輒擊取之,絕其糧道。積月餘日,賊食盡,夜遁去,追至館陶,大破之。受降未盡,而高湖、重連從東南來,與銅馬餘眾合,光武復與大戰於蒲陽,悉破降之,封其渠帥為列侯。降者猶不自安,光武知其意,敕令各歸營勒兵,乃自乘輕騎按行部陳。降者更相語曰:「蕭王推赤心置人腹中,安得不投死乎!」由是皆服。悉將降人分配諸將,眾遂數十萬,故關西號光武為「銅馬帝」。赤眉別帥與大肜、青犢十餘萬眾在射犬,光武進擊,大破之,眾皆散走。使吳漢、岑彭襲殺謝躬於鄴。

青犢、赤眉賊入函谷關,攻更始。光武乃遣鄧禹率六裨將引兵而西,以乘更始、赤眉之亂。時更始使大司馬朱鮪、舞陰王李軼等屯洛陽,光武亦令馮異守孟津以拒之。

建武元年春正月,平陵人方望立前孺子劉嬰為天子,更始遣丞相李松擊斬之。

光武北擊尤來、大搶、五幡於元氏,追至右北平,連破之。又戰於順水北,乘勝輕進,反為所敗。賊追急,短兵接,光武自投高岸,遇突騎王豐,下馬授光武,光武撫其肩而上,顧笑謂耿弇曰:「幾為虜嗤。」弇頻射卻賊,得免。士卒死者數千人,散兵歸保范陽。軍中不見光武,或云已歿,諸將不知所為。吳漢曰:「卿曹努力!王兄子在南陽,何憂無主?」眾恐懼,數日乃定。賊雖戰勝,而素懾大威,客主不相知,夜遂引去。大軍復進至安次,與戰,破之,斬首三千餘級。賊入漁陽,乃遣吳漢率耿弇、陳俊、馬武等十二將軍追戰于潞東,及平谷,大破滅之。

朱鮪遣討難將軍蘇茂攻溫,馮異、寇恂與戰,大破之,斬其將賈彊。

於是諸將議上尊號。馬武先進曰:「天下無主。如有聖人承敝而起,雖仲尼為相,孫子為將,猶恐無能有益。反水不收,後悔無及。大王雖執謙退,柰宗廟社稷何!宜且還薊即尊位,乃議征伐。今此誰賊而馳騖擊之乎?」光武驚曰:「何將軍出是言?可斬也!」武曰:「諸將盡然。」光武使出曉之,乃引軍還至薊。

夏四月,公孫述自稱天子。

光武從薊還,過范陽,命收葬吏士。至中山,諸將復上奏曰:「

漢遭王莽,宗廟廢絕,豪傑憤怒,兆人塗炭。王與伯升首舉義兵,更始因其資以據帝位,而不能奉承大統,敗亂綱紀,盜賊日多,群生危蹙。大王初征昆陽,王莽自潰;後拔邯鄲,北州弭定;參分天下而有其二,跨州據土,帶甲百萬。言武力則莫之敢抗,論文德則無所與辭。臣聞帝王不可以久曠,天命不可以謙拒,惟大王以社稷為計,萬姓為心。」光武又不聽。

行到南平棘,諸將復固請之。光武曰:「寇賊未平,四面受敵,何遽欲正號位乎?諸將且出。」耿純進曰:「天下士大夫捐親戚,棄土壤,從大王於矢石之閒者,其計固望其攀龍鱗,附鳳翼,以成其所志耳。今功業即定,天人亦應,而大王留時逆眾,不正號位,純恐士大夫望絕計窮,則有去歸之思,無為久自苦也。大眾一散,難可復合。時不可留,眾不可逆。」純言甚誠切,光武深感,曰:「吾將思之。」

行至鄗,光武先在長安時同舍生彊華自關中奉赤伏符,曰「劉秀發兵捕不道,四夷雲集龍鬥野,四七之際火為主」。群臣因復奏曰:「受命之符,人應為大,萬里合信,不議同情,周之白魚,曷足比焉?今上無天子,海內淆亂,符瑞之應,昭然著聞,宜荅天神,以塞群望。」光武於是命有司設壇場於鄗南千秋亭五成陌。

六月己未,即皇帝位。燔燎告天,禋于六宗,望於群神。其祝文曰:「皇天上帝,后土神祇,眷顧降命,屬秀黎元,為人父母,秀不敢當。群下百辟,不謀同辭,咸曰:『王莽篡位,秀發憤興兵,破王尋、王邑於昆陽,誅王郎、銅馬於河北,平定天下,海內蒙恩。上當天地之心,下為元元所歸。』讖記曰:『劉秀發兵捕不道,卯金修德為天子。』秀猶固辭,至于再,至于三。群下僉曰:『皇天大命,不可稽留。』敢不敬承。」於是建元為建武,大赦天下,改鄗為高邑。

是月,赤眉立劉盆子為天子。

甲子,前將軍鄧禹擊更始定國公王匡於安邑,大破之,斬其將劉均。

秋七月辛未,拜前將軍鄧禹為大司徒。丁丑,以野王令王梁為大司空。壬午,以大將軍吳漢為大司馬,偏將軍景丹為驃騎大將軍,大將軍耿弇為建威大將軍,偏將軍蓋延為虎牙大將軍,偏將軍朱祐為建義大將軍,中堅將軍杜茂為大將軍。

時宗室劉茂自號「厭新將軍」,率眾降,封為中山王。

己亥,幸懷。遣耿弇率彊弩將軍陳俊軍五社津,備滎陽以東。使吳漢率朱祐及廷尉岑彭、執金吾賈復、揚化將軍堅鐔等十一將軍圍朱鮪於洛陽。

八月壬子,祭社稷。癸丑,祠高祖、太宗、世宗於懷宮。進幸河陽。更始廩丘王田立降。

九月,赤眉入長安,更始奔高陵。辛未,詔曰:「更始破敗,棄城逃走,妻子裸袒,流冗道路。朕甚愍之。今封更始為淮陽王。吏人敢有賊害者,罪同大逆。」

甲申,以前高密令卓茂為太傅。

辛卯,朱鮪舉城降。

冬十月癸丑,車駕入洛陽,幸南宮卻非殿,遂定都焉。

遣岑彭擊荊州群賊。

十一月甲午,幸懷。

劉永自稱天子。

十二月丙戌,至自懷。

赤眉殺更始,而隗囂據隴右,盧芳起安定。破虜大將軍叔壽擊五校賊於曲梁,戰歿。

二年春正月甲子朔,日有食之。大司馬吳漢率九將軍擊檀鄉賊於鄴東,大破降之。庚辰,封功臣皆為列侯,大國四縣,餘各有差。下詔曰:「人情得足,苦於放縱,快須臾之欲,忘慎罰之義。惟諸將業遠功大,誠欲傳於無窮,宜如臨深淵,如履薄冰,戰戰慄慄,日慎一日。其顯效未詶,名籍未立者,大鴻臚趣上,朕將差而錄之。」博士丁恭議曰:「古帝王封諸侯不過百里,故利以建侯,取法於雷,強榦弱枝,所以為治也。今封諸侯四縣,不合法制。」帝曰:「古之亡國,皆以無道,未嘗聞功臣地多而滅亡者。」乃遣謁者即授印綬,策曰:「在上不驕,高而不危;制節謹度,滿而不溢。敬之戒之。傳爾子孫,長為漢藩。」

壬午,更始復漢將軍鄧曄、輔漢將軍于匡降,皆復爵位。

壬子,起高廟,建社稷於洛陽,立郊兆于城南,始正火德,色尚赤。

是月,赤眉焚西京宮室,發掘園陵,寇掠關中。大司徒鄧禹入長安,遣府掾奉十一帝神主,納於高廟。

真定王楊、臨邑侯讓謀反,遣前將軍耿純誅之。

二月己酉,幸修武。

大司空王梁免。壬子,以太中大夫宋弘為大司空。

遣驃騎大將軍景丹率征虜將軍祭遵等二將軍擊弘

農賊,破之,因遣祭遵圍蠻中賊張滿。

漁陽太守彭寵反,攻幽州牧朱浮於薊。

延岑自稱武安王於漢中。

辛卯,至自修武。

三月乙未,大赦天下,詔曰:「頃獄多冤人,用刑深刻,朕甚愍之。孔子云:『刑罰不中,則民無所措手足。』其與中二千石、諸大夫、博士、議郎議省刑法。」

遣執金吾賈復率二將軍擊更始郾王尹遵,破降之。

驍騎將軍劉植擊密賊,戰歿。

遣虎牙大將軍蓋延率四將軍伐劉永。夏四月,圍永於睢陽。更始將蘇茂殺淮陽太守潘蹇而附劉永。

甲午,封叔父良為廣陽王,兄子章為太原王,章弟興為魯王,舂陵侯嫡子祉為城陽王。

五月庚辰,封更始元氏王歙為泗水王,故真定王楊子得為真定王,周後姬常為周承休公。

癸未,詔曰:「民有嫁妻賣子欲歸父母者,恣聽之。敢拘執,論如律。」

六月戊戌,立貴人郭氏為皇后,子彊為皇太子,大赦天下。增郎、謁者、從官秩各一等。丙午,封宗子劉終為淄川王。

秋八月,帝自將征五校。丙辰,幸內黃,大破五校於羛陽,降之。

遣游擊將軍鄧隆救朱浮,與彭寵戰於潞,隆軍敗績。

蓋延拔睢陽,劉永奔譙。

破虜將軍鄧奉據淯陽反。

九月壬戌,至自內黃。

驃騎大將軍景丹薨。

延岑大破赤眉於杜陵。

關中饑,民相食。

冬十一月,以廷尉岑彭為征南大將軍,率八將軍討鄧奉於堵鄉。

銅馬、青犢、尤來餘賊共立孫登為天子於上郡。登將樂玄殺登,以其眾五萬餘人降。

遣偏將軍馮異代鄧禹伐赤眉。

使太中大夫伏隆持節安輯青徐二州,招張步降之。

十二月戊午,詔曰:「惟宗室列侯為王莽所廢,先靈無所依歸,朕甚愍之。其並復故國。若侯身已歿,屬所上其子孫見名尚書,封拜。」

是歲,蓋延等大破劉永於沛西。初,王莽末,天下旱蝗,黃金一斤易粟一斛;至是野穀旅生,麻缩尤盛,野蠶成繭,被於山阜,人收其利焉。

三年春正月甲子,以偏將軍馮異為征西大將軍,杜茂為驃騎大將軍,大司徒鄧禹及馮異與赤眉戰於回溪,禹、異敗績。

征虜將軍祭遵破蠻中,斬張滿。

辛巳,立皇考南頓君已上四廟。

壬午,大赦天下。

閏月乙巳,大司徒鄧禹免。

馮異與赤眉戰於崤底,大破之,餘眾南向宜陽,帝自將征之。己亥,幸宜陽。甲辰,親勒六軍,大陳戎馬,大司馬吳漢精卒當前,中軍次之,驍騎、武衛分陳左右。赤眉望見震怖,遣使乞降。丙午,赤眉君臣面縛,奉高皇帝璽綬,詔以屬城門校尉。戊申,至自宜陽,己酉,詔曰:「群盜縱橫,賊害元元,盆子竊尊號,亂惑天下。朕奮兵討擊,應時崩解,十餘萬眾束手降服,先帝璽綬歸之王府。斯皆祖宗之靈,士人之力,朕曷足以享斯哉!其擇吉日祠高廟,賜天下長子當為父後者爵,人一級。」

二月己未,祠高廟,受傳國璽。

劉永立董憲為海西王,張步為齊王。步殺光祿大夫伏隆而反。

幸懷。遣吳漢率二將軍擊青犢於軹西,大破降之。

三月壬寅,以大司徒司直伏湛為大司徒。

彭寵陷薊城,寵自立為燕王。

帝自將征鄧奉,幸堵陽。夏四月,大破鄧奉於小長安,斬之。

馮異與延岑戰於上林,破之。

吳漢率七將軍與劉永將蘇茂戰於廣樂,大破之。虎牙大將軍蓋延圍劉永於睢陽。

五月己酉,車駕還宮。

乙卯晦,日有食之。

六月壬戌,大赦天下。

耿弇與延岑戰於穰,大破之。

秋七月,征南大將軍岑彭率三將軍伐秦豐,戰於黎丘,大破之,獲其將蔡宏。

庚辰,詔曰:「吏不滿六百石,下至墨綬長、相,有罪先請。男子八十以上,十歲以下,及婦人從坐者,自非不道,詔所名捕,皆不得繫。當驗問者即就驗。女徒雇山歸家。」

蓋延拔睢陽,獲劉永,而蘇茂、周建立永子紆為梁王。

冬十月壬申,幸舂陵,祠園廟,因置酒舊宅,大會故人父老。十一月乙未,至自舂陵。

涿郡太守張豐反。

是歲,李憲自稱天子。西州大將軍隗囂奉奏。建義大將軍朱祐率祭遵與延岑戰於東陽,斬其將張成。

四年春正月甲申,大赦天下。

二月壬子,幸懷。壬申,至自懷。

遣右將軍鄧禹率二將軍與延岑戰於武當,破之。

夏四月丁巳,幸鄴。己巳,進幸臨平。

遣大司馬吳漢擊五校賊於箕山,大破之。

五月,進幸元氏。辛巳,進幸盧奴。

遣征虜將軍祭遵率四將軍討張豐於涿郡,斬豐。

六月辛亥,車駕還宮。

七月丁亥,幸譙。遣捕虜將軍馬武、偏將軍王霸圍劉紆於垂惠。

董憲將賁休以蘭陵城降,憲圍之。虎牙大將軍蓋延率平狄將軍龐萌救賁休,不克,蘭陵為憲所陷。

秋八月戊午,進幸壽春。

太中大夫徐惲擅殺臨淮太守劉度,惲坐誅。

遣揚武將軍馬成率三將軍伐李憲。九月,圍憲於舒。

冬十月甲寅,車駕還宮。

太傅卓茂薨。

十一月丙申,幸宛。遣建義大將軍朱祐率二將軍圍秦豐於黎丘。十二月丙寅,進幸黎丘。

是歲,征西大將軍馮異與公孫述將程焉戰於陳倉,破之。

五年春正月癸巳,車駕還宮。

二月丙午,大赦天下。

捕虜將軍馬武、偏將軍王霸拔垂惠。

乙丑,幸魏郡。

壬申,封殷後孔安為殷紹嘉公。

彭寵為其蒼頭所殺,漁陽平。

大司馬吳漢率建威大將軍耿弇擊富平、獲索賊於

平原,大破降之。復遣耿弇率二將軍討張步。

三月癸未,徙廣陽王良為趙王,始就國。

平狄將軍龐萌反,殺楚郡太守孫萌而東附董憲。

遣征南大將軍岑彭率二將軍伐田戎於津鄉,大破之。

夏四月,旱,蝗。

河西大將軍竇融始遣使貢獻。

五月丙子,詔曰:「久旱傷麥,秋種未下,朕甚憂之。將殘吏未勝,獄多冤結,元元愁恨,感動天氣乎?其令中都官、三輔、郡、國出繫囚,罪非犯殊死一切勿案,見徒免為庶人。務進柔良,退貪酷,各正厥事焉。」

六月,建義大將軍朱祐拔黎丘,獲秦豐;而龐萌、蘇茂圍桃城。帝時幸蒙,因自將征之。先理兵任城,乃進救桃城,大破萌等。

秋七月丁丑,幸沛,祠高原廟。詔修復西京園陵。進幸湖陵,征董憲。又幸蕃,遂攻董憲於昌慮,大破之。

八月己酉,進幸郯,留吳漢攻劉紆、董憲等,車駕轉徇彭城、下邳。吳漢拔郯,獲劉紆,漢進圍董憲、龐萌於朐。

冬十月,還,幸魯,使大司空祠孔子。

耿弇等與張步戰於臨淄,大破之。帝幸臨淄,進幸劇。張步斬蘇茂以降,齊地平。

初起太學。車駕還宮,幸太學,賜博士弟子各有差。

十一月壬寅,大司徒伏湛免,尚書令侯霸為大司徒。

十二月,盧芳自稱天子於九原。

西州大將軍隗囂子恂入侍。

交阯牧鄧讓率七郡太守遣使奉貢。

詔復濟陽二年傜役。

是歲,野穀漸少,田畝益廣焉。


\end{pinyinscope}