\article{崔駰列傳}

\begin{pinyinscope}
崔駰字亭伯,涿郡安平人也。高祖父朝,昭帝時為幽州從事,諫刺史無與燕剌王通。及剌王敗,擢為侍御史。生子舒,歷四郡太守,所在有能名。

舒小子篆,王莽時為郡文學,以明經徵詣公車。太保甄豐舉為步兵校尉,篆辭曰:「吾聞伐國不問仁人,戰陳不訪儒士。此舉奚為至哉?」遂投劾歸。

莽嫌諸不附己者,多以法中傷之。時篆兄發以佞巧幸於莽,位至大司空。母師氏能通經學、百家之言,莽寵以殊禮,賜號義成夫人,金印紫綬,文軒丹轂,顯於新世。

後以篆為建新大尹,篆不得已,乃歎曰:「吾生無妄之世,值澆、羿之君,上有老母,下有兄弟,安得獨潔己而危所生哉?」乃遂單車到官,稱疾不視事,三年不行縣。門下掾倪敞諫,篆乃強起班春。所至之縣,獄犴填滿。篆垂涕曰:「嗟乎!刑罰不中,乃陷人於阱。此皆何罪,而至于是!」遂平理,所出二千餘人。掾吏叩頭諫曰:「朝廷初政,州牧峻刻。宥過申枉,誠仁者之心;然獨為君子,將有悔乎!」篆曰:「邾文公不以一人易其身,君子謂之知命。如殺一大尹贖二千人,蓋所願也。」遂稱疾去。

建武初,朝廷多薦言之者,幽州刺史又舉篆賢良。篆自以宗門受莽偽寵,慚愧漢朝,遂辭歸不仕。客居滎陽,閉門潛思,著周易林六十四篇,用決吉凶,多所占驗。臨終作賦以自悼,名曰慰志。其辭曰:

嘉昔人之遘辰兮,美伊、傅之詳時。應規矩之淑質兮,過班、倕而裁之。協準矱之貞度兮,同斷金之玄策。何天衢於盛世兮,超千載而垂績。豈脩德之極致兮,將天祚之攸適?

愍余生之不造兮,丁漢氏之中微。氛霓鬱以橫厲兮,羲和忽以潛暉。六柄制于家門兮,王綱漼以陵遲。黎、共奮以跋扈兮,羿、浞狂以恣睢。睹嫚臧而乘釁兮,竊神器之萬機。思輔弼以媮存兮,亦號咷以詶咨。嗟三事之我負兮,乃迫余以天威。豈無熊僚之微介兮?悼我生之殲夷。庶明哲之末風兮,懼大雅之所譏。遂翕翼以委命兮,受符守乎艮維。恨遭閉而不隱兮,違石門之高蹤。揚蛾眉於復關兮,犯孔戒之冶容。懿氓蚩之悟悔兮,慕白駒之所從。乃稱疾而屢復兮,歷三祀而見許。悠輕舉以遠遁兮,託峻峗以幽處。竫潛思於至賾兮,騁六經之奧府。皇再命而紹卹兮,乃云眷乎建武。運欃槍以電埽兮,清六合之土宇。聖德滂以橫被兮,黎庶愷以鼓舞。闢四門以博延兮,彼幽牧之我舉。分畫定而計決兮,豈云賁乎鄙耇,遂懸車以縶馬兮,絕時俗之進取。歎暮春之成服兮,闔衡門以埽軌。聊優游以永日兮,守性命以盡齒。貴啟體之歸全兮,庶不忝乎先子。

篆生毅,以疾隱身不仕。

毅生駰,年十三能通詩、易、春秋,博學有偉才,盡通古今訓詁百家之言,善屬文。少游太學,與班固、傅毅同時齊名。常以典籍為業,未遑仕進之事。時人或譏其太玄靜,將以後名失實。駰擬楊雄解嘲,作達旨以荅焉。其辭曰:

或說己曰:「易稱『備物致用』,『可觀而有所合』,故能扶陽以出,順陰而入。春發其華,秋收其實,有始有極,爰登其質。今子韞櫝六經,服膺道術,歷世而游,高談有日,俯鉤深於重淵,仰探遠乎九乾,窮至賾於幽微,測潛隱之無源。然下不步卿相之廷,上不登王公之門,進不黨以讚己,退不黷於庸人。獨師友道德,合符曩真,抱景特立,與士不群。蓋高樹靡陰,獨木不林,隨時之宜,道貴從凡。于時太上運天德以君世,憲王僚而布官;臨雍泮以恢儒,疏軒冕以崇賢;率惇德以厲忠孝,揚茂化以砥仁義;選利器於良材,求鏌铮於明智。不以此時攀台階,闚紫闥,據高軒,望朱闕,夫欲千里而咫尺未發,蒙竊惑焉。故英人乘斯時也,猶逸禽之赴深林,虻蚋之趣大沛。胡為嘿嘿而久沈滯也?」

荅曰:「有是言乎?子苟欲勉我以世路,不知其跌而失吾之度也。古者陰陽始分,天地初制,皇綱云緒,帝紀乃設,傳序歷數,三代興滅。昔大庭尚矣,赫胥罔識。淳祓散離,人物錯乖。高辛攸降,厥趣各違。道無常稽,與時張弛。失仁為非,得義為是。君子通變,各審所履。故士或掩目而淵潛,或盥耳而山棲;或草耕而僅飽,或木茹而長飢;或重聘而不來,或屢黜而不去;或冒詬以干進,或望色而斯舉;或以役夫發夢於王公,或以漁父見兆於元龜。若夫紛驾塞路,凶虐播流,人有昏墊之厄,主有疇咨之憂,條垂藟蔓,上下相求。於是乎賢人授手,援世之災,跋涉赴俗,急斯時也。昔堯含慼而皋陶謨,高祖歎而子房慮;禍不散而曹、絳奮,結不解而陳平權。及其策合道從,克亂弭衝,乃將鏤玄珪,冊顯功,銘昆吾之冶,勒景、襄之鍾。與其有事,則褰裳濡足,冠挂不顧。人溺不拯,則非仁也。當其無事,則躐纓整襟,規矩其步。德讓不修,則非忠也。是以險則救俗,平則守禮,舉以公心,不私其體。

「今聖上之育斯人也,樸以皇質,雕以唐文。六合怡怡,比屋為仁。壹天下之眾異,齊品類之萬殊。參差同量,坏冶一陶。群生得理,庶績其凝。家家有以樂和,人人有以自優。威械臧而俎豆布,六典陳而九刑厝。濟茲兆庶,出於平易之路。雖有力牧之略,尚父之厲,伊、皋不論,奚事范、蔡?夫廣廈成而茂木暢,遠求存而良馬縶,陰事終而水宿臧,場功畢而大火入。方斯之際,處士山積,學者川流,衣裳被宇,冠蓋雲浮。譬猶衡陽之林,岱陰之麓,伐尋抱不為之稀,蓺拱把不為之數。悠悠罔極,亦各有得。彼採其華,我收其實。舍之則臧,己所學也。故進動以道,則不辭執珪而秉柱國;復靜以理,則甘糟糠而安藜藿。

「夫君子非不欲仕也。恥夸毗以求舉;非不欲室也,惡登牆而摟處。叫呼衒鬻,縣旌自表,非隨和之寶也。暴智燿世,因以干祿,非仲尼之道也。游不倫黨,苟以徇己,汗血競時,利合而友。子笑我之沈滯,吾亦病子饩饩而不已也。先人有則而我弗虧,行有枉徑而我弗隨。臧否在予,唯世所議。固將因天質之自然,誦上哲之高訓;詠太平之清風,行天下之至順。懼吾躬之穢德,勤百畝之不耘。縶余馬以安行,俟性命之所存。昔孔子起威於夾谷,晏嬰發勇於崔杼;曹劌舉節於柯盟,卞嚴克捷於彊禦;范蠡錯埶於會稽,五員樹功於柏舉;魯連辯言以退燕,包胥單辭而存楚;唐且華顛以悟秦,甘羅童牙而報趙;原衰見廉於壺飧,宣孟收德於束脯;吳札結信於丘木,展季效貞於門女;顏回明仁於度轂,程嬰顯義於趙武。僕誠不能編德於數者,竊慕古人之所序。」

元和中,肅宗始修古禮,巡狩方岳。駰上四巡頌以稱漢德,辭甚典美,文多故不載。帝雅好文章,自見駰頌後,帝嗟歎之,謂侍中竇憲曰:「卿寧知崔駰乎?」對曰:「班固數為臣說之,然未見也。」帝曰:「公愛班固而忽崔駰,此葉公之好龍也。試請見之。」駰由此候憲。憲屣履迎門,笑謂駰曰:「亭伯,吾受詔交公,公何得薄哉?」遂揖入為上客。居無幾何,帝幸憲第,時駰適在憲所,帝聞而欲召見之。憲諫,以為不宜與白衣會。帝悟曰:「吾能令駰朝夕在傍,何必於此!」適欲官之,會帝崩。

竇太后臨朝,憲以重戚出內詔命。駰獻書誡之曰:

駰聞交淺而言深者,愚也;在賤而望貴者,惑也;未信而納忠者,謗也。三者皆所不宜,而或蹈之者,思效其區區,憤盈而不能已也。竊見足下體淳淑之姿,躬高明之量,意美志厲,有上賢之風。駰幸得充下館,序後陳,是以竭其拳拳,敢進一言。

傳曰:「生而富者驕,生而貴者傲。」生富貴而能不驕傲者,未之有也。今寵祿初隆,百僚觀行,當堯舜之盛世,處光華之顯時,豈可不庶幾夙夜,以永眾譽,弘申伯之美,致周邵之事乎?語曰:「不患無位,患所以立。」昔馮野王以外戚居位,稱為賢臣,近陰衛尉克己復禮,終受多福。郯氏之宗,非不尊也;陽侯之族,非不盛也。重侯累將,建天樞,執斗柄。其所以獲譏於時,垂愆於後者,何也?蓋在滿而不挹,位有餘而仁不足也。漢興以後,迄于哀、平,外家二十,保族全身,四人而已。書曰:「鑒于有殷。」可不慎哉!

竇氏之興,肇自孝文。二君以淳淑守道,成名先日;安豐以佐命著德,顯自中興。內以忠誠自固,外以法度自守,卒享祚國,垂祉於今。夫謙德之光,周易所美;滿溢之位,道家所戒。故君子福大而愈懼,爵隆而益恭。遠察近覽,俯仰有則,銘諸几杖,刻諸盤杅。矜矜業業,無殆無荒。如此,則百福是荷,慶流無窮矣。

及憲為車騎將軍,辟駰為掾。憲府貴重,掾屬三十人,皆故刺史、二千石,唯駰以處士年少,擢在其閒。憲擅權驕恣,駰數諫之。及出擊匈奴,道路愈多不法,駰為主簿,前後奏記數十,指切長短。憲不能容,稍疏之,因察駰高第,出為長岑長。駰自以遠去,不得意,遂不之官而歸。永元四年,卒于家。所著詩、賦、銘、頌、書、記、表、七依、婚禮結言、達旨、酒警合二十一篇。中子瑗。

瑗字子玉,早孤,銳志好學,盡能傳其父業。年十八,至京師,從侍中賈逵質正大義,逵善待之,瑗因留游學,遂明天官、歷數、京房易傳、六日七分。諸儒宗之。與扶風馬融、南陽張衡特相友好。初,瑗兄章為州人所殺,瑗手刃報仇,因亡命。會赦,歸家。家貧,兄弟同居數十年,鄉邑化之。

年四十餘,始為郡吏。以事繫東郡發干獄。獄掾善為禮,瑗閒考訊時,輒問以禮說。其專心好學,雖顛沛必於是。後事釋歸家,為度遼將軍鄧遵所辟。居無何,遵被誅,瑗免歸。

後復辟車騎將軍閻顯府。時閻太后稱制,顯入參政事。先是安帝廢太子為濟陰王,而以北鄉侯為嗣。瑗以侯立不以正,知顯將敗,欲說令廢立,而顯日沈醉,不能得見。乃謂長史陳禪曰:「中常侍江京、陳達等,得以嬖寵惑蠱先帝,遂使廢黜正統,扶立疏孽。少帝即位,發病廟中,周勃之徵,於斯復見。今欲與長史君共求見,說將軍白太后,收京等,廢少帝,引立濟陰王,必上當天心,下合人望。伊、霍之功,不下席而立,則將軍兄弟傳祚於無窮。若拒違天意,久曠神器,則將以無罪并辜元惡。此所謂禍福之會,分功之時。」禪猶豫未敢從。會北鄉侯薨,孫程立濟陰王,是為順帝。閻顯兄弟悉伏誅,瑗坐被斥。門生蘇祇具知瑗謀,欲上書言狀,瑗聞而遽止之。時陳禪為司隸校尉,召瑗謂曰:「第聽祇上書,禪請為之證。」瑗曰:「此譬猶兒妾屏語耳,願使君勿復出口。」遂辭歸,不復應州郡命。

久之,大將軍梁商初開莫府,復首辟瑗。自以再為貴戚吏,不遇被斥,遂以疾固辭。歲中舉茂才,遷汲令。在事數言便宜,為人開稻田數百頃。視事七年,百姓歌之。

漢安初,大司農胡廣、少府竇章共薦瑗宿德大儒,從政有跡,不宜久在下位,由此遷濟北相。時李固為太山太守,美瑗文雅,奉書禮致殷勤。歲餘,光祿大夫杜喬為八使,徇行郡國,以臧罪奏瑗,徵詣廷尉。瑗上書自訟,得理出。會病卒,年六十六。臨終,顧命子寔曰:「夫人稟天地之氣以生,及其終也,歸精於天,還骨於地。何地不可臧形骸,勿歸鄉里。其賵贈之物,羊豕之奠,一不得受。」寔奉遺令,遂留葬洛陽。

瑗高於文辭,尤善為書、記、箴、銘,所著賦、碑、銘、箴、頌、七蘇、南陽文學官志、歎辭、移社文、悔祈、草書埶、七言,凡五十七篇。其南陽文學官志稱於後世,諸能為文者皆自以弗及。瑗愛士,好賓客,盛脩肴膳,單極滋味,不問餘產。居常蔬食菜羹而已。家無擔石儲,當世清之。

寔字子真,一名台,字元始。少沈靜,好典籍。父卒,隱居墓側。服竟,三公並辟,皆不就。

桓帝初,詔公卿郡國舉至孝獨行之士。寔以郡舉,徵詣公車,病不對策,除為郎。明於政體,吏才有餘,論當世便事數十條,名曰政論。指切時要,言辯而确,當世稱之。仲長統曰:「凡為人主,宜寫一通,置之坐側。」其辭曰:

自堯舜之帝,湯武之王,皆賴明哲之佐,博物之臣。故皋陶陳謨而唐虞以興,伊、箕作訓而殷周用隆。及繼體之君,欲立中興之功者,曷嘗不賴賢哲之謀乎!凡天下所以不理者,常由人主承平日久,俗漸敝而不悟,政寖衰而不改,習亂安危,怢不自睹。或荒耽嗜欲,不恤萬機;或耳蔽箴誨,厭偽忽真;或猶豫歧路,莫適所從;或見信之佐,括囊守祿;或疏遠之臣,言以賤廢。是以王綱縱弛於上,智士鬱伊於下。悲夫!

自漢興以來,三百五十餘歲矣。政令垢翫,上下怠懈,風俗彫敝,人庶巧偽,百姓囂然,咸復思中興之救矣。且濟時拯世之術,豈必體堯蹈舜然後乃理哉?期於補驽決壞,枝柱邪傾,隨形裁割,要措斯世於安寧之域而已。故聖人執權,遭時定制,步驟之差,各有云設。不彊人以不能,背急切而慕所聞也。蓋孔子對葉公以來遠,哀公以臨人,景公以節禮,非其不同,所急異務也。是以受命之君,每輒創制;中興之主,亦匡時失。昔盤庚愍殷,遷都易民;周穆有闕,甫侯正刑。俗人拘文牽古,不達權制,奇偉所聞,簡忽所見,烏可與論國家之大事哉!故言事者,雖合聖德,輒見掎奪。何者?其頑士闇於時權,安習所見,不知樂成,況可慮始,苟云率由舊章而已。其達者或矜名妒能,恥策非己,舞筆奮辭,以破其義,寡不勝眾,遂見擯棄。雖稷、契復存,猶將困焉。斯賈生之所以排於絳、灌,屈子之所以攄其幽憤者也。夫以文帝之明,賈生之賢,絳、灌之忠,而有此患,況其餘哉!

故宜量力度德,春秋之義。今既不能純法八世,故宜參以霸政,則宜重賞深罰以御之,明著法術以檢之。自非上德,嚴之則理,寬之則亂。何以明其然也?近孝宣皇帝明於君人之道,審於為政之理,故嚴刑峻法,破姦軌之膽,海內清肅,天下密如。薦勳祖廟,享號中宗。筭計見效,優於孝文。及元帝即位,多行寬政,卒以墮損,威權始奪,遂為漢室基禍之主。政道得失,於斯可監。昔孔子作春秋,褒齊桓,懿晉文,歎管仲之功。夫豈不美文、武之道哉?誠達權救敝之理也。故聖人能與世推移,而俗士苦不知變,以為結繩之約,可復理亂秦之緒,干戚之舞,足以解平城之圍。

夫熊經鳥伸,雖延歷之術,非傷寒之理;呼吸吐納,雖度紀之道,非續骨之膏。蓋為國之法,有似理身,平則致養,疾則攻焉。夫刑罰者,治亂之藥石也;德教者,興平之梁肉也。夫以德教除殘,是以梁肉理疾也;以刑罰理平,是以藥石供養也。方今承百王之敝,值厄運之會。自數世以來,政多恩貸,馭委其轡,馬駘其銜,四牡橫奔,皇路險傾。方將柑勒鞬輈以救之,豈暇鳴和鑾,清節奏哉?昔高祖令蕭何作九章之律,有夷三族之令,黥、劓、斬趾、斷舌、梟首,故謂之具五刑。文帝雖除肉刑,當劓者笞三百,當斬左趾者笞五百,當斬右趾者棄巿。右趾者既殞其命,笞撻者往往至死,雖有輕刑之名,其實殺也。當此之時,民皆思復肉刑。至景帝元年,乃下詔曰:「笞與重罪無異,幸而不死,不可為民。」乃定律,減笞輕捶。自是之後,笞者得全。以此言之,文帝乃重刑,非輕之也;以嚴致平,非以寬致平也。必欲行若言,當大定其本,使人主師五帝而式三王。盪亡秦之俗,遵先聖之風,棄苟全之政,蹈稽古之蹤,復五等之爵,立井田之制。然後選稷契為佐,伊呂為輔,樂作而鳳皇儀,擊石而百獸舞。若不然,則多為累而已。

其後辟太尉袁湯、大將軍梁冀府,並不應。大司農羊傅、少府何豹上書薦寔才美能高,宜在朝廷。召拜議郎,遷大將軍冀司馬,與邊韶、延篤等著作東觀。

出為五原太守。五原土宜麻枲,而俗不知織績,民冬月無衣,積細草而臥其中,見吏則衣草而出。寔至官,斥賣儲峙,為作紡績、織紝、綀縕之具以教之,民得以免寒苦。是時胡虜連入雲中、朔方,殺略吏民,一歲至九奔命。寔整厲士馬,嚴烽候,虜不敢犯,常為邊最。

以病徵,拜議郎,復與諸儒博士共雜定五經。會梁冀誅,寔以故吏免官,禁錮數年。

時鮮卑數犯邊,詔三公舉威武謀略之士,司空黃瓊薦寔,拜遼東太守。行道,母劉氏病卒,上疏求歸葬行喪。母有母儀淑德,博覽書傳。初,寔在五原,常訓以臨民之政,寔之善績,母有其助焉。服竟,召拜尚書。寔以世方阻亂,稱疾不視事,數月免歸。

初,寔父卒,剽賣田宅,起冢塋,立碑頌。葬訖,資產竭盡,因窮困,以酤釀販鬻為業。時人多以譏之,寔終不改。亦取足而已,不致盈餘。及仕官,歷位邊郡,而愈貧薄。建寧中病卒。家徒四壁立,無以殯斂,光祿勳楊賜、太僕袁逢、少府段熲為備棺槨葬具,大鴻臚袁隗樹碑頌德。所著碑、論、箴、銘、荅、七言、祠、文、表、記、書凡十五篇。

寔從兄烈,有重名於北州,歷位郡守、九卿。靈帝時,開鴻都門榜賣官爵,公卿州郡下至黃綬各有差。其富者則先入錢,貧者到官而後倍輸,或因常侍、阿保別自通達。是時段熲、樊陵、張溫等雖有功勤名譽,然皆先輸貨財而後登公位。烈時因傅母入錢五百萬,得為司徒。及拜日,天子臨軒,百僚畢會。帝顧謂親倖者曰:「悔不小靳,可至千萬。」程夫人於傍應曰:「崔公冀州名士,豈肯買官?賴我得是,反不知姝邪!」烈於是聲譽衰減。久之不自安,從容問其子鈞曰:「吾居三公,於議者何如?」鈞曰:「大人少有英稱,歷位卿守,論者不謂不當為三公;而今登其位,天下失望。」烈曰:「何為然也?」鈞曰:「論者嫌其銅臭。」烈怒,舉杖擊之。鈞時為虎賁中郎將,服武弁,戴鶡尾,狼狽而走。烈罵曰:「死卒,父檛而走,孝乎?」鈞曰:「舜之事父,小杖則受,大杖則走,非不孝也。」烈慚而止。烈後拜太尉。

鈞少交結英豪,有名稱,為西河太守。獻帝初,鈞與袁紹俱起兵山東,董卓以是收烈付郿獄,錮之,鋃鐺鐵鎖。卓既誅,拜烈城門校尉。及李傕入長安,為亂兵所殺。

烈有文才,所著詩、書、教、頌等凡四篇。

論曰:崔氏世有美才,兼以沈淪典籍,遂為儒家文林。駰、瑗雖先盡心於貴戚,而能終之以居正,則其歸旨異夫進趣者乎!李固,高絜之士也,與瑗鄰郡,奉贄以結好。由此知杜喬之劾,殆其過矣。寔之政論,言當世理亂,雖晁錯之徒不能過也。

贊曰:崔為文宗,世禪雕龍。建新恥潔,摧志求容。永矣長岑,于遼之陰。不有直道,曷取泥沈。瑗不言祿,亦離冤辱。子真持論,感起昏俗。


\end{pinyinscope}