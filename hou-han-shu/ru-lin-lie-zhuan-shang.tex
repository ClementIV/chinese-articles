\article{儒林列傳上}

\begin{pinyinscope}
昔王莽、更始之際,天下散亂,禮樂分崩,典文殘落。及光武中興,愛好經術,未及下車,而先訪儒雅,採求闕文,補綴漏逸。先是四方學士多懷協圖書,遁逃林藪。自是莫不抱負墳策,雲會京師,范升、陳元、鄭興、杜林、衛宏、劉昆、桓榮之徒,繼踵而集。於是立五經博士,各以家法教授,易有施、孟、梁丘、京氏,尚書歐陽、大小夏侯,詩齊、魯、韓,禮大小戴,春秋嚴、顏,凡十四博士,太常差次總領焉。

建武五年,乃修起太學,稽式古典,籩豆干戚之容,備之於列,服方領習矩步者,委它乎其中。中元元年,初建三雍。明帝即位,親行其禮。天子始冠通天,衣日月,備法物之駕,盛清道之儀,坐明堂而朝群后,登靈臺以望雲物,袒割辟雍之上,尊養三老五更。饗射禮畢,帝正坐自講,諸儒執經問難於前,冠帶縉紳之人,圜橋門而觀聽者蓋億萬計。其後復為功臣子孫、四姓末屬別立校舍,搜選高能以受其業,自期門羽林之士,悉令通孝經章句,匈奴亦遣子入學。濟濟乎,洋洋乎,盛於永平矣!

建初中,大會諸儒於白虎觀,考詳同異,連月乃罷。肅宗親臨稱制,如石渠故事,顧命史臣,著為通義。又詔高才生受古文尚書、毛詩、穀梁、左氏春秋,雖不立學官,然皆擢高第為講郎,給事近署,所以網羅遺逸,博存眾家。孝和亦數幸東觀,覽閱書林。及鄧后稱制,學者頗懈。時樊準、徐防並陳敦學之宜,又言儒職多非其人,於是制詔公卿妙簡其選,三署郎能通經術者,皆得察舉。自安帝覽政,薄於蓺文,博士倚席不講,朋徒相視怠散,學舍穨敝,鞠為園蔬,牧兒蕘豎,至於薪刈其下。順帝感翟酺之言,乃更脩黌宇,凡所造構二百四十房,千八百五十室。試明經下第補弟子,增甲乙之科員各十人,除郡國耆儒皆補郎、舍人。本初元年,梁太后詔曰:「大將軍下至六百石,悉遣子就學,每歲輒於鄉射月一饗會之,以此為常。」自是遊學增盛,至三萬餘生。然章句漸疏,而多以浮華相尚,儒者之風蓋衰矣。黨人既誅,其高名善士多坐流廢,後遂至忿爭,更相言告,亦有私行金貨,定蘭臺桼書經字,以合其私文。熹平四年,靈帝乃詔諸儒正定五經,刊於石碑,為古文、篆、隸三體書法以相參檢,樹之學門,使天下咸取則焉。

初,光武遷還洛陽,其經牒祕書載之二千餘兩,自此以後,參倍於前。及董卓移都之際,吏民擾亂,自辟雍、東觀、蘭臺、石室、宣明、鴻都諸藏典策文章,競共剖散,其縑帛圖書,大則連為帷蓋,小乃制為縢囊。及王允所收而西者,裁七十餘乘,道路艱遠,復棄其半矣。後長安之亂,一時焚蕩,莫不泯盡焉。

東京學者猥眾,難以詳載,今但錄其能通經名家者,以為儒林篇。其自有列傳者,則不兼書。若師資所承,宜標名為證者,乃著之云。

前書云:田何傳易授丁寬,丁寬授田王孫,王孫授沛人施讎、東海孟喜、琅邪梁丘賀,由是易有施、孟、梁丘之學。又東郡京房受易於梁國焦延壽,別為京氏學。又有東萊費直,傳易,授琅邪王橫,為費氏學。本以古字,號古文易。又沛人高相傳易,授子康及蘭陵毋將永,為高氏學。施、孟、梁丘、京氏四家皆立博士,費、高二家未得立。

劉昆字桓公,陳留東昏人,梁孝王之胤也。少習容禮。平帝時,受施氏易於沛人戴賓。能彈雅琴,知清角之操。

王莽世,教授弟子恆五百餘人。每春秋饗射,常備列典儀,以素木瓠葉為俎豆,桑弧蒿矢,以射「菟首」。每有行禮,縣宰輒率吏屬而觀之。王莽以昆多聚徒眾,私行大禮,有僭上心,乃繫昆及家屬於外黃獄。尋莽敗得免。既而天下大亂,昆避難河南負犢山中。

建武五年,舉孝廉,不行,遂逃,教授於江陵。光武聞之,即除為江陵令。時縣連年火災,昆輒向火叩頭,多能降雨止風。徵拜議郎,稍遷侍中、弘農太守。

先是崤、黽驛道多虎災,行旅不通。昆為政三年,仁化大行,虎皆負子度河。帝聞而異之。二十二年,徵代杜林為光祿勳。詔問昆曰:「前在江陵,反風滅火,後守弘農,虎北度河,行何德政而致是事?」昆對曰:「偶然耳。」左右皆笑其質訥。帝歎曰:「此乃長者之言也。」顧命書諸策。乃令入授皇太子及諸王小侯五十餘人。二十七年,拜騎都尉。三十年,以老乞骸骨,詔賜洛陽第舍,以千石祿終其身。中元二年卒。

子軼,字君文,傳昆業,門徒亦盛。永平中,為太子中庶子。建初中,稍遷宗正,卒官,遂世掌宗正焉。

洼丹字子玉,南陽育陽人也。世傳孟氏易。王莽時,常避世教授,專志不仕,徒眾數百人。建武初,為博士,稍遷,十一年,為大鴻臚。作易通論七篇,世號洼君通。丹學義研深,易家宗之,稱為大儒。十七年,卒於官,年七十。

時中山觟陽鴻,字孟孫,亦以孟氏易教授,有名稱,永平中為少府。

任安字定祖,廣漢綿竹人也。少遊太學,受孟氏易,兼通數經。又從同郡楊厚學圖讖,究極其術。時人稱曰:「欲知仲桓問任安。」又曰:「居今行古任定祖。」學終,還家教授,諸生自遠而至。初仕州郡。後太尉再辟,除博士,公車徵,皆稱疾不就。州牧劉焉表薦之,時王塗隔塞,詔命竟不至。年七十九,建安七年,卒于家。

楊政字子行,京兆人也。少好學,從代郡范升受梁丘易,善說經書。京師為之語曰:「說經鏗鏗楊子行。」教授數百人。

范升嘗為出婦所告,坐繫獄,政乃肉袒,以箭貫耳,抱升子潛伏道傍,候車駕,而持章叩頭大言曰:「范升三娶,唯有一子,今適三歲,孤之可哀。」武騎虎賁懼驚乘輿,舉弓射之,猶不肯去;旄頭又以戟叉政,傷胸,政猶不退。哀泣辭請,有感帝心,詔曰:「乞楊生師。」即尺一出升。政由是顯名。

為人嗜酒,不拘小節,果敢自矜,然篤於義。時帝婿梁松,皇后弟陰就,皆慕其聲名,而請與交友。政每共言論,常切磋懇至,不為屈撓。嘗詣楊虛侯馬武,武難見政,稱疾不為起。政入戶,徑升床排武,把臂責之曰:「卿蒙國恩,備位藩輔,不思求賢以報殊寵,而驕天下英俊,此非養身之道也。今日動者刀入脅。」武諸子及左右皆大驚,以為見劫,操兵滿側,政顏色自若。會陰就至,責數武,令為交友。其剛果任情,皆如此也。建初中,官至左中郎將。

張興字君上,潁川鄢陵人也。習梁丘易以教授。建武中,舉孝廉為郎,謝病去,復歸聚徒。後辟司徒馮勤府,勤舉為孝廉,稍遷博士。永平初,遷侍中祭酒。十年,拜太子少傅。顯宗數訪問經術。既而聲稱著聞,弟子自遠至者,著錄且萬人,為梁丘家宗。十四年,卒於官。

子魴,傳興業,位至張掖屬國都尉。

戴憑字次仲,汝南平輿人也。習京氏易。年十六,郡舉明經,徵試博士,拜郎中。

時詔公卿大會,群臣皆就席,憑獨立。光武問其意。憑對曰:「博士說經皆不如臣,而坐居臣上,是以不得就席。」帝即召上殿,令與諸儒難說,憑多所解釋。帝善之,拜為侍中,數進見問得失。帝謂憑曰:「侍中當匡補國政,勿有隱情。」憑對曰:「陛下嚴。」帝曰:「朕何用嚴?」憑曰:「伏見前太尉西曹掾蔣遵,清亮忠孝,學通古今,陛下納膚受之訴,遂致禁錮,世以是為嚴。」帝怒曰:「汝南子欲復黨乎?」憑出,自繫廷尉,有詔敕出。後復引見,憑謝曰:「臣無謇諤之節,而有狂瞽之言,不能以尸伏諫,偷生苟活,誠慚聖朝。」帝即敕尚書解遵禁錮,拜憑虎賁中郎將,以侍中兼領之。

正旦朝賀,百僚畢會,帝令群臣能說經者更相難詰,義有不通,輒奪其席以益通者,憑遂重坐五十餘席。故京師為之語曰:「解經不窮戴侍中。」在職十八年,卒於官,詔賜東園梓器,錢二十萬。

時南陽魏滿字叔牙,亦習京氏易,教授。永平中,至弘農太守。

孫期字仲彧,濟陰成武人也。少為諸生,習京氏易、古文尚書。家貧,事母至孝,牧豕於大澤中,以奉養焉。遠人從其學者,皆執經壟畔以追之,里落化其仁讓。黃巾賊起,過期里陌,相約不犯孫先生舍。郡舉方正,遣吏齎羊酒請期,期驅豕入草不顧。司徒黃琬特辟,不行,終於家。

建武中,范升傳孟氏易,以授楊政,而陳元、鄭眾皆傳費氏易,其後馬融亦為其傳。融授鄭玄,玄作易注,荀爽又作易傳,自是費氏興,而京氏遂衰。

前書云:濟南伏生傳尚書,授濟南張生及千乘歐陽生,歐陽生授同郡兒寬,寬授歐陽生之子,世世相傳,至曾孫歐陽高,為尚書歐陽氏學;張生授夏侯都尉,都尉授族子始昌,始昌傳族子勝,為大夏侯氏學;勝傳從兄子建,建別為小夏侯氏學:三家皆立博士。又魯人孔安國傳古文尚書授都尉朝,朝授膠東庸譚,為尚書古文學,未得立。

歐陽歙字正思,樂安千乘人也。自歐陽生傳伏生尚書,至歙八世,皆為博士。

歙既傳業,而恭謙好禮讓。王莽時,為長社宰。更始立,為原武令。世祖平河北,到原武,見歙在縣脩政,遷河南都尉,後行太守事。世祖即位,始為河南尹,封被陽侯。建武五年,坐事免官。明年,拜楊州牧,遷汝南太守。推用賢俊,政稱異跡。九年,更封夜侯。

歙在郡,教授數百人,視事九歲,徵為大司徒。坐在汝南臧罪千餘萬發覺下獄。諸生守闕為歙求哀者千餘人,至有自髡剔者。平原禮震,年十七,聞獄當斷,馳之京師,行到河內獲嘉縣,自繫,上書求代歙死。曰:「伏見臣師大司徒歐陽歙,學為儒宗,八世博士,而以臧咎當伏重辜。歙門單子幼,未能傳學,身死之後,永為廢絕,上令陛下獲殺賢之譏,下使學者喪師資之益。乞殺臣身以代歙命。」書奏,而歙已死獄中。歙掾陳元上書追訟之,言甚切至,帝乃賜棺木,贈印綬,賻縑三千匹。

子復嗣。復卒,無子,國除。

濟陰曹曾字伯山,從歙受尚書,門徒三千人,位至諫議大夫。子祉,河南尹,傳父業教授。

又陳留陳弇,字叔明,亦受歐陽尚書於司徒丁鴻,仕為蘄長。

牟長字君高,樂安臨濟人也。其先封牟,春秋之末,國滅,因氏焉。

長少習歐陽尚書,不仕王莽世。建武二年,大司空弘特辟,拜博士,稍遷河內太守,坐墾田不實免。

長自為博士及在河內,諸生講學者常有千餘人,著錄前後萬人。著尚書章句,皆本之歐陽氏,俗號為牟氏章句。復徵為中散大夫,賜告一歲,卒於家。

子紆,又以隱居教授,門生千人。肅宗聞而徵之,欲以為博士,道物故。

宋登字叔陽,京兆長安人也。父由,為太尉。

登少傳歐陽尚書,教授數千人。為汝陰令,政為明能,號稱「神父」。遷趙相,入為尚書僕射。順帝以登明識禮樂,使持節臨太學,奏定典律,轉拜侍中。數上封事,抑退權臣,由是出為潁川太守。市無二價,道不拾遺。病免,卒于家,汝陰人配社祠之。

張馴字子雋,濟陰定陶人也。少遊太學,能誦春秋左氏傳。以大夏侯尚書教授。辟公府,舉高第,拜議郎。與蔡邕共奏定六經文字。擢拜侍中,典領祕書近署,甚見納異。因便宜陳政得失,朝廷嘉之。遷丹陽太守,化有惠政。光和七年,徵拜尚書,遷大司農。初平中,卒於官。

尹敏字幼季,南陽堵陽人也。少為諸生。初習歐陽尚書,後受古文,兼善毛詩、穀梁、左氏春秋。

建武二年,上疏陳洪範消災之術。時世祖方草創天下,未遑其事,命敏待詔公車,拜郎中,辟大司空府。

帝以敏博通經記,令校圖讖,使蠲去崔發所為王莽著錄次比。敏對曰:「讖書非聖人所作,其中多近鄙別字,頗類世俗之辭,恐疑誤後生。」帝不納。敏因其闕文增之曰:「君無口,為漢輔。」帝見而怪之,召敏問其故。敏對曰:「臣見前人增損圖書,敢不自量,竊幸萬一。」帝深非之,雖竟不罪,而亦以此沈滯。

與班彪親善,每相遇,輒日旰忘食,夜分不寢,自以為鍾期伯牙、莊周惠施之相得也。

後三遷長陵令。永平五年,詔書捕男子周慮。慮素有名稱,而善於敏,敏坐繫免官。及出,歎曰:「瘖聾之徒,真世之有道者也,何謂察察而遇斯患乎?」十一年,除郎中,遷諫議大夫。卒於家。

周防字偉公,汝南汝陽人也。父揚,少孤微,常脩逆旅,以供過客,而不受其報。

防年十六,仕郡小吏。世祖巡狩汝南,召掾史試經,防尤能誦讀,拜為守丞。防以未冠,謁去。師事徐州刺史蓋豫,受古文尚書。經明,舉孝廉,拜郎中。撰尚書雜記三十二篇,四十萬言。太尉張禹薦補博士,稍遷陳留太守,坐法免。年七十八,卒於家。

子舉,自有傳。

孔僖字仲和,魯國魯人也。自安國以下,世傳古文尚書、毛詩。曾祖父子建,少遊長安,與崔篆友善。及篆仕王莽為建新大尹,嘗勸子建仕。對曰:「吾有布衣之心,子有袞冕之志,各從所好,不亦善乎!道既乖矣,請從此辭。」遂歸,終於家。

僖與崔篆孫駰復相友善,同遊太學,習春秋。因讀吳王夫差時事,僖廢書歎曰:「若是,所謂畫龍不成反為狗者。」駰曰:「然。昔孝武皇帝始為天子,年方十八,崇信聖道,師則先王,五六年閒,號勝文、景。及後恣己,忘其前之為善。」僖曰:「書傳若此多矣!」鄰房生梁郁儳和之曰:「如此,武帝亦是狗邪?」僖、駰默然不對。郁怒恨之,陰上書告駰、僖誹謗先帝,刺譏當世。事下有司,駰詣吏受訊。僖以吏捕方至,恐誅,乃上書肅宗自訟曰:「臣之愚意,以為凡言誹謗者,謂實無此事而虛加誣之也。至如孝武皇帝,政之美惡,顯在漢史,坦如日月。是為直說書傳實事,非虛謗也。夫帝者為善,則天下之善咸歸焉;其不善,則天下之惡亦萃焉。斯皆有以致之,故不可以誅於人也。且陛下即位以來,政教未過,而德澤有加,天下所具也,臣等獨何譏刺哉?假使所非實是,則固應悛改;儻其不當,亦宜含容,又何罪焉?陛下不推原大數,深自為計,徒肆私忿,以快其意。臣等受戮,死即死耳,顧天下之人,必回視易慮,以此事闚陛下心。自今以後,苟見不可之事,終莫復言者矣。臣之所以不愛其死,猶敢極言者,誠為陛下深惜此大業。陛下若不自惜,則臣何賴焉?齊桓公親揚其先君之惡,以唱管仲,然後群臣得盡其心。今陛下乃欲以十世之武帝,遠諱實事,豈不與桓公異哉?臣恐有司卒然見構,銜恨蒙枉,不得自敘,使後世論者,擅以陛下有所方比,寧可復使子孫追掩之乎?謹詣闕伏待重誅。」帝始亦無罪僖等意,及書奏,立詔勿問,拜僖蘭臺令史。

元和二年春,帝東巡狩,還過魯,幸闕里,以太牢祠孔子及七十二弟子,作六代之樂,大會孔氏男子二十以上者六十三人,命儒者講論。僖因自陳謝。帝曰:「今日之會,寧於卿宗有光榮乎?」對曰:「臣聞明王聖主,莫不尊師貴道。今陛下親屈萬乘,辱臨敝里,此乃崇禮先師,增煇聖德。至於光榮,非所敢承。」帝大笑曰:「非聖者子孫,焉有斯言乎!」遂拜僖郎中,賜褒成侯損及孔氏男女錢帛,詔僖從還京師,使校書東觀。

冬,拜臨晉令,崔駰以家林筮之,謂為不吉,止僖曰:「子盍辭乎?」僖曰:「學不為人,仕不擇官,凶吉由己,而由卜乎?」在縣三年,卒官,遺令即葬。

二子長彥、季彥,並十餘歲。蒲阪令許君然勸令反魯。對曰:「今載柩而歸,則違父令;舍墓而去,心所不忍。」遂留華陰。

長彥好章句學,季彥守其家業,門徒數百人。延光元年,河西大雨雹,大者如斗。安帝詔有道術之士極陳變眚,乃召季彥見於德陽殿,帝親問其故。對曰:「此皆陰乘陽之徵也。今貴臣擅權,母后黨盛,陛下宜脩聖德,慮此二者。」帝默然,左右皆惡之。舉孝廉,不就。三年,年四十七,終於家。

初,平帝時王莽秉政,乃封孔子後孔均為褒成侯,追謚孔子為褒成宣尼。及莽敗,失國。建武十三年,世祖復封均子志為褒成侯。志卒,子損嗣。永元四年,徙封褒亭侯。損卒,子曜嗣。曜卒,子完嗣。世世相傳,至獻帝初,國絕。

楊倫字仲理,陳留東昏人也。少為諸生,師事司徒丁鴻,習古文尚書。為郡文學掾。更歷數將,志乖於時,以不能人閒事,遂去職,不復應州郡命。講授於大澤中,弟子至千餘人。元初中,郡禮請,三府並辟,公車徵,皆辭疾不就。

後特徵博士,為清河王傅。是歲,安帝崩,倫輒棄官奔喪,號泣闕下不絕聲。閻太后以其專擅去職,坐抵罪。

順帝即位,詔免倫刑,遂留行喪于恭陵。服闋,徵拜侍中。是時邵陵令任嘉在職貪穢,因遷武威太守,後有司奏嘉臧罪千萬,徵考廷尉,其所牽染將相大臣百有餘人。倫乃上書曰:「臣聞春秋誅惡及本,本誅則惡消:振裘持領,領正則毛理。今任嘉所坐狼藉,未受辜戮,猥以垢身,改典大郡,自非案坐舉者,無以禁絕姦萌。往者湖陸令張疊、蕭令駟賢、徐州刺史劉福等,釁穢既章,咸伏其誅,而豺狼之吏至今不絕者,豈非本舉之主不加之罪乎?昔齊威之霸,殺姦臣五人,并及舉者,以弭謗讟。當斷不斷,黃石所戒。夫聖王所以聽僮夫匹婦之言者,猶塵加嵩岱,霧集淮海,雖未有益,不為損也。惟陛下留神省察。」奏御,有司以倫言切宜,辭不遜順,下之。尚書奏倫探知密事,激以求直。坐不敬,結鬼薪。詔書以倫數進忠言,特原之,免歸田里。

陽嘉二年,徵拜太中大夫。大將軍梁商以為長史。諫諍不合,出補常山王傅,病不之官。詔書敕司隸催促發遣,倫乃留河內朝歌,以疾自上,曰:「有留死一尺,無北行一寸。刎頸不易,九裂不恨。匹夫所執,彊於三軍。固敢有辭。」帝及下詔曰:「倫出幽升高,寵以藩傅,稽留王命,擅止道路,託疾自從,苟肆狷志。」遂徵詣廷尉,有詔原罪。

倫前後三徵,皆以直諫不合。既歸,閉門講授,自絕人事。公車復徵,遜遁不行,卒於家。

中興,北海牟融習大夏侯尚書,東海王良習小夏侯尚書,沛國桓榮習歐陽尚書。榮世習相傳授,東京最盛。扶風杜林傳古文尚書,林同郡賈逵為之作訓,馬融作傳,鄭玄注解,由是古文尚書遂顯于世。


\end{pinyinscope}