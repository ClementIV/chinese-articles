\article{馮衍傳下}

\begin{pinyinscope}
建武末,上疏自陳曰:

臣伏念高祖之略而陳平之謀,毀之則疏,譽之則親。以文帝之明而魏尚之忠,繩之以法則為罪,施之以德則為功。逮至晚世,董仲舒言道德,見妒於公孫弘,李廣奮節於匈奴,見排於衛青,此忠臣之常所為流涕也。臣衍自惟微賤之臣,上無無知之薦,下無馮唐之說,乏董生之才,寡李廣之埶,而欲免讒口,濟怨嫌,豈不難哉!

臣衍之先祖,以忠貞之故,成私門之禍。而臣衍復遭擾攘之時,值兵革之際,不敢回行求時之利,事君無傾邪之謀,將帥無虜掠之心。衛尉陰興,敬慎周密,內自修敕,外遠嫌疑,故敢與交通。興知臣之貧,數欲本業之。臣自惟無三益之才,不敢處三損之地,固讓而不受之。昔在更始,太原執貨財之柄,居蒼卒之閒,據位食祿二十餘年,而財產歲狹,居處日貧,家無布帛之積,出無輿馬之飾。於今遭清明之時,飭躬力行之秋,而怨讎叢興,譏議橫世。蓋富貴易為善,貧賤難為工也。疏遠壟畝之臣,無望高闕之下,惶恐自陳,以救罪尤。

書奏,猶以前過不用。

衍不得志,退而作賦,又自論曰:

馮子以為夫人之德,不碌碌如玉,落落如石。風興雲蒸,一龍一蛇,與道翱翔,與時變化,夫豈守一節哉?用之則行,舍之則臧,進退無主,屈申無常。故曰:「有法無法,因時為業,有度無度,與物趣舍。」常務道德之實,而不求當世之名,闊略杪小之禮,蕩佚人閒之事。正身直行,恬然肆志。顧嘗好俶儻之策,時莫能聽用其謀,喟然長歎,自傷不遭。久棲遲於小官,不得舒其所懷。抑心折節,意悽情悲。夫伐冰之家,不利雞豚之息;委積之臣,不操市井之利。況歷位食祿二十餘年,而財產益狹,居處益貧。惟夫君子之仕,行其道也。慮時務者不能興其德,為身求者不能成其功。去而歸家,復羇旅於州郡,身愈據職,家彌窮困,卒離飢寒之災,有喪元子之禍。

先將軍葬渭陵,哀帝之崩也,營之以為園。於是以新豐之東,鴻門之上,壽安之中,地埶高敞,四通廣大,南望酈山,北屬涇渭,東瞰河華,龍門之陽,三晉之路,西顧酆鄗,周秦之丘,宮觀之墟,通視千里,覽見舊都,遂定挞焉。退而幽居。蓋忠臣過故墟而歔欷,孝子入舊室而哀歎。每念祖考,著盛德於前,垂鴻烈於後,遭時之禍,墳墓蕪穢,春秋蒸嘗,昭穆無列。年衰歲暮,悼無成功,將西田牧肥饒之野,殖生產,修孝道,營宗廟,廣祭祀。然後闔門講習道德,觀覽乎孔老之論,庶幾乎松喬之福。上隴阪,陟高岡,游精宇宙,流目八紘。歷觀九州山川之體,追覽上古得失之風,愍道陵遲,傷德分崩。夫睹其終必原其始,故存其人而詠其道。疆理九野,經營五山,眇然有思陵雲之意。乃作賦自厲,命其篇曰顯志。顯志者,言光明風化之情,昭章玄妙之思也。其辭曰:

開歲發春兮,百卉含英。甲子之朝兮,汨吾西征。發軔新豐兮,裴回鎬京。陵飛廉而太息兮,登平陽而懷傷。悲時俗之險阨兮,哀好惡之無常。棄衡石而意量兮,隨風波而飛揚。紛綸流於權利兮,親雷同而妒異;獨耿介而慕古兮,豈時人之所憙?沮先聖之成論兮,横名賢之高風;忽道德之珍麗兮,務富貴之樂耽。遵大路而裴回兮,履孔德之窈冥;固眾夫之所眩兮,孰能觀於無形?行勁直以離尤兮,羌前人之所有;內自省而不慚兮,遂定志而弗改。欣吾黨之唐虞兮,愍吾生之愁勤;聊發憤而揚情兮,將以蕩夫憂心。往者不可攀援兮,來者不可與期;病沒世之不稱兮,願橫逝而無由。

陟雍畤而消搖兮,超略陽而不反。念人生之不再兮,悲六親之日遠。陟九嵕而臨战嶭兮,聽涇渭之波聲。顧鴻門而歔欷兮,哀吾孤之早零。何天命之不純兮,信吾罪之所生;傷誠善之無辜兮,齎此恨而入冥。嗟我思之不遠兮,豈敗事之可悔?雖九死而不眠兮,恐余殃之有再。淚汍瀾而雨集兮,氣滂浡而雲披;心怫鬱而紆結兮,意沈抑而內悲。

瞰太行之嵯峨兮,觀壺口之崢嶸;悼丘墓之蕪穢兮,恨昭穆之不榮。歲忽忽而日邁兮,壽冉冉其不與;恥功業之無成兮,赴原野而窮處。昔伊尹之干湯兮,七十說而乃信;皋陶釣於雷澤兮,賴虞舜而後親。無二士之遭遇兮,抱忠貞而莫達;率妻子而耕耘兮,委厥美而不伐。韓盧抑而不縱兮,騏驥絆而不試;獨慷慨而遠覽兮,非庸庸之所識。卑衛賜之阜貨兮,高顏回之所慕;重祖考之洪烈兮,故收功於此路。循四時之代謝兮,分五土之刑德;相林麓之所產兮,嘗水泉之所殖。修神農之本業兮,採軒轅之奇策;追周棄之遺教兮,軼范蠡之絕跡。陟隴山以踰望兮,眇然覽於八荒;風波飄其並興兮,情惆悵而增傷。覽河華之泱漭兮,望秦晉之故國。憤馮亭之不遂兮,慍去疾之遭惑。

流山岳而周覽兮,徇碣石與洞庭;浮江河而入海兮,泝淮濟而上征。瞻燕齊之舊居兮,歷宋楚之名都;哀群后之不祀兮,痛列國之為墟。馳中夏而升降兮,路紆軫而多艱;講聖哲之通論兮,心愊憶而紛紜。惟天路之同軌兮,或帝王之異政;堯舜煥其蕩蕩兮,禹承平而革命。并日夜而幽思兮,終悇憛而洞疑;高陽横其超遠兮,世孰可與論茲?訊夏啟於甘澤兮,傷帝典之始傾;頌成康之載德兮,詠南風之歌聲。思唐虞之晏晏兮,揖稷契與為朋;苗裔紛其條暢兮,至湯武而勃興。昔三后之純粹兮,每季世而窮禍;弔夏桀於南巢兮,哭殷紂於牧野。詔伊尹於亳郊兮,享呂望於酆洲;功與日月齊光兮,名與三王爭流。

楊朱號乎衢路兮,墨子泣乎白絲;知漸染之易性兮,怨造作之弗思。美關雎之識微兮,愍王道之將崩;拔周唐之盛德兮,捃桓文之譎功。忿戰國之遘禍兮,憎權臣之擅彊;黜楚子於南郢兮,執趙武於湨梁。善忠信之救時兮,惡詐謀之妄作;聘申叔於陳蔡兮,禽荀息於虞总。誅犁鉏之介聖兮,討臧倉之愬知;惫子反於彭城兮,爵管仲於夷儀。疾兵革之寖滋兮,苦攻伐之萌生;沈孫武於五湖兮,斬白起於長平。惡叢巧之亂世兮,毒從橫之敗俗;流蘇秦於洹水兮,幽張儀於鬼谷。澄德化之陵遲兮,烈刑罰之峭峻;燔商鞅之法術兮,燒韓非之說論。誚始皇之跋扈兮,投李斯於四裔;滅先王之法則兮,禍寖淫而弘大。援前聖以制中兮,矯二主之驕奢;饁女齊於絳臺兮,饗椒舉於章華。摛道德之光耀兮,匡衰世之眇風;褒宋襄於泓谷兮,表季札於延陵。摭仁智之英華兮,激亂國之末流;觀鄭僑於溱洧兮,訪晏嬰於營丘。日曀曀其將暮兮,獨於邑而煩惑;夫何九州之博大兮,迷不知路之南北。駟素虯而馳騁兮,乘翠雲而相佯;就伯夷而折中兮,得務光而愈明。款子高於中野兮,遇伯成而定慮;欽真人之德美兮,淹躊躇而弗去。意斟愖而不澹兮,俟回風而容與;求善卷之所存兮,遇許由於負黍。軔吾車於箕陽兮,秣吾馬於潁滸;聞至言而曉領兮,還吾反乎故宇。

覽天地之幽奧兮,統萬物之維綱;究陰陽之變化兮,昭五德之精光。躍青龍於滄海兮,豢白虎於金山;鑿巖石而為室兮,託高陽以養仙,神雀翔於鴻崖兮,玄武潛於嬰冥;伏朱樓而四望兮,採三秀之華英。纂前修之夸節兮,曜往昔之光勳;披綺季之麗服兮,揚屈原之靈芬。高吾冠之岌岌兮,長吾佩之洋洋;飲六醴之清液兮,食五芝之茂英。

揵六枳而為籬兮,築蕙若而為室;播蘭芷於中廷兮,列杜衡於外術。攢射干雜蘼蕪兮,搆木蘭與新夷;光扈扈而煬燿兮,紛郁郁而暢美;華芳曄其發越兮,時恍忽而莫貴;非惜身之埳軻兮,憐眾美之憔悴。游精神於大宅兮,抗玄妙之常操;處清靜以養志兮,實吾心之所樂。山峨峨而造天兮,林冥冥而暢茂;鸞回翔索其群兮,鹿哀鳴而求其友。誦古今以散思兮,覽聖賢以自鎮;嘉孔丘之知命兮,大老聃之貴玄;德與道其孰寶兮?名與身其孰親?陂山谷而閒處兮,守寂寞而存神。夫莊周之釣魚兮,辭卿相之顯位;於陵子之灌園兮,似至人之髣彿。蓋隱約而得道兮,羌窮悟而入術;離塵垢之窈冥兮,配喬、松之妙節。惟吾志之所庶兮,固與俗其不同;既俶儻而高引兮,願觀其從容。

顯宗即位,又多短衍以文過其實,遂廢於家。

衍娶北地女任氏為妻,悍忌,不得畜媵妾,兒女常自操井臼,老竟逐之,遂埳壈於時。然有大志,不戚戚於賤貧。居常慷慨歎曰:「衍少事名賢,經歷顯位,懷金垂紫,揭節奉使,不求苟得,常有陵雲之志。三公之貴,千金之富,不得其願,不概於懷。貧而不衰,賤而不恨,年雖疲曳,猶庶幾名賢之風。修道德於幽冥之路,以終身名,為後世法。」居貧年老,卒于家。所著賦、誄、銘、說、問交、德誥、慎情、書記說、自序、官錄說、策五十篇,肅宗甚重其文。子豹。

豹字仲文,年十二,母為父所出。後母惡之,嘗因豹夜寐,欲行毒害,豹逃走得免。敬事愈謹,而母疾之益深,時人稱其孝。長好儒學,以詩、春秋教麗山下。鄉里為之語曰:「道德彬彬馮仲文。」舉孝廉,拜尚書郎,忠勤不懈。每奏事未報,常俯伏省閤,或從昏至明。肅宗聞而嘉之,使黃門持被覆豹,敕令勿驚,由是數加賞賜。是時方平西域,以豹有才謀,拜為河西副校尉。和帝初,數言邊事,奏置戊己校尉,城郭諸國復率舊職。遷武威太守,視事二年,河西稱之,復徵入為尚書。永元十四年,卒於官。

論曰:夫貴者負埶而驕人,才士負能而遺行,其大略然也。二子不其然乎!馮衍之引挑妻之譬,得矣。夫納妻皆知取詈己者,而取士則不能。何也?豈非反妒情易,而恕義情難。光武雖得之於鮑永,猶失之於馮衍。夫然,義直所以見屈於既往,守節故亦彌阻於來情。嗚呼!

贊曰:譚非讖術,衍晚委質。道不相謀,詭時同失。體兼上才,榮微下秩。


\end{pinyinscope}