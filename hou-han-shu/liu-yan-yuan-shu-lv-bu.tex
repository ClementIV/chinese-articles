\article{劉焉袁術呂布列傳}

\begin{pinyinscope}
劉焉字君郎,江夏竟陵人也,魯恭王後也。肅宗時,徙竟陵。焉少任州郡,以宗室拜郎中。去官居陽城山,精學教授。舉賢良方正,稍遷南陽太守、宗正、太常。

時靈帝政化衰缺,四方兵寇,焉以為刺史威輕,既不能禁,且用非其人,輒增暴亂,乃建議改置牧伯,鎮安方夏,清選重臣,以居其任。焉乃陰求為交阯,以避時難。議未即行,會益州刺史郗儉在政煩擾,謠言遠聞,而并州刺史張懿、涼州刺史耿鄙並為寇賊所害,故焉議得用。出焉為監軍使者,領益州牧,太僕黃琬為豫州牧,宗正劉虞為幽州牧,皆以本秩居職。州任之重,自此而始。

是時益州賊馬相亦自號「黃巾」,合聚疲役之民數千人,先殺綿竹令,進攻雒縣,殺郗儉,又擊蜀郡、犍為,旬月之閒,破壞三郡。馬相自稱「天子」,眾至十餘萬人,遣兵破巴郡,殺郡守趙部。州從事賈龍,先領兵數百人在犍為,遂糾合吏人攻相,破之,龍乃遣吏卒迎焉。焉到,以龍為校尉,徙居綿竹。龍撫納離叛,務行寬惠,而陰圖異計。

沛人張魯,母有恣色,兼挾鬼道,往來焉家,遂任魯以為督義司馬,遂與別部司馬張脩將兵掩殺漢中太守蘇固,斷絕斜谷,殺使者。魯既得漢中,遂復殺張脩而并其眾。

焉欲立威刑以自尊大,乃託以佗事,殺州中豪彊十餘人,士民皆怨。初平二年,犍為太守任岐及賈龍並反,攻焉。焉擊破,皆殺之。自此意氣漸盛,遂造作乘輿車重千餘乘。焉四子,範為左中郎將,誕治書御史,璋奉車都尉,並從獻帝在長安,唯別部司馬瑁隨焉在益州。朝廷使璋曉譬焉,焉留璋不復遣。興平元年,征西將軍馬騰與範謀誅李傕,焉遣叟兵五千助之,戰敗,範及誕並見殺。焉既痛二子,又遇天火燒其城府車重,延及民家,館邑無餘,於是徙居成都,遂發背疽卒。

州大吏趙韙等貪璋溫仁,立為刺史。詔書因以璋為監軍使者,領益州牧,以韙為征東中郎將。先是荊州牧劉表表焉僭擬乘輿器服,韙以此遂屯兵朐颈備表。

初,南陽、三輔民數萬戶流入益州,焉悉收以為眾,名曰「東州兵」。璋性柔寬無威略,東州人侵暴為民患,不能禁制,舊士頗有離怨。趙韙之在巴中,甚得眾心,璋委之以權。韙因人情不輯,乃陰結州中大姓。建安五年,還共擊璋,蜀郡、廣漢、犍為皆反應。東州人畏見誅滅,乃同心并力,為璋死戰,遂破反者,進攻韙於江州,斬之。

張魯以璋闇懦,不復承順。璋怒,殺魯母及弟,而遣其將龐羲等攻魯,數為所破。魯部曲多在巴土,故以羲為巴郡太守。魯因襲取之,遂雄於巴漢。

十三年,曹操自將征荊州,璋乃遣使致敬。操加璋振威將軍,兄瑁平寇將軍。璋因遣別駕從事張松詣操,而操不相接禮。松懷恨而還,勸璋絕曹氏,而結好劉備。璋從之。

十六年,璋聞曹操當遣兵向漢中討張魯,內懷恐懼,松復說璋迎劉備以拒操。璋即遣法正將兵迎備。璋主簿巴西黃權諫曰:「劉備有梟名,今以部曲遇之,則不滿其心,以賓客待之,則一國不容二主,此非自安之道。」從事廣漢王累自倒懸於州門以諫。璋一無所納。

備自江陵馳至涪城,璋率步騎數萬與備會。張松勸備於會襲璋,備不忍。明年,出屯葭萌。松兄廣漢太守肅懼禍及己,乃以松謀白璋,收松斬之,敕諸關戍勿復通。備大怒,還兵擊璋,所在戰剋。十九年,進圍成都,數十日,城中有精兵三萬人,穀支一年,吏民咸欲拒戰。璋言:「父子在州二十餘歲,無恩德以加百姓,而攻戰三載,肌膏草野者,以璋故也。何心能安!」遂開城出降,群下莫不流涕。備遷璋於公安,歸其財寶,後以病卒。

明年,曹操破張魯,定漢中。

魯字公旗。初,祖父陵,順帝時客於蜀,學道鶴鳴山中,造作符書,以惑百姓。受其道者輒出米五斗,故謂之「米賊」。陵傳子衡,衡傳於魯,魯遂自號「師君」。其來學者,初名為「鬼卒」,後號「祭酒」。祭酒各領部眾,眾多者名曰「理頭」。皆校以誠信,不聽欺妄,有病但令首過而已。諸祭酒各起義舍於路,同之亭傳,縣置米肉以給行旅。食者量腹取足,過多則鬼能病之。犯法者先加三原,然後行刑。不置長吏,以祭酒為理,民夷信向。朝廷不能討,遂就拜魯鎮夷中郎將,領漢寧太守,通其貢獻。

韓遂、馬超之亂,關西民奔魯者數萬家。時人有地中得玉印者,群下欲尊魯為漢寧王。魯功曹閻圃諫曰:「漢川之民,戶出十萬,四面險固,財富土沃,上匡天子,則為桓文,次方竇融,不失富貴。今承制署置,埶足斬斷。遽稱王號,必為禍先。」魯從之。

魯自在漢川垂三十年,聞曹操征之,至陽平,欲舉漢中降。其弟衛不聽,率眾數萬,拒關固守。操破衛,斬之。魯聞陽平已陷,將稽顙歸降。閻圃說曰:「今以急往,其功為輕,不如且依巴中,然後委質,功必多也。」於是乃奔南山。左右欲悉焚寶貨倉庫。魯曰:「本欲歸命國家,其意未遂。今日之走,以避鋒銳,非有惡意。」遂封藏而去。操入南鄭,甚嘉之。又以魯本有善意,遣人慰安之。魯即與家屬出逆,拜鎮南將軍,封閬中侯,邑萬戶,將還中國,待以客禮。封魯五子及閻圃等皆為列侯。

魯卒,謚曰原侯。子富嗣。

論曰:劉焉睹時方艱,先求後亡之所,庶乎見幾而作。夫地廣則驕尊之心生,財衍則僭奢之情用,固亦恆人必至之期也。璋能閉隘養力,守案先圖,尚可與歲時推移,而遽輸利器,靜受流斥,所謂羊質虎皮,見豺則恐,吁哉!

袁術字公路,汝南汝陽人,司空逢之子也。少以俠氣聞,數與諸公子飛鷹走狗,後頗折節。舉孝廉,累遷至河南尹、虎賁中郎將。

時董卓將欲廢立,以術為後將軍。術畏卓之禍,出奔南陽。會長沙太守孫堅殺南陽太守張咨,引兵從術。劉表上術為南陽太守,術又表堅領豫州刺史,使率荊、豫之卒,擊破董卓於陽人。

術從兄紹因堅討卓未反,遠,遣其將會稽周昕奪堅豫州。術怒,擊昕走之。紹議欲立劉虞為帝,術好放縱,憚立長君,託以公義不肯同,積此釁隙遂成。乃各外交黨援,以相圖謀,術結公孫瓚,而紹連劉表。豪桀多附於紹,術怒曰:「群豎不吾從,而從吾家奴乎!」又與公孫瓚書,云紹非袁氏子,紹聞大怒。初平三年,術遣孫堅擊劉表於襄陽,堅戰死。公孫瓚使劉備與術合謀共逼紹,紹與曹操會擊,皆破之。四年,術引軍入陳留,屯封丘。黑山餘賊及匈奴於扶羅等佐術,與曹操戰於匡亭,大敗。術退保雍丘,又將其餘眾奔九江,殺楊州刺史陳溫而自領之,又兼稱徐州伯。李傕入長安,欲結術為援,乃授以左將軍,假節,封陽翟侯。

初,術在南陽,戶口尚數十百萬,而不修法度,以鈔掠為資,奢恣無猒,百姓患之。又少見識書,言「代漢者當塗高」,自云名字應之。又以袁氏出陳為舜後,以黃代赤,德運之次,遂有僭逆之謀。又聞孫堅得傳國璽,遂拘堅妻奪之。興平二年冬,天子播越,敗於曹陽。術大會群下,因謂曰:「今海內鼎沸,劉氏微弱。吾家四世公輔,百姓所歸,欲應天順民,於諸君何如?」眾莫敢對。主簿閻象進曰:「昔周自后稷至于文王,積德累功,參分天下,猶服事殷。明公雖奕世克昌,孰若有周之盛?漢室雖微,未至殷紂之敝也。」術嘿然,使召張範。範辭疾,遣弟承往應之。術問曰:「昔周室陵遲,則有桓文之霸;秦失其政,漢接而用之。今孤以土地之廣,士人之眾,欲徼福於齊桓,擬跡於高祖,可乎?」承對曰:「在德不在眾。苟能用德以同天下之欲,雖云匹夫,霸王可也。若陵僭無度,干時而動,眾之所棄,誰能興之!」術不說。

自孫堅死,子策復領其部曲,術遣擊楊州刺史劉繇,破之,策因據江東。策聞術將欲僭號,與書諫曰:「董卓無道,陵虐王室,禍加太后,暴及弘農,天子播越,宮廟焚毀,是以豪桀發憤,沛然俱起。元惡既斃,幼主東顧,乃使王人奉命,宣明朝恩,偃武修文,與之更始。然而河北異謀於黑山,曹操毒被於東徐,劉表僭亂於南荊,公孫叛逆於朔北,正禮阻兵,玄德爭盟,是以未獲從命,櫜弓戢戈。當謂使君與國同規,而舍是弗恤,完然有自取之志,懼非海內企望之意也。成湯討桀,稱『有夏多罪』;武王伐紂,曰『殷有重罰』。此二王者,雖有聖德,假使時無失道之過,無由逼而取也。今主上非有惡於天下,徒以幼子脅於彊臣,異於湯武之時也。又聞幼主明智聰敏,有夙成之德,天下雖未被其恩,咸歸心焉。若輔而興之,則旦、奭之美,率土所望也。使君五世相承,為漢宰輔,榮寵之盛,莫與為比,宜效忠守節,以報王室。時人多惑圖緯之言,妄牽非類之文,苟以悅主為美,不顧成敗之計,古今所慎,可不孰慮!忠言逆耳,駮議致憎,苟有益於尊明,無所敢辭。」術不納,策遂絕之。

建安二年,因河內張炯符命,遂果僭號,自稱「仲家」。以九江太守為淮南尹,置公卿百官,郊祀天地。乃遣使以竊號告呂布,并為子娉布女。布執術使送許。術大怒,遣其將張勳、橋蕤攻布,大敗而還。術又率兵擊陳國,誘殺其王寵及相駱俊,曹操乃自征之。術聞大駭,即走度淮,留張勳、橋蕤於蘄陽,以拒操。擊破斬蕤,而勳退走。術兵弱,大將死,眾情離叛。加天旱歲荒,士民凍餒,江、淮閒相食殆盡。時舒仲應為術沛相,術以米十萬斛與為軍糧,仲應悉散以給飢民。術聞怒,陳兵將斬之。仲應曰:「知當必死,故為之耳。寧可以一人之命,救百姓於塗炭。」術下馬牽之曰:「仲應,足下獨欲享天下重名,不與吾共之邪?」

術雖矜名尚奇,而天性驕肆,尊己陵物。及竊偽號,淫侈滋甚,媵御數百,無不兼羅紈,厭粱肉,自下飢困,莫之簡卹。於是資實空盡,不能自立。四年夏,乃燒宮室,奔其部曲陳簡、雷薄於灊山。復為簡等所拒,遂大困窮,士卒散走。憂懣不知所為,遂歸帝號於紹,曰:「祿去漢室久矣,天下提挈,政在家門。豪雄角逐,分割疆宇。此與周末七國無異,唯彊者兼之耳。袁氏受命當王,符瑞炳然。今君擁有四州,人戶百萬,以彊則莫與爭大,以位則無所比高。曹操雖欲扶衰獎微,安能續絕運,起已滅乎!謹歸大命,君其興之。」紹陰然其計。

術因欲北至青州從袁譚,曹操使劉備徼之,不得過,復走還壽春。六月,至江亭。坐簀床而歎曰:「袁術乃至是乎!」因憤慨結病,歐血死。妻子依故吏廬江太守劉勳。孫策破勳,復見收視,術女入孫權宮,子曜仕吳為郎中。

論曰:天命符驗,可得而見,未可得而言也。然大致受大福者,歸於信順乎!夫事不以順,雖彊力廣謀,不能得也。謀不可得之事,日失忠信,變詐妄生矣。況復苟肆行之,其以欺天乎!雖假符僭稱,歸將安所容哉!

呂布字奉先,五原九原人也。以弓馬驍武給并州。刺史丁原為騎都尉,原屯河內,以布為主簿,甚見親待。靈帝崩,原受何進召,將兵詣洛陽,為執金吾。會進敗,董卓誘布殺原而并其兵。

卓以布為騎都尉,誓為父子,甚愛信之。稍遷至中郎將,封都亭侯。卓自知凶恣,每懷猜畏,行止常以布自衛。嘗小失卓意,卓拔手戟擲之。布拳捷得免,而改容顧謝,卓意亦解。布由是陰怨於卓。卓又使布守中閤,而私與傅婢情通,益不自安。因往見司徒王允,自陳卓幾見殺之狀。時允與尚書僕射士孫瑞密謀誅卓,因以告布,使為內應。布曰:「如父子何?」曰:「君自姓呂,本非骨肉。今憂死不暇,何謂父子?擲戟之時,豈有父子情也?」布遂許之,乃於門刺殺卓,事已見卓傳。允以布為奮威將軍,假節,儀同三司,封溫侯。

允既不赦涼州人,由是卓將李傕等遂相結,還攻長安。布與傕戰,敗,乃將數百騎,以卓頭繫馬鞍,走出武關,奔南陽。袁術待之甚厚。布自恃殺卓,有德袁氏,遂恣兵鈔掠。術患之。布不安,復去從張楊於河內。時李傕等購募求布急,楊下諸將皆欲圖之。布懼,謂楊曰:「與卿州里,今見殺,其功未必多。不如生賣布,可大得傕等爵寵。」楊以為然。有頃,布得走投袁紹,紹與布擊張燕於常山。燕精兵萬餘,騎數千匹。布常御良馬,號曰赤菟,能馳城飛塹,與其健將成廉、魏越等數十騎馳突燕陣,一日或至三四,皆斬首而出。連戰十餘日,遂破燕軍。布既恃其功,更請兵於紹,紹不許,而將士多暴橫,紹患之。布不自安,因求還洛陽。紹聽之,承制使領司隸校尉,遣壯士送布而陰使殺之。布疑其圖己,乃使人鼓箏於帳中,潛自遁出。夜中兵起,而布已亡。紹聞,懼為患,募遣追之,皆莫敢逼,遂歸張楊。道經陳留,太守張邈遣使迎之,相待甚厚,臨別把臂言誓。

邈字孟卓,東平人,少以俠聞。初辟公府,稍遷陳留太守。董卓之亂,與曹操共舉義兵。及袁紹為盟主,有驕色,邈正義責之。紹既怨邈,且聞與布厚,乃令曹操殺邈。操不聽,然邈心不自安。興平元年,曹操東擊陶謙,令其將武陽人陳宮屯東郡。宮因說邈曰:「今天下分崩,雄桀並起,君擁十萬之眾,當四戰之地,撫劍顧眄,亦足以為人豪,而反受制,不以鄙乎!今州軍東征,其處空虛,呂布壯士,善戰無前,迎之共據兗州,觀天下形埶,俟時事變通,此亦從橫一時也。」邈從之,遂與弟超及宮等迎布為兗州牧,據濮陽,郡縣皆應之。

曹操聞而引軍擊布,累戰,相持百餘日。是時旱蝗少穀,百姓相食,布移屯山陽。二年閒,操復盡收諸城,破布於鉅野,布東奔劉備。邈詣袁術求救,留超將家屬屯雍丘。操圍超數月,屠之,滅其三族。邈未至壽春,為其兵所害。

時劉備領徐州,居下邳,與袁術相拒於淮上。術欲引布擊備,乃與布書曰:「術舉兵詣闕,未能屠裂董卓。將軍誅卓,為術報恥,功一也。昔金元休南至封丘,為曹操所敗。將軍伐之,令術復明目於遐邇,功二也。術生年以來,不聞天下有劉備,備乃舉兵與術對戰。憑將軍威靈,得以破備,功三也。將軍有三大功在術,術雖不敏,奉以死生。將軍連年攻戰,軍糧苦少,今送米二十萬斛。非唯此止,當駱驛復致。凡所短長亦唯命。」布得書大悅,即勒兵襲下邳,獲備妻子。備敗走海西,飢困,請降於布。布又恚術運糧不復至,乃具車馬迎備,以為豫州刺史,遣屯小沛。布自號徐州牧。術懼布為己害,為子求婚,布復許之。

術遣將紀靈等步騎三萬以攻備,備求救於布。諸將謂布曰:「將軍常欲殺劉備,今可假手於術。」布曰:「不然。術若破備,則北連太山,吾為在術圍中,不得不救也。」便率步騎千餘,馳往赴之。靈等聞布至,皆斂兵而止。布屯沛城外,遣人招備,并請靈等與共饗飲。布謂靈曰:「玄德,布弟也,為諸君所困,故來救之。布性不喜合鬥,但喜解鬥耳。」乃令軍候植戟於營門,布彎弓顧曰:「諸君觀布射小支,中者當各解兵,不中可留決鬥。」布即一發,正中戟支。靈等皆驚,言「將軍天威也」。明日復歡會,然後各罷。

術遣韓胤以僭號事告布,因求迎婦,布遣女隨之。沛相陳珪恐術報布成姻,則徐楊合從,為難未已。於是往說布曰:「曹公奉迎天子,輔贊國政,將軍宜與協同策謀,共存大計。今與袁術結姻,必受不義之名,將有累卵之危矣。」布亦素怨術,而女已在塗,乃追還絕婚,執胤送許,曹操殺之。

陳珪欲使子登詣曹操,布固不許,會使至,拜布為左將軍,布大喜,即聽登行,并令奉章謝恩。登見曹操,因陳布勇而無謀,輕於去就,宜早圖之。操曰:「布狼子野心,誠難久養,非卿莫究其情偽。」即增珪秩中二千石,拜登廣陵太守。臨別,操執登手曰:「東方之事,便以相付。」令陰合部眾,以為內應。始布因登求徐州牧,不得。登還,布怒,拔戟斫机曰:「卿父勸吾協同曹操,絕婚公路。今吾所求無獲,而卿父子並顯重,但為卿所賣耳。」登不為動容,徐對之曰:「登見曹公,言養將軍譬如養虎,當飽其肉,不飽則將噬人。公曰:『不如卿言。譬如養鷹,飢即為用,飽則颺去。』其言如此。」布意乃解。

袁術怒布殺韓胤,遣其大將張勳、橋蕤等與韓暹、楊奉連埶,步騎數萬,七道攻布。布時兵有三千,馬四百匹,懼其不敵,謂陳珪曰:「今致術軍,卿之由也,為之柰何?」珪曰;「暹、奉與術,卒合之師耳。謀無素定,不能相維。子登策之,比於連雞,埶不俱棲,立可離也。」布用珪策,與暹、奉書曰:「二將軍親拔大駕,而布手殺董卓,俱立功名,當垂竹帛。今袁術造逆,宜共誅討,柰何與賊還來伐布?可因今者同力破術,為國除害,建功天下,此時不可失也。」又許破術兵,悉以軍資與之。暹、奉大喜,遂共擊勳等於下邳,大破之,生禽橋蕤,餘眾潰走,其所殺傷、墯水死者殆盡。

時太山臧霸等攻破莒城,許布財幣以相結,而未及送,布乃自往求之。其督將高順諫止曰:「將軍威名宣播,遠近所畏,何求不得,而自行求賂。萬一不剋,豈不損邪?」布不從。既至莒,霸等不測往意,固守拒之,無獲而還。順為人清白有威嚴,少言辭,將眾整齊,每戰必剋。布性決易,所為無常。順每諫曰:「將軍舉動,不肯詳思,忽有失得,動輒言誤。誤事豈可數乎?」布知其忠而不能從。

建安三年,布遂復從袁術,遣順攻劉備於沛,破之。曹操遣夏侯惇救備。為順所敗。操乃自將擊布,至下邳城下。遺布書,為陳禍福。布欲降,而陳宮等自以負罪於操,深沮其計,而謂布曰:「曹公遠來,埶不能久。將軍若以步騎出屯於外,宮將餘眾閉守於內。若向將軍,宮引兵而攻其背;若但攻城,則將軍救於外。不過旬月,軍食畢盡,擊之可破也。」布然之。布妻曰:「昔曹氏待公臺如赤子,猶舍而歸我。今將軍厚公臺不過於曹氏,而欲委全城,捐妻子,孤軍遠出乎?若一旦有變,妾豈得為將軍妻哉!」布乃止。而潛遣人求救於袁術,自將千餘騎出。戰敗走還,保城不敢出。術亦不能救。

曹操塹圍之,壅沂、泗以灌其城,三月,上下離心。其將侯成使客牧其名馬,而客策之以叛。成追客得馬,諸將合禮以賀成。成分酒肉,先入詣布而言曰:「蒙將軍威靈,得所亡馬,諸將齊賀,未敢嘗也,故先以奉貢。」布怒曰:「布禁酒而卿等醞釀,為欲因酒共謀布邪?」成忿懼,乃與諸將共執陳宮、高順,率其眾降。布與麾下登白門樓。兵圍之急,令左右取其首詣操。左右不忍,乃下降。布見操曰:「今日已往,天下定矣。」操曰:「何以言之?」布曰:「明公之所患不過於布,今已服矣。令布將騎,明公將步,天下不足定也。」顧謂劉備曰:「玄德,卿為坐上客,我為降虜,繩縛我急,獨不可一言邪?」操笑曰:「縛虎不得不急。」乃命緩布縛。劉備曰:「不可。明公不見呂布事丁建陽、董太師乎?」操頷之。布目備曰:「大耳兒最叵信!」操謂陳宮曰:「公臺平生自謂智有餘,今意何如?」宮指布曰:「是子不用宮言,以至於此。若見從,未可量也。」操又曰:「柰卿老母何?」宮曰:「老母在公,不在宮也。夫以孝理天下者,不害人之親。」操復曰:「柰卿妻子何?」宮曰:「宮聞霸王之主,不絕人之祀。」固請就刑,遂出不顧,操為之泣涕。布及宮、順皆縊殺之,傳首許市。

贊曰:焉作庸牧,以希後福。曷云負荷?地墮身逐。術既叨貪,布亦侴覆。


\end{pinyinscope}