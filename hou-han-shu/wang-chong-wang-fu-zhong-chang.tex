\article{王充王符仲長統列傳}

\begin{pinyinscope}
王充字仲任,會稽上虞人也,其先自魏郡元城徙焉。充少孤,鄉里稱孝。後到京師,受業太學,師事扶風班彪。好博覽而不守章句。家貧無書,常游洛陽市肆,閱所賣書,一見輒能誦憶,遂博通眾流百家之言。後歸鄉里,屏居教授。仕郡為功曹,以數諫爭不合去。

充好論說,始若詭異,終有理實。以為俗儒守文,多失其真,乃閉門潛思,絕慶弔之禮,戶牖牆壁各置刀筆。箸論衡八十五篇,二十餘萬言,釋物類同異,正時俗嫌疑。

刺史董勤辟為從事,轉治中,自免還家。友人同郡謝夷吾上書薦充才學,肅宗特詔公車徵,病不行。年漸七十,志力衰耗,乃造養性書十六篇,裁節嗜欲,頤神自守。永元中,病卒于家。

王符字節信,安定臨涇人也。少好學,有志操,與馬融、竇章、張衡、崔瑗等友善。安定俗鄙庶孽,而符無外家,為鄉人所賤。自和、安之後,世務游宦,當塗者更相薦引,而符獨耿介不同於俗,以此遂不得升進。志意蘊憤,乃隱居著書三十餘篇,以譏當時失得,不欲章顯其名,故號曰潛夫論。其指訐時短,討謫物情,足以觀見當時風政,著其五篇云爾。

貴忠篇曰:

夫帝王之所尊敬者天也,皇天之所愛育者人也。今人臣受君之重位,牧天之所愛,焉可以不安而利之,養而濟之哉?是以君子任職則思利人,達上則思進賢,故居上而下不怨,在前而後不恨也。《書》稱「天工人其代之」。王者法天而建官,故明主不敢以私授,忠臣不敢以虛受。竊人之財猶謂之盜,況偷天官以私己乎!以罪犯人,必加誅罰,況乃犯天,得無咎乎?夫五世之臣,以道事君,澤及草木,仁被率土,是以福祚流衍,本支百世。季世之臣,以諂媚主,不思順天,專杖殺伐。白起、蒙恬,秦以為功,天以為賊;息夫、董賢,主以為忠,天以為盜。《易》曰:「德薄而位尊,智小而謀大,鮮不及矣。」是故德不稱,其禍必酷;能不稱,其殃必大。夫竊位之人,天奪其鑒。雖有明察之資,仁義之志,一旦富貴,則背親捐舊,喪其本心,疏骨肉而親便辟,薄知友而厚犬馬,寧見朽貫千萬,而不忍貸人一錢,情知積粟腐倉,而不忍貸人一斗,骨肉怨望於家,細人謗讟於道。前人以敗,後爭襲之,誠可傷也。

歷觀前政貴人之用心也,與嬰兒子其何異哉?嬰兒有常病,貴臣有常禍,父母有常失,人君有常過。嬰兒常病,傷於飽也;貴臣常禍,傷於寵也。哺乳多則生鸩病,富貴盛而致驕疾。愛子而賊之,驕臣而滅之者,非一也。極其罰者,乃有仆死深牢,銜刀都巿,豈非無功於天,有害於人者乎?夫鳥以山為埤而增巢其上,魚以泉為淺而穿穴其中,卒所以得者餌也。貴戚願其宅吉而制為令名,欲其門堅而造作鐵樞,卒其所以敗者,非苦禁忌少而門樞朽也,常苦崇財貨而行驕僭耳。

不上順天心,下育人物,而欲任其私智,竊弄君威,反戾天地,欺誣神明。居累卵之危,而圖太山之安,為朝露之行,而思傳世之功。豈不惑哉!豈不惑哉!

浮侈篇曰:

王者以四海為家,兆人為子。一夫不耕,天下受其飢;一婦不織,天下受其寒。今舉俗舍本農,趨商賈,牛馬車輿,填塞道路,游手為巧,充盈都邑,務本者少,浮食者眾。「商邑翼翼,四方是極。」今察洛陽,資末業者什於農夫,虛偽游手什於末業。是則一夫耕,百人食之,一婦桑,百人衣之,以一奉百,孰能供之!天下百郡千縣,巿邑萬數,類皆如此。本末不足相供,則民安得不飢寒?飢寒並至,則民安能無姦軌?姦軌繁多,則吏安能無嚴酷?嚴酷數加,則下安能無愁怨?愁怨者多,則咎徵並臻。下民無聊,而上天降災,則國危矣。

夫貧生於富,弱生於彊,亂生於化,危生於安。是故明王之養民,憂之勞之,教之誨之,慎微防萌,以斷其邪。故易美節以制度,不傷財,不害民。七月之詩,大小教之,終而復始。由此觀之,人固不可恣也。

今人奢衣服,侈飲食,事口舌而習調欺。或以謀姦合任為業,或以游博持掩為事。丁夫不扶犁鋤,而懷丸挾彈,攜手上山遨遊,或好取土作丸賣之,外不足禦寇盜,內不足禁鼠雀。或作泥車瓦狗諸戲弄之具,以巧詐小兒,此皆無益也。

詩刺「不績其麻,巿也婆娑」。又婦人不修中饋,休其蠶織,而起學巫祝,鼓舞事神,以欺誣細民,熒惑百姓妻女。羸弱疾病之家,懷憂憤憤,易為恐懼。至使奔走便時,去離正宅,崎嶇路側,風寒所傷,姦人所利,盜賊所中。或增禍重祟,至於死亡,而不知巫所欺誤,反恨事神之晚,此妖妄之甚者也。

或刻畫好繒,以書祝辭;或虛飾巧言,希致福祚;或糜折金綵,令廣分寸;或斷截眾縷,繞帶手腕;或裁切綺縠,繨紩成幡。皆單費百縑,用功千倍,破牢為偽,以易就難,坐食嘉穀,消損白日。夫山林不能給野火,江海不能實漏卮,皆所宜禁也。

昔孝文皇帝躬衣弋綈,革舄韋帶。而今京師貴戚,衣服飲食,車輿廬第,奢過王制,固亦甚矣。且其徒御僕妾,皆服文組綵牒,錦繡綺紈,葛子升越,筩中女布。犀象珠玉,虎魄玳瑁,石山隱飾,金銀錯鏤,窮極麗靡,轉相誇吒。其嫁娶者,車軿數里,緹帷竟道,騎奴侍童,夾轂並引。富者競欲相過,貧者恥其不逮,一饗之所費,破終身之業。古者必有命然後乃得衣繒絲而乘車馬,今雖不能復古,宜令細民略用孝文之制。

古之葬者,厚衣之以薪,葬之中野,不封不樹,喪期無數。後世聖人易之以棺槨,桐木為棺,葛采為緘,下不及泉,上不泄臭。中世以後,轉用楸梓槐柏杶樗之屬,各因方土,裁用膠漆,使其堅足恃,其用足任,如此而已。今者京師貴戚,必欲江南檽梓豫章之木。邊遠下土,亦競相放效。夫檽樟豫章,所出殊遠,伐之高山,引之窮谷,入海乘淮,逆河泝洛,工匠彫刻,連累日月,會眾而後動,多牛而後致,重且千斤,功將萬夫,而東至樂浪,西達敦煌,費力傷農於萬里之地。古者墓而不墳,中世墳而不崇。仲尼喪母,冢高四尺,遇雨而崩,弟子請修之,夫子泣曰:「古不修墓。」及鯉也死,有棺無槨。文帝葬芷陽,明帝葬洛南,皆不臧珠寶,不起山陵,墓雖卑而德最高。今京師貴戚,郡縣豪家,生不極養,死乃崇喪。或至金縷玉匣,檽梓楩柟,多埋珍寶偶人車馬,造起大冢,廣種松柏,廬舍祠堂,務崇華侈。案鄗畢之陵,南城之冢,周公非不忠,曾子非不孝,以為褒君愛父,不在於聚財,揚名顯親,無取於車馬。昔晉靈公多賦以雕牆,春秋以為非君;華元、樂舉厚葬文公,君子以為不臣。況於群司士庶,乃可僭侈主上,過天道乎?

實貢篇曰:

國以賢興,以諂衰;君以忠安,以佞危。此古今之常論,而時所共知也。然衰國危君,繼踵不絕者,豈時無忠信正直之士哉,誠苦其道不得行耳。夫十步之閒,必有茂草;十室之邑,必有忠信。是故亂殷有三仁,小衛多君子。今以大漢之廣土,士民之繁庶,朝廷之清明,上下之脩正,而官無善吏,位無良臣。此豈時之無賢,諒由取之乖實。夫志道者少與,逐俗者多疇,是以朋黨用私,背實趨華。其貢士者,不復依其質幹,準其才行,但虛造聲譽,妄生羽毛。略計所舉,歲且二百。覽察其狀,則德侔顏、冉,詳覈厥能,則鮮及中人,皆總務升官,自相推達。夫士者貴其用也,不必求備。故四友雖美,能不相兼;三仁齊致,事不一節。高祖佐命,出自亡秦;光武得士,亦資暴莽。況太平之時,而云無士乎!

夫明君之詔也若聲,忠臣之和也如響。長短大小,清濁疾徐,必相應也。且攻玉以石,洗金以鹽,濯錦以魚,浣布以灰。夫物固有以賤理貴,以醜化好者矣。智者棄短取長,以致其功。今使貢士必覈以實,其有小疵,勿彊衣飾,出處默語,各因其方,則蕭、曹、周、韓之倫,何足不致,吳、鄧、梁、竇之屬,企踵可待。孔子曰:「未之思也,夫何遠之有?」

愛日篇曰:

國之所以為國者,以有民也。民之所以為民者,以有穀也。穀之所以豐殖者,以有民功也。功之所以能建者,以日力也。化國之日舒以長,故其民閑暇而力有餘;亂國之日促以短,故其民困務而力不足。舒長者,非謂羲和安行,乃君明民靜而力有餘也。促短者,非謂分度損減,乃上闇下亂,力不足也。孔子稱「既庶則富之,既富乃教之」。是故禮義生於富足,盜竊起於貧窮;富足生於寬暇,貧窮起於無日。聖人深知力者民之本,國之基也,故務省徭役,使之愛日。是以堯敕羲和,欽若昊天,敬授民時。明帝時,公車以反支日不受章奏,帝聞而怪曰:「民廢農桑,遠來詣闕,而復拘以禁忌,豈為政之意乎!」於是遂蠲其制。令冤民仰希申訴,而令長以神自畜,百姓廢農桑而趨府廷者,相續道路,非朝餔不得通,非意氣不得見。或連日累月,更相瞻視;或轉請鄰里,饋糧應對。歲功既虧,天下豈無受其飢者乎?

孔子曰:「聽訟吾猶人也。」從此言之,中才以上,足議曲直,鄉亭部吏,亦有任決斷者,而類多枉曲,蓋有故焉。夫理直則恃正而不橈,事曲則諂意以行賕。不橈故無恩於吏,行賕故見私於法。若事有反覆,吏應坐之,吏以應坐之故,不得不枉之於庭。以羸民之少黨,而與豪吏對訟,其埶得無屈乎?縣承吏言,故與之同。若事有反覆,縣亦應坐之,縣以應坐之故,而排之於郡。以一民之輕,而與一縣為訟,其理豈得申乎?事有反覆,郡亦坐之,郡以共坐之故,而排之於州。以一民之輕,與一郡為訟,其事豈獲勝乎?既不肯理,故乃遠詣公府。公府復不能察,而當延以日月。貧弱者無以曠旬,彊富者可盈千日。理訟若此,何枉之能理乎?正士懷怨結而不見信,猾吏崇姦軌而不被坐,此小民所以易侵苦,而天下所以多困窮也。

且除上天感痛致災,但以人功見事言之。自三府州郡,至于鄉縣典司之吏,辭訟之民,官事相連,更相檢對者,日可有十萬人。一人有事,二人經營,是為日三十萬人廢其業也。以中農率之,則是歲三百萬人受其飢者也。然則盜賊何從而銷,太平何由而作乎?《詩》云:「莫肯念亂,誰無父母?」百姓不足,君誰與足?可無思哉!可無思哉!

述赦篇曰:

凡療病者,必知脈之虛實,氣之所結,然後為之方,故疾可愈而壽可長也。為國者,必先知民之所苦,禍之所起,然後為之禁,故姦可塞而國可安也。今日賊良民之甚者,莫大於數赦贖。赦贖數,則惡人昌而善人傷矣。何以明之哉?夫謹敕之人,身不蹈非,又有為吏正直,不避彊禦,而姦猾之黨橫加誣言者,皆知赦之不久故也。善人君子,被侵怨而能至闕庭自明者,萬無數人;數人之中得省問者,百不過一;既對尚書而空遣去者,復什六七矣。其輕薄姦軌,既陷罪法,怨毒之家冀其辜戮,以解畜憤,而反一概悉蒙赦釋,令惡人高會而誇吒,老盜服臧而過門,孝子見讎而不得討,遭盜者睹物而不敢取,痛莫甚焉!夫養稂莠者傷禾稼,惠姦軌者賊良民。書曰:「文王作罰,刑茲無赦。」先王之制刑法也,非好傷人肌膚,斷人壽命也;貴威姦懲惡,除人害也。故經稱「天命有德,五服五章哉,天討有罪,五刑五用哉」;詩刺「彼宜有罪,汝反脫之」。古者唯始受命之君,承大亂之極,寇賊姦軌,難為法禁,故不得不有一赦,與之更新,頤育萬民,以成大化。非以養姦活罪,放縱天賊也。夫性惡之民,民之豺狼,雖得放宥之澤,終無改悔之心。旦脫重梏,夕還囹圄,嚴明令尹,不能使其斷絕。何也?凡敢為大姦者,才必有過於眾,而能自媚於上者也。多散誕得之財,奉以諂諛之辭,以轉相驅,非有第五公之廉直,孰不為顧哉?論者多曰:「久不赦則姦軌熾而吏不制,宜數肆眚以解散之。」此未昭政亂之本源,不察禍福之所生也。

後度遼將軍皇甫規解官歸安定,鄉人有以貨得鴈門太守者,亦去職還家,書刺謁規。規臥不迎,既入而問:「卿前在郡食鴈美乎?」有頃,又白王符在門。規素聞符名,乃驚遽而起,衣不及帶,屣履出迎,援符手而還,與同坐,極歡。時人為之語曰:「徒見二千石,不如一縫掖。」言書生道義之為貴也。符竟不仕,終於家。

仲長統字公理,山陽高平人也。少好學,博涉書記,贍於文辭。年二十餘,游學青、徐、并、冀之閒,與交友者多異之。并州刺史高幹,袁紹甥也。素貴有名,招致四方遊士,士多歸附。統過幹,幹善待遇,訪以當時之事。統謂幹曰:「君有雄志而無雄才,好士而不能擇人,所以為君深戒也。」幹雅自多,不納其言,統遂去之。無幾,幹以并州叛,卒至於敗。并冀之士皆以是異統。

統性俶儻,敢直言,不矜小節,默語無常,時人或謂之狂生。每州郡命召,輒稱疾不就。常以為凡遊帝王者,欲以立身揚名耳,而名不常存,人生易滅,優遊偃仰,可以自娛,欲卜居清曠,以樂其志,論之曰:「使居有良田廣宅,背山臨流,溝池環匝,竹木周布,場圃築前,果園樹後。舟車足以代步涉之艱,使令足以息四體之役。養親有兼珍之膳,妻孥無苦身之勞。良朋萃止,則陳酒肴以娛之;嘉時吉日,則亨羔豚以奉之。躕躇畦苑,遊戲平林,濯清水,追涼風,釣游鯉,弋高鴻。諷於舞雩之下,詠歸高堂之上。安神閨房,思老氏之玄虛;呼吸精和,求至人之仿佛。與達者數子,論道講書,俯仰二儀,錯綜人物。彈南風之雅操,發清商之妙曲。消搖一世之上,睥睨天地之閒。不受當時之責,永保性命之期。如是,則可以陵霄漢,出宇宙之外矣。豈羨夫入帝王之門哉!」又作詩二篇,以見其志。辭曰:

飛鳥遺跡,蟬蛻亡殼。騰蛇棄鱗,神龍喪角。至人能變,達士拔俗。乘雲無轡,騁風無足。垂露成幃,張霄成幄。沆瀣當餐,九陽代燭。恆星豔珠,朝霞潤玉。六合之內,恣心所欲。人事可遺,何為局促?

大道雖夷,見幾者寡。任意無非,適物無可。古來繞繞,委曲如瑣。百慮何為,至要在我。寄愁天上,埋憂地下。叛散五經,滅棄風、雅。百家雜碎,請用從火。抗志山栖,游心海左。元氣為舟,微風為柂。敖翔太清,縱意容冶。

尚書令荀彧聞統名,奇之,舉為尚書郎。後參丞相曹操軍事。每論說古今及時俗行事,恆發憤歎息。因著論名曰昌言,凡三十四篇,十餘萬言。

獻帝遜位之歲,統卒,時年四十一。友人東海繆襲常稱統才章足繼西京董、賈、劉、楊。今簡撮其書有益政者,略載之云。

理亂篇曰:

豪傑之當天命者,未始有天下之分者也。無天下之分,故戰爭者競起焉。于斯之時,並偽假天威,矯據方國,擁甲兵與我角才智,程勇力與我競雌雄,不知去就,疑誤天下,蓋不可數也。角知者皆窮,角力者皆負,形不堪復伉,埶不足復校,乃始羈首係頸,就我之銜紲耳。夫或曾為我之尊長矣,或曾與我為等儕矣,或曾臣虜我矣,或曾執囚我矣。彼之蔚蔚,皆匈詈腹詛,幸我之不成,而以奮其前志,詎肯用此為終死之分邪?

及繼體之時,民心定矣。普天之下,賴我而得生育,由我而得富貴,安居樂業,長養子孫,天下晏然,皆歸心於我矣。豪傑之心既絕,士民之志已定,貴有常家,尊在一人。當此之時,雖下愚之才居之,猶能使恩同天地,威侔鬼神。暴風疾霆,不足以方其怒;陽春時雨,不足以喻其澤;周、孔數千,無所復角其聖;賁、育百萬,無所復奮其勇矣。

彼後嗣之愚主,見天下莫敢與之違,自謂若天地之不可亡也,乃奔其私嗜,騁其邪欲,君臣宣淫,上下同惡。目極角觝之觀,耳窮鄭衛之聲。入則耽於婦人,出則馳於田獵。荒廢庶政,棄亡人物,澶漫彌流,無所底極。信任親愛者,盡佞諂容說之人也;寵貴隆豐者,盡后妃姬妾之家也。使餓狼守庖廚,飢虎牧牢豚,遂至熬天下之脂膏,斲生人之骨髓。怨毒無聊,禍亂並起,中國擾攘,四夷侵叛,土崩瓦解,一朝而去。昔之為我哺乳之子孫者,今盡是我飲血之寇讎也。至於運徙埶去,猶不覺悟者,豈非富貴生不仁,沈溺致愚疾邪?存亡以之迭代,政亂從此周復,天道常然之大數也。

又政之為理者,取一切而已,非能斟酌賢愚之分,以開盛衰之數也。日不如古,彌以遠甚,豈不然邪?漢興以來,相與同為編戶齊民,而以財力相君長者,世無數焉。而清絜之士,徒自苦於茨棘之閒,無所益損於風俗也。豪人之室,連棟數百,膏田滿野,奴婢千群,徒附萬計。船車賈販,周於四方;廢居積貯,滿於都城。琦賂寶貨,巨室不能容;馬牛羊豕,山谷不能受。妖童美妾,填乎綺室;倡謳妓樂,列乎深堂。賓客待見而不敢去,車騎交錯而不敢進。三牲之肉,臭而不可食;清醇之酎,敗而不可飲。睇盼則人從其目之所視,喜怒則人隨其心之所慮。此皆公侯之廣樂,君長之厚實也。苟能運智詐者,則得之焉;苟能得之者,人不以為罪焉。源發而橫流,路開而四通矣。求士之舍榮樂而居窮苦,棄放逸而赴束縛,夫誰肯為之者邪!夫亂世長而化世短。亂世則小人貴寵,君子困賤。當君子困賤之時,跼高天,蹐厚地,猶恐有鎮厭之禍也。逮至清世,則復入於矯枉過正之檢。老者耄矣,不能及寬饒之俗;少者方壯,將復困於衰亂之時。是使姦人擅無窮之福利,而善士挂不赦之罪辜。苟目能辯色,耳能辯聲,口能辯味,體能辯寒溫者,將皆以脩絜為諱惡,設智巧以避之焉,況肯有安而樂之者邪?斯下世人主一切之愆也。

昔春秋之時,周氏之亂世也。逮乎戰國,則又甚矣。秦政乘并兼之埶,放虎狼之心,屠裂天下,吞食生人,暴虐不已,以招楚漢用兵之苦,甚於戰國之時也。漢二百年而遭王莽之亂,計其殘夷滅亡之數,又復倍乎秦、項矣。以及今日,名都空而不居,百里絕而無民者,不可勝數。此則又甚於亡新之時也。悲夫!不及五百年,大難三起,中閒之亂,尚不數焉。變而彌猜,下而加酷,推此以往,可及於盡矣。嗟乎!不知來世聖人救此之道,將何用也?又不知天若窮此之數,欲何至邪?

損益篇曰:

作有利於時,制有便於物者,可為也。事有乖於數,法有翫於時者,可改也。故行於古有其跡,用於今無其功者,不可不變。變而不如前,易有多所敗者,亦不可不復也。漢之初興,分王子弟,委之以士民之命,假之以殺生之權。於是驕逸自恣,志意無厭。魚肉百姓,以盈其欲;報蒸骨血,以快其情。上有篡叛不軌之姦,下有暴亂殘賊之害。雖藉親屬之恩,蓋源流形埶使之然也。降爵削土,稍稍割奪,卒至於坐食奉祿而已。然其洿穢之行,淫昏之罪,猶尚多焉。故淺其根本,輕其恩義,猶尚假一日之尊,收士民之用。況專之於國,擅之於嗣,豈可鞭笞叱吒,而使唯我所為者乎?時政彫敝,風俗移易,純樸已去,智惠已來。出於禮制之防,放於嗜欲之域久矣,固不可授之以柄,假之以資者也。是故收其奕世之權,校其從橫之埶,善者早登,否者早去,故下土無壅滯之士,國朝無專貴之人。此變之善,可遂行者也。

井田之變,豪人貨殖,館舍布於州郡,田畝連於方國。身無半通青綸之命,而竊三辰龍章之服;不為編戶一伍之長,而有千室名邑之役。榮樂過於封君,埶力侔於守令。財賂自營,犯法不坐。刺客死士,為之投命。至使弱力少智之子,被穿帷敗,寄死不斂,冤枉窮困,不敢自理。雖亦由網禁疏闊,蓋分田無限使之然也。今欲張太平之紀綱,立至化之基趾,齊民財之豐寡,正風俗之奢儉,非井田實莫由也。此變有所敗,而宜復者也。

肉刑之廢,輕重無品,下死則得髡鉗,下髡鉗則得鞭笞。死者不可復生,而髡者無傷於人。髡笞不足以懲中罪,安得不至於死哉!夫雞狗之攘竊,男女之淫奔,酒醴之賂遺,謬誤之傷害,皆非值於死者也。殺之則甚重,髡之則甚輕。不制中刑以稱其罪,則法令安得不參差,殺生安得不過謬乎?今患刑輕之不足以懲惡,則假臧貨以成罪,託疾病以諱殺。科條無所準,名實不相應,恐非帝王之通法,聖人之良制也。或曰:過刑惡人,可也;過刑善人,豈可復哉?曰:若前政以來,未曾枉害善人者,則有罪不死也,是為忍於殺人也,而不忍於刑人也。今令五刑有品,輕重有數,科條有序,名實有正,非殺人逆亂鳥獸之行甚重者,皆勿殺。嗣周氏之祕典,續呂侯之祥刑,此又宜復之善者也。

《易》曰:「陽一君二臣,君子之道也;陰二君一臣,小人之道也。」然則寡者,為人上者也;眾者,為人下者也。一伍之長,才足以長一伍者也;一國之君,才足以君一國者也;天下之王,才足以王天下者也。愚役於智,猶枝之附幹,此理天下之常法也。制國以分人,立政以分事,人遠則難綏,事總則難了。今遠州之縣,或相去數百千里,雖多山陵洿澤,猶有可居人種穀者焉。當更制其境界,使遠者不過二百里。明版籍以相數閱,審什伍以相連持,限夫田以斷并兼,定五刑以救死亡,益君長以興政理,急農桑以豐委積,去末作以一本業,敦教學以移情性,表德行以厲風俗,覈才蓺以敘官宜,簡精悍以習師田,修武器以存守戰,嚴禁令以防僭差,信實罰以驗懲勸,糾游戲以杜姦邪,察苛刻以絕煩暴。審此十六者以為政務,操之有常,課之有限,安寧勿懈墯,有事不迫遽,聖人復起,不能易也。

向者,天下戶過千萬,除其老弱,但戶一丁壯,則千萬人也。遺漏既多,又蠻夷戎狄居漢地者尚不在焉。丁壯十人之中,必有堪為其什伍之長,推什長已上,則百萬人也。又十取之,則佐史之才已上十萬人也。又十取之,則可使在政理之位者萬人也。以筋力用者謂之人,人求丁壯;以才智用者謂之士,士貴耆老。充此制以用天下之人,猶將有儲,何嫌乎不足也?故物有不求,未有無物之歲也;士有不用,未有少士之世也。夫如此,然後可以用天性,究人理,興頓廢,屬斷絕,網羅遺漏,拱柙天人矣。

或曰:善為政者,欲除煩去苛,并官省職,為之以無為,事之以無事,何子言之云云也?曰:若是,三代不足摹,聖人未可師也。君子用法制而至於化,小人用法制而至於亂。均是一法制也,或以之化,或以之亂,行之不同也。苟使豺狼牧羊豚,盜跖主征稅,國家昏亂,吏人放肆,則惡復論損益之閒哉!夫人待君子然後化理,國待蓄積乃無憂患。君子非自農桑以求衣食者也,蓄積非橫賦斂以取優饒者也。奉祿誠厚,則割剝貿易之罪乃可絕也;蓄積誠多,則兵寇水旱之災不足苦也。故由其道而得之,民不以為奢;由其道而取之,民不以為勞。天災流行,開倉庫以稟貸,不亦仁乎?衣食有餘,損靡麗以散施,不亦義乎?彼君子居位為士民之長,固宜重肉累帛,朱輪四馬。今反謂薄屋者為高,藿食者為清,既失天地之性,又開虛偽之名,使小智居大位,庶績不咸熙,未必不由此也。得拘絜而失才能,非立功之實也。以廉舉而以貪去,非士君子之志也。夫選用必取善士。善士富者少而貧者多,祿不足以供養,安能不少營私門乎?從而罪之,是設機置阱以待天下之君子也。

盜賊凶荒,九州代作,飢饉暴至,軍旅卒發,橫稅弱人,割奪吏祿,所恃者寡,所取者猥,萬里懸乏,首尾不救,徭役並起,農桑失業,兆民呼嗟於昊天,貧窮轉死於溝壑矣。今通肥饒之率,計稼穡之入,令畝收三斛,斛取一斗,未為甚多。一歲之閒,則有數年之儲,雖興非法之役,恣奢侈之欲,廣愛幸之賜,猶未能盡也。不循古法,規為輕稅,及至一方有警,一面被災,未逮三年,校計騫短,坐視戰士之蔬食,立望餓殍之滿道,如之何為君行此政也?二十稅一,名之曰貊,況三十稅一乎?夫薄吏祿以豐軍用,緣於秦征諸侯,續以四夷,漢承其業,遂不改更,危國亂家,此之由也。今田無常主,民無常居,吏食日稟,祿班未定。可為法制,畫一定科,租稅十一,更賦如舊。今者土廣民稀,中地未墾;雖然,猶當限以大家,勿令過制。其地有草者,盡曰官田,力堪農事,乃聽受之。若聽其自取,後必為姦也。

法誡篇曰:

周禮六典,冢宰貳王而理天下。春秋之時,諸侯明德者,皆一卿為政。爰及戰國,亦皆然也。秦兼天下,則置丞相,而貳之以御史大夫。自高帝逮于孝成,因而不改,多終其身。漢之隆盛,是惟在焉。夫任一人則政專,任數人則相倚。政專則和諧,相倚則違戾。和諧則太平之所興也,違戾則荒亂之所起也。光武皇帝慍數世之失權,忿彊臣之竊命,矯枉過直,政不任下,雖置三公,事歸臺閣。自此以來,三公之職,備員而已,然政有不理,猶加譴責。而權移外戚之家,寵被近習之豎,親其黨類,用其私人,內充京師,外布列郡,顛倒賢愚,貿易選舉,疲駑守境,貪殘牧民,撓擾百姓,忿怒四夷,招致乖叛,亂離斯瘼。怨氣並作,陰陽失和,三光虧缺,怪異數至,蟲螟食稼,水旱為災,此皆戚宦之臣所致然也。反以策讓三公,至於死免,乃足為叫呼蒼天,號咷泣血者也。又中世之選三公也,務於清愨謹慎,循常習故者。是婦女之檢柙,鄉曲之常人耳,惡足以居斯位邪?埶既如彼,選又如此,而欲望三公勳立於國家,績加於生民,不亦遠乎?昔文帝之於鄧通,可謂至愛,而猶展申徒嘉之志。夫見任如此,則何患於左右小臣哉?至如近世,外戚臣豎請託不行,意氣不滿,立能陷人於不測之禍,惡可得彈正者哉!曩者任之重而責之輕,今者任之輕而責之重。昔賈誼感絳侯之困辱,因陳大臣廉恥之分,開引自裁之端。自此以來,遂以成俗。繼世之主,生而見之,習其所常,曾莫之悟。嗚呼,可悲夫!左手據天下之圖,右手刎其喉,愚者猶知難之,況明哲君子哉!光武奪三公之重,至今而加甚,不假后黨以權,數世而不行,蓋親疏之埶異也。母后之黨,左右之人,有此至親之埶,故其貴任萬世。常然之敗,無世而無之,莫之斯鑒,亦可痛矣。未若置丞相自總之。若委三公,則宜分任責成。夫使為政者,不當與之婚姻;婚姻者,不當使之為政也。如此,在位病人,舉用失賢,百姓不安,爭訟不息,天地多變,人物多妖,然後可以分此罪矣。

或曰:政在一人,權甚重也。曰:人實難得,何重之嫌?昔者霍禹、竇憲、鄧騭、梁冀之徒,籍外戚之權,管國家之柄;及其伏誅,以一言之詔,詰朝而決,何重之畏乎?今夫國家漏神明於媟近,輸權重於婦黨,筭十世而為之者八九焉。不此之罪而彼之疑,何其詭邪!

論曰:百家之言政者尚矣。大略歸乎寧固根柢,革易時敝也。夫遭運無恆,意見偏雜,故是非之論,紛然相乖。嘗試妄論之,以為世非胥、庭,人乖鷇飲,化跡萬肇,情故萌生。雖周物之智,不能研其推變;山川之奧,未足況其紆險。則應俗適事,難以常條。如使用審其道,則殊塗同會;才爽其分,則一豪以乖。何以言之?若夫玄聖御世,則天同極,施舍之道,宜無殊典。而損益異運,文朴遞行。用明居晦,回泬於曩時;興戈陳俎,參差於上世。及至戴黃屋,服絺衣,豐薄不齊,而致化則一;亦有宥公族,黥國儲,寬慘巨隔,而防非必同。此其分波而共源,百慮而一致者也。若乃偏情矯用,則枉直必過。故葛屨履霜,敝由崇儉;楚楚衣服,戒在窮賒;疏禁厚下,以尾大陵弱;斂威峻罰,以苛薄分崩。斯曹、魏之刺,所以明乎國風;周、秦末軌,所以彰於微滅。故用舍之端,興敗資焉。是以繁簡唯時,寬猛相濟。刑書鐫鼎,事有可詳;三章在令,取貴能約。太叔致猛政之褒,國子流遺愛之涕,宣孟改冬日之和,平陽循畫一之法。斯實弛張之弘致,可以徵其統乎!數子之言當世失得皆究矣,然多謬通方之訓,好申一隅之說。貴清靜者,以席上為腐議;束名實者,以柱下為誕辭。或推前王之風,可行於當年,有引救敝之規,宜流於長世。稽之篤論,將為敝矣。如以舟無推陸之分,瑟非常調之音,不陽局以疑遠,不拘玄以妨素,則化樞各管其極,理略可得而言與?

贊曰:管視好偏,群言難一。救朴雖文,矯遲必疾。舉端自理,滯隅則失。詳觀時蠹,成昭政術。


\end{pinyinscope}