\article{方術列傳上}

\begin{pinyinscope}
仲尼稱易有君子之道四焉,曰「卜筮者尚其占」。占也者,先王所以定禍福,決嫌疑,幽贊於神明,遂知來物者也。若夫陰陽推步之學,往往見於墳記矣。然神經怪牒,玉策金繩,關扃於明靈之府,封縢於瑤壇之上者,靡得而闚也。至乃河洛之文,龜龍之圖,箕子之術,師曠之書,緯候之部,鈐決之符,皆所以探抽冥賾,參驗人區,時有可聞者焉。其流又有風角、遁甲、七政、元氣、六日七分、逢占、日者、挺專、須臾、孤虛之術,及望雲省氣,推處祥妖,時亦有以效於事也。而斯道隱遠,玄奧難原,故聖人不語怪神,罕言性命。或開末而抑其端,或曲辭以章其義,所謂「民可使由之,不可使知之」。

漢自武帝頗好方術,天下懷協道蓺之士,莫不負策抵掌,順風而屆焉。後王莽矯用符命,及光武尤信讖言,士之赴趣時宜者,皆騁馳穿鑿,爭談之也。故王梁、孫咸名應圖籙,越登槐鼎之任,鄭興、賈逵以附同稱顯,桓譚、尹敏以乖忤淪敗,自是習為內學,尚奇文,貴異數,不乏於時矣。是以通儒碩生,忿其姦妄不經,奏議慷慨,以為宜見藏擯。子長亦云:「觀陰陽之書,使人拘而多忌。」蓋為此也。

夫物之所偏,未能無蔽,雖云大道,其荬或同。若乃詩之失愚,書之失誣,然則數術之失,至於詭俗乎?如令溫柔敦厚而不愚,斯深於詩者也;疏通知遠而不誣,斯深於書者也;極數知變而不詭俗,斯深於數術者也。故曰:「苟非其人,道不虛行。」意者多迷其統,取遣頗偏,甚有雖流宕過誕亦失也。

中世張衡為陰陽之宗,郎顗咎徵最密,餘亦班班名家焉。其徒亦有雅才偉德,未必體極蓺能。今蓋糾其推變尤長,可以弘補時事,因合表之云。

任文公,巴郡閬中人也。父文孫,明曉天官風角祕要。文公少修父術,州辟從事。哀帝時,有言越巂太守欲反,刺史大懼,遣文公等五從事檢行郡界,潛伺虛實。共止傳舍,時暴風卒至,文公遽趣白諸從事促去,當有逆變來害人者,因起駕速驅。諸從事未能自發,郡果使兵殺之,文公獨得免。

後為治中從事。時天大旱,白刺史曰:「五月一日,當有大水,其變已至,不可防救,宜令吏人豫為其備。」刺史不聽,文公獨儲大船,百姓或聞,頗有為防者。到其日旱烈,文公急命促載,使白刺史,刺史笑之。日將中,天北雲起,須臾大雨,至晡時,湔水涌起十餘丈,突壞廬舍,所害數千人。文公遂以占術馳名。辟司空掾。平帝即位,稱疾歸家。

王莽篡後,文公推數,知當大亂,乃課家人負物百斤,環舍趨走,日數十,時人莫知其故。後兵寇並起,其逃亡者少能自脫,惟文公大小負糧捷步,悉得完免。遂奔子公山,十餘年不被兵革。

公孫述時,蜀武擔石折。文公曰:「噫!西州智士死,我乃當之。」自是常會聚子孫,設酒食。後三月果卒。故益部為之語曰:「任文公,智無雙。」

郭憲字子橫,汝南宋人也。少師事東海王仲子。時王莽為大司馬,召仲子,仲子欲往。憲諫曰:「禮有來學,無有往教之義。今君賤道畏貴,竊所不取。」仲子曰:「王公至重,不敢違之。」憲曰:「今正臨講業,且當訖事。」仲子從之,日晏乃往。莽問:「君來何遲?」仲子具以憲言對,莽陰奇之。及後篡位,拜憲郎中,賜以衣服。憲受衣焚之,逃于東海之濱。莽深忿恚,討逐不知所在。

光武即位,求天下有道之人,乃徵憲拜博士。再遷,建武七年,代張堪為光祿勳。從駕南郊。憲在位,忽回向東北,含酒三潠。執法奏為不敬。詔問其故。憲對曰:「齊國失火,故以此厭之。」後齊果上火災,與郊同日。

八年,車駕西征隗囂,憲諫曰:「天下初定,車駕未可以動。」憲乃當車拔佩刀以斷車靷。帝不從,遂上隴。其後潁川兵起,乃回駕而還。帝歎曰:「恨不用子橫之言。」

時匈奴數犯塞,帝患之,乃召百僚廷議。憲以為天下疲敝,不宜動眾。諫爭不合,乃伏地稱眩瞀,不復言。帝令兩郎扶下殿,憲亦不拜。帝曰:「常聞『關東觥觥郭子橫』,竟不虛也。」憲遂以病辭退,卒於家。

許楊字偉君,汝南平輿人也。少好術數。王莽輔政,召為郎,稍遷酒泉都尉。及莽篡位,楊乃變姓名為巫醫,逃匿它界。莽敗,方還鄉里。

汝南舊有鴻郤陂,成帝時,丞相翟方進奏毀敗之。建武中,太守鄧晨欲修復其功,聞楊曉水脈,召與議之。楊曰:「昔成帝用方進之言,尋而自夢上天,天帝怒曰:『何故敗我濯龍淵?』是後民失其利,多致飢困。時有謠歌曰:『敗我陂者翟子威,飴我大豆,亨我芋魁。反乎覆,陂當復。』昔大禹決江疏河以利天下,明府今興立廢業,富國安民,童謠之言,將有徵於此。誠願以死效力。」晨大悅,因署楊為都水掾,使典其事。楊因高下形埶,起塘四百餘里,數年乃立。百姓得其便,累歲大稔。

初,豪右大姓因緣陂役,競欲辜較在所,楊一無聽,遂共譖楊受取賕賂。晨遂收楊下獄,而械輒自解。獄吏恐,遽白晨。晨驚曰:「果濫矣。太守聞忠信可以感靈,今其效乎!」即夜出楊,遣歸。時天大陰晦,道中若有火光照之,時人異焉。後以病卒。晨於都官為楊起廟,圖畫形像,百姓思其功績,皆祭祀之。

高獲字敬公,汝南新息人也。為人尼首方面。少遊學京師,與光武有舊。師事司徒歐陽歙。歙下獄當斷,獲冠鐵冠,帶鈇鑕,詣闕請歙。帝雖不赦,而引見之。謂曰:「敬公,朕欲用子為吏,宜改常性。」獲對曰:「臣受性於父母,不可改之於陛下。」出便辭去。

三公爭辟不應。後太守鮑昱請獲,既至門,令主簿就迎,主簿曰但使騎吏迎之,獲聞之,即去。昱遣追請獲,獲顧曰:「府君但為主簿所欺,不足與談。」遂不留。時郡境大旱。獲素善天文,曉遁甲,能役使鬼神。昱自往問何以致雨,獲曰:「急罷三部督郵,明府當自北出,到三十里亭,雨可致也。」昱從之,果得大雨。每行縣,輒軾其閭。獲遂遠遁江南,卒於石城。石城人思之,共為立祠。

王喬者,河東人也。顯宗世,為葉令。喬有神術,每月朔望,常自縣詣臺朝。帝怪其來數,而不見車騎,密令太史伺望之。言其臨至,輒有雙鳧從東南飛來。於是候鳧至,舉羅張之,但得一隻舄焉。乃詔尚方邺視,則四年中所賜尚書官屬履也。每當朝時,葉門下鼓不擊自鳴,聞於京師。後天下玉棺於堂前,吏人推排,終不搖動。喬曰:「天帝獨召我邪?」乃沐浴服飾寢其中,蓋便立覆。宿昔葬於城東,土自成墳。其夕,縣中牛皆流汗喘乏,而人無知者。百姓乃為立廟,號葉君祠。牧守每班錄,皆先謁拜之。吏人祈禱,無不如應。若有違犯,亦立能為祟。帝乃迎取其鼓,置都亭下,略無復聲焉。或云此即古仙人王子喬也。

謝夷吾字堯卿,會稽山陰人也。少為郡吏,學風角占候。太守第五倫擢為督郵。時烏程長有臧釁,倫使收案其罪。夷吾到縣,無所驗,但望閤伏哭而還。一縣驚怪,不知所為。及還,白倫曰:「竊以占候,知長當死。近三十日,遠不過六十日,遊魂假息,非刑所加,故不收之。」倫聽其言,至月餘,果有驛馬齎長印綬,上言暴卒。倫以此益禮信之。

舉孝廉,為壽張令,稍遷荊州刺史,遷鉅鹿太守。所在愛育人物,有善績。及倫作司徒,令班固為文薦夷吾曰:「臣聞堯登稷、契,政隆太平;舜用皋陶,政致雍熙。殷、周雖有高宗、昌、發之君,猶賴傅說、呂望之策,故能克崇其業,允協大中。竊見鉅鹿太守會稽謝夷吾,出自東州,厥土塗泥,而英姿挺特,奇偉秀出。才兼四科,行包九德,仁足濟時,知周萬物。加以少膺儒雅,韜含六籍,推考星度,綜校圖錄,探賾聖祕,觀變歷徵,占天知地,與神合契,據其道德,以經王務。昔為陪隸,與臣從事,奮忠毅之操,躬史魚之節,董臣嚴綱,勖臣懦弱,得以免戾,寔賴厥勳。及其應選作宰,惠敷百里,降福彌異,流化若神,爰牧荊州,威行邦國。奉法作政,有周、召之風;居儉履約,紹公儀之操。尋功簡能,為外臺之表;聽聲察實,為九伯之冠。遷守鉅鹿,政合時雍。德量績謀,有伊、呂、管、晏之任;闡弘道奧,同史蘇、京房之倫。雖密勿在公,而身出心隱,不殉名以求譽,不馳騖以要寵,念存遜遁,演志箕山。方之古賢,實有倫序;採之於今,超焉絕俗。誠社稷之元龜,大漢之棟甍。宜當拔擢,使登鼎司,上令三辰順軌於歷象,下使五品咸訓于嘉時,必致休徵克昌之慶,非徒循法奉職而已。臣以頑駑,器非其疇,尸祿負乘,夕惕若厲。願乞骸骨,更授夷吾,上以光七曜之明,下以厭率土之望,庶令微臣塞咎免悔。」

後以行春乘柴車,從兩吏,冀州刺史上其儀序失中,有損國典,左轉下邳令。豫剋死日,如期果卒。敕其子曰:「漢末當亂,必有發掘露骸之禍。」使懸棺下葬,墓不起墳。

時博士勃海郭鳳亦好圖讖,善說災異,吉凶占應。先自知死期,豫令弟子市棺斂具,至其日而終。

楊由字哀侯,蜀郡成都人也。少習易,并七政、元氣、風雲占候。為郡文學掾。時有大雀夜集於庫樓上,太守廉范以問由。由對曰:「此占郡內當有小兵,然不為害。」後二十餘日,廣柔縣蠻夷反,殺傷長吏,郡發庫兵擊之。又有風吹削哺,太守以問由。由對曰:「方當有薦木實者,其色黃赤。」頃之,五官掾獻橘數包。

由嘗從人飲,敕御者曰:「酒若三行,便宜嚴駕。」既而趣去。後主人舍有鬥相殺者,人請問何以知之。由曰:「向社中木上有鳩鬥,此兵賊之象也。」其言多驗。著書十餘篇,名曰其平。終于家。

李南字孝山,丹陽句容人也。少篤學,明於風角。和帝永元中,太守馬棱坐盜賊事被徵,當詣廷尉,吏民不寧,南特通謁賀。棱意有恨,謂曰:「太守不德,今當即罪,而君反相賀邪?」南曰:「旦有善風,明日中時應有吉問,故來稱慶。」旦日,棱延望景晏,以為無徵;至晡,乃有驛使齎詔書原停棱事。南問其遲留之狀。使者曰:「向度宛陵浦里斻,馬踠足,是以不得速。」棱乃服焉。後舉有道,辟公府,病不行,終於家。

南女亦曉家術,為由拳縣人妻。晨詣爨室,卒有暴風,婦便上堂從姑求歸,辭其二親。姑不許,乃跪而泣曰:「家世傳術,疾風卒起,先吹灶突及井,此禍為婦女主爨者,妾將亡之應。」因著其亡日。乃聽還家,如期病卒。

李郃字孟節,漢中南鄭人也。父頡,以儒學稱,官至博士。郃襲父業,遊太學,通五經。善河洛風星,外質朴,人莫之識。縣召署幕門候吏。

和帝即位,分遣使者,皆微服單行,各至州縣,觀採風謠。使者二人當到益部,投郃候舍。時夏夕露坐,郃因仰觀,問曰:「二君發京師時,寧知朝廷遣二使邪?」二人默然,驚相視曰:「不聞也。」問何以知之。郃指星示云:「有二使星向益州分野,故知之耳。」

後三年,其使者一人拜漢中太守,郃猶為吏,太守奇其隱德,召署戶曹史。時大將軍竇憲納妻,天下郡國皆有禮慶,郡亦遣使。郃進諫曰:「竇將軍椒房之親,不修禮德,而專權驕恣,危亡之禍可翹足而待,願明府一心王室,勿與交通。」太守固遣之,郃不能止,請求自行,許之。郃遂所在留遲,以觀其變。行至扶風,而憲就國自殺,支黨悉伏其誅,凡交通憲者,皆為免官,唯漢中太守不豫焉。

郃歲中舉孝廉,五遷尚書令,又拜太常。元初四年,代袁敞為司空,數陳得失,有忠臣節。在位四年,坐請託事免。

安帝崩,北鄉侯立,復為司徒。及北鄉侯病,郃陰與少府河南陶範、步兵校尉趙直謀立順帝,會孫程等事先成,故郃功不顯。明年,坐吏民疾病,仍有災異,賜策免。將作大匠翟酺上郃「潛圖大計,以安社稷」,於是錄陰謀之功,封郃涉都侯,辭讓不受。年八十餘,卒於家。門人上黨馮冑獨制服,心喪三年,時人異之。

冑字世威,奉世之後也。常慕周伯況、閔仲叔之為人,隱處山澤,不應徵辟。

郃子固,已見前傳。弟子歷,字季子。清白有節,博學善交,與鄭玄、陳紀等相結。為新城長,政貴無為。亦好方術。時天下旱,縣界特雨。官至奉車都尉。

段翳字元章,廣漢新都人也。習易經,明風角。時有就其學者,雖未至,必豫知其姓名。嘗告守津吏曰:「某日當有諸生二人,荷擔問翳舍處者,幸為告之。」後竟如其言。又有一生來學,積年,自謂略究要術,辭歸鄉里。翳為合膏藥,并以簡書封於筒中,告生曰:「有急發視之。」生到葭萌,與吏爭度,津吏檛破從者頭。生開筒得書,言到葭萌,與吏鬥頭破者,以此膏裹之。生用其言,創者即愈。生歎服,乃還卒業。翳遂隱居竄跡,終于家。

廖扶字文起,汝南平輿人也。習韓詩、歐陽尚書,教授常數百人。父為北地太守,永初中,坐羌沒郡下獄死。扶感父以法喪身,憚為吏。及服終而歎曰:「老子有言:『名與身孰親?』吾豈為名乎!」遂絕志世外。專精經典,尤明天文、讖緯,風角、推步之術。州郡公府辟召皆不應。就問災異,亦無所對。

扶逆知歲荒,乃聚穀數千斛,悉用給宗族姻親,又斂葬遭疫死亡不能自收者。常居先人冢側,未曾入城市。太守謁煥,先為諸生,從扶學,後臨郡,未到,先遣吏脩門人之禮,又欲擢扶子弟,固不肯,當時人因號為北郭先生。年八十,終于家。

二子,孟舉、偉舉,並知名。

折像字伯式,廣漢雒人也。其先張江者,封折侯,曾孫國為鬱林太守,徙廣漢,因封氏焉。國生像。

國有貲財二億,家僮八百人。像幼有仁心,不殺昆蟲,不折萌牙。能通京氏易,好黃老言。及國卒,感多藏厚亡之義,乃散金帛資產,周施親疏。或諫像曰:「君三男兩女,孫息盈前,當增益產業,何為坐自殫竭乎?」像曰:「昔鬥子文有言:『我乃逃禍,非避富也。』吾門戶殖財日久,盈滿之咎,道家所忌。今世將衰,子又不才。不仁而富,謂之不幸。牆隙而高,其崩必疾也。」智者聞之咸服焉。

自知亡日,召賓客九族飲食辭訣,忽然而終。時年八十四。家無餘資,諸子衰劣如其言云。

樊英字季齊,南陽魯陽人也。少受業三輔,習京氏易,兼明五經。又善風角、星筭、河洛七緯,推步災異。隱於壺山之陽,受業者四方而至。州郡前後禮請不應;公卿舉賢良方正、有道,皆不行。

嘗有暴風從西方起,英謂學者曰:「成都市火甚盛。」因含水西向漱之,乃令記其日時。客後有從蜀都來,云「是日大火,有黑雲卒從東起,須臾大雨,火遂得滅」。於是天下稱其術蓺。

安帝初,徵為博士。至建光元年,復詔公車賜策書,徵英及同郡孔喬、李昺、北海郎宗、陳留楊倫、東平王輔六人,唯郎宗、楊倫到洛陽,英等四人並不至。

永建二年,順帝策書備禮,玄纁徵之,復固辭疾篤。乃詔切責郡縣,駕載上道。英不得已,到京,稱病不肯起。乃強輿入殿,猶不以禮屈。帝怒,謂英曰:「朕能生君,能殺君;能貴君,能賤君;能富君,能貧君。君何以慢朕命?」英曰:「臣受命於天。生盡其命,天也;死不得其命,亦天也。陛下焉能生臣,焉能殺臣!臣見暴君如見仇讎,立其朝猶不肯,可得而貴乎?雖在布衣之列,環堵之中,晏然自得,不易萬乘之尊,又可得而賤乎?陛下焉能貴臣,焉能賤臣!臣非禮之祿,雖萬鍾不受;若申其志,雖簞食不厭也。陛下焉能富臣,焉能貧臣!」帝不能屈,而敬其名,使出就太醫養疾,月致羊酒。

至四年三月,天子乃為英設壇席,令公車令導,尚書奉引,賜几杖,待以師傅之禮,延問得失。英不敢辭,拜五官中郎將。數月,英稱疾篤,詔以為光祿大夫,賜告歸。令在所送穀千斛,常以八月致牛一頭,酒三斛;如有不幸,祠以中牢。英辭位不受,有詔譬旨勿聽。

英初被詔命,僉以為必不降志,及後應對,又無奇謨深策,談者以為失望。初,河南張楷與英俱徵,既而謂英曰:「天下有二道,出與處也。吾前以子之出,能輔是君也,濟斯人也。而子始以不訾之身,怒萬乘之主;及其享受爵祿,又不聞匡救之術,進退無所據矣。」

英既善術,朝廷每有災異,詔輒下問變復之效,所言多驗。

初,英著易章句,世名樊氏學,以圖緯教授。潁川陳寔少從英學。嘗有疾,妻遣婢拜問,英下床荅拜。寔怪而問之。英曰:「妻,齊也,共奉祭祀,禮無不荅。」其恭謹若是。年七十餘,卒於家。

孫陵,靈帝時以諂事宦人為司徒。

陳郡郤巡學傳英業,官至侍中。

論曰:漢世之所謂名士者,其風流可知矣。雖弛張趣舍,時有未純,於刻情修容,依倚道蓺,以就其聲價,非所能通物方,弘時務也。及徵樊英、楊厚,朝廷若待神明,至竟無它異。英名最高,毀最甚。李固、朱穆等以為處士純盜虛名,無益於用,故其所以然也。然而後進希之以成名,世主禮之以得眾,原其無用亦所以為用,則其有用或歸於無用矣。何以言之?夫煥乎文章,時或乖用;本乎禮樂,適末或疏。及其陶搢紳,藻心性,使由之而不知者,豈非道邈用表,乖之數跡乎?而或者忽不踐之地,賒無用之功,至乃誚譟遠術,賤斥國華,以為力詐可以救淪敝,文律足以致寧平,智盡於猜察,道足於法令,雖濟萬世,其將與夷狄同也。孟軻有言曰:「以夏變夷,不聞變夷於夏。」況有未濟者乎!


\end{pinyinscope}