\article{天文下}

\begin{pinyinscope}
孝桓建和元年八月壬寅,熒惑犯輿鬼質星。二年二月辛卯,熒惑行在輿鬼中。三年五月己丑,太白行入太微右掖門,留十五日,出端門。丙申,熒惑入東井。八月己亥,鎮星犯輿鬼中南星。乙丑,彗星芒長五尺,見天巿中,東南指,色黃白,九月戊辰不見。熒惑犯輿鬼為死喪,質星為戮臣,入太微為亂臣。鎮星犯輿鬼為喪。彗星見天巿中為質貴人。至和平元年十二月甲寅,梁太后崩,梁冀益驕亂矣。

元嘉元年二月戊子,太白晝見。永興二年閏月丁酉,太白晝見。時上幸後宮采女鄧猛,明年,封猛兄演為南頓侯。後四歲,梁皇后崩,梁冀被誅,猛立為皇后,恩寵甚盛。

永壽元年三月丙申,鎮星逆行入太微中,七十四日去左掖門。七月己未,辰星入太微中,八十日去左掖門。八月己巳,熒惑入太微,二十一日出端門。太微,天子廷也。鎮星為貴臣妃后,逆行為匿謀。辰星入太微為大水,一曰後宮有憂。是歲雒水溢至津門,南陽大水。熒惑留入太微中,又為亂臣。是時梁氏專政。九月己酉,晝有流星長二尺所,色黃白。癸巳,熒惑犯歲星,為姦臣謀,大將戮。

二年六月甲寅,辰星入太微,遂伏不見。辰星為水,為兵,為妃后。八月戊午,太白犯軒轅大星,為皇后。其三年四月戊寅,熒惑入東井口中,為大臣有誅者。其七月丁丑,太白犯心前星,為大臣。後二年四月,懿獻皇后以憂死。大將軍梁冀使太倉令秦宮刺殺議郎邴尊,又欲殺鄧后母宣,事覺,桓帝收冀及妻壽襄城君印綬,皆自殺。誅諸梁及孫氏宗族,或徙邊。是其應也。

延熹四年三月甲寅,熒惑犯輿鬼質星。五月辛酉,客星在營室,稍順行,生芒長五尺所,至心一度,轉為彗。熒惑犯輿鬼質星,大臣有戮死者。五年十月,南郡太守李肅坐蠻夷賊攻盜郡縣,取財一億以上,入府取銅虎符,肅背敵走,不救城郭;又監黎陽謁者燕喬坐贓,重泉令彭良殺無辜,皆棄巿。京兆虎牙都尉宋謙坐贓,下獄死。客星在營室至心作彗,為大喪。後四年,鄧后以憂死。

六年十一月丁亥,太白晝見。是時鄧后家貴盛。

七年七月戊辰,辰星犯歲星。八月庚戌,熒惑犯輿鬼質星。庚申,歲星犯軒轅大星。十月丙辰,太白犯房北星。丁卯,辰星犯太白。十二月乙丑,熒惑犯軒轅第二星。辰星犯歲星為兵。熒惑犯質星有戮臣。歲星犯軒轅為女主憂。太白犯房北星為後宮。其八年二月,太僕南鄉侯左勝以罪賜死,勝弟中常侍上蔡侯悺、北鄉侯黨皆自殺。癸亥,皇后鄧氏坐執左道廢,遷于祠宮死,宗親侍中沘陽侯鄧康、河南尹鄧萬、越騎校尉鄧弼、虎賁中郎將安鄉侯鄧魯、侍中監羽林左騎鄧德、右騎鄧壽、昆陽侯鄧統、淯陽侯鄧秉、議郎鄧循皆繫暴室,萬、魯死,康等免官。又荊州刺史芝、交阯刺史葛祗皆為賊所拘略,桂陽太守任胤背敵走,皆棄巿,熒惑犯輿鬼質星之應也。

八年五月癸酉,太白犯輿鬼質星。壬午,熒惑入太微右執法。閏月己未,太白犯心前星。十月癸酉,歲星犯左執法。十一月戊午,歲星入太微,犯左執法。九年正月壬辰,歲星入太微中,五十八日出端門。六月壬戌,太白行入輿鬼。七月乙未,熒惑行輿鬼中,犯質星。九月辛亥,熒惑入太微西門,積五十八日。永康元年正月庚寅,熒惑逆行入太微東門,留太微中,百一日出端門。七月丙戌,太白晝見經天。太白犯心前星,太白犯輿鬼質星有戮臣。熒惑入太微為賊臣。太白犯心前星為兵喪。歲星入太微犯左執法,將相有誅者。歲星入守太微五十日,占為人主。太白、熒惑入輿鬼,皆為死喪,又犯質星為戮臣。熒惑留太微中百一日,占為人主。太白晝見經天為兵,憂在大人。其九年十一月,太原太守劉暧、南陽太守成档皆坐殺無辜,荊州刺史李隗為賊所拘,尚書郎孟璫坐受金漏言,皆棄巿。永康元年十二月丁丑,桓帝崩,太傅陳蕃,大將軍竇武、尚書令尹勳、黃門令山冰等皆枉死,太白犯心,熒惑留守太微之應也。

孝靈帝建寧元年六月,太白在西方,入太微,犯西蕃南頭星。太微,天廷也。太白行其中,宮門當閉,大將被甲兵,大臣伏誅。其八月,太傅陳蕃、大將軍竇武謀欲盡誅諸宦者;其九月辛亥,中常侍曹節、長樂五官史朱瑀覺之,矯制殺蕃、武等,家屬徙日南比景。

熹平元年十月,熒惑入南斗中。占曰:「熒惑所守為兵亂。」斗為吳。其十一月,會稽賊許昭聚眾自稱大將軍,昭父生為越王,攻破郡縣。

二年四月,有星出文昌,入紫宮,蛇行,有首尾無身,赤色,有光炤垣牆。八月丙寅,太白犯心前星。辛未,白氣如一匹練,衝北斗第四星。占曰:「文昌為上將貴相。太白犯心前星,為大臣。」後六年,司徒劉群為中常侍曹節所譖,下獄死。白氣衝北斗為大戰。明年冬,揚州刺史臧旻、丹陽太守陳寅,攻盜賊苴康,斬首數千級。

光和元年四月癸丑,流星犯軒轅第二星,東北行入北斗魁中。八月,彗星出亢北,入天巿中,長數尺,稍長至五六丈,赤色,經歷十餘宿,八十餘日,乃消於天菀中。流星為貴使,軒轅為內宮,北斗魁主殺。流星從軒轅出抵北斗魁,是天子大使將出,有伐殺也。至中平元年,黃巾賊起,上遣中郎將皇甫嵩、朱雋等征之,斬首十餘萬級。彗除天巿,天帝將徙,帝將易都。至初平元年,獻帝遷都長安。

三年冬,彗星出狼、弧,東行至于張乃去。張為周地,彗星犯之為兵亂。後四年,京都大發兵擊黃巾賊。

五年四月,熒惑在太微中,守屏。七月,彗星出三台下,東行入太微,至太子、幸臣,二十餘日而消。十月,歲星、熒惑、太白三合於虛,相去各五六寸。如連珠。占曰:「熒惑在太微為亂臣。」是時中常侍趙忠、張讓、郭勝、孫璋等,並為姦亂。彗星入太微,天下易主。至中平六年,宮車晏駕。歲星、熒惑、太白三合於虛為喪。虛,齊也。明年,琅邪王據薨。

光和中,國皇星東南角去地一二丈,如炬火狀,十餘日不見。占曰:「國皇星為內亂,外內有兵喪。」其後黃巾賊張角燒州郡,朝廷遣將討平,斬首十餘萬級。中平六年,宮車晏駕,大將軍何進令司隸校尉袁紹私募兵千餘人,陰跱雒陽城外,竊呼并州牧董卓使將兵至京都,共誅中官,對戰南、北宮闕下,死者數千人,燔燒宮室,遷都西京。及司徒王允與將軍呂布誅卓,卓部曲將郭汜、李傕旋兵攻長安,公卿百官吏民戰死者且萬人。天下之亂,皆自內發。

中平二年十月癸亥,客星出南門中,大如半筵,五色喜怒稍小,至後年六月消。占曰:「為兵。」至六年,司隸校尉袁紹誅滅中官,大將軍部曲將吳匡攻殺車騎將軍何苗,死者數千人。

三年四月,熒惑逆行守心後星。十月戊午,月食心後星。占曰:「為大喪。」後三年而靈帝崩。

五年二月,彗星出奎,逆行入紫宮,後三出,六十餘日乃消。六月丁卯,客星如三升碗,出貫索,西南行入天市,至尾而消。占曰:「彗除紫宮,天下易主。客星入天市,為貴人喪。」明年四月,宮車晏駕。中平中夏,流星赤如火,長三丈,起河鼓,入天市,抵觸宦者星,色白,長二三丈,後尾再屈,食頃乃滅,狀似枉矢。占曰:「枉矢流發,其宮射,所謂矢當直而枉者,操矢者邪枉人也。」中平六年,大將軍何進謀盡誅中官,於省中殺進:俱兩破滅,天下由此遂大壞亂。

六年八月丙寅,太白犯心前星,戊辰犯心中大星。其日未冥四刻,大將軍何進於省中為諸黃門所殺。己巳,車騎將軍何苗為進部曲將吳匡所殺。

孝獻初平三年九月,蚩尤旗見,長十餘丈,色白,出角、亢之南。占曰:「蚩尤旗見,則王征伐四方。」其後丞相曹公征討天下且三十年。

四年十月,孛星出兩角閒,東北行入天市中而滅。占曰:「彗除天市,天帝將徙,帝將易都。」是時上在長安,後二年東遷,明年七月,至雒陽,其八月,曹公迎上都許。

建安五年十月辛亥,有星孛于大梁,冀州分也。時袁紹在冀州。其年十一月,紹軍為曹公所破。七年夏,紹死,後曹公遂取冀州。

九年十一月,有星孛于東井輿鬼,入軒轅太微。十一年正月,星孛于北斗,首在斗中,尾貫紫宮,及北辰。占曰:「彗星掃太微宮,人主易位。」其後魏文帝受禪。

十二年十月辛卯,有星孛于鶉尾。荊州分也,時荊州牧劉表據荊州,時益州從事周群以荊州牧將死而失土。明年秋,表卒,以小子琮自代。曹公將伐荊州,琮懼,舉軍詣公降。

十七年十二月,有星孛于五諸侯。周群以為西方專據土地者,皆將失土。是時益州牧劉璋據益州,漢中太守張魯別據漢中,韓遂據涼州,宋建別據枹罕。明年冬,曹公遣偏將擊涼州。十九年,獲宋建;韓遂逃于羌中,病死。其年秋,璋失益州。二十年秋,公攻漢中,魯降。

十八年秋,歲星、鎮星、熒惑俱入太微,逆行留守帝坐百餘日。占曰:「歲星入太微,人主改。」

二十三年三月,孛星晨見東方二十餘日,夕出西方,犯歷五車、東井、五諸侯、文昌、軒轅、后妃、太微,鋒炎指帝坐。占曰:「除舊布新之象也。」

殤帝延平元年九月乙亥,隕石陳留四。春秋僖公十六年,隕石于宋五,傳曰隕星也。董仲舒以為從高反下之象。或以為庶人惟星,隕,民困之象也。

桓帝延熹七年三月癸亥,隕石右扶風一,鄠又隕石二,皆有聲如雷。


\end{pinyinscope}