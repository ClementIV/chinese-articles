\article{郡國三}

\begin{pinyinscope}
陳留郡

十七城,戶十七萬七千五百二十九,口八十六萬九千四百三十三。

陳留

有鳴鴈亭。浚儀本大梁。尉氏

雍丘

本杞國。襄邑有滑亭。有承匡城。外黃

有葵丘聚,齊桓公會此。城中有曲棘里。有繁陽城。小黃東昏

濟陽

平

丘

有臨濟亭,田儋死此。有匡。有黃池亭。封丘有桐牢亭,或曰古蟲牢。酸棗

長垣

侯國。有匡城。有蒲城。有祭城。己吾有大棘鄉。有首鄉。考城

故菑,章帝更名。故屬梁。圉

故屬淮陽。有高陽亭。扶溝

故屬淮陽。

東郡

十五城,戶十三萬六千八十八,口六十萬三千三百九十三。

濮陽

古昆吾國,春秋時曰濮。有鹹城,或曰古鹹國。有清丘。有鉏城。燕本南燕國。有雍鄉。有胙城,古胙國。有平陽亭。有瓦亭。有桃城。白馬有韋鄉。頓丘東阿有清亭。東武陽濕水出。范有秦亭。臨邑有沛廟。博平聊城有夷儀聚。有聶戚。發干樂平侯國。故清,章帝更名。陽平侯國。有莘亭。有岡成城。衛公國。本觀故國,姚姓,光武更名。有河牧城。有竿城。穀城春秋時小穀。有巂下聚。

東平國

七城,戶七萬九千一十二,口四十四萬八千二百七十。

無鹽

本宿國,任姓。有章城。東平陸六國時曰平陸。有闞亭。有堂陽亭。富成

章

壽張

春秋曰良,漢曰壽良,光武改曰壽張。有堂聚,故聚屬東郡。須昌故屬東郡。有致密城,古中都。有陽穀城。寧陽故屬泰山。

任城國

三城,戶三萬六千四百四十二,口十九萬四千一百五十六。

任城

本任國。有桃聚。亢父樊

泰山郡

十二城,戶八千九百二十九,口四十三萬七千三百一十七。

奉高

有明堂,武帝造。博有泰山廟。岱山在西北。有龜山。有龍鄉城。梁甫侯國。有菟裘聚。鉅平侯國。有亭禪山。有陽關亭。嬴有鐵。山茌侯國。萊蕪有原山,潘水出。蓋沂水出。南武陽侯國。有顓臾城。南城故屬東海。有東陽城。費侯國,故屬東海。有祊亭。有台亭。牟故國。

濟北國

五城,戶四萬五千六百八十九,口二十三萬五千八百九十七。

盧

有平陰城。有防門。有光里。有景茲山。有敖山。有清亭。有長城至東海。蛇丘有遂鄉。有下讙亭。有鑄鄉城。成

本國。茌平本屬東郡。剛。

山陽郡

十城,戶十萬九千八百九十八,口六十萬六千九十一。

昌邑

刺史治。有梁丘城。有甲父亭。東緡春秋時曰緡。鉅野有大野澤。高平侯國。故橐,章帝更名。有茅鄉城。湖陸故湖陵,章帝更名。南平陽侯國。有漆亭。有閭丘亭。方與有武唐亭,魯侯觀魚臺。有泥母亭,或曰古甯母。瑕丘金鄉防東

濟陰郡

十一城,戶十三萬三千七百一十五,口六十五萬七千五百五十四。

定陶

本曹國,古陶,堯所居。有三鬷亭。冤句有煮棗城。成陽有堯冢、靈臺,有雷澤。乘氏侯國。有泗水。有鹿城鄉。句陽有垂亭。鄄城離狐故屬東郡。廩丘故屬東郡。有高魚城。有運城。單父侯國,故屬山陽。成武故屬山陽。有郜城。己氏故屬梁。

右兗州刺史部,郡、國八,縣、邑、公、侯國八十。

東海郡

十三城,戶十四萬八千七百八十四,口七十萬六千四百一十六。

郯

本國,刺史治。蘭陵有次室亭。戚朐

有鐵。有伊盧鄉。襄賁

昌慮

有藍鄉。承

陰平

利城

合

城祝其

有羽山。春秋時曰祝其,夾谷地。厚丘贛榆本屬琅邪,建初五年復。

琅邪國

十

三城,戶二萬八百四,口五十七萬九百六十七。

開陽

故屬東海,建初五年屬。東武琅邪

東莞

有鄆亭。有邳鄉。有公來山,或曰古浮來。西海

諸

莒

本國,故屬城陽。有鐵。有崢嶸谷。東安

故屬城陽。陽都故屬城陽。有牟臺。臨沂故屬東海。有叢亭。即丘侯國,故屬東海,春秋曰祝丘。繒侯國,故屬東海。有概亭。姑幕

彭城國

八城,戶八萬六千一百七十,口四十九萬三千二十七。

彭城

有鐵。武原傅陽有柤水。呂

留

梧

菑丘廣戚

故屬沛國。

廣陵郡

十一城,戶八萬三千九百七,口四十一萬百九十。

廣陵

有東陵亭。江都有江水祠。高郵

平安

淩

故屬泗水。東陽故屬臨淮。有長洲澤,吳王濞太倉在此。射陽故屬臨淮。鹽瀆故屬臨淮。輿侯國,故屬臨淮。堂邑故屬臨淮。有鐵。春秋時曰堂。海西故屬東海。

下邳國

十

七城,戶十三萬六千三百八十九,口六十一萬一千八十三。

下邳

本屬東海。葛嶧山,本嶧陽山。有鐵。徐本國。有樓亭,或曰古蔞林。僮侯國。睢陵下相淮陰淮浦盱台

高山潘旌

淮陵

取慮

有蒲姑陂。東成曲陽侯國,故屬東海。司吾侯國,故屬東海。良成故屬東海。春秋時曰良。夏丘故屬沛。

右徐州刺史部,郡、國五,縣、邑、侯國六十二。


\end{pinyinscope}