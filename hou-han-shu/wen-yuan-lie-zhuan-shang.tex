\article{文苑列傳上}

\begin{pinyinscope}
杜篤字季雅,京兆杜陵人也。高祖延年,宣帝時為御史大夫。篤少博學,不修小節,不為鄉人所禮。居美陽,與美陽令遊,數從請託,不諧,頗相恨。令怒,收篤送京師。會大司馬吳漢薨,光武詔諸儒誄之,篤於獄中為誄,辭最高,帝美之,賜帛免刑。

篤以關中表裏山河,先帝舊京,不宜改營洛邑,乃上奏論都賦曰:

臣聞知而復知,是為重知。臣所欲言,陛下已知,故略其梗概,不敢具陳。昔般庚去奢,行儉於亳,成周之隆,乃即中洛。遭時制都,不常厥邑。賢聖之慮,蓋有優劣;霸王之姿,明知相絕。守國之埶,同歸異術;或棄去阻阨,務處平易;或據山帶河,并吞六國;或富貴思歸,不顧見襲;或掩空擊虛,自蜀漢出;即日車駕,策由一卒;或知而不從,久都墝埆。臣不敢有所據。竊見司馬相如、楊子雲作辭賦以諷主上,臣誠慕之,伏作書一篇,名曰論都,謹并封奏如左。

皇帝以建武十八年二月甲辰,升輿洛邑,巡于西岳。推天時,順斗極,排閶闔,入函谷,觀阨於崤、黽,圖險於隴、蜀。其三月丁酉,行至長安。經營宮室,傷愍舊京,即詔京兆,迺命扶風,齋肅致敬,告覲園陵。悽然有懷祖之思,喟乎以思諸夏之隆。遂天旋雲遊,造舟于渭,北錐涇流。千乘方轂,萬騎駢羅,衍陳於岐、梁,東橫乎大河。瘞后土,禮邠郊。其歲四月,反于洛都。明年,有詔復函谷關,作大駕宮、六王邸、高車廄於長安,脩理東都城門,橋涇、渭。往往繕離觀,東臨霸、滻,西望昆明,北登長平,規龍首,撫未央,覛平樂,儀建章。

是時山東翕然狐疑,意聖朝之西都,懼關門之反拒也。客有為篤言:「彼埳井之潢汙,固不容夫吞舟;且洛邑之渟瀯,曷足以居乎萬乘哉?咸陽守國利器,不可久虛,以示姦萌。」篤未甚然甚言也,故因為述大漢之崇,世據廱州之利,而今國家未暇之故,以喻客意。曰:

昔在強秦,爰初開畔,霸自岐、廱,國富人衍,卒以并兼,桀虐作亂。天命有聖,託之大漢。大漢開基,高祖有勳,斬白蛇,屯黑雲,聚五星於東井,提干將而呵暴秦。蹈滄海,跨崑崙,奮彗光,埽項軍,遂濟人難,蕩滌於泗、沂。劉敬建策,初都長安。太宗承流,守之以文。躬履節儉,側身行仁,食不二味,衣無異采,賑人以農桑,率下以約己,曼麗之容不悅於目,鄭衛之聲不過於耳,佞邪之臣不列於朝,巧偽之物不鬻於巿,故能理升平而刑幾措。富衍於孝景,功傳於後嗣。

是時孝武因其餘財府帑之蓄,始有鉤深圖遠之意,探冒頓之罪,校平城之讎。遂命票騎,勤任衛青,勇惟鷹揚,軍如流星,深之匈奴,割裂王庭,席卷漠北,叩勒祁連,橫分單于,屠裂百蠻。燒罽帳,繫閼氏,燔康居,灰珍奇,椎鳴鏑,釘鹿蠡,馳阬岸,獲昆彌,虜颓侲,驅騾驢,馭宛馬,鞭駃騠。拓地萬里,威震八荒。肇置四郡,據守敦煌。并域屬國,一郡領方。立候隅北,建護西羌。捶驅氐、僰,寥狼邛、莋。東攠烏桓,蹂轔濊貊。南羈鉤町,水劍強越。殘夷文身,海波沫血。郡縣日南,漂概朱崖。部尉東南,兼有黃支。連緩耳,瑣雕題,摧天督,牽象犀,椎蚌蛤,碎琉璃,甲玳瑁,戕觜觿。於是同穴裘褐之域,共川鼻飲之國,莫不袒跣稽顙,失氣虜伏。非夫大漢之盛,世藉廱土之饒,得御外理內之術,孰能致功若斯!故創業於高祖,嗣傳於孝惠,德隆於太宗,財衍於孝景,威盛於聖武,政行於宣、元,侈極於成、哀,祚缺於孝平。傳世十一,歷載三百,德衰而復盈,道微而復章,皆莫能遷於廱州,而背於咸陽。宮室寢廟,山陵相望,高顯弘麗,可思可榮,羲、農已來,無茲著明。

夫廱州本帝皇所以育業,霸王所以衍功,戰士角難之場也。禹貢所載,厥田惟上。沃野千里,原隰彌望。保殖五穀,桑麻條暢。濱據南山,帶以涇、渭,號曰陸海,蠢生萬類。楩柟檀柘,蔬果成實。畎瀆潤淤,水泉灌溉,漸澤成川,粳稻陶遂。厥土之膏,畝價一金。田田相如,鐇钁株林。火耕流種,功淺得深。既有蓄積,阨塞四臨:四被隴、蜀,南通漢中,北據谷口,東阻嶔巖。關函守嶢,山東道窮;置列洴、隴,廱偃西戎;拒守褒斜,嶺南不通;杜口絕津,朔方無從。鴻、渭之流,徑入于河;大船萬艘,轉漕相過;東綜滄海,西綱流沙;朔南暨聲,諸夏是和。城池百尺,阨塞要害。關梁之險,多所衿帶。一卒舉礧,千夫沈滯;一人奮戟,三軍沮敗。地埶便利,介冑剽悍,可與守近,利以攻遠。士卒易保,人不肉袒。肇十有二,是為贍腴。用霸則兼并,先據則功殊;修文則財衍,行武則士要;為政則化上,篡逆則難誅;進攻則百剋,退守則有餘:斯固帝王之淵囿,而守國之利器也。

逮及亡新,時漢之衰,偷忍淵囿,篡器慢違,徒以埶便,莫能卒危。假之十八,誅自京師。天畀更始,不能引維,慢藏招寇,復致赤眉。海內雲擾,諸夏滅微;群龍並戰,未知是非。于時聖帝,赫然申威。荷天人之符,兼不世之姿。受命於皇上,獲助於靈祇。立號高邑,搴旗四麾。首策之臣,運籌出奇;虓怒之旅,如虎如螭。師之攸向,無不靡披。蓋夫燔魚剸蛇,莫之方斯。大呼山東,響動流沙。要龍淵,首鏌铮,命騰太白,親發狼、弧,南禽公孫,北背強胡,西平隴、冀,東據洛都。乃廓平帝宇,濟蒸人於塗炭,成兆庶之亹亹,遂興復乎大漢。

今天下新定,矢石之勤始瘳,而主上方以邊垂為憂,忿葭萌之不柔,未遑於論都而遺思廱州也。方躬勞聖思,以率海內,厲撫名將,略地疆外,信威於征伐,展武乎荒裔。若夫文身鼻飲緩耳之主,椎結左衽鐻录之君,東南殊俗不羈之國,西北絕域難制之鄰,靡不重譯納貢,請為藩臣。上猶謙讓而不伐勤。意以為獲無用之虜,不如安有益之民;略荒裔之地,不如保殖五穀之淵;遠救於已亡,不若近而存存也。今國家躬脩道德,吐惠含仁,湛恩沾洽,時風顯宣。徒垂意於持平守實,務在愛育元元,苟有便於王政者,聖主納焉。何則?物罔挹而不損,道無隆而不移,陽盛則運,陰滿則虧,故存不忘亡,安不諱危,雖有仁義,猶設城池也。

客以利器不可久虛,而國家亦不忘乎西都,何必去洛邑之渟瀯與?

篤後仕郡文學掾,以目疾,二十餘年不闚京師。

篤之外高祖破羌將軍辛武賢,以武略稱。篤常歎曰:「杜氏文明善政,而篤不任為吏;辛氏秉義經武,而篤又怯於事。外內五世,至篤衰矣!」

女弟適扶風馬氏。建初三年,車騎將軍馬防擊西羌,請篤為從事中郎,戰沒於射姑山。

所著賦、誄、弔、書、讚、七言、女誡及雜文,凡十八篇。又著明世論十五篇。

子碩,豪俠,以貨殖聞。

王隆字文山,馮翊雲陽人也。王莽時,以父任為郎,後避難河西,為竇融左護軍。建武中,為新汲令。能文章,所著詩、賦、銘、書凡二十六篇。

初,王莽末,沛國史岑子孝亦以文章顯,莽以為謁者,著頌、誄、復神、說疾凡四篇。

夏恭字敬公,梁國蒙人也。習韓詩、孟氏易,講授門徒常千餘人。王莽末,盜賊從橫,攻沒郡縣,恭以恩信為眾所附,擁兵固守,獨安全。光武即位,嘉其忠果,召拜郎中,再遷太山都尉。和集百姓,甚得其歡心。

恭善為文,著賦、頌、詩、勵學凡二十篇。年四十九卒官,諸儒共謚曰宣明君。

子牙,少習家業,著賦、頌、讚、誄凡四十篇。舉孝廉,早卒,鄉人號曰文德先生。

傅毅字武仲,扶風茂陵人也。少博學。永平中,於平陵習章句,因作迪志詩曰:

咨爾庶士,迨時斯勗。日月逾邁,豈云旋復!哀我經營,旅力靡及。在茲弱冠,靡所庶立。

於赫我祖,顯于殷國。二跡阿衡,克光其則。武丁興商,伊宗皇士。爰作股肱,萬邦是紀。奕世載德,迄我顯考。保膺淑懿,纘脩其道。漢之中葉,俊乂式序。秩彼殷宗,光此勳緒。

伊余小子,穢陋靡逮。懼我世烈,自茲以墜。誰能革濁,清我濯溉?誰能昭闇,啟我童昧?先人有訓,我訊我誥。訓我嘉務,誨我博學。爰率朋友,尋此舊則。契闊夙夜,庶不懈忒。

秩秩大猷,紀綱庶式。匪勤匪昭,匪壹匪測。農夫不怠,越有黍稷,誰能云作,考之居息?二事敗業,多疾我力。如彼遵衢,則罔所極。二志靡成,聿勞我心。如彼兼聽,則溷於音。

於戲君子,無恆自逸。徂年如流,鮮茲暇日。行邁屢稅,胡能有迄。密勿朝夕,聿同始卒。

毅以顯宗求賢不篤,士多隱處,故作七激以為諷。

建初中,肅宗博召文學之士,以毅為蘭臺令史,拜郎中,與班固、賈逵共典校書。毅追美孝明皇帝功德最盛,而廟頌未立,乃依清廟作顯宗頌十篇奏之,由是文雅顯於朝廷。

車騎將軍馬防,外戚尊重,請毅為軍司馬,待以師友之禮。及馬氏敗,免官歸。

永元元年,車騎將軍竇憲復請毅為主記室,崔駰為主簿。及憲遷大將軍,復以毅為司馬,班固為中護軍。憲府文章之盛,冠於當世。

毅早卒,著詩、賦、誄、頌、祝文、七激、連珠凡二十八篇。

黃香字文彊,江夏安陸人也。年九歲,失母,思慕憔悴,殆不免喪,鄉人稱其至孝。年十二,太守劉護聞而召之,署門下孝子,甚見愛敬。香家貧,內無僕妾,躬執苦勤,盡心奉養。遂博學經典,究精道術,能文章,京師號曰「天下無雙江夏黃童」。

初除郎中,元和元年,肅宗詔香詣東觀,讀所未嘗見書。香後告休,及歸京師,時千乘王冠,帝會中山邸,乃詔香殿下,顧謂諸王曰:「此『天下無雙江夏黃童』者也。」左右莫不改觀。後召詣安福殿言政事,拜尚書郎,數陳得失,賞賚增加。常獨止宿臺上,晝夜不離省闥,帝聞善之。

永元四年,拜左丞,功滿當遷,和帝留,增秩。六年,累遷尚書令。後以為東郡太守,香上疏讓曰:「臣江淮孤賤,愚矇小生,經學行能,無可筭錄。遭值太平,先人餘福,得以弱冠特蒙徵用,連階累任,遂極臺閣。訖無纖介稱,報恩效死,誠不意悟,卒被非望,顯拜近郡,尊位千里。臣聞量能授官,則職無廢事;因勞施爵,則賢愚得宜。臣香小醜,少為諸生,典郡從政,固非所堪,誠恐矇頓,孤忝聖恩。又惟機密端首,至為尊要,復非臣香所當久奉。承詔驚惶,不知所裁。臣香年在方剛,適可驅使。願乞餘恩,留備冗官,賜以督責小職,任之宮臺煩事,以畢臣香螻蟻小志,誠瞑目至願,土灰極榮。」帝亦惜香幹用,久習舊事,復留為尚書令,增秩二千石,賜錢三十萬。是後遂管樞機,甚見親重,而香亦祗勤物務,憂公如家。

十二年,東平清河奏訞言卿仲遼等,所連及且千人。香科別據奏,全活甚眾。每郡國疑罪,輒務求輕科,愛惜人命,每存憂濟。又曉習邊事,均量軍政,皆得事宜。帝知其精勤,數加恩賞,疾病存問,賜醫藥。在位多所薦達,寵遇甚盛,議者譏其過倖。

延平元年,遷魏郡太守。郡舊有內外園田,常與人分種,收穀歲數千斛。香曰:「田令『商者不農』,王制『仕者不耕』,伐冰食祿之人,不與百姓爭利。」乃悉以賦人,課令耕種。時被水年飢,乃分奉祿及所得賞賜班贍貧者,於是豐富之家各出義穀,助官稟貸,荒民獲全。後坐水潦事免,數月,卒於家。

所著賦、牋、奏、書、令凡五篇。子瓊,自有傳。

劉毅,北海敬王子也。初封平望侯,永元中,坐事奪爵。毅少有文辯稱,元初元年,上漢德論并憲論十二篇。時劉珍、鄧耽、尹兌、馬融共上書稱其美,安帝嘉之,賜錢三萬,拜議郎。

李尤字伯仁,廣漢雒人也。少以文章顯。和帝時,侍中賈逵薦尤有相如、楊雄之風,召詣東觀,受詔作賦,拜蘭臺令史。稍遷,安帝時為諫議大夫,受詔與謁者僕射劉珍等俱撰漢記。後帝廢太子為濟陰王,尤上書諫爭。順帝立,遷樂安相。年八十三卒。所著詩、賦、銘、誄、頌、七歎、哀典凡二十八篇。

尤同郡李勝,亦有文才,為東觀郎,著賦、誄、頌、論數十篇。

蘇順,字孝山,京兆霸陵人也。和安閒以才學見稱。好養生術,隱處求道。晚乃仕,拜郎中,卒於官。所著賦、論、誄、哀辭、雜文凡十六篇。

時三輔多士,扶風曹眾伯師亦有才學,著誄、書、論四篇。

又有曹朔,不知何許人,作漢頌四篇。

劉珍字秋孫,一名寶,南陽蔡陽人也。少好學。永初中,為謁者僕射。鄧太后詔使與校書劉騊駼、馬融及五經博士,校定東觀五經、諸子傳記、百家蓺術,整齊脫誤,是正文字。永寧元年,太后又詔珍與騊駼作建武已來名臣傳,遷侍中、越騎校尉。延光四年,拜宗正。明年,轉衛尉,卒官。著誄、頌、連珠凡七篇。又撰釋名三十篇,以辯萬物之稱號云。

葛龔字元甫,梁國寧陵人也。和帝時,以善文記知名。性慷慨壯烈,勇力過人。安帝永初中,舉孝廉,為太官丞,上便宜四事,拜蕩陰令。辟太尉府,病不就。州舉茂才,為臨汾令。居二縣,皆有稱績。著文、賦、碑、誄、書記凡十二篇。

王逸字叔師,南郡宜城人也。元初中,舉上計吏,為校書郎。順帝時,為侍中。著楚辭章句行於世。其賦、誄、書、論及雜文凡二十一篇。又作漢詩百二十三篇。

子延壽,字文考,有俊才。少遊魯國,作靈光殿賦。後蔡邕亦造此賦,未成,及見延壽所為,甚奇之,遂輟翰而已。曾有異夢,意惡之,乃作夢賦以自厲。後溺水死,時年二十餘。

崔琦字子瑋,涿郡安平人,濟北相瑗之宗也。少遊學京師,以文章博通稱。初舉孝廉,為郎。河南尹梁冀聞其才,請與交。冀行多不軌,琦數引古今成敗以戒之,冀不能受。乃作外戚箴。其辭曰:

赫赫外戚,華寵煌煌。昔在帝舜,德隆英、皇。周興三母,有莘崇湯。宣王晏起,姜后脫簪。齊桓好樂,衛姬不音。皆輔主以禮,扶君以仁,達才進善,以義濟身。

爰暨末葉,漸已穨虧。貫魚不敘,九御差池。晉國之難,禍起於麗。惟家之索,牝雞之晨。專權擅愛,顯己蔽人。陵長閒舊,圮剝至親。並后匹嫡,淫女斃陳。匪賢是上,番為司徒。荷爵負乘,采食名都。詩人是刺,德用不憮。暴辛惑婦,拒諫自孤。蝠蛇其心,縱毒不辜。諸父是殺,孕子是刳。天怒地忿,人謀鬼圖。甲子昧爽,身首分離。初為天子,後為人螭。

非但耽色,母后尤然。不相率以禮,而競獎以權。先笑後號,卒以辱殘。家國泯絕,宗廟燒燔。末嬉喪夏,褒姒斃周,妲己亡殷,趙靈沙丘。戚姬人豕,呂宗以敗。陳后作巫,卒死於外。霍欲鴆子,身乃罹廢。

故曰:無謂我貴,天將爾摧;無恃常好,色有歇微;無怙常幸,愛有陵遲;無曰我能,天人爾違。患生不德,福有慎機。日不常中,月盈有虧。履道者固,杖埶者危。微臣司戚,敢告在斯。

琦以言不從,失意,復作白鵠賦以為風。梁冀見之,呼琦問曰:「百官外內,各有司存,天下云云,豈獨吾人之尤,君何激刺之過乎?」琦對曰:「昔管仲相齊,樂聞譏諫之言;蕭何佐漢,乃設書過之吏。今將軍累世台輔,任齊伊、公,而德政未聞,黎元塗炭,不能結納貞良,以救禍敗,反復欲鉗塞士口,杜蔽主聽,將使玄黃改色,馬鹿易形乎?」冀無以對,因遣琦歸。

後除為臨濟長,不敢之職,解印綬去。冀遂令刺客陰求殺之。客見琦耕於陌上,懷書一卷,息輒偃而詠之。客哀其志,以實告琦,曰:「將軍令吾要子,今見君賢者,情懷忍忍,可亟自逃,吾亦於此亡矣。」琦得脫走,冀後竟捕殺之。所著賦、頌、銘、誄、箴、弔、論、九咨、七言,凡十五篇。

邊韶字孝先,陳留浚儀人也。以文章知名,教授數百人。詔口辯,曾晝日假臥,弟子私仓之曰:「邊孝先,腹便便。嬾讀書,但欲眠。」韶潛聞之,應時對曰:「邊為姓,孝為字。腹便便,五經笥。但欲眠,思經事。寐與周公通夢,靜與孔子同意。師而可仓,出何典記?」仓者大慚。韶之才捷皆此類也。

桓帝時,為臨潁侯相,徵拜太中大夫,著作東觀。再遷北地太守,入拜尚書令。後為陳相,卒官。著詩、頌、碑、銘、書、策凡十五篇。


\end{pinyinscope}