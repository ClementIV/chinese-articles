\article{皇后紀下}

\begin{pinyinscope}
安思閻皇后諱姬,河南滎陽人也。祖父章,永平中為尚書,以二妹為貴人。章精力曉舊典,久次,當遷以重職,顯宗為後宮親屬,竟不用,出為步兵校尉。章生暢,暢生后。

后有才色。元初元年,以選入掖庭,甚見寵愛,為貴人。二年,立為皇后。后專房妒忌,帝幸宮人李氏,生皇子保,遂鴆殺李氏。三年,以后父侍中暢為長水校尉,封北宜春侯,食邑五千戶。四年,暢卒,謚曰文侯,子顯嗣。

建光元年,鄧太后崩,帝始親政事。顯及弟景、耀、晏並為卿校,典禁兵。延光元年,更封顯長社縣侯,食邑萬三千五百戶,追尊后母宗為滎陽君。顯、景諸子年皆童齔,並為黃門侍郎。后寵既盛,而兄弟頗與朝權,后遂與大長秋江京、中常侍樊豐等共譖皇太子保,廢為濟陰王。

四年春,后從帝幸章陵,帝道疾,崩於葉縣。后、顯兄弟及江京、樊豐等謀曰:「今晏駕道次,濟陰王在內,邂逅公卿立之,還為大害。」乃偽云帝疾甚,徙御臥車。行四日,驅馳還宮。明日,詐遣司徒劉喜詣郊廟社稷,告天請命。其夕,乃發喪。尊后曰皇太后。皇太后臨朝,以顯為車騎將軍儀同三司。

太后欲久專國政,貪立幼年,與顯等定策禁中,迎濟北惠王子北鄉侯懿,立為皇帝。顯忌大將軍耿寶位尊權重,威行前朝,乃風有司奏寶及其黨與中常侍樊豐、虎賁中郎將謝惲、惲弟侍中篤、篤弟大將軍長史宓、侍中周廣、阿母野王君王聖、聖女永、永婿黃門侍郎樊嚴等,更相阿黨,互作威福,探刺禁省,更為唱和,皆大不道。豐、惲、廣皆下獄死,家屬徙比景;宓、嚴減死,髡鉗;貶寶為則亭侯,遣就國,自殺;王聖母子徙鴈門。於是景為衛尉,耀城門校尉,晏執金吾,兄弟權要,威福自由。

少帝立二百餘日而疾篤,顯兄弟及江京等皆在左右。京引顯屏語曰:「北鄉侯病不解,國嗣宜時有定。前不用濟陰王,今若立之,後必當怨,又何不早徵諸王子,簡所置乎?」顯以為然。及少帝薨,京白太后,徵濟北、河閒王子。未至,而中黃門孫程合謀殺江京等,立濟陰王,是為順帝。顯、景、晏及黨與皆伏誅,遷太后於離宮,家屬徙比景。明年,太后崩。在位十二年,合葬恭陵。

帝母李氏瘞在洛陽城北,帝初不知,莫敢以聞。及太后崩,左右白之,帝感悟發哀,親到瘞所,更以禮殯,上尊謚曰恭愍皇后,葬恭北陵,為策書金匱,藏于世祖廟。

順烈梁皇后諱妠,大將軍商之女,恭懷皇后弟之孫也。后生,有光景之祥。少善女工,好史書,九歲能誦論語,治韓詩,大義略舉。常以列女圖畫置於左右,以自監戒。父商深異之,竊謂諸弟曰:「我先人全濟河西,所活者不可勝數。雖大位不究,而積德必報。若慶流子孫者,儻興此女乎?」

永建三年,與姑俱選入掖庭,時年十三。相工茅通見后,驚,再拜賀曰:「此所謂日角偃月,相之極貴,臣所未嘗見也。」太史卜兆得壽房,又筮得坤之比,遂以為貴人。常特被引御,從容辭於帝曰:「夫陽以博施為德,陰以不專為義,螽斯則百,福之所由興也。願陛下思雲雨之均澤,識貫魚之次序,使小妾得免罪謗之累。」由是帝加敬焉。

陽嘉元年春,有司奏立長秋宮,以乘氏侯商先帝外戚,春秋之義,娶先大國,梁小貴人宜配天祚,正位坤極。帝從之,乃於壽安殿立貴人為皇后。后既少聰惠,深覽前世得失,雖以德進,不敢有驕專之心,每日月見謫,輒降服求愆。

建康元年,帝崩。后無子,美人虞氏子炳立,是為沖帝。尊后為皇太后,太后臨朝。沖帝尋崩,復立質帝,猶秉朝政。

時楊、徐劇賊寇擾州郡,西羌、鮮卑及日南蠻夷攻城暴掠,賦斂煩數,官民困竭。太后夙夜勤勞,推心杖賢,委任太尉李固等,拔用忠良,務崇節儉。其貪叨罪慝,多見誅廢。分兵討伐,群寇消夷。故海內肅然,宗廟以寧。而兄大將軍冀鴆殺質帝,專權暴濫,忌害忠良,數以邪說疑誤太后,遂立桓帝而誅李固。太后又溺於宦官,多所封寵,以此天下失望。

和平元年春,歸政於帝,太后寢疾遂篤,乃御輦幸宣德殿,見宮省官屬及諸梁兄弟。詔曰:「朕素有心下結氣,從閒以來,加以浮腫,逆害飲食,寖以沈困,比使內外勞心請禱。私自忖度,日夜虛劣,不能復與群公卿士共相終竟。援立聖嗣,恨不久育養,見其終始。今以皇帝、將軍兄弟委付股肱,其各自勉焉。」後二日而崩。在位十九年,年四十五。合葬憲陵。

虞美人者,以良家子年十三選入掖庭,又生女舞陽長公主。自漢興,母氏莫不尊寵。順帝既未加美人爵號,而沖帝早夭,大將軍梁冀秉政,忌惡佗族,故虞氏抑而不登,但稱「大家」而已。

陳夫人者,家本魏郡,少以聲伎入孝王宮,得幸,生質帝。亦以梁氏故,榮寵不及焉。

熹平四年,小黃門趙祐、議郎卑整上言:「春秋之義,母以子貴。隆漢盛典,尊崇母氏,凡在外戚,莫不加寵。今沖帝母虞大家,質帝母陳夫人,皆誕生聖皇,而未有稱號。夫臣子雖賤,尚有追贈之典,況二母見在,不蒙崇顯之次,無以述遵先世,垂示後世也。」帝感其言,乃拜虞大家為憲陵貴人,陳夫人為渤海孝王妃,使中常侍持節授印綬,遣太常以三牲告憲陵、懷陵、靜陵焉。

孝崇匽皇后諱明,為蠡吾侯翼媵妾,生桓帝。桓帝即位,明年,追尊翼為孝崇皇,陵曰博陵,以后為博園貴人。和平元年,梁太后崩,乃就博陵尊后為孝崇皇后。遣司徒持節奉策授璽綬,齎乘輿器服,備法物。宮曰永樂。置太僕、少府以下,皆如長樂宮故事。又置虎賁、羽林衛士,起宮室,分鉅鹿九縣為后湯沐邑。在位三年,元嘉二年崩。以帝弟平原王石為喪主,斂以東園畫梓壽器、玉匣、飯含之具,禮儀制度比恭懷皇后。使司徒持節,大長秋奉弔祠,賻錢四千萬,布四萬匹,中謁者僕射典護喪事,侍御史護大駕鹵簿。詔安平王豹、河閒王建、勃海王悝,長社、益陽二長公主,與諸國侯三百里內者,及中二千石、二千石、令、長、相,皆會葬。將作大匠復土,繕廟,合葬博陵。

桓帝懿獻梁皇后諱女瑩,順烈皇后之女弟也。帝初為蠡吾侯,梁太后徵,欲與后為婚,未及嘉禮,會質帝崩,因以立帝。明年,有司奏太后曰:「春秋迎王后于紀,在塗則稱后。今大將軍冀女弟,膺紹聖善。結婚之際,有命既集,宜備禮章,時進徵幣。請下三公、太常案禮儀。」奏可。於是悉依孝惠皇帝納后故事,聘黃金二萬斤,納采鴈璧乘馬束帛,一如舊典。建和元年六月始入掖庭,八月立為皇后。

時太后秉政而梁冀專朝,故后獨得寵幸,自下莫得進見。后藉姊兄廕埶,恣極奢靡,宮幄彫麗,服御珍華,巧飾制度,兼倍前世。及皇太后崩,恩愛稍衰。后既無子,潛懷怨忌,每宮人孕育,鮮得全者。帝雖迫畏梁冀,不敢譴怒,然見御轉稀。至延熹三年,后以憂恚崩,在位十三年,葬懿陵。其歲,誅梁冀,廢懿陵為貴人冢焉。

桓帝鄧皇后諱猛女,和熹皇后從兄子鄧香之女也。母宣,初適香,生后。改嫁梁紀,紀者,大將軍梁冀妻孫壽之舅也。后少孤,隨母為居,因冒姓梁氏。冀妻見后貌美,永興中進入掖庭,為采女,絕幸。明年,封兄鄧演為南頓侯,位特進。演卒,子康嗣。及懿獻后崩,梁冀誅,立后為皇后。帝惡梁氏,改姓為薄,封后母宣為長安君。四年,有司奏后本郎中鄧香之女,不宜改易它姓,於是復為鄧氏。追封贈香車騎將軍安陽侯印綬,更封宣、康大縣,宣為昆陽君,康為沘陽侯,賞賜巨萬計。宣卒,賵贈葬禮,皆依后母舊儀。以康弟統襲封昆陽侯,位侍中;統從兄會襲安陽侯,為虎賁中郎將;又封統弟秉為淯陽侯。宗族皆列校、郎將。

帝多內幸,博採宮女至五六千人,及矫役從使,復兼倍於此。而后恃尊驕忌,與帝所幸郭貴人更相譖訴。八年,詔廢后,送暴室,以憂死。立七年。葬於北邙。從父河南尹萬世及會皆下獄死。統等亦繫暴室,免官爵,歸本郡,財物沒入縣官。

桓思竇皇后諱妙,章德皇后從祖弟之孫女也。父諱武。延熹八年,鄧皇后廢,后以選入掖庭為貴人,其冬,立為皇后,而御見甚稀,帝所寵唯采女田聖等。永康元年冬,帝寢疾,遂以聖等九女皆為貴人。及崩,無嗣,后為皇太后。太后臨朝定策,立解犢亭侯宏,是為靈帝。

太后素忌忍,積怒田聖等,桓帝梓宮尚在前殿,遂殺田聖。又欲盡誅諸貴人,中常侍管霸、蘇康苦諫,乃止。時太后父大將軍武謀誅宦官,而中常侍曹節等矯詔殺武,遷太后於南宮雲臺,家屬徙比景。

竇氏雖誅,帝猶以太后有援立之功,建寧四年十月朔,率群臣朝于南宮,親饋上壽。黃門令董萌因此數為太后訴怨,帝深納之,供養資奉有加於前。中常侍曹節、王甫疾萌附助太后,誣以謗訕永樂宮,萌坐下獄死。熹平元年,太后母卒於比景,后感疾而崩。立七年。合葬宣陵。

孝仁董皇后諱某,河閒人。為解犢亭侯萇夫人,生靈帝。建寧元年,帝即位,追尊萇為孝仁皇,陵曰慎陵,以后為慎園貴人。及竇氏誅,明年,帝使中常侍迎貴人,并徵貴人兄寵到京師,上尊號曰孝仁皇后,居南宮嘉德殿,宮稱永樂。拜寵執金吾。後坐矯稱永樂后屬請,下獄死。

及竇太后崩,始與朝政,使帝賣官求貨,自納金錢,盈滿堂室。中平五年,以后兄子衛尉脩侯重為票騎將軍,領兵千餘人。初,后自養皇子協,數勸帝立為太子,而何皇后恨之,議未及定而帝崩。何太后臨朝,重與太后兄大將軍進權埶相害,后每欲參干政事,太后輒相禁塞。后忿恚詈言曰:「汝今輈張,怙汝兄耶?當敕票騎斷何進頭來。」何太后聞,以告進。進與三公及弟車騎將軍苗等奏:「孝仁皇后使故中常侍夏惲、永樂太僕封諝等交通州郡,辜較在所珍寶貨賂,悉入西省。蕃后故事不得留京師,輿服有章,膳羞有品。請永樂后遷宮本國。」奏可。何進遂舉兵圍驃騎府,收重,免官自殺。后憂怖,疾病暴崩,在位二十二年。民閒歸咎何氏。喪還河閒,合葬慎陵。

靈帝宋皇后諱某,扶風平陵人也,肅宗宋貴人之從曾孫也。建寧三年,選入掖庭為貴人。明年,立為皇后。父酆,執金吾,封不其鄉侯。

后無寵而居正位,後宮幸姬眾,共譖毀。初,中常侍王甫枉誅勃海王悝及妃宋氏,妃即后之姑也。甫恐后怨之,及與太中大夫程阿共構言皇后挾左道祝詛,帝信之。光和元年,遂策收璽綬。后自致暴室,以憂死。在位八年。父及兄弟並被誅。諸常侍、小黃門在省闥者,皆憐宋氏無辜,共合錢物,收葬廢后及酆父子,歸宋氏舊塋皋門亭。

帝後夢見桓帝怒曰:「宋皇后有何罪過,而聽用邪孽,使絕其命?勃海王悝既已自貶,又受誅斃。今宋氏及悝自訴於天,上帝震怒,罪在難救。」夢殊明察。帝既覺而恐,以事問於羽林左監許永曰:「此何祥?其可攘乎?」永對曰:「宋皇后親與陛下共承宗廟,母臨萬國,歷年已久,海內蒙化,過惡無聞。而虛聽讒妒之說,以致無辜之罪,身嬰極誅,禍及家族,天下臣妾,咸為怨痛。勃海王悝,桓帝母弟也。處國奉藩,未嘗有過。陛下曾不證審,遂伏其辜。昔晉侯失刑,亦夢大厲被髮屬地。天道明察,鬼神難誣。宜并改葬,以安冤魂。反宋后之徙家,復勃海之先封,以消厥咎。」帝弗能用,尋亦崩焉。

靈思何皇后諱某,南陽宛人。家本屠者,以選入掖庭。長七尺一寸。生皇子辯,養於史道人家,號曰史侯。拜后為貴人,甚有寵幸。性彊忌,後宮莫不震懾。

光和三年,立為皇后。明年,追號后父真為車騎將軍、舞陽宣德侯,因封后母興為舞陽君。時王美人任娠,畏后,乃服藥欲除之,而胎安不動,又數夢負日而行。四年,生皇子協,后遂酖殺美人。帝大怒,欲廢后,諸宦官固請得止。董太后自養協,號曰董侯。

王美人,趙國人也。祖父苞,五官中郎將。美人豐姿色,聰敏有才明,能書會計,以良家子應法相選入掖庭。帝愍協早失母,又思美人,作追德賦、令儀頌。

中平六年,帝崩,皇子辯即位,尊后為皇太后。太后臨朝。后兄大將軍進欲誅宦官,反為所害;舞陽君亦為亂兵所殺。并州牧董卓被徵,將兵入洛陽,陵虐朝庭,遂廢少帝為弘農王而立協,是為獻帝。扶弘農王下殿,北面稱臣。太后鯁涕,群臣含悲,莫敢言。董卓又議太后踧迫永樂宮,至令憂死,逆婦姑之禮,乃遷於永安宮,因進酖,弒而崩。在位十年。董卓令帝出奉常亭舉哀,公卿皆白衣會,不成喪也。合葬文昭陵。

初,太后新立,當謁二祖廟,欲齋,輒有變故,如此者數,竟不克。時有識之士心獨怪之,後遂因何氏傾沒漢祚焉。

明年,山東義兵大起,討董卓之亂。卓乃置弘農王於閣上,使郎中令李儒進酖,曰:「服此藥,可以辟惡。」王曰:「我無疾,是欲殺我耳!」不肯飲。強飲之,不得已,乃與妻唐姬及宮人飲讌別。酒行,王悲歌曰:「天道易兮我何艱!棄萬乘兮退守蕃。逆臣見迫兮命不延,逝將去汝兮適幽玄!」因令唐姬起舞,姬抗袖而歌曰:「皇天崩兮后土穨,身為帝兮命夭摧。死生路異兮從此乖,柰我煢獨兮心中哀!」因泣下嗚咽,坐者皆歔欷。王謂姬曰:「卿王者妃,埶不復為吏民妻。自愛,從此長辭!」遂飲藥而死。時年十八。

唐姬,潁川人也。王薨,歸鄉里。父會稽太守瑁欲嫁之,姬誓不許。及李傕破長安,遣兵鈔關東,略得姬。傕因欲妻之,固不聽,而終不自名。尚書賈詡知之,以狀白獻帝。帝聞感愴,乃下詔迎姬,置園中,使侍中持節拜為弘農王妃。

初平元年二月,葬弘農王於故中常侍趙忠成壙中,謚曰懷王。

帝求母王美人兄斌,斌將妻子詣長安,賜第宅田業,拜奉車都尉。

興平元年,帝加元服。有司奏立長秋宮。詔曰:「朕稟受不弘,遭值禍亂,未能紹先,以光故典。皇母前薨,未卜宅兆,禮章有闕,中心如結。三歲之慼,蓋不言吉,且須其後。」於是有司乃奏追尊王美人為靈懷皇后,改葬文昭陵,儀比敬、恭二陵,使光祿大夫持節行司空事奉璽綬,斌與河南尹駱業復土。

斌還,遷執金吾,封都亭侯,食邑五百戶。病卒,贈前將軍印綬,謁者監護喪事。長子端襲爵。

獻帝伏皇后諱壽,琅邪東武人,大司徒湛之八世孫也。父完,沈深有大度,襲爵不其侯,尚桓帝女陽安公主,為侍中。

初平元年,從大駕西遷長安,后時入掖庭為貴人。興平二年,立為皇后,完遷執金吾。帝尋而東歸,李傕、郭汜等追敗乘輿於曹陽,帝乃潛夜度河走,六宮皆步行出營。后手持縑數匹,董承使符節令孫徽以刃脅奪之,殺傍侍者,血濺后衣。既至安邑,御服穿敝,唯以棗栗為糧。建安元年,拜完輔國將軍,儀比三司。完以政在曹操,自嫌尊戚,乃上印綬,拜中散大夫,尋遷屯騎校尉。十四年卒,子典嗣。

自帝都許,守位而已,宿衛兵侍,莫非曹氏黨舊姻戚。議郎趙彥嘗為帝陳言時策,曹操惡而殺之。其餘內外,多見誅戮。操後以事入見殿中,帝不任其憤,因曰:「君若能相輔,則厚;不爾,幸垂恩相捨。」操失色,俛仰求出。舊儀,三公領兵朝見,令虎賁執刃挾之。操出,顧左右,汗流浹背,自後不敢復朝請。董承女為貴人,操誅承而求貴人殺之。帝以貴人有妊,累為請,不能得。后自是懷懼,乃與父完書,言曹操殘逼之狀,令密圖之。完不敢發。至十九年,事乃露泄。操追大怒,遂逼帝廢后,假為策曰:「皇后壽,得由卑賤,登顯尊極,自處椒房,二紀于茲。既無任、姒徽音之美,又乏謹身養己之福,而陰懷妒害,苞藏禍心,弗可以承天命,奉祖宗。今使御史大夫郗慮持節策詔,其上皇后璽綬,退避中宮,遷于它館。嗚呼傷哉!自壽取之,未致于理,為幸多焉。」又以尚書令華歆為郗慮副,勒兵入宮收后。閉戶藏壁中,歆就牽后出。時帝在外殿,引慮於坐。后被髮徒跣行泣過訣曰:「不能復相活邪?」帝曰:「我亦不知命在何時!」顧謂慮曰:「郗公,天下寧有是邪?」遂將后下暴室,以幽崩。所生二皇子,皆酖殺之。后在位二十年,兄弟及宗族死者百餘人,母盈等十九人徙涿郡。

獻穆曹皇后諱節,魏公曹操之中女也。建安十八年,操進三女憲、節、華為夫人,聘以束帛玄纁五萬匹,小者待年於國。十九年,並拜為貴人。及伏皇后被弒,明年,立節為皇后。魏受禪,遣使求璽綬,后怒不與。如此數輩,后乃呼使者入,親數讓之,以璽抵軒下,因涕泣橫流曰:「天不祚爾!」左右皆莫能仰視。后在位七年。魏氏既立,以后為山陽公夫人。自後四十一年,魏景初元年薨,合葬禪陵,車服禮儀皆依漢制。

論曰:漢世皇后無謚,皆因帝謚以為稱。雖呂氏專政,上官臨制,亦無殊號。中興,明帝始建光烈之稱,其後並以德為配,至於賢愚優劣,混同一貫,故馬、竇二后俱稱德焉。其餘唯帝之庶母及蕃王承統,以追尊之重,特為其號,如恭懷、孝崇之比是也。初平中,蔡邕始追正和熹之謚,其安思、順烈以下,皆依而加焉。

贊曰:坤惟厚載,陰正乎內。詩美好逑,易稱歸妹。祁祁皇孋,言觀貞淑。媚茲良哲,承我天祿。班政蘭閨,宣禮椒屋。既云德升,亦曰幸進。身當隆極,族漸河潤。視景爭暉,方山並峻。乘剛多阻,行地必順。咎集驕滿,福協貞信。慶延自己,禍成誰釁。

漢制,皇女皆封縣公主,儀服同列侯。其尊崇者,加號長公主,儀服同蕃王。諸王女皆封鄉、亭公主,儀服同鄉、亭侯。肅宗唯特封東平憲王蒼、琅邪孝王京女為縣公主。其後安帝、桓帝妹亦封長公主,同之皇女。其皇女封公主者,所生之子襲母封為列侯,皆傳國於後。鄉、亭之封,則不傳襲。其職僚品秩,事在百官志。不足別載,故附于后紀末。

皇女義王,建武十五年封舞陽長公主,適延陵鄉侯太僕梁松。松坐誹謗誅。

皇女中禮,十五年封涅陽公主,適顯親侯大鴻臚竇固,肅宗尊為長公主。

皇女紅夫,十五年封館陶公主,適駙馬都尉韓光。光坐與淮陽王延謀反誅。皇女禮劉,十七年封淯陽公主,適陽安侯長樂少府郭璜。璜坐與竇憲謀反誅。

皇女綬,二十一年封酈邑公主,適新陽侯世子陰豐。豐害主,誅死。

皇女姬,永平二年封獲嘉長公主,適楊邑侯將作大匠馮柱。

皇女奴,三年封平陽公主,適大鴻臚馮順。

皇女迎,三年封隆慮公主,適牟平侯耿襲。

皇女次,三年封平氏公主。

皇女致,三年封沁水公主,適高密侯鄧乾。

皇女小姬,十二年封平皋公主,適昌安侯侍中鄧蕃。

皇女仲,十七年封浚儀公主,適軮侯黃門侍郎王度。

皇女惠,十七年封武安公主,適征羌侯世子黃門侍郎來棱,安帝尊為長公主。

皇女臣,建初元年封魯陽公主。

皇女小迎,元年封樂平公主。

皇女小民,元年封成安公主。

皇女男,建初四年封武德長公主。皇女王,四年封平邑公主,適黃門侍郎馮由。

皇女吉,永元五年封陰安公主。

皇女保,延平元年封脩武長公主。

皇女成,元年封共邑公主。

皇女利,元年封臨潁公主。適即墨侯侍中賈建。

皇女興,元年封聞喜公主。

和帝四女。

皇女生,永和三年封舞陽長公主。皇女成男,三年封冠軍長公主。

皇女廣,永和六年封汝陽長公主。

順帝三女。

皇女華,延熹元年封陽安長公主,適不其侯輔國將軍伏完。

皇女堅,七年封潁陰長公主。

皇女脩,九年封陽翟長公主。

桓帝三女。

皇女某,光和三年封萬年公主。

靈帝一女。


\end{pinyinscope}