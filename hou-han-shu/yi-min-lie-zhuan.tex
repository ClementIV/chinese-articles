\article{逸民列傳}

\begin{pinyinscope}
易稱「遯之時義大矣哉」。又曰:「不事王侯,高尚其事。」是以堯稱則天,不屈潁陽之高;武盡美矣,終全孤竹之絜。自茲以降,風流彌繁,長往之軌未殊,而感致之數匪一。或隱居以求其志,或回避以全其道,或靜己以鎮其躁,或去危以圖其安,或垢俗以動其概,或疵物以激其清。然觀其甘心畎畝之中,憔悴江海之上,豈必親魚鳥樂林草哉,亦云性分所至而已。故蒙恥之賓,屢黜不去其國;蹈海之節,千乘莫移其情。適使矯易去就,則不能相為矣。彼雖硜硜有類沽名者,然而蟬蛻囂埃之中,自致寰區之外,異夫飾智巧以逐浮利者乎!荀卿有言曰,「志意脩則驕富貴,道義重則輕王公」也。

漢室中微,王莽篡位,士之蘊藉義憤甚矣。是時裂冠毀冕,相攜持而去之者,蓋不可勝數。楊雄曰:「鴻飛冥冥,弋者何篡焉。」言其違患之遠也。光武側席幽人,求之若不及,旌帛蒲車之所徵賁,相望於巖中矣。若薛方、逢萌聘而不肯至,嚴光、周黨、王霸至而不能屈。群方咸遂,志士懷仁,斯固所謂「舉逸民天下歸心」者乎!肅宗亦禮鄭均而徵高鳳,以成其節。自後帝德稍衰,邪孽當朝,處子耿介,羞與卿相等列,至乃抗憤而不顧,多失其中行焉。蓋錄其絕塵不反,同夫作者,列之此篇。

野王二老者,不知何許人也。初,光武貳於更始,會關中擾亂,遣前將軍鄧禹西征,送之於道。既反,因於野王獵,路見二老者即禽。光武問曰:「禽何向?」並舉手西指,言「此中多虎,臣每即禽,虎亦即臣,大王勿往也」。光武曰:「苟有其備,虎亦何患。」父曰:「何大王之謬邪!昔湯即桀於鳴條,而大城於亳;武王亦即紂於牧野,而大城於郟鄏。彼二王者,其備非不深也。是以即人者,人亦即之,雖有其備,庸可忽乎!」光武悟其旨,顧左右曰:「此隱者也。」將用之,辭而去,莫知所在。

向長字子平,河內朝歌人也。隱居不仕,性尚中和,好通老、易。貧無資食,好事者更饋焉,受之取足而反其餘。王莽大司空王邑辟之,連年乃至,欲薦之於莽,固辭乃止。潛隱於家。讀易至損、益卦,喟然歎曰:「吾已知富不如貧,貴不如賤,但未知死何如生耳。」建武中,男女娶嫁既畢,敕斷家事勿相關,當如我死也。於是遂肆意,與同好北海禽慶俱遊五嶽名山,竟不知所終。

逢萌字子康,北海都昌人也。家貧,給事縣為亭長。時尉行過亭,萌候迎拜謁,既而擲楯歎曰:「大丈夫安能為人役哉!」遂去之長安學,通春秋經。時王莽殺其子宇,萌謂友人曰:「三綱絕矣!不去,禍將及人。」即解冠挂東都城門,歸,將家屬浮海,客於遼東。

萌素明陰陽,知莽將敗,有頃,乃首戴瓦盎,哭於巿曰:「新乎新乎!」因遂潛藏。

及光武即位,乃之琅邪勞山,養志脩道,人皆化其德。

北海太守素聞其高,遣吏奉謁致禮,萌不荅。太守懷恨而使捕之。吏叩頭曰:「子康大賢,天下共聞,所在之處,人敬如父,往必不獲,祇自毀辱。」太守怒,收之繫獄,更發它吏。行至勞山,人果相率以兵弩捍禦,吏被傷流血,奔而還。後詔書徵萌,託以老耄,迷路東西,語使者云:「朝廷所以徵我者,以其有益於政,尚不知方面所在,安能濟時乎?」即便駕歸。連徵不起,以壽終。

初,萌與同郡徐房、平原李子雲、王君公相友善,並曉陰陽,懷德穢行。房與子雲養徒各千人,君公遭亂獨不去,儈牛自隱。時人謂之論曰:「避世牆東王君公。」

周黨字伯況,太原廣武人也。家產千金。少孤,為宗人所養,而遇之不以理,及長,又不還其財。黨詣鄉縣訟,主乃歸之。既而散與宗族,悉免遣奴婢,遂至長安遊學。

初,鄉佐嘗眾中辱黨,黨久懷之。後讀春秋,聞復讎之義,便輟講而還,與鄉佐相聞,期剋鬥日。既交刃,而黨為鄉佐所傷,困頓。鄉佐服其義,輿歸養之,數日方蘇,既悟而去。自此敕身脩志,州里稱其高。

及王莽竊位,託疾杜門。自後賊暴從橫,殘滅郡縣,唯至廣武,過城不入。

建武中,徵為議郎,以病去職,遂將妻子居黽池。復被徵,不得已,乃著短布單衣,榖皮綃頭,待見尚書。及光武引見,黨伏而不謁,自陳願守所志,帝乃許焉。

博士范升奏毀黨曰:「臣聞堯不須許由、巢父,而建號天下;周不待伯夷、叔齊,而王道以成。伏見太原周黨、東海王良、山陽王成等,蒙受厚恩,使者三聘,乃肯就車。及陛見帝廷,黨不以禮屈,伏而不謁,偃蹇驕悍,同時俱逝。黨等文不能演義,武不能死君,釣采華名,庶幾三公之位。臣願與坐雲臺之下,考試圖國之道。不如臣言,伏虛妄之罪。而敢私竊虛名,誇上求高,皆大不敬。」書奏,天子以示公卿。詔曰:「自古明王聖主必有不賓之士。伯夷、叔齊不食周粟,太原周黨不受朕祿,亦各有志焉。其賜帛四十匹。」黨遂隱居黽池,著書上下篇而終。邑人賢而祠之。

初,黨與同郡譚賢伯升、鴈門殷謨君長,俱守節不仕王莽世。建武中,徵並不到。

王霸字儒仲,太原廣武人也。少有清節。及王莽篡位,棄冠帶,絕交宦。建武中,徵到尚書,拜稱名,不稱臣。有司問其故。霸曰:「天子有所不臣,諸侯有所不友。」司徒侯霸讓位於霸。閻陽毀之曰:「太原俗黨,儒仲頗有其風。」遂止。以病歸。隱居守志,茅屋蓬戶。連徵不至,以壽終。

嚴光字子陵,一名遵,會稽餘姚人也。少有高名,與光武同遊學。及光武即位,乃變名姓,隱身不見。帝思其賢,乃令以物色訪之。後齊國上言:「有一男子,披羊裘釣澤中。」帝疑其光,乃備安車玄纁,遣使聘之。三反而後至。舍於北軍,給床褥,太官朝夕進膳。

司徒侯霸與光素舊,遣使奉書。使人因謂光曰:「公聞先生至,區區欲即詣造,迫於典司,是以不獲。願因日暮,自屈語言。」光不荅,乃投札與之,口授曰:「君房足下:位至鼎足,甚善。懷仁輔義天下悅,阿諛順旨要領絕。」霸得書,封奏之。帝笑曰:「狂奴故態也。」車駕即日幸其館。光臥不起,帝即其臥所,撫光腹曰:「咄咄子陵,不可相助為理邪?」光又眠不應,良久,乃張目熟視,曰:「昔唐堯著德,巢父洗耳。士故有志,何至相迫乎!」帝曰:「子陵,我竟不能下汝邪?」於是升輿歎息而去。

復引光入,論道舊故,相對累日。帝從容問光曰:「朕何如昔時?」對曰:「陛下差增於往。」因共偃臥,光以足加帝腹上。明日,太史奏客星犯御坐甚急。帝笑曰:「朕故人嚴子陵共臥耳。」

除為諫議大夫,不屈,乃耕於富春山,後人名其釣處為嚴陵瀨焉。建武十七年,復特徵,不至。年八十,終於家。帝傷惜之,詔下郡縣賜錢百萬、穀千斛。

井丹字大春,扶風郿人也。少受業太學,通五經,善談論,故京師為之語曰:「五經紛綸井大春。」性清高,未嘗脩刺候人。

建武末,沛王輔等五王居北宮,皆好賓客,更遣請丹,不能致。信陽侯陰就,光烈皇后弟也,以外戚貴盛,乃詭說五王,求錢千萬,約能致丹,而別使人要劫之。丹不得已,既至,就故為設麥飯蔥葉之食,丹推去之,曰:「以君侯能供甘旨,故來相過,何其薄乎?」更置盛饌,乃食。及就起,左右進輦。丹笑曰:「吾聞桀駕人車,豈此邪?」坐中皆失色。就不得已而令去輦。自是隱閉不關人事,以壽終。

梁鴻字伯鸞,扶風平陵人也。父讓,王莽時為城門校尉,封脩遠伯,使奉少昊後,寓於北地而卒。鴻時尚幼,以遭亂世,因卷席而葬。

後受業太學,家貧而尚節介,博覽無不通,而不為章句。學畢,乃牧豕於上林苑中。曾誤遺火延及它舍,鴻乃尋訪燒者,問所去失,悉以豕償之。其主猶以為少。鴻曰:「無它財,願以身居作。」主人許之。因為執勤,不懈朝夕。鄰家耆老見鴻非恆人,乃共責讓主人,而稱鴻長者。於是始敬異焉,悉還其豕。鴻不受而去,歸鄉里。

埶家慕其高節,多欲女之,鴻並絕不娶。同縣孟氏有女,狀肥醜而黑,力舉石臼,擇對不嫁,至年三十。父母問其故。女曰:「欲得賢如梁伯鸞者。」鴻聞而娉之。女求作布衣、麻屨,織作筐緝績之具。及嫁,始以裝飾入門。七日而鴻不荅。妻乃跪床下請曰:「竊聞夫子高義,簡斥數婦,妾亦偃蹇數夫矣。今而見擇,敢不請罪。」鴻曰:「吾欲裘褐之人,可與俱隱深山者爾。今乃衣綺縞,傅粉墨,豈鴻所願哉?」妻曰:「以觀夫子之志耳。妾自有隱居之服。」乃更為椎髻,著布衣,操作而前。鴻大喜曰:「此真梁鴻妻也。能奉我矣!」字之曰德曜,孟光。

居有頃,妻曰:「常聞夫子欲隱居避患,今何為默默?無乃欲低頭就之乎?」鴻曰:「諾。」乃共入霸陵山中,以耕織為業,詠詩書,彈琴以自娛。仰慕前世高士,而為四皓以來二十四人作頌。

因東出關,過京師,作五噫之歌曰:「陟彼北芒兮,噫!顧覽帝京兮,噫!宮室崔嵬兮,噫!人之劬勞兮,噫!遼遼未央兮,噫!」肅宗聞而非之,求鴻不得。乃易姓運期,名燿,字侯光,與妻子居齊魯之閒。

有頃,又去適吳。將行,作詩曰:「逝舊邦兮遐征,將遙集兮東南。心惙怛兮傷悴,志菲菲兮升降。欲乘策兮縱邁,疾吾俗兮作讒。競舉枉兮措直,咸先佞兮唌唌。聊固靡慚兮獨建,冀異州兮尚賢。聊逍搖兮遨嬉,纘仲尼兮周流。儻云睹兮我悅,遂舍車兮即浮。過季札兮延陵,求魯連兮海隅。雖不察兮光貌,幸神靈兮與休。惟季春兮華阜,麥含含兮方秀。哀茂時兮逾邁,愍芳香兮日臭。悼吾心兮不獲,長委結兮焉究!口囂囂兮余訕,嗟恇恇兮誰留?」

遂至吳,依大家皋伯通,居廡下,為人賃舂。每歸,妻為具食,不敢於鴻前仰視,舉案齊眉。伯通察而異之,曰:「彼傭能使其妻敬之如此,非凡人也。」乃方舍之於家。鴻潛閉著書十餘篇。疾且困,告主人曰:「昔延陵季子葬子於嬴博之閒,不歸鄉里,慎勿令我子持喪歸去。」及卒,伯通等為求葬地於吳要離冢傍。咸曰:「要離烈士,而伯鸞清高,可令相近。」葬畢,妻子歸扶風。

初,鴻友人京兆高恢,少好老子,隱於華陰山中。及鴻東遊思恢,作詩曰:「鳥嚶嚶兮友之期,念高子兮僕懷思,想念恢兮爰集茲。」二人遂不復相見。恢亦高抗,終身不仕。

高鳳字文通,南陽葉人也。少為書生,家以農畝為業,而專精誦讀,晝夜不息。妻嘗之田,曝麥於庭,令鳳護雞。時天暴雨,而鳳持竿誦經,不覺潦水流麥。妻還怪問,鳳方悟之。其後遂為名儒,乃教授業於西唐山中。

鄰里有爭財者,持兵而鬥,鳳往解之,不已,乃脫巾叩頭,固請曰:「仁義遜讓,柰何棄之!」於是爭者懷感,投兵謝罪。

鳳年老,執志不倦,名聲著聞。太守連召請,恐不得免,自言本巫家,不應為吏,又詐與寡嫂訟田,遂不仕。建初中,將作大匠任隗舉鳳直言,到公車,託病逃歸。推其財產,悉與孤兄子。隱身漁釣,終於家。

論曰:先大夫宣侯嘗以講道餘隙,寓乎逸士之篇。至高文通傳,輟而有感,以為隱者也,因著其行事而論之曰:「古者隱逸,其風尚矣。潁陽洗耳,恥聞禪讓;孤竹長飢,羞食周粟。或高棲以違行,或疾物以矯情,雖軌跡異區,其去就一也。若伊人者,志陵青雲之上,身晦泥汙之下,心名且猶不顯,況怨累之為哉!與夫委體淵沙,鳴弦揆日者,不其遠乎!」

臺佟字孝威,魏郡鄴人也。隱於武安山,鑿穴為居,采藥自業。建初中,州辟不就。刺史行部,乃使從事致謁。佟載病往謝。刺史乃執贄見佟曰:「孝威居身如是,甚苦,如何?」佟曰:「佟幸得保終性命,存神養和。如明使君奉宣詔書,夕惕庶事,反不苦邪?」遂去,隱逸,終不見。

韓康字伯休,一名恬休,京兆霸陵人。家世著姓。常采藥名山,賣於長安市,口不二價,三十餘年。時有女子從康買藥,康守價不移。女子怒曰:「公是韓伯休那?乃不二價乎?」康歎曰:「我本欲避名,今小女子皆知有我,何用藥為?」乃遯入霸陵山中。博士公車連徵不至。桓帝乃備玄纁之禮,以安車聘之。使者奉詔造康,康不得已,乃許諾。辭安車,自乘柴車,冒晨先使者發。至亭,亭長以韓徵君當過,方發人牛脩道橋。及見康柴車幅巾,以為田叟也,使奪其牛。康即釋駕與之。有頃,使者至,奪牛翁乃徵君也。使者欲奏殺亭長。康曰:「此自老子與之,亭長何罪!」乃止。康因道逃遯,以壽終。

矯慎字仲彥,扶風茂陵人也。少好黃老,隱遯山谷,因穴為室,仰慕松、喬導引之術。與馬融、蘇章鄉里並時,融以才博顯名,章以廉直稱,然皆推先於慎。

汝南吳蒼甚重之,因遺書以觀其志曰:「仲彥足下:勤處隱約,雖乘雲行泥,棲宿不同,每有西風,何嘗不歎!蓋聞黃老之言,乘虛入冥,藏身遠遯,亦有理國養人,施於為政。至如登山絕跡,神不著其證,人不睹其驗。吾欲先生從其可者,於意何如?昔伊尹不懷道以待堯舜之君。方今明明,四海開闢,巢許無為箕山,夷齊悔入首陽。足下審能騎龍弄鳳,翔嬉雲閒者,亦非狐兔燕雀所敢謀也。」慎不荅。年七十餘,竟不肯娶。後忽歸家,自言死日,及期果卒。後人有見慎於敦煌者,故前世異之,或云神僊焉。

慎同郡馬瑤,隱於汧山,以兔罝為事。所居俗化,百姓美之,號馬牧先生焉。

戴良字叔鸞,汝南慎陽人也。曾祖父遵,字子高,平帝時,為侍御史。王莽篡位,稱病歸鄉里。家富,好給施,尚俠氣,食客常三四百人。時人為之語曰:「關東大豪戴子高。」

良少誕節,母憙驢鳴,良常學之以娛樂焉。及母卒,兄伯鸞居廬啜粥,非禮不行,良獨食肉飲酒,哀至乃哭,而二人俱有毀容。或問良曰:「子之居喪,禮乎?」良曰:「然。禮所以制情佚也,情苟不佚,何禮之論!夫食旨不甘,故致毀容之實。若味不存口,食之可也。」論者不能奪之。

良才既高達,而論議尚奇,多駭流俗。同郡謝季孝問曰:「子自視天下孰可為比?」良曰:「我若仲尼長東魯,大禹出西羌,獨步天下,誰與為偶!」

舉孝廉,不就。再辟司空府,彌年不到,州郡迫之,乃遯辭詣府,悉將妻子,既行在道,因逃入江夏山中。優遊不仕,以壽終。

初,良五女並賢,每有求姻,輒便許嫁,疏裳布被,竹笥木屐以遣之。五女能遵其訓,皆有隱者之風焉。

法真字高卿,扶風郿人,南郡太守雄之子也。好學而無常家,博通內外圖典,為關西大儒。弟子自遠方至者,陳留范冉等數百人。

性恬靜寡欲,不交人閒事。太守請見之,真乃幅巾詣謁。太守曰:「昔魯哀公雖為不肖,而仲尼稱臣。太守虛薄,欲以功曹相屈,光贊本朝,何如?」真曰:「以明府見待有禮,故敢自同賓末。若欲吏之,真將在北山之北,南山之南矣。」太守戄然,不敢復言。

辟公府,舉賢良,皆不就。同郡田弱薦真曰:「處士法真,體兼四業,學窮典奧,幽居恬泊,樂以忘憂,將蹈老氏之高蹤,不為玄纁屈也。臣願聖朝就加袞職,必能唱清廟之歌,致來儀之鳳矣。」會順帝西巡,弱又薦之。帝虛心欲致,前後四徵。真曰:「吾既不能遯形遠世,豈飲洗耳之水哉?」遂深自隱絕,終不降屈。友人郭正稱之曰:「法真名可得聞,身難得而見,逃名而名我隨,避名而名我追,可謂百世之師者矣!」乃共刊石頌之,號曰玄德先生。年八十九,中平五年,以壽終。

漢陰老父者,不知何許人也。桓帝延熹中,幸竟陵,過雲夢,臨沔水,百姓莫不觀者,有老父獨耕不輟。尚書郎南陽張溫異之,使問曰:「人皆來觀,老父獨不輟,何也?」老父笑而不對。溫下道百步,自與言。老父曰:「我野人耳,不達斯語。請問天下亂而立天子邪?理而立天子邪?立天子以父天下邪?役天下以奉天子邪?昔聖王宰世,茅茨采椽,而萬人以寧。今子之君,勞人自縱,逸遊無忌。吾為子羞之,子何忍欲人觀之乎!」溫大慚。問其姓名,不告而去。

陳留老父者,不知何許人也。桓帝世,黨錮事起,守外黃令陳留張升去官歸鄉里,道逢友人,共班草而言。升曰:「吾聞趙殺鳴犢,仲尼臨河而反;覆巢竭淵,龍鳳逝而不至。今宦豎日亂,陷害忠良,賢人君子其去朝乎?夫德之不建,人之無援,將性命之不免,柰何?」因相抱而泣。老父趨而過之,植其杖,太息言曰:「吁!二大夫何泣之悲也?夫龍不隱鱗,鳳不藏羽,網羅高縣,去將安所?雖泣何及乎!」二人欲與之語,不顧而去,莫知所終。

龐公者,南郡襄陽人也。居峴山之南,未嘗入城府。夫妻相敬如賓。荊州刺史劉表數延請,不能屈,乃就候之。謂曰:「夫保全一身,孰若保全天下乎?」龐公笑曰:「鴻鵠巢於高林之上,暮而得所栖;黿鼉穴於深淵之下,夕而得所宿。夫趣舍行止,亦人之巢穴也。且各得其栖宿而已,天下非所保也。」因釋耕於壟上,而妻子耘於前。表指而問曰:「先生苦居畎畝而不肯官祿,後世何以遺子孫乎?」龐公曰:「世人皆遺之以危,今獨遺之以安,雖所遺不同,未為無所遺也。」表歎息而去。後遂攜其妻子登鹿門山,因采藥不反。

贊曰:江海冥滅,山林長往。遠性風疏,逸情雲上。道就虛全,事違塵枉。


\end{pinyinscope}