\article{孝明八王列傳}

\begin{pinyinscope}
孝明皇帝九子:賈貴人生章帝;陰貴人生梁節王暢

;餘七王本書不載母氏。

千乘哀王建,永平三年封。明年薨。年少無子,國除。

陳敬王羨,永平三年封廣平王。建初三年,有司奏遣羨與鉅鹿王恭、樂成王黨俱就國。肅宗性篤愛,不忍與諸王乖離,遂皆留京師。明年,案輿地圖,令諸國戶口皆等,租入歲各八千萬。羨博涉經書,有威嚴,與諸儒講論於白虎殿。七年,帝以廣平在北,多有邊費,乃徙羨為西平王,分汝南八縣為國。及帝崩,遺詔徙封為陳王,食淮陽郡,其年就國。立三十七年薨,子思王鈞嗣。

鈞立,多不法,遂行天子大射禮。性隱賊,喜文法,國相二千石不與相得者,輒陰中之。憎怨敬王夫人李儀等,永元十一年,遂使客隗久殺儀家屬。吏捕得久,繫長平獄。鈞欲斷絕辭語,復使結客篡殺久。事發覺,有司舉奏,鈞坐削西華、項、新陽三縣。十二年,封鈞六弟為列侯。後鈞取掖庭出女李嬈為小妻,復坐削圉、宜祿、扶溝三縣。永初七年,封敬王孫安國為耕亭侯。

鈞立二十一年薨,子懷王竦嗣。立二年薨,無子,國絕。

永寧元年,立敬王子安壽亭侯崇為陳王,是為頃王。立五年薨,子孝王承嗣。

承薨,子愍王寵嗣。熹平二年,國相師遷追奏前相魏愔與寵共祭天神,希幸非冀,罪至不道。有司奏遣使者案驗。是時新誅勃海王悝,靈帝不忍復加法,詔檻車傳送愔、遷詣北寺詔獄,使中常侍王酺與尚書令、侍御史雜考。愔辭與王共祭黃老君,求長生福而已,無它冀幸。酺等奏愔職在匡正,而所為不端,遷誣告其王,罔以不道,皆誅死。有詔赦寵不案。

寵善弩射,十發十中,中皆同處。中平中,黃巾賊起,郡縣皆棄城走,寵有彊弩數千張,出軍都亭。國人素聞王善射,不敢反叛,故陳獨得完,百姓歸之者眾十餘萬人。及獻帝初,義兵起,寵率眾屯陽夏,自稱輔漢大將軍。國相會稽駱俊素有威恩,時天下飢荒,鄰郡人多歸就之,俊傾資賑贍,並得全活。後袁術求糧於陳而俊拒絕之,術忿恚,遣客詐殺俊及寵,陳由是破敗。

是時諸國無復租祿,而數見虜奪,并日而食,轉死溝壑者甚眾。夫人姬妾多為丹陽兵烏桓所略云。

彭城靖王恭,永平九年賜號靈壽王。十五年,封為鉅鹿王。建初三年,徙封江陵王,改南郡為國。元和二年,三公上言江陵在京師正南,不可以封,乃徙為六安王,以廬江郡為國。肅宗崩,遺詔徙封彭城王,食楚郡,其年就國。恭敦厚威重,舉動有節度,吏人敬愛之。永初六年,封恭子阿奴為竹邑侯。

元初三年,恭以事怒子酺,酺自殺。國相趙牧以狀上,因誣奏恭祠祀惡言,大逆不道。有司奏請誅之。恭上書自訟。朝廷以其素著行義,令考實,無徵,牧坐下獄,會赦免死。

恭立四十六年薨,子考王道嗣。元初五年,封道弟三人為鄉侯,恭孫順為東安亭侯。

道立二十八年薨,子頃王定嗣。本初元年,封定兄弟九人皆為亭侯。

定立四年薨,子孝王和嗣。和性至孝,太夫人薨,行喪陵次,毀胔過禮。傅相以聞。桓帝詔使奉牛酒迎王還宮。和敬賢樂施,國中愛之。初平中,天下大亂,和為賊昌務所攻,避奔東阿,後得還國。

立六十四年薨,孫祗嗣。立七年,魏受禪,以為崇德侯。

樂成靖王黨,永平九年賜號重熹王,十五年封樂成王。黨聰惠,善史書,喜正文字。與肅宗同年,尤相親愛。建初四年,以清河之游、觀津,勃海之東光、成平,涿郡之中水、饒陽、安平、南深澤八縣益樂成國。及帝崩,其年就國。黨急刻不遵法度。舊禁宮人出嫁,不得適諸國。有故掖庭技人哀置,嫁為男子章初妻,黨召哀置入宮與通,初欲上書告之,黨恐懼,乃密賂哀置姊焦使殺初。事發覺,黨乃縊殺內侍三人,以絕口語。又取故中山簡王傅婢李羽生為小妻。永元七年,國相舉奏之。和帝詔削東光、鄡二縣。

立二十五年薨,子哀王崇嗣。立二月薨,無子,國絕。

明年,和帝立崇兄脩侯巡為樂成王,是為釐王。立十五年薨,子隱王賓嗣。立八年薨,無子,國絕。

明年,復立濟北惠王子萇為樂成王後。萇到國數月,驕淫不法,愆過累積,冀州刺史與國相舉奏萇罪至不道。安帝詔曰:「萇有靦其面,而放逸其心。知陵廟至重,承繼有禮,不惟致敬之節,肅穆之慎,乃敢擅損犧牲,不備苾芬。慢易大姬,不震厥教。出入顛覆,風淫于家,娉取人妻,饋遺婢妾。毆擊吏人,專己凶暴。愆罪莫大,甚可恥也。朕覽八辟之議,不忍致之于理。其貶萇爵為臨湖侯。朕無『則哲』之明,致簡統失序,罔以尉承大姬,增懷永歎。」

延光元年,以河閒孝王子得嗣靖王後。以樂成比廢絕,故改國曰安平,是為安平孝王。

立三十年薨,子續立。中平元年,黃巾賊起,為所劫質,囚于廣宗。賊平復國。其年秋,坐不道被誅。立三十四年,國除。

下邳惠王衍,永平十五年封。衍有容貌,肅宗即位,常在左右。建初初冠,詔賜衍師傅已下官屬金帛各有差。四年,以臨淮郡及九江之鍾離、當塗、東城、歷陽、全椒合十七縣益下邳國。帝崩,其年就國。衍後病荒忽,而太子卬有罪廢,諸姬爭欲立子為嗣,連上書相告言。和帝憐之,使彭城靖王恭至下邳正其嫡庶,立子成為太子。

衍立五十四年薨,子貞王成嗣。永建元年,封成兄二人及惠王孫二人皆為列侯。

成立二年薨,子愍王意嗣。陽嘉元年,封意弟八人為鄉、亭侯。中平元年,意遭黃巾,棄國走。賊平復國,數月薨。立五十七年,年九十。

子哀王宜嗣,數月薨,無子,建安十一年國除。

梁節王暢,永平十五年封為汝南王。母陰貴人有寵,暢尤被愛幸,國土租入倍於諸國。肅宗立,緣先帝之意,賞賜恩寵甚篤。建初二年,封暢舅陰棠為西陵侯。四年,徙為梁王,以陳留之郾、寧陵,濟陰之薄、單父、己氏、成武,凡六縣,益梁國。帝崩,其年就國。

暢性聰惠,然少貴驕,頗不遵法度。歸國後,數有惡夢,從官卞忌自言能使六丁,善占夢,暢數使卜筮。又暢乳母王禮等,因此自言能見鬼神事,遂共占氣,祠祭求福。忌等諂媚,云神言王當為天子。暢心喜,與相應荅。永元五年,豫州刺史梁相舉奏暢不道,考訊,辭不服。有司請徵暢詣廷尉詔獄,和帝不許。有司重奏除暢國,徙九真,帝不忍,但削成武、單父二縣。暢慚懼,上疏辭謝曰:「臣天性狂愚,生在深宮,長養傅母之手,信惑左右之言。及至歸國,不知防禁。從官侍史利臣財物,熒惑臣暢。臣暢無所昭見,與相然諾,不自知陷死罪,以至考案。肌慄心悸,自悔無所復及。自謂當即時伏顯誅,魂魄去身,分歸黃泉。不意陛下聖德,枉法曲平,不聽有司,橫貸赦臣。戰慄連月,未敢自安。上念以負先帝而令陛下為臣收汙天下,誠無氣以息,筋骨不相連。臣暢知大貸不可再得,自誓束身約妻子,不敢復出入失繩墨,不敢復有所橫費。租入有餘,乞裁食睢陽、穀孰、虞、蒙、寧陵五縣,還餘所食四縣。臣暢小妻三十七人,其無子者願還本家。自選擇謹敕奴婢二百人,其餘所受虎賁、官騎及諸工技、鼓吹、倉頭、奴婢、兵弩、廄馬皆上還本署。臣暢以骨肉近親,亂聖化,汙清流,既得生活,誠無心面目以凶惡復居大宮,食大國,張官屬,藏什物。願陛下加大恩,開臣自悔之門,假臣小善之路,令天下知臣蒙恩,得去死就生,頗能自悔。臣以公卿所奏臣罪惡詔書常置於前,晝夜誦讀。臣小人,貪見明時,不能即時自引,惟陛下哀臣,令得喘息漏刻。若不聽許,臣實無顏以久生,下入黃泉,無以見先帝。此誠臣至心。臣欲多還所受,恐天恩不聽許,節量所留,於臣暢饒足。」詔報曰:「朕惟王至親之屬,淳淑之美,傅相不良,不能防邪,至令有司紛紛有言。今王深思悔過,端自克責,朕惻然傷之。志匪由于,咎在彼小子。一日克己復禮,天下歸仁。王其安心靜意,茂率休德。易不云乎:『一謙而四益。小有言,終吉。』強食自愛。」暢固讓,章數上,卒不許。

立二十七年薨,子恭王堅嗣。永元十六年,封堅弟二人為鄉、亭侯。

堅立二十六年薨,子懷王匡嗣。永建二年,封匡兄弟七人為鄉、亭侯。

匡立十一年薨,無子,順帝封匡弟孝陽亭侯成為梁王,是為夷王。

立二十九年薨,子敬王元嗣。

立十六年薨,子彌嗣。立四十年,魏受禪,以為崇德侯。

淮陽頃王疠,永平〈十〉五年封常山王,建初四年,徙為淮陽王,以汝南之新安、西華益淮陽國。

立十六年薨,未及立嗣,永元二年,和帝立疠小子側復為常山王,奉疠後,是為殤王。

立十三年薨,父子皆未之國,並葬京師。側無子,其月立兄防子侯章為常山王。和帝憐章早孤,數加賞賜。延平元年就國。

立二十五年薨,是為靖王。子頃王儀嗣。永建二年,封儀兄二人為亭侯。

儀立十七年薨,子節王豹嗣。永嘉元年,封豹兄四人為亭侯。

豹立八年薨,子暠嗣。三十二年,遭黃巾賊,棄國走,建安十一年國除。

濟陰悼王長,永平十五年封。建初四年,以東郡之離狐、陳留之長垣益濟陰國。立十三年,薨于京師,無子,國除。

論曰:晏子稱「夫人生厚而用利,於是乎正德以幅之,謂之幅利」。言人情須節以正其德,亦由布帛須幅以成其度焉。明帝封諸子,租歲不過二千萬,馬后為言而不得也。賢哉!豈徒儉約而已乎!知驕貴之無猒,嗜欲之難極也,故東京諸侯鮮有至於禍敗者也。

贊曰:孝明傳胤,維城八國。陳敬嚴重,彭城厚德。下邳嬰痾,梁節邪惑。三藩夙齡,黨惟荒忒。


\end{pinyinscope}