\article{黨錮列傳}

\begin{pinyinscope}
孔子曰:「性相近也,習相遠也。」言嗜惡之本同,而遷染之塗異也。夫刻意則行不肆,牽物則其志流。是以聖人導人理性,裁抑宕佚,慎其所與,節其所偏,雖情品萬區,質文異數,至於陶物振俗,其道一也。叔末澆訛,王道陵缺,而猶假仁以效己,憑義以濟功。舉中於理,則強梁裏氣;片言違正,則冢臺解情。蓋前哲之遺塵,有足求者。

霸德既衰,狙詐萌起。彊者以決勝為雄,弱者以詐劣受屈。至有畫半策而綰萬金,開一說而錫琛瑞。或起徒步而仕執珪,解草衣以升卿相。士之飾巧馳辯,以要能釣利者,不期而景從矣。自是愛尚相奪,與時回變,其風不可留,其敝不能反。

及漢祖杖劍,武夫绗興,憲令寬賒,文禮簡闊,緒餘四豪之烈,人懷陵上之心,輕死重氣,怨惠必讎,令行私庭,權移匹庶,任俠之方,成其俗矣。自武帝以後,崇尚儒學,懷經協術,所在霧會,至有石渠分爭之論,黨同伐異之說,守文之徒,盛於時矣。至王莽專偽,終於篡國,忠義之流,恥見纓紼,遂乃榮華丘壑,甘足枯槁。雖中興在運,漢德重開,而保身懷方,彌相慕襲,去就之節,重於時矣。逮桓靈之閒,主荒政繆,國命委於閹寺,士子羞與為伍,故匹夫抗憤,處士橫議,遂乃激揚名聲,互相題拂,品覈公卿,裁量執政,婞直之風,於斯行矣。

夫上好則下必甚,矯枉故直必過,其理然矣。若范滂、張儉之徒,清心忌惡,終陷黨議,不其然乎?

初,桓帝為蠡吾侯,受學於甘陵周福,及即帝位,擢福為尚書。時同郡河南尹房植有名當朝,鄉人為之謠曰:「天下規矩房伯武,因師獲印周仲進。」二家賓客,互相譏揣,遂各樹朋徒,漸成尤隙,由是甘陵有南北部,黨人之議,自此始矣。後汝南太守宗資任功曹范滂,南陽太守成档亦委功曹岑晊,二郡又為謠曰:「汝南太守范孟博,南陽宗資主畫諾。南陽太守岑公孝,弘農成档但坐嘯。」因此流言轉入太學,諸生三萬餘人,郭林宗、賈偉節為其冠,並與李膺、陳蕃、王暢更相褒重。學中語曰:「天下模楷李元禮,不畏強禦陳仲舉,天下俊秀王叔茂。」又渤海公族進階、扶風魏齊卿,並危言深論,不隱豪強。自公卿以下,莫不畏其貶議,屣履到門。

時河內張成善說風角,推占當赦,遂教子殺人。李膺為河南尹,督促收捕,既而逢宥獲免,膺愈懷憤疾,竟案殺之。初,成以方伎交通宦官,帝亦頗誶其占。成弟子牢脩因上書誣告膺等養太學遊士,交結諸郡生徒,更相驅馳,共為部黨,誹訕朝廷,疑亂風俗。於是天子震怒,班下郡國,逮捕黨人,布告天下,使同忿疾,遂收執膺等。其辭所連及陳寔之徒二百餘人,或有逃遁不獲,皆懸金購募。使者四出,相望於道。明年,尚書霍諝、城門校尉竇武並表為請,帝意稍解,乃皆赦歸田里,禁錮終身。而黨人之名,猶書王府。

自是正直廢放,邪枉熾結,海內希風之流,遂共相摽搒,指天下名士,為之稱號。上曰「三君」,次曰「八俊」,次曰「八顧」,次曰「八及」,次曰「八廚」,猶古之「八元」、「八凱」也。竇武、劉淑、陳蕃為「三君」。君者,言一世之所宗也。李膺、荀翌、杜密、王暢、劉祐、魏朗、趙典、朱宇為「八俊」。俊者,言人之英也。郭林宗、宗慈、巴肅、夏馥、范滂、尹勳、蔡衍、羊陟為「八顧」。顧者,言能以德行引人者也。張儉、岑晊、劉表、陳翔、孔昱、苑康、檀敷、翟超為「八及」。及者,言其能導人追宗者也。度尚、張邈、王考、劉儒、胡母班、秦周、蕃嚮、王章為「八廚」。廚者,言能以財救人者也。

又張儉鄉人朱並,承望中常侍侯覽意旨,上書告儉與同鄉二十四人別相署號,共為部黨,圖危社稷。以儉及檀彬、褚鳳、張肅、薛蘭、馮禧、魏玄、徐乾為「八俊」,田林、張隱、劉表、薛郁、王訪、劉祗、宣靖、公緒恭為「八顧」,朱楷、田槃、疏耽、薛敦、宋布、唐龍、嬴咨、宣褒為「八及」,刻石立墠,共為部黨,而儉為之魁。靈帝詔刊章捕儉等。大長秋曹節因此諷有司奏捕前黨故司空虞放、太僕杜密、長樂少府李膺、司隸校尉朱宇、潁川太守巴肅、沛相荀翌、河內太守魏朗、山陽太守翟超、任城相劉儒、太尉掾范滂等百餘人,皆死獄中。餘或先歿不及,或亡命獲免。自此諸為怨隙者,因相陷害,睚眥之忿,濫入黨中。又州郡承旨,或有未嘗交關,亦離禍毒。其死徙廢禁者,六七百人。

熹平五年,永昌太守曹鸞上書大訟黨人,言甚方切。帝省奏大怒,即詔司隸、益州檻車收鸞,送槐里獄掠殺之。於是又詔州郡更考黨人門生故吏父子兄弟,其在位者,免官禁錮,爰及五屬。

光和二年,上祿長和海上言:「禮,從祖兄弟別居異財,恩義已輕,服屬疏末。而今黨人錮及五族,既乖典訓之文,有謬經常之法。」帝覽而悟之,黨錮自從祖以下,皆得解釋。

中平元年,黃巾賊起,中常侍呂彊言於帝曰:「黨錮久積,人情多怨。若久不赦宥,輕與張角合謀,為變滋大,悔之無救。」帝懼其言,乃大赦黨人,誅徙之家皆歸故郡。其後黃巾遂盛,朝野崩離,綱紀文章蕩然矣。

凡黨事始自甘陵、汝南,成於李膺、張儉,海內塗炭,二十餘年,諸所蔓衍,皆天下善士。三君、八俊等三十五人,其名跡存者,並載乎篇。陳蕃、竇武、王暢、劉表、度尚、郭林宗別有傳。荀翌附祖淑傳。張邈附呂布傳。胡母班附袁紹傳。王考字文祖,東平壽張人,冀州刺史;秦周字平王,陳留平丘人,北海相;蕃嚮字嘉景,魯國人,郎中;王璋字伯儀,東萊曲城人,少府卿:位行並不顯。翟超,山陽太守,事見陳蕃傳,字及郡縣未詳。朱宇,沛人,與杜密等俱死獄中。唯趙典名見而已。

劉淑字仲承,河閒樂成人也。祖父稱,司隸校尉。淑少學明五經,遂隱居,立精舍講授,諸生常數百人。州郡禮請,五府連辟,並不就。永興二年,司徒种暠舉淑賢良方正,辭以疾。桓帝聞淑高名,切責州郡,使輿病詣京師。淑不得已而赴洛陽,對策為天下第一,拜議郎。又陳時政得失,災異之占,事皆效驗。再遷尚書,納忠建議,多所補益。又再遷侍中、虎賁中郎將。上疏以為宜罷宦官,辭甚切直,帝雖不能用,亦不罪焉。以淑宗室之賢,特加敬異,每有疑事,常密諮問之。靈帝既位,宦官譖淑與竇武等通謀,下獄自殺。

李膺字元禮,潁川襄城人也。祖父脩,安帝時為太尉。父益,趙國相。膺性簡亢,無所交接,唯以同郡荀淑、陳寔為師友。

初舉孝廉,為司徒胡廣所辟,舉高第,再遷青州刺史。守令畏威明,多望風棄官。復徵,再遷漁陽太守。尋轉蜀郡太守,以母老乞不之官。轉護烏桓校尉。鮮卑數犯塞,膺常蒙矢石,每破走之,虜甚憚懾。以公事免官,還居綸氏,教授常千人。南陽樊陵求為門徒,膺謝不受。陵後以阿附宦官,致位太尉,為節者所羞。荀爽嘗就謁膺,因為其御,既還,喜曰:「今日乃得御李君矣。」其見慕如此。

永壽二年,鮮卑寇雲中,桓帝聞膺能,乃復徵為度遼將軍。先是羌虜及疏勒、龜茲,數出攻鈔張掖、酒泉、雲中諸郡,百姓屢被其害。自膺到邊,皆望風懼服,先所掠男女,悉送還塞下。自是之後,聲振遠域。

延熹二年徵,再遷河南尹。時宛陵大姓羊元群罷北海郡,臧罪狼藉,郡舍溷軒有奇巧,乃載之以歸。膺表欲按其罪,元群行賂宦豎,膺反坐輸作左校。

初,膺與廷尉馮緄、大司農劉祐等共同心志,糾罰姦倖,緄、祐時亦得罪輸作。司隸校尉應奉上疏理膺等曰:「昔秦人觀寶於楚,昭奚恤蒞以群賢;梁惠王瑋其照乘之珠,齊威王荅以四臣。夫忠賢武將,國之心膂。竊見左校弛刑徒前廷尉馮緄、大司農劉祐、河南尹李膺等,執法不撓,誅舉邪臣,肆之以法,眾庶稱宜。昔季孫行父親逆君命,逐出莒僕,於舜之功二十之一。今膺等投身彊禦,畢力致罪,陛下既不聽察,而猥受譖訴,遂令忠臣同愆元惡。自春迄冬,不蒙降恕,遐邇觀聽,為之歎息。夫立政之要,記功忘失,是以武帝捨安國於徒中,宣帝徵張敞於亡命。緄前討蠻荊,均吉甫之功。祐數臨督司,有不吐茹之節。膺著威幽、并,遺愛度遼。今三垂蠢動,王旅未振。易稱『雷雨作解,君子以赦過宥罪』。乞原膺等,以備不虞。」書奏,乃悉免其刑。

再遷,復拜司隸校尉。時張讓弟朔為野王令,貪殘無道,至乃殺孕婦,聞膺厲威嚴,懼罪逃還京師,因匿兄讓弟舍,藏於合柱中。膺知其狀,率將吏卒破柱取朔,付洛陽獄。受辭畢,即殺之。讓訴冤於帝,詔膺入殿,御親臨軒,詰以不先請便加誅辟之意。膺對曰:「昔晉文公執衛成公歸于京師,春秋是焉。禮云公族有罪,雖曰宥之,有司執憲不從。昔仲尼為魯司寇,七日而誅少正卯。今臣到官已積一旬,私懼以稽留為愆,不意獲速疾之罪。誠自知釁責,死不旋踵,特乞留五日,剋殄元惡,退就鼎鑊,始生之願也。」帝無復言,顧謂讓曰:「此汝弟之罪,司隸何愆?」乃遣出之。自此諸黃門常侍皆鞠躬屏氣,休沐不敢復出宮省。帝怪問其故,並叩頭泣曰:「畏李校尉。」

是時朝庭日亂,綱紀穨阤,膺獨持風裁,以聲名自高。士有被其容接者,名為登龍門。及遭黨事,當考實膺等。案經三府,太尉陳蕃卻之。曰:「今所考案,皆海內人譽,憂國忠公之臣。此等猶將十世宥也,豈有罪名不章而致收掠者乎?」不肯平署。帝愈怒,遂下膺等於黃門北寺獄。膺等頗引宦官子弟,宦官多懼,請帝以天時宜赦,於是大赦天下。膺免歸鄉里,居陽城山中,天下士大夫皆高尚其道,而汙穢朝廷。

及陳蕃免太尉,朝野屬意於膺,荀爽恐其名高致禍,欲令屈節以全亂世,為書貽曰:「久廢過庭,不聞善誘,陟岵瞻望,惟日為歲。知以直道不容於時,悅山樂水,家于陽城。道近路夷,當即聘問,無狀嬰疾,闕於所仰。頃聞上帝震怒,貶黜鼎臣,人鬼同謀,以為天子當貞觀二五,利見大人,不謂夷之初旦,明而未融,虹蜺揚煇,棄和取同。方今天地氣閉,大人休否,智者見險,投以遠害。雖匱人望,內合私願。想甚欣然,不為恨也。願怡神無事,偃息衡門,任其飛沈,與時抑揚。」頃之,帝崩。陳蕃為太傅,與大將軍竇武共秉朝政,連謀誅諸宦官,故引用天下名士,乃以膺為長樂少府。及陳、竇之敗,膺等復廢。

後張儉事起,收捕鉤黨,鄉人謂膺曰:「可去矣。」對曰:「事不辭難,罪不逃刑,臣之節也。吾年已六十,死生有命,去將安之?」乃詣詔獄。考死,妻子徙邊,門生、故吏及其父兄,並被禁錮。

時侍御史蜀郡景毅子顧為膺門徒,而未有錄牒,故不及於譴。毅乃慨然曰:「本謂膺賢,遣子師之,豈可以漏奪名籍,苟安而已!」遂自表免歸,時人義之。

膺子瓚,位至東平相。初,曹操微時,瓚異其才,將沒,謂子宣等曰:「時將亂矣,天下英雄無過曹操。張孟卓與吾善,袁本初汝外親,雖爾勿依,必歸曹氏。」諸子從之,並免於亂世。

杜密字周甫,潁川陽城人也。為人沈質,少有厲俗志。為司徒胡廣所辟,稍遷代郡太守。徵,三遷太山太守、北海相。其宦官子弟為令長有姦惡者,輒捕案之。行春到高密縣,見鄭玄為鄉佐,知其異器,即召署郡職,遂遣就學。

後密去官還家,每謁守令,多所陳託。同郡劉勝,亦自蜀郡告歸鄉里,閉門埽軌,無所干及。太守王昱謂密曰:「劉季陵清高士,公卿多舉之者。」密知昱激己,對曰:「劉勝位為大夫,見禮上賓,而知善不薦,聞惡無言,隱情惜己,自同寒蟬,此罪人也。今志義力行之賢而密達之,違道失節之士而密糾之,使明府賞刑得中,令問休揚,不亦萬分之一乎?」昱慚服,待之彌厚。

後桓帝徵拜尚書令,遷河南尹,轉太僕。黨事既起,免歸本郡,與李膺俱坐,而名行相次,故時人亦稱「李杜」焉。後太傅陳蕃輔政,復為太僕。明年,坐黨事被徵,自殺。

劉祐字伯祖,中山安國人也。安國後別屬博陵。祐初察孝廉,補尚書侍郎,閑練故事,文札強辨,每有奏議,應對無滯,為僚類所歸。

除任城令,兗州舉為尤異,遷揚州刺史。是時會稽太守梁旻,大將軍冀之從弟也。祐舉奏其罪,旻坐徵。復遷祐河東太守。時屬縣令長率多中官子弟,百姓患之。祐到,黜其權強,平理冤結,政為三河表。

再遷,延熹四年,拜尚書令,又出為河南尹,轉司隸校尉。時權貴子弟罷州郡還入京師者,每至界首,輒改易輿服,隱匿財寶,威行朝廷。

拜宗正,三轉大司農。時中常侍蘇康、管霸用事於內,遂固天下良田美業,山林湖澤,民庶窮困,州郡累氣。祐移書所在,依科品沒入之。桓帝大怒,論祐輸左校。

後得赦出,復歷三卿,輒以疾辭,乞骸骨歸田里。詔拜中散大夫,遂杜門絕跡。每三公缺,朝廷皆屬意於祐,以譖毀不用。延篤貽之書曰:「昔太伯三讓,人無德而稱焉。延陵高揖,華夏仰風。吾子懷蘧氏之可卷,體甯子之如愚,微妙玄通,沖而不盈,蔑三光之明,未暇以天下為事,何其劭與!」

靈帝初,陳蕃輔政,以祐為河南尹。及蕃敗,祐黜歸,卒于家。明年,大誅黨人,幸不及禍。

魏朗字少英,會稽上虞人也。少為縣吏。兄為鄉人所殺,朗白日操刃報讎於縣中,遂亡命到陳國。從博士郤仲信學春秋圖緯,又詣太學受五經,京師長者李膺之徒爭從之。

初辟司徒府,再遷彭城令。時中官子弟為國相,多行非法,朗與更相章奏,幸臣忿疾,欲中之。會九真賊起,乃共薦朗為九真都尉。到官,獎厲吏兵,討破群賊,斬首二千級。桓帝美其功,徵拜議郎。頃之,遷尚書。屢陳便宜,有所補益。出為河內太守,政稱三河表。尚書令陳蕃薦朗公忠亮直,宜在機密,復徵為尚書。會被黨議,免歸家。

朗性矜嚴,閉門整法度,家人不見墯容。後竇武等誅,朗以黨被急徵,行至牛渚,自殺。著書數篇,號魏子云。

夏馥字子治,陳留圉人也。少為書生,言行質直。同縣高氏、蔡氏並皆富殖,郡人畏而事之,唯馥比門不與交通,由是為豪姓所仇。桓帝初,舉直言,不就。

馥雖不交時宦,然以聲名為中官所憚,遂與范滂、張儉等俱被誣陷,詔下州郡,捕為黨魁。

及儉等亡命,經歷之處,皆被收考,辭所連引,布遍天下。馥乃頓足而歎曰:「孽自己作,空汙良善,一人逃死,禍及萬家,何以生為!」乃自翦須變形,入林慮山中,隱匿姓名,為冶家傭。親突煙炭,形貌毀瘁,積二三年,人無知者。後馥弟靜,乘車馬,載縑帛,追之於涅陽市中。遇馥不識,聞其言聲,乃覺而拜之。馥避不與語,靜追隨至客舍,共宿。夜中密呼靜曰:「吾以守道疾惡,故為權宦所陷。且念營苟全,以庇性命,弟柰何載物相求,是以禍見追也。」明旦,別去。黨禁未解而卒。

宗慈字孝初,南陽安眾人也。舉孝廉,九辟公府,有道徵,不就。後為脩武令。時太守出自權豪,多取貨賂,慈遂棄官去。徵拜議郎,未到,道疾卒。南陽群士皆重其義行。

巴肅字恭祖,勃海高城人也。初察孝廉,歷慎令、貝丘長,皆以郡守非其人,辭病去。辟公府,稍遷拜議郎。與竇武、陳蕃等謀誅閹官,武等遇害,肅亦坐黨禁錮。中常侍曹節後聞其謀,收之。肅自載詣縣,縣令見肅,入閤解印綬與俱去。肅曰:「為人臣者,有謀不敢隱,有罪不逃刑。既不隱其謀矣,又敢逃其刑乎?」遂被害。刺史賈琮刊石立銘以記之。

范滂字孟博,汝南征羌人也。少厲清節,為州里所服,舉孝廉、光祿四行。時冀州飢荒,盜賊群起,乃以滂為清詔使,案察之。滂登車攬轡,慨然有澄清天下之志。及至州境,守令自知臧汙,望風解印綬去。其所舉奏,莫不厭塞眾議。遷光祿勳主事。時陳蕃為光祿勳,滂執公儀詣蕃,蕃不止之,滂懷恨,投版棄官而去。郭林宗聞而讓蕃曰:「若范孟博者,豈宜以公禮格之?今成其去就之名,得無自取不優之議也?」蕃乃謝焉。

復為太尉黃瓊所辟。後詔三府掾屬舉謠言,滂奏刺史、二千石權豪之黨二十餘人。尚書責滂所劾猥多,疑有私故。滂對曰:「臣之所舉,自非叨穢姦暴,深為民害,豈以汙簡札哉!閒以會日迫促,故先舉所急,其未審者,方更參實。臣聞農夫去草,嘉穀必茂;忠臣除姦,王道以清。若臣言有貳,甘受顯戮。」吏不能詰。滂睹時方艱,知意不行,因投劾去。

太守宗資先聞其名,請署功曹,委任政事。滂在職,嚴整疾惡。其有行違孝悌,不軌仁義者,皆埽跡斥逐,不與共朝。顯薦異節,抽拔幽陋。滂外甥西平李頌,公族子孫,而為鄉曲所棄,中常侍唐衡以頌請資,資用為吏。滂以非其人,寑而不召。資遷怒,捶書佐朱零。零仰曰:「范滂清裁,猶以利刃齒腐朽。今日寧受笞死,而滂不可違。」資乃止。郡中中人以下,莫不歸怨,乃指滂之所用以為「范黨」。

後牢脩誣言鉤黨,滂坐繫黃門北寺獄。獄吏謂曰:「凡坐繫皆祭皋陶。」滂曰:「皋陶賢者,古之直臣。知滂無罪,將理之於帝;如其有罪,祭之何益!」眾人由此亦止。獄吏將加掠考,滂以同囚多嬰病,乃請先就格,遂與同郡袁忠爭受楚毒。桓帝使中常侍王甫以次辨詰,滂等皆三木囊頭,暴於階下。餘人在前,或對或否,滂、忠於後越次而進。王甫詰曰:「君為人臣,不惟忠國,而共造部黨,自相褒舉,評論朝廷,虛搆無端,諸所謀結,並欲何為?皆以情對,不得隱飾。」滂對曰:「臣聞仲尼之言,『見善如不及,見惡如探湯』。欲使善善同其清,惡惡同其汙,謂王政之所願聞,不悟更以為黨。」甫曰:「卿更相拔舉,迭為脣齒,有不合者,見則排斥,其意如何?」滂乃慷慨仰天曰:「古之循善,自求多福;今之循善,身陷大戮。身死之日,願埋滂於首陽山側,上不負皇天,下不愧夷、齊。」甫愍然為之改容。乃得並解桎梏。

滂後事釋,南歸。始發京師,汝南、南陽士大夫迎之者數千兩。同囚鄉人殷陶、黃穆,亦免俱歸,並衛侍於滂,應對賓客。滂顧謂陶等曰:「今子相隨,是重吾禍也。」遂遁還鄉里。

初,滂等繫獄,尚書霍諝理之。及得免,到京師,往候諝而不為謝。或有讓滂者。對曰:「昔叔向嬰罪,祁奚救之,未聞羊舌有謝恩之辭,祁老有自伐之色。」竟無所言。

建寧二年,遂大誅黨人,詔下急捕滂等。督郵吳導至縣,抱詔書,閉傳舍,伏床而泣。滂聞之,曰:「必為我也。」即自詣獄。縣令郭揖大驚,出解印綬,引與俱亡。曰:「天下大矣,子何為在此?」滂曰:「滂死則禍塞,何敢以罪累君,又令老母流離乎!」其母就與之訣。滂白母曰:「仲博孝敬,足以供養,滂從龍舒君歸黃泉,存亡各得其所。惟大人割不可忍之恩,勿增感戚。」母曰:「汝今得與李、杜齊名,死亦何恨!既有令名,復求壽考,可兼得乎?」滂跪受教,再拜而辭。顧謂其子曰:「吾欲使汝為惡,則惡不可為;使汝為善,則我不為惡。」行路聞之,莫不流涕。時年三十三。

論曰:李膺振拔汙險之中,蘊義生風,以鼓動流俗,激素行以恥威權,立廉尚以振貴埶,使天下之士奮迅感概,波蕩而從之,幽深牢破室族而不顧,至于子伏其死而母歡其義。壯矣哉!子曰:「道之將廢也與?命也!」

尹勳字伯元,河南鞏人也。家世衣冠。伯父睦為司徒,兄頌為太尉,宗族多居貴位者,而勳獨持清操,不以地埶尚人。州郡連辟,察孝廉,三遷邯鄲令,政有異跡。後舉高第,五遷尚書令。及桓帝誅大將軍梁冀,勳參建大謀,封都鄉侯。遷汝南太守。上書解釋范滂、袁忠等黨議禁錮。尋徵拜將作大匠,轉大司農。坐竇武等事,下獄自殺。

蔡衍字孟喜,汝南項人也。少明經講授,以禮讓化鄉里。鄉里有爭訟者,輒詣衍決之,其所平處,皆曰無怨。

舉孝廉,稍遷冀州刺史。中常侍具瑗託其弟恭舉茂才,衍不受,乃收齎書者案之。又劾奏河閒相曹鼎臧罪千萬。鼎者,中常侍騰之弟也。騰使大將軍梁冀為書請之,衍不荅,鼎竟坐輸作左校。乃徵衍拜議郎、符節令。梁冀聞衍賢,請欲相見,衍辭疾不往,冀恨之。時南陽太守成档等以收糾宦官考廷尉,衍與議郎劉瑜表救之,言甚切厲,坐免官還家,杜門不出。靈帝即位,徵拜議郎,會病卒。

羊陟字嗣祖,太山梁父人也。家世冠族。陟少清直有學行,舉孝廉,辟太尉李固府,舉高第,拜侍御史。會固被誅,陟以故吏禁錮歷年。復舉高第,再遷冀州刺史。奏案貪濁,所在肅然。又再遷虎賁中郎將、城門校尉,三遷尚書令。時太尉張顥、司徒樊陵、大鴻臚郭防、太僕曹陵、大司農馮方並與宦豎相姻私,公行貨賂,並奏罷黜之,不納。以前太尉劉寵、司隸校尉許冰、幽州刺史楊熙、涼州刺史劉恭、益州刺史龐艾清亮在公,薦舉升進。帝嘉之,拜陟河南尹。計日受奉,常食乾飯茹菜,禁制豪右,京師憚之。會黨事起,免官禁錮,卒於家。

張儉字元節,山陽高平人,趙王張耳之後也。父成,江夏太守。儉初舉茂才,以刺史非其人,謝病不起。

延熹八年,太守翟超請為東部督郵。時中常侍侯覽家在防東,殘暴百姓,所為不軌。儉舉劾覽及其母罪惡,請誅之。覽遏絕章表,並不得通,由是結仇。鄉人朱並,素性佞邪,為儉所棄,並懷怨恚,遂上書告儉與同郡二十四人為黨,於是刊章討捕。儉得亡命,困迫遁走,望門投止,莫不重其名行,破家相容。後流轉東萊,止李篤家。外黃令毛欽操兵到門,篤引欽謂曰:「張儉知名天下,而亡非其罪。縱儉可得,寧忍執之乎?」欽因起撫篤曰:「蘧伯玉恥獨為君子,足下如何自專仁義?」篤曰:「篤雖好義,明廷今日載其半矣。」欽歎息而去。篤因緣送儉出塞,以故得免。其所經歷,伏重誅者以十數,宗親並皆殄滅,郡縣為之殘破。

中平元年,黨事解,乃還鄉里。大將軍、三公並辟,又舉敦朴,公車特徵,起家拜少府,皆不就。獻帝初,百姓飢荒,而儉資計差溫,乃傾竭財產,與邑里共之,賴其存者以百數。

建安初,徵為衛尉,不得已而起。儉見曹氏世德已萌,乃闔門懸車,不豫政事。歲餘卒于許下。年八十四。

論曰:昔魏齊違死,虞卿解印;季布逃亡,朱家甘罪。而張儉見怒時王,顛沛假命,天下聞其風者,莫不憐其壯志,而爭為之主。至乃捐城委爵、破族屠身,蓋數十百所,豈不賢哉!然儉以區區一掌,而欲獨堙江河,終嬰疾甚之亂,多見其不知量也。

岑晊字公孝,南陽棘陽人也。父像,為南郡太守,以貪叨誅死。晊年少未知名,往候同郡宗慈,慈方以有道見徵,賓客滿門,以晊非良家子,不肯見。晊留門下數日,晚乃引入。慈與語,大奇之,遂將俱至洛陽,因詣太學受業。

晊有高才,郭林宗、朱公叔等皆為友,李膺、王暢稱其有幹國器,雖在閭里,慨然有董正天下之志。太守弘農成档下車,欲振威嚴,聞晊高名,請為功曹,又以張牧為中賊曹吏。档委心晊、牧,褒善糾違,肅清朝府。宛有富賈張汎者,桓帝美人之外親,善巧雕鏤玩好之物,頗以賂遺中官,以此並得顯位,恃其伎巧,用埶縱橫。晊與牧勸档收捕汎等,既而遇赦,晊竟誅之,并收其宗族賓客,殺二百餘人,後乃奏聞。於是中常侍侯覽使汎妻上書訟其冤。帝大震怒,徵档,下獄死。晊與牧亡匿齊魯之閒。會赦出。後州郡察舉,三府交辟,並不就。及李、杜之誅,因復逃竄,終于江夏山中云。

陳翔字子麟,汝南邵陵人也。祖父珍,司隸校尉。翔少知名,善交結。察孝廉,太尉周景辟舉高第,拜侍御史。時正旦朝賀,大將軍梁冀威儀不整,奏冀恃貴不敬,請收案罪,時人奇之。遷定襄太守,徵拜議郎,遷揚州刺史。舉奏豫章太守王永奏事中官,吳郡太守徐參在職貪穢,並徵詣廷尉。參,中常侍璜之弟也。由此威名大振。又徵拜議郎,補御史中丞。坐黨事考黃門北寺獄,以無驗見原,卒于家。

孔昱字元世,魯國魯人也。七世祖霸,成帝時歷九卿,封褒成侯。自霸至昱,爵位相係,其卿相牧守五十三人,列侯七人。昱少習家學,大將軍梁冀辟,不應。太尉舉方正,對策不合,乃辭病去。後遭黨事禁錮。靈帝即位,公車徵拜議郎,補洛陽令,以師喪棄官,卒於家。

苑康字仲真,勃海重合人也。少受業太學,與郭林宗親善。舉孝廉,再遷潁陰令,有能跡。

遷太山太守。郡內豪姓多不法,康至,奮威怒,施嚴令,莫有干犯者。先所請奪人田宅,皆遽還之。

是時山陽張儉殺常侍侯覽母,案其宗黨賓客,或有迸匿太山界者,康既常疾閹官,因此皆窮相收掩,無得遺脫。覽大怨之,誣康與兗州刺史第五種及都尉壺嘉詐上賊降,徵康詣廷尉獄,減死罪一等,徙日南。潁陰人及太山羊陟等詣闕為訟,乃原還本郡,卒於家。

檀鲍字文有,山陽瑕丘人也。少為諸生,家貧而志清,不受鄉里施惠。舉孝廉,連辟公府,皆不就。立精舍教授,遠方至者常數百人。桓帝時,博士徵,不就。靈帝即位,太尉黃瓊舉方正,對策合時宜,再遷議郎,補蒙令。以郡守非其人,棄官去。家無產業,子孫同衣而出。年八十,卒於家。

劉儒字叔林,東郡陽平人也。郭林宗常謂儒口訥心辯,有珪璋之質。察孝廉,舉高第,三遷侍中。桓帝時,數有災異,下策博求直言,儒上封事十條,極言得失,辭甚忠切。帝不能納,出為任城相。頃之,徵拜議郎。會竇武事,下獄自殺。

賈彪字偉節,潁川定陵人也。少遊京師,志節慷慨,與同郡荀爽齊名。

初仕州郡,舉孝廉,補新息長。小民困貧,多不養子,彪嚴為其制,與殺人同罪。城南有盜劫害人者,北有婦人殺子者,彪出案發,而掾吏欲引南。彪怒曰:「賊寇害人,此則常理,母子相殘,逆天違道。」遂驅車北行,案驗其罪。城南賊聞之,亦面縛自首。數年閒,人養子者千數,僉曰「賈父所長」,生男名為「賈子」,生女名為「賈女」。

延熹九年,黨事起,太尉陳蕃爭之不能得,朝廷寒心,莫敢復言。彪謂同志曰:「吾不西行,大禍不解。」乃入洛陽,說城門校尉竇武、尚書霍諝,武等訟之,桓帝以此大赦黨人。李膺出,曰:「吾得免此,賈生之謀也。」

先是岑晊以黨事逃亡,親友多匿焉,彪獨閉門不納,時人望之。彪曰:「傳言『相時而動,無累後人』。公孝以要君致釁,自遺其咎,吾以不能奮戈相待,反可容隱之乎?」於是咸服其裁正。

以黨禁錮,卒于家。初,彪兄弟三人,並有高名,而彪最優,故天下稱曰「賈氏三虎,偉節最怒」。

何顒字伯求,南陽襄鄉人也。少遊學洛陽。顒雖後進,而郭林宗、賈偉節等與之相好,顯名太學。友人虞偉高有父讎未報,而篤病將終,顒往候之,偉高泣而訴。顒感其義,為復讎,以頭醊其墓。

及陳蕃、李膺之敗,顒以與蕃、膺善,遂為宦官所陷,乃變姓名,亡匿汝南閒。所至皆親其豪桀,有聲荊豫之域。袁紹慕之,私與往來,結為奔走之友。是時黨事起,天下多離其難,顒常私入洛陽,從紹計議。其窮困閉厄者,為求援救,以濟其患。有被掩捕者,則廣設權計,使得逃隱,全免者甚眾。

及黨錮解,顒辟司空府。每三府會議,莫不推顒之長。累遷。及董卓秉政,逼顒以為長史,託疾不就,乃與司空荀爽、司徒王允等共謀卓。會爽薨,顒以它事為卓所繫,憂憤而卒。初,顒見曹操,歎曰:「漢家將亡,安天下者必此人也。」操以是嘉之。嘗稱「潁川荀彧,王佐之器」。及彧為尚書令,遣人西迎叔父爽,并致顒屍,而葬之爽之冢傍。

贊曰:渭以涇濁,玉以礫貞。物性既區,嗜惡從形。蘭蕕無並,銷長相傾。徒恨芳膏,煎灼燈明。


\end{pinyinscope}