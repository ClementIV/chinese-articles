\article{律歷上}

\begin{pinyinscope}
律準候氣

古之人論數也,曰「物生而後有象,象而後有滋,滋而後有數」。然則天地初形,人物既著,則筭數之事生矣。記稱大橈作甲子,隸首作數。二者既立,以比日表,以管萬事。夫一、十、百、千、萬,所同用也;律、度、量、衡、曆,其別用也。故體有長短,檢以度;物有多少,受以量;量有輕重,平以權衡;聲有清濁,協以律呂;三光運行,紀以曆數:然後幽隱之情,精微之變,可得而綜也。

漢興,北平侯張蒼首治律曆。孝武正樂,置協律之官。至元始中,博徵通知鍾律者,考其意義,羲和劉歆典領條奏,前史班固取以為志。而元帝時,郎中京房房字君明知五聲之音,六律之數。上使太子太傅韋玄成、字少翁諫議大夫章,雜試問房於樂府。房對:「受學故小黃令焦延壽。六十律相生之法:以上生下,皆三生二,以下生上,皆三生四,陽下生陰,陰上生陽,終於中呂,而十二律畢矣。中呂上生執始,執始下生去滅,上下相生,終於南事,六十律畢矣。夫十二律之變至於六十,猶八卦之變至於六十四也。宓羲作易,紀陽氣之初,以為律法。建日冬至之聲,以黃鍾為宮,太蔟為商,姑洗為角,林鍾為徵,南呂為羽,應鍾為變宮,蕤賓為變徵。此聲氣之元,五音之正也。故各終一日。其餘以次運行,當日者各自為宮,而商徵以類從焉。禮運篇曰『五聲、六律、十二管還相為宮』,此之謂也。以六十律分期之日,黃鍾自冬至始,及冬至而復,陰陽寒燠風雨之占生焉。於以檢攝群音,考其高下,苟非草木之聲,則無不有所合。虞書曰『律和聲』,此之謂也。」房又曰:「竹聲不可以度調,故作準以定數。準之狀如瑟,長丈而十三弦,隱閒九尺,以應黃鍾之律九寸;中央一弦,下有畫分寸,以為六十律清濁之節。」房言律詳於歆所奏,其術施行於史官,候部用之。文多不悉載。故總其本要,以續前志。

律術曰:陽以圓為形,其性動。陰以方為節,其性靜。動者數三,靜者數二。以陽生陰,倍之;以陰生陽,四之:皆三而一。陽生陰曰下生,陰生陽曰上生。上生不得過黃鍾之清濁,下生不得及黃鍾之數實。皆參天兩地,圓蓋方覆,六耦承奇之道也。黃鍾,律呂之首,而生十一律者也。其相生也,皆三分而損益之。是故十二律之,得十七萬七千一百四十七,是為黃鍾之實。又以二乘而三約之,是為下生林鍾之實。又以四乘而三約之,是為上生太蔟之實。推此上下,以定六十律之實。以九三之,數萬九千六百八十三為法。律為寸,於準為尺。不盈者十之,所得為分。又不盈十之,所得為小分。以其餘正其強弱。

黃鍾,十七萬七千一百四十七。下生林鍾。黃鍾為宮,太蔟商,林鍾徵。一日。律,九寸。準,九尺。

色育,十七萬六千七百七十六。下生謙待。色育為宮,未知商,謙待徵。六日。律,八寸九分小分八微強。準,八尺九寸萬五千九百七十三。執始,十七萬四千七百六十二。下生去滅。執始為宮,時息商,去滅徵。六日。律,八寸八分小分七大強。準,八尺八寸萬五千五百一十六。丙盛,十七萬二千四百一十。下生安度。丙盛為宮,屈齊商,安度徵。六日。律,八寸七分小分六微弱。準,八尺七寸萬一千六百七十九。分動,十七萬八十九。下生歸嘉。分動為宮,隨期商,歸嘉徵。六日。律,八寸六分小分四強。準,八尺六寸八千一百五十二。質末,十六萬七千八百。下生否與。質末為宮,形晉商,否與徵。六日。律,八寸五分小分二強。準,八尺五寸四千九百四十五。大呂,十六萬五千八百八十八。下生夷則。大呂為宮,夾鍾商,夷則徵。

八日。律,八寸四分小分三弱。準,八尺四寸五千五百八。分否,十六萬三千六百五十四。下生解形。分否為宮,開時商,解形徵。八日。律,八寸三分小分一強。準,八尺三寸二千八百五十一。凌陰,十六萬一千四百五十二。下生去南。凌陰為宮,族嘉商,去南徵。八日。律,八寸二分小分一弱。準,八尺二寸五百一十四。少出,十五萬九千二百八十。下生分積。少出為宮,爭南商,分積徵。六日。律,八寸小分九強。準,八尺萬八千一百六十。太蔟,十五萬七千四百六十四。下生南呂。太蔟為宮,姑洗商,南呂徵。一日。律,八寸。準,八尺。未知,十五萬七千一百三十四。下生白呂。未知為宮,南授商,白呂徵。六日。律,七寸九分小分八強。準,七尺九寸萬六千三百八十三。時息,十五萬五千三百四十四。

下生結躬。時息為宮,變虞商,結躬徵。六日。律,七寸八分小分九少強。準,七尺八寸萬八千一百六十六。屈齊,十五萬三千二百五十三。下生歸期。屈齊為宮,路時商,歸期徵。六日。律,七寸七分小分九弱。準,七尺七寸萬六千九百三十九。隨期,十五萬一千一百九十。下生未卯。隨期為宮,形始商,未卯徵。六日。律,七寸六分小分八強。準,七尺六寸萬五千九百九十二。形晉,十四萬九千一百五十五。下生夷汗。形晉為宮,依行商,夷汗徵。六日。律,七寸五分小分八弱。準,七尺五寸萬五千三百二十五。夾鍾,十四萬七千四百五十六。下生無射。夾鍾為宮,中呂商,無射徵。六日。律,七寸四分小分九強。準,七尺四寸萬八千一十八。開時,十四萬五千四百七十。下生閉掩。開時為宮,南中商,閉掩徵。八日。律,七寸三分小分九微弱。準,七尺三寸萬七千八百四十一。

族嘉,十四萬三千五百一十三。下生鄰齊。族嘉為宮,內負商,鄰齊徵。八日。律,七寸二分小分九微強。準,七尺二寸萬七千九百五十四。爭南,十四萬一千五百八十二。下生期保。爭南為宮,物應商,期保徵。八日。律,七寸一分小分九強。準,七尺一寸萬八千三百二十七。姑洗,十三萬九千九百六十八。下生應鍾。姑洗為宮,蕤賓商,應鍾徵。一日。律,七寸一分小分一微強。準,七尺一寸二千一百八十七。南授,十三萬九千六百七十〈四〉。下生分烏。南授為宮,南事商,分烏徵。六日。律,七寸小分九大強。準,七尺萬八千九百三十。變虞,十三萬八千八十四。下生遲內。變虞為宮,盛變商,遲內徵。六日。律,七寸小分一半強。準,七尺三千三十。路時,十三萬六千二百二十五。下生未育。路時為宮,離宮商,未育徵。

六日。律,六寸九分小分二微強。準,六尺九寸四千一百二十三。形始,十三萬四千三百九十二。下生遲時。形始為宮,制時商,遲時徵。五日。律,六寸八分小分三弱。準,六尺八寸五千四百七十六。依行,十三萬二千五百八十二。上生色育。依行為宮,謙待商,色育徵。七日。律,六寸七分小分三大強。準,六尺七寸七千五十九。中呂,十三萬一千七十二。上生執始。中呂為宮,去滅商,執始徵。八日。律,六寸六分小分六弱。準,六尺六寸萬一千六百四十二。南中,十二萬九千三百八。上生丙盛。南中為宮,安度商,丙盛徵。七日。律,六寸五分小分七微弱。準,六尺五寸萬三千六百八十五。內負,十二萬七千五百六十七。上生分動。內負為宮,歸嘉商,分動徵。八日。律,六寸四分小分八強。準,六尺四寸萬五千九百五十八。物應,十二萬五千八百五十。

上生質末。物應為宮,否與商,質末徵。七日。律,六寸三分小分九強。準,六尺三寸萬八千四百七十一。蕤賓,十二萬四千四百一十六。上生大呂。蕤賓為宮,夷則商,大呂徵。一日。律,六寸三分小分二微強。準,六尺三寸四千一百三十一。南事,十二萬四千一百五十四。下生。南事窮,無商、徵,不為宮。七日。律,六寸三分小分一弱。準,六尺三寸一千五百三十一。盛變,十二萬二千七百四十一。上生分否。盛變為宮,解形商,分否徵。七日。律,六寸二分小分三大強。準,六尺二寸七千六十四。離宮,十二萬一千八百一十九。上生凌陰。離宮為宮,去南商,凌陰徵。七日。律,六寸一分小分五微強。準,六尺一寸萬二百二十七。制時,十一萬九千四百六十。上生少出。制時為宮,分積商,少出徵。八日。律,六寸小分七弱。準,六尺萬三千六百二十。

林鍾,十一萬八千九十八。上生太蔟。林鍾為宮,南呂商,太蔟徵。一日。律,六寸。準,六尺。謙待,十一萬七千八百五十一。上生未知。謙待為宮,白呂商,未知徵。五日。律,五寸九分小分九弱。準,五尺九寸萬七千二百一十三。去滅,十一萬六千五百八。上生時息。去滅為宮,結躬商,時息徵。七日。律,五寸九分小分二弱。準,五尺九寸三千七百八十三。安度,十一萬四千九百四十。上生屈齊。安度為宮,歸期商,屈齊徵。六日。律,五寸八分小分四弱。準,五尺八寸七千七百八十六。歸嘉,十一萬三千三百九十三。上生隨期。歸嘉為宮,未卯商,隨期徵。六日。律,五寸七分小分六微強。準,五尺七寸萬一千九百九十九。否與,十一萬一千八百六十七。上生形晉。否與為宮,夷汗商,形晉徵。

五日。律,五寸六分小分八強。準,五尺六寸萬六千四百二十二。夷則,十一萬五百九十二。上生夾鍾。夷則為宮,無射商,夾鍾徵。八日。律,五寸六分小分二弱。準,五尺六寸三千六百七十二。解形,十一萬九千一百三。上生開時。解形為宮,閉掩商,開時徵。八日。律,五寸五分小分四強。準,五尺五寸八千四百六十五。去南,十萬七千六百三十五。上生族嘉。去南為宮,鄰齊商,族嘉徵。八日。律,五寸四分小分六大強。準,五尺四寸萬三千四百六十八。分積,十萬六千一百八十八〈七〉。上生爭南。分積為宮,期保商,爭南徵。七日。律,五寸三分小分九半強。準,五尺三寸萬八千六百八〈七〉十一。南呂,十萬四千九百七十六。上生姑洗。南呂為宮,應鍾商,姑洗徵。一日。律,五寸三分小分三強。準,五尺三寸六千五百六十一。

白呂,十萬四千七百五十六。上生南授。白呂為宮,分烏商,南授徵。五日。律,五寸三分小分二強。準,五尺三寸四千三百七〈六〉十一。結躬,十萬三千五百六十三。上生變虞。結躬為宮,遲內商,變虞徵。六日。律,五寸二分小分六少強。準,五尺二寸萬二千一百一十四。歸期,十萬二千一百六十九。上生路時。歸期為宮,未育商,路時徵。六日。律,五寸一分小分九微強。準,五尺一寸萬七千八百五十七。未卯,十萬七百九十四。上生形始。未卯為宮,遲時商,形始徵。六日。律,五寸一分小分二微強。準,五尺一寸四千八十〈一百〉七。夷汗,九萬九千四百三十七。上生依行。夷汗為宮,色育商,依行徵。七日。律,五寸小分五強。準,五尺萬二百二十。無射,九萬八千三百四。上生中呂。無射為宮,執始商,中呂徵。

八日。律,四寸九分小分九強。準,四尺九寸萬八千五百七十三。閉掩,九萬六千九百八十。上生南中。閉掩為宮,丙盛商,南中徵。八日。律,四寸九分小分三弱。準,四尺九寸五千三百三十三。鄰齊,九萬五千六百七十五。上生內負。鄰齊為宮,分動商,內負徵。七日。律,四寸八分小分六微強。準,四尺八寸萬一千九百六十六。期保,九萬四千三百八十八。上生物應。期保為宮,質末商,物應徵。八日。律,四寸七分小分九微強。準,四尺七寸萬八千七百七十九。應鍾,九萬三千三百一十二。上生蕤賓。應鍾為宮,大呂商,蕤賓徵。一日。律,四寸七分小分四微強。準,四尺七寸八千十九。分烏,九萬三千一百一十七〈六〉。上生南事。分烏窮次,無徵,不為宮。七日。律,四寸七分小分三微強。準,四尺七寸六千五十九。遲內,九萬二千五十六。

上生盛變。遲內為宮,分否商,盛變徵。八日。律,四寸六分小分八弱。準,四尺六寸萬五千一百四十二。未育,九萬八百一十七。上生離宮。未育為宮,凌陰商,離宮徵。八日。律,四寸六分小分一少強。準,四尺六寸二千七百五十二。遲時,八萬九千五百九十五。上生制時。遲時為宮,少出商,制時徵。六日。律,四寸五分小分五強。準,四尺五寸萬二百一十五。

截管為律,吹以考聲,列以物氣,道之本也。術家以其聲微而體難知,其分數不明,故作準以代之。準之聲,明暢易達,分寸又粗。然弦以緩急清濁,非管無以正也。均其中弦,令與黃鍾相得,案畫以求諸律,無不如數而應者矣。

音聲精微,綜之者解。元和元年,待詔候鍾律殷肜上言:「官無曉六十律以準調音者。故待詔嚴崇具以準法教子男宣,宣通習。願召宣補學官,主調樂器。」詔曰:「崇子學審曉律,別其族,協其聲者,審試。不得依託父學,以聾為聰。聲微妙,獨非莫知,獨是莫曉。以律錯吹,能知命十二律不失一,方為能傳崇學耳。」太史丞弘試十二律,其二中,其四不中,其六不知何律,宣遂罷。自此律家莫能為準施弦,候部莫知復見。熹平六年,東觀召典律者太子舍人張光等問準意。光等不知,歸閱舊藏,乃得其器,形制如房書,猶不能定其弦緩急,音不可書以時人,知之者欲教而無從,心達者體知而無師,故史官能辨清濁者遂絕。其可以相傳者,唯大搉常數及候氣而已。

夫五音生於陰陽,分為十二律,轉生六十,皆所以紀斗氣,效物類也。天效以景,地效以響,即律也。陰陽和則景至,律氣應則灰除。是故天子常以日冬夏至御前殿,合八能之士,陳八音,聽樂均,度晷景,候鍾律,權土灰,放陰陽。冬至陽氣應,則樂均清,景長極,黃鍾通,土灰輕而衡仰。夏至陰氣應,則樂均濁,景短極,蕤賓通,土灰重而衡低。進退於先後五日之中,八能各以候狀聞,太史封上。效則和,否則占。候氣之法,為室三重,戶閉,塗釁必周,密布緹縵。室中以木為案,每律各一,內庳外高,從其方位,加律其上,以葭莩灰抑其內端,案曆而候之。氣至者灰去。其為氣所動者其灰散,人及風所動者其灰聚。殿中候,用玉律十二。惟二至乃候靈臺,用竹律六十。候日如其曆。


\end{pinyinscope}