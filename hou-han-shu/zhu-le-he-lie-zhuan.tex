\article{朱樂何列傳}

\begin{pinyinscope}
朱暉字文季,南陽宛人也。家世衣冠。暉早孤,有氣決。年十三,王莽敗,天下亂,與外氏家屬從田閒奔入宛城。道遇群賊,白刃劫諸婦女,略奪衣物。昆弟賓客皆惶迫,伏地莫敢動。暉拔劍前曰:「財物皆可取耳,諸母衣不可得。今日朱暉死日也!」賊見其小,壯其志,笑曰:「童子內刀。」遂捨之而去。

初,光武與暉父岑俱學長安,有舊故。及即位,求問岑,時已卒,乃召暉拜為郎。暉尋以病去,卒業於太學。性矜嚴,進止必以禮,諸儒稱其高。

永平初,顯宗舅新陽侯陰就慕暉賢,自往候之,暉避不見。復遣家丞致禮,暉遂閉門不受。就聞,歎曰:「志士也,勿奪其節。」後為郡吏,太守阮況嘗欲市暉牛,暉不從。及況卒,暉乃厚贈送其家。人或譏焉,暉曰:「前阮府君有求於我,所以不敢聞命,誠恐以財貨污君。今而相送,明吾非有愛也。」驃騎將軍東平王蒼聞而辟之,甚禮敬焉。正月朔旦,蒼當入賀。故事,少府給璧。是時陰就為府卿,貴驕,吏傲不奉法。蒼坐朝堂,漏且盡,而求璧不可得,顧謂掾屬曰:「若之何?」暉望見少府主簿持璧,即往紿之曰:「我數聞璧未嘗見,試請觀之。」主簿以授暉,暉顧召令史奉之。主簿大驚,遽以白就。就曰:「朱掾義士,勿復求。」更以它璧朝。蒼既罷,召暉謂曰:「屬者掾自視孰與藺相如?」帝聞壯之。及當幸長安,欲嚴宿衛,故以暉為衛士令。再遷臨淮太守。

暉好節概,有所拔用,皆厲行士。其諸報怨,以義犯率,皆為求其理,多得生濟。其不義之囚,即時僵仆。吏人畏愛,為之歌曰:「彊直自遂,南陽朱季。吏畏其威,人懷其惠。」數年,坐法免。

暉剛於為吏,見忌於上,所在多被劾。自去臨淮,屏居野澤,布衣蔬食,不與邑里通,鄉黨譏其介。建初中,南陽大飢,米石千餘,暉盡散其家資,以分宗里故舊之貧羸者,鄉族皆歸焉。初,暉同縣張堪素有名稱,嘗於太學見暉,甚重之,接以友道,乃把暉臂曰:「欲以妻子託朱生。」暉以堪先達,舉手未敢對,自後不復相見。堪卒,暉聞其妻子貧困,乃自往候視,厚賑贍之。暉少子頡怪而問曰:「大人不與堪為友,平生未曾相聞,子孫竊怪之。」暉曰:「堪嘗有知己之言,吾以信於心也。」暉又與同郡陳揖交善,揖早卒,有遺腹子友,暉常哀之。及司徒桓虞為南陽太守,召暉子駢為吏,暉辭駢而薦友。虞嘆息,遂召之。其義烈若此。

元和中,肅宗巡狩,告南陽太守問暉起居,召拜為尚書僕射。歲中遷太山太守。暉上疏乞留中,詔許之。因上便宜,陳密事,深見嘉納。詔報曰:「補公家之闕,不累清白之素,斯善美之士也。俗吏苟合,阿意面從,進無謇謇之志,卻無退思之念,患之甚久。惟今所言,適我願也。生其勉之!」

是時穀貴,縣官經用不足,朝廷憂之。尚書張林上言:「穀所以貴,由錢賤故也。可盡封錢,一取布帛為租,以通天下之用。又鹽,食之急者,雖貴,人不得不須,官可自煮。又宜因交阯、益州上計吏往來,市珍寶,收采其利,武帝時所謂均輸者也。」於是詔諸尚書通議。暉奏據林言不可施行,事遂寢。後陳事者復重述林前議,以為於國誠便,帝然之,有詔施行。暉復獨奏曰:「王制,天子不言有無,諸侯不言多少,祿食之家不與百姓爭利。今均輸之法與賈販無異,鹽利歸官,則下人窮怨,布帛為租,則吏多姦盜,誠非明主所當宜行。」帝卒以林等言為然,得暉重議,因發怒,切責諸尚書。暉等皆自繫獄。三日,詔敕出之。曰:「國家樂聞駮議,黃髮無愆,詔書過耳,何故自繫?」暉因稱病篤,不肯復署議。尚書令以下惶怖,謂暉曰:「今臨得譴讓,柰何稱病,其禍不細!」暉曰:「行年八十,蒙恩得在機密,當以死報。若心知不可而順旨雷同,負臣子之義。今耳目無所聞見,伏待死命。」遂閉口不復言。諸尚書不知所為,乃共劾奏暉。帝意解,寑其事。後數日,詔使直事郎問暉起居,太醫視疾,太官賜食。暉乃起謝,復賜錢十萬,布百匹,衣十領。

後遷為尚書令,以老病乞身,拜騎都尉,賜錢二十萬。和帝即位,竇憲北征匈奴,暉復上疏諫。頃之,病卒。

子頡,修儒術,安帝時至陳相。頡子穆。

穆字公叔。年五歲,便有孝稱。父母有病,輒不飲食,差乃復常。及壯耽學,銳意講誦,或時思至,不自知亡失衣冠,顛隊阬岸。其父常以為專愚,幾不知數馬足。穆愈更精篤。

初舉孝廉。順帝末,江淮盜賊群起,州郡不能禁。或說大將軍梁冀曰:「朱公叔兼資文武,海內奇士,若以為謀主,賊不足平也。」冀亦素聞穆名,乃辟之,使典兵事,甚見親任。及桓帝即位,順烈太后臨朝,穆以冀埶地親重,望有以扶持王室,因推災異,奏記以勸戒冀曰:「穆伏念明年丁亥之歲,刑德合於乾位,易經龍戰之會。其文曰:『龍戰于野,其道窮也。』謂陽道將勝而陰道負也。今年九月天氣鬱冒,五位四候連失正氣,此互相明也。夫善道屬陽,惡道屬陰,若修正守陽,摧折惡類,則福從之矣。穆每事不逮,所好唯學,傳受於師,時有可試。願將軍少察愚言,申納諸儒,而親其忠正,絕其姑息,專心公朝,割除私欲,廣求賢能,斥遠佞惡。夫人君不可不學,當以天地順道漸漬其心。宜為皇帝選置師傅及侍講者,得小心忠篤敦禮之士,將軍與之俱入,參勸講授,師賢法古,此猶倚南山坐平原也,誰能傾之!今年夏,月暈房星,明年當有小厄。宜急誅姦臣為天下所怨毒者,以塞災咎。議郎、大夫之位,本以式序儒術高行之士,今多非其人;九卿之中,亦有乖其任者。惟將軍察焉。」又薦种暠、欒巴等。而明年嚴鮪謀立清河王蒜,又黃龍二見沛國。冀無術學,遂以穆「龍戰」之言為應,於是請暠為從事中郎,薦巴為議郎,舉穆高第,為侍御史。

時同郡趙康叔盛者,隱于武當山,清靜不仕,以經傳教授。穆時年五十,乃奉書稱弟子。及康歿,喪之如師。其尊德重道,為當時所服。

常感時澆薄,慕尚敦篤,乃作崇厚論。其辭曰:

夫俗之薄也,有自來矣。故仲尼歎曰:「大道之行也,而丘不與焉。」蓋傷之也。夫道者,以天下為一,在彼猶在己也。故行違於道則愧生於心,非畏義也;事違於理則負結于意,非憚禮也。故率性而行謂之道,得其天性謂之德。德性失然後貴仁義,是以仁義起而道德遷,禮法興而淳樸散。故道德以仁義為薄,淳樸以禮法為賊也。夫中世之所敦,已為上世之所薄,況又薄於此乎!

故夫天不崇大則覆幬不廣,地不深厚則載物不博,人不敦厖則道數不遠。昔在仲尼不失舊於原壤,楚嚴不忍章於絕纓。由此觀之,聖賢之德敦矣。老氏之經曰:「大丈夫處其厚不處其薄,居其實不居其華,故去彼取此。」夫時有薄而厚施,行有失而惠用。故覆人之過者,敦之道也;救人之失者,厚之行也。往者,馬援深昭此道,可以為德,誡其兄子曰:「吾欲汝曹聞人之過如聞父母之名。耳可得聞,口不得言。」斯言要矣。遠則聖賢履之上世,近則丙吉、張子孺行之漢廷。故能振英聲於百世,播不滅之遺風,不亦美哉!

然而時俗或異,風化不敦,而尚相誹謗,謂之臧否。記短則兼折其長,貶惡則并伐其善。悠悠者皆是,其可稱乎!凡此之類,豈徒乖為君子之道哉,將有危身累家之禍焉。悲夫!行之者不知憂其然,故害興而莫之及也。斯既然矣,又有異焉。人皆見之而不能自遷。何則?務進者趨前而不顧後,榮貴者矜己而不待人,智不接愚,富不賑貧,貞士孤而不恤,賢者厄而不存。故田蚡以尊顯致安國之金,淳于以貴埶引方進之言。夫以韓、翟之操,為漢之名宰,然猶不能振一貧賢,薦一孤士,又況其下者乎!此禽息、史魚所以專名於前,而莫繼於後者也。故時敦俗美,則小人守正,利不能誘也;時否俗薄,雖君子為邪,義不能止也。何則?先進者既往而不反,後來者復習俗而追之,是以虛華盛而忠信微,刻薄稠而純篤稀。斯蓋谷風有「棄予」之歎,伐木有「鳥鳴」之悲矣!

嗟乎!世士誠躬師孔聖之崇則,嘉楚嚴之美行,希李老之雅誨,思馬援之所尚,鄙二宰之失度,美韓稜之抗正,貴丙、張之弘裕,賤時俗之誹謗,則道豐績盛,名顯身榮,載不刊之德,播不滅之聲。然知薄者之不足,厚者之有餘也。彼與草木俱朽,此與金石相傾,豈得同年而語,並日而談哉?」

穆又著絕交論,亦矯時之作。

梁冀驕暴不悛,朝野嗟毒,穆以故吏,懼其釁積招禍,復奏記諫曰:「古之明君,必有輔德之臣,規諫之官,下至器物,銘書成敗,以防遺失。故君有正道,臣有正路,從之如升堂,違之如赴壑。今明將軍地有申伯之尊,位為群公之首,一日行善,天下歸仁,終朝為惡,四海傾覆。頃者,官人俱匱,加以水蟲為害。京師諸官費用增多,詔書發調或至十倍。各言官無見財,皆當出民,搒掠割剝,彊令充足。公賦既重,私斂又深。牧守長吏,多非德選,貪聚無猒,遇人如虜,或絕命於箠楚之下,或自賊於迫切之求。又掠奪百姓,皆託之尊府。遂令將軍結怨天下,吏人酸毒,道路歎嗟。昔秦政煩苛,百姓土崩,陳勝奮臂一呼,天下鼎沸,而面諛之臣,猶言安耳。諱惡不悛,卒至亡滅。昔永和之末,綱紀少弛,頗失人望。四五歲耳,而財空戶散,下有離心。馬免之徒乘敝而起,荊揚之閒幾成大患。幸賴順烈皇后初政清靜,內外同力,僅乃討定。今百姓戚戚,困於永和,內非仁愛之心可得容忍,外非守國之計所宜久安也。夫將相大臣,均體元首,共輿而馳,同舟而濟,輿傾舟覆,患實共之。豈可以去明即昧,履危自安,主孤時困,而莫之卹乎!宜時易宰守非其人者,減省第宅園池之費,拒絕郡國諸所奉送。內以自明,外解人惑,使挾姦之吏無所依託,司察之臣得盡耳目。憲度既張,遠邇清壹,則將軍身尊事顯,德燿無窮。天道明察,無言不信,惟垂省覽。」冀不納,而縱放日滋,遂復賂遺左右,交通宦者,任其子弟、賓客以為州郡要職。穆又奏記極諫,冀終不悟。報書云:「如此,僕亦無一可邪?」穆言雖切,然亦不甚罪也。

永興元年,河溢,漂害人庶數十萬戶,百姓荒饉,流移道路。冀州盜賊尤多,故擢穆為冀州刺史。州人有宦者三人為中常侍,並以檄謁穆。穆疾之,辭不相見。冀部令長聞穆濟河,解印綬去者四十餘人。及到,奏劾諸郡,至有自殺者。以威略權宜,盡誅賊渠帥。舉劾權貴,或乃死獄中。有宦者趙忠喪父,歸葬安平,僭為璵璠、玉匣、偶人。穆聞之,下郡案驗。吏畏其嚴明,遂發墓剖棺,陳尸出之,而收其家屬。帝聞大怒,徵穆詣廷尉,輸作左校。太學書生劉陶等數千人詣闕上書訟穆曰:「伏見施刑徒朱穆,處公憂國,拜州之日,志清姦惡。誠以常侍貴寵,父兄子弟布在州郡,競為虎狼,噬食小人,故穆張理天網,補綴漏目,羅取殘禍,以塞天意。由是內官咸共恚疾,謗讟煩興,讒隙仍作,極其刑謫,輸作左校。天下有識,皆以穆同勤禹、稷而被共、鯀之戾,若死者有知,則唐帝怒於崇山,重華忿於蒼墓矣。當今中官近習,竊持國柄,手握王爵,口含天憲,運賞則使餓隸富於季孫,呼唿則令伊、顏化為桀、跖。而穆獨亢然不顧身害。非惡榮而好辱,惡生而好死也,徒感王綱之不攝,懼天網之久失,故竭心懷憂,為上深計。臣願黥首繫趾,代穆校作。」帝覽其奏,乃赦之。

穆居家數年,在朝諸公多有相推薦者,於是徵拜尚書。穆既深疾宦官,及在臺閣,旦夕共事,志欲除之。乃上疏曰:「案漢故事,中常侍參選士人。建武以後,乃悉用宦者。自延平以來,浸益貴盛,假貂璫之飾,處常伯之任,天朝政事,一更其手,灌傾海內,寵貴無極,子弟親戚,並荷榮任,故放濫驕溢,莫能禁禦。凶狡無行之徒,媚以求官,恃埶怙寵之輩,漁食百姓,窮破天下,空竭小人。愚臣以為可悉罷省,遵復往初,率由舊章,更選海內清淳之士,明達國體者,以補其處。即陛下可為堯舜之君,眾僚皆為稷契之臣,兆庶黎萌蒙被聖化矣。」帝不納。後穆因進見,口復陳曰:「臣聞漢家舊典,置侍中、中常侍各一人,省尚書事,黃門侍郎一人,傳發書奏,皆用姓族。自和熹太后以女主稱制,不接公卿,乃以閹人為常侍,小黃門通命兩宮。自此以來,權傾人主,窮困天下。宜皆罷遣,博選耆儒宿德,與參政事。」帝怒,不應。穆伏不肯起。左右傳出,良久乃趨而去。自此中官數因事稱詔詆毀之。

穆素剛,不得意,居無幾,憤懣發疽。延熹六年,卒,時年六十四。祿仕數十年,蔬食布衣,家無餘財。公卿共表穆立節忠清,虔恭機密,守死善道,宜蒙旌寵。策詔褒述,追贈益州太守。所著論、策、奏、教、書、詩、記、嘲,凡二十篇。

穆前在冀州,所辟用皆清德長者,多至公卿、州郡。子野,少有名節,仕至河南尹。初,穆父卒,穆與諸儒考依古義,謚曰貞宣先生。及穆卒,蔡邕復與門人共述其體行,謚為文忠先生。

論曰:朱穆見比周傷義,偏黨毀俗,志抑朋游之私,遂著絕交之論。蔡邕以為穆貞而孤,又作正交而廣其致焉。蓋孔子稱「上交不諂,下交不黷」,又曰「晏平仲善與人交」,子夏之門人亦問交於子張。故易明「斷金」之義,詩載「讌朋」之謠。若夫文會輔仁,直諒多聞之友,時濟其益,紵衣傾蓋,彈冠結綬之夫,遂隆其好,斯固交者之方焉。至乃田、竇、衛、霍之游客,廉頗、翟公之門賓,進由埶合,退因衰異。又專諸、荊卿之感激,侯生、豫子之投身,情為恩使,命緣義輕。皆以利害移心,懷德成節,非夫交照之本,未可語失得之原也。穆徒以友分少全,因絕同志之求;黨俠生敝,而忘得朋之義。蔡氏貞孤之言,其為然也!古之善交者詳矣。漢興稱王陽、貢禹、陳遵、張竦,中世有廉范、慶鴻、陳重、雷義云。

樂恢字伯奇,京兆長陵人也。父親,為縣吏,得罪於令,收將殺之。恢年十一,常俯伏寺門,晝夜號泣。令聞而矜之,即解出親。

恢長好經學,事博士焦永。永為河東太守,恢隨之官,閉廬精誦,不交人物。後永以事被考,諸弟子皆以通關被繫,恢獨皦然不污於法,遂篤志為名儒。性廉直介立,行不合己者,雖貴不與交。信陽侯陰就數致禮請恢,恢絕不荅。

後仕本郡吏,太守坐法誅,故人莫敢往,恢獨奔喪行服,坐以抵罪。歸,復為功曹,選舉不阿,請託無所容。同郡楊政數眾毀恢,後舉政子為孝廉,由是鄉里歸之。辟司空牟融府。會蜀郡太守第五倫代融為司空,恢以與倫同郡,不肯留,薦潁川杜安而退。諸公多其行,連辟之,遂皆不應。

後徵拜議郎。會車騎將軍竇憲出征匈奴,恢數上書諫爭,朝廷稱其忠。入為尚書僕射。是時河南尹王調、洛陽令李阜與竇憲厚善,縱舍自由。恢劾奏調、阜,并及司隸校尉。諸所刺舉,無所回避,貴戚惡之。憲弟夏陽侯瑰欲往候恢,恢謝不與通。憲兄弟放縱,而忿其不附己。妻每諫恢曰:「昔人有容身避害,何必以言取怨?」恢歎曰:「吾何忍素餐立人之朝乎!」遂上疏諫曰:「臣聞百王之失,皆由權移於下。大臣持國,常以埶盛為咎。伏念先帝,聖德未永,早棄萬國。陛下富於春秋,纂承大業,諸舅不宜幹正王室,以示天下之私。經曰:『天地乖互,眾物夭傷。君臣失序,萬人受殃。』政失不救,其極不測。方今之宜,上以義自割,下以謙自引。四舅可長保爵土之榮,皇太后永無慚負宗廟之憂,誠策之上者也。」書奏不省。時竇太后臨朝,和帝未親萬機,恢以意不得行,乃稱疾乞骸骨。詔賜錢,太醫視疾。恢薦任城郭均、成陽高鳳,而遂稱篤。拜騎都尉,上書辭謝曰:「仍受厚恩,無以報效。夫政在大夫,孔子所疾;世卿持權,春秋以戒。聖人懇惻,不虛言也。近世外戚富貴,必有驕溢之敗。今陛下思慕山陵,未遑政事;諸舅寵盛,權行四方。若不能自損,誅罰必加。臣壽命垂盡,臨死竭愚,惟蒙留神。」詔聽上印綬,乃歸鄉里。竇憲因是風厲州郡迫脅,恢遂飲藥死。弟子縗絰輓者數百人,眾庶痛傷之。

後竇氏誅,帝始親事,恢門生何融等上書陳恢忠節,除子己為郎中。

何敞字文高,扶風平陵人也。其先家于汝陰。六世祖比干,學尚書於朝錯,武帝時為廷尉正,與張湯同時。湯持法深而比干務仁恕,數與湯爭,雖不能盡得,然所濟活者以千數。後遷丹楊都尉,因徙居平陵。敞父寵,建武中為千乘都尉,以病免,遂隱居不仕。

敞性公正。自以趣舍不合時務,每請召,常稱疾不應。元和中,辟太尉宋由府,由待以殊禮。敞論議高,常引大體,多所匡正。司徒袁安亦深敬重之。是時京師及四方累有奇異鳥獸草木,言事者以為祥瑞。敞通經傳,能為天官,意甚惡之。乃言於二公曰:「夫瑞應依德而至,災異緣政而生。故鴝鵒來巢,昭公有乾侯之厄;西狩獲麟,孔子有兩楹之殯。海鳥避風,臧文祀之,君子譏焉。今異鳥翔於殿屋,怪草生於庭際,不可不察。」由、安懼然不敢荅。居無何而肅宗崩。

時竇氏專政,外戚奢侈,賞賜過制,倉帑為虛。敞奏記由曰:「敞聞事君之義,進思盡忠,退思補過。歷觀世主時臣,無不各欲為化,垂之無窮,然而平和之政萬無一者,蓋以聖主賢臣不能相遭故也。今國家秉聰明之弘道,明公履晏晏之純德,君臣相合,天下翕然,治平之化,有望於今。孔子曰:『如有用我者,三年有成。』今明公視事,出入再期,宜當克己,以酬四海之心。禮,一穀不升,則損服徹膳。天下不足,若己使然。而比年水旱,人不收穫,涼州緣邊,家被凶害,男子疲於戰陳,妻女勞於轉運,老幼孤寡,歎息相依,又中州內郡,公私屈竭,此實損膳節用之時。國恩覆載,賞賚過度,但聞臘賜,自郎官以上,公卿王侯以下,至於空竭帑藏,損耗國資。尋公家之用,皆百姓之力。明君賜賚,宜有品制,忠臣受賞,亦應有度,是以夏禹玄圭,周公束帛。今明公位尊任重,責深負大,上當匡正綱紀,下當濟安元元,豈但空空無違而已哉!宜先正己以率群下,還所得賜,因陳得失,奏王侯就國,除苑囿之禁,節省浮費,賑卹窮孤,則恩澤下暢,黎庶悅豫,上天聰明,必有立應。使百姓歌誦,史官紀德,豈但子文逃祿,公儀退食之比哉!」由不能用。

時齊殤王子都鄉侯暢奔弔國憂,上書未報,侍中竇憲遂令人刺殺暢於城門屯衛之中,而主名不立。敞又說由曰:「劉暢宗室肺府,茅土藩臣,來弔大憂,上書須報,親在武衛,致此殘酷。奉憲之吏,莫適討捕,蹤跡不顯,主名不立。敞備數股肱,職典賊曹,故欲親至發所,以糾其變,而二府以為故事三公不與賊盜。昔陳平生於征戰之世,猶知宰相之分,云『外鎮四夷,內撫諸侯,使卿大夫各得其宜』。今二府執事不深惟大義,惑於所聞,公縱姦慝,莫以為咎。惟明公運獨見之明,昭然勿疑,敞不勝所見,請獨奏案。」由乃許焉。二府聞敞行,皆遣主者隨之,於是推舉具得事實,京師稱其正。

以高第拜侍御史。時遂以竇憲為車騎將軍,大發軍擊匈奴,而詔使者為憲弟篤、景並起邸第,興造勞役,百姓愁苦。敞上疏諫曰:「臣聞匈奴之為桀逆久矣。平城之圍,嫚書之恥,此二辱者,臣子所為捐軀而必死,高祖、呂后忍怒還忿,舍而不誅。伏惟皇太后秉文母之操,陛下履晏晏之姿,匈奴無逆節之罪,漢朝無可慚之恥,而盛春東作,興動大役,元元怨恨,咸懷不悅。而猥復為衛尉篤、奉車都尉景繕修館第,彌街絕里。臣雖斗筲之人,誠竊懷怪,以為篤、景親近貴臣,當為百僚表儀。今眾軍在道,朝廷焦脣,百姓愁苦,縣官無用,而遽起大第,崇飾玩好,非所以垂令德,示無窮也。宜且罷工匠,專憂北邊,恤人之困。」書奏不省。

後拜為尚書,復上封事曰:「夫忠臣憂世,犯主嚴顏,譏刺貴臣,至以殺身滅家而猶為之者,何邪?君臣義重,有不得已也。臣伏見往事,國之危亂,家之將凶,皆有所由,較然易知。昔鄭武姜之幸叔段,衛莊公之寵州吁,愛而不教,終至凶戾。由是觀之,愛子若此,猶飢而食之以毒,適所以害之也。伏見大將軍憲,始遭大憂,公卿比奏,欲令典幹國事。憲深執謙退,固辭盛位,懇懇勤勤,言之深至,天下聞之,莫不悅喜。今踰年無幾,大禮未終,卒然中改,兄弟專朝。憲秉三軍之重,篤、景總宮衛之權,而虐用百姓,奢侈僭偪,誅戮無罪,肆心自快。今者論議凶凶,咸謂叔段、州吁復生於漢。臣觀公卿懷持兩端,不肯極言者,以為憲等若有匪懈之志,則己受吉甫褒申伯之功,如憲等陷於罪辜,則自取陳平、周勃順呂后之權,終不以憲等吉凶為憂也。臣敞區區,誠欲計策兩安,絕其綿綿,塞其涓涓,上不欲令皇太后損文母之號,陛下有誓泉之譏,下使憲等得長保其福祐。然臧獲之謀,上安主父,下存主母,猶不免於嚴怒。臣伏惟累祖蒙恩,至臣八世,復以愚陋,旬年之閒,歷顯位,備機近,每念厚德,忽然忘生。雖知言必夷滅,而冒死自盡者,誠不忍目見其禍而懷默苟全。駙馬都尉瑰,雖在弱冠,有不隱之忠,比請退身,願抑家權。可與參謀,聽順其意,誠宗廟至計,竇氏之福。」

敞數切諫,言諸竇罪過,憲等深怨之。時濟南王康尊貴驕甚,憲乃白出敞為濟南太傅。敞至國,輔康以道義,數引法度諫正之,康敬禮焉。

歲餘,遷汝南太守。敞疾文俗吏以苛刻求當時名譽,故在職以寬和為政。立春日,常召督郵還府,分遣儒術大吏案行屬縣,顯孝悌有義行者。及舉冤獄,以春秋義斷之。是以郡中無怨聲,百姓化其恩禮。其出居者,皆歸養其父母,追行喪服,推財相讓者二百許人。置立禮官,不任文吏。又修理鮦陽舊渠,百姓賴其利,墾田增三萬餘頃。吏人共刻石,頌敞功德。

及竇氏敗,有司奏敞子與夏陽侯瑰厚善,坐免官。永元十二年復徵,三遷五官中郎將。常忿疾中常侍蔡倫,倫深憾之。元興元年,敞以祠廟嚴肅,微疾不齋,後鄧皇后上太傅禹冢,敞起隨百官會,倫因奏敞詐病,坐抵罪。卒于家。

論曰:永元之際,天子幼弱,太后臨朝,竇氏憑盛戚之權,將有呂、霍之變。幸漢德未衰,大臣方忠,袁、任二公正色立朝,樂、何之徒抗議柱下,故能挾幼主斷,勦姦回之偪。不然,國家危矣。夫竇氏之閒,唯何敞可以免,而特以子失交之故廢黜,不顯大位。惜乎,過矣哉!

贊曰:朱生受寄,誠不愆義。公叔辟梁,允納明刺。絕交面朋,崇厚浮偽。恢舉謗己,敞非祥瑞。永言國偪,甘心彊詖。


\end{pinyinscope}