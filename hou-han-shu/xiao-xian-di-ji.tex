\article{孝獻帝紀}

\begin{pinyinscope}
孝獻皇帝諱協,靈帝中子也。母王美人,為何皇后所害。中平六年四月,少帝即位,封帝為勃海王,徙封陳留王。

九月甲戌,即皇帝位,年九歲。遷皇太后於永安宮。大赦天下。改昭寧為永漢。丙子,董卓殺皇太后何氏。

初令侍中、給事黃門侍郎員各六人。賜公卿以下至黃門侍郎家一人為郎,以補宦官所領諸署,侍於殿上。

乙酉,以太尉劉虞為大司馬。董卓自為太尉,加鈇鉞、虎賁。丙戌,太中大夫楊彪為司空。甲午,豫州牧黃琬為司徒。

遣使弔祠故太傅陳蕃、大將軍竇武等。冬十月乙巳,葬靈思皇后。

白波賊寇河東,董卓遣其將牛輔擊之。

十一月癸酉,董卓為相國。十二月戊戌,司徒黃琬為太尉,司空楊彪為司徒,光祿勳荀爽為司空。

省扶風都尉,置漢安都護。

詔除光熹、昭寧、永漢三號,還復中平六年。

初平元年春正月,山東州郡起兵以討董卓。

辛亥,大赦天下。

癸酉,董卓殺弘農王。

白波賊寇東郡。

二月乙亥,太尉黃琬、司徒楊彪免。

庚辰,董卓殺城門校尉伍瓊、督軍校尉周珌。以光祿勳趙謙為太尉,太僕王允為司徒。

丁亥,遷都長安。董卓驅徙京師百姓悉西入關,自留屯畢圭苑。

壬辰,白虹貫日。

三月乙巳,車駕入長安,幸未央宮。

己酉,董卓焚洛陽宮廟及人家。

戊午,董卓殺太傅袁隗、太僕袁基,夷其族。

夏五月,司空荀爽薨。六月辛丑,光祿大夫种拂為司空。

大鴻臚韓融、少府陰脩、執金吾胡母班、

將作大匠吳脩、越騎校尉王瑰安集關東,後將軍袁術、河內太守王匡各執而殺之,唯韓融獲免。

董卓壞五銖錢,更鑄小錢。

冬十一月庚戌,鎮星、熒惑、太白合於尾。

是歲,有司奏,和、安、順、桓四帝無功德,不宜稱宗,又恭懷、敬隱、恭愍三皇后並非正嫡,不合稱后,皆請除尊號。制曰:「可。」孫堅殺荊州刺史王叡,又殺南陽太守張咨。

二年春正月辛丑,大赦天下。

二月丁丑,董卓自為太師。

袁術遣將孫堅與董卓將胡軫戰於陽人,軫軍大敗。董卓遂發掘洛陽諸帝陵。夏四月,董卓入長安。

六月丙戌,地震。

秋七月,司空种拂免,光祿大夫濟南淳于嘉為司空。太尉趙謙罷,太常馬日磾為太尉。

九月,蚩尤旗見于角、亢。

冬十月壬戌,董卓殺衛尉張溫。

十一月,青州黃巾寇太山,太山太守應劭擊破之。黃巾轉寇勃海,公孫瓚與戰於東光,復大破之。

是歲,長沙有人死經月復活。

三年春正月丁丑,大赦天下。

袁術遣將孫堅攻劉表於襄陽,堅戰歿。

袁紹及公孫瓚戰于界橋,瓚軍大敗。

夏四月辛巳,誅董卓,夷三族。司徒王允錄尚書事,總朝政,遣使者張种撫慰山東。

青州黃巾擊殺兗州刺史劉岱於東平。東郡太守曹操大破黃巾於壽張,降之。

五月丁酉,大赦天下。

丁未,征西將軍皇甫嵩為車騎將軍。

董卓部曲將李傕、郭汜、樊稠、張濟等反,攻京師。六月戊午,陷長安城,太常种拂、太僕魯旭、大鴻臚周奐、城門校尉崔烈、越騎校尉王頎並戰歿,吏民死者萬餘人。李傕等並自為將軍。

己未,大赦天下。

李傕殺司隸校尉黃琬,甲子,殺司徒王允,皆滅其族。丙子,前將軍趙謙為司徒。

秋七月庚子,太尉馬日磾為太傅,錄尚書事。八月,遣日磾及太僕趙岐,持節慰撫天下。車騎將軍皇甫嵩為太尉。司徒趙謙罷。

九月,李傕自為車騎將軍,郭汜後將軍,樊稠右將軍,張濟鎮東將軍。濟出屯弘農。

甲申,司空淳于嘉為司徒,光祿大夫楊彪為司空,並錄尚書事。

冬十二月,太尉皇甫嵩免。光祿大夫周忠為太尉,參錄尚書事。

四年春正月甲寅朔,日有食之。

丁卯,大赦天下。

三月,袁術殺楊州刺史陳溫,據淮南。

長安宣平城門外屋自壞。

夏五月癸酉,無雲而雷。六月,扶風大風,雨雹。華山崩裂。

太尉周忠免,太僕朱雋為太尉,錄尚書事。

下邳賊闕宣自稱天下。

雨水。遣侍御史裴茂訊詔獄,原輕繫。六月辛丑,天狗西北行。

九月甲午,試儒生四十餘人,上第賜位郎中,次太子舍人,下第者罷之。詔曰:「孔子歎『學之不講』,不講則所識日忘。今耆儒年踰六十,去離本土,營求糧資,不得專業。結童入學,白首空歸,長委農野,永絕榮望,朕甚愍焉。其依科罷者,聽為太子舍人。」

冬十月,太學行禮,車駕幸永福城門,臨觀其儀,賜博士以下各有差。

辛丑,京師地震。有星孛于天市。

司空楊彪免,太常趙溫為司空。

公孫瓚殺大司馬劉虞。

十二月辛丑,地震。

司空趙溫免,乙巳,衛尉張喜為司空。

是歲,琅邪王容薨。

興平元年春正月辛酉,大赦天下,改元興平。甲子,帝加元服。二月壬午,追尊謚皇妣王氏為靈懷皇后,甲申,改葬于文昭陵。丁亥,帝耕于藉田。

三月,韓遂、馬騰與郭汜、樊稠戰於長平觀,遂、騰敗績,左中郎將劉範、前益州刺史种劭戰歿。

夏六月丙子,分涼州河西四郡為廱州。

丁丑,地震;戊寅,又震。乙巳晦,日有食之,帝避正殿,寢兵,不聽事五日。大蝗。

秋七月壬子,太尉朱雋免。戊午,太常楊彪為太尉,錄尚書事。

三輔大旱,自四月至于是月。帝避正殿請雨,遣使者洗囚徒,原輕繫。是時穀一斛五十萬,豆麥一斛二十萬,人相食啖,白骨委積。帝使侍御史侯汶出太倉米豆,為飢人作縻粥,經日而死者無降。帝疑賦卹有虛,乃親於御坐前量試作糜,乃知非實,使侍中劉艾出讓有司。於是尚書令以下皆詣省閣謝,奏收侯汶考實。詔曰:「未忍致汶于理,可杖五十。」自是之後,多得全濟。

八月,馮翊羌叛,寇屬縣,郭汜、樊稠擊破之。

九月,桑復生椹,人得以食。

司徒淳于嘉罷。

冬十月,長安市門自壞。

以衛尉趙溫為司徒,錄尚書事。

十二月,分安定、扶風為新平郡。

是歲,楊州刺史劉繇與袁術將孫策戰于曲阿,繇軍敗績,孫策遂據江東。太傅馬日磾薨于壽春。

二年春正月癸丑,大赦天下。

二月乙亥,李傕殺樊稠而與郭汜相攻。三月丙寅,李傕脅帝幸其營,焚宮室。

夏四月甲午,立貴人伏氏為皇后。

丁酉,郭汜攻李傕,矢及御前。是日,李傕移帝幸北塢。

大旱。

五月壬午,李傕自為大司馬。六月庚午,張濟自陝來和傕、汜。

秋七月甲子,車駕東歸。郭汜自為車騎將軍,楊定為後將軍,楊奉為興義將軍,董承為安集將軍,並侍送乘輿。張濟為票騎將軍,還屯陝。八月甲辰,幸新豐。冬十月戊戌,郭汜使其將伍習夜燒所幸學舍,逼脅乘輿。楊定、楊奉與郭汜戰,破之。壬寅,幸華陰,露次道南。是夜,有赤氣貫紫宮。張濟復反,與李傕、郭汜合。十一月庚午,李傕、郭汜等追乘輿,戰於東澗,王師敗績,殺光祿勳鄧泉、衛尉士孫瑞、廷尉宣播、大長秋苗祀、步兵校尉魏桀、侍中朱展、射聲校尉沮雋。壬申,幸曹陽,露次田中。楊奉、董承引白波帥胡才、李樂、韓暹及匈奴左賢王去卑,率師奉迎,與李傕等戰,破之。十二月庚辰,車駕乃進。李傕等復來追戰,王師大敗,殺略宮人,少府田芬、大司農張義等皆戰歿。進幸陝,夜度河。乙亥,幸安邑。

是歲,袁紹遣將麴義與公孫瓚戰於鮑丘,瓚軍大敗。

建安元年春正月癸酉,郊祀上帝於安邑,大赦天下,改元建安。

二月,韓暹攻衛將軍董承。

夏六月乙未,幸聞喜。秋七月甲子,車駕至洛陽,幸故中常侍趙忠宅。丁丑,郊祀上帝,大赦天下。己卯,謁太廟。八月辛丑,幸南宮楊安殿。

癸卯,安國將軍張楊為大司馬,韓暹為將軍,楊奉為車騎將軍。

是時,宮室燒盡,百官披荊棘,依牆壁閒。州郡各擁彊兵,而委輸不至,群僚飢乏,尚書郎以下自出採髋,或飢死牆壁閒,或為兵士所殺。

辛亥,鎮東將軍曹操自領司隸校尉,錄尚書事。曹操殺侍中臺崇、尚書馮碩等。封衛將軍董承為輔國將軍伏完等十三人為列侯,贈沮雋為弘農太守。

庚申,遷都許。己巳,幸曹操營。

九月,太尉楊彪、司空張喜罷。冬十一月丙戌,曹操自為司空,行車騎將軍事,百官總己以聽。

二年春,袁術自稱天子。三月,袁紹自為大將軍。

夏五月,蝗。秋九月,漢水溢。

是歲飢,江淮閒民相食。袁術殺陳王寵。孫策遣使奉貢。

三年夏四月,遣謁者裴茂率中郎將段煨討李傕,夷三族。

呂布叛。

冬十一月,盜殺大司馬張楊。

十二月癸酉,曹操擊呂布於徐州,斬之。

四年春三月,袁紹攻公孫瓚于易京,獲之。

衛將軍董承為車騎將軍。

夏六月,袁術死。

是歲,初置尚書左右僕射。武陵女子死十四日復活。

五年春正月,車騎將軍董承、偏將軍王服、越騎校尉种輯受密詔誅曹操,事洩。壬午,曹操殺董承等,夷三族。

秋七月,立皇子馮為南陽王。壬午,南陽王馮薨。

九月庚午朔,日有食之。詔三公舉至孝二人,九卿、校尉、郡國守相各一人。皆上封事,靡有所諱。

曹操與袁紹戰於官度,紹敗走。

冬十月辛亥,有星孛于大梁。

東海王祗薨。

是歲,孫策死,弟權襲其餘業。

六年春三月丁卯朔,日有食之。

七年夏五月庚戌,袁紹薨。

于窴國獻馴象。

是歲,越巂男子化為女子。

八年冬十月己巳,公卿初迎冬於北郊,總章始復備八佾舞。

初置司直官,督中都官。

九年秋八月戊寅,曹操大破袁尚,平冀州,自領冀州牧。

冬十月,有星孛于東井。

十二月,賜三公已下金帛各有差。自是三年一賜,以為常制。

十年春正月,曹操破袁譚於青州,斬之。

夏四月,黑山賊張燕率眾降。

秋九月,賜百官尤貧者金帛各有差。

十一年春正月,有星孛于北斗。

三月,曹操破高幹於并州,獲之。

秋七月,武威太守張猛殺雍州刺史邯鄲商。

是歲,立故琅邪王容子熙為琅邪王。齊、北海、阜陵、下邳、常山、甘陵、濟陰、平原八國皆除。

十二年秋八月,曹操大破烏桓於柳城,斬其蹋頓。

冬十月辛卯,有星孛于鶉尾。

乙巳,黃巾賊殺濟南王贇。

十一月,遼東太守公孫康殺袁尚、袁熙。

十三年春正月,司徒趙溫免。

夏六月,罷三公官,置丞相、御史大夫。癸巳,曹操自為丞相。

秋七月,曹操南征劉表。

八月丁未,光祿勳郗慮為御史大夫。

壬子,曹操殺太中大夫孔融,夷其族。

是月,劉表卒,少子琮立,琮以荊州降操。

冬十月癸未朔,日有食之。

曹操以舟師伐孫權,權將周瑜敗之於烏林、赤壁。

十四年冬十月,荊州地震。

十五年春二月乙巳朔,日有食之。

十六年秋九月庚戌,曹操與韓遂、馬超戰於渭南,遂等大敗,關西平。

是歲,趙王赦薨。

十七年夏五月癸未,誅衛尉馬騰,夷三族。

六月庚寅晦,日有食之。

秋七月,洧水、潁水溢。螟。

八月,馬超破涼州,殺刺史韋康。

九月庚戌,立皇子熙為濟陰王,懿為山陽王,邈為濟北王,敦為東海王。

冬十二月,星孛于五諸侯。

十八年春正月庚寅,復禹貢九州。

夏五月丙申,曹操自立為魏公,加九錫。

大雨水。

徙趙王珪為博陵王。

是歲,歲星、鎮星、熒惑俱入太微。彭城王和薨。

十九年,夏四月,旱。五月,雨水。

劉備破劉璋,據益州。

冬十月,曹操遣將夏侯淵討宋建于枹罕,獲之。

十一月丁卯,曹操殺皇后伏氏,滅其族及二皇子。

二十年春正月甲子,立貴人曹氏為皇后。賜天下男子爵,人一級,孝悌、力田二級。賜諸王侯公卿以下穀各有差。

秋七月,曹操破漢中,張魯降。

二十一年夏四月甲午,曹操自進號魏王。

五月己亥朔,日有食之。

秋七月,匈奴南單于來朝。

是歲,曹操殺琅邪王熙,國除。

二十二年夏六月,丞相軍師華歆為御史大夫。

冬,有星孛于東北。

是歲大疫。

二十三年春正月甲子,少府耿紀、丞相司直韋晃起兵誅曹操,不克,夷三族。

三月,有星孛于東方。

二十四年春二月壬子晦,日有食之。

夏五月,劉備取漢中。

秋七月庚子,劉備自稱漢中王。

八月,漢水溢。

冬十一月,孫權取荊州。

二十五年春正月庚子,魏王曹操薨。子丕襲位。

二月丁未朔,日有食之。

三月,改元延康。

冬十月乙卯,皇帝遜位,魏王丕稱天子。奉帝為山陽公,邑一萬戶,位在諸侯王上,奏事不稱臣,受詔不拜,以天子車服郊祀天地,宗廟、祖、臘皆如漢制,都山陽之濁鹿城。四皇子封王者,皆降為列侯。

明年,劉備稱帝于蜀,孫權亦自王於吳,於是天下遂三分矣。

魏青龍二年三月庚寅,山陽公薨。自遜位至薨,十有四年,年五十四,謚孝獻皇帝。八月壬申,以漢天子禮儀葬于禪陵,置園邑令丞。

太子早卒,孫康立五十一年,晉太康六年薨。子瑾立四年,太康十年薨。子秋立二十年,永嘉中為胡賊所殺,國除。

論曰:傳稱鼎之為器,雖小而重,故神之所寶,不可奪移。至令負而趨者,此亦窮運之歸乎!天厭漢德久矣,山陽其何誅焉!

贊曰:獻生不辰,身播國屯。終我四百,永作虞賓,


\end{pinyinscope}