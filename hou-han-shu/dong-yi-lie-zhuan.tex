\article{東夷列傳}

\begin{pinyinscope}
王制云:「東方曰夷。」夷者,柢也,言仁而好生,萬物柢地而出。故天性柔順,易以道御,至有君子、不死之國焉。夷有九種,曰畎夷,于夷,方夷,黃夷,白夷,赤夷,玄夷,風夷,陽夷。故孔子欲居九夷也。

昔堯命羲仲宅嵎夷,曰暘谷,蓋日之所出也。夏后氏太康失德,夷人始畔。自少康已後,世服王化,遂賓於王門,獻其樂舞。桀為暴虐,諸夷內侵,殷湯革命,伐而定之。至于仲丁,藍夷作寇。自是或服或畔,三百餘年。武乙衰敝,東夷寖盛,遂分遷淮、岱,漸居中土。

及武王滅紂,肅慎來獻石砮、楛矢。管、蔡畔周,乃招誘夷狄,周公征之,遂定東夷。康王之時,肅慎復至。後徐夷僭號,乃率九夷以伐宗周,西至河上。穆王畏其方熾,乃分東方諸侯,命徐偃王主之。偃王處潢池東,地方五百里,行仁義,陸地而朝者三十有六國。穆王後得驥騄之乘,乃使造父御以告楚,令伐徐,一日而至。於是楚文王大舉兵而滅之。偃王仁而無權,不忍鬥其人,故致於敗。乃北走彭城武原縣東山下,百姓隨之者以萬數,因名其山為徐山。厲王無道,淮夷入寇,王命虢仲征之,不克,宣王復命召公伐而平之。及幽王淫亂,四夷交侵,至齊桓修霸,攘而卻焉。及楚靈會申,亦來豫盟。後越遷琅邪,與共征戰,遂陵暴諸夏,侵滅小邦。

秦并六國,其淮、泗夷皆散為民戶。陳涉起兵,天下崩潰,燕人衛滿避地朝鮮,因王其國。百有餘歲,武帝滅之,於是東夷始通上京。王莽篡位,貊人寇邊。建武之初,復來朝貢。時遼東太守祭肜威讋北方,聲行海表,於是濊、貊、倭、韓萬里朝獻,故章、和已後,使聘流通。逮永初多難,始入寇鈔;桓、靈失政,漸滋曼焉。

自中興之後,四夷來賓,雖時有乖畔,而使驛不絕,故國俗風土,可得略記。東夷率皆土著,憙飲酒歌舞,或冠弁衣錦,器用俎豆。所謂中國失禮,求之四夷者也。凡蠻、夷、戎、狄總名四夷者,猶公、侯、伯、子、男皆號諸侯云。

夫餘國,在玄菟北千里。南與高句驪,東與挹婁,西與鮮卑接,北有弱水。地方二千里,本濊地也。

初,北夷索離國王出行,其侍兒於後妊身,王還,欲殺之。侍兒曰:「前見天上有氣,大如雞子,來降我,因以有身。」王囚之,後遂生男。王令置於豕牢,豕以口氣噓之,不死。復徙於馬蘭,馬亦如之。王以為神,乃聽母收養,名曰東明。東明長而善射,王忌其猛,復欲殺之。東明奔走,南至掩箫水,以弓擊水,魚鱉皆聚浮水上,東明乘之得度,因至夫餘而王之焉。於東夷之域,最為平敞,土宜五穀。出名馬、赤玉、貂豽,大珠如酸棗。以員柵為城,有宮室、倉庫、牢獄。其人麤大彊勇而謹厚,不為寇鈔。以弓矢刀矛為兵。以六畜名官,有馬加、牛加、狗加,其邑落皆主屬諸加。食飲用俎豆,會同拜爵洗爵,揖讓升降。以臘月祭天,大會連日,飲食歌舞,名曰「迎鼓」。是時斷刑獄,解囚徒。有軍事亦祭天,殺牛,以蹄占其吉凶。行人無晝夜,好歌吟,音聲不絕。其俗用刑嚴急,被誅者皆沒其家人為奴婢。盜一責十二。男女淫皆殺之,尤治惡妒婦,既殺,復尸於山上。兄死妻嫂。死則有槨無棺。殺人殉葬,多者以百數。其王葬用玉匣,漢朝常豫以玉匣付玄菟郡,王死則迎取以葬焉。

建武中,東夷諸國皆來獻見。二十五年,夫餘王遣使奉貢,光武厚荅報之,於是使命歲通。至安帝永初五年,夫餘王始將步騎七八千人寇鈔樂浪,殺傷吏民,後復歸附。永寧元年,乃遣嗣子尉仇台印闕貢獻,天子賜尉仇台印綬金綵。順帝永和元年,其王來朝京師,帝作黃門鼓吹、角抵戲以遣之。桓帝延熹四年,遣使朝賀貢獻。永康元年,王夫台將二萬餘人寇玄菟,玄菟太守公孫域擊破之,斬首千餘級。至靈帝熹平三年,復奉章貢獻。夫餘本屬玄菟,獻帝時,其王求屬遼東云。

挹婁,古肅慎之國也。在夫餘東北千餘里,東濱大海,南與北沃沮接,不知其北所極。土地多山險。人形似夫餘,而言語各異。有五穀、麻布,出赤玉、好貂。無君長,其邑落各有大人。處於山林之閒,土氣極寒,常為穴居,以深為貴,大家至接九梯。好養豕,食其肉,衣其皮。冬以豕膏塗身,厚數分,以禦風寒。夏則裸袒,以尺布蔽其前後。其人臭穢不絜,作廁於中,圜之而居。自漢興已後,臣屬夫餘。種眾雖少,而多勇力,處山險,又善射,發能入人目。弓長四尺,力如弩。矢用楛,長一尺八寸,青石為鏃,鏃皆施毒,中人即死。便乘船,好寇盜,鄰國畏患,而卒不能服。東夷夫餘飲食類此皆用俎豆,唯挹婁獨無,法俗最無綱紀者也。

高句驪,在遼東之東千里,南與朝鮮、濊貊,東與沃沮,北與夫餘接。地方二千里,多大山深谷,人隨而為居。少田業,力作不足以自資,故其俗節於飲食,而好修宮室。東夷相傳以為夫餘別種,故言語法則多同,而跪拜曳一腳,行步皆走。凡有五族,有消奴部,絕奴部,順奴部,灌奴部,桂婁部。本消奴部為王,稍微弱,後桂婁部代之。其置官,有相加、對盧、沛者、古鄒大加、主簿、優台、使者、帛衣先人。武帝滅朝鮮,以高句驪為縣,使屬玄菟,賜鼓吹伎人。其俗淫,皆絜淨自憙,暮夜輒男女群聚為倡樂。好祠鬼神、社稷、零星,以十月祭天大會,名曰「東盟」。其國東有大穴,號禭神,亦以十月迎而祭之。其公會衣服皆錦繡,金銀以自飾。大加、主簿皆著幘,如冠幘而無後;其小加著折風,形如弁。無牢獄,有罪,諸加評議便殺之,沒入妻子為奴婢。其昏姻皆就婦家,生子長大,然後將還,便稍營送終之具。金銀財幣盡於厚葬,積石為封,亦種松柏。其人性凶急,有氣力,習戰鬥,好寇鈔,沃沮、東濊皆屬焉。

句驪一名貊耳。有別種,依小水為居,因名曰小水貊。出好弓,所謂「貊弓」是也。

王莽初,發句驪兵以伐匈奴,其人不欲行,彊迫遣之,皆亡出塞為寇盜。遼西大尹田譚追擊,戰死。莽令其將嚴尤擊之,誘句驪侯騶入塞,斬之,傳首長安。莽大說,更名高句驪王為下句驪侯,於是貊人寇邊愈甚。建武八年,高句驪遣使朝貢,光武復其王號。二十三年冬,句驪蠶支落大加戴升等萬餘口詣樂浪內屬。二十五年春,句驪寇右北平、漁陽、上谷、太原,而遼東太守祭肜以恩信招之,皆復款塞。

後句驪王宮生而開目能視,國人懷之,及長勇壯,數犯邊境。和帝元興元年春,復入遼東,寇略六縣,太守耿夔擊破之,斬其渠帥。安帝永初五年,宮遣使貢獻,求屬玄菟。元初五年,復與濊貊寇玄菟,攻華麗城。建光元年春,幽州刺史馮煥、玄菟太守姚光、遼東太守蔡諷等將兵出塞擊之,捕斬濊貊渠帥,獲兵馬財物。宮乃遣嗣子遂成將二千餘人逆光等,遣使詐降;光等信之,遂成因據險阨以遮大軍,而潛遣三千人攻玄菟、遼東,焚城郭,殺傷二千餘人。於是發廣陽、漁陽、右北平、涿郡屬國三千餘騎同救之,而貊人已去。夏,復與遼東鮮卑八千餘人攻遼隊,殺略吏人。蔡諷等追擊於新昌,戰歿,功曹耿耗、兵曹掾龍端、兵馬掾公孫酺以身扞諷,俱沒於陳,死者百餘人。秋,宮遂率馬韓、濊貊數千騎圍玄菟。夫餘王遣子尉仇台將二萬餘人,與州郡并力討破之,斬首五百餘級。

是歲宮死,子遂成立。姚光上言欲因其喪發兵擊之,議者皆以為可許。尚書陳忠曰:「宮前桀黠,光不能討,死而擊之,非義也。宜遣弔問,因責讓前罪,赦不加誅,取其後善。」安帝從之。明年,遂成還漢生口,詣玄菟降。詔曰:「遂成等桀逆無狀,當斬斷葅醢,以示百姓,幸會赦令,乞罪請降。鮮卑、濊貊連年寇鈔,驅略小民,動以千數,而裁送數十百人,非向化之心也。自今已後,不與縣官戰鬥而自以親附送生口者,皆與贖直,縑人四十匹,小口半之。」

遂成死,子伯固立。其後濊貊率服,東垂少事。順帝陽嘉元年,置玄菟郡屯田六部。質、桓之閒,復犯遼東西安平,殺帶方令,掠得樂浪太守妻子。建寧二年,玄菟太守耿臨討之,斬首數百級,伯固降服,乞屬玄菟云。

東沃沮在高句驪蓋馬大山之東,東濱大海;北與挹婁、夫餘,南與濊貊接。其地東西夾,南北長,可折方千里。土肥美,背山向海,宜五穀,善田種,有邑落長帥。人性質直彊勇,便持矛步戰。言語、食飲、居處、衣服有似句驪。其葬,作大木槨,長十餘丈,開一頭為戶,新死者先假埋之,令皮肉盡,乃取骨置槨中。家人皆共一槨,刻木如主,隨死者為數焉。

武帝滅朝鮮,以沃沮地為玄菟郡。後為夷貊所侵,徙郡於高句驪西北,更以沃沮為縣,屬樂浪東部都尉。至光武罷都尉官,後皆以封其渠帥,為沃沮侯。其土迫小,介於大國之閒,遂臣屬句驪。句驪復置其中大人遂為使者,以相監領,貴其租稅,貂布魚鹽,海中食物,發美女為婢妾焉。

又有北沃沮,一名置溝婁,去南沃沮八百餘里。其俗皆與南同。界南接挹婁。挹婁人憙乘船寇抄,北沃沮畏之,每夏輒臧於巖穴,至冬船道不通,乃下居邑落。其耆老言,嘗於海中得一布衣,其形如中人衣,而兩袖長三丈。又於岸際見一人乘破船,頂中復有面,與語不通,不食而死。又說海中有女國,無男人。或傳其國有神井,闚之輒生子云。

濊北與高句驪、沃沮,南與辰韓接,東窮大海,西至樂浪。濊及沃沮、句驪,本皆朝鮮之地也。昔武王封箕子於朝鮮,箕子教以禮義田蠶,又制八條之教。其人終不相盜,無門戶之閉。婦人貞信。飲食以籩豆。其後四十餘世,至朝鮮侯準,自稱王。漢初大亂,燕、齊、趙人往避地者數萬口,而燕人衛滿擊破準而自王朝鮮,傳國至孫右渠。元朔元年,濊君南閭等畔右渠,率二十八萬口詣遼東內屬,武帝以其地為蒼海郡,數年乃罷。至元封三年,滅朝鮮,分置樂浪、臨屯、玄菟、真番四部。至昭帝始元五年,罷臨屯、真番,以并樂浪、玄菟。玄菟復徙居句驪。自單單大領已東,沃沮、濊貊悉屬樂浪。後以境土廣遠,復分領東七縣,置樂浪東部都尉。自內屬已後,風俗稍薄,法禁亦浸多,至有六十餘條。建武六年,省都尉官,遂棄領東地,悉封其渠帥為縣侯,皆歲時朝賀。

無大君長,其官有侯、邑君、三老。耆舊自謂與句驪同種,言語法俗大抵相類。其人性愚愨,少嗜欲,不請饨。男女皆衣曲領。其俗重山川,山川各有部界,不得妄相干涉。同姓不昏。多所忌諱,疾病死亡,輒捐棄舊宅,更造新居。知種麻,養蠶,作綿布。曉候星宿,豫知年歲豐約。常用十月祭天,晝夜飲酒歌舞,名之為「舞天」。又祠虎以為神。邑落有相侵犯者,輒相罰,責生口牛馬,名之為「責禍」。殺人者償死。少寇盜。能步戰,作矛長三丈,或數人共持之。樂浪檀弓出其地。又多文豹,有果下馬,海出班魚,使來皆獻之。

韓有三種:一曰馬韓,二曰辰韓,三曰弁辰。馬韓在西,有五十四國,其北與樂浪,南與倭接。辰韓在東,十有二國,其北與濊貊接。弁辰在辰韓之南,亦十有二國,其南亦與倭接。凡七十八國,伯濟是其一國焉。大者萬餘戶,小者數千家,各在山海閒,地合方四千餘里,東西以海為限,皆古之辰國也。馬韓最大,共立其種為辰王,都目支國,盡王三韓之地。其諸國王先皆是馬韓種人焉。

馬韓人知田蠶,作綿布。出大栗如梨。有長尾雞,尾長五尺。邑落雜居,亦無城郭。作土室,形如冢,開戶在上。不知跪拜。無長幼男女之別。不貴金寶錦罽,不知騎乘牛馬,唯重瓔珠,以綴衣為飾,及縣頸垂耳。大率皆魁頭露紒,布袍草履。其人壯勇,少年有築室作力者,輒以繩貫脊皮,縋以大木,嚾呼為健。常以五月田竟祭鬼神,晝夜酒會,群聚歌舞,舞輒數十人相隨蹋地為節。十月農功畢,亦復如之。諸國邑各以一人主祭天神,號為「天君」。又立蘇塗,建大木以縣鈴鼓,事鬼神。其南界近倭,亦有文身者。

辰韓,耆老自言秦之亡人,避苦役,適韓國,馬韓割東界地與之。其名國為邦,弓為弧,賊為寇,行酒為行觴,相呼為徒,有似秦語,故或名之為秦韓。有城柵屋室。諸小別邑,各有渠帥,大者名臣智,次有儉側,次有樊秖,次有殺奚,次有邑借。土地肥美,宜五穀。知蠶桑,作縑布。乘駕牛馬。嫁娶以禮。行者讓路。國出鐵,濊、倭、馬韓並從巿之。凡諸貨易,皆以鐵為貨。俗憙歌舞飲酒鼓瑟。兒生欲令其頭扁,皆押之以石。

弁辰與辰韓雜居,城郭衣服皆同,言語風俗有異。其人形皆長大,美髮,衣服絜清。而刑法嚴峻。其國近倭,故頗有文身者。

初,朝鮮王準為衛滿所破,乃將其餘眾數千人走入海,攻馬韓,破之,自立為韓王。準後滅絕,馬韓人復自立為辰王。建武二十年,韓人廉斯人蘇馬諟等詣樂浪貢獻。光武封蘇馬諟為漢廉斯邑君,使屬樂浪郡,四時朝謁。靈帝末,韓、濊並盛,郡縣不能制,百姓苦亂,多流亡入韓者。

馬韓之西,海島上有州胡國。其人短小,髡頭,衣韋衣,有上無下。好養牛豕。乘船往來,貨市韓中。

倭在韓東南大海中,依山島為居,凡百餘國。自武帝滅朝鮮,使驛通於漢者三十許國,國皆稱王,世世傳統。其大倭王居邪馬臺國。樂浪郡徼,去其國萬二千里,去其西北界拘邪韓國七千餘里。其地大較在會稽東冶之東,與朱崖、儋耳相近,故其法俗多同。

土宜禾稻、麻紵、蠶桑,知織績為縑布。出白珠、青玉。其山有丹土。氣溫鹏,冬夏生菜茹。無牛馬虎豹羊鵲。其兵有矛、楯、木弓,竹矢或以骨為鏃。男子皆黥面文身,以其文左右大小別尊卑之差。其男衣皆橫幅結束相連。女人被髮屈紒,衣如單被,貫頭而著之;並以丹朱坋身,如中國之用粉也。有城柵屋室。父母兄弟異處,唯會同男女無別。飲食以手,而用籩豆。俗皆徒跣,以蹲踞為恭敬。人性嗜酒。多壽考,至百餘歲者甚眾。國多女子,大人皆有四五妻,其餘或兩或三。女人不淫不妒。又俗不盜竊,少爭訟。犯法者沒其妻子,重者滅其門族。其死停喪十餘日,家人哭泣,不進酒食,而等類就歌舞為樂。灼骨以卜,用決吉凶。行來度海,令一人不櫛沐,不食肉,不近婦人,名曰「持衰」。若在塗吉利,則雇以財物;如病疾遭害,以為持衰不謹,便共殺之。

建武中元二年,倭奴國奉貢朝賀,使人自稱大夫,倭國之極南界也。光武賜以印綬。安帝永初元年,倭國王帥升等獻生口百六十人,願請見。

桓、靈閒,倭國大亂,更相攻伐,歷年無主。有一女子名曰卑彌呼,年長不嫁,事鬼神道,能以妖惑眾,於是共立為王。侍婢千人,少有見者,唯有男子一人給飲食,傳辭語。居處宮室樓觀城柵,皆持兵守衛。法俗嚴峻。

自女王國東度海千餘里至拘奴國,雖皆倭種,而不屬女王。自女王國南四千餘里至朱儒國,人長三四尺。自朱儒東南行船一年,至裸國、黑齒國,使驛所傳,極於此矣。

會稽海外有東鯷人,分為二十餘國。又有夷洲及澶洲。傳言秦始皇遣方士徐福將童男女數千人入海,求蓬萊神仙不得,徐福畏誅不敢還,遂止此洲,世世相承,有數萬家。人民時至會稽市。會稽東冶縣人有入海行遭風,流移至澶洲者。所在絕遠,不可往來。

論曰:昔箕子違衰殷之運,避地朝鮮。始其國俗未有聞也,及施八條之約,使人知禁,遂乃邑無淫盜,門不夜扃,回頑薄之俗,就寬略之法,行數百千年,故東夷通以柔謹為風,異乎三方者也。苟政之所暢,則道義存焉。仲尼懷憤,以為九夷可居。或疑其陋。子曰:「君子居之,何陋之有!」亦徒有以焉爾。其後遂通接商賈,漸交上國。而燕人衛滿擾雜其風,於是從而澆異焉。老子曰:「法令滋章,盜賊多有。」若箕子之省簡文條而用信義,其得聖賢作法之原矣!

贊曰:宅是嵎夷,曰乃暘谷。巢山潛海,厥區九族。嬴末紛亂,燕人違難。雜華澆本,遂通有漢。眇眇偏譯,或從或畔。


\end{pinyinscope}