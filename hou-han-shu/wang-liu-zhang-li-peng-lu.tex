\article{王劉張李彭盧列傳}

\begin{pinyinscope}
王昌一名郎,趙國邯鄲人也。素為卜相工,明星歷,常以為河北有天子氣。時趙繆王子林好奇數,任俠於趙、魏閒,多通豪猾,而郎與之親善。初,王莽篡位,長安中或自稱成帝子子輿者,莽殺之。郎緣是詐稱真子輿,云「母故成帝謳者,嘗下殿卒僵,須臾有黃氣從上下,半日乃解,遂妊身就館。趙后欲害之,偽易他人子,以故得全。輿年十二,識命者郎中李曼卿,與俱至蜀;十七,到丹陽;二十,還長安;展轉中山,來往燕、趙,以須天時」。林等愈動疑惑;乃與趙國大豪李育、張參等通謀,規共立郎。會人閒傳赤眉將度河,林等因此宣言赤眉當,立劉子輿以觀眾心,百姓多信之。

更始元年十二月,林等遂率車騎數百,晨入邯鄲城,止於王宮,立郎為天子。林為丞相,李育為大司馬,張參為大將軍。分遣將帥,徇下幽、冀。移檄州郡曰:「制詔部剌史、郡太守曰:朕,孝成皇帝子子輿者也。昔遭趙氏之禍,因以王莽篡殺,賴知命者將護朕躬,解形河濱,削跡趙、魏。王莽竊位,獲罪於天,天命佑漢,故使東郡太守翟義、嚴鄉侯劉信,擁兵征討,出入胡、漢。普天率土,知朕隱在人閒。南嶽諸劉,為其先驅。朕仰觀天文,乃興于斯,以今月壬辰即位趙宮。休氣熏蒸,應時獲雨。蓋聞為國,子之襲父,古今不易。劉聖公未知朕,故且持帝號。諸興義兵,咸以助朕,皆當裂土享祚子孫。已詔聖公及翟太守,亟與功臣詣行在所。疑刺史、二千石皆聖公所置,未睹朕之沈滯,或不識去就,強者負力,弱者惶惑。今元元創痍,已過半矣,朕甚悼焉,故遣使者班下詔書。」郎以百姓思漢,既多言翟義不死,故詐稱之,以從人望。於是趙國以北,遼東以西,皆從風而靡。

明年,光武自薊得郎檄,南走信都,發兵徇旁縣,遂攻柏人,不下。議者以為守柏人不如定鉅鹿,光武乃引兵東北圍鉅鹿。郎太守王饒據城,數十日連攻不剋。耿純說曰:「久守王饒,士眾疲敝,不如及大兵精銳,進攻邯鄲。若王郎已誅,王饒不戰自服矣。」光武善其計,乃留將軍鄧滿守鉅鹿,而進軍邯鄲,屯其郭北門。

郎數出戰不利,乃使其諫議大夫杜威持節請降。威雅稱郎實成帝遺體。光武曰:「設使成帝復生,天下不可得,況詐子輿者乎!」威請求萬戶侯。光武曰:「顧得全身可矣。」威曰:「邯鄲雖鄙,并力固守,尚曠日月,終不君臣相率但全身而已。」遂辭而去。急攻之,二十餘日,郎少傅李立為反閒,開門內漢兵,遂拔邯鄲。郎夜亡走,道死,追斬之。

劉永者,梁郡睢陽人,梁孝王八世孫也。傳國至父立。元始中,立與平帝外家衛氏交通,為王莽所誅。

更始即位,永先詣洛陽,紹封為梁王,都睢陽。永聞更始政亂,遂據國起兵,以弟防為輔國大將軍,防弟少公御史大夫,封魯王。遂招諸豪傑沛人周建等,並署為將帥,攻下濟陰、山陽、沛、楚、淮陽、汝南,凡得二十八城。又遣使拜西防賊帥山陽佼彊為橫行將軍。是時東海人董憲起兵據其郡,而張步亦定齊地。永遣使拜憲翼漢大將軍,步輔漢大將軍,與共連兵,遂專據東方。及更始敗,永自稱天子。

建武二年夏,光武遣虎牙大將軍蓋延等伐永。初,陳留人蘇茂為更始討難將軍,與朱鮪等守洛陽。鮪既降漢,茂亦歸命,光武因使茂與蓋延俱攻永。軍中不相能,茂遂反,殺淮陽太守,掠得數縣,據廣樂而臣於永。永以茂為大司馬、淮陽王。蓋延遂圍睢陽,數月,拔之,永將家屬走虞。虞人反,殺其母及妻子,永與麾下數十人奔譙。蘇茂、佼彊、周建合軍救永,為蓋延所敗,茂奔還廣樂,彊、建從永走保湖陵。三年春,永遣使立張步為齊王,董憲為海西王。於是遣大司馬吳漢等圍蘇茂於廣樂,周建率眾救茂,茂、建戰敗,棄城復還湖陵,而睢陽人反城迎永。吳漢與蓋延等合軍圍之,城中食盡,永與茂、建走酇。諸將追急,永將慶吾斬永首降,封吾為列侯。蘇茂、周建奔垂惠,共立永子紆為梁王。佼彊還保西防。

四年秋,遣捕虜將軍馬武、騎都尉王霸圍紆、建於垂惠,蘇茂將五校兵救之,紆、建亦出兵與武等戰,不剋,而建兄子誦反,閉城門拒之。建、茂、紆等皆走,建於道死,茂奔下邳與董憲合,紆奔佼彊。五年,遣驃騎大將軍杜茂攻佼彊於西防,彊與劉紆奔董憲。

時平狄將軍龐萌反叛,遂襲破蓋延,引兵與董憲連和,自號東平王,屯桃鄉之北。

龐萌,山陽人。初亡在下江兵中。更始立,以為冀州牧,將兵屬尚書令謝躬,共破王郎。及躬敗,萌乃歸降。光武即位,以為侍中。萌為人遜順,甚見信愛。帝常稱曰:「可以託六尺之孤,寄百里之命者,龐萌是也。」拜為平狄將軍,與蓋延共擊董憲。

時詔書獨下延而不及萌,萌以為延譖己,自疑,遂反。帝聞之,大怒,乃自將討萌。與諸將書曰:「吾常以龐萌社稷之臣,將軍得無笑其言乎?老賊當族。其各厲兵馬,會睢陽!」憲聞帝自討龐萌,乃與劉紆、蘇茂、佼彊去下邳,還蘭陵,使茂、彊助萌,合兵三萬,急圍桃城。

帝時幸蒙,聞之,乃留輜重,自將輕騎三千,步卒數萬,晨夜馳赴,次任城,去桃鄉六十里。旦日,諸將請進,賊亦勒兵挑戰,帝不聽,乃休士養銳,以挫其鋒。城中聞車駕至,眾心益固。時吳漢等在東郡,馳使召之。萌等乃悉兵攻城,二十餘日,眾疲困而不能下。及吳漢與諸將到,乃率眾軍進桃城,而帝親自搏戰,大破之。萌、茂、彊夜棄輜重逃奔,董憲乃與劉紆悉其兵數萬人屯昌慮,自將銳卒拒新陽。帝先遣吳漢擊破之,憲走還昌慮。漢進守之,憲恐,乃招誘五校餘賊步騎數千人屯建陽,去昌慮三十里。

帝至蕃,去憲所百餘里。諸將請進,帝不聽,知五校乏食當退,敕各堅壁以待其敝。頃之,五校糧盡,果引去。帝乃親臨,四面攻憲,三日,復大破之,眾皆奔散。遣吳漢追擊之,佼彊將其眾降,蘇茂奔張步,憲及龐萌走入繒山。數日,吏士聞憲尚在,復往往相聚,得數百騎,迎憲入郯城。吳漢等復攻拔郯,憲與龐萌走保朐。劉紆不知所歸,軍士高扈斬其首降,梁地悉平。

吳漢進圍朐。明年,城中穀盡,憲、萌潛出,襲取贛榆,琅邪太守陳俊攻之,憲、萌走澤中。會吳漢下朐城,進盡獲其妻子。憲乃流涕謝其將士曰:「妻子皆已得矣。嗟乎!久苦諸卿。」乃將數十騎夜去,欲從閒道歸降,而吳漢校尉韓湛追斬憲於方與,方與人黔陵亦斬萌,皆傳首洛陽。封韓湛為列侯,黔陵關內侯。

張步字文公,琅邪不其人也。漢兵之起,步亦聚眾數千,轉攻傍縣,下數城,自為五威將軍,遂據本郡。

更始遣魏郡王閎為琅邪太守,步拒之,不得進。閎為檄,曉喻吏人降,得贛榆等六縣,收兵數千人,與步戰,不勝。時梁王劉永自以更始所立,貪步兵彊,承制拜步輔漢大將軍、忠節侯,督青徐二州,使征不從命者,步貪其爵號,遂受之。乃理兵於劇,以弟弘為衛將軍,弘弟藍玄武大將軍,藍弟壽高密太守。遣將徇太山、東萊、城陽、膠東、北海、濟南、齊諸郡,皆下之。

步拓地寖廣,兵甲日盛。王閎懼其眾散,乃詣步相見,欲誘以義方。步大陳兵引閎,怒曰:「步有何過,君前見攻之甚乎!」閎按劍曰:「太守奉朝命,而文公擁兵相距,閎攻賊耳,何謂甚邪!」步嘿然,良久,離席跪謝,乃陳樂獻酒,待以上賓之禮,令閎關掌郡事。

建武三年,光武遣光祿大夫伏隆持節使齊,拜步為東萊太守。劉永聞隆至劇,乃馳遣立步為齊王,步即殺隆而受永命。

是時帝方北憂漁陽,南事梁、楚,故步得專集齊地,據郡十二。及劉永死,步等欲立永子紆為天子,自為定漢公,置百官。王閎諫曰:「梁王以奉本朝之故,是以山東頗能歸之。今尊立其子,將疑眾心。且齊人多詐,宜且詳之。」步乃止。五年,步聞帝將攻之,以其將費邑為濟南王,屯歷下。冬,建威大將軍耿弇破斬費邑,進拔臨淄。步以弇兵少遠客,可一舉而取,乃悉將其眾攻弇於臨淄。步兵大敗,還奔劇。帝自幸劇。步退保平壽,蘇茂將萬餘人來救之。茂讓步曰:「以南陽兵精,延岑善戰,而耿弇走之。大王柰何就攻其營?既呼茂,不能待邪?」步曰:「負負,無可言者。」帝乃遣使告步、茂,能相斬降者,封為列侯。步遂斬茂,使使奉其首降。步三弟各自繫所在獄,皆赦之。封步為安丘侯,後與家屬居洛陽。王閎亦詣劇降。

八年夏,步將妻子逃奔臨淮,與弟弘、藍欲招其故眾,乘船入海,琅邪太守陳俊追擊斬之。

王閎者,王莽叔父平阿侯譚之子也,哀帝時為中常侍。時倖臣董賢為大司馬,寵愛貴盛,閎屢諫,忤旨。哀帝臨崩,以璽綬付賢曰:「無妄以與人。」時國無嗣主,內外恇懼,閎白元后,請奪之;即帶劍至宣德後闥,舉手叱賢曰:「宮車晏駕,國嗣未立,公受恩深重,當俯伏號泣,何事久持璽綬以待禍至邪!」賢知閎必死,不敢拒之,乃跪授璽綬。閎持上太后,朝廷壯之。及王莽篡位,僭忌閎,乃出為東郡太守。閎懼誅,常繫藥手內。莽敗,漢兵起,閎獨完全東郡三十餘萬戶,歸降更始。

李憲者,潁川許昌人也。王莽時為廬江屬令。莽末,江賊王州公等起眾十餘萬,攻掠郡縣,莽以憲為偏將軍、廬江連率,擊破州公。莽敗,憲據郡自守。更始元年,自稱淮南王。建武三年,遂自立為天子,置公卿百官,擁九城,眾十餘萬。

四年秋,光武幸壽春,遣揚武將軍馬成等擊憲,圍舒。至六年正月,拔之。憲亡走,其軍士帛意追斬憲而降,憲妻子皆伏誅。封帛意漁浦侯。

後憲餘黨淳于臨等猶聚眾數千人,屯灊山,攻殺安風令。楊州牧歐陽歙遣兵不能剋,帝議欲討之。廬江人陳眾為從事,白歙請得喻降臨;於是乘單車,駕白馬,往說而降之。灊山人共生為立祠,號「白馬陳從事」云。

彭寵字伯通,南陽宛人也。父宏,哀帝時為漁陽太守,偉容貌,能飲飯,有威於邊。王莽居攝,誅不附己者,宏與何武、鮑宣並遇害。

寵少為郡吏,地皇中,為大司空士,從王邑東拒漢軍。到洛陽,聞同產弟在漢兵中,懼誅,即與鄉人吳漢亡至漁陽,抵父時吏。更始立,使謁者韓鴻持節徇北州,承制得專拜二千石已下。鴻至薊,以寵、漢並鄉閭故人,相見歡甚,即拜寵偏將軍,行漁陽太守事,漢安樂令。

及光武鎮慰河北,至薊,以書招寵。寵具牛酒,將上謁。會王郎詐立,傳檄燕、趙,遣將徇漁陽、上谷,急發其兵,北州眾多疑惑,欲從之。吳漢說寵從光武,語在漢傳。會上谷太守耿況亦使功曹寇恂詣寵,結謀共歸光武。寵乃發步騎三千人,以吳漢行長史,及都尉嚴宣、護軍蓋延、狐奴令王梁,與上谷軍合而南,及光武於廣阿。光武承制封寵建忠侯,賜號大將軍。遂圍邯鄲,寵轉糧食,前後不絕。

及王郎死,光武追銅馬,北至薊。寵上謁,自負其功,意望甚高,光武接之不能滿,以此懷不平。光武知之,以問幽州牧朱浮。浮對曰:「前吳漢北發兵時,大王遺寵以所服劍,又倚以為北道主人。寵謂至當迎閤握手,交歡並坐。今既不然,所以失望。」浮因曰:「王莽為宰衡時,甄豐旦夕入謀議,時人語曰『夜半客,甄長伯。』及莽篡位後,豐意不平,卒以誅死。」光武大笑,以為不至於此。及即位,吳漢、王梁,寵之所遣,並為三公,而寵獨無所加,愈怏怏不得志。歎曰:「我功當為王;但爾者,陛下忘我邪?」

是時北州破散,而漁陽差完,有舊鹽鐵官,寵轉以貿穀,積珍寶,益富彊。朱浮與寵不相能,浮數譖搆之。建武二年春,詔徵寵,寵意浮賣己,上疏願與浮俱徵。又與吳漢、蓋延等書,盛言浮枉狀,固求同徵。帝不許,益以自疑。而其妻素剛,不堪抑屈,固勸無受召。寵又與常所親信吏計議,皆懷怨於浮,莫有勸行者。帝遣寵從弟子后蘭卿喻之,寵因留子后蘭卿,遂發兵反,拜署將帥,自將二萬餘人攻朱浮於薊,分兵徇廣陽、上谷、右北平。又自以與耿況俱有重功,而恩賞並薄,數遣使要誘況,況不受,輒斬其使。

秋,帝使游擊將軍鄧隆救薊。隆軍潞南,浮軍雍奴,遣吏奏狀。帝讀檄,怒謂使吏曰:「營相去百里,其勢豈可得相及?比若還,北軍必敗矣。」寵果盛兵臨河以拒隆,又別發輕騎三千襲其後,大破隆軍。浮遠,遂不能救。引而去。明年春,寵遂拔右北平、上谷數縣。遣使以美女繒綵賂遺匈奴,要結和親。單于使左南將軍七八千騎,往來為游兵以助寵。又南結張步及富平獲索諸豪傑,皆與交質連衡。遂攻拔薊城,自立為燕王。

其妻數惡夢,又多見怪變,卜筮及望氣者皆言兵當從中起。寵疑子后蘭卿質漢歸,故不信之,使將兵居外,無親於中。五年春,寵齋,獨在便室。蒼頭子密等三人因寵臥寐,共縛著床,告外吏云:「大王齋禁,皆使吏休。」偽稱寵命教,收縛奴婢,各置一處。又以寵命呼其妻。妻入,大驚。寵急呼曰:「趣為諸將軍辦裝。」於是兩奴將妻入取寶物,留一奴守寵。寵謂守奴曰:「若小兒,我素愛也,今為子密所迫劫耳。解我縛,當以女珠妻汝,家中財物皆與若。」小奴意欲解之,視戶外,見子密聽其語,遂不敢解。於是收金玉衣物,至寵所裝之,被馬六疋,使妻縫兩縑囊。昏夜後,解寵手,令作記告城門將軍云:「今遣子密等至子后蘭卿所,速開門出,勿稽留之。」書成,即斬寵及妻頭,置囊中,便持記馳出城,因以詣闕,封為不義侯。明旦,閤門不開,官屬踰牆而入,見寵屍,驚怖。其尚書韓立等共立寵子午為王,以子后蘭卿為將軍。國師韓利斬午首,詣征虜將軍祭遵降。夷其宗族。

盧芳字君期,安定三水人也,居左谷中,王莽時,天下咸思漢德,芳由是詐自稱武帝曾孫劉文伯。曾祖母匈奴谷蠡渾邪王之姊為武帝皇后,生三子。遭江充之亂,太子誅,皇后坐死,中子次卿亡之長陵,小子回卿逃於左谷。霍將軍立次卿,迎回卿,回卿不出,因居左谷,生子孫卿,孫卿生文伯。常以是言誑惑安定閒。王莽末,乃與三水屬國羌胡起兵。更始至長安,徵芳為騎都尉,使鎮撫安定以西。

更始敗,三水豪傑共計議,以芳劉氏子孫,宜承宗廟,乃共立芳為上將軍、西平王,使使與西羌、匈奴結和親。單于曰:「匈奴本與漢約為兄弟。後匈奴中衰,呼韓邪單于歸漢,漢為發兵擁護,世世稱臣。今漢亦中絕,劉氏來歸我,亦當立之,令尊事我。」乃使句林王將數千騎迎芳,芳與兄禽、弟程俱入匈奴。單于遂立芳為漢帝。以程為中郎將,將胡騎還入安定。初,五原人李興、隨昱,朔方人田颯,代郡人石鮪、閔堪,各起兵自稱將軍。建武四年,單于遣無樓且渠王入五原塞,與李興等和親,告興欲令芳還漢地為帝。五年,李興、閔堪引兵至單于庭迎芳,與俱入塞,都九原縣。掠有五原、朔方、雲中、定襄、鴈門五郡,並置守令,與胡通兵,侵苦北邊。

六年,芳將軍賈覽將胡騎擊殺代郡太守劉興。芳後以事誅其五原太守李興兄弟,而其朔方太守田颯、雲中太守橋扈恐懼,叛芳,舉郡降,光武令領職如故。後大司馬吳漢、驃騎大將軍杜茂數擊芳,並不剋。十二年,芳與賈覽共攻雲中,久不下,其將隨昱留守九原,欲脅芳降。芳知羽翼外附,心膂內離,遂棄輜重,與十餘騎亡入匈奴,其眾盡歸隨昱。昱乃隨使者程恂詣闕。拜昱為五原太守,封鐫胡侯,昱弟憲武進侯。

十六年,芳復入居高柳,與閔堪兄林使使請降。乃立芳為代王,堪為代相,林為代太傅,賜繒二萬匹,因使和集匈奴。芳上疏謝曰:「臣芳過託先帝遺體,棄在邊陲。社稷遭王莽廢絕,以是子孫之憂,所宜共誅,故遂西連羌戎,北懷匈奴。單于不忘舊德,權立救助。是時兵革並起,往往而在。臣非敢有所貪覬,期於奉承宗廟,興立社稷,是以久僭號位,十有餘年,罪宜萬死。陛下聖德高明,躬率眾賢,海內賓服,惠及殊俗。以胏附之故,赦臣芳罪,加以仁恩,封為代王,使備北藩。無以報塞重責,冀必欲和輯匈奴,不敢遺餘力,負恩貸。謹奉天子玉璽,思望闕庭。」詔報芳朝明年正月。其冬,芳入朝,南及昌平,有詔止,令更朝明歲。芳自道還,憂恐,乃復背叛,遂反,與閔堪、閔林相攻連月。匈奴遣數百騎迎芳及妻子出塞。芳留匈奴中十餘年,病死。

初,安定屬國胡與芳為寇,及芳敗,胡人還鄉里,積苦縣官徭役,其中有駮馬少伯者,素剛壯;二十一年,遂率種人反叛,與匈奴連和,屯聚青山。乃遣將兵長史陳訢,率三千騎擊之,少伯乃降。徙於冀縣。

論曰:傳稱「盛德必百世祀」,孔子曰「寬則得眾」。夫能得眾心,則百世不忘矣。觀更始之際,劉氏之遺恩餘烈,英雄豈能抗之哉!然則知高祖、孝文之寬仁,結於人心深矣。周人之思邵公,愛其甘棠,又況其子孫哉!劉氏之再受命,蓋以此乎!若數子者,豈有國之遠圖哉!因時擾攘,苟恣縱而已耳,然猶以附假宗室,能掘強歲月之閒。觀其智略,固無足以憚漢祖,發其英靈者也。

贊曰:天地閉革,野戰群龍。昌、芳僭詐,梁、齊連鋒。寵負強地,憲縈深江。實惟非律,代委神邦。


\end{pinyinscope}