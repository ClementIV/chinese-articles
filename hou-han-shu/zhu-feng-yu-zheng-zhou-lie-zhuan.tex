\article{朱馮虞鄭周列傳}

\begin{pinyinscope}
朱浮字叔元,沛國蕭人也。初從光武為大司馬主簿,遷偏將軍,從破邯鄲。光武遣吳漢誅更始幽州牧苗曾,乃拜浮為大將軍幽州牧,守薊城,遂討定北邊。建武二年,封舞陽侯,食三縣。

浮年少有才能,頗欲厲風跡,收士心,辟召州中名宿涿郡王岑之屬,以為從事,及王莽時故吏二千石,皆引置幕府,乃多發諸郡倉穀,稟贍其妻子。漁陽太守彭寵以為天下未定,師旅方起,不宜多置官屬,以損軍實,不從其令。浮性矜急自多,頗有不平,因以峻文詆之;寵亦佷強,兼負其功,嫌怨轉積。浮密奏寵遣吏迎妻而不迎其母,又受貨賄,殺害友人,多聚兵穀,意計難量。寵既積怨,聞,遂大怒,而舉兵攻浮。浮以書質責之曰:

蓋聞知者順時而謀,愚者逆理而動,常竊悲京城太叔以不知足而無賢輔,卒自棄於鄭也。

伯通以名字典郡,有佐命之功,臨人親職,愛惜倉庫,而浮秉征伐之任,欲權時救急,二者皆為國耳。即疑浮相譖,何不詣闕自陳,而為族滅之計乎?朝廷之於伯通,恩亦厚矣,委以大郡,任以威武,事有柱石之寄,情同子孫之親。匹夫媵母尚能致命一餐,豈有身帶三綬,職典大邦,而不顧恩義,生心外畔者乎!伯通與吏人語,何以為顏?行步拜起,何以為容?坐臥念之,何以為心?引鏡窺影,何施眉目?舉措建功,何以為人?惜乎棄休令之嘉名,造梟鴟之逆謀,捐傳世之慶祚,招破敗之重災,高論堯舜之道,不忍桀紂之性,生為世笑,死為愚鬼,不亦哀乎!

伯通與耿俠遊俱起佐命,同被國恩。俠遊謙讓,屢有降挹之言;而伯通自伐,以為功高天下。往時遼東有豕,生子白頭,異而獻之,行至河東,見群豕皆白,懷慚而還。若以子之功論於朝廷,則為遼東豕也。今乃愚妄,自比六國。六國之時,其埶各盛,廓土數千里,勝兵將百萬,故能據國相持,多歷年世。今天下幾里,列郡幾城,柰何以區區漁陽而結怨天子?此猶河濱之人捧土以塞孟津,多見其不知量也!

方今天下適定,海內願安,士無賢不肖,皆樂立名於世。而伯通獨中風狂走,自捐盛時,內聽驕婦之失計,外信讒邪之諛言,長為群后惡法,永為功臣鑒戒,豈不誤哉!定海內者無私讎,勿以前事自誤,願留意顧老母幼弟。凡舉事無為親厚者所痛,而為見讎者所快。

寵得書愈怒,攻浮轉急。明年,涿郡太守張豐亦舉兵反。

時二郡畔戾,北州憂恐,浮以為天子必自將兵討之,而但遣游擊將軍鄧隆陰助浮。浮懷懼,以為帝怠於敵,不能救之,乃上疏曰:「昔楚宋列國,俱為諸侯,莊王以宋執其使,遂有投袂之師。魏公子顧朋友之要,觸冒強秦之鋒。夫楚魏非有分職匡正之大義也,莊王但為爭強而發忿,公子以一言而立信耳。今彭寵反畔,張豐逆節,以為陛下必棄捐它事,以時滅之。既歷時月,寂寞無音。從圍城而不救,放逆虜而不討,臣誠惑之。昔高祖聖武,天下既定,猶身自征伐,未嘗寧居。陛下雖興大業,海內未集,而獨逸豫,不顧北垂,百姓遑遑,無所繫心,三河、冀州,曷足以傳後哉!今秋稼已孰,復為漁陽所掠。張豐狂悖,姦黨日增,連年拒守,吏士疲勞,甲冑生蟣蝨,弓弩不得弛,上下燋心,相望救護,仰希陛下生活之恩。」詔報曰:「往年赤眉跋扈長安,吾策其無穀必東,果來歸降。今度此反虜,埶無久全,其中必有內相斬者。今軍資未充,故須後麥耳。」浮城中糧盡,人相食。會上谷太守耿況遣騎來救浮,浮乃得遁走。南至良鄉,其兵長反遮之,浮恐不得脫,乃下馬刺殺其妻,僅以身免,城降於寵。尚書令侯霸奏浮敗亂幽州,構成寵罪,徒勞軍師,不能死節,罪當伏誅。帝不忍,以浮代賈復為執金吾,徙封父城侯。後豐、寵並自敗。

帝以二千石長吏多不勝任,時有纖微之過者,必見斥罷,交易紛擾,百姓不寧。六年有日食之異,浮因上疏曰:「臣聞日者眾陽之所宗,君上之位也。凡居官治民,據郡典縣,皆為陽為上,為尊為長。若陽上不明,尊長不足,則干動三光,垂示王者。五典紀國家之政,鴻範別災異之文,皆宣明天道,以徵來事者也。陛下哀龟海內新離禍毒,保宥生人,使得蘇息。而今牧人之吏,多未稱職,小違理實,輒見斥罷,豈不粲然黑白分明哉!然以堯舜之盛,猶加三考,大漢之興,亦累功效,吏皆積久,養老於官,至名子孫,因為氏姓。當時吏職,何能悉理;論議之徒,豈不諠譁。蓋以為天地之功不可倉卒,艱難之業當累日也。而閒者守宰數見換易,迎新相代,疲勞道路。尋其視事日淺,未足昭見其職,既加嚴切,人不自保,各相顧望,無自安之心。有司或因睚眥以騁私怨,苟求長短,求媚上意。二千石及長吏迫於舉劾,懼於刺譏,故爭飾詐偽,以希虛譽。斯皆群陽騷動,日月失行之應。夫物暴長者必夭折,功卒成者必亟壞,如摧長久之業,而造速成之功,非陛下之福也。天下非一時之用也,海內非一旦之功也。願陛下遊意於經年之外,望化於一世之後。天下幸甚。」帝下其議,群臣多同於浮,自是牧守易代頗簡。

舊制,州牧奏二千石長吏不任位者,事皆先下三公,三公遣掾史案驗,然後黜退。帝時用明察,不復委任三府,而權歸刺舉之吏。浮復上疏曰:「陛下清明履約,率禮無違,自宗室諸王、外家后親,皆奉遵繩墨,無黨埶之名。至或乘牛車,齊於編人。斯固法令整齊,下無作威者也。求之於事,宜以和平,而災異猶見者,而豈徒然?天道信誠,不可不察。竊見陛下疾往者上威不行,下專國命,即位以來,不用舊典,信刺舉之官,黜鼎輔之任,至於有所劾奏,便加免退,覆案不關三府,罪譴不蒙澄察。陛下以使者為腹心,而使者以從事為耳目,是為尚書之平,決於百石之吏,故群下苛刻,各自為能。兼以私情容長,憎愛在職,皆競張空虛,以要時利,故有罪者心不厭服,無咎者坐被空文,不可經盛衰,貽後王也。夫事積久則吏自重。吏安則人自靜。傳曰:『五年再閏,天道乃備。』夫以天地之靈,猶五載以成其化,況人道哉!臣浮愚戇,不勝惓惓,願陛下留心千里之任,省察偏言之奏。」

七年,轉太僕。浮又以國學既興,宜廣博士之選,乃上書曰:「夫太學者,禮義之宮,教化所由興也。陛下尊敬先聖,垂意古典,宮室未飾,干戈未休,而先建太學,進立橫舍,比日車駕親臨觀饗,將以弘時雍之化,顯勉進之功也。尋博士之官,為天下宗師,使孔聖之言傳而不絕。舊事,策試博士,必廣求詳選,爰自畿夏,延及四方,是以博舉明經,唯賢是登,學者精勵,遠近同慕。伏聞詔書更試五人,唯取見在洛陽城者。臣恐自今以往,將有所失。求之密邇,容或未盡,而四方之學,無所勸樂。凡策試之本,貴得其真,非有期會,不及遠方也。又諸所徵試,皆私自發遣,非有傷費煩擾於事也。語曰:『中國失禮,求之於野。』臣浮幸得與講圖讖,故敢越職。」帝然之。

二十年,代竇融為大司空。二十二年,坐賣弄國恩免。二十五年,徙封新息侯。

帝以浮陵轢同列,每銜之,惜其功能,不忍加罪。永平中,有人單辭告浮事者,顯宗大怒,賜浮死。長水校尉樊儵言於帝曰:「唐堯大聖,兆人獲所,尚優遊四凶之獄,厭服海內之心,使天下咸知,然後殛罰。浮事雖昭明,而未達人聽,宜下廷尉,章著其事。」帝亦悔之。

論曰:吳起與田文論功,文不及者三,朱買臣難公孫弘十策,弘不得其一,終之田文相魏,公孫宰漢,誠知宰相自有體也。故曾子曰:「君子所貴乎道者三,籩豆之事則有司存。」而光武、明帝躬好吏事,亦以課覈三公,其人或失而其禮稍薄,至有誅斥詰辱之累。任職責過,一至於此,追感賈生之論,不亦篤乎!朱浮譏諷苛察欲速之弊,然矣,焉得長者之言哉!

馮魴字孝孫,南陽湖陽人也。其先魏之支別,食菜馮城,因以氏焉。秦滅魏,遷于湖陽,為郡族姓。

王莽末,四方潰畔,魴乃聚賓客,招豪桀,作營鹗,以待所歸。是時湖陽大姓虞都尉反城稱兵,先與同縣申屠季有仇,而殺其兄,謀滅季族。季亡歸魴,魴將季欲還其營,道逢都尉從弟長卿來,欲執季。魴叱長卿曰:「我與季雖無素故,士窮相歸,要當以死任之,卿為何言?」遂與俱歸。季謝曰:「蒙恩得全,死無以為報恩,有牛馬財物,願悉獻之。」魴作色曰:「吾老親弱弟皆賊城中,今日相與,尚無所顧,何云財物乎?」季慚不敢復言。魴自是為縣邑所敬信,故能據營自固。

時天下未定,而四方之士擁兵矯稱者甚眾,唯魴自守,兼有方略。光武聞而嘉之,建武三年,徵詣行在所,見於雲臺,拜虞令。為政敢殺伐,以威信稱。遷郟令。後車駕西征隗囂,潁川盜賊群起,郟賊延褒等眾三千餘人,攻圍縣舍,魴率吏士七十許人,力戰連日,弩矢盡,城陷,魴乃遁去。帝聞郡國反,即馳赴潁川,魴詣行在所。帝案行鬥處,知魴力戰,乃嘉之曰:「此健令也。所當討擊,勿拘州郡。」褒等聞帝至,皆自髡剔,負鈇鑕,將其眾請罪。帝且赦之,使魴轉降諸聚落,縣中平定,詔乃悉以褒等還魴誅之。魴責讓以行軍法,皆叩頭曰:「今日受誅,死無所恨。」魴曰:「汝知悔過伏罪,今一切相赦,聽各反農桑,為令作耳目。」皆稱萬歲。是時每有盜賊,並為褒等所發,無敢動者,縣界清靜。

十三年,遷魏郡太守。二十七年,以高第入代趙憙為太僕。中元元年,從東封岱宗,行衛尉事。還,代張純為司空,賜爵關內侯。二年,帝崩,使魴持節起原陵,更封楊邑鄉侯,食三百五十戶。永平四年,坐考隴西太守鄧融,聽任姦吏,策免,削爵土。六年,顯宗幸魯,復行衛尉事。七年,代陰嵩為執金吾。

魴性矜嚴公正,在位數進忠言,多見納用。十四年,詔復爵土。明年,東巡郡國,留魴宿衛南宮。建初三年,以老病乞身,肅宗許之。其冬為五更,詔魴朝賀,就列侯位。元和二年,卒,時年八十六。

子柱嗣,尚顯宗女獲嘉長公主,少為侍中,以恭肅謙約稱,位至將作大匠。柱卒,子定嗣,官至羽林中郎將。定卒,無子,國除。

定弟石,襲母公主封獲嘉侯,亦為侍中,稍遷衛尉。能取悅當世,為安帝所寵。帝嘗幸其府,留飲十許日,賜駁犀具劍、佩刀、紫艾綬、玉玦各一,拜子世為黃門侍郎,世弟二人皆郎中。自永初兵荒,王侯租秩多不充,於是特詔以它縣租稅足石,令如舊限,歲入穀三萬斛,錢四萬。遷光祿勳,遂代楊震為太尉。及北鄉侯立,遷太傅,與太尉東萊劉喜參錄尚書事。順帝既立,石與喜皆以阿黨閻顯、江京等策免,復為衛尉。卒,子代嗣。代卒,弟承嗣,為步兵校尉。

石弟珖,和帝時詔封楊邑侯,亦以石寵,官至城門校尉。卒,子肅嗣,為黃門侍郎。

虞延字子大,陳留東昏人也。延初生,其上有物若一匹練,遂上升天,占者以為吉。及長,長八尺六寸,要帶十圍,力能扛鼎。少為戶牖亭長。時王莽貴人魏氏賓客放從,延率吏卒突入其家捕之,以此見怨,故位不升。性敦朴,不拘小節,又無鄉曲之譽。王莽末,天下大亂,延常嬰甲冑,擁衛親族,扞禦鈔盜,賴其全者甚眾。延從女弟年在孩乳,其母不能活之,棄於溝中,延聞其號聲,哀而收之,養至成人。建武初,仕執金吾府,除細陽令。每至歲時伏臘,輒休遣徒繫,各使歸家,並感其恩德,應期而還。有囚於家被病,自載詣獄,既至而死,延率掾吏,殯于門外,百姓感悅之。

後去官還鄉里,太守富宗聞延名,召署功曹。宗性奢靡,車服器物,多不中節。延諫曰:「昔晏嬰輔齊,鹿裘不完,季文子相魯,妾不衣帛,以約失之者鮮矣。」宗不悅,延即辭退。居有頃,宗果以侈從被誅,臨當伏刑,攬涕而歎曰:「恨不用功曹虞延之諫!」光武聞而奇之。二十年東巡,路過小黃,高帝母昭靈后園陵在焉,時延為部督郵,詔呼引見,問園陵之事。延進止從容,占拜可觀,其陵樹株櫱,皆諳其數,俎豆犧牲,頗曉其禮。帝善之,敕延從駕到魯。還經封丘城門,門下小,不容羽蓋,帝怒,使撻侍御史,延因下見引咎,以為罪在督郵。言辭激揚,有感帝意,乃制誥曰:「以陳留督郵虞延故,貰御史罪。」延從送車駕西盡郡界,賜錢及劍帶佩刀還郡,於是聲名遂振。

二十三年,司徒玉況辟焉。時元正朝賀,帝望而識延,遣小黃門馳問之,即日召拜公車令。明年,遷洛陽令。是時陰氏有客馬成者,常為姦盜,延收考之。陰氏屢請,獲一書輒加篣二百。信陽侯陰就乃訴帝,譖延多所冤枉。帝乃臨御道之館,親錄囚徒。延陳其獄狀可論者在東,無理者居西。成乃回欲趨東,延前執之,謂曰:「爾人之巨蠹,久依城社,不畏熏燒。今考實未竟,宜當盡法!」成大呼稱枉,陛戟郎以戟刺延,叱使置之。帝知延不私,謂成曰:「汝犯王法,身自取之!」呵使速去。後數日伏誅。於是外戚斂手,莫敢干法。在縣三年,遷南陽太守。

永平初,有新野功曹鄧衍,以外戚小侯每豫朝會,而容姿趨步,有出於眾,顯宗目之,顧左右曰:「朕之儀貌,豈若此人!」特賜輿馬衣服。延以衍雖有容儀而無實行,未嘗加禮。帝既異之,乃詔衍令自稱南陽功曹詣闕。既到,拜郎中,遷玄武司馬。衍在職不服父喪,帝聞之,乃歎曰:「『知人則哲,惟帝難之。』信哉斯言!」衍慚而退,由是以延為明。

三年,徵代趙憙為太尉;八年,代范遷為司徒。歷位二府,十餘年無異政績。會楚王英謀反,陰氏欲中傷之,使人私以楚謀告延,延以英藩戚至親,不然其言,又欲辟幽州從事公孫弘,以弘交通楚王而止,並不奏聞。及英事發覺,詔書切讓,延遂自殺。家至清貧,子孫不免寒餧。

延從曾孫放,字子仲。少為太尉楊震門徒,及震被讒自殺,順帝初,放詣闕追訟震罪,由是知名。桓帝時為尚書,以議誅大將軍梁冀功封都亭侯,後為司空,坐水災免。性疾惡宦官,遂為所陷,靈帝初,與長樂少府李膺等俱以黨事誅。

鄭弘字巨君,會稽山陰人也。從祖吉,宣帝時為西域都護。弘少為鄉嗇夫,太守第五倫行春,見而深奇之,召署督郵,舉孝廉。

弘師同郡河東太守焦貺。楚王英謀反發覺,以疏引貺,貺被收捕,疾病於道亡沒,妻子閉繫詔獄,掠考連年。諸生故人懼相連及,皆改變名姓,以逃其禍,弘獨髡頭負鈇鑕,詣闕上章,為貺訟罪。顯宗覺悟,即赦其家屬,弘躬送貺喪及妻子還鄉里,由是顯名。

拜為騶令,政有仁惠,民稱蘇息。遷淮陰太守。四遷,建初,為尚書令。舊制,尚書郎限滿補縣長令史丞尉。弘奏以為臺職雖尊,而酬賞甚薄,至於開選,多無樂者,請使郎補千石,令史為長。帝從其議。弘前後所陳有補益王政者,皆著之南宮,以為故事。

出為平原相,徵拜侍中。建初八年,代鄭眾為大司農。舊交阯七郡貢獻轉運,皆從東冶汎海而至,風波艱阻,沈溺相係。弘奏開零陵、桂陽嶠道,於是夷通,至今遂為常路。在職二年,所息省三億萬計。時歲天下遭旱,邊方有警,人食不足,而帑藏殷積。弘又奏宜省貢獻,減徭費,以利飢人。帝順其議。

元和元年,代鄧彪為太尉。時舉將第五倫為司空,班次在下,每正朔朝見,弘曲躬而自卑。帝問知其故,遂聽置雲母屏風,分隔其閒,由此以為故事。在位四年,奏尚書張林阿附侍中竇憲,而素行臧穢,又上洛陽令楊光,憲之賓客,在官貪殘,並不宜處位。書奏,吏與光故舊,因以告之。光報憲,憲奏弘大臣漏泄密事。帝詰讓弘,收上印綬。弘自詣廷尉,詔敕出之,因乞骸骨歸,未許。病篤,上書陳謝,并言竇憲之短。帝省章,遣醫占弘病,比至已卒。臨歿悉還賜物,敕妻子褐巾布衣素棺殯殮,以還鄉里。

周章字次叔,南陽隨人也。初仕郡為功曹。時大將軍竇憲免,封冠軍侯就國。章從太守行春到冠軍,太守猶欲謁之。章進諫曰:「今日公行春,豈可越儀私交。且憲椒房之親,埶傾王室,而退就藩國,禍福難量。明府剖符大臣,千里重任,舉止進退,其可輕乎?」太守不聽,遂便升車。章前拔佩刀絕馬鞅,於是乃止。及憲被誅,公卿以下多以交關得罪,太守幸免,以此重章。舉孝廉,六遷為五官中郎將。延平元年,為光祿勳。

永初元年,代魏霸為太常。其冬,代尹勤為司空。是時中常侍鄭眾、蔡倫等皆秉埶豫政,章數進直言。初,和帝崩,鄧太后以皇子勝有痼疾,不可奉承宗廟,貪殤帝孩抱,養為己子,故立之,以勝為平原王。及殤帝崩,群臣以勝疾非痼,意咸歸之,太后以前既不立,恐後為怨,乃立和帝兄清河孝王子祐,是為安帝。章以眾心不附,遂密謀閉宮門,誅車騎將軍鄧騭兄弟及鄭眾、蔡倫,劫尚書,廢太后於南宮,封帝為遠國王,而立平原王。事覺,勝策免,章自殺。家無餘財,諸子易衣而出,并日而食。

論曰:孔子稱「可與立,未可與權」。權也者,反常者也。將從反常之事,必資非常之會,使夫舉無違妄,志行名全。周章身非負圖之託,德乏萬夫之望,主無絕天之舋,地有既安之埶,而創慮於難圖,希功於理絕,不已悖乎!如令君器易以下議,即斗筲必能叨天業,狂夫豎臣亦自奮矣。孟軻有言曰:「有伊尹之心則可,無伊尹之心則篡矣。」於戲,方來之人戒之哉!

贊曰:朱定北州,激成寵尤。魴用降帑,延感歸囚。鄭、竇怨偶,代相為仇。周章反道,小智大謀。


\end{pinyinscope}