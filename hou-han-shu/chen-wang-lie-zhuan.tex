\article{陳王列傳}

\begin{pinyinscope}
陳蕃字仲舉,汝南平輿人也。祖河東太守。蕃年十五,嘗閑處一室,而庭宇蕪穢。父友同郡薛勤來候之,謂蕃曰:「孺子何不洒埽以待賓客?」蕃曰:「大丈夫處世,當埽除天下,安事一室乎!」勤知其有清世志,甚奇之。

初仕郡,舉孝廉,除郎中。遭母憂,棄官行喪。服闋,刺史周景辟別駕從事,以諫爭不合,投傳而去。後公府辟舉方正,皆不就。

太尉李固表薦,徵拜議郎,再遷為樂安太守。時李膺為青州刺史,名有威政,屬城聞風,皆自引去,蕃獨以清績留。郡人周璆,高絜之士。前後郡守招命莫肯至,唯蕃能致焉。字而不名,特為置一榻,去則縣之。璆字孟玉,臨濟人,有美名。民有趙宣葬親而不閉埏隧,因居其中,行服二十餘年,鄉邑稱孝,州郡數禮請之。郡內以薦蕃,蕃與相見,問及妻子,而宣五子皆服中所生。蕃大怒曰:「聖人制禮,賢者俯就,不肖企及。且祭不欲數,以其易黷故也。況乃寢宿冢藏,而孕育其中,誑時惑眾,誣汙鬼神乎?」遂致其罪。

大將軍梁冀威震天下,時遣書詣蕃,有所請託,不得通,使者詐求謁,蕃怒,笞殺之,坐左轉脩武令。稍遷,拜尚書。

時零陵、桂陽山賊為害,公卿議遣討之,又詔下州郡,一切皆得舉孝廉、茂才。蕃上疏駮之曰:「昔高祖創業,萬邦息肩,撫養百姓,同之赤子。今二郡之民,亦陛下赤子也。致令赤子為害,豈非所在貪虐,使其然乎?宜嚴敕三府,隱覈牧守令長,其有在政失和,侵暴百姓者,即便舉奏,更選清賢奉公之人,能班宣法令情在愛惠者,可不勞王師,而群賊弭息矣。又三署郎吏二千餘人,三府掾屬過限未除,但當擇善而授之,簡惡而去之。豈煩一切之詔,以長請屬之路乎!」以此忤左右,故出為豫章太守。性方峻,不接賓客,士民亦畏其高。徵為尚書令,送者不出郭門。

遷大鴻臚。會白馬令李雲抗疏諫,桓帝怒,當伏誅。蕃上書救雲,坐免歸田里。

復徵拜議郎,數日遷光祿勳。時封賞踰制,內寵猥盛,蕃乃上疏諫曰:「臣聞有事社稷者,社稷是為;有事人君者,容悅是為。今臣蒙恩聖朝,備位九列,見非不諫,則容悅也。夫諸侯上象四七,垂燿在天,下應分土,藩屏上國。高祖之約,非功臣不侯。而聞追錄河南尹鄧萬世父遵之微功,更爵尚書令黃雋先人之絕封,近習以非義授邑,左右以無功傳賞,授位不料其任,裂土莫紀其功,至乃一門之內,侯者數人,故緯象失度,陰陽謬序,稼用不成,民用不康。臣知封事已行,言之無及,誠欲陛下從是而止。又比年收斂,十傷五六,萬人飢寒,不聊生活,而采女數千,食肉衣綺,脂油粉黛,不可貲計。鄙諺言『盜不過五女門』,以女貧家也。今後宮之女,豈不貧國乎!是以傾宮嫁而天下化,楚女悲而西宮災。且聚而不御,必生憂悲之感,以致并隔水旱之困。夫獄以禁止姦違,官以稱才理物。若法虧於平,官失其人,則王道有缺。而令天下之論,皆謂獄由怨起,爵以賄成。夫不有臭穢,則蒼蠅不飛。陛下宜採求失得,擇從忠善。尺一選舉,委尚書三公,使褒責誅賞,各有所歸,豈不幸甚!」帝頗納其言,為出宮女五百餘人,但賜雋爵關內侯,而萬世南鄉侯。

延熹六年,車駕幸廣城校獵。蕃上疏諫曰:「臣聞人君有事於苑囿,唯仲秋西郊,順時講武,殺禽助祭,以敦孝敬。如或違此,則為肆縱。故皋陶戒舜『無教逸遊』,周公戒成王『無槃于遊田』。虞舜、成王猶有此戒,況德不及二主者乎!夫安平之時,尚宜有節,況當今之世,有三空之厄哉!田野空,朝廷空,倉庫空,是謂三空。加兵戎未戢,四方離散,是陛下焦心毀顏,坐以待旦之時也。豈宜揚旗曜武,騁心輿馬之觀乎!又前秋多雨,民始種麥。今失其勸種之時,而令給驅禽除路之役,非賢聖恤民之意也。齊景公欲觀於海,放乎琅邪,晏子為陳百姓惡聞旌旗輿馬之音,舉首嚬眉之感,景公為之不行。周穆王欲肆車轍馬跡,祭公謀父為誦祈招之詩,以止其心。誠惡逸遊之害人也。」書奏不納。

自蕃為光祿勳,與五官中郎將黃琬共典選舉,不偏權富,而為埶家郎所譖訴,坐免歸。頃之,徵為尚僕射,轉太中大夫。八年,代楊秉為太尉。蕃讓曰:「『不愆不忘,率由舊章,』臣不如太常胡廣。齊七政,訓五典,臣不如議郎王暢。聰明亮達,文武兼姿,臣不如弛刑徒李膺。」帝不許。

中常侍蘇康、管霸等復被任用,遂排陷忠良,共相阿媚。大司農劉祐、廷尉馮緄、河南尹李膺,皆以忤旨,為之抵罪。蕃因朝會,固理膺等,請加原宥,升之爵任。言及反覆,誠辭懇切。帝不聽,因流涕而起。時小黃門趙津、南陽大猾張汜等,奉事中官,乘埶犯法,二郡太守劉暧、成档考案其罪,雖經赦令,而並竟考殺之。宦官怨恚,有司承旨,遂奏暧、档罪當棄市。又山陽太守翟超,沒入中常侍侯覽財產,東海相黃浮,誅殺下邳令徐宣,超、浮並坐髡鉗,輸作左校。蕃與司徒劉矩、司空劉茂共諫請暧、档、超、浮等,帝不悅。有司劾奏之,矩、茂不敢復言。蕃乃獨上疏曰:「臣聞齊桓修霸,務為內政;春秋於魯,小惡必書。宜先自整敕,後以及人。今寇賊在外,四支之疾;內政不理,心腹之患。臣寢不能寐,食不能飽,實憂左右日親,忠言以疏,內患漸積,外難方深。陛下超從列侯,繼承天位。小家畜產百萬之資,子孫尚恥愧失其先業,況乃產兼天下,受之先帝,而欲懈怠以自輕忽乎?誠不愛己,不當念先帝得之勤苦邪?前梁氏五族,毒遍海內,天啟聖意,收而戮之,天下之議,冀當小平。明鑒未遠,覆車如昨,而近習之權,復相扇結。小黃門趙津、大猾張汜等,肆行貪虐,姦媚左右,前太原太守劉暧、南陽太守成档,糾而戮之。雖言赦後不當誅殺,原其誠心,在乎去惡。至於陛下,有何悁悁?而小人道長,營惑聖聽,遂使天威為之發怒。如加刑謫,已為過甚,況乃重罰,令伏歐刀乎!又前山陽太守翟超、東海相黃浮,奉公不橈,疾惡如讎,超沒侯覽財物,浮誅徐宣之罪,並蒙刑坐,不逢赦恕,覽之從橫,沒財已幸;宣犯釁過,死有餘辜。昔丞相申屠嘉召責鄧通,洛陽令董宣折辱公主,而文帝從而請之,光武加以重賞,未聞二臣有專命之誅。而今左右群豎,惡傷黨類,妄相交搆,致此刑譴。聞臣是言,當復啼訴。陛下深宜割塞近習豫政之源,引納尚書朝省之事,公卿大官,五日壹朝,簡練清高,斥黜佞邪。如是天和於上,地洽於下,休禎符瑞,豈遠乎哉!陛下雖厭毒臣言,凡人主有自勉強,敢以死陳。」帝得奏愈怒,竟無所納。朝廷眾庶莫不怨之。宦官由此疾蕃彌甚,選舉奏議,輒以中詔譴卻,長吏已下多至抵罪。猶以蕃名臣,不敢加害。暧字文理,高唐人。档字幼平,陝人。並有經術稱,處位敢直言,多所搏擊,知名當時,皆死於獄中。

九年,李膺等以黨事下獄考實。蕃因上疏極諫曰:「臣聞賢明之君,委心輔佐;亡國之主,諱聞直辭。故湯武雖聖,而興於伊呂;桀紂迷惑,亡在失人。由此言之,君為元首,臣為股肱,同體相須,共成美惡者也。伏見前司隸校尉李膺、太僕杜密、太尉掾范滂等,正身無玷,死心社稷。以忠忤旨,橫加考案,或禁錮閉隔,或死徙非所。杜塞天下之口,聾盲一世之人,與秦焚書阬儒,何以為異?昔武王克殷,表閭封墓,今陛下臨政,先誅忠賢。遇善何薄?待惡何優?夫讒人似實,巧言如簧,使聽之者惑,視之者昏。夫吉凶之效,存乎識善;成敗之機,在於察言。人君者,攝天地之政,秉四海之維,舉動不可以違聖法,進退不可以離道規。謬言出口,則亂及八方,何況髡無罪於獄,殺無辜於市乎!昔禹巡狩蒼梧,見市殺人,下車而哭之曰:『萬方有罪,在予一人!』故其興也勃焉。又青、徐炎旱,五穀損傷,民物流遷,茹菽不足。而宮女積於房掖,國用盡於羅紈,外戚私門,貪財受賂,所謂『祿去公室,政在大夫』。昔春秋之末,周德衰微,數十年閒無復災眚者,天所棄也。天之於漢,悢悢無已,故殷勤示變,以悟陛下。除妖去孽,實在脩德。臣位列台司,憂責深重,不敢尸祿惜生,坐觀成敗。如蒙採錄,使身首分裂,異門而出,所不恨也。」帝諱其言切,託以蕃辟召非其人,遂策免之。

永康元年,帝崩。竇后臨朝,詔曰:「夫民生樹君,使司牧之,必須良佐,以固王業。前太尉陳蕃,忠清直亮。其以蕃為太傅,錄尚書事。」時新遭大喪,國嗣未立,諸尚書畏懼權官,託病不朝。蕃以書責之曰:「古人立節,事亡如存。今帝祚未立,政事日蹙,諸君柰何委荼蓼之苦,息偃在床?於義不足,焉得仁乎!」諸尚書惶怖,皆起視事。

靈帝即位,竇太后復優詔蕃曰:「蓋褒功以勸善,表義以厲俗,無德不報,大雅所歎。太傅陳蕃,輔弼先帝,出內累年。忠孝之美,德冠本朝;謇愕之操,華首彌固。今封蕃高陽鄉侯,食邑三百戶。」蕃上疏讓曰:「使者即臣廬,授高陽鄉侯印綬,臣誠悼心,不知所裁。臣聞讓,身之文,德之昭也,然不敢盜以為名。竊惟割地之封,功德是為。臣孰自思省,前後歷職,無它異能,合亦食祿,不合亦食祿。臣雖無素絜之行,竊慕『君子不以其道得之,不居也』。若受爵不讓,掩面就之,使皇天震怒,災流下民,於臣之身,亦何所寄?顧惟陛下哀臣朽老,戒之在得。」竇太后不許,蕃復固讓,章前後十上,竟不受封。

初,桓帝欲立所幸田貴人為皇后。蕃以田氏卑微,竇族良家,爭之甚固。帝不得已,乃立竇后。及后臨朝,故委用於蕃。蕃與后父大將軍竇武,同心盡力,徵用名賢,共參政事,天下之士,莫不延頸想望太平。而帝乳母趙嬈,旦夕在太后側,中常侍曹節、王甫等與共交搆,諂事太后。太后信之,數出詔命,有所封拜,及其支類,多行貪虐。蕃常疾之,志誅中官,會竇武亦有謀。蕃自以既從人望而德於太后,必謂其志可申,乃先上疏曰:「臣聞言不直而行不正,則為欺乎天而負乎人。危言極意,則群凶側目,禍不旋踵。鈞此二者,臣寧得禍,不敢欺天也。今京師囂囂,道路諠譁,言侯覽、曹節、公乘昕、王甫、鄭颯等與趙夫人諸女尚書並亂天下。附從者升進,忤逆者中傷。方今一朝群臣,如河中木耳,汎汎東西,耽祿畏害。陛下前始攝位,順天行誅,蘇康、管霸並伏其辜。是時天地清明,人鬼歡喜,柰何數月復縱左右?元惡大姦,莫此之甚。今不急誅,必生變亂,傾危社稷,其禍難量。願出臣章宣示左右,並令天下諸姦知臣疾之。」太后不納,朝廷聞者莫不震恐。蕃因與竇武謀之,語在武傳。

及事泄,曹節等矯詔誅武等。蕃時年七十餘,聞難作,將官屬諸生八十餘人,並拔刃突入承明門,攘臂呼曰:「大將軍忠以衛國,黃門反逆,何云竇氏不道邪?」王甫時出,與蕃相迕,適聞其言,而讓蕃曰:「先帝新棄天下,山陵未成,竇武何功,兄弟父子,一門三侯?又多取掖庭宮人,作樂飲讌,旬月之閒,貲財億計。大臣若此,是為道邪?公為棟梁,枉橈阿黨,復焉求賊!」遂令收蕃。蕃拔劍叱甫,甫兵不敢近,乃益人圍之數十重,遂執蕃送黃門北寺獄。黃門從官騶蹋踧蕃曰:「死老魅!復能損我曹員數,奪我曹稟假不?」即日害之。徙其家屬於比景,宗族、門生、故吏皆斥免禁錮。

蕃友人陳留朱震,時為銍令,聞而棄官哭之,收葬蕃尸,匿其子逸於甘陵界中。事覺繫獄,合門桎梏。震受考掠,誓死不言,故逸得免。後黃巾賊起,大赦黨人,乃追還逸,官至魯相。

震字伯厚,初為州從事,奏濟陰太守單匡臧罪,并連匡兄中常侍車騎將軍超。桓帝收匡下廷尉,以譴超,超詣獄謝。三府諺曰:「車如雞栖馬如狗,疾惡如風朱伯厚。」

論曰:桓、靈之世,若陳蕃之徒,咸能樹立風聲,抗論惛俗。而驅馳嶮阨之中,與刑人腐夫同朝爭衡,終取滅亡之禍者,彼非不能絜情志,違埃霧也。愍夫世士以離俗為高,而人倫莫相恤也。以遯世為非義,故屢退而不去;以仁心為己任,雖道遠而彌厲。及遭際會,協策竇武,自謂萬世一遇也。懍懍乎伊、望之業矣!功雖不終,然其信義足以攜持民心。漢世亂而不亡,百餘年閒,數公之力也。

王允字子師,太原祁人也。世仕州郡為冠蓋。同郡郭林宗嘗見允而奇之,曰:「王生一日千里,王佐才也。」遂與定交。

年十九,為郡吏。時小黃門晉陽趙津貪橫放恣,為一縣巨患,允討捕殺之。而津兄弟諂事宦官,因緣譖訴,桓帝震怒,徵太守劉暧,遂下獄死。允送喪還平原,終畢三年,然後歸家。復還仕,郡人有路佛者,少無名行,而太守王球召以補吏,允犯顏固爭,球怒,收允欲殺之。刺史鄧盛聞而馳傳辟為別駕從事。允由是知名,而路佛以之廢棄。

允少好大節,有志於立功,常習誦經傳,朝夕試馳射。三公並辟,以司徒高第為侍御史。中平元年,黃巾賊起,特選拜豫州刺史。辟荀爽、孔融等為從事,上除禁黨。討擊黃巾別帥,大破之,與左中郎將皇甫嵩、右中郎將朱雋等受降數十萬。於賊中得中常侍張讓賓客書疏,與黃巾交通,允具發其姦,以狀聞。靈帝責怒讓,讓叩頭陳謝,竟不能罪之。而讓懷協忿怨,以事中允。明年,遂傳下獄。

會赦,還復刺史。旬日閒,復以它罪被捕。司徒楊賜以允素高,不欲使更楚辱,乃遣客謝之曰:「君以張讓之事,故一月再徵。凶慝難量,幸為深計。」又諸從事好氣決者,共流涕奉藥而進之。允厲聲曰:「吾為人臣,獲罪於君,當伏大辟以謝天下,豈有乳藥求死乎!」投杯而起,出就檻車。既至廷尉,左右皆促其事,朝臣莫不歎息。大將軍何進、太尉袁隗、司徒楊賜共上疏請之曰:「夫內視反聽,則忠臣竭誠;寬賢矜能,則義士厲節。是以孝文納馮唐之說,晉悼宥魏絳之罪。允以特選受命,誅逆撫順,曾未期月,州境澄清。方欲列其庸勳,請加爵賞,而以奉事不當,當肆大戮。責輕罰重,有虧眾望。臣等備位宰相,不敢寢默。誠以允宜蒙三槐之聽,以昭忠貞之心。」書奏,得以減死論。是冬大赦,而允獨不在宥,三公咸復為言。至明年,乃得解釋。是時宦者橫暴,睚眥觸死。允懼不免,乃變易名姓,轉側河內、陳留閒。

及帝崩,乃奔喪京師。時大將軍何進欲誅宦官,召允與謀事,請為從事中郎,轉河南尹。獻帝即位,拜太僕,再遷守尚書令。

初平元年,代楊彪為司徒,守尚書令如故。及董卓遷都關中,允悉收斂蘭臺、石室圖書秘緯要者以從。既至長安,皆分別條上。又集漢朝舊事所當施用者,一皆奏之。經籍具存,允有力焉。時董卓尚留洛陽,朝政大小,悉委之於允。允矯情屈意,每相承附,卓亦推心,不生乖疑,故得扶持王室於危亂之中,臣主內外,莫不倚恃焉。

允見卓禍毒方深,篡逆已兆,密與司隸校尉黃琬、尚書鄭公業等謀共誅之。乃上護羌校尉楊瓚行左將軍事,執金吾士孫瑞為南陽太守,並將兵出武關道,以討袁術為名,實欲分路征卓,而後拔天子還洛陽。卓疑而留之,允乃引內瑞為僕射,瓚為尚書。

二年,卓還長安,錄入關之功,封允為溫侯,食邑五千戶。固讓不受。士孫瑞說允曰:「夫執謙守約,存乎其時。公與董太師並位俱封,而獨崇高節,豈和光之道邪?」允納其言,乃受二千戶。

三年春,連雨六十餘日,允與士孫瑞、楊瓚登臺請霽,復結前謀。瑞曰:「自歲末以來,太陽不照,霖雨積時,月犯執法,彗孛仍見,晝陰夜陽,霧氣交侵,此期應促盡,內發者勝。幾不可後,公其圖之。」允然其言,乃潛結卓將呂布,使為內應。會卓入賀,呂布因刺殺之。語在卓傳。

允初議赦卓部曲,呂布亦數勸之。既而疑曰:「此輩無罪,從其主耳。今若名為惡逆而特赦之,適足使其自疑,非所以安之之道也。」呂布又欲以卓財物班賜公卿、將校,允又不從。而素輕布,以劍客遇之。布亦負其功勞,多自誇伐,既失意望,漸不相平。

允性剛棱疾惡,初懼董卓豺狼,故折節圖之。卓既殲滅,自謂無復患難,及在際會,每乏溫潤之色,杖正持重,不循權宜之計,是以群下不甚附之。

董卓將校及在位者多涼州人,允議罷其軍。或說允曰:「涼州人素憚袁氏而畏關東。今若一旦解兵關東,則必人人自危。可以皇甫義真為將軍,就領其眾,因使留陝以安撫之,而徐與關東通謀,以觀其變。」允曰:「不然。關東舉義兵者,皆吾徒耳。今若距險屯陝,雖安涼州,而疑關東之心,甚不可也。」時百姓訛言,當悉誅涼州人,遂轉相恐動。其在關中者,皆擁兵自守。更相謂曰:「丁彥思、蔡伯喈但以董公親厚,並尚從坐。今既不赦我曹,而欲解兵,今日解兵,明日當復為魚肉矣。」卓部曲將李傕、郭汜等先將兵在關東,因不自安,遂合謀為亂,攻圍長安。城陷,呂布奔走。布駐馬青瑣門外,招允曰:「公可以去乎?」允曰:「若蒙社稷之靈,上安國家,吾之願也。如其不獲,則奉身以死之。朝廷幼少,恃我而已,臨難苟免,吾不忍也。努力謝關東諸公,勤以國家為念。」

初,允以同郡宋翼為左馮翊,王宏為右扶風。是時三輔民庶熾盛,兵穀富實,李傕等欲即殺允,懼二郡為患,乃先徵翼、宏。宏遣使謂翼曰:「郭汜、李傕以我二人在外,故未危王公。今日就徵,明日俱族。計將安出?」翼曰:「雖禍福難量,然王命所不得避也。」宏曰:「義兵鼎沸,在於董卓,況其黨與乎!若舉兵共討君側惡人,山東必應之,此轉福為福之計也。」翼不從。宏不能獨立,遂俱就徵,下廷尉。傕乃收允及翼、宏,并殺之。

允時年五十六。長子侍中蓋、次子景、定及宗族十餘人皆見誅害,唯兄子晨、陵得脫歸鄉里。天子感慟,百姓喪氣,莫敢收允尸者,唯故吏平陵令趙戩棄官營喪。

王宏字長文,少有氣力,不拘細行。初為弘農太守,考案郡中有事宦官買爵位者,雖位至二千石,皆掠考收捕,遂殺數十人,威動鄰界。素與司隸校尉胡种有隙,及宏下獄,种遂迫促殺之。宏臨命詬曰:「宋翼豎儒,不足議大計。胡种樂人之禍,禍將及之。」种後眠輒見宏以杖擊之,因發病,數日死。

後遷都於許,帝思允忠節,使改殯葬之,遣虎賁中郎將奉策弔祭,賜東園祕器,贈以本官印綬,送還本郡。封其孫黑為安樂亭侯,食邑三百戶。

士孫瑞字君策,扶風人,頗有才謀。瑞以允自專討董卓之勞,故歸功不侯,所以獲免於難。後為國三老、光祿大夫。每三公缺,楊彪、皇甫嵩皆讓位於瑞。興平二年,從駕東歸,為亂兵所殺。

趙戩字叔茂,長陵人,性質正多謀。初平中,為尚書,典選舉。董卓數欲有所私授,戩輒堅拒不聽,言色強厲。卓怒,召將殺之,眾人悚慄,而戩辭貌自若。卓悔,謝釋之。長安之亂,容於荊州,劉表厚禮焉。及曹操平荊州,乃辟之,執戩手曰:「恨相見晚。」卒相國鍾繇長史。

論曰:士雖以正立,亦以謀濟。若王允之推董卓而引其權,伺其閒而敝其罪,當此之時,天子懸解矣。而終不以猜忤為釁者,知者本於忠義之誠也。故推卓不為失正,分權不為苟冒,伺閒不為狙詐。及其謀濟意從,則歸成於正也。

贊曰:陳蕃蕪室,志清天綱。人謀雖緝,幽運未當。言觀殄瘁,曷非云亡?子師圖難,晦心傾節。功全元醜,身殘餘孽。時有隆夷,事亦工拙。


\end{pinyinscope}