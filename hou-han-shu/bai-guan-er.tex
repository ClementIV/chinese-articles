\article{百官二}

\begin{pinyinscope}
太常光祿勳衛尉太僕廷尉大鴻臚

太常,卿一人,中二千石。本注曰:掌禮儀祭祀,每祭祀,先奏其禮儀;及行事,常贊天子。每選試博士,奏其能否。大射、養老、大喪,皆奏其禮儀。每月前晦,察行陵廟。丞一人,比千石。本注曰:掌凡行禮及祭祀小事,總署曹事。其署曹掾史,隨事為員,諸卿皆然。

太史令一人,六百石。本注曰:掌天時、星曆。凡歲將終,奏新年曆。凡國祭祀、喪、娶之事,掌奏良日及時節禁忌。凡國有瑞應、災異,掌記之。丞一人。明堂及靈臺丞一人,二百石。本注曰:二丞,掌守明堂、靈臺。靈臺掌候日月星氣,皆屬太史。

博士祭酒一人,六百石。本僕射,中興轉為祭酒。博士十四人,比六百石。本注曰:易四,施、孟、梁丘、京氏。尚書三,歐陽、大小夏侯氏。詩三,魯、齊、韓氏。禮二,大小戴氏。春秋二,公羊嚴、顏氏。掌教弟子。國有疑事,掌承問對。本四百石,宣帝增秩。

太祝令一人,六百石。本注曰:凡國祭祀,掌讀祝,及迎送神。丞一人。本注曰:掌祝小神事。

太宰令一人,六百石。本注曰:掌宰工鼎俎饌具之物。凡國祭祀,掌陳饌具。丞一人。

大子樂令一人,六百石。本注曰:掌伎樂。凡國祭祀,掌請奏樂,及大饗用樂,掌其陳序。丞一人。

高廟令一人,六百石。本注曰:守廟,掌案行掃除。無丞。

世祖廟令一人,六百石。本注曰:如高廟。

先帝陵,每陵園令各一人,六百石。本注曰:掌守陵園,案行掃除。丞及校長各一人。本注曰:校長,主兵戎盜賊事。

先帝陵,每陵食官令各一人,六百石。本注曰:掌望晦時節祭祀。

右屬太常。本注曰:有祠祀令一人,後轉屬少府。有太卜令,六百石,後省并太史。

中興以來,省前凡十官。

光祿勳,卿一人,中二千石。本注曰:掌宿衛宮殿門戶,典謁署郎更直執戟,宿衛門戶,考其德行而進退之。郊祀之事,掌三獻。丞一人,比千石。

五官中郎將一人,比二千石。本注曰:主五官郎。五官中郎,比六百石。本注曰:無員。五官侍郎,比四百石。本注曰:無員。五官郎中,比三百石。本注曰:無員。凡郎官皆主更直執戟,宿衛諸殿門,出充車騎。唯議郎不在直中。

左中郎將,比二千石。本注曰:主左署郎。中郎,比六百石。侍郎,比四百石。郎中,比三百石。本注曰:皆無員。

右中郎將,比二千石。本注曰:主右署郎。中郎,比六百石。侍郎,比四百石。郎中,比三百石。本注曰:皆無員。

虎賁中郎將,比二千石。本注曰:主虎賁宿衛。左右僕射、左右陛長各一人,比六百石。本注曰:僕射,主虎賁郎習射。陛長,主直虎賁,朝會在殿中。虎賁中郎,比六百石。虎賁侍郎,比四百石。虎賁郎中,比三百石。節從虎賁,比二百石。本注曰:皆無員。掌宿衛侍從。自節從虎賁久者轉遷,才能差高至中郎。

羽林中郎將,比二千石。本注曰:主羽林郎。羽林郎,此三百石。本注曰:無員。掌宿衛侍從。常選漢陽、隴西、安定、北地、上郡、西河凡六郡良家補。本武帝以便馬從獵,還宿殿陛巖下室中,故號巖郎。

羽林左監一人,六百石。本注曰:主羽林左騎。丞一人。

羽林右監一人,六百石。本注曰:主羽林右騎。丞一人。

奉車都尉,比二千石。本注曰:無員。掌御乘輿車。

駙馬都尉,比二千石。本注曰:無員。掌駙馬。

騎都尉,比二千石。本注曰:無員。本監羽林騎。

光祿大夫,比二千石。本注曰:無員。凡大夫、議郎皆掌顧問應對,無常事,唯詔令所使。凡諸國嗣之喪,則光祿大夫掌弔。

太中大夫,千石。本注曰:無員。

中散大夫,六百石。本注曰:無員。

諫議大夫,六百石。本注曰:無員。

議郎,六百石。本注曰:無員。

謁者僕射一人,比千石。本注曰:為謁者臺率,主謁者,天子出,奉引。古重習武,有主射以督錄之,故曰僕射。常侍謁者五人,比六百石。本注曰:主殿上時節威儀。謁者三十人。其給事謁者,四百石。其灌謁者郎中,比三百石。本注曰:掌賓贊受事,及上章報問。將、大夫以下之喪,掌使弔。本員七十人,中興但三十人。初為灌謁者,滿歲為給事謁者。

右屬光祿勳。本注曰:職屬光祿者,自五官將至羽林右監,凡七署。自奉車都尉至謁者,以文屬焉。舊有左右曹,秩以二千石,上殿中,主受尚書奏事,平省之。世祖省,使小黃門郎受事,車駕出,給黃門郎兼。有請室令,車駕出,在前請所幸,徼車迎白,示重慎。中興但以郎兼,事訖罷,又省車、戶、騎凡三將,及羽林令。

衛尉,卿一人,中二千石。本注曰:掌宮門衛士,宮中徼循事。丞一人,比千石。

公車司馬令一人,六百石。本注曰:掌宮南闕門,凡吏民上章,四方貢獻,及徵詣公車者。丞、尉各一人。本注曰:丞選曉諱,掌知非法。尉主闕門兵禁,戒非常。

南宮衛士令一人,六百石。本注曰:掌南宮衛士。丞一人。

北宮衛士令一人,六百石,本注曰:掌北宮衛士。丞一人。

左右都候各一人,六百石。本注曰:主劍戟士,徼循宮,及天子有所收考。丞各一人。

宮掖門,每門司馬一人,比千石。木注曰:南宮南屯司馬,主平城門;北宮門蒼龍司馬,主東門;玄武司馬,主玄武門;北屯司馬,主北門;北宮朱爵司馬,主南掖門;東明司馬,主東門;朔平司馬,主北門:凡七門。凡居宮中者,皆有口籍於門之所屬。宮名兩字,為鐵印文符,案省符乃內之。若外人以事當入,本宮長史為封棨傳;其有官位,出入令御者言其官。

右屬衛尉。本注曰:中興省旅賁令,衛士一人丞。

太僕,卿一人,中二千石。本注曰:掌車馬。天子每出,奏駕上鹵簿用;大駕則執馭。丞一人,比千石。

考工令一人,六百石。本注曰:主作兵器弓弩刀鎧之屬,成則傳執金吾入武庫,及主織綬諸雜工。左右丞各一人。

車府令一人,六百石。本注曰:主乘輿諸車。丞一人。

未央廄令一人,六百石。本注曰:主乘輿及廄中諸馬。長樂廄丞一人。

右屬太僕。本注曰:舊有六廄,皆六百石令,中興省約,但置一廄。後置左駿令、廄,別主乘輿御馬,後或并省。又有牧師菀,皆令官,主養馬,分在河西六郡界中,中興皆省,唯漢陽有流馬菀,但以羽林郎監領。

廷尉,卿一人,中二千石。本注曰:掌平獄,奏當所應。凡郡國讞疑罪,皆處當以報。正、左監各一人。左平一人,六百石。本注曰:掌平決詔獄。

右屬廷尉。本注曰:孝武帝以下,置中都官獄二十六所,各令長名世祖中興皆省,

唯廷尉及雒陽有詔獄。

大鴻臚,卿一人,中二千石。本注曰:掌諸侯及四方歸義蠻夷。其郊廟行禮,贊導,請行事,既可,以命群司。諸王入朝,當郊迎,典其禮儀。及郡國上計,匡四方來,亦屬焉。皇子拜王,贊授印綬。及拜諸侯、諸侯嗣子及四方夷狄封者,臺下鴻臚召拜之。王薨則使弔之,及拜王嗣。丞一人,比千石。

大行令一人,六百石。本注曰:主諸郎。丞一人。治禮郎四十七人。

右屬大鴻臚。本注曰:承秦有典屬國,別主四方夷狄朝貢侍子,成帝時省并大鴻臚。

中興省驛官、別火二令、丞,及郡邸長、丞,但令郎治郡邸。


\end{pinyinscope}