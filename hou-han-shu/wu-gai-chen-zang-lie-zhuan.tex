\article{吳蓋陳臧列傳}

\begin{pinyinscope}
吳漢字子顏,南陽宛人也。家貧,給事縣為亭長。王莽末,以賓客犯法,乃亡命至漁陽。資用乏,以販馬自業,往來燕、薊閒,所至皆交結豪傑。更始立,使使者韓鴻徇河北。或謂鴻曰:「吳子顏,奇士也,可與計事。」鴻召見漢,甚悅之,遂承制拜為安樂令。

會王郎起,北州擾惑。漢素聞光武長者,獨欲歸心。乃說太守彭寵曰:「漁陽、上谷突騎,天下所聞也。君何不合二郡精銳,附劉公擊邯鄲,此一時之功也。」寵以為然,而官屬皆欲附王郎,寵不能奪。漢乃辭出,止外亭,念所以譎眾,未知所出。望見道中有一人似儒生者,漢使人召之,為具食,問以所聞。生因言劉公所過,為郡縣所歸;邯鄲舉尊號者,實非劉氏。漢大喜,即詐為光武書,移檄漁陽,使生齎以詣寵,令具以所聞說之,漢復隨後入。寵甚然之。於是遣漢將兵與上谷諸將并軍而南,所至擊斬王郎將帥。及光武於廣阿,拜漢為偏將軍。既拔邯鄲,賜號建策侯。

漢為人質厚少文,造次不能以辭自達。鄧禹及諸將多知之,數相薦舉,及得召見,遂見親信,常居門下。

光武將發幽州兵,夜召鄧禹,問可使行者。禹曰:「閒數與吳漢言,其人勇鷙有智謀,諸將鮮能及者。」即拜漢大將軍,持節北發十郡突騎。更始幽州牧苗曾聞之,陰勒兵,敕諸郡不肯應調。漢乃將二十騎先馳至無終。曾以漢無備,出迎於路,漢即撝兵騎,收曾斬之,而奪其軍。北州震駭,城邑莫不望風弭從。遂悉發其兵,引而南,與光武會清陽。諸將望見漢還,士馬甚盛,皆曰:「是寧肯分兵與人邪?」及漢至莫府,上兵簿,諸將人人多請之。光武曰:「屬者恐不與人,今所請又何多也?」諸將皆慚。

初,更始遣尚書令謝躬率六將軍攻王郎,不能下。會光武至,共定邯鄲,而躬裨將虜掠不相承稟,光武深忌之。雖俱在邯鄲,遂分城而處,然每有以慰安之。躬勤於職事,光武常稱曰「謝尚書真吏也」,故不自疑。躬既而率其兵數萬,還屯於鄴。時光武南擊青犢,謂躬曰:「我追賊於射犬,必破之。尤來在山陽者,埶必當驚走。若以君威力,擊此散虜,必成禽也。」躬曰:「善。」及青犢破,而尤來果北走隆慮山,躬乃留大將軍劉慶、魏郡太守陳康守鄴,自率諸將軍擊之。窮寇死戰,其鋒不可當,躬遂大敗,死者數千人。光武因躬在外,乃使漢與岑彭襲其城。漢先令辯士說陳康曰:「蓋聞上智不處危以僥倖,中智能因危以為功,下愚安於危以自亡。危亡之至,在人所由,不可不察。今京師敗亂,四方雲擾,公所聞也。蕭王兵彊士附,河北歸命,公所見也。謝躬內背蕭王,外失眾心,公所知也。公今據孤危之城,待滅亡之禍,義無所立,節無所成。不若開門內軍,轉禍為福,免下愚之敗,收中智之功,此計之至者也。」康然之。於是康收劉慶及躬妻子,開門內漢等。及躬從隆慮歸鄴,不知康已反之,乃與數百騎輕入城。漢伏兵收之,手擊殺躬,其眾悉降。躬字子張,南陽人。初,其妻知光武不平之,常戒躬曰:「君與劉公積不相能,而信其虛談,不為之備,終受制矣。」躬不納,故及於難。

光武北擊群賊,漢常將突騎五千為軍鋒,數先登陷陳。及河北平,漢與諸將奉圖書,上尊號。光武即位,拜為大司馬,更封舞陽侯。

建武二年春,漢率大司空王梁,建義大將軍朱祐,大將軍杜茂,執金吾賈復,揚化將軍堅鐔,偏將軍王霸,騎都尉劉隆、馬武、陰識,共擊檀鄉賊於鄴東漳水上,大破之,降者十餘萬人。帝使使者璽書定封漢為廣平侯,食廣平、斥漳、曲周、廣年,凡四縣。復率諸將擊鄴西山賊黎伯卿等,及河內脩武,悉破諸屯聚。車駕親幸撫勞。復遣漢進兵南陽,擊宛、涅陽、酈、穰、新野諸城,皆下之。引兵南,與秦豐戰黃郵水上,破之。又與偏將軍馮異擊昌城五樓賊張文等,又攻銅馬、五幡於新安,皆破之。

明年春,率建威大將軍耿弇、虎牙大將軍蓋延,擊青犢於軹西,大破降之。又率驃騎大將軍杜茂、彊弩將軍陳俊等,圍蘇茂於廣樂。劉永將周建別招聚收集得十餘萬人,救廣樂。漢將輕騎迎與之戰,不利,墯馬傷膝,還營,建等遂連兵入城。諸將謂漢曰:「大敵在前而公傷臥,眾心懼矣。」漢乃勃然裹創而起,椎牛饗士,令軍中曰:「賊眾雖多,皆劫掠群盜,『勝不相讓,敗不相救』,非有仗節死義者也。今日封侯之秋,諸君勉之!」於是軍士激怒,人倍其氣。旦日,建、茂出兵圍漢。漢選四部精兵黃頭吳河等,及烏桓突騎三千餘人,齊鼓而進。建軍大潰,反還奔城。漢長驅追擊,爭門並入,大破之,茂、建突走。漢留杜茂、陳俊等守廣樂,自將兵助蓋延圍劉永於睢陽。永既死,二城皆降。

明年,又率陳俊及前將軍王梁,擊破五校賊於臨平,追至東郡箕山,大破之。北擊清河長直及平原五里賊,皆平之。時鬲縣五姓共逐守長,據城而反。諸將爭欲攻之,漢不聽,曰:「使鬲反者,皆守長罪也。敢輕冒進兵者斬。」乃移檄告郡,使收守長,而使人謝城中。五姓大喜,即相率歸降。諸將乃服,曰:「不戰而下城,非眾所及也。」

冬,漢率建威大將軍耿弇、漢中將軍王常等,擊富平、獲索二賊於平原。明年春,賊率五萬餘人夜攻漢營,軍中驚亂,漢堅臥不動,有頃乃定。即夜發精兵出營突擊,大破其眾。因追討餘黨,遂至無鹽,進擊勃海,皆平之。又從征董憲,圍朐城。明年春,拔朐,斬憲。事以見劉永傳。東方悉定,振旅還京師。

會隗囂畔,夏,復遣漢西屯長安。八年,從車駕上隴,遂圍隗囂於西城。帝敕漢曰:「諸郡甲卒但坐費糧食,若有逃亡,則沮敗眾心,宜悉罷之。」漢等貪并力攻囂,遂不能遣,糧食日少,吏士疲役,逃亡者多,及公孫述救至,漢遂退敗。

十一年春,率征南大將軍岑彭等伐公孫述。及彭破荊門,長驅入江關,漢留夷陵,裝露橈船,將南陽兵及弛刑募士三萬人泝江而上。會岑彭為刺客所殺,漢并將其軍。十二年春,與公孫述將魏黨、公孫永戰於魚涪津,大破之,遂圍武陽。述遣子婿史興將五千人救之。漢迎擊興,盡殄其眾,因入犍為界。諸縣皆城守。漢乃進軍攻廣都,拔之。遣輕騎燒成都市橋,武陽以東諸小城皆降。

帝戒漢曰:「成都十餘萬眾,不可輕也。但堅據廣都,待其來攻,勿與爭鋒。若不敢來,公轉營迫之,須其力疲,乃可擊也。」漢乘利,遂自將步騎二萬餘人進逼成都,去城十餘里,阻江北為營,作浮橋,使副將武威將軍劉尚將萬餘人屯於江南,相去二十餘里。帝聞大驚,讓漢曰:「比敕公千條萬端,何意臨事勃亂!既輕敵深入,又與尚別營,事有緩急,不復相及。賊若出兵綴公,以大眾攻尚,尚破,公即敗矣。幸無它者,急引兵還廣都。」詔書未到,述果使其將謝豐、袁吉將眾十許萬,分為二十餘營,并出攻漢。使別將萬餘人劫劉尚,令不得相救。漢與大戰一日,兵敗,走入壁,豐因圍之。漢乃召諸將厲之曰:「吾共諸君踰越險阻,轉戰千里,所在斬獲,遂深入敵地,至其城下。而今與劉尚二處受圍,埶既不接,其禍難量。欲潛師就尚於江南,并兵禦之。若能同心一力,人自為戰,大功可立;如其不然,敗必無餘。成敗之機,在此一舉。」諸將皆曰「諾」。於是饗士秣馬,閉營三日不出,乃多樹幡旗,使煙火不絕,夜銜枚引兵與劉尚合軍。豐等不覺,明日,乃分兵拒江北,自將攻江南。漢悉兵迎戰,自旦至晡,遂大破之,斬謝豐、袁吉,獲甲首五千餘級。於是引還廣都,留劉尚拒述,具以狀上,而深自譴責。帝報曰:「公還廣都,甚得其宜,述必不敢略尚而擊公也。若先攻尚,公從廣都五十里悉步騎赴之,適當值其危困,破之必矣。」自是漢與述戰於廣都、成都之閒,八戰八剋,遂軍于其郭中。述自將數萬人出城大戰,漢使護軍高午、唐邯將數萬銳卒擊之。述兵敗走,高午奔陳刺述,殺之。事已見述傳。旦日城降,斬述首傳送洛陽。明年正月,漢振旅浮江而下。至宛,詔令過家上冢,賜穀二萬斛。

十五年,復率揚武將軍馬成、捕虜將軍馬武北擊匈奴,徙鴈門、代郡、上谷吏人六萬餘口,置居庸、常關以東。

十八年,蜀郡守將史歆反於成都,自稱大司馬,攻太守張穆,穆踰城走廣都,歆遂移檄郡縣,而宕渠楊偉、朐颈徐容等,起兵各數千人以應之。帝以歆昔為岑彭護軍,曉習兵事,故遣漢率劉尚及太中大夫臧宮將萬餘人討之。漢入武都,乃發廣漢、巴、蜀三郡兵圍成都,百餘日城破,誅歆等。漢乃乘桴沿江下巴郡,楊偉、徐容等惶恐解散,漢誅其渠帥二百餘人,徙其黨與數百家於南郡、長沙而還。

漢性彊力,每從征伐,帝未安,恆側足而立。諸將見戰陳不利,或多惶懼,失其常度。漢意氣自若,方整厲器械,激揚士吏。帝時遣人觀大司馬何為,還言方脩戰攻之具,乃歎曰:「吳公差彊人意,隱若一敵國矣!」每當出師,朝受詔,夕即引道,初無辦嚴之日。故能常任職,以功名終。及在朝廷,斤斤謹質,形於體貌。漢嘗出征,妻子在後買田業。漢還,讓之曰:「軍師在外,吏士不足,何多買田宅乎!」遂盡以分與昆弟外家。

二十年,漢病篤。車駕親臨,問所欲言。對曰:「臣愚無所知識,唯願陛下慎無赦而已。」及薨,有詔悼愍,賜謚曰忠侯。發北軍五校、輕車、介士送葬,如大將軍霍光故事。

子哀侯成嗣,為奴所殺。二十八年,分漢封為三國:成子旦為灈陽侯,以奉漢嗣;旦弟盱為筑陽侯;成弟國為新蔡侯。旦卒,無子,國除。建初八年,徙封盱為平春侯,以奉漢後。盱卒,子勝嗣。初,漢兄尉為將軍,從征戰死,封尉子彤為安陽侯。帝以漢功大,復封弟翕為褒親侯。吳氏侯者凡五國。

初,漁陽都尉嚴宣,與漢俱會光武於廣阿,光武以為偏將軍,封建信侯。

論曰:吳漢自建武世,常居上公之位,終始倚愛之親,諒由質簡而彊力也。子曰「剛毅木訥近仁」,斯豈漢之方乎!昔陳平智有餘以見疑,周勃資朴忠而見信。夫仁義不足以相懷,則智者以有餘為疑,而朴者以不足取信矣。

蓋延字巨卿,漁陽要陽人也。身長八尺,彎弓三百斤。邊俗尚勇力,而延以氣聞。歷郡列掾、州從事,所在職辦。彭寵為太守,召延署營尉,行護軍。

及王郎起,延與吳漢同謀歸光武。延至廣阿,拜偏將軍,號建功侯,從平河北。光武即位,以延為虎牙將軍。

建武二年,更封安平侯。遣南擊敖倉,轉攻酸棗、封丘,皆拔。其夏,督駙馬都尉馬武、騎都尉劉隆、護軍都尉馬成、偏將軍王霸等南伐劉永,先攻拔襄邑,進取麻鄉,遂圍永於睢陽。數月,盡收野麥,夜梯其城入。永驚懼,引兵走出東門,延追擊,大破之。永棄軍走譙,延進攻,拔薛,斬其魯郡太守,而彭城、扶陽、杼秋、蕭皆降。又破永沛郡太守,斬之。永將蘇茂、佼彊、周建等三萬餘人救永,共攻延,延與戰於沛西,大破之。永軍亂,遁沒溺死者太半。永棄城走湖陵,蘇茂奔廣樂。延遂定沛、楚、臨淮,修高祖廟,置嗇夫、祝宰、樂人。

三年,睢陽復反城迎劉永,延復率諸將圍之百日,收其野穀。永乏食,突走,延追擊,盡得輜重。永為其將所殺,永弟防舉城降。

四年春,延又擊蘇茂、周建於蘄,進與董憲戰留下,皆破之。因率平敵將軍龐萌攻西防,拔之。復追敗周建、蘇茂於彭城,茂、建亡奔董憲,將賁休舉蘭陵城降。憲聞之,自郯圍休,時延及龐萌在楚,請往救之,帝敕曰:「可直往擣郯,則蘭陵必自解。」延等以賁休城危,遂先赴之。憲逆戰而陽敗,延等遂逐退,因拔圍入城。明日,憲大出兵合圍,延等懼,遽出突走,因往攻郯。帝讓之曰:「閒欲先赴郯者,以其不意故耳。今既奔走,賊計已立,圍豈可解乎!」延等至郯,果不能克,而董憲遂拔蘭陵,殺賁休。延等往來要擊憲別將於彭城、郯、邳之閒,戰或日數合,頗有剋獲。帝以延輕敵深入,數以書誡之。及龐萌反,攻殺楚郡太守,引軍襲敗延,延走,北度泗水,破舟楫,壞津梁,僅而得免。帝自將而東,徵延與大司馬吳漢、漢忠將軍王常、前將軍王梁、捕虜將軍馬武、討虜將軍王霸等會任城,討龐萌於桃鄉,又並從征董憲於昌慮,皆破平之。六年春,遣屯長安。

九年,隗囂死,延西擊街泉、略陽、清水諸屯聚,皆定。

十一年,與中郎將來歙攻河池,未剋,以病引還,拜為左馮翊,將軍如故。十三年,增封定食萬戶。十五年,薨於位。

子扶嗣。扶卒,子側嗣。永平十三年,坐與舅王平謀反,伏誅,國除。永初七年,鄧太后紹封延曾孫恢為蘆亭侯。恢卒,子遂嗣。

陳俊字子昭,南陽西鄂人也。少為郡吏。更始立,以宗室劉嘉為太常將軍,俊為長史。光武徇河北,嘉遣書薦俊,光武以為安集掾。

從擊銅馬於清陽,進至滿陽,拜彊弩將軍。與五校戰於安次,俊下馬,手接短兵,所向必破,追奔二十餘里,斬其渠帥而還。光武望而歎曰:「戰將盡如是,豈有憂哉!」五校引退入漁陽,所過虜掠。俊言於光武曰:「宜令輕騎出賊前,使百姓各自堅壁,以絕其食,可不戰而殄也。」光武然之,遣俊將輕騎馳出賊前。視人保壁堅完者,敕令固守;放散在野者,因掠取之。賊至無所得,遂散敗。及軍還,光武謂俊曰:「困此虜者,將軍策也。」及即位,封俊為列侯。

建武二年春,攻匡賊,下四縣,更封新處侯。引擊頓丘,降三城。其秋,大司馬吳漢承制拜俊為彊弩大將軍,別擊金門、白馬賊於河內,皆破之。四年,轉徇汝陽及項,又拔南武陽。是時太山豪傑多擁眾與張步連兵,吳漢言於帝曰:「非陳俊莫能定此郡。」於是拜俊太山太守,行大將軍事。張步聞之,遣其將擊俊,戰於嬴下,俊大破之,追至濟南,收得印綬九十餘,稍攻下諸縣,遂定太山。五年,與建威大將軍耿弇共破張步。事在弇傳。

時琅邪未平,乃徙俊為琅邪太守,領將軍如故。齊地素聞俊名,入界,盜賊皆解散。俊將兵擊董憲於贛榆,進破朐賊孫陽,平之。八年,張步畔,還琅邪,俊追討,斬之。帝美其功,詔俊得專征青、徐。俊撫貧弱,表有義,檢制軍吏,不得與郡縣相干,百姓歌之。數上書自請,願奮擊隴、蜀。詔報曰:「東州新平,大將軍之功也。負海猾夏,盜賊之處,國家以為重憂,且勉鎮撫之。」

十三年,增邑,定封祝阿侯。明年,徵奉朝請。二十三年卒。

子浮嗣,徙封蘄春侯。浮卒,子專諸嗣。專諸卒,子篤嗣。

臧宮字君翁,潁川郟人也。少為縣亭長、游徼,後率賓客入下江兵中為校尉,因從光武征戰,諸將多稱其勇。光武察宮勤力少言,甚親納之。及至河北,以為偏將軍,從破群賊,數陷陳卻敵。

光武即位,以為侍中、騎都尉。建武二年,封成安侯。明年,將突騎與征虜將軍祭遵擊更始將左防、韋顏於沮陽、酈,悉降之。五年,將兵徇江夏,擊代鄉、鐘武、竹里,皆下之。帝使太中大夫持節拜宮為輔威將軍。七年,更封期思侯。擊梁郡、濟陰,皆平之。

十一年,將兵至中盧,屯駱越。是時公孫述將田戎、任滿與征南大將軍岑彭相拒於荊門,彭等戰數不利,越人謀畔從蜀。宮兵少,力不能制。會屬縣送委輸車數百乘至,宮夜使鋸斷城門限。令車聲回轉出入至旦。越人候伺者聞車聲不絕,而門限斷,相告以漢兵大至。其渠帥乃奉牛酒以勞軍營。宮陳兵大會,擊牛釃酒,饗賜慰納之,越人由是遂安。

宮與岑彭等破荊門,別至垂鵲山,通道出秭歸,至江州。岑彭下巴郡,使宮將降卒五萬,從涪水上平曲。公孫述將延岑盛兵於沅水,時宮眾多食少,轉輸不至,而降者皆欲散畔,郡邑復更保聚,觀望成敗。宮欲引還,恐為所反。會帝遣謁者將兵詣岑彭,有馬七百匹,宮矯制取以自益,晨夜進兵,多張旗幟,登山鼓噪,右步左騎,挾船而引,呼聲動山谷。岑不意漢軍卒至,登山望之,大震恐。宮因從擊,大破之。斬首溺死者萬餘人,水為之濁流。延岑奔成都,其眾悉降,盡獲其兵馬珍寶。自是乘勝追北,降者以十萬數。

軍至平陽鄉,蜀將王元舉眾降。進拔綿竹,破涪城,斬公孫述弟恢,復攻拔繁、郫。前後收得節五,印綬千八百。是時大司馬吳漢亦乘勝進營逼成都。宮連屠大城,兵馬旌旗甚盛,乃乘兵入小雒郭門,歷成都城下,至吳漢營,飲酒高會。漢見之甚歡,謂宮曰:「將軍向者經虜城下,震揚威靈,風行電照。然窮寇難量,還營願從它道矣。」宮不從,復路而歸,賊亦不敢近之。進軍咸門,與吳漢並滅公孫述。

帝以蜀地新定,拜宮為廣漢太守。十三年,增邑,更封酇侯。十五年,徵還京師,以列侯奉朝請,定封朗陵侯。十八年,拜太中大夫。

十九年,妖巫維汜弟子單臣、傅鎮等,復妖言相聚,入原武城,劫吏人,自稱將軍。於是遣宮將北軍及黎陽營數千人圍之。賊穀食多,數攻不下,士卒死傷。帝召公卿諸侯王問方略,皆曰「宜重其購賞」。時顯宗為東海王,獨對曰:「妖巫相劫,埶無久立,其中必有悔欲亡者。但外圍急,不得走耳。宜小挺緩,令得逃亡,逃亡則一亭長足以禽矣。」帝然之,即敕宮徹圍緩賊,賊眾分散,遂斬臣、鎮等。宮還,遷城門校尉,復轉左中郎將。擊武谿賊,至江陵,降之。

宮以謹信質樸,故常見任用。後匈奴飢疫,自相分爭,帝以問宮,宮曰:「願得五千騎以立功。」帝笑曰:「常勝之家,難與慮敵,吾方自思之。」二十七人,宮乃與楊虛侯馬武上書曰:「匈奴貪利,無有禮信,窮則稽首,安則侵盜,緣邊被其毒痛,中國憂其抵突。虜今人畜疫死,旱蝗赤地,疫困之力,不當中國一郡。萬里死命,縣在陛下。福不再來,時或易失,豈宜固守文德而墮武事乎?今命將臨塞,厚縣購賞,喻告高句驪、烏桓、鮮卑攻其左,發河西四郡、天水、隴西羌胡擊其右。如此,北虜之滅,不過數年。臣恐陛下仁恩不忍,謀臣狐疑,令萬世刻石之功不立於聖世。」詔報曰:「黃石公記曰,『柔能制剛,弱能制彊』。柔者德也,剛者賊也,弱者仁之助也,彊者怨之歸也。故曰有德之君,以所樂樂人;無德之君,以所樂樂身。樂人者其樂長,樂身者不久而亡。舍近謀遠者,勞而無功;舍遠謀近者,逸而有終。逸政多忠臣,勞政多亂人。故曰務廣地者荒,務廣德者彊。有其有者安,貪人有者殘。殘滅之政,雖成必敗。今國無善政,災變不息,百姓驚惶,人不自保,而復欲遠事邊外乎?孔子曰:『吾恐季孫之憂,不在顓臾。』且北狄尚彊,而屯田警備傳聞之事,恆多失實。誠能舉天下之半以滅大寇,豈非至願;苟非其時,不如息人。」自是諸將莫敢言兵事者。

宮永平元年卒,謚曰愍侯。子信嗣。信卒,子震嗣。震卒,子松嗣。元初四年,與母別居,國除。永寧元年,鄧太后紹封松弟由為朗陵侯。

論曰:中興之業,誠艱難也。然敵無秦、項之彊,人資附漢之思,雖懷璽紆紱,跨陵州縣,殊名詭號,千隊為群,尚未足以為比功上烈也。至於山西既定,威臨天下,戎羯喪其精膽,群帥賈其餘壯,斯誠雄心尚武之幾,先志翫兵之日。臧宮、馬武之徒,撫嗚劍而扺掌,志馳於伊吾之北矣。光武審黃石,存包桑,閉玉門以謝西域之質,卑詞幣以禮匈奴之使,其意防蓋已弘深。豈其顛沛平城之圍,忍傷黥王之陳乎?

贊曰:吳公鷙彊,實為龍驤。電埽群孽,風行巴、梁。虎牙猛力,功立睢陽。宮、俊休休,是亦鷹揚。


\end{pinyinscope}