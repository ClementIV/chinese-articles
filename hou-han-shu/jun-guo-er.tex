\article{郡國二}

\begin{pinyinscope}
潁川郡

十七城,戶二十六萬三千四百四十,口百四十三萬六千五百一十三。

陽翟

禹所都。有鈞臺。有高氏亭。有雍氏城。襄

有養陰里。襄城有西不羹。有氾城。有汾丘。有魚齒山。昆陽有湛水。定陵有東不羹。舞陽邑。郾臨潁潁陽潁陰有狐宗鄉,或曰古狐人亭。有岸亭。許

新汲

衝陵

春秋時曰衝。長社有長葛城。有向鄉。有蜀城,有蜀津。陽城有嵩高山,洧水、潁水出。有鐵。有負黍聚。父城有應鄉。輪氏建初四年置。

汝南郡

三十七城,戶四十萬四千四百四十八,口二百一十萬七百八十八。

平輿

有沈亭,故國,姬姓。新陽侯國。西平有鐵。有柏亭,故柏國。上蔡本蔡國。南頓本頓國。汝陰本胡國。汝陽新息國。北宜春濦強侯國。灈陽期思有蔣鄉,故蔣國。陽安

道亭,故國。項西華細陽安城侯國。有武城亭。吳房有棠谿亭。鮦陽侯國。慎陽慎新蔡有大呂亭。安陽侯國。有江亭,故國,嬴姓。富波侯國,永元中復。宜祿永元中復。朗陵侯國。弋陽侯國。有黃亭,故黃國,嬴姓。召陵有陘亭。有安陵鄉。征羌侯國。有安陵亭。思善侯國。宋

公國,周名郪丘,漢改為新郪,章帝建初四年徙宋公於此。有繁陽亭。褒信侯國。有賴亭,故國。原鹿侯國。定潁侯國。固始侯國。故寢也,光武中興更名。有寢丘。山桑侯國,故屬沛。有下城父聚。有垂惠聚。城父故屬沛,春秋時曰夷。有章華臺。

梁國

九

城,戶八萬三千三百,口四十三萬一千二百八十三。

下邑

睢陽

本宋國閼伯墟。有盧門亭。有魚門。有陽梁聚。虞有空桐地,有桐地,有桐亭。有綸城,少康邑。碭

山出文石。蒙有蒙澤。穀熟有新城。有邳亭。衝故屬陳留。寧陵故屬陳留。有葛鄉,故葛伯國。薄故屬山陽,所都。

沛國

二十一城,戶二十萬四百九十五,口二十五萬一千三百九十三。

相

蕭

本

國。沛有泗水亭。豐

西有大澤,高祖斬白蛇於此。有枌榆亭。酇有鄍聚。穀陽譙刺史治。洨

有垓下聚。蘄有大澤鄉,陳涉起此。銍鄲

建平

臨睢

故芒

,光武更名。竹邑侯國,故竹。公丘本膠國。龍亢

向

本國。符離虹太丘

杼秋

故屬梁國,有檀淵聚。

陳國

九城,戶十一萬二千六百五十三,口百五十四萬七千五百七十二。

陳

陽夏

有

固陵聚。寧平苦

春秋時曰相。有賴鄉。柘新平扶樂武平

長平

故屬汝南。有辰亭。有赭丘城。

魯國

六城,戶七萬八千四百四十七,口四十一萬一千五百九十。

魯

國,奄國。有大庭氏庫。有鐵。有闕里,孔子所居。有牛首亭。有五父衢。騶本邾國。蕃有南梁水。薛本國,六國時曰徐州。卞有盜泉。有郚鄉城。汶陽

右豫州刺史部,郡、國六,縣、邑、、侯國九十九。

魏郡

十五城,戶十二萬九千三百一十,口六十九萬五千六百六。

鄴

有故大河。有滏水。有汙水,有汙城。有平陽城。有武城。有九侯城。繁陽內黃清河水出。有羛陽聚。有黃澤。魏元城墟,故沙鹿,有沙亭。黎陽

陰安

邑。館陶清淵

平恩

沙

侯國。斥丘有葛。武安有鐵。曲梁侯國,故屬廣平。有雞澤。梁期

鉅鹿郡

十五

城,戶十萬九千五百一十七,口六十萬二千九十六。

廮陶

有薄落亭。鉅鹿故大鹿,有大陸澤。楊氏

鄡

下曲陽

有鼓聚,故翟鼓子國。有昔陽亭。任

南和

廣平

斥

章

廣宗曲周

列人

廣年

平鄉

南

讀

常山國

十三城,戶九萬七千五百,口六十三萬一千一百八十四。

元氏

高邑

故

鄗,光武更名。刺史治。有千秋亭、五成陌,光武即位於此矣。都鄉侯國。有鐵。南行唐有石臼谷。房子贊皇山,濟水所出。平棘有塞。欒城九門靈壽衛水出。蒲吾井陘真定

上艾

故屬太原。

中山國

十三城,戶九萬七千四百一十二,口六十五萬八千一百九十五。

盧奴

北平

有鐵

。母極新市

有鮮虞亭,故國,子姓。望都

唐

有中人亭,有左人鄉。安國安憙本安險,章帝更名。漢昌本苦陘,章帝更名。蠡吾侯國,故屬涿。上曲陽故屬常山。恒山在西北。蒲陰本曲逆,章帝更名。有陽城。廣昌故屬代郡。

安平國

十

三城,戶九萬一千四百四十,口六十五萬五千一百一十八。

信都

有絳水、呼沱河。阜城故昌城。南宮扶柳下博武邑觀津

經

西有漳水,津名薄落津。堂陽故屬鉅鹿。武遂

故屬河閒。饒陽故名饒,屬涿。有無蔞亭。安平故屬涿。南深國故屬涿。

河閒國

十一城,戶九萬三千七百五十四,口六十三萬四千四百二十一。

樂成

弓高易

故屬涿。武垣故屬涿。中水故屬涿。鄚故屬涿。高陽故屬涿。有葛城。文安故屬勃海。束州故屬勃海。成平故屬勃海。東平舒故屬勃海。

清河國

七城,戶十二萬三千九百六十四,口七十六萬四百一十八。

甘陵

故厝,安帝更名。貝丘東武成鄃

靈

和帝永元九年復。繹幕廣川故屬信都。有棘津城。

趙國

五城,戶三萬二千七百一十九,口十八萬八千三百八十一。

邯鄲

有叢臺。易陽襄國本邢國,秦為信都,項羽更名。有檀臺。有蘇人亭。柏人中丘

勃海郡

八城,戶十三萬二千三百八十九,口百一十萬六千五百。

南皮

高城

侯國

。重合侯國。浮陽侯國。東光武

陽信

延光元年復。脩故屬信都。

右冀州刺史部,郡、國九,縣、邑、侯國百。


\end{pinyinscope}