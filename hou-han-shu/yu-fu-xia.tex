\article{輿服下}

\begin{pinyinscope}
通天冠遠遊冠高山冠進賢

冠法冠

武冠建華冠方山冠巧士冠

卻非冠

卻敵冠樊噲冠術氏冠鶡冠

幘佩

刀印黃赤綬赤綬綠綬

紫綬青綬

黑綬黃綬青紺綸后夫人服

上古穴居而野處,衣毛而冒皮,未有制度。後世聖人易之以絲麻,觀翬翟之文,榮華之色,乃染帛以效之,始作五采,成以為服。見鳥獸有冠角签胡之制,遂作冠冕纓蕤,以為首飾。凡十二章。故《易》曰:「庖犧氏之王天下也,仰觀象於天,俯觀法於地,觀鳥獸之文,與地之宜,近取諸身,遠取諸物,於是始作八卦,以通神明之德,以類萬物之情。」黃帝堯舜垂衣裳而天下治,蓋取諸乾巛。乾巛有文,故上衣玄,下裳黃。日月星辰,山龍華蟲,作繢宗彝,藻火粉米,黼黻絺繡,以五采章施于五色作服。天子備章,公自山以下,侯伯自華蟲以下,子男自藻火以下,卿大夫自粉米以下。至周而變之,以三辰為旂旗。王祭上帝,則大裘而冕;公侯卿大夫之服用九章以下。秦以戰國即天子位,滅去禮學,郊祀之服皆以袀玄。漢承秦故。至世祖踐祚,都于土中,始修三雍,正兆七郊。顯宗遂就大業,初服旒冕,衣裳文章,赤舄絇屨,以祠天地,養三老五更於三雍,于時致治平矣。

天子、三公、九卿、特進侯、侍祠侯,祀天地明堂,皆冠旒冕,衣裳玄上纁下。乘輿備文,日月星辰十二章,三公、諸侯用山龍九章,九卿以下用華蟲七章,皆備五采,大佩,赤舄絇履,以承大祭。百官執事者,冠長冠,皆祗服。五獄、四瀆、山川、宗廟、社稷諸沾秩祠,皆袀玄長冠,五郊各如方色云。百官不執事,各服常冠袀玄以從。

冕冠,垂旒,前後邃延,玉藻。孝明皇帝永平二年,初詔有司采周官、禮記、尚書皋陶篇,乘輿服從歐陽氏說,公卿以下從大小夏侯氏說。冕皆廣七寸,長尺二寸,前圓後方,朱綠裏,玄上,前垂四寸,後垂三寸,係白玉珠為十二旒,以其綬采色為組纓。三公諸侯七旒,青玉為珠;卿大夫五旒,黑玉為珠。皆有前無後,各以其綬采色為組纓,旁垂黈纊。郊天地,宗祀,明堂,則冠之。衣裳玉佩備章采,乘輿刺史,公侯九卿以下皆織成,陳留襄邑獻之云。

長冠,一曰齋冠,高七寸,廣三寸,促漆纚為之,制如板,以竹為裏。初,高祖微時,以竹皮為之,謂之劉氏冠,楚冠制也。民謂之鵲尾冠,非也。祀宗廟諸祀則冠之。皆服袀玄,絳緣領袖為中衣,絳恊収,示其赤心奉神也。五郊,衣幘恊収各如其色。此冠高祖所造,故以為祭服,尊敬之至也。

委貌冠、皮弁冠同制,長七寸,高四寸,制如覆杯,前高廣,後卑銳,所謂夏之母追,殷之章甫者也。委貌以皁絹為之,皮弁以鹿皮為之。行大射禮於辟雍,公卿諸侯大夫行禮者,冠委貌,衣玄端素裳。執事者冠皮弁,衣緇麻衣,皁領袖,下素裳,所謂皮弁素積者也。

爵弁,一名冕。廣八寸,長尺二寸,如爵形,前小後大,繒其上似爵頭色,有收持笄,所謂夏收殷冔者也。祠天地五郊明堂,雲翹舞樂人服之。禮曰:「朱干玉鏚,冕而舞大夏。」此之謂也。

通天冠,高九寸,正豎,頂少邪卻,乃直下為鐵卷梁,前有山,展筩為述,乘輿所常服。服衣,深衣制,有袍,隨五時色。袍者,或曰周公抱成王宴居,故施袍。禮記「孔子衣逢掖之衣」。縫掖其袖,合而縫大之,近今袍者也。今下至賤更小史,皆通制袍,單衣,皁緣領袖中衣,為朝服云。

遠遊冠,制如通天,有展筩橫之於前,無山述,諸王所服也。

高山冠,一曰側注。制如通天,不邪卻,直豎,無山述展筩,中外官、謁者、僕射所服。太傅胡廣說曰:「高山冠,蓋齊王冠也。秦滅齊,以其君冠賜近臣謁者服之。」

進賢冠,古緇布冠也,文儒者之服也。前高七寸,後高三寸,長八寸。公侯三梁,中二千石以下至博士兩梁,自博士以下至小史私學弟子,皆一梁。宗室劉氏亦兩梁冠,示加服也。

法冠,一曰柱後。高五寸,以纚為展筩,鐵柱卷,執法者服之,侍御史、廷尉正監平也。或謂之獬豸冠。獬豸神羊,能別曲直,楚王嘗獲之,故以為冠。胡廣說曰:「春秋左氏傳有南冠而縶者,則楚冠也。秦滅楚,以其君服賜執法近臣御史服之。」

武冠,一曰武弁大冠,諸武官冠之。侍中、中常侍加黃金璫,附蟬為文,貂尾為飾,謂之「趙惠文冠」。胡廣說曰:「趙武靈王效胡服,以金璫飾首,前插貂尾,為貴職。秦滅趙,以其君冠賜近臣。」建武時,匈奴內屬,世祖賜南單于衣服,以中常侍惠文冠,中黃門童子佩刀云。

建華冠,以鐵為柱卷,貫大銅珠九枚,制似縷鹿。記曰:「知天者冠述,知地者履絇。」春秋左傳曰:「鄭子臧好鷸冠。」前圓,以為此則是也。天地、五郊、明堂,育命舞樂人服之。

方山冠,似進賢,以五采縠為之。祠宗朝,大予、八佾、四時、五行樂人服之,冠衣各如其行方之色而舞焉。

巧士冠,高七寸,要後相通,直豎。不常服,唯郊天,黃門從官四人冠之,在鹵簿中,次乘輿車前,以備宦者四星云。

卻非冠,制似長冠,下促。宮殿門吏僕射冠之。負赤幡,青翅燕尾,諸僕射幡皆如之。

卻敵冠,前高四寸,通長四寸,後高三寸,制似進賢,衛士服之。

樊噲冠,漢將樊噲造次所冠,以入項羽軍。廣九寸,高七寸,前後出各四寸,制似冕。司馬殿門大難衛士服之。或曰,樊噲常持鐵楯,聞項羽有意殺漢王,噲裂裳以裹楯,冠之入軍門,立漢王旁,視項羽。

術氏冠,前圓,吳制,差池邐迆四重。趙武靈王好服之。今不施用,官有其圖注。

諸冠皆有纓蕤,執事及武吏皆縮纓,垂五寸。

武冠,俗謂之大冠,環纓無蕤,以青系為緄,加雙鶡尾,豎左右,為鶡冠云。五官、左右虎賁、羽林、五中郎將、羽林左右監皆冠鶡冠,紗縠單衣。虎賁將虎文恊,白虎文劍佩刀。虎賁武騎皆鶡冠,虎文單衣。襄邑歲獻織成虎文云。鶡者,勇雉也,其鬥對一死乃止,故趙武靈王以表武士,秦施之焉。

安帝立皇太子,太子謁高祖廟、世祖廟,門大夫從,冠兩梁進賢;洗馬冠高山。罷廟,侍御史任方奏請非乘從時,皆冠一梁,不宜以為常服。事下有司。尚書陳忠奏:「門大夫職如諫大夫,洗馬職如謁者,故皆服其服,先帝之舊也。方言可寢。」奏可。謁者,古者一名洗馬。

古者有冠無幘,其戴也,加首有頍,所以安物。故《詩》曰「有頍者弁」,此之謂也。三代之世,法制滋彰,下至戰國,文武並用。秦雄諸侯,乃加其武將首飾為絳袙,以表貴賤,其後稍稍作顏題。漢興,續其顏,卻摞之,施巾連題,卻覆之,今喪幘是其制也。名之曰幘。幘者,賾也,頭首嚴賾也。至孝文乃高顏題,續之為耳,崇其巾為屋,合後施收,上下群臣貴賤皆服之。文者長耳,武者短耳,稱其冠也。尚書幘收,方三寸,名曰納言,示以忠正,顯近職也。迎氣五郊,各如其色,從章服也。皁衣群吏春服青幘,立夏乃止,助微順氣,尊其方也。武吏常赤幘,成其威也。未冠童子幘無屋者,示未成人也。入學小童幘也句卷屋者,示尚幼少,未遠冒也。喪幘卻摞,反本禮也。升數如冠,與冠偕也。期喪起耳有收,素幘亦如之,禮輕重有制,變除從漸,文也。

古者君臣佩玉,尊卑有度;上有韍,貴賤有殊。佩,所以章德,服之衷也。韍,所以執事,禮之共也。故禮有其度,威儀之制,三代同之。五霸迭興,戰兵不息,佩非戰器,韍非兵旗,於是解去韍佩,留其係璲,以為章表。故《詩》曰「鞙鞙佩璲」,此之謂也。韍佩既廢,秦乃以采組連結於璲,光明章表,轉相結受,故謂之綬。漢承秦制,用而弗改,故加之以雙印佩刀之飾。至孝明皇帝,乃為大佩,衝牙雙瑀璜,皆以白玉。乘輿落以白珠,公卿諸侯以采絲,其視冕旒,為祭服云。

佩刀,乘輿黃金通身貂錯,半鮫魚鱗,金漆錯,雌黃室,五色罽隱室華。諸侯王黃金錯,環挾半鮫,黑室。公卿百官皆純黑,不半鮫。小黃門雌黃室,中黃門朱室,童子皆虎爪文,虎賁黃室虎文,其將白虎文,皆以白珠鮫為諱口之飾。乘輿者,加翡翠山,紆嬰其側。

佩雙印,長寸二分,方六分。乘輿、諸侯王、公、列侯以白玉,中二千石以下至四百石皆以黑犀,二百石以至私學弟子皆以象牙。上合絲,乘輿以縢貫白珠,赤罽蕤,諸侯王以下以綔赤絲蕤,縢綔各如其印質。刻書文曰:「正月剛卯既決,靈殳四方,赤青白黃,四色是當。帝令祝融,以教夔龍,庶疫剛癉,莫我敢當。疾日嚴卯,帝令夔化,慎爾周伏,化茲靈殳。既正既直,既觚既方,庶疫剛癉,莫我敢當。」凡六十六字。

乘輿黃赤綬,四采,黃赤紺縹,淳黃圭,長〈二〉丈九尺九寸,五百首。

諸侯王赤綬,四采,赤黃縹紺,淳赤圭,長二丈一尺,三百首。

太皇太后、皇太后,其綬皆與乘輿同,皇后亦如之。

長公主、天子貴人與諸侯王同綬者,加特也。

諸國貴人、相國皆綠綬,三采,綠紫紺,淳綠圭,長二丈一尺,二百四十首。

公、侯、將軍紫綬,二采,紫白,淳紫圭,長丈七尺,百八十首。公主封君服紫綬。

九卿、中二千石、二千石青綬,三采,青白紅,淳青圭,長丈七尺,百二十首。自青綬以上,縌皆長三尺二寸,與綬同采而首半之。縌者,古佩璲也。佩綬相迎受,故曰縌。紫綬以上,縌綬之閒得施玉環鐍云。

千石、六百石黑綬,三采,青赤紺,淳青圭,長丈六尺,八十首。四百石、三百石長同。

四百石、三百石、二百石黃綬,〈一采〉,淳黃圭,一采長丈五尺,六十首。自黑綬以下,縌綬皆長三尺,與綬同采而首半之。

百石青紺綸,一采,宛轉繆織,長丈二尺。

凡先合單紡為一系,四系為一扶,五扶為一首,五首成一文,文采淳為一圭。首多者系細,少者系麤,皆廣尺六寸。

太皇太后、皇太后入廟服,紺上皁下,蠶,青上縹下,皆深衣制,隱領袖緣以絛。翦氂蔮,簪珥。珥,耳璫垂珠也。簪以玳瑁為擿,長一尺,端為華勝,上為鳳皇爵,以翡翠為毛羽,下有白珠,垂黃金鑷。左右一橫簪之,以安蔮結。諸簪珥皆同制,其擿有等級焉。

皇后謁廟服,紺上皁下,蠶,青上縹下,皆深衣制,隱領袖緣以絛。假結,步搖,簪珥。步搖以黃金為山題,貫白珠為桂枝相繆,一爵九華,熊、虎、赤羆、天鹿、辟邪、南山豐大特六獸,詩所謂「副笄六珈」者。諸爵獸皆以翡翠為毛羽。金題,白珠璫繞,以翡翠為華云。

貴人助蠶服,純縹上下,深衣制。大手結,墨玳瑁,又加簪珥。長公主見會衣服,加步搖,公主大手結,皆有簪珥,衣服同制。自公主封君以上皆帶綬,以采組為緄帶,各如其綬色。黃金辟邪,首為帶鐍,飾以白珠。

公、卿、列侯、中二千石、二千石夫人,紺繒蔮,黃金龍首銜白珠,魚須擿,長一尺,為簪珥。入廟佐祭者皁絹上下,助蠶者縹絹上下,皆深衣制,緣。自二千石夫人以上至皇后,皆以蠶衣為朝服。

公主、貴人、妃以上,嫁娶得服錦綺羅縠繒,采十二色,重緣袍。特進、列侯以上錦繒,采十二色。六百石以上重練,采九色,禁丹紫紺。三百石以上五色采,青絳黃紅綠。二百石以上四采,青黃紅綠。賈人,緗縹而已。

公、列侯以下皆單緣耻,制文繡為祭服。自皇后以下,皆不得服諸古麗圭襂閨緣加上之服。建武、永平禁絕之,建初、永元又復中重,於是世莫能有制其裁者,乃遂絕矣。

凡冠衣諸服,旒冕、長冠、委貌、皮弁、爵弁、建華、方山、巧士,衣裳文繡,赤舄,服絇履,大佩,皆為祭服,其餘悉為常用朝服。唯長冠,諸王國謁者以為常朝服云。宗廟以下,祠祀皆冠長冠,皁繒袍單衣,絳緣領袖中衣,絳恊収,五郊各從其色焉。

贊曰:車輅各庸,旌旂異局。冠服致美,佩紛璽玉。敬敬報情,尊尊下欲。孰夸華文,匪豪麗縟。


\end{pinyinscope}