\article{亡徵}

\begin{pinyinscope}
凡人主之國小而家大,權輕而臣重者,可亡也。簡法禁而務謀慮,荒封內而恃交援者,可亡也。群臣為學,門子好辯,商賈外積,小民右仗者,可亡也。好宮室臺榭陂池,事車服器玩好,罷露百姓,煎靡貨財者,可亡也。用時日,事鬼神,信卜筮,而好祭祀者,可亡也。聽以爵不待參驗,用一人為門戶者,可亡也。官職可以重求,爵祿可以貨得者,可亡也。緩心而無成,柔茹而寡斷,好惡無決,而無所定立者,可亡也。饕貪而無饜,近利而好得者,可亡也。喜淫而不周於法,好辯說而不求其用,濫於文麗而不顧其功者,可亡也,淺薄而易見,漏泄而無藏,不能周密,而通群臣之語者,可亡也。很剛而不和,愎諫而好勝,不顧社稷而輕為自信者,可亡也。恃交援而簡近鄰,怙強大之救,而侮所迫之國者,可亡也。羈旅僑士,重帑在外,上閒謀計,下與民事者,可亡也。民信其相,下不能其上,主愛信之而弗能廢者,可亡也。境內之傑不事,而求封外之士,不以功伐課試,而好以名問舉錯,羈旅起貴以陵故常者,可亡也。輕其適正,庶子稱衡,太子未定而主即世者,可亡也。大心而無悔,國亂而自多,不料境內之資而易其鄰敵者,可亡也。國小而不處卑,力少而不畏強,無禮而侮大鄰,貪愎而拙交者,可亡也。太子已置,而娶於強敵以為后妻,則太子危,如是,則群臣易慮,群臣易慮者,可亡也。怯懾而弱守,蚤見而心柔懦,知有謂可,斷而弗敢行者,可亡也。出君在外而國更置,質太子未反而君易子,如是則國攜,國攜者,可亡也,挫辱大臣而狎其身,刑戮小民而逆其使,懷怒思恥而專習則賊生,賊生者,可亡也。大臣兩重,父兄眾強,內黨外援以爭事勢者,可亡也。婢妾之言聽,愛玩之智用,外內悲惋而數行不法者,可亡也。簡侮大臣,無禮父兄,勞苦百姓,殺戮不辜者,可亡也。好以智矯法,時以行集公,法禁變易,號令數下者,可亡也。無地固,城郭惡,無畜積,財物寡,無守戰之備而輕攻伐者,可亡也。種類不壽,主數即世,嬰兒為君,大臣專制,樹羈旅以為黨,數割地以待交者,可亡也。太子尊顯,徒屬眾強,多大國之交,而威勢蚤具者,可亡也。變褊而心急,輕疾而易動發,心悁忿而不訾前後者,可亡也。主多怒而好用兵,簡本教而輕戰攻者,可亡也。貴臣相妒,大臣隆盛,外藉敵國,內困百姓,以攻怨讎,而人主弗誅者,可亡也。君不肖而側室賢,太子輕而庶子伉,官吏弱而人民桀,如此則國躁,國躁者,可亡也。藏怒而弗發,懸罪而弗誅,使群臣陰憎而愈憂懼,而久未可知者,可亡也。出軍命將太重,邊地任守太尊,專制擅命,徑為而無所請者,可亡也。后妻淫亂,主母畜穢,外內混通,男女無別,是謂兩主,兩主者,可亡也。后妻賤而婢妾貴,太子卑而庶子尊,相室輕而典謁重,如此則內外乖,內外乖者,可亡也。大臣甚貴,偏黨眾強,壅塞主斷而重擅國者,可亡也。私門之官用,馬府之世,鄉曲之善舉,官職之勞廢,貴私行而賤公功者,可亡也。公家虛而大臣實,正戶貧而寄寓富,耕戰之士困,末作之民利者,可亡也。見大利而不趨,聞禍端而不備,淺薄於爭守之事,而務以仁義自飾者,可亡也。不為人主之孝,而慕匹夫之孝,不顧社稷之利,而聽主母之令,女子用國,刑餘用事者,可亡也。辭辯而不法,心智而無術,主多能而不以法度從事者,可亡也。親臣進而故人退,不肖用事而賢良伏,無功貴而勞苦賤,如是則下怨,下怨者,可亡也。父兄大臣祿秩過功,章服侵等,宮室供養太侈,而人主弗禁,則臣心無窮,臣心無窮者,可亡也。公婿公孫與民同門,暴傲其鄰者,可亡也。亡徵者,非曰必亡,言其可亡也。夫兩堯不能相王,兩桀不能相亡,亡王之機,必其治亂、其強弱相踦者也。木之折也必通蠹,牆之壞也必通隙。然木雖蠹,無疾風不折;牆雖隙,無大雨不壞。萬乘之主,有能服術行法以為亡徵之君風雨者,其兼天下不難矣。


\end{pinyinscope}