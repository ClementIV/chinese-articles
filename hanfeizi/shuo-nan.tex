\article{說難}

\begin{pinyinscope}
凡說之難:非吾知之,有以說之之難也;又非吾辯之,能明吾意之難也;又非吾敢橫失,而能盡之難也。凡說之難,在知所說之心,可以吾說當之。

所說出於為名高者也,而說之以厚利,則見下節而遇卑賤,必棄遠矣。所說出於厚利者也,而說之以名高,則見無心而遠事情,必不收矣。所說陰為厚利而顯為名高者也,而說之以名高,則陽收其身而實疏之,說之以厚利,則陰用其言顯棄其身矣。此不可不察也。

夫事以密成,語以泄敗,未必其身泄之也,而語及所匿之事,如此者身危。彼顯有所出事,而乃以成他故,說者不徒知所出而已矣,又知其所以為,如此者身危。規異事而當,知者揣之外而得之,事泄於外,必以為己也,如此者身危。周澤未渥也,而語極知,說行而有功則德忘,說不行而有敗則見疑,如此者身危。貴人有過端,而說者明言禮義以挑其惡,如此者身危。貴人或得計而欲自以為功,說者與知焉,如此者身危。彊以其所不能為,止以其所不能已,如此者身危。故與之論大人則以為閒己矣,與之論細人則以為賣重,論其所愛則以為藉資,論其所憎則以為嘗己也。徑省其說則以為不智而拙之,米鹽博辯則以為多而交之。略事陳意則曰怯懦而不盡,慮事廣肆則曰草野而倨侮。此說之難,不可不知也。

凡說之務,在知飾所說之所矜而滅其所恥。彼有私急也,必以公義示而強之。其意有下也,然而不能己,說者因為之飾其美而少其不為也。其心有高也,而實不能及,說者為之舉其過而見其惡而多其不行也。有欲矜以智能,則為之舉異事之同類者,多為之地,使之資說於我,而佯不知也以資其智。欲內相存之言,則必以美名明之,而微見其合於私利也。欲陳危害之事,則顯其毀誹而微見其合於私患也。譽異人與同行者,規異事與同計者。有與同汙者,則必以大飾其無傷也;有與同敗者,則必以明飾其無失也。彼自多其力,則毋以其難概之也;自勇其斷,則無以其謫怒之;自智其計,則毋以其敗窮之。大意無所拂悟,辭言無所繫縻,然後極騁智辯焉,此道所得親近不疑而得盡辭也。伊尹為宰,百里奚為虜,皆所以干其上也,此二人者,皆聖人也,然猶不能無役身以進,如此其汙也。今以吾言為宰虜,而可以聽用而振世,此非能仕之所恥也。夫曠日離久,而周澤既渥,深計而不疑,引爭而不罪,則明割利害以致其功,直指是非以飾其身,以此相持,此說之成也。

昔者鄭武公欲伐胡,故先以其女妻胡君以娛其意。因問於群臣:「吾欲用兵,誰可伐者?」大夫關其思對曰:「胡可伐。」武公怒而戮之,曰:「胡,兄弟之國也,子言伐之何也?」胡君聞之,以鄭為親己,遂不備鄭,鄭人襲胡,取之。宋有富人,天雨牆壞,其子曰:「不築,必將有盜。」其鄰人之父亦云。暮而果大亡其財,其家甚智其子,而疑鄰人之父。此二人說者皆當矣,厚者為戮,薄者見疑,則非知之難也,處知則難也。故繞朝之言當矣,其為聖人於晉,而為戮於秦也。此不可不察。

昔者彌子瑕有寵於衛君。衛國之法,竊駕君車者罪刖。彌子瑕母病,人閒往夜告彌子,彌子矯駕君車以出,君聞而賢之曰:「孝哉,為母之故,忘其刖罪。」異日,與君遊於果園,食桃而甘,不盡,以其半啗君,君曰:「愛我哉,忘其口味,以啗寡人。」及彌子色衰愛弛,得罪於君,君曰:「是固嘗矯駕吾車,又嘗啗我以餘桃。」故彌子之行未變於初也,而以前之所以見賢,而後獲罪者,愛憎之變也。故有愛於主則智當而加親,有憎於主則智不當見罪而加疏。故諫說談論之士,不可不察愛憎之主而後說焉。夫龍之為蟲也,柔可狎而騎也,然其喉下有逆鱗徑尺,若人有嬰之者則必殺人。人主亦有逆鱗,說者能無嬰人主之逆鱗,則幾矣。


\end{pinyinscope}