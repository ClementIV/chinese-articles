\article{五蠹}

\begin{pinyinscope}
上古之世,人民少而禽獸眾,人民不勝禽獸蟲蛇,有聖人作,搆木為巢以避群害,而民悅之,使王天下,號曰有巢氏。民食果蓏蚌蛤,腥臊惡臭而傷害腹胃,民多疾病,有聖人作,鑽燧取火以化腥臊,而民說之,使王天下,號之曰燧人氏。中古之世,天下大水,而鯀、禹決瀆。近古之世,桀、紂暴亂,而湯、武征伐。今有搆木鑽燧於夏后氏之世者,必為鯀、禹笑矣。有決瀆於殷、周之世者,必為湯、武笑矣。然則今有美堯、舜、湯、武、禹之道於當今之世者,必為新聖笑矣。是以聖人不期脩古,不法常可,論世之事,因為之備。宋人有耕田者,田中有株,兔走,觸株折頸而死,因釋其耒而守株,冀復得兔,兔不可復得,而身為宋國笑。今欲以先王之政,治當世之民,皆守株之類也。

古者丈夫不耕,草木之實足食也;婦人不織,禽獸之皮足衣也。不事力而養足,人民少而財有餘,故民不爭。是以厚賞不行,重罰不用而民自治。今人有五子不為多,子又有五子,大父未死而有二十五孫,是以人民眾而貨財寡,事力勞而供養薄,故民爭,雖倍賞累罰而不免於亂。

堯之王天下也,茅茨不翦,采椽不斲,糲粢之食,藜藿之羹,冬日麑裘,夏日葛衣,雖監門之服養,不虧於此矣。禹之王天下也,身執耒臿以為民先,股無胈,脛不生毛,雖臣虜之勞不苦於此矣。以是言之,夫古之讓天子者,是去監門之養而離臣虜之勞也,古傳天下而不足多也。今之縣令,一日身死,子孫累世絜駕,故人重之;是以人之於讓也,輕辭古之天子,難去今之縣令者,薄厚之實異也。夫山居而谷汲者,膢臘而相遺以水;澤居苦水者,買庸而決竇。故饑歲之春,幼弟不饟;穰歲之秋,疏客必食;非疏骨肉愛過客也,多少之實異也。是以古之易財,非仁也,財多也;今之爭奪,非鄙也,財寡也;輕辭天子,非高也,勢薄也;爭土橐,非下也,權重也。故聖人議多少、論薄厚為之政,故罰薄不為慈,誅嚴不為戾,稱俗而行也。故事因於世,而備適於事。

古者文王處豐、鎬之間,地方百里,行仁義而懷西戎,遂王天下。徐偃王處漢東,地方五百里,行仁義,割地而朝者三十有六國,荊文王恐其害己也,舉兵伐徐,遂滅之。故文王行仁義而王天下,偃王行仁義而喪其國,是仁義用於古不用於今也。故曰:世異則事異。當舜之時,有苗不服,禹將伐之,舜曰:「不可。上德不厚而行武,非道也。」乃修教三年,執干戚舞,有苗乃服。共工之戰,鐵銛矩者及乎敵,鎧甲不堅者傷乎體,是干戚用於古不用於今也。故曰:事異則備變。上古競於道德,中世逐於智謀,當今爭於氣力。齊將攻魯,魯使子貢說之,齊人曰:「子言非不辯也,吾所欲者土地也,非斯言所謂也。」遂舉兵伐魯,去門十里以為界。故偃王仁義而徐亡,子貢辯智而魯削。以是言之,夫仁義辯智,非所以持國也。去偃王之仁,息子貢之智,循徐、魯之力使敵萬乘,則齊、荊之欲不得行於二國矣。

夫古今異俗,新故異備,如欲以寬緩之政、治急世之民,猶無轡策而御駻馬,此不知之患也。今儒、墨皆稱先王兼愛天下,則視民如父母。何以明其然也?曰:「司寇行刑,君為之不舉樂;聞死刑之報,君為流涕。」此所舉先王也。夫以君臣為如父子則必治,推是言之,是無亂父子也。人之情性,莫先於父母,皆見愛而未必治也,雖厚愛矣,奚遽不亂?今先王之愛民,不過父母之愛子,子未必不亂也,則民奚遽治哉!且夫以法行刑而君為之流涕,此以效仁,非以為治也。夫垂泣不欲刑者仁也,然而不可不刑者法也,先王勝其法不聽其泣,則仁之不可以為治亦明矣。且民者固服於勢,寡能懷於義。仲尼,天下聖人也,修行明道以遊海內,海內說其仁,美其義,而為服役者七十人,蓋貴仁者寡,能義者難也。故以天下之大,而為服役者七十人,而仁義者一人。魯哀公,下主也,南面君國,境內之民莫敢不臣。民者固服於勢,誠易以服人,故仲尼反為臣,而哀公顧為君。仲尼非懷其義,服其勢也。故以義則仲尼不服於哀公,乘勢則哀公臣仲尼。今學者之說人主也,不乘必勝之勢,而務行仁義則可以王,是求人主之必及仲尼,而以世之凡民皆如列徒,此必不得之數也。

今有不才之子,父母怒之弗為改,鄉人譙之弗為動,師長教之弗為變。夫以父母之愛,鄉人之行,師長之智,三美加焉,而終不動其脛毛,不改;州部之吏,操官兵、推公法而求索姦人,然後恐懼,變其節,易其行矣。故父母之愛不足以教子,必待州部之嚴刑者,民固驕於愛、聽於威矣。故十仞之城,樓季弗能踰者,峭也;千仞之山,跛牂易牧者,夷也。故明王峭其法、而嚴其刑也。布帛尋常,庸人不釋;鑠金百溢,盜跖不掇。不必害則不釋尋常,必害手則不掇百溢,故明主必其誅也。是以賞莫如厚而信,使民利之;罰莫如重而必,使民畏之;法莫如一而固,使民知之。故主施賞不遷,行誅無赦。譽輔其賞,毀隨其罰,則賢不肖俱盡其力矣。

今則不然,以其有功也爵之,而卑其士官也;以其耕作也賞之,而少其家業也;以其不收也外之,而高其輕世也;以其犯禁也罪之,而多其有勇也。毀譽、賞罰之所加者相與悖繆也,故法禁壞而民愈亂。今兄弟被侵必攻者廉也,知友被辱隨仇者貞也,廉貞之行成,而君上之法犯矣。人主尊貞廉之行,而忘犯禁之罪,故民程於勇而吏不能勝也。不事力而衣食則謂之能,不戰功而尊則謂之賢,賢能之行成而兵弱而地荒矣。人主說賢能之行,而忘兵弱地荒之禍,則私行立而公利滅矣。

儒以文亂法,俠以武犯禁,而人主兼禮之,此所以亂也。夫離法者罪,而諸先生以文學取;犯禁者誅,而群俠以私劍養。故法之所非,君之所取;吏之所誅,上之所養也。法趣上下四相反也,而無所定,雖有十黃帝不能治也。故行仁義者非所譽,譽之則害功;文學者非所用,用之則亂法。楚之有直躬,其父竊羊而謁之吏,令尹曰:「殺之,」以為直於君而曲於父,報而罪之。以是觀之,夫君之直臣,父之暴子也。魯人從君戰,三戰三北,仲尼問其故,對曰:「吾有老父,身死莫之養也。」仲尼以為孝,舉而上之。以是觀之,夫父之孝子,君之背臣也。故令尹誅而楚姦不上聞,仲尼賞而魯民易降北。上下之利若是其異也,而人主兼舉匹夫之行,而求致社稷之福,必不幾矣。古者蒼頡之作書也,自環者謂之私,背私謂之公,公私之相背也,乃蒼頡固以知之矣。今以為同利者,不察之患也。然則為匹夫計者,莫如脩行義而習文學。行義脩則見信,見信則受事;文學習則為明師,為明師則顯榮;此匹夫之美也。然則無功而受事,無爵而顯榮,為有政如此,則國必亂,主必危矣。故不相容之事,不兩立也。斬敵者受賞,而高慈惠之行;拔城者受爵祿,而信廉愛之說;堅甲厲兵以備難,而美薦紳之飾;富國以農,距敵恃卒,而貴文學之士;廢敬上畏法之民,而養遊俠私劍之屬。舉行如此,治強不可得也。國平養儒俠,難至用介士,所利非所用,所用非所利。是故服事者簡其業,而游學者日眾,是世之所以亂也。

且世之所謂賢者,貞信之行也。所謂智者,微妙之言也。微妙之言,上智之所難知也。今為眾人法,而以上智之所難知,則民無從識之矣。故糟糠不飽者不務梁肉,短褐不完者不待文繡。夫治世之事,急者不得,則緩者非所務也。今所治之政,民閒之事,夫婦所明知者不用,而慕上知之論,則其於治反矣。故微妙之言,非民務也。若夫賢良貞信之行者,必將貴不欺之士。不欺之士者,亦無不欺之術也。布衣相與交,無富厚以相利,無威勢以相懼也,故求不欺之士。今人主處制人之勢,有一國之厚,重賞嚴誅,得操其柄,以修明術之所燭,雖有田常、子罕之臣,不敢欺也,奚待於不欺之士?今貞信之士不盈於十,而境內之官以百數,必任貞信之士,則人不足官,人不足官則治者寡而亂者眾矣。故明主之道,一法而不求智,固術而不慕信,故法不敗,而群官無姦詐矣。

今人主之於言也,說其辯而不求其當焉;其用於行也,美其聲而不責其功焉。是以天下之眾,其談言者務為辯而不周於用,故舉先王言仁義者盈廷,而政不免於亂;行身者競於為高而不合於功,故智士退處巖穴、歸祿不受,而兵不免於弱,政不免於亂,此其故何也?民之所譽,上之所禮,亂國之術也。今境內之民皆言治,藏商、管之法者家有之,而國愈貧,言耕者眾,執耒者寡也;境內皆言兵,藏孫、吳之書者家有之,而兵愈弱,言戰者多,被甲者少也。故明主用其力,不聽其言;賞其功,必禁無用;故民盡死力以從其上。夫耕之用力也勞,而民為之者,曰:可得以富也。戰之為事也危,而民為之者,曰:可得以貴也。今修文學、習言談,則無耕之勞、而有富之實,無戰之危、而有貴之尊,則人孰不為也?是以百人事智而一人用力,事智者眾則法敗,用力者寡則國貧,此世之所以亂也。故明主之國,無書簡之文,以法為教;無先王之語,以吏為師;無私劍之捍,以斬首為勇。是境內之民,其言談者必軌於法,動作者歸之於功,為勇者盡之於軍。是故無事則國富,有事則兵強,此之謂王資。既畜王資而承敵國之舋,超五帝,侔三王者,必此法也。

今則不然,士民縱恣於內,言談者為勢於外,外內稱惡以待強敵,不亦殆乎!故群臣之言外事者,非有分於從衡之黨,則有仇讎之忠,而借力於國也。從者,合眾弱以攻一強也;而衡者,事一強以攻眾弱也;皆非所以持國也。今人臣之言衡者皆曰:「不事大則遇敵受禍矣。」事大未必有實,則舉圖而委,效璽而請兵矣。獻圖則地削,效璽則名卑,地削則國削,名卑則政亂矣。事大為衡未見其利也,而亡地亂政矣。人臣之言從者皆曰:「不救小而伐大則失天下,失天下則國危,國危而主卑。」救小未必有實,則起兵而敵大矣。救小未必能存,而交大未必不有疏,有疏則為強國制矣。出兵則軍敗,退守則城拔,救小為從未見其利,而亡地敗軍矣。是故事強則以外權士官於內,救小則以內重求利於外,國利未立,封土厚祿至矣;主上雖卑,人臣尊矣;國地雖削,私家富矣。事成則以權長重,事敗則以富退處。人主之於其聽說也,於其臣,事未成則爵祿已尊矣;事敗而弗誅,則游說之士,孰不為用矰繳之說而徼倖其後?故破國亡主以聽言談者之浮說,此其故何也?是人君不明乎公私之利,不察當否之言,而誅罰不必其後也。皆曰「外事大可以王,小可以安。」夫王者,能攻人者也;而安,則不可攻也。強,則能攻人者也;治,則不可攻也。治強不可責於外,內政之有也。今不行法術於內,而事智於外,則不至於治強矣。鄙諺曰:「長袖善舞,多錢善賈。」此言多資之易為工也。故治強易為謀,弱亂難為計。故用於秦者十變而謀希失,用於燕者一變而計希得,非用於秦者必智,用於燕者必愚也,蓋治亂之資異也。故周去秦為從,期年而舉;衛離魏為衡,半歲而亡。是周滅於從,衛亡於衡也。使周、衛緩其從衡之計,而嚴其境內之治,明其法禁,必其賞罰,盡其地力以多其積,致其民死以堅其城守,天下得其地則其利少,攻其國則其傷大,萬乘之國、莫敢自頓於堅城之下,而使強敵裁其弊也,此必不亡之術也。舍必不亡之術而道必滅之事,治國者之過也。智困於內而政亂於外,則亡不可振也。

民之故計,皆就安利如辟危窮。今為之攻戰,進則死於敵,退則死於誅則危矣。棄私家之事而必汗馬之勞,家困而上弗論則窮矣。窮危之所在也,民安得勿避。故事私門而完解舍,解舍完則遠戰,遠戰則安。行貨賂而襲當塗者則求得,求得則私安,私安則利之所在,安得勿就?是以公民少而私人眾矣。夫明王治國之政,使其商工游食之民少而名卑,以寡趣本務而趨末作。今世近習之請行則官爵可買,官爵可買則商工不卑也矣;姦財貨賈得用於市則商人不少矣。聚斂倍農而致尊過耕戰之士,則耿介之士寡而高價之民多矣。

是故亂國之俗,其學者則稱先王之道,以籍仁義,盛容服而飾辯說,以疑當世之法而貳人主之心。其言古者,為設詐稱,借於外力,以成其私而遺社稷之利。其帶劍者,聚徒屬,立節操,以顯其名而犯五官之禁。其患御者,積於私門,盡貨賂而用重人之謁,退汗馬之勞。其商工之民,修治苦窳之器,聚弗靡之財,蓄積待時而侔農夫之利。此五者,邦之蠹也。人主不除此五蠹之民,不養耿介之士,則海內雖有破亡之國,削滅之朝,亦勿怪矣。


\end{pinyinscope}