\article{難言}

\begin{pinyinscope}
臣非非難言也,所以難言者:言順比滑澤,洋洋纚纚然,則見以為華而不實。敦祗恭厚,鯁固慎完,則見以為掘而不倫。多言繁稱,連類比物,則見以為虛而無用。摠微說約,徑省而不飾,則見以為劌而不辯。激急親近,探知人情,則見以為譖而不讓。閎大廣博,妙遠不測,則見以為夸而無用。家計小談,以具數言,則見以為陋。言而近世,辭不悖逆,則見以為貪生而諛上。言而遠俗,詭躁人間,則見以為誕。捷敏辯給,繁於文采,則見以為史。殊釋文學,以質信言,則見以為鄙。時稱詩書,道法往古,則見以為誦。此臣非之所以難言而重患也。

故度量雖正,未必聽也;義理雖全,未必用也。大王若以此不信,則小者以為毀訾誹謗,大者患禍災害死亡及其身。故子胥善謀而吳戮之,仲尼善說而匡圍之,管夷吾實賢而魯囚之。故此三大夫豈不賢哉?而三君不明也。上古有湯至聖也,伊尹至智也;夫至智說至聖,然且七十說而不受,身執鼎俎為庖宰,昵近習親,而湯乃僅知其賢而用之。故曰以至智說至聖,未必至而見受,伊尹說湯是也;以智說愚必不聽,文王說紂是也。故文王說紂而紂囚之,翼侯炙,鬼侯腊,比干剖心,梅伯醢,夷吾束縛,而曹羈奔陳,伯里子道乞,傅說轉鬻,孫子臏腳於魏,吳起收泣於岸門、痛西河之為秦、卒枝解於楚,公叔痤言國器、反為悖,公孫鞅奔秦,關龍逢斬,萇宏分胣,尹子阱於棘,司馬子期死而浮於江,田明辜射,宓子賤、西門豹不鬥而死人手,董安于死而陳於市,宰予不免於田常,范睢折脅於魏。此十數人者,皆世之仁賢忠良有道術之士也,不幸而遇悖亂闇惑之主而死,然則雖賢聖不能逃死亡避戮辱者何也?則愚者難說也,故君子不少也。且至言忤於耳而倒於心,非賢聖莫能聽,願大王熟察之也。


\end{pinyinscope}