\article{定法}

\begin{pinyinscope}
問者曰:「申不害、公孫鞅,此二家之言孰急於國?」應之曰:「是不可程也。人不食,十日則死;大寒之隆,不衣亦死。謂之衣食孰急於人,則是不可一無也,皆養生之具也。今申不害言術,而公孫鞅為法。術者,因任而授官,循名而責實,操殺生之柄,課群臣之能者也,此人主之所執也。法者,憲令著於官府,刑罰必於民心,賞存乎慎法,而罰加乎姦令者也,此臣之所師也。君無術則弊於上,臣無法則亂於下,此不可一無,皆帝王之具也。」

問者曰:「徒術而無法,徒法而無術,其不可何哉?」對曰:「申不害,韓昭侯之佐也。韓者,晉之別國也。晉之故法未息,而韓之新法又生;先君之令未收,而後君之令又下。申不害不擅其法,不一其憲令則姦多故。利在故法前令則道之,利在新法後令則道之,利在故新相反,前後相勃。則申不害雖十使昭侯用術,而姦臣猶有所譎其辭矣。故託万乘之勁韓,七十年而不至於霸王者,雖用術於上,法不勤飾於官之患也。公孫鞅之治秦也,設告相坐而責其實,連什伍而同其罪,賞厚而信,刑重而必,是以其民用力勞而不休,逐敵危而不卻,故其國富而兵強。然而無術以知姦,則以其富強也資人臣而已矣。及孝公、商君死,惠王即位,秦法未敗也,而張儀以秦殉韓、魏。惠王死,武王即位,甘茂以秦殉周。武王死,昭襄王即位,穰侯越韓、魏而東攻齊,五年而秦不益尺土之地,乃城其陶邑之封,應侯攻韓八年,成其汝南之封;自是以來,諸用秦者皆應、穰之類也。故戰勝則大臣尊,益地則私封立,主無術以知姦也。商君雖十飾其法,人臣反用其資。故乘強秦之資,數十年而不至於帝王者,法不勤飾於官,主無術於上之患也。」

問者曰:「主用申子之術、而官行商君之法,可乎?」對曰:「申子未盡於法也。申子言『治不踰官,雖知弗言』。治不踰官,謂之守職也可;知而弗言,是不謂過也。人主以一國目視,故視莫明焉;以一國耳聽,故聽莫聰焉。今知而弗言,則人主尚安假借矣?商君之法曰:『斬一首者爵一級,欲為官者為五十石之官;斬二首者爵二級,欲為官者為百石之官。』官爵之遷與斬首之功相稱也。今有法曰:斬首者令為醫匠,則屋不成而病不已。夫匠者,手巧也;而醫者,齊藥也;而以斬首之功為之,則不當其能。今治官者,智能也;今斬首者,勇力之所加也。以勇力之所加、而治智能之官,是以斬首之功為醫匠也。故曰:二子之於法術,皆未盡善也。」


\end{pinyinscope}