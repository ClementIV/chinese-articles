\article{揚權}

\begin{pinyinscope}
天有大命,人有大命。夫香美脆味,厚酒肥肉,甘口而病形;曼理皓齒,說情而捐精。故去甚去泰,身乃無害。權不欲見,素無為也。事在四方,要在中央。聖人執要,四方來效。虛而待之,彼自以之。四海既藏,道陰見陽。左右既立,開門而當。勿變勿易,與二俱行,行之不已,是謂履理也。夫物者有所宜,材者有所施,各處其宜,故上下無為。使雞司夜,令狸執鼠,皆用其能,上乃無事。上有所長,事乃不方。矜而好能,下之所欺。辯惠好生,下因其材。上下易用,國故不治。

用一之道,以名為首。名正物定,名倚物徙。故聖人執一以靜,使名自命,令事自定。不見其采,下故素正。因而任之,使自事之。因而予之,彼將自舉之。正與處之,使皆自定之。上以名舉之,不知其名,復脩其形。形名參同,用其所生。二者誠信,下乃貢情。謹脩所事,待命於天。毋失其要,乃為聖人。聖人之道,去智與巧,智巧不去,難以為常。民人用之,其身多殃,主上用之,其國危亡。因天之道,反形之理,督參鞠之,終則有始。虛以靜後,未嘗用己。凡上之患,必同其端。信而勿同,萬民一從。

夫道者、弘大而無形,德者、覈理而普至。至於群生,斟酌用之,萬物皆盛,而不與其寧。道者、下周於事,因稽而命,與時生死。參名異事,通一同情。故曰道不同於萬物,德不同於陰陽,衡不同於輕重,繩不同於出入,和不同於燥溼,君不同於群臣。凡此六者,道之出也。道無雙,故曰一。是故明君貴獨道之容。君臣不同道,下以名禱,君操其名,臣效其形,形名參同,上下和調也。

凡聽之道,以其所出,反以為之入。故審名以定位,明分以辯類。聽言之道,溶若甚醉。脣乎齒乎,吾不為始乎,齒乎脣乎,愈惛惛乎。彼自離之,吾因以知之。是非輻湊,上不與構。虛靜無為,道之情也;參伍比物,事之形也。參之以比物,伍之以合虛。根幹不革,則動泄不失矣。動之溶之,無為而改之。喜之則多事,惡之則生怨。故去喜去惡,虛心以為道舍。上不與共之,民乃寵之。上不與義之,使獨為之。上固閉內扃,從室視庭,參咫尺已具,皆之其處。以賞者賞,以刑者刑。因其所為,各以自成。善惡必及,孰敢不信!規矩既設,三隅乃列。

主上不神,下將有因。其事不當,下考其常。若天若地,是謂累解。若地若天,孰疏孰親?能象天地,是謂聖人。欲治其內,置而勿親;欲治其外,官置一人;不使自恣,安得移并。大臣之門,唯恐多人。凡治之極,下不能得。周合刑名,民乃守職。去此更求,是謂大惑。猾民愈眾,姦邪滿側。故曰:毋富人而貸焉,毋貴人而逼焉,毋專信一人而失其都國焉。腓大於股,難以趣走。主失其神,虎隨其後。主上不知,虎將為狗。主不蚤止,狗益無已。虎成其群,以弒其母。為主而無臣,奚國之有!主施其法,大虎將怯;主施其刑,大虎自寧。法刑狗信,虎化為人,復反其真。

欲為其國,必伐其聚,不伐其聚,彼將聚眾。欲為其地,必適其賜,不適其賜,亂人求益。彼求我予,假仇人斧,假之不可,彼將用之以伐我。黃帝有言曰:「上下一日百戰。」下匿其私,用試其上;上操度量,以割其下。故度量之立,主之寶也;黨與之具,臣之寶也。臣之所不弒其君者,黨與不具也。故上失扶寸,下得尋常。有國之君,不大其都。有道之臣,不貴其家。有道之君,不貴其臣。貴之富之,備將代之。備危恐殆,急置太子,禍乃無從起。內索出圉,必身自執其度量。厚者虧之,薄者靡之。虧靡有量,毋使民比周,同欺其上。虧之若月,靡之若熱。簡令謹誅,必盡其罰。毋弛而弓,一棲兩雄。一棲兩雄,其鬬㘖㘖。豺狼在牢,其羊不繁。一家二貴,事乃無功。夫妻持政,子無適從。為人君者,數披其木,毋使木枝扶疏;木枝扶疏,將塞公閭,私門將實,公庭將虛,主將壅圍。數披其木,無使木枝外拒;木枝外拒,將逼主處。數披其木,毋使枝大本小,枝大本小,將不勝春風,不勝春風,枝將害心。公子既眾,宗室憂吟。止之之道,數披其木,毋使枝茂。木數披,黨與乃離。掘其根本,木乃不神。填其洶淵,毋使水清。探其懷,奪之威。主上用之,若電若雷。


\end{pinyinscope}