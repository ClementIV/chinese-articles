\article{解老}

\begin{pinyinscope}
德者,內也。得者,外也。上德不德,言其神不淫於外也。神不淫於外則身全,身全之謂德。德者,得身也。凡德者,以無為集,以無欲成,以不思安,以不用固。為之欲之,則德無舍,德無舍則不全。用之思之則不固,不固則無功,無功則生於德。德則無德,不德則在有德。故曰:「上德不德,是以有德。」

所以貴無為無思為虛者,謂其意無所制也。夫無術者,故以無為無思為虛也。夫故以無為無思為虛者,其意常不忘虛,是制於為虛也。虛者,謂其意無所制也。今制於為虛,是不虛也。虛者之無為也,不以無為為有常,不以無為為有常則虛,虛則德盛,德盛之謂上德,故曰:「上德無為而無不為也。」

仁者,謂其中心欣然愛人也。其喜人之有福,而惡人之有禍也。生心之所不能已也,非求其報也。故曰:「上仁為之而無以為也。」

義者,君臣上下之事,父子貴賤之差也,知交朋友之接也,親疏內外之分也。臣事君宜,下懷上宜,子事父宜,賤敬貴宜,知交友朋之相助也宜,親者內而疏者外宜。義者,謂其宜也,宜而為之,故曰:「上義為之而有以為也。」

禮者,所以貌情也,群義之文章也,君臣父子之交也,貴賤賢不肖之所以別也。中心懷而不諭,故疾趨卑拜而明之。實心愛而不知,故好言繁辭以信之。禮者,外節之所以諭內也。故曰:「禮以貌情也。」凡人之為外物動也,不知其為身之禮也。眾人之為禮也,以尊他人也,故時勸時衰。君子之為禮,以為其身,以為其身,故神之為上禮,上禮神而眾人貳,故不能相應,不能相應,故曰:「上禮為之而莫之應。」眾人雖貳,聖人之復恭敬盡手足之禮也不衰,故曰:「攘臂而仍之。」道有積而德有功,德者道之功。功有實而實有光,仁者德之光。光有澤而澤有事,義者仁之事也。事有禮而禮有文,禮者義之文也。故曰:「失道而後失德,失德而後失仁,失仁而後失義,失義而後失禮。」

禮為情貌者也,文為質飾者也。夫君子取情而去貌,好質而惡飾。夫恃貌而論情者,其情惡也;須飾而論質者,其質衰也。何以論之?和氏之璧,不飾以五采,隋侯之珠,不飾以銀黃,其質至美,物不足以飾之。夫物之待飾而後行者,其質不美也。是以父子之間,其禮樸而不明,故曰:「禮薄也。」凡物不並盛,陰陽是也。理相奪予,威德是也。實厚者貌薄,父子之禮是也。由是觀之,禮繁者實心衰也。然則為禮者,事通人之樸心者也。眾人之為禮也,人應則輕歡,不應則責怨。今為禮者事通人之樸心,而資之以相責之分,能毋爭乎?有爭則亂,故曰:「禮者,忠信之薄也,而亂之首乎。」

先物行先理動之謂前識,前識者,無緣而忘意度也。何以論之?詹何坐,弟子侍,有牛鳴於門外,弟子曰:「是黑牛也而白題。」詹何曰:「然,是黑牛也,而白在其角。」使人視之,果黑牛而以布裹其角。以詹子之術,嬰眾人之心,華焉殆矣,故曰「道之華也」。嘗試釋詹子之察,而使五尺之愚童子視之,亦知其黑牛而以布裹其角也。故以詹子之察,苦心傷神,而後與五尺之愚童子同功,是以曰「愚之首也」。故曰:「前識者道之華也,而愚之首也。」

所謂大丈夫者,謂其智之大也。所謂處其厚不處其薄者,行情實而去禮貌也。所謂處其實不處其華者,必緣理不徑絕也。所謂去彼取此者,去貌徑絕而取緣理好情實也。故曰:「去彼取此。」

人有禍則心畏恐,心畏恐則行端直,行端直則思慮熟,思慮熟則得事理,行端直則無禍害,無禍害則盡天年,得事理則必成功,盡天年則全而壽,必成功則富與貴,全壽富貴之謂福。而福本於有禍,故曰:「禍兮福之所倚。」以成其功也。

人有福則富貴至,富貴至則衣食美,衣食美則驕心生,驕心生則行邪僻而動棄理,行邪僻則身死夭,動棄理則無成功。夫內有死夭之難,而外無成功之名者,大禍也。而禍本生於有福,故曰:「福兮禍之所伏」。

夫緣道理以從事者無不能成。無不能成者,大能成天子之勢尊,而小易得卿相將軍之賞祿。夫棄道理而忘舉動者,雖上有天子諸侯之勢尊,而下有猗頓、陶朱、卜祝之富,猶失其民人而亡其財資也。眾人之輕棄道理而易忘舉動者,不知其禍福之深大而道闊遠若是也,故諭人曰:「熟知其極。」人莫不欲富貴全壽,而未有能免於貧賤死夭之禍也,心欲富貴全壽,而今貧賤死夭,是不能至於其所欲至也。凡失其所欲之路而妄行者之謂迷,迷則不能至於其所欲至矣。今眾人之不能至於其所欲至,故曰「迷」。眾人之所不能至於其所欲至也,自天地之剖判以至于今,故曰:「人之迷也,其日故以久矣。」

所謂方者,內外相應也,言行相稱也。所謂廉者,必生死之命也,輕恬資財也。所謂直者,義必公正,公心不偏黨也。所謂光者,官爵尊貴,衣裘壯麗也。今有道之士,雖中外信順,不以誹謗窮墮;雖死節輕財,不以侮罷羞貪;雖義端不黨,不以去邪罪私;雖勢尊衣美,不以夸賤欺貧。其故何也?使失路者而肯聽習問知,即不成迷也。今眾人之所以欲成功而反為敗者,生於不知道理而不肯問知而聽能。眾人不肯問知聽能,而聖人強以其禍敗適之,則怨。眾人多而聖人寡,寡之不勝眾,數也。今舉動而與天下之為讎,非全身長生之道也,是以行軌節而舉之也。故曰:「方而不割,廉而不劌,直而不肆,光而不耀。」

聰明睿智天也,動靜思慮人也。人也者,乘於天明以視,寄於天聰以聽,託於天智以思慮。故視強則目不明,聽甚則耳不聰,思慮過度則智識亂。目不明則不能決黑白之分,耳不聰則不能別清濁之聲,智識亂則不能審得失之地。目不能決黑白之色則謂之盲,耳不能別清濁之聲則謂之聾,心不能審得失之地則謂之狂。盲則不能避晝日之險,聾則不能知雷霆之害,狂則不能免人間法令之禍。書之所謂治人者,適動靜之節,省思慮之費也。所謂事天者,不極聰明之力,不盡智識之任。苟極盡則費神多,費神多則盲聾悖狂之禍至,是以嗇之。嗇之者,愛其精神,嗇其智識也。故曰:「治人事天莫如嗇。」

眾人之用神也躁,躁則多費,多費之謂侈。聖人之用神也靜,靜則少費,少費之謂嗇。嗇之謂術也生於道理。夫能嗇也,是從於道而服於理者也。眾人離於患,陷於禍,猶未知退,而不服從道理。聖人雖未見禍患之形,虛無服從於道理,以稱蚤服。故曰:「夫謂嗇,是以蚤服。」

知治人者其思慮靜,知事天者其孔竅虛。思慮靜,故德不去。孔竅虛,則和氣日入。故曰:「重積德。」夫能令故德不去,新和氣日至者,蚤服者也。故曰:「蚤服是謂重積德。」積德而後神靜,神靜而後和多,和多而後計得,計得而後能御萬物,能御萬物則戰易勝敵,戰易勝敵而論必蓋世,論必蓋世,故曰「無不克」。無不克本於重積德,故曰「重積德則無不克」。戰易勝敵則兼有天下,論必蓋世則民人從。進兼天下而退從民人,其術遠,則眾人莫見其端末。莫見其端末,是以莫知其極,故曰:「無不克則莫知其極。」

凡有國而後亡之,有身而後殃之,不可謂能有其國能保其身。夫能有其國、必能安其社稷,能保其身、必能終其天年,而後可謂能有其國、能保其身矣。夫能有其國、保其身者必且體道,體道則其智深,其智深則其會遠,其會遠眾人莫能見其所極。唯夫能令人不見其事極,不見事極者為保其身、有其國,故曰:「莫知其極;莫知其極,則可以有國。」

所謂有國之母,母者,道也,道也者生於所以有國之術,所以有國之術,故謂之有國之母。夫道以與世周旋者,其建生也長,持祿也久,故曰:「有國之母,可以長久。」樹木有曼根,有直根。根者,書之所謂柢也。柢也者,木之所以建生也;曼根者,木之所以持生也。德也者,人之所以建生也;祿也者,人之所以持生也。今建於理者其持祿也久,故曰:「深其根。」體其道者,其生日長,故曰:「固其柢。」柢固則生長,根深則視久,故曰:「深其根,固其柢,長生久視之道也。」

工人數變業則失其功,作者數搖徙則亡其功。一人之作,日亡半日,十日則亡五人之功矣。萬人之作,日亡半日,十日則亡五萬人之功矣。然則數變業者,其人彌眾,其虧彌大矣。凡法令更則利害易,利害易則民務變,務變之謂變業。故以理觀之,事大眾而數搖之則少成功,藏大器而數徙之則多敗傷,烹小鮮而數撓之則賊其澤,治大國而數變法則民苦之,是以有道之君貴靜,不重變法,故曰:「治大國者若烹小鮮。」

人處疾則貴醫,有禍則畏鬼。聖人在上則民少欲,民少欲則血氣治,而舉動理則少禍害。夫內無痤疽癉痔之害,而外無刑罰法誅之禍者,其輕恬鬼也甚,故曰:「以道蒞天下,其鬼不神。」治世之民不與鬼神相害也,故曰:「非其鬼不神也,其神不傷人也。」鬼崇也疾人之謂鬼傷人,人逐除之之謂人傷鬼也;民犯法令之謂民傷上,上刑戮民之謂上傷民;民不犯法則上亦不行刑,上不行刑之謂上不傷人;故曰:「聖人亦不傷民。」上不與民相害,而人不與鬼相傷,故曰:「兩不相傷。」民不敢犯法,則上內不用刑罰,而外不事利其產業,上內不用刑罰、而外不事利其產業則民蕃息,民蕃息而畜積盛,民蕃息而畜積盛之謂有德。凡所謂崇者,魂魄去而精神亂,精神亂則無德。鬼不崇人則魂魄不去,魂魄不去而精神不亂,精神不亂之謂有德。上盛畜積,而鬼不亂其精神,則德盡在於民矣。故曰:「兩不相傷,則德交歸焉。」言其德上下交盛而俱歸於民也。

有道之君,外無怨讎於鄰敵,而內有德澤於人民。夫外無怨讎於鄰敵者,其遇諸侯也外有禮義。內有德澤於人民者,其治人事也務本。遇諸侯有禮義則役希起,治民事務本則淫奢止。凡馬之所以大用者,外供甲兵,而內給淫奢也。今有道之君,外希用甲兵,而內禁淫奢。上不事馬於戰鬥逐北,而民不以馬遠淫通物,所積力唯田疇,積力於田疇必且糞灌,故曰:「天下有道,卻走馬以糞也。」

人君者無道,則內暴虐其民,而外侵欺其鄰國。內暴虐則民產絕,外侵欺則兵數起。民產絕則畜生少,兵數起則士卒盡。畜生少則戎馬乏,士卒盡則軍危殆。戎馬乏則將馬出,軍危殆則近臣役。馬者,軍之大用;郊者,言其近也。今所以給軍之具於將馬近臣,故曰:「天下無道,戎馬生於郊矣。」

人有欲則計會亂,計會亂而有欲甚,有欲甚則邪心勝,邪心勝則事經絕,事經絕則禍難生。由是觀之,禍難生於邪心,邪心誘於可欲。可欲之類,進則教良民為姦,退則令善人有禍。姦起則上侵弱君,禍至則民人多傷。然則可欲之類,上侵弱君而下傷人民。夫上侵弱君而下傷人民者,大罪也。故曰:「禍莫大於可欲。」是以聖人不引五色,不淫於聲樂,明君賤玩好而去淫麗。人無毛羽,不衣則不犯寒。上不屬天,而下不著地,以腸胃為根本,不食則不能活。是以不免於欲利之心,欲利之心不除,其身之憂也。故聖人衣足以犯寒,食足以充虛,則不憂矣。眾人則不然,大為諸侯,小餘千金之資,其欲得之憂不除也,胥靡有免,死罪時活,今不知足者之憂,終身不解,故曰:「禍莫大於不知足。」故欲利甚於憂,憂則疾生,疾生而智慧衰,智慧衰則失度量,失度量則妄舉動,妄舉動則禍害至,禍害至而疾嬰內,疾嬰內則痛禍薄外,痛禍薄外則苦痛雜於腸胃之間,苦痛雜於腸胃之間則傷人也憯,憯則退而自咎,退而自咎也生於欲利,故曰:「咎莫憯於欲利。」

道者,萬物之所然也,萬理之所稽也。理者,成物之文也;道者,萬物之所以成也。故曰:「道,理之者也。」物有理不可以相薄,物有理不可以相薄故理之為物之制。萬物各異理,萬物各異理而道盡。稽萬物之理,故不得不化;不得不化,故無常操;無常操,是以死生氣稟焉,萬智斟酌焉,萬事廢興焉。天得之以高,地得之以藏,維斗得之以成其威,日月得之以恆其光,五常得之以常其位,列星得之以端其行,四時得之以御其變氣,軒轅得之以擅四方,赤松得之與天地統,聖人得之以成文章。道與堯、舜俱智,與接輿俱狂,與桀、紂俱滅,與湯、武俱昌。以為近乎,遊於四極;以為遠乎,常在吾側;以為暗乎,其光昭昭;以為明乎,其物冥冥;而功成天地,和化雷霆,宇內之物,恃之以成。凡道之情,不制不形,柔弱隨時,與理相應。萬物得之以死,得之以生;萬事得之以敗,得之以成。道譬諸若水,溺者多飲之即死,渴者適飲之即生。譬之若劍戟,愚人以行忿則禍生,聖人以誅暴則福成。故得之以死,得之以生,得之以敗,得之以成。

人希見生象也,而得死象之骨,案其圖以想其生也,故諸人之所以意想者皆謂之象也。今道雖不可得聞見,聖人執其見功以處見其形,故曰:「無狀之狀,無物之象。」

凡理者,方圓、短長、麤靡、堅脆之分也。故理定而後可得道也。故定理有存亡,有死生,有盛衰。夫物之一存一亡,乍死乍生,初盛而後衰者,不可謂常。唯夫與天地之剖判也具生,至天地之消散也不死不衰者謂常。而常者,無攸易,無定理,無定理非在於常所,是以不可道也。聖人觀其玄虛,用其周行,強字之曰道,然而可論,故曰:「道之可道,非常道也。」

人始於生而卒於死。始之謂出,卒之謂入,故曰:「出生入死。」人之身三百六十節,四肢,九竅,其大具也。四肢與九竅十有三者,十有三者之動靜盡屬於生焉。屬之謂徒也,故曰:「生之徒也十有三者。」至死也十有三具者皆還而屬之於死,死之徒亦有十三,故曰:「生之徒,十有三;死之徒,十有三。」凡民之生生而生者固動,動盡則損也,而動不止,是損而不止也,損而不止則生盡,生盡之謂死,則十有三具者皆為死死地也。故曰:「民之生,生而動,動皆之死地,之十有三。」是以聖人愛精神而貴處靜,此甚大於兕虎之害。夫兕虎有域,動靜有時,避其域,省其時,則免其兕虎之害矣。民獨知兕虎之有爪角也,而莫知萬物之盡有爪角也,不免於萬物之害。何以論之?時雨降集,曠野閒靜,而以昏晨犯山川,則風露之爪角害之。事上不忠,輕犯禁令,則刑法之爪角害之。處鄉不節,憎愛無度,則爭鬥之爪角害之。嗜慾無限,動靜不節,則痤疽之爪角害之。好用其私智而棄道理,則網羅之爪角害之。兕虎有域,而萬害有原,避其域,塞其原,則免於諸害矣。凡兵革者,所以備害也。重生者雖入軍無忿爭之心,無忿爭之心則無所用救害之備。此非獨謂野處之軍也,聖人之遊世也無害人之心,無害人之心則必無人害,無人害則不備人,故曰:「陸行不遇兕虎。」入山不恃備以救害,故曰:「入軍不備甲兵。」遠諸害,故曰:「兕無所投其角,虎無所錯其爪,兵無所容其刃。」不設備而必無害,天地之道理也。體天地之道,故曰:「無死地焉。」動無死地,而謂之「善攝生」矣。

愛子者慈於子,重生者慈於身,貴功者慈於事。慈母之於弱子也,務致其福,務致其福則事除其禍,事除其禍則思慮熟,思慮熟則得事理,得事理則必成功,必成功則其行之也不疑,不疑之謂勇。聖人之於萬事也,盡如慈母之為弱子慮也,故見必行之道,見必行之道則明,其從事亦不疑,不疑之謂勇。不疑生於慈,故曰:「慈故能勇。」

周公曰:「冬日之閉凍也不固,則春夏之長草木也不茂。」天地不能常侈常費,而況於人乎?故萬物必有盛衰,萬事必有弛張,國家必有文武,官治必有賞罰。是以智士儉用其財則家富,聖人愛寶其神則精盛,人君重戰其卒則民眾。民眾則國廣,是以舉之曰:「儉故能廣。」

凡物之有形者易裁也,易割也。何以論之?有形則有短長,有短長則有小大,有小大則有方圓,有方圓則有堅脆,有堅脆則有輕重,有輕重則有白黑。短長、大小、方圓、堅脆、輕重、白黑之謂理。理定而物易割也。故議於大庭而後言則立,權議之士知之矣。故欲成方圓而隨其規矩,則萬事之功形矣。而萬物莫不有規矩。議言之士,計會規矩也。聖人盡隨於萬物之規矩,故曰:「不敢為天下先。」不敢為天下先則事無不事,功無不功,而議必蓋世,欲無處大官,其可得乎?處大官之謂為成事長,是以故曰:「不敢為天下先,故能為成事長。」

慈於子者不敢絕衣食,慈於身者不敢離法度,慈於方圓者不敢舍規矩。故臨兵而慈於士吏則戰勝敵,慈於器械則城堅固。故曰:「慈於戰則勝,以守則固。」夫能自全也而盡隨於萬物之理者,必且有天生。天生也者,生心也。故天下之道盡之生也,若以慈衛之也。事必萬全,而舉無不當,則謂之寶矣。故曰:「吾有三寶,持而寶之。」

書之所謂大道也者,端道也。所謂貌施也者,邪道也。所謂徑大也者,佳麗也。佳麗也者,邪道之分也。朝甚除也者,獄訟繁也。獄訟繁則田荒,田荒則府倉虛,府倉虛則國貧,國貧而民俗淫侈,民俗淫侈則衣食之業絕,衣食之業絕則民不得無飾巧詐,飾巧詐則知采文,知采文之謂服文采。獄訟繁、倉廩虛、而有以淫侈為俗,則國之傷也若以利劍刺之。故曰:「帶利劍。」諸夫飾智故以至於傷國者,其私家必富,私家必富,故曰:「資貨有餘。」國有若是者,則愚民不得無術而效之,效之則小盜生。由是觀之,大姦作則小盜隨,大姦唱則小盜和。竽也者,五聲之長者也,故竽先則鍾瑟皆隨,竽唱則諸樂皆和。今大姦作則俗之民唱,俗之民唱則小盜必和,故服文采,帶利劍,厭飲食,而貨資有餘者,是之謂盜竽矣。

人無愚智,莫不有趨舍。恬淡平安,莫不知禍福之所由來。得於好惡,怵於淫物,而後變亂。所以然者,引於外物,亂於玩好也。恬淡有趨舍之義,平安知禍福之計。而今也玩好變之,外物引之,引之而往,故曰:「拔。」至聖人不然,一建其趨舍,雖見所好之物不能引,不能引之謂不拔。一於其情,雖有可欲之類,神不為動,神不為動之謂不脫。為人子孫者體此道,以守宗廟不滅之謂祭祀不絕。身以積精為德,家以資財為德,鄉國天下皆以民為德。今治身而外物不能亂其精神,故曰:「脩之身,其德乃真。」真者,慎之固也。治家,無用之物不能動其計則資有餘,故曰:「脩之家,其德有餘。」治鄉者行此節,則家之有餘者益眾,故曰:「脩之鄉,其德乃長。」治邦者行此節,則鄉之有德者益眾,故曰:「脩之邦,其德乃豐。」蒞天下者行此節,則民之生莫不受其澤,故曰:「脩之天下,其德乃普。」脩身者以此別君子小人,治鄉治邦蒞天下者各以此科適觀息耗則萬不失一,故曰:「以身觀身,以家觀家,以鄉觀鄉,以邦觀邦,以天下觀天下,吾奚以知天下之然也?以此。」


\end{pinyinscope}