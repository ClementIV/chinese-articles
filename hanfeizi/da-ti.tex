\article{大體}

\begin{pinyinscope}
古之全大體者:望天地,觀江海,因山谷,日月所照,四時所行,雲布風動;不以智累心,不以私累己;寄治亂於法術,託是非於賞罰,屬輕重於權衡;不逆天理,不傷情性;不吹毛而求小疵,不洗垢而察難知;不引繩之外,不推繩之內;不急法之外,不緩法之內;守成理,因自然;禍福生乎道法而不出乎愛惡,榮辱之責在乎己,而不在乎人。故至安之世,法如朝露,純樸不散;心無結怨,口無煩言。故車馬不疲弊於遠路,旌旗不亂於大澤,萬民不失命於寇戎,雄駿不創壽於旗幢;豪傑不著名於圖書,不錄功於盤盂,記年之牒空虛。故曰:利莫長於簡,福莫久於安。使匠石以千歲之壽操鉤,視規矩,舉繩墨,而正太山;使賁、育帶干將而齊萬民;雖盡力於功,極盛於壽,太山不正,民不能齊。故曰:古之牧天下者,不使匠石極巧以敗太山之體,不使賁、育盡威以傷萬民之性。因道全法,君子樂而大姦止;澹然閒靜,因天命,持大體。故使人無離法之罪,魚無失水之禍。如此,故天下少不可。

上不天則下不遍覆,心不地則物不畢載。太山不立好惡,故能成其高;江海不擇小助,故能成其富。故大人寄形於天地而萬物備,歷心於山海而國家富。上無忿怒之毒,下無伏怨之患,上下交撲,以道為舍。故長利積,大功立,名成於前,德垂於後,治之至也。


\end{pinyinscope}