\article{南面}

\begin{pinyinscope}
人主之過,在己任在臣矣,又必反與其所不任者備之,此其說必與其所任者為讎,而主反制於其所不任者。今所與備人者,且曩之所備也。人主不能明法而以制大臣之威,無道得小人之信矣。人主釋法而以臣備臣,則相愛者比周而相譽,相憎者朋黨而相非,非譽交爭,則主惑亂矣。人臣者,非名譽請謁無以進取,非背法專制無以為威,非假於忠信無以不禁,三者,惛主壞法之資也。人主使人臣雖有智能不得背法而專制,雖有賢行不得踰功而先勞,雖有忠信不得釋法而不禁,此之謂明法。


人主有誘於事者,有壅於言者,二者不可不察也。人臣易言事者,少索資,以事誣主,主誘而不察,因而多之,則是臣反以事制主也,如是者謂之誘,誘於事者困於患。其進言少,其退費多,雖有功其進言不信,不信者有罪,事有功者必賞,則群臣莫敢飾言以惛主。主道者,使人臣前言不復於後,後言不復於前,事雖有功,必伏其罪,謂之任下。人臣為主設事而恐其非也,則先出說設言曰:「議是事者,妒事者也。」人主藏是言不更聽群臣,群臣畏是言不敢議事,二勢者用,則忠臣不聽而譽臣獨任,如是者謂之壅於言,壅於言者制於臣矣。主道者,使人臣必有言之責,又有不言之責。言無端末、辯無所驗者,此言之責也。以不言避責、持重位者,此不言之責也。人主使人臣言者必知其端以責其實,不言者必問其取舍以為之責,則人臣莫敢妄言矣,又不敢默然矣,言默則皆有責也。人主欲為事,不通其端末,而以明其欲,有為之者,其為不得利,必以害反,知此者,任理去欲。舉事有道,計其入多,其出少者,可為也。惑主不然,計其入不計其出,出雖倍其入,不知其害,則是名得而實亡,如是者功小而害大矣。凡功者,其入多、其出少乃可謂功。今大費無罪而少得為功,則人臣出大費而成小功,小功成而主亦有害。


不知治者,必曰:「無變古,毋易常。」變與不變,聖人不聽,正治而已。然則古之無變,常之毋易,在常古之可與不可。伊尹毋變殷,太公毋變周,則湯、武不王矣。管仲毋易齊,郭偃毋更晉,則桓、文不霸矣。凡人難變古者,憚易民之安也。夫不變古者,襲亂之跡;適民心者,恣姦之行也。民愚而不知亂,上懦而不能更,是治之失也。人主者,明能知治,嚴必行之,故雖拂於民心立其治。說在商君之內外而鐵殳,重盾而豫戒也。故郭偃之始治也,文公有官卒;管仲始治也,桓公有武車;戒民之備也。是以愚贛窳墯之民,苦小費而忘大利也,故夤虎受阿謗。𨌑小變而失長便,故鄒賈非載旅。狎習於亂而容於治,故鄭人不能歸。

\end{pinyinscope}