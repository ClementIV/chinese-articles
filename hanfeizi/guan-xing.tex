\article{觀行}

\begin{pinyinscope}
古之人目短於自見,故以鏡觀面;智短於自知,故以道正己。故鏡無見疵之罪,道無明過之怨。目失鏡則無以正鬚眉,身失道則無以知迷惑。西門豹之性急,故佩韋以自緩;董安于之心緩,故佩弦以自急。故以有餘補不足,以長續短之謂明主。

天下有信數三:一曰智有所不能立,二曰力有所不能舉,三曰彊有所不能勝。故雖有堯之智,而無眾人之助,大功不立。有烏獲之勁,而不得人助,不能自舉。有賁、育之彊,而無法術,不得長生。故勢有不可得,事有不可成。故烏獲輕千鈞而重其身,非其身重於千鈞也,勢不便也;離朱易百步而難眉睫,非百步近而眉睫遠也,道不可也。故明主不窮烏獲,以其不能自舉;不困離朱,以其不能自見。因可勢,求易道,故用力寡而功名立。時有滿虛,事有利害,物有生死,人主為三者發喜怒之色,則金石之士離心焉。聖賢之撲淺深矣。故明主觀人,不使人觀己。明於堯不能獨成,烏獲不能自舉,賁、育之不能自勝,以法術則觀行之道畢矣。


\end{pinyinscope}