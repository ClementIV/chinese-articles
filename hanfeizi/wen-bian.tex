\article{問辯}

\begin{pinyinscope}
或問曰:「辯安生乎?」對曰:「生於上之不明也。」問者曰:「上之不明因生辯也何哉?」對曰:「明主之國,令者、言最貴者也,法者、事最適者也。言無二貴,法不兩適,故言行而不軌於法令者必禁。若其無法令而可以接詐應變生利揣事者,上必采其言而責其實,言當則有大利,不當則有重罪,是以愚者畏罪而不敢言,智者無以訟,此所以無辯之故也。亂世則不然,主有令而民以文學非之,官府有法民以私行矯之,人主顧漸其法令,而尊學者之智行,此世之所以多文學也。夫言行者,以功用為之的彀者也。夫砥礪殺矢而以妄發,其端未嘗不中秋毫也,然而不可謂善射者,無常儀的也。設五寸之的,引十步之遠,非羿、逢蒙不能必中者,有常也。故有常則羿、逢蒙以五寸的為巧,無常則以妄發之中秋毫為拙。今聽言觀行,不以功用為之的彀,言雖至察,行雖至堅,則妄發之說也。是以亂世之聽言也,以難知為察,以博文為辯;其觀行也,以離群為賢,以犯上為抗。人主者說辯察之言,尊賢抗之行,故夫作法術之人,立取舍之行,別辭爭之論,而莫為之正。是以儒服帶劍者眾,而耕戰之士寡;堅白無厚之詞章,而憲令之法息。故曰:上不明,則辯生焉。」


\end{pinyinscope}