\article{難四}

\begin{pinyinscope}
衛孫文子聘於魯,公登亦登。叔孫穆子趨進曰:「諸侯之會,寡君未嘗後衛君也。今子不後寡君一等,寡君未知所過也,子其少安。」孫子無辭,亦無悛容。穆子退而告人曰:「孫子必亡。亡臣而不後君,過而不悛,亡之本也。」

或曰:天子失道,諸侯伐之,故有湯、武。諸侯失道,大夫伐之,故有齊、晉。臣而伐君者必亡,則是湯、武不王,晉、齊不立也。孫子君於衛,而後不臣於魯,臣之君也。君有失也,故臣有得也。不命亡於有失之君,而命亡於有得之臣,不察。魯不得誅衛大夫,而衛君之明不知不悛之臣,孫子雖有是二也臣以亡?其所以亡其失所以得君也。

或曰:臣主之施分也。臣能奪君者,以得相踦也。故非其分而取者,眾之所奪也;辭其分而取者,民之所予也。是以桀索崏山之女,紂求比干之心,而天下離;湯身易名,武身受詈,而海內服;趙咺走山,田外僕,而齊、晉從。則湯、武之所以王,齊、晉之所以立,非必以其君也,彼得之而後以君處之也。今未有其所以得,而行其所以處,是倒義而逆德也。倒義,則事之所以敗也,逆德,則怨之所以聚也;敗亡之不察何也!

魯陽虎欲攻三桓,不剋而奔齊,景公禮之。鮑文子諫曰:「不可。陽虎有寵於季氏而欲伐於季孫,貪其富也。今君富於季孫,而齊大於魯,陽虎所以盡詐也。」景公乃囚陽虎。

或曰:千金之家,其子不仁,人之急利甚也。桓公,五伯之上也,爭國而殺其兄,其利大也。臣主之間,非兄弟之親也。劫殺之功,制萬乘而享大利,則群臣孰非陽虎也。事以微巧成,以疏拙敗。群臣之未起難也,其備未具也。群臣皆有陽虎之心,而君上不知,是微而巧也。陽虎貪,於天下,以欲攻上,是疏而拙也。不使景公加誅於拙虎,是鮑文子之說反也。臣之忠詐,在君所行也。君明而嚴則群臣忠,君懦而闇則群臣詐。知微之謂明,無赦之謂嚴。不知齊之巧臣而誅魯之成亂,不亦妄乎!

或曰:仁貪不同心。故公子目夷辭宋,而楚商臣弒父,鄭去疾予弟,而魯桓弒兄,五伯兼并,而以桓律人;則是皆無貞廉也。且君明而嚴則群臣忠,陽虎為亂於魯,不成而走,入齊而不誅,是承為亂也。君明則誅,知陽虎之可以濟亂也,此見微之情也。語曰:「諸侯以國為親。」君嚴則陽虎之罪不可失,此無赦之實也。則誅陽虎,所以使群臣忠也。未知齊之巧臣,而廢明亂之罰;責以未然,而不誅昭昭之罪;此則妄矣。今誅魯之罪亂以威群臣之有姦心者,而可以得季、孟、叔孫之親,鮑文之說,何以為反?

鄭伯將以高渠彌為卿,昭公惡之,固諫不聽。及昭公即位,懼其殺己也,辛卯,弒昭公而立子亶也。君子曰:「昭公知所惡矣。」公子圉曰:「高伯其為戮乎,報惡已甚矣。」

或曰:公子圉之言也不亦反乎!昭公之及於難者,報惡晚也。然則高伯之晚於死者,報惡甚也。明君不懸怒,懸怒則臣罪輕舉以行計,則人主危。故靈臺之飲,衛侯怒而不誅,故褚師作難;食黿之羹,鄭君怒而不誅,故子公殺君。君子之舉知所惡,非甚之也,曰知之若是其明也,而不行誅焉,以及於死,故知所惡,以見其無權也。人君非獨不足於見難而已,或不足於斷制。今昭公見惡稽罪而不誅,使渠彌含憎懼死以徼幸,故不免於殺,是昭公之報惡不甚也。

或曰:報惡甚者,大誅報小罪。大誅報小罪也者,獄之至也。獄之患,故非在所以誅也,以讎之眾也。是以晉厲公滅三郤而欒中行作難,鄭子都殺伯咺而食鼎起禍,吳王誅子胥而越句踐成霸。則衛侯之逐,鄭靈之弒,不以褚師之不死而子公之不誅也,以未可以怒而有怒之色,未可誅而有誅之心。怒其當罪,而誅不逆人心,雖懸奚害?夫未立有罪,即位之後,宿罪而誅,齊胡之所以滅也。君行之臣,猶有後患,況為臣而行之君乎?誅既不當,而以盡為心,是與天下為讎也,則雖為戮,不亦可乎!

衛靈公之時,彌子瑕有寵,於衛國。侏儒有見公者曰:「臣之夢淺矣。」公曰:「奚夢?」「夢見灶者,為見公也。」公怒曰:「吾聞見人主者夢見日,奚為見寡人而夢見灶乎?」侏儒曰:「夫日兼照天下,一物不能當也。人君兼照一國,一人不能壅也,故將見人主而夢日也。夫灶,一人煬焉,則後人無從見矣。或者一人煬君邪?則臣雖夢灶,不亦可乎!」公曰:「善。」遂去雍鉏,退彌子瑕,而用司空狗。

或曰:侏儒善假於夢以見主道矣,然靈公不知侏儒之言也。去雍鉏,退彌子瑕,而用司空狗者,是去所愛而用所賢也。鄭子都賢慶建而壅焉,燕子噲賢子之而壅焉,夫去所愛而用所賢,未免使一人煬己也。不肖者煬主不足以害明,今不加知而使賢者煬己,則必危矣。

或曰屈到嗜芰,文王嗜菖蒲葅,非正味也,而二賢尚之,所味不必美。晉靈侯說參無恤,燕噲賢子之,非正士也,而二君尊之,所賢不必賢也。非賢而賢用之,與愛而用之同。賢誠賢而舉之,與用所愛異狀。故楚莊舉叔孫而霸,商辛用費仲而滅,此皆用所賢而事相反也。燕噲雖舉所賢而同於用所愛,衛奚距然哉?則侏儒之未可見也。君壅而不知其壅也,已見之後而知其壅也,故退壅臣,是加知之也。日「不加知而使賢者煬己則必危」,而今以加知矣,則雖煬己必不危矣。


\end{pinyinscope}