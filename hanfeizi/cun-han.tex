\article{存韓}

\begin{pinyinscope}
韓事秦三十餘年,出則為扞蔽,入則為蓆薦,秦特出銳師取韓地,而隨之怨懸於天下,功歸於強秦。且夫韓入貢職,與郡縣無異也。今臣竊聞貴臣之計,舉兵將伐韓。夫趙氏聚士卒,養從徒,欲贅天下之兵,明秦不弱,則諸侯必滅宗廟,欲西面行其意,非一日之計也。今釋趙之患,而攘內臣之韓,則天下明趙氏之計矣。夫韓、小國也,而以應天下四擊,主辱臣苦,上下相與同憂久矣。修守備,戒強敵,有蓄積、築城池以守固。今伐韓未可一年而滅,拔一城而退,則權輕於天下,天下摧我兵矣。韓叛則魏應之,趙據齊以為原,如此,則以韓、魏資趙假齊以固其從,而以與爭強,趙之福而秦之禍也。夫進而擊趙不能取,退而攻韓弗能拔,則陷銳之卒,懃於野戰,負任之旅,罷於內攻,則合群苦弱以敵而共二萬乘,非所以亡趙之心也。均如貴臣之計,則秦必為天下兵質矣。陛下雖以金石相弊,則兼天下之日未也。

今賤臣之愚計:使人使荊,重弊用事之臣,明趙之所以欺秦者;與魏質以安其心,從韓而伐趙,趙雖與齊為一,不足患也。二國事畢,則韓可以移書定也。是我一舉,二國有亡形,則荊、魏又必自服矣。故曰:「兵者,凶器也,」不可不審用也。以秦與趙敵,衡加以齊,今又背韓,而未有以堅荊、魏之心。夫一戰而不勝,則禍搆矣。計者、所以定事也,不可不察也。韓、秦強弱在今年耳。且趙與諸侯陰謀久矣。夫一動而弱於諸侯,危事也;為計而使諸侯有意我之心,至殆也;見二疏,非所以強於諸侯也。臣竊願陛下之幸熟圖之。夫攻伐而使從者閒焉,不可悔也。

詔以韓客之所上書,書言韓子之未可舉,下臣斯,臣斯甚以為不然。秦之有韓,若人之有腹心之病也,虛處則㤥然,若居濕地,著而不去,以極走則發矣。夫韓雖臣於秦,未嘗不為秦病,今若有卒報之事,韓不可信也。秦與趙為難,荊蘇使齊,未知何如?以臣觀之,則齊、趙之交未必以荊蘇絕也;若不絕,是悉趙而應二萬乘也。夫韓不服秦之義,而服於強也。今專於齊、趙,則韓必為腹心之病而發矣。韓與荊有謀,諸侯應之,則秦必復見崤塞之患。

非之來也,未必不以其能存韓也,為重於韓也。辯說屬辭,飾非詐謀,以釣利於秦,而以韓利闚陛下。夫秦、韓之交親,則非重矣,此自便之計也。

臣視非之言,文其淫說,靡辯才甚。臣恐陛下淫非之辯而聽其盜心,因不詳察事情。今以臣愚議:秦發兵而未名所伐,則韓之用事者,以事秦為計矣。臣斯請往見韓王,使來入見,大王見、因內其身而勿遣,稍召其社稷之臣,以與韓人為市,則韓可深割也。因令象武發東郡之卒,闚兵於境上而未名所之,則齊人懼而從蘇之計,是我兵未出而勁韓以威擒,強齊以義從矣。聞於諸侯也,趙氏破膽,荊人狐疑,必有忠計。荊人不動,魏不足患也,則諸侯可蠶食而盡,趙氏可得與敵矣。願陛下幸察愚臣之計,無忽。

秦遂遣斯使韓也。

李斯往詔韓王,未得見,因上書曰:「昔秦、韓戮力一意以不相侵,天下莫敢犯,如此者數世矣。前時五諸侯嘗相與共伐韓,秦發兵以救之。韓居中國,地不能滿千里,而所以得與諸侯班位於天下、君臣相保者,以世世相教事秦之力也。先時五諸侯共伐秦,韓反與諸侯先為鴈行以嚮秦軍於關下矣。諸侯兵困力極,無奈何,諸侯兵罷。杜倉相秦,起兵發將以報天下之怨而先攻荊,荊令尹患之曰:「夫韓以秦為不義,而與秦兄弟共苦天下。已又背秦,先為鴈行以攻關。韓則居中國,展轉不可知。」天下共割韓上地十城以謝秦,解其兵。夫韓嘗一背秦而國迫地侵,兵弱至今;所以然者,聽姦臣之浮說,不權事實,故雖殺戮姦臣不能使韓復強。

「今趙欲聚兵士卒,以秦為事,使人來借道,言欲伐秦,其勢必先韓而後秦。且臣聞之:『脣亡則齒寒。』夫秦、韓不得無同憂,其形可見。魏欲發兵以攻韓,秦使人將使者於韓。今秦王使臣斯來而不得見,恐左右襲曩姦臣之計,使韓復有亡地之患。臣斯不得見,請歸報,秦、韓之交必絕矣。斯之來使,以奉秦王之歡心,願效便計,豈陛下所以逆賤臣者邪?臣斯願得一見,前進道愚計,退就葅戮,願陛下有意焉。今殺臣於韓,則大王不足以強,若不聽臣之計,則禍必搆矣。秦發兵不留行,而韓之社稷憂矣。臣斯暴身於韓之市,則雖欲察賤臣愚忠之計,不可得已。邊鄙殘,國固守,鼓鐸之聲於耳,而乃用臣斯之計晚矣。且夫韓之兵於天下可知也,今又背強秦。夫棄城而敗軍,則反掖之寇必襲城矣。城盡則聚散,聚散則無軍矣。城固守,則秦必興兵而圍王一都,道不通,則難必謀,其勢不救,左右計之者不用,願陛下熟圖之。若臣斯之所言有不應事實者,願大王幸使得畢辭於前,乃就吏誅不晚也。秦王飲食不甘,遊觀不樂,意專在圖趙,使臣斯來言,願得身見,因急與陛下有計也。今使臣不通,則韓之信未可知也。夫秦必釋趙之患而移兵於韓,願陛下幸復察圖之,而賜臣報決。」


\end{pinyinscope}