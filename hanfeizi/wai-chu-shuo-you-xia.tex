\article{外儲說右下}

\begin{pinyinscope}
一、賞罰共則禁令不行,何以明之,明之以造父、於期。子罕為出彘,田恆為圃池,故宋君、簡公弒。患在王良、造父之共車,田連、成竅之共琴也。

二、治強生於法,弱亂生於阿,君明於此,則正賞罰而非仁下也。爵祿生於功,誅罰生於罪,臣明於此,則盡死力而非忠君也。君通於不仁,臣通於不忠,則可以王矣。昭襄知主情,而不發五苑;田鮪知臣情,故教田章;而公儀辭魚。

三、明主者、鑒於外也,而外事不得不成,故蘇代非齊王。人主鑒於士也,而居者不適不顯,故潘壽言禹情。人主無所覺悟,方吾知之,故恐同衣於族,而況借於權乎?吳章知之,故說以佯,而況借於誠乎?趙王惡虎目而壅;明主之道,如周行人之卻衛侯也。

四、人主者,守法責成以立功者也。聞有吏雖亂而有獨善之民,不聞有亂民而有獨治之吏,故明主治吏不治民。說在搖木之本,與引網之綱。故失火之嗇夫,不可不論也。救火者,吏操壺走火、則一人之用也,操鞭使人、則役萬夫。故所遇術者,如造父之遇驚馬,牽馬推車則不能進,代御執轡持筴則馬咸騖矣。是以說在椎鍛平夷,榜檠矯直。不然,敗在淖齒用齊戮閔王,李兌用趙餓主父也。

五、因事之理則不勞而成,故茲鄭之踞轅而歌以上高梁也。其患在趙簡主稅吏請輕重,薄疑之言國中飽;簡主喜而府庫虛,百姓餓而姦吏富也。故桓公巡民而管仲省腐財怨女。不然,則在延陵乘馬不得進,造父過之而為之泣也。

右經

說一

造父御四馬,馳驟周旋而恣欲於馬。恣欲於馬者,擅轡筴之制也。然馬驚於出彘,而造父不能禁制者,非轡筴之嚴不足也,威分於出彘也。王子於期為駙駕,轡筴不用而擇欲於馬,擅芻水之利也。然馬過於圃池而駙馬敗者,非芻水之利不足也,德分於圃池也。故王良、造父,天下之善御者也,然而使王良操左革而叱吒之,使造父操右革而鞭笞之,馬不能行十里,共故也。田連、成竅,天下善鼓琴者也,然而田連鼓上,成竅擫下,而不能成曲,亦共故也。夫以王良、造父之巧,共轡而御不能使馬,人主安能與其臣共權以為治?以田連、成竅之巧,共琴而不能成曲,人主又安能與其臣共勢以成功乎?

一曰。造父為齊王駙駕,渴馬服成,效駕圃中,渴馬見圃池,去車走池,駕敗。王子於期為趙簡主取道爭千里之表,其始發也,彘伏溝中,王子於期齊轡筴而進之,彘突出於溝中,馬驚駕敗。

司城子罕謂宋君曰:「慶賞賜與,民之所喜也,君自行之。殺戮誅罰,民之所惡也,臣請當之。」宋君曰:「諾。」於是出威令,誅大臣,君曰「問子罕」也。於是大臣畏之,細民歸之。處期年,子罕殺宋君而奪政。故子罕為出彘以奪其君國。

簡公在上位,罰重而誅嚴,厚賦斂而殺戮民。田成恆設慈愛,明寬厚。簡公以齊民為渴馬,不以恩加民,而田成恆以仁厚為圃池也。

一曰。造父為齊王駙駕,以渴服馬,百日而服成,服成請效駕,齊王王曰:「效駕於圃中。」造父驅車入圃,馬見圃池而走,造父不能禁。造父以渴服馬久矣,今馬見池,駻而走,雖造父不能治。今簡公之以法禁其眾久矣,而田成恆利之,是田成恆傾圃池而示渴民也。

一曰。王子於期為宋君為千里之逐。已駕,察手吻文。且發矣,驅而前之,輪中繩引而卻之,馬掩跡。拊而發之,彘逸出於竇中,馬退而卻,筴不能進前也,馬駻而走,轡不能正也。

一曰。司城子罕謂宋君曰:「慶賞賜予者,民之所好也,君自行之。誅罰殺戮者,民之所惡也,臣請當之。」於是戮細民而誅大臣,君曰「與子罕議之」。居期年,民知殺生之命制於子罕也,故一國歸焉。故子罕劫宋君而奪其政,法不能禁也。故曰子罕為出彘,而田成常為圃池也。令王良、造父共車,人操一邊轡而入門閭,駕必敗而道不至也。令田連、成竅共琴,人撫一絃而揮,則音必敗曲不遂矣。

說二

秦昭王有病,百姓里買牛而家為王禱。公孫述出見之,入賀王曰:「百姓乃皆里買牛為王禱。」王使人問之,果有之。王曰:「訾之人二甲。夫非令而擅禱,是愛寡人也。夫愛寡人,寡人亦且改法而心與之相循者,是法不立,法不立,亂亡之道也。不如人罰二甲而復與為治。」

一曰。秦襄王病,百姓為之禱,病愈,殺牛塞禱。郎中閻遏、公孫衍出見之曰:「非社臘之時也,奚自殺牛而祠社?」怪而問之。百姓曰:「人主病,為之禱,今病愈,殺牛塞禱。」閻遏、公孫衍說,見王,拜賀曰:「過堯、舜矣。」王驚曰:「何謂也?」對曰:「堯、舜,其民未至為之禱也,今王病,而民以牛禱,病愈,殺牛塞禱,故臣竊以王為過堯、舜也。」王因使人問之何里為之,訾其里正與伍老屯二甲。閻遏、公孫衍媿不敢言。居數月,王飲酒酣樂,閻遏、公孫衍謂王曰:「前時臣竊以王為過堯、舜,非直敢諛也。堯、舜病,且其民未至為之禱也。今王病而民以牛禱,病愈,殺牛塞禱。今乃訾其里正與伍老屯二甲,臣竊怪之。」王曰:「子何故不知於此。彼民之所以為我用者,非以吾愛之為我用者也,以吾勢之為我用者也。吾釋勢與民相收,若是,吾適不愛,而民因不為我用也,故遂絕愛道也。」

秦大饑,應侯請曰:「五苑之草著、蔬菜、橡果、棗栗,足以活民,請發之。」昭襄王曰:「吾秦法,使民有功而受賞,有罪而受誅。今發五苑之蔬草者,使民有功與無功俱賞也。夫使民有功與無功俱賞者,此亂之道也。夫發五苑而亂,不如棄棗蔬而治。」一曰。「今發五苑之蓏蔬棗栗足以活民,是用民有功與無功爭取也。夫生而亂,不如死而治,大夫其釋之。」

田鮪教其子田章曰:「欲利而身,先利而君;欲富而家,先富而國。」

一曰。田鮪教其子田章曰:「主賣官爵,臣賣智力,故自恃無恃人。」

公儀休相魯而嗜魚,一國盡爭買魚而獻之,公儀子不受,其弟諫曰:「夫子嗜魚而不受者何也?」對曰:「夫唯嗜魚,故不受也。夫即受魚,必有下人之色,有下人之色,將枉於法,枉於法則免於相,雖嗜魚,此不必能自給致我魚,我又不能自給魚。即無受魚而不免於相,雖嗜魚,我能長自給魚。」此明夫恃人不如自恃也,明於人之為己者不如己之自為也。

說三

子之相燕,貴而主斷。蘇代為齊使燕,王問之曰:「齊王亦何如主也?」對曰:「必不霸矣。」燕王曰:「何也?」對曰:「昔桓公之霸也,內事屬鮑叔,外事屬管仲,桓公被髮而御婦人,日遊於市。今齊王不信其大臣。」於是燕王因益大信子之。子之聞之,使人遺蘇代金百鎰,而聽其所使之。

一曰。蘇代為秦使燕,見無益子之,則必不得事而還,貢賜又不出,於是見燕王乃譽齊王。燕王曰:「齊王何若是之賢也!則將必王乎?」蘇代曰:「救亡不暇,安得王哉?」燕王曰:「何也?」曰:「其任所愛不均。」燕王曰:「其亡何也?」曰:「昔者齊桓公愛管仲,置以為仲父,內事理焉,外事斷焉,舉國而歸之,故一匡天下,九合諸侯。今齊任所愛不均,是以知其亡也。」燕王曰:「今吾任子之,天下未之聞也。」於是明日張朝而聽子之。

潘壽謂燕王曰:「王不如以國讓子之。人所以謂堯賢者,以其讓天下於許由,許由必不受也,則是堯有讓許由之名而實不失天下也。今王以國讓子之,子之必不受也,則是王有讓子之之名而與堯同行也。」於是燕王因舉國而屬之,子之大重。

一曰。潘壽,闞者。燕使人聘之。潘壽見燕王曰:「臣恐子之之如益也。」王曰:「何益哉?」對曰:「古者禹死,將傳天下於益,啟之人因相與攻益而立啟。今王信愛子之,將傳國子之,太子之人盡懷印為,子之之人無一人在朝廷者,王不幸棄群臣,則子之亦益也。」王因收吏璽自三百石以上皆效之子之,子之大重。

夫人主之所以鏡照者,諸侯之士徒也,今諸侯之士徒皆私門之黨也。人主之所以自淺娟者,巖穴之士徒也,今巖穴之士徒皆私門之舍人也。是何也?奪褫之資在子之也。故吳章曰:「人主不佯憎愛人,佯愛人不得復憎也,佯憎人不得復愛也。」

一曰。燕王欲傳國於子之也,問之潘壽,對曰:「禹愛益,而任天下於益,已而以啟人為吏。及老,而以啟為不足任天下,故傳天下於益,而勢重盡在啟也。已而啟與友黨攻益而奪之天下,是禹名傳天下於益,而實令啟自取之也。此禹之不及堯、舜明矣。今王欲傳之子之,而吏無非太子之人者也。是名傳之,而實令太子自取之也。」燕王乃收璽自三百石以上皆效之子之,子之遂重。

方吾子曰:「吾聞之古禮,行不與同服者同車,不與同族者共家,而況君人者乃借其權而外其勢乎!」

吳章謂韓宣王曰:「人主不可佯愛人,一日不可復憎;不可以佯憎人,一日不可復愛也。故佯憎佯愛之徵見,則諛者因資而毀譽之,雖有明主不能復收,而況於以誠借人也!」

趙王遊於圃中,左右以菟與虎而輟,盼然環其眼,王曰:「可惡哉,虎目也!」左右曰:「平陽君之目可惡過此。見此未有害也,見平陽君之目如此者則必死矣。」其明日,平陽君聞之,使人殺言者,而王不誅也。

衛君入朝於周,周行人問其號,對曰:「諸侯辟疆。」周行人卻之曰:「諸侯不得與天子同號。」衛君乃自更曰「諸侯燬」而後內之。仲尼聞之曰:「遠哉禁偪,虛名不以借人,況實事乎!」

說四

搖木者一一攝其葉則勞而不遍,左右拊其本而葉遍搖矣。臨淵而搖木,鳥驚而高,魚恐而下。善張網者引其綱,不一一攝萬目而後得則是勞而難,引其綱而魚已囊矣。故吏者,民之本綱者也,故聖人治吏不治民。

救火者,令吏挈壺甕而走火則一人之用也,操鞭箠指麾而趣使人則制萬夫。是以聖人不親細民,明主不躬小事。

造父方耨,得有子父乘車過者,馬驚而不行,其子下車牽馬,父子推車請造父助我推車,造父因收器輟而寄載之,援其子之乘,乃始檢轡持筴,未之用也而馬轡驚矣。使造父而不能御,雖盡力勞身助之推車,馬猶不肯行也。今身使佚,且寄載,有德於人者,有術而御之也。故國者君之車也,勢者君之馬也,無術以御之,身雖勞猶不免亂,有術以御之,身處佚樂之地,又致帝王之功也。

椎鍛者所以平不夷也,榜檠者所以矯不直也,聖人之為法也,所以平不夷矯不直也。

淖齒之用齊也擢閔王之筋,李兌之用趙也餓殺主父。此二君者皆不能用其椎鍛榜檠,故身死為戮而為天下笑。

一曰。入齊則獨聞淖齒而不聞齊王,入趙則獨聞李兌而不聞趙王。故曰:人主者不操術,則威勢輕而臣擅名。

一曰。田嬰相齊,人有說王者曰:「終歲之計,王不一以數日之間自聽之,則無以知吏之姦邪得失也。」王曰:「善。」田嬰聞之,即遽請於王而聽其計,王將聽之矣,田嬰令官具押券斗石參升之計,王自聽計,計不勝聽,罷食,後復坐,不復暮食矣。田嬰復謂曰:「群臣所終歲日夜不敢偷怠之事也,王以一夕聽之,則群臣有為勸勉矣。」王曰:「諾。」俄而王已睡矣,吏盡揄刀削其押券升石之計。王自聽之,亂乃始生。

一曰。武靈王使惠文王蒞政,李兌為相,武靈王不以身躬親殺生之柄,故劫於李兌。

說五

茲鄭子引輦上高梁而不能支。茲鄭踞轅而歌,前者止,後者趨,輦乃上。使茲鄭無術以致人,則身雖絕力至死,輦猶不上也。今身不至勞苦而輦以上者,有術以致人之故也。

趙簡主出稅者,吏請輕重,簡主曰:「勿輕勿重。重則利入於上,若輕則利歸於民,吏無私利而正矣。」薄疑謂趙簡主曰:「君之國中飽。」簡主欣然而喜曰:「何如焉?」對曰:「府庫空虛於上,百姓貧餓於下,然而姦吏富矣。」

齊桓公微服以巡民家,人有年老而自養者,桓公問其故,對曰:「臣有子三人,家貧,無以妻之,傭未反。」桓公歸,以告管仲,管仲曰:「畜積有腐棄之財則人飢餓,宮中有怨女則民無妻。」桓公曰:「善。」乃論宮中有婦人而嫁之,下令於民曰:「丈夫二十而室,婦人十五而嫁。」

一曰。桓公微服而行於民間,有鹿門稷者,行年七十而無妻,桓公問管仲曰:「有民老而無妻者乎?」管仲曰:「有鹿門稷者,行年七十矣而無妻」桓公曰:「何以令之有妻?」管仲曰:「臣聞之,上有積財則民臣必匱乏於下,宮中有怨女則有老而無妻者。」桓公曰:「善。」令於宮中女子未嘗御出嫁之,乃令男子年二十而室,女年十五而嫁。則內無怨女,外無曠夫。

延陵卓子乘蒼龍挑文之乘,鉤飾在前,錯錣在後,馬欲進則鉤飾禁之,欲退則錯錣貫之,馬因旁出。造父過而為之泣涕曰:「古之治人亦然矣。夫賞所以勸之而毀存焉,罰所以禁之而譽加焉,民中立而不知所由,此亦聖人之所為泣也。」

一曰。延陵卓子乘蒼龍與翟文之乘,前則有錯飾,後則有利錣,進則引之,退則筴之,馬前不得進,後不得退,遂避而逸,因下抽刀而刎其腳。造父見之、泣,終日不食,因仰天而歎曰:「筴所以進之也,錯飾在前;引所以退之也,利錣在後。今人主以其清潔也進之,以其不適左右也退之,以其公正也譽之,以其不聽從也廢之,民懼,中立而不知所由,此聖人之所為泣也。」


\end{pinyinscope}