\article{制分}

\begin{pinyinscope}
夫凡國博君尊者,未嘗非法重而可以至乎令行禁止於天下者也。是以君人者分爵制祿,則法必嚴以重之。夫國治則民安,事亂則邦危。法重者得人情,禁輕者失事實。且夫死力者,民之所有者也,情莫不出其死力以致其所欲。而好惡者,上之所制也,民者好利祿而惡刑罰。上掌好惡以御民力,事實不宜失矣,然而禁輕事失者,刑賞失也。其治民不秉法,為善也如是,則是無法也。故治亂之理,宜務分刑賞為急。治國者莫不有法,然而有存有亡,亡者、其制刑賞不分也,治國者、其刑賞莫不有分。有持、以異為分,不可謂分。至於察君之分,獨分也,是以其民重法而畏禁,願毋抵罪而不敢胥賞。故曰:不待刑賞而民從事矣。

是故夫至治之國,善以止姦為務。是何也?其法通乎人情,關乎治理也。然則去微姦之道奈何?其務令之相規其情者也。則使相闚奈何?曰:蓋里相坐而已。禁尚有連於己者,理不得相闚,惟恐不得免。有姦心者不令得忘,闚者多也。如此,則慎己而闚彼。發姦之密,告過者免罪受賞,失姦者必誅連刑。如此,則姦類發矣。姦不容細,私告任坐使然也。

夫治法之至明者,任數不任人。是以有術之國,不用譽則毋適,境內必治,任數也;亡國使兵公行乎其地、而弗能圉禁者,任人而無數也。自攻者人也,攻人者數也。故有術之國,去言而任法。凡畸功之循約者難知,過刑之於言者難見也,是以刑賞惑乎貳。所謂循約難知者,姦功也;臣過之難見者,失根也。循理不見虛功,度情殖乎姦根,則二者安得無兩失也。是以虛士立名於內,而談者為略於外,故愚怯勇慧相連而以虛道屬俗而容乎世,故其法不用,而刑罰不加乎僇人。如此,則刑賞安得不容其二?故實有所至,而理失其量,量之失,非法使然也,法定而任慧也。釋法而任慧者,則受事者安得其務?務不與事相得,則法安得無失、而刑安得無煩?是以賞罰擾亂,邦道差誤,刑賞之不分白也。


\end{pinyinscope}