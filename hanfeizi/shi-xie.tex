\article{飾邪}

\begin{pinyinscope}
鑿龜數筴,兆曰大吉,而以攻燕者趙也。鑿龜數筴,兆曰大吉,而以攻趙者燕也。劇辛之事,燕無功而社稷危。鄒衍之事,燕無功而國道絕。趙代先得意於燕,後得意於齊,國亂節高,自以為與秦提衡,非趙龜神而燕龜欺也。趙又嘗鑿龜數筴而北伐燕,將劫燕以逆秦,兆曰大吉,始攻大梁而秦出上黨矣,兵至釐而六城拔矣,至陽城,秦拔鄴矣,龐援揄兵而南則鄣盡矣。臣故曰:趙龜雖無遠見於燕,且宜近見於秦。秦以其大吉,辟地有實,救燕有有名。趙以其大吉,地削兵辱,主不得意而死。又非秦龜神而趙龜欺也。初時者魏數年東鄉攻盡陶、衛,數年西鄉以失其國,此非豐隆、五行、太一、王相、攝提、六神、五括、天河、殷搶、歲星非數年在西也,又非天缺、弧逆、刑星、熒惑、奎台非數年在東也。故曰:龜筴鬼神不足舉勝,左右背鄉不足以專戰。然而恃之,愚莫大焉。

古者先王盡力於親民,加事於明法。彼法明則忠臣勸,罰必則邪臣止。忠勸邪止而地廣主尊者,秦是也。群臣朋黨比周以隱正道、行私曲而地削主卑者,山東是也。亂弱者亡,人之性也。治強者王,古之道也。越王勾踐恃大朋之龜與吳戰而不勝,身臣入宦於吳,反國棄龜,明法親民以報吳,則夫差為擒。故恃鬼神者慢於法,恃諸侯者危其國。曹恃齊而不聽宋,齊攻荊而宋滅曹。荊恃吳而不聽齊,越伐吳而齊滅荊。許恃荊而不聽魏,荊攻宋而魏滅許。鄭恃魏而不聽韓,魏攻荊而韓滅鄭。今者韓國小而恃大國,主慢而聽秦魏、恃齊荊為用,而小國愈亡。故恃人不足以廣壤,而韓不見也。荊為攻魏而加兵許、鄢,齊攻任扈而削魏,不足以存鄭,而韓弗知也。此皆不明其法禁以治其國,恃外以滅其社稷者也。

臣故曰:明於治之數,則國雖小,富。賞罰敬信,民雖寡,強。賞罰無度,國雖大兵弱者,地非其地,民非其民也。無地無民,堯、舜不能以王,三代不能以強。人主又以過予;人臣又以徒取。舍法律而言先王明君之功者,上任之以國,臣故曰:是願古之功,以古之賞賞今之人也,主以是過予,而臣以此徒取矣。主過予則臣偷幸,臣徒取則功不尊。無功者受賞則財匱而民望,財匱而民望則民不盡力矣。故用賞過者失民,用刑過者民不畏。有賞不足以勸,有刑不足以禁,則國雖大,必危。故曰:小知不可使謀事,小忠不可使主法。荊恭王與晉厲公戰於鄢陵,荊師敗,恭王傷,酣戰而司馬子反渴而求飲,其友豎穀陽奉卮酒而進之,子反曰:「去之,此酒也。」豎穀陽曰:「非也。」子反受而飲之。子反為人嗜酒,甘之,不能絕之於口,醉而臥。恭王欲復戰而謀事,使人召子反,子反辭以心疾,恭王駕而往視之,入幄中聞酒臭而還,曰:「今日之戰,寡人目親傷,所恃者司馬,司馬又如此,是亡荊國之社稷而不恤吾眾也,寡人無與復戰矣。」罷師而去之,斬子反以為大戮。故曰:豎穀陽之進酒也,非以端惡子反也,實心以忠愛之而適足以殺之而已矣。此行小忠而賊大忠者也。故曰:小忠,大忠之賊也。若使小忠主法,則必將赦罪以相愛,是與下安矣,然而妨害於治民者也。

當魏之方明立辟、從憲令行之時,有功者必賞,有罪者必誅,強匡天下,威行四鄰;及法慢,妄予,而國日削矣。當趙之方明國律、從大軍之時,人眾兵強,辟地齊、燕;及國律慢,用者弱,而國日削矣。當燕之方明奉法、審官斷之時,東縣齊國,南盡中山之地;及奉法已亡,官斷不用,左右交爭,論從其下,則兵弱而地削,國制於鄰敵矣。故曰:明法者強,慢法者弱。強弱如是其明矣,而世主弗為,國亡宜矣。語曰:「家有常業,雖饑不餓。國有常法,雖危不亡。」夫舍常法而從私意,則臣下飾於智能,臣下飾於智能則法禁不立矣。是妄意之道行,治國之道廢也。治國之道,去害法者,則不惑於智能、不矯於名譽矣。昔者舜使吏決鴻水,先令有功而舜殺之;禹朝諸侯之君會稽之上,防風之君後至而禹斬之。以此觀之,先令者殺,後令者斬,則古者先貴如令矣。故鏡執清而無事,美惡從而比焉;衡執正而無事,輕重從而載焉。夫搖鏡則不得為明,搖衡則不得為正,法之謂也。故先王以道為常,以法為本,本治者名尊,本亂者名絕。凡智能明通,有以則行,無以則止。故智能單道,不可傳於人。而道法萬全,智能多失。夫懸衡而知平,設規而知圓,萬全之道也。明主使民飾於道之故,故佚而則功。釋規而任巧,釋法而任智,惑亂之道也。亂主使民飾於智,不知道之故,故勞而無功。

釋法禁而聽請謁,群臣賣官於上,取賞於下,是以利在私家而威在群臣。故民無盡力事主之心,而務為交於上。民好上交則貨財上流,而巧說者用。若是,則有功者愈少。姦臣愈進而材臣退,則主惑而不知所行,民聚而不知所道,此廢法禁、後功勞、舉名譽、聽請謁之失也。凡敗法之人,必設詐託物以來親,又好言天下之所希有,此暴君亂主之所以惑也,人臣賢佐之所以侵也。故人臣稱伊尹、管仲之功,則背法飾智有資;稱比干、子胥之忠而見殺,則疾強諫有辭。夫上稱賢明,下稱暴亂,不可以取類,若是者禁。君之立法,以為是也,今人臣多立其私智。以法為非,者是邪以智。過法立智,如是者禁,主之道也。禁主之道,必明於公私之分,明法制,去私恩。夫令必行,禁必止,人主之公義也;必行其私,信於朋友,不可為賞勸,不可為罰沮,人臣之私義也。私義行則亂,公義行則治,故公私有分。人臣有私心,有公義。修身潔白而行公行正,居官無私,人臣之公義也。汙行從欲,安身利家,人臣之私心也。明主在上則人臣去私心行公義,亂主在上則人臣去公義行私心,故君臣異心。君以計畜臣,臣以計事君,君臣之交,計也。害身而利國,臣弗為也;富國而利臣,君不行也。臣之情,害身無利;君之情,害國無親。君臣也者,以計合者也。至夫臨難必死,盡智竭力,為法為之。故先王明賞以勸之,嚴刑以威之。賞刑明則民盡死,民盡死則兵強主尊。刑賞不察則民無功而求得,有罪而幸免,則兵弱主卑。故先王賢佐盡力竭智。故曰:公私不可不明,法禁不可不審,先王知之矣。


\end{pinyinscope}