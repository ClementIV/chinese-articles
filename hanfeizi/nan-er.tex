\article{難二}

\begin{pinyinscope}
景公過晏子曰:「子宮小,近市,請徙子家豫章之圃。」晏子再拜而辭曰:「且嬰家貧,待市食,而朝暮趨之,不可以遠。」景公笑曰:「子家習市,識貴賤乎?」是時景公繁於刑,晏子對曰:「踴貴而屨賤。」景公曰:「何故?」對曰:「刑多也。」景公造然變色曰:「寡人其暴乎!」於是損刑五。

或曰:晏子之貴踴,非其誠也,欲便辭以止多刑也,此不察治之患也。夫刑當無多,不當無少,無以不當聞,而以太多說,無術之患也。敗軍之誅以千百數,猶北不止。即治亂之刑如恐不勝,而姦尚不盡。今晏子不察其當否,而以太多為說,不亦妄乎!夫惜草茅者耗禾穗,惠盜賊者傷良民。今緩刑罰,行寬惠,是利姦邪而害善人也,此非所以為治也。

齊桓公飲酒醉,遺其冠,恥之,三日不朝。管仲曰:「此非有國之恥也,公胡其不雪之以政?」公曰:「胡其善。」因發倉囷,賜貧窮;論囹圄,出薄惱。處三日而民歌之曰:「公胡不復遺冠乎!」

或曰:管仲雪桓公之恥於小人,而生桓公之恥於君子矣。使桓公發倉囷而賜貧窮,論囹圄而出薄惱,非義也,不可以雪恥使之而義也。桓公宿義,須遺冠而後行之,則是桓公行義,非為遺冠也。是雖雪遺冠之恥於小人,而亦遺義之恥於君子矣。且夫發囷倉而賜貧窮者,是賞無功也;論囹圄而出薄惱者,是不誅過也。夫賞無功則民偷幸而望於上,不誅過則民不懲而易為非,此亂之本也,安可以雪恥哉?

昔者文王侵孟、克莒、舉酆,三舉事而紂惡之,文王乃懼,請入洛西之地、赤壤之國、方千里以請解炮烙之刑,天下皆說。仲尼聞之曰:「仁哉文王!輕千里之國而請解炮烙之刑。智哉文王!出千里之地而得天下之心。」

或曰:仲尼以文王為智也,不亦過乎!夫智者知禍難之地而辟之者也,是以身不及於患也。使文王所以見惡於紂者,以其不得人心耶?則雖索人心以解惡可也。紂以其大得人心而惡之,己又輕地以收人心,是重見疑也。固其所以桎梏囚於羑里也。鄭長者有言:「體道,無為、無見也。」此最宜於文王矣,不使人疑之也。仲尼以文王為智,未及此論也。

晉平公問叔向曰:「昔者齊桓公九合諸侯,一匡天下,不識臣之力也?君之力也?」叔向對曰:「管仲善制割,賓胥無善削縫,隰朋善純緣,衣成,君舉而服之,亦臣之力也,君何力之有?」師曠伏琴而笑之。公曰:「太師奚笑也?」師曠對曰:「臣笑叔向之對君也。凡為人臣者,猶炮宰和五味而進之君,君弗食,孰敢強之也。臣請譬之:君者、壤地也,臣者、草木也,必壤地美然後草木碩大,亦君之力也,臣何力之有?」

或曰:叔向、師曠之對皆偏辭也。夫一匡天下,九合諸侯,美之大者也,非專君之力也,又非專臣之力也。昔者宮之奇在虞,僖負眾在曹,二臣之智,言中事,發中功,虞、曹俱亡者何也?此有其臣而無其君者也。且蹇叔處干而干亡,處秦而秦霸,非蹇叔愚於干而智於秦也,此有君與無臣也。向曰「臣之力也」不然矣。昔者桓公宮中二市,婦閭二百,被髮而御婦人,得管仲為五伯長,失管仲得豎刁,而身死,蟲流出尸不葬。以為非臣之力也,且不以管仲為霸;以為君之力也,且不以豎刁為亂。昔者晉文公慕於齊女而亡歸,咎犯極諫,故使反晉國。故桓公以管仲合,文公以舅犯霸,而師曠曰「君之力也」又不然矣。凡五霸所以能成功名於天下者,必君臣俱有力焉。故曰:叔向、師曠之對皆偏辭也。

齊桓公之時,晉客至,有司請禮,桓公曰「告仲父」者三。而優笑曰:「易哉為君,一曰仲父,二曰仲父。」桓公曰:「吾聞君人者勞於索人,佚於使人。吾得仲父已難矣,得仲父之後,何為不易乎哉!」

或曰:桓公之所應優,非君人者之言也。桓公以君人為勞於索人,何索人為勞哉?伊尹自以為宰干湯,百里奚自以為虜干穆公,虜所辱也,宰所羞也,蒙羞辱而接君上,賢者之憂世急也;然則君人者無道賢而已矣,索賢不為人主難。且官職所以任賢也,爵祿所以賞功也,設官職,陳爵祿,而士自至,君人者奚其勞哉!使人又非所佚也,人主雖使人必以度量準之,以刑名參之,以事;遇於法則行,不遇於法則止;功當其言則賞,不當則誅;以刑名收臣,以度量準下;此不可釋也,君人者焉佚哉?索人不勞,使人不佚,而桓公曰「勞於索人,佚於使人」者,不然。且桓公得管仲又不難,管仲不死其君而歸桓公,鮑叔輕官讓能而任之,桓公得管仲又不難明矣。已得管仲之後,奚遽易哉!管仲非周公旦,周公旦假為天子七年,成王壯,授之以政,非為天下計也,為其職也。夫不奪子而行天下者,必不背死君而事其讎,背死君而事其讎者,必不難奪子而行天下,不難奪子而行天下者,必不難奪其君國矣。管仲,公子糾之臣也,謀殺桓公而不能,其君死而臣桓公,管仲之取舍非周公旦未可知也。若使管仲大賢也,且為湯、武,湯、武,桀、紂之臣也,桀、紂作亂,湯、武奪之,今桓公以易居其上,是以桀、紂之行居湯、武之上,桓公危矣。若使管仲不肖人也,且為田常,田常,簡公之臣也,而弒其君,今桓公以易居其上,是以簡公之易居田常之上也,桓公又危矣。管仲非周公旦以明矣,然為湯、武與田常未可知也,為湯、武有桀、紂之危,為田常有簡公之亂也。已得仲父之後,桓公奚遽易哉!若使桓公之任管仲必知不欺己也,是知不欺主之臣也;然雖知不欺主之臣,今桓公以任管仲之專借豎刁、易牙,蟲流出尸而不葬,桓公不知臣欺主與不欺主已明矣,而任臣如彼其專也,故曰:桓公闇主。

李兌治中山,苦陘令上計而入多。李兌曰:「語言辨,聽之說,不度於義,謂之窕言。無山林澤谷之利而入多者,謂之窕貨。君子不聽窕言,不受窕貨,子姑免矣。」

或曰:李子設辭曰:「夫言語辨,聽之說,不度於義者,謂之窕言。」辯、在言者,說、在聽者,言非聽者也。所謂不度於義,非謂聽者必謂所聽也。聽者非小人則君子也,小人無義必不能度之義也,君子度之義必不肯說也。夫曰「言語辨,聽之說,不度於義」者,必不誠之言也。入多之為窕貨也,未可遠行也。李子之姦弗蚤禁,使至於計,是遂過也。無術以知而入多,入多者,穰也,雖倍入將奈何!舉事慎陰陽之和,種樹節四時之適,無早晚之失,寒溫之災,則入多。不以小功妨大務,不以私欲害人事,丈夫盡於耕農,婦人力於織紝,則入多。務於畜養之理,察於土地之宜,六畜遂,五穀殖,則入多。明於權計,審於地形、舟車機械之利,用力少致功大,則入多。利商市關梁之行,能以所有致所無,客商歸之,外貨留之,儉於財用,節於衣食,宮室器械,周於資用,不事玩好,則入多。入多、皆人為也。若天事、風雨時,寒溫適,土地不加大,而有豐年之功,則入多。人事、天功,二物者皆入多,非山林澤谷之利也。夫無山林澤谷之利入多,因謂之窕貨者,無術之言也。

趙簡子圍衛之郛郭,犀楯、犀櫓立於矢石之所不及,鼓之而士不起,簡子投枹曰:「烏乎!吾之士數弊也。」行人燭過免冑而對曰:「臣聞之,亦有君之不能耳,士無弊者。昔者吾先君獻公并國十七,服國三十八,戰十有二勝,是民之用也。獻公沒,惠公即位,淫衍暴亂,身好玉女,秦人恣侵,去絳十七里,亦是人之用也。惠公沒,文公授之,圍衛、取鄴,城濮之戰,五敗荊人,取尊名於天下,亦此人之用也。亦有君不能耳,士無弊也。」簡子乃去楯、櫓立矢石之所及,鼓之而士乘之,戰大勝。簡子曰:「與吾得革車千乘,不如聞行人燭過之一言也。」

或曰:行人未有以說也,乃道惠公以此人是敗,文公以此人是霸,未見所以用人也;簡子未可以速去楯、櫓也。嚴親在圍,輕犯矢石,孝子之所愛親也。孝子愛親,百數之一也。今以為身處危而人尚可戰,是以百族之子於上皆若孝子之愛親也,是行人之誣也。好利惡害,夫人之所有也。賞厚而信,人輕敵矣;刑重而必,失人不北矣。長行徇上,數百不一失。喜利畏罪,人莫不然。將眾者不出乎莫不然之數,而道乎百無一人之行,行人未知用眾之道也。


\end{pinyinscope}