\article{三守}

\begin{pinyinscope}
人主有三守。三守完則國安身榮,三守不完則國危身殆。何謂三守?人臣有議當途之失、用事之過、舉臣之情,人主不心藏而漏之近習能人,使人臣之欲有言者,不敢不下適近習能人之心而乃上以聞人主,然則端言直道之人不得見,而忠直日疏。愛人不獨利也,待譽而後利之;憎人不獨害也,待非而後害之;然則人主無威而重在左右矣。惡自治之勞憚,使群臣輻湊之變,因傳柄移藉,使殺生之機、奪予之要在大臣,如是者侵。此謂三守不完。三守不完則劫殺之徵也。

凡劫有三:有明劫,有事劫,有刑劫。人臣有大臣之尊,外操國要以資群臣,使外內之事非己不得行。雖有賢良,逆者必有禍,而順者必有福。然則群臣直莫敢忠主憂國以爭社稷之利害。人主雖賢不能獨計,而人臣有不敢忠主,則國為亡國矣,此謂國無臣。國無臣者,豈郎中虛而朝臣少哉?群臣持祿養交,行私道而不效公忠。此謂明劫。鬻寵擅權,矯外以勝內,險言禍福得失之形,以阿主之好惡,人主聽之,卑身輕國以資之,事敗與主分其禍,而功成則臣獨專之。諸用事之人,壹心同辭以語其美,則主言惡者必不信矣。此謂事劫。至於守司囹圄,禁制刑罰,人臣擅之,此謂刑劫。三守不完則三劫者起,三守完則三劫者止,三劫止塞則王矣。


\end{pinyinscope}