\article{姦劫弒臣}

\begin{pinyinscope}
凡姦臣皆欲順人主之心以取親幸之勢者也。是以主有所善,臣從而譽之;主有所憎,臣因而毀之。凡人之大體,取舍同者則相是也,取舍異者則相非也。今人臣之所譽者,人主之所是也,此之謂同取。人臣之所毀者,人主之所非也,此之謂同舍。夫取舍合而相與逆者,未嘗聞也,此人臣之所以取信幸之道也。夫姦臣得乘信幸之勢以毀譽進退群臣者,人主非有術數以御之也,非參驗以審之也,必將以曩之合己信今之言,此幸臣之所以得欺主成私者也。故主必欺於上,而臣必重於下矣,此之謂擅主之臣。國有擅主之臣,則群下不得盡其智力以陳其忠,百官之吏不得奉法以致其功矣。何以明之?夫安利者就之,危害者去之,此人之情也。今為臣盡力以致功,竭智以陳忠者,其身困而家貧,父子罹其害;為姦利以弊人主,行財貨以事貴重之臣者,身尊家富,父子被其澤;人焉能去安利之道而就危害之處哉?治國若此其過也,而上欲下之無姦,吏之奉法,其不可得亦明矣。故左右知貞信之不可以得安利也,必曰:「我以忠信事上積功勞而求安,是猶盲而欲知黑白之情,必不幾矣。若以道化行正理不趨富貴事上而求安,是猶聾而欲審清濁之聲也,愈不幾矣。二者不可以得安,我安能無相比周、蔽主上、為姦私以適重人哉?」此必不顧人主之義矣。其百官之吏,亦知方正之不可以得安也,必曰:「我以清廉事上而求安,若無規矩而欲為方圓也,必不幾矣。若以守法不朋黨治官而求安,是猶以足搔頂也,愈不幾也。二者不可以得安,能無廢法行私以適重人哉?」此必不顧君上之法矣。故以私為重人者眾,而以法事君者少矣。是以主孤於上而臣成黨於下,此田成之所以弒簡公者也。

夫有術者之為人臣也,得效度數之言,上明主法,下困姦臣,以尊主安國者也。是以度數之言得效於前,則賞罰必用於後矣。人主誠明於聖人之術,而不苟於世俗之言,循名實而定是非,因參驗而審言辭。是以左右近習之臣,知偽詐之不可以得安也,必曰:「我不去姦私之行盡力竭智以事主,而乃以相與比周妄毀譽以求安,是猶負千鈞之重,陷於不測之淵而求生也,必不幾矣。」百官之吏,亦知為姦利之不可以得安也,必曰:「我不以清廉方正奉法,乃以貪污之心枉法以取私利,是猶上高陵之顛,墮峻谿之下而求生,必不幾矣。」安危之道若此其明也,左右安能以虛言惑主,而百官安敢以貪漁下?是以臣得陳其忠而不弊,下得守其職而不怨。此管仲之所以治齊,而商君之所以強秦也。從是觀之,則聖人之治國也,固有使人不得不愛我之道,而不恃人之以愛為我也。恃人之以愛為我者危矣,恃吾不可不為者安矣。夫君臣非有骨肉之親,正直之道可以得利,則臣盡力以事主;正直之道不可以得安,則臣行私以干上。明主知之,故設利害之道以示天下而已矣。夫是以人主雖不口教百官,不目索姦邪,而國已治矣。人主者,非目若離婁乃為明也,非耳若師曠乃為聰也。目必,不任其數,而待目以為明,所見者少矣,非不弊之術也。耳必,不因其勢,而待耳以為聰,所聞者寡矣,非不欺之道也。明主者,使天下不得不為己視,天下不得不為己聽。故身在深宮之中而明照四海之內,而天下弗能蔽、弗能欺者何也?闇亂之道廢,而聰明之勢興也。故善任勢者國安,不知因其勢者國危。古秦之俗,君臣廢法而服私,是以國亂兵弱而主卑。商君說秦孝公以變法易俗而明公道,賞告姦,困末作而利本事。當此之時,秦民習故俗之有罪可以得免、無功可以得尊顯也,故輕犯新法。於是犯之者其誅重而必,告之者其賞厚而信,故姦莫不得而被刑者眾,民疾怨而眾過日聞。孝公不聽,遂行商君之法,民後知有罪之必誅,而私姦者眾也,故民莫犯,其刑無所加。是以國治而兵強,地廣而主尊。此其所以然者,匿罪之罰重,而告姦之賞厚也。此亦使天下必為己視聽之道也。至治之法術已明矣,而世學者弗知也。

且夫世之愚學,皆不知治亂之情,讘䛟多誦先古之書,以亂當世之治;智慮不足以避阱井之陷,又妄非有術之士。聽其言者危,用其計者亂,此亦愚之至大,而患之至甚者也。俱與有術之士,有談說之名,而實相去千萬也,此夫名同而實有異者也。夫世愚學之人比有術之士也,猶螘垤之比大陵也,其相去遠矣。而聖人者,審於是非之實,察於治亂之情也。故其治國也,正明法,陳嚴刑,將以救群生之亂,去天下之禍,使強不陵弱,眾不暴寡,耆老得遂,幼孤得長,邊境不侵,君臣相親,父子相保,而無死亡係虜之患,此亦功之至厚者也。愚人不知,顧以為暴。愚者固欲治而惡其所以治,皆惡危而喜其所以危者。何以知之?夫嚴刑重罰者,民之所惡也,而國之所以治也;哀憐百姓、輕刑罰者,民之所喜,而國之所以危也。聖人為法國者,必逆於世,而順於道德。知之者,同於義而異於俗;弗知之者,異於義而同於俗。天下知之者少,則義非矣。

處非道之位,被眾口之譖,溺於當世之言,而欲當嚴天子而求安,幾不亦難哉!此夫智士所以至死而不顯於世者也。楚莊王之弟春申君有愛妾曰余,春申君之正妻子曰甲,余欲君之棄其妻也,因自傷其身以視君而泣,曰:「得為君之妾,甚幸。雖然,適夫人非所以事君也,適君非所以事夫人也。身故不肖,力不足以適二主,其勢不俱適,與其死夫人所者,不若賜死君前。妾以賜死,若復幸於左右,願君必察之,無為人笑。」君因信妾余之詐,為棄正妻。余又欲殺甲而以其子為後,因自裂其親身衣之,以示君而泣,曰:「余之得幸君之日久矣,甲非弗知也,今乃欲強戲余,余與爭之,至裂余之衣,而此子之不孝,莫大於此矣。」君怒,而殺甲也。故妻以妾余之詐棄,而子以之死。從是觀之,父之愛子也,猶可以毀而害也。君臣之相與也,非有父子之親也,而群臣之毀言非特一妾之口也,何怪夫賢聖之戮死哉!此商君之所以車裂於秦,而吳起之所以枝解於楚者也。凡人臣者,有罪固不欲誅,無功者皆欲尊顯。而聖人之治國也,賞不加於無功,而誅必行於有罪者也。然則有術數者之為人也,固左右姦臣之所害,非明主弗能聽也。

世之學術者說人主,不曰「乘威嚴之勢以困姦邪之臣」,而皆曰「仁義惠愛而已矣」。世主美仁義之名而不察其實,是以大者國亡身死,小者地削主卑。何以明之?夫施與貧困者,此世之所謂仁義;哀憐百姓不忍誅罰者,此世之所謂惠愛也。夫有施與貧困,則無功者得賞;不忍誅罰,則暴亂者不止。國有無功得賞者,則民不外務當敵斬首,內不急力田疾作,皆欲行貨財、事富貴、為私善、立名譽以取尊官厚俸。故姦私之臣愈眾,而暴亂之徒愈勝,不亡何待?夫嚴刑者,民之所畏也;重罰者,民之所惡也。故聖人陳其所畏以禁其邪,設其所惡以防其姦。是以國安而暴亂不起。吾以是明仁義愛惠之不足用,而嚴刑重罰之可以治國也。無捶策之威,銜橛之備,雖造父不能以服馬。無規矩之法,繩墨之端,雖王爾不能以成方圓。無威嚴之勢,賞罰之法,雖堯、舜不能以為治。今世主皆輕釋重罰、嚴誅,行愛惠,而欲霸王之功,亦不可幾也。故善為主者,明賞設利以勸之,使民以功賞,而不以仁義賜;嚴刑重罰以禁之,使民以罪誅而不以愛惠免。是以無功者不望,而有罪者不幸矣。託於犀車良馬之上,則可以陸犯阪阻之患;乘舟之安,持楫之利,則可以水絕江河之難;操法術之數,行重罰嚴誅,則可以致霸王之功。治國之有法術賞罰,猶若陸行之有犀車良馬也,水行之有輕舟便楫也,乘之者遂得其成。伊尹得之湯以王,管仲得之齊以霸,商君得之秦以強。此三人者,皆明於霸王之術,察於治強之數,而不以牽於世俗之言;適當世明主之意,則有直任布衣之士,立為卿相之處;處位治國,則有尊主廣地之實;此之謂足貴之臣。湯得伊尹,以百里之地立為天子;桓公得管仲,立為五霸主,九合諸侯,一匡天下;孝公得商君,地以廣,兵以強。故有忠臣者,外無敵國之患,內無亂臣之憂,長安於天下,而名垂後世,所謂忠臣也。若夫豫讓為智伯臣也,上不能說人主使之明法術、度數之理,以避禍難之患,下不能領御其眾,以安其國;及襄子之殺智伯也,豫讓乃自黔劓,敗其形容,以為智伯報襄子之仇;是雖有殘刑殺身以為人主之名,而實無益於智伯若秋毫之末。此吾之所下也,而世主以為忠而高之。古有伯夷、叔齊者,武王讓以天下而弗受,二人餓死首陽之陵;若此臣者,不畏重誅,不利重賞,不可以罰禁也,不可以賞使也。此之謂無益之臣也,吾所少而去也,而世主之所多而求也。

諺曰:「厲憐王。」此不恭之言也。雖然,古無虛諺,不可不察也。此謂劫殺死亡之主言也。人主無法術以御其臣,雖長年而美材,大臣猶將得勢擅事主斷,而各為其私急。而恐父兄豪傑之士,借人主之力,以禁誅於己也,故弒賢長而立幼弱,廢正的而立不義。故春秋記之曰:「楚王子圍將聘於鄭,未出境,聞王病而反,因入問病,以其冠纓絞王而殺之,遂自立也。齊崔杼,其妻美,而莊公通之,數如崔氏之室,及公往,崔子之徒賈舉率崔子之徒而攻公,公入室,請與之分國,崔子不許,公請自刃於廟,崔子又不聽,公乃走踰於北牆,賈舉射公,中其股,公墜,崔子之徒以戈斫公而死之,而立其弟景公。」近之所見:李兌之用趙也,餓主父百日而死;卓齒之用齊也,擢湣王之筋,懸之廟梁,宿昔而死。故厲雖癰腫疕瘍,上比於春秋,未至於絞頸射股也;下比於近世,未至餓死擢筋也。故劫殺死亡之君,此其心之憂懼、形之苦痛也,必甚於厲矣。由此觀之,雖「厲憐王」可也。


\end{pinyinscope}