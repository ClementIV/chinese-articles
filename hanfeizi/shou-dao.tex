\article{守道}

\begin{pinyinscope}
聖王之立法也,其賞足以勸善,其威足以勝暴,其備足以必完法。治世之臣,功多者位尊,力極者賞厚,情盡者名立。善之生如春,惡之死如秋,故民勸極力而樂盡情,此之謂上下相得。上下相得,故能使用力者自極於權衡,而務至於任鄙;戰士出死,而願為賁、育;守道者皆懷金石之心,以死子胥之節。用力者為任鄙,戰如賁、育,中為金石,則君人者高枕而守己完矣。

古之善守者,以其所重禁其所輕,以其所難止其所易。故君子與小人俱正,盜跖與曾、史俱廉。何以知之?夫貪盜不赴谿而掇金,赴谿而掇金則身不全;賁、育不量敵則無勇名,盜跖不計可則利不成。明主之守禁也,賁、育見侵於其所不能勝,盜跖見害於其所不能取。故能禁賁、育之所不能犯,守盜跖之所不能取,則暴者守愿,邪者反正。大勇愿,巨盜貞,則天下公平,而齊民之情正矣。

人主離法失人,則危於伯夷不妄取,而不免於田成、盜跖之耳可也。今天下無一伯夷,而姦人不絕世,故立法度量。度量信則伯夷不失是,而盜跖不得非。法分明則賢不得奪不肖,強不得侵弱,眾不得暴寡。託天下於堯之法,則貞士不失分,姦人不徼幸。寄千金於羿之矢,則伯夷不得亡,而盜跖不敢取。堯明於不失姦,故天下無邪;羿巧於不失發,故千金不亡。邪人不壽而盜跖止,如此,故圖不載宰予,不舉六卿;書不著子胥,不明夫差。孫、吳之略廢,盜跖之心伏。人主甘服於玉堂之中,而無瞋目切齒傾取之患。人臣垂拱於金城之內,而無扼捥聚脣嗟唶之禍。服虎而不以柙,禁姦而不以法,塞偽而不以符,此賁、育之所患,堯、舜之所難也。故設柙非所以備鼠也,所以使怯弱能服虎也;立法非所以備曾、史也,所以使庸主能止盜跖也;為符非所以豫尾生也,所以使眾人不相謾也。不獨恃比干之死節,不幸亂臣之無詐也,恃怯之所能服,握庸主之所易守。當今之世,為人主忠計,為天下結德者,利莫長於此。故君人者無亡國之圖,而忠臣無失身之畫。明於尊位必賞,故能使人盡力於權衡,死節於官職。通賁、育之情,不以死易生;惑於盜跖之貪,不以財易身;則守國之道畢備矣。


\end{pinyinscope}