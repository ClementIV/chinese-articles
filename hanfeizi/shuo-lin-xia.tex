\article{說林下}

\begin{pinyinscope}
伯樂教二人相踶馬,相與之簡子廄觀馬。一人舉踶馬,其一人從後而循之,三撫其尻而馬不踶,此自以為失相。其一人曰:「子非失相也。此其為馬也,踒肩而腫膝。夫踶馬也者,舉後而任前,腫膝不可任也,故後不舉。子巧於相踶馬而拙於任腫膝。」夫事有所必歸,而以有所,腫膝而不任,智者之所獨知也。惠子曰:「置猿於柙中,則與豚同。」故勢不便,非所以逞能也。

衛將軍文子見曾子,曾子不起而延於坐席,正身於奧。文子謂其御曰:「曾子,愚人也哉!以我為君子也,君子安可毋敬也?以我為暴人也,暴人安可侮也?曾子不僇命也。」

鳥有翢翢者,重首而屈尾,將欲飲於河則必顛,乃銜其羽而飲之。人之所有飲不足者,不可不索其羽也。

鱣似蛇,蠶似蠋。人見蛇則驚駭,見蠋則毛起。漁者持鱣,婦人拾蠶,利之所在,皆為賁、諸。

伯樂教其所憎者相千里之馬,教其所愛者相駑馬。千里之馬時一,其利緩,駑馬日售,其利急。此《周書》所謂「下言而上用者惑也。」

桓赫曰:「刻削之道,鼻莫如大,目莫如小。鼻大可小,小不可大也。目小可大,大不可小也。」舉事亦然,為其不可復者也,則事寡敗矣。

崇侯、惡來知不適紂之誅也,而不見武王之滅之也。比干、子胥知其君之必亡也,而不知身之死也。故曰:「崇侯、惡來知心而不知事,比干、子胥知事而不知心。」聖人其備矣。

宋太宰貴而主斷。季子將見宋君,梁子聞之曰:「語必可與太宰三坐乎,不然,將不免。」季子因說以貴主而輕國。

楊朱之弟楊布衣素衣而出,天雨,解素衣,衣緇衣而反,其狗不知而吠之。楊布怒,將擊之。楊朱曰:「子毋擊也,子亦猶是。曩者使女狗白而往,黑而來,子豈能毋怪哉!」

惠子曰:「羿執鞅持扞,操弓關機,越人爭為持的。弱子扞弓,慈母入室閉戶。故曰:可必,則越人不疑羿;不可必,則慈母逃弱子。」

桓公問管仲「富有涯乎」?答曰:「水之以涯,其無水者也。富之以涯,其富已足者也。人不能自止於足,而亡其富之涯乎。」

宋之富賈有監止子者,與人爭買百金之璞玉,因佯失而毀之,負其百金,而理其毀瑕,得千溢焉。事有舉之而有敗而賢其毋舉之者,負之時也。

有欲以御見荊王者,眾騶妒之,因曰:「臣能撽鹿。」見王,王為御,不及鹿,自御及之。王善其御也,乃言眾騶妒之。

荊令公子將伐陳,丈人送之曰:「晉強,不可不慎也。」公子曰:「丈人奚憂,吾為丈人破晉。」丈人曰:「可。吾方廬陳南門之外。」公子曰:「是何也?」曰:「我笑句踐也,為人之如是其易也,己獨何為密密十年難乎?」

堯以天下讓許由,許由逃之,舍於家人,家人藏其皮冠。夫棄天下而家人藏其皮冠,是不知許由者也。

三蝨相與訟,一蝨過之,曰:「訟者奚說?」三蝨曰:「爭肥饒之地。」一蝨曰:「若亦不患臘之至而茅之燥耳,若又奚患?」於是乃相與聚嘬其母而食之。彘臞,人乃弗殺。

蟲有就1者,一身兩口,爭食相齕也。遂相殺,因自殺。人臣之爭事而亡其國者,皆蚘類也。1. 就 : 或作「蚘」。原注:「或作蚘」。

宮有堊器,有滌則潔矣。行身亦然,無滌堊之地則寡非矣。

公子糾將為亂,桓公使使者視之,使者報曰:「笑不樂,視不見,必為亂。」乃使魯人殺之。

公孫弘斷髮而為越王騎,公孫喜使人絕之曰:「吾不與子為昆弟矣。」公孫弘曰:「我斷髮,子斷頸而為人用兵,我將謂子何?」周南之戰,公孫喜死焉。

有與悍者鄰,欲賣宅而避之。人曰:「是其貫將滿矣,子姑待之。」答曰:「吾恐其以我滿貫也。」遂去之。故曰:「物之幾者,非所靡也。」

孔子謂弟子曰:「孰能導子西之釣名也?」子貢曰:「賜也能。」乃導之,不復疑也。孔子曰:「寬哉,不被於利;絜哉,民性有恆。曲為曲,直為直。孔子曰子西不免。」白公之難,子西死焉。故曰:「直於行者曲於欲。」

晉中行文子出亡,過於縣邑,從者曰:「此嗇夫,公之故人,公奚不休舍?且待後車。」文子曰:「吾嘗好音,此人遺我鳴琴;吾好珮,此人遺我玉環;是振我過者也。以求容於我者,吾恐其以我求容於人也。」乃去之。果收文子後車二乘而獻之其君矣。

周趮謂宮他曰:「為我謂齊王曰:以齊資我於魏,請以魏事王。」宮他曰:「不可,是示之無魏也,齊王必不資於無魏者,而以怨有魏者。公不如曰:以王之所欲,臣請以魏聽王。齊王必以公為有魏也,必因公。是公有齊也,因以有齊、魏矣。」

白圭謂宋令尹曰:「君長自知政,公無事矣。今君少主也而務名,不如令荊賀君之孝也,則君不奪公位,而大敬重公,則公常用宋矣。」

管仲、鮑叔相謂曰:「君亂甚矣,必失國。齊國之諸公子其可輔者,非公子糾則小白也,與子人事一人焉,先達者相收。」管仲乃從公子糾,鮑叔從小白。國人果弒君,小白先入為君,魯人拘管仲而效之,鮑叔言而相之。故諺曰:「巫咸雖善祝,不能自祓也;秦醫雖善除,不能自彈也。」以管仲之聖而待鮑叔之助,此鄙諺所謂「虜自賣裘而不售,士自譽辯而不信」者也。

荊王伐吳,吳使沮衛蹙融犒於荊師而將軍曰「縛之,殺以釁鼓。」問之曰:「汝來卜乎?」答曰:「卜。」「卜吉乎?」曰:「吉。」荊人曰:「今荊將與女釁鼓其何也?」答曰:「是故其所以吉也。吳使人來也,固視將軍怒。將軍怒,將深溝高壘;將軍不怒,將懈怠。今也將軍殺臣,則吳必警守矣。且國之卜,非為一臣卜。夫殺一臣而存一國,其不言吉何也?且死者無知,則以臣釁鼓無益也;死者有知也,臣將當戰之時,臣使鼓不鳴。」荊人因不殺也。

知伯將伐仇由,而道難不通。乃鑄大鐘遺仇由之君,仇由之君大說,除道將內之。赤章曼枝曰:「不可。此小之所以事大也,而今也大以來,卒必隨之,不可內也。」仇由之君不聽,遂內之。赤章曼枝因斷轂而驅,至於齊七月,而仇由亡矣。

越已勝吳,又索卒於荊而攻晉,左史倚相謂荊王曰:「夫越破吳,豪士死,銳卒盡,大甲傷,今又索卒以攻晉,示我不病也,不如起師與分吳。」荊王曰:「善。」因起師而從越,越王怒,將擊之,大夫種曰:「不可。吾豪士盡,大甲傷,我與戰必不剋,不如賂之。」乃割露山之陰五百里以賂之。

荊伐陳,吳救之,軍閒三十里,雨十日夜,星。左史倚相謂子期曰:「雨十日,甲輯而兵聚,吳人必至,不如備之。」乃為陳,陳未成也而吳人至,見荊陳而反。左史曰:「吳反覆六十里,其君子必休,小人必食,我行三十里擊之,必可敗也。」乃從之,遂破吳軍。

韓、趙相與為難。韓子索兵於魏,曰:「願借師以伐趙。」魏文侯曰:「寡人與趙兄弟,不可以從。」趙又索兵以攻韓,文侯曰:「寡人與韓兄弟,不敢從。」二國不得兵,怒而反。已乃知文侯以搆於己,乃皆朝魏。

齊伐魯,索讒鼎,魯以其鴈往,齊人曰:「鴈也。」魯人曰:「真也。」齊曰:「使樂正子春來,吾將聽子。」魯君請樂正子春,樂正子春曰:「胡不以其真往也?」君曰:「我愛之。」答曰:「臣亦愛臣之信。」

韓咎立為君,未定也。弟在周,周欲重之,而恐韓咎不立也。綦毋恢曰:「不若以車百乘送之。得立,因曰為戒;不立,則曰來效賊也。」

靖郭君將城薛,客多以諫者。靖郭君謂謁者曰:「毋為客通。」齊人有請見者曰:「臣請三言而已,過三言,臣請烹。」靖郭君因見之,客趨進曰:「海大魚。」因反走。靖郭君曰:「請聞其說。」客曰:「臣不敢以死為戲。」靖郭君曰:「願為寡人言之。」答曰:「君聞大魚乎?網不能止,繳不能絓也,蕩而失水,螻蟻得意焉。今夫齊亦君之海也,君長有齊,奚以薛為?君失齊,雖隆薛城至於天猶無益也。」靖郭君曰:「善。」乃輟,不城薛。

荊王弟在秦,秦不出也。中射之士曰:「資臣百金,臣能出之。」因載百金之晉,見叔向,曰:「荊王弟在秦,秦不出也,請以百金委叔向。」叔向受金,而以見之晉平公曰:「可以城壺丘矣。」平公曰:「何也?」對曰:「荊王弟在秦,秦不出也,是秦惡荊也,必不敢禁我城壺丘。若禁之,我曰:為我出荊王之弟,吾不城也。彼如出之,可以德荊。彼不出,是卒惡也,必不敢禁我城壺丘矣。」公曰:「善。」乃城壺丘,謂秦公曰:「為我出荊王之弟,吾不城也。」秦因出之,荊王大說,以鍊金百鎰遺晉。

闔廬攻郢,戰三勝,問子胥曰:「可以退乎?」子胥對曰:「溺人者一飲而止則無逆者,以其不休也,不如乘之以沈之。」

鄭人有一子,將宦,謂其家曰:「必築壞牆,是不善人將竊。」其巷人亦云。不時築,而人果竊之。以其子為智,以巷人告者為盜。


\end{pinyinscope}