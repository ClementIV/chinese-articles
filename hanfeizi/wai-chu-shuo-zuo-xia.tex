\article{外儲說左下}

\begin{pinyinscope}
一、以罪受誅,人不怨上,刖危坐子皋。以功受賞,臣不德君,翟璜操右契而乘軒。襄王不知,故昭卯五乘而履屩。上不過任,臣不誣能,即臣將為失少室周。

二、恃勢而不恃信,故東郭牙議管仲。恃術而不恃信,故渾軒非文公。故有術之主,信賞以盡能,必罰以禁邪,雖有駮行,必得所利,簡主之相陽虎,哀公問一足。

三、失臣主之理,則文王自履而矜。不易朝燕之處,則季孫終身莊而遇賊。

四、利所禁,禁所利,雖神不行;譽所罪,毀所賞,雖堯不治。夫為門而不使入,委利而不使進,亂之所以產也。齊侯不聽左右,魏主不聽譽者,而明察照群臣,則鉅不費金錢,孱不用璧,西門豹請復治鄴足以知之。猶盜嬰兒之矜裘,與刖危子榮衣。子綽左右畫,去蟻驅蠅,安得無桓公之憂索官,與宣王之患臞馬也。

五、臣以卑儉為行,則爵不足以觀賞;寵光無節,則臣下侵偪。說在苗賁皇非獻伯,孔子議晏嬰,故仲尼論管仲與叔孫敖。而出入之容變,陽虎之言見其臣也。而簡主之應人臣也失主術。朋黨相和,臣下得欲,則人主孤;群臣公舉,下不相和,則人主明。陽虎將為趙武之賢、解狐之公。而簡主以為枳棘,非所以教國也。

六、公室卑則忌直言,私行勝則少公功。說在文子之直言,武子之用杖;子產忠諫,子國譙怒;梁車用法,而成侯收璽;管仲以公,而國人謗怨。

右經

說一

孔子相衛,弟子子皋為獄吏,刖人足,所刖者守門,人有惡孔子於衛君者曰:「尼欲作亂。」衛君欲執孔子,孔子走,弟子皆逃,子皋從出門,刖危引之而逃之門下室中,吏追不得,夜半,子皋問刖危曰:「吾不能虧主之法令而親刖子之足,是子報仇之時也,而子何故乃肯逃我?我何以得此於子?」刖危曰:「吾斷足也,固吾罪當之,不可奈何。然方公之獄治臣也,公傾側法令,先後臣以言,欲臣之免也甚,而臣知之。及獄決罪定,公憱然不悅,形於顏色,臣見又知之。非私臣而然也,夫天性仁心固然也,此臣之所以悅而德公也。」

田子方從齊之魏,望翟黃乘軒騎駕出,方以為文侯也,移車異路而避之,則徒翟黃也,方問曰:「子奚乘是車也?」曰:「君謀欲伐中山,臣薦翟角而謀得果。且伐之,臣薦樂羊而中山拔。得中山,憂欲治之,臣薦李克而中山治。是以君賜此車。」方曰:「寵之稱功尚薄」。

秦、韓攻魏,昭卯西說而秦、韓罷。齊、荊攻魏,卯東說而齊、荊罷。魏襄王養之以五乘將軍,卯曰:「伯夷以將軍葬於首陽山之下,而天下曰:夫以伯夷之賢與其稱仁,而以將軍葬,是手足不掩也。今臣罷四國之兵,而王乃與臣五乘,此其稱功,猶贏勝而履蹻。」

孔子曰:「善為吏者樹德,不能為吏者樹怨。概者、平量者也,吏者、平法者也,治國者、不可失平也。」

少室周者,古之貞廉潔愨者也,為趙襄主力士,與中牟徐子角力,不若也,入言之襄主以自代也,襄主曰:「子之處,人之所欲也,何為言徐子以自代?」曰:「臣以力事君者也,今徐子力多臣,臣不以自代,恐他人言之而為罪也。」

一曰。少室周為襄主驂乘,至晉陽,有力士牛子耕與角力而不勝,周言於主曰:「主之所以使臣騎乘者,以臣多力也,今有多力於臣者,願進之。」

說二

齊桓公將立管仲,令群臣曰:「寡人將立管仲為仲父,善者入門而左,不善者入門而右。」東郭牙中門而立,公曰:「寡人立管仲為仲父,令曰善者左,不善者右,今子何為中門而立?」牙曰:「以管仲之智為能謀天下乎?」公曰:「能」。「以斷為敢行大事乎?」公曰:「敢」。牙曰:「君知能謀天下,斷敢行大事,君因專屬之國柄焉。以管仲之能,乘公之勢以治齊國,得無危乎?」公曰:「善」。乃令隰朋治內,管仲治外以相參。

晉文公出亡,箕鄭挈壺餐而從,迷而失道,與公相失,飢而道泣,寢餓而不敢食。及文公反國,舉兵攻原,克而拔之,文公曰:「夫輕忍飢餒之患而必全壺餐,是將不以原叛」。乃舉以為原令。大夫渾軒聞而非之曰:「以不動壺餐之故,怙其不以原叛也,不亦無術乎!故明主者,不恃其不我叛也,恃吾不可叛也;不恃其不我欺也,恃吾不可欺也。」

陽虎議曰:「主賢明則悉心以事之,不肖則飾姦而試之。」逐於魯,疑於齊,走而之趙,趙簡主迎而相之,左右曰:「虎善竊人國政,何故相也?」簡主曰:「陽虎務取之,我務守之。」遂執術而御之,陽虎不敢為非,以善事簡主,興主之強,幾至於霸也。

魯哀公問於孔子曰:「吾聞古者有夔一足,其果信有一足乎?孔子對曰:「不也,夔非一足也。夔者忿戾惡心,人多不說喜也。雖然,其所以得免於人害者,以其信也,人皆曰獨此一足矣,夔非一足也,一而足也。」哀公曰:「審而是固足矣。」

一曰。哀公問於孔子曰:「吾聞夔一足,信乎?」曰:「夔,人也,何故一足?彼其無他異,而獨通於聲,堯曰:『夔一而足矣。』使為樂正。故君子曰:『夔有一足,非一足也。』」

說三

文王伐崇,至鳳黃虛,襪繫解,因自結,太公望曰:「何為也?」王曰:「君與處皆其師,中皆其友,下盡其使也。今皆先君之臣,故無可使也。」

一曰。晉文公與楚戰,至黃鳳之陵,履係解,因自結之,左右曰:「不可以使人乎?」公曰:「吾聞上君所與居,皆其所畏也;中君之所與居,皆其所愛也;下君之所與居,皆其所侮也。寡人雖不肖,先君之人皆在,是以難之也。」

季孫好士,終身莊,居處衣服,常如朝廷,而季孫適懈,有過失,而不能長為也,故客以為厭易己,相與怨之,遂殺季孫。故君子去泰去甚。

南宮敬子問顏涿聚曰:「季孫養孔子之徒,所朝服與坐者以十數而遇賊,何也?」曰:「昔周成王近優侏儒以逞其意,而與君子斷事,是能成其欲於天下。今季孫養孔子之徒,所朝服而與坐者以十數,而與優侏儒斷事,是以遇賊。故曰:不在所與居,在所與謀也。」

孔子御坐於魯哀公,哀公賜之桃與黍,哀公:「請用。」仲尼先飯黍而後啗桃,左右皆揜口而笑,哀公曰:「黍者,非飯之也,以雪桃也。」仲尼對曰:「丘知之矣。夫黍者五穀之長也,祭先王為上盛。果蓏有六,而桃為下,祭先王不得入廟。丘之聞也,君子以賤雪貴,不聞以貴雪賤。今以五穀之長雪果蓏之下,是從上雪下也,丘以為妨義,故不敢以先於宗廟之盛也。」

趙簡子謂左右曰:「車席泰美。夫冠雖賤,頭必戴之;屨雖貴,足必履之。今車席如此,大美,吾將何屩以履之?夫美下而耗上,妨義之本也。」

費仲說紂曰:「西伯昌賢,百姓悅之,諸侯附焉,不可不誅,不誅必為殷患。」紂曰:「子言,義主,何可誅?」費仲曰:「冠雖穿弊,必戴於頭;履雖五采,必踐之於地。今西伯昌,人臣也,修義而人向之,卒為天下患,其必昌乎!人人不以其賢為其主,非可不誅也。且主而誅臣,焉有過?」紂曰:「夫仁義者,上所以勸下也。今昌好仁義,誅之不可。」三說不用,故亡。

齊宣王問匡倩曰:「儒者博乎?」曰:「不也。」王曰:「何也?」匡倩對曰:「博者貴梟,勝者必殺梟,殺梟者,是殺所貴也,儒者以為害義,故不博也。」又問曰:「儒者弋乎?」曰:「不也。弋者從下害於上者也,是從下傷君也,儒者以為害義,故不弋。」又問儒者鼓瑟乎?曰:「不也。夫瑟以小絃為大聲,以大絃為小聲,是大小易序,貴賤易位,儒者以為害義,故不鼓也。」宣王曰:「善。」仲尼曰:「與其使民諂下也,寧使民諂上。」

說四

鉅者,齊之居士。孱者,魏之居士。齊、魏之君不明,不能親照境內,而聽左右之言,故二子費金璧而求入仕也。

西門豹為鄴令,清剋潔愨,秋毫之端無私利也,而甚簡左右,左右因相與比周而惡之,居期年,上計,君收其璽,豹自請曰:「臣昔者不知所以治鄴,今臣得矣,願請璽復以治鄴,不當,請伏斧鑕之罪。」文侯不忍而復與之,豹因重斂百姓,急事左右,期年,上計,文侯迎而拜之,豹對曰:「往年臣為君治鄴,而君奪臣璽,今臣為左右治鄴,而君拜臣,臣不能治矣。」遂納璽而去,文侯不受,曰:「寡人曩不知子,今知矣,願子勉為寡人治之。」遂不受。

齊有狗盜之子與刖危子戲而相誇,盜子曰:「吾父之裘獨有尾。」危子曰:「吾父獨冬不失褲。」

子綽曰:「人莫能左畫方而右畫圓也。以肉去蟻蟻愈多,以魚驅蠅蠅愈至。」

桓公謂管仲曰:「官少而索者眾,寡人憂之。」管仲曰:「君無聽左右之謂請,因能而受祿,錄功而與官,則莫敢索官,君何患焉?」

韓宣子曰:「吾馬菽粟多矣,甚臞,何也?寡人患之。」周市對曰:「使騶盡粟以食,雖無肥,不可得也。名為多與之,其實少,雖無臞,亦不可得也。主不審其情實,坐而患之,馬猶不肥也。」

桓公問置吏於管仲,管仲曰:「辯察於辭,清潔於貨,習人情,夷吾不如弦商,請立以為大理。登降肅讓,以明禮待賓,臣不如隰朋,請立以為大行。墾草仞邑,辟地生粟,臣不如甯武,請以為大田。三軍既成陳,使士視死如歸,臣不如公子成父,請以為大司馬。犯顏極諫,臣不如東郭牙,請立以為諫臣。治齊此五子足矣,將欲霸王,夷吾在此。」

說五

孟獻伯相魯,堂下生藿藜,門外長荊棘,食不二味,坐不重席,晉無衣帛之妾,居不粟馬,出不從車,叔向聞之,以告苗賁皇,賁皇非之曰:「是出主之爵祿以附下也。」

一曰。孟獻伯拜上卿,叔向往賀,門有御,馬不食禾,向曰:「子無二馬二輿何也?」獻伯曰:「吾觀國人尚有飢色,是以不秣馬。班白者多以徒行,故不二輿。」向曰:「吾始賀子之拜卿,今賀子之儉也。」向出,語苗賁皇曰:「助吾賀獻伯之儉也。」苗子曰:「何賀焉!夫爵祿旂章,所以異功伐別賢不肖也。故晉國之法,上大夫二輿二乘,中大夫二輿一乘,下大夫專乘,此明等級也。且夫卿必有軍事,是故循車馬,比卒乘,以備戎事。有難則以備不虞,平夷則以給朝事。今亂晉國之政,乏不虞之備,以成節,以絜私名,獻伯之儉也可與?又何賀!」

管仲相齊,曰:「臣貴矣,然而臣貧。」桓公曰:「使子有三歸之家。」曰:「臣富矣,然而臣卑。」桓公使立於高、國之上。曰:「臣尊矣,然而臣疏。」乃立為仲父。孔子聞而非之曰:「泰侈偪上。」

一曰。管仲父,出、朱蓋青衣,置鼓而歸,庭有陳鼎,家有三歸,孔子曰:「良大夫也,其侈偪上。」

孫叔敖相楚,棧車牝馬,糲餅菜羹,枯魚之膳,冬羔裘,夏葛衣,面有飢色,則良大夫也,其儉偪下。

陽虎去齊走趙,簡主問曰:「吾聞子善樹人。」虎曰:「臣居魯,樹三人,皆為令尹,及虎抵罪於魯,皆搜索於虎也。臣居齊,薦三人,一人得近王,一人為縣令,一人為候吏,及臣得罪,近王者不見臣,縣令者迎臣執縛,候吏者追臣至境上,不及而止。虎不善樹人。」主俛而笑曰:「夫樹橘柚者,食之則甘,嗅之則香;樹枳棘者,成而刺人;故君子慎所樹。」

中牟無令,晉平公問趙武曰:「中牟,三國之股肱,邯鄲之肩髀,寡人欲得其良令也,誰使而可?」武曰:「邢伯子可。」公曰:「非子之讎也?」曰:「私讎不入公門。」公又問曰:「中府之令誰使而可?」曰:「臣子可。」故曰:「外舉不避讎,內舉不避子。」趙武所薦四十六人,及武死,各就賓位,其無私德若此也。

平公問叔向曰:「群臣孰賢?」曰:「趙武。」公曰:「子黨於師人。」曰:「武立如不勝衣,言如不出口,然所舉士也數十人,皆得其意,而公家甚賴之,及武子之生也不利於家,死不託於孤,臣敢以為賢也。」

解狐薦其讎於簡主以為相,其讎以為且幸釋己也,乃因往拜謝,狐乃引弓送而射之,曰:「夫薦汝公也,以汝能當之也。夫讎汝,吾私怨也,不以私怨汝之故擁汝於吾君。故私怨不入公門。」

一曰。解狐舉邢伯柳為上黨守,柳往謝之曰:「子釋罪,敢不再拜。」曰:「舉子公也,怨子私也,子往矣,怨子如初也。」

鄭縣人賣豚,人問其價,曰:「道日暮安暇語汝。」

說六

范文子喜直言,武子擊之以杖:「夫直議者不為人所容,無所容則危身,非徒危身,又將危父。」

子產者,子國之子也。子產忠於鄭君,子國譙怒之曰:「夫介異於人臣,而獨忠於主,主賢明,能聽汝,不明,將不汝聽,聽與不聽,未可必知,而汝已離於群臣,離於群臣則必危汝身矣,非徒危己也,又且危父矣。」

梁車新為鄴令,其姊往看之,暮而後門閉,因踰郭而入,車遂刖其足,趙成侯以為不慈,奪之璽而免之令。

管仲束縛,自魯之齊,道而飢渴,過綺烏封人而乞食,烏封人跪而食之,甚敬,封人因竊謂仲曰:「適幸及齊不死而用齊,將何報我?」曰:「如子之言,我且賢之用,能之使,勞之論,我何以報子?」封人怨之。


\end{pinyinscope}