\article{初見秦}

\begin{pinyinscope}
臣聞不知而言不智,知而不言不忠,為人臣不忠當死,言而不當亦當死。雖然,臣願悉言所聞,唯大王裁其罪。

臣聞天下陰燕陽魏,連荊固齊,收韓而成從,將西面以與秦強為難,臣竊笑之。世有三亡,而天下得之,其此之謂乎!臣聞之曰:「以亂攻治者亡,以邪攻正者亡,以逆攻順者亡。」今天下之府庫不盈,囷倉空虛,悉其士民,張軍數十百萬。其頓首戴羽為將軍,斷死於前,不至千人,皆以言死。白刃在前,斧鑕在後,而卻走不能死也。非其士民不能死也,上不能故也。言賞則不與,言罰則不行,賞罰不信,故士民不死也。

今秦出號令而行賞罰,有功無功相事也。出其父母懷衽之中,生未嘗見寇耳。聞戰,頓足徒裼,犯白刃,蹈鑪炭,斷死於前者皆是也。夫斷死與斷生者不同,而民為之者,是貴奮死也。夫一人奮死可以對十,十可以對百,百可以對千,千可以對萬,萬可以剋天下矣。今秦地折長補短,方數千里,名師數十百萬。秦之號令賞罰、地形利害,天下莫若也。以此與天下,天下不足兼而有也。是故秦戰未嘗不剋,攻未嘗不取,所當未嘗不破,開地數千里,此其大功也。然而兵甲頓,士民病,蓄積索,田疇荒,囷倉虛,四鄰諸侯不服,霸王之名不成,此無異故,其謀臣皆不盡其忠也。

臣敢言之,往者齊南破荊,東破宋,西服秦,北破燕,中使韓、魏,土地廣而兵強,戰剋攻取,詔令天下。齊之清濟濁河,足以為限;長城巨防,足以為塞。齊五戰之國也,一戰不剋而無齊。由此觀之,夫戰者,萬乘之存亡也。且聞之曰:「削跡無遺根,無與禍鄰,禍乃不存。」秦與荊人戰,大破荊,襲郢,取洞庭、五湖、江南,荊王君臣亡走,東服於陳。當此時也,隨荊以兵則荊可舉,荊可舉,則民足貪也,地足利也。東以弱齊、燕,中以凌三晉。然則是一舉而霸王之名可成也,四鄰諸侯可朝也。而謀臣不為,引軍而退,復與荊人為和,令荊人得收亡國,聚散民,立社稷,主置宗廟,令率天下西面以與秦為難,此固以失霸王之道一矣。天下又比周而軍華下,大王以詔破之,兵至梁郭下,圍梁數旬則梁可拔,拔梁則魏可舉,舉魏則荊、趙之意絕,荊、趙之意絕則趙危,趙危而荊狐疑,東以弱齊、燕,中以凌三晉。然則是一舉而霸王之名可成也,四鄰諸侯可朝也。而謀臣不為,引軍而退,復與魏氏為和,令魏氏反收亡國,聚散民,立社稷,主置宗廟,令,此固以失霸王之道二矣。前者穰侯之治秦也,用一國之兵而欲以成兩國之功。是故兵終身暴露於外,士民疲病於內,霸王之名不成,此固以失霸王之道三矣。

趙氏,中央之國也,雜民所居也。其民輕而難用也。號令不治,賞罰不信,地形不便,下不能盡其民力。彼固亡國之形也,而不憂民萌。悉其士民,軍於長平之下,以爭韓上黨。大王以詔破之,拔武安。當是時也,趙氏上下不相親也,貴賤不相信也。然則邯鄲不守。拔邯鄲,筦山東河間,引軍而去,西攻脩武,踰華,絳上黨。代四十六縣,上黨七十縣,不用一領甲,不苦一士民,此皆秦有也。以代、上黨不戰而畢為秦矣,東陽、河外不戰而畢反為齊矣,中山、呼沱以北不戰而畢為燕矣。然則是趙舉,趙舉則韓亡,韓亡則荊、魏不能獨立,荊、魏不能獨立則是一舉而壞韓、蠹魏、拔荊,東以弱齊、燕,決白馬之口以沃魏氏,是一舉而三晉亡,從者敗也。大王垂拱以須之,天下編隨而服矣,霸王之名可成。而謀臣不為,引軍而退,復與趙氏為和。夫以大王之明,秦兵之強,棄霸王之業,地曾不可得,乃取欺於亡國,是謀臣之拙也。且夫趙當亡而不亡,秦當霸而不霸,天下固以量秦之謀臣一矣。乃復悉士卒以攻邯鄲,不能拔也,棄甲負弩,戰竦而卻,天下固已量秦力二矣。軍乃引而復,并於孚下,大王又并軍而至,與戰不能剋之也,又不能反運,罷而去,天下固量秦力三矣。內者量吾謀臣,外者極吾兵力。由是觀之,臣以為天下之從,幾不難矣。內者,吾甲兵頓,士民病,蓄積索,田疇荒,囷倉虛;外者、天下皆比意甚固。願大王有以慮之也。

且臣聞之曰:「戰戰栗栗,日慎一日,苟慎其道,天下可有。」何以知其然也?昔者紂為天子,將率天下甲兵百萬,左飲於淇溪,右飲於洹谿,淇水竭而洹水不流,以與周武王為難。武王將素甲三千,戰一日,而破紂之國,禽其身,據其地而有其民,天下莫傷。知伯率三國之眾以攻趙襄主於晉陽,決水而灌之三月,城且拔矣;襄主鑽龜筮占兆,以視利害,何國可降。乃使其臣張孟談於是乃潛於行而出,知伯之約,得兩國之眾以攻知伯,禽其身以復襄主之初。今秦地折長補短,方數千里,名師數十百萬,秦國之號令賞罰,地形利害,天下莫如也,以此與天下,天下可兼而有也。臣昧死願望見大王言所以破天下之從,舉趙、亡韓,臣荊、魏,親齊、燕,以成霸王之名,朝四鄰諸侯之道。大王誠聽其說,一舉而天下之從不破,趙不舉,韓不亡,荊、魏不臣,齊、燕不親,霸王之名不成,四鄰諸侯不朝,大王斬臣以徇國,以為王謀不忠者也。


\end{pinyinscope}