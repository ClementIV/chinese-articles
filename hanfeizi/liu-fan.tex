\article{六反}

\begin{pinyinscope}
畏死難,降北之民也,而世尊之曰貴生之士;學道立方,離法之民也,而世尊之曰文學之士;遊居厚養,牟食之民也,而世尊之曰有能之士;語曲牟知,偽詐之民也,而世尊之曰辯智之士;行劍攻殺,暴憿之民也,而世尊之曰磏勇之士;活湧匿姦,當死之民也,而世尊之曰任譽之士;此六民者,世之所譽也。赴險殉誠,死節之民,而世少之曰失計之民也;寡聞從令,全法之民也,而世少之曰樸陋之民也;力作而食,生利之民也,而世少之曰寡能之民也;嘉厚純粹,整穀之民也,而世少之曰愚戇之民也;重命畏事,尊上之民也,而世少之曰怯懾之民也;挫賊遏姦,明上之民也,而世少之曰諂讒之民也;此六民者,世之所毀也。姦偽無益之民六,而世譽之如彼;耕戰有益之民六,而世毀之如此;此之謂六反。布衣循私利而譽之,世主聽虛聲而禮之,禮之所在,利必加焉。百姓循私害而訾之,世主壅於俗而賤之,賤之所在,害必加焉。故名賞在乎私惡當罪之民,而毀害在乎公善宜賞之士,索國之富強,不可得也。

古者有諺曰:「為政、猶沐也,雖有棄髮、必為之。」愛棄髮之費,而忘長髮之利,不知權者也。夫彈痤者痛,飲藥者苦,為苦憊之故,不彈痤、飲藥,則身不活、病不已矣。

今上下之接,無子父之澤,而欲以行義禁下,則交必有郤矣。且父母之於子也,產男則相賀,產女則殺之。此俱出父母之懷衽,然男子受賀,女子殺之者,慮其後便、計之長利也。故父母之於子也,猶用計算之心以相待也,而況無父子之澤乎!

今學者之說人主也,皆去求利之心,出相愛之道,是求人主之過父母之親也,此不熟於論恩詐而誣也,故明主不受也。聖人之治也,審於法禁,法禁明著則官法;必於賞罰,賞罰不阿則民用。官官治則國富,國富則兵強,而霸王之業成矣。霸王者,人主之大利也。人主挾大利以聽治,故其任官者當能,其賞罰無私。使士民明焉盡力致死、則功伐可立而爵祿可致,爵祿致而富貴之業成矣。富貴者,人臣之大利也。人臣挾大利以從事,故其行危至死,其力盡而不望。此謂君不仁,臣不忠,則不可以霸王矣。

夫姦必知則備,必誅則止;不知則肆,不誅則行。夫陳輕貨於幽隱,雖曾、史可疑也;懸百金於市,雖大盜不取也。不知則曾、史可疑於幽隱,必知則大盜不取懸金於市。故明主之治國也眾其守、而重其罪,使民以法禁而不以廉止。母之愛子也倍父,父令之行於子者十母;吏之於民無愛,令之行於民也萬父。母積愛而令窮,吏用威嚴而民聽從,嚴愛之筴亦可決矣。且父母之所以求於子也,動作則欲其安利也,行身則欲其遠罪也;君上之於民也,有難則用其死,安平則盡其力。親以厚愛關子於安利而不聽,君以無愛利求民之死力而令行。明主知之,故不養恩愛之心而增威嚴之勢。故母厚愛處,子多敗,推愛也;父薄愛教笞,子多善,用嚴也。

今家人之治產也,相忍以飢寒,相強以勞苦,雖犯軍旅之難,饑饉之患,溫衣美食者,必是家也;相憐以衣食,相惠以佚樂,天饑歲荒,嫁妻賣子者,必是家也。故法之為道,前苦而長利;仁之為道,偷樂而後窮。聖人權其輕重,出其大利,故用法之相忍,而棄仁人之相憐也。學者之言,皆曰輕刑,此亂亡之術也。凡賞罰之必者,勸禁也。賞厚、則所欲之得也疾,罰重、則所惠之禁也急。夫欲利者必惡害,害者,利之反也,反於所欲,焉得無惡。欲治者必惡亂,亂者,治之反也。是故欲治甚者,其賞必厚矣;其惡亂甚者,其罰必重矣。今取於輕刑者,其惡亂不甚也,其欲治又不甚也,此非特無術也,又乃無行。是故決賢不肖愚知之美,在賞罰之輕重。且夫重刑者,非為罪人也。明主之法,揆也。治賊,非治所揆也;治所揆也者,是治死人也。刑盜,非治所刑也;治所刑也者,是治胥靡也。故曰重一姦之罪而止境內之邪,此所以為治也。重罰者,盜賊也;而悼懼者,良民也;欲治者奚疑於重刑!若夫厚賞者,非獨賞功也,又勸一國。受賞者甘利,未賞者慕業,是報一人之功而勸境內之眾也,欲治者何疑於厚賞!今不知治者,皆曰重刑傷民,輕刑可以止姦,何必於重哉?此不察於治者也。夫以重止者,未必以輕止也;以輕止者,必以重止矣。是以上設重刑者而姦盡止,姦盡止則此奚傷於民也?所謂重刑者,姦之所利者細,而上之所加焉者大也;民不以小利蒙大罪,故姦必止者也。所謂輕刑者,姦之所利者大,上之所加焉者小也;民慕其利而傲其罪,故姦不止也。故先聖有諺曰:「不躓於山,而躓於垤。」山者大、故人順之,垤微小、故人易之也。今輕刑罰,民必易之。犯而不誅,是驅國而棄之也;犯而誅之,是為民設陷也。是故輕罪者,民之垤也。是以輕罪之為民道也,非亂國也則設民陷也,此則可謂傷民矣!

今學者皆道書筴之頌語,不察當世之實事,曰:「上不愛民,賦斂常重,則用不足而下恐上,故天下大亂。」此以為足其財用以加愛焉,雖輕刑罰可以治也。此言不然矣。凡人之取重賞罰,固已足之之後也。雖財用足而厚愛之,然而輕刑猶之亂也。夫當家之愛子,財貨足用,財貨足用則輕用,輕用則侈泰;親愛之則不忍,不忍則驕恣;侈泰則家貧,驕恣則行暴,此雖財用足而愛厚,輕利之患也。凡人之生也,財用足則隳於用力,上治懦則肆於為非;財用足而力作者神農也,上治懦而行修者曾、史也;夫民之不及神農、曾、史亦已明矣。老聃有言曰:「知足不辱,知止不殆。」夫以殆辱之故而不求於足之外者老聃也,今以為足民而可以治,是以民為皆如老聃也。故桀貴在天子而不足於尊,富有四海之內而不足於寶。君人者雖足民,不能足使為君,天子而桀未必為天子為足也,則雖足民,何可以為治也?故明主之治國也,適其時事以致財物,論其稅賦以均貧富,厚其爵祿以盡賢能,重其刑罰以禁姦邪,使民以力得富,以事致貴,以過受罪,以功致賞而不念慈惠之賜,此帝王之政也。

人皆寐、則盲者不知,皆嘿、則喑者不知。覺而使之視,問而使之對,則喑盲者窮矣。不聽其言也,則無術者不知;不任其身也,則不肖者不知;聽其言而求其當,任其身而責其功,則無術不肖者窮矣。夫欲得力士而聽其自言,雖庸人與烏獲不可別也,授之以鼎俎則罷健效矣。故官職者,能士之鼎俎也,任之以事,而愚智分矣。故無術者得於不用,不肖者得於不任,言不用而自文以為辯,身不任而自飾以為高,世主眩其辯、濫其高而尊貴之,是不須視而定明也,不待對而定辯也,喑盲者不得矣。明主聽其言必責其用,觀其行必求其功,然則虛舊之學不談,矜誣之行不飾矣。


\end{pinyinscope}