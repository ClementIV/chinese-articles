\article{難三}

\begin{pinyinscope}
魯穆公問於子思曰:「吾聞龐𥼴氏之子不孝,其行奚如?」子思對曰:「君子尊賢以崇德,舉善以觀民。若夫過行,是細人之所識也,臣不知也。」子思出,子服厲伯入見,問龐𥼴氏子,子服厲伯對曰:「其過三,皆君之所未嘗聞。」自是之後,君貴子思而賤子服厲伯也。


或曰:魯之公室,三世劫於季氏,不亦宜乎!明君求善而賞之,求姦而誅之,其得之一也。故以善聞之者,以說善同於上者也;以姦聞之者,以惡姦同於上者也;此宜賞譽之所力也。不以姦聞,是異於上而下比周於姦者也,此宜毀罰之所及也。今子思不以過聞,而穆公貴之,厲伯以姦聞而穆公賤之,人情皆喜貴而惡賤,故季氏之亂成而不上聞,此魯君之所以劫也。且此亡王之俗,取、魯之民所以自美,而穆公獨貴之,不亦倒乎!


文公出亡,獻公使寺人披攻之蒲城,披斬其袪,文公奔翟。惠公即位,又使攻之惠竇,不得也。及文公反國,披求見。公曰:「蒲城之役,君令一宿,而汝即至;惠竇之難,君令三宿,而汝一宿,何其速也?」披對曰:「君令不二,除君之惡,惟恐不堪,蒲人、翟人余何有焉?今公即位,其無蒲、翟乎!且桓公置射鉤而相管仲。」君乃見之。


或曰:齊、晉絕祀,不亦宜乎!桓公能用管仲之功而忘射鉤之怨,文公能聽寺人之言而棄斬袪之罪,桓公、文公能容二子者也。後世之君,明不及二公;後世之臣,賢不如二子。以不忠之臣事不明之君。君不知,則有燕操、子罕、田常之賊;知之,則以管仲、寺人自解。君必不誅,而自以為有桓、文之德,是臣讎而明不能燭,多假之資。自以為賢而不戒,則雖無後嗣,不亦可乎!且寺人之言也,直飾君令而不貳者,則是貞於君也。死君後生臣不愧而復為貞,今惠公朝卒而暮事文公,寺人之不貳何如?

人有設桓公隱者曰:「一難,二難,三難,何也?」桓公不能對,以告管仲。管仲對曰:「一難也、近優而遠士。二難也、去其國而數之海。三難也、君老而晚置太子。」桓公曰:「善。」不擇日而廟禮太子。


或曰:管仲之射隱不得也。士之用不在近遠。而俳優侏儒,固人主之所與燕也。則近優而遠士,而以為治,非其難者也。夫處勢而不能用其有,而悖不去國,是以一人之力禁一國。以一人之力禁一國者,少能勝之。明能照遠姦而見隱微,必行之令,雖遠於海,內必無變;然則去國之海而不劫殺,非其難者也。楚成王置商臣以為太子,又欲置公子職,商臣作難,遂弒成王。公子宰,周太子也,公子根有寵,遂以東州反,分而為兩國。此皆非晚置太子之患也。夫分勢不二,庶孽卑,寵無藉,雖處大臣,晚置太子可也;然則晚置太子,庶孽不亂,又非其難也。物之所謂難者;必借人成勢而勿使侵害己,可謂一難也。貴妾不使二后,二難也。愛孽不使危正適,專聽一臣而不敢隅君,此則可謂三難也。


葉公子高問政於仲尼,仲尼曰:「政在悅近而來遠。」哀公問政於仲尼,仲尼曰:「政在選賢。」齊景公問政於仲尼,仲尼曰:「政在節財。」三公出,子貢問曰:「三公問夫子政一也,夫子對之不同,何也?」仲尼曰:「葉都大而國小,民有背心,故曰政在悅近而來遠。魯哀公有大臣三人,外障距諸侯四鄰之士,內比周而以愚其君,使宗廟不掃除,社稷不血食者,必是三臣也,故曰政在選賢。齊景公築雍門,為路寢,一朝而以三百乘之家賜者三,故曰政在節財。」


或曰:仲尼之對,亡國之言也。葉民有倍心,而說之悅近而來遠,則是教民懷惠。惠之為政,無功者受賞,而有罪者免,此法之所以敗也。法敗而政亂,以亂政治敗民,未見其可也。且民有倍心者,君上之明有所不及也。不紹葉公之明,而使之悅近而來遠,是舍吾勢之所能禁而使與不行惠以爭民,非能持勢者也。夫堯之賢,六王之冠也,舜一從而咸包,而堯無天下矣。有人無術以禁下,恃為舜而不失其民,不亦無術乎!明君見小姦於微,故民無大謀;行小誅於細,故民無大亂;此謂圖難於其所易也,為大者於其所細也。今有功者必賞,賞者不得君,力之所致也;有罪者必誅,誅者不怨上,罪之所生也。民知誅罰之皆起於身也,故疾功利於業,而不受賜於君。「太上、下智有之。」此言太上之下民無說也,安取懷惠之民?上君之民無利害,說以悅近來遠,亦可舍己。哀公有臣外障距內比周以愚其君,而說之以選賢,此非功伐之論也,選其心之所謂賢者也。使哀公知三子外障距內比周也,則三子不一日立矣。哀公不知選賢,選其心之所謂賢,故三子得任事。燕子噲賢子之而非孫卿,故身死為僇。夫差智太宰嚭而愚子胥,故滅於越。魯君不必知賢,而說以選賢,是使哀公有夫差、燕噲之患也。明君不自舉臣,臣相進也;不自賢,功自徇也。論之於任,試之於事,課之於功。故群臣公政而無私,不隱賢,不進不肖,然則人主奚勞於選賢?景公以百乘之家賜,而說以節財,是使景公無術使智君之侈,而獨儉於上,未免於貧也。有君以千里養其口腹,則雖桀、紂不侈焉。齊國方三千里,而桓公以其半自養,是侈於桀、紂也,然而能為五霸冠者,知侈儉之地也。為君不能禁下而自禁者謂之劫,不能飾下而自飾者謂之亂,不節下而自節者謂之貧。明君使人無私,以詐而食者禁;力盡於事,歸利於上者必聞,聞者必賞;污穢為私者必知,知者必誅。然故忠臣盡忠於方公,民士竭力於家,百官精剋於上,侈倍景公,非國之患也。然則說之以節財,非其急者也。夫對三公一言而三公可以無患,知下之謂也。知下明則禁於微,禁於微則姦無積,姦無積則無比周。無比周則公私分,公私分則朋黨散,朋黨散則無外障距內比周之患。知下明則見精沐,見精沐則誅賞明,誅賞明則國不貧,故曰一對而三公無患,知下之謂也。


鄭子產晨出,過東匠之閭,聞婦人之哭,撫其御之手而聽之。有閒,遣吏執而問之,則手絞其夫者也。異日,其御問曰:「夫子何以知之?」子產曰:「其聲懼。凡人於其親愛也,始病而憂,臨死而懼,已死而哀。今哭已死不哀而懼,是以知其有姦也。」


或曰:子產之治,不亦多事乎?姦必待耳目之所及而後知之,則鄭國之得姦者寡矣。不任典成之吏,不察參伍之政,不明度量,恃盡聰明,勞智慮,而以知姦,不亦無術乎?且夫物眾而智寡,寡不勝眾,智不足以遍知物,故因物以治物。下眾而上寡,寡不勝眾,者言君不足以遍知臣也,故因人以知人。是以形體不勞而事治,智慮不用而姦得。故宋人語曰:「一雀過羿,羿必得之,則羿誣矣。以天下為之羅,則雀不失矣。」夫知姦亦有大羅,不失其一而已矣。不修其理,而以己之胸察為之弓矢,則子產誣矣。老子曰:「以智治國,國之賊也。」其子產之謂矣。


秦昭王問於左右曰:「今時韓、魏孰與始強?」左右對曰:「弱於始也。」「今之如耳、魏齊孰與曩之孟常、芒卯?」對曰:「不及也。」王曰:「孟常、芒卯率強韓、魏猶無奈寡人何也!」左右對曰:「甚然!」中期推琴而對曰:「王之料天下過矣!夫六晉之時,知氏最強,滅范、中行而從韓、魏之兵以伐趙,灌以晉水,城之未沈者三板。知伯出,魏宣子御,韓康子為驂乘,知伯曰:「始吾不知水可以滅人之國,吾乃今知之。汾水可以灌安邑,絳水可以灌平陽。」魏宣子肘韓康子,康子踐宣子之足,肘足接乎車上,而知氏分於晉陽之下。今足下雖強,未若知氏;韓、魏雖弱,未至如其在晉陽之下也。此天下方用肘足之時,願王勿易之也。」


或曰:昭王之問也有失,左右中期之對也有過。凡明主之治國也,任其勢。勢不可害,則雖強天下無奈何也,而況孟常、芒卯、韓、魏能奈我何!其勢可害也,則不肖如如耳、魏齊,及韓、魏猶能害之。然則害與不侵,在自恃而已矣,奚問乎?自恃其不可侵,則強與弱奚其擇焉?失在不自恃,而問其奈何也,其不侵也幸矣!《申子》曰:「失之數而求之信則疑矣,」其昭王之謂也。知伯無度,從韓康、魏宣而圖以水灌滅其國,此知伯之所以國亡而身死、頭為飲杯之故也。今昭王乃問孰與始強,其畏有水人之患乎?雖有左右非韓、魏之二子也,安有肘足之事,而中期曰「勿易」,此虛言也。且中期之所官、琴瑟也,絃不調,弄不明,中期之任也,此中期所以事昭王者也。中期善承其任,未慊昭王也,而為所不知,豈不妄哉!左右對之曰「弱於始」與「不及」則可矣,其曰「甚然」則諛也。《申子》曰:「治不踰官,雖知不言。」今中期不知而尚言之。故曰昭王之問有失,左右中期之對皆有過也。


管子曰:「見其可說之有證,見其不可惡之有形,賞罰信於所見,雖所不見,其敢為之乎?見其可說之無證,見其不可惡之無形,賞罰不信於所見,而求所不見之外,不可得也。」


或曰:廣廷嚴居,眾人之所肅也;晏室獨處,曾、史之所僈也。觀人之所肅,非行、情也。且君上者,臣下之所為飾也。好惡在所見,臣下之飾姦物以愚其君,必也。明不能燭遠姦,見隱微,而待之以觀飾行,定賞罰,不亦弊乎!


管子曰:「言於室滿於室,言於堂滿於堂,是謂天下王。」


或曰:管仲之所謂言室滿室、言堂滿堂者,非特謂遊戲飲食之言也,必謂大物也。人主之大物,非法則術也。法者,編著之圖籍,設之於官府,而布之於百姓者也。術者,藏之於胸中,以偶眾端而潛御群臣者也。故法莫如顯,而術不欲見。是以明主言法,則境內卑賤莫不聞知也,不獨滿於堂。用術,則親愛近習莫之得聞也,不得滿室。而管子猶曰「言於室滿室,言於堂滿堂」,非法術之言也。

\end{pinyinscope}