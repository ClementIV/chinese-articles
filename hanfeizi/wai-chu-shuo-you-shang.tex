\article{外儲說右上}

\begin{pinyinscope}
君所以治臣者有三:

一、勢不足以化則除之。師曠之對,晏子之說,皆合勢之易也而道行之難,是與獸逐走也,未知除患。患之可除,在子夏之說春秋也。善持勢者蚤絕其姦萌,故季孫讓仲尼以遇勢,而況錯之於君乎?是以太公望殺狂矞,而臧獲不乘驥。嗣公知之,故不駕鹿。薛公知之,故與二欒博。此皆知同異之反也。故明主之牧臣也,說在畜烏。

二、人主者,利害之軺轂也,射者眾,故人主共矣。是以好惡見則下有因,而人主惑矣;辭言通則臣難言,而主不神矣。說在申子之言六慎,與唐易之言弋也。患在國羊之請變,與宣王之太息也。明之以靖郭氏之獻十珥也,與犀首、甘茂之道穴聞也。堂谿公知術,故問玉卮。昭侯能術,故以聽獨寢。明主之道,在申子之勸獨斷也。

三、術之不行,有故。不殺其狗則酒酸。夫國亦有狗,且左右皆社鼠也。人主無堯之再誅,與莊王之應太子,而皆有薄媼之決蔡嫗也。知貴不能以教歌之法先揆之,吳起之出愛妻,文公之斬顛頡,皆違其情者也。故能使人彈疽者,必其忍痛者也。

右經

說一

賞之譽之不勸,罰之毀之不畏,四者加焉不變,則其除之。

齊景公之晉,從平公飲,師曠侍坐,始坐,景公問政於師曠曰:「太師將奚以教寡人?」師曠曰:「君必惠民而已。」中坐,酒酣,將出,又復問政於師曠曰:「太師奚以教寡人?」曰:「君必惠民而已矣。」景公出之舍,師曠送之,又問政於師曠,師曠曰:「君必惠民而已矣。」景公歸,思,未醒,而得師曠之所謂。「公子尾、公子夏者,景公之二弟也,甚得齊民,家富貴而民說之,擬於公室,此危吾位者也,今謂我惠民者,使我與二弟爭民邪?」於是反國發廩粟以賦眾貧,散府餘財以賜孤寡,倉無陳粟,府無餘財,宮婦不御者出嫁之,七十受祿米,鬻德惠施於民也,已與二弟爭。居二年,二弟出走,公子夏逃楚,公子尾走晉。

景公與晏子游於少海,登柏寢之臺而還望其國,曰:「美哉,泱泱乎,堂堂乎,後世將孰有此?」晏子對曰:「其田成氏乎?」景公曰:「寡人有此國也,而曰田成氏有之,何也?」晏子對曰:「夫田成氏甚得齊民,其於民也,上之請爵祿行諸大臣,下之私大斗斛區釜以出貸,小斗斛區釜以收之。殺一牛,取一豆肉,餘以食士。終歲,布帛取二制焉,餘以衣士。故市木之價不加貴於山,澤之魚鹽龜鱉蠃蚌不加貴於海。君重斂,而田成氏厚施。齊嘗大飢,道旁餓死者不可勝數也,父子相牽而趨田成氏者不聞不生。故周秦之民相與歌之曰:謳乎,其己乎苞乎,其往歸田成子乎!《詩》曰:『雖無德與女,式歌且舞。』今田成氏之德,而民之歌舞,民德歸之矣。故曰:其田成氏乎。」公泫然出涕曰:「不亦悲乎!寡人有國而田成氏有之,今為之奈何?」晏子對曰:「君何患焉!若君欲奪之,則近賢而遠不肖,治其煩亂,緩其刑罰,振貧窮而恤孤寡,行恩惠而給不足,民將歸君,則雖有十田成氏,其如君何?」

或曰:景公不知用勢,而師曠、晏子不知除患。夫獵者,託車輿之安,用六馬之足,使王良佐轡,則身不勞而易及輕獸矣。今釋車輿之利,捐六馬之足與王良之御,而下走逐獸,則雖樓季之足無時及獸矣,託良馬固車則臧獲有餘。國者、君之車也,勢者、君之馬也。夫不處勢以禁誅擅愛之臣,而必德厚以與天下齊行以爭民,是皆不乘君之車,不因馬之利車而下走者也。故曰:景公不知用勢之主也,而師曠、晏子不知除患之臣也。

子夏曰:「春秋之記臣殺君,子殺父者,以十數矣,皆非一日之積也,有漸而以至矣。」凡姦者,行久而成積,積成而力多,力多而能殺,故明主蚤絕之。今田常之為亂,有漸見矣,而君不誅。晏子不使其君禁侵陵之臣,而使其主行惠,故簡公受其禍。故子夏曰:「善持勢者蚤絕姦之萌。」

季孫相魯,子路為郈令。魯以五月起眾為長溝,當此之為,子路以其私秩粟為漿飯,要作溝者於五父之衢而餐之。孔子聞之,使子貢往覆其飯,擊毀其器,曰:「魯君有民,子奚為乃餐之?」子路怫然怒,攘肱而入請曰:「夫子疾由之為仁義乎?所學於夫子者仁義也,仁義者,與天下共其所有而同其利者也。今以由之秩粟而餐民,不可何也?」孔子曰:「由之野也!吾以女知之,女徒未及也,女故如是之不知禮也!女之餐之,為愛之也。夫禮,天子愛天下,諸侯愛境內,大夫愛官職,士愛其家,過其所愛曰侵。今魯君有民而子擅愛之,是子侵也,不亦誣乎!」言未卒,而季孫使者至,讓曰:「肥也起民而使之,先生使弟子令徒役而餐之,將奪肥之民耶?」孔子駕而去魯。以孔子之賢,而季孫非魯君也,以人臣之資,假人主之術,蚤禁於未形,而子路不得行其私惠,而害不得生,況人主乎?以景公之勢而禁田常之侵也,則必無劫弒之患矣。

太公望東封於齊,齊東海上有居士曰狂矞、華士,昆弟二人者立議曰:「吾不臣天子,不友諸侯,耕作而食之,掘井而飲之,吾無求於人也。無上之名,無君之祿,不事仕而事力。」太公望至於營丘,使吏執殺之以為首誅。周公旦從魯聞之,發急傳而問之曰:「夫二子,賢者也。今日饗國而殺賢者,何也?」太公望曰:「是昆弟二人立議曰:『吾不臣天子,不友諸侯,耕作而食之,掘井而飲之,吾無求於人也,無上之名,無君之祿,不事仕而事力。』彼不臣天子者,是望不得而臣也。不友諸侯者,是望不得而使也。耕作而食之,掘井而飲之,無求於人者,是望不得以賞罰勸禁也。且無上名,雖知、不為望用;不仰君祿,雖賢、不為望功。不仕則不治,不任則不忠。且先王之所以使其臣民者,非爵祿則刑罰也。今四者不足以使之,則望當誰為君乎?不服兵革而顯,不親耕耨而名,又所以教於國也。今有馬於此,如驥之狀者,天下之至良也。然而驅之不前,卻之不止,左之不左,右之不右,則臧獲雖賤,不託其足。臧獲之所願託其足於驥者,以驥之可以追利辟害也。今不為人用,臧獲雖賤,不託其足焉。已自謂以為世之賢士,而不為主用,行極賢而不用於君,此非明主之所臣也,亦驥之不可左右矣,是以誅之。」

一曰。太公望東封於齊,海上有賢者狂矞,太公望聞之往請焉,三卻馬於門而狂矞不報見也,太公望誅之。當是時也,周公旦在魯,馳往止之,比至,已誅之矣。周公旦曰:「狂矞,天下賢者也,夫子何為誅之?」太公望曰:「狂矞也議不臣天子,不友諸侯,吾恐其亂法易教也,故以為首誅。今有馬於此,形容似驥也,然驅之不往,引之不前,雖臧獲不託足以旋其軫也。」

如耳說衛嗣公,衛嗣公說而太息。左右曰:「公何為不相也?」公曰:「夫馬似鹿者而題之千金,然而有百金之馬而無一金之鹿者,馬為人用而鹿不為人用也。今如耳,萬乘之相也,外有大國之意,其心不在衛,雖辯智,亦不為寡人用,吾是以不相也。」

薛公之相魏昭侯也,左右有欒子者曰陽胡、潘,其於王甚重,而不為薛公,薛公患之。於是乃召與之博,予之人百金,令之昆弟博,俄又益之人二百金。方博有閒,謁者言客張季之子在門,公怫然怒,撫兵而授謁者曰:「殺之,吾聞季之不為文也。」立有閒,時季羽在側,曰:「不然。竊聞季為公甚,顧其人陰未聞耳。」乃輟不殺客,而大禮之曰:「曩者聞季之不為文也,故欲殺之。今誠為文也,豈忘季哉!」告廩獻千石之粟,告府獻五百金,告騶私廄獻良馬固車二乘,因令奄將宮人之美妾二十人並遺季也。欒子因相謂曰:「為公者必利,不為公者必害,吾曹何愛不為公?」因私競勸而遂為之。薛公以人臣之勢,假人主之術也,而害不得生,況錯之人主乎?夫馴烏者斷其下翎焉,斷其下翎則必恃人而食,焉得不馴乎?夫明主畜臣亦然,令臣不得不利君之祿,不得無服上之名;夫利君之祿,服上之名,焉得不服?

說二

《申子》曰:「上明見,人備之;其不明見,人惑之。其知見,人惑之;不知見,人匿之。其無欲見,人司之;其有欲見,人餌之。故曰:吾無從知之,惟無為可以規之。」

一曰。《申子》曰:「慎而言也,人且知女;慎而行也,人且隨女。而有知見也,人且匿女;而無知見也,人且意女。女有知也,人且臧女;女無知也,人且行女。故曰:惟無為可以規之。」

田子方問唐易鞠曰:「弋者何慎?」對曰:「鳥以數百目視子,子以二目御之,子謹周子廩。」田子方曰:「善。子加之弋,我加之國。」鄭長者聞之曰:「田子方知欲為廩,而未得所以為廩。夫虛無無見者廩也。」

一曰。齊宣王問弋於唐易子曰:「弋者奚貴?」唐易子曰:「在於謹廩。」王曰:「何謂謹廩?」對曰:「鳥以數十目視人,人以二目視鳥,奈何不謹廩也?故曰在於謹廩也。」王曰:「然則為天下何以為此廩?今人主以二目視一國,一國以萬目視人主,將何以自為廩乎?」對曰:「鄭長者有言曰:『夫虛靜無為而無見也。』其可以為此廩乎。」

國羊重於鄭君,聞君之惡己也,侍飲,因先謂君曰:「臣適不幸而有過,願君幸而告之,臣請變更,則臣免死罪矣。」

客有說韓宣王,宣王說而太息,左右引王之說之以先告客以為德。

靖郭君之相齊也,王后死,未知所置,乃獻玉珥以知之。

一曰。薛公相齊,齊威王夫人死,中有十孺子皆貴於王,薛公欲知王所欲立而請置一人以為夫人,王聽之、則是說行於王而重於置夫人也,王不聽、是說不行而輕於置夫人也,欲先知王之所欲置以勸王置之,於是為十玉珥而美其一而獻之,王以賦十孺子,明日坐,視美珥之所在而勸王以為夫人。

甘茂相秦惠王,惠王愛公孫衍,與之閒有所言,曰:「寡人將相子。」甘茂之吏道穴聞之,以告甘茂,甘茂入見王,曰:「王得賢相,臣敢再拜賀。」王曰:「寡人託國於子,安更得賢相?」對曰:「將相犀首。」王曰:「子安聞之?」對曰:「犀首告臣。」王怒犀首之泄,乃逐之。

一曰。犀首,天下之善將也,梁王之臣也。秦王欲得之與治天下,犀首曰:「衍其人臣者也,不敢離主之國。」居期年,犀首抵罪於梁王,逃而入秦,秦王甚善之。樗里疾,秦之將也,恐犀首之代之將也,鑿穴於王之所常隱語者,俄而王果與犀首計曰:「吾欲攻韓,奚如?」犀首曰:「秋可矣。」王曰:「吾欲以國累子,子必勿泄也。」犀首反走再拜曰:「受命。」於是樗里疾也道穴聽之,矣郎中皆曰:「兵秋起攻韓犀首為將。」於是日也郎中盡知之,於是月也境內盡知之。王召樗里疾曰:「是何匈匈也,何道出?」樗里疾曰:「似犀首也。」王曰:「吾無與犀首言也,其犀首何哉?」樗里疾曰:「犀首也羈旅,新抵罪,其心孤,是言自嫁於眾。」王曰:「然。」使人召犀首,已逃諸侯矣。

堂谿公謂昭侯曰:「今有千金之玉卮,通而無當,可以盛水乎?」昭侯曰:「不可。」「有瓦器而不漏,可以盛酒乎?」昭侯曰:「可。」對曰:「夫瓦器至賤也,不漏,可以盛酒。雖有乎千金之玉卮,至貴,而無當,漏,不可盛水,則人孰注漿哉?今為人主而漏其群臣之語,是猶無當之玉卮也,雖有聖智,莫盡其術,為其漏也。」昭侯曰:「然。」昭侯聞堂谿公之言,自此之後,欲發天下之大事,未嘗不獨寢,恐夢言而使人知其謀也。

一曰。堂谿公見昭侯曰:「今有白玉之卮而無當,有瓦卮而有當,君渴,將何以飲?」君曰:「以瓦卮。」堂谿公曰:「白玉之卮美,而君不以飲者,以其無當耶?」君曰:「然。」堂谿公曰:「為人主而漏泄其群臣之語,譬猶玉卮之無當。」堂谿公每見而出,昭侯必獨臥,惟恐夢言泄於妻妾。

《申子》曰:「獨視者謂明,獨聽者謂聰。能獨斷者,故可以為天下主。」

說三

宋人有酤酒者,升概甚平,遇客甚謹,為酒甚美,縣幟甚高,著然不售,酒酸,怪其故,問其所知閭1長者楊倩,倩曰:「汝狗猛耶。」曰:「狗猛則酒何故而不售?」曰:「人畏焉。或令孺子懷錢挈壺罋而往酤,而狗迓而齕之,此酒所以酸而不售也。」夫國亦有狗,有道之士懷其術而欲以明萬乘之主,大臣為猛狗迎而齕之,此人主之所以蔽脅,而有道之士所以不用也。故桓公問管仲「治國最奚患?」對曰:「最患社鼠矣。」公曰:「何患社鼠哉?」對曰:「君亦見夫為社者乎?樹木而塗之,鼠穿其間,掘穴託其中,燻之則恐焚木,灌之則恐塗阤,此社鼠之所以不得也。今人君之左右,出則為勢重而收利於民,入則比周而蔽惡於君,內閒主之情以告外,外內為重,諸臣百吏以為富,吏不誅則亂法,誅之則君不安,據而有之,此亦國之社鼠也。」故人臣執柄而擅禁,明為己者必利,而不為己者必害,此亦猛狗也。夫大臣為猛狗而齕有道之士矣,左右又為社鼠而閒主之情,人主不覺,如此,主焉得無壅,國焉得無亡乎?1. 閭 : 原作「,問」。據:《意林》等,見《韓非子集解》。

一曰。宋之酤酒者有莊氏者,其酒常美,或使僕往酤莊氏之酒,其狗齕人,使者不敢往,乃酤佗家之酒,問曰:「何為不酤莊氏之酒?」對曰:「今日莊氏之酒酸。」故曰:不殺其狗則酒酸。

一曰。桓公問管仲曰:「治國何患?」對曰:「最苦社鼠。夫社木而塗之,鼠因自託也。燻之則木焚,灌之則塗阤,此所以苦於社鼠也。今人君左右,出則為勢重以收利於民,入則比周謾侮蔽惡以欺於君,不誅則亂法,誅之則人主危,據而有之,此亦社鼠也。」故人臣執柄擅禁,明為己者必利,不為己者必害,亦猛狗也。故左右為社鼠,用事者為猛狗,則術不行矣。

堯欲傳天下於舜,鯀諫曰:「不祥哉!孰以天下而傳之於匹夫乎?」堯不聽,舉兵而誅,殺鯀於羽山之郊。共工又諫曰:「孰以天下而傳之於匹夫乎?」堯不聽,又舉兵而誅,共工於幽州之都。於是天下莫敢言無傳天下於舜。仲尼聞之曰:「堯之知,舜之賢,非其難者也。夫至乎誅諫者必傳之舜,乃其難也。」一曰。「不以其所疑敗其所察則難也。」

荊莊王有茅門之法曰:「群臣大夫諸公子入朝,馬蹄踐霤者,廷理斬其輈,戮其御。」於是太子入朝,馬蹄踐霤,廷理斬其輈,戮其御。太子怒,入為王泣曰:「為我誅戮廷理。」王曰:「法者所以敬宗廟,尊社稷。故能立法從令尊敬社稷者,社稷之臣也,焉可誅也?夫犯法廢令不尊敬社稷者,是臣乘君而下尚校也。臣乘君則主失威,下尚校則上位危。威失位危,社稷不守,吾將何以遺子孫?」於是太子乃還走,避舍露宿三日,北面再拜請死罪。

一曰。楚王急召太子。楚國之法,車不得至於茆門。天雨,廷中有潦,太子遂驅車至於茆門。廷理曰:「車不得至茆門,非法也。」太子曰:「王召急,不得須無潦。」遂驅之,廷理舉殳而擊其馬,敗其駕。太子入為王泣曰:「廷中多潦,驅車至茆門,廷理曰非法也,舉殳擊臣馬,敗臣駕,王必誅之。」王曰:「前有老主而不踰,後有儲主而不屬,矜矣。是真吾守法之臣也。」乃益爵二級,而開後門出太子。「勿復過。」

衛嗣君謂薄疑曰:「子小寡人之國以為不足仕,則寡人力能仕子,請進爵以子為上卿。」乃進田萬頃。薄子曰:「疑之母親疑,以疑為能相萬乘所不窕也。然疑家巫有蔡嫗者,疑母甚愛信之,屬之家事焉。疑智足以信言家事,疑母盡以聽疑也。然已與疑言者,亦必復決之於蔡嫗也。故論疑之智能,以疑為能相萬乘而不窕也;論其親,則子母之間也;然猶不免議之於蔡嫗也。今疑之於人主也,非子母之親也,而人主皆有蔡嫗。人主之蔡嫗,必其重人也。重人者,能行私者也。夫行私者,繩之外也;而疑之所言,法之內也。繩之外與法之內,讎也,不相受也。」

一曰。衛君之晉,謂薄疑曰:「吾欲與子皆行。」薄疑曰:「媼也在中,請歸與媼計之。衛君自請薄媼,薄媼曰:「疑,君之臣也,君有意從之,甚善。」衛君曰:「吾以請之媼,媼許我矣。」薄疑歸言之媼也,曰:「衛君之愛疑奚與媼?」媼曰:「不如吾愛子也。」「衛君之賢疑奚與媼也?」曰:「不如吾賢子也。」「媼與疑計家事,已決矣,乃請決之於卜者蔡嫗。今衛君從疑而行,雖與疑決計,必與他蔡嫗敗之,如是則疑不得長為臣矣。」

夫教歌者,使先呼而詘之,其聲反清徵者乃教之。

一曰。教歌者,先揆以法,疾呼中宮,徐呼中徵。疾不中宮,徐不中徵,不可謂教。

吳起,衛左氏中人也。使其妻織組而幅狹於度,吳子使更之,其妻曰:「諾。」及成,復度之,果不中度,吳子大怒。其妻對曰:「吾始經之而不可更也。」吳子出之,其妻請其兄而索入,其兄曰:「吳子,為法者也。其為法也,且欲以與萬乘致功,必先踐之妻妾然後行之,子毋幾索入矣。」其妻之弟又重於衛君,乃因以衛君之重請吳子,吳子不聽,遂去衛而入荊也。

一曰。吳起示其妻以組曰:「子為我織組,令之如是。」組已就而效之,其組異善。起曰:「使子為組,令之如是,而今也異善何也?」其妻曰:「用財若一也,加務善之。」吳起曰:「非語也。」使之衣歸。其父往請之,吳起曰:「起家無虛言。」

晉文公問於狐偃曰:「寡人甘肥周於堂,卮酒豆肉集於宮,壺酒不清,生肉不布,殺一牛遍於國中,一歲之功盡以衣士卒,其足以戰民乎?」狐子曰:「不足。」文公曰:「吾弛關市之征而緩刑罰,其足以戰民乎?」狐子曰:「不足。」文公曰:「吾民之有喪資者,寡人親使郎中視事;有罪者赦之;貧窮不足者與之;其足以戰民乎?」狐子對曰:「不足。此皆所以慎產也。而戰之者,殺之也。民之從公也,為慎產也,公因而迎殺之,失所以為從公矣。」曰:「然則何如足以戰民乎?」狐子對曰:「令無得不戰。」公曰:「無得不戰奈何?」狐子對曰:「信賞必罰,其足以戰。」公曰:「刑罰之極安至?」對曰:「不辟親貴,法行所愛。」文公曰:「善。」明日令田於圃陸,期以日中為期,後期者行軍法焉。於是公有所愛者曰顛頡後期,吏請其罪,文公隕涕而憂。吏曰:「請用事焉。」遂斬顛頡之脊,以徇百姓,以明法之信也。而後百姓皆懼曰:「君於顛頡之貴重如彼甚也,而君猶行法焉,況於我則何有矣?」文公見民之可戰也,於是遂興兵伐原,克之。伐衛,東其畝,取五鹿。攻陽,勝虢,伐曹。南圍鄭,反之陴。罷宋圍,還與荊人戰城濮,大敗荊人,返為踐土之盟,遂成衡雍之義。一舉而八有功。所以然者,無他故異物,從狐偃之謀,假顛頡之脊也。

夫痤疽之痛也,非刺骨髓,則煩心不可支也;非如是不能使人以半寸砥石彈之。今人主之於治亦然,非不知有苦則安;欲治其國,非如是不能聽聖知而誅亂臣。亂臣者,必重人。重人者,必人主所甚親愛也。人主所甚親愛也者,是同堅白也。夫以布衣之資,欲以離人主之堅白、所愛,是以解左髀說右髀者,是身必死而說不行者也。


\end{pinyinscope}