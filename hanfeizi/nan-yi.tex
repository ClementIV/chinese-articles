\article{難一}

\begin{pinyinscope}
晉文公將與楚人戰,召舅犯問之,曰:「吾將與楚人戰,彼眾我寡,為之奈何?」舅犯曰:「臣聞之,繁禮君子,不厭忠信;戰陣之閒,不厭詐偽。君其詐之而已矣。」文公辭舅犯,因召雍季而問之,曰:「我將與楚人戰,彼眾我寡,為之奈何?」雍季對曰:「焚林而田,偷取多獸,後必無獸;以詐遇民,偷取一時,後必無復。」文公曰:「善。」辭雍季,以舅犯之謀與楚人戰以敗之。歸而行爵,先雍季而後舅犯。群臣曰:「城濮之事,舅犯謀也,夫用其言而後其身可乎?」文公曰:「此非君所知也。夫舅犯言,一時之權也;雍季言,萬世之利也。」仲尼聞之,曰:「文公之霸也宜哉!既知一時之權,又知萬世之利。」

或曰:雍季之對,不當文公之問。凡對問者,有因問小大緩急而對也,所問高大而對以卑狹,則明主弗受也。今文公問以少遇眾,而對曰「後必無復」,此非所以應也。且文公不知一時之權,又不知萬世之利。戰而勝,則國安而身定,兵強而威立,雖有後復,莫大於此,萬世之利,奚患不至?戰而不勝,則國亡兵弱,身死名息,拔拂今日之死不及,安暇待萬世之利?待萬世之利在今日之勝,今日之勝在詐於敵,詐敵,萬世之利而已。故曰:雍季之對不當文公之問。且文公又不知舅犯之言,舅犯所謂不厭詐偽者,不謂詐其民,請詐其敵也。敵者,所伐之國也,後雖無復,何傷哉?文公之所以先雍季者,以其功耶?則所以勝楚破軍者,舅犯之謀也;以其善言耶?則雍季乃道其後之無復也,此未有善言也。舅犯則以兼之矣。舅犯曰「繁禮君子,不厭忠信」者,忠、所以愛其下也,信、所以不欺其民也。夫既以愛而不欺矣,言孰善於此?然必曰出於詐偽者,軍旅之計也。舅犯前有善言,後有戰勝,故舅犯有二功而後論,雍季無一焉而先賞。「文公之霸,不亦宜乎,」仲尼不知善賞也。

歷山之農者侵畔,舜往耕焉,期年,甽畝正。河濱之漁者爭坻,舜往漁焉,期年,而讓長。東夷之陶者器苦窳,舜往陶焉,期年而器牢。仲尼歎曰:「耕、漁與陶,非舜官也,而舜往為之者,所以救敗也。舜其信仁乎!乃躬藉處苦而民從之,故曰:聖人之德化乎!」

或問儒者曰:「方此時也,堯安在?」其人曰:「堯為天子。」「然則仲尼之聖堯奈何?聖人明察在上位,將使天下無姦也。今耕漁不爭,陶器不窳,舜又何德而化?舜之救敗也,則是堯有失也;賢舜則去堯之明察,聖堯則去舜之德化;不可兩得也。楚人有鬻楯與矛者,譽之曰:『吾楯之堅,莫能陷也。』又譽其矛曰:『吾矛之利,於物無不陷也。』或曰:『以子之矛陷子之楯,何如?』其人弗能應也。夫不可陷之楯與無不陷之矛,不可同世而立。今堯、舜之不可兩譽,矛楯之說也。且舜救敗,期年已一過,三年已三過,舜有盡,壽有盡,天下過無已者,以有盡逐無已,所止者寡矣。賞罰使天下必行之,令曰:『中程者賞,弗中程者誅。』令朝至暮變,暮至朝變,十日而海內畢矣,奚待期年?舜猶不以此說堯令從己,乃躬親,不亦無術乎?且夫以身為苦而後化民者,堯、舜之所難也;處勢而驕下者,庸主之所易也。將治天下,釋庸主之所易,道堯、舜之所難,未可與為政也。」

管仲有病,桓公往問之,曰:「仲父病,不幸卒於大命,將奚以告寡人?」管仲曰:「微君言,臣故將謁之。願君去豎刁,除易牙,遠衛公子開方。易牙為君主味,君惟人肉未嘗,易牙烝其子首而進之;夫人情莫不愛其子,今弗愛其子,安能愛君?君妒而好內,豎刁自宮以治內,人情莫不愛其身,身且不愛,安能愛君?聞開方事君十五年,齊、衛之間不容數日行,棄其母久宦不歸,其母不愛,安能愛君?臣聞之:『矜偽不長,蓋虛不久。』願君去此三子者也。」管仲卒死,桓公弗行,及桓公死,蟲出尸不葬。

或曰:管仲所以見告桓公者,非有度者之言也。所以去豎刁、易牙者,以不愛其身,適君之欲也。曰「不愛其身,安能愛君」,然則臣有盡死力以為其主者,管仲將弗用也。曰「不愛其死力,安能愛君」,是君去忠臣也。且以不愛其身,度其不愛其君,是將以管仲之不能死公子糾度其不死桓公也,是管仲亦在所去之域矣。明主之道不然,設民所欲以求其功,故為爵祿以勸之;設民所惡以禁其姦,故為刑罰以威之。慶賞信而刑罰必,故君舉功於臣,而姦不用於上,雖有豎刁,其奈君何?且臣盡死力以與君市,君垂爵祿以與臣市,君臣之際,非父子之親也,計數之所出也。君有道,則臣盡力而姦不生;無道,則臣上塞主明而下成私。管仲非明此度數於桓公也,使去豎刁,一豎刁又至,非絕姦之道也。且桓公所以身死蟲流出尸不葬者,是臣重也;臣重之實,擅主也。有擅主之臣,則君令不下究,臣情不上通,一人之力能隔君臣之間,使善敗不聞,禍福不通,故有不葬之患也。明主之道,一人不兼官,一官不兼事。卑賤不待尊貴而進,論,大臣不因左右而見。百官修通,群臣輻湊。有賞者君見其功,有罰者君知其罪。見知不悖於前,賞罰不弊於後,安有不葬之患?管仲非明此言於桓公也,使去三子,故曰管仲無度矣。

襄子圍於晉陽中,出圍,賞有功者五人,高赫為賞首。張孟談曰:「晉陽之事,赫無大功,今為賞首何也?」襄子曰:「晉陽之事,寡人國家危,社稷殆矣。吾群臣無有不驕侮之意者,惟赫子不失君臣之禮,是以先之。」仲尼聞之曰:「善賞哉襄子!賞一人而天下為人臣者莫敢失禮矣。」

或曰:仲尼不知善賞矣。夫善賞罰者,百官不敢侵職,群臣不敢失禮。上設其法,而下無姦詐之心,如此,則可謂善賞罰矣。使襄子於晉陽也,令不行,禁不止,是襄子無國,晉陽無君也,尚誰與守哉?今襄子於晉陽也,知氏灌之,臼灶生龜,而民無反心,是君臣親也;襄子有君臣親之澤,操令行禁止之法,而猶有驕侮之臣,是襄子失罰也。為人臣者,乘事而有功則賞。今赫僅不驕侮而襄子賞之,是失賞也。明主賞不加於無功,罰不加於無罪。今襄子不誅驕侮之臣,而賞無功之赫,安在襄子之善賞也?故曰仲尼不知善賞。

晉平公與群臣飲,飲酣,乃喟然歎曰:「莫樂為人君!惟其言而莫之違。」師曠侍坐於前,援琴撞之,公披衽而避,琴壞於壁。公曰:「太師誰撞?」師曠曰:「今者有小人言於側者,故撞之。」公曰:「寡人也。」師曠曰:「啞!是非君人者之言也。」左右請除之。公曰:「釋之,以為寡人戒。」

或曰:平公失君道,師曠失臣禮。夫非其行而誅其身,君之於臣也;非其行則陳其言,善諫不聽則遠其身者,臣之於君也。今師曠非平公之行,不陳人臣之諫,而行人主之誅,舉琴而親其體,是逆上下之位,而失人臣之禮也。夫為人臣者,君有過則諫,諫不聽則輕爵祿以待之,此人臣之禮義也。今師曠非平公之過,舉琴而親其體,雖嚴父不加於子,而師曠行之於君,此大逆之術也。臣行大逆,平公喜而聽之,是失君道也。故平公之跡,不可明也,使人主過於聽而不悟其失。師曠之行亦不可明也,使姦臣襲極諫而飾弒君之道。不可謂兩明,此為兩過。故曰:平公失君道,師曠亦失臣禮矣。

齊桓公時,有處士曰小臣稷,桓公三往而弗得見。桓公曰:「吾聞布衣之士,不輕爵祿,無以易萬乘之主;萬乘之主,不好仁義,亦無以下布衣之士。」於是五往乃得見之。

或曰:桓公不知仁義。夫仁義者,憂天下之害,趨一國之患,不避卑辱謂之仁義。故伊尹以中國為亂,道為宰于湯;百里奚以秦為亂,道為虜于穆公;皆憂天下之害,趨一國之患,不辭卑辱,故謂之仁義。今桓公以萬乘之勢,下匹夫之士,將欲憂齊國,而小臣不行,見小臣之忘民也,忘民不可謂仁義。仁義者,不失人臣之禮,不敗君臣之位者也。是故四封之內,執會而朝名曰臣,臣吏分職受事名曰萌。今小臣在民萌之眾,而逆君上之欲,故不可謂仁義。仁義不在焉,桓公又從而禮之。使小臣有智能而遁桓公,是隱也,宜刑;若無智能而虛驕矜桓公,是誣也,宜戮;小臣之行,非刑則戮。桓公不能領臣主之理,而禮刑戮之人,是桓公以輕上侮君之俗教於齊國也,非所以為治也。故曰:桓公不知仁義。

靡笄之役,韓獻子將斬人,郤獻子聞之,駕往救之,比至,則已斬之矣。郤子因曰:「胡不以徇?」其僕曰:「曩不將救之乎?」郤子曰:「吾敢不分謗乎?」

或曰:郤子言不可不察也,非分謗也。韓子之所斬也,若罪人則不可救,救罪人,法之所以敗也,法敗則國亂;若非罪人,則勸之以徇,勸之以徇,是重不辜也,重不辜,民所以起怨者也,民怨則國危。郤子之言,非危則亂,不可不察也。且韓子之所斬若罪人,郤子奚分焉?斬若非罪人,則已斬之矣,而郤子乃至,是韓子之謗已成,而郤子且後至也。夫郤子曰「以徇」,不足以分斬人之謗,而又生徇之謗。是子言分謗也?昔者紂為炮烙,崇侯、惡來又曰斬涉者之脛也,奚分於紂之謗?且民之望於上也甚矣,韓子弗得,且望郤子之得之也;今郤子俱弗得,則民絕望於上矣,故曰:郤子之言非分謗也,益謗也。且郤子之往救罪也,以韓子為非也,不道其所以為非,而勸之「以徇」,是使韓子不知其過也。夫下使民望絕於上,又使韓子不知其失,吾未得郤子之所以分謗者也。

桓公解管仲之束縛而相之。管仲曰:「臣有寵矣,然而臣卑。」公曰:「使子立高、國之上。」管仲曰:「臣貴矣,然而臣貧。」公曰:「使子有三歸之家。」管仲曰:「臣富矣,然而臣疏。」於是立以為仲父。霄略曰:「管仲以賤為不可以治國,故請高、國之上;以貧為不可以治富,故請三歸;以疏為不可以治親,故處仲父。管仲非貪,以便治也。」

或曰:今使臧獲奉君令詔卿相,莫敢不聽,非卿相卑而臧獲尊也,主令所加,莫敢不從也。今使管仲之治,不緣桓公,是無君也,國無君不可以為治。若負桓公之威,下桓公之令,是臧獲之所以信也,奚待高、國、仲父之尊而後行哉?當世之行事都丞之下徵令者,不辟尊貴,不就卑賤。故行之而法者,雖巷伯信乎卿相;行之而非法者,雖大吏詘乎民萌。今管仲不務尊主明法,而事增寵益爵,是非管仲貪欲富貴,必闇而不知術也。故曰:管仲有失行,霄略有過譽。

韓宣王問於樛留:「吾欲兩用公仲、公叔其可乎?」樛留對曰:「昔魏兩用樓、翟而亡西河,楚兩用昭、景而亡鄢、郢,今君兩用公仲、公叔,此必將爭事而外市,則國必憂矣。」

或曰:昔者齊桓公兩用管仲、鮑叔,成湯兩用伊尹、仲虺。夫兩用臣者國之憂,則是桓公不霸,成湯不王也。湣王一用淖齒而手死乎東廟,主父一用李兌,減食而死。主有術,兩用不為患;無術,兩用則爭事而外市,一則專制而劫弒。今留無術以規上,使其主去兩用一,是不有西河、鄢、郢之憂,則必有身死減食之患。是樛留未有善以知言也。


\end{pinyinscope}