\article{備內}

\begin{pinyinscope}
人主之患在於信人,信人則制於人。人臣之於其君,非有骨肉之親也,縛於勢而不得不事也。故為人臣者,窺覘其君心也無須臾之休,而人主怠傲處其上,此世所以有劫君弒主也。為人主而大信其子,則姦臣得乘於子以成其私,故李兌傅趙王而餓主父。為人主而大信其妻,則姦臣得乘於妻以成其私,故優施傅麗姬,殺申生而立奚齊。夫以妻之近與子之親而猶不可信,則其餘無可信者矣。

且萬乘之主,千乘之君,后妃、夫人、適子為太子者,或有欲其君之蚤死者。何以知其然?夫妻者,非有骨肉之恩也,愛則親,不愛則疏。語曰:「其母好者其子抱。」然則其為之反也,其母惡者其子釋。丈夫年五十而好色未解也,婦人年三十而美色衰矣。以衰美之婦人事好色之丈夫,則身死見疏賤,而子疑不為後,此后妃、夫人之所以冀其君之死者也。唯母為后而子為主,則令無不行,禁無不止,男女之樂不減於先君,而擅萬乘不疑,此鴆毒扼昧之所以用也。故桃左春秋曰:「人主之疾死者不能處半。」人主弗知則亂多資,故曰:利君死者眾則人主危。故王良愛馬,越王勾踐愛人,為戰與馳。醫善吮人之傷,含人之血,非骨肉之親也,利所加也。故輿人成輿則欲人之富貴,匠人成棺則欲人之夭死也,非輿人仁而匠人賊也,人不貴則輿不售,人不死則棺不買,情非憎人也,利在人之死也。故后妃、夫人、太子之黨成而欲君之死也,君不死則勢不重,情非憎君也,利在君之死也,故人主不可以不加心於利己死者。故日月暈圍於外,其賊在內,備其所憎,禍在所愛。是故明王不舉不參之事,不食非常之食,遠聽而近視以審內外之失,省同異之言以知朋黨之分,偶參伍之驗以責陳言之實,執後以應前,按法以治眾,眾端以參觀,士無幸賞,無踰行,殺必當,罪不赦,則姦邪無所容其私。徭役多則民苦,民苦則權勢起,權勢起則復除重,復除重則貴人富,苦民以富貴人起勢,以藉人臣,非天下長利也。故曰徭役少則民安,民安則下無重權,下無重權則權勢滅,權勢滅則德在上矣。今夫水之勝火亦明矣,然而釜鬵閒之,水煎沸竭盡其上,而火得熾盛焚其下,水失其所以勝者矣。今夫治之禁姦又明於此,然守法之臣為釜鬵之行,則法獨明於胸中,而已失其所以禁姦者矣。上古之傳言,春秋所記,犯法為逆以成大姦者,未嘗不從尊貴之臣也。然而法令之所以備,刑罰之所以誅,常於卑賤,是以其民絕望,無所告愬。大臣比周,蔽上為一,陰相善而陽相惡,以示無私,相為耳目,以候主隙,人主掩蔽,無道得聞,有主名而無實,臣專法而行之,周天子是也。偏借其權勢則上下易位矣,此言人臣之不可借權勢也。


\end{pinyinscope}