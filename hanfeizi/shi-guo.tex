\article{十過}

\begin{pinyinscope}
十過:一曰、行小忠則大忠之賊也。二曰、顧小利則大利之殘也。三曰、行僻自用,無禮諸侯,則亡身之至也。四曰、不務聽治而好五音,則窮身之事也。五曰、貪愎喜利則滅國殺身之本也。六曰、耽於女樂,不顧國政,則亡國之禍也。七曰、離內遠遊而忽於諫士,則危身之道也。八曰、過而不聽於忠臣,而獨行其意,則滅高名為人笑之始也。九曰、內不量力,外恃諸侯,則削國之患也。十曰、國小無禮,不用諫臣,則絕世之勢也。

奚謂小忠?昔者楚共王與晉厲公戰於鄢陵,楚師敗,而共王傷其目。酣戰之時,司馬子反渴而求飲,豎穀陽操觴酒而進之。子反曰:「嘻,退!酒也。」穀陽曰:「非酒也。」子反受而飲之。子反之為人也,嗜酒而甘之,弗能絕於口,而醉。戰既罷,共王欲復戰,令人召司馬子反,司馬子反辭以心疾。共王駕而自往,入其幄中,聞酒臭而還,曰:「今日之戰,不穀親傷,所恃者司馬也。而司馬又醉如此,是亡楚國之社稷而不恤吾眾也,不穀無復戰矣。」於是還師而去,斬司馬子反以為大戮。故豎穀陽之進酒不以讎子反也,其心忠愛之而適足以殺之。故曰:行小忠則大忠之賊也。

奚謂顧小利?昔者晉獻公欲假道於虞以伐虢。荀息曰:「君其以垂棘之璧、與屈產之乘,賂虞公,求假道焉,必假我道。」君曰:「垂棘之璧,吾先君之寶也;屈產之乘,寡人之駿馬也。若受吾幣不假之道將奈何?」荀息曰:「彼不假我道,必不敢受我幣。若受我幣而假我道,則是寶猶取之內府而藏之外府也,馬猶取之內廄而著之外廄也。君勿憂。」君曰:「諾。」乃使荀息以垂棘之璧、與屈產之乘,賂虞公而求假道焉。虞公貪利其璧與馬而欲許之。宮之奇諫曰:「不可許。夫虞之有虢也,如車之有輔,輔依車,車亦依輔,虞、虢之勢正是也。若假之道,則虢朝亡而虞夕從之矣。不可,願勿許。」虞公弗聽,遂假之道。荀息伐虢之,還反處三年,興兵伐虞,又剋之。荀息牽馬操璧而報獻公,獻公說曰:「璧則猶是也。雖然,馬齒亦益長矣。」故虞公之兵殆而地削者何也?愛小利而不慮其害。故曰:顧小利則大利之殘也。

奚謂行僻?昔者楚靈王為申之會,宋太子後至,執而囚之,狎徐君,拘齊慶封。中射士諫曰:「合諸侯不可無禮,此存亡之機也。昔者桀為有戎之會,而有緡叛之;紂為黎丘之蒐,而戎、狄叛之;由無禮也。君其圖之。」君不聽,遂行其意。居未期年,靈王南遊,群臣從而劫之,靈王餓而死乾溪之上。故曰:行僻自用,無禮諸侯,則亡身之至也。

奚謂好音?昔者衛靈公將之晉,至濮水之上,稅車而放馬,設舍以宿,夜分,而聞鼓新聲者而說之,使人問左右,盡報弗聞。乃召師涓而告之,曰:「有鼓新聲者,使人問左右,盡報弗聞,其狀似鬼神,子為我聽而寫之。」師涓曰:「諾。」因靜坐撫琴而寫之。師涓明日報曰:「臣得之矣,而未習也,請復一宿習之。」靈公曰:「諾。」因復留宿,明日,而習之,遂去之晉。晉平公觴之於施夷之臺,酒酣,靈公起,公曰:「有新聲,願請以示。」平公曰:「善。」乃召師涓,令坐師曠之旁,援琴鼓之。未終,師曠撫止之,曰:「此亡國之聲,不可遂也。」平公曰:「此道奚出?」師曠曰:「此師延之所作,與紂為靡靡之樂也,及武王伐紂,師延東走,至於濮水而自投,故聞此聲者必於濮水之上。先聞此聲者其國必削,不可遂。」平公曰:「寡人所好者音也,子其使遂之。」師涓鼓究之。平公問師曠曰:「此所謂何聲也?」師曠曰:「此所謂清商也。」公曰:「清商固最悲乎?」師曠曰:「不如清徵。」公曰:「清徵可得而聞乎?」師曠曰:「不可,古之聽清徵者皆有德義之君也,今吾君德薄,不足以聽。」平公曰:「寡人之所好者音也,願試聽之。」師曠不得已,援琴而鼓。一奏之,有玄鶴二八,道南方來,集於郎門之垝。再奏之而列。三奏之,延頸而鳴,舒翼而舞。音中宮商之聲,聲聞於天。平公大說,坐者皆喜。平公提觴而起為師曠壽,反坐而問曰:「音莫悲於清徵乎?」師曠曰:「不如清角。」平公曰:「清角可得而聞乎?」師曠曰:「不可。昔者黃帝合鬼神於泰山之上,駕象車而六蛟龍,畢方並轄,蚩尤居前,風伯進掃,雨師灑道,虎狼在前,鬼神在後,騰蛇伏地,鳳皇覆上,大合鬼神,作為清角。今主君德薄,不足聽之,聽之將恐有敗。」平公曰:「寡人老矣,所好者音也,願遂聽之。」師曠不得已而鼓之。一奏之,有玄雲從西北方起;再奏之,大風至,大雨隨之,裂帷幕,破俎豆,隳廊瓦,坐者散走,平公恐懼,伏於廊室之間。晉國大旱,赤地三年。平公之身遂癃病。故曰:不務聽治,而好五音不已,則窮身之事也。

奚謂貪愎?昔者智伯瑤率趙、韓、魏而伐范、中行,滅之,反歸,休兵數年,因令人請地於韓,韓康子欲勿與。段規諫曰:「不可不與也。夫知伯之為人也,好利而驁愎。彼來請地而弗與,則移兵於韓必矣。君其與之。與之彼狃,又將請地他國,他國且有不聽,不聽,則知伯必加之兵。如是韓可以免於患而待其事之變。」康子曰:「諾。」因令使者致萬家之縣一於知伯,知伯說。又令人請地於魏,宣子欲勿與,趙葭諫曰:「彼請地於韓,韓與之,今請地於魏,魏弗與,則是魏內自強,而外怒知伯也。如弗予,其措兵於魏必矣,不如予之。」宣子「諾」。因令人致萬家之縣一於知伯。知伯又令人之趙請蔡、皋狼之地,趙襄子弗與,知伯因陰約韓、魏將以伐趙。襄子召張孟談而告之曰:「夫知伯之為人也,陽規而陰疏,三使韓、魏而寡人不與焉,其措兵於寡人必矣,今吾安居而可?」張孟談曰:「夫董閼于,簡主之才臣也,其治晉陽,而尹鐸循之,其餘教猶存,君其定居晉陽而已矣。」君曰:「諾。」乃召延陵生,令將軍車騎先至晉陽,君因從之。君至,而行其城郭及五官之藏,城郭不治,倉無積粟,府無儲錢,庫無甲兵,邑無守具,襄子懼,乃召張孟談曰:「寡人行城郭及五官之藏,皆不備具,吾將何以應敵?」張孟談曰:「臣聞聖人之治,藏於臣不藏於府庫,務修其教不治城郭。君其出令,令民自遺三年之食,有餘粟者入之倉,遺三年之用,有餘錢者入之府,遺,有奇人者使治城郭之繕。」君夕出令,明日,倉不容粟,府無積錢,庫不受甲兵,居五日而城郭已治,守備已具。君召張孟談而問之曰:「吾城郭已治,守備已具,錢粟已足,甲兵有餘,吾奈無箭何?」張孟談曰:「臣聞董子之治晉陽也,公宮之垣皆以荻蒿楛楚牆之,有楛高至於丈,君發而用之。」於是發而試之,其堅則雖菌輅之勁弗能過也。君曰:「吾箭已足矣,奈無金何?」張孟談曰:「臣聞董子之治晉陽也,公宮令舍之堂,皆以鍊銅為柱、質,君發而用之。」於是發而用之,有餘金矣。號令已定,守備已具,三國之兵果至,至則乘晉陽之城,遂戰,三月弗能拔。因舒軍而圍之,決晉陽之水以灌之,圍晉陽三年。城中巢居而處,懸釜而炊,財食將盡,士大夫羸病。襄子謂張孟談曰:「糧食匱,財力盡,士大夫羸病,吾恐不能守矣,欲以城下,何國之可下?」張孟談曰:「臣聞之,亡弗能存,危弗能安,則無為貴智矣,君失此計者。臣請試潛行而出,見韓、魏之君。」張孟談見韓、魏之君曰:「臣聞脣亡齒寒。今知伯率二君而伐趙,趙將亡矣。趙亡,則二君為之次。」二君曰:「我知其然也。雖然,知伯之為人也麤中而少親,我謀而覺,則其禍必至矣,為之奈何?」張孟談曰:「謀出二君之口而入臣之耳,人莫之知也。」二君因與張孟談約三軍之反,與之期日。夜遣孟談入晉陽以報二君之反於襄子,襄子迎孟談而再拜之,且恐且喜。二君以約遣張孟談,因朝知伯而出,遇智過於轅門之外,智過怪其色,因入見知伯曰:「二君貌將有變。」君曰:「何如?」曰:「其行矜而意高,非他時之節也,君不如先之。」君曰:「吾與二主約謹矣,破趙而三分其地,寡人所以親之,必不侵欺。兵之著於晉陽三年,今旦暮將拔之而嚮其利,何乃將有他心,必不然,子釋勿憂,勿出於口。」明旦,二主又朝而出,復見智過於轅門,智過入見曰:「君以臣之言告二主乎?」君曰:「何以知之?」曰:「今日二主朝而出,見臣而其色動,而視屬臣,此必有變,君不如殺之。」君曰:「子置勿復言。」智過曰:「不可,必殺之。若不能殺,遂親之。」君曰:「親之奈何?」智過曰:「魏宣子之謀臣曰趙葭,韓康子之謀臣曰段規,此皆能移其君之計,君與其二君約,破趙國因封二子者各萬家之縣一,如是則二主之心可以無變矣。知伯曰:「破趙而三分其地,又封二子者各萬家之縣一,則吾所得者少,不可。」智過見其言之不聽也,出,因更其族為輔氏。至於期日之夜,趙氏殺其守隄之吏而決其水灌知伯軍,知伯軍救水而亂,韓、魏翼而擊之,襄子將卒犯其前,大敗知伯之軍而擒知伯。知伯身死軍破,國分為三,為天下笑。故曰:貪愎好利,則滅國殺身之本也。

奚謂耽於女樂?昔者戎王使由余聘於秦,穆公問之曰:「寡人嘗聞道而未得目見之也,願聞古之明主得國失國何常以?」由余對曰:「臣嘗得聞之矣,常以儉得之,以奢失之。」穆公曰:「寡人不辱而問道於子,子以儉對寡人何也?」由余對曰:「臣聞昔者堯有天下,飯於土簋,飲於土鉶,其地南至交趾,北至幽都,東西至日月之所出入者,莫不賓服。堯禪天下,虞舜受之,作為食器,斬山木而財之,削鋸修之跡流漆墨其上,輸之於宮以為食器,諸侯以為益侈,國之不服者十三。舜禪天下而傳之於禹,禹作為祭器,墨染其外,而朱畫其內,縵帛為茵,蔣席頗緣,觴酌有采,而樽俎有飾,此彌侈矣,而國之不服者三十三。夏后氏沒,殷人受之,作為大路,而建九旒,食器雕琢,觴酌刻鏤,四壁堊墀,茵席雕文,此彌侈矣,而國之不服者五十三。君子皆知文章矣,而欲服者彌少,臣故曰儉其道也。」由余出,公乃召內史廖而告之,曰:「寡人聞鄰國有聖人,敵國之憂也。今由余,聖人也,寡人患之,吾將奈何?」內史廖曰:「臣聞戎王之居,僻陋而道遠,未聞中國之聲,君其遺之女樂,以亂其政,而後為由余請期,以疏其諫,彼君臣有間而後可圖也。」君曰:「諾。」乃使史廖以女樂二八遺戎王,因為由余請期,戎王許諾。見其女樂而說之,設酒張飲,日以聽樂,終歲不遷,牛馬半死。由余歸,因諫戎王,戎王弗聽,由余遂去之秦,秦穆公迎而拜之上卿,問其兵勢與其地形,既以得之,舉兵而伐之,兼國十二,開地千里。故曰:耽於女樂,不顧國政,亡國之禍也。

奚謂離內遠遊?昔者田成子遊於海而樂之,號令諸大夫曰:「言歸者死。」顏涿聚曰:「君遊海而樂之,奈臣有圖國者何?君雖樂之,將安得?」田成子曰:「寡人布令曰言歸者死,今子犯寡人之令。」援戈將擊之。顏涿聚曰:「昔桀殺關龍逢而紂殺王子比干,今君雖殺臣之身以三之可也。臣言為國,非為身也。」延頸而前曰:「君擊之矣!」君乃釋戈趣駕而歸,至三日,而聞國人有謀不內田成子者矣。田成子所以遂有齊國者,顏涿聚之力也。故曰:離內遠遊,則危身之道也。

奚謂過而不聽於忠臣?昔者齊桓公九合諸侯,一匡天下,為五伯長,管仲佐之。管仲老,不能用事,休居於家,桓公從而問之曰:「仲父家居有病,即不幸而不起此病,政安遷之?」管仲曰:「臣老矣,不可問也。雖然,臣聞之,知臣莫若君,知子莫若父,君其試以心決之。」君曰:「鮑叔牙何如?」管仲曰:「不可。鮑叔牙為人,剛愎而上悍。剛則犯民以暴,愎則不得民心,悍則下不為用,其心不懼。非霸者之佐也。」公曰:「然則豎刁何如?」管仲曰:「不可。夫人之情莫不愛其身,公妒而好內,豎刁自獖以為治內,其身不愛,又安能愛君?」公曰:「然則衛公子開方何如?」管仲曰:「不可。齊、衛之間不過十日之行,開方為事君,欲適君之故,十五年不歸見其父母,此非人情也,其父母之不親也,又能親君乎?」公曰:「然則易牙何如?」管仲曰:「不可。夫易牙為君主味,君之所未嘗食唯人肉耳,易牙蒸其子首而進之,君所知也。人之情莫不愛其子,今蒸其子以為膳於君,其子弗愛,又安能愛君乎?」公曰:「然則孰可?」管仲曰:「隰朋可。其為人也,堅中而廉外,少欲而多信。夫堅中則足以為表,廉外則可以大任,少欲則能臨其眾,多信則能親鄰國,此霸者之佐也,君其用之。」君曰:「諾。」居一年餘,管仲死,君遂不用隰朋而與豎刁。刁蒞事三年,桓公南遊堂阜,豎刁率易牙、衛公子開方及大臣為亂,桓公渴餒而死南門之寢、公守之室,身死三月不收,蟲出於戶。故桓公之兵橫行天下,為五伯長,卒見弒於其臣,而滅高名,為天下笑者,何也?不用管仲之過也。故曰:過而不聽於忠臣,獨行其意,則滅其高名為人笑之始也。

奚謂內不量力?昔者秦之攻宜陽,韓氏急,公仲朋謂韓君曰:「與國不可恃也,豈如因張儀為和於秦哉?因賂以名都而南與伐楚,是患解於秦而害交於楚也。」公曰:「善。」乃警公仲之行,將西和秦。楚王聞之,懼,召陳軫而告之曰:「韓朋將西和秦,今將奈何?」陳軫曰:「秦得韓之都一,驅其練甲,秦、韓為一以南鄉楚,此秦王之所以廟祠而求也,其為楚害必矣,王其趣發信臣,多其車,重其幣,以奉韓曰:『不穀之國雖小,卒已悉起,願大國之信意於秦也。因願大國令使者入境視楚之起卒也。』」韓使人之楚,楚王因發車騎陳之下路,謂韓使者曰:「報韓君言弊邑之兵今將入境矣。」使者還報韓君,韓君大悅,止公仲,公仲曰:「不可。夫以實告我者秦也,以名救我者楚也,聽楚之虛言而輕誣強秦之實禍,則危國之本也。」韓君弗聽,公仲怒而歸,十日不朝。宜陽益急,韓君令使者趣卒於楚,冠蓋相望而卒無至者,宜陽果拔,為諸侯笑。故曰:內不量力,外恃諸侯者,則國削之患也。

奚謂國小無禮?昔者晉公子重耳出亡過於曹。曹君袒裼而觀之。釐負羈與叔瞻侍於前。叔瞻謂曹君曰。臣觀晉公子非常人也。君遇之無禮。彼若有時反國而起兵。即恐為曹傷。君不如殺之。曹君弗聽。釐負羈歸而不樂。其妻問之曰。公從外來而有不樂之色何也。負羈曰。吾聞之。有福不及。禍來連我。今日吾君召晉公子。其遇之無禮。我與在前。吾是以不樂。其妻曰。吾觀晉公子。萬乘之主也。其左右從者。萬乘之相也。今窮而出亡過於曹。曹遇之無禮。此若反國。必誅無禮。則曹其首也。子奚不先自貳焉。負羈曰。諾。盛黃金於壺。充之以餐。加璧其上。夜令人遺公子。公子見使者。再拜受其餐而辭其璧。公子自曹入楚自楚入秦。入秦三年。秦穆公召群臣而謀曰。昔者晉獻公與寡人交。諸侯莫弗聞。獻公不幸離群臣。出入十年矣。嗣子不善。吾恐此將令其宗廟不祓除而社稷不血食也。如是弗定。則非與人交之道。吾欲輔重耳而入之晉。何如?群臣皆曰善。公因起卒。革車五百乘。疇騎二千。步卒五萬。輔重耳入之于晉。立為晉君。重耳即位三年。舉兵而伐曹矣。因令人告曹君曰。懸叔瞻而出之。我且殺而以為大戮。又令人告釐負羈曰。軍旅薄城。吾知子不違也。其表子之閭。寡人將以為令。令軍勿敢犯。曹人聞之率其親戚而保釐負羈之閭者七百餘家。此禮之所用也。故曹小國也。而迫於晉、楚之間。其君之危猶累卵也。而以無禮蒞之。此所以絕世也。故曰。國小無禮。不用諫臣。則絕世之勢也。


\end{pinyinscope}