\article{有度}

\begin{pinyinscope}
國無常強,無常弱。奉法者強則國強,奉法者弱則國弱。荊莊王并國二十六,開地三千里,莊王之氓社稷也,而荊以亡。齊桓公并國三十,啟地三千里,桓公之氓社稷也,而齊以亡。燕襄王以河為境,以薊為國,襲涿、方城,殘齊,平中山,有燕者重,無燕者輕,襄王之氓社稷也,而燕以亡。魏安釐王攻趙救燕,取地河東;攻盡陶、魏之地;加兵於齊,私平陸之都;攻韓拔管,勝於淇下;睢陽之事,荊軍老而走;蔡、召陵之事,荊軍破;兵四布於天下,威行於冠帶之國;安釐死而魏以亡。故有荊莊、齊桓則荊、齊可以霸,有燕襄、魏安釐則燕、魏可以強。今皆亡國者,其群臣官吏皆務所以亂,而不務所以治也。其國亂弱矣,又皆釋國法而私其外,則是負薪而救火也,亂弱甚矣。

故當今之時,能去私曲就公法者,民安而國治;能去私行行公法者,則兵強而敵弱。故審得失有法度之制者加以群臣之上,則主不可欺以詐偽;審得失有權衡之稱者以聽遠事,則主不可欺以天下之輕重。今若以譽進能,則臣離上而下比周;若以黨舉官,則民務交而不求用於法。故官之失能者其國亂。以譽為賞,以毀為罰也,則好賞惡罰之人,釋公行、行私術、比周以相為也。忘主外交,以進其與,則其下所以為上者薄矣。交眾與多,外內朋黨,雖有大過,其蔽多矣。故忠臣危死於非罪,姦邪之臣安利於無功。忠臣危死而不以其罪,則良臣伏矣;姦邪之臣安利不以功,則姦臣進矣;此亡之本也。若是、則群臣廢法而行私重,輕公法矣。數至能人之門,不壹至主之廷;百慮私家之便,不壹圖主之國。屬數雖多,非所以尊君也;百官雖具,非所以任國也。然則主有人主之名,而實託於群臣之家也。故臣曰:亡國之廷無人焉。廷無人者,非朝廷之衰也。家務相益,不務厚國;大臣務相尊,而不務尊君;小臣奉祿養交,不以官為事。此其所以然者,由主之不上斷於法,而信下為之也。故明主使法擇人,不自舉也;使法量功,不自度也。能者不可弊,敗者不可飾,譽者不能進,非者弗能退,則君臣之間明辨而易治,故主讎法則可也。

賢者之為人臣,北面委質,無有二心,朝廷不敢辭賤,軍旅不敢辭難,順上之為,從主之法,虛心以待令而無是非也。故有口不以私言,有目不以私視,而上盡制之。為人臣者,譬之若手,上以脩頭,下以脩足,清暖寒熱,不得不救,入,鏌邪傅體,不敢弗搏。無私賢哲之臣,無私事能之士。故民不越鄉而交,無百里之慼。貴賤不相踰,愚智提衡而立,治之至也。今夫輕爵祿,易去亡,以擇其主,臣不謂廉。詐說逆法,倍主強諫,臣不謂忠。行惠施利,收下為名,臣不謂仁。離俗隱居,而以作非上,臣不謂義。外使諸侯,內耗其國,伺其危嶮之陂以恐其主曰:「交非我不親,怨非我不解」,而主乃信之,以國聽之,卑主之名以顯其身,毀國之厚以利其家,臣不謂智。此數物者,險世之說也,而先王之法所簡也。先王之法曰:「臣毋或作威,毋或作利,從王之指;無或作惡,從王之路。」古者世治之民,奉公法,廢私術,專意一行,具以待任。

夫為人主而身察百官,則日不足,力不給。且上用目則下飾觀,上用耳則下飾聲,上用慮則下繁辭。先王以三者為不足,故舍己能,而因法數,審賞罰。先王之所守要,故法省而不侵。獨制四海之內,聰智不得用其詐,險躁不得關其佞,姦邪無所依。遠在千里外,不敢易其辭;勢在郎中,不敢蔽善飾非。朝廷群下,直湊單微,不敢相踰越。故治不足而日有餘,上之任勢使然也。

夫人臣之侵其主也,如地形焉,即漸以往,使人主失端、東西易面而不自知。故先王立司南以端朝夕。故明主使其群臣不遊意於法之外,不為惠於法之內,動無非法。法所以凌過遊外私也,嚴刑所以遂令懲下也。威不貸錯,制不共門。威制共則眾邪彰矣,法不信則君行危矣,刑不斷則邪不勝矣。故曰:巧匠目意中繩,然必先以規矩為度;上智捷舉中事,必以先王之法為比。故繩直而枉木斲,準夷而高科削,權衡縣而重益輕,斗石設而多益少。故以法治國,舉措而已矣。法不阿貴,繩不撓曲。法之所加,智者弗能辭,勇者弗敢爭。刑過不避大臣,賞善不遺匹夫。故矯上之失,詰下之邪,治亂決繆,絀羨齊非,一民之軌,莫如法。屬官威民,退淫殆,止詐偽,莫如刑。刑重則不敢以貴易賤,法審則上尊而不侵,上尊而不侵則主強,而守要,故先王貴之而傳之。人主釋法用私,則上下不別矣。


\end{pinyinscope}