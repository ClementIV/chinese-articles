\article{難勢}

\begin{pinyinscope}
慎子曰:「飛龍乘雲,騰蛇遊霧,雲罷霧霽,而龍蛇與螾螘同矣,則失其所乘也。賢人而詘於不肖者,則權輕位卑也;不肖而能服於賢者,則權重位尊也。堯為匹夫不能治三人,而桀為天子能亂天下,吾以此知勢位之足恃,而賢智之不足慕也。夫弩弱而矢高者,激於風也;身不肖而令行者,得助於眾也。堯教於隸屬而民不聽,至於南面而王天下,令則行,禁則止。由此觀之,賢智未足以服眾,而勢位足以詘賢者也。」

應慎子曰:飛龍乘雲,騰蛇遊霧,吾不以龍蛇為不託於雲霧之勢也。雖然,夫釋賢而專任勢,足以為治乎?則吾未得見也。夫有雲霧之勢,而能乘遊之者,龍蛇之材美也。今雲盛而螾弗能乘也,霧醲而螘不能遊也,夫有盛雲醲霧之勢而不能乘遊者,螾螘之材薄也。今桀、紂南面而王天下,以天子之威為之雲霧,而天下不免乎大亂者,桀、紂之材薄也。且其人以堯之勢以治天下也,其勢何以異桀之勢也,亂天下者也。夫勢者,非能必使賢者用已,而不肖者不用已也,賢者用之則天下治,不肖者用之則天下亂。人之情性,賢者寡而不肖者眾,而以威勢之利濟亂世之不肖人,則是以勢亂天下者多矣,以勢治天下者寡矣。夫勢者,便治而利亂者也,故《周書》曰:「毋為虎傅翼,將飛入邑,擇人而食之。」夫乘不肖人於勢,是為虎傅翼也。桀、紂為高臺深池以盡民力,為炮烙以傷民性,桀、紂得乘四行者,南面之威為之翼也。使桀、紂為匹夫,未始行一而身在刑戮矣。勢者,養虎狼之心,而成暴亂之事者也,此天下之大患也。勢之於治亂,本末有位也,而語專言勢之足以治天下者,則其智之所至者淺矣。夫良馬固車,使臧獲御之則為人笑,王良御之而日取千里,車馬非異也,或至乎千里,或為人笑,則巧拙相去遠矣。今以國位為車,以勢為馬,以號令為轡,以刑罰為鞭筴,使堯、舜御之則天下治,桀、紂御之則天下亂,則賢不肖相去遠矣。夫欲追速致遠,不知任王良;欲進利除害,不知任賢能;此則不知類之患也。夫堯、舜亦治民之王良也。

復應之曰:其人以勢為足恃以治官。客曰「必待賢乃治」,則不然矣。夫勢者,名一而變無數者也。勢必於自然,則無為言於勢矣。吾所為言勢者,言人之所設也。今日堯、舜得勢而治,桀、紂得勢而亂,吾非以堯、桀為不然也。雖然,非一人之所得設也。夫堯、舜生而在上位,雖有十桀、紂不能亂者,則勢治也;桀、紂亦生而在上位,雖有十堯、舜而亦不能治者,則勢亂也。故曰:「勢治者,則不可亂;而勢亂者,則不可治也。」此自然之勢也,非人之所得設也。若吾所言,謂人之所得勢也而已矣,賢何事焉?何以明其然也?客曰:「人有鬻矛與楯者,譽其楯之堅,物莫能陷也,俄而又譽其矛曰:『吾矛之利,物無不陷也。』人應之曰:『以子之矛陷子之楯何如?』其人弗能應也。」以為不可陷之楯,與無不陷之矛,為名不可兩立也。夫賢之為勢不可禁,而勢之為道也無不禁,以不可禁之勢,此矛楯之說也;夫賢勢之不相容亦明矣。且夫堯、舜、桀、紂千世而一出,是比肩隨踵而生也,世之治者不絕於中。吾所以為言勢者,中也。中者,上不及堯、舜,而下亦不為桀、紂。抱法處勢則治,背法去勢則亂。今廢勢背法而待堯、舜,堯、舜至乃治,是千世亂而一治也。抱法處勢而待桀、紂,桀、紂至乃亂,是千世治而一亂也。且夫治千而亂一,與治一而亂千也,是猶乘驥駬而分馳也,相去亦遠矣。夫棄隱栝之法,去度量之數,使奚仲為車,不能成一輪。無慶賞之勸,刑罰之威,釋勢委法,堯、舜戶說而人辯之,不能治三家。夫勢之足用亦明矣,而曰必待賢則亦不然矣。且夫百日不食以待粱肉,餓者不活;今待堯、舜之賢乃治當世之民,是猶待粱肉而救餓之說也。夫曰良馬固車,臧獲御之則為人笑,王良御之則日取乎千里,吾不以為然。夫待越人之善海遊者以救中國之溺人,越人善游矣,而溺者不濟矣。夫待古之王良以馭今之馬,亦猶越人救溺之說也,不可亦明矣。夫良馬固車,五十里而一置,使中手御之,追速致遠,可以及也,而千里可日致也,何必待古之王良乎!且御,非使王良也,則必使臧獲敗之;治,非使堯、舜也,則必使桀、紂亂之。此味非飴蜜也,必苦萊亭歷也。此則積辯累辭,離理失術,兩末之議也,奚可以難,失道理之言乎哉!客議未及此論也。


\end{pinyinscope}