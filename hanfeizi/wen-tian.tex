\article{問田}

\begin{pinyinscope}
徐渠問田鳩曰:「臣聞智士不襲下而遇君,聖人不見功而接上。令陽成義渠,明將也,而措於毛伯;公孫亶回,聖相也,而關於州部;何哉?」田鳩曰:「此無他故異物,主有度,上有術之故也。且足下獨不聞楚將宋觚而失其政,魏相馮離而亡其國。二君者驅於聲詞,眩乎辯說,不試於毛伯,不關乎州部,故有失政亡國之患。由是觀之,夫無毛伯之試,州部之關,豈明主之備哉!」

堂谿公謂韓子曰:「臣聞服禮辭讓,全之術也;修行退智,遂之道也。今先生立法術,設度數,臣竊以為危於身而殆於軀。何以效之?所聞先生術曰:『楚不用吳起而削亂,秦行商君而富彊,二子之言已當矣,然而吳起支解而商君車裂者,不逢世遇主之患也。』逢遇不可必也,患禍不可斥也,夫舍乎全遂之道而肆乎危殆之行,竊為先生無取焉。」韓子曰:「臣明先生之言矣。夫治天下之柄,齊民萌之度,甚未易處也。然所以廢先王之教,而行賤臣之所取者,竊以為立法術,設度數,所以利民萌便眾庶之道也。故不憚亂主闇上之患禍,而必思以齊民萌之資利者,仁智之行也。憚亂主闇上之患禍,而避乎死亡之害,知明夫身而不見民萌之資利者,貪鄙之為也。臣不忍嚮貪鄙之為,不敢傷仁智之行。先王有幸臣之意,然有大傷臣之實。」


\end{pinyinscope}