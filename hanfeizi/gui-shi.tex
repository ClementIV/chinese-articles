\article{詭使}

\begin{pinyinscope}
聖人之所以為治道者三:一曰利,二曰威,三曰名。夫利者所以得民也,威者所以行令也,名者上下之所同道也。非此三者,雖有不急矣。今利非無有也而民不化,上威非不存也而下不聽從,官非無法也而治不當名。三者非不存也,而世一治一亂者何也?夫上之所貴與其所以為治相反也。

夫立名號所以為尊也,今有賤名輕實者,世謂之高。設爵位所以為賤貴基也,而簡上不求見者,世謂之賢。威利所以行令也,而無利輕威者,世謂之重。法令所以為治也,而不從法令、為私善者,世謂之忠。官爵所以勸民也,而好名義、不進仕者,世謂之烈士。刑罰所以擅威也,而輕法、不避刑戮死亡之罪者,世謂之勇夫。民之急名也甚,其求利也如此,則士之飢餓乏絕者,焉得無巖居苦身以爭名於天下哉?故世之所以不治者,非下之罪,上失其道也。常貴其所以亂,而賤其所以治,是故下之所欲,常與上之所以為治相詭也。今下而聽其上,上之所急也。而惇愨純信、用心怯言,則謂之窶。守法固、聽令審,則謂之愚。敬上畏罪,則謂之怯。言時節,行中適,則謂之不肖。無二心私學,聽吏從教者,則謂之陋。難致謂之正。難予謂之廉。難禁謂之齊。有令不聽從謂之勇。無利於上謂之愿。少欲寬惠行德謂之仁。重厚自尊謂之長者。私學成群謂之師徒。閑靜安居謂之有思。損仁逐利謂之疾。險躁佻反覆謂之智。先為人而後自為,類名號,言,汎愛天下,謂之聖。言大本稱而不可用,行而乘於世者,謂之大人。賤爵祿,不撓上者,謂之傑。下漸行如此,入則亂民,出則不便也。上宜禁其欲、滅其跡而不止也,又從而尊之,是教下亂上以為治也。

凡所治者刑罰也,今有私行義者尊。社稷之所以立者安靜也,而譟險讒諛者任。四封之內所以聽從者信與德也,而陂知傾覆者使。令之所以行、威之所以立者恭儉聽上,而巖居非世者顯。倉廩之所以實者耕農之本務也,而綦組錦繡刻劃為末作者富。名之所以成、城池之所以廣者戰士也,今死士之孤飢餓乞於道,而優笑酒徒之屬乘車衣絲。賞祿所以盡民力易下死也,今戰勝攻取之士勞而賞不霑,而卜筮視手理狐蟲為順辭於前者日賜。上握度量所以擅生殺之柄也,今守度奉量之士欲以忠嬰上而不得見,巧言利辭行姦軌以倖偷世者數御。據法直言、名刑相當、循繩墨、誅姦人所以為上治也而愈疏遠,諂施順意從欲以危世者近。習悉租稅、專民力所以備難充倉府也,而士卒之逃事狀匿附託有威之門以避傜賦、而上不得者萬數。夫陳善田利宅所以戰士卒也,而斷頭裂腹播骨乎平原野者,無宅容身,身死田奪;而女妹有色、大臣左右無功者,擇宅而受,擇田而食。賞利一從上出、所以擅剬下也,而戰介之士不得職,而閒居之士尊顯。上以此為教,名安得無卑,位安得無危。夫卑名位者,必下之不從法令、有二心無私學、反逆世者也,而不禁其行,不破其群,以散其黨,又從而尊之,用事者過矣。上世之所以立廉恥者,所以屬下也;今士大夫不羞汙泥醜辱而宦,女妹私義之門不待次而宦。賞賜之所以為重也,而戰鬥有功之士貧賤,而便辟優徒超級。名號誠信,所以通威也,而主揜障。近習女謁並行,百官主爵遷人,用事者過矣。大臣官人與下先謀比周,雖不法行,威利在下則主卑而大臣重矣。

夫立法令者以廢私也,法令行而私道廢矣。私者所以亂法也。而士有二心私學、巖居窞處、託伏深慮,大者非世,細者惑下;上不禁,又從而尊之,以名化之以實,是無功而顯,無勞而富也。如此,則士之有二心私學者,焉得無深慮、勉知詐、與誹謗法令以求索,與世相反者也。凡亂上反世者,常士有二心私學者也。故本言曰:「所以治者法也,所以亂者私也;法立,則莫得為私矣。」故曰:道私者亂,道法者治。上無其道,則智者有私詞,賢者有私意。上有私惠,下有私欲,聖智成群,造言作辭,以非法措於上。上不禁塞,又從而尊之,是教下不聽上、不從法也。是以賢者顯名而居,姦人賴賞而富。賢者顯名而居,姦人賴賞而富,是以上不勝下也。


\end{pinyinscope}