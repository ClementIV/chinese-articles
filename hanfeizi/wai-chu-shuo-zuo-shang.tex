\article{外儲說左上}

\begin{pinyinscope}
一、明主之道,如有若之應密子也。明主之聽言也美其辯,其觀行也賢其遠,故群臣士民之道言者迂弘,其行身也離世。其說在田鳩對荊王也。故墨子為木鳶,謳癸築武宮。夫藥酒用言,明君聖主之以獨知也。

二、人主之聽言也,不以功用為的,則說者多棘刺白馬之說;不以儀的為關,則射者皆如羿也。人主於說也,皆如燕王學道也;而長說者,皆如鄭人爭年也。是以言有纖察微難而非務也,故李、惠、宋、墨皆畫策也;論有迂深閎大非用也,故畏震瞻車狀皆鬼魅也;言而拂難堅确非功也,故務、卞、鮑、介、墨翟皆堅瓠也。且虞慶詘匠也而屋壤,范且窮工而弓折。是故求其誠者,非歸餉也不可。

三、挾夫相為則責望,自為則事行。故父子或怨譟,取庸作者進美羹。說在文公之先宣言,與句踐之稱如皇也。故桓公藏蔡怒而攻楚,吳起懷瘳實而吮傷。且先王之賦頌,鍾鼎之銘,皆播吾之跡,華山之博也。然先王所期者利也,所用者力也。築社之諺,目辭說也。請許學者而行宛曼於先王,或者不宜今乎?如是不能更也。鄭縣人得車厄也,衛人佐弋也,卜子妻寫弊褲也,而其少者也。先王之言,有其所為小而世意之大者,有其所為大而世意之小者,未可必知也。說在宋人之解書,與梁人之讀記也。故先王有郢書而後世多燕說。夫不適國事而謀先王,皆歸取度者也。

四、利之所在民歸之,名之所彰士死之。是以功外於法而賞加焉,則上不能得所利於下;名外於法而譽加焉,則士勸名而不畜之於君。故中章、胥己仕,而中牟之民棄田圃而隨文學者邑之半;平公腓痛足痺而不敢壞坐,晉國之辭仕託者國之錘。此三士者,言襲法則官府之籍也,行中事則如令之民也,二君之禮太甚;若言離法而行遠功,則繩外民也,二君又何禮之,禮之當亡。且居學之士,國無事不用力,有難不被甲;禮之則惰修耕戰之功,不禮則周主上之法;國安則尊顯,危則為屈公之威;人主奚得於居學之士哉?故明王論李疵視中山也。

五、《詩》曰:「不躬不親,庶民不信。」傅說之以無衣紫,緩之以鄭簡、宋襄,責之以尊厚耕戰。夫不明分,不責誠,而以躬親位下,且為下走睡臥,與夫揜弊微服。孔丘不知,故稱猶盂。鄒君不知,故先自僇。明主之道,如叔向賦獵,與昭侯之奚聽也。

六、小信成則大信立,故明主積於信。賞罰不信,則禁令不行。說在文公之攻原與箕鄭救餓也。是以吳起須故人而食,文侯會虞人而獵。故明主表信,如曾子殺彘也。患在尊厲王擊警鼓與李悝謾兩和也。

右經

說一

宓子賤治單父,有若見之曰:「子何臞也?」宓子曰:「君不知賤不肖,使治單父,官事急,心憂之,故臞也。」有若曰:「昔者舜鼓五絃之琴,歌南風之詩而天下治。今以單父之細也,治之而憂,治天下將奈何乎?故有術而御之,身坐於廟堂之上,有處女子之色,無害於治;無術而御之,身雖瘁臞,猶未有益。」

楚王謂田鳩曰:「墨子者,顯學也。其身體則可,其言多而不辯何也?」曰:「昔秦伯嫁其女於晉公子,令晉為之飾裝,從衣文之媵七十人,至晉,晉人愛其妾而賤公女,此可謂善嫁妾而未可謂善嫁女也。楚人有賣其珠於鄭者,為木蘭之櫃,薰以桂椒,綴以珠玉,飾以玫瑰,輯以翡翠,鄭人買其櫝而還其珠,此可謂善賣櫝矣,未可謂善鬻珠也。今世之談也,皆道辯說文辭之言,人主覽其文而忘有用。墨子之說,傳先王之道,論聖人之言以宣告人,若辯其辭,則恐人懷其文忘其直,以文害用也。此與楚人鬻珠,秦伯嫁女同類,故其言多不辯。」

墨子為木鳶,三年而成,蜚一日而敗。弟子曰:「先生之巧,至能使木鳶飛。」墨子曰:「吾不如為車輗者巧也,用咫尺之木,不費一朝之事,而引三十石之任致遠,力多,久於歲數。今我為鳶,三年成,蜚一日而敗。」惠子聞之曰:「墨子大巧,巧為輗,拙為鳶。」

宋王與齊仇也,築武宮。謳癸倡,行者止觀,築者不倦,王聞召而賜之,對曰:「臣師射稽之謳又賢於癸。」王召射稽使之謳,行者不止,築者知倦,王曰:「行者不止,築者知倦,其謳不勝如癸美何也?」對曰:「王試度其功,癸四板,射稽八板;擿其堅,癸五寸,射稽二寸。」

夫良藥苦於口,而智者勸而飲之,知其入而已己疾也。忠言拂於耳,而明主聽之,知其可以致功也。

說二

宋人有請為燕王以棘刺之端為母猴者,必三月齋然後能觀之,燕王因以三乘養之。右御、治工言王曰:「臣聞人主無十日不燕之齋。今知王不能久齋以觀無用之器也,故以三月為期。凡刻削者,以其所以削必小。今臣治人也,無以為之削,此不然物也,王必察之。」王因囚而問之,果妄,乃殺之。治人謂王曰:「計無度量,言談之士多棘刺之說也。」

一曰。燕王好微巧,衛人曰:「能以棘刺之端為母猴。」燕王說之,養之以五乘之奉。王曰:「吾試觀客為棘刺之母猴。」客曰:「人主欲觀之,必半歲不入宮,不飲酒食肉,雨霽日出視之晏陰之間,而棘刺之母猴乃可見也。」燕王因養衛人不能觀其母猴。鄭有臺下之治者謂燕王曰:「臣為削者也,諸微物必以削削之,而所削必大於削。今棘刺之端不容削鋒,難以治棘刺之端。王試觀客之削能與不能可知也。」王曰:「善。」謂衛人曰:「客為棘削之?」曰:「以削。」王曰:「吾欲觀見之。」客曰:「臣請之舍取之。」因逃。

兒說,宋人,善辯者也。持白馬非馬也服齊稷下之辯者,乘白馬而過關,則顧白馬之賦。故籍之虛辭則能勝一國,考實按形不能謾於一人。

夫新砥礪殺矢,彀弩而射,雖冥而妄發,其端未嘗不中秋毫也,然而莫能復其處,不可謂善射,無常儀的也;設五寸之的,引十步之遠,非羿、逢蒙不能必全者,有常儀的也;有度難而無度易也。有常儀的則羿、逢蒙以五寸為巧,無常儀的則以妄發而中秋毫為拙,故無度而應之則辯士繁說,設度而持之雖知者猶畏失也不敢妄言。今人主聽說不應之以度,而說其辯不度以功,譽其行而不入關,此人主所以長欺、而說者所以長養也。

客有教燕王為不死之道者,王使人學之,所使學者未及學而客死。王大怒,誅之。王不知客之欺己,而誅學者之晚也。夫信不然之物,而誅無罪之臣,不察之患也。且人所急無如其身,不能自使其無死,安能使王長生哉?

鄭人有相與爭年者,一人曰:「吾與堯同年。」其一人曰:「我與黃帝之兄同年。」訟此而不決,以後息者為勝耳。

客有為周君畫莢者,三年而成,君觀之,與髹莢者同狀,周君大怒,畫莢者曰:「築十版之牆,鑿八尺之牖,而以日始出時加之其上而觀。」周君為之,望見其狀盡成龍蛇禽獸車馬,萬物之狀備具,周君大悅。此莢之功非不微難也,然其用與素髹筴同。

客有為齊王畫者,齊王問曰:「畫孰最難者?」曰:「犬馬最難。」「孰最易者?」曰:「鬼魅最易。夫犬馬、人所知也,旦暮罄於前,不可類之,故難。鬼魅、無形者,不罄於前,故易之也。」

齊有居士田仲者,宋人屈穀見之曰:「穀聞先生之義,不恃仰人而食。今穀有樹瓠之道,堅如石,厚而無竅,獻之。」仲曰:「夫瓠所貴者,謂其可以盛也。今厚而無竅,則不可剖以盛物,而任重如堅石,則不可以剖而以斟,吾無以瓠為也。」曰:「然,穀將棄之。今田仲不恃仰人而食,亦無益人之國,亦堅瓠之類也。」

虞慶為屋,謂匠人曰:「屋太尊。」匠人對曰:「此新屋也,塗濡而椽生。」虞慶曰:「不然。夫濡塗重而生椽撓,以撓椽任重塗,此宜卑。更日久則塗乾而椽燥,塗乾則輕,椽燥則直,以直椽任輕塗,此益尊。」匠人詘,為之而屋壞。

一曰。虞慶將為屋,匠人曰:「材生而塗濡。夫材生則撓,塗濡則重,以撓任重,今雖成,久必壞。」虞慶曰:「材乾則直,塗乾則輕,今誠得乾,日以輕直,雖久必不壞。」匠人詘,作之,成,有間,屋果壞。

范且曰:「弓之折必於其盡也,不於其始也。夫工人張弓也,伏檠三旬而蹈弦,一日犯機,是節之其始而暴之其盡也,焉得無折。」范且曰,「不然。伏檠一日而蹈弦,三旬而犯機,是暴之其始而節之其盡也。」工人窮也,為之,弓折。

范且、虞慶之言皆文辯辭勝而反事之情,人主說而不禁,此所以敗也。夫不謀治強之功,而豔乎辯說文麗之聲,是卻有術之士而任壞屋折弓也。故人主之於國事也,皆不達乎工匠之搆屋張弓也,然而士窮乎范且、虞慶者,為虛辭、其無用而勝,實事、其無易而窮也。人主多無用之辯,而少無易之言,此所以亂也。今世之為范且、虞慶者不輟,而人主說之不止,是貴敗折之類而以知術之人為工匠也。不得施其技巧,故屋壞弓折。知治之人不得行其方術,故國亂而主危。

夫嬰兒相與戲也,以塵為飯,以塗為羹,以木為胾,然至日晚必歸饟者,塵飯塗羹可以戲而不可食也。夫稱上古之傳頌,辯而不愨,道先王仁義而不能正國者,此亦可以戲而不可以為治也。夫慕仁義而弱亂者,三晉也;不慕而治強者,秦也;然而未帝者,治未畢也。

說三

人為嬰兒也,父母養之簡,子長而怨。子盛壯成人,其供養薄,父母怒而誚之。子、父,至親也,而或譙、或怨者,皆挾相為而不周於為己也。夫賣庸而播耕者,主人費家而美食、調布而求易錢者,非愛庸客也,曰:如是,耕者且深耨者熟耘也。庸客致力而疾耘耕者,盡巧而正畦陌畦畤者,非愛主人也,曰:如是,羹且美錢布且易云也。此其養功力,有父子之澤矣,而心調於用者,皆挾自為心也。故人行事施予,以利之為心,則越人易和;以害之為心,則父子離且怨。

文公伐宋,乃先宣言曰:「吾聞宋君無道,蔑侮長老,分財不中,教令不信,余來為民誅之。」

越伐吳,乃先宣言曰:「我聞吳王築如皇之臺,掘深池,罷苦百姓,煎靡財貨,以盡民力,余來為民誅之。」

蔡女為桓公妻,桓公與之乘舟,夫人蕩舟,桓公大懼,禁之不止,怒而出之,乃且復召之,因復更嫁之,桓公大怒,將伐蔡,仲父諫曰:「夫以寢席之戲,不足以伐人之國,功業不可冀也,請無以此為稽也。」桓公不聽,仲父曰:「必不得已,楚之菁茅不貢於天子三年矣,君不如舉兵為天子伐楚,楚服,因還襲蔡曰:余為天子伐楚而蔡不以兵聽從,因遂滅之。此義於名而利於實,故必有為天子誅之名,而有報讎之實。」

吳起為魏將而攻中山,軍人有病疽者,吳起跪而自吮其膿,傷者之母立泣,人問曰:「將軍於若子如是,尚何為而泣?」對曰:「吳起吮其父之創而父死,今是子又將死也,今吾是以泣。」

趙主父令工施鉤梯而緣播吾,刻疏人跡其上,廣三尺,長五尺,而勒之曰:「主父常遊於此。」

秦昭王令工施鉤梯而上華山,以松柏之心為博,箭長八尺,棋長八寸,而勒之曰「昭王嘗與天神博於此」矣。

文公反國,至河,令籩豆捐之,席蓐捐之,手足胼胝,面目黧黑者後之,咎犯聞之而夜哭,公曰:「寡人出亡二十年,乃今得反國,咎犯聞之不喜而哭,意不欲寡人反國邪?」犯對曰:「籩豆所以食也,席蓐所以臥也,而君捐之;手足胼胝、面目黧黑,勞有功者也,而君後之。今臣有與在後,中不勝其哀,故哭。且臣為君行詐偽以反國者眾矣,臣尚自惡也,而況於君?」再拜而辭,文公止之曰:「諺曰:築社者,攐撅而置之,端冕而祀之。今子與我取之,而不與我治之;與我置之,而不與我祀之;焉可?」解左驂而盟于河。

鄭縣人卜子,使其妻為褲,其妻問曰:「今褲何如?」夫曰:「象吾故苦。」妻子因毀新令如故褲。

鄭縣人有得車軛者,而不知其名,問人曰:「此何種也?」對曰:「此車軛也。」俄又復得一,問人曰:「此是何種也?」對曰:「此車軛也。」問者大怒曰:「曩者曰車軛,今又曰車軛,是何眾也?此女欺我也。」遂與之鬥。

衛人有佐弋者,鳥至,因先以其裷麾之,鳥驚而不射也。

鄭縣人卜子妻之市,買鱉以歸,過潁水,以為渴也,因縱而飲之,遂亡其鱉。

夫少者侍長者飲,長者飲亦自飲也。

一曰。魯人有自喜者,見長年飲酒不能釂則唾之,亦效唾之。

一曰。宋人有少者亦欲效善,見長者飲無餘,非斟酒飲也而欲盡之。

《書》曰:「紳之束之。」宋人有治者,因重帶自紳束也。人曰:「是何也?」對曰:「書言之,固然。」

《書》曰:「既雕既琢,還歸其樸。」梁人有治者,動作言學,舉事於文,曰難之,顧失其實,人曰:「是何也?」對曰:「書言之固然。」

郢人有遺燕相國書者,夜書,火不明,因謂持燭者曰:「舉燭。」云而過書舉燭,舉燭,非書意也,燕相受書而說之,曰:「舉燭者,尚明也,尚明也者,舉賢而任之。」燕相白王,王大說,國以治,治則治矣,非書意也。今世舉學者多似此類。

鄭人有且置履者,先自度其足而置之其坐,至之市而忘操之,已得履,乃曰:「吾忘持度。」反歸取之,及反,市罷,遂不得履,人曰:「何不試之以足?」曰:「寧信度,無自信也。」

說四

王登為中牟令,上言於襄主曰:「中牟有士曰中章、胥己者,其身甚修,其學甚博,君何不舉之?」主曰:「子見之,我將為中大夫。」相室諫曰:「中大夫,晉重列也,今無功而受,非晉臣之意。君其耳而未之目邪?」襄主曰:「我取登既耳而目之矣,登之所取又耳而目之,是耳目人絕無已也。」王登一日而見二中大夫,予之田宅,中牟之人棄其田耘、賣宅圃,而隨文學者邑之半。

叔向御坐平公請事,公腓痛足痺轉筋而不敢壞坐,晉國聞之,皆曰「叔向賢者,平公禮之,轉筋而不敢壞坐。」晉國之辭仕託、慕叔向者國之錘矣。

鄭縣人有屈公者,聞敵恐,因死;恐已,因生。

趙主父使李疵視中山可攻不也?還報曰:「中山可伐也,君不亟伐,將後齊、燕。」主父曰:「何故可攻?」李疵對曰:「其君見好巖穴之士,所傾蓋與車以見窮閭隘巷之士以十數,伉禮下布衣之士以百數矣。」君曰:「以子言論,是賢君也,安可攻?」疵曰:「不然。夫好顯巖穴之士而朝之,則戰士怠於行陣;上尊學者,下士居朝,則農夫惰於田。戰士怠於行陳者則兵弱也,農夫惰於田者則國貧也。兵弱於敵,國貧於內,而不亡者,未之有也,伐之不亦可乎?」主父曰:「善。」舉兵而伐中山,遂滅也。

說五

齊桓公好服紫,一國盡服紫,當是時也,五素不得一紫,桓公患之,謂管仲曰:「寡人好服紫,紫貴甚,一國百姓好服紫不已,寡人奈何?」管仲曰:「君欲何不試勿衣紫也,謂左右曰,吾甚惡紫之臭。」於是左右適有衣紫而進者,公必曰:「少卻,吾惡紫臭。」公曰:「諾。」於是日郎中莫衣紫,其明日國中莫衣紫,三日境內莫衣紫也。

一曰。齊王好衣紫,齊人皆好也。齊國五素不得一紫,齊王患紫貴。傅說王曰:「《詩》云:不躬不親,庶民不信。今王欲民無衣紫者,王以自解紫衣而朝,群臣有紫衣進者,曰益遠,寡人惡臭。」是日也,郎中莫衣紫;是月也,國中莫衣紫;是歲也,境內莫衣紫。

鄭簡公謂子產曰:「國小,迫於荊、晉之間。今城郭不完,兵甲不備,不可以待不虞。」子產曰:「臣閉其外也已遠矣,而守其內也已固矣,雖國小猶不危之也。君其勿憂。」是以沒簡公身無患。

子產相鄭,簡公謂子產曰:「飲酒不樂也,俎豆不大,鍾鼓竽瑟不鳴,寡人之事不一,國家不定,百姓不治,耕戰不輯睦,亦子之罪。子有職,寡人亦有職,各守其職。」子產退而為政五年,國無盜賊,道不拾遺,桃棗蔭於街者莫有援也,錐刀遺道三日可反,三年不變,民無飢也。

宋襄公與楚人戰於涿谷上,宋人既成列矣,楚人未及濟,右司馬購強趨而諫曰:「楚人眾而宋人寡,請使楚人半涉未成列而擊之,必敗。」襄公曰:「寡人聞君子曰:不重傷,不擒二毛,不推人於險,不迫人於阨,不鼓不成列。今楚未濟而擊之,害義。請使楚人畢涉成陣而後鼓士進之。」右司馬曰:「君不愛宋民,腹心不完,特為義耳。」公曰:「不反列,且行法。」右司馬反列,楚人已成列撰陣矣,公乃鼓之,宋人大敗,公傷股,三日而死,此乃慕自親仁義之禍。夫必恃人主之自躬親而後民聽從,是則將令人主耕以為上,服戰鴈行也民乃肯耕戰,則人主不泰危乎?而人臣不泰安乎?

齊景公游少海,傳騎從中來謁曰:「嬰疾甚,且死,恐公後之。」景公遽起,傳騎又至。景公曰:「趨駕煩且之乘,使騶子韓樞御之。」行數百步,以騶為不疾,奪轡代之,御可數百步,以馬為不進,盡釋車而走。以煩且之良,而騶子韓樞之巧,而以為不如下走也。

魏昭王欲與官事,謂孟嘗君曰:「寡人欲與官事。」君曰:「王欲與官事,則何不試習讀法?」昭王讀法十餘簡而睡臥矣,王曰:「寡人不能讀此法。」夫不躬親其勢柄,而欲為人臣所宜為者也,睡不亦宜乎。

孔子曰:「為人君者猶盂也,民猶水也,盂方水方,盂圜水圜。」

鄒君好服長纓,左右皆服長纓,纓甚貴,鄒君患之,問左右,左右曰:「君好服,百姓亦多服,是以貴。」君因先自斷其纓而出,國中皆不服長纓。君不能下令為百姓服度以禁之,乃斷纓出以示民,是先戮以蒞民也。

叔向賦獵,功多者受多,功少者受少。

韓昭侯謂申子曰:「法度甚易行也。」申子曰:「法者見功而與賞,因能而受官。今君設法度而聽左右之請,此所以難行也。」昭侯曰:「吾自今以來知行法矣,寡人奚聽矣。」一日,申子請仕其從兄官,昭侯曰:「非所學於子也。聽子之謁敗子之道乎?亡其用子之謁。」申子辟舍請罪。

說六

晉文公攻原,裹十日糧,遂與大夫期十日,至原十日而原不下,擊金而退,罷兵而去,士有從原中出者曰:「原三日即下矣。」群臣左右諫曰:「夫原之食竭力盡矣,君姑待之。」公曰:「吾與士期十日,不去,是亡吾信也。得原失信,吾不為也。」遂罷兵而去。原人聞曰:「有君如彼其信也,可無歸乎?」乃降公。衛人聞曰:「有君如彼其信也,可無從乎?」乃降公。孔子聞而記之曰:「攻原得衛者信也。」

文公問箕鄭曰:「救餓奈何?」對曰:「信。」公曰:「安信?」曰:「信名。信名,則群臣守職,善惡不踰,百事不怠。信事,則不失天時,百姓不踰。信義,則近親勸勉而遠者歸之矣。」

吳起出,遇故人而止之食,故人曰:「諾,今返而御。」吳子曰:「待公而食。」故人至暮不來,起不食待之,明日早,令人求故人,故人來方與之食。

魏文侯與虞人期獵,明日,會天疾風,左右止,文侯不聽,曰:「不可。以風疾之故而失信,吾不為也。」遂自驅車往,犯風而罷虞人。

曾子之妻之市,其子隨之而泣,其母曰:「女還,顧反為女殺彘。」妻適市來,曾子欲捕彘殺之,妻止之曰:「特與嬰兒戲耳。」曾子曰:「嬰兒非與戲也。嬰兒非有知也,待父母而學者也,聽父母之教,今子欺之,是教子欺也。母欺子,子而不信其母,非所以成教也。」遂烹彘也。

楚厲王有警,為鼓以與百姓為戍,飲酒醉,過而擊之也,民大驚,使人止之。曰:「吾醉而與左右戲,過擊之也。」民皆罷。居數月,有警,擊鼓而民不赴,乃更令明號而民信之。

李悝警其兩和曰:「謹警敵人,旦暮且至擊汝。」如是者再三而敵不至,兩和懈怠,不信李悝,居數月,秦人來襲之,至,幾奪其軍,此不信患也。

一曰。李悝與秦人戰,謂左和曰:「速上,右和已上矣。」又馳而至右和曰:「左和已上矣。」左右和曰:「上矣。」於是皆爭上。其明年與秦人戰,秦人襲之,至,幾奪其軍,此不信之患。

有相與訟者,子產離之而毋得使通辭,到至其言以告而知也。

惠嗣公使人偽關市,關市呵難之,因事關市以金,關市乃舍之,嗣公謂關市曰:「某時有客過而予汝金,因譴之。」關市大恐,以嗣公為明察。


\end{pinyinscope}