\article{說疑}

\begin{pinyinscope}
凡治之大者,非謂其賞罰之當也。賞無功之人,罰不辜之民,非所謂明也。賞有功,罰有罪,而不失其人,方在於人者也,非能生功止過者也。是故禁姦之法,太上禁其心,其次禁其言,其次禁其事。今世皆曰「尊主安國者,必以仁義智能」,而不知卑主危國者之必以仁義智能也。故有道之主,遠仁義,去智能,服之以法。是以譽廣而名威,民治而國安,知用民之法也。凡術也者,主之所以執也;法也者,官之所以師也。然使郎中日聞道於郎門之外,以至於境內日見法,又非其難者也。

昔者有扈氏有失度,讙兜氏有孤男,三苗有成駒,桀有侯侈,紂有崇侯虎,晉有優施,此六人者,亡國之臣也。言是如非,言非如是,內險以賊其外,小謹以徵其善,稱道往古、使良事沮,善禪其主、以集精微,亂之以其所好,此夫郎中左右之類者也。往世之主,有得人而身安國存者,有得人而身危國亡者,得人之名一也,而利害相千万也,故人主左右不可不慎也。為人主者誠明於臣之所言,則別賢不肖如黑白矣。

若夫許由、續牙、晉伯陽、秦顛頡、衛僑如、狐不稽、重明、董不識、卞隨、務光、伯夷、叔齊,此十二人者,皆上見利不喜,下臨難不恐,或與之天下而不取,有萃辱之名,則不樂食穀之利。夫見利不喜,上雖厚賞無以勸之;臨難不恐,上雖嚴刑無以威之;此之謂不令之民也。此十二人者,或伏死於窟穴,或槁死於草木,或飢餓於山谷,或沉溺於水泉。有民如此,先古聖王皆不能臣,當今之世,將安用之?

若夫關龍逢、王子比干、隨季梁、陳泄冶、楚申胥、吳子胥,此六人者,皆疾爭強諫以勝其君。言聽事行,則如師徒之勢;一言而不聽,一事而不行,則陵其主以語,待之以其身,雖死家破,要領不屬,手足異處,不難為也。如此臣者,先古聖王皆不能忍也,當今之時,將安用之?

若夫齊田恆、宋子罕、魯季孫意如、晉僑如、衛子南勁、鄭太宰欣、楚白公、周單荼、燕子之,此九人者之為其臣也,皆朋黨比周以事其君,隱正道而行私曲,上逼君,下亂治,援外以撓內、親下以謀上,不難為也。如此臣者,唯聖王智主能禁之,若夫昏亂之君,能見之乎?

若夫后稷、皋陶、伊尹、周公旦、太公望、管仲、隰朋、百里奚、蹇叔、舅犯、趙衰、范蠡、大夫種、逢同、華登,此十五人者為其臣也,皆夙興夜寐,卑身賤體,竦心白意,明刑辟、治官職以事其君,進善言、通道法而不敢矜其善,有成功立事而不敢伐其勞,不難破家以便國,殺身以安主,以其主為高天泰山之尊,而以其身為壑谷釜洧之卑,主有明名廣譽於國,而身不難受壑谷釜洧之卑。如此臣者,雖當昏亂之主尚可致功,況於顯明之主乎?此謂霸王之佐也。

若夫周滑之、鄭王孫申、陳公孫寧、儀行父、荊芋尹申亥、隨少師越、種干、吳王孫額、晉陽成泄、齊豎刁、易牙,此十二人者之為其臣也,皆思小利而忘法義,進則揜蔽賢良以陰闇其主,退則撓亂百官而為禍難,皆輔其君、共其欲,苟得一說於主,雖破國殺眾不難為也。有臣如此,雖當聖王尚恐奪之,而況昏亂之君,其能無失乎?有臣如此者,皆身死國亡,為天下笑。故周威公身殺,國分為二;鄭子陽身殺,國分為三;陳靈公身死於夏徵舒氏;荊靈王死於乾谿之上;隨亡於荊;吳併於越;智伯滅於晉陽之下;桓公身死七日不收。故曰,諂諛之臣,唯聖王知之,而亂主近之,故至身死國亡。

聖王明君則不然,內舉不避親,外舉不避讎。是在焉從而舉之,非在焉從而罰之。是以賢良遂進而姦邪并退,故一舉而能服諸侯。其在記曰:「堯有丹朱,而舜有商均,啟有五觀,商有太甲,武王有管、蔡」,五王之所誅者,皆父兄子弟之親也,而所殺亡其身殘破其家者何也?以其害國傷民敗法類也。觀其所舉,或在山林藪澤巖穴之間,或在囹圄緤紲纏索之中,或在割烹芻牧飯牛之事。然明主不羞其卑賤也,以其能、為可以明法,便國利民,從而舉之,身安名尊。

亂主則不然,不知其臣之意行,而任之以國。故小之名卑地削,大之國亡身死,不明於用臣也。無數以度其臣者,必以其眾人之口斷之。眾之所譽,從而說之;眾之所非,從而憎之。故為人臣者破家殘賥,內構黨與,外接巷族以為譽,從陰約結以相固也,虛相與爵祿以相勸也。曰:「與我者將利之,不與我者將害之。」眾貪其利,劫其威。彼誠喜、則能利己,忌怒、則能害己。眾歸而民留之,以譽盈於國,發聞於主,主不能理其情,因以為賢。彼又使譎詐之士,外假為諸侯之寵使,假之以輿馬,信之以瑞節,鎮之以辭令,資之以幣帛,使諸侯淫說其主,微挾私而公議。所為使者,異國之主也,所為談者,左右之人也。主說其言而辯其辭,以此人者天下之賢士也。內外之於左右,其諷一而語同,大者不難卑身尊位以下之,小者高爵重祿以利之。夫姦人之爵祿重而黨與彌眾,又有姦邪之意,則姦臣愈反而說之,曰:「古之所謂聖君明王者,非長幼弱也及以次序也。以其搆黨與,聚巷族,偪上弒君而求其利也。」彼曰:「何知其然也?」因曰:「舜偪堯,禹偪舜,湯放桀,武王伐紂,此四王者,人臣弒其君者也,而天下譽之。察四王之情,貪得人之意也;度其行,暴亂之兵也。然四王自廣措也,而天下稱大焉;自顯名也,而天下稱明焉。則威足以臨天下,利足以蓋世,天下從之。」又曰:「以今時之所聞田成子取齊,司城子罕取宋,太宰欣取鄭,單氏取周,易牙之取衛,韓、魏、趙三子分晉,此六人,臣之弒其君者也。」姦臣聞此,蹙然舉耳以為是也。故內搆黨與,外攄巷族,觀時發事,一舉而取國家。且夫內以黨與劫弒其君,外以諸侯之權矯易其國,隱敦適,持私曲,上禁君,下撓治者,不可勝數也。是何也?則不明於擇臣也。記曰:「周宣王以來,亡國數十,其臣弒其君而取國者眾矣。」然則難之從內起,與從外作者相半也。能一盡其民力,破國殺身者,尚皆賢主也。若夫轉法易位,全眾傳國,最其病也。

為人主者,誠明於臣之所言,則雖罼弋馳騁,撞鐘舞女,國猶且存也。不明臣之所言,雖節儉勤勞,布衣惡食,國猶自亡也。趙之先君敬侯,不修德行,而好縱慾,適身體之所安,耳目之所樂,冬日罼弋,夏浮淫,為長夜,數日不廢御觴,不能飲者以筩灌其口,進退不肅、應對不恭者斬於前。故居處飲食如此其不節也,制刑殺戮如此其無度也,然敬侯享國數十年,兵不頓於敵國,地不虧於四鄰,內無君臣百官之亂,外無諸侯鄰國之患,明於所以任臣也。燕君子噲,邵公奭之後也,地方數千里,持戟數十萬,不安子女之樂,不聽鍾石之聲,內不湮汙池臺榭,外不罼弋田獵,又親操耒耨以修畎畝,子噲之苦身以憂民如此其甚也,雖古之所謂聖王明君者,其勤身而憂世不甚於此矣。然而子噲身死國亡,奪於子之,而天下笑之,此其何故也?不明乎所以任臣也。故曰:人臣有五姦,而主不知也。為人臣者,有侈用財貨賂以取譽者,有務慶賞賜予以移眾者,有務朋黨徇智尊士以擅逞者,有務解免赦罪獄以事威者,有務奉下直曲、怪言偉服瑰稱、以眩民耳目者。此五者明君之所疑也,而聖主之所禁也。去此五者,則譟詐之人不敢北面談立,文言多、實行寡、而不當法者不敢誣情以談說。是以群臣居則修身,動則任力,非上之令、不敢擅作疾言誣事,此聖王之所以牧臣下也。彼聖主明君,不適疑物以闚其臣也。見疑物而無反者,天下鮮矣。故曰:孽有擬適之子,配有擬妻之妾,廷有擬相之臣,臣有擬主之寵,此四者國之所危也。故曰:內寵並后,外寵貳政,枝子配適,大臣擬主,亂之道也。故周記曰:「無尊妾而卑妻,無孽適子而尊小枝,無尊嬖臣而匹上卿,無尊大臣以擬其主也。」四擬者破,則上無意、下無怪也。四擬不破,則隕身滅國矣。


\end{pinyinscope}