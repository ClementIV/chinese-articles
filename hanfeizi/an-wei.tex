\article{安危}

\begin{pinyinscope}
安術有七,危道有六。

安術:一曰、賞罰隨是非,二曰、禍福隨善惡,三曰、死生隨法度,四曰、有賢不肖而無愛惡,五曰、有愚智而無非譽,六曰、有尺寸而無意度,七曰、有信而無詐。

危道:一曰、斲削於繩之內,二曰、斷割於法之外,三曰、利人之所害,四曰、樂人之所禍,五曰、危人於所安,六曰、所愛不親,所惡不疏。如此,則人失其所以樂生,而忘其所以重死,人不樂生則人主不尊,不重死則令不行也。

使天下皆極智能於儀表,盡力於權衡,以動則勝,以靜則安。治世使人樂生於為是,愛身於為非。小人少而君子多,故社稷常立,國家久安。奔車之上無仲尼,覆舟之下無伯夷。故號令者,國之舟車也。安則智廉生,危則爭鄙起。故安國之法,若饑而食,寒而衣,不令而自然也。先王寄理於竹帛,其道順,故後世服。今使人去饑寒,雖賁、育不能行;廢自然,雖順道而不立。強勇之所不能行,則上不能安。上以無厭責,己盡,則下對無有,無有則輕法,法所以為國也而輕之,則功不立、名不成。聞古扁鵲之治其病也,以刀刺骨;聖人之救危國也,以忠拂耳。刺骨,故小痛在體而長利在身;拂耳,故小逆在心而久福在國。故甚病之人利在忍痛,猛毅之君以福拂耳。忍痛,故扁鵲盡巧;拂耳,則子胥不失;壽安之術也。病而不忍痛,則失扁鵲之巧;危而不拂耳,則失聖人之意。如此,長利不遠垂,功名不久立。

人主不自刻以堯而責人臣以子胥,是幸殷人之盡如比干,盡如比干則上不失、下不亡。不權其力而有田成,而幸其身盡如比干,故國不得一安。廢堯、舜而立桀、紂,則人不得樂所長而憂所短。失所長則國家無功,守所短則民不樂生,以無功御不樂生,不可行於齊民。如此,則上無以使下,下無以事上。

安危在是非,不在於強弱。存亡在虛實,不在於眾寡。故齊、萬乘也,而名實不稱,上空虛於國內,不充滿於名實,故臣得奪主。殺天子也,而無是非,賞於無功;使讒諛,以詐偽為貴;誅於無罪,使傴以天性剖背;以詐偽為是,天性為非,小得勝大。

明主堅內,故不外失。失之近而不亡於遠者無有。故周之奪殷也,拾遺於庭,使殷不遺於朝,則周不敢望秋毫於境,而況敢易位乎。

明主之道忠法,其法忠心,故臨之而法,去之而思。堯無膠漆之約於當世而道行,舜無置錐之地於後世而德結。能立道於往古,而垂德於萬世者之謂明主。


\end{pinyinscope}