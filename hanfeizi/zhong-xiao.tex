\article{忠孝}

\begin{pinyinscope}
天下皆以孝悌忠順之道為是也,而莫知察孝悌忠順之道而審行之,是以天下亂。皆以堯、舜之道為是而法之,是以有弒君,有曲於父。堯、舜、湯、武,或反君臣之義,亂後世之教者也。堯為人君而君其臣,舜為人臣而臣其君,湯、武為人臣而弒其主、刑其尸,而天下譽之,此天下所以至今不治者也。夫所謂明君者,能畜其臣者也;所謂賢臣者,能明法辟、治官職以戴其君者也。今堯自以為明而不能以畜舜,舜自以為賢而不能以戴堯,湯、武自以為義而弒其君長,此明君且常與,而賢臣且常取也。故至今為人子者有取其父之家,為人臣者有取其君之國者矣。父而讓子,君而讓臣,此非所以定位一教之道也。臣之所聞曰:「臣事君,子事父,妻事夫,三者順則天下治,三者逆則天下亂,此天下之常道也,明王賢臣而弗易也。」則人主雖不肖,臣不敢侵也。今夫上賢任智無常,逆道也;而天下常以為治,是故田氏奪呂氏於齊,戴氏奪子氏於宋,此皆賢且智也,豈愚且不肖乎?是廢常、上賢則亂,舍法、任智則危。故曰:「上法而不上賢。」

記曰:「舜見瞽瞍,其容造焉。孔子曰:當是時也,危哉!天下岌岌,有道者、父固不得而子,君固不得而臣也。」臣曰:孔子本未知孝悌忠順之道也。然則有道者,進不為臣主,退不為父子耶?父之所以欲有賢子者,家貧則富之,父苦則樂之;君之所以欲有賢臣者,國亂則治之,主卑則尊之。今有賢子而不為父,則父之處家也苦;有賢臣而不為君,則君之處位也危。然則父有賢子,君有賢臣,適足以為害耳,豈得利哉!焉所謂忠臣不危其君,孝子不非其親?今舜以賢取君之國,而湯、武以義放弒其君,此皆以賢而危主者也,而天下賢之。古之烈士,進不臣君,退不為家,是進則非其君,退則非其親者也。且夫進不臣君,退不為家,亂世絕嗣之道也。是故賢堯、舜、湯、武而是烈士,天下之亂術也。瞽瞍為舜父而舜放之,象為舜弟而殺之。放父殺弟,不可謂仁;妻帝二女而取天下,不可謂義。仁義無有,不可謂明。《詩》云:「普天之下,莫非王土,率土之濱,莫非王臣。」信若詩之言也,是舜出則臣其君,入則臣其父、妾其母、妻其主女也。故烈士內不為家,亂世絕嗣;而外矯於君,朽骨爛肉,施於土地,流於川谷,不避蹈水火,使天下從而效之,是天下遍死而願夭也,此皆釋世而不治是也。世之所為烈士者,雖眾獨行,取異於人,為恬淡之學而理恍惚之言。臣以為恬淡,無用之教也;恍惚,無法之言也。言出於無法,教出於無用者,天下謂之察。臣以為人生必事君養親,事君養親不可以恬淡;之人必以言論忠信法術,言論忠信法術不可以恍惚。恍惚之言,恬淡之學,天下之惑術也。孝子之事父也,非競取父之家也;忠臣之事君也,非競取君之國也。夫為人子而常譽他人之親曰:「某子之親,夜寢早起,強力生財以養子孫臣妾」,是誹謗其親者也。為人臣常譽先王之德厚而願之,是誹謗其君者也。非其親者知謂之不孝,而非其君者天下此賢之,此所以亂也。故人臣毋稱堯、舜之賢,毋譽湯、武之伐,毋言烈士之高,盡力守法,專心於事主者為忠臣。

古者黔首悗密惷愚,故可以虛名取也。今民儇詗智慧,欲自用,不聽上,上必且勸之以賞然後可進,又且畏之以罰然後不敢退。而世皆曰:「許由讓天下,賞不足以勸;盜跖犯刑赴難,罰不足以禁。」臣曰:未有天下而無以天下為者許由是也,已有天下而無以天下為者堯、舜是也;毀廉求財,犯刑趨利,忘身之死者,盜跖是也。此二者殆物也,治國用民之道也不以此二者為量。治也者,治常者也;道也者,道常者也。殆物妙言,治之害也。天下太平之士,不可以賞勸也;天下太平之士,不可以刑禁也。然為太上士不設賞,為太下士不設刑,則治國用民之道失矣。故世人多不言國法而言從橫。諸侯言從者曰:「從成必霸」,而言橫者曰「橫成必王」,山東之言從橫未嘗一日而止也,然而功名不成,霸王不立者,虛言非所以成治也。王者獨行謂之王,是以三王不務離合而正,五霸不待從橫而察,治內以裁外而已矣。


\end{pinyinscope}