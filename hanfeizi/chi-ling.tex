\article{飭令}

\begin{pinyinscope}
飭令則法不遷,法平則吏無姦。法已定矣,不以善言售法。任功則民少言,任善則民多言。行法曲斷,以五里斷者王,以九里斷者強,宿治者削。

以刑治,以賞戰,厚祿以用術。行都之過,則都無姦市。物多末眾,農弛姦勝,則國必削。民有餘食,使以粟出,爵必以其力,則震不怠。三寸之管毋當,不可滿也。授官爵、出利祿不以功,是無當也。國以功授官與爵,此謂以成智謀,以威勇戰,其國無敵。國以功授官與爵,則治見者省,言有塞,此謂以治去治,以言去言。以功與爵者也故國多力,而天下莫之能侵也。兵出必取,取必能有之;案兵不攻必當。朝廷之事,小者不毀,效功取官爵,廷雖有辟言,不得以相干也,是謂以數治。以力攻者,出一取十;以言攻者,出十喪百。國好力,此謂以難攻;國好言,此謂以易攻。其能,勝其害,輕其任,而道壞餘力於心,莫負乘宮之責於君,內無伏怨,使明者不相干,故莫訟;使士不兼官,故技長;使人不同功,故莫爭。言此謂易攻。

重刑少賞,上愛民,民死賞。多賞輕刑,上不愛民,民不死賞。利出一空者,其國無敵;利出二空者,其兵半用;利出十空者民不守。重刑明民大制使人則上利。行刑、重其輕者,輕者不至,重者不來,此謂以刑去刑。罪重而刑輕,刑輕則事生,此謂以刑致刑,其國必削。


\end{pinyinscope}