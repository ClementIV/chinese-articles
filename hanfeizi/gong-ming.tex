\article{功名}

\begin{pinyinscope}
明君之所以立功成名者四:一曰天時,二曰人心,三曰技能,四曰勢位。非天時雖十堯不能冬生一穗,逆人心雖賁、育不能盡人力。故得天時則不務而自生,得人心則不趣而自勸,因技能則不急而自疾,得勢位則不進而名成。若水之流,若船之浮,守自然之道,行毋窮之令,故曰明主。

夫有材而無勢,雖賢不能制不肖。故立尺材於高山之上,則臨千仞之谿,材非長也,位高也。桀為天子,能制天下,非賢也,勢重也;堯為匹夫,不能正三家,非不肖也,位卑也。千鈞得船則浮,錙銖失船則沈,非千鈞輕錙銖重也,有勢之與無勢也。故短之臨高也以位,不肖之制賢也以勢。人主者,天下一力以共載之,故安;眾同心以共立之,故尊。人臣守所長,盡所能,故忠。以尊主主御忠臣,則長樂生而功名成。名實相持而成,形影相應而立,故臣主同欲而異使。人主之患在莫之應,故曰:一手獨拍,雖疾無聲。人臣之憂在不得一,故曰:右手畫圓,左手畫方,不能兩成。故曰:至治之國,君若桴,臣若鼓,技若車,事若馬。故人有餘力易於應,而技有餘巧便於事。立功者不足於力,親近者不足於信,成名者不足於勢。近者已親,而遠者不結,則名不稱實者也。聖人德若堯、舜,行若伯夷,而位不載於世,則功不立,名不遂。故古之能致功名者,眾人助之以力,近者結之以成,遠者譽之以名,尊者載之以勢。如此,故太山之功長立於國家,而日月之名久著於天地。此堯之所以南面而守名,舜之所以北面而效功也。


\end{pinyinscope}