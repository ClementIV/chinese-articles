\article{孤憤}

\begin{pinyinscope}
智術之士,必遠見而明察,不明察不能燭私;能法之士,必強毅而勁直,不勁直不能矯姦。人臣循令而從事,案法而治官,非謂重人也。重人也者,無令而擅為,虧法以利私,耗國以便家,力能得其君,此所為重人也。智術之士,明察聽用,且燭重人之陰情;能法之士,勁直聽用,且矯重人之姦行。故智術能法之士用,則貴重之臣必在繩之外矣。是智法之士與當塗之人,不可兩存之仇也。

當塗之人擅事要,則外內為之用矣。是以諸侯不因則事不應,故敵國為之訟。百官不因則業不進,故群臣為之用。郎中不因則不得近主,故左右為之匿。學士不因則養祿薄禮卑,故學士為之談也。此四助者,邪臣之所以自飾也。重人不能忠主而進其仇,人主不能越四助而燭察其臣,故人主愈弊,而大臣愈重。凡當塗者之於人主也,希不信愛也,又且習故。若夫即主心同乎好惡,固其所自進也。官爵貴重,朋黨又眾,而一國為之訟。則法術之士欲干上者,非有所信愛之親,習故之澤也;又將以法術之言矯人主阿辟之心,是與人主相反也。處勢卑賤,無黨孤特。夫以疏遠與近愛信爭,其數不勝也;以新旅與習故爭,其數不勝也;以反主意與同好爭,其數不勝也;以輕賤與貴重爭,其數不勝也;以一口與一國爭,其數不勝也。法術之士,操五不勝之勢,以歲數而又不得見;當塗之人,乘五勝之資,而旦暮獨說於前;故法術之士,奚道得進,而人主奚時得悟乎?故資必不勝而勢不兩存,法術之士焉得不危?其可以罪過誣者,以公法而誅之;其不可被以罪過者,以私劍而窮之。是明法術而逆主上者,不僇於吏誅,必死於私劍矣。

朋黨比周以弊主,言曲以便私者,必信於重人矣。故其可以功伐借者,以官爵貴之;其不可借以美名者,以外權重之。是以弊主上而趨於私門者,不顯於官爵,必重於外權矣。今人主不合參驗而行誅,不待見功而爵祿,故法術之士安能蒙死亡而進其說,姦邪之臣安肯乘利而退其身?故主上愈卑,私門益尊。夫越雖國富兵彊,中國之主皆知無益於己也,曰:「非吾所得制也。」今有國者雖地廣人眾,然而人主壅蔽,大臣專權,是國為越也。智不類越,而不智不類其國,不察其類者也。人主所以謂齊亡者,非地與城亡也,呂氏弗制,而田氏用之。所以謂晉亡者,亦非地與城亡也,姬氏不制,而六卿專之也。今大臣執柄獨斷,而上弗知收,是人主不明也。與死人同病者,不可生也;與亡國同事者,不可存也。今襲跡於齊、晉,欲國安存,不可得也。

凡法術之難行也,不獨萬乘,千乘亦然。人主之左右不必智也,人主於人有所智而聽之,因與左右論其言,是與愚人論智也。人主之左右不必賢也,人主於人有所賢而禮之,因與左右論其行,是與不肖論賢也。智者決策於愚人,賢士程行於不肖,則賢智之士羞而人主之論悖矣。人臣之欲得官者,其修士且以精絜固身,其智士且以治辯進業。其修士不能以貨賂事人,恃其精潔,而更不能以枉法為治,則修智之士,不事左右,不聽請謁矣。人主之左右,行非伯夷也,求索不得,貨賂不至,則精辯之功息,而毀誣之言起矣。治辯之功制於近習,精潔之行決於毀譽,則修智之吏廢,則人主之明塞矣。不以功伐決智行,不以參伍審罪過,而聽左右近習之言,則無能之士在廷,而愚污之吏處官矣。

萬乘之患,大臣太重;千乘之患,左右太信;此人主之所公患也。且人臣有大罪,人主有大失,臣主之利與相異者也。何以明之哉?曰:主利在有能而任官,臣利在無能而得事;主利在有勞而爵祿,臣利在無功而富貴;主利在豪傑使能,臣利在朋黨用私。是以國地削而私家富,主上卑而大臣重。故主失勢而臣得國,主更稱蕃臣,而相室剖符,此人臣之所以譎主便私也。故當世之重臣,主變勢而得固寵者,十無二三。是其故何也?人臣之罪大也。臣有大罪者,其行欺主也,其罪當死亡也。智士者遠見,而畏於死亡,必不從重人矣。賢士者修廉,而羞與姦臣欺其主,必不從重人矣。是當塗者之徒屬,非愚而不知患者,必污而不避姦者也。大臣挾愚污之人,上與之欺主,下與之收利侵漁,朋黨比周,相與一口,惑主敗法,以亂士民,使國家危削,主上勞辱,此大罪也。臣有大罪而主弗禁,此大失也。使其主有大失於上,臣有大罪於下,索國之不亡者,不可得也。


\end{pinyinscope}