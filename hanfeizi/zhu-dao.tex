\article{主道}

\begin{pinyinscope}
道者、萬物之始,是非之紀也。是以明君守始以知萬物之源,治紀以知善敗之端。故虛靜以待令,令名自命也,令事自定也。虛則知實之情,靜則知動者正。有言者自為名,有事者自為形,形名參同,君乃無事焉,歸之其情。故曰:君無見其所欲,君見其所欲,臣自將雕琢;君無見其意,君見其意,臣將自表異。故曰:去好去惡,臣乃見素,去舊去智,臣乃自備。故有智而不以慮,使萬物知其處;有行而不以賢,觀臣下之所因;有勇而不以怒,使群臣盡其武。是故去智而有明,去賢而有功,去勇而有強。群臣守職,百官有常,因能而使之,是謂習常。故曰:寂乎其無位而處,漻乎莫得其所。明君無為於上,群臣竦懼乎下。明君之道,使智者盡其慮,而君因以斷事,故君不窮於智;賢者敕其材,君因而任之,故君不窮於能;有功則君有其賢,有過則臣任其罪,故君不窮於名。是故不賢而為賢者師,不智而為智者正。臣有其勞,君有其成功,此之謂賢主之經也。

道在不可見,用在不可知。虛靜無事,以闇見疵。見而不見,聞而不聞,知而不知。知其言以往,勿變勿更,以參合閱焉。官有一人,勿令通言,則萬物皆盡。函;掩其跡,匿其端,下不能原;去其智,絕其能,下不能意。保吾所以往而稽同之,謹執其柄而固握之。絕其能望,破其意,毋使人欲之。不謹其閉,不固其門,虎乃將存。不慎其事,不掩其情,賊乃將生。弒其主,代其所,人莫不與,故謂之虎。處其主之側,為姦臣,聞其主之忒,故謂之賊。散其黨,收其餘,閉其門,奪其輔,國乃無虎。大不可量,深不可測,同合刑名,審驗法式,擅為者誅,國乃無賊。是故人主有五壅:臣閉其主曰壅,臣制財利曰壅,臣擅行令曰壅,臣得行義曰壅,臣得樹人曰壅。臣閉其主則主失位,臣制財利則主失德,臣擅行令則主失制,臣得行義則主失明,臣得樹人則主失黨。此人主之所以獨擅也,非人臣之所以得操也。

人主之道,靜退以為寶。不自操事而知拙與巧,不自計慮而知福與咎。是以不言而善應,不約而善增。言已應則執其契,事已增則操其符。符契之所合,賞罰之所生也。故群臣陳其言,君以其言授其事,事以責其功。功當其事,事當其言則賞;功不當其事,事不當其言則誅。明君之道,臣不陳言而不當。是故明君之行賞也,曖乎如時雨,百姓利其澤;其行罰也,畏乎如雷霆,神聖不能解也。故明君無偷賞,無赦罰。賞偷則功臣墮其業,赦罰則姦臣易為非。是故誠有功則雖疏賤必賞,誠有過則雖近愛必誅。近愛必誅,則疏賤者不怠,而近愛者不驕也。


\end{pinyinscope}