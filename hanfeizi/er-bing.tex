\article{二柄}

\begin{pinyinscope}
明主之所導制其臣者,二柄而已矣。二柄者,刑、德也。何謂刑德?曰:殺戮之謂刑,慶賞之謂德。為人臣者畏誅罰而利慶賞,故人主自用其刑德,則群臣畏其威而歸其利矣。故世之姦臣則不然,所惡則能得之其主而罪之,所愛則能得之其主而賞之。今人主非使賞罰之威利出於己也,聽其臣而行其賞罰,則一國之人皆畏其臣而易其君,歸其臣而去其君矣,此人主失刑德之患也。夫虎之所以能服狗者、爪牙也,使虎釋其爪牙而使狗用之,則虎反服於狗矣。人主者、以刑德制臣者也,今君人者、釋其刑德而使臣用之,則君反制於臣矣。故田常上請爵祿而行之群臣,下大斗斛而施於百姓,此簡公失德而田常用之也,故簡公見弒。子罕謂宋君曰:「夫慶賞賜予者,民之所喜也,君自行之;殺戮刑罰者,民之所惡也,臣請當之。」於是宋君失刑而子罕用之,故宋君見劫。田常徒用德而簡公弒,子罕徒用刑而宋君劫。故今世為人臣者兼刑德而用之,則是世主之危甚於簡公、宋君也。故劫殺擁蔽之主,非失刑德而使臣用之而不危亡者,則未嘗有也。

人主將欲禁姦,則審合刑名者,言異事也。為人臣者陳而言,君以其言授之事,專以其事責其功。功當其事,事當其言,則賞;功不當其事,事不當其言,則罰。故群臣其言大而功小者則罰,非罰小功也,罰功不當名也。群臣其言小而功大者亦罰,非不說於大功也,以為不當名也害甚於有大功,故罰。昔者韓昭侯醉而寢,典冠者見君之寒也,故加衣於君之上,覺寢而說,問左右曰:「誰加衣者?」左右對曰:「典冠。」君因兼罪典衣與典冠。其罪典衣、以為失其事也,其罪典冠、以為越其職也。非不惡寒也,以為侵官之害甚於寒。故明主之畜臣,臣不得越官而有功,不得陳言而不當。越官則死,不當則罪,守業其官所言者貞也,則群臣不得朋黨相為矣。

人主有二患:任賢,則臣將乘於賢以劫其君;妄舉,則事沮不勝。故人主好賢,則群臣飾行以要君欲,則是群臣之情不效;群臣之情不效,則人主無以異其臣矣。故越王好勇,而民多輕死;楚靈王好細腰,而國中多餓人;齊桓公妒而好內,故豎刁自宮以治內,桓公好味,易牙蒸其子首而進之;燕子噲好賢,故子之明不受國。故君見惡則群臣匿端,君見好則群臣誣能。人主欲見,則群臣之情態得其資矣。故子之託於賢以奪其君者也,豎刁、易牙因君之欲以侵其君者也,其卒子噲以亂死,桓公蟲流出戶而不葬。此其故何也?人君以情借臣之患也。人臣之情非必能愛其君也,為重利之故也。今人主不掩其情,不匿其端,而使人臣有緣以侵其主,則群臣為子之、田常不難矣。故曰:去好去惡,群臣見素。群臣見素,則大君不蔽矣。


\end{pinyinscope}