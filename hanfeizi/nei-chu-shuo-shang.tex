\article{內儲說上}

\begin{pinyinscope}
主之所用也七術,所察也六微。七術:一曰、眾端參觀,二曰、必罰明威,三曰、信賞盡能,四曰、一聽責下,五曰、疑詔詭使,六曰、挾知而問,七曰、倒言反事。此七者,主之所用也。

觀聽不參則誠不聞,聽有門戶則臣壅塞。其說在侏儒之夢見灶,哀公之稱莫眾而迷。故齊人見河伯,與惠子之言亡其半也。其患在豎牛之餓叔孫,而江乙之說荊俗也。嗣公欲治不知,故使有敵。是以明主推積鐵之類,而察一市之患。

參觀一

愛多者則法不立,威寡者則下侵上。是以刑罰不必則禁令不行。其說在董子之行石邑,與子產之教游吉也。故仲尼說隕霜,而殷法刑棄灰;將行去樂池,而公孫鞅重輕罪。是以麗水之金不守,而積澤之火不救。成歡以太仁弱齊國,卜皮以慈惠亡魏王。管仲知之,故斷死人。嗣公知之,故買胥靡。

必罰二

賞譽薄而謾者,下不用也,賞譽厚而信者下輕死。其說在文子稱若獸鹿。故越王焚宮室,而吳起倚車轅,李悝斷訟以射,宋崇門以毀死。句踐知之,故式怒蛙。昭侯知之,故藏弊褲。厚賞之使人為賁、諸也,婦人之拾蠶,漁者之握鱣,是以效之。

賞譽三

一聽則智愚不分,責下則人臣不參。其說在索鄭與吹竽。其患在申子之以趙紹、韓沓為嘗試。故公子氾議割河東,而應侯謀弛上黨。

一聽四

數見久待而不任,姦則鹿散。使人問他則不鬻私。是以龐敬還公大夫,而戴讙詔視轀車。周主亡玉簪,商太宰論牛矢。

詭使五

挾智而問,則不智者至;深智一物,眾隱皆變。其說在昭侯之握一爪也。故必南門而三鄉得。周主索曲杖而群臣懼,卜皮事庶子,西門豹詳遺轄。

挾智六

倒言反事以嘗所疑則姦情得。故陽山謾樛豎,淖齒為秦使,齊人欲為亂,子之以白馬,子產離訟者,嗣公過關市。

倒言七

右經

說一

衛靈公之時,彌子瑕有寵,專於衛國,侏儒有見公者曰:「臣之夢踐矣。」公曰:「何夢?」對曰:「夢見灶,為見公也。」公怒曰:「吾聞見人主者夢見日,奚為見寡人而夢見灶?」對曰:「夫日兼燭天下,一物不能當也。人君兼燭一國,一人不能壅也,故將見人主者夢見日。夫灶一人煬焉,則後人無從見矣。今或者一人、有煬君者乎?則臣雖夢見灶,不亦可乎!」

魯哀公問於孔子曰:「鄙諺曰:莫眾而迷。今寡人舉事,與群臣慮之,而國愈亂,其故何也?」孔子對曰:「明主之問臣,一人知之,一人不知也。如是者,明主在上,群臣直議於下。今群臣無不一辭同軌乎季孫者,舉魯國盡化為一,君雖問境內之人,猶不免於亂也。」

一曰。晏子聘魯,哀公問曰:「語曰:莫三人而迷。今寡人與一國慮之,魯不免於亂何也?」晏子曰:「古之所謂莫三人而迷者,一人失之,二人得之,三人足以為眾矣,故曰莫三人而迷。今魯國之群臣以千百數,一言於季氏之私,人數非不眾,所言者一人也,安得三哉?」

齊人有謂齊王曰:「河伯,大神也。王何不試與之遇乎?臣請使王遇之。」乃為壇場大水之上,而與王立之焉。有閒,大魚動,因曰:「此河伯。」

張儀欲以秦、韓與魏之勢伐齊、荊,而惠施欲以齊、荊偃兵。二人爭之,群臣左右皆為張子言,而以攻齊、荊為利,而莫為惠子言,王果聽張子,而以惠子言為不可。攻齊、荊事已定,惠子入見,王言曰:「先生毋言矣。攻齊、荊之事果利矣,一國盡以為然。」惠子因說:「不可不察也。夫齊、荊之事也誠利,一國盡以為利,是何智者之眾也?攻齊、荊之事誠不利,一國盡以為利,何愚者之眾也?凡謀者,疑也。疑也者,誠疑,以為可者半,以為不可者半。今一國盡以為可,是王亡半也。劫主者固亡其半者也。」

叔孫相魯,貴而主斷。其所愛者曰豎牛,亦擅用叔孫之令。叔孫有子曰壬,豎牛妒而欲殺之,因與壬游於魯君所,魯君賜之玉環,壬拜受之而不敢佩,使豎牛請之叔孫,豎牛欺之曰:「吾已為爾請之矣,使爾佩之。」壬因佩之,豎牛因謂叔孫:「何不見壬於君乎?」叔孫曰:「孺子何足見也。」豎牛曰:「壬固已數見於君矣。君賜之玉環,壬已佩之矣。」叔孫召壬見之,而果佩之,叔孫怒而殺壬。壬兄曰丙,豎牛又妒而欲殺之,叔孫為丙鑄鐘,鐘成,丙不敢擊,使豎牛請之叔孫,豎牛不為請,又欺之曰:「吾已為爾請之矣。使爾擊之。」丙因擊之,叔孫聞之曰:「丙不請而擅擊鐘。」怒而逐之。丙出走齊,居一年,豎牛為謝叔孫,叔孫使豎牛召之,又不召而報之曰:「吾已召之矣,丙怒甚,不肯來。」叔孫大怒,使人殺之。二子已死,叔孫有病,豎牛因獨養之而去左右,不內人,曰:「叔孫不欲聞人聲。」因不食而餓殺。叔孫已死,豎牛因不發喪也,徙其府庫重寶空之而奔齊。夫聽所信之言,而子父為人僇,此不參之患也。

江乙為魏王使荊,謂荊王曰:「臣入王之境內,聞王之國俗曰:君子不蔽人之美,不言人之惡,誠有之乎?」王曰:「有之。」「然則若白公之亂,得庶無危乎!誠得如此,臣免死罪矣。」

衛嗣君重如耳,愛世姬,而恐其皆因其愛重以壅己也,乃貴薄疑以敵如耳,尊魏姬以耦世姬,曰:「以是相參也。」嗣君知欲無壅,而未得其術也。夫不使賤議貴,下必坐上,而必待勢重之鈞也,而後敢相議,則是益樹壅塞之臣也。嗣君之壅乃始。

夫矢來有鄉,則積鐵以備一鄉;矢來無鄉,則為鐵室以盡備之。備之則體不傷。故彼以盡備之不傷,此以盡敵之無姦也。

龐恭與太子質於邯鄲,謂魏王曰:「今一人言市有虎,王信之乎?」曰:「不信。」「二人言市有虎,王信之乎?」曰:「不信。」「三人言市有虎,王信之乎?」王曰:「寡人信之。」龐恭曰:「夫市之無虎也明矣,然而三人言而成虎。今邯鄲之去魏也遠於市,議臣者過於三人,願王察之。」龐恭從邯鄲反,竟不得見。

說二

董閼于為趙上地守,行石邑山中,澗深,峭如牆,深百仞,因問其旁鄉左右曰:「人嘗有入此者乎?」對曰:「無有。」曰:「嬰兒癡聾狂悖之人嘗有入此者乎?」對曰:「無有。」「牛馬犬彘嘗有入此者乎?」對曰:「無有。」董閼于喟然太息曰:「吾能治矣。使吾法之無赦,猶入澗之必死也,則人莫之敢犯也,何為不治?」

子產相鄭,病將死,謂游吉曰:「我死後,子必用鄭,必以嚴蒞人。夫火形嚴,故人鮮灼;水形懦,人多溺。子必嚴子之形,無令溺子之懦。」故子產死,游吉不肯嚴形,鄭少年相率為盜,處於雚澤,將遂以為鄭禍。游吉率車騎與戰,一日一夜,僅能剋之。游吉喟然歎曰:「吾蚤行夫子之教,必不悔至於此矣。」

魯哀公問於仲尼曰:「春秋之記曰:冬十二月霣霜不殺菽,何為記此?」仲尼對曰:「此言可以殺而不殺也。夫宜殺而不殺,桃李冬實。天失道,草木猶犯干之,而況於人君乎?」

殷之法刑棄灰於街者,子貢以為重,問之仲尼,仲尼曰:「知治之道也。夫棄灰於街必掩人,掩人人必怒,怒則鬥,鬥必三族相殘也。此殘三族之道也,雖刑之可也。且夫重罰者,人之所惡也,而無棄灰,人之所易也。使人行之所易,而無離所惡,此治之道。」

一曰。殷之法,棄灰于公道者斷其手,子貢曰:「棄灰之罪輕,斷手之罰重,古人何太毅也?」曰:「無棄灰所易也,斷手所惡也,行所易不關所惡,古人以為易,故行之。」

中山之相樂池以車百乘使趙,選其客之有智能者以為將行,中道而亂,樂池曰:「吾以公為有智,而使公為將行,今中道而亂何也?」客因辭而去曰:「公不知治,有威足以服人,而利足以勸之,故能治之。今臣,君之少客也。夫從少正長,從賤治貴,而不得操其利害之柄以制之,此所以亂也。嘗試使臣彼之善者我能以為卿相,彼不善者我得以斬其首,何故而不治?」

公孫鞅之法也重輕罪。重罪者人之所難犯也,而小過者人之所易去也,使人去其所易無離其所難,此治之道。夫小過不生,大罪不至,是人無罪而亂不生也。

一曰。公孫鞅曰:「行刑重其輕者,輕者不至,重者不來,是謂以刑去刑。」

荊南之地、麗水之中生金,人多竊采金,采金之禁,得而輒辜磔於市,甚眾,壅離其水也,而人竊金不止。夫罪莫重辜磔於市,猶不止者,不必得也。故今有於此,曰:「予汝天下而殺汝身」,庸人不為也。夫有天下,大利也,猶不為者,知必死。故不必得也,則雖辜磔,竊金不止;知必死,則天下不為也。

魯人燒積澤,天北風,火南倚,恐燒國,哀公懼,自將眾趣救火,左右無人,盡逐獸而火不救,乃召問仲尼,仲尼曰:「夫逐獸者樂而無罰,救火者苦而無賞,此火之所以無救也。」哀公曰:「善。」仲尼曰:「事急,不及以賞,救火者盡賞之,則國不足以賞於人,請徒行罰。」哀公曰:「善。」於是仲尼乃下令曰:「不救火者比降北之罪,逐獸者比入禁之罪。」令下未遍而火已救矣。

成驩謂齊王曰:「王太仁,太不忍人。」王曰:「太仁、太不忍人,非善名邪?」對曰:「此人臣之善也,非人主之所行也。夫人臣必仁而後可與謀,不忍人而後可近也。不仁則不可與謀,忍人則不可近也。」王曰:「然則寡人安所太仁、安不忍人?」對曰:「王太仁於薛公,而太不忍於諸田。太仁薛公則大臣無重,太不忍諸田則父兄犯法。大臣無重則兵弱於外,父兄犯法則政亂於內。兵弱於外、政亂於內,此亡國之本也。」

魏惠王謂卜皮曰:「子聞寡人之聲聞亦何如焉?」對曰:「臣聞王之慈惠也。」王欣然喜曰:「然則功且安至?」對曰:「王之功至於亡。」王曰:「慈惠,行善也,行之而亡何也?」卜皮對曰:「夫慈者不忍,而惠者好與也。不忍則不誅有過,好予則不待有功而賞。有過不罪,無功受賞,雖亡不亦可乎?」

齊國好厚葬,布帛盡於衣衾,材木盡於棺槨,桓公患之,以告管仲曰:「布帛盡則無以為蔽,材木盡則無以為守備,而人厚葬之不休,禁之奈何?」管仲對曰:「凡人之有為也,非名之則利之也。」於是乃下令曰:「棺槨過度者戮其尸,罪夫當喪者。」夫戮死無名,罪當喪者無利,人何故為之也?

衛嗣君之時,有胥靡逃之魏,因為襄王之后治病,衛嗣君聞之,使人請以五十金買之,五反而魏王不予,乃以左氏易之。群臣左右諫曰:「夫以一都買胥靡可乎?」王曰:「非子之所知也。夫治無小而亂無大,法不立而誅不必,雖有十左氏無益也。法立而誅必,雖失十左氏無害也。」魏王聞之曰:「主欲治而不聽之,不祥。」因載而往,徒獻之。

說三

齊王問於文子曰:「治國何如?」對曰:「夫賞罰之為道,利器也。君固握之,不可以示人。若如臣者,猶獸鹿也,唯薦草而就。」

越王問於大夫文種曰:「吾欲伐吳可乎?」對曰:「可矣。吾賞厚而信,罰嚴而必。君欲知之,何不試焚宮室?」於是遂焚宮室,人莫救之,乃下令曰:「人之救火者,死,比死敵之賞。救火而不死者,比勝敵之賞。不救火者,比降北之罪。」人塗其體、被濡衣而走火者,左三千人,右三千人。此知必勝之勢也。

吳起為魏武侯西河之守,秦有小亭臨境,吳起欲攻之。不去,則甚害田者;去之,則不足以徵甲兵。於是乃倚一車轅於北門之外而令之曰:「有能徙此南門之外者賜之上田上宅。」人莫之徙也,及有徙之者,還,賜之如令。俄又置一石赤菽東門之外而令之曰:「有能徙此於西門之外者賜之如初。」人爭徙之。乃下令曰:「明日且攻亭,有能先登者,仕之國大夫,賜之上田宅。」人爭趨之,於是攻亭一朝而拔之。

李悝為魏文侯上地之守,而欲人之善射也,乃下令曰:「人之有狐疑之訟者,令之射的,中之者勝,不中者負。」令下而人皆疾習射,日夜不休。及與秦人戰,大敗之,以人之善射也。

宋崇門之巷人服喪,而毀甚瘠,上以為慈愛於親,舉以為官師。明年,人之所以毀死者歲十餘人。子之服親喪者為愛之也,而尚可以賞勸也,況君上之於民乎?

越王慮伐吳,欲人之輕死也,出見怒蛙乃為之式,從者曰:「奚敬於此?」王曰:「為其有氣故也。」明年之請以頭獻王者歲十餘人。由此觀之,譽之足以殺人矣。

一曰。越王句踐見怒蛙而式之,御者曰:「何為式?」王曰:「蛙有氣如此,可無為式乎?」士人聞之曰:「蛙有氣,王猶為式,況士人之有勇者乎!」是歲人有自剄死以其頭獻者。故越王將復吳而試其教,燔臺而鼓之,使民赴火者,賞在火也,臨江而鼓之,使人赴水者,賞在水也,臨戰而使人絕頭刳腹而無顧心者,賞在兵也,又況據法而進賢,其助甚此矣。

韓昭侯使人藏弊褲,侍者曰:「君亦不仁矣,弊褲不以賜左右而藏之。」昭侯曰:「非子之所知也,吾聞明主之愛,一嚬一笑,嚬有為嚬,而笑有為笑。今夫褲豈特嚬笑哉!褲之與嚬笑相去遠矣,吾必待有功者,故藏之未有予也。」

鱣似蛇,蠶似蠋。人見蛇則驚駭,見蠋則毛起。然而婦人拾蠶,漁者握鱣,利之所在,則忘其所惡,皆為孟賁。

說四

魏王謂鄭王曰:「始鄭、梁一國也,已而別,今願復得鄭而合之梁。」鄭君患之,召群臣而與之謀所以對魏,鄭公子謂鄭君曰:「此甚易應也。君對魏曰:以鄭為故魏而可合也,則弊邑亦願得梁而合之鄭。」魏王乃止。

齊宣王使人吹竽,必三百人,南郭處士請為王吹竽,宣王說之,廩食以數百人。宣王死,湣王立,好一一聽之,處士逃。

一曰。韓昭侯曰:「吹竽者眾,吾無以知其善者。」田嚴對曰:「一一而聽之。」

趙令人因申子於韓請兵,將以攻魏,申子欲言之君,而恐君之疑己外市也,不則恐惡於趙,乃令趙紹、韓沓嘗試君之動貌而後言之,內則知昭侯之意,外則有得趙之功。

三國兵至韓,秦王謂樓緩曰:「三國之兵深矣,寡人欲割河東而講,何如?」對曰:「夫割河東,大費也;免國於患,大功也。此父兄之任也,王何不召公子氾而問焉?」王召公子氾而告之,對曰:「講亦悔,不講亦悔。王今割河東而講,三國歸,王必曰:三國固且去矣,吾特以三城送之。不講,三國也入韓,則國必大舉矣,王必大悔,王曰:不獻三城也。臣故曰:王講亦悔,不講亦悔。」王曰:「為我悔也,寧亡三城而悔,無危乃悔。寡人斷講矣。」

應侯謂秦王曰:「王得宛葉、藍田、陽夏,斷河內,因梁、鄭,所以未王者,趙未服也。弛上黨在一而已以臨東陽,則邯鄲口中虱也。王拱而朝天下,後者以兵中之。然上黨之安樂,其處甚劇,臣恐弛之而不聽,奈何?」王曰:「必弛易之矣。」

說五

龐敬,縣令也,遣市者行,而召公大夫而還之,立有間,無以詔之,卒遣行,市者以為令與公大夫有言,不相信,以至無姦。

戴驩、宋太宰,夜使人曰:「吾聞數夜有乘轀車至李史門者,謹為我伺之。」使人報曰:「不見轀車,見有奉笥而與李史語者,有間,李史受笥。」

周主亡玉簪,令吏求之,三日不能得也,周主令人求而得之家人之屋閒,周主曰:「吾知吏之不事事也。求簪,三日不得之,吾令人求之,不移日而得之。」於是吏皆聳懼,以為君、神明也。

商太宰使少庶子之市,顧反而問之曰:「何見於市?」對曰:「無見也。」太宰曰:「雖然何見也?」對曰:「市南門之外甚眾牛車,僅可以行耳。」太宰因誡使者無敢告人吾所問於女,因召市吏而誚之曰:「市門之外何多牛屎?」市吏甚怪太宰知之疾也,乃悚懼其所也。

說六

韓昭侯握爪而佯亡一爪,求之甚急,左右因割其爪而效之,昭侯以此察左右之誠不。

韓昭侯使騎於縣,使者報,昭侯問曰:「何見也?」對曰:「無所見也。」昭侯曰:「雖然何見?」曰:「南門之外,有黃犢食苗道左者。」昭侯謂使者「毋敢洩吾所問於女」,乃下令曰:「當苗時,禁牛馬入人田中固有令,而吏不以為事,牛馬甚多入人田中,亟舉其數上之,不得,將重其罪。」於是三鄉舉而上之,昭侯曰:「未盡也。」復往審之,乃得南門之外黃犢,吏以昭侯為明察,皆悚懼其所而不敢為非。

周主下令索曲杖,吏求之數日不能得,周主私使人求之,不移日而得之,乃謂吏曰:「吾知吏不事事也。曲杖甚易也,而吏不能得,我令人求之,不移日而得之,豈可謂忠哉?」吏乃皆悚懼其所,以君為神明。

卜皮為縣令。其御史汙穢,而有愛妾,卜皮乃使少庶子佯愛之以知御史陰情。

西門豹為鄴令,佯亡其車轄,令吏求之不能得,使人求之而得之家人屋間。

說七

陽山君相衛,聞王之疑己也,乃偽謗樛豎以知之。

淖齒聞齊王之惡己也,乃矯為秦使以知之。

齊人有欲為亂者,恐王知之,因詐逐所愛者,令走王知之。

子之相燕,坐而佯言曰:「走出門者何白馬也?」左右皆言不見。有一人走追之,報曰:「有。」子之以此知左右之誠信不。

有相與訟者,子產離之而無使得通辭,倒其言以告而知之。

衛嗣公使人為客過關市,關市苛難之,因事關市以金,關吏乃舍之,嗣公為關吏曰:「某時有客過而所,與汝金,而汝因遣之。」關市乃大恐,而以嗣公為明察。


\end{pinyinscope}