\article{用人}

\begin{pinyinscope}
聞古之善用人者,必循天順人而明賞罰。循天則用力寡而功立,順人則刑罰省而令行,明賞罰則伯夷、盜跖不亂。如此,則白黑分矣。治國之臣,效功於國以履位,見能於官以受職,盡力於權衡以任事。人臣皆宜其能,勝其官,輕其任,而莫懷餘力於心,莫負兼官之責於君。故內無伏怨之亂,外無馬服之患。明君使事不相干,故莫訟;使士不兼官,故技長,使人不同功,故莫爭。爭訟止,技長立,則彊弱不觳力,冰炭不合形,天下莫得相傷,治之至也。

釋法術而心治,堯不能正一國。去規矩而妄意度,奚仲不能成一輪。廢尺寸而差短長,王爾不能半中。使中主守法術,拙匠守規矩尺寸,則萬不失矣。君人者,能去賢巧之所不能,守中拙之所萬不失,則人力盡而功名立。

明主立可為之賞,設可避之罰。故賢者勸賞而不見子胥之禍,不肖者少罪而不見傴剖背,盲者處平而不遇深谿,愚者守靜而不陷險危。如此,則上下之恩結矣。古之人曰:「其心難知,喜怒難中也。」故以表示目,以鼓語耳,以法教心。君人者釋三易之數而行一難知之心,如此,則怒積於上,而怨積於下,以積怒而御積怨則兩危矣。明主之表易見,故約立;其教易知,故言用;其法易為,故令行。三者立而上無私心,則下得循法而治,望表而動,隨繩而斲,因攢而縫。如此,則上無私威之毒,而下無愚拙之誅。故上君明而少怒,下盡忠而少罪。

聞之曰:「舉事無患者,堯不得也。」而世未嘗無事也。君人者不輕爵祿,不易富貴,不可與救危國。故明主厲廉恥,招仁義。昔者介子推無爵祿而義隨文公,不忍口腹而仁割其肌,故人主結其德,書圖著其名。人主樂乎使人以公盡力,而苦乎以私奪威。人臣安乎以能受職,而苦乎以一負二。故明主除人臣之所苦,而立人主之所樂,上下之利,莫長於此。不察私門之內,輕慮重事,厚誅薄惱,久怨細過,長侮偷快,數以德追禍,是斷手而續以玉也,故世有易身之患。

人主立難為而罪不及,則私怨生;人臣失所長而奉難給,則伏怨結。勞苦不撫循,憂悲不哀憐。喜則譽小人,賢不肖俱賞;怒則毀君子,使伯夷與盜跖俱辱;故臣有叛主。

使燕王內憎其民而外愛魯人,則燕不用而魯不附。民見憎,不能盡力而務功;魯見說,而不能離死命而親他主。如此,則人臣為隙穴,而人主獨立。以隙穴之臣而事獨立之主,此之謂危殆。

釋儀的而妄發,雖中小不巧;釋法制而妄怒,雖殺戮而姦人不恐。罪生甲,禍歸乙,伏怨乃結。故至治之國,有賞罰,而無喜怒,故聖人極;有刑法而死,無螫毒,故姦人服。發矢中的,賞罰當符,故堯復生,羿復立。如此,則上無殷、夏之患,下無比干之禍,君高枕而臣樂業,道蔽天地,德極萬世矣。

夫人主不塞隙穴,而勞力於赭堊,暴雨疾風必壞。不去眉睫之禍,而慕賁、育之死;不謹蕭牆之患,而固金城於遠境;不用近賢之謀,而外結萬乘之交於千里。飄風一旦起,則賁、育不及救,而外交不及至,禍莫大於此。當今之世,為人主忠計者,必無使燕王說魯人,無使近世慕賢於古,無思越人以救中國溺者,如此,則上下親,內功立,外名成。


\end{pinyinscope}