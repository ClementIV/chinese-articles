\article{和氏}

\begin{pinyinscope}
楚人和氏得玉璞楚山中,奉而獻之厲王,厲王使玉人相之,玉人曰:「石也。」王以和為誑,而刖其左足。及厲王薨,武王即位,和又奉其璞而獻之武王,武王使玉人相之,又曰「石也」,王又以和為誑,而刖其右足。武王薨,文王即位,和乃抱其璞而哭於楚山之下,三日三夜,泣盡而繼之以血。王聞之,使人問其故,曰:「天下之刖者多矣,子奚哭之悲也?」和曰:「吾非悲刖也,悲夫寶玉而題之以石,貞士而名之以誑,此吾所以悲也。」王乃使玉人理其璞而得寶焉,遂命曰:「和氏之璧。」

夫珠玉人主之所急也,和雖獻璞而未美,未為主之害也,然猶兩足斬而寶乃論,論寶若此其難也。今人主之於法術也,未必和璧之急也,而禁群臣士民之私邪;然則有道者之不僇也,特帝王之璞未獻耳。主用術則大臣不得擅斷,近習不敢賣重;官行法則浮萌趨於耕農,而游士危於戰陳。則法術者乃群臣士民之所禍也。人主非能倍大臣之議,越民萌之誹,獨周乎道言也。則法術之士雖至死亡,道必不論矣。

昔者吳起教楚悼王以楚國之俗曰:「大臣太重,封君太眾,若此則上偪主而下虐民,此貧國弱兵之道也。不如使封君之子孫三世而收爵祿,絕滅百吏之祿秩,損不急之枝官,以奉選練之士。」悼王行之期年而薨矣,吳起枝解於楚。商君教秦孝公以連什伍,設告坐之過,燔詩書而明法令,塞私門之請而遂公家之勞,禁游宦之民而顯耕戰之士。孝公行之,主以尊安,國以富強,八年而薨,商君車裂於秦。楚不用吳起而削亂,秦行商君法而富強,二子之言也已當矣,然而枝解吳起而車裂商君者何也?大臣苦法而細民惡治也。當今之世,大臣貪重,細民安亂,甚於秦、楚之俗,而人主無悼王、孝公之聽,則法術之士,安能蒙二子之危也而明己之法術哉!此世所以亂無霸王也。


\end{pinyinscope}