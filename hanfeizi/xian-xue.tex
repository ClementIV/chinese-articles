\article{顯學}

\begin{pinyinscope}
世之顯學,儒、墨也。儒之所至,孔丘也。墨之所至,墨翟也。自孔子之死也,有子張之儒,有子思之儒,有顏氏之儒,有孟氏之儒,有漆雕氏之儒,有仲良氏之儒,有孫氏之儒,有樂正氏之儒。自墨子之死也,有相里氏之墨,有相夫氏之墨,有鄧陵氏之墨。故孔、墨之後,儒分為八,墨離為三,取舍相反、不同,而皆自謂真孔、墨,孔、墨不可復生,將誰使定世之學乎?孔子、墨子俱道堯、舜,而取舍不同,皆自謂真堯、舜,堯、舜不復生,將誰使定儒、墨之誠乎?殷、周七百餘歲,虞、夏二千餘歲,而不能定儒、墨之真,今乃欲審堯、舜之道於三千歲之前,意者其不可必乎!無參驗而必之者、愚也,弗能必而據之者、誣也。故明據先王,必定堯、舜者,非愚則誣也。愚誣之學,雜反之行,明主弗受也。

墨者之葬也,冬日冬服,夏日夏服,桐棺三寸,服喪三月,世主以為儉而禮之。儒者破家而葬,服喪三年,大毀扶杖,世主以為孝而禮之。夫是墨子之儉,將非孔子之侈也;是孔子之孝,將非墨子之戾也。今孝戾、侈儉俱在儒、墨,而上兼禮之。漆雕之議,不色撓,不目逃,行曲則違於臧獲,行直則怒於諸侯,世主以為廉而禮之。宋榮子之議,設不鬥爭,取不隨仇,不羞囹圄,見侮不辱,世主以為寬而禮之。夫是漆雕之廉,將非宋榮之恕也;是宋榮之寬,將非漆雕之暴也。今寬廉、恕暴俱在二子,人主兼而禮之。自愚誣之學、雜反之辭爭,而人主俱聽之,故海內之士,言無定術,行無常議。夫冰炭不同器而久,寒暑不兼時而至,雜反之學不兩立而治,今兼聽雜學繆行同異之辭,安得無亂乎?聽行如此,其於治人又必然矣。

今世之學士語治者多曰:「與貧窮地以實無資。」今夫與人相若也,無豐年旁入之利而獨以完給者,非力則儉也。與人相若也,無饑饉疾疚禍罪之殃獨以貧窮者,非侈則墯也。侈而墯者貧,而力而儉者富。今上徵斂於富人以布施於貧家,是奪力儉而與侈墯也。而欲索民之疾作而節用,不可得也。

今有人於此,義不入危城,不處軍旅,不以天下大利易其脛一毛,世主必從而禮之,貴其智而高其行,以為輕物重生之士也。夫上所以陳良田大宅、設爵祿,所以易民死命也,今上尊貴輕物重生之士、而索民之出死而重殉上事,不可得也。藏書策、習談論、聚徒役、服文學而議說,世主必從而禮之,曰:「敬賢士,先王之道也。」夫吏之所稅,耕者也;而上之所養,學士也。耕者則重稅,學士則多賞,而索民之疾作而少言談,不可得也。立節參民,執操不侵,怨言過於耳必隨之以劍,世主必從而禮之,以為自好之士。夫斬首之勞不賞,而家鬥之勇尊顯,而索民之疾戰距敵而無私鬥,不可得也。國平則養儒俠,難至則用介士,所養者非所用,所用者非所養,此所以亂也。且夫人主於聽學也,若是其言、宜布之官而用其身,若非其言、宜去其身而息其端。今以為是也而弗布於官,以為非也而不息其端,是而不用,非而不息,亂亡之道也。

澹臺子羽,君子之容也,仲尼幾而取之,與處久而行不稱其貌。宰予之辭,雅而文也,仲尼幾而取之,與處而智不充其辯。故孔子曰:「以容取人乎,失之子羽;以言取人乎,失之宰予。」故以仲尼之智而有失實之聲。今之新辯濫乎宰予,而世主之聽眩乎仲尼,為悅其言,因任其身,則焉得無失乎?是以魏任孟卯之辯而有華下之患,趙任馬服之辯而有長平之禍;此二者,任辯之失也。夫視鍛錫而察青黃,區冶不能以必劍;水擊鵠雁,陸斷駒馬,則臧獲不疑鈍利。發齒吻形容,伯樂不能以必馬;授車就駕而觀其末塗,則臧獲不疑駑良。觀容服,聽辭言,仲尼不能以必士;試之官職,課其功伐,則庸人不疑於愚智。故明主之吏,宰相必起於州部,猛將必發於卒伍。夫有功者必賞,則爵祿厚而愈勸;遷官襲級,則官職大而愈治。夫爵祿大而官職治,王之道也。

磐石千里,不可謂富;象人百萬,不可謂強。石非不大,數非不眾也,而不可謂富強者,磐不生粟,象人不可使距敵也。今商官技藝之士亦不墾而食,是地不墾與磐石一貫也。儒俠毋軍勞、顯而榮者則民不使,與象人同事也。夫禍知磐石象人,而不知禍商官儒俠為不墾之地、不使之民,不知事類者也。

故敵國之君王雖說吾義,吾弗入貢而臣;關內之侯雖非吾行,吾必使執禽而朝。是故力多則人朝,力寡則朝於人,故明君務力。夫嚴家無悍虜,而慈母有敗子,吾以此知威勢之可以禁暴,而德厚之不足以止亂也。

夫聖人之治國,不恃人之為吾善也,而用其不得為非也。恃人之為吾善也,境內不什數;用人不得為非,一國可使齊。為治者用眾而舍寡,故不務德而務法。夫必恃自直之箭,百世無矢;恃自圜之木,千世無輪矣。自直之箭、自圜之木,百世無有一,然而世皆乘車射禽者何也?隱栝之道用也。雖有不恃隱栝而有自直之箭、自圜之木,良工弗貴也,何則?乘者非一人,射者非一發也。不恃賞罰而恃自善之民,明主弗貴也,何則?國法不可失,而所治非一人也。故有術之君,不隨適然之善,而行必然之道。

今或謂人曰:「使子必智而壽」,則世必以為狂。夫智、性也,壽、命也,性命者,非所學於人也,而以人之所不能為說人,此世之所以謂之為狂也。謂之不能,然則是諭也。夫諭、性也。以仁義教人,是以智與壽說也,有度之主弗受也。故善毛嗇、西施之美,無益吾面,用脂澤粉黛則倍其初。言先王之仁義,無益於治,明吾法度,必吾賞罰者亦國之脂澤粉黛也。故明主急其助而緩其頌,故不道仁義。

今巫祝之祝人曰:「使若千秋萬歲。」千秋萬歲之聲聒耳,而一日之壽無徵於人,此人所以簡巫祝也。今世儒者之說人主,不善今之所以為治,而語已治之功;不審官法之事,不察姦邪之情,而皆道上古之傳,譽先王之成功。儒者飾辭曰:「聽吾言則可以霸王。」此說者之巫祝,有度之主不受也。故明主舉實事,去無用;不道仁義者故,不聽學者之言。

今不知治者必曰:「得民之心。」欲得民之心而可以為治,則是伊尹、管仲無所用也,將聽民而已矣。民智之不可用,猶嬰兒之心也。夫嬰兒不剔首則腹痛,不揊痤則寖益,剔首、揊痤必一人抱之,慈母治之,然猶啼呼不止,嬰兒子不知犯其所小苦致其所大利也。今上急耕田墾草以厚民產也,而以上為酷;修刑重罰以為禁邪也,而以上為嚴;徵賦錢粟以實倉庫、且以救饑饉備軍旅也,而以上為貪;境內必知介,而無私解,并力疾鬥所以禽虜也,而以上為暴。此四者所以治安也,而民不知悅也。夫求聖通之士者,為民知之不足師用。昔禹決江濬河而民聚瓦石,子產開畝樹桑鄭人謗訾。禹利天下,子產存鄭,皆以受謗,夫民智之不足用亦明矣。故舉士而求賢智,為政而期適民,皆亂之端,未可與為治也。


\end{pinyinscope}