\article{八姦}

\begin{pinyinscope}
凡人臣之所道成姦者有八術:一曰在同床。何謂同床?曰:貴夫人,愛孺子,便僻好色,此人主之所惑也。託於燕處之虞,乘醉飽之時,而求其所欲,此必聽之術也。為人臣者內事之以金玉,使惑其主,此之謂同床。二曰在旁。何謂在旁?曰:優笑侏儒,左右近習,此人主未命而唯唯,未使而諾諾,先意承旨,觀貌察色以先主心者也。此皆俱進俱退,皆應皆對,一辭同軌以移主心者也。為人臣者內事之以金玉玩好,外為之行不法,使之化其主,此之謂在旁。三曰父兄。何謂父兄?曰:側室公子,人主之所親愛也,大臣廷吏,人主之所與度計也,此皆盡力畢議,人主之所必聽也。為人臣者事公子側室以音聲子女,收大臣廷吏以辭言,處約言事事成則進爵益祿,以勸其心使犯其主,此之謂父兄。四曰養殃。何謂養殃?曰:人主樂美宮室臺池、好飾子女狗馬以娛其心,此人主之殃也。為人臣者盡民力以美宮室臺池,重賦歛以飾子女狗馬,以娛其主而亂其心、從其所欲,而樹私利其間,此謂養殃。五曰民萌。何謂民萌?曰:為人臣者散公財以說民人,行小惠以取百姓,使朝廷市井皆勸譽己,以塞其主而成其所欲,此之謂民萌。六曰流行。何謂流行?曰:人主者,固壅其言談,希於聽論議,易移以辯說。為人臣者求諸侯之辯士、養國中之能說者,使之以語其私,為巧文之言,流行之辭,示之以利勢,懼之以患害,施屬虛辭以壞其主,此之謂流行。七曰威強。何謂威強?曰:君人者,以群臣百姓為威強者也。群臣百姓之所善則君善之,非群臣百姓之所善則君不善之。為人臣者,聚帶劍之客、養必死之士以彰其威,明為己者必利,不為己者必死,以恐其群臣百姓而行其私,此之謂威強。八曰四方。何謂四方?曰:君人者,國小則事大國,兵弱則畏強兵,大國之所索,小國必聽,強兵之所加,弱兵必服。為人臣者,重賦歛,盡府庫,虛其國以事大國,而用其威求誘其君;甚者舉兵以聚邊境而制歛於內,薄者數內大使以震其君,使之恐懼,此之謂四方。凡此八者,人臣之所以道成姦,世主所以壅劫,失其所有也,不可不察焉。

明君之於內也,娛其色而不行其謁,不使私請。其於左右也,使其身必責其言,不使益辭。其於父兄大臣也,聽其言也必使以罰任於後,不令妄舉。其於觀樂玩好也,必令之有所出,不使擅進不使擅退,群臣虞其意。其於德施也,縱禁財,發墳倉,利於民者,必出於君,不使人臣私其德。其於說議也,稱譽者所善,毀疵者所惡,必實其能、察其過,不使群臣相為語。其於勇力之士也,軍旅之功無踰賞,邑鬥之勇無赦罪,不使群臣行私財。其於諸侯之求索也,法則聽之,不法則距之。

所謂亡君者,非莫有其國也,而有之者,皆非己有也。令臣以外為制於內,則是君人者亡也,聽大國為救亡也,而亡亟於不聽,故不聽。群臣知不聽則不外諸侯,諸侯之不聽則不受之,臣誣其君矣。

明主之為官職爵祿也,所以進賢材勸有功也。故曰:賢材者,處厚祿任大官;功大者,有尊爵受重賞。官賢者量其能,賦祿者稱其功。是以賢者不誣能以事其主,有功者樂進其業,故事成功立。今則不然,不課賢不肖,論有功勞,用諸侯之重,聽左右之謁,父兄大臣上請爵祿於上,而下賣之以收財利及以樹私黨。故財利多者買官以為貴,有左右之交者請謁以成重。功勞之臣不論,官職之遷失謬。是以吏偷官而外交,棄事而財親。是以賢者懈怠而不勸,有功者隳而簡其業,此亡國之風也。


\end{pinyinscope}