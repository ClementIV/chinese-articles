\article{八說}

\begin{pinyinscope}
為故人行私謂之不棄,以公財分施謂之仁人,輕祿重身謂之君子,枉法曲親謂之有行,棄官寵交謂之有俠,離世遁上謂之高傲,交爭逆令謂之剛材,行惠取眾謂之得民。不棄者吏有姦也,仁人者公財損也,君子者民難使也,有行者法制毀也,有俠者官職曠也,高傲者民不事也,剛材者令不行也,得民者君上孤也。此八者匹夫之私譽,人主之大敗也。反此八者,匹夫之私毀,人主之公利也。人主不察社稷之利害,而用匹夫之私譽,索國之無危亂,不可得矣。

任人以事,存亡治亂之機也。無術以任人,無所任而不敗。人君之所任,非辯智則修潔也。任人者,使有勢也;智士者未必信也;為多其智,因惑其信也;以智士之計,處乘勢之資而為其私急,則君必欺焉。為智者之不可信也,故任修士;者,使斷事也,修士者未必智;為潔其身,因惑其智;以愚人之所惛,處治事之官而為其所然,則事必亂矣。故無術以用人,任智則君欺,任修則君事亂,此無術之患也。明君之道,賤德義貴,下必坐上,決誠以參,聽無門戶,故智者不得詐欺。計功而行賞,程能而授事,察端而觀失,有過者罪,有能者得,故愚者不任事。智者不敢欺,愚者不得斷,則事無失矣。

察士然後能知之,不可以為令,夫民不盡察。賢者然後能行之,不可以為法,夫民不盡賢。楊朱、墨翟,天下之所察也,干世亂而卒不決,雖察而不可以為官職之令。鮑焦、華角,天下之所賢也,鮑焦木枯,華角赴河,雖賢不可以為耕戰之士。故人主之察,智士盡其辯焉;人主之所尊,能士盡其行焉。今世主察無用之辯,尊遠功之行,索國之富強,不可得也。博習辯智如孔、墨,孔、墨不耕耨,則國何得焉?修孝寡欲如曾、史,曾、史不戰攻,則國何利焉?匹夫有私便,人主有公利。不作而養足,不仕而名顯,此私便也。息文學而明法度,塞私便而一功勞,此公利也。錯法以道民也而又貴文學,則民之所師法也疑。賞功以勸民也而又尊行修,則民之產利也惰。夫貴文學以疑法,尊行修以貳功,索國之富強,不可得也。

搢笏干戚,不適有方鐵銛;登降周旋,不逮日中奏百;狸首射侯,不當強弩趨發;干城距衝,不若堙穴伏櫜。古人亟於德,中世逐於智,當今爭於力。古者寡事而備簡,樸陋而不盡,故有珧銚而推車者。古者人寡而相親,物多而輕利易讓,故有揖讓而傳天下者。然則行揖讓,高慈惠,而道仁厚,皆推政也。處多事之時,用寡事之器,非智者之備也;當大爭之世而循揖讓之軌,非聖人之治也。故智者不乘推車,聖人不行推政也。

法所以制事,事所以名功也。法有立而有難,權其難而事成則立之;事成而有害,權其害而功多則為之。無難之法,無害之功,天下無有也。是以拔千丈之都,敗十萬之眾,死傷者軍之乘,甲兵折挫,士卒死傷,而賀戰勝得地者,出其小害計其大利也。夫沐者有棄髮,除者傷血肉,為人見其難,因釋其業,是無術之事也。先聖有言曰:「規有摩,而水有波,我欲更之,無奈之何!」此通權之言也。是以說有必立而曠於實者,言有辭拙而急於用者,故聖人不求無害之言,而務無易之事。人之不事衡石者,非貞廉而遠利也,石不能為人多少,衡不能為人輕重,求索不能得,故人不事也。明主之國,官不敢枉法,吏不敢為私,貨賂不行,是境內之事盡如衡石也。此其臣有姦者必知,知者必誅。是以有道之主,不求清潔之吏,而務必知之術也。

慈母之於弱子也,愛不可為前。然而弱子有僻行,使之隨師;有惡病,使之事醫。不隨師則陷於刑,不事醫則疑於死。慈母雖愛,無益於振刑救死。則存子者非愛也,子母之性,愛也。臣主之權,筴也。母不能以愛存家,君安能以愛持國?明主者,通於富強則可以得欲矣。故謹於聽治,富強之法也。明其法禁,察其謀計。法明則內無變亂之患,計得則外無死虜之禍。故存國者,非仁義也。仁者,慈惠而輕財者也;暴者,心毅而易誅者也。慈惠則不忍,輕財則好與。心毅則憎心見於下,易誅則妄殺加於人。不忍則罰多宥赦,好與則賞多無功。憎心見則下怨其上,妄誅則民將背叛。故仁人在位,下肆而輕犯禁法,偷幸而望於上;暴人在位,則法令妄而臣主乖,民怨而亂心生。故曰:仁暴者,皆亡國者也。

不能具美食而勸餓人飯,不為能活餓者也;不能辟草生粟而勸貸施賞賜,不能為富民者也。今學者之言也,不務本作而好末事,知道虛聖以說民,此勸飯之說。勸飯之說,明主不受也。

書約而弟子辯,法省而民訟簡。是以聖人之書必著論,明主之法必詳事。盡思慮,揣得失,智者之所難也;無思無慮,挈前言而責後功,愚者之所易也。明主慮愚者之所易,以責智者之所難,故智慮力勞不用而國治也。

酸甘鹹淡,不以口斷而決於宰尹,則廚人輕君而重於宰尹矣。上下清濁,不以耳斷而決於樂正,則瞽工輕君而重於樂正矣。治國是非,不以術斷而決於寵人,則臣下輕君而重於寵人矣。人主不親觀聽,而制斷在下,託食於國者也。

使人不衣不食而不飢不寒,又不惡死,則無事上之意。意欲不宰於君,則不可使也。

今生殺之柄在大臣,而主令得行者,未嘗有也。虎豹必不用其爪牙而與鼷鼠同威,萬金之家、必不用其富厚而與監門同資。有土之君,說人不能利,惡人不能害,索人欲畏重己,不可得也。

人臣肆意陳欲曰俠,人主肆意陳欲曰亂;人臣輕上曰驕,人主輕下曰暴。行理同實,下以受譽,上以得非,人臣大得,人主大亡。

明主之國,有貴臣無重臣。貴臣者,爵尊而官大也;重臣者,言聽而力多者也。明主之國,遷官襲級,官爵受功,故有貴臣。言不度行,而有偽必誅,故無重臣也。


\end{pinyinscope}