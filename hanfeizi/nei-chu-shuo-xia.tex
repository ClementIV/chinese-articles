\article{內儲說下}

\begin{pinyinscope}
六微:一曰、權借在下,二曰、利異外借,三曰、託於似類,四曰、利害有反,五曰、參疑內爭,六曰、敵國廢置。此六者,主之所察也。

權勢不可以借人,上失其一,臣以為百。故臣得借則力多,力多則內外為用,內外為用則人主壅。其說在老聃之言失魚也。是以人主久語,而左右鬻懷刷。其患在胥僮之諫厲公,與州侯之一言,而燕人浴矢也。

權借一

君臣之利異,故人臣莫忠,故臣利立而主利滅。是以姦臣者,召敵兵以內除,舉外事以眩主,苟成其私利,不顧國患。其說在衛人之夫妻禱祝也。故戴歇議子弟,而三桓攻昭公;公叔內齊軍,而翟黃召韓兵;太宰嚭說大夫種,大成牛教申不害;司馬喜告趙王,呂倉規秦、楚;宋石遺衛君書,白圭教暴譴。

利異二

似類之事,人主之所以失誅,而大臣之所以成私也。是以門人捐水而夷射誅,濟陽自矯而二人罪,司馬喜殺爰騫而季辛誅,鄭袖言惡臭而新人劓,費無忌教郤宛而令尹誅,陳需殺張壽而犀首走。故燒芻廥而中山罪,殺老儒而濟陽賞也。

似類三

事起而有所利,其尸主之;有所害,必反察之。是以明主之論也,國害則省其利者,臣害則察其反者。其說在楚兵至而陳需相,黍種貴而廩吏覆。是以昭奚恤執販茅,而不僖侯譙其次;文公髮繞炙,而穰侯請立帝。

有反四

參疑之勢,亂之所由生也,故明主慎之。是以晉驪姬殺太子申生,而鄭夫人用毒藥,郤州吁殺其君完,公子根取東周,王子職甚有寵,而商臣果作亂,嚴遂、韓廆爭而哀侯果遇賊,田常、闞止、戴驩、皇喜敵而宋君、簡公殺。其說在狐突之稱二好,與鄭昭之對未生也。

參疑五

敵之所務在淫察而就靡,人主不察則敵廢置矣。故文王資費仲,而秦王患楚使,黎且去仲尼,而干象沮甘茂。是以子胥宣言而子常用,內美人而虞、虢亡,佯遺書而萇宏死,用雞猳而鄶桀盡。

廢置六

參疑廢置之事,明主絕之於內而施之於外。資其輕者,輔其弱者,此謂廟攻。參伍既用於內,觀聽又行於外,則敵偽得。其說在秦侏儒之告惠文君也。故襄疵言襲鄴,而嗣公賜令蓆。

廟攻

右經

說一

勢重者,人主之淵也;臣者,勢重之魚也。魚失於淵而不可復得也,人主失其勢重於臣而不可復收也。古之人難正言,故託之於魚。

賞罰者,利器也。君操之以制臣,臣得之以擁主。故君先見所賞則臣鬻之以為德,君先見所罰則臣鬻之以為威。故曰:「國之利器不可以示人。」

靖郭君相齊,與故人久語則故人富,懷左右刷則左右重。久語懷刷,小資也,猶以成富,況於吏勢乎?

晉厲公之時,六卿貴。胥僮長魚矯諫曰:「大臣貴重,敵主爭事,外市樹黨,下亂國法,上以劫主,而國不危者,未嘗有也。」公曰:「善。」乃誅三卿。胥僮長魚矯又諫曰:「夫同罪之人偏誅而不盡,是懷怨而借之閒也。」公曰:「吾一朝而夷三卿,予不忍盡也。」長魚矯對曰:「公不忍之,彼將忍公。」公不聽,居三月,諸卿作難,遂殺厲公而分其地。

州侯相荊,貴而主斷,荊王疑之,因問左右,左右對曰「無有」,如出一口也。

燕人無惑,故浴狗矢。燕人、其妻有私通於士,其夫早自外而來,士適出,夫曰:「何客也?」其妻曰:「無客。」問左右,左右言無有,如出一口。其妻曰:「公惑易也。」因浴之以狗矢。

一曰。燕人李季好遠出,其妻私有通於士,季突至,士在內中,妻患之,其室婦曰:「令公子裸而解髮直出門,吾屬佯不見也。」於是公子從其計,疾走出門,季曰:「是何人也?」家室皆曰:「無有。」季曰:「吾見鬼乎?」婦人曰:「然。」「為之奈何?」曰:「取五姓之矢浴之。」季曰:「諾。」乃浴以矢。一曰浴以蘭湯。

說二

衛人有夫妻禱者,而祝曰:「使我無故,得百束布。」其夫曰:「何少也?」對曰:「益是,子將以買妾。」

荊王欲宦諸公子於四鄰,戴歇曰:「不可。」「宦公子於四鄰,四鄰必重之」。曰:「子出者重,重則必為所重之國黨,則是教子於外市也,不便。」

魯孟孫、叔孫、季孫相戮力劫昭公,遂奪其國而擅其制。魯三桓公偪,昭公攻季孫氏,而孟孫氏、叔孫氏相與謀曰:「救之乎?」叔孫氏之御者曰:「我,家臣也,安知公家?凡有季孫與無季孫於我孰利?」皆曰:「無季孫必無叔孫。」「然則救之。」於是撞西北隅而入,孟孫見叔孫之旗入,亦救之,三桓為一,昭公不勝,逐之死於乾侯。

公叔相韓而有攻齊,公仲甚重於王,公叔恐王之相公仲也,使齊、韓約而攻魏,公叔因內齊軍於鄭,以劫其君,以固其位,而信兩國之約。

翟璜,魏王之臣也,而善於韓,乃召韓兵令之攻魏,因請為魏王搆之以自重也。

越王攻吳,王吳王謝而告服,越王欲許之,范蠡、大夫種曰:「不可。昔天以越與吳,吳不受,今天反夫差,亦天禍也。以吳予越,再拜受之,不可許也。」太宰嚭遺大夫種書曰:「狡兔盡則良犬烹,敵國滅則謀臣亡。大夫何不釋吳而患越乎?」大夫種受書讀之,太息而歎曰:「殺之,越與吳同命。」

大成牛從趙謂申不害於韓曰:「以韓重我於趙,請以趙重子於韓,是子有兩韓,我有兩趙。」

司馬喜,中山君之臣也,而善於趙,嘗以中山之謀微告趙王。

呂倉,魏王之臣也,而善於秦、荊,微諷秦、荊令之攻魏,因請行和以自重也。

宋石,魏將也。衛君,荊將也。兩國搆難,二子皆將,宋石遺衛君書曰:「二軍相當,兩旗相望,唯毋一戰,戰必不兩存,此乃兩主之事也,與子無有私怨,善者相避也。」

白圭相魏,暴譴相韓。白圭謂暴譴曰:「子以韓輔我於魏,我請以魏待子於韓,臣長用魏,子長用韓。」

說三

齊中大夫有夷射者,御飲於王,醉甚而出,倚於郎門,門者刖跪請曰:「足下無意賜之餘瀝乎?」夷射曰:「叱去!刑餘之人,何事乃敢乞飲長者?」刖跪走退,及夷射去,刖跪因捐水郎門霤下,類溺者之狀。明日,王出而訶之曰:「誰溺於是?」刖跪對曰:「臣不見也。雖然,昨日中大夫夷射立於此。」王因誅夷射而殺之。

魏王臣二人不善濟陽君,濟陽君因偽令人矯王命而謀攻己,王使人問濟陽君曰:「誰與恨?」對曰:「無敢與恨,雖然,嘗與二人不善,不足以至於此。」王問左右,左右曰:「固然。」王因誅二人者。

季辛與爰騫相怨。司馬喜新與季辛惡,因微令人殺爰騫,中山之君以為季辛也,因誅之。

荊王所愛妾有鄭袖者。荊王新得美女,鄭袖因教之曰:「王甚喜人之掩口也,為近王,必掩口。」美女入見,近王,因掩口,王問其故,鄭袖曰:「此固言惡王之臭。」及王與鄭袖、美女三人坐,袖因先誡御者曰:「王適有言,必亟聽從。」王言美女前,近王,甚數掩口,王悖然怒曰:「劓之。」御因揄刀而劓美人。

一曰。魏王遺荊王美人,荊王甚悅之,夫人鄭袖知王悅愛之也,亦悅愛之,甚於王,衣服玩好擇其所欲為之,王曰:「夫人知我愛新人也,其悅愛之甚於寡人,此孝子所以養親,忠臣之所以事君也。」夫人知王之不以己為妒也,因為新人曰:「王甚悅愛子,然惡子之鼻,子見王,常掩鼻,則王長幸子矣。」於是新人從之,每見王,常掩鼻,王謂夫人曰:「新人見寡人常掩鼻何也?」對曰:「不己知也。」王強問之,對曰:「頃嘗言惡聞王臭。」王怒曰:「劓之。」夫人先誡御者曰:「王適有言,必可從命。」御者因揄刀而劓美人。

費無極,荊令尹之近者也。郤宛新事令尹,令尹甚愛之,無極因謂令尹曰:「君愛宛甚,何不一為酒其家?」令尹曰:「善。」因令之為具於郤宛之家。無極教宛曰:「令尹甚傲而好兵,子必謹敬,先亟陳兵堂下及門庭。」宛因為之。令尹往而大驚曰:「此何也?」無極曰:「君殆去之,事未可知也。」令尹大怒,舉兵而誅郤宛,遂殺之。

犀首與張壽為怨,陳需新入,不善犀首,因使人微殺張壽,魏王以為犀首也,乃誅之。

中山有賤公子,馬甚瘦,車甚弊,左右有私不善者,乃為之請王曰:「公子甚貧,馬甚瘦,王何不益之馬食?」王不許,左右因微令夜燒芻廄,王以為賤公子也,乃誅之。

魏有老儒而不善濟陽君,客有與老儒私怨者,因攻老儒殺之以德於濟陽君曰:「臣為其不善君也,故為君殺之。」濟陽君因不察而賞之。

一曰。濟陽君有少庶子,有不見知,欲入愛於君者,齊使老儒掘藥於馬梨之山,濟陽少庶子欲以為功,入見於君曰:「齊使老儒掘藥於馬梨之山,名掘藥也,實閒君之國,君殺之,是將以濟陽君抵罪於齊矣。臣請刺之。」君曰:「可。」於是明日得之城陰而刺之,濟陽君還益親之。

說四

陳需,魏王之臣也,善於荊王,而令荊攻魏,荊攻魏,陳需因請為魏王行解之,因以荊勢相魏。

韓昭侯之時,黍種嘗貴甚,昭侯令人覆廩,吏果竊黍種而糶之甚多。

昭奚恤之用荊也,有燒倉廥窌者,而不知其人,昭奚恤令吏執販茅者而問之,果燒也。

昭僖侯之時,宰人上食而羹中有生肝焉。昭侯召宰人之次而誚之曰:「若何為置生肝寡人羹中?」宰人頓首服死罪曰:「竊欲去尚宰人也。」

一曰。僖侯浴,湯中有礫,僖侯曰:「尚浴免則有當代者乎?」左右對曰:「有。」僖侯曰:「召而來。」譙之曰:「何為置礫湯中?」對曰:「尚浴免,則臣得代之,是以置礫湯中。」

文公之時,宰臣上炙而髮繞之,文公召宰人而譙之曰:「女欲寡人之哽邪?奚為以髮繞炙。」宰人頓首再拜請曰:「臣有死罪三:援礪砥刀,利猶干將也,切肉,肉斷而髮不斷,臣之罪一也;援木而貫臠而不見髮,臣之罪二也;奉熾爐,炭火盡赤紅,而炙熟而髮不燒,臣之罪三也。堂下得無微有疾臣者乎?」公曰:「善。」乃召其堂下而譙之,果然,乃誅之。

一曰。晉平公觴客,少庶子進炙而髮繞之,平公趣殺炮人,毋有反令,炮人呼天曰:「嗟乎!臣有三罪,死而不自知乎?」平公曰:「何謂也?」對曰:「臣刀之利,風靡骨斷而髮不斷,是臣之一死也;桑炭炙之,肉紅白而髮不焦,是臣之二死也;炙熟又重睫而視之,髮繞炙而目不見,是臣之三死也。意者堂下其有翳憎臣者乎?殺臣不亦蚤乎!」

穰侯相秦而齊強,穰侯欲立秦為帝而齊不聽,因請立齊為東帝而不能成也。

說五

晉獻公之時,驪姬貴,擬於后妻,而欲以其子奚齊代太子申生,因患申生於君而殺之,遂立奚齊為太子。

鄭君已立太子矣,而有所愛美女欲以其子為後,夫人恐,因用毒藥賊君殺之。

衛州吁重於衛,擬於君,群臣百姓盡畏其勢重,州吁果殺其君而奪之政。

公子朝,周太子也,弟公子根甚有寵於君,君死,遂以東周叛,分為兩國。

楚成王以商臣為太子,既而又欲置公子職。商臣作亂,遂攻殺成王。

一曰。楚成王商臣為太子,既欲置公子職。商臣聞之,未察也,乃為其傅潘崇曰:「奈何察之也?」潘崇曰:「饗江芊而勿敬也。」太子聽之。江芊曰:「呼役夫!宜君王之欲廢女而立職也。」商臣曰:「信矣。」潘崇曰:「能事之乎?」曰:「不能。」「能為之諸侯乎?」曰:「不能。」「能舉大事乎?」曰:「能。」於是乃起宿營之甲而攻成王,成王請食能膰而死,不許,遂自殺。

韓廆相韓哀侯,嚴遂重於君,二人甚相害也,嚴遂乃令人刺韓廆於朝,韓廆走君而抱之,遂刺韓廆而兼哀侯。

田恆相齊,闞止重於簡公,二人相憎而欲相賊也,田恆因行私惠以取其國,遂殺簡公而奪之政。

戴驩為宋太宰,皇喜重於君,二人爭事而相害也,皇喜遂殺宋君而奪其政。

狐突曰:「國君好內則太子危,好外則相室危。」

鄭君問鄭昭曰:「太子亦何如?」對曰:「太子未生也。」君曰:「太子已置而曰未生何也?」對曰:「太子雖置,然而君之好色不已,所愛有子,君必愛之,愛之則必欲以為後,臣故曰太子未生也。」

說六

文王資費仲而游於紂之旁,令之諫紂而亂其心。

荊王使人之秦,秦王甚禮之。王曰:「敵國有賢者,國之憂也。今荊王之使者甚賢,寡人患之。」群臣諫曰:「以王之賢聖與國之資厚,願荊王之賢人。王何不深知之而陰有之,荊以為外用也,則必誅之。」

仲尼為政於魯,道不拾遺,齊景公患之,梨且謂景公曰:「去仲尼猶吹毛耳。君何不迎之以重祿高位,遺哀公女樂以驕榮其意。哀公新樂之,必怠於政,仲尼必諫,諫必輕絕於魯。」景公曰:「善。」乃令梨且以女樂二八遺哀公,哀公樂之,果怠於政,仲尼諫,不聽,去而之楚。

楚王謂干象曰:「吾欲以楚扶甘茂而相之秦可乎?」干象對曰:「不可也。」王曰:「何也?」曰:「甘茂少而事史舉先生,史舉,上蔡之監門也,大不事君,小不事家,以苛刻聞天下,茂事之順焉。惠王之明,張儀之辨也,茂事之,取十官而免於罪,是茂賢也。」王曰:「相人敵國而相賢,其不可何也?」干象曰:「前時王使邵滑之越,五年而能亡越,所以然者,越亂而楚治也。日者知用之越,今亡之秦,不亦太亟忘乎!」王曰:「然則為之奈何?」干象對曰:「不如相共立。」王曰:「共立可相何也?」對曰:「共立少見愛幸,長為貴卿,被王衣,含杜若,握玉環,以聽於朝。且利以亂秦矣。」

吳政荊,子胥使人宣言於荊曰:「子期用,將擊之。子常用,將去之。」荊人聞之,因用子常而退子期也。吳人擊之,遂勝之。

晉獻公伐虞、虢,乃遺之屈產之乘,垂棘之璧,女樂二八,以榮其意而亂其政。

叔向之讒萇弘也,為書曰:「萇弘謂叔向曰:子為我謂晉君,所與君期者時可矣,何不亟以兵來?」因佯遺其書周君之庭而急去行,周以萇弘為賣周也,乃誅萇弘而殺之。

鄭桓公將欲襲鄶,先問鄶之豪傑良臣辯智果敢之士,盡與其姓名,擇鄶之良田賂之,為官爵之名而書之,因為設壇場郭門之外而埋之,釁之以雞豭,若盟狀。鄶君以為內難也而盡殺其良臣,桓公襲鄶,遂取之。

說七

七秦侏儒善於荊王,而陰有善荊王左右而內重於惠文君,荊適有謀,侏儒常先聞之以告惠文君。

鄴令襄疵,陰善趙王左右,趙王謀襲鄴,襄疵常輒聞而先言之魏王,魏王備之,趙乃輒還。

衛嗣君之時,有人於令之左右,縣令有發蓐而席弊甚,嗣公還令人遺之席曰:「吾聞汝今者發蓐而席弊甚,賜汝席。」縣令大驚,以君為神也。


\end{pinyinscope}