\article{人主}

\begin{pinyinscope}
人主之所以身危國亡者,大臣太貴,左右太威也。所謂貴者,無法而擅行,操國柄而便私者也。所謂威者,擅權勢而輕重者也。此二者,不可不察也。夫馬之所以能任重引車致遠道者,以筋力也。萬乘之主、千乘之君所以制天下而征諸侯者,以其威勢也。威勢者,人主之筋力也。今大臣得威,左右擅勢,是人主失力,人主失力而能有國者,千無一人。虎豹之所以能勝人執百獸者,以其爪牙也,當使虎豹失其爪牙,則人必制之矣。今勢重者,人主之爪牙也,君人而失其爪牙,虎豹之類也。宋君失其爪牙於子罕,簡公失其爪牙於田常,而不蚤奪之,故身死國亡。今無術之主,皆明知宋、簡之過也,而不悟其失,不察其事類者也。

且法術之士,與當途之臣,不相容也。何以明之?主有術士,則大臣不得制斷,近習不敢賣重,大臣左右權勢息,則人主之道明矣。今則不然,其當途之臣得勢擅事以環其私,左右近習朋黨比周以制疏遠,則法術之士奚時得進用,人主奚時得論裁?故有術不必用,而勢不兩立,法術之士焉得無危?故君人者非能退大臣之議,而背左右之訟,獨合乎道言也;則法術之士安能蒙死亡之危而進說乎?此世之所以不治也。明主者,推功而爵祿,稱能而官事,所舉者必有賢,所用者必有能,賢能之士進,則私門之請止矣。夫有功者受重祿,有能者處大官,則私劍之士安得無離於私勇而疾距敵,游宦之士焉得無撓於私門而務於清潔矣?此所以聚賢能之士,而散私門之屬也。今近習者不必智,人主之於人也或有所知而聽之,入因與近習論其言,聽近習而不計其智,是與愚論智也。其當途者不必賢,人主之於人或有所賢而禮之,入因與當途者論其行,聽其言而不用賢,是與不肖論賢也。故智者決策於愚人,賢士程行於不肖,則賢智之士奚時得用,而主之明塞矣。昔關龍逢說桀而傷其四肢,王子比干諫紂而剖其心,子胥忠直夫差而誅於屬鏤。此三子者,為人臣非不忠,而說非不當也。然不免於死亡之患者,主不察賢智之言,而蔽於愚不肖之患也。今人主非肯用法術之士,聽愚不肖之臣,則賢智之士、孰敢當三子之危而進其智能者乎?此世之所以亂也。


\end{pinyinscope}