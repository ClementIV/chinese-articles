\article{喻老}

\begin{pinyinscope}
天下有道無急患則曰靜,遽傳不用,故曰:「卻走馬以糞。」天下無道,攻擊不休,相守數年不已,甲冑生蟣蝨,鷰雀處帷幄,而兵不歸,故曰:「戎馬生於郊。」

翟人有獻豐狐、玄豹之皮於晉文公,文公受客皮而歎曰:「此以皮之美自為罪。」夫治國者以名號為罪,徐偃王是也。以城與地為罪,虞、虢是也。故曰:「罪莫大於可欲。」

智伯兼范、中行而攻趙不已,韓、魏反之,軍敗晉陽,身死高梁之東,遂卒被分,漆其首以為溲器,故曰:「禍莫大於不知足。」

虞君欲屈產之乘,與垂棘之璧,不聽宮之奇,故邦亡身死,故曰:「咎莫憯於欲得。」

邦以存為常,霸王其可也。身以生為常,富貴其可也。不欲自害則邦不亡身不死,故曰:「知足之為足矣。」

楚莊王既勝狩於河雍,歸而賞孫叔敖,孫叔敖請漢間之地,沙石之處。楚邦之法,祿臣再世而收地,唯孫叔敖獨在。此不以其邦為收者,瘠也,故九世而祀不絕。故曰:「善建不拔,善抱不脫,子孫以其祭祀世世不輟」,孫叔敖之謂也。

制在己曰重,不離位曰靜。重則能使輕,靜則能使躁。故曰:「重為輕根,靜為躁君。故曰君子終日行不離輜重也。」邦者,人君之輜重也。主父生傳其邦,此離其輜重者也。故雖有代、雲中之樂,超然已無趙矣。主父,萬乘之主,而以身輕於天下,無勢之謂輕,離位之謂躁,是以生幽而死。故曰:「輕則失臣,躁則失君」,主父之謂也。

勢重者,人君之淵也。君人者勢重於人臣之閒,失則不可復得也。簡公失之於田成,晉公失之於六卿,而邦亡身死。故曰:「魚不可脫於深淵。」賞罰者,邦之利器也,在君則制臣,在臣則勝君。君見賞,臣則損之以為德;君見罰,臣則益之以為威。人君見賞而人臣用其勢,人君見罰而人臣乘其威。故曰:「邦之利器不可以示人。」

越王入宦於吳,而觀之伐齊以弊吳。吳兵既勝齊人於艾陵,張之於江、濟,強之於黃池,故可制於五湖。故曰:「將欲翕之,必固張之;將欲弱之,必固強之。」晉獻公將欲襲虞,遺之以璧馬;知伯將襲仇由,遺之以廣車。故曰:「將欲取之,必固與之。」起事於無形,而要大功於天下,是謂微明。處小弱而重自卑謂損弱勝強也。

有形之類,大必起於小;行久之物,族必起於少。故曰:「天下之難事必作於易,天下之大事必作於細。」是以欲制物者於其細也,故曰:「圖難於其易也,為大於其細也。」千丈之隄以螻蟻之穴潰,百尺之室以突隙之煙焚。故曰:白圭之行隄也塞其穴,丈人之慎火也塗其隙。是以白圭無水難,丈人無火患。此皆慎易以避難,敬細以遠大者也。扁鵲見蔡桓公,立有間,扁鵲曰:「君有疾在腠理,不治將恐深。」桓侯曰:「寡人無。」扁鵲出,桓侯曰:「醫之好治不病以為功。」居十日,扁鵲復見曰:「君之病在肌膚,不治將益深。」桓侯不應。扁鵲出,桓侯又不悅。居十日,扁鵲復見曰:「君之病在腸胃,不治將益深。」桓侯又不應。扁鵲出,桓侯又不悅。居十日,扁鵲望桓侯而還走。桓侯故使人問之,扁鵲曰:「疾在腠理,湯熨之所及也;在肌膚,鍼石之所及也;在腸胃,火齊之所及也;在骨髓,司命之所屬,無奈何也。今在骨髓,臣是以無請也。」居五日,桓公體痛,使人索扁鵲,已逃秦矣,桓侯遂死。故良醫之治病也,攻之於腠理,此皆爭之於小者也。夫事之禍福亦有腠理之地,故曰:聖人蚤從事焉。

昔晉公子重耳出亡過鄭,鄭君不禮,叔瞻諫曰:「此賢公子也,君厚待之,可以積德。」鄭君不聽。叔瞻又諫曰:「不厚待之,不若殺之,無令有後患。」鄭君又不聽。及公子返晉邦,舉兵伐鄭,大破之,取八城焉。晉獻公以垂棘之璧假道於虞而伐虢,大夫宮之奇諫曰:「不可。脣亡而齒寒,虞、虢相救,非相德也。今日晉滅虢,明日虞必隨之亡。」虞君不聽,受其璧而假之道。晉已取虢,還,反滅虞。此二臣者皆爭於腠理者也,而二君不用也。然則叔瞻、宮之奇亦虞、鄭之扁鵲也,而二君不聽,故鄭以破,虞以亡。故曰:「其安易持也,其未兆易謀也。」

昔者紂為象箸而箕子怖。以為象箸必不加於土鉶,必將犀玉之杯。象箸玉杯必不羹菽藿,則必旄象豹胎。旄象豹胎必不衣短褐而食於茅屋之下,則錦衣九重,廣室高臺。吾畏其卒,故怖其始。居五年,紂為肉圃,設炮烙,登糟邱,臨酒池,紂遂以亡。故箕子見象箸以知天下之禍,故曰:「見小曰明。」

句踐入宦於吳,身執干戈為吳王洗馬,故能殺夫差於姑蘇。文王見詈於王門,顏色不變,而武王擒紂於牧野。故曰:「守柔曰強。」越王之霸也不病宦,武王之王也不病詈。故曰:「聖人之不病也,以其不病,是以無病也。」

宋之鄙人得璞玉而獻之子罕,子罕不受,鄙人曰:「此寶也,宜為君子器,不宜為細人用。」子罕曰:「爾以玉為寶,我以不受子玉為寶。」是鄙人欲玉,而子罕不欲玉。故曰:「欲不欲,而不貴難得之貨。」

王壽負書而行,見徐馮於周塗,馮曰:「事者,為也。為生於時,知者無常事。書者,言也。言生於知,知者不藏書。今子何獨負之而行?」於是王壽因焚其書而舞之。故知者不以言談教,而慧者不以藏書篋。此世之所過也,而王壽復之,是學不學也。故曰:「學不學,復歸眾人之所過也。」

夫物有常容,因乘以導之,因隨物之容。故靜則建乎德,動則順乎道。宋人有為其君以象為楮葉者,三年而成。豐殺莖柯,毫芒繁澤,亂之楮葉之中而不可別也。此人遂以功食祿於宋邦。列子聞之曰:「使天地三年而成一葉,則物之有葉者寡矣。」故不乘天地之資,而載一人之身;不隨道理之數,而學一人之智;此皆一葉之行也。故冬耕之稼,后稷不能羨也;豐年大禾,臧獲不能惡也。以一人力,則后稷不足;隨自然,則臧獲有餘。故曰:「恃萬物之自然而不敢為也。」

空竅者,神明之戶牖也。耳目竭於聲色,精神竭於外貌,故中無主。中無主則禍福雖如丘山無從識之,故曰:「不出於戶,可以知天下;不闚於牖,可以知天道。」此言神明之不離其實也。

趙襄主學御於王子期,俄而與於期逐,三易馬而三後。襄主曰:「子之教我御術未盡也。」對曰:「術已盡,用之則過也。凡御之所貴,馬體安於車,人心調於馬,而後可以進速致遠。今君後則欲逮臣,先則恐逮於臣。夫誘道爭遠,非先則後也。而先後心皆在於臣,上何以調於馬,此君之所以後也。」白公勝慮亂,罷朝,倒杖而策銳貫顊,血流至於地而不知。鄭人聞之曰:「顊之忘,將何為忘哉!」故曰:「其出彌遠者,其智彌少。」此言智周乎遠,則所遺在近也,是以聖人無常行也。能並智,故曰:「不行而知。」能並視,故曰:「不見而明。」隨時以舉事,因資而立功,用萬物之能而獲利其上,故曰:「不為而成。」

楚莊王蒞政三年,無令發,無政為也。右司馬御座而與王隱曰:「有鳥止南方之阜,三年不翅不飛不鳴,嘿然無聲,此為何名?」王曰:「三年不翅,將以長羽翼。不飛不鳴,將以觀民則。雖無飛,飛必沖天;雖無鳴,鳴必驚人。子釋之,不穀知之矣。」處半年,乃自聽政,所廢者十,所起者九,誅大臣五,舉處士六,而邦大治。舉兵誅齊,敗之徐州,勝晉於河雍,合諸侯於宋,遂霸天下。莊王不為小害善,故有大名;不蚤見示,故有大功。故曰:「大器晚成,大音希聲。」

楚莊王欲伐越,杜子諫曰:「王之伐越何也?」曰:「政亂兵弱。」杜子曰:「臣愚患之。智如目也,能見百步之外而不能自見其睫。王之兵自敗於秦、晉,喪地數百里,此兵之弱也。莊蹻為盜於境內而吏不能禁,此政之亂也。王之弱亂非越之下也,而欲伐越,此智之如目也。」王乃止。故知之難,不在見人,在自見。故曰:「自見之謂明。」

子夏見曾子,曾子曰:「何肥也?」對曰:「戰勝故肥也。」曾子曰:「何謂也?」子夏曰:「吾入見先王之義則榮之,出見富貴之樂又榮之,兩者戰於胸中,未知勝負,故臞。今先王之義勝,故肥。」是以志之難也,不在勝人,在自勝也。故曰:「自勝之謂強。」

周有玉版,紂令膠鬲索之,文王不予,費仲來求,因予之。是膠鬲賢而費仲無道也。周惡賢者之得志也,故予費仲。文王舉太公於渭濱者,貴之也;而資費仲玉版者,是愛之也。故曰:「不貴其師,不愛其資,雖知大迷,是謂要妙。」


\end{pinyinscope}