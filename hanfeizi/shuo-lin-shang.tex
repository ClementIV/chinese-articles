\article{說林上}

\begin{pinyinscope}
湯以伐桀,而恐天下言己為貪也,因乃讓天下於務光。而恐務光之受之也,乃使人說務光曰:「湯殺君而欲傳惡聲于子,故讓天下於子。」務光因自投於河。

秦武王令甘茂擇所欲為於僕與行事,孟卯曰:「公不如為僕。公所長者、使也,公雖為僕,王猶使之於公也。公佩僕璽而為行事,是兼官也。」

子圉見孔子於商太宰,孔子出,子圉入,請問客,太宰曰:「吾已見孔子,則視子猶蚤蝨之細者也。吾今見之於君。」子圉恐孔子貴於君也,因謂太宰曰:「君已見孔子,亦將視子猶蚤蝨也。」太宰因弗復見也。

魏惠王為臼里之盟,將復立於天子,彭喜謂鄭君曰:「君勿聽,大國惡有天子,小國利之。若君與大不聽,魏焉能與小立之。」

晉人伐邢,齊桓公將救之,鮑叔曰:「太蚤。邢不亡,晉不敝,晉不敝,齊不重。且夫持危之功,不如存亡之德大。君不如晚救之以敝晉,齊實利。待邢亡而復存之,其名實美。」桓公乃弗救。

子胥出走,邊候得之,子胥曰:「上索我者,以我有美珠也。今我已亡之矣,我且曰子取吞之。」候因釋之。

慶封為亂於齊而欲走越,其族人曰:「晉近,奚不之晉?」慶封曰:「越遠,利以避難。」族人曰:「變是心也,居晉而可。不變是心也,雖遠越,其可以安乎!」

智伯索地於魏宣子,魏宣子弗予,任章曰:「何故不予?」宣子曰:「無故請地,故弗予。」任章曰:「無故索地,鄰國必恐,彼重欲無厭,天下必懼,君予之地,智伯必驕而輕敵,鄰邦必懼而相親,以相親之兵待輕敵之國,則智伯之命不長矣。《周書》曰:「將欲敗之,必姑輔之,將欲取之,必姑予之。」君不如予之以驕智伯。且君何釋以天下圖智氏,而獨以吾國為智氏質乎?」君曰:「善。」乃與之萬戶之邑,智伯大悅。因索地於趙,弗與,因圍晉陽,韓、魏反之外,趙氏應之內,智氏自亡。

秦康公築臺三年,荊人起兵,將欲以兵攻齊,任妄曰:「饑召兵,疾召兵,勞召兵,亂召兵。君築臺三年,今荊人起兵將攻齊,臣恐其攻齊為聲,而以襲秦為實也,不如備之。」戍東邊,荊人輟行。

齊攻宋,宋使臧孫子南求救於荊,荊大說,許救之,甚歡,臧孫子憂而反,其御曰:「索救而得,今子有憂色何也?」臧孫子曰:「宋小而齊大,夫救小宋而惡於大齊,此人之所以憂也,而荊王說,必以堅我也。我堅而齊敝,荊之所利也。」臧孫子乃歸,齊人拔五城於宋而荊救不至。

魏文侯借道於趙而攻中山,趙肅侯將不許,趙刻曰:「君過矣。魏攻中山而弗能取,則魏必罷,罷則魏輕,魏輕則趙重。魏拔中山,必不能越趙而有中山也,是用兵者魏也,而得地者趙也。君必許之。許之而大歡,彼將知君利之也,必將輟行。君不如借之道,示以不得已也。」

鴟夷子皮事田成子,田成子去齊,走而之燕,鴟夷子皮負傳而從,至望邑,子皮曰:「子獨不聞涸澤之蛇乎?澤涸,蛇將徙,有小蛇謂大蛇曰:子行而我隨之,人以為蛇之行者耳,必有殺子,不如相銜負我以行,人以我為神君也。乃相銜負以越公道,人皆避之,曰:神君也。今子美而我惡,以子為我上客,千乘之君也;以子為我使者,萬乘之卿也。子不如為我舍人。」田成子因負傳而隨之,至逆旅,逆旅之君待之甚敬,因獻酒肉。

溫人之周,周不納客,問之曰:「客耶?」對曰:「主人。」問其巷人而不知也,吏因囚之,君使人問之曰:「子非周人也,而自謂非客何也?」對曰:「臣少也誦《詩》曰:普天之下,莫非王土,率土之濱,莫非王臣。今君,天子,則我天子之臣也,豈有為人之臣而又為之客哉?故曰主人也。」君使出之。

韓宣王謂樛留曰:「吾欲兩用公仲、公叔其可乎?」對曰:「不可。晉用六卿而國分,簡公兩用田成、闞止而簡公殺,魏兩用犀首、張儀而西河之外亡。今王兩用之,其多力者樹其黨,寡力者借外權。群臣有內樹黨以驕主,有外為交以削地,則王之國危矣。」

紹績昧醉寐而亡其裘,宋君曰:「醉足以亡裘乎?」對曰:「桀以醉亡天下,而。《康誥》曰:『毋彝酒。』者,彝酒、常酒也,常酒者,天子失天下,匹夫失其身。」

管仲、隰朋從於桓公而伐孤竹,春往冬反,迷惑失道,管仲曰:「老馬之智可用也。」乃放老馬而隨之,遂得道。行山中無水,隰朋曰:「蟻冬居山之陽,夏居山之陰,蟻壤一寸而仞有水。」乃掘地,遂得水。以管仲之聖,而隰朋之智,至其所不知,不難師於老馬與蟻,今人不知以其愚心而師聖人之智,不亦過乎。

有獻不死之藥於荊王者,謁者操之以入,中射之士問曰:「可食乎?」曰:「可。」因奪而食之,王大怒,使人殺中射之士,中射之士使人說王曰:「臣問謁者曰可食,臣故食之,是臣無罪,而罪在謁者也。且客獻不死之藥,臣食之而王殺臣,是死藥也,是客欺王也。夫殺無罪之臣,而明人之欺王也,不如釋臣。」王乃不殺。

田駟欺鄒君,鄒君將使人殺之,田駟恐,告惠子,惠子見鄒君曰:「今有人見君,則眇其一目,奚如?」君曰:「我必殺之。」惠子曰:「瞽,兩目眇,君奚為不殺?」君曰:「不能勿犁。」惠子曰:「田駟東慢齊侯,南欺荊王,駟之於欺人,瞽也,君奚怨焉?」鄒君乃不殺。

魯穆公使眾公子或宦於晉,或宦於荊,犁鉏曰:「假人於越而救溺子,越人雖善遊,子必不生矣。失火而取水於海,海水雖多,火必不滅矣,遠水不救近火也。今晉與荊雖強,而齊近,魯患其不救乎?」

嚴遂不善周君,患之,馮沮曰:「嚴遂相,而韓傀貴於君,不如行賊於韓傀,則君必以為嚴氏也。」

張譴相韓,病將死,公乘無正懷三十金而問其疾,居一月自問張譴曰:「若子死,將誰使代子?」答曰:「無正重法而畏上,雖然,不如公子食我之得民也。」張譴死,因相公乘無正。

樂羊為魏將而攻中山,其子在中山,中山之君烹其子而遺之羹,樂羊坐於幕下而啜之,盡一杯,文侯謂堵師贊曰:「樂羊以我故而食其子之肉。」答曰:「其子而食之,且誰不食?」樂羊罷中山,文侯賞其功而疑其心。

孟孫獵得麑,使秦西巴持之歸,其母隨之而啼,秦西巴弗忍而與之,孟孫歸,至而求麑,答曰:「余弗忍而與其母。」孟孫大怒,逐之,居三月,復召以為其子傅,其御曰:「曩將罪之,今召以為子傅何也?」孟孫曰:「夫不忍麑,又且忍吾子乎?」故曰:「巧詐不如拙誠。」樂羊以有功見疑,秦西巴以有罪益信。

曾從子,善相劍者也。衛君怨吳王,曾從子曰:「吳王好劍,臣相劍者也,臣請為吳王相劍,拔而示之,因為君刺之。」衛君曰:「子為之是也,非緣義也,為利也。吳強而富,衛弱而貧,子必往,吾恐子為吳王用之於我也。」乃逐之。

紂為象箸而箕子怖,以為象箸必不盛羹於土簋,則必犀玉之杯,玉杯象箸必不盛菽藿,則必旄象豹胎,旄象豹胎必不衣短褐,而舍茅茨之下,則必錦衣九重,高臺廣室也。稱此以求,則天下不足矣。聖人見微以知萌,見端以知末,故見象箸而怖,知天下不足也。

周公旦已勝殷,將攻商、蓋,辛公甲曰:「大難攻,小易服,不如服眾小以劫大。」乃攻九夷而商、蓋服矣。

紂為長夜之飲,懼以失日,問其左右盡不知也,乃使人問箕子,箕子謂其徒曰:「為天下主而一國皆失日,天下其危矣。一國皆不知而我獨知之,吾其危矣。」辭以醉而不知。

魯人身善織屨,妻善織縞,而欲徒於越,或謂之曰:「子必窮矣。」魯人曰:「何也?」曰:「屨為履之也,而越人跣行;縞為冠之也,而越人被髮。以子之所長,游於不用之國,欲使無窮,其可得乎?」

陳軫貴於魏王,惠子曰:「必善事左右,夫楊橫樹之即生,倒樹之即生,折而樹之又生。然使十人樹之而一人拔之,則毋生楊至。以十人之眾,樹易生之物,而不勝一人者何也?樹之難而去之易也。子雖工自樹於王,而欲去子者眾,子必危矣。」

魯季孫新弒其君,吳起仕焉。或謂起曰:「夫死者,始死而血,已血而衄,已衄而灰,已灰而土,及其土也,無可為者矣。今季孫乃始血,其毋乃未可知也。」吳起因去之晉。

隰斯彌見田成子,田成子與登臺四望,三面皆暢,南望,隰子家之樹蔽之,田成子亦不言,隰子歸,使人伐之,斧離數創,隰子止之,其相室曰:「何變之數也?」隰子曰:「古者有諺曰:知淵中之魚者不祥。夫田子將有大事,而我示之知微,我必危矣。不伐樹未有罪也,知人之所不言,其罪大矣。」乃不伐也。

楊子過於宋東之逆旅,有妾二人,其惡者貴,美者賤。楊子問其故,逆旅之父答曰:「美者自美,吾不知其美也,惡者自惡,吾不知其惡也。」楊子謂弟子曰:「行賢而去自賢之心,焉往而不美。」

衛人嫁其子而教之曰:「必私積聚。為人婦而出,常也。其成居,幸也。」其子因私積聚,其姑以為多私而出之,其子所以反者倍其所以嫁。其父不自罪於教子非也,而自知其益富。今人臣之處官者皆是類也。

魯丹三說中山之君而不受也,因散五十金事其左右,復見,未語,而君與之食。魯丹出,而不反舍,遂去中山。其御曰:「反見,乃始善我,何故去之?」魯丹曰:「夫以人言善我,必以人言罪我。」未出境,而公子惡之曰:「為趙來閒中山。」君因索而罪之。

田伯鼎好士而存其君,白公好士而亂荊,其好士則同,其所以為則異。公孫友自刖而尊百里,豎刁自宮而諂桓公,其自刑則同,其所以自刑之為則異。慧子曰:「狂者東走,逐者亦東走,其東走則同,其所以東走之為則異。故曰:同事之人,不可不審察也。」


\end{pinyinscope}