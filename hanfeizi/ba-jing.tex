\article{八經}

\begin{pinyinscope}
一、凡治天下,必因人情。人情者,有好惡,故賞罰可用;賞罰可用則禁令可立而治道具矣。君執柄以處勢,故令行禁止。柄者,殺生之制也;勢者,勝眾之資也。廢置無度則權瀆,賞罰下共則威分。是以明主不懷愛而聽,不留說而計。故聽言不參則權分乎姦,智力不用則君窮乎臣。故明主之行制也天,其用人也鬼。天則不非,鬼則不困。勢行教嚴逆而不違,毀譽一行而不議。故賞賢罰暴,舉善之至者也;賞暴罰賢,舉惡之至者也;是謂賞同罰異。賞莫如厚,使民利之;譽莫如美,使民榮之;誅莫如重,使民畏之;毀莫如惡,使民恥之。然後一行其法,禁誅於私。家不害功罪,賞罰必知之,知之道盡矣。

因情

二、力不敵眾,智不盡物。與其用一人,不如用一國。故智力敵而群物勝,揣中則私勞,不中則在過。下君盡己之能,中君盡人之力,上君盡人之智。是以事至而結智,一聽而公會。聽不一則後悖於前,後悖於前則愚智不分;不公會則猶豫而不斷,不斷則事留。自取一,則毋墮壑之累。故使之諷,諷定而怒。是以言陳之日,必有筴籍,結智者事發而驗,結能者功見而。謀成敗,成敗有徵,賞罰隨之。事成則君收其功,規敗則臣任其罪。君人者合符猶不親,而況於力乎?事智猶不親,而況於懸乎?故非用人也不取同,同則君怒。使人相用則君神,君神則下盡。下盡下,則臣、上不因君而主道畢矣。

主道

三、知臣主之異利者王,以為同者劫,與共事者殺。故明主審公私之分,審利害之地,姦乃無所乘。亂之所生六也:主母,后姬,子姓,弟兄,大臣,顯賢。任吏責臣,主母不放。禮施異等,后姬不疑。分勢不貳,庶適不爭。權籍不失,兄弟不侵。下不一門,大臣不擁。禁賞必行,顯賢不亂。臣有二因,謂外內也。外曰畏,內曰愛。所畏之求得,所愛之言聽,此亂臣之所因也。外國之置諸吏者,結誅親暱重帑,則外不籍矣。爵祿循功,請者俱罪,則內不因矣。外不籍,內不因,則姦宄塞矣。官襲節而進,以至大任,智也。其位至而任大者,以三節持之,曰質、曰鎮、曰固。親戚妻子,質也。爵祿厚而必,鎮也。參伍貴帑,固也。賢者止於質,貪饕化於鎮,姦邪窮於固。忍不制則下上,小不除則大誅,而名實當則徑之。生害事,死傷名,則行飲食;不然,而與其讎;此謂除陰姦也。醫曰詭,詭曰易。易功而賞,見罪而罰,而詭乃止。是非不泄,說諫不通,而易乃不用。父兄賢良播出曰遊禍,其患鄰敵多資。僇辱之人近習曰狎賊,其患發忿疑辱之心生。藏怒持罪而不發曰增亂,其患徼幸妄舉之人起。大臣兩重、提衡而不踦曰卷禍,其患家隆劫殺之難作。脫易不自神曰彈威,其患賊夫酖毒之亂起。此五患者,人主之不知,則有劫殺之事。廢置之事,生於內則治,生於外則亂。是以明主以功論之內,而以利資之外,故其國治而敵亂。即亂之道,臣憎則起外若眩,臣愛則起內若藥。

起亂

四、參伍之道:行參以謀多,揆伍以責失;行參必拆,揆伍必怒。不拆則瀆上,不怒則相和。拆之徵足以知多寡,怒之前不及其眾。觀聽之勢,其徵在比周而賞異也。誅毋謁而罪同。言會眾端,必揆之以地,謀之以天,驗之以物,參之以人。四徵者符,乃可以觀矣。參言以知其誠,易視以改其澤。執見以得非常。一用以務近習,重言以懼遠使,舉往以悉其前,即邇以知其內,疏置以知其外,握明以問所闇,詭使以絕黷泄,倒言以嘗所疑,論反以得陰姦,設諫以綱獨為,舉錯以觀姦動,明說以誘避過,卑適以觀直諂,宣聞以通未見,作鬥以散朋黨,深一以警眾心,泄異以易其慮。似類則合其參,陳過則明其固,知罪辟罪以止威,陰使時循以省衰,漸更以離通比,下約以侵其上,相室約其廷臣,廷臣約其官屬,兵士約其軍吏,遣使約其行介,縣令約其辟吏,郎中約其左右,后姬約其宮媛,此之謂條達之道。言通事泄則術不行。

立道

五、明主,其務在周密。是以喜見則德償,怒見則威分。故明主之言隔塞而不通,周密而不見。故以一得十者下道也,以十得一者上道也。明主兼行上下,故姦無所失。伍、官、連、縣而鄰,謁過賞,失過誅。上之於下,下之於上,亦然。是故上下貴賤相畏以法,相誨以和。民之性,有生之實,有生之名。為君者有賢知之名,有賞罰之實。名實俱至,故福善必聞矣。

參言

六、聽不參則無以責下,言不督乎用則邪說當上。言之為物也以多信,不然之物,十人云疑,百人然乎,千人不可解也。吶者言之疑,辯者言之信。姦之食上也,取資乎眾,籍信乎辯,而以類飾其私。人主不饜忿而待合參,其勢資下也。有道之主,聽言、督其用,課其功,功課而賞罰生焉,故無用之辯不留朝。任事者知不足以治職,則放官收。說大而誇則窮端,故姦得而怒。無故而不當為誣,誣而罪,臣言必有報,說必責用也,故朋黨之言不上聞。凡聽之道,人臣忠論以聞姦,博論以內一,人主不智則姦得資。明主之道,己喜則求其所納,己怒則察其所搆;論於已變之後,以得毀譽公私之徵。眾諫以效智故,使君自取一以避罪。故眾之諫也,敗、君之取也。無副言於上以設將然,今符言於後以知謾誠語。明主之道,臣不得兩諫,必任其一語;不得擅行,必合其參;故姦無道進矣。

聽法

七、官之重也,毋法也;法之息也,上闇也。上闇無度則官擅為,官擅為故奉重,無前則徵多,徵多故富。官之富重也,亂功之所生也。明主之道,取於任,賢於官,賞於功;言程、主喜俱必利,不當、主怒俱必害,則人不私父兄而進其仇讎。勢足以行法,奉足以給事,而私無所生,故民勞苦而輕官。任事也毋重,使其寵必在爵;處官者毋私,使其利必在祿;故民尊爵而重祿。爵祿所以賞也,民重所以賞也則國治。刑之煩也,名之繆也,賞譽不當則民疑。民之重名與其重賞也均。賞者有誹焉,不足以勸;罰者有譽焉,不足以禁。明主之道,賞必出乎公利,名必在乎為上。賞譽同軌,非誅俱行,然則民無榮於賞之內。有重罰者必有惡名,故民畏。罰所以禁也,民畏所以禁則國治矣。

類柄

八、行義示則主威分,慈仁聽則法制毀。民以制畏上,而上以勢卑下,故下肆很觸而榮於輕君之俗則主威分。民以法難犯上,而上以法撓慈仁,故下明愛施而務賕紋之政,是以法令隳。尊私行以貳主威,行賕紋以疑法,聽之則亂治,不聽則謗主,故君輕乎位而法亂乎官,此之謂無常之國。明主之道,臣不得以行義成榮,不得以家利為功。功名所生,必出於官法;法之所外,雖有難行,不以顯焉;故民無以私名。設法度以齊民,信賞罰以盡民能,明誹譽以勸沮,名號、賞罰、法令三隅,故大臣有行則尊君,百姓有功則利上,此之謂有道之國也。


\end{pinyinscope}