\article{愛臣}

\begin{pinyinscope}
愛臣太親,必危其身;人臣太貴,必易主位;主妾無等,必危嫡子;兄弟不服,必危社稷。臣聞千乘之君無備,必有百乘之臣在其側,以徙其民而傾其國;萬乘之君無備,必有千乘之家在其側,以徙其威而傾其國。是以姦臣蕃息,主道衰亡。是故諸侯之博大,天子之害也;群臣之太富,君主之敗也。將相之管主而隆國家,此君人者所外也。萬物莫如身之至貴也,位之至尊也,主威之重,主勢之隆也,此四美者不求諸外,不請於人,議之而得之矣。故曰人主不能用其富,則終於外也。此君人者之所識也。

昔者紂之亡,周之卑,皆從諸侯之博大也;晉之分也,齊之奪也,皆以群臣之太富也。夫燕、宋之所以弒其君者,皆以類也。故上比之殷、周,中比之燕、宋,莫不從此術也。是故明君之蓄其臣也,盡之以法,質之以備。故不赦死,不宥刑,赦死宥刑,是謂威淫,社稷將危,國家偏威。是故大臣之祿雖大,不得藉威城市;黨與雖眾,不得臣士卒。故人臣處國無私朝,居軍無私交,其府庫不得私貸於家,此明君之所以禁其邪。是故不得四從;不載奇兵;非傳非遽,載奇兵革,罪死不赦。此明君之所以備不虞者也。


\end{pinyinscope}