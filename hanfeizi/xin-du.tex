\article{心度}

\begin{pinyinscope}
聖人之治民,度於本,不從其欲,期於利民而已。故其與之刑,非所以惡民,愛之本也。刑勝而民靜,賞繁而姦生,故治民者,刑勝、治之首也,賞繁、亂之本也。夫民之性,喜其亂而不親其法。故明主之治國也,明賞則民勸功,嚴刑則民親法。勸功則公事不犯,親法則姦無所萌。故治民者,禁姦於未萌;而用兵者,服戰於民心。禁先其本者治,兵戰其心者勝。聖人之治民也,先治者強,先戰者勝。夫國事務先而一民心,專舉公而私不從,賞告而姦不生,明法而治不煩,能用四者強,不能用四者弱。夫國之所以強者,政也;主之所以尊者,權也。故明君有權有政,亂君亦有權有政,積而不同,其所以立異也。故明君操權而上重,一政而國治。故法者,王之本也;刑者,愛之自也。

夫民之性,惡勞而樂佚,佚則荒,荒則不治,不治則亂而賞刑不行於天下,者必塞。故欲舉大功而難致而力者,大功不可幾而舉也;欲治其法而難變其故者,民亂,不可幾而治也。故治民無常,唯治為法。法與時轉則治,治與世宜則有功。故民樸、而禁之以名則治,世知、維之以刑則從。時移而治不易者亂,能治眾而禁不變者削。故聖人之治民也,法與時移而禁與能變。

能越力於地者富,能起力於敵者強,強不塞者王。故王道在所聞,在所塞。塞其姦者必王,故王術不恃外之不亂也,恃其不可亂也。恃外不亂而治立者削,恃其不可亂而行法者興。故賢君之治國也,適於不亂之術。貴爵則上重,故賞功爵任而邪無所關。好力者其爵貴,爵貴則上尊,上尊則必王。國不事力而恃私學者,其爵賤,爵賤則上卑,上卑者必削。故立國用民之道也,能閉外塞私而上自恃者,王可致也。


\end{pinyinscope}